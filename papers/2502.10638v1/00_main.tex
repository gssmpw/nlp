%%
%% This is file `sample-manuscript.tex',
%% generated with the docstrip utility.
%%
%% The original source files were:
%%
%% samples.dtx  (with options: `manuscript')
%% 
%% IMPORTANT NOTICE:
%% 
%% For the copyright see the source file.
%% 
%% Any modified versions of this file must be renamed
%% with new filenames distinct from sample-manuscript.tex.
%% 
%% For distribution of the original source see the terms
%% for copying and modification in the file samples.dtx.
%% 
%% This generated file may be distributed as long as the
%% original source files, as listed above, are part of the
%% same distribution. (The sources need not necessarily be
%% in the same archive or directory.)
%%
%% Commands for TeXCount
%TC:macro \cite [option:text,text]
%TC:macro \citep [option:text,text]
%TC:macro \citet [option:text,text]
%TC:envir table 0 1
%TC:envir table* 0 1
%TC:envir tabular [ignore] word
%TC:envir displaymath 0 word
%TC:envir math 0 word
%TC:envir comment 0 0
%%
%%
%% The first command in your LaTeX source must be the \documentclass command.
%%%% Small single column format, used for CIE, CSUR, DTRAP, JACM, JDIQ, JEA, JERIC, JETC, PACMCGIT, TAAS, TACCESS, TACO, TALG, TALLIP (formerly TALIP), TCPS, TDSCI, TEAC, TECS, TELO, THRI, TIIS, TIOT, TISSEC, TIST, TKDD, TMIS, TOCE, TOCHI, TOCL, TOCS, TOCT, TODAES, TODS, TOIS, TOIT, TOMACS, TOMM (formerly TOMCCAP), TOMPECS, TOMS, TOPC, TOPLAS, TOPS, TOS, TOSEM, TOSN, TQC, TRETS, TSAS, TSC, TSLP, TWEB.
% \documentclass[acmsmall]{acmart}

%%%% Large single column format, used for IMWUT, JOCCH, PACMPL, POMACS, TAP, PACMHCI
% \documentclass[acmlarge,screen]{acmart}

%%%% Large double column format, used for TOG
% \documentclass[acmtog, authorversion]{acmart}

%%%% Generic manuscript mode, required for submission
%%%% and peer review
% \documentclass[manuscript,screen,review, anonymous]{acmart}
\documentclass[sigconf]{acmart}
%% Fonts used in the template cannot be substituted; margin 
%% adjustments are not allowed.
%%
%% \BibTeX command to typeset BibTeX logo in the docs
\AtBeginDocument{%
  \providecommand\BibTeX{{%
    \normalfont B\kern-0.5em{\scshape i\kern-0.25em b}\kern-0.8em\TeX}}}

%% Rights management information.  This information is sent to you
%% when you complete the rights form.  These commands have SAMPLE
%% values in them; it is your responsibility as an author to replace
%% the commands and values with those provided to you when you
%% complete the rights form.
% \setcopyright{acmcopyright}
% \copyrightyear{2018}
% \acmYear{2018}
% \acmDOI{XXXXXXX.XXXXXXX}

% %% These commands are for a PROCEEDINGS abstract or paper.
% \acmConference[Conference acronym 'XX]{Make sure to enter the correct
%   conference title from your rights confirmation emai}{June 03--05,
%   2018}{Woodstock, NY}
% %
% %  Uncomment \acmBooktitle if th title of the proceedings is different
% %  from ``Proceedings of ...''!
% %
% \acmBooktitle{Woodstock '18: ACM Symposium on Neural Gaze Detection,
%  June 03--05, 2018, Woodstock, NY} 
% \acmPrice{15.00}
% \acmISBN{978-1-4503-XXXX-X/18/06}

\usepackage{xcolor}
\newcommand{\hari}[1]{{\textcolor{purple}{\bf [*** HS: #1]}}}
\newcommand{\roy}[1]{{\textcolor{blue}{\bf [*** RP: #1]}}}
\newcommand{\momin}[1]{{\textcolor{teal}{\bf [*** MS: #1]}}}
\newcommand{\change}[1]{{\textcolor{black}{#1}}}

\usepackage{xspace}
\newcommand{\system}{Script\&Shift\xspace} 
\newcommand{\systems}{Script\&Shift's\xspace}

\usepackage{fontawesome}
\usepackage{tikz}
\usepackage{graphicx}
\usepackage{subcaption}

\definecolor{skyblue}{RGB}{135, 206, 235}
\definecolor{lightestgray}{RGB}{230, 230, 230}
\definecolor{legalyellow}{RGB}{255, 255, 204}

\definecolor{ToneTara}{RGB}{52, 130, 206} % Blue
\definecolor{IdeaIvy}{RGB}{128, 0, 128}  % Purple
\definecolor{DetailDanny}{RGB}{249, 186, 97} % Light Orange
\definecolor{FeedbackFelix}{RGB}{34, 211, 238} % Cyan
\definecolor{AudienceAli}{RGB}{255, 102, 26} % Orange
\definecolor{StructureSam}{RGB}{123, 183, 116} % Green

% \newcommand{\pageiconwithnumber}[3][black]{%
%   \begin{tikzpicture}
%     \node[inner sep=0pt, text=#1] {\faFile}; % Draw the page icon
%     \node[align=center, inner sep=0pt, text=#2] at (0,0) {\normalsize \textbf{#3}}; % Overlay the large number
%   \end{tikzpicture}%
% }

\newcommand{\coloredcircle}[1]{%
  \tikz\draw[fill=#1, draw=none] (0,0) circle (1.2ex); % Larger radius for the circle
}


\newcommand{\pageiconwithnumber}[3][black]{%
  \raisebox{-.1\height}{% Lower the icon to align it with the text bottom
    \begin{tikzpicture}
      \node[inner sep=0pt, text=#1] {\faFile}; % Draw the page icon
      \node[align=center, inner sep=0pt, text=#2] at (0,0) {\normalsize\textbf{#3}}; % Overlay the number
    \end{tikzpicture}%
  }%
}


% Define the command for text with a dashed border and customizable colors
\newcommand{\customdashedbordertext}[3]{%
  \tikz[baseline=(X.base)] \node[draw=#2, rectangle, dashed, fill=#3, fill opacity=0.1, text opacity=1, inner sep=2pt, rounded corners] (X) {\color{black}#1};%
}

\newcommand{\customsolidbordertext}[3]{%
  \tikz[baseline=(X.base)] \node[draw=#2, rectangle, fill=#3, fill opacity=0.1, text opacity=1, inner sep=2pt, rounded corners] (X) {\color{black}#1};%
}

\newcommand{\twopageicon}[2]{%
  \begin{tikzpicture}[baseline=(current bounding box.south)]
    % Top page
    \draw[fill=#1, draw=black] (0, 0) rectangle ++(1em, 1.2ex);
    % Bottom page, directly below the top page
    \draw[fill=#2, draw=black] (0, -1.5ex) rectangle ++(1em, 1.2ex);
  \end{tikzpicture}%
}

% Define the command for a label/tag icon with a customizable color
\newcommand{\tagicon}[1]{%
  \begin{tikzpicture}[baseline=(current bounding box.south)]
    % Tag shape
    \draw[fill=#1, draw=black] 
      (0, 0) -- (1em, 0) -- (1em, 1.2ex) -- (0.6em, 1.2ex) -- (0.4em, 0.6ex) -- (0.6em, 0) -- cycle;
  \end{tikzpicture}%
}




%%
%% Submission ID.
%% Use this when submitting an article to a sponsored event. You'll
%% receive a unique submission ID from the organizers
%% of the event, and this ID should be used as the parameter to this command.
%%\acmSubmissionID{123-A56-BU3}

%%
%% For managing citations, it is recommended to use bibliography
%% files in BibTeX format.
%%
%% You can then either use BibTeX with the ACM-Reference-Format style,
%% or BibLaTeX with the acmnumeric or acmauthoryear sytles, that include
%% support for advanced citation of software artefact from the
%% biblatex-software package, also separately available on CTAN.
%%
%% Look at the sample-*-biblatex.tex files for templates showcasing
%% the biblatex styles.
%%

%%
%% The majority of ACM publications use numbered citations and
%% references.  The command \citestyle{authoryear} switches to the
%% "author year" style.
%%
%% If you are preparing content for an event
%% sponsored by ACM SIGGRAPH, you must use the "author year" style of
%% citations and references.
%% Uncommenting
%% the next command will enable that style.
%%\citestyle{acmauthoryear}

\copyrightyear{2025}
\acmYear{2025}
% \setcopyright{cc}
% \setcctype{by}
\acmConference[CHI '25]{CHI Conference on Human Factors in Computing
Systems}{April 26-May 1, 2025}{Yokohama, Japan}
\acmBooktitle{CHI Conference on Human Factors in Computing Systems (CHI
'25), April 26-May 1, 2025, Yokohama,
Japan}\acmDOI{10.1145/3706598.3714119}
\acmISBN{979-8-4007-1394-1/25/04}

%%
%% end of the preamble, start of the body of the document source.
\begin{document}

%%
%% The "title" command has an optional parameter,
%% allowing the author to define a "short title" to be used in page headers.
\title[A Layered Interface for Writing with LLMs]{\system: A Layered Interface Paradigm for Integrating Content Development and Rhetorical Strategy with LLM Writing Assistants}

%%
%% The "author" command and its associated commands are used to define
%% the authors and their affiliations.
%% Of note is the shared affiliation of the first two authors, and the
%% "authornote" and "authornotemark" commands
%% used to denote shared contribution to the research.



\author{Momin Siddiqui}
 \affiliation{%
  \institution{Georgia Institute of Technology}
   \country{USA}
 }
 \email{msiddiqui66@gatech.edu}


\author{Roy Pea}
 \affiliation{%
  \institution{Stanford University}
   \country{USA}
 }
 \email{roypea@stanford.edu}


\author{Hari Subramonyam}
 \affiliation{%
  \institution{Stanford University}
   \country{USA}
 }
 \email{harihars@stanford.edu}





%%
%% By default, the full list of authors will be used in the page
%% headers. Often, this list is too long, and will overlap
%% other information printed in the page headers. This command allows
%% the author to define a more concise list
%% of authors' names for this purpose.
\renewcommand{\shortauthors}{Siddiqui, et al.}

%%
%% The abstract is a short summary of the work to be presented in the
%% article.
\begin{abstract}
  Good writing is a dynamic process of knowledge transformation, where writers refine and evolve ideas through planning, translating, and reviewing. Generative AI-powered writing tools can enhance this process but may also disrupt the natural flow of writing, such as when using LLMs for complex tasks like restructuring content across different sections or creating smooth transitions. We introduce \system, a \textit{layered interface paradigm} designed to minimize these disruptions by aligning writing intents with LLM capabilities to support diverse content development and rhetorical strategies. By bridging envisioning, semantic, and articulatory distances, \systems interactions allow writers to leverage LLMs for various content development tasks (\textit{scripting}) and experiment with diverse organization strategies while tailoring their writing for different audiences (\textit{shifting}). This approach preserves creative control while encouraging divergent and iterative writing. Our evaluation shows that \system enables writers to creatively and efficiently incorporate LLMs while preserving a natural flow of composition.
\end{abstract}

%%
%% The code below is generated by the tool at http://dl.acm.org/ccs.cfm.
%% Please copy and paste the code instead of the example below.
%%
% \begin{CCSXML}
% <ccs2012>
%  <concept>
%   <concept_id>10010520.10010553.10010562</concept_id>
%   <concept_desc>Computer systems organization~Embedded systems</concept_desc>
%   <concept_significance>500</concept_significance>
%  </concept>
%  <concept>
%   <concept_id>10010520.10010575.10010755</concept_id>
%   <concept_desc>Computer systems organization~Redundancy</concept_desc>
%   <concept_significance>300</concept_significance>
%  </concept>
%  <concept>
%   <concept_id>10010520.10010553.10010554</concept_id>
%   <concept_desc>Computer systems organization~Robotics</concept_desc>
%   <concept_significance>100</concept_significance>
%  </concept>
%  <concept>
%   <concept_id>10003033.10003083.10003095</concept_id>
%   <concept_desc>Networks~Network reliability</concept_desc>
%   <concept_significance>100</concept_significance>
%  </concept>
% </ccs2012>
% \end{CCSXML}

% \ccsdesc[500]{Computer systems organization~Embedded systems}
% \ccsdesc[300]{Computer systems organization~Redundancy}
% \ccsdesc{Computer systems organization~Robotics}
% \ccsdesc[100]{Networks~Network reliability}

\begin{CCSXML}
<ccs2012>
   <concept>
       <concept_id>10003120.10003123.10011760</concept_id>
       <concept_desc>Human-centered computing~Systems and tools for interaction design</concept_desc>
       <concept_significance>500</concept_significance>
       </concept>
   <concept>
       <concept_id>10003120.10003123.10011759</concept_id>
       <concept_desc>Human-centered computing~Empirical studies in interaction design</concept_desc>
       <concept_significance>500</concept_significance>
       </concept>
 </ccs2012>
\end{CCSXML}

\ccsdesc[500]{Human-centered computing~Systems and tools for interaction design}
\ccsdesc[500]{Human-centered computing~Empirical studies in interaction design}
%%
%% Keywords. The author(s) should pick words that accurately describe
%% the work being presented. Separate the keywords with commas.
% \keywords{datasets, neural networks, gaze detection, text tagging}
\keywords{Human-AI collaborative writing, large language models, writing assistants, creativity support}

%% A "teaser" image appears between the author and affiliation
%% information and the body of the document, and typically spans the
%% page.
\begin{teaserfigure}
  \includegraphics[width=\textwidth]{figures/teaser_figure_cr.png}
  \caption{\system is an AI-assisted writing interface that empowers writers to query LLMs using familiar visual design elements. (A) Scripting enables users to query several specialized Writer's Friends inline. (B) Shifting allows writers to reorganize layers and combine them to see what the final document looks like. }
  \Description{}
  \label{fig:teaser}
\end{teaserfigure}

% \received{20 February 2007}
% \received[revised]{12 March 2009}
% \received[accepted]{5 June 2009}

%%
%% This command processes the author and affiliation and title
%% information and builds the first part of the formatted document.
\maketitle


\section{Introduction}
\label{sec:intro}


\ps{Challenges of technology scaling}

The growing demand for computing performance has always been met by increasing the number of transistors per chip, which is only possible due to CMOS technology scaling.
However, as we keep pushing the boundaries of technology scaling, we encounter multiple challenges.
Firstly, whenever we transition to a more advanced technology node, the non-recurring cost due to physical design, verification, software, mask sets, and prototyping almost doubles \cite{cost-tech-node}.
As a result, designing a chip in an advanced technology node is only economically viable if the chip is manufactured in vast quantities.
Secondly, many chip components such as I/O drivers, analog circuits, or \gls{srams} have reached their scaling limit.
This means that we cannot shrink these components further, even if we use a more advanced technology with a smaller feature size.
Thirdly, advanced technology nodes suffer from high defect rates, diminishing the yield and inflating the recurring cost.
To tackle these challenges, new chip-design paradigms have been developed.

\ps{Why 2.5D integration?}

One of these new paradigms is 2.5D integration, where multiple silicon dies called chiplets are integrated into the same package.
Once designed, a single chiplet can be reused in multiple 2.5D stacked chips, which increases the ratio of production volume to non-recurring cost.
Another advantage is that multiple chiplets - fabricated in different technologies - can be integrated into the same package.
This means that only components that can take full advantage of technology scaling are built in bleeding-edge technologies.
Components that have reached their scaling limit are fabricated in more mature and hence less costly technology nodes.
Furthermore, chiplets are smaller than monolithic chips.
Therefore, manufacturing chiplets results in less silicon area loss due to fabrication defects and hence a higher yield.
Due to these economic advantages, chip vendors such as AMD \cite{amd-chiplet} and NVIDIA \cite{chiplet-book} have adopted the 2.5D integration paradigm.  

\ps{Challenges of 2.5D integration}

An important challenge when designing 2.5D stacked chips is the construction of a low-latency and high-throughput \gls{ici}. 
To build an \gls{ici}, we connect different chiplets using \gls{d2d} links.
These links are fabricated in an organic package substrate, silicon bridge, or silicon interposer, and they are connected to the chiplets using \gls{c4} bumps or microbumps.
The number of bumps per chiplet is limited, and so is the bandwidth of \gls{d2d} links.
In addition to having lower bandwidth than links in monolithic chips, \gls{d2d} links also have higher latency.
This latency is caused by wire delay and by \gls{phys} that are necessary in both the sending and the receiving chiplet.
\gls{phys} are needed to convert between protocols, voltage levels, and frequencies, which are usually different between on-chiplet links and \gls{d2d} links.
Due to these limitations, the \gls{ici} can quickly become a bottleneck.

\ps{How we solve these challenges differently than the related work does.}

Existing approaches to maximize the performance of the \gls{ici} either optimize the placement of chiplets (with potentially heterogeneous shapes) for a predetermined \gls{ici} topology 
\cite{ho,liu,seemuth,eris,osmolovskyi,tap25d,chiou}, select one topology out of a set of candidates \cite{coskun-1, coskun-2}, or they optimize the \gls{ici} topology for a 2D grid of homogeneously shaped chiplets on an active interposer \cite{butterdonut, cluscross, kite}.
To the best of our knowledge, there is no prior work on \gls{ici} topologies for chips with heterogeneously shaped chiplets or with passive silicon interposers or silicon bridges.
To fill this gap, we propose \name, a novel optimization methodology to jointly optimize the chiplet placement and \gls{ici} topology of such architectures.
\ifnb
\else
\newpage
\fi

\ps{Details on \name~and the key idea}

The key idea is as follows: 
We optimize the chiplet placement without a predetermined topology.
For each placement generated by an optimization algorithm, we infer a placement-based \gls{ici} topology by connecting chiplets that are in close proximity in that specific placement.
We then compute the latency and throughput of this combination of placement and topology for different traffic types.
These latencies and throughputs together with the total chip area are used to compute a user-defined quality-score of the placement, which is returned to the optimization algorithm.
Based on this quality score, the algorithm can further optimize the placement.
By following this iterative process, we jointly optimize the chiplet placement and the \gls{ici} topology.

\ps{Short evaluation-summary}

We provide our open-source framework implementing the proposed placement and topology co-optimization methodology, which we evaluate using both synthetic traffic and traffic traces.
A 2D grid of chiplets with a mesh topology is used as a baseline since many proposals for 2.5D stacked chips \cite{dataflow_accel_dnn, cifher, simba, hecaton, dojo} use such an architecture.
We reduce the latency of synthetic L1-to-L2 and L2-to-memory traffic, the two most important traffic types for cache coherency traffic, by up to 28\% and 62\% respectively.
For real traffic traces, we reduce the average packet latency for almost all traces and architectures considered (reduced by an 8\% or 18\% on average depending on the configuration of \gls{phys} within a chiplet).

\section{Related Work}

\subsection{Penetration Depth Computation}

The computation of penetration depth often utilizes the Minkowski sum, a well-regarded algorithm documented in Dobkin et al.'s work~\cite{dobkin1993computing}.
This method shows high efficacy for convex shapes, where the simplicity of the objects allows for accurate and computationally efficient penetration depth calculations~\cite{dobkin1993computing,varadhan2004accurate,hachenberger2009exact}.
However, applying this algorithm to concave shapes significantly increases computational complexity.  
As a result, research has focused on developing methods to approximate penetration depth more efficiently for these shapes~\cite{cameron1997enhancing,bergen1999fast,lien2010simple,je2012polydepth}.  

Beyond the Minkowski sum, other methods have been explored, including techniques such as utilizing distance fields or the Hausdorff distance for penetration depth calculations~\cite{fisher2001fast,sud2006fast,SIG09HIST}.

Tang et al.\cite{SIG09HIST} devised an efficient algorithm for calculating the Hausdorff distance between two objects within a given error bound.
They also demonstrated that the proposed algorithm can accelerate penetration depth computation by focusing on the Hausdorff distance in overlapping regions of objects.
Building upon Tang et al.'s method, Zheng et al.\cite{zheng2022economic} improved performance using a BVH-based framework with a four-point strategy.
This method has achieved a performance improvement of up to 20 times compared to Tang et al.'s technique~\cite{SIG09HIST}.
\revision{A common feature of these works, known as the culling-based method, is computing bounds for the Hausdorff distance and reducing the search space.}

\revision{Although culling-based methods have demonstrated significant performance gains, they face challenges in leveraging parallel hardware.  
Updating and sharing bounds require synchronization, which is not well-suited for massively parallel processing architectures such as GPUs.}

\revision{In this work, we propose a GPU-based penetration depth algorithm that specifically accelerates two key processes using RT core technology:  
(1) detecting the overlapping volume and (2) calculating the Hausdorff distance.  
To highlight the effectiveness of our approach, we also implemented a CPU-based penetration depth algorithm based on Tang et al.~\cite{SIG09HIST} and Zheng et al.~\cite{zheng2022economic} for performance comparison.}

%In this work, we propose a GPU-based penetration depth algorithm, specifically accelerating two key processes with RT core technology:  
%first, detecting the overlapping volume; and second, calculating the Hausdorff distance.  
%To highlight our method's effectiveness, we also implemented a CPU-based penetration depth algorithm based on Tang et al.~\cite{SIG09HIST} and Zheng et al.~\cite{zheng2022economic} for performance comparison.  

%utilize a Hausdorff distance-based method for penetration depth calculation, accelerating two key processes with RT core technology: 

%A notable development in this area is the work of Tang et al., who devised algorithms for the rapid calculation of the Hausdorff distance between two objects~\cite{SIG09HIST}.
%Their approach is geared towards efficient penetration depth calculation by focusing on the Hausdorff distance in overlapping object regions.


%One of the algorithms for calculating penetration depth is the Minkowski sum.\cite{dobkin1993computing} The Minkowski sum is useful to compute penetration depth between two convex objects because they have a simple shape so the Minkowski sum can calculate accurate penetration depth with low computational complexity~\cite{dobkin1993computing,varadhan2004accurate,hachenberger2009exact}.
%However, applying the Minkowski sum in cases involving concave objects is challenging due to higher computational complexity. As a result, prior research has focused on quickly computing an approximate penetration depth in these scenarios~\cite{cameron1997enhancing,bergen1999fast,lien2010simple,je2012polydepth}.

%Instead of the Minkowski sum method, there have also been attempts to calculate the penetration depth based on the distance field or the vertices that make up the objects~\cite{fisher2001fast,sud2006fast,SIG09HIST}. Tang et al.~\cite{SIG09HIST} proposed the algorithms that compute the Hausdorff distance between two objects quickly and showed that can be computed penetration depth to fast by calculating the Hausdorff distance for the overlapping area of two objects.

%In this paper, the proposed method is based on Tang's methods~\cite{SIG09HIST}, and then partially divided into steps detecting overlapping volume step and the Hausdorff distance step. These two steps accelerated with RT core.

\subsection{Ray-Tracing Core-Based Acceleration}

\revision{Recent advancements in GPU technology have led to the integration of dedicated ray-tracing cores (RT cores), enabling hardware-accelerated ray tracing.
These cores optimize intersection checks between rays and objects, allowing for efficient ray-bounding box and ray-triangle intersection tests.
To utilize RT cores, various frameworks such as DXR, OptiX~\cite{parker2010optix}, and Vulkan have been developed.
RT cores primarily accelerate ray intersection tasks by efficiently traversing acceleration hierarchies.}

%The Ray-Tracing Core (RT-core) is NVIDIA’s specialized hardware for accelerating ray tracing.
%Integrated into RTX GPUs like the GeForce RTX series

%\revision{Notably, OptiX~\cite{parker2010optix} is an NVIDIA-supported SDK.
%The ray-tracing core primarily facilitates two tasks: building an acceleration hierarchy and executing ray intersection tasks with traversal.}
%OptiX operates by launching a CUDA kernel and invoking a ray generation ($ray_{gen}$) shader.
%Each CUDA core thread makes requests to the ray-tracing core, which then executes appropriate shaders like intersection ($IS$), miss($miss_{hit}$), closest hit($closest$), and any hit($any_{hit}$).
%Consequently, OptiX enables access to the results of ray-primitive intersection tests.

While the core purpose of ray-tracing cores is to expedite ray tracing, recent studies have explored their application beyond this traditional scope~\cite{wald2019rtx,zhu2022rtnn,thoman2022multi,nagarajan2023rt,meneses2023accelerating,morrical2023attribute}.
Wald et al.~\cite{wald2019rtx} addressed the problem of locating points within tetrahedra using ray-tracing cores.
Zhu et al.~\cite{zhu2022rtnn} introduced a K-Nearest Neighbor (K-NN) algorithm utilizing ray-tracing cores, achieving performance improvements of 2.2 to 65.0 times compared to previous GPU-based nearest neighbor search algorithms.
Thoman et al.~\cite{thoman2022multi} employed RT cores for Room Impulse Response (RIR) simulation.
Nagarajan et al.~\cite{nagarajan2023rt} implemented RT core-based DBSCAN clustering, reporting up to 4 times higher performance enhancement.
Meneses et al.~\cite{meneses2023accelerating} proposed RT core-based Range Minimum Query (RMQ) algorithms, yielding performance up to 2.3 times faster than existing RMQ methods.

\revision{
For collision detection between objects, one of the fundamental proximity queries, researchers have explored ray-tracing approaches even before the introduction of RT-core technology.
Hermann et al.\cite{hermann2008ray} proposed ray-tracing-based collision detection methods for deformable bodies.
Youngjun et al.\cite{kim2010mesh} applied Hermann's idea to medical simulation.
Lehericey et al.\cite{lehericey2015gpu} introduced GPU ray-traced collision detection algorithms for cloth simulation.
Recently, these approaches have been extended to utilize RT cores, as demonstrated by Sui et al.\cite{sui2024hardware}, who proposed discrete and continuous collision detection algorithms using ray-tracing cores.
Unlike these works, which focus on determining when and where collisions occur, our work focuses on calculating penetration depth.
}

In line with these advancements, this study uniquely applies RT-core technology to compute penetration depth, diverging from traditional ray-tracing applications and thereby contributing a novel approach to this field.

%\subsection{Collision detection with Ray-tracing}

%\YW{There have been attempts to apply the ray tracing approaches for collision detection~\cite{hermann2008ray, kim2010mesh, lehericey2015gpu}. Hermann et al~\cite{hermann2008ray} proposed ray tracing collision detection methods for deformable bodies. Youngjun et al~\cite{kim2010mesh} apply Hermann's idea for Medical simulation. Lehericey et al~\cite{lehericey2015gpu} introduced GPU ray-traced collision detection algorithms for cloth simulation.
%However, these methods proposed deformable objects, not solid- or discrete- objects, and there is no report about the result using ray tracing core yet. Therefore, our research implements the penetration depth algorithm with ray tracing methods and reports the benefit of ray tracing core.}

%\YW{Sui et al~\cite{sui2024hardware} proposed the method for discrete and continuous collision detection with ray tracing core. They generate the ray candidate as much as the edge of the source mesh and investigate the intersections to solve discrete collision detection. And also, to solve continuous collision detection, they build sphere-swept volumes with OptiX B-Spline curves using continuous trajectory points that are pre-computed and trace the ray samely. However, their implementation only considers non-penetrating collision, and because of that reason, there need for other approaches to compute penetration cases.}

%\YW{To address this issue, our approacthe has propose the methods to find penetration surface with RT core (that called RT-PPE). Not only that, our methods report the penetration depth as computing the Hausdorff distance between the penetration surface.}

%Recently, modern GPU embedded ray tracing core for hardware accelerated ray tracing.

%To access the ray tracing core, we can use DXR, OptiX~\cite{parker2010optix}, and Vulkan.
%Above all, OptiX~\cite{parker2010optix} is NVIDIA NVIDIA-supported SDK. The ray tracing core actually works about two tasks. One is a built acceleration hierarchy, and another is ray intersection task with traversal. Therefore OptiX launches one CUDA kernel and called $ray\_gen$ shader. Each CUDA core thread requests to ray tracing core, and then ray tracing core executes a suitable shader such as $IS$, $miss\_hit$, $closest$, $any\_hit$ shader.
%Finally, we can access ray-primitive intersection test results using OptiX shader.

%While the ray tracing core is designed for accelerating ray tracing, recent research tried using the ray tracing core for other purposes~\cite{wald2019rtx,zhu2022rtnn,thoman2022multi,nagarajan2023rt,meneses2023accelerating,morrical2023attribute}.
%%Beyond ray tracing
%Wald et al~\cite{wald2019rtx} solved the point in location of tetrahedron problem using ray tracing cores.
%%RTNN
%Zhu et al~\cite{zhu2022rtnn} proposed K-NN(K-Nearest Neighbor) algorithms using ray tracing cores. They achieved a performance of 2.2-65.0 times faster than prior GPU-based nearest neighbor search algorithms.
%%RIR Simulation
%Thoman et al~\cite{thoman2022multi} utilized the RT core to RIR(Room impulse response) simulation,
%%RT-DBSCAN
%Nagarajan et al~\cite{nagarajan2023rt} implemented DBSCAN clustering with RT core and achieved performance up to 4x times.
%%RTX-RMQ
%Meneses et al~\cite{meneses2023accelerating} proposed RT core-based RMQ(Range minimum query) algorithms, and they got performance up to 2.3x than state-of-the-art RMQ algorithms.

%%
%Similar to prior research, this study is distinguished by utilizing RT-core for computing penetration depth, as opposed to conventional ray tracing problems.



\section{\change{\system: A Layered Interface For LLM Co-Writing} }
\begin{figure*}[t!]
    \centering
    \includegraphics[width=\textwidth]{figures/workspace_1.pdf}
    \caption{Workspace view of \system: (a) Zoomable,  scrollable writing workspace, (b) workspace level operations, (c) a text layer with layer toolbar on top, and (d) the compiled document viewer.}
    \label{fig:canvas}
\end{figure*}


\system is designed to support flexible generative writing workflows by interleaving content development and structural organization. In designing \system, our goal was to lower the \textit{envisioning gaps} around AI capabilities, instructions, and intentionality as well as to reduce the \textit{semantic} and \textit{articulatory} distance when interfacing with Generative AI. As shown in figure~\ref{fig:canvas}, the primary interface consists of an infinite zoomable \textit{workspace} where the writer can create and organize individual \textit{layers}. A \textit{layer} is a discrete, modular content unit within the writing workspace designed to encapsulate specific elements of the writing or organizational process. 
% Layers serve as flexible sandboxes where tasks such as tone adjustments, elaboration, or structural reorganization can occur in isolation, fostering iterative and exploratory workflows.

\change{Here, we use the ``layered'' metaphor because it aligns with how individuals think, by separating, refining, and integrating complex ideas. Unlike layers in visual tools such as Photoshop or Illustrator, which focus on the compositional blending of visual elements, \systems layers emphasize cognitive composition, representing distinct aspects of the writing or organizational process—such as brainstorming, tone adjustments, and meta-information. This metaphor naturally accommodates iterative workflows, enabling users to ``tear apart,'' ``combine,'' or ``stack'' layers for experimentation, similar to rearranging layers on a desk. Each layer acts as a \textit{sandbox} for specific tasks, preserving creative control and enabling non-linear exploration. By reimagining the familiar concept of layers, \system bridges user expectations with the flexibility and modularity required for writing tasks, where the process of exploration and refinement is as critical as the outcome.}



\system supports three types of layers, which include (1) a templated metadata layer for specifying the overall writing goals, (2) content layers for authoring the main text, and (3) a scratchpad for gathering relevant information to support the writing process. \system also supports `tags' for labeling individual layers and collections of layers (e.g., a stack of layers). Further, the interface provides contextual toolbars at the layer and workspace level. The layer-level toolbar consists of standard text formatting operations and specific AI functionality. The workspace-level toolbar supports adding layers and intelligent structuring operations. Further, \system implements an extensible collection of AI ``friends'' that provide targeted writing support for ideation, content organization, research, adjusting the tone, etc. To better understand how \system supports writing, let us follow Sashi, a freelance writer whose goal is to write an opinion piece about ``LLMs and Creative Ownership.'' 

\begin{figure*}[htbp!]
    \centering
    \begin{subfigure}[b]{0.28\textwidth}
        \centering
        \includegraphics[width=\textwidth]{figures/meta_layer.png}
        \caption{Specifying context of writing in meta layer}
    \end{subfigure}
    \hfill
    \begin{subfigure}[b]{0.28\textwidth}
        \centering
        \includegraphics[width=\textwidth]{figures/intro_layer.png}
        \caption{Starting free writing in Introduction Layer}
    \end{subfigure}
    \hfill
    \begin{subfigure}[b]{0.28\textwidth}
        \centering
        \includegraphics[width=\textwidth]{figures/slash_layer.png}
        \caption{Calling Detail Danny from the `/' trigger menu}
    \end{subfigure}
    
    \vspace{1em}
    
    \begin{subfigure}[b]{0.28\textwidth}
        \centering
        \includegraphics[width=\textwidth]{figures/danny_layer.png}
        \caption{Detail Danny generating two variations for user-specified prompt for detail/elaboration}
    \end{subfigure}
    \hfill
    \begin{subfigure}[b]{0.28\textwidth}
        \centering
        \includegraphics[width=\textwidth]{figures/ramesh_layer.png}
        \caption{Research friend as a scratchpad for researching topics specified in meta layer}
    \end{subfigure}
    \hfill
    \begin{subfigure}[b]{0.28\textwidth}
        \centering
        \includegraphics[width=\textwidth]{figures/child_layer.png}
        \caption{Adding links for child layers that the current layer will extract content from}
    \end{subfigure}
    
    \vspace{1em}
    
    \begin{subfigure}[b]{0.28\textwidth}
        \centering
        \includegraphics[width=\textwidth]{figures/import_layer.png}
        \caption{Tunneling into content from another layer into current layer}
    \end{subfigure}
    \hfill
    \begin{subfigure}[b]{0.28\textwidth}
        \centering
        \includegraphics[width=\textwidth]{figures/peak_layer.png}
        \caption{Peaking at (possible) future content using the bottom right}
    \end{subfigure}
    \hfill
    \begin{subfigure}[b]{0.28\textwidth}
        \centering
        \includegraphics[width=\textwidth]{figures/felix_layer.png}
        \caption{Invoking Feedback Felix for paragraph level feedback}
    \end{subfigure}
    \caption{Key Interactions supported by \system.}
    \label{fig:nine_figures}
\end{figure*}

\change{\subsection{Use Scenario}}
\subsubsection{Goal Setting}
To begin with, Sashi opens \system on his web browser and creates a new project. \system displays the workspace, adding a blank metadata layer~\textcolor{skyblue}{\faFile} to it (Figure~\ref{fig:nine_figures}a). The \textbf{metadata layer} contains a set of guiding questions and response text fields for writing goal, audience, topic context, and intent for writing. The metadata layer optionally allows Sashi to upload relevant documents to the writing task. In this case, Sashi enters the goal as \textit{``to write an opinion piece that explores the complex interplay between Large Language Models (LLMs), creative processes, and copyright law,''} the target audience as ``technology creators and potentially legal professionals,''  and intent as ``arguing for a reevaluation of current copyright frameworks to address the challenges posed by AI-generated content.'' Here, Sashi can also upload relevant documents, such as the news article about the AI-generated photo that won the art prize at the Colorado State Fair\footnote{\url{https://www.nytimes.com/2022/09/02/technology/ai-artificial-intelligence-artists.html}}, the New York Times Lawsuit against OpenAI\footnote{\url{https://nytco-assets.nytimes.com/2023/12/NYT_Complaint_Dec2023.pdf}}, and any other relevant legal documents. Sashi can always revisit this layer throughout his writing to provide additional context. At this point, Sashi clicks on the ``Begin Writing'' button at the bottom of the metadata layer. This adds a new layer to the workspace, which Sashi names `Introduction'  \pageiconwithnumber[lightestgray]{black}{I}.

\subsubsection{Content Development}
On this layer \pageiconwithnumber[lightestgray]{black}{I}, Sashi begins by engaging in \textit{free-writing}, exploring various real-world instances of LLM creativity he has encountered to motivate the article (Figure~\ref{fig:nine_figures}b). Once he has written several examples, he realizes that he must also include a brief description of LLMs and their creative abilities. Instead of manually describing LLMs, Sashi opts to leverage the content development features of \system. Inspired by `personas' for writing feedback~\cite{benharrak2024writer}, we have implemented a set of \textbf{writer's friends} that embody various writing skills, including \coloredcircle{IdeaIvy} Idea Ivy for content ideation, \coloredcircle{DetailDanny} Detail Danny for elaboration, \coloredcircle{StructureSam} Structure Sam for content structuring, \coloredcircle{ToneTara} Tone Tara for stylistic suggestions, \coloredcircle{FeedbackFelix} Feedback Felix for feedback on content, and \coloredcircle{AudienceAli} Audience Ali for audience-specific feedback. 


In this case, Sashi selects the Detail Danny friend by typing a forward slash `/' at the beginning of a new paragraph. This action opens a dropdown menu displaying a list of all friends, from which Sashi chooses \coloredcircle{DetailDanny} Detail Danny (Figure~\ref{fig:nine_figures}c). \system inserts an \customdashedbordertext{inline prompt box}{DetailDanny}{DetailDanny} where he writes \textit{``Write a concise description of LLMs and their creative abilities, explain how they function for the creative tasks listed above''} and presses the enter key. In response, \system generates a paragraph of text about LLMs and creativity. The \customsolidbordertext{generated text}{DetailDanny}{DetailDanny} is highlighted with a background color corresponding to the friend (Figure~\ref{fig:nine_figures}d). Sashi can accept this content by pressing enter at the end of the content or delete it and regenerate new content. Sashi accepts the suggestion and continues editing. Sashi has a high-level vision about a few topics his opinion essay should cover. He creates new layers for `Traditional Copyright Law - \pageiconwithnumber[lightestgray]{black}{L},' `Authorship -- \pageiconwithnumber[lightestgray]{black}{A},' and `Ethical Considerations -- \pageiconwithnumber[lightestgray]{black}{E}' and, as before, writes his thoughts. When writing about copyright law, Sashi realizes he needs to understand copyright principles better. He creates a new \textbf{scratchpad} layer ~\textcolor{legalyellow}{\faFile}, which is supported by Research Ramesh, where he can ask questions about the purpose and intent of copyright protection or how copyright has adapted to previous technological changes to help with his main content writing (Figure~\ref{fig:nine_figures}e). 

In the course of writing, Sashi creates a new layer for Stakeholder Perspectives -- \pageiconwithnumber[lightestgray]{black}{S} and brainstorms using  \coloredcircle{IdeaIvy} Idea Ivy about whose perspectives to include, and in response, Ivy suggests talking about \customsolidbordertext{Content Creators}{IdeaIvy}{IdeaIvy}, \customsolidbordertext{AI Companies}{IdeaIvy}{IdeaIvy}, \customsolidbordertext{Consumers}{IdeaIvy}{IdeaIvy}, and \customsolidbordertext{Legal Experts.}{IdeaIvy}{IdeaIvy}. By individually selecting each stakeholder, Sashi can create \textbf{sub-layers} to flesh out the stakeholder perspective (Figure~\ref{fig:nine_figures}f). There is a persistent link between the text in the Stakeholder layer \pageiconwithnumber[lightestgray]{black}{S} and the sub-layers [\pageiconwithnumber[lightestgray]{black}{$S^{CC}$}, \pageiconwithnumber[lightestgray]{black}{$S^{AI}$}, \pageiconwithnumber[lightestgray]{black}{$S^C$}, \pageiconwithnumber[lightestgray]{black}{$S^{LE}$}]. Sashi can also \textbf{split} the content into two layers and develop the content separately.  

Idea Ivy can also be used to support more complex writing procedures across layers, such as how some specific text about the economic impact in the Ethical Considerations layer \pageiconwithnumber[lightestgray]{black}{E} might relate to the current text in the content-creator layer \pageiconwithnumber[lightestgray]{black}{$S^{CC}$}. To support such complex prompting using content across layers, \system supports a \textit{tunneling} function where the current layer split opens at the cursor position to reveal a desired layer selected from the dropdown (Figure~\ref{fig:nine_figures}g). From here, Sashi can choose the relevant text and import it to the current prompt he is composing. On any layer, if he is stuck on what to write, he can click on the bottom right corner of the layer to \textbf{`peek'} into what the Generative AI might write (Figure~\ref{fig:nine_figures}h). The peek function shows continuing text based on his current writing in a greyed-out format. For creating structure from unstructured text, he can invoke \coloredcircle{StructureSam} Structure Sam, who will create a new layer with a structured representation of the content, including headings and subheadings. He can invoke the \coloredcircle{AudienceAli} audience and \coloredcircle{FeedbackFelix}feedback friends on any layer using the toolbar option to receive comments inline, which he can toggle on or off (Figure~\ref{fig:nine_figures}i). He can call upon \coloredcircle{ToneTara} Tone Tara to generate a different stylistic variations of the current content rendered as separate layers from the original. 

Lastly, as he is writing, Sashi can \textbf{compare} two layers by bringing them close, such that the right edge of a layer is touching the left edge of the other layer \pageiconwithnumber[lightestgray]{black}{$S^{AI}$}\pageiconwithnumber[lightestgray]{black}{$S^{CC}$}. \system recognizes this as an intent to compare and provides a floating button, ``Compare the two layers?'' On clicking this, the button expands into a text prompt where Sashi can input specific instructions for comparison, such as ``how the stakeholder perspectives of content creators and AI companies align or conflict regarding LLM-generated content and copyright.'' The system uses color highlighting and inline annotations to indicate similarities and differences. To mitigate possible confusions, these annotations only persist while the layers remain in proximity. 

\subsubsection{Structure and Rhetoric}
Throughout this process, \system provided several features for Sashi to develop the structure for his final essay and tailor it to specific audiences. To organize the content, \system supports \textbf{combining} two layers into a single layer by bringing the second layer to the bottom of the first layer \twopageicon{lightestgray}{lightestgray} and optionally prompting for specific transitional text between layers. For instance, he can connect a layer on `Challenges posed by LLMs' \pageiconwithnumber[lightestgray]{black}{Ch} with the layer on `Copyright Law' \pageiconwithnumber[lightestgray]{black}{C} by generating transitional text from identifying the problems to analyzing the legal framework meant to address them. Alternately, Sashi can \textbf{tear} layers into parts and glue them together to reorganize the text, i.e. to have content inform structure. Sashi can fold layers he wishes to exclude, which show the text summary on the folded side of the layer. He can \colorbox{yellow}{Tag} individual layers or even clusters and stacks of layers in the workspace. When Sashi is ready to \textit{compose} his essay, he can \textbf{stack} the layers manually in an order [\pageiconwithnumber[lightestgray]{black}{I},\pageiconwithnumber[lightestgray]{black}{C},\pageiconwithnumber[lightestgray]{black}{E},\pageiconwithnumber[lightestgray]{black}{S}\ldots] and generate the final document by concatenating them or asking the LLM to generate the order based on an unordered stack. Here he can issue specific prompts such as generating an audience-specific version, editing for consistency across layers, or even adaptive summarization to meet some target length. The final document is generated with any edited text highlighted for Sashi to review. Sashi can click on the text in the final document and track back to the source layer. In this manner, \systems features allow Sashi to use generative AI for content development and rhetorical strategies, flexibly, fluidly, and iteratively. Table~\ref{fig:layered_action} summarizes the different scripting and shifting affordances of \system.



% \change{The Figures \ref{fig:layered_action} and \ref{fig:writers_friend} below show all the features of \system.}


\begin{figure*}[t!]
    \centering
    \includegraphics[width=0.8\textwidth]{figures/generation_table.png}
    \caption{\change{Content transformation through user prompts. The following 10 features allow writers to issue descriptive instructions for invoking specialized LLM assistance. The example column consists of real prompts issued by participants in the usability assessment.}}
    \label{fig:layered_action}
\end{figure*}








\section{User Studies Overview}
To evaluate the effectiveness and usability of \theDevice, we conducted two user studies with signers ranging from novice to native/ fluent in ASL proficiency. The first study focused on word-level recognition, assessing the system's ability to accurately recognize individual fingerspelled words across a diverse group of participants. The second study expanded on these findings by examining phrase-level recognition in real-time scenarios, providing insights into the system's performance in more natural, context-rich environments. These studies aimed to validate SpellRing's performance across different levels of signing experience, explore the impact of signing speed and habits on recognition accuracy, and investigate how users adapt to the system in real-time use. The user study was approved by the Institutional Review Board (IRB) at the authors' institution. Participants were compensated \$40 per hour for their participation in the study.

\section{Example Writing Workflows with \system}

In this section, we demonstrate the flexibility of writing with \system through three distinct workflows: freewriting, document-based-question, and parallel topic development. Each example highlights how the layered interface and AI tools support fluid, iterative writing across different starting points and content structures.


\subsection{Freewriting to Argumentative Writing Workflow}
In this workflow, the writer begins with freewriting, allowing ideas to flow without worrying about organization or structure. Once they have written their thoughts, they determine that an argument style structure can be effective to organize the text, consisting of the following components: Claim, Grounds, Warrant, Backing, Qualifier, and Rebuttal~\cite{toulmin2003uses,marshall1989representing}. The writer adds the \textit{argument template}  to \systems workspace and drops the layer with the freewriting text onto the template. Using the template, \system takes the unstructured text and generates six layers, one for each component. Next, the writer can continue to flesh out each layer using Writer's Friends such as ``Idea Ivy'' and later refine, reorganize, and link ideas across layers, transforming initial thoughts into a structured argumentative essay.

\begin{figure*}[t]
    \centering
    \includegraphics[width=\textwidth, height=0.3\textheight, keepaspectratio]{figures/FreeWriting.pdf}
    \caption{Free Writing Example with \system \change{ \textbf{(A)} Writer calls \textit{Template} for organizing their writing into milestones for argumentative writing. \textbf{(B)} They select the ``Ground'' layer for further development. \textbf{(C)} They invoke \textit{Idea Ivy} to brainstorm what to write next and then use \textit{Structure Sam} to organize that into subheadings and paragraphs.}}
    \label{fig:freewriting}
\end{figure*}

\begin{figure*}[t]
    \centering
    \includegraphics[width=\textwidth, height=0.3\textheight, keepaspectratio]{figures/DBQ.pdf}
    \caption{Document-Based Question Example with \system \change{ \textbf{(A)} The writer specifies the context of their writing and upload their assignment to the \textit{Meta Layer}. \textbf{(B)} They call \textit{Research Ramesh} to understand details of their assignment based on the context document. \textbf{(C)} They \textit{Tunnel} into another layer to extract some details. \textbf{(D)} They combine a stack of layers to generate \textbf{E}.}}
    \label{fig:freewriting}
\end{figure*}


\subsection{DBQ Writing Workflow for Students}
In this workflow, the student working on a Document-Based Question (DBQ) can upload primary source documents directly into the system. The metadata page allows the student to provide information about the assignment goals, guiding questions, and document context. The student can then create additional layers for organizing their arguments, categorizing evidence drawn from the primary sources. For example, they might have separate layers for economic, political, and social factors. LLM friends like ``Research Ramesh'' assist by analyzing the documents and suggesting relevant excerpts or summaries that will fit the argument. As the student develops their essay, cross-layer interactions allow them to pull evidence from the source layers into argument layers, ensuring that evidence is seamlessly linked back to the original documents. Using \systems stacking feature, the student can then combine the various layers into a cohesive essay. \system allows them to specify prompts for transitions in order to ensure that all arguments are well-supported by evidence. This workflow encourages students to engage deeply with primary sources, while providing an intuitive way to structure their argument, making the ballistic process of evidence-based writing more fluid and interactive.


\subsection{Layered Topic Development for Research Paper}

In this workflow, the writer develops different sections of the research paper in parallel, creating distinct layers for the literature review, methodology, and findings. \system allows the writer to work on these topics simultaneously, using cross-layer interactions to draw connections between sections. For instance, the writer can link methodology details to key studies in the literature review. Within the findings section, the writer can create separate layers for different findings or topics, allowing them to focus on each result individually. Once these layers are complete, the writer can combine them into a unified findings section, ensuring a cohesive and structured presentation of the results. Writer's friends like ``Research Ramesh'' assist by sourcing and summarizing relevant research, while ``Tone Tara'' and ``Feedback Felix'' help ensure stylistic and argumentative consistency across sections. Once the layers are fully developed, the document structuring features in \system help weave them together into a cohesive research paper.
\section{Evaluation}

To evaluate the effectiveness of our layered interface paradigm and generative authoring workflow, we conducted two studies: (1) a mixed-methods user experience evaluation with a Subjective Evidence-Based Ethnography (SEBE) protocol, and (2) a between-subjects deployment study on Prolific.

\subsection{User Experience Evaluation}

In the first study, our goal was to evaluate \system's impact on cognitive load, usability, and its influence on creative output. \change{ We wanted to understand the manners in which personas and layered affordances could bridge the envisioning gap~\cite{subramonyam2024bridging} for writers. }

\subsubsection{Participants}
We recruited participants via LinkedIn and Twitter, selecting individuals who engaged in creative writing at least a few times per month. The final sample included 12 participants (8 male, 4 female), aged 18 to 54. The majority (7 participants) were in the 18-24 age group, with 3 in the 25-34 range, and one each in the 35-44 and 45-54 groups. All participants were native English speakers. Educational backgrounds varied: 6 participants held bachelor's degrees, 5 had master's degrees, and 1 had some college experience. Participants reported diverse creative writing habits and experiences, with most having taken creative writing courses. Self-reported confidence in writing ranged from 2 to 5 on a 5-point scale ($\mu = 3.54$, $\sigma = 0.82$).

\subsubsection{Study Procedure}
Each study session lasted not more than 2 hours. At the start of the session, participants received a live demonstration of the application and were encouraged to ask questions. The participants were then asked to access the application on their own computer through the web url and complete a brief guided writing task about a tree, a cat and a dog to familiarize themselves with the interface. Next, participants engaged in the main writing task. They were given 40 minutes to write an 800-word essay connecting a recent real-life event to a film, using one of the 5 specified writing styles (First-Person Narrative, Journalistic Style, Dialogue Format, Letter or Diary Entry, or Screenplay Format). They were asked to identify parallels, extract insights, and reflect on personal growth. After completing the task, the participants were asked to fill out the NASA Task Load Index (TLX) for cognitive load, the Post-Study System Usability Questionnaire (PSSUQ) for usability assessment, and the Creative Support Index for evaluating creative support. The participants completed a survey gathering demographic information, writing habits, and experience with AI writing tools. 

Finally, participants engaged in a retrospective interview while reviewing the snippets of the video recordings of their session. We wanted to understand the (1) affordances, in terms of effects of the application on their writing process,  (2) their embodied competencies, skills, and knowledge they drew upon during the task, and (3) any perceived rules, norms, or expectations that guided their approach. We wrapped up the session by asking about overall impressions and additional insights. The entire writing session was recorded, capturing both the participants' audio recording and the participants' screen. Each participant received \$30 for their participant in the form of an Amazon gift card.


\subsubsection{Results}
\textbf{NASA TLX.} The survey measures six subscales: Mental Demand, Physical Demand, Temporal Demand, Performance, Effort, and Frustration. Analysis of the data (N = 12) revealed varying levels of perceived workload across these dimensions. Effort ($\mu = 8.42$, $\sigma = 6.50$) and Mental Demand ($\mu = 6.17$, $\sigma = 3.43$) were the highest-rated factors, indicating that participants found the task mentally taxing and requiring substantial effort. This aligns with expectations, as the task was designed to require deep thinking and was non-trivial. Temporal Demand also had a notable score ($\mu = 6.42$, $\sigma = 6.35$), suggesting that time pressure played a significant role. \change{The Performance score ($\mu = 5.25$, $\sigma = 3.33$) indicates participants generally performed well, as lower values on this scale represent better perceived performance. Similarly, low Frustration scores ($\mu = 3.58$, $\sigma = 3.96$) suggest minimal participant frustration.} The high standard deviations, particularly for Effort and Temporal Demand, reflect significant variability in study participants' experiences, likely due to differences in their task approach or expertise characteristics. Overall, these findings suggest that \system imposed a low cognitive load on participants.

% \begin{figure}
%   \centering
%   \includesvg[width=0.8\textwidth]{figures/plots/nasa_tlx_plot.svg}
%   \caption{Boxplot for NASA Task Load Index across the 6 dimensions}
%   \label{fig:nasaltx}
% \end{figure}

\begin{figure}[!t]
  \centering
  \includegraphics[width=0.9\columnwidth]{figures/NASATLX.pdf}
  \caption{NASA-TLX Workload response distribution across relevant dimensions}
  \label{fig:nasatlx}
\end{figure}


\begin{figure}[!t]
  \centering
  \includegraphics[width=0.9\columnwidth]{figures/PSSUQ.pdf}
  \caption{PSSUQ Response dimensions across System Usefulness, Information Quality, Interface Quality categories and Overall dimension}
  \label{fig:pssuq}
\end{figure}
\textbf{PSSUQ.} The Post-Study System Usability Questionnaire was used to assess the usability of the AI-assisted writing tool across three key dimensions: System Usefulness, Information Quality, and Interface Quality. Analysis of the data (N = 12) revealed generally positive usability scores, with all dimensions receiving mean ratings below the midpoint of the 7-point scale (where lower scores indicate better usability). The Overall PSSUQ score ($\mu = 2.18$, $\sigma = 1.29$) suggests that participants found the system to be reasonably usable. System Usefulness ($\mu = 1.99$, $\sigma = 1.01$) and Interface Quality ($\mu = 2.00$, $\sigma = 1.39$) were rated the highest, indicating that users found the tool functional and easy to interact with. The Information Quality dimension received a slightly lower but still positive rating ($\mu = 2.51$, $\sigma = 1.45$), suggesting room for improvement in the clarity and organization of information provided by the system.

% \begin{figure}
%   \centering
%   \includesvg[width=0.9\textwidth]{figures/plots/pssuq_plot.svg}
%   \caption{PSSUQ Response dimensions across System Usefulness, Information Quality, Interface Quality
% categories and Overall dimension}
%   \label{fig:pssuq}
% \end{figure}

Participants reported that they could not always tell if they had made a mistake in the interface, though this may not be a significant concern for a writing tool where many `mistakes' are subjective or stylistic choices. Within the System Usefulness dimension, items related to ease of use ($\mu = 1.67$, $\sigma = 0.89$) and efficiency ($\mu = 1.58$, $\sigma = 0.79$) received particularly positive ratings, suggesting that users found the tool intuitive and time-saving. However, the item related to system capabilities (System Usefulness: $\mu = 2.08$, $\sigma = 1.73$) showed higher variability, indicating diverse opinions on whether the system had all the expected functions and system capabilities, from the participants' perspectives.

In the Information Quality dimension, the item related to error message clarity ($\mu = 4.38$, $\sigma = 1.41$) received the lowest rating, highlighting a significant area for improvement. Interface Quality items were consistently rated positively, with low variability, suggesting a well-designed user interface. The relatively high standard deviations across most items indicate varied user experiences, possibly due to differences in expectations, prior experience with similar tools, or the specific writing tasks undertaken during the study.

\textbf{CSI.} The Creative Support Index was used to evaluate participants' experiences with \system, using a scale of 0 to 20, where lower scores indicate more positive outcomes. The analysis focused on five key dimensions: Exploration, Enjoyment, Results Worth Effort, Immersion, and Expressiveness. Data from 12 participants revealed generally positive experiences across these dimensions. Enjoyment was the highest-rated factor ($\mu = 2.67$, $\sigma = 3.50$), suggesting that users found the tool particularly enjoyable. This was closely followed by Results Worth Effort ($\mu = 3.83$, $\sigma = 3.81$), indicating that participants felt their input produced valuable results. Exploration ($\mu = 4.00$, $\sigma = 4.31$) and Expressiveness ($\mu = 4.33$, $\sigma = 3.17$) also received favorable ratings, implying that the tool effectively supported idea generation and self-expression.

\begin{figure}[!t]
  \centering
  \includegraphics[width=0.9\columnwidth]{figures/CSI.pdf}
  \caption{Creative Support Index (CSI) response distribution across relevant dimensions}
  \label{fig:csi}
\end{figure}

% \begin{figure*}[t!]
%   \centering
%   \includegraphics[width=0.4\textwidth]{figures/CSI.pdf}
%   \caption{Creative Support Index response distribution across relevant dimensions}
%   \label{fig:csi}
% \end{figure*}





\begin{figure*}[t!]
    \centering
    \includegraphics[width=0.9\textwidth]{figures/usage_tree.pdf}
    \caption{\change{Tree visualization of layer manipulation and LLM calls in \system. We show three diverse ways participants leveraged the interface affordances for completing the assigned task. The diverging gray lines depict `tear', the converging gray lines depict `combine', and meaning of other symbols is present in the legend}}
    \label{fig:participant-journey}
\end{figure*}

Interestingly, Immersion received the lowest rating ($\mu = 7.50$, $\sigma = 5.79$), though it still fell on the positive side of the scale. This suggests that while users found the tool engaging, there may be opportunities to enhance its ability to create a more immersive experience. The high standard deviations across all dimensions, particularly for Immersion and Exploration, reflect significant variability in user experiences, possibly due to individual differences in writing styles. The Collaboration dimension was excluded from our findings, as participants rated its weight factor as 0. 

\textbf{Spatial organization of content.} \system provides users with the ability to spatially organize their layers and associated content, a feature that was highly valued by participants for improving their focus and workflow. P9 praised the interface's immersive quality, stating: \textit{``One nice thing is that the fact that it's all in one big interface makes it less immersion-breaking than, say, opening a bunch of Google Docs tabs, where you have to make more major context switches and get distracted by other tabs.''} P9 further emphasized the advantages of a dedicated workspace: \textit{``Or you break out of the immersion of writing and being in the zone by going more into your desktop environment, where you're reminded of your work, or end up seeing social media. So it's nice to have a dedicated, isolated workspace.''}  This sentiment was shared by other participants who recognized the organizational benefits of \system over traditional text editors. P4, for instance, pointed out the limitations of conventional file systems: \textit{``The structure and organizational possibilities of this kind of thing would be huge because you end up with folder upon folder upon folder.''} P4's comment highlights \system's potential to address the organizational challenges typically encountered with standard text editors, offering a more intuitive and flexible way to manage content throughout the writing process. 

The system's spatial design supported expressive content management strategies\footnote{We thank psycholinguist George Miller, author Roy Pea's postdoctoral mentor at Rockefeller University in the 1970's for his seminal insights half a century ago in foregrounding 'the human tendency to locate information spatially', which we expressly leverage in the design of \system~\cite{millerpsychology}}. P9 emphasized how this shaped their writing process: \textit{``Being able to fold or bin my writing without permanently removing them made me much more willing to experiment. In single-page interfaces, I often feel pressured to be certain about content placement before typing.''} Users demonstrated remarkable spatial awareness of their content, organizing it in ways that enhanced accessibility. P11 noted: \textit{``Being able to [tunnel] into content from all over and get contextually-informed responses from the friends made everything feel so accessible.''} Through reduced cognitive load, users focused more on writing and experimentation rather than on content management.

\textbf{Paper metaphor for writing.} A central metaphor that \system aims to embody is the ability to manipulate and move layers much like rearranging sheets of paper on a desk. P11 captured this sentiment, stating, \textit{``It's a much more visual, like a desk with pieces of paper all over it,''} highlighting the intuitive, tactile nature of organizing content in the workspace. This visual and spatial approach offers users a more flexible and natural way to manage their writing, akin to physically handling documents in a traditional environment. It also opens up future promising research directions to develop and study a gesture- and voice-based rendition of the \system functionalities. 

\textbf{Flexibility in testing rhetorical strategies.} The affordances granted by \system, including the ability to tear, split, combine, stack, and fold layers, were clearly evident in how participants utilized the tool. These features not only supported writing tasks but also aligned with users' conceptual models of the writing process. P1 drew a parallel between the system's structure and traditional writing approaches: \textit{``Layers are like, if you're writing a paper, you need an outline and goal. Intuitively, it's like a tree which is the outline, and you work on each part of the node. In this case, you start with the intro, and you spend time collaborating with Idea Ivy. It was natural to break down [the writing process].''} This natural breakdown of the writing process was further enhanced by the system's manipulable interface, as P4 highlighted: \textit{``I really liked it. Being able to push something over there until later, and then bring it back and like smoosh it all together. That was really nice; that I liked that a lot.''} 

\change{The interface's bottom-up approach to LLM integration particularly enhanced exploratory writing. P5, whose usage journey is visualized in Fig. ~\ref{fig:participant-journey} C, observed: \textit{``Clicking a layer, moving it around and expected something to happen is so intuitive. The interface interactions felt natural and expected - exactly what I envisioned would happen.''} Participants often combined different features creatively, as P5 described: \textit{``I think order to exposition is really important. Tearing and recombining in different orders helped me rapidly see what narrative flow made sense''} P5 also wrote initially in first person, utilizing Tone Tara to shift to third person narration, demonstrating the system's flexibility in supporting various narrative styles.}

\textbf{Collaboration with Writer's Friends:} Participants demonstrated diverse and interesting approaches to leveraging the capability of the Writer's Friends in \system. These distinct personas, each representing different writing assistance features, effectively bridged the gulf of envisioning~\cite{subramonyam2024bridging}. P8 expressed a particular fondness for one such friend: \textit{``I like Danny. My Danny's a good guy.''} Similarly, P11 highlighted the value of constructive criticism: \textit{``I really like the friends, especially the feedback. I liked getting negative feedback.''} \change{Through writing in the interface and receiving feedback from the ``friends'', users frequently refined their meta layer. P2 explained: \textit{``Sometimes I felt my ideas were being misconstrued so updating the meta layer helped''} The iterative nature of getting feedback helped clarify writing goals and better understand the target readers. On the anthropomorphized LLM scaffolding, P8 remarked their indifference, ``I don't think it influenced my usage positively or negatively''} 

A thematic analysis of user-defined prompts revealed that participants accurately matched their requirements to the appropriate friend in the majority of cases. Out of the 57 times Detail Danny was used, the participants gave it prompts for detail and elaboration 77\% of the time. In the case of Idea Ivy, out of the 52 instances used across sessions 86\% of prompts were for ideation and brainstorming. In case of Tone Tara and  Structure Sam, 85\% (20) and 95\% (20) mapped to tone transformation and structuring respectively. Importantly, the Writer's Friends seemed to enhance rather than replace the creative process. P12 noted, \textit{``I honestly felt like I was still using this tool\ldots to come up with my own ideas. And I think that was good for me.''} This sentiment suggests that \system successfully balanced AI assistance with the preservation of user agency in the writing process. 

\textbf{Usability and User Interface.} The intuitive design and user-friendly interface of \system were frequently highlighted by participants, emphasizing the tool's ease of use and its ability to seamlessly integrate into the writing process. P11 expressed enthusiasm for the command interface, noting its natural feel: \textit{``I love the backslash and having the pull down menu and immediately being able to select. I liked that. I didn't even think about it\ldots I naturally did it.''} This comment showcases the effectiveness of the interface in reducing the cognitive load on users, allowing them to access features quickly and intuitively without disrupting their writing flow. The system's design also contributed to a positive emotional experience for users. P7 remarked on the sense of control and comfort provided by the interface: \textit{``I wasn't stressed at all, I felt completely in control.''}

In figure \change{~\ref{fig:participant-journey} we can see the journey through the interface taken by users with three distinct usage patterns. These visualizations give us a sense of how different writers leverage \system to test rhetorical strategies.  In the case of P2 we see that they created a total of 5 layers, two of which were alternative structures suggested by Structure Sam. In case of P1 they created 13 layers and used 4 of them in their final essay generation. Similarly, P5 created 17 layers, counting all tears, combination, and alternative suggestions layers, and used only two to generate their final essay. }



\subsection{Comparative Analysis: Between-Subjects Evaluation on Prolific}
\change{Building on the insights from the usability assessment, we designed a comparative evaluation. We wanted to understand what facet of \system supported the dynamic knowledge transformation we observed by the writers in study 1. To isolate the effects from different features, we constructed two separate interfaces in addition to \system. We conducted a between-subject study with three conditions: (1) a layered interface with in-line LLM (\system condition), (2) a writing interface with in-line LLM but no layers or spatial component (In-Line-LLM), and (3) a writing interface with a separate AI chat window for LLM interaction (Chat-LLM condition).}


% \begin{figure}
%     \centering
%     \includegraphics[width=0.9\textwidth]{figures/qual_comp.pdf}
%     \caption{\change{Visualization of LLM Interaction Across Conditions. Each square represents a different writing subprocess, with their meaning defined in the legend.}}
%     \label{fig:all_conditions}
% \end{figure}

\begin{figure*}[t!]
  \centering
  \includegraphics[width=0.95\textwidth]{figures/study_2_qual.pdf}
  \caption{Visualization of LLM Interaction Across Conditions. Each square represents a different writing subprocess, with their meaning defined in the legend.}
  \label{fig:all_conditions}
\end{figure*}


\subsubsection{Design Decision For Conditions}
In order to have a fully functional In-Line-LLM application that supports the Writer's Friends without the layered paradigm, we had to make some design choices. Detail Danny, Idea Ivy, Feedback Felix and Audience Ali, have their operations constrained on the active layer so they will remain the same but in the case of Structure Sam and Tone Tara, where new layers with transformed content are generated, we had to change their behavior for the In-Line-LLM condition. Both Sam and Tara, instead of generating new layers, replaced the existing content in the editor and provided the user the option in the toolbar to reverse it to their previous content before the transformation. In the case of the Chat-LLM interface, we added a chat interface right beside the editor, to resemble having a word processor and an LLM chat interface open in split view. The rationale for creating our own interface instead of using a baseline of an existing LLM chat interface with a word processor was to have control over the data logging.

\subsubsection{Participants}
We conducted the study through Prolific and had a screening for people who wrote professionally, roles included journalist, copywriter/marketing/communications, and creative writing. We also screened for participants who spent more than 5 hours a week on Prolific, to increase chances of high-quality participation. We recruited 84 participants (F=48, M=30, NB=6). Participants reported ages between 18 and 64, the median age being in the 45-54 age group. 

\subsubsection{Study Procedure}
Each participant was randomly assigned to one of the three conditions (\system, In-Line-LLM and Chat-LLM) such that there was a distribution of 28 participants per condition.  On a scale of 1 to 5, participants reported confidence in their writing skills as follows: chat-llm ($\mu = 3.96$, $\sigma = 0.98$), in-line-llm ($\mu = 4.20$, $\sigma = 0.61$), and \system ($\mu = 4.42$, $\sigma = 0.64$) in case of \system, This study design would allow us to isolate the effect of different components and formulate a comprehensive assessment. Each participant first filled out a demographic survey, and consented to having their writing data logged into our database. The participants were each asked to watch a video tutorial for their respective condition interface. Afterwards,  they had to take a quiz about the interface. This helped ensure that the participants understood the interface well before using it. They were not allowed to proceed without getting all the answers right. Following this task they were assigned their task, which they were given 40 minutes to complete. The task required writing about an 800 words essay on a \textit{crowdworker's experiences with Large Language Models (LLMs):} They were specifically asked that it cover LLMs' impact on their work, adaptation strategies, and future outlook. They were instructed that the essay should include specific examples, data, and reflections on both positive and negative aspects of LLMs in crowd-work. The task was intentionally made non-trivial as we wanted to simulate a real writing task for the participants in a creative capacity. In order to ensure the participants did not write someplace outside of the interface before pasting it later, we tracked their activity in the interface and made persistent activity a criterion for their submission to be valid. We also manually validated the essays written to ensure they were actual essays about the topic and not something off-topic or incoherent.

% \change{The words per minute we report are only from keystrokes registered from the participants. LLM generations do not count towards the words per minute.}  

\subsubsection{Results}




\change{Figure ~\ref{fig:all_conditions} shows the sequence of invocation of features across conditions. We coded the prompts issued by users into one of the following categories: (1) Review, (2) Tone Transformation, (3) Elaboration, (4) Organize, (5) Brainstorm, and (6) Research. Between the In-Line-LLM and \system condition, we combined Feedback Felix and Audience Ali into Review. We also marked prompts that did not fit into any of these categories as miscellaneous. We also do not show instances where users reissue commands when they are not content with the initial generation. Notably, analysis of prompt reissuance revealed a higher density in the Chat-LLM condition, suggesting users were frequently dissatisfied with initial responses and attempted to regenerate content. In contrast, we observed substantially lower reissuance in the In-Line-LLM and \system conditions. As we can see in the visualization, usage differences exist between the interfaces that support in-line LLM support and the chat interface. Looking at the prompts issued by the user, we observed difficulty in understanding how the LLM could support their writing. We observed prompts like, \textit{``You're not a very comprehensive model, are you?''} and\textit{  ``So, your main function is to serve as a research tool?''.} Despite providing an introduction video for all conditions, users experienced the most difficulty in the chat interface. We attribute this to challenges in formulating specific writing intentions and planning, which we believe the task-specific LLM personas were successful at bridging. Users in the In-Line-LLM and \system asked for assistance with transformation or elaboration. In the case of the chat interface, we observed users attempting to solicit complete essays based on points they specified in the prompt.}

\change{When comparing the In-Line-LLM and \system conditions, we observed greater diversity in writing strategies among \system users. A notable pattern was the tendency to seek feedback earlier in their writing process, suggesting that users aimed to align their work with audience expectations and prevent significant deviations as their drafts progressed. Conversely, the In-Line condition exhibited a more fixed usage pattern, where users typically began with review tasks, spent the bulk of their time elaborating on content, and concluded with tone transformations. Interestingly, this pattern deviated from the tutorial video shared with participants, leaving us uncertain about the underlying reasons for this behavior. Through this analysis, we conclude that \system effectively helps users bridge the envisioning and articulatory gap, enabling them to articulate their writing intentions and integrate feedback more seamlessly into their workflows. While in-line support fosters more thoughtful usage of LLM,  the spatial organization of \system propelled the dynamic knowledge transformation.}



\section{Discussion}


In this paper, we adopted a learner-centered design approach, beginning with a formative study to identify students' challenges with existing tools. Based on these insights, we developed DBox, a tool that scaffolds students in breaking problems into smaller parts and provides personalized, adaptive support. Our user study demonstrated that DBox improved learners' performance on similar algorithmic problems, increased perceived learning gains, and fostered greater cognitive engagement, achievement, and satisfaction. In this section, we discuss design implications and generalizability based on our key findings.


\ms{
\subsection{Chaining Learners' Thoughts with Visualized Structured UI Components}

Decomposition requires students to effectively organize their thoughts. While visual elements are known to promote structured thinking and support mental model construction \cite{mcdougall2001effects, liu2010mental}, our formative and user studies revealed shortcomings in existing tools like LeetCode and ChatGPT, which rely on textual representations without adequately supporting structured mental models. In contrast, DBox uses an interactive step tree to visually organize learners' thoughts. This feature was praised by 22 of 24 participants for enhancing algorithmic thinking, serving as a progress tracker, and providing value even without AI assistance.

DBox's interactive step tree and tree-based scaffolding demonstrate the broader potential of intelligent tutoring systems (ITS) to promote active learning and self-regulated problem-solving in fields requiring problem decomposition. Similar principles could benefit STEM education, such as physics or engineering, by externalizing abstract concepts and facilitating multi-step problem-solving. Additionally, progress-tracking visual components may inspire designs for professional training tools in areas like medical diagnostics or software engineering.

\subsection{Promoting Independent Thinking and Active Decomposition Learning}

\subsubsection{\textbf{Transforming Learners from Passive Readers to Active Thinkers}}

Many coding tools provide direct answers or solutions \cite{kazemitabaar2023novices, phung2023generating}, which, while efficient, often bypass opportunities to develop critical problem-solving skills. In contrast, DBox cultivates students' decomposition abilities through structured scaffolding, fostering critical thinking and self-regulated learning in line with learning by doing \cite{anzai1979theory} and constructivist principles \cite{tobias2009constructivist}.

To strengthen decomposition skills, DBox first encourages students to develop their own decomposition strategies by coding or building a step tree from scratch. While DBox can generate parts of a step tree from a student's existing code, these steps are derived from the learner's own reasoning, with DBox acting solely as a modality converter. Besides, DBox provides feedback on tree node statuses, identifying potential errors or missing steps without directly showing the correct answer, challenging students to critically evaluate and refine their decomposition plans.


DBox's scaffolded hint system further supports decomposition skill development by providing adaptive guidance tailored to the student’s progress without overwhelming them. All hints are based on the learner's current decomposition skeleton, with the most detailed hint—``reveal substep''—triggered only after repeated attempts and struggles. Notably, even the most detailed hints prompt only one substep, requiring students to complete the rest independently. As shown in Sec \ref{hintusage}, only 19\% of hints are this detailed, with students primarily relying on simpler, thought-provoking question hints. This scaffolded support system balances guidance and independent thinking, keeping students engaged during challenges without compromising their ability to independently decompose problems \cite{kinnunen2006students}.

Based on these findings, we recommend fostering active problem-solving by shifting students from passive content consumption to active solution creation. Designers could adopt layered scaffolding, starting with minimal guidance and increasing support as needed, to help students progressively master decomposition skills while maintaining confidence and avoiding frustration. Additionally, adaptive learning techniques, such as real-time feedback and progress tracking, can further tailor the support to individual decomposition barriers, encouraging deeper engagement with decomposition tasks. Moreover, designers could integrate metacognitive strategies, such as encouraging students to articulate or reflect on their decomposition approaches, to further enhance critical thinking and foster habits of independent thinking.




\subsubsection{\textbf{Choice of Scaffolding: Balancing Independent Problem-Solving and Efforts}}

Scaffolding involves providing tailored support to help learners accomplish tasks they cannot yet complete independently \cite{kim2011scaffolding, tobias2009constructivist}. Broadly, scaffolding strategies fall into two categories \cite{van2010scaffolding}: (1) gradually reducing assistance as learners gain proficiency, and (2) encouraging independent problem-solving while offering incremental support to address challenges. DBox adopts the second approach, emphasizing independent thinking and encouraging learners to actively decompose problems \cite{zimmerman2013theories}. While our scaffolding strategies successfully enhanced critical thinking, satisfaction, and perceived usefulness, they also led to increased cognitive effort (Sec. \ref{Effects_on_UX}). This tradeoff underscores the importance of carefully balancing cognitive effort with the promotion of independent thinking.

Future designs could incorporate adaptive scaffolding that adjusts support dynamically based on learner proficiency, reducing unnecessary effort in areas where students have demonstrated competence. Additionally, while incremental scaffolding was effective for algorithmic problem-solving, tailoring strategies to different educational contexts could enhance their applicability in diverse domains. Such adaptive, context-specific approaches could further optimize the balance between support and independence in learning environments.


\subsection{Supporting Personalized Algorithmic Programming Learning}

\subsubsection{\textbf{Prioritizing Learners' Own Solutions Over Optimality}}

Algorithmic problems often have multiple solutions with varying time and space complexities. DBox prioritizes independent exploration by supporting learners' strategies rather than steering them toward a single ``optimal'' solution. Using LLM-driven prompts, it evaluates and guides each step based on the learner's reasoning, preserving their step decomposition and respecting their input—even when errors occur. While some solutions may not be the most efficient, this approach fosters autonomy by aligning feedback with learners’ thought processes instead of enforcing rigid standards.

Our user study showed that this approach improves learning outcomes and is well-received by students. We recommend designing systems that respect personalized problem-solving strategies by aligning feedback with learners' reasoning while allowing for diverse approaches. Designers should balance flexibility and rigor, using prompts and interfaces that support varied strategies while gently guiding learners toward effective solutions.


\subsubsection{\textbf{Catering to Individual Learning Styles and Contextual Needs}}

DBox accommodates diverse problem-solving approaches with two input modes: coding and natural language descriptions. Each mode offers distinct advantages tailored to different learners, stages, and situations. Learners can switch seamlessly between modes, with progress automatically synced across the interface. Features such as verifying code-step alignment ensure strong integration between modes.

Our findings reveal that this flexibility enhances user experience. Participant interaction logs and interviews revealed three usage patterns, highlighting that each mode fits different needs: code mode works well for students with a clear and detailed problem-solving plan already, while the step tree with natural language descriptions helps less experienced students with only a basic idea who are not ready to write code directly, boosting their confidence.


We argue there is no universal “best” mode for programming education—each has unique benefits depending on the learner habits, expertise, and context. Future tools should provide flexibility, like DBox, or use adaptive algorithms to recommend modes based on user needs and context. This flexibility highlights the importance of designing educational tools that accommodate varying levels of expertise and problem-solving styles, which can be generalized to other domains requiring personalized learning \cite{bernacki2021systematic}.

\subsection{Appropriate Usage of LLMs for Supporting Algorithmic Programming Learning}

\subsubsection{\textbf{Caution About LLM Errors}}

Although LLMs have shown strong performance in coding tasks \cite{finnie2023my, leinonen2023using}, they remain prone to errors. Our technical evaluation and user study revealed that even with comprehensive context—such as problem statements, user code, and natural language steps—LLM sometimes misinterprets user descriptions. These errors likely arise from discrepancies between the natural language used by students and the formal, precise language the LLM was trained on, which is primarily sourced from web-based code and comments \cite{liu2023wants}.

Such misinterpretations can hinder learning by causing confusion or frustration. While future improvements to training data and GPT versions may mitigate these issues, design strategies can help address them. \textbf{First}, LLMs should avoid giving direct solutions and instead focus on fostering active problem-solving through explanations and hints. \textbf{Second}, feedback could be paired with interactive features, like a ``Run Code'' option, allowing students to validate their reasoning. \textbf{Third}, simple tutorials could teach users how to phrase their descriptions more clearly, improving LLM's understanding. Additionally, future tools could integrate a ``Language Enhancement'' feature to suggest improvements or assess the clarity of descriptions, aiding LLM in accurately capturing user intent. Most importantly, we recommend designers prioritize technical feasibility, such as conducting rigorous evaluations like ours, before fully integrating LLMs into programming learning tools.
}



\subsubsection{\textbf{Learner-LLM Co-Decomposition of Solutions: Learner as Leader, LLM as Aid}}

A central feature of DBox is the construction of a step tree, where students break solutions into steps and sub-steps. The LLM supports this by mapping code to step descriptions, evaluating them, and offering hints. However, students maintain full control, deciding how to decompose problems and define each step, fostering independent thinking. The LLM acts solely as an aid, using a scaffolding approach to support the development of learners' Zone of Proximal Development (ZPD) \cite{chaiklin2003zone}. Unlike tools like ChatGPT or Copilot that dominate problem-solving, DBox fosters deeper cognitive engagement. Students reported greater accomplishment and found this approach more effective for learning.

This contrasts with existing human-AI collaboration paradigms in non-educational scenarios where AI usually suggest options, leaving final decisions to users \cite{dang2023choice, gao2024collabcoder, gebreegziabher2023patat, ma2019smarteye, ma2022glancee}, such as in human-AI decision-making \cite{ma2023should, ma2024towards, ma2024you}. Some educational tools, like Jin et al. \cite{jin2024teach}, use LLMs to generate solutions for students to evaluate, which aids in syntax learning but such ``LLM-generate then learner-evaluate'' approach is less effective for algorithmic problem-solving, where constructing solutions is key. Just evaluating LLM-generated contents can place a cognitive anchor on learners \cite{furnham2011literature}, limiting independent thinking and creativity. Thus, task allocation between humans and AI should align with the educational context (e.g., whether it is basic knowledge/concept learning or higher-level creative thinking). Future LLM-based educational tools should carefully define the division of roles between LLMs and learners, tailoring it to specific learning contexts and goals.




% \subsubsection{Human-LLM Co-Decomposition of Solution: AI Should Judge Instead of Recommending}

% A core interaction in DBox is the construction of a step tree, where the entire solution is broken down into a series of steps and sub-steps. We refer to this as the human-LLM co-decomposition process. In this process, the LLM behind DBox plays three roles: First, it maps the student's written code into step descriptions. Second, it evaluates the status of each step and sub-step (whether they are correct, incorrect, missing, or need further decomposition). Third, it provides hints for incorrect or missing steps or sub-steps. However, the actual construction of the step tree—such as dividing the solution into steps and sub-steps and determining the content of each node—remains primarily the student's responsibility.

% This division of labor maximizes student engagement in independent thinking and problem-solving. The LLM does not provide any suggestions for decomposition nor directly recommend content for specific steps, aligning with the scaffolding educational approach, where guidance is provided appropriately, but the main task of forming the solution is left to the students.

% In contrast, when students directly seek help from an LLM, such as asking questions in ChatGPT or using Copilot for code completion, the LLM takes too much initiative by directly offering ideas or code. In our co-decomposition design, however, students demonstrated higher cognitive engagement and more active critical thinking. Furthermore, students reported that constructing solutions in this way gave them a greater sense of achievement and made them feel the process was more beneficial for learning, leading to higher satisfaction with the experience.

% Related work has proposed similar approaches. For instance, XXX, in the context of problem-solving, uses the "learning by teaching" concept, where students take on the tasks of judging and teaching, while the LLM generates most of the solutions. Compared to our approach, their division of labor between the student and the LLM is reversed. This method works well in introductory programming, where the focus is on mastering syntax. Having students guide the LLM to generate code or evaluate potentially incorrect code produced by the LLM is an effective way to quiz them. However, in our work, which focuses on algorithmic programming, the key step is constructing a solution from scratch. If the LLM builds the solution, leaving students only to judge it, it hampers their independent thinking.

% Thus, when designing LLM-based educational tools in the future, it is crucial to consider the specific context to effectively allocate tasks between the student and the LLM, ensuring that students derive the maximum benefit from the co-decomposition process.


% \subsection{Future Design Opportunities}

% \emph{Providing Appropriate Generative Assistance:} While DBox promotes independent problem-solving, some users showed interest in features like auto-completion for trivial coding tasks. Future versions could balance promoting independence with targeted assistance by enabling adjustable difficulty levels and offering contextual suggestions when appropriate.

% \emph{Covering All Stages of Algorithmic Programming:} DBox currently lacks a focus on foundational algorithm instruction and problem comprehension. Future iterations could include features like generating distractor solutions, input-output tests, and step-by-step rephrasing to help students grasp key concepts and understand the coding problem.

% \emph{Combining Step Trees with Dialogue:} Users can currently describe their thought processes but cannot ask questions. Adding a dialogue system to the step tree would allow students to share challenges and ask follow-up questions. GPT could then provide guided feedback without giving direct answers, supporting independent problem-solving.





% \emph{Other Important Features.} DBox could offer more control by allowing users to select specific parts of their code for targeted evaluation and guidance. A ``review'' feature could also help students reflect on key stumbling points, understand where their thought process went wrong, and how they eventually solved the problem.


% \subsection{Future Design Opportunities}

% \emph{Providing Appropriate Generative Assistance.} Our tool primarily focuses on encouraging users to create the step tree and write the code independently, with the system mainly serving as a judge. However, users expressed a desire for some intelligent completion features, particularly for repetitive or simple code, allowing them to focus their efforts on learning the key parts. Future improvements should strike a balance between fostering independent thinking and providing appropriate assistance. One approach could be designing basic rules where the tool offers intelligent suggestions and completions for parts unrelated to the core logic, while maintaining the current level of independence for key learning areas. Additionally, the system could offer different modes, allowing users to choose the level of assistance, from basic judgment-only feedback to a combination of judgment, guidance, necessary completions, and even on-demand suggestions.

% \emph{Covering All Stages of Algorithmic Programming.} Currently, our system does not cover the basic teaching of algorithms or the problem comprehension stage. In the future, to address the diversity and uncertainty in solutions and help students grasp multiple approaches, we could expand assistance during the idea formation phase. For example, GPT could generate multiple potential solutions with distractors, prompting students to identify the one that meets the problem's complexity requirements. We could also introduce specialized algorithm training, where students select a specific algorithm, and the system’s guidance focuses solely on that algorithm. To assist with problem comprehension, we could incorporate input-output tests to check students' understanding of the problem and step-by-step rephrasing to help them grasp more complex problems.

% \emph{Combining Interactive Step Trees with Dialogue Boxes.} Sometimes users want to describe their difficulties, and currently, we ask them to outline their thought processes. Additionally, users may want to ask follow-up questions. In the future, we could combine the structured step tree with a small dialogue box. The primary goal would still be to construct the step tree, but users could engage in a conversation with GPT in the context of the current step tree or a specific step. Importantly, GPT should guide the user without revealing direct answers.

% \emph{Other Important Features.} First, DBox could offer learners more control, such as allowing users to select specific parts of the code for targeted evaluation and guidance. We could also introduce a summary feature for key stumbling points, helping students reflect on the challenges they faced, where their thought process went wrong, and how they eventually overcame the problem.




\subsection{Limitations and Future Work}

This study has several limitations. \emph{First}, we tested DBox's effectiveness on only two problem types; future work should examine a broader range of algorithms. \emph{Second}, participants engaged in just one learning session per condition due to time constraints, whereas mastering algorithmic problems typically requires extended practice. Longitudinal studies should explore how DBox supports skill development over time, including changes in mental models and skill retention. \emph{Third}, we assessed learning gains based on correctness in a test session using similar learning and test problems. Future research should evaluate knowledge transfer to less similar problems. Due to time constraints, we conducted a single post-test rather than a pre-post comparison. While pre-test expertise filtering and randomization minimized prior familiarity effects, a more rigorous pre-post design would yield more accurate learning gain measurements. Looking ahead, we plan to release DBox as a Chrome plugin for integration with existing coding platforms, enabling large-scale field studies. This will allow for the collection of long-term usage data and periodic surveys to identify usage patterns and learning experiences over time.



% This study has several limitations. First, in our within-subject design, we selected two types of algorithm problems—Greedy and Binary Search—and randomly assigned them to two conditions (DBox and baseline). However, selection bias may still exist, as some participants might naturally excel at one type of algorithm. Although we addressed this by filtering participants' proficiency through a pre-test and using a Latin Square design, further validation across a broader range of algorithms is needed in future work.

% Second, students experienced only one learning session per condition before the test session. While this allowed for a fair comparison, mastering algorithmic problems typically requires extended practice. Future work should explore how DBox supports students' long-term improvement in algorithmic skills. Longitudinal studies could provide insights into changes in learners' mental models, allowing students more time to deepen their understanding and refine their decomposition methods. Additionally, retention tests could assess whether students can still apply learned problem-solving methods after a time gap.

% We measured learning gains through correctness scores in the test session, with relatively similar learning and test problems. Future work should explore students' ability to transfer their knowledge to problems with lower similarity. Due to time constraints, we opted for a single post-test rather than a pre-post comparison. While we minimized prior familiarity effects by filtering participants and randomizing problem assignments, future studies could adopt a more rigorous pre-post test design for better measurement of learning gains.

% Looking ahead, we plan to release DBox as a Chrome plugin for integration with existing online coding platforms and large-scale real-world testing. In such settings, where students may be more motivated (e.g., preparing for algorithm interviews), we can gather long-term usage data while ensuring privacy. We also plan to conduct periodic surveys to track changes in students' usage patterns and learning experiences over time.



% \subsection{Limitations and Future Work}

% This study has several limitations. First, in our within-subjects study, we selected two types of algorithm problems, Greedy and Binary Search, and randomly assigned them to two conditions, DBox and the baseline. However, there may still be selection bias, where some participants were naturally better at one type of algorithm. While we mitigated this issue to a large extent by filtering participants' proficiency through a pre-test and employing a Latin Square design to randomize the problem-condition assignment, there is still room for improvement. Future work should validate DBox's effectiveness across a broader range of problem types.

% Second, in our experiment, students only experienced one learning session in each condition before moving on to the test session. Although this comparison was fair (as both conditions had only one learning session), mastering an algorithmic problem often requires extended practice. Future work should explore how DBox can help students gradually improve their algorithmic programming skills over time. Longitudinal studies may reveal significant changes in learners' mental models, providing more time for them to understand a specific algorithm and enhance their decomposition methods. Additionally, future studies could include retention tests to measure whether students can still effectively apply previously learned problem-solving methods after a period of time.

% Furthermore, when objectively measuring students' learning gains, we calculated their correctness score in the test session. On the one hand, the learning session and test session problems had a relatively high degree of similarity. Future work should investigate whether students can transfer what they have learned to solve problems of the same algorithm type with lower similarity. On the other hand, due to time constraints, we did not include a pre-post test comparison, opting for a single post-test instead. This result might be influenced by students' pre-existing familiarity with the problems. Although we mitigated this issue by filtering for familiarity (ensuring participants were not too familiar with the problems) and randomizing the problem assignments, future work could include a more rigorous pre-post test design to better calculate students' learning gains.

% Moreover, DBox is currently only applied in algorithmic programming, specifically solving algorithm problems. However, this decomposition-based computational thinking approach could be extended to other learning scenarios, such as project-based learning. Future work could explore how to adapt DBox to broader educational contexts outside of algorithmic programming.

% Looking forward, we aim to deploy DBox in real-world algorithm courses. Since algorithms are a core required subject in undergraduate computer science curricula, we hope to investigate how students who have just learned algorithm concepts use DBox to develop their problem-solving skills. Additionally, we plan to convert DBox into a Chrome plugin and release it in the Chrome Web Store for real-world testing. This would allow DBox to seamlessly integrate with existing online coding platforms, enabling large-scale experiments. In such settings, students' motivation may be stronger (e.g., a graduate preparing for an algorithm interview), leading to more realistic usage patterns. Students could use DBox to tackle a wide variety of algorithm problems. We hope to collect long-term (e.g., six-month) usage data from real-world users while ensuring privacy, and use periodic surveys to capture changes in students' usage patterns and learning experiences over time.





\section{Conclusion}
% In this paper, we introduced Decomposition Box (DBox), a novel tool designed to scaffold learners in decomposing problems during algorithmic programming learning. Based on insights from a formative study, we identified key design goals to address the limitations of existing tools in algorithmic programming education. DBox supports two critical stages of the programming process: idea formation and idea implementation. By offering two modes (code mode and language mode), it encourages users to independently develop their solution strategies. The interactive, visual step tree helps students break down problems and build a structured mental model. DBox provides fine-grained, step-level feedback, enabling students to quickly identify issues, while its multi-level guidance offers targeted support without undermining independent thinking.

% Our user study demonstrated that DBox led to significantly higher learning gains, cognitive engagement, and critical thinking. Students reported a stronger sense of achievement and found the assistance both appropriate and effective for their learning. We identified three main usage patterns, underscoring the importance of respecting students' problem-solving habits and offering them autonomy. The learner-LLM co-decomposition model we designed promotes independent thinking while allowing the LLM to contribute meaningfully, even with occasional imperfections. 

% We hope the formative study, design goals, features, technical evaluation, and key findings from this work will inspire future research on developing educational tools for broader programming learning.
In this paper, we introduced DBox, an interactive tool designed to help learners decompose algorithmic programming problems by supporting both solution formation and implementation. Featuring an intuitive tree-like box widget, DBox accepts input in both code and natural language, fostering independent problem-solving while its step tree structure helps learners develop structured mental models. It provides step-level feedback and layered guidance without compromising learner autonomy.
Our user study showed that DBox significantly improved learning outcomes, cognitive engagement, and critical thinking, with students reporting a greater sense of achievement and finding the support highly effective. Additionally, we identified three key usage patterns, highlighting the importance of accommodating individual problem-solving styles. Moreover, our findings suggest that the learner-LLM co-decomposition approach fosters independent thinking while providing meaningful guidance, even with occasional imperfections.
We hope the insights from our system design will inspire future research on integrating LLMs into educational tools for programming learning.



In this study, we performed the first large-scale analysis of data leakage across 83 software engineering (SE) benchmarks, covering three popular programming languages—Python, Java, and C/C++. By combining an efficient near-duplicate detection algorithm with extensive manual labeling, we ensured the accurate identification of leaked data.



Our findings show that while data leakage is generally low, with average leakage ratios of 4.8\%, 2.8\%, and 0.7\% for Python, Java, and C/C++ benchmarks respectively, some benchmarks exhibit higher leakage that requires attention. We identified four main causes of leakage: direct inclusion of benchmark data in pre-training datasets, overlap between source repositories, reliance on platforms like LeetCode, and shared data sources such as GitHub issues.
We also found that automatic detection methods, like Perplexity-based metrics, struggle to distinguish between leaked and non-leaked samples. Additionally, our experiments reveal that data leakage inflates evaluation metrics, with models performing significantly better on leaked samples. For instance, StarCoder-7b achieved a Pass@1 score 4.9 times higher on leaked samples, underlining the need to address leakage to ensure fair evaluations.
This study offers insights into data leakage status in SE benchmarks and its impact on LLM evaluation.


In the future, we aim to expand the analysis to additional benchmarks and explore new methods to prevent or further reduce data leakage.





\vspace{0.2cm}
\noindent \textbf{Acknowledgement.}  This research / project is supported by the National Research Foundation, under its Investigatorship Grant (NRF-NRFI08-2022-0002). Any opinions, findings and conclusions or recommendations expressed in this material are those of the author(s) and do not reflect the views of National Research Foundation, Singapore.





\bibliographystyle{ACM-Reference-Format}
\bibliography{99_refs}

\end{document}

