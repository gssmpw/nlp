\section{\change{\system: A Layered Interface For LLM Co-Writing} }
\begin{figure*}[t!]
    \centering
    \includegraphics[width=\textwidth]{figures/workspace_1.pdf}
    \caption{Workspace view of \system: (a) Zoomable,  scrollable writing workspace, (b) workspace level operations, (c) a text layer with layer toolbar on top, and (d) the compiled document viewer.}
    \label{fig:canvas}
\end{figure*}


\system is designed to support flexible generative writing workflows by interleaving content development and structural organization. In designing \system, our goal was to lower the \textit{envisioning gaps} around AI capabilities, instructions, and intentionality as well as to reduce the \textit{semantic} and \textit{articulatory} distance when interfacing with Generative AI. As shown in figure~\ref{fig:canvas}, the primary interface consists of an infinite zoomable \textit{workspace} where the writer can create and organize individual \textit{layers}. A \textit{layer} is a discrete, modular content unit within the writing workspace designed to encapsulate specific elements of the writing or organizational process. 
% Layers serve as flexible sandboxes where tasks such as tone adjustments, elaboration, or structural reorganization can occur in isolation, fostering iterative and exploratory workflows.

\change{Here, we use the ``layered'' metaphor because it aligns with how individuals think, by separating, refining, and integrating complex ideas. Unlike layers in visual tools such as Photoshop or Illustrator, which focus on the compositional blending of visual elements, \systems layers emphasize cognitive composition, representing distinct aspects of the writing or organizational process—such as brainstorming, tone adjustments, and meta-information. This metaphor naturally accommodates iterative workflows, enabling users to ``tear apart,'' ``combine,'' or ``stack'' layers for experimentation, similar to rearranging layers on a desk. Each layer acts as a \textit{sandbox} for specific tasks, preserving creative control and enabling non-linear exploration. By reimagining the familiar concept of layers, \system bridges user expectations with the flexibility and modularity required for writing tasks, where the process of exploration and refinement is as critical as the outcome.}



\system supports three types of layers, which include (1) a templated metadata layer for specifying the overall writing goals, (2) content layers for authoring the main text, and (3) a scratchpad for gathering relevant information to support the writing process. \system also supports `tags' for labeling individual layers and collections of layers (e.g., a stack of layers). Further, the interface provides contextual toolbars at the layer and workspace level. The layer-level toolbar consists of standard text formatting operations and specific AI functionality. The workspace-level toolbar supports adding layers and intelligent structuring operations. Further, \system implements an extensible collection of AI ``friends'' that provide targeted writing support for ideation, content organization, research, adjusting the tone, etc. To better understand how \system supports writing, let us follow Sashi, a freelance writer whose goal is to write an opinion piece about ``LLMs and Creative Ownership.'' 

\begin{figure*}[htbp!]
    \centering
    \begin{subfigure}[b]{0.28\textwidth}
        \centering
        \includegraphics[width=\textwidth]{figures/meta_layer.png}
        \caption{Specifying context of writing in meta layer}
    \end{subfigure}
    \hfill
    \begin{subfigure}[b]{0.28\textwidth}
        \centering
        \includegraphics[width=\textwidth]{figures/intro_layer.png}
        \caption{Starting free writing in Introduction Layer}
    \end{subfigure}
    \hfill
    \begin{subfigure}[b]{0.28\textwidth}
        \centering
        \includegraphics[width=\textwidth]{figures/slash_layer.png}
        \caption{Calling Detail Danny from the `/' trigger menu}
    \end{subfigure}
    
    \vspace{1em}
    
    \begin{subfigure}[b]{0.28\textwidth}
        \centering
        \includegraphics[width=\textwidth]{figures/danny_layer.png}
        \caption{Detail Danny generating two variations for user-specified prompt for detail/elaboration}
    \end{subfigure}
    \hfill
    \begin{subfigure}[b]{0.28\textwidth}
        \centering
        \includegraphics[width=\textwidth]{figures/ramesh_layer.png}
        \caption{Research friend as a scratchpad for researching topics specified in meta layer}
    \end{subfigure}
    \hfill
    \begin{subfigure}[b]{0.28\textwidth}
        \centering
        \includegraphics[width=\textwidth]{figures/child_layer.png}
        \caption{Adding links for child layers that the current layer will extract content from}
    \end{subfigure}
    
    \vspace{1em}
    
    \begin{subfigure}[b]{0.28\textwidth}
        \centering
        \includegraphics[width=\textwidth]{figures/import_layer.png}
        \caption{Tunneling into content from another layer into current layer}
    \end{subfigure}
    \hfill
    \begin{subfigure}[b]{0.28\textwidth}
        \centering
        \includegraphics[width=\textwidth]{figures/peak_layer.png}
        \caption{Peaking at (possible) future content using the bottom right}
    \end{subfigure}
    \hfill
    \begin{subfigure}[b]{0.28\textwidth}
        \centering
        \includegraphics[width=\textwidth]{figures/felix_layer.png}
        \caption{Invoking Feedback Felix for paragraph level feedback}
    \end{subfigure}
    \caption{Key Interactions supported by \system.}
    \label{fig:nine_figures}
\end{figure*}

\change{\subsection{Use Scenario}}
\subsubsection{Goal Setting}
To begin with, Sashi opens \system on his web browser and creates a new project. \system displays the workspace, adding a blank metadata layer~\textcolor{skyblue}{\faFile} to it (Figure~\ref{fig:nine_figures}a). The \textbf{metadata layer} contains a set of guiding questions and response text fields for writing goal, audience, topic context, and intent for writing. The metadata layer optionally allows Sashi to upload relevant documents to the writing task. In this case, Sashi enters the goal as \textit{``to write an opinion piece that explores the complex interplay between Large Language Models (LLMs), creative processes, and copyright law,''} the target audience as ``technology creators and potentially legal professionals,''  and intent as ``arguing for a reevaluation of current copyright frameworks to address the challenges posed by AI-generated content.'' Here, Sashi can also upload relevant documents, such as the news article about the AI-generated photo that won the art prize at the Colorado State Fair\footnote{\url{https://www.nytimes.com/2022/09/02/technology/ai-artificial-intelligence-artists.html}}, the New York Times Lawsuit against OpenAI\footnote{\url{https://nytco-assets.nytimes.com/2023/12/NYT_Complaint_Dec2023.pdf}}, and any other relevant legal documents. Sashi can always revisit this layer throughout his writing to provide additional context. At this point, Sashi clicks on the ``Begin Writing'' button at the bottom of the metadata layer. This adds a new layer to the workspace, which Sashi names `Introduction'  \pageiconwithnumber[lightestgray]{black}{I}.

\subsubsection{Content Development}
On this layer \pageiconwithnumber[lightestgray]{black}{I}, Sashi begins by engaging in \textit{free-writing}, exploring various real-world instances of LLM creativity he has encountered to motivate the article (Figure~\ref{fig:nine_figures}b). Once he has written several examples, he realizes that he must also include a brief description of LLMs and their creative abilities. Instead of manually describing LLMs, Sashi opts to leverage the content development features of \system. Inspired by `personas' for writing feedback~\cite{benharrak2024writer}, we have implemented a set of \textbf{writer's friends} that embody various writing skills, including \coloredcircle{IdeaIvy} Idea Ivy for content ideation, \coloredcircle{DetailDanny} Detail Danny for elaboration, \coloredcircle{StructureSam} Structure Sam for content structuring, \coloredcircle{ToneTara} Tone Tara for stylistic suggestions, \coloredcircle{FeedbackFelix} Feedback Felix for feedback on content, and \coloredcircle{AudienceAli} Audience Ali for audience-specific feedback. 


In this case, Sashi selects the Detail Danny friend by typing a forward slash `/' at the beginning of a new paragraph. This action opens a dropdown menu displaying a list of all friends, from which Sashi chooses \coloredcircle{DetailDanny} Detail Danny (Figure~\ref{fig:nine_figures}c). \system inserts an \customdashedbordertext{inline prompt box}{DetailDanny}{DetailDanny} where he writes \textit{``Write a concise description of LLMs and their creative abilities, explain how they function for the creative tasks listed above''} and presses the enter key. In response, \system generates a paragraph of text about LLMs and creativity. The \customsolidbordertext{generated text}{DetailDanny}{DetailDanny} is highlighted with a background color corresponding to the friend (Figure~\ref{fig:nine_figures}d). Sashi can accept this content by pressing enter at the end of the content or delete it and regenerate new content. Sashi accepts the suggestion and continues editing. Sashi has a high-level vision about a few topics his opinion essay should cover. He creates new layers for `Traditional Copyright Law - \pageiconwithnumber[lightestgray]{black}{L},' `Authorship -- \pageiconwithnumber[lightestgray]{black}{A},' and `Ethical Considerations -- \pageiconwithnumber[lightestgray]{black}{E}' and, as before, writes his thoughts. When writing about copyright law, Sashi realizes he needs to understand copyright principles better. He creates a new \textbf{scratchpad} layer ~\textcolor{legalyellow}{\faFile}, which is supported by Research Ramesh, where he can ask questions about the purpose and intent of copyright protection or how copyright has adapted to previous technological changes to help with his main content writing (Figure~\ref{fig:nine_figures}e). 

In the course of writing, Sashi creates a new layer for Stakeholder Perspectives -- \pageiconwithnumber[lightestgray]{black}{S} and brainstorms using  \coloredcircle{IdeaIvy} Idea Ivy about whose perspectives to include, and in response, Ivy suggests talking about \customsolidbordertext{Content Creators}{IdeaIvy}{IdeaIvy}, \customsolidbordertext{AI Companies}{IdeaIvy}{IdeaIvy}, \customsolidbordertext{Consumers}{IdeaIvy}{IdeaIvy}, and \customsolidbordertext{Legal Experts.}{IdeaIvy}{IdeaIvy}. By individually selecting each stakeholder, Sashi can create \textbf{sub-layers} to flesh out the stakeholder perspective (Figure~\ref{fig:nine_figures}f). There is a persistent link between the text in the Stakeholder layer \pageiconwithnumber[lightestgray]{black}{S} and the sub-layers [\pageiconwithnumber[lightestgray]{black}{$S^{CC}$}, \pageiconwithnumber[lightestgray]{black}{$S^{AI}$}, \pageiconwithnumber[lightestgray]{black}{$S^C$}, \pageiconwithnumber[lightestgray]{black}{$S^{LE}$}]. Sashi can also \textbf{split} the content into two layers and develop the content separately.  

Idea Ivy can also be used to support more complex writing procedures across layers, such as how some specific text about the economic impact in the Ethical Considerations layer \pageiconwithnumber[lightestgray]{black}{E} might relate to the current text in the content-creator layer \pageiconwithnumber[lightestgray]{black}{$S^{CC}$}. To support such complex prompting using content across layers, \system supports a \textit{tunneling} function where the current layer split opens at the cursor position to reveal a desired layer selected from the dropdown (Figure~\ref{fig:nine_figures}g). From here, Sashi can choose the relevant text and import it to the current prompt he is composing. On any layer, if he is stuck on what to write, he can click on the bottom right corner of the layer to \textbf{`peek'} into what the Generative AI might write (Figure~\ref{fig:nine_figures}h). The peek function shows continuing text based on his current writing in a greyed-out format. For creating structure from unstructured text, he can invoke \coloredcircle{StructureSam} Structure Sam, who will create a new layer with a structured representation of the content, including headings and subheadings. He can invoke the \coloredcircle{AudienceAli} audience and \coloredcircle{FeedbackFelix}feedback friends on any layer using the toolbar option to receive comments inline, which he can toggle on or off (Figure~\ref{fig:nine_figures}i). He can call upon \coloredcircle{ToneTara} Tone Tara to generate a different stylistic variations of the current content rendered as separate layers from the original. 

Lastly, as he is writing, Sashi can \textbf{compare} two layers by bringing them close, such that the right edge of a layer is touching the left edge of the other layer \pageiconwithnumber[lightestgray]{black}{$S^{AI}$}\pageiconwithnumber[lightestgray]{black}{$S^{CC}$}. \system recognizes this as an intent to compare and provides a floating button, ``Compare the two layers?'' On clicking this, the button expands into a text prompt where Sashi can input specific instructions for comparison, such as ``how the stakeholder perspectives of content creators and AI companies align or conflict regarding LLM-generated content and copyright.'' The system uses color highlighting and inline annotations to indicate similarities and differences. To mitigate possible confusions, these annotations only persist while the layers remain in proximity. 

\subsubsection{Structure and Rhetoric}
Throughout this process, \system provided several features for Sashi to develop the structure for his final essay and tailor it to specific audiences. To organize the content, \system supports \textbf{combining} two layers into a single layer by bringing the second layer to the bottom of the first layer \twopageicon{lightestgray}{lightestgray} and optionally prompting for specific transitional text between layers. For instance, he can connect a layer on `Challenges posed by LLMs' \pageiconwithnumber[lightestgray]{black}{Ch} with the layer on `Copyright Law' \pageiconwithnumber[lightestgray]{black}{C} by generating transitional text from identifying the problems to analyzing the legal framework meant to address them. Alternately, Sashi can \textbf{tear} layers into parts and glue them together to reorganize the text, i.e. to have content inform structure. Sashi can fold layers he wishes to exclude, which show the text summary on the folded side of the layer. He can \colorbox{yellow}{Tag} individual layers or even clusters and stacks of layers in the workspace. When Sashi is ready to \textit{compose} his essay, he can \textbf{stack} the layers manually in an order [\pageiconwithnumber[lightestgray]{black}{I},\pageiconwithnumber[lightestgray]{black}{C},\pageiconwithnumber[lightestgray]{black}{E},\pageiconwithnumber[lightestgray]{black}{S}\ldots] and generate the final document by concatenating them or asking the LLM to generate the order based on an unordered stack. Here he can issue specific prompts such as generating an audience-specific version, editing for consistency across layers, or even adaptive summarization to meet some target length. The final document is generated with any edited text highlighted for Sashi to review. Sashi can click on the text in the final document and track back to the source layer. In this manner, \systems features allow Sashi to use generative AI for content development and rhetorical strategies, flexibly, fluidly, and iteratively. Table~\ref{fig:layered_action} summarizes the different scripting and shifting affordances of \system.



% \change{The Figures \ref{fig:layered_action} and \ref{fig:writers_friend} below show all the features of \system.}


\begin{figure*}[t!]
    \centering
    \includegraphics[width=0.8\textwidth]{figures/generation_table.png}
    \caption{\change{Content transformation through user prompts. The following 10 features allow writers to issue descriptive instructions for invoking specialized LLM assistance. The example column consists of real prompts issued by participants in the usability assessment.}}
    \label{fig:layered_action}
\end{figure*}







