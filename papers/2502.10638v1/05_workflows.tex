\section{Example Writing Workflows with \system}

In this section, we demonstrate the flexibility of writing with \system through three distinct workflows: freewriting, document-based-question, and parallel topic development. Each example highlights how the layered interface and AI tools support fluid, iterative writing across different starting points and content structures.


\subsection{Freewriting to Argumentative Writing Workflow}
In this workflow, the writer begins with freewriting, allowing ideas to flow without worrying about organization or structure. Once they have written their thoughts, they determine that an argument style structure can be effective to organize the text, consisting of the following components: Claim, Grounds, Warrant, Backing, Qualifier, and Rebuttal~\cite{toulmin2003uses,marshall1989representing}. The writer adds the \textit{argument template}  to \systems workspace and drops the layer with the freewriting text onto the template. Using the template, \system takes the unstructured text and generates six layers, one for each component. Next, the writer can continue to flesh out each layer using Writer's Friends such as ``Idea Ivy'' and later refine, reorganize, and link ideas across layers, transforming initial thoughts into a structured argumentative essay.

\begin{figure*}[t]
    \centering
    \includegraphics[width=\textwidth, height=0.3\textheight, keepaspectratio]{figures/FreeWriting.pdf}
    \caption{Free Writing Example with \system \change{ \textbf{(A)} Writer calls \textit{Template} for organizing their writing into milestones for argumentative writing. \textbf{(B)} They select the ``Ground'' layer for further development. \textbf{(C)} They invoke \textit{Idea Ivy} to brainstorm what to write next and then use \textit{Structure Sam} to organize that into subheadings and paragraphs.}}
    \label{fig:freewriting}
\end{figure*}

\begin{figure*}[t]
    \centering
    \includegraphics[width=\textwidth, height=0.3\textheight, keepaspectratio]{figures/DBQ.pdf}
    \caption{Document-Based Question Example with \system \change{ \textbf{(A)} The writer specifies the context of their writing and upload their assignment to the \textit{Meta Layer}. \textbf{(B)} They call \textit{Research Ramesh} to understand details of their assignment based on the context document. \textbf{(C)} They \textit{Tunnel} into another layer to extract some details. \textbf{(D)} They combine a stack of layers to generate \textbf{E}.}}
    \label{fig:freewriting}
\end{figure*}


\subsection{DBQ Writing Workflow for Students}
In this workflow, the student working on a Document-Based Question (DBQ) can upload primary source documents directly into the system. The metadata page allows the student to provide information about the assignment goals, guiding questions, and document context. The student can then create additional layers for organizing their arguments, categorizing evidence drawn from the primary sources. For example, they might have separate layers for economic, political, and social factors. LLM friends like ``Research Ramesh'' assist by analyzing the documents and suggesting relevant excerpts or summaries that will fit the argument. As the student develops their essay, cross-layer interactions allow them to pull evidence from the source layers into argument layers, ensuring that evidence is seamlessly linked back to the original documents. Using \systems stacking feature, the student can then combine the various layers into a cohesive essay. \system allows them to specify prompts for transitions in order to ensure that all arguments are well-supported by evidence. This workflow encourages students to engage deeply with primary sources, while providing an intuitive way to structure their argument, making the ballistic process of evidence-based writing more fluid and interactive.


\subsection{Layered Topic Development for Research Paper}

In this workflow, the writer develops different sections of the research paper in parallel, creating distinct layers for the literature review, methodology, and findings. \system allows the writer to work on these topics simultaneously, using cross-layer interactions to draw connections between sections. For instance, the writer can link methodology details to key studies in the literature review. Within the findings section, the writer can create separate layers for different findings or topics, allowing them to focus on each result individually. Once these layers are complete, the writer can combine them into a unified findings section, ensuring a cohesive and structured presentation of the results. Writer's friends like ``Research Ramesh'' assist by sourcing and summarizing relevant research, while ``Tone Tara'' and ``Feedback Felix'' help ensure stylistic and argumentative consistency across sections. Once the layers are fully developed, the document structuring features in \system help weave them together into a cohesive research paper.