\section{Introduction}
Writing is rarely a linear process. Instead, it involves a complex interplay of cognitive activities  spanning various levels of abstractions and perspectives~\cite{kellogg1999psychology, pea1987chapter}. Writers must juggle fine-grained details --- such as word choice and sentence structure --- while simultaneously managing broader concerns such as overall organization and rhetorical strategy~\cite{hayes2012modeling, hayes1980identifying}. For instance, imagine writing the introduction to this CHI paper. The author might start by crafting a compelling opening sentence (e.g., \textit{``Large Language Models (LLMs) are effective writing collaborators.''}), only to pause and reconsider how this framing aligns with the paper's overall contribution. They might zoom out to then enumerate  what the key contributions are before diving back in to flesh out a particular argument. As they write, they must anticipate potential questions and critiques from readers, adjusting their narrative to address each of these imagined interlocutors. This dynamic interplay between knowledge creation and rhetorical problem-solving is what distinguishes good writing as a process of \textit{knowledge transformation} rather than mere knowledge telling~\cite{bereiter1987psychology}. 

However, the quality of \textit{writing tools} can make or break this dynamic process of composition, largely depending on the affordances they provide -- or fail to provide~\cite{kozma1991computer, kellogg1999psychology}. Tools such as pen and paper afford a flexible tactile medium for writing, but quickly become limiting when writers need to make extensive revisions or rearrange large blocks of text. Word processors, such as Microsoft Word, allow easy editing and formatting, but the page-based metaphor makes it challenging to escape the rigid top-to-bottom organization. For instance, in the CHI paper example, the author may want to try multiple ways of presenting the argument, such as structuring it around the design process, or components of a theoretical framework. Similarly, they may want to explore alternative arguments to benefits of LLMs in writing. However, such a process can be labor intensive and require duplicating sections, manually copying content, or creating entirely new documents just to assess how these alternative structures and arguments might develop. As a third alternative, platforms such as Overleaf allow writers to manage large sections of papers more efficiently through modular organization and referential code-like syntax. On the other hand, the precise formatting requirements can hinder exploratory writing. Ultimately, writers are left with tools that offer affordances for certain tasks, but which hinder accomplishing others. As a result, writers often avoid engaging in iterative, exploratory, and divergent writing, even though exploratory writing \cite{elbow1998writing,bean2021engaging} can produce good writing~\cite{bereiter1987psychology}. 

As evidenced by rapidly expanding landscape of LLM writing tools~\cite{Lee2024design}, LLMs hold the potential to alleviate many of the process-related challenges, but interface issues largely remain. At the process level, writers can prompt LLMs to generate alternative phrasings, improve sentence clarity, or provide stylistic variations, enabling more efficient iteration on the small details without becoming bogged down in manual rewriting~\cite{reza2023abscribe}. For instance, while drafting a section of this CHI paper, the author could request the model to suggest multiple ways to introduce a complex argument, allowing for immediate exploration of tone and clarity. Or they might use LLMs to receive feedback from different perspectives~\cite{benharrak2024writer} as well as structure suggestions for their text.

Despite these benefits, the exceptional potential of LLMs is often \textit{constrained} by the traditional, linear paradigms of existing writing tools or multi-turn conversational interfaces. These LLM environments don't naturally support the fluidity required for expansive writing. For instance, prompting LLMs across content levels remains a challenge due to these rigid document structures. A writer may wish to ask an LLM to refine a sentence within a section while simultaneously considering how that change affects the cohesion in a different section of the paper. To achieve this aim, they would need to prompt the LLM twice: first, referencing the specific sentence or paragraph for fine-grained revision, and then referencing the broader section or entire document to assess how the changes influence overall coherence. This process can be challenging because current writing tools often do not support seamless cross-referencing between micro and macro levels. Writers must manually manage the integration of these prompts, which disrupts the flow ballistics of the writing process. Therefore, the motivating question for our work is: \textit{How can writing tools be designed to better support seamless, iterative prompting across both micro and macro levels without disrupting the writer's workflow and associated cognitive flow? }

We introduce \system, a novel layered interface paradigm for LLM-augmented writing. \system is designed to address the limitations of traditional writing tools by offering a layered, zoomable workspace that allows writers to seamlessly interleave content development and structural organization. Our layered paradigm supports flexible prompting across both micro and macro levels, allowing users to issue complex queries to LLMs for content generation, restructuring, and refinement. As shown in Figure~\ref{fig:teaser} \system reduces the gulf of envisioning~\cite{subramonyam2024bridging} and execution~\cite{hutchins1985direct} by offering intuitive interactions such as inline friends, contextual toolbars, intelligent structuring, and other layer-based user interactions such as stacking and folding. These features make it easier for writers to issue complex prompts to LLMs and receive targeted feedback, streamlining the management of complex writing tasks. Results from our usability study and comparative evaluation show that \system supports expressive and exploratory writing with LLM assistants. 

Our main contributions include:
\begin{enumerate}
    \item A \textbf{Layered Interface Paradigm} for co-writing with LLMs, supporting flexible transitions between content creation and structural organization.
    \item A \textbf{Generalizable Architecture for Layered Interaction}, enabling dynamic and intuitive LLM prompting across different document based tasks.
    \item A comprehensive \textbf{Evaluation} demonstrating the system's effectiveness in enhancing writing workflows compared to chat- and page-based paradigms.
\end{enumerate}





