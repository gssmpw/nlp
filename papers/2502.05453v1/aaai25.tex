%File: formatting-instructions-latex-2025.tex
%release 2025.0
\documentclass[letterpaper]{article} % DO NOT CHANGE THIS
\usepackage{aaai25}  % DO NOT CHANGE THIS
\usepackage{times}  % DO NOT CHANGE THIS
\usepackage{helvet}  % DO NOT CHANGE THIS
\usepackage{courier}  % DO NOT CHANGE THIS
\usepackage[hyphens]{url}  % DO NOT CHANGE THIS
\usepackage{graphicx} % DO NOT CHANGE THIS
\usepackage[T1]{fontenc}
\urlstyle{rm} % DO NOT CHANGE THIS
\def\UrlFont{\rm}  % DO NOT CHANGE THIS
\usepackage{natbib}  % DO NOT CHANGE THIS AND DO NOT ADD ANY OPTIONS TO IT
\usepackage{caption} % DO NOT CHANGE THIS AND DO NOT ADD ANY OPTIONS TO IT
\frenchspacing  % DO NOT CHANGE THIS
\setlength{\pdfpagewidth}{8.5in}  % DO NOT CHANGE THIS
\setlength{\pdfpageheight}{11in}  % DO NOT CHANGE THIS


%File: anonymous-submission-latex-2025.tex
% \documentclass[letterpaper]{article} % DO NOT CHANGE THIS
% \usepackage[submission]{aaai25}  % DO NOT CHANGE THIS
% \usepackage{times}  % DO NOT CHANGE THIS
% \usepackage{helvet}  % DO NOT CHANGE THIS
% \usepackage{courier}  % DO NOT CHANGE THIS
% \usepackage[hyphens]{url}  % DO NOT CHANGE THIS
% \usepackage{graphicx} % DO NOT CHANGE THIS
% \urlstyle{rm} % DO NOT CHANGE THIS
% \def\UrlFont{\rm}  % DO NOT CHANGE THIS
% \usepackage{natbib}  % DO NOT CHANGE THIS AND DO NOT ADD ANY OPTIONS TO IT
% \usepackage{caption} % DO NOT CHANGE THIS AND DO NOT ADD ANY OPTIONS TO IT
% \frenchspacing  % DO NOT CHANGE THIS
% \setlength{\pdfpagewidth}{8.5in} % DO NOT CHANGE THIS
% \setlength{\pdfpageheight}{11in} % DO NOT CHANGE THIS
%
% These are recommended to typeset algorithms but not required. See the subsubsection on algorithms. Remove them if you don't have algorithms in your paper.
\usepackage{algorithm}
\usepackage{algorithmic}
\usepackage{amsmath,amssymb}
\usepackage{subcaption}
\usepackage{booktabs}
\usepackage{enumitem}
\usepackage{pdfpages}
%\usepackage{hyperref}
\makeatletter
\newcommand*{\rom}[1]{\romannumeral #1}
\makeatother

\newcommand{\framework}{Decentralized Cooperative Generative Agents}
\newcommand{\frameworkAbbr}{DCGA}
\newcommand{\hq}[1]{{\color{black} #1}}
\newcommand{\jd}[1]{{\color{black} #1}}

\usepackage{cite}



% These are are recommended to typeset listings but not required. See the subsubsection on listing. Remove this block if you don't have listings in your paper.
\usepackage{newfloat}
\usepackage{listings}
\DeclareCaptionStyle{ruled}{labelfont=normalfont,labelsep=colon,strut=off} % DO NOT CHANGE THIS
\lstset{%
	basicstyle={\footnotesize\ttfamily},% footnotesize acceptable for monospace
	numbers=left,numberstyle=\footnotesize,xleftmargin=2em,% show line numbers, remove this entire line if you don't want the numbers.
	aboveskip=0pt,belowskip=0pt,%
	showstringspaces=false,tabsize=2,breaklines=true}
\floatstyle{ruled}
\newfloat{listing}{tb}{lst}{}
\floatname{listing}{Listing}
%
% Keep the \pdfinfo as shown here. There's no need
% for you to add the /Title and /Author tags.
\pdfinfo{
/TemplateVersion (2025.1)
}

% DISALLOWED PACKAGES
% \usepackage{authblk} -- This package is specifically forbidden
% \usepackage{balance} -- This package is specifically forbidden
% \usepackage{color (if used in text)
% \usepackage{CJK} -- This package is specifically forbidden
% \usepackage{float} -- This package is specifically forbidden
% \usepackage{flushend} -- This package is specifically forbidden
% \usepackage{fontenc} -- This package is specifically forbidden
% \usepackage{fullpage} -- This package is specifically forbidden
% \usepackage{geometry} -- This package is specifically forbidden
% \usepackage{grffile} -- This package is specifically forbidden
% \usepackage{hyperref} -- This package is specifically forbidden
% \usepackage{navigator} -- This package is specifically forbidden
% (or any other package that embeds links such as navigator or hyperref)
% \indentfirst} -- This package is specifically forbidden
% \layout} -- This package is specifically forbidden
% \multicol} -- This package is specifically forbidden
% \nameref} -- This package is specifically forbidden
% \usepackage{savetrees} -- This package is specifically forbidden
% \usepackage{setspace} -- This package is specifically forbidden
% \usepackage{stfloats} -- This package is specifically forbidden
% \usepackage{tabu} -- This package is specifically forbidden
% \usepackage{titlesec} -- This package is specifically forbidden
% \usepackage{tocbibind} -- This package is specifically forbidden
% \usepackage{ulem} -- This package is specifically forbidden
% \usepackage{wrapfig} -- This package is specifically forbidden
% DISALLOWED COMMANDS
% \nocopyright -- Your paper will not be published if you use this command
% \addtolength -- This command may not be used
% \balance -- This command may not be used
% \baselinestretch -- Your paper will not be published if you use this command
% \clearpage -- No page breaks of any kind may be used for the final version of your paper
% \columnsep -- This command may not be used
% \newpage -- No page breaks of any kind may be used for the final version of your paper
% \pagebreak -- No page breaks of any kind may be used for the final version of your paperr
% \pagestyle -- This command may not be used
% \tiny -- This is not an acceptable font size.
% \vspace{- -- No negative value may be used in proximity of a caption, figure, table, section, subsection, subsubsection, or reference
% \vskip{- -- No negative value may be used to alter spacing above or below a caption, figure, table, section, subsection, subsubsection, or reference

\setcounter{secnumdepth}{2} %May be changed to 1 or 2 if section numbers are desired.

% The file aaai25.sty is the style file for AAAI Press
% proceedings, working notes, and technical reports.
%

% Titl

%Example, Single Author, ->> remove \iffalse,\fi and place them surrounding AAAI title to use it
\iffalse
\title{LLM-Powered Decentralized Generative Agents with Adaptive Hierarchical Knowledge Graph for Cooperative Planning}
\author {
    Author Name
}
\affiliations{
    Affiliation\\
    Affiliation Line 2\\
    name@example.com
}
\fi

%\iffalse
%Example, Multiple Authors, ->> remove \iffalse,\fi and place them surrounding AAAI title to use it

\title{
\hq{
LLM-Powered Decentralized Generative Agents with Adaptive Hierarchical Knowledge Graph for Cooperative Planning
}
}
\author {
    % Authors
    Hanqing Yang\textsuperscript{\rm 1},
    Jingdi Chen\textsuperscript{\rm 1},
    Marie Siew\textsuperscript{\rm 2},
    Tania Lorido-Botran\textsuperscript{\rm 3 4},
    Carlee Joe-Wong\textsuperscript{\rm 1}
}
\affiliations {
    % Affiliations
    \textsuperscript{\rm 1}Carnegie Mellon University\\
    \textsuperscript{\rm 2}Singapore University of Technology and Design\\
    \textsuperscript{\rm 3}Roblox\\
    \textsuperscript{\rm 4}Northeastern University\\
    \{hanqing3, jingdic, cjoewong\}@andrew.cmu.edu, marie\_siew@sutd.edu.sg, t.loridobotran@northeastern.edu
}
%\fi

\renewcommand{\baselinestretch}{1}
\begin{document}

\maketitle

\begin{abstract}
Developing intelligent agents for long-term cooperation in dynamic open-world scenarios is a major challenge in multi-agent systems. Traditional Multi-agent Reinforcement Learning (MARL) frameworks like centralized training decentralized execution (CTDE) struggle with scalability and flexibility. They require centralized long-term planning, which is difficult without custom reward functions, and face challenges in processing multi-modal data. CTDE approaches also assume fixed cooperation strategies, making them impractical in dynamic environments where agents need to adapt and plan independently.
To address decentralized multi-agent cooperation, we propose Decentralized Adaptive Knowledge Graph Memory and Structured Communication System (\textbf{DAMCS}) in a novel Multi-agent Crafter environment. Our generative agents, powered by Large Language Models (LLMs), are more scalable than traditional MARL agents by leveraging external knowledge and language for long-term planning and reasoning. 
Instead of fully sharing information from all past experiences, DAMCS introduces a multi-modal memory system organized as a hierarchical knowledge graph and a structured communication protocol to optimize agent cooperation. This allows agents to reason from past interactions and share relevant information efficiently. Experiments on novel multi-agent open-world tasks show that DAMCS outperforms both MARL and LLM baselines in task efficiency and collaboration. Compared to single-agent scenarios, the two-agent scenario achieves the same goal with 63\% fewer steps, and the six-agent scenario with 74\% fewer steps, highlighting the importance of adaptive memory and structured communication in achieving long-term goals. \hq{We publicly release our project at: https://happyeureka.github.io/damcs}.
\end{abstract}

% Uncomment the following to link to your code, datasets, an extended version or similar.
%
% \begin{links}
%     \link{Code}{https://aaai.org/example/code}
%     \link{Datasets}{https://aaai.org/example/datasets}
%     \link{Extended version}{https://aaai.org/example/extended-version}
% \end{links}

\section{Introduction}
%具身智能体在复杂场景下 manipulation 的 performance robustness 和泛化能力始终是一个广受关注的研究方向。其中,visuomotor imitation learning 是具身智能体 Policy 的主流范式之一,它允许 agent 从高维视觉观察和机器人本体感知中 effectively 学习 manipulation skills。
%然而,增加场景的复杂度和 visual distraction,会导致在简单场景下表现良好的决策模型性能下降。实际上,不仅是 simple imitation learning policy,先进的多模态 foundation models such as GPT-4o 或 vision language action models (VLA),也不能很好地关注一张语义丰富的图片中的特定的局部问题。对于 robot control or 多模态大模型,其往往侧重于 action prediction, observation mapping or 多模态 alignment,而缺少直观的视觉感知增强。模型需要隐性地或遵循 high-level text instruction 从相关的视觉区域中获得面向任务语义的定位知识。
%To tackle this challenge problem, we introduce Imit Diff, a diffusion transformer imitation learning framework with dual resolution enhancement guided by fine-grained semantics information。具体来说,our work 有三个关键组成部分。
%1) Semanstic Injection. Imit Diff 通过 vision language models (VLM) 和 vision foundation models 的 pretrain knowledge 将面向任务的语义信息和高层文本指导转化为显式的 pixel-level 视觉定位标签,注入到 environment observation中。
%2) Dual Res Fusion。 我们构建了双分辨率图像观测流,使用双分辨率视觉编码器分别提取全局和细粒度视觉特征。多尺度视觉信息随后在 attention block 中进行融合,在保证计算 effiency 的前提下,为全局视觉观测引入多尺度细粒度信息,提升场景理解能力。
%3) Consistency policy on diffusion transformer。Diffusion based imitation policies 通常受到 denoise times 的困扰。我们建立了基于 consistency policy 的 DiT action head。Policy 的决策层可以通过 single step denoise 实现系统高频响应。额外地,受益于较快的 inference time,我们引入 temperal ensemble 改善预测动作的平滑性。
%我们设计了四个在 manipulation 精细度上具有挑战性的现实世界任务来评估 Imit Diff,并通过增加场景复杂度和 visual distraction 来测试模型的场景理解能力。额外地,我们设计了 visual distraction 和 category generalization 的 zero shot 实验来验证模型是否受益于 dual res enhancement framework and fine-grained semantics injection。实验结果表明,Imit Diff outperforms 现有的 strong baselines。
%In summary, the contributions of our work are three-fold:
%1) We propose Imit Diff, a DiT architecture imitation learning framework with dual res enhancement guied by fine-grained semantics information.
%2) 我们构建了 open-set vision foundation models pipeline 来获得显式视觉遮罩。该方法能够有效处理机器人控制场景的运动模糊、遮挡、物体丢失情况。并将其作为 fine-grained 语义信息引导 policy decision。
%3) 我们在DiT上实现了consistency policy,显著减少了模型推理时间。通过异步控制框架,实现了 open-set vision foundation models 工作流下的实时控制。
%The code will be publicly available soon。

\section{Introduction}


\label{Intro}
The performance robustness and generalization capabilities of embodied agents in complex manipulation scenarios have long been a focus of significant research interest \citep{ju2025robo, yuan2024learning}. Visuomotor imitation learning is one of the mainstream paradigms of robot manipulation policy \citep{chi2023diffusion, shridhar2023perceiver, ze2023gnfactor, florence2022implicit, hansen2022pre}. This approach enables agents to derive state estimation and decision-making capabilities from expert demonstrations that incorporate high-dimensional visual observations and robot proprioception \citep{ze20243d}.

However, as scene complexity and visual distractions increase, the performance of decision models that excel in simpler environments tends to degrade \citep{zheng2024instruction, liurobustness}. Not only do simple imitation learning policies face challenges, but even advanced multimodal foundation models, such as GPT-4o \citep{hurst2024gpt} or vision language action models (VLA) \citep{liu2024rdt, brohan2022rt, brohan2023rt, o2023open, kim2024openvla, wen2024diffusion}, struggle to accurately focus on specific details within semantically complex images. In fact, in robot control and embodied multimodal foundation models, the focus is often on action prediction, observation mapping, or multimodal alignment. Therefore, intuitive visual perception enhancement is typically lacking. Models can only acquire task-oriented semantic localization knowledge from relevant visual regions either implicitly or when guided by high-level text instructions \citep{reuss2023multimodal}.

To tackle this challenge problem, we introduce \textbf{Imit Diff}, a diffusion transformer imitation learning framework with dual resolution enhancement guided by fine-grained semantics information. Specifically, our work has three key components:

\begin{enumerate}

\item \textbf{Semanstic injection.} Imit Diff transforms task-oriented semantic information and high-level textual guidance into explicit pixel-level visual localization labels through the pretrain knowledge of vision language models (VLM) and vision foundation models, and injects them into the policy observation.

\item \textbf{Dual resolution (dual res) fusion.} We develop a dual res image observation stream and employed a dual res vision encoder to extract global and fine-grained visual features. The extracted multi-scale visual information is subsequently fused within an attention block, integrating fine-grained details into the global visual feature. This approach enhances scene understanding while maintaining computational efficiency.

\item \textbf{Consistency policy on diffusion transformer (DiT).} Diffusion-based imitation policies often suffer from inefficiencies due to the required denoising steps. To address this, we design a DiT \citep{peebles2023scalable} action head incorporating a consistency policy \citep{song2023consistency}, enabling the decision layer to achieve high-frequency system responses through single-step denoising. Furthermore, leveraging faster inference times, we introduce temperal ensemble to enhance the smoothness of predicted actions.

\end{enumerate}

We design four real-world tasks with challenging manipulation precision to evaluate Imit Diff and test the model's scene understanding capabilities by introducing increased scene complexity and visual distractions. Additionally, we conducted zero-shot experiments on visual distraction and category generalization to assess the benefits of the dual res enhancement framework and fine-grained semantic injection. Experimental results demonstrate that Imit Diff significantly outperforms existing strong baselines. 

In summary, the contributions of our work are three-fold:

\begin{enumerate}

\item We propose Imit Diff, a DiT architecture imitation learning framework with dual res enhancement guied by fine-grained semantics information.

\item We developed an open-set vision foundation model pipeline to generate explicit visual masks. This approach effectively addresses challenges such as motion blur, occlusion, and object loss in robot control scenarios, leveraging the generated masks as fine-grained semantic information to guide policy decisions.

\item We implemented a consistency policy on DiT, which significantly reduced the model inference time. Through the asynchronous control framework, we achieved real-time control under the workflow of open-set vision foundation models.

\end{enumerate}

The code will be made publicly available soon.



\section{Related Work}\label{sec:related}
\section{Related Works and Discussions}
\subsection{General Reasoning with LLMs}
Prompting techniques have greatly improved the reasoning abilities of LLMs.
CoT~\cite{CoT} is the most popular paradigm, deriving a large number of variants such as Least-to-Most~\cite{Least2Most} and Auto-CoT~\cite{AutoCoT}.
The central concept of these approaches is ``divide and conquer"--prompting LLMs to deconstruct complex problems into simpler sub-tasks, systematically address each one by reporting the process and then synthesize a comprehensive final answer.
Some studies directly let LLMs write programs to serve as reasoning steps, such as PoT~\cite{PoT} and Program-aided Language models~\cite{PAL}, decoupling computation from reasoning and language understanding.
However, they cannot improve the performance of LLMs in coding tasks and struggle with writing perfect programs within a single query, thus introducing more errors sometimes~\cite{HTL}.
Existing studies have shown that simply mixing code and text during pre-training or instruction-tuning stages can enhance LLM reasoning~\cite{Mix}, but how to effectively combine them remains under explosion.

\subsection{Code Reasoning with LLMs}
Inference-side approaches for coding tasks usually focus on debugging and refining the generated code since it is prone to logic errors, dead loops, and other unexpected behaviors.
Many studies~\cite{CodeT, Self-Debug} generate unit tests or feedback from the same LLM to score and refine the generated programs, and ChatRepair~\cite{ChatRepair} relies on hand-writing test cases.
Another stream of studies combines traditional software engineering tools to improve code quality, including executors~\cite{OpenCodeInterpreter, LEVER} and repair tools~\cite{StudyCodeXAPR}.
Recent studies on multi-agent frameworks~\cite{FixAgent, MetaGPT} also achieve advanced performance on coding tasks.
They borrow the information provided by software analysis tools and embed such information into prompts to expand the ability bounds of LLMs in code reasoning.

\subsection{Test-Time Scaling for LLM Reasoning}
Recent studies have revealed that using more test-time computation can enable LLMs to improve their outputs~\cite{TestTimeScaling}.
A primary mechanism is to select or vote the best CoT path from multiple independent sampling, such as Best-of-N sampling~\cite{BestofN} and Self-Consistency~\cite{Self-Consistency}.
Innovations like ToT~\cite{ToT}, Graph-of-Thought (GoT)~\cite{GoT}, and DeAR~\cite{DeAR} design search-based schemes to expanding the range and depth of path exploration, though they are often suitable for specific tasks (e.g., the Game of 24) as they require to pre-define a fixed candidate size for each node, leading to redundancy or insufficiency.

Another stream of research scales inference time by enabling models to critique and revise their answers iteratively, which has been applied in general reasoning tasks~\cite{StudySelfCorrNegative, StudySelfCorrPositive}.
Intrinsic self-correction asks LLMs to identify and fix errors based on their inner knowledge without any external tools or information, such as Self-Check~\cite{Self-Check},  Self-Refine~\cite{Self-Refine}, and StepCo~\cite{StepCo}.
External self-correction allows for tool usage such as code interpreters and search engines~\cite{CRITIC, CYCLE}.
Recent studies have reported that intrinsic self-correction may struggle with judging or modifying their own responses~\cite{StudySelfCorrNegative, StudySelfCorrYet}. Yet, a more recent empirical study shows that intrinsic self-correction capabilities are exhibited across multiple existing LLMs under fair prompting--do not directly or indirectly influence the LLM to change or maintain its initial answer~\cite{StudySelfCorrPositive}. 
% Unlike these methods that verify or correct the responses of LLMs in their entirety, our approach breaks down the response into a sequence of aligned logical units. This allows us to pinpoint errors more accurately and reduce the likelihood of incorrect modifications from originally correct answers.








\section{Framework: DAMCS}\label{sec:method}
% \carlee{We should change the title to something specific to our method, e.g., \framework}

In this section, we give an overview of our framework. We first formally define how this framework interacts with our problem environment (Section~\ref{sec:setting}) and then describe the design of our multi-modal, adaptive memory system (Section~\ref{sec:memory}), structured LLM output for making agent decisions (Section~\ref{sec:output}) and communication protocol that enables agent cooperation (Section~\ref{sec:communication}).
\subsection{Problem Setting}\label{sec:setting} 
Our goal is to demonstrate that Large Language Models (LLMs) can effectively plan, coordinate, and execute tasks in a multi-agent environment where collaboration and resource management are critical. % In our extension of the Crafter environment \cite{hafner2021benchmarking}, 
We consider an environment model that follows a Decentralized Partially Observable Markov Decision Process (Dec-POMDP)~\cite{bernstein2002complexity,chen2024rgmcomm}, as is common in cooperative MARL, where agents lack complete information about the environment and have only local observations. Figure \ref{fig:framework} gives an overview of this framework. We model the environment as a Dec-POMDP with communication as a tuple $D=\langle I, n, S, A, P, \Omega, O, g, R \rangle$, where $I = \{1,2,\dots,n\}$ is a set of $n$ agents, $S$ is the joint \textbf{state} space, and $A=A_1\times A_2 \times \dots \times A_n$ is the joint \textbf{action} space, where $\boldsymbol{a}=(a_1,a_2,\dots,a_n)\in A$ denotes the joint action of all agents. $P(\boldsymbol{s}'|\boldsymbol{s},\boldsymbol{a}): S \times A \times S \to [0,1] $ is the \textbf{state transition function} that describes how the environment state evolves, given the actions taken by the agents.

We consider an episode that is divided into a series of timeslots $t = 1,2,\ldots$; at the start of each episode, agents respawn in the center of the map. Within each timeslot, each agent can take an \textit{action}, e.g., sharing resources with another agent or working towards a goal. 
%\carlee{is this right? Or is the action simply deciding which direction to travel in?} \hq{Yes. The action space includes moving to different directions, sharing, do, crafting tools, etc.}
Agents decide their action based on their observations, which are contained in the \textbf{observation} space $\Omega$, and $O(\boldsymbol{s}, i): S \times I \to \Omega$ denotes the function that maps from the joint state space to distributions of observations for each agent $i$.
Each agent's observations, as shown in Figure~\ref{fig:framework}, include its own environment input, as well as communication messages from the other agents. We use $g: \Omega \to M$ to denote the \textbf{communication message generation function} that each agent $j$ uses to encode its local observation $o_j$ into a communication message for other agents $i \neq j$. 
We use $\boldsymbol{m_{-i}}=\{m_{j}=g(o_j), \forall j \neq i\}$ to denote the collection of messages agent $i$ receives from all other agents $j \neq i$. 

 
% $\Omega$ is the \textbf{observation} space. $O(\boldsymbol{s}, i): S \times I \to \Omega$ is a function that maps from the joint state space to distributions of observations for each agent $i$. 
In deciding which actions to take, the agents' goal is to maximize the long-term reward. More formally, they aim to find a policy $\pi$ that maximizes the average expected return $\lim_{T \to \infty} (1/T) E_{\pi} [{\sum_{t=0}^T R_{t}}]$, where $R(\boldsymbol{s}, \boldsymbol{a}): S \times A \to \mathbb{R}$ is the reward of the current state $\boldsymbol{s}$ and joint action $\boldsymbol{a}$ and $R_t$ is the reward incurred in timeslot $t$. As shown in Figure~\ref{fig:framework}, this policy goal is enforced in our framework by including it in a prompt that is fed to a \textbf{multi-modal large language model (MLLM)} along with a prompt to generate plans and actions for the current timestep, thus forming the policy $\pi$. For example, Agent 6 in Figure~\ref{fig:framework} is told to find a diamond.
% and $\gamma$ is the discount factor. 
To ensure the LLM finds a good policy based on historical data, each agent maintains its own memory, consisting of  both \textbf{Short-Term Working Memory (\textbf{STWM})} and \textbf{Long-Term Memory (LTM)}. The STWM holds information for decision-making at the current timestep, combining current environmental perceptions with relevant information retrieved from LTM. The STWM is then included in the MLLM prompt. % fed into a \textbf{multi-modal large language model (MLLM)} along with a prompt to generate plans and actions for the current timestep, thus forming the policy $\pi$. 
The STWM and MLLM responses are then consolidated into the agent’s LTM, enabling agents to make strategic decisions based on historical context.

% In Dec-POMDP with communications, each agent $i$ considers an individual policy $\pi_i(a_i|o_i,\boldsymbol{m_{-i}})$ conditioned on local observation $o_i$ and messages $\boldsymbol{m_{-i}}$, i.e., $\pi=[\pi_i(a_i|o_i,\boldsymbol{m_{-i}}),\forall i]$.
% The objective is to find a policy $\pi$ that maximizes the average expected return $J(\pi) =\lim_{T \to \infty} (1/T) E_{\pi} [{\sum_{t=0}^T R_{t}}]$. The core system is structured to enable agents to learn from past interactions and \textcolor{orange}{transfer} those experiences to \textcolor{orange}{learn to collaborate in }new scenarios.

% \textcolor{orange}{Jingdi: I added this section, but I think the message generation function could not be highlighted here.}
% \carlee{Maybe we can define it and say it is connected to the communication protocol}

\begin{figure}[h]
  \centering
  \includegraphics[width=1\linewidth]{AnonymousSubmission/LaTeX/figures/framework.pdf}
  \caption{Framework Overview. Multiple agents respawn on the map and interact with each other through a memory system and communication protocol, aiming to collect a diamond as fast as possible.}
  \label{fig:framework}
  %\Description{Framework Overview.}
\end{figure}

% \subsection{Framework Overview}
% Our decentralized cooperative agents operate within a modified \textit{Dec-POMDP} framework, where each agent receives partial observations and makes decisions independently. Agents collaborate by sharing resources and updating each other on their goals and progress. The core system is structured to enable agents to learn from past interactions and \textcolor{orange}{transfer} those experiences to \textcolor{orange}{learn to collaborate in }new scenarios.

% \carlee{Integrate this with Section 3.1 (this seems to explain how agents take actions, while 3.1 explains how the environment evolves)}
% At the core of our framework is the interaction between working memory and long-term memory. Figure \ref{fig:framework} shows the framework. At the start of each episode, agents respawn in the center of the map. Each agent maintains its own memory, consisting of two components:  \textcolor{orange}{\textbf{Short-Term Working Memory (\textbf{STWM})}} and \textcolor{orange}{\textbf{Long-Term Memory (LTM)}}. The \textbf{STWM} holds information for decision-making at the current timestep, combining current environmental perceptions with relevant information retrieved from LTM. The \textbf{STWM} is fed into a \textbf{multi-modal large language model (MLLM)} along with a prompt to generate plans and actions for the current timestep. The \textbf{STWM} and MLLM responses are then consolidated into the agent’s LTM, enabling agents to make strategic decisions based on historical context. \carlee{refer to Sections 3.3 and 3.4 here for details of the memory structure}

\subsection{Adaptive Knowledge Graph Memory System}\label{sec:memory}
Recent work in multi-task learning has demonstrated the benefits of integrating heterogeneous data sources for optimized decision-making \cite{baltruvsaitis2018multimodal, ngiam2011multimodal, xu2024predicting}. In the proposed \textbf{Adaptive Knowledge Graph Memory System (A-KGMS)}, inspired by human cognitive processes \cite{sumers2023cognitive}, each agent uses a \textit{multi-modal memory system} combining short-term and long-term memories that facilitates storing and retrieving experiences across different memory types. While existing memory systems focus on aspects like semantic understanding \cite{li2024optimus}, our system is goal-oriented.
%\carlee{How does this compare to existing LLM memory systems (do any exist)?} 
This memory system allows agents to learn from past experiences, facilitating task completion in open-world environments. %\carlee{how is the memory shared across agents?} %is essential for enabling agents to learn from past interactions and apply those experiences to new scenarios.
% \carlee{refer to the figure more in explaining short term and long term memory}

%As illustrated in Figure~\ref{fig:memory_system}, the memory system is divided into two main components: \textit{working memory} and \textit{long-term memory}. The working memory captures immediate environmental inputs, while the long-term memory retains historical experiences and knowledge. We describe the components of our memory system in detail:

\begin{figure}[h]
  \centering
  \includegraphics[width=1\linewidth]{AnonymousSubmission/LaTeX/figures/memory.pdf}
  \caption{Memory System. 
  The system consists of \textit{working memory} and \textit{long-term memory}. \textit{Sensory inputs} (1) are captured in \textit{working memory} (2), alongside relevant information retrieved from \textit{long-term memory} (4). The agent 'thinks' using an \textit{MLLM} (3) to generate responses and action plans, which are then stored in long-term memory. A \textit{consolidation process} updates the \textit{goal-oriented hierarchical knowledge graph} (5), linking new experiences to past events. This graph comprises \textit{experience nodes} $E$, \textit{goal nodes} $G$, and \textit{long-term goal nodes} $LTG$.}
  \label{fig:memory_system}
  \vspace{-0.1in}
  %\Description{Memory System.}
\end{figure}
%The system is divided into two main components: working memory and long-term memory. The \textit{sensory inputs} (1) from the environment are captured in the \textit{working memory} (2), along with relevant information retrieved from the long-term memory (4). The agent then "thinks" by feeding the working memory into a \textit{MLLM} (3) along with a prompt to generate responses and action plans. The responses, along with the working memory, are then stored in the experience pool in \textit{long-term memory} (4). A consolidation process is then triggered to update the \textit{goal-oriented hierarchical knowledge graph} (5), connecting the current experience with past events. The knowledge graph consists of experience nodes $E$, goal nodes $G$, and long-term goal nodes $LTG$.
\textbf{Experience.} The \textbf{experience} for each time step in a learning episode consists of two stages: \textbf{pre-stage} and \textbf{post-stage}, as shown in Parts 2 and 3 of Figure~\ref{fig:memory_system} The \textbf{pre-stage} refers to the information available to the agent at the current timestep for decision-making. The \textbf{post-stage} is the thought process generated by the language model, then consolidated into \textbf{Long-Term Memory}. The post-stage contains full information, including environment cues and the agent's thoughts, which help generalize actions in similar scenarios by emphasizing decision-making and consequences.
%\carlee{explain the rationale for this design}

%Each episode consists of multiple timesteps, and during each timestep, an \textbf{experience} is created, which consists of two stages: \textbf{pre-stage} and \textbf{post-stage}. The information that is available for the agent to use in making a decision at the current timestep is the pre-stage. The thought process, which is the response generated from the language model \textcolor{blue}{[what types of questions does the prompt ask the agents?]}, is referred to as the post-stage, which is then consolidated into the long-term memory.

\textbf{Short-Term Working Memory (STWM, Part 2 of Figure~\ref{fig:memory_system}).} STWM refers to the pre-stage experience and consists of four parts: (\rom{1}). \textbf{Sensory memory} captures raw environmental observations, such as visual inputs and communication messages; (\rom{2}). \textbf{Episodic memory} stores contextual details, including the agent's health, location, time, and inventory; (\rom{3}). \textbf{Feedback}, retrieved from long-term semantic and procedural memory, provides available actions and their prerequisites; (\rom{4}). \textbf{Retrospection} offers context from the hierarchical knowledge graph, including recent events, achievements, goals, and progress. STWM, along with a prompt, is processed by a multi-modal large language model (MLLM) to help the agent `think' and `plan' its next action.

%\textbf{\textcolor{orange}{Short-Term Working Memory (\textbf{STWM})}} The working memory is also referred to as the pre-stage experience. The \textbf{sensory memory} refers to the raw observations from the environment, including visual input and communication messages. The \textbf{episodic memory} stores the \textcolor{blue}{episode's} contextual information, including the agent's health stats, location, time, and inventory items. The \textbf{feedback} \textcolor{blue}{is the agent's available actions and these action's prerequisites. This information} is retrieved from the long-term memory, specifically from the semantic memory and procedural memory. %, which provide the agent's available actions and their prerequisites. 
%Finally, \textbf{retrospection} contains information retrieved from the hierarchical knowledge graph from the long-term memory, providing more contextual information, such as recent events, past accomplishments, goals, and current progress. The working memory is fed into a multi-modal large language model (MLLM) along with a prompt, allowing the agent to "think" and make plans to determine the action to take.

\textbf{Long-Term Memory (LTM, Part 4 of Figure~\ref{fig:memory_system}).} LTM consists of an experience pool of post-stage experiences. A consolidation process updates the goal-oriented hierarchical knowledge graph (further explained below) by organizing experiences according to their goals, connecting current experiences with past events and allowing agents to access memories useful to their short- and long-term goals.
%\carlee{and allowing agents to access memories useful to their short- and long-term goals}. 
\textbf{Semantic memory} holds factual knowledge, specifically the hierarchical crafting tree of the environment, which is programmed explicitly using logical expressions. This factual knowledge provides accurate feedback on action prerequisites, 
%\textcolor{blue}{[what kinds of factual knowledge, and how was it obtained?]} about the environment, providing accurate feedback on action prerequisites, 
while \textbf{procedural memory} stores all available actions. The consolidation process is triggered whenever a new experience is added, updating the hierarchical knowledge graph.

%\textcolor{orange}{\textbf{Long-Term Memory (LTM)}} The long-term memory is composed of an \textbf{experience pool} of post-stage experiences. A consolidation process is then triggered to update the \textbf{goal-oriented hierarchical knowledge graph} \textcolor{blue}{through organizing the post-stage experiences according to their respective goals}, hence connecting the current experience with past events. The \textbf{semantic memory} consists of factual knowledge \textcolor{blue}{[what kinds of factual knowledge, and how was it obtained?]} about the environment that can provide accurate feedback to the agent on the prerequisites of actions. The \textbf{procedural memory} retains all available actions in the environment. The consolidation process happens when a new experience is added to the long-term memory, which involves updating the hierarchical knowledge graph.

\textbf{Goal-Oriented Hierarchical Knowledge Graph (Part 5 of Figure~\ref{fig:memory_system}).} % As shown in Figure~\ref{fig:memory_system}, 
The agent maintains an adaptive goal-oriented hierarchical knowledge graph within its LTM. Each node represents an experience ($E$), and nodes are linked sequentially based on goal-related sequences, reflecting the agent's progress. We link each experience node to a goal node corresponding to the goal it tries to achieve, derived from the LLM output.
%\carlee{We link each experience node to a goal node corresponding to the goal it tries to achieve, derived from the LLM output.} % A specific algorithm \textcolor{blue}{[state the algorithm or technique that helps us link the experience according to goals]} links \textbf{experience} to goals. 
When a new goal begins, a new \textbf{goal node} ($G$) is created and connected to the previous one, forming a sequence that tracks the agent's journey. A higher-level \textbf{Long-Term Goal node} ($LTG$) is generated from goal nodes, providing an overview of the agent’s long-term progress. At the end of the \textbf{consolidation process}, a summary is updated for the most recent goal node, including the long-term goal, current goal, past goals, and recent experiences. \textbf{At the planning stage}, the agent retrieves information from the most recent goal node ($G$) and combines it with pre-stage experiences $\boldsymbol{E}$ to form its STWM. This enables the agent to reason and make decisions by integrating past and present data, as well as adjusting strategies in real-time to optimize progress toward current and long-term goals.

%\textbf{Goal-Oriented Hierarchical Knowledge Graph.} As depicted in Figure~\ref{fig:memory_system}, the agent maintains an adaptive goal-oriented hierarchical knowledge graph (Part 5 of Figure~\ref{fig:memory_system}) within its long-term memory. Each node in the knowledge graph represents an experience ($E$), and nodes are connected based on goal-related sequences. When the agent is working on a goal, experience nodes ($E$) are linked sequentially, \textcolor{blue}{[state the algorithm or technique that helps us link the experience according to goals]} reflecting the agent's progress on that goal. Upon starting a new goal, a goal node ($G$) is created and linked to the previous goal node, forming a sequence of goals that records the agent’s journey. On top of the goal node is the long-term goal node ($LTG$), which is generated based on goal nodes in a similar way. Long-term goal nodes provide a higher-level view of the agent’s overall progress, guiding the agent toward its long-term objectives. At the end of the consolidation process, a summary will be updated for the most recent goal node: long-term goal, current goal, past accomplished goal, and recent experiences toward completing the current goal.

%When planning, the agent retrieves relevant information from the most recent goal node and combines it with pre-stage experiences to form the working memory. This helps the agent reason and make decisions based on both past and present information, adjusting strategies in real-time and optimizing progress toward both current and long-term goals.

\subsection{Structured Reasoning Output}\label{sec:output}
%Converting unstructured inputs into structured data is crucial for developing multi-step agent workflows that enable LLMs to perform actions \cite{pokrass2023structured}. Structured outputs provide a framework that constrains language models to adhere to predefined schemas. In our reasoning process, we utilize structured prompting techniques to achieve this. We employ a carefully tuned structured output format along with an environment explanation as the prompt. This prompt organizes the working memory components into actionable insights, enabling the agent to generate well-informed decisions.
% \carlee{Is this about how agents process the outputs of the LLM?}
%\textcolor{blue}{[maybe some brief examples of what unstructured input (free flow text?) vs structured data/outputs are. cause the above paragraph is kinda abstract. an alternative is to put the specifics described below first.]}
%\textcolor{blue}{[if this structured output helps reduce communication required, we could mention it too.]}

Converting unstructured inputs, such as free-form text, into structured data is crucial for developing multi-step agent workflows that enable LLMs to perform actions \cite{pokrass2023structured}. Structured outputs provide a framework that constrains language models to follow predefined \textbf{schemas}. For example, instead of processing unstructured text like \textit{`The agent moved north to pick up a key'}, we format it into structured data such as \textit{`[Action: Move North, Reason: Pick up a key]'}. We utilize structured prompting techniques, combining a carefully tuned output format with environment explanations, to organize working memory into actionable insights. This reduces communication needs and helps the agent make well-informed decisions. Meanwhile, the number of output tokens is significantly reduced due to formatted and focused responses, resulting in faster generation speed.

\textbf{Schemas.} The schemas are built around three core components: (\rom{1}) \textbf{Reflection}, which enables agents to review recent actions, summarize outcomes, and reflect on lessons learned to adjust future strategies; (\rom{2}) \textbf{Goal}, which tracks both current and long-term objectives, including sub-goals and progress updates, helping the agent stay focused and break down tasks into manageable steps; and (\rom{3}) \textbf{NextAction}, which determines the agent’s upcoming actions and the reasoning behind them, evaluating prerequisites and ensuring alignment with both short-term and long-term goals. Each component is represented by a data class with fields specifying required responses and data types, using the Python \textit{Pydantic} library.

%\textbf{Schemas.} The schemas are defined by three core components: NextAction, Reflection, and Goal. Each component is represented by a data class with fields specifying required responses and data types using the Python Pydantic library.

% \mycodebox[red!20!white]{
% class NextAction(BaseModel):\\
% \hspace*{5mm}next\_action: ActionType\\
% \hspace*{5mm}next\_action\_reason: str\\
% \hspace*{5mm}final\_next\_action: ActionType\\
% \hspace*{5mm}final\_next\_action\_reason: str\\
% }

% \mycodebox[blue!20]{%
% class Reflection(BaseModel):\\
% \hspace*{5mm}vision: list[MaterialType]\\
% \hspace*{5mm}last\_action: ActionType\\
% \hspace*{5mm}last\_action\_result: ResultType\\
% \hspace*{5mm}last\_action\_result\_reflection: str\\
% \hspace*{5mm}last\_action\_repeated\_reflection: str\\
% }

% \mycodebox[yellow!20]{%
% class Goal(BaseModel):\\
% \hspace*{5mm}ultimate\_goal: LongTermGoalType\\
% \hspace*{5mm}long\_term\_goal: LongTermGoalType\\
% \hspace*{5mm}long\_term\_goal\_subgoals: str\\
% \hspace*{5mm}long\_term\_goal\_progress: GoalType\\
% \hspace*{5mm}current\_goal: GoalType\\
% \hspace*{5mm}current\_goal\_reason: str\\
% }

%The \textbf{Reflection} component enables agents to utilize their working memory by reviewing their most recent actions. It summarizes these actions, their outcomes, and the agent’s reflections on the results, helping identify lessons learned and adapt future strategies accordingly. The \textbf{Goal} component tracks both the agent’s current and long-term objectives, including sub-goals and progress updates. This helps the agent stay focused on overarching goals while managing immediate tasks. By linking current objectives with long-term plans, this component allows agents to articulate their goals and break them down into manageable sub-goals. The \textbf{NextAction} component determines the agent's upcoming actions and the reasoning behind them. It allows agents to evaluate the prerequisites for their next move, why they chose it, and how it aligns with both their current and long-term goals.
\subsection{Structured Communication System}\label{sec:communication}
\begin{figure}[h]
  \centering
  \includegraphics[width=0.8\linewidth]{AnonymousSubmission/LaTeX/figures/communication.pdf}
  \caption{Communication Protocol. Agents collaborate by exchanging messages to coordinate tasks and share resources. An arrow from agent $i$ to agent $j$ indicates that agent $i$ is helping agent $j$; communication then flows in the opposite direction.}%\carlee{An arrow from agent $i$ to agent $j$ indicates that agent $i$ is helping agent $j$; communication then flows in the opposite direction.}}
  \label{fig:communication_protocol}
  %\Description{Communication protocol.}
    \vspace{-0.1in}
\end{figure}

%In a multi-agent environment, communication between agents is crucial for achieving efficient cooperation and collaboration. Our communication framework allows agents to share their current status, resource availability, crafting progress, and requests for assistance, with a hierarchical focus—each agent prioritizes helping its preceding agent.

In a multi-agent environment, communication is key for effective cooperation. Our communication framework, consisting of message generation modules $g=\{g_1,\dots,g_n\}$ for all agents, where $m_i = g_i (o_i,rs_i,c_i,rq_i)$, enables agents to share their current observations $o_i$, includes status $s_i$, resource availability $rs_i$, short-term goal %\carlee{this should be more generic, maybe call it ``current short-term goal''} 
$c_i$, and assistance requests $rq_i$. This follows a hierarchical structure, where each agent $i$ prioritizes helping the preceding agent $i-1$.

We propose a novel \textbf{Collaboration} schema $\boldsymbol{C_i}=\Phi(h_i, I_i, \Delta p_i)$ for each agent $i$ and add this to the structured outputs, which is based on the target agents $h_i$ who needs help from agent $i$, intentions $I_i$ to assist target agents from agent $i$, and how the collaboration impacts agent $i$'s current plan, denoted by $\Delta p_i$. In our multi-agent system, the message generation function $g_i$ can be augmented by incorporating the collaboration schema $\boldsymbol{C_i}$ to refine and guide the message generation process, then the message generation process is enhanced by the information encoded in $\boldsymbol{C_i}$, i.e., $m_i = g_i (o_i,rs_i,c_i,rq_i, C_i)$. Therefore, the Collaboration schema enables agents to interpret and generate actions $a_i=\pi_i(o_i,\boldsymbol{m}_{-i})$, where $\boldsymbol{m_{-i}}=\{m_{j}=g(o_j), \forall j \neq i\}$ to denote the collection of messages agent $i$ receives from all other agents $j \neq i$. This structure ensures that our collaborative agents act in a goal-oriented manner with collaboration as a key consideration. 

%A new schema, \textbf{Collaboration}, has been added to the structured output. This schema enables agents to focus on interpreting and generating actions based on who needs help, how to help, and how the collaboration affects their current plan. This ensures that agents act in a goal-oriented manner, considering collaboration.

%\textcolor{orange}{As illustrated in Figure~\ref{fig:communication_protocol}, agents collaborate by communicating and sharing resources, using message generation modules $g={g_1, \dots, g_n}$ to coordinate actions such as task allocation and resource sharing. Agents are ordered from 1 to $n$, with each agent $i$ responsible for assisting the preceding agent $i-1$ and the leader agent $1$ with communication module $g_1$. The first agent's communication module, $g_1$, acts as the leader, tasked with crafting essential tools and distributing them to other agents as needed. The second agent's communication module, $g_2$, focuses on gathering materials and aiding $g_1$ with crafting tasks. The last agent's communication module, $g_n$, is responsible for supporting $g_{n-1}$ and eventually shifting its focus to finding a diamond, deciding when to switch from assisting other agents to locating the diamond based on the collaboration schema $\boldsymbol{C}_n=\Phi(h_n, I_n, \Delta p_n)$. This protocol is simple yet effective in a hierarchical environment, parallelizing tasks and encouraging cooperation among agents while maintaining low communication costs. Since the environment is hierarchical, this collaborative approach effectively speeds up the crafting process and naturally scales with any arbitrary number of agents $n$.}

\textbf{An Illustrative Example.} As illustrated in Figure~\ref{fig:communication_protocol}, agents collaborate by communicating and sharing resources through message generation modules $g={g_1, \dots, g_n}$ to coordinate tasks like allocation and resource sharing. Agents are ordered from 1 to $n$, with each agent $i$ assisting the preceding agent $i-1$ and the leader agent $1$. The first agent, acts as the leader, crafting essential tools and distributing them to others. The second agent gathers materials and assists the agent $1$ with crafting. The last agent $n$, supports agent $n-1$ and eventually shifts its focus to finding a diamond, deciding when to switch goals using the collaboration schema $\boldsymbol{C}_n=\Phi(h_n, I_n, \Delta p_n)$. This simple yet effective protocol works in hierarchical environments by parallelizing tasks, fostering cooperation, and keeping communication costs low. It naturally scales with any number of agents $n$, speeding up the crafting process.

%As illustrated in Figure~\ref{fig:communication_protocol}, agents collaborate by communicating and sharing resources. They exchange messages to coordinate actions such as task allocation and resource sharing. Agents are ordered from 1 to $N$, with each agent designated to assist the preceding agent and the leader agent. The first agent is the leader agent, responsible for crafting tools and sharing them with other agents in need. \textcolor{blue}{[personal take: can be even more specific, on what tasks the first and second agent does]} The last agent is tasked with helping its previous agent and finding a diamond, determining when to switch its goal from assisting agents to locating the diamond. This protocol is simple yet effective in a hierarchical environment, parallelizing tasks and encouraging cooperation among agents while maintaining low communication costs. Since the environment is hierarchical, this collaborative approach effectively speeds up the crafting process and naturally scales with any arbitrary number of agents.
%\textcolor{blue}{[personal take: can be even more specific, on what tasks the first and second agent does]} 
%\textcolor{blue}{[it may not scale at times right, depending on the task. sometimes less is better]}

% \subsection{Libraries and Packages}
% Our implementation leverages several key libraries and packages to build the decentralized cooperative generative agents framework. We use Python 3.10.14, along with Pydantic 2.9.2 for structured output, OpenAI 1.44.1, and Torch 2.3.1. Additionally, we use GPT-4o (2024-08-06 version) as the backbone language model. \carlee{I'd move this to the experiments section or appendix. Also say here that we will release this code publicly}

% \textcolor{orange}{Jingdi: could move this to appendix}

% \mycodebox[green!20]{%
% class ResponseEvent(BaseModel):\\
% \hspace*{5mm}episode\_number: int\\
% \hspace*{5mm}timestep: int\\
% \hspace*{5mm}past\_events: str\\
% \hspace*{5mm}current\_inventory: list[InventoryItemsCount]\\
% \hspace*{5mm}collaboration: Collaboration\\
% \hspace*{5mm}reflection: Reflection\\
% \hspace*{5mm}goal: Goal\\
% \hspace*{5mm}action: NextAction\\
% \hspace*{5mm}summary: str\\
% }

\hq{

\section{Evaluation Challenges of LLM Agents}\label{sec:evaluation_challenges}
Evaluating LLM-powered multi-agent systems presents unique challenges. Unlike MARL-based agents, which are trained to optimize carefully crafted rewards, LLM agents rely on prompts and contextual information, making them highly adaptable but sensitive to the evaluation environment.

\textbf{Limitations of Existing Environments.} Existing multi-agent benchmarks are often too simple for meaningful collaboration~\cite{terry2021pettingzoo} or too complex~\cite{berner2019dota,vinyals2019grandmaster,fan2022minedojo}. Many focus on micro-level action management, whereas our work emphasizes macro-level planning, communication, and cooperation. Furthermore, MARL frameworks are known for scalability challenges, and existing environments are often not designed to support cooperative tasks that scale well with an increasing number of agents.

\textbf{Evaluation of Cooperation.} LLM-based collaboration is highly adaptable but difficult to quantify. Unlike RL agents that optimize reward signals, LLM-based collaboration relies on context and commonsense reasoning, making responses variable. No standardized metric exists for evaluating cooperation among LLM agents, and extensive modifications to benchmarks are often required. Testing with environment-specific prompts is also time-consuming.

\textbf{Quantifying LLM Agents' Capabilities.} Evaluating memory quality and adaptability in LLM agents is non-trivial. While our \textbf{A-KGMS} organizes past experiences, determining the quality of stored information and its impact on decision-making remains challenging. Adaptability is also difficult to measure, as LLM agents adjust dynamically rather than optimizing predefined objectives.

To address these challenges, we introduce Multi-Agent Crafter to evaluate strategic coordination, planning, and resource sharing in open-ended, scalable cooperative tasks.
}

\section{Multi-Agent Crafter: A Novel Testbed}\label{sec:crafter}
% \subsection{Environment Details} 
%The original Crafter environment \cite{hafner2021benchmarking} is a procedurally generated, open-world survival game designed for benchmarking reinforcement learning (RL) algorithms. It features a grid world with a discrete action space of size 17 and provides information on the player's inventory, health, food, water, and crafting progress. Crafter includes 22 achievements organized in a tech tree with a depth of 7, with the ultimate goal of exploring the environment. Inspired by Minecraft, Crafter simplifies the game’s mechanics to enable faster experimentation and results collection.

The original Crafter environment \cite{hafner2021benchmarking} is a procedurally generated, open-world survival game used to benchmark RL algorithms. It features a 17 discrete action grid world and tracks player metrics like inventory, health, and crafting progress, with 22 achievements organized in a 7-depth tech tree. Inspired by Minecraft, Crafter simplifies game mechanics for faster experimentation and results collection.
\hq{We proposed a novel multi-agent Crafter for multi-agent tasks, enabling cooperative agent interaction and introducing new actions and challenges. These changes, shown in Figure \ref{fig:crafter}, make the environment suitable for studying multi-agent cooperation. Key modifications are outlined below.}

%We have made several significant modifications to transform Crafter into a robust multi-agent environment. These changes allow agents to interact cooperatively within the environment and introduce additional challenges, making it more suitable for studying multi-agent cooperation, as shown in Figure \ref{fig:crafter}. The key changes are outlined below:
\hq{
\textbf{A Scalable Cooperative Environment.} We extended the Crafter environment to support an arbitrary number of agents, each with independent observations, inventories, and health stats, enabling cooperative agent interaction and introducing new actions and challenges (Figure~\ref{fig:crafter}). Agents can collaborate by sharing resources, coordinating actions, and balancing individual roles to achieve collective goals efficiently. Unlike traditional MARL environments, which often focus on micro-level action management, our testbed is designed to evaluate strategic planning, coordination, and shared decision-making.

Our environment allows agents to share items, including resources and tools, fostering teamwork by enabling task delegation and resource management. Crafting dependencies and environmental prompts can be easily customized, increasing task complexity with more participants. This ensures that agents must coordinate and efficiently allocate roles, enabling effective large-scale parallel collaboration. The flexible design makes the testbed suitable for evaluating cooperative behavior potentially for any number of agents.

\textbf{Evaluation of Cooperation and LLM Agents' Capabilities.}  
Unlike the original Crafter environment, which focused on open-ended exploration, we define a clear objective: agents must collaborate to craft necessary tools and obtain a diamond as quickly as possible while managing their needs for food, water, and energy. This setup allows us to evaluate whether agents can effectively cooperate and reason toward both short- and long-term goals, making the environment ideal for testing multi-agent coordination, planning, and resource optimization.

To assess cooperative efficiency, agents share resources and tools, requiring negotiation, task division, and decision-making. Unlike previous MARL settings, where collaboration is forced or predefined, our testbed allows agents to develop teamwork strategies. Our environment quantifies multi-agent cooperation through indirect measurements, such as tracking the steps an agent takes to craft items, providing insights into decision-making and adaptability.

\textbf{Support for Language Agents.} We added a navigation skill that allows agents to move toward specific resources, reducing the burden of manual low-level movement control. This enables agents to focus on higher-level decision-making, such as strategic planning and collaboration.

\textbf{Customizability and Compatibility.}
Our multi-agent Crafter environment is designed to be highly flexible and extensible, supporting RL, MARL, and LLM-powered agents. The single-agent version follows the Gymnasium API, ensuring integration with standard RL libraries, while the multi-agent version aligns with the PettingZoo API, ensuring compatibility with existing MARL frameworks. We provide example training scripts for single-agent experiments using Stable-Baselines3 (SB3) and multi-agent experiments using AgileRL, allowing researchers to efficiently test new ideas, integrate with existing RL libraries, and adapt the environment for diverse multi-agent challenges.
}


% While this paper evaluates our decision-making framework using the multi-agent Crafter environment, we believe it will also serve as a valuable platform for future research on multi-agent coordination, communication, long-term planning, and resource optimization in complex, real-time multi-agent environments.









\section{Evaluations}\label{sec:evaluation}
% \section{Simulation Evaluation \& Results}\label{sec:results}

\subsection{Baseline Planners}

To evaluate the performance of \PlannerName, we compare it against several baseline methods. In the following section, we describe these baselines, their implementation details, and their respective advantages and limitations, particularly in the context of information gathering in large, high-dimensional search spaces. The simulation framework and vehicle parameters remain consistent across all planners, and each method is allowed to replan during testing.

\subsubsection{Monte-Carlo Tree Search}

Monte Carlo Tree Search (MCTS) can be a powerful technique for finding feasible and optimal paths in complex environments. It is a heuristic search algorithm that builds a search tree incrementally through repeated simulations. At each iteration, it selects a node to explore based on a selection policy (often the Upper Confidence Bound or UCB1 algorithm), expands the tree by adding possible actions from that node, runs a simulation from the newly added node, and updates the statistics of nodes along the path traversed during the simulation. 

The UCB1 (Upper Confidence Bound) algorithm is a technique commonly used in the context of multi-armed bandit problems and Monte Carlo Tree Search (MCTS) for balancing exploration and exploitation. It helps in selecting actions or nodes that are likely to yield high rewards while also exploring less-frequented options to gather more information about their potential rewards. 

We formulate our UCB score in the following manner, \\
\begin{equation*}
    UCB_\text{node} = \frac{I(X_{\text{node}})}{\alpha} + C \times \sqrt{\frac{\ln(N_\text{tree})}{N_\text{node}}}
\end{equation*}
%  $
% UCB_\text{node} = \frac{\overline{X_\text{node}}}{\alpha} + C \times \sqrt{\frac{\ln(N_\text{tree})}{N_\text{node}}}
% $ \\
Here $I(X_{\text{node}})$ denotes the estimated information gain from the node, $\alpha$ denotes the normalization factor which is given by $\frac{B}{v_\text{desired}}$, $B$ being the maximum planning budget and $v_\text{desired}$ being the desired speed of our UAV. $C$ denotes the exploration weight, and $N_\text{tree}$ denotes the number of visits to the tree root node while $N_\text{node}$ denotes the number of times the present node has been visited.

After selecting a candidate node, if it has been visited before, it is expanded by applying motion primitives to generate child nodes, growing the tree. Unvisited nodes skip this step. Following expansion, either the unvisited candidate node or one of its children is selected for the simulation phase, where the future values of nodes along the path are estimated to update the total potential information gain. This informs the selection policy in subsequent iterations. Once planning time is exhausted, the path with the highest information gain is returned.

% with authors goes here
\begin{figure}[t]
\centering
\includegraphics[trim={.7cm 0cm .5cm 1.4cm},clip,width=\columnwidth]{figs/5_/Results1v3.pdf}
\caption{The Monte Carlo simulation results for the planners. The plots show the average percent reduction in entropy over the course of the simulations, and the shading shows the 95\% confidence intervals. IA-TIGRIS outperforms all of the baselines.}
\label{fig:mc_results}
\end{figure}

While MCTS is probabilistically guaranteed to converge to the optimal path \cite{mcts_ref_1}, it is constrained to actions within a predefined set of motion primitives. Its reliance on random sampling to estimate the future value of nodes can result in poor approximations, particularly in environments with sparse, localized pockets of high information gain. This limitation is especially pronounced in large search areas or scenarios with large budgets constraints, where estimating future node values becomes increasingly expensive. As a result, in such scenarios, MCTS is often implemented with a finite planning horizon, which can restrict its ability to account for long-term consequences or dependencies in the environment.

% This property of MCTS, which causes unguided exploration of the environment, leads to increased convergence times on the optimal path, as a result of a lot of budget being spent in exploring information sparse areas of the map. 
% Also, the computation time of MCTS increases exponentially with the depth of the search tree. The time complexity of MCTS is given by $\mathcal{O}(\frac{T}{t_\text{iter}} \cdot |A|^d)$. Here, $T$ is the total planning time and $t_\text{iter}$ is the time taken per iteration of the planning loop. $|A|$ is the number of actions and $d$ represents the average depth of the search tree. 

% The above limitations are not inconsequential in the context of performing informative path planning in large high-dimensional search spaces. We compare MCTS with \PlannerName, in \ref{}, and empirically demonstrate its drawbacks and how \PlannerName, is able to outperform MCTS in the context of the mission parameters we examine in this work.  

\subsubsection{Greedy}

For the greedy planner, we iterated through each cell within the search bounds and calculated the reward for a given cell $i$ as $g_i = R(X_i) / d_i$ where $R(X_i)$ is given through \eqref{equ:reward} and $d_i$ represents the Euclidean distance between the current position the robot at the current time $t$ and the closest viewpoint to the cell. To compute this viewpoint, the yaw between the current pose of the robot and the intersected cell is first calculated. Using the robot's sensor configuration and this yaw, $x$ and $y$ coordinates are calculated that view the cell at the desired flight altitude. With this formulation, the planner prioritizes regions with a high ratio of entropy to distance. This can lead to locally optimal choices that contradict with paths that lead to higher information gain over the entire trajectory. 

% without authors goes here
% \begin{figure}[t]
% \centering
% \includegraphics[trim={.7cm 0cm .5cm 1.4cm},clip,width=\columnwidth]{figs/5_/Results1v3.pdf}
% \caption{The Monte Carlo simulation results for the planners. The plots show the average percent reduction in entropy over the course of the simulations, and the shading shows the 95\% confidence intervals. IA-TIGRIS outperforms all of the baselines.}
% \label{fig:mc_results}
% \end{figure}


\begin{figure*}[t]
    \centering
    \begin{subfigure}[b]{0.99\textwidth}
        \centering
        \includegraphics[trim={0cm 0.3cm 0cm 0cm},clip,width=\textwidth]{figs/5_/Fig2v1_target.png}
        % \caption{Slice by targets}
        % \vspace{.1cm}
    \end{subfigure}
    
    \begin{subfigure}[b]{0.99\textwidth}
        \centering
        \includegraphics[trim={0cm 0cm 0cm 0cm},clip,width=\textwidth]{figs/5_/Fig2v1_sigma.png}
        % \caption{Slice by sigma }
    \end{subfigure}
    \caption{A comparison of the methods based on the number of sampled prior clusters and the standard deviation of sampled prior clusters. IA-TIGRIS is most effective compared to the baselines when there is high variation in the search space. As the search space prior information becomes more evenly spread out, the performance gap between the methods tends to decrease.}
    \label{fig:targets_sigmas}
\end{figure*}

\subsubsection{Random}

The random planner operates by iteratively sampling points within the defined search bounds and calculating the minimum-cost path to observe each sampled point. This process is repeated until the available budget is fully expended. The random planner does not utilize any prior information about the environment or target distribution. Additionally, it does not optimize the sequence of actions, instead treating each sampled point independently without considering the global structure of the search problem. This simplicity allows the random planner to highlight the performance benefits of more sophisticated methods by providing a lower-bound comparison for evaluation.

\subsubsection{Coverage}

The coverage planner generates a plan that systematically covers the entire search space using a straightforward lawn-mower pattern. The spacing between each pass is set to match the width of the projected observation footprint at 20\% from the bottom, ensuring that no grid cells are missed. This spacing also maintains a distance that enables high-quality sensor measurements. However, due to the size of the search spaces considered, the coverage planner spends significant time surveying empty regions. This approach results in inefficient use of the budget, as it prioritizes full coverage with safe sensor overlap, even in areas with little or no valuable information. While simple and robust, this method highlights the tradeoff between exhaustive coverage and efficient, targeted exploration.

% \subsubsection{Branch and Bound}
% The branch and bound baseline is based on motion primitive planning. In each future step the drone has a set of motion primitives with future states and each of these future states also has a set of motion primitives. In this way, a tree can be built with multiple path candidates. The path candidate with the highest information gain will be selected and form the output. 

% By adding branch and bound, there will be an estimation of a node's upper bound information reward, using the node's current information reward, updated information map and the remaining budget. If this upper bound is already lower than the information reward of any other node in the tree, the corresponding node will be closed and not expanded in the future to accelerate the expansion of the tree. 



\subsection{Tests and Analysis}
% To evaluate the efficacy of IA-TIGRIS compared to the baseline methods, we conduct Monte Carlo testing as well as analyze how the prior and budget affect the performance of each method. In all of these test cases, there are no time-based or priority rewards and have horizon lengths set to the full budget. All tests were performed using an Intel Xeon CPU E5-2620 v4 @ 2.10GHz.
To evaluate the efficacy of IA-TIGRIS against baseline methods, we perform Monte Carlo testing and analyze the impact of the prior and budget on the performance of each method. In all test cases, rewards are calculated using \eqref{equ:reward}, and horizon lengths are set to match the full budget. The tests are conducted on an Intel Xeon CPU E5-2620 v4 @ 2.10GHz, ensuring consistent computational conditions across all evaluations.

% Random sample across which parameters.

% Quantitative ideas. Look into number and std of prior (metric for this? std of grid cell values, mediuan, mean,). 
% Uniform prior? 
% Split distinct regions, not smooth. 
% Compare to coverage and amount of time to reach specific amount. 
% Compare with different budgets. 
% Repeatability test. 
% Graph size vs time. 
% Look at coverage with different altitudes or widths. Something that shows long horizon vs not nature of things?
% Shape of search space?
% Time/budget to get x\% of all info gain. Have to do moving horizon. 
% Targets detected? 

% Key thought for results where I show time, our optimization does not optimize for time, only final value. Key thing to show across the different budgets. 

% \BM{Qualitative. Nayana idea of plot with example sampled case. Should add one here.} 



\subsubsection{Monte Carlo Testing}
Our simulated testing environment is a $5000\times5000$ m square with Gaussian-distributed prior information randomly placed throughout the search space. The number of prior clusters was sampled uniformly between $[4,20]$, with standard deviations between $[60,450]$, and maximum value between $[0.05,0.5]$. 

The results of $100$ Monte Carlo tests are shown in Fig.~\ref{fig:mc_results}. IA-TIGRIS clearly outperforms the other methods, achieving nearly a $40\%$ greater reduction in entropy than the next best method. Early in the simulation, the greedy method initially gains information more quickly, as expected, but this does not translate to better long-term performance. Since our method optimizes for total information gain, it generates paths that maximize information collection over the entire budget. MCTS performed slightly worse than the greedy approach.

The random paths slightly outperformed the coverage paths. This is likely because the lawnmower strategy requires sufficient overlap between passes to avoid missing areas, and its long straight paths often lead to redundant observations due to the UAV’s forward-facing camera. Changing the heading of the UAV is beneficial to viewing more of the search space, which may explain why random paths performed better.

We also conducted Monte Carlo tests where either the number of prior clusters or their standard deviation was held constant to analyze how variations in the information map affect planner performance. The results, shown in Fig.~\ref{fig:targets_sigmas}, include two cases: the upper figure fixes the number of priors, while the lower figure fixes their standard deviation. All other agent and simulation parameters remained unchanged.


% The first thing to note from these results is that for all tests the proportional performance gap between IA-TIGRIS and the baselines increases as the number and standard deviation of the Gaussian priors decreases. As the search space becomes more uniformly filled with entropy in the information map, the need for longer-horizon planning decreases and other simple or random approaches can perform satisfactorily given the testing budget. As the information becomes more sparsely distribution in the space, such as when the information is contained in separated pockets of areas, there is a greater need to plan longer-horizon paths that reason about the given budget.
% \BM{Could have figures here or refer to others}

Across these tests, the performance gap between IA-TIGRIS and the baselines widens as the number and standard deviation of the Gaussian priors decrease. When entropy is more uniformly distributed across the search space, simpler methods perform reasonably well within the given budget. However, when information is concentrated in sparse, distinct regions, longer-horizon planning becomes essential. In such cases, IA-TIGRIS demonstrates a significant advantage by effectively reasoning about the budget and prioritizing high-value regions.

% Show plot of first plans expected info gain versus planning time. (plans not executed)


\subsubsection{Budget Analysis}
To evaluate the impact of budget constraints on performance, we conducted additional tests beyond our initial Monte Carlo experiments, evaluating budgets of $5000$ m, $10000$ m, $30000$ m, and $60000$ m. Table~\ref{tab:budgets} summarizes the average entropy reduction across these budgets.

\definecolor{tabfirst}{rgb}{1, 0.7, 0.7} % red
\definecolor{tabsecond}{rgb}{1, 0.85, 0.7} % orange
\definecolor{tabthird}{rgb}{1, 1, 0.7} % yellow
\begin{table}[t]
    \centering
    \resizebox{\linewidth}{!}{
    \begin{tabular}{l|ccccc}
    & $5000$ m & 10000 m  & 15000 m& 30000 m& 60000 m\\ \hline

    % \hline
    IA-TIGRIS  &  \cellcolor{tabfirst}$9.41\pm1.0$ &  \cellcolor{tabfirst}$18.28\pm1.8$ & \cellcolor{tabfirst}$25.36\pm2.3$ & \cellcolor{tabfirst}$41.08\pm2.9$ & \cellcolor{tabfirst}$58.85\pm2.9$ \\
    Greedy  &  \cellcolor{tabsecond}$6.99\pm0.8$ &  \cellcolor{tabsecond}$13.10\pm1.5$ & \cellcolor{tabsecond}$17.97\pm2.0$ & \cellcolor{tabthird}$30.00\pm2.3$ & \cellcolor{tabsecond}$49.38\pm3.5$ \\
    MCTS  &  \cellcolor{tabthird}$6.06\pm0.7$ &  \cellcolor{tabthird}$11.80\pm1.1$ & \cellcolor{tabthird}$17.11\pm1.4$ & \cellcolor{tabsecond}$30.21\pm2.2$ & \cellcolor{tabthird}$48.68\pm2.7$ \\
    Random  &  $2.19\pm0.3$ & $4.29\pm0.7$ & $6.61\pm0.6$ & $17.50\pm1.2$ & $22.47\pm1.4$ \\
    Coverage  &  $1.58\pm0.3$ &  $2.82\pm0.4$ & $4.09\pm0.7$ & $12.04\pm1.9$ & $16.77\pm2.4$ \\

    \end{tabular}
    }
    \caption{Monte Carlo testing results given different budgets. The values are the average percent reduction in entropy and the 95\% confidence bounds. \mbox{IA-TIGRIS} had the best performance for all budgets.}
    \label{tab:budgets}
\end{table}
%$\uparrow$ 

IA-TIGRIS consistently achieved the highest entropy reduction across all budget constraints, with a statistically significant margin over alternative methods. Greedy generally ranked second but was slightly outperformed by MCTS at the $30000$ m budget level. Greedy and MCTS exhibited comparable performance throughout the tests, with their results closely tracking each other. Consistent with our previous findings, Random and Coverage methods yielded the lowest results.


Among the tested methods, only IA-TIGRIS and MCTS explicitly incorporate budget constraints into their planning algorithms. Notably, at lower budgets ($5000$ m and $10000$ m), these methods achieved higher entropy reduction compared to the equivalent time steps ($200$ s and $400$ s) in the $15000$ m budget scenario shown in Fig.~\ref{fig:mc_results}. This improved performance stems from IA-TIGRIS's optimization of total path reward under budget constraints, contrasting with the myopic next-best-action approach of the greedy method. The remaining methods---Greedy, Random, and Coverage---maintain consistent behavior regardless of budget constraints, as their planning strategies do not account for resource limitations.


The performance gap between IA-TIGRIS and the next-best method varied with budget size, showing margins of $34.6\%$, $39.5\%$, $41.1\%$, $36.0\%$, and $19.2\%$ in ascending budget order. This gap widened through the first three budget levels as problem complexity increased, before declining significantly at higher budgets. This performance pattern suggests that implementing a planning horizon could enhance efficiency by limiting tree search depth, enabling the planner to prioritize path quality optimization over exhaustive space exploration.


% percent improved from next best
% 34.6, 39.5, 41.1, 36.0, 19.2
% reasons, too long horizon is a larger search space, so less quality paths closer. Or larger horizon, more packing in


% with authors goes here
\begin{figure}[t] 
    \centering
    \renewcommand\arraystretch{0} % Adjust the height between rows here
    \setlength{\tabcolsep}{1pt} % Adjust the column separation here
    \begin{tabular}{c}
        \begin{tikzpicture}
            \node[anchor=south west, inner sep=0] (image) at (0,0) {
                \includegraphics[width=0.9\linewidth]{figs/5_/google_earth_prior.png}
            };
            \begin{scope}[x={(image.south east)},y={(image.north west)}]
                % \fill[OrangeRed] (0.02, 0.03) circle (2pt); 
                % \fill[OrangeRed] (0.51, 0.04) circle (2pt); 
                % \fill[OrangeRed] (0.61, 0.04) arc (0:90:2pt); 
                \fill[Orange, opacity=0.8] (0.74, 0.45) circle (3pt); % Adjust 
                \fill[Orange, opacity=0.8] (0.27, 0.42) circle (3pt); % Adjust 
                \fill[Orange, opacity=0.8] (0.39, 0.63) circle (3pt); % Adjust 
            \end{scope}
        \end{tikzpicture} \\
        % \includegraphics[width=0.9\linewidth]{figs/5_/google_earth_prior.png} \\
        \\
        \includegraphics[width=0.9\linewidth]{figs/5_/google_earth_path.png} 
    \end{tabular}
    \caption{Google Earth screenshots illustrating the mission planning process and execution. Top: Areas of high entropy targeted for search are highlighted in red, representing regions with a binary occupied/unoccupied probability of 0.2. Three points of particular interest, each assigned a 0.5 probability, are marked in orange. Bottom: The executed drone flight path (yellow) shows the optimized path for maximum information gain across the search space.} 
    \label{fig:google_earth}
\end{figure}
\begin{figure}[t]
\centering
% https://docs.google.com/presentation/d/1RjI-QqHpBRLHN60UAxzmQYs4EaWaVCOoSBkEkA39kk0/edit?usp=sharing
\includegraphics[width=\columnwidth]{figs/5_/m600_labeled.jpg}
\caption{Hexarotor system (DJI M600 Pro) with onboard compute and camera. Left image shows drone on the ground, right image shows drone in flight.}
\label{fig:m600}
\end{figure}


\section{Field Deployments}\label{sec:field}


\subsection{Hexarotor Deployment}
The first field experiment that we present uses a hexarotor drone to cover an urban area shown in Fig.~\ref{fig:fig1}.
We designed this field experiment to simulate classifying where cars are within a search area.  
Hence, we set the plan request to focus on parking lots at the field test site (Fig.~\ref{fig:google_earth}, top), with the addition of three chosen grid cells within the parking lots being marked as having a higher uncertainty. The plan request boundaries and priors were created with GPS coordinates in Google Earth, exported as kml files, and then converted into our plan request message format. 

The following sections details the hardware, autonomy, and experimental results for our hexarotor deployments.

% without the authors goes here
% \begin{figure}[t] 
%     \centering
%     \renewcommand\arraystretch{0} % Adjust the height between rows here
%     \setlength{\tabcolsep}{1pt} % Adjust the column separation here
%     \begin{tabular}{c}
%         \begin{tikzpicture}
%             \node[anchor=south west, inner sep=0] (image) at (0,0) {
%                 \includegraphics[width=0.9\linewidth]{figs/5_/google_earth_prior.png}
%             };
%             \begin{scope}[x={(image.south east)},y={(image.north west)}]
%                 % \fill[OrangeRed] (0.02, 0.03) circle (2pt); 
%                 % \fill[OrangeRed] (0.51, 0.04) circle (2pt); 
%                 % \fill[OrangeRed] (0.61, 0.04) arc (0:90:2pt); 
%                 \fill[Orange, opacity=0.8] (0.74, 0.45) circle (3pt); % Adjust 
%                 \fill[Orange, opacity=0.8] (0.27, 0.42) circle (3pt); % Adjust 
%                 \fill[Orange, opacity=0.8] (0.39, 0.63) circle (3pt); % Adjust 
%             \end{scope}
%         \end{tikzpicture} \\
%         % \includegraphics[width=0.9\linewidth]{figs/5_/google_earth_prior.png} \\
%         \\
%         \includegraphics[width=0.9\linewidth]{figs/5_/google_earth_path.png} 
%     \end{tabular}
%     \caption{Google Earth screenshots illustrating the mission planning process and execution. Top: Areas of high entropy targeted for search are highlighted in red, representing regions with a binary occupied/unoccupied probability of 0.2. Three points of particular interest, each assigned a 0.5 probability, are marked in orange. Bottom: The executed drone flight path (yellow) shows the optimized path for maximum information gain across the search space.} 
%     \label{fig:google_earth}
% \end{figure}
% \begin{figure}[t]
% \centering
% % https://docs.google.com/presentation/d/1RjI-QqHpBRLHN60UAxzmQYs4EaWaVCOoSBkEkA39kk0/edit?usp=sharing
% \includegraphics[width=\columnwidth]{figs/5_/m600_labeled.jpg}
% \caption{Hexarotor system (DJI M600 Pro) with onboard compute and camera. Left image shows drone on the ground, right image shows drone in flight.}
% \label{fig:m600}
% \end{figure}

\subsubsection{Hardware System}
The hardware consists of the DJI M600 Pro, shown in Fig.~\ref{fig:m600}, along with the physical sensing and onboard computer payload. The DJI M600 Pro contains a flight controller that handles pose estimation and position-based control. The DJI M600 Pro’s flight controller also handles teleloperation if human intervention is necessary. Beneath the drone's base, we mount a custom hardware payload.
That payload consists of an onboard computer, a Jetson Xavier, to run the autonomy software shown in Fig.~\ref{fig:functional_diagram}.
The payload also contains a downward-facing a camera for sensing the environment. The camera is a Seek S304SP thermal camera.
The camera intrinsics are used to calculate the frustum's intersection with the search map's cells in IA-TIGRIS.

% without authors goes here
\begin{figure}[t]
\centering
% https://lucid.app/lucidchart/f750ddb4-2809-4773-8361-d5fbb1ba49eb/edit?viewport_loc=-257%2C-116%2C2219%2C1140%2C0_0&invitationId=inv_56e8a3a9-e8cf-4cad-a280-48bd967ff651
\includegraphics[trim={0cm 0cm 0cm 0cm},clip,width=\columnwidth]{figs/5_/functional_diagram.jpeg}
\caption{Functional diagram of the DJI M600 Pro autonomy software.}
\label{fig:functional_diagram}
\end{figure}
\begin{figure}[b]
    \centering
    \begin{subfigure}[b]{0.48\columnwidth}
        \centering
        \includegraphics[width=1.0\linewidth]{figs/5_/field_test_altitude_over_time.png}
        \caption{}
        \label{fig:m600_altitude_over_time}
    \end{subfigure}
    \begin{subfigure}[b]{0.48\columnwidth}
        \centering
        \includegraphics[width=1.0\linewidth]{figs/5_/field_test_entropy_over_time.png}
        \caption{}
        \label{fig:m600_entropy_over_time}
    \end{subfigure}
    \caption{The results for our hexarotor field deployment. (a) Plot of flown altitude over time, showing large variation throughout the experiment. (b) Reduction in entropy percentage over time of field experiment.}
\end{figure}

\subsubsection{Autonomy System}
Fig.~\ref{fig:functional_diagram} illustrates the functional system diagram for the real world field test on the DJI M600. The user specifies the initial plan request prior to takeoff. The TIGRIS planner makes an initial plan on that plan request and sends a global path to the waypoint manager. The waypoint manager tracks the current waypoint within the plan and sends the next waypoint to the DJI software development kit, which then sends actuation commands to the motors. The position of the drone is used to calculate the distance from the drone to the ground and sends that distance parameter to the sensor model. The sensor model's true positive and false positive rate is used to calculate the per-cell entropy updates in the search map manager. The search map manager publishes the current information map, and the replanning node sends an updated plan request to the IA-TIGRIS planner every ten seconds.

The drone started at an altitude of $50$ m above the origin of the reference frame. The informed sampler in IA-TIGRIS was set to add states at altitudes of either $30$ m or $60$ m, creating a trade-off between observation area and detector accuracy. The budget was $2000$ m, the planning horizon was $600$ m, and the planning time was $10$ seconds. 

% % without authors goes here
% \begin{figure}[t]
% \centering
% % https://lucid.app/lucidchart/f750ddb4-2809-4773-8361-d5fbb1ba49eb/edit?viewport_loc=-257%2C-116%2C2219%2C1140%2C0_0&invitationId=inv_56e8a3a9-e8cf-4cad-a280-48bd967ff651
% \includegraphics[trim={0cm 0cm 0cm 0cm},clip,width=\columnwidth]{figs/5_/functional_diagram.jpeg}
% \caption{Functional diagram of the DJI M600 Pro autonomy software.}
% \label{fig:functional_diagram}
% \end{figure}
% \begin{figure}[b]
%     \centering
%     \begin{subfigure}[b]{0.48\columnwidth}
%         \centering
%         \includegraphics[width=1.0\linewidth]{figs/5_/field_test_altitude_over_time.png}
%         \caption{}
%         \label{fig:m600_altitude_over_time}
%     \end{subfigure}
%     \begin{subfigure}[b]{0.48\columnwidth}
%         \centering
%         \includegraphics[width=1.0\linewidth]{figs/5_/field_test_entropy_over_time.png}
%         \caption{}
%         \label{fig:m600_entropy_over_time}
%     \end{subfigure}
%     \caption{The results for our hexarotor field deployment. (a) Plot of flown altitude over time, showing large variation throughout the experiment. (b) Reduction in entropy percentage over time of field experiment.}
% \end{figure}

\subsubsection{Experimental Results}


The bottom image of Fig.~\ref{fig:google_earth} shows the path selected by IA-TIGRIS in the search area. The figure highlights how the planner dynamically adjusts altitudes over time to balance coverage and sensing resolution, maximizing information gain. Higher altitudes allow for broader area coverage, while lower altitudes provide more detailed observations where needed. Additionally, the planner prioritizes revisiting the three regions of higher uncertainty, recognizing the need for repeated observations reduce entropy. This adaptive strategy ensures that uncertain areas receive sufficient attention to improve the belief map. As a result, the entropy of the information map decreases to near zero by the end of the mission, as shown in Fig.~\ref{fig:m600_entropy_over_time}, indicating that the planner has effectively gathered the necessary information. This behavior demonstrates the planner’s ability to optimize sensing actions, balancing altitude selection, revisit frequency, and exploration to maximize mission success.

\begin{figure}[t]
\centering
% \includegraphics[width=2.5in]{fig1}
\includegraphics[trim={4cm 4cm 0cm 4cm},clip,width=\columnwidth]{figs/5_/TL1.jpg}
\caption{Fixed-wing platform used for autonomous flights with an onboard camera pitched at 10 degrees\cite{alarewebsite}}
\label{fig:tl1}
\end{figure}






\subsection{Fixed-wing Deployments}

Our proposed approach was extensively tested on the fixed-wing AlareTech TL-1 UAV, shown in Fig.~\ref{fig:tl1}. The UAV is equipped with an onboard camera pitched at 10 degrees, which introduces a more challenging planning problem due to the non-holonomic motion model and the camera's field of view. Over more than 20 flight hours and 100 flights running IA-TIGRIS, we validated our approach with the objective to search for objects of interest in a large search space across a variety of test scenarios, including different terrain types, varying environmental conditions, and diverse target distributions. An example mission from these tests is shown in Fig.~\ref{fig:fwd}. In this scenario, the planner was given the search bounds and a designated high-priority region. The resulting flight path prioritized revisiting the high-priority area twice, optimizing sensor use and ensuring maximum information gain. This strategy led to the successful detection of the object of interest, with its estimated position marked by the red dot in the figure. 

The map on the upper right in Fig.~\ref{fig:fwd} shows the information map after plan execution was complete. Due to the UAV's limited budget, the upper right and lower left corners of the map are not searched by the agent. The budget is instead utilized to search over the area of higher priority two times. Compared to the paths in Fig.~\ref{fig:google_earth}, we observe that the paths for the fixed wing are smoother and have a larger turning radius, demonstrating how IA-TIGRIS respects the motion constraints of the vehicle. We can also see the effect of wind on the path execution, where the flown path shown in green deviates from the planned path shown in yellow. This illustrates the importance of online planning in the cases where this deviation is large or would accumulate over the course of a longer mission and cause the expected observed area to be much different than actual observed area. 

\begin{figure}[t]
\centering
% \includegraphics[width=2.5in]{fig1}
% [trim={left bottom right top},clip]
\includegraphics[trim={3.0cm, 1.0cm, 3.0cm, 1.0cm},clip,width=\columnwidth]{figs/5_/ONRFig_v3.pdf}
\caption{An example path generated for the fixed-wing platform conducting a large-area search for an object of interest. The larger black rectangle denotes the search bounds, while the smaller black rectangle highlights a region of higher uncertainty. The red dot marks the estimated position of the detected object based on image detections. The upper-right map displays the information state after planning is complete, while the middle plot shows the percent change in entropy over mission time. The flown path illustrates a balance between allocating resources to the high-priority region and exploring other areas within the search space.}
\label{fig:fwd}
\end{figure}

% Also tested extensively on the AlareTech TL-1 (citation?) tube launched UAV seen in Fig.~\ref{fig:tl1}.

% Talk about amount of flights, hours. Platform. Compute. Show visualization fo example flight. Talk about objects of interest in a broad sense (no mention of water/ocean/land for targets). Follow similar figure format as previous section. Main thing we want to highlight is the differences introduced in plans by having a fixed-wing platform compared to a drone. Include image of Alare TL-1 somewhere.

% One big figure showing all the info we want to convey. 

% \BM{Pitch 10 degrees, onboard computer type, etc}


% \subsection{VTOL?}
% what would it bring?



\section{Conclusion}\label{sec:conclusion}
\section{Concluding Remarks}
In this paper, we proposed a novel approach utilizing multimodal LLMs to generate gesture-aware speech recognition transcripts for patients with language disorders. Our framework integrates verbal speech and iconic gestures, enabling the generation of enriched transcripts that capture the latent meaning conveyed through both modalities. Through extensive experimentation, we demonstrated that the proposed method effectively contextualizes incomplete or disfluent speech by incorporating gesture information, leading to more accurate and meaningful representations of the speaker's intent. These findings highlight the potential of our approach to significantly contribute to the field of speech and language therapy, offering innovative tools that can enhance the quality of life for individuals with language disorders by facilitating better communication and assessment methods.

\subsection{Ethical Statement} 
Our dataset was obtained from AphasiaBank with the approval of the Institutional Review Board (IRB) and adheres to the data sharing guidelines set by TalkBank\footnote{https://talkbank.org/share/ethics.html}. This includes complying with the Ground Rules for all TalkBank databases, which are based on the American Psychological Association Code of Ethics~\cite{american2002ethical}.

\subsection{Limitation \& Future Work} 
%This study represents a preliminary investigation into using multimodal LLMs to generate gesture-aware speech recognition transcripts. 
While the results are promising, we recognize several limitations and outline our plans to extend this work further.

One primary limitation is the absence of a definitive ground truth for quantitative evaluation. Since our model generates transcripts by synthesizing speech and gesture data from scratch, traditional benchmarks, such as comparisons with standard speech recognition outputs, are insufficient. Moreover, existing original transcripts lack gesture annotations, making direct comparisons challenging. In future work, we aim to address this gap by collaborating with certified pathologists to conduct qualitative assessments, such as A-B preference tests, to evaluate the effectiveness of gesture-enriched transcripts in accurately conveying the speaker's intentions.

To support quantitative evaluations, we plan to develop novel metrics that assess transcript quality, including grammar accuracy, semantic consistency, and the integration of multimodal information. Such metrics will provide a more objective basis for assessing our model's performance and facilitate comparisons with other multimodal and unimodal approaches.

Another limitation of this study is its focus on structured gestures from a specific task, the Peanut Butter Sandwich Task. While this task offers a controlled context for testing our approach, it does not encompass the diversity of gestures and communication patterns seen in everyday scenarios. As part of our future work, we plan to expand the scope of our model to include tasks such as the Cinderella Story Recall Task~\cite{bird1996cinderella}, which involves unstructured and complex narrative gestures. This expansion will allow us to evaluate the adaptability and robustness of our model in handling varied linguistic and gestural contexts.

In summary, while this study establishes a strong foundation for gesture-aware speech recognition, we aim to refine and extend our methods through collaborative qualitative evaluations, the development of robust quantitative metrics, and broader task applications. These efforts will ensure that our approach continues to evolve, ultimately contributing to more effective communication tools and interventions for individuals with language disorders.







%\bibliographystyle{aaai25}
\bibliography{aaai25}


\section{Metric}
\label{sec:metric}

\textbf{Mean Squared Error (MSE)} Mean Squared Error (MSE) is a common statistical metric used to assess the difference between predicted and actual values. The formula is:
\begin{equation}
    MSE = \frac{1}{n} \sum_{i=1}^{n} (y_i - \hat{y}_i)^2
\end{equation}
where $ n $ is the number of samples, $ y_i $ is the actual value, and $ \hat{y}_i $ is the predicted value.

\textbf{Relative L2 Error} Relative L2 error measures the relative difference between predicted and actual values, commonly used in time series prediction. The formula is:
\begin{equation}
    \text{Relative L2 Error} = \frac{\| Y_{\text{pred}} - Y_{\text{true}} \|_2}{\| Y_{\text{true}} \|_2}
\end{equation}
where $ Y_{\text{pred}} $ is the predicted value and $ Y_{\text{true}} $ is the actual value.

\textbf{Structural Similarity Index Measure (SSIM)} The Structural Similarity Index (SSIM) measures the similarity between two images in terms of luminance, contrast, and structure. The formula is:
\begin{equation}
    SSIM(x, y) = \frac{(2\mu_x \mu_y + C_1)(2\sigma_{xy} + C_2)}{(\mu_x^2 + \mu_y^2 + C_1)(\sigma_x^2 + \sigma_y^2 + C_2)}
\end{equation}
where $ \mu_x $ and $ \mu_y $ are the mean values, $ \sigma_x $ and $ \sigma_y $ are the standard deviations, $ \sigma_{xy} $ is the covariance.

\section{Related Work}
\subsection{Deep Learning based Weather Forecasting}
\textbf{Global Weather Forecasting.} Global weather forecasting has seen significant progress with deep learning models. FourCastNet, based on Fourier neural operators, provides global forecasts comparable to traditional numerical methods like IFS, but at much higher speeds~\cite{pathak2022fourcastnet}. Pangu, utilizing the Swin Transformer, exceeds NWP methods, incorporating earth-specific location embeddings for better performance~\cite{bi2023accurate}. The Spherical Fourier Neural Operator (SFNO) extends Fourier methods using spherical harmonics, offering more stable long-term predictions~\cite{bonev2023spherical}. FuXi focuses on long-term forecasting, achieving a 15-day forecasts comparable to ECMWF~\cite{chen2023fuxi}. GraphCast leverages message-passing networks to improve efficiency and forecasting accuracy~\cite{lam2023learning}, and GenCast builds on this to enhance ensemble forecasting~\cite{price2023gencast}. Further, diffusion models like those in~\cite{li2024generative} generate probabilistic ensembles by sampling, while NeuralGCM~\cite{kochkov2024neural} focuses on atmospheric circulation with a dynamic core, offering climate simulation capabilities but at higher training and inference costs. 

\textbf{Regional Weather Forecasting.} The goal of regional weather forecasting is to enhance local prediction accuracy with high-resolution models. CorrDiff~\cite{mardani2023generative} combines U-Net and diffusion models to improve local forecasts. MetaWeather~\cite{kim2024metaweather} adapts global forecasts to regional contexts using meta-learning. GNNs are also widely applied in regional forecasting, with Graphcast~\cite{lam2023learning} enhancing accuracy by modeling complex spatial dependencies. MetNet-3~\cite{espeholt2022deep} offers high-accuracy forecasts for weather variables, such as precipitation, temperature, and wind speed, at 2-minute intervals and 1–4 km resolution, outperforming traditional models like HRRR. NowcastNet~\cite{zhang2023skilful} and DGMR~\cite{ravuri2021skilful} excel in short-term extreme precipitation forecasts using deep generative models and radar data. In spatiotemporal prediction, NMO~\cite{wu2024neural} models the evolution of physical dynamics, providing new insights for local weather forecasting. Similarly, SimVP~\cite{gao2022simvp} and PastNet~\cite{wu2024pastnet} achieve good results in forecasting local precipitation evolution using spatiotemporal convolution methods.
    
% Despite these advances, none of these methods effectively address the challenge of balancing global and regional high-resolution forecasts or capturing the fine-grained, dynamic interactions important for extreme event prediction.
    
\subsection{Numerical analysis methods}
Multigrid methods~\cite{mccormick1987multigrid,wesseling1995introduction,hackbusch2013multi,bramble2019multigrid,hiptmair1998multigrid,brandt1983multigrid,borzi2009multigrid} and nested grid strategies~\cite{miyakoda1977one,zhang2012nested,sullivan1996grid} are widely used to solve PDEs and handle multi-scale problems~\cite{debreu2008two,xue2000advanced}. Multigrid methods use grids of different resolutions to transfer information and accelerate iterations. They efficiently solve large-scale problems and improve computational accuracy. By eliminating low-frequency errors on coarse grids and high-frequency errors on fine grids, multigrid methods effectively handle error convergence at different scales~\cite{he2019mgnet,he2023mgno,shao2022fast}. Nested grid strategies embed higher-resolution fine grids into regions of interest based on a global coarse grid to capture local complex physical phenomena in detail. In weather forecasting, this method provides large-scale background fields on a global scale while refining the grid for target regions to accurately simulate the evolution of local weather systems and the occurrence of extreme events~\cite{bacon2000dynamically}. 

% Our proposed neural nested grid method helps address challenges like boundary information loss in regional forecasting and multi-scale feature capture.

\section{Additional Results}
%
We present more additional results in Figure \ref{fig_0.25-day}, \ref{fig_0.5-day}, \ref{fig_1.0-day} \ref{fig_1.5-day}, \ref{fig_2.0-day}, \ref{fig_2.5-day}, \ref{fig_3.0-day}, \ref{fig_3.5-day}, \ref{fig_4.0-day}, \ref{fig_4.5-day}, \ref{fig_5.0-day}, \ref{fig_5.5-day}, \ref{fig_6.0-day}, \ref{fig_6.5-day}, \ref{fig_7.0-day}, \ref{fig_7.5-day},
\ref{fig_8.0-day}, \ref{fig_8.5-day}, \ref{fig_9.0-day}, \ref{fig_9.5-day},
\ref{fig_10.0-day}, including 18 variables that are importmant to weather forecasting, each with results ranging from 6 hours to 10 days. These additional results further demonstrate the effectiveness of OneForecast. Same as the Figure \ref{fig:visual_results}
, the initial conditions is 00:00 UTC, 1 January 2020.


\begin{figure*}[h]
\centering
\includegraphics[width=1\linewidth]{figures/fig_0.25-day.jpg}
\vspace{-20pt}
\caption{6-hour forecast results of different models.}
\label{fig_0.25-day}
\end{figure*}

\begin{figure*}[h]
\centering
\includegraphics[width=1\linewidth]{figures/fig_0.5-day.jpg}
\vspace{-20pt}
\caption{0.5-day forecast results of different models.}
\label{fig_0.5-day}
\end{figure*}

\begin{figure*}[h]
\centering
\includegraphics[width=1\linewidth]{figures/fig_1.0-day.jpg}
\vspace{-20pt}
\caption{1-day forecast results of different models.}
\label{fig_1.0-day}
\end{figure*}

\begin{figure*}[h]
\centering
\includegraphics[width=1\linewidth]{figures/fig_1.5-day.jpg}
\vspace{-20pt}
\caption{1.5-day forecast results of different models.}
\label{fig_1.5-day}
\end{figure*}

\begin{figure*}[h]
\centering
\includegraphics[width=1\linewidth]{figures/fig_2.0-day.jpg}
\vspace{-20pt}
\caption{2-day forecast results of different models.}
\label{fig_2.0-day}
\end{figure*}


\begin{figure*}[h]
\centering
\includegraphics[width=1\linewidth]{figures/fig_2.5-day.jpg}
\vspace{-20pt}
\caption{2.5-day forecast results of different models.}
\label{fig_2.5-day}
\end{figure*}

\begin{figure*}[h]
\centering
\includegraphics[width=1\linewidth]{figures/fig_3.0-day.jpg}
\vspace{-20pt}
\caption{3-day forecast results of different models.}
\label{fig_3.0-day}
\end{figure*}

\begin{figure*}[h]
\centering
\includegraphics[width=1\linewidth]{figures/fig_3.5-day.jpg}
\vspace{-20pt}
\caption{3.5-day forecast results of different models.}
\label{fig_3.5-day}
\end{figure*}

\begin{figure*}[h]
\centering
\includegraphics[width=1\linewidth]{figures/fig_4.0-day.jpg}
\vspace{-20pt}
\caption{4-day forecast results of different models.}
\label{fig_4.0-day}
\end{figure*}

\begin{figure*}[h]
\centering
\includegraphics[width=1\linewidth]{figures/fig_4.5-day.jpg}
\vspace{-20pt}
\caption{4.5-day forecast results of different models.}
\label{fig_4.5-day}
\end{figure*}


\begin{figure*}[h]
\centering
\includegraphics[width=1\linewidth]{figures/fig_5.0-day.jpg}
\vspace{-20pt}
\caption{5.0-day forecast results of different models.}
\label{fig_5.0-day}
\end{figure*}

\begin{figure*}[h]
\centering
\includegraphics[width=1\linewidth]{figures/fig_5.5-day.jpg}
\vspace{-20pt}
\caption{5.5-day forecast results of different models.}
\label{fig_5.5-day}
\end{figure*}

\begin{figure*}[h]
\centering
\includegraphics[width=1\linewidth]{figures/fig_6.0-day.jpg}
\vspace{-20pt}
\caption{6.0-day forecast results of different models.}
\label{fig_6.0-day}
\end{figure*}

\begin{figure*}[h]
\centering
\includegraphics[width=1\linewidth]{figures/fig_6.5-day.jpg}
\vspace{-20pt}
\caption{6.5-day forecast results of different models.}
\label{fig_6.5-day}
\end{figure*}

\begin{figure*}[h]
\centering
\includegraphics[width=1\linewidth]{figures/fig_7.0-day.jpg}
\vspace{-20pt}
\caption{7.0-day forecast results of different models.}
\label{fig_7.0-day}
\end{figure*}

\begin{figure*}[h]
\centering
\includegraphics[width=1\linewidth]{figures/fig_7.5-day.jpg}
\vspace{-20pt}
\caption{7.5-day forecast results of different models.}
\label{fig_7.5-day}
\end{figure*}

\begin{figure*}[h]
\centering
\includegraphics[width=1\linewidth]{figures/fig_8.0-day.jpg}
\vspace{-20pt}
\caption{8.0-day forecast results of different models.}
\label{fig_8.0-day}
\end{figure*}

\begin{figure*}[h]
\centering
\includegraphics[width=1\linewidth]{figures/fig_8.5-day.jpg}
\vspace{-20pt}
\caption{8.5-day forecast results of different models.}
\label{fig_8.5-day}
\end{figure*}

\begin{figure*}[h]
\centering
\includegraphics[width=1\linewidth]{figures/fig_9.0-day.jpg}
\vspace{-20pt}
\caption{9.0-day forecast results of different models.}
\label{fig_9.0-day}
\end{figure*}

\begin{figure*}[h]
\centering
\includegraphics[width=1\linewidth]{figures/fig_9.5-day.jpg}
\vspace{-20pt}
\caption{9.5-day forecast results of different models.}
\label{fig_9.5-day}
\end{figure*}

\begin{figure*}[h]
\centering
\includegraphics[width=1\linewidth]{figures/fig_10.0-day.jpg}
\vspace{-20pt}
\caption{10.0-day forecast results of different models.}
\label{fig_10.0-day}
\end{figure*}


\section{Detailed Mathematical Proof}
\label{sec:proof}
\textbf{Proof of Theorem 1}

Now we have N augmented data and we need to select the best from them. We consider both the quality and the diversity of these data and get the sampling strategy from an optimization problem.

We model the sampling strategy as a multinomial distribution supported on all the augmented data $S = \{\mathbf{X}_j\}_{j=1}^N$, which means that the sampling strategy $\pi=(\pi_1,...,\pi_N)^\top$ is the corresponding probabilities of selecting $\mathbf{X}_1,...,\mathbf{X}_N$, then we can model the expectation of the similarity as:
$$\begin{aligned}
 & \mathbb{E}_{Y_x,Y_{x^{\prime}}\in\mathcal{C}}\{g(x,x^{\prime})\mid S\} \\
 & =\quad\int g(\mathbf{x},\mathbf{x}^{\prime})\boldsymbol{\pi}(\mathbf{x})\mathrm{Pr}_{S}(Y_{x}\in\mathcal{C}\mid\boldsymbol{x}=\mathbf{x})\boldsymbol{\pi}(\mathbf{x}^{\prime})\mathrm{Pr}_{S}(Y_{x}\in\mathcal{C}\mid\boldsymbol{x}=\mathbf{x}^{\prime})d\mathbf{x}d\mathbf{x}^{\prime} \\
 & =\quad\sum_{i,j=1}^Ng(\mathbf{X}_i,\mathbf{X}_j)\pi_i\pi_j\mathrm{Pr}_{S}(Y_x\in\mathcal{C}\mid\boldsymbol{x}=\mathbf{X}_i)\mathrm{Pr}_{S}(Y_x\in\mathcal{C}\mid\boldsymbol{x}=\mathbf{X}_j),
\end{aligned}$$
where the set $\mathcal{C}$ denotes the criterion of selection we are using, the function $g$ can be chosen as any similarity metric function and $x$ means a random variable.

The core to solving the above optimization problem is to use predictive inference to approximate the conditional probability of $\{Y_x\in\mathcal{C}\}$ given $x = \mathbf{X}$
Let $\mu ( \mathbf{x} ) : = \mathbb{E} ( Y\mid \mathbf{X} = \mathbf{x} )$ be the oracle associated with $( \mathbf{X} , Y) .$ Denote $\theta_j=\mathbb{I}\{Y_j\in\mathcal{C}\}$. As the augmented data
$\mathbf{X}_1,...,\mathbf{X}_N$ are independently identically distributed, $\theta_1,...,\theta_N$ can be regarded as independent Bernoulli($q)$ variables with $q=\Pr(Y_j\in\mathcal{C}).$ The probability distribution of the predicted result $W_j$ for $j=1,...,N$ is
$$\Pr(W_j\mid\theta_j)=(1-\theta_j)f_0+\theta_jf_1,\quad$$
where $f_0$ and $f_1$ are the conditional distributions of $W_j$ on $Y_j \in \mathcal{C}$ or not.

Denote $T(w) = \frac{(1-q)f_0(W_j)}{f(W_j)}$, we can rewrite the expectation of the similarity as
$$\mathbb{E}_{Y_x,Y_{x^{\prime}}\in\mathcal{C}}\{g(x,x^{\prime})|S\}=\sum_{i,j=1}^Ng(\mathbf{X}_i,\mathbf{X}_j)\pi_i\pi_j(1-T_i)(1-T_j)=\boldsymbol{\pi}^\top A_\mathbb{T}\boldsymbol{\pi},$$

Next, we use the expectation to control the quality of the data.
$$\mathbb{E}\{\mathbb{I}(Y_x\not\in\mathcal{C})\mid S\}=\sum_{i=1}^N\Pr(Y_i\not\in\mathcal{C}\mid\mathbf{X}_i)\pi_i=\sum_{i=1}^N\pi_iT_i\leq\alpha,$$

In all, the optimization problem can be modeled as 
\begin{align}
    & \arg\min_{\boldsymbol{\pi}}\quad h(\boldsymbol{\pi},\mathbb{T}):=\boldsymbol{\pi}^\top A_\mathbb{T}\boldsymbol{\pi}, \\
    & \text{subject to} \quad
        \begin{cases}
            \sum_{i = 1}^N\pi_iT_i\leq\alpha, \\
            \sum_{i = 1}^N\pi_i = 1, \\
            0\leq\pi_i\leq m^{-1}, \quad 1\leq i\leq N.
        \end{cases}
\end{align}

where $m$ is used to control the maximum selection.

The best selection of K is determined by the strategy $\pi$ which serves as the solution to the above optimization problem.

\section{Additional Experiments}
\label{sec:more_experiments}
\subsection{Long-term forecasting experiment expansion}

In the long-term forecasting experiments, we compare the performance of different backbone models on the SWE benchmark, evaluating the relative L2 error for three variables (U, V, and H). Our setup inputs 5 frames and predicts 50 frames. For the SimVP-v2 model, using \method{} reduces the relative L2 error for SWE (u) from 0.0187 to 0.0154, SWE (v) from 0.0387 to 0.0342, and SWE (h) from 0.0443 to 0.0397. We visualize SWE (h) in 3D as shown in Figure~\ref{fig:case} [\textcolor{red}{I}]. For the ConvLSTM model, applying \method{} reduces the relative L2 error for SWE (u) from 0.0487 to 0.0321, SWE (v) from 0.0673 to 0.0351, and SWE (h) from 0.0762 to 0.0432. For the FNO model, using \method{} reduces the relative L2 error for SWE (u) from 0.0571 to 0.0502, SWE (v) from 0.0832 to 0.0653, and SWE (h) from 0.0981 to 0.0911. Overall, \method{} significantly improves the long-term forecasting accuracy of different backbone models.

\begin{figure*}[h]
    \centering
    \includegraphics[width=\textwidth]{image/casestudy.pdf}
    \caption{
    \textcolor{red}{I.} 3D visualization of the SWE(h), showing Ground-truth, SimVP-V2+BeamVQ predictions, and Error at T=1, 10, 20, 30, 40, 50. The first row shows Ground-truth, the second SimVP-V2+BeamVQ predictions, and the third Error. \textcolor{red}{II.} A case study. Building fire simulation with ventilation settings added to Wu's Prometheus~\cite{wu2024prometheus}. (a) Layout and HRR growth. (b) Comparison of physical metrics for different methods. (c) Ground-truth, ResNet+BeamVQ, and ResNet predictions.
    }
    \label{fig:case} 
\end{figure*}


\subsection{Experiment Statistical Significance}
\label{sec:significance}
To measure the statistical significance of our main experiment results, we choose three backbones to train on two datasets to run 5 times. 
Table~\ref{tab:significance} records the average and standard deviation of the test MSE loss.
The results prove that our method is statistically significant to outperform the baselines
because our confidence interval is always upper than the confidence interval of the baselines. 
Due to limited computation resources, we do not cover all ten backbones and five datasets, 
but we believe these results have shown that our method has consistent advantages.


\begin{table}[h]
\label{tab:significance}
\centering
\begin{scriptsize}
    \begin{sc}
    \caption{ The average and standard deviation of MSE in 5 runs}
    \label{tab:significance}
    \centering
        \renewcommand{\multirowsetup}{\centering}
        \setlength{\tabcolsep}{10pt}
        \begin{tabular}{l|cc|cc}
            \toprule
            
            \multirow{4}{*}{Model} & \multicolumn{4}{c}{Benchmarks}  \\
            \cmidrule(lr){2-5}
            & \multicolumn{2}{c}{NSE} &   \multicolumn{2}{c}{SEVIR}   \\
            \cmidrule(lr){2-5}
           & Ori & + BeamVQ & Ori & + BeamVQ  \\
            \midrule
            ConvLSTM &0.4092$\pm$0.0002 &\textbf{0.1277$\pm$0.0001}  & 0.1762 0.0007  & \textbf{0.1279$\pm$0.0009}  \\
            FNO &  0.2227$\pm$0.0003 &\textbf{0.1007 $\pm$0.0002}& 0.0787$\pm$0.0012 & \textbf{ 0.0437$\pm$0.0013} \\
            CNO & 0.2192 $\pm$0.0008 &\textbf{ 0.1492$\pm$0.0011}& 0.0057$\pm$0.0005 & \textbf{ 0.0053$\pm$0.0006} \\
            \bottomrule
        \end{tabular}
    \end{sc}

\end{scriptsize}
\end{table}


\end{document}
