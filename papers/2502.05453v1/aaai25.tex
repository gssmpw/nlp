%File: formatting-instructions-latex-2025.tex
%release 2025.0
\documentclass[letterpaper]{article} % DO NOT CHANGE THIS
\usepackage{aaai25}  % DO NOT CHANGE THIS
\usepackage{times}  % DO NOT CHANGE THIS
\usepackage{helvet}  % DO NOT CHANGE THIS
\usepackage{courier}  % DO NOT CHANGE THIS
\usepackage[hyphens]{url}  % DO NOT CHANGE THIS
\usepackage{graphicx} % DO NOT CHANGE THIS
\usepackage[T1]{fontenc}
\urlstyle{rm} % DO NOT CHANGE THIS
\def\UrlFont{\rm}  % DO NOT CHANGE THIS
\usepackage{natbib}  % DO NOT CHANGE THIS AND DO NOT ADD ANY OPTIONS TO IT
\usepackage{caption} % DO NOT CHANGE THIS AND DO NOT ADD ANY OPTIONS TO IT
\frenchspacing  % DO NOT CHANGE THIS
\setlength{\pdfpagewidth}{8.5in}  % DO NOT CHANGE THIS
\setlength{\pdfpageheight}{11in}  % DO NOT CHANGE THIS


%File: anonymous-submission-latex-2025.tex
% \documentclass[letterpaper]{article} % DO NOT CHANGE THIS
% \usepackage[submission]{aaai25}  % DO NOT CHANGE THIS
% \usepackage{times}  % DO NOT CHANGE THIS
% \usepackage{helvet}  % DO NOT CHANGE THIS
% \usepackage{courier}  % DO NOT CHANGE THIS
% \usepackage[hyphens]{url}  % DO NOT CHANGE THIS
% \usepackage{graphicx} % DO NOT CHANGE THIS
% \urlstyle{rm} % DO NOT CHANGE THIS
% \def\UrlFont{\rm}  % DO NOT CHANGE THIS
% \usepackage{natbib}  % DO NOT CHANGE THIS AND DO NOT ADD ANY OPTIONS TO IT
% \usepackage{caption} % DO NOT CHANGE THIS AND DO NOT ADD ANY OPTIONS TO IT
% \frenchspacing  % DO NOT CHANGE THIS
% \setlength{\pdfpagewidth}{8.5in} % DO NOT CHANGE THIS
% \setlength{\pdfpageheight}{11in} % DO NOT CHANGE THIS
%
% These are recommended to typeset algorithms but not required. See the subsubsection on algorithms. Remove them if you don't have algorithms in your paper.
\usepackage{algorithm}
\usepackage{algorithmic}
\usepackage{amsmath,amssymb}
\usepackage{subcaption}
\usepackage{booktabs}
\usepackage{enumitem}
\usepackage{pdfpages}
%\usepackage{hyperref}
\makeatletter
\newcommand*{\rom}[1]{\romannumeral #1}
\makeatother

\newcommand{\framework}{Decentralized Cooperative Generative Agents}
\newcommand{\frameworkAbbr}{DCGA}
\newcommand{\hq}[1]{{\color{black} #1}}
\newcommand{\jd}[1]{{\color{black} #1}}

\usepackage{cite}



% These are are recommended to typeset listings but not required. See the subsubsection on listing. Remove this block if you don't have listings in your paper.
\usepackage{newfloat}
\usepackage{listings}
\DeclareCaptionStyle{ruled}{labelfont=normalfont,labelsep=colon,strut=off} % DO NOT CHANGE THIS
\lstset{%
	basicstyle={\footnotesize\ttfamily},% footnotesize acceptable for monospace
	numbers=left,numberstyle=\footnotesize,xleftmargin=2em,% show line numbers, remove this entire line if you don't want the numbers.
	aboveskip=0pt,belowskip=0pt,%
	showstringspaces=false,tabsize=2,breaklines=true}
\floatstyle{ruled}
\newfloat{listing}{tb}{lst}{}
\floatname{listing}{Listing}
%
% Keep the \pdfinfo as shown here. There's no need
% for you to add the /Title and /Author tags.
\pdfinfo{
/TemplateVersion (2025.1)
}

% DISALLOWED PACKAGES
% \usepackage{authblk} -- This package is specifically forbidden
% \usepackage{balance} -- This package is specifically forbidden
% \usepackage{color (if used in text)
% \usepackage{CJK} -- This package is specifically forbidden
% \usepackage{float} -- This package is specifically forbidden
% \usepackage{flushend} -- This package is specifically forbidden
% \usepackage{fontenc} -- This package is specifically forbidden
% \usepackage{fullpage} -- This package is specifically forbidden
% \usepackage{geometry} -- This package is specifically forbidden
% \usepackage{grffile} -- This package is specifically forbidden
% \usepackage{hyperref} -- This package is specifically forbidden
% \usepackage{navigator} -- This package is specifically forbidden
% (or any other package that embeds links such as navigator or hyperref)
% \indentfirst} -- This package is specifically forbidden
% \layout} -- This package is specifically forbidden
% \multicol} -- This package is specifically forbidden
% \nameref} -- This package is specifically forbidden
% \usepackage{savetrees} -- This package is specifically forbidden
% \usepackage{setspace} -- This package is specifically forbidden
% \usepackage{stfloats} -- This package is specifically forbidden
% \usepackage{tabu} -- This package is specifically forbidden
% \usepackage{titlesec} -- This package is specifically forbidden
% \usepackage{tocbibind} -- This package is specifically forbidden
% \usepackage{ulem} -- This package is specifically forbidden
% \usepackage{wrapfig} -- This package is specifically forbidden
% DISALLOWED COMMANDS
% \nocopyright -- Your paper will not be published if you use this command
% \addtolength -- This command may not be used
% \balance -- This command may not be used
% \baselinestretch -- Your paper will not be published if you use this command
% \clearpage -- No page breaks of any kind may be used for the final version of your paper
% \columnsep -- This command may not be used
% \newpage -- No page breaks of any kind may be used for the final version of your paper
% \pagebreak -- No page breaks of any kind may be used for the final version of your paperr
% \pagestyle -- This command may not be used
% \tiny -- This is not an acceptable font size.
% \vspace{- -- No negative value may be used in proximity of a caption, figure, table, section, subsection, subsubsection, or reference
% \vskip{- -- No negative value may be used to alter spacing above or below a caption, figure, table, section, subsection, subsubsection, or reference

\setcounter{secnumdepth}{2} %May be changed to 1 or 2 if section numbers are desired.

% The file aaai25.sty is the style file for AAAI Press
% proceedings, working notes, and technical reports.
%

% Titl

%Example, Single Author, ->> remove \iffalse,\fi and place them surrounding AAAI title to use it
\iffalse
\title{LLM-Powered Decentralized Generative Agents with Adaptive Hierarchical Knowledge Graph for Cooperative Planning}
\author {
    Author Name
}
\affiliations{
    Affiliation\\
    Affiliation Line 2\\
    name@example.com
}
\fi

%\iffalse
%Example, Multiple Authors, ->> remove \iffalse,\fi and place them surrounding AAAI title to use it

\title{
\hq{
LLM-Powered Decentralized Generative Agents with Adaptive Hierarchical Knowledge Graph for Cooperative Planning
}
}
\author {
    % Authors
    Hanqing Yang\textsuperscript{\rm 1},
    Jingdi Chen\textsuperscript{\rm 1},
    Marie Siew\textsuperscript{\rm 2},
    Tania Lorido-Botran\textsuperscript{\rm 3 4},
    Carlee Joe-Wong\textsuperscript{\rm 1}
}
\affiliations {
    % Affiliations
    \textsuperscript{\rm 1}Carnegie Mellon University\\
    \textsuperscript{\rm 2}Singapore University of Technology and Design\\
    \textsuperscript{\rm 3}Roblox\\
    \textsuperscript{\rm 4}Northeastern University\\
    \{hanqing3, jingdic, cjoewong\}@andrew.cmu.edu, marie\_siew@sutd.edu.sg, t.loridobotran@northeastern.edu
}
%\fi

\renewcommand{\baselinestretch}{1}
\begin{document}

\maketitle

\begin{abstract}
Developing intelligent agents for long-term cooperation in dynamic open-world scenarios is a major challenge in multi-agent systems. Traditional Multi-agent Reinforcement Learning (MARL) frameworks like centralized training decentralized execution (CTDE) struggle with scalability and flexibility. They require centralized long-term planning, which is difficult without custom reward functions, and face challenges in processing multi-modal data. CTDE approaches also assume fixed cooperation strategies, making them impractical in dynamic environments where agents need to adapt and plan independently.
To address decentralized multi-agent cooperation, we propose Decentralized Adaptive Knowledge Graph Memory and Structured Communication System (\textbf{DAMCS}) in a novel Multi-agent Crafter environment. Our generative agents, powered by Large Language Models (LLMs), are more scalable than traditional MARL agents by leveraging external knowledge and language for long-term planning and reasoning. 
Instead of fully sharing information from all past experiences, DAMCS introduces a multi-modal memory system organized as a hierarchical knowledge graph and a structured communication protocol to optimize agent cooperation. This allows agents to reason from past interactions and share relevant information efficiently. Experiments on novel multi-agent open-world tasks show that DAMCS outperforms both MARL and LLM baselines in task efficiency and collaboration. Compared to single-agent scenarios, the two-agent scenario achieves the same goal with 63\% fewer steps, and the six-agent scenario with 74\% fewer steps, highlighting the importance of adaptive memory and structured communication in achieving long-term goals. \hq{We publicly release our project at: https://happyeureka.github.io/damcs}.
\end{abstract}

% Uncomment the following to link to your code, datasets, an extended version or similar.
%
% \begin{links}
%     \link{Code}{https://aaai.org/example/code}
%     \link{Datasets}{https://aaai.org/example/datasets}
%     \link{Extended version}{https://aaai.org/example/extended-version}
% \end{links}

\section{Introduction}
\section{Introduction}%

Decision-making is at the heart of artificial intelligence systems, enabling agents to navigate complex environments, achieve goals, and adapt to changing conditions. Traditional decision-making frameworks often rely on associations or statistical correlations between variables, which can lead to suboptimal outcomes when the underlying causal relationships are ignored \citep{pearl2009causal}. 
The rise of causal inference as a field has provided powerful frameworks and tools to address these challenges, such as structural causal models and potential outcomes frameworks \citep{rubin1978bayesian,pearl2000causality}. 
Unlike traditional methods, \textit{causal decision-making} focuses on identifying and leveraging cause-effect relationships, allowing agents to reason about the consequences of their actions, predict counterfactual scenarios, and optimize decisions in a principled way \citep{spirtes2000causation}. In recent years, numerous decision-making methods based on causal reasoning have been developed, finding applications in diverse fields such as recommender systems \citep{zhou2017large}, clinical trials \citep{durand2018contextual}, finance \citep{bai2024review}, and ride-sharing platforms \citep{wan2021pattern}. Despite these advancements, a fundamental question persists: 

\begin{center}
    \textit{When and why do we need causal modeling in decision-making?}
\end{center} 

% Numerous decision-making methods based on causal reasoning have been developed recently with wide applications 
% %Decision makings based on causal reasoning have been widely applied 
% in a variety of fields, including 
% recommender systems \citep{zhou2017large}, clinical trials \citep{durand2018contextual}, 
% finance \citep{bai2024review}, 
% ride-sharing platforms \citep{wan2021pattern}, and so on. 


 

% At the intersection of these fields, causal decision-making seeks to answer critical questions: How can agents make decisions when causal knowledge is incomplete? How do we integrate learning and reasoning about causality into real-world decision-making systems? What role do interventions, counterfactuals, and observational data play in guiding decisions? 

% Our review is structured as follows: 
 

This question is closely tied to the concept of counterfactual thinking—reasoning about what might have happened under alternative decisions or actions. Counterfactual analysis is crucial in domains where the outcomes of unchosen decisions are challenging, if not impossible, to observe. For instance, a business leader selecting one marketing strategy over another may never fully know the outcome of the unselected option \citep{rubin1974estimating, pearl2009causal}. Similarly, in econometrics, epidemiology, psychology, and social sciences, \textit{the inability to observe counterfactuals directly often necessitates causal approaches} \citep{morgan2015counterfactuals, imbens2015causal}. 
Conversely, non-causal analysis may suffice in scenarios where alternative outcomes are readily determinable. For example, a personal investor's actions may have negligible impact on stock market dynamics, enabling potential outcomes of alternate investment decisions to be inferred from existing stock price time series \citep{angrist2008mostly}. However, even in cases where counterfactual outcomes are theoretically calculable—such as in environments with known models like AlphaGo—exhaustively computing all possible outcomes is computationally infeasible \citep{silver2017mastering, silver2018general}. 
In such scenarios, causal modeling remains advantageous by offering \textit{structured ways to infer outcomes efficiently and make robust decisions}. 


%This perspective not only enhances the interpretability of decisions but also provides a principled framework for addressing uncertainty, guiding actions, and improving performance across a broad range of applications.

% Data-driven decision-making exists before the causal revolution. \textit{So when and why do we need causal modelling in decision-making?} 
% This is closely related to the presence of counterfactuals in many applications. 
% The counterfactual thinking involves considering what would have happened in an alternate scenario where a different decision or action was taken. 
% In many fields, including econometrics, epidemiology, psychology, and social sciences, accessing outcomes from unchosen decisions is often challenging if not impossible. 
% For example, a business leader who selects one marketing strategy over another may never know the outcome of the unselected option. 
% Conversely, non-causal analysis may be adequate in situations where potential outcomes of alternate actions are more readily determinable: for example, the investment of a personal investor may have minimal impact on the market, therefore her counterfactual investment decision's outcomes can still be calculated with the data of stock price time series. 
% However, it is important to note that even when counterfactuals are theoretically calculable, as in environments with known models like AlphaGo, computing all possible outcomes may not be feasible. 
% In such scenarios, a causal perspective  remains beneficial. 


 

% 1. significance of decision making
% 2. role of causal in decision making
% 3. refer to the https://jair.org/index.php/jair/article/view/13428/26917

% Decision makings based on causal reasoning have been widely applied in a variety of fields, including recommender systems \citep{zhou2017large}, clinical trials \citep{durand2018contextual}, 
% business economics scenarios \citep{shen2015portfolio}, 
% ride-sharing platforms \citep{wan2021pattern}, and so on. 
% However, most existing works primarily assume either sophisticated prior knowledge or strong causal models to conduct follow-up decision-making. To make effective and trustworthy decisions, it is critical to have a thorough understanding of the causal connections between actions, environments, and outcomes.

\begin{figure}[!t]
    \centering
    \includegraphics[width = .75\linewidth]{Figure/3Steps_V2.png}
    \caption{Workflow of the \acrlong{CDM}. $f_1$, $f_2$, and $f_3$ represent the impact sizes of the directed edges. Variables enclosed in solid circles are observed, while those in dashed circles are actionable.}\label{fig:cdm}
\end{figure}


Most existing works primarily assume either sophisticated prior knowledge or strong causal models to conduct follow-up decision-making. To make effective and trustworthy decisions, it is critical to have a thorough understanding of the causal relationships among actions, environments, and outcomes. This review synthesizes the current state of research in \acrfull{CDM}, providing an overview of foundational concepts, recent advancements, and practical applications. Specifically, this work discusses the connections of \textbf{three primary components of decision-making} through a causal lens: 1) discovering causal relationships through \textit{\acrfull{CSL}}, 2) understanding the impacts of these relationships through \textit{\acrfull{CEL}}, and 3) applying the knowledge gained from the first two aspects to decision making via \textit{\acrfull{CPL}}. 

Let $\boldsymbol{S}$ denote the state of the environment, which includes all relevant feature information about the environment the decision-makers interact with, $A$ the action taken, $\pi$ the action policy that determines which action to take, and $R$ the reward observed after taking action $A$. As illustrated in Figure \ref{fig:cdm}, \acrshort{CDM} typically begins with \acrshort{CSL}, which aims to uncover the unknown causal relationships among various variables of interest. Once the causal structure is established, \acrshort{CEL} is used to assess the impact of a specific action on the outcome rewards. To further explore more complex action policies and refine decision-making strategies, \acrshort{CPL} is employed to evaluate a given policy or identify an optimal policy. In practice, it is also common to move directly from \acrshort{CSL} to \acrshort{CPL} without conducting \acrshort{CEL}. Furthermore, \acrshort{CPL} has the potential to improve both \acrshort{CEL} and \acrshort{CSL} by facilitating the development of more effective experimental designs \citep{zhu2019causal,simchi2023multi} or adaptively refining causal structures \citep{sauter2024core}. %However, these are beyond the scope of this paper.

\begin{figure}[!t]
    \centering
    \includegraphics[width = .9\linewidth]{Figure/Table_of_Six_Scenarios_S.png}
    \caption{Common data dependence structures (paradigms) in \acrshort{CDM}. Detailed notations and explanations can be found in Section \ref{sec:paradigms}.}
    \label{Fig:paradigms}
\end{figure}
Building on this framework, decision-making problems discussed in the literature can be further categorized into \textbf{six paradigms}, as summarized in Figure \ref{Fig:paradigms}. These paradigms summarize the common assumptions about data dependencies frequently employed in practice. Paradigms 1-3 describe the data structures in offline learning settings, where data is collected according to an unknown and fixed behavior policy. In contrast, paradigms 4-6 capture the online learning settings, where policies dynamically adapt to newly collected data, enabling continuous policy improvement. These paradigms also reflect different assumptions about state dependencies. The simplest cases, paradigms 1 and 4, assume that all observations are independent, implying no long-term effects of actions on future observations. To account for sequental dependencies, the \acrfull{MDP} framework, summarized in paradigms 2 and 5, assumes Markovian state transition. Specifically, it assumes that given the current state-action pair $(S_t, A_t)$, the next state $S_{t+1}$ and reward $R_t$ are independent of all prior states $\{S_j\}_{j < t}$ and actions $\{A_j\}_{j < t}$. When such independence assumptions do not hold, paradigms 3 and 6 account for scenarios where all historical observations may impact state transitions and rewards. This includes but not limited to researches on \acrfull{POMDP} \citep{hausknecht2015deep, littman2009tutorial}, panel data analysis \citep{hsiao2007panel,hsiao2022analysis}, \acrfull{DTR} with finite stages \citep{chakraborty2014dynamic, chakraborty2013statistical}. 

Each \acrshort{CDM} task has been studied under different paradigms, with \acrshort{CSL} extensively explored within paradigm 1. \acrshort{CEL} and offline \acrshort{CPL} encompass paradigms 1-3, while online \acrshort{CPL} spans paradigms 4-6. By organizing the discussion around these three tasks and six paradigms, this review aims to provide a cohesive framework for understanding the field of \acrlong{CDM} across diverse tasks and data structures.

%Recognizing the importance of long-term effects in decision-making

%Further discussions on these paradigms and their connections to various causal decision-making problems are provided in Section \ref{sec:paradigms}.


\textbf{Contribution.} In this paper, we conduct a comprehensive survey of \acrshort{CDM}. 
Our contributions are as follows. 
\begin{itemize}
    \item We for the first time organize the related causal decision-making areas into three tasks and six paradigms, connecting previously disconnected areas (including economics, statistics, machine learning, and reinforcement learning) using a consistent language. For each paradigm and task, we provide a few taxonomies to establish a unified view of the recent literature.
    \item We provide a comprehensive overview of \acrshort{CDM}, covering all three major tasks and six classic problem structures, addressing gaps in existing reviews that either focus narrowly on specific tasks or paradigms or overlook the connection between decision-making and causality (detailed in Section \ref{sec::related_work}).
    %\item We outline three key challenges that emerge when utilizing CDM in practice. Moreover, we delve into a comprehensive discussion on the recent advancements and progress made in addressing these challenges. We also suggest six future directions for these problems.
    \item We provide real-world examples to illustrate the critical role of causality in decision-making and to reveal how \acrshort{CSL}, \acrshort{CEL} and \acrshort{CPL} are inherently interconnected in daily applications, often without explicit recognition.
    \item We are actively maintaining and expanding a GitHub repository and online book, providing detailed explanations of key methods reviewed in this paper, along with a code package and demos to support their implementation, with URL: \url{https://causaldm.github.io/Causal-Decision-Making}.
\end{itemize}
% Our review is structured as follows: 


%%%%%%%%%%%%%%%%%%%%%%%%%%%%%%%%%%
%  causal helps over "Correlational analysis"
%Correlational analysis, though widely used in various fields, has inherent limitations, particularly when it comes to decision-making. While it identifies relationships between variables, it fails to establish causality, often leading to misinterpretations and misguided decisions. For example, the positive correlation between ice cream sales and drowning incidents is a classic example of how correlational data can be misleading, as both are influenced by a third factor, temperature, rather than causing each other. Such spurious correlations, due to oversight of confounding variables, underscore the necessity of causal modeling in decision making. Causal models excel where correlational analysis falls short, offering predictive power and a deeper understanding of underlying mechanisms. They enable us to predict the outcomes of interventions, even under untested conditions, and provide insights into the processes leading to these outcomes, thereby informing more effective strategies. Moreover, causal models are good at generalizing findings across different contexts, a capability often limited in purely correlational studies. 

%  causal helps in causal RL 
%From another complementary angle, although causal concepts have traditionally not been explicitly incorporated in fields like online bandits \citep{lattimore2020bandit} and \acrfull{RL} \citep{sutton2018reinforcement}, much of the literature in these areas implicitly relies on basic assumptions outlined in Section \ref{sec:prelim_assump} to utilize observed data in place of potential outcomes in their analyses, and there is also a growing recognition of the significance of the causal perspective \citep{lattimore2016causal, zeng2023survey} in these areas. 
% \textbf{Read causal RL survey and summarize. } However, by integrating causal concepts and leverging existing methodologies, we open up possibilities for developing more robust models to remove spurious correlation and selection bias \citep{xu2023instrumental, forney2017counterfactual}, designing more sample-efficient \citep{sontakke2021causal, seitzer2021causal} and robust \citep{dimakopoulou2019balanced, ye2023doubly} algorithms, and improving the generalizability \citep{zhang2017transfer, eghbal2021learning}, explanability \citep{foerster2018counterfactual, herlau2022reinforcement}, and fairness \citep{zhang2018fairness,huang2022achieving,balakrishnan2022scales} of these methods. %, and safety \cite{hart2020counterfactual}

%


%\subsection{Paper Structure}
The remainder of this paper is organized as follows: Section \ref{sec::related_work} provides an overview of related survey papers. Section \ref{sec:preliminary} introduces the foundational concepts, assumptions, and notations that form the foundation for the subsequent discussions. In Section \ref{sec:3task6paradigm}, we offer a detailed introduction to the three key tasks and six learning paradigms in \acrshort{CDM}. Sections \ref{Sec:CSL} through \ref{sec:Online CPL} form the core of the paper, with each section dedicated to a specific topic within \acrshort{CDM}: \acrshort{CSL}, \acrshort{CEL}, Offline \acrshort{CPL}, and Online \acrshort{CPL}, respectively. Section \ref{sec:assump_violated} then explores extensions needed when standard causal assumptions are violated. To illustrate the practical application of the \acrshort{CDM} framework, Section \ref{sec:real_data} presents two real-world case studies. Finally, Section \ref{sec:conclusion} concludes the paper with a summary of our contributions and a discussion of additional research directions that are actively being explored.





\section{Related Work}\label{sec:related}

\section{Related Work}

\subsection{Instruction Generation}

Instruction tuning is essential for aligning Large Language Models (LLMs) with user intentions~\cite{ouyang2022training,cao2023instruction}. Initially, this involved collecting and cleaning existing data, such as open-source NLP datasets~\cite{wang2023far,ding2023enhancing}. With the importance of instruction quality recognized, manual annotation methods emerged~\cite{wang2023far,zhou2024lima}. As larger datasets became necessary, approaches like Self-Instruct~\cite{wang2022self} used models to generate high-quality instructions~\cite{guo2024human}. However, complex instructions are rare, leading to strategies for synthesizing them by extending simpler ones~\cite{xu2023wizardlm,sun2024conifer,he2024can}. However, existing methods struggle with scalability and diversity.


\subsection{Back Translation}

Back-translation, a process of translating text from the target language back into the source language, is mainly used for data augmentation in tasks like machine translation~\cite{sennrich2015improving, hoang2018iterative}. ~\citet{li2023self} first applied this to large-scale instruction generation using unlabeled data, with Suri~\cite{pham2024suri} and Kun~\cite{zheng2024kun} extending it to long-form and Chinese instructions, respectively. ~\citet{nguyen2024better} enhanced this method by adding quality assessment to filter and revise data. Building on this, we further investigated methods to generate high-quality complex instruction dataset using back-translation.





\section{Framework: DAMCS}\label{sec:method}
% \carlee{We should change the title to something specific to our method, e.g., \framework}

In this section, we give an overview of our framework. We first formally define how this framework interacts with our problem environment (Section~\ref{sec:setting}) and then describe the design of our multi-modal, adaptive memory system (Section~\ref{sec:memory}), structured LLM output for making agent decisions (Section~\ref{sec:output}) and communication protocol that enables agent cooperation (Section~\ref{sec:communication}).
\subsection{Problem Setting}\label{sec:setting} 
Our goal is to demonstrate that Large Language Models (LLMs) can effectively plan, coordinate, and execute tasks in a multi-agent environment where collaboration and resource management are critical. % In our extension of the Crafter environment \cite{hafner2021benchmarking}, 
We consider an environment model that follows a Decentralized Partially Observable Markov Decision Process (Dec-POMDP)~\cite{bernstein2002complexity,chen2024rgmcomm}, as is common in cooperative MARL, where agents lack complete information about the environment and have only local observations. Figure \ref{fig:framework} gives an overview of this framework. We model the environment as a Dec-POMDP with communication as a tuple $D=\langle I, n, S, A, P, \Omega, O, g, R \rangle$, where $I = \{1,2,\dots,n\}$ is a set of $n$ agents, $S$ is the joint \textbf{state} space, and $A=A_1\times A_2 \times \dots \times A_n$ is the joint \textbf{action} space, where $\boldsymbol{a}=(a_1,a_2,\dots,a_n)\in A$ denotes the joint action of all agents. $P(\boldsymbol{s}'|\boldsymbol{s},\boldsymbol{a}): S \times A \times S \to [0,1] $ is the \textbf{state transition function} that describes how the environment state evolves, given the actions taken by the agents.

We consider an episode that is divided into a series of timeslots $t = 1,2,\ldots$; at the start of each episode, agents respawn in the center of the map. Within each timeslot, each agent can take an \textit{action}, e.g., sharing resources with another agent or working towards a goal. 
%\carlee{is this right? Or is the action simply deciding which direction to travel in?} \hq{Yes. The action space includes moving to different directions, sharing, do, crafting tools, etc.}
Agents decide their action based on their observations, which are contained in the \textbf{observation} space $\Omega$, and $O(\boldsymbol{s}, i): S \times I \to \Omega$ denotes the function that maps from the joint state space to distributions of observations for each agent $i$.
Each agent's observations, as shown in Figure~\ref{fig:framework}, include its own environment input, as well as communication messages from the other agents. We use $g: \Omega \to M$ to denote the \textbf{communication message generation function} that each agent $j$ uses to encode its local observation $o_j$ into a communication message for other agents $i \neq j$. 
We use $\boldsymbol{m_{-i}}=\{m_{j}=g(o_j), \forall j \neq i\}$ to denote the collection of messages agent $i$ receives from all other agents $j \neq i$. 

 
% $\Omega$ is the \textbf{observation} space. $O(\boldsymbol{s}, i): S \times I \to \Omega$ is a function that maps from the joint state space to distributions of observations for each agent $i$. 
In deciding which actions to take, the agents' goal is to maximize the long-term reward. More formally, they aim to find a policy $\pi$ that maximizes the average expected return $\lim_{T \to \infty} (1/T) E_{\pi} [{\sum_{t=0}^T R_{t}}]$, where $R(\boldsymbol{s}, \boldsymbol{a}): S \times A \to \mathbb{R}$ is the reward of the current state $\boldsymbol{s}$ and joint action $\boldsymbol{a}$ and $R_t$ is the reward incurred in timeslot $t$. As shown in Figure~\ref{fig:framework}, this policy goal is enforced in our framework by including it in a prompt that is fed to a \textbf{multi-modal large language model (MLLM)} along with a prompt to generate plans and actions for the current timestep, thus forming the policy $\pi$. For example, Agent 6 in Figure~\ref{fig:framework} is told to find a diamond.
% and $\gamma$ is the discount factor. 
To ensure the LLM finds a good policy based on historical data, each agent maintains its own memory, consisting of  both \textbf{Short-Term Working Memory (\textbf{STWM})} and \textbf{Long-Term Memory (LTM)}. The STWM holds information for decision-making at the current timestep, combining current environmental perceptions with relevant information retrieved from LTM. The STWM is then included in the MLLM prompt. % fed into a \textbf{multi-modal large language model (MLLM)} along with a prompt to generate plans and actions for the current timestep, thus forming the policy $\pi$. 
The STWM and MLLM responses are then consolidated into the agent’s LTM, enabling agents to make strategic decisions based on historical context.

% In Dec-POMDP with communications, each agent $i$ considers an individual policy $\pi_i(a_i|o_i,\boldsymbol{m_{-i}})$ conditioned on local observation $o_i$ and messages $\boldsymbol{m_{-i}}$, i.e., $\pi=[\pi_i(a_i|o_i,\boldsymbol{m_{-i}}),\forall i]$.
% The objective is to find a policy $\pi$ that maximizes the average expected return $J(\pi) =\lim_{T \to \infty} (1/T) E_{\pi} [{\sum_{t=0}^T R_{t}}]$. The core system is structured to enable agents to learn from past interactions and \textcolor{orange}{transfer} those experiences to \textcolor{orange}{learn to collaborate in }new scenarios.

% \textcolor{orange}{Jingdi: I added this section, but I think the message generation function could not be highlighted here.}
% \carlee{Maybe we can define it and say it is connected to the communication protocol}

\begin{figure}[h]
  \centering
  \includegraphics[width=1\linewidth]{AnonymousSubmission/LaTeX/figures/framework.pdf}
  \caption{Framework Overview. Multiple agents respawn on the map and interact with each other through a memory system and communication protocol, aiming to collect a diamond as fast as possible.}
  \label{fig:framework}
  %\Description{Framework Overview.}
\end{figure}

% \subsection{Framework Overview}
% Our decentralized cooperative agents operate within a modified \textit{Dec-POMDP} framework, where each agent receives partial observations and makes decisions independently. Agents collaborate by sharing resources and updating each other on their goals and progress. The core system is structured to enable agents to learn from past interactions and \textcolor{orange}{transfer} those experiences to \textcolor{orange}{learn to collaborate in }new scenarios.

% \carlee{Integrate this with Section 3.1 (this seems to explain how agents take actions, while 3.1 explains how the environment evolves)}
% At the core of our framework is the interaction between working memory and long-term memory. Figure \ref{fig:framework} shows the framework. At the start of each episode, agents respawn in the center of the map. Each agent maintains its own memory, consisting of two components:  \textcolor{orange}{\textbf{Short-Term Working Memory (\textbf{STWM})}} and \textcolor{orange}{\textbf{Long-Term Memory (LTM)}}. The \textbf{STWM} holds information for decision-making at the current timestep, combining current environmental perceptions with relevant information retrieved from LTM. The \textbf{STWM} is fed into a \textbf{multi-modal large language model (MLLM)} along with a prompt to generate plans and actions for the current timestep. The \textbf{STWM} and MLLM responses are then consolidated into the agent’s LTM, enabling agents to make strategic decisions based on historical context. \carlee{refer to Sections 3.3 and 3.4 here for details of the memory structure}

\subsection{Adaptive Knowledge Graph Memory System}\label{sec:memory}
Recent work in multi-task learning has demonstrated the benefits of integrating heterogeneous data sources for optimized decision-making \cite{baltruvsaitis2018multimodal, ngiam2011multimodal, xu2024predicting}. In the proposed \textbf{Adaptive Knowledge Graph Memory System (A-KGMS)}, inspired by human cognitive processes \cite{sumers2023cognitive}, each agent uses a \textit{multi-modal memory system} combining short-term and long-term memories that facilitates storing and retrieving experiences across different memory types. While existing memory systems focus on aspects like semantic understanding \cite{li2024optimus}, our system is goal-oriented.
%\carlee{How does this compare to existing LLM memory systems (do any exist)?} 
This memory system allows agents to learn from past experiences, facilitating task completion in open-world environments. %\carlee{how is the memory shared across agents?} %is essential for enabling agents to learn from past interactions and apply those experiences to new scenarios.
% \carlee{refer to the figure more in explaining short term and long term memory}

%As illustrated in Figure~\ref{fig:memory_system}, the memory system is divided into two main components: \textit{working memory} and \textit{long-term memory}. The working memory captures immediate environmental inputs, while the long-term memory retains historical experiences and knowledge. We describe the components of our memory system in detail:

\begin{figure}[h]
  \centering
  \includegraphics[width=1\linewidth]{AnonymousSubmission/LaTeX/figures/memory.pdf}
  \caption{Memory System. 
  The system consists of \textit{working memory} and \textit{long-term memory}. \textit{Sensory inputs} (1) are captured in \textit{working memory} (2), alongside relevant information retrieved from \textit{long-term memory} (4). The agent 'thinks' using an \textit{MLLM} (3) to generate responses and action plans, which are then stored in long-term memory. A \textit{consolidation process} updates the \textit{goal-oriented hierarchical knowledge graph} (5), linking new experiences to past events. This graph comprises \textit{experience nodes} $E$, \textit{goal nodes} $G$, and \textit{long-term goal nodes} $LTG$.}
  \label{fig:memory_system}
  \vspace{-0.1in}
  %\Description{Memory System.}
\end{figure}
%The system is divided into two main components: working memory and long-term memory. The \textit{sensory inputs} (1) from the environment are captured in the \textit{working memory} (2), along with relevant information retrieved from the long-term memory (4). The agent then "thinks" by feeding the working memory into a \textit{MLLM} (3) along with a prompt to generate responses and action plans. The responses, along with the working memory, are then stored in the experience pool in \textit{long-term memory} (4). A consolidation process is then triggered to update the \textit{goal-oriented hierarchical knowledge graph} (5), connecting the current experience with past events. The knowledge graph consists of experience nodes $E$, goal nodes $G$, and long-term goal nodes $LTG$.
\textbf{Experience.} The \textbf{experience} for each time step in a learning episode consists of two stages: \textbf{pre-stage} and \textbf{post-stage}, as shown in Parts 2 and 3 of Figure~\ref{fig:memory_system} The \textbf{pre-stage} refers to the information available to the agent at the current timestep for decision-making. The \textbf{post-stage} is the thought process generated by the language model, then consolidated into \textbf{Long-Term Memory}. The post-stage contains full information, including environment cues and the agent's thoughts, which help generalize actions in similar scenarios by emphasizing decision-making and consequences.
%\carlee{explain the rationale for this design}

%Each episode consists of multiple timesteps, and during each timestep, an \textbf{experience} is created, which consists of two stages: \textbf{pre-stage} and \textbf{post-stage}. The information that is available for the agent to use in making a decision at the current timestep is the pre-stage. The thought process, which is the response generated from the language model \textcolor{blue}{[what types of questions does the prompt ask the agents?]}, is referred to as the post-stage, which is then consolidated into the long-term memory.

\textbf{Short-Term Working Memory (STWM, Part 2 of Figure~\ref{fig:memory_system}).} STWM refers to the pre-stage experience and consists of four parts: (\rom{1}). \textbf{Sensory memory} captures raw environmental observations, such as visual inputs and communication messages; (\rom{2}). \textbf{Episodic memory} stores contextual details, including the agent's health, location, time, and inventory; (\rom{3}). \textbf{Feedback}, retrieved from long-term semantic and procedural memory, provides available actions and their prerequisites; (\rom{4}). \textbf{Retrospection} offers context from the hierarchical knowledge graph, including recent events, achievements, goals, and progress. STWM, along with a prompt, is processed by a multi-modal large language model (MLLM) to help the agent `think' and `plan' its next action.

%\textbf{\textcolor{orange}{Short-Term Working Memory (\textbf{STWM})}} The working memory is also referred to as the pre-stage experience. The \textbf{sensory memory} refers to the raw observations from the environment, including visual input and communication messages. The \textbf{episodic memory} stores the \textcolor{blue}{episode's} contextual information, including the agent's health stats, location, time, and inventory items. The \textbf{feedback} \textcolor{blue}{is the agent's available actions and these action's prerequisites. This information} is retrieved from the long-term memory, specifically from the semantic memory and procedural memory. %, which provide the agent's available actions and their prerequisites. 
%Finally, \textbf{retrospection} contains information retrieved from the hierarchical knowledge graph from the long-term memory, providing more contextual information, such as recent events, past accomplishments, goals, and current progress. The working memory is fed into a multi-modal large language model (MLLM) along with a prompt, allowing the agent to "think" and make plans to determine the action to take.

\textbf{Long-Term Memory (LTM, Part 4 of Figure~\ref{fig:memory_system}).} LTM consists of an experience pool of post-stage experiences. A consolidation process updates the goal-oriented hierarchical knowledge graph (further explained below) by organizing experiences according to their goals, connecting current experiences with past events and allowing agents to access memories useful to their short- and long-term goals.
%\carlee{and allowing agents to access memories useful to their short- and long-term goals}. 
\textbf{Semantic memory} holds factual knowledge, specifically the hierarchical crafting tree of the environment, which is programmed explicitly using logical expressions. This factual knowledge provides accurate feedback on action prerequisites, 
%\textcolor{blue}{[what kinds of factual knowledge, and how was it obtained?]} about the environment, providing accurate feedback on action prerequisites, 
while \textbf{procedural memory} stores all available actions. The consolidation process is triggered whenever a new experience is added, updating the hierarchical knowledge graph.

%\textcolor{orange}{\textbf{Long-Term Memory (LTM)}} The long-term memory is composed of an \textbf{experience pool} of post-stage experiences. A consolidation process is then triggered to update the \textbf{goal-oriented hierarchical knowledge graph} \textcolor{blue}{through organizing the post-stage experiences according to their respective goals}, hence connecting the current experience with past events. The \textbf{semantic memory} consists of factual knowledge \textcolor{blue}{[what kinds of factual knowledge, and how was it obtained?]} about the environment that can provide accurate feedback to the agent on the prerequisites of actions. The \textbf{procedural memory} retains all available actions in the environment. The consolidation process happens when a new experience is added to the long-term memory, which involves updating the hierarchical knowledge graph.

\textbf{Goal-Oriented Hierarchical Knowledge Graph (Part 5 of Figure~\ref{fig:memory_system}).} % As shown in Figure~\ref{fig:memory_system}, 
The agent maintains an adaptive goal-oriented hierarchical knowledge graph within its LTM. Each node represents an experience ($E$), and nodes are linked sequentially based on goal-related sequences, reflecting the agent's progress. We link each experience node to a goal node corresponding to the goal it tries to achieve, derived from the LLM output.
%\carlee{We link each experience node to a goal node corresponding to the goal it tries to achieve, derived from the LLM output.} % A specific algorithm \textcolor{blue}{[state the algorithm or technique that helps us link the experience according to goals]} links \textbf{experience} to goals. 
When a new goal begins, a new \textbf{goal node} ($G$) is created and connected to the previous one, forming a sequence that tracks the agent's journey. A higher-level \textbf{Long-Term Goal node} ($LTG$) is generated from goal nodes, providing an overview of the agent’s long-term progress. At the end of the \textbf{consolidation process}, a summary is updated for the most recent goal node, including the long-term goal, current goal, past goals, and recent experiences. \textbf{At the planning stage}, the agent retrieves information from the most recent goal node ($G$) and combines it with pre-stage experiences $\boldsymbol{E}$ to form its STWM. This enables the agent to reason and make decisions by integrating past and present data, as well as adjusting strategies in real-time to optimize progress toward current and long-term goals.

%\textbf{Goal-Oriented Hierarchical Knowledge Graph.} As depicted in Figure~\ref{fig:memory_system}, the agent maintains an adaptive goal-oriented hierarchical knowledge graph (Part 5 of Figure~\ref{fig:memory_system}) within its long-term memory. Each node in the knowledge graph represents an experience ($E$), and nodes are connected based on goal-related sequences. When the agent is working on a goal, experience nodes ($E$) are linked sequentially, \textcolor{blue}{[state the algorithm or technique that helps us link the experience according to goals]} reflecting the agent's progress on that goal. Upon starting a new goal, a goal node ($G$) is created and linked to the previous goal node, forming a sequence of goals that records the agent’s journey. On top of the goal node is the long-term goal node ($LTG$), which is generated based on goal nodes in a similar way. Long-term goal nodes provide a higher-level view of the agent’s overall progress, guiding the agent toward its long-term objectives. At the end of the consolidation process, a summary will be updated for the most recent goal node: long-term goal, current goal, past accomplished goal, and recent experiences toward completing the current goal.

%When planning, the agent retrieves relevant information from the most recent goal node and combines it with pre-stage experiences to form the working memory. This helps the agent reason and make decisions based on both past and present information, adjusting strategies in real-time and optimizing progress toward both current and long-term goals.

\subsection{Structured Reasoning Output}\label{sec:output}
%Converting unstructured inputs into structured data is crucial for developing multi-step agent workflows that enable LLMs to perform actions \cite{pokrass2023structured}. Structured outputs provide a framework that constrains language models to adhere to predefined schemas. In our reasoning process, we utilize structured prompting techniques to achieve this. We employ a carefully tuned structured output format along with an environment explanation as the prompt. This prompt organizes the working memory components into actionable insights, enabling the agent to generate well-informed decisions.
% \carlee{Is this about how agents process the outputs of the LLM?}
%\textcolor{blue}{[maybe some brief examples of what unstructured input (free flow text?) vs structured data/outputs are. cause the above paragraph is kinda abstract. an alternative is to put the specifics described below first.]}
%\textcolor{blue}{[if this structured output helps reduce communication required, we could mention it too.]}

Converting unstructured inputs, such as free-form text, into structured data is crucial for developing multi-step agent workflows that enable LLMs to perform actions \cite{pokrass2023structured}. Structured outputs provide a framework that constrains language models to follow predefined \textbf{schemas}. For example, instead of processing unstructured text like \textit{`The agent moved north to pick up a key'}, we format it into structured data such as \textit{`[Action: Move North, Reason: Pick up a key]'}. We utilize structured prompting techniques, combining a carefully tuned output format with environment explanations, to organize working memory into actionable insights. This reduces communication needs and helps the agent make well-informed decisions. Meanwhile, the number of output tokens is significantly reduced due to formatted and focused responses, resulting in faster generation speed.

\textbf{Schemas.} The schemas are built around three core components: (\rom{1}) \textbf{Reflection}, which enables agents to review recent actions, summarize outcomes, and reflect on lessons learned to adjust future strategies; (\rom{2}) \textbf{Goal}, which tracks both current and long-term objectives, including sub-goals and progress updates, helping the agent stay focused and break down tasks into manageable steps; and (\rom{3}) \textbf{NextAction}, which determines the agent’s upcoming actions and the reasoning behind them, evaluating prerequisites and ensuring alignment with both short-term and long-term goals. Each component is represented by a data class with fields specifying required responses and data types, using the Python \textit{Pydantic} library.

%\textbf{Schemas.} The schemas are defined by three core components: NextAction, Reflection, and Goal. Each component is represented by a data class with fields specifying required responses and data types using the Python Pydantic library.

% \mycodebox[red!20!white]{
% class NextAction(BaseModel):\\
% \hspace*{5mm}next\_action: ActionType\\
% \hspace*{5mm}next\_action\_reason: str\\
% \hspace*{5mm}final\_next\_action: ActionType\\
% \hspace*{5mm}final\_next\_action\_reason: str\\
% }

% \mycodebox[blue!20]{%
% class Reflection(BaseModel):\\
% \hspace*{5mm}vision: list[MaterialType]\\
% \hspace*{5mm}last\_action: ActionType\\
% \hspace*{5mm}last\_action\_result: ResultType\\
% \hspace*{5mm}last\_action\_result\_reflection: str\\
% \hspace*{5mm}last\_action\_repeated\_reflection: str\\
% }

% \mycodebox[yellow!20]{%
% class Goal(BaseModel):\\
% \hspace*{5mm}ultimate\_goal: LongTermGoalType\\
% \hspace*{5mm}long\_term\_goal: LongTermGoalType\\
% \hspace*{5mm}long\_term\_goal\_subgoals: str\\
% \hspace*{5mm}long\_term\_goal\_progress: GoalType\\
% \hspace*{5mm}current\_goal: GoalType\\
% \hspace*{5mm}current\_goal\_reason: str\\
% }

%The \textbf{Reflection} component enables agents to utilize their working memory by reviewing their most recent actions. It summarizes these actions, their outcomes, and the agent’s reflections on the results, helping identify lessons learned and adapt future strategies accordingly. The \textbf{Goal} component tracks both the agent’s current and long-term objectives, including sub-goals and progress updates. This helps the agent stay focused on overarching goals while managing immediate tasks. By linking current objectives with long-term plans, this component allows agents to articulate their goals and break them down into manageable sub-goals. The \textbf{NextAction} component determines the agent's upcoming actions and the reasoning behind them. It allows agents to evaluate the prerequisites for their next move, why they chose it, and how it aligns with both their current and long-term goals.
\subsection{Structured Communication System}\label{sec:communication}
\begin{figure}[h]
  \centering
  \includegraphics[width=0.8\linewidth]{AnonymousSubmission/LaTeX/figures/communication.pdf}
  \caption{Communication Protocol. Agents collaborate by exchanging messages to coordinate tasks and share resources. An arrow from agent $i$ to agent $j$ indicates that agent $i$ is helping agent $j$; communication then flows in the opposite direction.}%\carlee{An arrow from agent $i$ to agent $j$ indicates that agent $i$ is helping agent $j$; communication then flows in the opposite direction.}}
  \label{fig:communication_protocol}
  %\Description{Communication protocol.}
    \vspace{-0.1in}
\end{figure}

%In a multi-agent environment, communication between agents is crucial for achieving efficient cooperation and collaboration. Our communication framework allows agents to share their current status, resource availability, crafting progress, and requests for assistance, with a hierarchical focus—each agent prioritizes helping its preceding agent.

In a multi-agent environment, communication is key for effective cooperation. Our communication framework, consisting of message generation modules $g=\{g_1,\dots,g_n\}$ for all agents, where $m_i = g_i (o_i,rs_i,c_i,rq_i)$, enables agents to share their current observations $o_i$, includes status $s_i$, resource availability $rs_i$, short-term goal %\carlee{this should be more generic, maybe call it ``current short-term goal''} 
$c_i$, and assistance requests $rq_i$. This follows a hierarchical structure, where each agent $i$ prioritizes helping the preceding agent $i-1$.

We propose a novel \textbf{Collaboration} schema $\boldsymbol{C_i}=\Phi(h_i, I_i, \Delta p_i)$ for each agent $i$ and add this to the structured outputs, which is based on the target agents $h_i$ who needs help from agent $i$, intentions $I_i$ to assist target agents from agent $i$, and how the collaboration impacts agent $i$'s current plan, denoted by $\Delta p_i$. In our multi-agent system, the message generation function $g_i$ can be augmented by incorporating the collaboration schema $\boldsymbol{C_i}$ to refine and guide the message generation process, then the message generation process is enhanced by the information encoded in $\boldsymbol{C_i}$, i.e., $m_i = g_i (o_i,rs_i,c_i,rq_i, C_i)$. Therefore, the Collaboration schema enables agents to interpret and generate actions $a_i=\pi_i(o_i,\boldsymbol{m}_{-i})$, where $\boldsymbol{m_{-i}}=\{m_{j}=g(o_j), \forall j \neq i\}$ to denote the collection of messages agent $i$ receives from all other agents $j \neq i$. This structure ensures that our collaborative agents act in a goal-oriented manner with collaboration as a key consideration. 

%A new schema, \textbf{Collaboration}, has been added to the structured output. This schema enables agents to focus on interpreting and generating actions based on who needs help, how to help, and how the collaboration affects their current plan. This ensures that agents act in a goal-oriented manner, considering collaboration.

%\textcolor{orange}{As illustrated in Figure~\ref{fig:communication_protocol}, agents collaborate by communicating and sharing resources, using message generation modules $g={g_1, \dots, g_n}$ to coordinate actions such as task allocation and resource sharing. Agents are ordered from 1 to $n$, with each agent $i$ responsible for assisting the preceding agent $i-1$ and the leader agent $1$ with communication module $g_1$. The first agent's communication module, $g_1$, acts as the leader, tasked with crafting essential tools and distributing them to other agents as needed. The second agent's communication module, $g_2$, focuses on gathering materials and aiding $g_1$ with crafting tasks. The last agent's communication module, $g_n$, is responsible for supporting $g_{n-1}$ and eventually shifting its focus to finding a diamond, deciding when to switch from assisting other agents to locating the diamond based on the collaboration schema $\boldsymbol{C}_n=\Phi(h_n, I_n, \Delta p_n)$. This protocol is simple yet effective in a hierarchical environment, parallelizing tasks and encouraging cooperation among agents while maintaining low communication costs. Since the environment is hierarchical, this collaborative approach effectively speeds up the crafting process and naturally scales with any arbitrary number of agents $n$.}

\textbf{An Illustrative Example.} As illustrated in Figure~\ref{fig:communication_protocol}, agents collaborate by communicating and sharing resources through message generation modules $g={g_1, \dots, g_n}$ to coordinate tasks like allocation and resource sharing. Agents are ordered from 1 to $n$, with each agent $i$ assisting the preceding agent $i-1$ and the leader agent $1$. The first agent, acts as the leader, crafting essential tools and distributing them to others. The second agent gathers materials and assists the agent $1$ with crafting. The last agent $n$, supports agent $n-1$ and eventually shifts its focus to finding a diamond, deciding when to switch goals using the collaboration schema $\boldsymbol{C}_n=\Phi(h_n, I_n, \Delta p_n)$. This simple yet effective protocol works in hierarchical environments by parallelizing tasks, fostering cooperation, and keeping communication costs low. It naturally scales with any number of agents $n$, speeding up the crafting process.

%As illustrated in Figure~\ref{fig:communication_protocol}, agents collaborate by communicating and sharing resources. They exchange messages to coordinate actions such as task allocation and resource sharing. Agents are ordered from 1 to $N$, with each agent designated to assist the preceding agent and the leader agent. The first agent is the leader agent, responsible for crafting tools and sharing them with other agents in need. \textcolor{blue}{[personal take: can be even more specific, on what tasks the first and second agent does]} The last agent is tasked with helping its previous agent and finding a diamond, determining when to switch its goal from assisting agents to locating the diamond. This protocol is simple yet effective in a hierarchical environment, parallelizing tasks and encouraging cooperation among agents while maintaining low communication costs. Since the environment is hierarchical, this collaborative approach effectively speeds up the crafting process and naturally scales with any arbitrary number of agents.
%\textcolor{blue}{[personal take: can be even more specific, on what tasks the first and second agent does]} 
%\textcolor{blue}{[it may not scale at times right, depending on the task. sometimes less is better]}

% \subsection{Libraries and Packages}
% Our implementation leverages several key libraries and packages to build the decentralized cooperative generative agents framework. We use Python 3.10.14, along with Pydantic 2.9.2 for structured output, OpenAI 1.44.1, and Torch 2.3.1. Additionally, we use GPT-4o (2024-08-06 version) as the backbone language model. \carlee{I'd move this to the experiments section or appendix. Also say here that we will release this code publicly}

% \textcolor{orange}{Jingdi: could move this to appendix}

% \mycodebox[green!20]{%
% class ResponseEvent(BaseModel):\\
% \hspace*{5mm}episode\_number: int\\
% \hspace*{5mm}timestep: int\\
% \hspace*{5mm}past\_events: str\\
% \hspace*{5mm}current\_inventory: list[InventoryItemsCount]\\
% \hspace*{5mm}collaboration: Collaboration\\
% \hspace*{5mm}reflection: Reflection\\
% \hspace*{5mm}goal: Goal\\
% \hspace*{5mm}action: NextAction\\
% \hspace*{5mm}summary: str\\
% }

\hq{

\section{Evaluation Challenges of LLM Agents}\label{sec:evaluation_challenges}
Evaluating LLM-powered multi-agent systems presents unique challenges. Unlike MARL-based agents, which are trained to optimize carefully crafted rewards, LLM agents rely on prompts and contextual information, making them highly adaptable but sensitive to the evaluation environment.

\textbf{Limitations of Existing Environments.} Existing multi-agent benchmarks are often too simple for meaningful collaboration~\cite{terry2021pettingzoo} or too complex~\cite{berner2019dota,vinyals2019grandmaster,fan2022minedojo}. Many focus on micro-level action management, whereas our work emphasizes macro-level planning, communication, and cooperation. Furthermore, MARL frameworks are known for scalability challenges, and existing environments are often not designed to support cooperative tasks that scale well with an increasing number of agents.

\textbf{Evaluation of Cooperation.} LLM-based collaboration is highly adaptable but difficult to quantify. Unlike RL agents that optimize reward signals, LLM-based collaboration relies on context and commonsense reasoning, making responses variable. No standardized metric exists for evaluating cooperation among LLM agents, and extensive modifications to benchmarks are often required. Testing with environment-specific prompts is also time-consuming.

\textbf{Quantifying LLM Agents' Capabilities.} Evaluating memory quality and adaptability in LLM agents is non-trivial. While our \textbf{A-KGMS} organizes past experiences, determining the quality of stored information and its impact on decision-making remains challenging. Adaptability is also difficult to measure, as LLM agents adjust dynamically rather than optimizing predefined objectives.

To address these challenges, we introduce Multi-Agent Crafter to evaluate strategic coordination, planning, and resource sharing in open-ended, scalable cooperative tasks.
}

\section{Multi-Agent Crafter: A Novel Testbed}\label{sec:crafter}
% \subsection{Environment Details} 
%The original Crafter environment \cite{hafner2021benchmarking} is a procedurally generated, open-world survival game designed for benchmarking reinforcement learning (RL) algorithms. It features a grid world with a discrete action space of size 17 and provides information on the player's inventory, health, food, water, and crafting progress. Crafter includes 22 achievements organized in a tech tree with a depth of 7, with the ultimate goal of exploring the environment. Inspired by Minecraft, Crafter simplifies the game’s mechanics to enable faster experimentation and results collection.

The original Crafter environment \cite{hafner2021benchmarking} is a procedurally generated, open-world survival game used to benchmark RL algorithms. It features a 17 discrete action grid world and tracks player metrics like inventory, health, and crafting progress, with 22 achievements organized in a 7-depth tech tree. Inspired by Minecraft, Crafter simplifies game mechanics for faster experimentation and results collection.
\hq{We proposed a novel multi-agent Crafter for multi-agent tasks, enabling cooperative agent interaction and introducing new actions and challenges. These changes, shown in Figure \ref{fig:crafter}, make the environment suitable for studying multi-agent cooperation. Key modifications are outlined below.}

%We have made several significant modifications to transform Crafter into a robust multi-agent environment. These changes allow agents to interact cooperatively within the environment and introduce additional challenges, making it more suitable for studying multi-agent cooperation, as shown in Figure \ref{fig:crafter}. The key changes are outlined below:
\hq{
\textbf{A Scalable Cooperative Environment.} We extended the Crafter environment to support an arbitrary number of agents, each with independent observations, inventories, and health stats, enabling cooperative agent interaction and introducing new actions and challenges (Figure~\ref{fig:crafter}). Agents can collaborate by sharing resources, coordinating actions, and balancing individual roles to achieve collective goals efficiently. Unlike traditional MARL environments, which often focus on micro-level action management, our testbed is designed to evaluate strategic planning, coordination, and shared decision-making.

Our environment allows agents to share items, including resources and tools, fostering teamwork by enabling task delegation and resource management. Crafting dependencies and environmental prompts can be easily customized, increasing task complexity with more participants. This ensures that agents must coordinate and efficiently allocate roles, enabling effective large-scale parallel collaboration. The flexible design makes the testbed suitable for evaluating cooperative behavior potentially for any number of agents.

\textbf{Evaluation of Cooperation and LLM Agents' Capabilities.}  
Unlike the original Crafter environment, which focused on open-ended exploration, we define a clear objective: agents must collaborate to craft necessary tools and obtain a diamond as quickly as possible while managing their needs for food, water, and energy. This setup allows us to evaluate whether agents can effectively cooperate and reason toward both short- and long-term goals, making the environment ideal for testing multi-agent coordination, planning, and resource optimization.

To assess cooperative efficiency, agents share resources and tools, requiring negotiation, task division, and decision-making. Unlike previous MARL settings, where collaboration is forced or predefined, our testbed allows agents to develop teamwork strategies. Our environment quantifies multi-agent cooperation through indirect measurements, such as tracking the steps an agent takes to craft items, providing insights into decision-making and adaptability.

\textbf{Support for Language Agents.} We added a navigation skill that allows agents to move toward specific resources, reducing the burden of manual low-level movement control. This enables agents to focus on higher-level decision-making, such as strategic planning and collaboration.

\textbf{Customizability and Compatibility.}
Our multi-agent Crafter environment is designed to be highly flexible and extensible, supporting RL, MARL, and LLM-powered agents. The single-agent version follows the Gymnasium API, ensuring integration with standard RL libraries, while the multi-agent version aligns with the PettingZoo API, ensuring compatibility with existing MARL frameworks. We provide example training scripts for single-agent experiments using Stable-Baselines3 (SB3) and multi-agent experiments using AgileRL, allowing researchers to efficiently test new ideas, integrate with existing RL libraries, and adapt the environment for diverse multi-agent challenges.
}


% While this paper evaluates our decision-making framework using the multi-agent Crafter environment, we believe it will also serve as a valuable platform for future research on multi-agent coordination, communication, long-term planning, and resource optimization in complex, real-time multi-agent environments.









\section{Evaluations}\label{sec:evaluation}
% \begin{table}[!t]
% \centering
% \scalebox{0.68}{
%     \begin{tabular}{ll cccc}
%       \toprule
%       & \multicolumn{4}{c}{\textbf{Intellipro Dataset}}\\
%       & \multicolumn{2}{c}{Rank Resume} & \multicolumn{2}{c}{Rank Job} \\
%       \cmidrule(lr){2-3} \cmidrule(lr){4-5} 
%       \textbf{Method}
%       &  Recall@100 & nDCG@100 & Recall@10 & nDCG@10 \\
%       \midrule
%       \confitold{}
%       & 71.28 &34.79 &76.50 &52.57 
%       \\
%       \cmidrule{2-5}
%       \confitsimple{}
%     & 82.53 &48.17
%        & 85.58 &64.91
     
%        \\
%        +\RunnerUpMiningShort{}
%     &85.43 &50.99 &91.38 &71.34 
%       \\
%       +\HyReShort
%         &- & -
%        &-&-\\
       
%       \bottomrule

%     \end{tabular}
%   }
% \caption{Ablation studies using Jina-v2-base as the encoder. ``\confitsimple{}'' refers using a simplified encoder architecture. \framework{} trains \confitsimple{} with \RunnerUpMiningShort{} and \HyReShort{}.}
% \label{tbl:ablation}
% \end{table}
\begin{table*}[!t]
\centering
\scalebox{0.75}{
    \begin{tabular}{l cccc cccc}
      \toprule
      & \multicolumn{4}{c}{\textbf{Recruiting Dataset}}
      & \multicolumn{4}{c}{\textbf{AliYun Dataset}}\\
      & \multicolumn{2}{c}{Rank Resume} & \multicolumn{2}{c}{Rank Job} 
      & \multicolumn{2}{c}{Rank Resume} & \multicolumn{2}{c}{Rank Job}\\
      \cmidrule(lr){2-3} \cmidrule(lr){4-5} 
      \cmidrule(lr){6-7} \cmidrule(lr){8-9} 
      \textbf{Method}
      & Recall@100 & nDCG@100 & Recall@10 & nDCG@10
      & Recall@100 & nDCG@100 & Recall@10 & nDCG@10\\
      \midrule
      \confitold{}
      & 71.28 & 34.79 & 76.50 & 52.57 
      & 87.81 & 65.06 & 72.39 & 56.12
      \\
      \cmidrule{2-9}
      \confitsimple{}
      & 82.53 & 48.17 & 85.58 & 64.91
      & 94.90&78.40 & 78.70& 65.45
       \\
      +\HyReShort{}
       &85.28 & 49.50
       &90.25 & 70.22
       & 96.62&81.99 & \textbf{81.16}& 67.63
       \\
      +\RunnerUpMiningShort{}
       % & 85.14& 49.82
       % &90.75&72.51
       & \textbf{86.13}&\textbf{51.90} & \textbf{94.25}&\textbf{73.32}
       & \textbf{97.07}&\textbf{83.11} & 80.49& \textbf{68.02}
       \\
   %     +\RunnerUpMiningShort{}
   %    & 85.43 & 50.99 & 91.38 & 71.34 
   %    & 96.24 & 82.95 & 80.12 & 66.96
   %    \\
   %    +\HyReShort{} old
   %     &85.28 & 49.50
   %     &90.25 & 70.22
   %     & 96.62&81.99 & 81.16& 67.63
   %     \\
   % +\HyReShort{} 
   %     % & 85.14& 49.82
   %     % &90.75&72.51
   %     & 86.83&51.77 &92.00 &72.04
   %     & 97.07&83.11 & 80.49& 68.02
   %     \\
      \bottomrule

    \end{tabular}
  }
\caption{\framework{} ablation studies. ``\confitsimple{}'' refers using a simplified encoder architecture. \framework{} trains \confitsimple{} with \RunnerUpMiningShort{} and \HyReShort{}. We use Jina-v2-base as the encoder due to its better performance.
}
\label{tbl:ablation}
\end{table*}

\section{Results}
\label{sec:results}

In this section, we present detailed results demonstrating \emph{CellFlow}'s state-of-the-art performance in cellular morphology prediction under perturbations, outperforming existing methods across multiple datasets and evaluation metrics.

\subsection{Datasets}

Our experiments were conducted using three cell imaging perturbation datasets: BBBC021 (chemical perturbation)~\cite{caie2010high}, RxRx1 (genetic perturbation)~\cite{sypetkowski2023rxrx1}, and the JUMP dataset (combined perturbation)~\cite{chandrasekaran2023jump}. We followed the preprocessing protocol from IMPA~\cite{palma2023predicting}, which involves correcting illumination, cropping images centered on nuclei to a resolution of 96×96, and filtering out low-quality images. The resulting datasets include 98K, 171K, and 424K images with 3, 5, and 6 channels, respectively, from 26, 1,042, and 747 perturbation types. Examples of these images are provided in Figure~\ref{fig:comparison}. Details of datasets are provided in \S\ref{sec:data}.

\subsection{Experimental Setup}

\textbf{Evaluation metrics.} We evaluate methods using two types of metrics: (1) FID and KID, which measure image distribution similarity via Fréchet and kernel-based distances, computed on 5K generated images for BBBC021 and 100 randomly selected perturbation classes for RxRx1 and JUMP; we report both overall scores across all samples and conditional scores per perturbation class. (2) Mode of Action (MoA) classification accuracy, which assesses biological fidelity by using a trained classifier to predict a drug’s effect from perturbed images and comparing it to its known MoA from the literature.

\textbf{Baselines.} We compare our approach against two baselines, PhenDiff~\cite{bourou2024phendiff} and IMPA~\cite{palma2023predicting}, the only two baselines that incorporate control images into their model design --- a crucial setup for distinguishing true perturbation effects from artifacts such as batch effects. PhenDiff uses diffusion models to first map control images to noise and then transform the noise into target images. In contrast, IMPA employs GANs with an AdaIN layer to transfer the style of control images to target images, specifically designed for paired image-to-image mappings. Our method uses flow matching, which is tailored for distribution-to-distribution mapping, providing a more suitable solution for our problem. We reproduce these baselines with official codes.

\textbf{Training details.} \emph{CellFlow} employs a UNet-based velocity field with a four-stage design. Perturbations are encoded following IMPA~\cite{palma2023predicting}. Training is conducted for 100 epochs on 4 A100 GPUs. Details are in \S\ref{sec:experimental}.


\begin{table*}[ht!]
\centering
\footnotesize
\renewcommand{\arraystretch}{1.2}
\begin{tabular}{p{2.15cm} p{14.05cm}}
\toprule
 & \includegraphics[height=2.6cm]{figure/qualitative_example.pdf}\\
\midrule
Original \phantom{00000} Description & Bromazepam binds to the GABA receptor GABA-A, causing a conformational change and \textbf{increasing inhibitory effects of GABA}. Other neurotransmitters are not influenced. One of the \textbf{benzodiazepines} that is \textbf{used in the treatment of anxiety disorders.} \\
\midrule
User & Explain the features of the given molecule that are most distinct from those of other molecules. \\
\midrule
Mol-LLaMA (Ours) & The given molecule is a member of the \textcolor{tabblue}{benzodiazepine class}, which is characterized by a fusion of a benzene ring and a seven-membered diazepine ring. This structure is crucial for its biological activity, particularly its role as \textcolor{tabblue}{a sedative and hypnotic agent}. ...$<$omitted$>$...\\

& The given molecule is also characterized by a ketone group within the diazepine ring, which is a common feature in benzodiazepines. \textcolor{tabblue}{This functional group is important for the interaction with the GABA-A receptor, enhancing the inhibitory effects of GABA}. ...$<$omitted$>$...\\

\midrule
GPT-4o & 1. **Quinazoline Core**: The molecule contains a \textcolor{tabred}{quinazoline core}, ...$<$omitted$>$... \\
 & Overall, the combination of a \textcolor{tabred}{quinazoline} structure with unique substitution patterns, including bromine halogenation and the presence of a pyridinyl group, ...$<$omitted$>$... . \\
\midrule
LLaMo & The molecule has a \textcolor{tabblue}{benzodiazepine} structure with a bromo substituent at the 7-position and \textcolor{tabred}{a methyl group at the 1-position}. It is distinct from other molecules in that it contains a bromine atom, a nitrogen atom, and a methyl group, which are not present in the other molecules. \\
\midrule
3D-MoLM & ...$<$omitted$>$... It belongs to the class of compounds called \textcolor{tabred}{quinazolines}, which are characterized by a bicyclic structure consisting of a benzene ring fused to a pyrimidine ring. ...$<$omitted$>$...\\
& The molecule's structure suggests potential applications in medicinal chemistry, as quinazolines have been found to possess various biological activities, including \textcolor{tabred}{antitumor, antimicrobial, and anti-inflammatory properties.} \\
\midrule
Mol-Instructions & The molecule is a \textcolor{tabred}{quinoxaline derivative}. \\
\bottomrule
\end{tabular}
\vspace{-0.1in}
\caption{\small Case study to compare molecular understanding and reasoning ability. Mol-LLaMA accurately understands the molecular features, answering a correct molecular taxonomy and providing its distinct properties that are relevant to the given molecule.}
\label{tab:qualitative}
\vspace{-0.1in}
\end{table*}

\subsection{Main Results}

\textbf{\emph{CellFlow} generates highly realistic cell images.}  
\emph{CellFlow} outperforms existing methods in capturing cellular morphology across all datasets (Table~\ref{tab:results}a), achieving overall FID scores of 18.7, 33.0, and 9.0 on BBBC021, RxRx1, and JUMP, respectively --- improving FID by 21\%–45\% compared to previous methods. These gains in both FID and KID metrics confirm that \emph{CellFlow} produces significantly more realistic cell images than prior approaches.

\textbf{\emph{CellFlow} accurately captures perturbation-specific morphological changes.}  
As shown in Table~\ref{tab:results}a, \emph{CellFlow} achieves conditional FID scores of 56.8 (a 26\% improvement), 163.5, and 84.4 (a 16\% improvement) on BBBC021, RxRx1, and JUMP, respectively. These scores are computed by measuring the distribution distance for each specific perturbation and averaging across all perturbations.   
Table~\ref{tab:results}b further highlights \emph{CellFlow}’s performance on six representative chemical and three genetic perturbations. For chemical perturbations, \emph{CellFlow} reduces FID scores by 14–55\% compared to prior methods.
The smaller improvement (5–12\% improvements) on RxRx1 is likely due to the limited number of images per perturbation type.

\textbf{\emph{CellFlow} preserves biological fidelity across perturbation conditions.} 
Table~\ref{tab:ablation}a presents mode of action (MoA) classification accuracy on the BBBC021 dataset using generated cell images. MoA describes how a drug affects cellular function and can be inferred from morphology. To assess this, we train an image classifier on real perturbed images and test it on generated ones. \emph{CellFlow} achieves 71.1\% MoA accuracy, closely matching real images (72.4\%) and significantly surpassing other methods (best: 63.7\%), demonstrating its ability to maintain biological fidelity across perturbations. Qualitative comparisons in Figure~\ref{fig:comparison} further highlight \emph{CellFlow}’s accuracy in capturing key biological effects. For example, demecolcine produces smaller, fragmented nuclei, which other methods fail to reproduce accurately.

\textbf{\emph{CellFlow} generalizes to out-of-distribution (OOD) perturbations.}  
On BBBC021, \emph{CellFlow} demonstrates strong generalization to novel chemical perturbations never seen during training (Table~\ref{tab:ablation}b). It achieves 6\% and 28\% improvements in overall and conditional FID over the best baseline. This OOD generalization is critical for biological research, enabling the exploration of previously untested interventions and the design of new drugs.

\textbf{Ablations highlight the importance of each component in \emph{CellFlow}.}  
Table~\ref{tab:ablation}c shows that removing conditional information, classifier-free guidance, or noise augmentation significantly degrades performance, leading to higher FID scores. These underscore the critical role of each component in enabling \emph{CellFlow}’s state-of-the-art performance.  

\begin{figure*}[!tb]
    \centering
     \includegraphics[width=\linewidth]{imgs/interpolation.pdf}
     \vspace{-2em}
    \caption{
    \textbf{\emph{CellFlow} enables new capabilities.} 
\textit{(a.1) Batch effect calibration.}  
\emph{CellFlow} initializes with control images, enabling batch-specific predictions. Comparing predictions from different batches highlights actual perturbation effects (smaller cell size) while filtering out spurious batch effects (cell density variations).  
\textit{(a.2) Interpolation trajectory.}  
\emph{CellFlow}'s learned velocity field supports interpolation between cell states, which might provide insights into the dynamic cell trajectory. 
\textit{(b) Diffusion model comparison.}  
Unlike flow matching, diffusion models that start from noise cannot calibrate batch effects or support interpolation.  
\textit{(c) Reverse trajectory.}  
\emph{CellFlow}'s reversible velocity field can predict prior cell states from perturbed images, offering potential applications such as restoring damaged cells.
    }
    \label{fig:interpolation}
    \vspace{-1em}
\end{figure*}

\subsection{New Capabilities}

\textbf{\emph{CellFlow} addresses batch effects and reveals true perturbation effects.}  
\emph{CellFlow}’s distribution-to-distribution approach effectively addresses batch effects, a significant challenge in biological experimental data collection. As shown in Figure~\ref{fig:interpolation}a, when conditioned on two distinct control images with varying cell densities from different batches, \emph{CellFlow} consistently generates the expected perturbation effect (cell shrinkage due to mevinolin) while recapitulating batch-specific artifacts, revealing the true perturbation effect. Table~\ref{tab:ablation}d further quantifies the importance of conditioning on the same batch. By comparing generated images conditioned on control images from the same or different batches against the target perturbation images, we find that same-batch conditioning reduces overall and conditional FID by 21\%. This highlights the importance of modeling control images to more accurately capture true perturbation effects—an aspect often overlooked by prior approaches, such as diffusion models that initialize from noise (Figure~\ref{fig:interpolation}b).

\textbf{\emph{CellFlow} has the potential to model cellular morphological change trajectories.}
Cell trajectories could offer valuable information about perturbation mechanisms, but capturing them with current imaging technologies remains challenging due to their destructive nature. Since \emph{CellFlow} continuously transforms the source distribution into the target distribution, it can generate smooth interpolation paths between initial and final predicted cell states, producing video-like sequences of cellular transformation based on given source images (Figure~\ref{fig:interpolation}a). This suggests a possible approach for simulating morphological trajectories during perturbation response, which diffusion methods cannot achieve (Figure~\ref{fig:interpolation}b). Additionally, the reversible distribution transformation learned through flow matching enables \emph{CellFlow} to model backward cell state reversion (Figure~\ref{fig:interpolation}c), which could be useful for studying recovery dynamics and predicting potential treatment outcomes.


\section{Conclusion}\label{sec:conclusion}
\paragraph{Summary}
Our findings provide significant insights into the influence of correctness, explanations, and refinement on evaluation accuracy and user trust in AI-based planners. 
In particular, the findings are three-fold: 
(1) The \textbf{correctness} of the generated plans is the most significant factor that impacts the evaluation accuracy and user trust in the planners. As the PDDL solver is more capable of generating correct plans, it achieves the highest evaluation accuracy and trust. 
(2) The \textbf{explanation} component of the LLM planner improves evaluation accuracy, as LLM+Expl achieves higher accuracy than LLM alone. Despite this improvement, LLM+Expl minimally impacts user trust. However, alternative explanation methods may influence user trust differently from the manually generated explanations used in our approach.
% On the other hand, explanations may help refine the trust of the planner to a more appropriate level by indicating planner shortcomings.
(3) The \textbf{refinement} procedure in the LLM planner does not lead to a significant improvement in evaluation accuracy; however, it exhibits a positive influence on user trust that may indicate an overtrust in some situations.
% This finding is aligned with prior works showing that iterative refinements based on user feedback would increase user trust~\cite{kunkel2019let, sebo2019don}.
Finally, the propensity-to-trust analysis identifies correctness as the primary determinant of user trust, whereas explanations provided limited improvement in scenarios where the planner's accuracy is diminished.

% In conclusion, our results indicate that the planner's correctness is the dominant factor for both evaluation accuracy and user trust. Therefore, selecting high-quality training data and optimizing the training procedure of AI-based planners to improve planning correctness is the top priority. Once the AI planner achieves a similar correctness level to traditional graph-search planners, strengthening its capability to explain and refine plans will further improve user trust compared to traditional planners.

\paragraph{Future Research} Future steps in this research include expanding user studies with larger sample sizes to improve generalizability and including additional planning problems per session for a more comprehensive evaluation. Next, we will explore alternative methods for generating plan explanations beyond manual creation to identify approaches that more effectively enhance user trust. 
Additionally, we will examine user trust by employing multiple LLM-based planners with varying levels of planning accuracy to better understand the interplay between planning correctness and user trust. 
Furthermore, we aim to enable real-time user-planner interaction, allowing users to provide feedback and refine plans collaboratively, thereby fostering a more dynamic and user-centric planning process.




%\bibliographystyle{aaai25}
\bibliography{aaai25}

\section{Secure Token Pruning Protocols}
\label{app:a}
We detail the encrypted token pruning protocols $\Pi_{prune}$ in Figure \ref{fig:protocol-prune} and $\Pi_{mask}$ in Figure \ref{fig:protocol-mask} in this section.

%Optionally include supplemental material (complete proofs, additional experiments and plots) in appendix.
%All such materials \textbf{SHOULD be included in the main submission.}
\begin{figure}[h]
%vspace{-0.2in}
\begin{protocolbox}
\noindent
\textbf{Parties:} Server $P_0$, Client $P_1$.

\textbf{Input:} $P_0$ and $P_1$ holds $\{ \left \langle Att \right \rangle_{0}^{h}, \left \langle Att \right \rangle_{1}^{h}\}_{h=0}^{H-1} \in \mathbb{Z}_{2^{\ell}}^{n\times n}$ and $\left \langle x \right \rangle_{0}, \left \langle x \right \rangle_{1} \in \mathbb{Z}_{2^{\ell}}^{n\times D}$ respectively, where H is the number of heads, n is the number of input tokens and D is the embedding dimension of tokens. Additionally, $P_1$ holds a threshold $\theta \in \mathbb{Z}_{2^{\ell}}$.

\textbf{Output:} $P_0$ and $P_1$ get $\left \langle y \right \rangle_{0}, \left \langle y \right \rangle_{1} \in \mathbb{Z}_{2^{\ell}}^{n'\times D}$, respectively, where $y=\mathsf{Prune}(x)$ and $n'$ is the number of remaining tokens.

\noindent\rule{13.2cm}{0.1pt} % This creates the horizontal line
\textbf{Protocol:}
\begin{enumerate}[label=\arabic*:, leftmargin=*]
    \item For $h \in [H]$, $P_0$ and $P_1$ compute locally with input $\left \langle Att \right \rangle^{h}$, and learn the importance score in each head $\left \langle s \right \rangle^{h} \in \mathbb{Z}_{2^{\ell}}^{n} $, where $\left \langle s \right \rangle^{h}[j] = \frac{1}{n} \sum_{i=0}^{n-1} \left \langle Att \right \rangle^{h}[i,j]$.
    \item $P_0$ and $P_1$ compute locally with input $\{ \left \langle s \right \rangle^{i} \in \mathbb{Z}_{2^{\ell}}^{n}  \}_{i=0}^{H-1}$, and learn the final importance score $\left \langle S \right \rangle \in \mathbb{Z}_{2^{\ell}}^{n}$ for each token, where  $\left \langle S \right \rangle[i] = \frac{1}{H} \sum_{h=0}^{H-1} \left \langle s \right \rangle^{h}[i]$.
    \item  For $i \in [n]$, $P_0$ and $P_1$ invoke $\Pi_{CMP}$ with inputs  $\left \langle S \right \rangle$ and $ \theta $, and learn  $\left \langle M \right \rangle \in \mathbb{Z}_{2^{\ell}}^{n}$, such that$\left \langle M \right \rangle[i] = \Pi_{CMP}(\left \langle S \right \rangle[i] - \theta) $, where: \\
    $M[i] = \begin{cases}
        1  &\text{if}\ S[i] > \theta, \\
        0  &\text{otherwise}.
            \end{cases} $
    % \item If the pruning location is insensitive, $P_0$ and $P_1$ learn real mask $M$ instead of shares $\left \langle M \right \rangle$. $P_0$ and $P_1$ compute $\left \langle y \right \rangle$ with input $\left \langle x \right \rangle$ and $M$, where  $\left \langle x \right \rangle[i]$ is pruned if $M[i]$ is $0$.
    \item $P_0$ and $P_1$ invoke $\Pi_{mask}$ with inputs  $\left \langle x \right \rangle$ and pruning mask $\left \langle M \right \rangle$, and set outputs as $\left \langle y \right \rangle$.
\end{enumerate}
\end{protocolbox}
\setlength{\abovecaptionskip}{-1pt} % Reduces space above the caption
\caption{Secure Token Pruning Protocol $\Pi_{prune}$.}
\label{fig:protocol-prune}
\end{figure}




\begin{figure}[h]
\begin{protocolbox}
\noindent
\textbf{Parties:} Server $P_0$, Client $P_1$.

\textbf{Input:} $P_0$ and $P_1$ hold $\left \langle x \right \rangle_{0}, \left \langle x \right \rangle_{1} \in \mathbb{Z}_{2^{\ell}}^{n\times D}$ and  $\left \langle M \right \rangle_{0}, \left \langle M \right \rangle_{1} \in \mathbb{Z}_{2^{\ell}}^{n}$, respectively, where n is the number of input tokens and D is the embedding dimension of tokens.

\textbf{Output:} $P_0$ and $P_1$ get $\left \langle y \right \rangle_{0}, \left \langle y \right \rangle_{1} \in \mathbb{Z}_{2^{\ell}}^{n'\times D}$, respectively, where $y=\mathsf{Prune}(x)$ and $n'$ is the number of remaining tokens.

\noindent\rule{13.2cm}{0.1pt} % This creates the horizontal line
\textbf{Protocol:}
\begin{enumerate}[label=\arabic*:, leftmargin=*]
    \item For $i \in [n]$, $P_0$ and $P_1$ set $\left \langle M \right \rangle$ to the MSB of $\left \langle x \right \rangle$ and learn the masked tokens $\left \langle \Bar{x} \right \rangle \in Z_{2^{\ell}}^{n\times D}$, where
    $\left \langle \Bar{x}[i] \right \rangle = \left \langle x[i] \right \rangle + (\left \langle M[i] \right \rangle << f)$ and $f$ is the fixed-point precision.
    \item $P_0$ and $P_1$ compute the sum of $\{\Pi_{B2A}(\left \langle M \right \rangle[i]) \}_{i=0}^{n-1}$, and learn the number of remaining tokens $n'$ and the number of tokens to be pruned $m$, where $m = n-n'$.
    \item For $k\in[m]$, for $i\in[n-k-1]$, $P_0$ and $P_1$ invoke $\Pi_{msb}$ to learn the highest bit of $\left \langle \Bar{x}[i] \right \rangle$, where $b=\mathsf{MSB}(\Bar{x}[i])$. With the highest bit of $\Bar{x}[i]$, $P_0$ and $P_1$ perform a oblivious swap between $\Bar{x}[i]$ and $\Bar{x}[i+1]$:
    $\begin{cases}
        \Tilde{x}[i] = b\cdot \Bar{x}[i] + (1-b)\cdot \Bar{x}[i+1] \\
        \Tilde{x}[i+1] = b\cdot \Bar{x}[i+1] + (1-b)\cdot \Bar{x}[i]
    \end{cases} $ \\
    $P_0$ and $P_1$ learn the swapped token sequence $\left \langle \Tilde{x} \right \rangle$.
    \item $P_0$ and $P_1$ truncate $\left \langle \Tilde{x} \right \rangle$ locally by keeping the first $n'$ tokens, clear current MSB (all remaining token has $1$ on the MSB), and set outputs as $\left \langle y \right \rangle$.
\end{enumerate}
\end{protocolbox}
\setlength{\abovecaptionskip}{-1pt} % Reduces space above the caption
\caption{Secure Mask Protocol $\Pi_{mask}$.}
\label{fig:protocol-mask}
%\vspace{-0.2in}
\end{figure}

% \begin{wrapfigure}{r}{0.35\textwidth}  % 'r' for right, and the width of the figure area
%   \centering
%   \includegraphics[width=0.35\textwidth]{figures/msb.pdf}
%   \caption{Runtime of $\Pi_{prune}$ and $\Pi_{mask}$ in different layers. We compare different secure pruning strategies based on the BERT Base model.}
%   \label{fig:msb}
%   \vspace{-0.1in}
% \end{wrapfigure}

% \begin{figure}[h]  % 'r' for right, and the width of the figure area
%   \centering
%   \includegraphics[width=0.4\textwidth]{figures/msb.pdf}
%   \caption{Runtime of $\Pi_{prune}$ and $\Pi_{mask}$ in different layers. We compare different secure pruning strategies based on the BERT Base model.}
%   \label{fig:msb}
%   % \vspace{-0.1in}
% \end{figure}

\textbf{Complexity of $\Pi_{mask}$.} The complexity of the proposed $\Pi_{mask}$ mainly depends on the number of oblivious swaps. To prune $m$ tokens out of $n$ input tokens, $O(mn)$ swaps are needed. Since token pruning is performed progressively, only a small number of tokens are pruned at each layer, which makes $\Pi_{mask}$ efficient during runtime. Specifically, for a BERT base model with 128 input tokens, the pruning protocol only takes $\sim0.9$s on average in each layer. An alternative approach is to invoke an oblivious sort algorithm~\citep{bogdanov2014swap2,pang2023bolt} on $\left \langle \Bar{x} \right \rangle$. However, this approach is less efficient because it blindly sort the whole token sequence without considering $m$. That is, even if only $1$ token needs to be pruned, $O(nlog^{2}n)\sim O(n^2)$ oblivious swaps are needed, where as the proposed $\Pi_{mask}$ only need $O(n)$ swaps. More generally, for an $\ell$-layer Transformer with a total of $m$ tokens pruned, the overall time complexity using the sort strategy would be $O(\ell n^2)$ while using the swap strategy remains an overall complexity of $O(mn).$ Specifically, using the sort strategy to prune tokens in one BERT Base model layer can take up to $3.8\sim4.5$ s depending on the sorting algorithm used. In contrast, using the swap strategy only needs $0.5$ s. Moreover, alternative to our MSB strategy, one can also swap the encrypted mask along with the encrypted token sequence. However, we find that this doubles the number of swaps needed, and thus is less efficient the our MSB strategy, as is shown in Figure \ref{fig:msb}.

\section{Existing Protocols}
\label{app:protocol}
\noindent\textbf{Existing Protocols Used in Our Private Inference.}  In our private inference framework, we reuse several existing cryptographic protocols for basic computations. $\Pi_{MatMul}$ \citep{pang2023bolt} processes two ASS matrices and outputs their product in SS form. For non-linear computations, protocols $\Pi_{SoftMax}, \Pi_{GELU}$, and $\Pi_{LayerNorm}$\citep{lu2023bumblebee, pang2023bolt} take a secret shared tensor and return the result of non-linear functions in ASS. Basic protocols from~\citep{rathee2020cryptflow2, rathee2021sirnn} are also utilized. $\Pi_{CMP}$\citep{EzPC}, for example, inputs ASS values and outputs a secret shared comparison result, while $\Pi_{B2A}$\citep{EzPC} converts secret shared Boolean values into their corresponding arithmetic values.

\section{Polynomial Reduction for Non-linear Functions}
\label{app:b}
The $\mathsf{SoftMax}$ and $\mathsf{GELU}$ functions can be approximated with polynomials. High-degree polynomials~\citep{lu2023bumblebee, pang2023bolt} can achieve the same accuracy as the LUT-based methods~\cite{hao2022iron-iron}. While these polynomial approximations are more efficient than look-up tables, they can still incur considerable overheads. Reducing the high-degree polynomials to the low-degree ones for the less important tokens can imporve efficiency without compromising accuracy. The $\mathsf{SoftMax}$ function is applied to each row of an attention map. If a token is to be reduced, the corresponding row will be computed using the low-degree polynomial approximations. Otherwise, the corresponding row will be computed accurately via a high-degree one. That is if $M_{\beta}'[i] = 1$, $P_0$ and $P_1$ uses high-degree polynomials to compute the $\mathsf{SoftMax}$ function on token $x[i]$:
\begin{equation}
\mathsf{SoftMax}_{i}(x) = \frac{e^{x_i}}{\sum_{j\in [d]}e^{x_j}}
\end{equation}
where $x$ is a input vector of length $d$ and the exponential function is computed via a polynomial approximation. For the $\mathsf{SoftMax}$ protocol, we adopt a similar strategy as~\citep{kim2021ibert, hao2022iron-iron}, where we evaluate on the normalized inputs $\mathsf{SoftMax}(x-max_{i\in [d]}x_i)$. Different from~\citep{hao2022iron-iron}, we did not used the binary tree to find max value in the given vector. Instead, we traverse through the vector to find the max value. This is because each attention map is computed independently and the binary tree cannot be re-used. If $M_{\beta}[i] = 0$, $P_0$ and $P_1$ will approximate the $\mathsf{SoftMax}$ function with low-degree polynomial approximations. We detail how $\mathsf{SoftMax}$ can be approximated as follows:
\begin{equation}
\label{eq:app softmax}
\mathsf{ApproxSoftMax}_{i}(x) = \frac{\mathsf{ApproxExp}(x_i)}{\sum_{j\in [d]}\mathsf{ApproxExp}(x_j)}
\end{equation}
\begin{equation}
\mathsf{ApproxExp}(x)=\begin{cases}
    0  &\text{if}\ x \leq T \\
    (1+ \frac{x}{2^n})^{2^n} &\text{if}\ x \in [T,0]\\
\end{cases}
\end{equation}
where the $2^n$-degree Taylor series is used to approximate the exponential function and $T$ is the clipping boundary. The value $n$ and $T$ determines the accuracy of above approximation. With $n=6$ and $T=-13$, the approximation can achieve an average error within $2^{-10}$~\citep{lu2023bumblebee}. For low-degree polynomial approximation, $n=3$ is used in the Taylor series.

Similarly, $P_0$ or $P_1$ can decide whether or not to approximate the $\mathsf{GELU}$ function for each token. If $M_{\beta}[i] = 1$, $P_0$ and $P_1$ use high-degree polynomials~\citep{lu2023bumblebee} to compute the $\mathsf{GELU}$ function on token $x[i]$ with high-degree polynomial:
% \begin{equation}
% \mathsf{GELU}(x) = 0.5x(1+\mathsf{Tanh}(\sqrt{2/\pi}(x+0.044715x^3)))
% \end{equation}
% where the $\mathsf{Tanh}$ and square root function are computed via a OT-based lookup-table.

\begin{equation}
\label{eq:app gelu}
\mathsf{ApproxGELU}(x)=\begin{cases}
    0  &\text{if}\ x \leq -5 \\
    P^3(x), &\text{if}\ -5 < x \leq -1.97 \\
    P^6(x), &\text{if}\ -1.97 < x \leq 3  \\
    x, &\text{if}\ x >3 \\
\end{cases}
\end{equation}
where $P^3(x)$ and $P^6(x)$ are degree-3 and degree-6 polynomials respectively. The detailed coefficient for the polynomial is: 
\begin{equation*}
    P^3(x) = -0.50540312 -  0.42226581x - 0.11807613x^2 - 0.01103413x^3
\end{equation*}
, and
\begin{equation*}
    P^6(x) = 0.00852632 + 0.5x + 0.36032927x^2 - 0.03768820x^4 + 0.00180675x^6
\end{equation*}

For BOLT baseline, we use another high-degree polynomial to compute the $\mathsf{GELU}$ function.

\begin{equation}
\label{eq:app gelu}
\mathsf{ApproxGELU}(x)=\begin{cases}
    0  &\text{if}\ x < -2.7 \\
    P^4(x), &\text{if}\   |x| \leq 2.7 \\
    x, &\text{if}\ x >2.7 \\
\end{cases}
\end{equation}
We use the same coefficients for $P^4(x)$ as BOLT~\citep{pang2023bolt}.

\begin{figure}[h]
 % \vspace{-0.1in}
    \centering
    \includegraphics[width=1\linewidth]{figures/bumble.pdf}
    % \captionsetup{skip=2pt}
    % \vspace{-0.1in}
    \caption{Comparison with prior works on the BERT model. The input has 128 tokens.}
    \label{fig:bumble}
\end{figure}

If $M_{\beta}'[i] = 0$, $P_0$ and $P_1$ will use low-degree 
polynomial approximation to compute the $\mathsf{GELU}$ function instead. Encrypted polynomial reduction leverages low-degree polynomials to compute non-linear functions for less important tokens. For the $\mathsf{GELU}$ function, the following degree-$2$ polynomial~\cite{kim2021ibert} is used:
\begin{equation*}
\mathsf{ApproxGELU}(x)=\begin{cases}
    0  &\text{if}\ x <  -1.7626 \\
    0.5x+0.28367x^2, &\text{if}\ x \leq |1.7626| \\
    x, &\text{if}\ x > 1.7626\\
\end{cases}
\end{equation*}


\section{Comparison with More Related Works.}
\label{app:c}
\textbf{Other 2PC frameworks.} The primary focus of CipherPrune is to accelerate the private Transformer inference in the 2PC setting. As shown in Figure \ref{fig:bumble}, CipherPrune can be easily extended to other 2PC private inference frameworks like BumbleBee~\citep{lu2023bumblebee}. We compare CipherPrune with BumbleBee and IRON on BERT models. We test the performance in the same LAN setting as BumbleBee with 1 Gbps bandwidth and 0.5 ms of ping time. CipherPrune achieves more than $\sim 60 \times$ speed up over BOLT and $4.3\times$ speed up over BumbleBee.

\begin{figure}[t]
 % \vspace{-0.1in}
    \centering
    \includegraphics[width=1\linewidth]{figures/pumab.pdf}
    % \captionsetup{skip=2pt}
    % \vspace{-0.1in}
    \caption{Comparison with MPCFormer and PUMA on the BERT models. The input has 128 tokens.}
    \label{fig:pumab}
\end{figure}

\begin{figure}[h]
 % \vspace{-0.1in}
    \centering
    \includegraphics[width=1\linewidth]{figures/pumag.pdf}
    % \captionsetup{skip=2pt}
    % \vspace{-0.1in}
    \caption{Comparison with MPCFormer and PUMA on the GPT2 models. The input has 128 tokens. The polynomial reduction is not used.}
    \label{fig:pumag}
\end{figure}

\textbf{Extension to 3PC frameworks.} Additionally, we highlight that CipherPrune can be also extended to the 3PC frameworks like MPCFormer~\citep{li2022mpcformer} and PUMA~\citep{dong2023puma}. This is because CipherPrune is built upon basic primitives like comparison and Boolean-to-Arithmetic conversion. We compare CipherPrune with MPCFormer and PUMA on both the BERT and GPT2 models. CipherPrune has a $6.6\sim9.4\times$ speed up over MPCFormer and $2.8\sim4.6\times$ speed up over PUMA on the BERT-Large and GPT2-Large models.


\section{Communication Reduction in SoftMax and GELU.}
\label{app:e}

\begin{figure}[h]
    \centering
    \includegraphics[width=0.9\linewidth]{figures/layerwise.pdf}
    \caption{Toy example of two successive Transformer layers. In layer$_i$, the SoftMax and Prune protocol have $n$ input tokens. The number of input tokens is reduced to $n'$ for the Linear layers, LayerNorm and GELU in layer$_i$ and SoftMax in layer$_{i+1}$.}
    \label{fig:layer}
\end{figure}

\begin{table*}[h]
\captionsetup{skip=2pt}
\centering
\scriptsize
\caption{Communication cost (in MB) of the SoftMax and GELU protocol in each Transformer layer.}
\begin{tblr}{
    colspec = {c |c c c c c c c c c c c c},
    row{1} = {font=\bfseries},
    row{2-Z} = {rowsep=1pt},
    % row{4} = {bg=LightBlue},
    colsep = 2.5pt,
    }
\hline
\textbf{Layer Index} & \textbf{0}  & \textbf{1}  & \textbf{2} & \textbf{3} & \textbf{4} & \textbf{5} & \textbf{6} & \textbf{7} & \textbf{8} & \textbf{9} & \textbf{10} & \textbf{11} \\
\hline
Softmax & 642.19 & 642.19 & 642.19 & 642.19 & 642.19 & 642.19 & 642.19 & 642.19 & 642.19 & 642.19 & 642.19 & 642.19 \\
Pruned Softmax & 642.19 & 129.58 & 127.89 & 119.73 & 97.04 & 71.52 & 43.92 & 21.50 & 10.67 & 6.16 & 4.65 & 4.03 \\
\hline
GELU & 698.84 & 698.84 & 698.84 & 698.84 & 698.84 & 698.84 & 698.84 & 698.84 & 698.84 & 698.84 & 698.84 & 698.84\\
Pruned GELU  & 325.10 & 317.18 & 313.43 & 275.94 & 236.95 & 191.96 & 135.02 & 88.34 & 46.68 & 16.50 & 5.58 & 5.58\\
\hline
\end{tblr}
\label{tab:layer}
\end{table*}

{
In Figure \ref{fig:layer}, we illustrate why CipherPrune can reduce the communication overhead of both  SoftMax and GELU. Suppose there are $n$ tokens in $layer_i$. Then, the SoftMax protocol in the attention module has a complexity of $O(n^2)$. CipherPrune's token pruning protocol is invoked to select $n'$ tokens out of all $n$ tokens, where $m=n-n'$ is the number of tokens that are removed. The overhead of the GELU function in $layer_i$, i.e., the current layer, has only $O(n')$ complexity (which should be $O(n)$ without token pruning). The complexity of the SoftMax function in $layer_{i+1}$, i.e., the following layer, is reduced to $O(n'^2)$ (which should be $O(n^2)$ without token pruning). The SoftMax protocol has quadratic complexity with respect to the token number and the GELU protocol has linear complexity. Therefore, CipherPrune can reduce the overhead of both the GELU protocol and the SoftMax protocols by reducing the number of tokens. In Table \ref{tab:layer}, we provide detailed layer-wise communication cost of the GELU and the SoftMax protocol. Compared to the unpruned baseline, CipherPrune can effectively reduce the overhead of the GELU and the SoftMax protocols layer by layer.
}

\section{Analysis on Layer-wise redundancy.}
\label{app:f}

\begin{figure}[h]
    \centering
    \includegraphics[width=0.9\linewidth]{figures/layertime0.pdf}
    \caption{The number of pruned tokens and pruning protocol runtime in different layers in the BERT Base model. The results are averaged across 128 QNLI samples.}
    \label{fig:layertime}
\end{figure}

{
In Figure \ref{fig:layertime}, we present the number of pruned tokens and the runtime of the pruning protocol for each layer in the BERT Base model. The number of pruned tokens per layer was averaged across 128 QNLI samples, while the pruning protocol runtime was measured over 10 independent runs. The mean token count for the QNLI samples is 48.5. During inference with BERT Base, input sequences with fewer tokens are padded to 128 tokens using padding tokens. Consistent with prior token pruning methods in plaintext~\citep{goyal2020power}, a significant number of padding tokens are removed at layer 0.  At layer 0, the number of pruned tokens is primarily influenced by the number of padding tokens rather than token-level redundancy.
%In Figure \ref{fig:layertime}, we demonstrate the number of pruned tokens and the pruning protocol runtime in each layer in the BERT Base model. We averaged the number of pruned tokens in each layer across 128 QNLI samples and then tested the pruning protocol runtime in 10 independent runs. The mean token number of the QNLI samples is 48.5. During inference with BERT Base, input sequences with small token number are padded to 128 tokens with padding tokens. Similar to prior token pruning methods in the plaintext~\citep{goyal2020power}, a large number of padding tokens can be removed at layer 0. We remark that token-level redundancy builds progressively throughout inference~\citep{goyal2020power, kim2022LTP}. The number of pruned tokens in layer 0 mostly depends on the number of padding tokens instead of token-level redundancy.
}

{
%As shown in Figure \ref{fig:layertime}, more tokens are removed in the intermediate layers, e.g., layer $4$ to layer $7$. This suggests there is more redundant information in these intermediate layers. 
In CipherPrune, tokens are removed progressively, and once removed, they are excluded from computations in subsequent layers. Consequently, token pruning in earlier layers affects computations in later layers, whereas token pruning in later layers does not impact earlier layers. As a result, even if layers 4 and 7 remove the same number of tokens, layer 7 processes fewer tokens overall, as illustrated in Figure \ref{fig:layertime}. Specifically, 8 tokens are removed in both layer $4$ and layer $7$, but the runtime of the pruning protocol in layer $4$ is $\sim2.4\times$ longer than that in  layer $7$.
}

\section{Related Works}
\label{app:g}

{
In response to the success of Transformers and the need to safeguard data privacy, various private Transformer Inferences~\citep{chen2022thex,zheng2023primer,hao2022iron-iron,li2022mpcformer, lu2023bumblebee, luo2024secformer, pang2023bolt}  are proposed. To efficiently run private Transformer inferences, multiple cryptographic primitives are used in a popular hybrid HE/MPC method IRON~\citep{hao2022iron-iron}, i.e., in a Transformer, HE and SS are used for linear layers, and SS and OT are adopted for nonlinear layers. IRON and BumbleBee~\citep{lu2023bumblebee} focus on optimizing linear general matrix multiplications; SecFormer~\cite{luo2024secformer} improves the non-linear operations like the exponential function with polynomial approximation; BOLT~\citep{pang2023bolt} introduces the baby-step giant-step (BSGS) algorithm to reduce the number of HE rotations, proposes a word elimination (W.E.) technique, and uses polynomial approximation for non-linear operations, ultimately achieving state-of-the-art (SOTA) performance.
}

{Other than above hybrid HE/MPC methods, there are also works exploring privacy-preserving Transformer inference using only HE~\citep{zimerman2023converting, zhang2024nonin}. The first HE-based private Transformer inference work~\citep{zimerman2023converting} replaces \mysoftmax function with a scaled-ReLU function. Since the scaled-ReLU function can be approximated with low-degree polynomials more easily, it can be computed more efficiently using only HE operations. A range-loss term is needed during training to reduce the polynomial degree while maintaining high accuracy. A training-free HE-based private Transformer inference was proposed~\citep{zhang2024nonin}, where non-linear operations are approximated by high-degree polynomials. The HE-based methods need frequent bootstrapping, especially when using high-degree polynomials, thus often incurring higher overhead than the hybrid HE/MPC methods in practice.
}


\end{document}
