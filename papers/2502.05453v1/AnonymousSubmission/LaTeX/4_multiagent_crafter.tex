% \subsection{Environment Details} 
%The original Crafter environment \cite{hafner2021benchmarking} is a procedurally generated, open-world survival game designed for benchmarking reinforcement learning (RL) algorithms. It features a grid world with a discrete action space of size 17 and provides information on the player's inventory, health, food, water, and crafting progress. Crafter includes 22 achievements organized in a tech tree with a depth of 7, with the ultimate goal of exploring the environment. Inspired by Minecraft, Crafter simplifies the game’s mechanics to enable faster experimentation and results collection.

The original Crafter environment \cite{hafner2021benchmarking} is a procedurally generated, open-world survival game used to benchmark RL algorithms. It features a 17 discrete action grid world and tracks player metrics like inventory, health, and crafting progress, with 22 achievements organized in a 7-depth tech tree. Inspired by Minecraft, Crafter simplifies game mechanics for faster experimentation and results collection.
\hq{We proposed a novel multi-agent Crafter for multi-agent tasks, enabling cooperative agent interaction and introducing new actions and challenges. These changes, shown in Figure \ref{fig:crafter}, make the environment suitable for studying multi-agent cooperation. Key modifications are outlined below.}

%We have made several significant modifications to transform Crafter into a robust multi-agent environment. These changes allow agents to interact cooperatively within the environment and introduce additional challenges, making it more suitable for studying multi-agent cooperation, as shown in Figure \ref{fig:crafter}. The key changes are outlined below:
\hq{
\textbf{A Scalable Cooperative Environment.} We extended the Crafter environment to support an arbitrary number of agents, each with independent observations, inventories, and health stats, enabling cooperative agent interaction and introducing new actions and challenges (Figure~\ref{fig:crafter}). Agents can collaborate by sharing resources, coordinating actions, and balancing individual roles to achieve collective goals efficiently. Unlike traditional MARL environments, which often focus on micro-level action management, our testbed is designed to evaluate strategic planning, coordination, and shared decision-making.

Our environment allows agents to share items, including resources and tools, fostering teamwork by enabling task delegation and resource management. Crafting dependencies and environmental prompts can be easily customized, increasing task complexity with more participants. This ensures that agents must coordinate and efficiently allocate roles, enabling effective large-scale parallel collaboration. The flexible design makes the testbed suitable for evaluating cooperative behavior potentially for any number of agents.

\textbf{Evaluation of Cooperation and LLM Agents' Capabilities.}  
Unlike the original Crafter environment, which focused on open-ended exploration, we define a clear objective: agents must collaborate to craft necessary tools and obtain a diamond as quickly as possible while managing their needs for food, water, and energy. This setup allows us to evaluate whether agents can effectively cooperate and reason toward both short- and long-term goals, making the environment ideal for testing multi-agent coordination, planning, and resource optimization.

To assess cooperative efficiency, agents share resources and tools, requiring negotiation, task division, and decision-making. Unlike previous MARL settings, where collaboration is forced or predefined, our testbed allows agents to develop teamwork strategies. Our environment quantifies multi-agent cooperation through indirect measurements, such as tracking the steps an agent takes to craft items, providing insights into decision-making and adaptability.

\textbf{Support for Language Agents.} We added a navigation skill that allows agents to move toward specific resources, reducing the burden of manual low-level movement control. This enables agents to focus on higher-level decision-making, such as strategic planning and collaboration.

\textbf{Customizability and Compatibility.}
Our multi-agent Crafter environment is designed to be highly flexible and extensible, supporting RL, MARL, and LLM-powered agents. The single-agent version follows the Gymnasium API, ensuring integration with standard RL libraries, while the multi-agent version aligns with the PettingZoo API, ensuring compatibility with existing MARL frameworks. We provide example training scripts for single-agent experiments using Stable-Baselines3 (SB3) and multi-agent experiments using AgileRL, allowing researchers to efficiently test new ideas, integrate with existing RL libraries, and adapt the environment for diverse multi-agent challenges.
}


% While this paper evaluates our decision-making framework using the multi-agent Crafter environment, we believe it will also serve as a valuable platform for future research on multi-agent coordination, communication, long-term planning, and resource optimization in complex, real-time multi-agent environments.





