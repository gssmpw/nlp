\newpage
\appendix
\onecolumn
\section{Environment Description} \label{appendix:env}
Below is the instruction prompt that describes the environment. The instruction prompt is sent to the Language Agent at every step to ensure it remains aware of the environment.

\subsubsection{Multi-agent Crafter}
Multi-Agent Crafter is a sandbox game where players gather resources, craft tools, and survive. You are tasked with collaborating with other agents. The ultimate goal is to mine a diamond as quickly as possible, and only one agent needs to obtain the diamond.

\subsubsection{I. Key Features}
\begin{itemize}[label={}]
    \item Unique worlds with varying terrains.
    \item Resource gathering: wood, stone, coal, iron, diamond.
    \item Crafting system for tools.
    \item Survival mechanics such as health, drink, hunger, and energy management.
\end{itemize}

\subsubsection{II. Getting Started}
\begin{itemize}[label={}]
    \item Collect resources from the environment.
    \item Use resources to craft basic tools.
    \item Gather advanced materials with improved tools.
    \item Craft advanced tools to achieve higher-level goals.
    \item Ultimately, collect a diamond to complete the main objective.
\end{itemize}

\subsubsection{III. Tips for Success}
\begin{itemize}[label={}]
    \item Continuously upgrade tools based on the crafting hierarchy.
    \item Progress to new goals once the current one is complete.
    \item Learn from setbacks and adapt your approach.
\end{itemize}

\subsubsection{IV. Prerequisites and Crafting Hierarchy}
\begin{itemize}[label={}]
    \item Collect Cow: \{facing: cow\}
    \item Collect Drink: \{facing: water\}
    \item Collect Wood: \{facing: tree\}
    \item Collect Stone: \{facing: stone, wood pickaxe: 1\}
    \item Collect Coal: \{facing: coal, wood pickaxe: 1\}
    \item Collect Iron: \{facing: iron, stone pickaxe: 1\}
    \item Collect Diamond: \{facing: diamond, iron pickaxe: 1\}
    \item Place Table: \{facing: grass, wood: 2\}
    \item Place Furnace: \{facing: grass, stone: 4\}
    \item Make Wood Pickaxe: \{facing: table, wood: 1\}
    \item Make Stone Pickaxe: \{facing: table, stone: 1, wood: 1\}
    \item Make Iron Pickaxe: \{facing: furnace, iron: 1, coal: 1, wood: 1\}
\end{itemize}

\subsubsection{V. Rules About the World}
\begin{itemize}[label={}]
    \item Assumptions or guesses are not permitted; all actions must be verified using inventory or other references.
    \item Actions may fail; inventory checks should be performed before proceeding.
    \item Health stats are crucial. If hunger is low, navigate to a cow and collect it; if drink is low, collect water; if energy is low, sleep.
    \item To collect or attack, face the material, ensure the correct tool is available, and perform the "do" action repeatedly.
    \item The Navigator tool can be used to locate targets.
    \item Placed items should not be placed again unnecessarily, as this consumes additional materials.
\end{itemize}



\newpage
\section{Structured Output Format}
The structured output ensures that the model always generates responses adhering to the supplied schema. The schema is designed using the \textit{pydantic} library. The Response Event is structured with four main components: collaboration, reflection, goal, and action. For each component, the language agent is required to answer a different set of questions, guiding its next action selection. The Enum class guarantees that the response is a valid option in the environment, enhancing the language agent's ability to run more smoothly. Below is the detailed schema.\\

\begin{lstlisting}[language=Python]
from pydantic import BaseModel, Field
from enum import Enum

class ResultType(str, Enum):
    SUCCESS = "success"
    FAILURE = "failure"
    IN_PROGRESS = "in_progress"
    
class ActionType(str, Enum):
    noop = "noop"
    move_left = "move_left"
    move_right = "move_right"
    move_up = "move_up"
    move_down = "move_down"
    do = "do"
    sleep = "sleep"
    place_stone = "place_stone"
    place_table = "place_table"
    place_furnace = "place_furnace"
    place_plant = "place_plant"
    make_wood_pickaxe = "make_wood_pickaxe"
    make_stone_pickaxe = "make_stone_pickaxe"
    make_iron_pickaxe = "make_iron_pickaxe"
    Navigator = "Navigator"
    share = "share"
    
class GoalType(str, Enum):
    COLLECT_WOOD = "collect_wood"
    MAKE_WOOD_PICKAXE = "make_wood_pickaxe"
    COLLECT_STONE = "collect_stone"
    MAKE_STONE_PICKAXE = "make_stone_pickaxe"
    COLLECT_IRON = "collect_iron"
    MAKE_IRON_PICKAXE = "make_iron_pickaxe"
    COLLECT_DIAMOND = "collect_diamond"
    
    PLACE_TABLE = "place_table"
    PLACE_FURNACE = "place_furnace"
    COLLECT_COAL = "collect_coal"
    SHARE = "share"

class LongTermGoalType(str, Enum):
    MAKE_WOOD_PICKAXE = "make_wood_pickaxe"
    MAKE_STONE_PICKAXE = "make_stone_pickaxe"
    MAKE_IRON_PICKAXE = "make_iron_pickaxe"
    PLACE_TABLE = "place_table"
    PLACE_FURNACE = "place_furnace"
    COLLECT_DIAMOND = "collect_diamond"
    HELP_AGENT = "help_agent"
    
class MaterialType(str, Enum):
    TABLE = "table"
    FURNACE = "furnace"
    GRASS = "grass"
    SAND = "sand"
    LAVA = "lava"
    TREE = "tree"
    WATER = "water"
    STONE = "stone"
    COAL = "coal"
    IRON = "iron"
    DIAMOND = "diamond"
    
class NavigationDestinationItems(str, Enum):
    TREE = "tree"
    WATER = "water"
    STONE = "stone"
    IRON = "iron"
    DIAMOND = "diamond"
    COAL = "coal"
    GRASS = "grass"
    TABLE = "table"
    FURNACE = "furnace"
    NOT_APPICABLE = "not_applicable"
    
class ShareableItems(str, Enum):
    WOOD = "wood"
    STONE = "stone"
    COAL = "coal"
    IRON = "iron"
    DIAMOND = "diamond"
    WOOD_PICKAXE = "wood_pickaxe"
    STONE_PICKAXE = "stone_pickaxe"
    IRON_PICKAXE = "iron_pickaxe"
    NOT_APPLICABLE = "not_applicable"

class InventoryItems(str, Enum):
    WOOD = "wood"
    STONE = "stone"
    COAL = "coal"
    IRON = "iron"
    DIAMOND = "diamond"
    WOOD_PICKAXE = "wood_pickaxe"
    STONE_PICKAXE = "stone_pickaxe"
    IRON_PICKAXE = "iron_pickaxe"
    
class Reflection(BaseModel):
    vision: list[MaterialType] = Field(description="List of materials you see around you.")
    last_action: ActionType 
    last_action_result: ResultType
    last_action_result_reflection: str 
    last_action_repeated_reflection: str = Field(description="Did you repeat the last action? If so, why?")

class Goal(BaseModel):
    ultimate_goal: LongTermGoalType = Field(description="What is your ultimate goal?")
    
    long_term_goal: LongTermGoalType = Field(description="Working towards the ultimate goal, what should be your next goal?")
    long_term_goal_subgoals: str = Field(Description="What are the subgoals to complete the long term goal?")
    long_term_goal_progress: GoalType = Field(Description="What is the progress of the long term goal?")
    long_term_goal_status: ResultType
    
    current_goal: GoalType = Field(description="The current goal that you are working on.")
    current_goal_reason: str 
    current_goal_status: ResultType
 
class InventoryItemsCount(BaseModel):
    item: InventoryItems
    count: int

class NextAction(BaseModel):
    next_action: ActionType = Field(description="What is the next action you plan to take?")
    next_action_reason: str 
    next_action_prerequisites_status: ResultType = Field(description="Are the prerequisites met?")
    next_action_prerequisites: str = Field(description="What prerequisites are not met?")
    final_next_action: ActionType = Field(description="What is your final decision on next action.")
    final_next_action_reason: str 
    final_target_material_to_collect: NavigationDestinationItems = Field(description="Navigate to where?")
    final_target_material_to_share: ShareableItems = Field(description="Share what?")
    final_target_agent_id: int = Field(description="Which agent to share with, if applicable, or return -1.")
    
class Collaboration(BaseModel):
    target_agent_to_help: int = Field(description="Which agent should you help, if applicable?")
    target_agent_need: ShareableItems = Field(description="What does the target agent need, if applicable?")
    help_method: str = Field(description="What can you do to help the agent, if applicable?")
    can_help_now: ResultType = Field(description="Can you help the agent now? Do you have the resources in inventory?")
    being_helped_by_agent: int = Field(description="Which agent is helping you, if applicable?")
    help_method_by_agent: str = Field(description="What is the agent doing to help you, if applicable?")
    change_in_plan: str = Field(description="How does the help from the agent change your plan, if applicable?")
    
class ResponseEvent(BaseModel):
    epsiode_number: int = Field(Description="What is the current episode?")
    timestep: int = Field(Description="What is the current timestep in the episode?")
    past_events: str = Field(Description="Briefly describe the past events in the episode.")
    current_facing_direction: MaterialType
    current_inventory: list[InventoryItemsCount] = Field(Description="What is in your current inventory? Only list items with item count greater than 0.")
    collaboration: Collaboration
    reflection: Reflection
    goal: Goal
    action: NextAction
    summary: str = Field(Description=(
                                "Summarize the episode, including the timestep, long-term goal, progress, significant events, and plan. "
                                "Explain your actions, the rationale behind your decisions. Treat as if you have done the next actions aleardy. Explain your intended support for other agents (if applicable). What should come next?"
                                "Keep the summary concise and focused on key information, using *past tense* for everything as it serves as a note for future reference. Use clear and plain language."
                                "Use PAST TENSE!!!\n")
                         )
\end{lstlisting}










\newpage
\section{Adaptive Hierarchical Knowledge Graph}
Below is the Adaptive Knowledge Graph of Agent 0 in a six-agent communication setting. Blue nodes represent step nodes, green nodes represent goal nodes, and red nodes represent long-term goal nodes. Note that a goal node may be associated with multiple step nodes, and a long-term goal node may be associated with multiple goal nodes.

\begin{figure}[h]
    \centering
    \includegraphics[width=0.8\textwidth, height=0.5\textwidth]{AnonymousSubmission/LaTeX/figures/knowledge_graph.png}
    \caption{Example of an Adaptive Hierarchical Knowledge Graph for an Agent in an Episode.}
    \label{fig:knowledge_graph}
\end{figure}

\begin{figure}[h]
    \centering
    % Subfigure 1
    \begin{subfigure}[b]{0.4\textwidth}
        \centering
        \includegraphics[width=\textwidth]{AnonymousSubmission/LaTeX/figures/knowledge_graph_step.png}
        \caption{Goal node.}
    \end{subfigure}
    \hfill % Adds horizontal space between subfigures
    % Subfigure 2
    \begin{subfigure}[b]{0.4\textwidth}
        \centering
        \includegraphics[width=\textwidth]{AnonymousSubmission/LaTeX/figures/knowledge_graph_goal.png}
        \caption{Step node.}
    \end{subfigure}
    \caption{Example of a step node and a goal node in the Adaptive Hierarchical Knowledge Graph.}
    \label{fig:knowledge_graph_node}
\end{figure}




\newpage
\section{Six Agents with Communication - Memory of Each Agent in the Same Game} \label{appendix:agent_memories}
\hq{Figure \ref{fig:memory_of_each_agent} illustrates each agent’s memory structure during gameplay. While each agent independently controls its own behavior and maintains its own memory, the Structured Communication System (S-CS) ensures they remain aware of others’ progress, enabling timely and adaptive cooperation.

Agent 0, responsible for tool crafting, follows a sequential memory structure, reflecting hierarchical goal progression. Agent 1, tasked with assisting Agent 0, develops clustered memories centered on crafting and resource gathering, helping Agent 0 with its needs. Similarly, Agent 2 supports Agent 1, with memory clusters focused on cooperative material collection and crafting tasks. These agents dynamically adjust their strategies based on shared information in a decentralized manner.

Agents 3 and 4, focused on resource sharing, exhibit simpler, less interconnected memory structures since their role is primarily to collect and distribute materials rather than craft tools. Agent 5, which monitors the overall team’s progress, integrates information from all agents and determines when to transition toward diamond collection.

The S-CS plays a crucial role in shaping these memory patterns. Crafting agents exhibit structured, sequential goal formation, while resource-gathering agents maintain more discrete clusters, prioritizing aid based on real-time assessments. Decentralized decision-making, enhanced by communication, ensures that agents act autonomously while dynamically adapting their goals to support the team. By optimizing task allocation and minimizing redundant efforts, S-CS enables more effective decentralized cooperation.}

\begin{figure}[!b]
    \centering
    \includegraphics[width=1\linewidth]{AnonymousSubmission/LaTeX/figures/memory_of_each_agent.png}
    \caption{Memory of each agent in a game play.}
    \label{fig:memory_of_each_agent}
\end{figure}

\newpage
\section{Complete Game Trajectories}
The complete game trajectory of six agents with memory and communication is presented below.

\includepdf[pages=-]{AnonymousSubmission/LaTeX/six_agent_mem_comm.pdf}