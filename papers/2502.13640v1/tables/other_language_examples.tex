
% \onecolumn
\clearpage
 \begin{center}
 \footnotesize
	\tablefirsthead{%
		\toprule
		   \textbf{Field} & \textbf{Content} \\
		\midrule}
	\tablehead{%
		\toprule
		   \textbf{Field} & \textbf{Content} \\
		\midrule}
	\tabletail{%
		\bottomrule
	}
	\tablelasttail{\bottomrule}
	\tablecaption{\textbf{Examples of other languages} across different domains by ChatGPT and \textit{davinci-text-003}.}
	\begin{supertabular}{l | p{12cm}}
        Language &  \textbf{Arabic} \\
        LLM Generator & \textsc{ChatGPT} \\
        Domain &  \textit{Wikipedia} \\
        \midrule
        Title & \<ركمجة>\\
        \midrule
        Human &  {\small \begin{RLtext}
        الرَكْمَجَة من (ركوب + موج) هي رياضة ركوب متن الأمواج المتكسرة على الشاطئ بواسطة ألواح خاصة. الموجات المناسبة لهذه الرياضة غالباً ما تكون في المحيط ولكن هذا لا يمنع ركوب امواج البحيرات والانهار والمسابح أيضا.  ظهرت أنواع وتعاريف مختلفة لهذه الرياضة التي تطورت على مر السنين  كمميزات الموجة المناسبة وما هو اللوح ومن هو راكب الامواج. يعتبر ركوب الموج بدون لوح من انقى أنواع هذه الرياضة لدى البعض كما أن هناك أنواع مختلفة ظهرت منذ قرون مثل  صعود بايبو واستخدام المجداف وركوب  الامواج  بالزوارق  والقوارب. حاليا تستخدم العديد من المركبات المطوره مثل القوارب المطاطية والألواح المعدنية.وكما وظهر في أفلام  وثائقية مختلفة عن أنواع أخرى من ركوب الامواج مثل (فير بايتس) حيث يتم فيها استخدام أدوات أخرى لهذه الرياضة  تشمل زحاليق المياه والطاولات والقيثارات والأبواب. عندما يركب أكثر من شخص نفس اللوح سويّا يسمى هذا النوع جنباً إلى جنب.
        
        هناك نوعان أساسيان  لركوب الامواج  وهما:
        
        الركوب الطويل.
        
        الركوب القصير. حيث يوجد اختلافات بينها من ناحية تصميم اللوح وطوله وطريقة الركوب ونوع الموجة.
        
        عند ركوب الامواج الكبيرة تقوم مركبة مائية كالزورق بسحب الراكب اتجاه الموجة وذلك لمساعدته باللحاق بسرعة الموجة الهائلة.
        
        \end{RLtext}}\\
        
        % Human &
        % \<الرَكْمَجَة من (ركوب + موج) هي رياضة ركوب متن الأمواج المتكسرة على>
        % \<الشاطئ بواسطة ألواح خاصة. الموجات المناسبة لهذه الرياضة غالباً ما تكون>
        % \<في المحيط ولكن هذا لا يمنع ركوب امواج البحيرات والانهار والمسابح أيضا.>
        % \<ظهرت أنواع وتعاريف مختلفة لهذه الرياضة التي تطورت على مر السنين>
        % \<كمميزات الموجة المناسبة وما هو اللوح ومن هو راكب الامواج. يعتبر ركوب>
        % \<الموج بدون لوح من انقى أنواع هذه الرياضة لدى البعض كما أن هناك>
        % \<أنواع مختلفة ظهرت منذ قرون مثل  صعود بايبو واستخدام المجداف وركوب >
        % \<الامواج  بالزوارق  والقوارب. حاليا تستخدم العديد من المركبات المطوره>
        % \<مثل القوارب المطاطية والألواح المعدنية.وكما وظهر في أفلام  وثائقية> \\
        % & \<مختلفة عن أنواع أخرى من ركوب الامواج مثل (فير بايتس) حيث يتم فيها>
        % \<استخدام أدوات أخرى لهذه الرياضة  تشمل زحاليق المياه والطاولات>
        % \<والقيثارات والأبواب. عندما يركب أكثر من شخص نفس اللوح سويّا يسمى هذا>
        % \<النوع جنباً إلى جنب. هناك نوعان أساسيان  لركوب الامواج  وهما:الركوب>
        % \<الطويل.الركوب القصير. حيث يوجد اختلافات بينها من ناحية تصميم اللوح>
        % \<وطوله وطريقة الركوب ونوع الموجة.عند ركوب الامواج الكبيرة تقوم مركبة>
        % \<مائية كالزورق بسحب الراكب اتجاه الموجة وذلك لمساعدته باللحاق بسرعة>
        % \<الموجة الهائلة.>\\ 
        
        \midrule
        Prompt & 
        This is a sample Arabic Wikipedia summary section for the title ``\<سحلية>'':
        
        {\small \begin{RLtext}
        
        الحِرْذَون (الجمع: حراذين) أو السحالي (المفرد: سحلية) أو العظاء (المفرد: عظاءة) (باللاتينية: Lacertilia) هي رتيبة من الزواحف تتبع طائفة العظايا الحرشفية من شعبة الحبليات
        
        وهي زواحف قريبة الصلة بالأفاعي. وبعضها ـ مثل الأفاعي ـ لا أرجل له، بينما يشبه بعضها الأفاعي لحد ما ولكن له أرجل، أما السحالي كبيرة الحجم فهي أكثر شبها بالتماسيح. تتباين السحالي فيما بينها في الحجم والشكل واللون، ولديها طرق عديدة للتنقل وللدفاع عن النفس. ولقد تعرّف العلماء على أكثر من 3,750 نوعاً مختلفًا من السحالي، وهنالك أكثر من 500 نوع تعيش في قارة إستراليا. تعد السحالي من الحيوانات الفقارية التي يغطي جسمها قشور جذورها من البشرة بين القشور يظل الجلد رفيعا ولينا.
        
        تكبر القشور عند الرأس وتكون شفرة طويلة على البطن. لا يمكن انتزاع تلك القشور الواحدة تلو الأخرى. دورياً ينزع هذا الجلد على مراحل. وهو ما يعرف بانسلاخ السحلية.
        
        أن السحلية حيوان ذو حرارة متغيرة (ذوات الدم البارد)، تفضل الأماكن الجافة والمشمسة جداً. لا تكون السحلية رشيقة إلا عندما يكون جسمها ساخنا، وحين تنخفض درجة الحرارة ويبرد الجسم يخمل الحيوان. في الشتاء لا تستطيع السحالي الحركة ولا الأكل ويعيش في حالة كمون.

        \end{RLtext}}\\
        & {\small \begin{RLtext} 



        الرأس بها فتحات الأنف في الطرف. أما الأعين فتحميهم جفون متحركة، وخلف كل واحدة يصفو الجلد ويتمدد ليكون الطبلة. الفم الواسع يسمح بمرور لسان رفيع متشعب من خلال فتحة في الشفة العلوية. أما الأسنان فهي عديدة وأصغر من أن تسمح له بمضغ فرائسه التي يبتلعها كاملة.
        
        الجسم رفيع ورشيق. ينتهي بذيل طويل.
        
        
        
        عند الإمساك بسحلية من ذيلها فإنها تصارع وتنفصل بإرادتها عن ذيلها وذلك بكسره عند مستوى منطقة ضعيفة.
        
        ظاهرة البتر الذاتي تتبعها عادة عملية تجديد لا تحدث إلا مرة واحدة.
        
        للسحالي قوائم قصيرة متصلة بجوانب الجسم ولا تستطيع سند الحيوان الذي يلامس الأرض.
        \end{RLtext}}\\
        & {\small \begin{RLtext}        
        تنتهي كل واحدة بخمسة أصابع طويلة متفرقة تحمل مخالب حادة مفيدة جداً للتعلق. حركة السحلية تكون على شكل تموجات تقوم بها بمساعدة قوائمها وبطنها وذيلها. السحلية من فصيلة القناص آكل اللحوم: تتغذى على الحشرات، العناكب، الدود، تختلف السحلية عن الضفدع فهي تملك تنفس رئوي فقط لكونها طوال حياتها تتنفس الهواء. في الربيع بعد التزاوج، تضع الأنثى من خمسة إلى عشرة بيضات في حفرة تحفرها بنفسها في الأرض ثم تسدها.
        
        سحلية الحائط حيوان يبيض. وليس للبيض قشرة صلبة ولكنها محاطة بغشاء يشبه الورق.
        
        لا ترقد الأنثى على البيض. إن حرارة الصيف هي التي تجعله يفقس في خلال ستة إلى ثمانية أسابيع. منذ الخروج من البيض، تصطاد السحلية الصغيرة الحشرات لتتغذى عليها. بعض أنواع السحالي تحتضن البيض بداخلها. وهي تعيش لعدة سنوات.
        
        \end{RLtext}}
        
        Using similar formatting and style, write an Arabic Wikipedia summary section for the title "\<ركمجة>". The summary should at least have 250 words\\
        
        \midrule
        Machine & 
        {\small \begin{RLtext}

        ركمجة هي رياضة جماعية تقوم على استخدام دراجات صغيرة وخفيفة ومزودة بإطارات عريضة وشاسيه معزز لتحمل الصدمات. تُعرف أيضًا باسم "الدراجات الجبلية". بدأت الرياضة في التطور في الولايات المتحدة في أوائل القرن العشرين، وانتشرت بسرعة في جميع أنحاء العالم. تتضمن الركمجة العديد من الانواع والتقنيات التي تختلف تبعًا للأنماط والمنافسات. فمن بينها: الركمجة العادية، ركوب الغابات، النزهة، إطلاق النار، السباقات، الهواء الطلق، الفوركس، والتحديات. تعتمد تقنيات الركمجة على الحركة الفردية والفعالية. تحتوي الدراجات على عدد من المكونات الرئيسية، بما في ذلك الإطارات الصلبة والمتينة والتي تجعل الدوران أكثر سلاسة والتحكم أسهل. وتستخدم الإطارات الأمامية والخلفية مكابح التشريد والسرعات العالية لتحسين الأداء والأمان. يكتشف اللاعبون الأماكن الوعرة والمناظر الطبيعية المعرضة للخطر ويتنافسون ضد بعضهم البعض وضد الخطوط الزمنية. وتَزيد العوامل الخطرة وازدياد التحدي فرص المزيد من الإثارة والتشويق لهذه الرياضة. كما تعتبر تلك الأماكن مناسبة لتخويف الانتقام، وذلك بإخراج المنافسين عن المسار أو تعريضهم للطين والحجر المغبر. يمكن للركمجة أن تكون رياضة نمطية ريادية مع التقدم التكنولوجي. ويمكن زيادة الأمان وتحسين جودة الدراجات وتحسين الأداء. كما أن تطوير الدراجات الهجينة تنتج دراجات للسفر على المسارات الوعرة في الحدائق والغابات. تشهد الركمجة شعبية متزايدة في جميع أنحاء العالم وتعتبر وسيلة ممتازة لتحسين المهارات البدنية والتحمل والصحة العامة وكذلك الاندماج الاجتماعي. وتُنظم العديد من المنافسات على المستوى المحلي والإقليمي والعالمي لتحقيق الإثارة والتحديات الجديدة.

            \end{RLtext}}\\
    
        % \<ركمجة هي رياضة جماعية تقوم على استخدام دراجات صغيرة وخفيفة ومزودة>
        % \<بإطارات عريضة وشاسيه معزز لتحمل الصدمات. تُعرف أيضًا باسم>
        % \<"الدراجات الجبلية". بدأت الرياضة في التطور في الولايات المتحدة في>
        % \<أوائل القرن العشرين، وانتشرت بسرعة في جميع أنحاء العالم. تتضمن>
        % \<الركمجة العديد من الانواع والتقنيات التي تختلف تبعًا للأنماط و>
        % \<المنافسات. فمن بينها: الركمجة العادية، ركوب الغابات، النزهة، إطلاق>
        % \<النار، السباقات، الهواء الطلق، الفوركس، والتحديات. تعتمد تقنيات>
        % \<الركمجة على الحركة الفردية والفعالية. تحتوي الدراجات على عدد من>
        % \<المكونات الرئيسية، بما في ذلك الإطارات الصلبة والمتينة والتي تجعل >
        % \<الدوران أكثر سلاسة والتحكم أسهل. وتستخدم الإطارات الأمامية والخلفية>
        % \<مكابح التشريد والسرعات العالية لتحسين الأداء والأمان. يكتشف اللاعبون>
        % \<الأماكن الوعرة والمناظر الطبيعية المعرضة للخطر ويتنافسون ضد بعضهم>
        % \<البعض وضد الخطوط الزمنية. وتَزيد العوامل الخطرة وازدياد التحدي فرص>
        % \<المزيد من الإثارة والتشويق لهذه الرياضة. كما تعتبر تلك الأماكن>
        % \<مناسبة لتخويف الانتقام، وذلك بإخراج المنافسين عن المسار أو تعريضهم>
        % \<للطين والحجر المغبر. يمكن للركمجة أن تكون رياضة نمطية ريادية مع>\\
        % &\<التقدم التكنولوجي. ويمكن زيادة الأمان وتحسين جودة الدراجات وتحسين>
        % \<الأداء. كما أن تطوير الدراجات الهجينة تنتج دراجات للسفر على>
        % \<المسارات الوعرة في الحدائق والغابات. تشهد الركمجة شعبية متزايدة>
        % \<في جميع أنحاء العالم وتعتبر وسيلة ممتازة لتحسين المهارات البدنية>
        % \<والتحمل والصحة العامة وكذلك الاندماج الاجتماعي. وتُنظم العديد من>
        % \<المنافسات على المستوى المحلي والإقليمي والعالمي لتحقيق الإثارة>
        % \<والتحديات الجديدة.>
        % \\
        % \midrule 
        % \midrule 
        Language &  \textbf{Russian} \\
        LLM Generator & \textsc{Davinci-003} \\
        Domain &  \textit{RuATD} \\
        \midrule
        Human & \foreignlanguage{russian}{День был посвящен празднику на «линии движения» его сына. Гнетущее осталось впечатление от некоторых (от трех лиц) явно и (одно из них) даже оформленно приближенного к его жизни с особыми правами на близость. Угашаю вспыхнувшую потребность судить и осуждать одну женщину и одну мужскую фигуру, их лица, глаза, выражения, манеры смотреть, говорить, двигаться. Осталось больным местом только искаженное их присутствием и тоном отношение к ним главы дома. Жена и мать ее, и место, какое они заняли в обновлении внешнего и внутреннего лика Игоря, хочется верить, не смогут в главных чертах лика его внутренней жизни внести изменение в сторону разрушительную или застойную. Внешне облегчат и приукрасят жизнь. Одни ямочки в улыбке жены на ее девически-свежем лице, при твердом до жесткости взгляде ясных серых глаз, делают понятной ее привлекательность для мягкосердечного, но в какой-то части своего существа угрюмого мужа. Вспомнилось, как 4 года тому назад после знакомства его с одной красивой, в возрасте между 30-40 годами, женщиной, которая картиной душевного мира и взглядом на жизнь, по-моему, должна была бы разбудить интерес к себе у Игоря, он согласился со мной, что «она, по-видимому, хороший человек». Но тут же прибавил: — Разве вы не видите, тетя Вава, что мне лучше всего одному, как я теперь живу. Дико и оскорбительно представить себе вот в этих стенах какую-то другую жену, кроме той, которая у меня отнята смертью.}\\ 
        \midrule
        Prompt & \foreignlanguage{russian}{Пожалуйста, переформулируйте этот текст, чтобы он содержал более 1000 символов. Текст: 'День был посвящен празднику на «линии движения» его сына. Гнетущее осталось впечатление от некоторых (от трех лиц) явно и (одно из них) даже оформленно приближенного к его жизни с особыми правами на близость. Угашаю вспыхнувшую потребность судить и осуждать одну женщину и одну мужскую фигуру, их лица, глаза, выражения, манеры смотреть, говорить, двигаться. Осталось больным местом только искаженное их присутствием и тоном отношение к ним главы дома. Жена и мать ее, и место, какое они заняли в обновлении внешнего и внутреннего лика Игоря, хочется верить, не смогут в главных чертах лика его внутренней жизни внести изменение в сторону разрушительную или застойную. Внешне облегчат и приукрасят жизнь. Одни ямочки в улыбке жены на ее девически-свежем лице, при твердом до жесткости взгляде ясных серых глаз, делают понятной ее привлекательность для мягкосердечного, но в какой-то части своего существа угрюмого мужа. Вспомнилось, как 4 года тому назад после знакомства его с одной красивой, в возрасте между 30-40 годами, женщиной, которая картиной душевного мира и взглядом на жизнь, по-моему, должна была бы разбудить интерес к себе у Игоря, он согласился со мной, что «она, по-видимому, хороший человек». Но тут же прибавил: — Разве вы не видите, тетя Вава, что мне лучше всего одному, как я теперь живу. Дико и оскорбительно представить себе вот в этих стенах какую-то другую жену, кроме той, которая у меня отнята смертью.'}\\
        \midrule
        Machine & \foreignlanguage{russian}{День праздника был посвящен пути движения Игоря, его сына. Взгляд, которым его судьба была окрашена, был властный, а к мимолетным правам на близость был дан официальный статус. Желание судить и осуждать пару, видимо, комплексное, так как сразу трое - мужчина, женщина и в какой-то мере дети, занимаются браконьерством. Даже брутальное отношение главы дома к ним не могло исказить более-менее нормальное настроение.Одержимость жены, ее свежесть, девичья улыбка с ямочками на лице, а также твердый взгляд серых глаз очень привлекла мягкосердечного, но застаревшего мужа. Назад вспомнили, как 4 года назад он встретился с красавицей, которая должна была привлечь его интерес, по мнению Игоря, в душевном мире и представлении жизни. Однако, овладение собой было столь сильно, что он дошел до того, что «лучше для него жить в одиночестве, чем представлять себе другую жену внутри этих чертовых стен». Даже сейчас после потери жены, чья память приукрасит и облегчит жизнь Игоря и его сына, до разрушительных, можно догадываться, или загнутых ходов, они не доводятся.}\\

        \midrule 
        \midrule 
        Language &  \textbf{Russian} \\
        LLM Generator & \textsc{ChatGPT} \\
        Domain &  \textit{RuATD} \\
        \midrule
        Human & \foreignlanguage{russian}{Регулирование градостроительной деятельности на территории муниципального образования село Ныда (далее – муниципальное образование) должно осуществляться в соответствии с основными принципами законодательства о градостроительной деятельности, направленными на устойчивое развитие территории, создание условий для привлечения инвестиций и активизации строительства, формирование экологически безопасной, благоприятной среды жизнедеятельности, комплексное и эффективное развитие социальной, производственной, инженерной и транспортной инфраструктуры, бережное природопользование, повышение уровня архитектурно-художественной выразительности застройки муниципального образования, на основании утвержденных документов территориального планирования и градостроительного зонирования, а также документации по планировке территории.}\\ 
        \midrule
        Prompt & \foreignlanguage{russian}{Пожалуйста, переформулируйте этот текст, чтобы он содержал более 1000 символов. Текст:'Регулирование градостроительной деятельности на территории муниципального образования село Ныда (далее – муниципальное образование) должно осуществляться в соответствии с основными принципами законодательства о градостроительной деятельности, направленными на устойчивое развитие территории, создание условий для привлечения инвестиций и активизации строительства, формирование экологически безопасной, благоприятной среды жизнедеятельности, комплексное и эффективное развитие социальной, производственной, инженерной и транспортной инфраструктуры, бережное природопользование, повышение уровня архитектурно-художественной выразительности застройки муниципального образования, на основании утвержденных документов территориального планирования и градостроительного зонирования, а также документации по планировке территории.'}\\
        \midrule
        Machine & \foreignlanguage{russian}{Муниципальное образование село Ныда должно регулировать свою градостроительную деятельность в соответствии с законодательством о градостроительстве. Эти законы направлены на устойчивое развитие территории и создание обстановки, способствующей инвестициям и активизации строительства. Они также ставят целью сформировать экологически безопасную, благоприятную среду жизнедеятельности и развивать всю инфраструктуру, как социальную, так и производственную, инженерную и транспортную. Помимо этого, они должны бережно использовать ресурсы природы и повышать уровень архитектурной экспрессии в чертежах своих построек.
        
        Муниципальное образование должно использовать утвержденные документы по территориальному планированию и градостроительному зонированию, а также документацию по планировке территории. Эти правила регулируют различные аспекты градостроительства, такие как масштаб и высота зданий, открытые пространства, пешеходные и транспортные маршруты, зеленые насаждения и многое другое.
        
        Местные власти должны заботиться о том, чтобы градостроительство было эффективным и комплексным, учитывая различные потребности граждан в доступном жилье и инфраструктуре. Они также должны уделять большое внимание экологической устойчивости и сохранению биоразнообразия.
        
        В целом, здесь требуется сбалансированное подход к градостроительству в муниципальном образовании, искреннее взаимодействие с обществом и эффективное использование ресурсов, чтобы создать благоприятную и устойчивую среду для жизни.}\\

        % \midrule 
        % \midrule 
        Language &  \textbf{Bulgarian} \\
        LLM Generator & \textsc{Davinci-003} \\
        Domain &  \textit{Bulgarian True \& Fake News} \\
        \midrule
        Title & \foreignlanguage{russian}{Скандал! РДНСК прати багер да маха буните на най-скъпия плаж в България} \\
        \midrule
        Human & \foreignlanguage{russian}{28 Март 2017, Вторник Автор: Иван КОЛЕВ, снимки: Георги РУСЧЕВИвицата е тясна, ако ги няма буните плажът ще изчезне, алармира стопанинът на плажа - „Акра лидо”Багери, пратени от РДНСК Бургас, тази сутрин „захапаха” двете буни на къмпинг „Черноморец”. От строителния надзор твърдят, че съоръженията са незаконни и трябвало да бъдат премахнати. Машините са спрени с жива верига от 15-ина души и в момента се очаква полицията.„Няма да допуснем да продължат. Ако се премахнат буните за един месец ще изчезне целият плаж”, каза един от протестиращите пред Флагман.бг.Наш репортер се свърза и със стопанина на плажа – бургаската фирма „Акра лидо” ООД, която има 5-годишен договор за ивицата и плаща най-скъпия плажен наем в България – по 107 хил.лева на година за изключително малка площ.От компанията останаха изненадани от действията на строителния наздор. Буните са там от много години, спират свлачищния процес и осигурят плажна ивица, макар и много тясна.Само преди седмица от „Акра лидо” са платили накуп наемната цена за 2017 година, а сега ако бъдат премахнати буните ще останат без плаж.Скандалът ще има международен отзвук, защото 500 руски семейства имат жилища в непосредствена близост до къмпинг „Черноморец” и са ги закупили заради това, че плажът им е само на няколко метра.Това е друга от машините, която ще участва в акцията по разрушаването на буните„Ние сме изрядни платци, наели сме плажа в това състояние от държавата и до тази сутрин смятахме, че всичко е изрядно. Не сме поставяли под съмнение, че държавата ни е отдала под наем плаж с незаконни буни”, каза един от собствениците на фирмата Милен Янков пред Флагман.бг.Припомняме, че миналата година също имаше много проблеми с този плаж. В основата бяха отношенията със стария стопанин на ивицата (ВИЖ НАШАТА ТЕМА В РАЗВИТИЕ)ОЧАКВАЙТЕ ПОДРОБНОСТИ! В категории: Созопол}\\ 
        \midrule
        Prompt & \foreignlanguage{russian}{Моля, генерирайте статия с дължина над 1000 символа и заглавие "Скандал! РДНСК прати багер да маха буните на най-скъпия плаж в България"} \\
        \midrule
        Machine & \foreignlanguage{russian}{Скандал! РДНСК прати багер да маха буните на най-скъпия плаж в България
        
        В този момент, най-високото бюро в България за разрешаване на спорове между плавателни клубове и партньорите им предприе нещо неочаквано. Речникът Държавен надзор на стандартите (РДНСК) изпрати багер да махне буните на най-скъпия плаж в България - Golden Sands.
        
        Бунита е в Балтийското море, в близост до Варна. По неизвестни причини РДНСК реши да започне да маха буни, а при това привлече много необичайно интересни детайли.
        
        Ако се сещате по-добре, то освен това, че РДНСК разреши махането на буните на Golden Sands, той поръча да се изтрият буните от другите плажове в България в същото време. Това е правило, което РДНСК не следва преди това и което създаде много недоволство в индустрията.
        
        Докато правните изследвания са все още в процес на продължаване за да се разбере точно какви са причините за багера, се очаква тази политика да продължи и да се извършват постоянни разрешения за някои интереси. Някои пленници на ансамбъла на бунита подози, че максимумът на разрешенията е 100, които могат да се дават годишно според правилата, които сме предвидили.
        
        Защо РДНСК решил да направи това и защо именно сега, това продължава да бъде неизвестно. Но понеже това се оказва наистина забранено, много от плавателните клубове и индивиди на плажа ще бъдат принудени да се преместят на други места.
        
        Бановете, разположени в гледната точка за потоците, да се зачитат на рилско море - това ли ще стане след махането на буните? Ще има ли някаква защита за плавателните инсталации, една от важните атракции на Golden Sands? Ще плажът продължи да съществува или ще бъде претърпял значителни изменения? Тези и много други въпроси очакват отговори.
        
        За да се оцени тайнственото поведение на РДНСК, много информация трябва да се прецени. Ето защо е особено важно да се осигури правопорядъчно намиране на логичен разследвателен процес за да се разберат всички мотиви за действието на РДНСК и да се донесат промени, които да се правят в добро на всички страни.}\\

        % \midrule
        % \midrule 
        Language &  \textbf{Bulgarian} \\
        LLM Generator & \textsc{ChatGPT} \\
        Domain &  \textit{Bulgarian True \& Fake News} \\
        \midrule
        Title & \foreignlanguage{russian}{Неподозирани разкрития за кралското семейство} \\
        \midrule
        Human & \foreignlanguage{russian}{Обикновените хора се обръщат към кралица Елизабет Втора с „Нейно кралско величество“ и придружават обръщението с поклон и израз на любезност. Но зад кулисите отношенията са доста по-неформални, разказват от „Новини Лондон“. Потребител на сайта Quora зададе въпрос: „Как внуците на кралица Елизабет се обръщат към нея на четири очи?“. Макар че отговорите варират, проучване на „Индипендънт“ показва, че внуците на кралицата са й измислили обръщения, изразяващи обичта им към нея. Уилям и Хари окуражаваха баба си при нейната неочаквана, спираща дъха роля на церемонията по откриването на Олимпийските игри през 2012 г. При вида на баба им, която се появи заедно с Даниел Крейг като Джеймс Бонд, принцовете извикаха с цяло гърло: „Давай, бабо!“. Но това не е единственото име, използвано от Уилям и Хари. Според светския хроникьор Ричард Кей веднъж в Бъкингамския дворец малкият Уилям казал на баба си „Гари“. Когато служители на двореца попитали „Кой е Гари?“, кралицата отговорила: „Аз съм Гари“. Уил още не се е научил да казва „Granny” (бабо – английски език, бел. ред.), обяснила Елизабет. Тригодишният правнук на Елизабет Втора принц Джордж също има любвеобилно обръщение към кралицата, както стана ясно от интервю, излъчено по телевизия ITV миналото лято по повод 90-годишнината й. Херцогинята на Кеймбридж Кейт Мидълтън сподели, че Джордж нарича баба си „Ган Ган“. Но това не е просто мило обръщение, а е израз на дълготрайна кралска традиция. Американският биограф Кити Кели твърди, че принц Чарлз, Уилям и Хари също се обръщали така към своята прабаба.}\\ 
        \midrule
        Prompt & \foreignlanguage{russian}{Моля, генерирайте статия с дължина над 1000 символа и заглавие "Неподозирани разкрития за кралското семейство"} \\
        \midrule
        Machine & \foreignlanguage{russian}{Неподозирани разкрития за кралското семейство
        
        Кралските семейства винаги са били обект на внимание и интерес от страна на медиите и обществото. Известни са скандалите и скритите истории, които остават извън зрителния ъгъл и несъмнено експлоатирането на подобни теми продължава да държи високи места в таблоидите.
        
        Но дали знаете всичко за кралските семейства, което медиите не съобщават? Има ли неща, които остават скрити дълбоко в историята или в семейните тайни?
        
        Изглежда, отговорът е да. В последните години имаше повече и повече изисквания за откриване на тайните на кралските династии, но много от тези разкрития остават скрити за публика.
        
        Една от най-известните скрити истории, свързани с британското кралско семейство, е тази на принца Джон, третият син на крал Джордж V и кралица Мери. Принц Джон беше диагностициран с епилепсия и склероза на 7-годишна възраст. Той беше изолиран от обществото и от своето семейство и беше принуден да живее в бездейственост и самота, за да не покаже "ненормалното" си поведение. Принц Джон почина на 13-годишна възраст, покрит с тайните и позволено му беше да забрави света на живите.
        
        Разказите за скритата история на Американското кралско семейство са също известни. Съществува теория, че Сежър, който се хвали, че е третият наследник на трона след Алфонсо и Джон, наистина има връзки с кралското семейство на САЩ. През 1937 г. Сежър заяви, че е бил роден като дете на Сесилия, дъщерята на крал Едуард VIII и Уолис Симпсън.
        
        Макар това да е слух, наличието на тайни и тъмни истории в кралските семейства е добре документирано. Все още има много да се разкрие, но кралските династии на света ще продължат да привличат интриги и вълнение през годините.}\\
        

        % \midrule 
        Language &  \textbf{Indonesian} \\
        LLM Generator & \textsc{ChatGPT} \\
        Domain &  \textit{News} \\
        \midrule
        Title & Ahmad Dhani Klaim Tak Punya Informasi soal Sri Bintang \\
        \midrule
        Human & Jakarta, CNN Indonesia -- Musisi Ahmad Dhani memenuhi panggilan Polda Metro Jaya untuk diperiksa sebagai saksi tersangka dugaan makar Sri Bintang Pamungkas, Selasa (20/12). Berdasarkan pantauan CNNIndonesia.com, dia tiba pukul 15.00 WIB. Sedangkan tim kuasa hukumnya yang tergabung dalam Advokat Cinta Tanah Air sudah tiba satu jam sebelum kedatangannya. Tak lama kemudian, Farhat Abbas juga datang untuk mendampingi Ahmad Dhani. Ahmad Dhani mengatakan, dirinya tidak akan memberikan informasi apapun soal Sri Bintang. Dia mengklaim tidak kenal dengan Sri Bintang. Buni Yani dan Ahmad Dhani Jadi Saksi Kasus Sri Bintang Besok Saksi Mengaku Dapat Aliran Dana dari Tersangka Makar Buni Yani Diperiksa Soal Pidato Sri Bintang di Kalijodo "Informasi pasti tidak ada, karena saya tidak kenal dengan Sri Bintang Pamungkas. Saya pernah ketemu beliau ketika di Mako Brimob," ucapnya. Meski demikian, Ahmad Dhani mengaku hadir saat ada pertemuan di Universitas Bung Karno. Namun, ia mengklaim tidak mendengar pidato yang disampaikan oleh Sri Bintang karena terlambat datang. Awalnya, Kepala Subdirektorat Kejahatan dan Kekerasan Direktorat Reserse Kriminal Umum Polda Metro Jaya AKBP Hendy Kurniawan mengatakan, karena Ahmad Dhani sedang sakit maka pemeriksaan terhadap dia yang akan dilakukan hari ini harus ditunda Kamis (22/12). "Ahmad Dhani ditunda karena sakit. Surat sakit sudah kami terima. Kami sudah koordinasi dengan kuasa hukumnya kemarin," ujarnya. Kuasa Hukum Ahmad Dhani, Ali Lubis mengklaim, kliennya tersebut merupakan warga negara yang baik sehingga datang memenuhi panggilan kepolisian. "Beliau kooperatif, beliau ingin membantu kepolisian jadi lebih cepat lebih baik," ucapnya. Ahmad Dhani telah ditetapkan sebagai tersangka dugaan penghinaan terhadap penguasa. Dia ikut ditangkap pada Jumat (2/12) dengan sejumlah tersangka dugaan makar lainnya. Sepuluh tersangka dugaan makar itu adalah Sri Bintang, Kivlan Zein, Adityawarman Thahar, Ratna Sarumpaet, Firza Huzein, Eko Santjojo, Alvin Indra, Rachmawati Soekarnoputri, dan kakak beradik Rizal Kobar dan Jamran.\\ 
        \midrule
        Prompt & Tulislah suatu artikel berita yang berjudul "Ahmad Dhani Klaim Tak Punya Informasi soal Sri Bintang". Artikelnya harus mengandung paling tidak 300 kata dalam bahasa Indonesia.     Tambahkan "Jakarta, CNN Indonesia -- " di awal artikel. \\
        \midrule
        Machine & Jakarta, CNN Indonesia -- Ahmad Dhani, personel grup band Dewa 19, mengaku tak memiliki informasi mengenai sosok Sri Bintang yang tengah ramai dibicarakan di media sosial. Seperti diketahui, Sri Bintang menjadi perbincangan publik setelah dirinya mengunggah video yang menyebutkan bahwa personel band Dewa 19, Ahmad Dhani, pernah menyebut Sri Bintang sebagai sosok yang berperan dalam pembuatannya dari awal. Klaim tersebut kemudian dibantah oleh Ahmad Dhani melalui sebuah video yang diunggahnya di akun media sosialnya. "Saya sendiri nggak tahu siapa Sri Bintang itu. Jadi saya gak bisa bicara, saya gak bisa komen, kenapa orang lain mau ngomongin saya tentang Sri Bintang," ujar Ahmad Dhani dalam video tersebut. Ahmad Dhani juga mengatakan bahwa beberapa orang yang mengklaim tahu tentang perjalanannya dalam industri musik sebenarnya tidak memiliki informasi yang tepat. "Saya sekali lagi ingin tegaskan bahwa ada beberapa orang yang bicara itu nggak tahu apa yang mereka bicarakan. Intinya, kita perlu memperhatikan orang yang kita ngomongin," tegasnya. Kendati begitu, Ahmad Dhani mengungkapkan bahwa dirinya berharap Sri Bintang bisa berkarya dan memperlihatkan kemampuan yang dimilikinya di industri musik tanah air. Ahmad Dhani juga berharap bahwa sosok Sri Bintang tidak menjadi obyek perdebatan yang tak berkesudahan di antara para penggemar musik dan publik di Indonesia. "Setidaknya semua artis harus diberikan kesempatan yang sama untuk berkarya. Saya berharap Sri Bintang bisa berkarya dengan baik dan melebihi ekspektasi dari semua orang," harapnya. Meski begitu, klaim dari Sri Bintang mengenai Ahmad Dhani telah menjadi perhatian banyak pengguna media sosial. Banyak netizen yang menyatakan bahwa hal tersebut merupakan sebuah bukti bahwa industri musik di Indonesia selalu menarik perhatian masyarakat dan penuh dengan kontroversi. Maka dari itu, publik diharapkan mampu menghormati karya dari masing-masing musisi dan menghindari terjadinya perdebatan yang tak perlu di media sosial.\\
       
        % \midrule 
        Language &  \textbf{Chinese} \\
        LLM Generator & \textsc{ChatGPT} \\
        Domain &  \textit{Baike QA} \\
        \midrule
        Question & \begin{CJK}{UTF8}{gbsn}关于百宝箱的问题我本人这里网络连接经常会不时中断,请问一下我的网\end{CJK} \\
        \midrule
        Human & \begin{CJK}{UTF8}{gbsn}我也是传3玩家,你的问题我经常遇到,其实这种问题是很难解决的,问题是GT的总服务器设定为无反映30秒下线,我们挂 机如果身边有怪突然掉线30秒的时间不喝红战士还有可能没问题,法师必挂,解决的话只能换换网络或者带普通装备挂机~!\end{CJK}\\ 
        \midrule
        Prompt & \begin{CJK}{UTF8}{gbsn}关于百宝箱的问题我本人这里网络连接经常会不时中断,请问一下我的网,详细来说,我本人这里连接经常会不时中断,请问一下我的网络连接断了以后人物还会保留在游戏中间一段时间吗?有的话是几秒?顺便还后突然重起机子的情况, 我有时在打怪的时候网络连接中断,再上的时,有时候是满血,好象没被怪打过,但是有几次就发现回城了,还掉了装备,请懂的详细告诉我一下,能解决突然网络中断人物立刻就下线的问题吗?(我用的是百宝挂机)\end{CJK}\\
        \midrule
        Machine & \begin{CJK}{UTF8}{gbsn}如果您的网络连接中断,您的角色可能会保留在游戏中的一段时间,具体时间取决于游戏服务器和您的网络连接速度。一般来说,如果您在游戏中掉线,您的角色会在几秒钟之内被服务器标记为离线状态,如果您的网络恢复连接,您可以尝试重新登录游戏。但如果您的网络连接中断时间太长,可能会导致您的角色被游戏服务器强制下线。\end{CJK}
        \begin{CJK}{UTF8}{gbsn}关于网络连接不稳定导致角色死亡或掉装备的问题,这是由于在网络中断的情况下,游戏服务器无法正确接收您的移动或攻击操作,因此可能会导致您的角色无法正常反应或被怪物攻击致死。而掉装备的情况可能是由于您的人物在网络中断时正好被怪物攻击,并且该物品尚未在游戏服务器上保存,因此游戏服务器无法将该物品恢复至您的背包中。
        解决突然网络中断导致角色掉线的最佳方法是改善您的网络连接,例如通过使用更稳定的网络连接方式、更快的网络速度、更好的路由器等来提高网络信号的稳定性。另外,您也可以考虑使用其他的游戏挂机工具来降低游戏对网络连接的依赖。\end{CJK}\\		
	\end{supertabular}
\end{center}