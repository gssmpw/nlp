\section{Conclusion}

% Building on previous work\todo{Preslav: Starting the conclusion by pointing this is based on previous work diminishes the contribution. Better start with what is new, then say it builds on previous work.} in LLMs safety evaluation, 
In this work, we introduce a new Chinese dataset that includes both general and cultural-specificity red-teaming questions, to examine the safety risks in Chinese and multilingual LLMs.
It consists of more than 3k prompts, covering three attack perspectives, with a focus on risk perception and sensitivity to keywords and phrases. 
We collect 15k responses from five different LLMs, and further propose new fine-grained guidelines for both manual and automatic harmfulness evaluation. 
Experiments show that LLMs can produce harmful responses even when presented with non-risky input prompts.
The five LLMs perform almost equally well over general questions, and the response harmfulness over culture-specific questions is the major determinant of the final safety rank.


% \todo{Preslav: Better somehow repeat the list of contributions from the end of teh introduction in some form. That you have Chinese, cultural-specificity, new categories of questions, that you are releasing the dataset, etc.} Notably, our findings suggest that fine-grained evaluation criteria yield more accurate insights into the safety of LLMs. Importantly, our research also highlights a crucial observation: LLMs can produce harmful responses even when presented with non-risky input prompts.

% \todo[inline]{Preslav: You need to say something about future work.}


%%% Local Variables:
%%% mode: latex
%%% TeX-master: "../ARR_2024"
%%% End:
