\section{Related Work}
\subsection{3D Feature Matching}
Feature matching is a crucial step in 
correspondence-based point cloud registration~\cite{FPFH,zeng20173dmatch,ao2021spinnet,huang2021predator,yu2021cofinet,qin2022geometric,yu2023peal}. 
According to the approach of feature extraction, feature matching methods 
can be categorized into two types: traditional descriptor-based methods~\cite{PFH, FPFH, salti2014shot} and 
learning-based methods~\cite{ao2021spinnet,huang2021predator,yu2021cofinet,qin2022geometric,yu2023peal}. 
Before the widespread adoption of deep learning, features were manually designed descriptors tailored to represent local information. 
They are classified into LRF-based methods~\cite{guo2013rotational, salti2014shot} and LRF-free methods~\cite{PFH, FPFH} 
based on whether a local reference frame (LRF) is needed. With the advancements in deep learning, 
numerous learning-based methods have emerged. Based on the representation strategy, 
they can be further divided into patch-based methods~\cite{gojcic2019perfect,ao2021spinnet,zhao2023spherenet} 
and fragment-based methods~\cite{choy2019fully,huang2021predator,yu2021cofinet,gath1989unsupervised}. 
SpinNet~\cite{ao2021spinnet} and BUFFER~\cite{ao2023buffer} have achieved excellent generalization ability through 3D cylindrical convolution. Additionally, CoFiNet~\cite{yu2021cofinet} and GeoTransformer~\cite{qin2022geometric} propose a coarse-to-fine matching strategy and a Transformer-based framework, which has garnered widespread attention. Subsequent methods~\cite{yang2022one, yu2023rotation, yu2023peal, jin2024multiway} have further 
improved performance by incorporating position encoding and prior information into this framework.
ODIN~\cite{jin2024multiway} achieves the best performance through diffusion strategy and global optimization.

\subsection{Geometry-based Outlier Removal}
Geometry-based outlier removal involves fitting models from noisy correspondences using the geometric properties of 3D scenes to estimate poses. The most classic method is random sampling consensus (RANSAC)~\cite{fischler1981random}, which 
samples correspondences multiple times and validates to 
obtain the best pose, thus mitigating the impact of noisy correspondences. Subsequently, numerous variants of RANSAC~\cite{barath2018graph, barath2022space, chum2008optimal, schnabel2007efficient} have been proposed to address issues 
of time consumption and instability. TEASER~\cite{yang2020teaser} introduces the truncated least squares (TLS) cost to solve for pose and employ rotation invariant measurements to handle outliers. Then, SC$^2$-PCR~\cite{chen2022sc2, chen2023sc2} propose second-order consistency for robust outlier removal. In recent years, methods~\cite{zhang20233d, yang2023mutual} based on 
geometric consistency have been widely applied in point cloud registration, leveraging graph-theoretic approaches~\cite{eppstein2010listing} and geometric consistency to 
remove outliers, thereby achieving robust registration.

\begin{figure*}[t]
    \centering{\includegraphics[width=1.0\textwidth]{new_v2.pdf}}  %0.32
    \caption{
       Overall framework of our method.
       We extract features from the original point cloud, 
       obtaining features $\mathbf{{F} }^{\mathcal{P}}$ and $\mathbf{{F} }^{\mathcal{Q}}$ 
       as input for our method. 
       % Initially, global 
       % feature matching is conducted using these features, followed by simple geometric constraints 
       % for preliminary spatial filtering to obtain initial correspondences $\mathcal{{G}}^{0}$. 
       Subsequently, a progressive 
       process is applied to iteratively regenerate more accurate and denser 
       correspondences $\mathcal{{G}}^{t}$. At each iteration, the output correspondences $\mathcal{{G}}^{t-1}$ from the previous 
       iteration serve as input. Firstly, prior-guided local grouping is employed to sample 
       seed corresponding points and form local correspondence regions 
       $\mathbf{P}^{t}_i$ and $\mathbf{Q}^{t}_i$. Then, for each pair of local correspondence regions, 
       generalized mutual matching is performed to get new correspondences. Next, these correspondences are refined locally and globally using our center-aware three-point consistency, 
       followed by a merging operation $\oplus$ of local correspondences $\mathcal{{G}}^{t}_{i}$ using a hash table.
       % Finally, pose $\mathbf T\{\mathbf R,\mathbf t\}$ is estimated using refined correspondences $\mathcal{{G}}^{T}$ through efficient SVD
       % , and Point-level Correspondence Verification is proposed for further pose refinement.
       Finally, using these refined correspondences,
       we achieve robust and accurate transformation estimation $\mathbf T\{\mathbf R,\mathbf t\}$ only with SVD.
       }
    \label{fig1}
 \end{figure*}

\subsection{Learning-based Outlier Removal}
Recent researches~\cite{3DRegNet, PointDSC, jiang2023robust} have integrated 
deep learning into outlier removal and pose estimation, 
showcasing promising performance through training. 
Inspired by image matching~\cite{CN-Net}, 3DRegNet~\cite{3DRegNet} adapts CN-Net~\cite{CN-Net} to remove outliers in point cloud registration. Subsequently, PointDSC~\cite{PointDSC} embeds the spatial consistency 
into feature maps and employs neural spectral matching to estimate confidence of correspondences, 
achieving robust registration. VBReg~\cite{jiang2023robust} utilizes variational Bayesian inference 
to achieve better outlier removal. Hunter~\cite{yao2023hunter} addresses severe outlier issues 
by introducing higher-order consistency using hypergraphs. However, while both geometry-based 
and learning-based methods have shown effectiveness in outlier removal, 
they inherently rely on the initial correspondences and only trim down on them, 
which poses challenges when initial inliers are scarce.