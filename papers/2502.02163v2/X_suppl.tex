\clearpage
\setcounter{page}{1}
\maketitlesupplementary


%************Appendx**********************
\section{Analysis and Proof}

\subsection{Proof of Theorem 1}
\ptitle{Theorem 1.}
\emph{
        % Assuming event $\Phi $, where the geometric consistency of correct correspondences is greater than the maximum consistency of incorrect correspondences, 
        % has a probability ${P}(\Phi)\approx  1 $,
        % if the score of the local correspondences satisfies $Score(\mathcal{G}^{t}_i)\geqslant  1$, then the proportion of correct correspondences $b>=a$. 
        Assuming event $\Phi $, where the consistency of correct correspondences is greater than the maximum consistency of incorrect correspondences, 
        has a probability ${P}(\Phi)\approx  1 $ (which means our method can correctly identify inliers),
        if the score of the local correspondences satisfies $Score(\mathcal{G}^{t}_i)\geqslant  1$, then the proportion of correct correspondences $b \geqslant a$.
}


\emph{Proof.}
    For simplicity, we denote the consistency matrix $\mathbf{S}^t_{i}$ as $\mathbf{S}_{\mathrm{CTC}}$. Thus, we can demonstrate that:
    % \begin{equation}
    %    \begin{aligned} 
    %     &P(Score(\mathcal{G}^{t}_i)\geqslant1) = P(\frac{|| \mathbf{S}_{\mathrm{SOG}}||_1}{a\cdot n^2}\geqslant 1)=1 \\
    %     =& P(\frac{  \mathop{\max}\limits_{1\leqslant j \leqslant n} \sum_{i=1}^{n} |{\mathbf{S}_{\mathrm{SOG}}}_{ij}|   }{a\cdot n^2}\geqslant 1) \\
    %     =& P(\frac{  \mathop{\max}\limits_{1\leqslant j \leqslant n} \sum_{i=1}^{n} |{\mathbf{S}_{\mathrm{SOG}}}_{ij}|   }{a\cdot n^2}\geqslant 1 | \Phi) P(\Phi)\\
    %     +& P(\frac{  \mathop{\max}\limits_{1\leqslant j \leqslant n} \sum_{i=1}^{n} |{\mathbf{S}_{\mathrm{SOG}}}_{ij}|   }{a\cdot n^2}\geqslant 1 | \overline\Phi) P(\overline\Phi) \\
    %     \overset{(1)}{\approx}& P(\frac{  \mathop{\max}\limits_{1\leqslant j \leqslant n} \sum_{i=1}^{n} |{\mathbf{S}_{\mathrm{SOG}}}_{ij}|   }{a\cdot n^2}\geqslant 1 | \Phi) P(\Phi)\\
    %     =& \left[1- P(\frac{  \mathop{\max}\limits_{1\leqslant j \leqslant n} \sum_{i=1}^{n} |{\mathbf{S}_{\mathrm{SOG}}}_{ij}|   }{a\cdot n^2}< 1 | \Phi) \right]P(\Phi)\\
    %     =& \left[1- P(\frac{ \sum_{i=1}^{n} |{\mathbf{S}_{\mathrm{SOG}}}_{il}|   }{a\cdot n^2}< 1 | \Phi)^n \right]P(\Phi)\\
    %     \overset{(2)}{\leqslant }& \left[1- P(\frac{ (b \cdot n) \cdot n   }{a\cdot n^2}< 1 | \Phi)^n \right]P(\Phi)\\
    %     \overset{(3)}{=}& \left[1- P(\frac{ (b \cdot n) \cdot n   }{a\cdot n^2}< 1)^n \right]P(\Phi)\\
    %     =& \left[1- P(\frac{b   }{a}< 1)^n \right]P(\Phi)\\
    %    \end{aligned} 
    % \label{proof1}
    % \end{equation}
        \begin{equation}
       \begin{aligned} 
        &P(Score(\mathcal{G}^{t}_i)\geqslant1) = P(\frac{|| \mathbf{S}_{\mathrm{CTC}}||_1}{a\cdot N}\geqslant 1)=1 \\
        =& P(\frac{  \mathop{\max}\limits_{1\leqslant j \leqslant N} \sum_{i=1}^{N} |{\mathbf{S}_{\mathrm{CTC}}}_{ij}|   }{a\cdot N}\geqslant 1) \\
        =& P(\frac{  \mathop{\max}\limits_{1\leqslant j \leqslant N} \sum_{i=1}^{N} |{\mathbf{S}_{\mathrm{CTC}}}_{ij}|   }{a\cdot N}\geqslant 1 | \Phi) P(\Phi)\\
        +& P(\frac{  \mathop{\max}\limits_{1\leqslant j \leqslant N} \sum_{i=1}^{N} |{\mathbf{S}_{\mathrm{CTC}}}_{ij}|   }{a\cdot N}\geqslant 1 | \overline\Phi) P(\overline\Phi) \\
        \overset{(1)}{\approx}& P(\frac{  \mathop{\max}\limits_{1\leqslant j \leqslant N} \sum_{i=1}^{N} |{\mathbf{S}_{\mathrm{CTC}}}_{ij}|   }{a\cdot N}\geqslant 1 | \Phi) P(\Phi)\\
        =& \left[1- P(\frac{  \mathop{\max}\limits_{1\leqslant j \leqslant N} \sum_{i=1}^{N} |{\mathbf{S}_{\mathrm{CTC}}}_{ij}|   }{a\cdot N}< 1 | \Phi) \right]P(\Phi)\\
        =& \left[1- P(\frac{ \sum_{i=1}^{N} |{\mathbf{S}_{\mathrm{CTC}}}_{il}|   }{a\cdot N}< 1 | \Phi)^N \right]P(\Phi)\\
        \overset{(2)}{\leqslant }& \left[1- P(\frac{ b \cdot N   }{a\cdot N}< 1 )^N \right]P(\Phi)\\
        % \overset{(3)}{=}& \left[1- P(\frac{ b \cdot N   }{a\cdot N}< 1)^N \right]P(\Phi)\\
        =& \left[1- P(\frac{b   }{a}< 1)^N \right]P(\Phi)\\
       \end{aligned} 
    \label{proof1}
    \end{equation}
where step (1) is based on our assumption that ${P}(\Phi)\approx  1 $. 
Step (2) can be easily derived 
from the definition of consistency matrix. By definition, the number of 
correct correspondences is $b\cdot N$. According to the definition of our consistency matrix, 
When the geometric consistency of correct correspondences is greater than the maximum consistency of incorrect correspondences, that is, event $\Phi$ occurs, there is always:
% the matrix element ${\mathbf{S}_{\mathrm{SOG}}}_{ij}$ represents the number of correspondences 
% that are consistent with both correspondence $i$ and correspondence $j$. Thus, ${\mathbf{S}_{\mathrm{SOG}}}_{ij} \leqslant b \cdot N$. 
% The equality holds if and only if both $i$ and $j$ are correct correspondences. Therefore, we have:
\begin{equation}
    \sum_{i=1}^{n} |{\mathbf{S}_{\mathrm{CTC}}}_{il}| \leqslant b \cdot N.
    \label{proof2}
\end{equation}
According to Eq.~\ref{proof2} and~\ref{proof1}, we have already proved that $\left[1- P(\frac{b   }{a}< 1)^N \right]P(\Phi) \geqslant 1$.
Below is our final derivation:
\begin{equation}
\begin{aligned} 
    &\left[1- P(\frac{b   }{a}< 1)^N \right]P(\Phi) \geqslant 1 \\
    \Rightarrow & \left[1- P(\frac{b   }{a}< 1)^N \right] \geqslant 1 \\
    \Rightarrow & P(\frac{b   }{a}< 1)^N \leqslant 0 \\
    \Rightarrow & P(\frac{b   }{a}< 1) \leqslant 0 \\
    \Rightarrow & P(\frac{b   }{a}\geqslant  1) =  1
\end{aligned} 
\end{equation}
Thus, we have proved the proportion of correct correspondences 
$b$ exceeds the threshold $a$.
Therefore, given $a$, we can distinguish local regions where the inlier ratio is greater than $a$.

This theorem is crucial for our correspondence refinement, as it allows control over correspondences through hyperparameter $a$ tuning. Additionally, It provides a theoretical guarantee for the quality of correspondence at each step. It is also valuable for other outlier removal methods.

\subsection{Proof of Theorem 2}
\ptitle{Theorem 2.}
\emph{
Our generalized mutual matching (GMM) generates at least as many correct correspondences as those produced by mutual matching (MM), that is, $|\mathcal{{G}}_{GMM}|\geqslant |\mathcal{{G}}_{MM}|$ where 
$\mathcal{{G}}_{GMM}$ and $\mathcal{{G}}_{MM}$ represent the correspondences obtained from the GMM and MM methods, respectively.  
}

\emph{Proof.}
According to the definition of the matching matrices $\mathbf{M}^{\mathcal{P}\rightarrow\mathcal{Q} }_1$ and $\mathbf{M}^{\mathcal{Q}\rightarrow\mathcal{P} }_1$ provided in Sec.~\ref{Grouping}, the mutual matching (MM) process can be expressed as follows:
\begin{equation}
    \mathcal{{G}}_{MM} = \mathbb{G}   \left( \mathbf{M}^{\mathcal{P}\rightarrow\mathcal{Q} }_1 \odot \mathbf{M}^{\mathcal{Q}\rightarrow\mathcal{P} }_1  \right) 
\end{equation}
where the function $\mathbb{G}(\mathbf{M})$ extracts correspondences from the matching matrix $\mathbf{M}$, which is defined as follows:
\begin{equation}
     \mathbb{G} \left( \mathbf{M}\right) = \left\{(\mathbf{p}_i, \mathbf{q}_j) | \mathbf{M}(i,j)=1, \mathbf{p}_i\in \mathbf{P}, \mathbf{q}_j\in \mathbf{Q}  \right\}
\end{equation}
In contrast, our GMM process can be expressed as:
\begin{equation}
    \mathcal{{G}}_{GMM} = \mathbb{G}   \left( \left( \mathbf{M}^{\mathcal{P}\rightarrow\mathcal{Q} }_1 \odot \mathbf{M}^{\mathcal{Q}\rightarrow\mathcal{P} }_2  \right) \otimes \left( \mathbf{M}^{\mathcal{Q}\rightarrow\mathcal{P} }_1 \odot \mathbf{M}^{\mathcal{P}\rightarrow\mathcal{Q} }_2  \right) \right) 
\end{equation}
Based on the definition an properties of function $\mathbb{G}$, we can derive the following:
\begin{equation}
    \begin{aligned} 
     & |\mathcal{{G}}_{GMM}|\\
     =&\left|\mathbb{G}   \left( \left( \mathbf{M}^{\mathcal{P}\rightarrow\mathcal{Q} }_1 \circ \mathbf{M}^{\mathcal{Q}\rightarrow\mathcal{P} }_2  \right) \otimes \left( \mathbf{M}^{\mathcal{Q}\rightarrow\mathcal{P} }_1 \circ \mathbf{M}^{\mathcal{P}\rightarrow\mathcal{Q} }_2  \right) \right) \right|\\
     \geqslant & \left|\mathbb{G}   \left( \left( \mathbf{M}^{\mathcal{P}\rightarrow\mathcal{Q} }_1 \circ \mathbf{M}^{\mathcal{Q}\rightarrow\mathcal{P} }_2  \right) \right) \right|\\
     \geqslant & \left|\mathbb{G}   \left( \left( \mathbf{M}^{\mathcal{P}\rightarrow\mathcal{Q} }_1 \circ \mathbf{M}^{\mathcal{Q}\rightarrow\mathcal{P} }_1  \right) \right) \right|\\
     =&|\mathcal{{G}}_{MM}|\\
    \end{aligned} 
 \label{proof4}
\end{equation}
 Thus, we complete the proof that  $|\mathcal{{G}}_{GMM}|\geqslant |\mathcal{{G}}_{MM}|$.


\subsection{Why our Regor perform well with few inliers?}
\begin{figure}[ht]
    \centering{\includegraphics[width=1.0\linewidth]{analysis.pdf}}  %0.32
    \caption{
        Illustration of our progressive correspondence regenerator.
        The green and red lines indicate correct and incorrect correspondences at the current stage, respectively. The purple and yellow dots represent the seed correspondences (center points) and really correct corresponding points, respectively.
        The luminous part is the overlapping area. 
        The dotted circles are the local corresponding regions.
        % With each iteration, we regenerate more correct correspondences.
       }
    \label{analysis}
\end{figure}

The advantage of our algorithm lies in its ability to achieve robust registration even with very few inliers or under extreme outlier rates, which is unattainable for existing outlier removal methods.
As shown in Fig.~\ref{analysis}a, the initial correspondences are filled with numerous incorrect matches, with only one inlier. Methods relying on geometric consistency, such as~\cite{chen2022sc2,zhang20233d}, would fail to extract this inlier.  Moreover, it is impossible to solve the 6DoF pose with only a single correct correspondence.%relying on a single correct correspondence is insufficient for solving the 6DoF pose accurately.

In contrast, our method still performs well in such harsh situations. The reasons can be summarized in the following two aspects. \textbf{First}, the whole process is gradually generated and corrected, which does not depend on the quality of the initial correspondence. \textbf{Second}, outliers with small errors conduce to regenerate new correct correspondences by local matching. %The key to our method's success in such cases lies in its ability to leverage erroneous correspondences with small errors to generate new correspondences through local matching.  
For example, in Figure~\ref{analysis}, our approach begins by sampling a seed correspondence (the purple point in the figure).  For clarity, we illustrate a single seed point here.  Although the seed correspondence is incorrect (the correct match is the yellow point in the figure), our next step, local matching, is crucial.
If the seed correspondence has a relatively small distance error, it can still guide the matching of overlapping local regions.  In these regions, despite the initial error, the overlap enables the generation of new correct correspondences.  More importantly, performing local matching within these regions significantly reduces the search space compared to global matching across the entire point cloud, making it much easier to obtain correct correspondences.

Additionally, in the early iterations of our algorithm, we adopt a larger region radius for local matching.  This increases the probability of overlap between local regions, mitigating the impact of erroneous correspondences during the initial stages.  As the iterations progress, the region radius is gradually reduced to refine the matches and improve the overall quality of the correspondences.
As a result, even with very few inliers, our method can utilize erroneous correspondences with small errors to generate new correct correspondences, making accurate pose estimation possible under such challenging conditions.


\section{Implementation details}


\subsection{Point-level Pose Refinement}
Pose refinement is crucial for achieving accurate pose estimation. In the final stage of correspondence regeneration, we perform pose refinement based on the established correspondences. Unlike correspondence-level pose refinement such as SC2-PCR~\cite{chen2022sc2} and MAC~\cite{zhang20233d}, our approach conducts pose refinement at the point level.

A key advantage of our approach is its ability to generate dense and accurate correspondences. However, traditional pose refinement evaluation metrics, such as inlier count (IC)~\cite{chen2022sc2}, mean square error (MSE), truncated chamfer distance (TCD), and feature-and-spatial-consistency-constrained truncated chamfer distance (FS-TCD)~\cite{chen2023sc2}, rely heavily on initial correspondences. Given the limited number of correct initial correspondences, these metrics fail to fully capture the benefits of our dense and accurate correspondences. To address this limitation, we propose a novel point-level evaluation metric, termed point-level truncated chamfer distance (PO-TCD):
\begin{equation}
    { \mathbf{R}}^{*}, { \mathbf{t}}^{*}=\max _{\mathbf{R}, \mathbf{t}} 
    \sum_{\mathbf{{p}}_i \in \mathbf{{P}}} 
    \mathds {1}\left(
        \min _{\mathbf{{q}}_j \in \mathbf{{Q}}}
        \left\|  \mathbf{\mathbf{R} {p}}_i+\mathbf{t}-\mathbf{{q}}_j  \right\|< {\sigma_d}
        \right).
\end{equation}
It eliminates reliance on initial correspondences and instead computes the optimal truncation distance directly from dense point clouds, fully leveraging the benefits of our dense correspondences.


\subsection{Hyperparameter Settings}

\begin{table}[htbp]
    % \setlength{\tabcolsep}{3.5pt}
    \centering
    % \setlength{\tabcolsep}{2pt}
    % \scriptsize
    \resizebox{1.0\linewidth}{!}{
    \begin{tabular}{l|cccccccc}
    \toprule
    Dataset  &$k_0$ &$r_0$ &$s_0$ &$\omega_k$ &$\omega_r$ &$\omega_s$ &$k_{GMM}$ &$a$\\
    \midrule
    3DMatch~\cite{zeng20173dmatch} &20 &1&500 &5 &0.5 &0.2 &3 &0.5   \\
    KITTI~\cite{KITTIdataset}   &20 &10 &500 &5 &0.5 &0.2 &3 &0.5   \\
    \bottomrule
    \end{tabular}
    }
    \caption{Hyperparameter settings in different datasets. }
    \label{Hyperparameters Settings}
\end{table}

The detailed hyperparameter settings are provided in Table~\ref{Hyperparameters Settings}. Due to the differing scales of indoor and outdoor datasets, we specify two distinct parameter configurations, differing only in the local region radius. Specifically, for indoor datasets, we set $r_0$ to 1m, while for outdoor datasets, $r_0$ is set to 10m. All other parameters remain consistent across both settings.

Here, we outline the implementation details of our approach. First, for local grouping, during the
$t$th iteration, we perform random sampling with sampling quantity
${s_0}\cdot{(w_s)^{t}}$ on the correspondences obtained from the $t-1$th iteration. Using a radius of ${r_0}\cdot{(r_s)^t}$, we construct local point sets and standardize the number of points in each local set to ${k_0}\cdot{(w_k)^{t}}$ by sampling to facilitate parallel computation. Next, we apply generalized mutual matching with 
$k_{GMM}=3$ to obtain reliable correspondences. Finally, through a refinement process that combines local and global correspondence optimization, the global correspondences are updated.
It is worth noting that before the first iteration of sampling, we utilize the SC2-PCR~\cite{chen2022sc2} module to ensure the initial quality of the correspondences. Additionally, after the final correspondence refinement stage, we perform point-level pose refinement to further enhance the results.

Throughout the entire process, point sets for each stage are stored and updated using index-based representations, significantly reducing memory overhead. Similarly, during Correspondence Merging, we employ an index-based hash $\mathbb{H}$ table to efficiently obtain unique global correspondences, ensuring both speed and memory efficiency.

\section{Detailed Metrics}

\ptitle{Inlier Number (IN). }
 It represents the number of correct correspondences in the final correspondences, reflecting our ability to generate inliers.

\ptitle{Inlier Number Ratio (INR).}
 It is the ratio of inliers $\mathrm{IN}_{a}$ in the final correspondences to the inliers $\mathrm{IN}_{b}$ in the initial correspondences, which can be calculated by:
%  \begin{equation}
%     \mathrm{INR} = \frac{1}{H} \sum_{h=1}^H \mathrm{INR}_h,  \mathrm{INR}_h = 
%     \begin{cases}
%         \frac{\mathrm{IN}^{a}_h}{\mathrm{IN}^{b}_h}, \mathrm{IN}^{b}_h\neq 0\\
%         \mathrm{IN}^{a}_h, \mathrm{IN}^{b}_h = 0
%     \end{cases}
%  \end{equation}
\begin{equation}
    \mathrm{INR} = \frac{1}{H} \sum_{h=1}^H \mathrm{INR}_h,\,  \mathrm{INR}_h = 
    \begin{cases}
        \frac{\mathrm{IN}^{a}_h}{\mathrm{IN}^{b}_h}, \mathrm{IN}^{b}_h\neq 0\\
        \mathrm{IN}^{a}_h, \mathrm{IN}^{b}_h = 0
    \end{cases}
 \end{equation}
where $H$ represents the total number of point cloud pairs in the dataset, and 
$\mathrm{INR}_h$ denotes the INR for the $h$th point cloud pair.

\ptitle{Registration Recall (RR).}
Following~\cite{PointDSC,chen2022sc2}, we consider registration to be successful if the translation error (TE) and rotation error (RE) are within specified thresholds. For indoor scenes, the thresholds are $15^{\circ}$ and $30$cm, while for outdoor scenes, they are 
$5^{\circ}$ and $60$cm. The registration recall rate is calculated as:
\begin{equation}
    \mathrm{RR}=\frac{1}{H} \sum_{h=1}^H \mathds {1}\left(\mathrm{TE}_h<\sigma_d  \wedge  \mathrm{RE}_h<\sigma_\theta\right),
\end{equation}
where
$\mathds {1}$ represents the indicator function, and 
$\sigma_d$ and $\sigma_\theta $ are the thresholds for rotation error and translation error, respectively. $\mathrm{RE}_h$ and 
$\mathrm{TE}_h$ denote the rotation error and translation error for the 
for the $h$th point cloud pair. They can be computed using the following formulas:
\begin{equation}
    \begin{cases}
        \mathrm{RE}_h=\arccos \left(\frac{\operatorname{trace}\left(\hat{\mathbf{R}}_h^T \mathbf{R}_h\right)-1}{2}\right) \\
        \mathrm{TE}_h=\left\|\hat{\mathbf{t}}_h-\mathbf{t}_h\right\|
    \end{cases}
\end{equation}
where $\hat{\mathbf{R}}_h$, $\mathbf{R}_h$, $\hat{\mathbf{h}}_t$ and $\mathbf{h}_t$ are our predicted rotation matrix, the ground-truth rotation matrix,
predicted translation vector and the ground-truth translation vector, respectively.

% \section{3DMatch-EOR Benchmark}

\begin{table*}[htbp]
    % \setlength{\tabcolsep}{3.5pt}
    \centering
    % \setlength{\tabcolsep}{2pt}
    % \scriptsize
    \resizebox{1.0\linewidth}{!}{
    \begin{tabular}{l|cccccc}
    \toprule
    Benchmark  &Indoor scene &Nuisances &Application scenarion &\# Matching pairs  \\
    \midrule
    3DMatch~\cite{zeng20173dmatch} &Indoor scene  &Occlusion, real noise &Registration &1623  \\
    3DLoMatch~\cite{huang2021predator} &Indoor scene  &Limited overlap, occlusion, real noise &Registration &1781  \\
    3DMatch-EOR 90\% (\emph{ours}) &Indoor scene  &High outlier ratio, occlusion, real noise &Registration &  1270\\
    3DMatch-EOR 99\% (\emph{ours}) &Indoor scene  &Extreme high outlier ratio, occlusion, real noise &Registration &  142\\
    KITTI~\cite{KITTIdataset} &Outdoor scene &Clutter, occlusion, real noise &Detection, registration, segmentation &555   \\
    \bottomrule
    \end{tabular}
    }
    \caption{Information of all tested datasets.}
    \label{tested datasets}
\end{table*}


\section{Additional Experiments}
The information of all tested datasets is given in Table~\ref{tested datasets}.

\subsection{Results at different numbers of inliers}

\begin{table*}[htbp]
    \centering
  %    \scriptsize
      \resizebox{1.0\linewidth}{!}{
          \begin{tabular}{cc|cccccc|cccccc}
              \toprule
              &  &\multicolumn{6}{c|}{Regor (\emph{ours})} & \multicolumn{6}{c}{Outlier removal~\cite{chen2022sc2}}  \\
              \# Inlier number &\# Pairs  &RR($\uparrow$) & RE($\downarrow$) & TE($\downarrow$) & IP($\uparrow$)  & INR($\uparrow$) & IN($\uparrow$) & RR($\uparrow$) & RE($\downarrow$) & TE($\downarrow$) & IP($\uparrow$)  & INR($\uparrow$) & IN($\uparrow$)     
              \\ \midrule 
                0$\sim$20   &55 & 3.87 &3.57  &10.36 &3.59 &273.01 &38.44
                &3.64 &4.82 &23.9 &1.49 &3.86 &0.02\\ 
                20$\sim$40  &95 &36.84  &3.98  &9.22 &29.65 &1510.95 &471.55
                &8.42 &5.16 &13.79  &10.33 &15.87 &0.29\\ 
                40$\sim$60   &80& 62.50  &2.71 &8.00 &52.83 &2536.91  &1275.64
                &27.5 &5.70 &12.86 &28.47 &28.47 &0.73\\ 
                60$\sim$80    &90 &80.00 &2.60 &8.50 &71.21 &2434.90 &1656.36   
                &63.33 &4.15 &11.62 &51.59 &61.64 &2.35\\ 
                80$\sim$100   &95 &91.58  &1.98 &8.04 &78.83 &2288.78 &2000.84 
                &86.32 &3.82  &11.22 &63.77 &72.58 &3.73\\ 
                100$\sim$200  &333 &95.50 &1.89 &7.00  &86.85 &1625.94 &2318.18 
                &92.19 &2.71  &8.24 &78.66 &85.15 &26.11\\ 
              \bottomrule
      \end{tabular}
      }
      \caption{Comparison between our Regor and outlier removal method~\cite{chen2022sc2} under different numbers of inliers. }
      \label{inlier_number}
\end{table*}


In Sec.~\ref{Indoor1}, we propose the 3DMatch-EOR Benchmark, which includes two baselines: one with an outlier ratio exceeding 99\% and another exceeding 90\%. From these experiments, we conclude that our method demonstrates greater robustness under extreme outlier ratios. However, since these experiments are designed based on ratios, they do not verify the impact of the number of inliers. To address this, we conduct additional comparative experiments using FPFH as the feature extraction baseline, varying the number of inliers. Specifically, we designed seven scenarios with inlier counts ranging from 0$\sim$20, 20$\sim$40, 40$\sim$60, 60$\sim$80, 80$\sim$100, to 100$\sim$200, comparing our method against the outlier removal method SC2-PCR~\cite{chen2022sc2}. The experimental results are summarized in Table~\ref{inlier_number}.

In all conditions, our method outperforms SC2-PCR~\cite{chen2022sc2} across all metrics. Even when the number of correct correspondences is extremely low (0$\sim$100), our method delivers excellent performance, thanks to its strong capability for generating new correspondences, as reflected in the INR and IN metrics. Furthermore, 
as the number of inliers decreases, feature matching becomes increasingly challenging, leading to a gradual decline in the RR of our method. However, the rate of decline for our method is significantly slower than that of SC2-PCR~\cite{chen2022sc2}, indicating a certain degree of robustness in handling scenarios with very few inliers. However, when IN is less than 20, our method also cannot achieve robust registration.

\subsection{Detailed Ablation Study}

\begin{figure}[ht]
    \centering{\includegraphics[width=1.0\linewidth]{iteration.pdf}}  %0.32
    \caption{
        Ablation study on the progressive iteration.
       }
    \label{iteration}
\end{figure}


\ptitle{Progressive Iteration.}
In Section~\ref{sec.Ablation}, we conduct an ablation study to compare progressive and one-stage approaches. To further analyze the effect of iterations, we vary the number of iterations from 0 to 10 and evaluate the performance on the 3DMatch dataset. To balance registration success and speed, our study reports two metrics: registration recall (RR) and running time. The results, presented in Figure~\ref{iteration}, show that RR increases significantly as the number of iterations rises from 1 to 4. Beyond 4th iterations, the improvement slows, eventually saturating, with occasional fluctuations. Meanwhile, runtime increases steadily with more iterations. To balance performance and efficiency, we ultimately select 4 iterations as the optimal configuration.



\begin{table}[htbp]
    % \setlength{\tabcolsep}{3.5pt}
    \centering
    % \setlength{\tabcolsep}{2pt}
    % \scriptsize
    \resizebox{1.0\linewidth}{!}{
    \begin{tabular}{l|ccc|ccc}
    \toprule
    &\multicolumn{3}{c|}{3DMatch} & \multicolumn{3}{c}{3DMatch-EOR} \\
    Methods        & RR  & IP& IN& RR & IP & IN \\
    \midrule
    1) K nearest neighbor  &88.32 &{82.55} &2522.09  &26.70 &21.40 &\textbf{670.23}  \\
    2) Radius nearest neighbor*  &\textbf{88.48} &82.68 &\textbf{2532.40} &\textbf{26.76} &\textbf{21.54} &{666.94}\\
    \midrule
    3) Spectral technique~\cite{leordeanu2005spectral,chen2022sc2}  &88.36 &\textbf{84.77} &2501.41  &26.19 &\textbf{24.33} &630.92\\
    4) Sampling with  consistency score   &88.06 &84.09 & 2510.91   &20.81 &22.19 &578.09 \\
    5) Random sampling*    &\textbf{88.48} &82.68 &\textbf{2532.40} &\textbf{26.76} &{21.54} &\textbf{666.94}\\
    \bottomrule
    \end{tabular}
    }
    % \vspace{-5pt}
    % \setlength{\belowcaptionskip}{-6pt}
    \caption{Ablation study on prior-guided local grouping. }
    \label{grouping}
\end{table}

\ptitle{Prior-guided Local Grouping.}
We conduct a more detailed ablation study on the prior-guided local grouping module, focusing on two aspects: sampling strategies and neighbor grouping strategies. For neighbor grouping, we evaluate two approaches: K-nearest neighbors (KNN) and radius-based neighbors (RNN). For sampling, we test three methods: spectral technique~\cite{leordeanu2005spectral}, sampling with consistency scores, and random sampling.
As shown in Table~\ref{grouping}, the RNN strategy outperforms KNN. 
This is because RNN defines local region sizes independently of the point cloud's sparsity, while KNN is highly affected by variations in point cloud density. Surprisingly, as seen in Rows 3, 4, and 5 of Table~\ref{grouping}, random sampling outperforms the other strategies, achieving the highest RR, albeit with relatively lower IP. This is because random sampling promotes a more uniform distribution of local regions across the point cloud, reducing the risk of getting trapped in local optima and improving robustness. Meanwhile, ablated models focus local regions on areas with higher-quality correspondences as determined by our method, resulting in higher IP.


% \begin{table*}[htbp]
%     % \setlength{\tabcolsep}{3.5pt}
%     \centering
%     % \setlength{\tabcolsep}{2pt}
%     % \scriptsize
%     \resizebox{1.0\linewidth}{!}{
%     \begin{tabular}{l|cccccc|ccccc|ccccc}
%     \toprule
%     & \multicolumn{6}{c|}{Hyperparameter} &\multicolumn{5}{c|}{3DMatch} & \multicolumn{5}{c}{3DMatch-EOR} \\
%     No.  &$k_0$ &$r_0$ &$s_0$ &$\omega_k$ &$\omega_r$ &$\omega_s$       & RR & RE &TE & IP& IN& RR & RE &TE &  IP & IN \\
%     \midrule
%     1)         &100 &0.8 &2 &2 & &   \\
%     2)         & & & & & &  \\
%     \bottomrule
%     \end{tabular}
%     }
%     % \vspace{-5pt}
%     % \setlength{\belowcaptionskip}{-6pt}
%     \caption{Ablation study with different hyperparameters. }
%     \label{hyperparameters}
% \end{table*}

\begin{table}[htbp]
    % \setlength{\tabcolsep}{3.5pt}
    \centering
    % \setlength{\tabcolsep}{2pt}
    % \scriptsize
    \resizebox{1.0\linewidth}{!}{
    \begin{tabular}{l|ccc|ccc}
    \toprule
    &\multicolumn{3}{c|}{3DMatch} & \multicolumn{3}{c}{3DMatch-EOR} \\
    Models        & RR  & IP& IN& RR & IP & IN \\
    \midrule
    1) $k_0=50$, $\omega_k=5$  &88.42 &80.48 &\textbf{2601.12}  &24.20 &20.11 &\textbf{681.23}  \\
    2) $k_0=20$, $\omega_k=10$  &\textbf{88.48} &81.82 &{2580.45} &{26.60} &{20.90} &{678.14}\\
    3) $k_0=20$, $\omega_k=5$*  &\textbf{88.48} &82.68 &{2532.40} &\textbf{26.76} &\textbf{21.54} &{666.94}\\
    \midrule
    4) $r_0=2$, $\omega_r=0.5$ &88.42 &84.09 & \textbf{2598.01}   &26.44 &20.91 &\textbf{702.29} \\
    5) $r_0=0.5$, $\omega_r=0.5$    &87.21&80.92 & 2423.12   &22.13 &21.49 &590.77 \\
    6) $r_0=1$, $\omega_r=0.2$   &87.43 &82.56 & 2499.86   &22.81 &21.88 &553.51 \\
    7) $r_0=1$, $\omega_r=0.5$*    &\textbf{88.48} &82.68 &{2532.40} &\textbf{26.76} &\textbf{21.54} &{666.94}\\
    \bottomrule
    \end{tabular}
    }
    % \vspace{-5pt}
    % \setlength{\belowcaptionskip}{-6pt}
    \caption{Ablation study with different hyperparameters. }
    \label{Ablation:hyperparameters}
\end{table}



\ptitle{Hyperparameter Setting.}
We conduct an ablation study on hyperparameter settings using the 3DMatch dataset. To simplify hyperparameter tuning, we design the required sampling number $s_t$, the local region radius $r_t$, and the number of points in the local region $k_t$ to follow a geometric progression. This approach allows all parameters to be determined by setting only the initial values $s_0$, $r_0$, $k_0$, and the common ratios $w_s$, $w_r$, $w_k$. The results of the ablation study are presented in Table~\ref{Ablation:hyperparameters}. As shown in Rows 2, 3, 4, and 7 of Table~\ref{Ablation:hyperparameters}, increasing $k_0$ and $r_0$ leads to a higher number of inliers (IN), but does not necessarily result in a higher RR. Balancing accuracy and efficiency, we determine $k_0=20$ and $\omega_k=5$ based on Rows 1, 2, and 3. Similarly, Rows 4, 5, 6, and 7 indicated that $r_0=1$ and $\omega_r=0.5$ are the optimal radius hyperparameters.


\subsection{Additional Qualitative Results}
As demonstrated in Sec.~\ref{Indoor1} and \ref{Outdoor}, our Regor achieves superior quantitative performance on the 3DMatch and KITTI datasets. In this section, we provide additional visualizations to illustrate our qualitative results.

\begin{figure*}[ht]
    \centering{\includegraphics[width=1.0\linewidth]{qr1.pdf}}  %0.32
    \caption{
        Correspondence results on 3DMatch.
       }
    \label{qr1}
\end{figure*}

\begin{figure*}[ht]
    \centering{\includegraphics[width=0.95\linewidth]{qr2.pdf}}  %0.32
    \caption{
        Registration results on 3DMatch.
       }
    \label{qr2}
\end{figure*}

\ptitle{Additional Qualitative Results on 3DMatch.}
On the 3DMatch dataset, we compare the qualitative results of our method with those of SC2-PCR~\cite{chen2022sc2} and PointDSC~\cite{PointDSC}. The visualization of the correspondences is shown in Figure~\ref{qr1}. As illustrated, when the initial correspondences are of poor quality, outlier removal methods~\cite{fischler1981random,chen2022sc2,PointDSC} struggle to establish high-quality correspondences, which significantly hinders accurate registration. In contrast, our Regor progressively refines correspondences, generating dense and reliable inliers. This provides a robust foundation for accurate 6DoF pose estimation and demonstrates the adaptability of our approach to scenarios with extremely high outlier rates.

Figure~\ref{qr2} presents the qualitative results and registration visualizations. In some scenarios, SC2-PCR~\cite{chen2022sc2} and PointDSC~\cite{PointDSC} fail to achieve robust registration due to the limitations in correspondence quality. Compared to outlier removal methods~\cite{fischler1981random,chen2022sc2,PointDSC}, our method not only achieves robust registration but also yields lower pose errors.

\begin{figure*}[ht]
    \centering{\includegraphics[width=1.0\linewidth]{qr3.pdf}}  %0.32
    \caption{
        Qualitative results on KITTI odometry.
       }
    \label{qr3}
\end{figure*}

\ptitle{Additional Qualitative Results on KITTI.}
On the KITTI dataset, we use FPFH~\cite{FPFH} for feature extraction and the results of the qualitative experiments are shown in Figure~\ref{qr3}. Our method achieves accurate and robust registration due to high-quality dense correspondences.

% {\small
% \bibliographystyle{ieeenat_fullname}
% \bibliography{cvpr24}
% }