\section{Related Work}
The most common setting in fair division is equally-entitled agents
having additive valuations over goods. For this setting, ____
studied implications among 5 fairness notions (CEEI, EF, PROP, MMS, min-max-share).
____ consider implications between EF, PROP, EF1, and PROP1 for mixed manna instead.
Over time, as new fairness notions were proposed
____,
their connections with other well-established notions were studied.
However, the above works only consider a limited number of fairness notions and fair division settings.
Our work, on the other hand, aims to be exhaustive, and thus, have broader applicability.

For the popular setting of equally-entitled agents having additive valuations over goods,
many implications were already known. However, it wasn't clear if this was the final picture,
since many of the counterexamples were not known, a gap that we have now filled.
For less common settings, e.g., chores, mixed manna, or unequal entitlements, much less was known,
so our work makes a significant advancement towards understanding the fair division landscape.