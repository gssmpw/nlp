
\begin{table*}[t]
\normalsize
\centering
\renewcommand{\arraystretch}{1.6} 
\setlength{\tabcolsep}{6pt} 
\resizebox{\textwidth}{!}{%
\begin{tabular}{cc|p{0.3\textwidth} p{0.3\textwidth} p{0.3\textwidth}}
\toprule
\multicolumn{2}{c}{} & \multicolumn{3}{c}{\textbf{\Large\textsc{Nature of Attention}}} \\ 
\multicolumn{2}{c}{} & \multicolumn{1}{c}{\large\textbf{Content}} & \multicolumn{1}{c}{\large\textbf{Relationship-Building}} & \multicolumn{1}{c}{\large\textbf{Management}} \\ 
\midrule
\multirow{4}{*}{\rotatebox[origin=c]{90}{\Large\textbf{\textsc{\parbox[c]{5cm}{\centering Recipient of \\ Attention}}}}} 
& \large\textbf{Only Student A} & "Okay, [Student A], can you use this word in a sentence, please?" & "Okay, [Student A], tell me one fun thing from the weekend." & "Oh, [Student A], I can't see you." \\ 
& \large\textbf{Only Student B} & "Okay, [Student B], what's the middle sound?" & "I know [Student B] likes to read." & "And [Student B], can you hit mute?" \\ 
& \large\textbf{Both Students} & "Ooh, mushroom does have the M sound." & "You guys are awesome." & "Let's keep our listening ears on and our focus, and let's do this page." \\ 
& \large\textbf{One of the Students} & "Ball." & "Today you're six years old?" & "Okay, keep your earphones on, stop moving the computer screen." \\ 
\bottomrule
\end{tabular}
}
\caption{
\textbf{A sample of utterances illustrating our two-dimensional Attention Framework.} 
Our framework classifies a teacher utterance based on both the \textsc{Nature of Attention} (content, relationship-building, or management) and the \textsc{Recipient of Attention}. 
The classification of each utterance relies on the broader conversational context (not shown in the table). 
While some utterances can be clearly categorized, many require careful consideration of context for accurate classification. 
\label{tab:framework}}
\end{table*}