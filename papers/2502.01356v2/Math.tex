\section{Mathematical Background}
\subsection{Quasi-Conformal Geometry}

\begin{definition}[Quasi-conformal map]
A quasi-conformal map is a map $f: \mathbb{C} \rightarrow \mathbb{C}$ that satisfies the Beltrami equation
\begin{equation}
\frac{\partial f}{\partial \bar{z}}=\mu(z) \frac{\partial f}{\partial z}
\label{eq:beleq}
\end{equation}
for some complex-valued function named as Beltrami coefficient $\mu$ satisfying $\|\mu\|_{\infty}<1$ and $\frac{\partial f}{\partial z}$ is non-vanishing almost everywhere. The complex partial derivatives are given by
\begin{equation}
\frac{\partial f}{\partial z}:=\frac{1}{2}\left(\frac{\partial f}{\partial x}-i \frac{\partial f}{\partial y}\right) 
\quad \text{ and } \quad 
\frac{\partial f}{\partial \bar{z}}:=\frac{1}{2}\left(\frac{\partial f}{\partial x}+i \frac{\partial f}{\partial y}\right).
\end{equation}

\begin{figure}
    \centering
    \includegraphics[width=0.5\textwidth]{im/conformality.PNG}
    \caption{Illustration of how the Beltrami coefficient measures the conformality distortion of a quasi-conformal map}
    \label{fig:qcmap}
\end{figure}

\end{definition}
$\mu$ is the Beltrami representation, which is also called the Beltrami coefficient, of the quasi-conformal map $f$. It's worthy to mention that $\mu$ is a measure of non-conformality. Particularly, for a point $p$, the associated quasi-conformal map $f$ is conformal around a small neighbourhood of $p$ if $\mu(p)=0$. In this case, Equation \ref{eq:beleq} becomes the Cauchy-Riemann equation. This can also illustrate that conformality analysis of a quasi-conformal map $f$ can be simplified into the analysis of its associated Beltrami coefficient $\mu$. Infinitesimally, such a map $f$ can be rewritten as follows in a local neighbourhood around a point $p$:
\begin{equation}
\begin{aligned}
f(z) &=f(p)+f_{z}(p) z+f_{\bar{z}}(p) \bar{z} \\
&=f(p)+f_{z}(p)(z+\mu(p) \bar{z}).
\end{aligned}
\end{equation}
This further enhances our discussion before that $f$ is conformal when $\mu(p) = 0$. To explain the equation above, $f(p)$ is a translation, while $f_z(p)$ is a dilation. Since both of them are conformal, all the non-conformality of $f$ is brought by $D(z)=z+\mu(p) \bar{z}$. Hence, the Beltrami coefficient $\mu$ actually encodes the conformality of $f$. Analyzing quasi-conformal $f$ is equivalent to that for its associated Beltrami coefficient $\mu$. To be detail, the angle of maximal magnification is $\arg (\mu(p)) / 2$ with magnifying factor $1 + |\mu(p)|$; the maximal shrinking is the orthogonal angle $(\arg (\mu(p))-\pi) / 2$ with shrinking factor $1 - |\mu(p)|$. 

The maximal quasi-conformal dilation of $f$ is given by
\begin{equation}
K=\frac{1+\|\mu\|_{\infty}}{1-\|\mu\|_{\infty}}.
\end{equation}
Figure \ref{fig:qcmap} illustrates the geometry of a quasi-conformal map.

Another important relationship between a map and its Beltrami coefficients is the diffeomorphism property. By a norm constraint on $\mu$, the bijectivity of $f$ can be preserved which is explained by the following theory.

\begin{theorem}
If $f: \mathbb{C} \rightarrow \mathbb{C}$ is a $C^{1}$ map. Define 
\begin{equation}
\mu=\frac{\partial f}{\partial \bar{z}} / \frac{\partial f}{\partial z}.
\end{equation}
If $\mu$ satisfies $\left\|\mu_{f}\right\|_{\infty}<1$, then $f$ is bijective.
\end{theorem}

\begin{theorem}[Measurable Riemann mapping theorem \cite{gardiner2000quasiconformal}]
\label{them:RiemannMapping}
Suppose $\mu: \mathbb{C}\rightarrow\mathbb{C}$ is Lebesgue measurable satisfying $\|\mu\|_{\infty}<1$, then there exists a quasi-conformal mapping $f:\mathbb{C}\rightarrow \mathbb{C}$ in the Sobolev space $W^{1,2}$ that satisfies the Beltrami equation in the distribution sense. Furthermore, assuming that the mapping is stationary at $0, 1$ and $\infty$, then the associated quasi-conformal mapping $f$ is uniquely determined.
\end{theorem}

\begin{figure}
    \centering
    \includegraphics[width=0.8\textwidth]{im/beltramidifferential.pdf}
    \caption{Illustration of quasi-conformal mapping between Riemann surfaces.}
    \label{fig:bcdifferential}
\end{figure}

The Beltrami coefficient of a composition of quasi-conformal maps is related to the Beltrami coefficients of the original maps. Suppose $f: \Omega \rightarrow f(\Omega)$ and $g: f(\Omega) \rightarrow \mathbb{C}$ are two quasi-conformal maps with Beltrami coefficients $\mu_f$ and $\mu_g$ correspondingly. The Beltrami coefficient of the composition map $g \circ f$ is given by
$$
\mu_{g \circ f}=\frac{\mu_f+\frac{\overline{f_z}}{f_z}\left(\mu_g \circ f\right)}{1+\frac{\overline{f_z}}{f_z} \overline{\mu_f}\left(\mu_g \circ f\right)} .
$$

Quasi-conformal maps can also be defined between two Riemann surfaces. In this case, the Beltrami differential is used. A Beltrami differential $\mu(z) \frac{d\overline{z}}{d z}$ on a Riemann surface $S$ is an assignment to each chart ($U_\alpha, \phi_\alpha$) of an $L_{\infty}$ complex-valued function $\mu_\alpha$, defined on local parameter $z_\alpha$ such that
$$
\mu_\alpha \frac{d \overline{z_\alpha}}{d z_\alpha}=\mu_\beta \frac{d \overline{z_\beta}}{d z_\beta},
$$
on the domain which is also covered by another chart $\left(U_\beta, \phi_\beta\right)$. Here, $\frac{d z_\beta}{d z_\alpha}=\frac{d}{d z_\alpha} \phi_{\alpha \beta}$ and $\phi_{\alpha \beta}=\phi_\beta \circ \phi_\alpha$. An orientation preserving diffeomorphism $f: M \rightarrow N$ is called quasi-conformal associated with $\mu(z) \frac{d z}{d z}$ if for any chart ($U_\alpha, \phi_\alpha$) on $M$ and any chart $\left(U_\beta, \psi_\beta\right)$ on $N$, the mapping $f_{\alpha \beta}:=\psi_\beta \circ f \circ f_\alpha^{-1}$ is quasi-conformal associated with $\mu_\alpha \frac{d \overline{z}}{d z_\alpha}$. Readers are referred to \cite{gardiner2000quasiconformal,lehto1973quasiconformal} for more details about quasi-conformal theories.