\documentclass[onefignum,onetabnum]{siamonline171218}

\usepackage[utf8]{inputenc} % allow utf-8 input
\usepackage[T1]{fontenc}    % use 8-bit T1 fonts
\usepackage{color}
\usepackage{hyperref}       % hyperlinks
\usepackage{url}            % simple URL typesetting
\usepackage{makecell}
\usepackage{amsmath,amsfonts,amssymb}
\usepackage{mathtools}
\usepackage{pifont}
\usepackage{booktabs}       % professional-quality tables
\usepackage{amsfonts}       % blackboard math symbols
\usepackage{nicefrac}       % compact symbols for 1/2, etc.
\usepackage{microtype}      % microtypography
\usepackage{lipsum,bm}
\usepackage{graphicx}
\usepackage{ragged2e}
\usepackage{multirow}
\usepackage[labelfont=bf]{caption}
\usepackage{enumitem}
% \usepackage{bm}
\setlist[enumerate]{leftmargin=.5in}
\setlist[itemize]{leftmargin=.5in}
\setlength{\textfloatsep}{3pt}
\graphicspath{ {./images/} }
\def\etal{\textit{et al. }}
\newcommand{\han}[1]{{\color{red}{#1}}}
\newcommand{\enoch}[1]{{\color{blue}{#1}}}
\newcommand{\creflastconjunction}{, and~}
\newsiamremark{remark}{Remark}
\newsiamremark{hypothesis}{Hypothesis}
\crefname{hypothesis}{Hypothesis}{Hypotheses}
\newsiamthm{claim}{Claim}
\usepackage{bm}

\headers{Quasi-Conformal Convolution}{H. Zhang, T.L. Ip and L.M. Lui}

\title{Quasi-Conformal Convolution : A Learnable Convolution for Deep Learning on Riemann Surfaces
\thanks{Submitted to the editors DATE.
% \funding{This work was funded by the Fog Research Institute under contract no.~FRI-454.}
}}


\author{Han Zhang \thanks{Department of Mathematics, City University of Hong Kong, Hong Kong, China. (hzhang863-c@my.cityu.edu.hk)
}
\and Tsz Lok Ip \thanks{Department of Mathematics, Chinese University of Hong Kong, Hong Kong, China;(enochitl@link.cuhk.edu.hk)
}
\and Lok Ming Lui \thanks{Department of Mathematics, Chinese University of Hong Kong, Hong Kong, China;(lmlui@math.cuhk.edu.hk)
}}

\begin{document}
\maketitle
Since 2020, GitGuardian has been detecting checked-in hard-coded secrets in GitHub repositories. During 2020-2023, GitGuardian has observed an upward annual trend and a four-fold increase in hard-coded secrets, with 12.8 million exposed in 2023. However, removing all the secrets from software artifacts is not feasible due to time constraints and technical challenges. Additionally, the security risks of the secrets are not equal, protecting assets ranging from obsolete databases to sensitive medical data. Thus, secret removal should be prioritized by security risk reduction, which existing secret detection tools do not support. \textit{The goal of this research is to aid software practitioners in prioritizing secrets removal efforts through our security risk-based tool}. We present RiskHarvester, a risk-based tool to compute a security risk score based on the value of the asset and ease of attack on a database. We calculated the value of asset by identifying the sensitive data categories present in a database from the database keywords in the source code. We utilized data flow analysis, SQL, and Object Relational Mapper (ORM) parsing to identify the database keywords. To calculate the ease of attack, we utilized passive network analysis to retrieve the database host information. To evaluate RiskHarvester, we curated RiskBench, a benchmark of 1,791 database secret-asset pairs with sensitive data categories and host information manually retrieved from 188 GitHub repositories. RiskHarvester demonstrates precision of (95\%) and recall (90\%) in detecting database keywords for the value of asset and precision of (96\%) and recall (94\%) in detecting valid hosts for ease of attack. Finally, we conducted a survey (52 respondents) to understand whether developers prioritize secret removal based on security risk score. We found that 86\% of the developers prioritized the secrets for removal with descending security risk scores.
\section{Introduction}

Large language models (LLMs) have achieved remarkable success in automated math problem solving, particularly through code-generation capabilities integrated with proof assistants~\citep{lean,isabelle,POT,autoformalization,MATH}. Although LLMs excel at generating solution steps and correct answers in algebra and calculus~\citep{math_solving}, their unimodal nature limits performance in plane geometry, where solution depends on both diagram and text~\citep{math_solving}. 

Specialized vision-language models (VLMs) have accordingly been developed for plane geometry problem solving (PGPS)~\citep{geoqa,unigeo,intergps,pgps,GOLD,LANS,geox}. Yet, it remains unclear whether these models genuinely leverage diagrams or rely almost exclusively on textual features. This ambiguity arises because existing PGPS datasets typically embed sufficient geometric details within problem statements, potentially making the vision encoder unnecessary~\citep{GOLD}. \cref{fig:pgps_examples} illustrates example questions from GeoQA and PGPS9K, where solutions can be derived without referencing the diagrams.

\begin{figure}
    \centering
    \begin{subfigure}[t]{.49\linewidth}
        \centering
        \includegraphics[width=\linewidth]{latex/figures/images/geoqa_example.pdf}
        \caption{GeoQA}
        \label{fig:geoqa_example}
    \end{subfigure}
    \begin{subfigure}[t]{.48\linewidth}
        \centering
        \includegraphics[width=\linewidth]{latex/figures/images/pgps_example.pdf}
        \caption{PGPS9K}
        \label{fig:pgps9k_example}
    \end{subfigure}
    \caption{
    Examples of diagram-caption pairs and their solution steps written in formal languages from GeoQA and PGPS9k datasets. In the problem description, the visual geometric premises and numerical variables are highlighted in green and red, respectively. A significant difference in the style of the diagram and formal language can be observable. %, along with the differences in formal languages supported by the corresponding datasets.
    \label{fig:pgps_examples}
    }
\end{figure}



We propose a new benchmark created via a synthetic data engine, which systematically evaluates the ability of VLM vision encoders to recognize geometric premises. Our empirical findings reveal that previously suggested self-supervised learning (SSL) approaches, e.g., vector quantized variataional auto-encoder (VQ-VAE)~\citep{unimath} and masked auto-encoder (MAE)~\citep{scagps,geox}, and widely adopted encoders, e.g., OpenCLIP~\citep{clip} and DinoV2~\citep{dinov2}, struggle to detect geometric features such as perpendicularity and degrees. 

To this end, we propose \geoclip{}, a model pre-trained on a large corpus of synthetic diagram–caption pairs. By varying diagram styles (e.g., color, font size, resolution, line width), \geoclip{} learns robust geometric representations and outperforms prior SSL-based methods on our benchmark. Building on \geoclip{}, we introduce a few-shot domain adaptation technique that efficiently transfers the recognition ability to real-world diagrams. We further combine this domain-adapted GeoCLIP with an LLM, forming a domain-agnostic VLM for solving PGPS tasks in MathVerse~\citep{mathverse}. 
%To accommodate diverse diagram styles and solution formats, we unify the solution program languages across multiple PGPS datasets, ensuring comprehensive evaluation. 

In our experiments on MathVerse~\citep{mathverse}, which encompasses diverse plane geometry tasks and diagram styles, our VLM with a domain-adapted \geoclip{} consistently outperforms both task-specific PGPS models and generalist VLMs. 
% In particular, it achieves higher accuracy on tasks requiring geometric-feature recognition, even when critical numerical measurements are moved from text to diagrams. 
Ablation studies confirm the effectiveness of our domain adaptation strategy, showing improvements in optical character recognition (OCR)-based tasks and robust diagram embeddings across different styles. 
% By unifying the solution program languages of existing datasets and incorporating OCR capability, we enable a single VLM, named \geovlm{}, to handle a broad class of plane geometry problems.

% Contributions
We summarize the contributions as follows:
We propose a novel benchmark for systematically assessing how well vision encoders recognize geometric premises in plane geometry diagrams~(\cref{sec:visual_feature}); We introduce \geoclip{}, a vision encoder capable of accurately detecting visual geometric premises~(\cref{sec:geoclip}), and a few-shot domain adaptation technique that efficiently transfers this capability across different diagram styles (\cref{sec:domain_adaptation});
We show that our VLM, incorporating domain-adapted GeoCLIP, surpasses existing specialized PGPS VLMs and generalist VLMs on the MathVerse benchmark~(\cref{sec:experiments}) and effectively interprets diverse diagram styles~(\cref{sec:abl}).

\iffalse
\begin{itemize}
    \item We propose a novel benchmark for systematically assessing how well vision encoders recognize geometric premises, e.g., perpendicularity and angle measures, in plane geometry diagrams.
	\item We introduce \geoclip{}, a vision encoder capable of accurately detecting visual geometric premises, and a few-shot domain adaptation technique that efficiently transfers this capability across different diagram styles.
	\item We show that our final VLM, incorporating GeoCLIP-DA, effectively interprets diverse diagram styles and achieves state-of-the-art performance on the MathVerse benchmark, surpassing existing specialized PGPS models and generalist VLM models.
\end{itemize}
\fi

\iffalse

Large language models (LLMs) have made significant strides in automated math word problem solving. In particular, their code-generation capabilities combined with proof assistants~\citep{lean,isabelle} help minimize computational errors~\citep{POT}, improve solution precision~\citep{autoformalization}, and offer rigorous feedback and evaluation~\citep{MATH}. Although LLMs excel in generating solution steps and correct answers for algebra and calculus~\citep{math_solving}, their uni-modal nature limits performance in domains like plane geometry, where both diagrams and text are vital.

Plane geometry problem solving (PGPS) tasks typically include diagrams and textual descriptions, requiring solvers to interpret premises from both sources. To facilitate automated solutions for these problems, several studies have introduced formal languages tailored for plane geometry to represent solution steps as a program with training datasets composed of diagrams, textual descriptions, and solution programs~\citep{geoqa,unigeo,intergps,pgps}. Building on these datasets, a number of PGPS specialized vision-language models (VLMs) have been developed so far~\citep{GOLD, LANS, geox}.

Most existing VLMs, however, fail to use diagrams when solving geometry problems. Well-known PGPS datasets such as GeoQA~\citep{geoqa}, UniGeo~\citep{unigeo}, and PGPS9K~\citep{pgps}, can be solved without accessing diagrams, as their problem descriptions often contain all geometric information. \cref{fig:pgps_examples} shows an example from GeoQA and PGPS9K datasets, where one can deduce the solution steps without knowing the diagrams. 
As a result, models trained on these datasets rely almost exclusively on textual information, leaving the vision encoder under-utilized~\citep{GOLD}. 
Consequently, the VLMs trained on these datasets cannot solve the plane geometry problem when necessary geometric properties or relations are excluded from the problem statement.

Some studies seek to enhance the recognition of geometric premises from a diagram by directly predicting the premises from the diagram~\citep{GOLD, intergps} or as an auxiliary task for vision encoders~\citep{geoqa,geoqa-plus}. However, these approaches remain highly domain-specific because the labels for training are difficult to obtain, thus limiting generalization across different domains. While self-supervised learning (SSL) methods that depend exclusively on geometric diagrams, e.g., vector quantized variational auto-encoder (VQ-VAE)~\citep{unimath} and masked auto-encoder (MAE)~\citep{scagps,geox}, have also been explored, the effectiveness of the SSL approaches on recognizing geometric features has not been thoroughly investigated.

We introduce a benchmark constructed with a synthetic data engine to evaluate the effectiveness of SSL approaches in recognizing geometric premises from diagrams. Our empirical results with the proposed benchmark show that the vision encoders trained with SSL methods fail to capture visual \geofeat{}s such as perpendicularity between two lines and angle measure.
Furthermore, we find that the pre-trained vision encoders often used in general-purpose VLMs, e.g., OpenCLIP~\citep{clip} and DinoV2~\citep{dinov2}, fail to recognize geometric premises from diagrams.

To improve the vision encoder for PGPS, we propose \geoclip{}, a model trained with a massive amount of diagram-caption pairs.
Since the amount of diagram-caption pairs in existing benchmarks is often limited, we develop a plane diagram generator that can randomly sample plane geometry problems with the help of existing proof assistant~\citep{alphageometry}.
To make \geoclip{} robust against different styles, we vary the visual properties of diagrams, such as color, font size, resolution, and line width.
We show that \geoclip{} performs better than the other SSL approaches and commonly used vision encoders on the newly proposed benchmark.

Another major challenge in PGPS is developing a domain-agnostic VLM capable of handling multiple PGPS benchmarks. As shown in \cref{fig:pgps_examples}, the main difficulties arise from variations in diagram styles. 
To address the issue, we propose a few-shot domain adaptation technique for \geoclip{} which transfers its visual \geofeat{} perception from the synthetic diagrams to the real-world diagrams efficiently. 

We study the efficacy of the domain adapted \geoclip{} on PGPS when equipped with the language model. To be specific, we compare the VLM with the previous PGPS models on MathVerse~\citep{mathverse}, which is designed to evaluate both the PGPS and visual \geofeat{} perception performance on various domains.
While previous PGPS models are inapplicable to certain types of MathVerse problems, we modify the prediction target and unify the solution program languages of the existing PGPS training data to make our VLM applicable to all types of MathVerse problems.
Results on MathVerse demonstrate that our VLM more effectively integrates diagrammatic information and remains robust under conditions of various diagram styles.

\begin{itemize}
    \item We propose a benchmark to measure the visual \geofeat{} recognition performance of different vision encoders.
    % \item \sh{We introduce geometric CLIP (\geoclip{} and train the VLM equipped with \geoclip{} to predict both solution steps and the numerical measurements of the problem.}
    \item We introduce \geoclip{}, a vision encoder which can accurately recognize visual \geofeat{}s and a few-shot domain adaptation technique which can transfer such ability to different domains efficiently. 
    % \item \sh{We develop our final PGPS model, \geovlm{}, by adapting \geoclip{} to different domains and training with unified languages of solution program data.}
    % We develop a domain-agnostic VLM, namely \geovlm{}, by applying a simple yet effective domain adaptation method to \geoclip{} and training on the refined training data.
    \item We demonstrate our VLM equipped with GeoCLIP-DA effectively interprets diverse diagram styles, achieving superior performance on MathVerse compared to the existing PGPS models.
\end{itemize}

\fi 

\section{Related Works}
\subsection{Computational Quasi-Conformal Mapping}
Computational quasi-conformal mapping is a powerful tool to control the geometric variation and topology in image science \cite{lam2014landmark} and surface processing \cite{levy2002least,gu2004genus}. Benefitting from the Beltrami representation, the mapping between two different domains can preserve good geometric properties like bijectivity and smoothness, through controlling the Beltrami coefficients with such representation of mappings. Driven by the motivation to preserve different geometric information, ways of parameterization methods are proposed~\cite{gu2003global}. Such convenient representations are also popular and succeed in the computational fabrication community~\cite{Soliman:2018:OCS,Crane:2013:RFC,panetta2019x}. With the capability to handle large deformations, the quasi-conformal method also succeeds in registration for images~\cite{lam2014landmark} and surfaces~\cite{choi2015fast} and segmentation with topology- and convexity prior~\cite{zhang2021topology,zhang2024qis}. In \cite{zhang2022nondeterministic,zhang2022new}, quasi-conformality is used for deformation analysis with uncertainties to study medical images for disease analysis. 


\subsection{Deformable Convolution}

Deformable convolution has been proposed as a solution to the limitations of the traditional convolution operation in Convolutional Neural Networks (CNNs). Jeon \etal proposed the Active Convolution (AC) \cite{jeon2017active}, which integrates a trainable attention mechanism into the convolution operation to adaptively select informative features for different input instances. Another related approach is the Spatial Transformer Network (STN) \cite{jaderberg2015spatial}, which introduces a learnable transformation module that can warp the input feature map based on a set of learnable parameters. Zhang \etal \cite{zhang2024learning,zhang2023deformation} extend it with a Relu-Jacobian regularization to make the produced mapping bijective. By introducing an explicit spatial transformation module, the STN allows the network to learn spatial transformations that can better align the input with the task at hand, leading to improved performance in tasks such as digit recognition and image classification. 

Building on the STN, Dai \etal proposed the Deformable Convolution (DCN) \cite{dai2017deformable}, which extends the idea of spatial transformation to the convolution operation itself, by introducing learnable offsets for each position in the convolutional kernel. This allows the DCN to dynamically adjust the sampling locations of the convolution kernel for each input instance, leading to improved performance on tasks such as object detection and semantic segmentation. However, the original DCN has limitations in handling large deformations and invariance to occlusion. To address these limitations, researchers have proposed several variations, such as the Deformable Convolution v2 (DCNv2) \cite{zhu2019deformable}, which introduces additional deformable offsets for the intermediate feature maps, and the Deformable RoI Pooling (DRoIPool) \cite{dai2017deformable}, which extends the DCN to the task of region-based object detection. However, Luo \etal found that the contribution of each pixel is not equal to the final results in DCN \cite{luo2016understanding}. These findings suggest the need for further improvements in the deformable convolution operation to address its limitations and maximize its performance.

\subsection{Geometric Learning}
In the field of geometric modelling, Bronstein \etal introduced manifold convolution with geodesic patch operators, demonstrating its success in various applications~\cite{bronstein2017geometric, masci2015geodesic}. Similarly, Boscaini \etal utilized an anisotropic heat kernel to define the convolution window, further contributing to the field~\cite{boscaini2016learning}. Other convolution definitions have also succeeded in registration tasks~\cite{bouritsas2019neural, gong2019spiralnet++}. Additionally, the MeshCNN framework by Hanocka \etal is noteworthy, as it redefined convolution using edges rather than vertices, offering a natural and straightforward approach to the concept~\cite{hanocka2019meshcnn}. Schonsheck \etal propose \cite{schonsheck2022parallel} Parallel Transport Convolution to enhance the translation invariance and allow the construction of compactly supported filters in manifold neural networks.
\section{Mathematical Background}
\subsection{Quasi-Conformal Geometry}

\begin{definition}[Quasi-conformal map]
A quasi-conformal map is a map $f: \mathbb{C} \rightarrow \mathbb{C}$ that satisfies the Beltrami equation
\begin{equation}
\frac{\partial f}{\partial \bar{z}}=\mu(z) \frac{\partial f}{\partial z}
\label{eq:beleq}
\end{equation}
for some complex-valued function named as Beltrami coefficient $\mu$ satisfying $\|\mu\|_{\infty}<1$ and $\frac{\partial f}{\partial z}$ is non-vanishing almost everywhere. The complex partial derivatives are given by
\begin{equation}
\frac{\partial f}{\partial z}:=\frac{1}{2}\left(\frac{\partial f}{\partial x}-i \frac{\partial f}{\partial y}\right) 
\quad \text{ and } \quad 
\frac{\partial f}{\partial \bar{z}}:=\frac{1}{2}\left(\frac{\partial f}{\partial x}+i \frac{\partial f}{\partial y}\right).
\end{equation}

\begin{figure}
    \centering
    \includegraphics[width=0.5\textwidth]{im/conformality.PNG}
    \caption{Illustration of how the Beltrami coefficient measures the conformality distortion of a quasi-conformal map}
    \label{fig:qcmap}
\end{figure}

\end{definition}
$\mu$ is the Beltrami representation, which is also called the Beltrami coefficient, of the quasi-conformal map $f$. It's worthy to mention that $\mu$ is a measure of non-conformality. Particularly, for a point $p$, the associated quasi-conformal map $f$ is conformal around a small neighbourhood of $p$ if $\mu(p)=0$. In this case, Equation \ref{eq:beleq} becomes the Cauchy-Riemann equation. This can also illustrate that conformality analysis of a quasi-conformal map $f$ can be simplified into the analysis of its associated Beltrami coefficient $\mu$. Infinitesimally, such a map $f$ can be rewritten as follows in a local neighbourhood around a point $p$:
\begin{equation}
\begin{aligned}
f(z) &=f(p)+f_{z}(p) z+f_{\bar{z}}(p) \bar{z} \\
&=f(p)+f_{z}(p)(z+\mu(p) \bar{z}).
\end{aligned}
\end{equation}
This further enhances our discussion before that $f$ is conformal when $\mu(p) = 0$. To explain the equation above, $f(p)$ is a translation, while $f_z(p)$ is a dilation. Since both of them are conformal, all the non-conformality of $f$ is brought by $D(z)=z+\mu(p) \bar{z}$. Hence, the Beltrami coefficient $\mu$ actually encodes the conformality of $f$. Analyzing quasi-conformal $f$ is equivalent to that for its associated Beltrami coefficient $\mu$. To be detail, the angle of maximal magnification is $\arg (\mu(p)) / 2$ with magnifying factor $1 + |\mu(p)|$; the maximal shrinking is the orthogonal angle $(\arg (\mu(p))-\pi) / 2$ with shrinking factor $1 - |\mu(p)|$. 

The maximal quasi-conformal dilation of $f$ is given by
\begin{equation}
K=\frac{1+\|\mu\|_{\infty}}{1-\|\mu\|_{\infty}}.
\end{equation}
Figure \ref{fig:qcmap} illustrates the geometry of a quasi-conformal map.

Another important relationship between a map and its Beltrami coefficients is the diffeomorphism property. By a norm constraint on $\mu$, the bijectivity of $f$ can be preserved which is explained by the following theory.

\begin{theorem}
If $f: \mathbb{C} \rightarrow \mathbb{C}$ is a $C^{1}$ map. Define 
\begin{equation}
\mu=\frac{\partial f}{\partial \bar{z}} / \frac{\partial f}{\partial z}.
\end{equation}
If $\mu$ satisfies $\left\|\mu_{f}\right\|_{\infty}<1$, then $f$ is bijective.
\end{theorem}

\begin{theorem}[Measurable Riemann mapping theorem \cite{gardiner2000quasiconformal}]
\label{them:RiemannMapping}
Suppose $\mu: \mathbb{C}\rightarrow\mathbb{C}$ is Lebesgue measurable satisfying $\|\mu\|_{\infty}<1$, then there exists a quasi-conformal mapping $f:\mathbb{C}\rightarrow \mathbb{C}$ in the Sobolev space $W^{1,2}$ that satisfies the Beltrami equation in the distribution sense. Furthermore, assuming that the mapping is stationary at $0, 1$ and $\infty$, then the associated quasi-conformal mapping $f$ is uniquely determined.
\end{theorem}

\begin{figure}
    \centering
    \includegraphics[width=0.8\textwidth]{im/beltramidifferential.pdf}
    \caption{Illustration of quasi-conformal mapping between Riemann surfaces.}
    \label{fig:bcdifferential}
\end{figure}

The Beltrami coefficient of a composition of quasi-conformal maps is related to the Beltrami coefficients of the original maps. Suppose $f: \Omega \rightarrow f(\Omega)$ and $g: f(\Omega) \rightarrow \mathbb{C}$ are two quasi-conformal maps with Beltrami coefficients $\mu_f$ and $\mu_g$ correspondingly. The Beltrami coefficient of the composition map $g \circ f$ is given by
$$
\mu_{g \circ f}=\frac{\mu_f+\frac{\overline{f_z}}{f_z}\left(\mu_g \circ f\right)}{1+\frac{\overline{f_z}}{f_z} \overline{\mu_f}\left(\mu_g \circ f\right)} .
$$

Quasi-conformal maps can also be defined between two Riemann surfaces. In this case, the Beltrami differential is used. A Beltrami differential $\mu(z) \frac{d\overline{z}}{d z}$ on a Riemann surface $S$ is an assignment to each chart ($U_\alpha, \phi_\alpha$) of an $L_{\infty}$ complex-valued function $\mu_\alpha$, defined on local parameter $z_\alpha$ such that
$$
\mu_\alpha \frac{d \overline{z_\alpha}}{d z_\alpha}=\mu_\beta \frac{d \overline{z_\beta}}{d z_\beta},
$$
on the domain which is also covered by another chart $\left(U_\beta, \phi_\beta\right)$. Here, $\frac{d z_\beta}{d z_\alpha}=\frac{d}{d z_\alpha} \phi_{\alpha \beta}$ and $\phi_{\alpha \beta}=\phi_\beta \circ \phi_\alpha$. An orientation preserving diffeomorphism $f: M \rightarrow N$ is called quasi-conformal associated with $\mu(z) \frac{d z}{d z}$ if for any chart ($U_\alpha, \phi_\alpha$) on $M$ and any chart $\left(U_\beta, \psi_\beta\right)$ on $N$, the mapping $f_{\alpha \beta}:=\psi_\beta \circ f \circ f_\alpha^{-1}$ is quasi-conformal associated with $\mu_\alpha \frac{d \overline{z}}{d z_\alpha}$. Readers are referred to \cite{gardiner2000quasiconformal,lehto1973quasiconformal} for more details about quasi-conformal theories.
\section{Adaptive Convolution on Riemannian Manifolds}

Convolution is a fundamental mathematical operator in mathematics, physics, and engineering. It combines two functions to illustrate how the characteristics of one function are modified by the other. The necessity of defining convolution stems from its applications in various contexts. In this section, we will define convolution on Riemann surfaces, starting with an investigation of convolution on general manifolds and progressing to the definition of convolution on simply connected surfaces through Quasi-conformal Convolution (QCC), which is applicable to many real-world situations.

\subsection{Convolution on Riemannian $n$-manifold}

\subsubsection{Convolution on Manifold}
Before providing a definition of convolution on manifolds, we first examine the standard convolution operation in Euclidean space, as outlined in the following definition.
% In an Euclidean space, the convolution operation can be given as the following definition.

\begin{definition}[Convolution]
For two functions $h, k : \mathbb{R}^n \to \mathbb{R}$, the convolution of $h$ and $k$ is defined as:
\begin{equation}
(h \ast k)(x) = \int_{\mathbb{R}^n} h(y) k(x - y) \, dy,
\end{equation}
where:
\begin{itemize}
    \item $x \in \mathbb{R}^n$ is the point at which the convolution is evaluated,
    \item $y \in \mathbb{R}^n$ is the integration variable,
    \item $k(x - y)$ translates $k$ to align it with $h$.
\end{itemize}
\end{definition}

Defining convolution on manifolds or surfaces is more complex than in Euclidean space due to the absence of a global linear structure on manifolds. In the Euclidean case, convolution involves translating the kernel function $k$ using the displacement vector $x - y$. However, this approach does not directly apply to manifolds. A suitable notion of displacement must be established before performing convolution on manifolds.
%Defining convolution on manifolds or surfaces is more complex than in Euclidean space due to the absence of a global linear structure on manifolds.  In the Euclidean case, convolution involves translating the function $k$ by $x-y$, a shift-invariant operation that does not directly apply to manifolds. Consequently, convolution on an $n$-manifold $\mathcal{M}$ can only be defined between a function on $\mathcal{M}$ and a kernel function on $\mathbb{R}^n$. Since the displacement vector $x-y$ is not well-defined when $x$ and $y$ are points on the manifold, it becomes essential to establish a suitable notion of displacement before performing convolution on manifolds.

\begin{definition}[Displacement function and displacement vector]
    Let $\mathcal{M}$ be a Riemannian $n$-manifold, and $U \subseteq \mathcal{M}$ be a subset of $\mathcal{M}$. A function $d: U\times U \to \mathbb{R}^n$ is a displacement function on $U$ if it satisfies the following properties:
    \begin{enumerate}
        \item For all $p, q \in U$, $d(p,q) = 0$ if and only if $p=q$.
        \item For all $p, q, r \in U$, $d(p,r) = d(p,q) + d(q,r)$.
    \end{enumerate}
    Then, the vector $d(p, q)$ is referred to as the displacement vector from $p$ to $q$.
    
    Moreover, if the functions $d(\cdot, q_0)$ and $d(p_0, \cdot)$ are orientation-preserving homeomorphisms from $U$ to subsets of $\mathbb{R}^n$ depending on $p_0$ and $q_0$ for all fixed $p_0, q_0 \in U$, then the displacement function $d$ is said to be regular.
\end{definition}

% \begin{theorem}
%     Suppose $d: \mathcal{M}\times\mathcal{M} \to \mathbb{R}^n$ is a displacement function, then $d$ satisfies:
%     \begin{enumerate}
%         \item For all $p \in \mathcal{M}$, $d(p,p) = 0$.
%         \item For all $p, q \in \mathcal{M}$, $d(p,q) = -d(q,p)$.
%         \item 
%     \end{enumerate}
% \end{theorem}

The above displacement function is introduced to replace the standard expression $ x-y $ in $ \mathbb{R}^n $. This displacement function closely mimics the fundamental property of translation symmetry \cite{schonsheck2022parallel}. According to our definition, the displacement between any two distinct points on a manifold is always nonzero, and the sum of vectors along a path aligns precisely with the vector directly connecting the path’s endpoints. By adhering to these properties, the kernel can be effectively translated across different points on the manifold.

With the displacement function $d$ on $\mathcal{M}$, we are now ready to give a general definition of manifold convolution.

\begin{definition}[Manifold convolution]\label{mani_conv}
Let $\mathcal{M}$ be a Riemannian $n$-manifold with a metric $g$, and let $h: \mathcal{M} \to \mathbb{R}$ be a manifold function with a kernel function $k: \mathbb{R}^{n}\to \mathbb{R}$. The convolution of $h$ and $k$ on $\mathcal{M}$ is defined as:

\begin{equation}\label{eq:mani_conv}
(h \ast_{\mathcal{M},d,g} k)(p) = \int_{\mathcal{M}} h(q) k(d(p, q)) \, dq,
\end{equation}

where:
\begin{itemize}
    \item $p, q \in \mathcal{M}$,
    \item $d: \mathcal{M}\times\mathcal{M} \to \mathbb{R}^n $ is a global displacement function on $\mathcal{M}$.
    % \item $d\nu(q)$ is the volume form on $\mathcal{M}$ derived from the metric $g$.
\end{itemize}
For simplicity, we denote $*_{\mathcal{M}, d,g}$ as $*_{d,g}$.
Moreover, $*_{d,g}$ is said to be regular if the displacement function $d$ is regular.

\end{definition}

%In Euclidean space, convolution relies on translating a kernel function $ k(x) $ into $ k(x-y) $. However, on a manifold, translation is not well-defined due to curvature. Therefore, a displacement function is introduced to replace the standard expression $ x-y $ in $ \mathbb{R}^n $. This displacement function should closely mimic the fundamental property of translation symmetry \cite{schonsheck2022parallel}. According to our definition, the displacement between any two distinct points on a manifold is always nonzero, and the sum of vectors along a path aligns precisely with the vector directly connecting the path’s endpoints. By adhering to these properties, the kernel can be effectively translated across different points on the manifold.

% These observations highlight the close relationship between the displacement function $d$ and volume form $d\nu$ derived from the Riemannian metric $g_{\mu\nu}$, as formalized in Definition \ref{mani_conv}. Ideally, $d\nu$ should be directly induced by the displacement function $d$, ensuring metric consistency throughout the integral. This relationship and its implications will be further explored in the next subsection.

Note that the manifold convolution above is not commutative as the manifold function $h$ is defined on $\mathcal{M}$ while kernel function $k$ is defined on $\mathbb{R}^n$. This lack of commutativity does not hinder the definition of the convolution operation for deep learning tasks on Riemann surfaces. 

% Before we proceed to the next section, let's make another remark about the volume form $d\nu$ and the Riemannian metric $g$ to make the discussion later more convenient.

Before we proceed to the next section, we shall emphasize that the convolution operator $*_{d,g}$ does depend on the Riemannian metric $g$ of $\mathcal{M}$ due to the standard definition of integration on manifolds used in Equation \ref{eq:mani_conv}. More details are discussed in the following remark.
\begin{remark}\label{rm:volume_form}
    Let $\mathcal{M}$ be a Riemannian $n$-manifold and let $h: \mathcal{M} \to \mathbb{R}$ be a manifold function. 
    Suppose $\mathcal{M}$ is covered by a collection of coordinate charts $\{(U_\alpha, \phi_\alpha)\}$, and let $\{\psi_\alpha\}$ be a partition of unity subordinate to the cover $\{U_\alpha\}$. In local coordinates, the {\it volume form $d\nu$ induced by $g$} is given by:
\begin{equation}
d\nu =  \sqrt{\det(g(x))} \, dx^1 \wedge dx^2 \wedge \cdots dx^n,
\end{equation}
\noindent where $g(x)=(g_{ij}(x))_{1\leq i,j\leq n}$. The manifold integral of $h$ is then defined as
    % \begin{equation}
    % \int_\mathcal{M} h \, d\nu = \int_\Omega h(\phi(x)) \sqrt{\det(g(x))} \, dx.
    % \end{equation}
    \begin{equation}
    \int_\mathcal{M} h \, d\nu = \sum_\alpha \int_{U_\alpha} \psi_\alpha(x) h(\phi_\alpha^{-1}(x)) \sqrt{\det(g(x))} \, dx.
    \end{equation}
  \end{remark}

%\begin{remark}
 %   Let $\mathcal{M}$ be a Riemannian $n$-manifold, and let $h: \mathcal{M} \to \mathbb{R}$ be a manifold function. For a diffeomorphism $\phi : \Omega \to \mathcal{M}$, where $\Omega \in \mathbb{R}^{n}$, the manifold integral could be defined as    
 %   \begin{equation}
  %  \int_\mathcal{M} h \, d\nu = \int_U h(\phi(x)) \sqrt{\det(g(x))} \, dx,    
   % \end{equation}
    %where $g=(g_{ij})$ and:  
    %\begin{equation}
    %g_{ij} = \langle \frac{\partial \phi}{\partial x_i}, \frac{\partial \phi}{\partial x_j} \rangle,
    %\end{equation}
    %we say the volume form $d\nu$ is derived by metric $g=(g_{ij})$ via $\phi$.
%\end{remark}

\subsubsection{Convolution on Manifold via Parameterization}

The concept of the convolution operation on a manifold is not well-established, primarily due to the curvature of the manifold. In the Euclidean domain, the plain convolution operator involves shifting the kernel function. This shifting is straightforward in Euclidean space, where the geometry is flat and the displacement from one point to another is well-defined. In this subsection, we introduce the idea of defining convolution on a Riemann surface on its 2D parametric flat domain.
%When defining convolution on a manifold, one is confronted with the challenges introduced by the manifold's curvature and the lack of a proper manifold metric definition. The typical convolution operation, which relies on shifting and integrating a kernel over a space, is straightforward in Euclidean spaces, where the geometry is flat and distances are well-defined. Therefore, it is natural to define a convolution on the parameterization domain.

\begin{definition}[Parametrized manifold convolution]
    Let $\mathcal{M}$ be a Riemannian $n$-manifold, and let $h: \mathcal{M} \to \mathbb{R}$ be a manifold function with a kernel function $k: \mathbb{R}^{n}\to \mathbb{R}$. Suppose there exists a bijective parametrization $\phi:\Omega \to \mathcal{M}$, where $\Omega \subset \mathbb{R}^n$. The parametrized manifold convolution of $h$ and $k$ on $\mathcal{M}$ is defined as:
    \begin{equation}
        (h *_\phi k)(p) = \int_\Omega h(\phi(y))k(\phi^{-1}(p)-y)dy.
    \end{equation}
    
    % \begin{align*}
    %     (h *_\phi g)(p) &= \int_\Omega h(\phi(y))g(\phi^{-1}(p)-y)dy\\
    %     &= \int_\mathcal{M} h(q)g(\phi^{-1}(p)-\phi^{-1}(q))d\phi^{-1}(q)
    % \end{align*}
    
    % $$(h *_\phi g)(p) = \int_\Omega h(\phi(y))g(\phi^{-1}(p)-y)dy$$
    % \int_\mathcal{M} h(q)g(\phi^{-1}(p) - \phi^{-1}(q))d\phi{-1}(q),$$
    % ($\phi$ doesn't need to be bijective here(?))
    \label{them:parameterconv}
\end{definition}

The definition above simplifies the computation of manifold convolution and provides a more intuitive viewpoint by performing convolution on an Euclidean domain. To explore the relationship between parameterized manifold convolution and manifold convolution, the following lemma is essential to demonstrate that a displacement function on a manifold gives rise to a surface parameterization, enabling us to define the parameterized manifold convolution.

% This allows us to consider a substantial subset of manifold convolutions, where the displacement function $d$ and metric $g$ are properly related, as parametrized manifold convolution, or vice versa, whichever is more intuitive.


% \begin{theorem}
%     Given a $n$-manifold $\mathcal{M}$, then there exists a displacement function $d:\mathcal{M} \times \mathcal{M} \to \mathbb{R}$ if and only if there exists a bijective parametrization $\phi: \Omega \to \mathcal{M}$ such that $$d(\phi(x),\phi(y)) = x-y$$ for all $x, y \in \Omega$, where $\Omega \subset \mathbb{R}^n$.
% \end{theorem}

\begin{lemma}\label{disp2phi}
     Let $\mathcal{M}$ be a Riemannian $n$-manifold. If there exists a displacement function $d:\mathcal{M} \times \mathcal{M} \to \mathbb{R}^n$, then there exists a bijective parametrization $\phi: \Omega \to \mathcal{M}$ such that
     \begin{equation}
         d(p,q) = \phi^{-1}(p)-\phi^{-1}(q)
     \end{equation}
     for all $p, q \in \mathcal{M}$, where $\Omega \subset \mathbb{R}^n$.

    Conversely, if there exists a bijective parametrization $\phi: \Omega \to \mathcal{M}$, where $\Omega \subset \mathbb{R}^n$, then a function $d:\mathcal{M} \times \mathcal{M} \to \mathbb{R}^n$ defined by 
    \begin{equation}
         d(p,q) = \phi^{-1}(p)-\phi^{-1}(q)
    \end{equation}
    for all $p, q \in \mathcal{M}$, is a displacement function on $\mathcal{M}$.
\end{lemma}

\begin{proof}
    $(\Rightarrow)$ Suppose there exists a displacement function $d:\mathcal{M} \times \mathcal{M} \to \mathbb{R}$. Pick any $p_0 \in \mathcal{M}$ and let $\Omega = \{d(p, p_0): p \in \mathcal{M}\} \subset \mathbb{R}^n$. Define $\tilde{\phi}: \mathcal{M} \to \Omega$ by $\tilde{\phi}(p) = d(p, p_0)$. Then
    \begin{equation}
        d(p, q) = d(p, p_0) - d(q, p_0) = \tilde{\phi}(p)-\tilde{\phi}(q).
    \end{equation}
    Note that this equation shows that $\tilde{\phi}$ is injective as $d(p,q)=0$ implies $p=q$, and the surjectivity of $\tilde{\phi}$ is guaranteed by the definition of $\Omega$, $\tilde{\phi}$ is therefore bijective. 
    % Suppose $\tilde{\phi}(p) = \tilde{\phi}(q)$ for some $p, q \in \mathcal{M}$, then
    % $$d(p, q) = d(p, p_0) - d(q, p_0) = \tilde{\phi}(p) - \tilde{\phi}(q) = 0,$$
    % which follows that $p=q$, and so $\tilde{\phi}$ is injective. Note that the surjectivity of $\tilde{\phi}$ is guaranteed by the definition of $\Omega$, $\tilde{\phi}$ is therefore bijective. For all $p, q \in \mathcal{M}$, we can check that
    % $$d(p, q) = d(p, p_0) - d(q, p_0) = \tilde{\phi}(p)-\tilde{\phi}(q).$$
    Thus $\phi = \tilde{\phi}^{-1}$ is the desired bijective parametrization.
    % $(\Leftarrow)$ It is a direct consequence of the definition of displacement functions.
    % $(\Leftarrow)$ Suppose there exists a bijective parametrization $\phi: \Omega \to \mathcal{M}$ such that $d(\phi(x),\phi(y)) = x-y$ for all $x, y \in \Omega$, where $\Omega \subset \mathbb{R}^n$. Define $d:\mathcal{M} \times \mathcal{M} \to \mathbb{R}$ by
    % $$d(p, q) = \phi^{-1}(p) - \phi^{-1}(q),$$
    % then we can
    
    $(\Leftarrow)$ Check that for all $p, q, r \in \mathcal{M}$,
    $$d(p, q) = 0 \iff \phi^{-1}(p) = \phi^{-1}(q) \iff p=q$$
    and
    $$d(p, r) = \phi^{-1}(p) - \phi^{-1}(r) = (\phi^{-1}(p) - \phi^{-1}(q)) + (\phi^{-1}(q) - \phi^{-1}(r)) = d(p, q) + d(q, r).$$
    Therefore, $d$ is a displacement function for $\mathcal{M}$.
    % , which obviously satisfies $d(\phi(x),\phi(y)) = x-y$ for all $x, y \in \Omega$.
\end{proof}

% (We need to choose one of the three theorems below)
% \begin{theorem}[original]
%     Let $\mathcal{M}$ be a Riemannian $n$-manifold and $*_\mathcal{M}$ be an operator. Then $*_\mathcal{M} = *_d$ for some displacement function $d$ on $\mathcal{M}$ if and only if $*_\mathcal{M} = *_\mathcal{\phi}$ for some bijective parametrization $\phi: \Omega \to \mathcal{M}$, where $\Omega \subset \mathbb{R}^n$.
% \end{theorem}

% \begin{proof}
%     ($\Rightarrow$) Suppose $*_\mathcal{M} = *_d$ for some displacement function $d$ on $\mathcal{M}$. By Lemma \ref{disp2phi}, there exists a bijective parametrization $\phi: \Omega \to \mathcal{M}$ such that $$d(p,q) = \phi^{-1}(p)-\phi^{-1}(q)$$ for all $p, q \in \mathcal{M}$, where $\Omega \subset \mathbb{R}^n$. Then for any manifold function $h:\mathcal{M} \to \mathbb{R}$ and kernel function $g: \mathbb{R}^n \to \mathbb{R}$, we have 
%     \begin{align*}
%         (h *_\phi g)(p) &= \int_\mathcal{M} h(q)g(\phi^{-1}(p)-\phi^{-1}(q))d\phi^{-1}(q) \\
%         &= \int_\mathcal{M} h(q)g(d(p,q))d\phi^{-1}(q) \\
%         &= (h *_d g)(p)
%     \end{align*}
%     where $\nu(q)$ is induced by $\nu = L^1 \circ \phi^{-1}$.

%     ($\Leftarrow$) By Lemma \ref{disp2phi}, the function $d(p,q) = \phi^{-1}(p)-\phi^{-1}(q)$ is a displacement function. Therefore, the result follows similarly.
% \end{proof}

\begin{theorem}\label{equiv_conv}
    Let $\mathcal{M}$ be a Riemannian $n$-manifold and $d$ be a displacement function on $\mathcal{M}$. Then there exists a bijective parametrization $\phi: \Omega \to \mathcal{M}$, where $\Omega \subset \mathbb{R}^n$, along with a metric $g$ of $\mathcal{M}$, such that $*_{d, g} = *_\phi$. Conversely, for any bijective parametrization $\phi: \Omega \to \mathcal{M}$, there exists a displacement function $d$ on $\mathcal{M}$ and a metric $g$ of $\mathcal{M}$ such that $*_{d, g} = *_\phi$.
\end{theorem}

\begin{proof}
    ($\Rightarrow$) Suppose $*_\mathcal{M} = *_{d,g}$ for some displacement function $d$ on $\mathcal{M}$. By Lemma \ref{disp2phi}, there exists a bijective parametrization $\phi: \Omega \to \mathcal{M}$ such that $$d(p,q) = \phi^{-1}(p)-\phi^{-1}(q)$$ for all $p, q \in \mathcal{M}$, where $\Omega \subset \mathbb{R}^n$. Consider $\phi^{-1}$ as the coordinate chart function and define the Riemannian metric $g = (\phi^{-1})^*g_{\mathbb{R}^n}$ as a pullback metric of $\mathcal{M}$, where $g_{\mathbb{R}^n}$ is the standard Euclidean metric. The volume form $d\nu$ can be obtained by Remark \ref{rm:volume_form}. 
    
    Since the distance $dg(p,q) = |\phi^{-1}(p)-\phi^{-1}(q)|$ for all $p,q \in \mathcal{M}$, $\phi^{-1}$ is an isometric mapping with respect to the metric $g$ and $\det(D\phi^{-1}) = 1$, therefore $dy = |\det(D\phi^{-1})| \, dq = dq$ for $\phi(y) = q$. For any manifold function $h:\mathcal{M} \to \mathbb{R}$, kernel function $k: \mathbb{R}^n \to \mathbb{R}$ and $p \in \mathcal{M}$, we now have 
    \begin{equation}
        \begin{aligned}
            (h *_\phi k)(p) &= \int_\Omega h(\phi(y))k(\phi^{-1}(p)-y)dy \\
            &= \int_\mathcal{M} h(q)k(\phi^{-1}(p)-\phi^{-1}(q))dq \\
            &= \int_\mathcal{M} h(q)k(d(p,q))dq \\
            &= (h *_{d,g} k)(p).
        \end{aligned}
    \end{equation}
    Therefore $*_{d, g} = *_\phi$.

    ($\Leftarrow$) By Lemma \ref{disp2phi}, the function $d(p,q) = \phi^{-1}(p)-\phi^{-1}(q)$ is a displacement function. Therefore, the result follows similarly.
\end{proof}


% **********************************NEED DISCUSSION HERE******************************************
% \begin{remark}
%     The parameterization $\phi$ is an isometric mapping with respect to the volume form $d\nu$, which is directly induced by $\phi$ in the construction.
% \end{remark}


% \begin{theorem}
%     Let $\mathcal{M}$ be a Riemannian $n$-manifold. An operator $*_\mathcal{M}$ is a generalized manifold convolution with respect to some displacement function $d$ on $\mathcal{M}$ if and only if it is a parametrized manifold convolution with respect to some bijective parametrization $\phi: \Omega \to \mathcal{M}$, where $\Omega \subset \mathbb{R}^n$.
% \end{theorem}


% \begin{theorem}
%     Let $\mathcal{M}$ be a Riemannian $n$-manifold and $*_\mathcal{M}$ be a manifold convolution on $\mathcal{M}$. Then there exists a bijective parametrization $\phi: \Omega \to \mathcal{M}$, where $\Omega \subset \mathbb{R}^n$, such that $*_\phi = *_\mathcal{M}$. (This only includes a single direction.)
% \end{theorem}
% \begin{proof}
%     ...
% \end{proof}

Theorem \ref{equiv_conv} establishes that the space of manifold convolutions, equipped with a displacement function $d$ and an associated metric $g$, is equivalent to the space of parameterized manifold convolutions. Hence, the set of all parameterized manifold convolutions encompasses a substantial portion of manifold convolutions, suggesting that many spatially defined manifold convolutions can be effectively represented within this parameterized framework. To conclude this section, we will investigate the regularity properties of both types of convolutions.
%Theorem \ref{equiv_conv} establishes that the space of manifold convolutions, equipped with a displacement function $d$ and an associated volume form $d\nu$ derived from a suitable metric, is equivalent to the space of parametrized manifold convolutions. Nonetheless, the set of all parametrized manifold convolutions encompasses a substantial portion of manifold convolutions, indicating that many manifold convolutions which are spatially defined can be effectively represented within this parametrized framework. To conclude this section, we investigate the regularity properties of both types of convolutions.

%Theorem \ref{equiv_conv} establishes that the space of manifold convolutions, equipped with a displacement function $d$ and a volume form $d\nu$ derived from a suitable metric, is equivalent to the space of parametrized manifold convolutions. Nonetheless, the set of all parametrized manifold convolutions represents a significant subset of manifold convolutions that are geometrically defined, in contrast to those that are not, such as spectral convolutions \cite{bruna2014spectral}. To conclude this section, we investigate the regularity properties of both types of convolutions. 


\begin{corollary}
    Let $\mathcal{M}$ be a Riemannian $n$-manifold equipped with a displacement function $d$ and a metric $g$, is parametrized by a function $\phi$ such that $*_{d,g} = *_\phi$. Then the following statements are equivalent:
    \begin{enumerate}
        \item $d$ is regular.
        \item $\phi$ is an orientation-preserving homeomorphism.
        \item $*_{d,g} = *_\phi$ is regular.
    \end{enumerate}
    \label{them:manifoldparametrized}
\end{corollary}

% \begin{corollary}
%     Suppose a Riemannian $n$-manifold $\mathcal{M}$ is parametrized by a diffeomorphic function $\phi$. Then the manifold convolution $*_\phi$ is regular.
%     % in the sense that there exists an equivalent convolution operator $*_\mathcal{M}$, such that the corresponding displacement function $d$ is regular.
% \end{corollary}
\begin{proof}
    Note that (1) and (3) are equivalent by definition, and (2) implies (1) as $*_{d,g} = *_\phi$ implies $d(p,q) = \phi^{-1}(p) - \phi^{-1}(q)$, which is regular if $\phi$ is an orientation-preserving homeomorphism.
    
    (1) $\Rightarrow$ (2): Suppose $d$ is regular, then using the construction of $\phi$ in the previous theorem, we immediately see $\phi$ is an orientation-preserving homeomorphism. Assume $*_{d,g} = *_{\phi_1}$ for some bijective parametrization $\phi_1: \Omega_1 \to \mathcal{M}$, where $\Omega_1 \subset \mathbb{R}^n$, then for all $p, q \in \mathcal{M}$,
    \begin{equation}
    \begin{aligned}
        &\phi^{-1}(p) - \phi^{-1}(q) = \phi_1^{-1}(p) - \phi_1^{-1}(q) \\
        \Rightarrow \quad &\phi^{-1}(p) - \phi_1^{-1}(p) = \phi^{-1}(q) - \phi_1^{-1}(q) \\
        \Rightarrow \quad &\phi^{-1} - \phi_1^{-1} \equiv c \qquad \text{for some constant vector } c \in \mathbb{R}^n.
    \end{aligned}
    \end{equation}
    Hence $\phi_1$ is also an orientation-preserving homeomorphism. 
\end{proof}

\subsection{Convolution on Riemann Surfaces}
In this work, we focus our problems on simply connected open surfaces embedded in $\mathbb{R}^3$, now we will move on to describe how we can define convolution on Riemann surfaces.

\subsubsection{Conformal Convolution}
A natural and useful approach to produce a parametrized manifold convolution is to employ a conformal parametrization of the manifold. A conformal parametrization $\phi$ is a map from a domain $\Omega \subset \mathbb{R}^2$ to the manifold $\mathcal{M} \subset \mathbb{R}^3$ that preserves angles. By mapping the curved surface to a flat Euclidean space, we can take advantage of the well-established theory of Euclidean convolutions. 

To define the convolution of a manifold function $h : \mathcal{M} \to \mathbb{R}$ on a $2$-manifold $\mathcal{M}$ and a kernel function $k : \mathbb{R}^2 \to \mathbb{R}$, we begin by pulling the function back to the parameter domain $\Omega$ using the conformal parametrization $\phi$. This transforms the problem of manifold convolution into a more manageable Euclidean convolution problem. Specifically, for each function $h$ defined on the surface and $k$ defined on $\mathbb{R}^2$, we define the pullback function $\tilde{h}$ on the flat domain $\Omega$ as follows:
\begin{equation}
\tilde{h}:=\phi^* h = h \circ \phi.
\end{equation}
% *********************************************************************************
% \enoch{(A more standard notation should be $\phi^* h = h \circ \phi$. How about we write $\tilde{h}:=\phi^* h = h \circ \phi$?)\\}
% *********************************************************************************
The pullback functions $\tilde{h}$ are now defined on the Euclidean space $\Omega \subset \mathbb{R}^2$, where the convolution operation can be performed using the standard Euclidean formulation.

The convolution of $\tilde{h}$ and $k$ in the Euclidean domain $\Omega$ is then defined as:
\begin{equation}
(\tilde{h} \ast k)(x) = \int_{\Omega} \tilde{h}(y) k(x - y) \, dy,
\end{equation}
where the integral is taken over the domain $\Omega \subset \mathbb{R}^2$. This is the classical convolution in Euclidean space, which is computationally efficient and well-understood.

Once the convolution is computed in the Euclidean domain, the result must be mapped back to the manifold $\mathcal{M}$. This is done by applying the inverse of the conformal parametrization $\phi^{-1}$, which maps the Euclidean result back onto the manifold's coordinates. The convolution on the manifold is then given by:
\begin{equation}
(h \ast_\phi k)(p) = (\tilde{h} \ast k)(\phi^{-1}(p)),
\end{equation}
where $p \in \mathcal{M}$ is a point on the manifold, and $\phi^{-1}(p)$ returns the corresponding point in the parameter domain $\Omega$.

Then, the manifold convolution is reduced to a Euclidean convolution performed in the parametrized space $\Omega$, followed by a pullback to the manifold using the inverse of the conformal map. The final formal definition is then written as:

\begin{definition}[Conformal Convolution]
Let $\mathcal{M} \subset \mathbb{R}^3$ be a 2-manifold, and let $\phi: \Omega \to \mathcal{M}$ be a conformal parametrization, where $\Omega \subset \mathbb{R}^2$ is a domain in the Euclidean plane. Let $h: \mathcal{M} \to \mathbb{R}$ be a manifold function with a kernel function $k: \mathbb{R}^{n}\to \mathbb{R}$. The conformal convolution of $h$ and $k$ is defined as:
\begin{equation}
\begin{aligned}
(h \ast_\phi k)(p) &= \int_\mathcal{M} h(q) k(\phi^{-1}(p) - \phi^{-1}(q)) dq \\
&= \int_{\Omega} \tilde{h}(y) k(x - y) dy \\
&= (\tilde{h} \ast k)(\phi^{-1}(p)),
\end{aligned}
\end{equation}
where:
\begin{itemize}
    \item $p,q \in \mathcal{M}$,
    \item $x=\phi^{-1}(p), y=\phi^{-1}(q) \in \Omega$,
    \item $\tilde{h} = h \circ \phi$ is the pullback function,
    \item $g = (\phi^{-1})^* g_{\mathbb{R}^2}$ is the Riemannian metric of $\mathcal{M}$.
\end{itemize}
\label{def:conformalconv}
\end{definition}

It is also important to note that, according to the Riemann Mapping Theorem (Theorem \ref{them:RiemannMapping}), the conformal parameterization is unique when three points are fixed. In practice, by mapping the surface to a disk or rectangle while fixing the boundary points, the parametrization becomes determined. The conformal convolution is a special case of parametrized convolution, and thus a manifold convolution by taking $d(p,q) \coloneqq   \phi^{-1}(p) - \phi^{-1}(q)$ and $d\nu$ derived from metric $g$  via $\phi^{-1}$, according to Theorem \ref{equiv_conv}. 

Although conformal parametrization retains nice geometric properties once the surface is mapped to the 2D domain, there is no evidence that it is the best parametrization to define a parametrized convolution. As a special case of parametrized convolution, conformal convolution is too restrictive for advanced usage, such as when implemented into deep learning tasks. Therefore, a much more flexible convolution, namely Quasi-Conformal Convolution, will be introduced in the next subsection.

\subsubsection{Quasi-Conformal Convolution}

The proposed Quasi-Conformal Convolution (QCC) is to define convolution operation on manifolds using quasi-conformal mappings. Quasi-conformal theory offers a mathematically robust framework for studying deformations between surfaces while preserving local geometric structures. By leveraging this theory, QCC extends the convolution operation to non-Euclidean domains such as manifolds, enabling deep learning methods to process irregular and geometrically distorted data.

Here, similar to how we define parametrized convolution, we have the following definition for Quasi-conformal Convolution.

\begin{definition}[Quasi-conformal Convolution]
Let $\mathcal{M} \subset \mathbb{R}^3$ be a 2-manifold, and let $\phi: \Omega \to \mathcal{M}$ be a conformal parametrization, where $\Omega \subset \mathbb{R}^2$ is a domain in the Euclidean plane. Let $h: \mathcal{M} \to \mathbb{R}$ be a manifold function with a kernel function $k: \mathbb{R}^{n}\to \mathbb{R}$.  Let $f: \Omega \to \Omega$ be a quasi-conformal mapping. The quasi-conformal convolution of $h$ and $k$ with respect to $\phi$ and $f$ is defined as:

\begin{equation}
\begin{aligned}
(h \ast_{\phi, f} k)(p)
&= \int_{\mathcal{M}}h(q)k(f\circ\phi^{-1}(p) - f\circ\phi^{-1}(q)) dq\\
&= \int_{\Omega} h\circ\phi(y) k(f(x) - f(y)) \, df(y) = \int_{\Omega} \tilde{h}(y)k(f(x) - f(y)) \, df(y)\\
&= \int_{\Omega} \tilde{h}\circ f^{-1}(y')k(x' - y') \, dy' = \int_{\Omega} h^\#(y')k(x' - y') \, dy'\\
&= h^\# \ast k(f\circ\phi^{-1}(p))
\label{eq:parametrizedQCC}
\end{aligned}
\end{equation}

where:
\begin{itemize}
    \item $p,q \in \mathcal{M}$ is a point on the manifold,
    \item $p = \phi(x), q = \phi(y) \in \Omega$,
    \item $x'=f(x), y'=f(y) \in \Omega$,
    \item $f:\Omega\to\Omega$ is a quasi-conformal mapping on the parametrized domain $\Omega$,
    \item $\tilde{h}=h\circ\phi$ is the pullback function,
    \item $h^\#=\tilde{h}\circ f^{-1}$ is the transformed pullback function,
    \item $g = (f \circ \phi^{-1})^* g_{\mathbb{R}^2}$ is the Riemannian metric of $\mathcal{M}$.
\end{itemize}
\end{definition}

Recall that, according to the Riemann Mapping Theorem (Theorem \ref{them:RiemannMapping}), the conformal parameterization is unique when three points are fixed which is generally ensured through the fix the boundaries. Therefore, the subscript $\phi$ of the operation $\ast_{\phi, f}$ can be omitted into $\ast_{f}$.

\begin{theorem}
    Let $\mathcal{M} \subset \mathbb{R}^3$ be a 2-manifold and $*_\mathcal{M}$ be a parametrized manifold convolution. Then $*_\mathcal{M}$ is regular if and only if $*_\mathcal{M}$ is a quasi-conformal convolution.
\end{theorem}
\begin{proof}
    The mapping $\phi: \Omega \to \mathcal{M}$ is quasi-conformal if and only if it is an orientation-preserving homeomorphism, which is equivalent to saying that the parametrized manifold convolution $*_\phi$ is regular.
\end{proof}

\begin{remark}
    Quasi-conformal Convolution is a parametrized manifold convolution with parametrization $\phi \circ f^{-1}$.
\end{remark}

The theorem and remark above show that we can generalize any regular parametrized manifold convolution on Riemann surfaces into quasi-conformal convolution, through which a substantial subset of manifold convolution could be represented. As different Quasi-conformal parameterizations would yield distinct convolution operators on the Riemann surface, one can find the best convolution operator for a specific task by optimizing the corresponding Quasi-conformal parameterization. Within this framework, we can identify the most effective convolution operator on a Riemann surface by optimizing the Quasi-conformal parameterization. 

By representing parametrized manifold convolutions through quasi-conformal mappings, learning the convolution operations in neural networks becomes feasible. In the next section, we will develop a deep neural network framework to learn the optimal Quasi-conformal parameterization associated with the best convolution operator for a given task. Quasi-conformal mappings preserve local geometric structures while allowing controlled deformations. This characteristic ensures that the convolution operation aligns with the intrinsic geometry of the manifold, enabling more robust and effective feature extraction when implementing quasi-conformal convolution into deep neural networks.
We applied Recurrency Sequence Processing to address the lack of consistency in the coarse dance representation of the~\cite{li2024lodge} model. We named this Recurrency Sequence Representation Learning as Dance Recalibration (DR). Dance recalibration uses \(n\) Dance Recalibration Blocks (DRB) corresponding to the length of the rough dance sequence to add sequential information to the rough dance representation to improve the consistency of the whole dance. The overall structure of our model is illustrated in Figure 1.

\begin{figure}[!t]
    \centering
    \includegraphics[width=\textwidth]{Figure1.eps}
    \caption{overall procedure of Pooling processing by our Pooling Block}
    \label{fig:enter-label4}
\end{figure}


\subsection{Dance Recalibration (DR)}
When the dance motion representation passes through the Dance Decoder Process using the~\cite{li2024lodge} model, it yields a coarse dance motion representation. During this process, the dance motion representations that pass through Global Diffusion follow a distribution but can output unstable values. This results in awkward dance motions when viewed from a sequential perspective. To address this issue, we added a Dance Recalibration Process.

DR fundamentally follows a structure similar to RNNs. Although RNNs are known to suffer from the gradient vanishing problem as they get deeper, the sequence length of the coarse dance representation in \cite{li2024lodge} is not long enough to cause this issue, making it suitable for use. Using LSTM or GRU, which solve the gradient vanishing problem, would make the model too complex and computationally expensive, making them unsuitable for use with the Denoising Diffusion Process \cite{ho2020denoising, song2020denoising}.

The coarse dance representation has 139 channels, consisting of 4-dim foot positions, 3-dim root translation, 6-dim rotaion information and 126-dim joint rotation channels. Of these, the 126-dim channels directly impact the dance motion, and all DR operations are performed using these 126 channels.

The values output from the Global Dance Decoder \(GD_{i}\), contain unstable dance motion information that follows a distribution. We construct Global Recalibrated Dance \(GRD_{i}\) by concatenating \(C\) the information from \(GRD_{i-1}\) with \(GD_{i}\) and applying pooling \(P\), thereby adding sequential information. However, using previous information as is may result in overly simple and smoothly connected dance motions. To prevent this, we add Gaussian noise \(G\) to the previous information \(GRD_{i-1}\) to produce more varied dance motions. This process is represented in Equations 1 below. The entire procedure is illustrated in Figure 2, 3.
\begin{equation}
    GRD_{i} = P(C(GD_{i} , GRD_{i-1} + G(Threshold))
\end{equation}



\begin{figure}[!t]
    \centering
    \includegraphics[width=\textwidth]{DanceRecalibration.eps}
    \caption{Overall of the Dance Recalibration Block Structure}
    \label{fig:enter-label1}
\end{figure}

\begin{figure}[!t]
    \centering
    \includegraphics[width=\textwidth]{DanceRecalibrationBlock.eps}
    \caption{The structure of the dance recalibration block}
    \label{fig:enter-label2}
\end{figure}

\subsection{Pooling Block}
Pooling \(P\) uses a simple pooling method. When \(GRD_{i}\) with added \(G\) and \(GD_{i+1}\) are input, they are concatenated into a \((Batch\times2\times126)\). First, we perform Layer Normalization to minimize differences between layers. Then, we pass through three simple 1D-Convolution Blocks, each followed by an activation function and batch normalization, to construct \(GRD_{i+1}\) that includes information from the previous dance motion. This procedure is illustrated in Figure 4.

\begin{figure}[!t]
    \centering
    \includegraphics[width=\textwidth]{Figure3.eps}
    \caption{overall procedure of Pooling processing by our Pooling Block}
    \label{fig:enter-label3}
\end{figure}

By following all these steps, each dance motion incorporates a bit of information from the previous dance motions, producing an overall coarse dance motion that follows the distribution of Global Diffusion while also retaining sequential information. This process is expressed in Equation 2:

\begin{equation}
    Total Coarse Dance Motion = C_{i=1}^{n}(P(C(GD_{i} , GRD_{i-1} + G(Threshold))), P(GD_{0}))
\end{equation}

We did not use bi-directional information because it complicates the calculations and can destabilize sequential information when using more than two \(GD_{i}\). Since there is a trade-off between generating complex dance motions and maintaining consistency, it is crucial to add appropriate noise. However, due to time constraints, we could not conduct various ablation studies.
\subsection{Experimental Setup}
\label{section:experimental_setup}
\textbf{Datasets:} Table~\ref{tab:datasets} provides a detailed breakdown of the SOTA intrusion datasets utilized in our study. 
%For each dataset we follow the data preparation steps outlined in section~\ref{section:data_preparation}. 
% \sean{is this section necessary with reduced page limit?}
% \begin{enumerate}
%     \item X-IIoTID \cite{al2021x}: The dataset consists of 59 features which are collected with the independence of devices and connectivity, generating a holistic intrusion data set to represent the heterogeneity of IIoT systems. It includes novel IIoT connectivity protocols, activities of various devices, and attack scenarios.  
%     \item WUSTL-IIoT \cite{zolanvari2021wustl}: WUSTL-IIoT aims to emulate real-world industrial systems. The dataset is deliberately unbalanced to imitate real-world industrial control systems, consisting of 41 features and 1,194,464 observations.
%     \item CICIDS2017 \cite{Sharafaldin2018TowardGA} The CICIDS2017 dataset includes a comprehensive collection of benign and malicious network traffic. It contains 80 features and represents a broad range of attacks, such as DoS, DDoS, Brute Force, XSS, and SQL Injection, across more than 2.8 million network flows. The dataset is widely used in evaluating intrusion detection systems.
%     \item UNSW-NB15 \cite{moustafa2015unsw, moustafa2016evaluation, moustafa2017novel, moustafa2017big, sarhan2020netflow} UNSW-NB15 is a comprehensive network intrusion dataset created by the University of New South Wales. It contains 49 features representing normal and malicious activities generated using IXIA's network traffic generator, covering a variety of contemporary attack types. 
% \end{enumerate}
For IIoT intrusion, we use IIoT datasets X-IIoTID \cite{al2021x} and WUSTL-IIoT \cite{zolanvari2021wustl}. We also include commonly used network intrusion datasets CICIDS2017 \cite{Sharafaldin2018TowardGA} and UNSW-NB15 \cite{moustafa2015unsw}. For X-IIoTID \cite{al2021x}, CICIDS2017 \cite{Sharafaldin2018TowardGA}, and UNSW-NB15 \cite{moustafa2015unsw}, we split the data across five experiences such that each experience contains two to four attacks. For WUSTL-IIoT \cite{zolanvari2021wustl}, we split the data across four experiences such that each experience contains one attack. We perform this data split to simulate an evolving data stream with emerging cyber attacks over time where each experience contains different attacks. 


%%%%%%%%%%%%%%%%%%%%%%%%%%%%%%%%%%%%%%%%%%%%%%%%%%%%%%%%%%%%%%%%%%%%%%%%%%%
\begin{table}[h]
    \caption{Selected Intrusion Datasets}
    \centering
    \label{tab:datasets}
    \resizebox{.99\columnwidth}{!}{
    \begin{tabular}{c|c|c|c|c}
    \hline
    Dataset    & Size      & Normal Data & Attack Data & Attack Types \\ 
    \hline
    X-IIoTID \cite{al2021x}   & 820,502   & 421,417     & 399,417     & 18           \\
    \hline
    WUSTL-IIoT \cite{zolanvari2021wustl} & 1,194,464 & 1,107,448   & 87,016      & 4       \\
    \hline
    CICIDS2017 \cite{Sharafaldin2018TowardGA} & 2,830,743 & 2,273,097 & 557,646 & 15 \\
    \hline
    UNSW-NB15 \cite{moustafa2015unsw}
 & 257,673 & 164,673 & 93,000 & 10 \\
    \hline
    \end{tabular}}
\end{table}
%%%%%%%%%%%%%%%%%%%%%%%%%%%%%%%%%%%%%%%%%%%%%%%%%%%%%%%%%%%%%%%%%%%%%%%%%%%

\textbf{Baselines:} %Due to the novelty of this problem formulation, there are no directly comparable methods. However, the most similar widely studied problem would be unsupervised continual learning (UCL). Therefore, 
We evaluate our algorithm against two SOTA unsupervised continual learning (UCL) algorithms: the Autonomous Deep Clustering Network (\textbf{ADCN}) \cite{ashfahani2023unsupervised}, and an autoencoder paired with K-Means clustering. The autoencoder K-Means model is combined with Learning without Forgetting \cite{lwf2019Li} continual learning loss; we refer to this model as \textbf{LwF}. Note that both \textbf{ADCN} and \textbf{LwF} require a small amount of labeled normal and attack data to perform classification. We also compare our approach against SOTA ND methods: local outlier factor (\textbf{LOF})\cite{Faber_2024}, one-class support vector machine (\textbf{OC-SVM})\cite{Faber_2024}, principal component analysis (\textbf{PCA})\cite{rios2022incdfm}, and Deep Isolation Forest (\textbf{DIF}) \cite{xu2023deep}. 
%We train the ND algorithms on the clean subset of normal data, $N_c$, and evaluate their performance on the remainder of the dataset. 
Since these ND models cannot be retrained on unlabeled contaminated data, continual learning is not feasible for these methods.

%an autoencoder with K-Means clustering paired with SOTA Learning without Forgetting (LwF) continual loss (LwF) \cite{lwf2019Li}.
%Notably, many SOTA UCL algorithms rely on image-specific contrastive pairs, which is not directly applicable to intrusion detection \cite{madaan2022representational, yu2023scale, fini2022self, liu2024unsupervised}.

%%%%%%%%%%%%%%%%%%%%%%%%%%%%%%%%%%%%%%%%%%%%%%%%%%%%%%%
\begin{figure*}
    \centering
    \includegraphics[width=.95\linewidth]{figures/cl_experiments.pdf}
    \caption{Continual learning metric results of ADCN\cite{ashfahani2023unsupervised}, LwF\cite{lwf2019Li}, and \Design{}}
    \label{fig:continual_methods_results}
\end{figure*}
%%%%%%%%%%%%%%%%%%%%%%%%%%%%%%%%%%%%%%%%%%%%%%%%%%%%%%%

\textbf{Evaluation Metrics:} To evaluate the model performance, we report $F_{1}$ score. Since there is a class imbalance within these datasets, to simulate real world IDS, $F_{1}$ score gives an accurate idea on attack detection. For the continual learning methods, we evaluate their performance at the end of each training experience on all experience test sets. This generates a matrix of $F_{1}$ score results $R_{ij}$ such that $i$ is the current training experience, and $j$ is the testing experience. To summarize this matrix of results, we report widely used CL metrics \cite{diaz2018don}: average $F_{1}$ score on current experience (AVG), forward transfer (FwdTrans), and backward transfer (BwdTrans). For a matrix $R_{ij}$ with $m$ total experiences, our metrics are formulated as follows: $\text{AVG}_{F_1} = \frac{\sum_{i = j} R_{ij}}{m}$; $\text{FwdTrans}_{F_1} = \frac{\sum_{j>i} R_{ij}}{\frac{m * (m-1)}{2}}$; $\text{BwdTrans}_{F_1} = \frac{\sum_{i}^m R_{mi} - R_{ii}}{\frac{m * (m-1)}{2}}$.
AVG is the average performance on the current test experience at every point of training. FwdTrans is the average performance on ``future'' experiences, which simulates performance on zero-day attacks. Finally, BwdTrans is the average change in performance of ``past'' test experiences at a ``future'' point of training. A negative BwdTrans indicates catastrophic forgetting, whereas a positive BwdTrans  indicates the model actually improved performance on past experiences after learning a future experience. Overall, AVG measures seen attacks, FwdTrans measures zero-day attacks, and BwdTrans measures forgetting. For all metrics, a higher positive result indicates a better performance. 

We also report the threshold-free metric Precision-Recall Area Under the Curve (PR-AUC) \cite{praucDavid06}. Since \Design{} requires selecting a threshold, PR-AUC allows us to assess model performance independently of the threshold. We choose PR-AUC over Receiver Operating Characteristic Area Under the Curve (ROC-AUC) because ROC-AUC can give misleadingly high results in the presence of class imbalance \cite{praucDavid06}.

\textbf{Hyperparameters:} %For $L_{CND}$ hyperparameters are the number of K-Means clusters $K$, the reconstruction loss strength $\lambda_R$,  the continual learning loss strength $\lambda_{CL}$, and the cluster separation loss margin $m$. 
We utilize \textit{elbow method} \cite{han2011data} for determining the number of clusters $K$. 
%It tests a range of $K$ values and then selects the value   where there is a significant change in slope, called the elbow point. 
%This resulted in $K$ values between 100-500. 
We set $\lambda_R$ and $\lambda_{CL}$ to 0.1, and for $m$ we use 2 after careful experimentation. For the AE modules of \Design{}, we use 4-layer MLP with 256 neurons in the hidden layers. We train it using Adam optimizer \cite{kingma2017adammethods} with a learning rate of 0.001. For PCA, we use the explained variance method and set it to 95\% \cite{rios2022incdfm}.

\textbf{Hardware:} We run our experiments on NVIDIA GeForce RTX 3090 GPU, with a AMD EPYC 7343 16-Core processor.

\subsection{Results}

\textbf{Continual Learning Comparison:} Fig.~\ref{fig:continual_methods_results} presents the results of our approach \Design{} compared with ADCN\cite{ashfahani2023unsupervised} and LwF\cite{lwf2019Li}. \Design{} shows the best performance on both seen (AVG) and unseen (FwdTrans) attacks across all datasets. \Design{} also has the highest BwdTrans on all except one dataset (UNSW-NB15). The average BwdTrans of \Design{} (0.87\%) is higher than the average BwdTrans of both ADCN (-0.06\%) and LwF (0.09\%). Notably, the BwdTrans of \Design{} is positive for three datasets. Indicating past experiences actually improve after training on future experiences for these datasets. Given the high FwdTrans as well, our approach finds features that generalize well to future experiences. 

Table~\ref{tab:improvement} shows the improvement of \Design{} over the UCL baselines on all datasets. Bold and underlined cases indicate the best and the second best improvements with respect to each metric, respectively. These improvements were calculated by comparing the performance of \Design{} to the baselines, where the improvement values represent the proportional increase over the baseline performance. We do not include BwdTrans because a proportional increase does not make sense for a metric that can be negative. \Design{} has up to $4.50\times$ and $6.1\times$ AVG improvement on ADCN and LwF, respectively. In addition, \Design{} has up to $6.47\times$ and $3.47\times$ FwdTrans improvement on ADCN and LwF. Averaged across all datasets, \Design{} shows a $1.88\times$ and $1.78\times$ improvement on AVG, and a $2.63\times$ and $1.60\times$ improvement on FwdTrans, compared to ADCN and LwF, respectively. %These results underscore the benefit of our continual novelty detection method \Design{}. The notably high FwdTrans score emphasizes how novelty detection can be used to identify unseen anomalous data, thereby significantly enhancing performance on zero-day attacks.

Overall, these results highlight the benefit of continual ND over UCL methods for IDS. \Design{}, with its PCA-based novelty detector, excels by effectively harnessing the normal data to identify attacks. A key strength of our approach lies in the assumption that normal data forms a distinct class, while everything else is treated as anomalous. This assumption is particularly well-suited to IDS. In contrast, methods like ADCN and LwF do not make this distinction where they handle both normal and attack data similarly, limiting their ability to fully exploit the inherent structure of the data. 



% %%%%%%%%%%%%%%%%%%%%%%%%%%%%%%%%%%%%%%%%%%%%%%%%%%%%%%%
% \begin{table}[]
% \centering
% \caption{\Design{} Percentage Improvement over UCL Baselines on AVG and FwdTrans}
% \label{tab:improvement}
% \begin{tabular}{|c|c|c|c|}
% \hline
% Baseline      & Dataset    & AVG  & FwdTrans  \\ \hline
% ADCN\cite{ashfahani2023unsupervised}          & X-IIoTID   & 101.88\%        & 400.35\%        \\ \cline{2-4} 
%               & WUSTL-IIoT & 349.86\%        & 546.68\%        \\ \cline{2-4} 
%               & CICIDS2017 & 37.19\%         & 73.46\%         \\ \cline{2-4} 
%               & UNSW-NB15  & 29.25\%         & 43.90\%         \\ \hline
% LwF\cite{lwf2019Li} & X-IIoTID   & 46.43\%         & 35.39\%         \\ \cline{2-4} 
%               & WUSTL-IIoT & 510.92\%        & 246.81\%        \\ \cline{2-4} 
%               & CICIDS2017 & 92.72\%         & 163.81\%        \\ \cline{2-4} 
%               & UNSW-NB15  & 11.07\%         & 2.20\%          \\ \hline
% \end{tabular}
% \end{table}
% %%%%%%%%%%%%%%%%%%%%%%%%%%%%%%%%%%%%%%%%%%%%%%%%%%%%%%%

%%%%%%%%%%%%%%%%%%%%%%%%%%%%%%%%%%%%%%%%%%%%%%%%%%%%%%%
\begin{table}[]
\centering
\caption{\Design{} Improvement over UCL Baselines}
\label{tab:improvement}
\scalebox{1}{
\begin{tabular}{|c|c|c|c|}
\hline
Baseline      & Dataset    & AVG  & FwdTrans  \\ \hline
ADCN\cite{ashfahani2023unsupervised}  & X-IIoTID   & $\underline{2.02\times}$  & $\underline{5.00\times}$   \\ \cline{2-4} 
                                      & WUSTL-IIoT & $\mathbf{4.50\times}$  & $\mathbf{6.47\times}$   \\ \cline{2-4} 
                                      & CICIDS2017 & $1.37\times$  & $1.73\times$   \\ \cline{2-4} 
                                      & UNSW-NB15  & $1.29\times$  & $1.44\times$   \\ \hline
LwF\cite{lwf2019Li}                   & X-IIoTID   & $1.46\times$  & $1.35\times$   \\ \cline{2-4} 
                                      & WUSTL-IIoT & $\mathbf{6.11\times}$  & $\mathbf{3.47\times}$   \\ \cline{2-4} 
                                      & CICIDS2017 & $\underline{1.93\times}$  & $\underline{2.64\times}$   \\ \cline{2-4} 
                                      & UNSW-NB15  & $1.11\times$  & $1.02\times$   \\ \hline
\end{tabular}}
\end{table}

%%%%%%%%%%%%%%%%%%%%%%%%%%%%%%%%%%%%%%%%%%%%%%%%%%%%%%%

%Figure~\ref{fig:XIIoT_graph} shows the $F_{1}$ score of ADCN and \Design{} for each experience on both datasets. Similarly, we use green and red colors for \Design{} and ADCN respectively. Notably for \Design{}, the $F_{1}$ score of each experience has little change over training time. This highlights the strength of novelty detection for IDSs, as even before seeing attacks \Design{} has good performance. On the other hand, ADCN test experiences do not improve until the associated training experience, meaning ADCN does not have an ability to generalize to future attacks. ADCN utilizes a subset of labeled data to assign labels to clusters. This subset of labeled might be causing ADCN to overfit to the attacks within the current experience, therefore leading ADCN to not generalize well. We can also clearly see that our approach is consistently better (higher $F_{1}$ score) than the state-of-the-art ADCN. 

% %%%%%%%%%%%%%%%%%%%%%%%%%%%%%%%%%%%%%%%%%%%%%%%%%%%%%%%
% \begin{figure*}[t]
%     \centering
%     \begin{subfigure}[t]{\linewidth}
%         \centering
%         \includegraphics[width=\linewidth]{figures/X-IIoTID-experiences.pdf}
%         \caption{X-IIoTID}
%         \label{fig:ADCN_XIIoT_results}
%     \end{subfigure}
%     \begin{subfigure}[t]{\linewidth}
%         \centering
%         \includegraphics[width=\linewidth]{figures/WUSTL-IIoT-experiences.pdf}
%         \caption{WUSTL-IIoT}
%         \label{fig:WUSTL-}
%     \end{subfigure}
%     \caption{$F_1$ Score of ADCN and \Design{} of each test experience over training experiences.}
%     \label{fig:XIIoT_graph}
% \end{figure*}
% %%%%%%%%%%%%%%%%%%%%%%%%%%%%%%%%%%%%%%%%%%%%%%%%%%%%%%%

\textbf{Novelty Detectors Comparison:} Fig.~\ref{fig:novelty_methods_results} compares LOF\cite{Faber_2024}, OC-SVM\cite{Faber_2024}, PCA\cite{rios2022incdfm}, and DIF \cite{xu2023deep} with \Design{} on all datasets. The average $F_{1}$ score of the novelty detection methods are compared to the AVG of \Design{}.  It can be seen \Design{} outperforms all other methods across all datasets. The two best performing methods are DIF and PCA. The average $F_{1}$ score improvement across all datasets of \Design{} is $1.16\times$ and $1.08\times$ over DIF and PCA, respectively. These results highlight the critical role of leveraging information from unsupervised data streams. Unlike these ND algorithms, \Design{} is capable of continuously learning from this unsupervised data, enabling it to enhance PCA reconstruction over time. By integrating evolving data patterns, \Design{} not only adapts to new anomalies but also improves its overall detection accuracy, demonstrating a clear advantage in dynamic environments.

%Given that \Design{} employs PCA detection, this indicates that the CFE effectively extracts useful features from the unlabeled training experiences. T

%%%%%%%%%%%%%%%%%%%%%%%%%%%%%%%%%%%%%%%%%%%%%%%%%%%%%%%   
\begin{figure}
    \centering
    \includegraphics[width=0.9\linewidth]{figures/novelty_detectors_experiments.pdf}
    \caption{Average $F_1$ score on all experiences of \Design{} and novelty detection methods: LOF, OC-SVM, PCA, DIF}
    \label{fig:novelty_methods_results}
\end{figure}
%%%%%%%%%%%%%%%%%%%%%%%%%%%%%%%%%%%%%%%%%%%%%%%%%%%%%%%
%%%%%%%%%%%%%%%%%%%%%%%%%%%%%%%%%%%%%%%%%%%%%%%%%%%%%%% 
\begin{figure}
    \centering
    \includegraphics[width=0.86\linewidth]{figures/novelty_detectors_pr_auc.pdf}
    \caption{Thresholding Free Evaluation of \Design{}}
    \label{fig:thresholding_free}
\end{figure}

%%%%%%%%%%%%%%%%%%%%%%%%%%%%%%%%%%%%%%%%%%%%%%%%%%%%%%%

\textbf{Pre-threshold Evaluation:} While thresholding plays a crucial role in attack decision-making, evaluating model prediction performance before applying threshold is also important. The UCL algorithms (ADCN\cite{ashfahani2023unsupervised} and LwF\cite{lwf2019Li}) do not output anomaly scores because they select classes based on the closest labeled cluster. Therefore we compare against the two best ND methods: DIF\cite{xu2023deep} and PCA\cite{rios2022incdfm}. Fig.~\ref{fig:thresholding_free} presents the PR-AUC values of DIF, PCA, and \Design{}. It can be seen that \Design{} provides the best threshold free results, which aligns with the threshold-based results presented earlier. The strong performance of \Design{} in both pre-threshold and threshold-based evaluations demonstrates that the model is robust regardless of the decision threshold. 

\subsection{Ablation Study}

To demonstrate the impact of our loss function components, we perform an ablation study. Table~\ref{tab:ablation_loss} shows the results of \Design{} with each loss function removed to demonstrate their individual effectiveness. Bold and underlined cases indicate the best and the second best performances with respect to each metric, respectively. \Design{} without reconstruction loss ($L_R$) and \Design{} without cluster separation loss ($L_{CS}$) performs worse in all categories. \Design{} without both $L_R$ and continual learning loss ($L_{CL}$) actually performs better AVG but has worse BwdTrans and FwdTrans. AVG does not account for past experiences, so the significantly negative BwdTrans indicates \Design{} w/o $L_R$ and $L_{CL}$ forgets, and therefore would perform worse on those experiences in the future. This would make sense as a regularization loss to improve continual learning would slightly decrease performance in non-continual scenario. Overall \Design{} has the best results when taking every metric category into account. Notably the low BwdTrans and FwdTrans of \Design{} (w/o $L_R$) showcases how the reconstruction loss helps \Design{} generalize better to unseen and past data. This highlights the power of $L_R$ to provide good features for continual learning. 

%%%%%%%%%%%%%%%%%%%%%%%%%%%%%%%%%%%%%%%%%%%%%%%%%%%%%%%%%%%%%%%%%%%%%
\begin{table}[]
\caption{Ablation Study of \Design{} Loss Functions}
\label{tab:ablation_loss}
\centering
\begin{tabular}{|c|c|c|c|}
\hline
Strategy                         & AVG              & BwdTrans        & FwdTrans         \\ \hline
CND-IDS                          &\underline{76.92\%}    & \textbf{0.87\%} & \textbf{73.70\%} \\ \hline
CND-IDS (w/o $L_{CS}$)           & 66.23\%          & \underline{0.09\%}    & 70.26\%          \\ \hline
CND-IDS (w/o $L_R$)              & 72.86\%          & -5.44\%         & 67.82\%          \\ \hline
CND-IDS (w/o $L_R$ and $L_{CL}$) & \textbf{79.92\%} & -11.26\%        & \underline{71.01\%}    \\ \hline
\end{tabular}
\end{table}
%%%%%%%%%%%%%%%%%%%%%%%%%%%%%%%%%%%%%%%%%%%%%%%%%%%%%%%%%%%%%%%%%%%%%%%

\subsection{Overhead Analysis}
%%%%%%%%%%%%%%%%%%%%%%%%%%%%%%%%%%%%%%%%%%%%%%%%%%%%%%%%%%%
% \begin{table}[]
% \centering
% \caption{Average training time and inference time per sample across all datasets in milliseconds}
% \label{tab:overhead}
% \begin{tabular}{|c|c|c|}
% \hline
% Strategy               & Inference Time(ms) \\ \hline
% \Design{}                   & 0.0019             \\ \hline
% ADCN\cite{ashfahani2023unsupervised}    & 0.4061             \\ \hline
% LwF\cite{lwf2019Li}           & 0.0677             \\ \hline
% DIF\cite{xu2023deep}         & 1.0535             \\ \hline
% PCA\cite{rios2022incdfm}       & 0.0018             \\ \hline
% \end{tabular}
% \end{table}
%%%%%%%%%%%%%%%%%%%%%%%%%%%%%%%%%%%%%%%%%%%%%%%%%%%%%%%%%%%%%
\begin{table}[]
\centering

\caption{Average inference time (in ms) per test sample}
\label{tab:overhead}
\scalebox{0.95}{
\begin{tabular}{|c|c|c|c|c|c|}
\hline
Strategy           & \Design{} & ADCN   & LwF    & DIF    & PCA    \\ \hline
Inference Time (ms) & \underline{0.0019}                     & 0.4061 & 0.0677 & 1.0535 & \textbf{0.0018} \\ \hline
\end{tabular}}
\end{table}
%%%%%%%%%%%%%%%%%%%%%%%%%%%%%%%%%%%%%%%%%%%%%%%%%%%%%%%%
Table~\ref{tab:overhead} evaluates the inference overhead of \Design{} compared to ADCN \cite{ashfahani2023unsupervised}, LwF \cite{lwf2019Li}, DIF \cite{xu2023deep}, and PCA \cite{rios2022incdfm}. %, excluding OC-SVM \cite{Faber_2024} and LOF \cite{Faber_2024} due to poor performance. 
\Design{} offers the fastest inference time among continual learning methods. Out of novelty detection methods, \Design{} is second only to PCA. We attribute the efficiency of \Design{} to avoiding the clustering classification used by LwF and ADCN. %\Design{} instead uses PCA reconstruction, which is much quicker than comparing data points to clusters. In addition, 
The difference between \Design{} and PCA is minimal, only 0.0001 milliseconds slower, due to the additional but lightweight step of encoding the data. Considering that the average median flow duration across datasets is 27.77 milliseconds, the overhead introduced by \Design{} is negligible in the context of real-time traffic flow.

%In this section we analyze the inference overhead of \Design{} compared to ADCN\cite{ashfahani2023unsupervised}, LwF\cite{lwf2019Li}, DIF\cite{xu2023deep}, and PCA\cite{rios2022incdfm}. We do not include OC-SVM\cite{Faber_2024} and LOF \cite{Faber_2024} due to weak performance. Table~\ref{tab:overhead} shows the average inference time in milliseconds per sample across all datasets. \Design{} has the best inference time besides PCA. We attribute this good inference time to \Design{} not using clustering classification like LwF and ADCN. Evidently, PCA reconstruction utilized by \Design{} is more time efficient than having to compare a data point to all saved clusters. Compared to pure PCA reconstruction, \Design{} is only 0.0001 ms slower. This small increase in inference time is due to the only added computation at inference is encoding the data with the encoder, which is simply a 4 layer MLP. Across all datasets, the average median travel flow duration is 27.77 ms, and the dataset with the quickest median travel flow is UNSW with 4.29 ms. Therefore the overhead introduced by \Design{} is irrelevant compared to the speed of the traffic flow. 

%\label{section:ablation_study}
%To assess the impact of our design choices, we perform an ablation study. Our goal is to analyze (i) threshold function evaluation, and (ii) novelty detection algorithm selection. 

 

%\textbf{Threshold Function Evaluation:} AE, PCA, and \Design{} all require a threshold to classify an anomaly based on the anomaly score. In all previously reported results, we select a widely used threshold that maximizes the $F_{1}$ score on the test set, i.e., Best-F. %This is not realistic but was used to compare the effectiveness of these methods. In this section 
%Here, we analyze three different threshold methods, which we denote: Best-F \cite{su2019robust}, Top-k \cite{zong2018deep}, and validation percentile (ValPer). Best-F uses the threshold that maximizes the $F_{1}$ score on test set. Top-k utilizes the contamination ratio $r$ of the test set, such that $r$ is the percentage of anomalies within the test set. Top-k selects a threshold so that the percentile of data within the test set classified as anomalies is equal to $r$. ValPer utilizes a validation set of normal data, and selects a threshold such that 99.7\% (3 standard deviations) of the normal data is within this threshold. 
%ValPer is the most realistic method as it does not rely on any information from the test set. 
%A breakdown of the $F_{1}$ score results for the different threshold methods is show in Table~\ref{tab:thresholding_results} where the best within each category is bolded. Overall Best-F performs significantly better than the other threshold methods, which is obvious as Best-F is an upper-bound for threshold selection. However the significant gap highlights the importance of threshold selection. Most importantly, \Design{} still performs better than PCA and AE through all threshold methods. 

%%%%%%%%%%%%%%%%%%%%%%%%%%%%%%%%%%%%%%%%%%%%%%%%%%%%%%%
%\begin{table}[]
%    \centering
%    \caption{Threshold Function Evaluation}
%    \resizebox{.97\columnwidth}{!}{
%    \begin{tabular}{c|c|c|c|c}
%        \hline
%         Dataset & Stategy & Best-F & Top-k & ValPer\\
%         \hline
%         & PCA  & 70.9 & 4.03 & 3.56 \\
%         \cline{2-5}
%         X-IIoTID & AE  & 75.6 & 4.03 & 29.4 \\
%         \cline{2-5}
%         & \Design{} & \textbf{78.8} & \textbf{5.63} &  %\textbf{52.9} \\	
%         \hline
%        & PCA  & 85.6 &19.9 & 52.8\\
%         \cline{2-5}
%         WUSTL-IIoT & AE  & 79.6 &19.7 & 37.8\\
%         \cline{2-5}
%         & \Design{} & \textbf{88.2} & \textbf{21.1} & \textbf{55.6}\\	
%         \hline
%    \end{tabular}}
%    \label{tab:thresholding_results}
%\end{table}
%%%%%%%%%%%%%%%%%%%%%%%%%%%%%%%%%%%%%%%%%%%%%%%%%%%%%%%

% %%%%%%%%%%%%%%%%%%%%%%%%%%%%%%%%%%%%%%%%%%%%%%%%%%%%%%%
% \begin{figure}
%     \centering
%     \includegraphics[width=0.95\linewidth]{figures/novelty_ablation.pdf}
%     \caption{Comparison of \Design{} with PCA and AE novelty detection models}
%     \label{fig:novelty_ablation_results}
% \end{figure}
% %%%%%%%%%%%%%%%%%%%%%%%%%%%%%%%%%%%%%%%%%%%%%%%%%%%%%%%

% \textbf{Novelty Detection Algorithm Selection:} For \Design{}, we select PCA as the novelty detection algorithm. As shown in Figure~\ref{fig:novelty_methods_results}, both PCA and AE perform well for detecting intrusions. Therefore, we test both AE and PCA as the novelty detection methods for \Design{}. Figure~\ref{fig:novelty_ablation_results} illustrates the AVG performance of \Design{} with AE and PCA as the novelty detection models. It is evident that PCA outperforms AE, justifying our selection of this algorithm for novelty detection. This could be because the CFE utilizes SAEs, which generate features based on the same reconstruction loss used by AE to classify anomalies. It may be beneficial to use PCA as it deconstructs the input in a different manner, thereby identifying different features and functioning better in conjunction with the SAE-based CFE.

We present RiskHarvester, a risk-based tool to compute a security risk score based on the value of the asset and ease of attack on a database. We calculated the value of asset by identifying the sensitive data categories present in a database from the database keywords. We utilized data flow analysis, SQL, and Object Relational Mapper (ORM) parsing to identify the database keywords. To calculate the ease of attack, we utilized passive network analysis to retrieve the database host information. To evaluate RiskHarvester, we curated RiskBench, a benchmark of 1,791 database secret-asset pairs with sensitive data categories and host information manually retrieved from 188 GitHub repositories. RiskHarvester demonstrates precision of (95\%) and recall (90\%) in detecting database keywords for the value of asset and precision of (96\%) and recall (94\%) in detecting valid hosts for ease of attack. Finally, we conducted an online survey to understand whether developers prioritize secret removal based on security risk score. We found that 86\% of the developers prioritized the secrets for removal with descending security risk scores.
% \appendix
% \section{An example appendix} 
\section*{Acknowledgments}
This work was supported by HKRGC GRF (Project ID: 14306721), and Hong Kong Centre for Cerebro- Cardiovascular Health Engineering (COCHE).

\bibliographystyle{siamplain}
\bibliography{references}

\end{document}
