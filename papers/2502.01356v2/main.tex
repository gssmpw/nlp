\documentclass[onefignum,onetabnum]{siamonline171218}

\usepackage[utf8]{inputenc} % allow utf-8 input
\usepackage[T1]{fontenc}    % use 8-bit T1 fonts
\usepackage{color}
\usepackage{hyperref}       % hyperlinks
\usepackage{url}            % simple URL typesetting
\usepackage{makecell}
\usepackage{amsmath,amsfonts,amssymb}
\usepackage{mathtools}
\usepackage{pifont}
\usepackage{booktabs}       % professional-quality tables
\usepackage{amsfonts}       % blackboard math symbols
\usepackage{nicefrac}       % compact symbols for 1/2, etc.
\usepackage{microtype}      % microtypography
\usepackage{lipsum,bm}
\usepackage{graphicx}
\usepackage{ragged2e}
\usepackage{multirow}
\usepackage[labelfont=bf]{caption}
\usepackage{enumitem}
% \usepackage{bm}
\setlist[enumerate]{leftmargin=.5in}
\setlist[itemize]{leftmargin=.5in}
\setlength{\textfloatsep}{3pt}
\graphicspath{ {./images/} }
\def\etal{\textit{et al. }}
\newcommand{\han}[1]{{\color{red}{#1}}}
\newcommand{\enoch}[1]{{\color{blue}{#1}}}
\newcommand{\creflastconjunction}{, and~}
\newsiamremark{remark}{Remark}
\newsiamremark{hypothesis}{Hypothesis}
\crefname{hypothesis}{Hypothesis}{Hypotheses}
\newsiamthm{claim}{Claim}
\usepackage{bm}

\headers{Quasi-Conformal Convolution}{H. Zhang, T.L. Ip and L.M. Lui}

\title{Quasi-Conformal Convolution : A Learnable Convolution for Deep Learning on Riemann Surfaces
\thanks{Submitted to the editors DATE.
% \funding{This work was funded by the Fog Research Institute under contract no.~FRI-454.}
}}


\author{Han Zhang \thanks{Department of Mathematics, City University of Hong Kong, Hong Kong, China. (hzhang863-c@my.cityu.edu.hk)
}
\and Tsz Lok Ip \thanks{Department of Mathematics, Chinese University of Hong Kong, Hong Kong, China;(enochitl@link.cuhk.edu.hk)
}
\and Lok Ming Lui \thanks{Department of Mathematics, Chinese University of Hong Kong, Hong Kong, China;(lmlui@math.cuhk.edu.hk)
}}

\begin{document}
\maketitle
\begin{abstract}

To develop generalizable models in multi-agent reinforcement learning, recent approaches have been devoted to discovering task-independent skills for each agent, which generalize across tasks and facilitate agents' cooperation. However, particularly in partially observed settings, such approaches struggle with sample efficiency and generalization capabilities due to two primary challenges: (a) How to incorporate global states into coordinating the skills of different agents? (b) How to learn generalizable and consistent skill semantics when each agent only receives partial observations? To address these challenges, we propose a framework called \textbf{M}asked \textbf{A}utoencoders for \textbf{M}ulti-\textbf{A}gent \textbf{R}einforcement \textbf{L}earning (MA2RL), which encourages agents to infer unobserved entities by reconstructing entity-states from the entity perspective. The entity perspective helps MA2RL generalize to diverse tasks with varying agent numbers and action spaces. Specifically, we treat local entity-observations as masked contexts of the global entity-states, and MA2RL can infer the latent representation of dynamically masked entities, facilitating the assignment of task-independent skills and the learning of skill semantics. Extensive experiments demonstrate that MA2RL achieves significant improvements relative to state-of-the-art approaches, demonstrating extraordinary performance, remarkable zero-shot generalization capabilities and advantageous transferability.

 % Additional rewards transform the original MTRL problem into a multi-objective MTRL problem, and the coupling relationship between the outputs of SP and ACP further complicates the optimization process. To solve this challenge, TSAC assigns a virtual expected budget to convert the multi-objective MTRL into a constrained single-objective formulation and then employs the Lagrangian method to transform a constrained single-objective optimization into an unconstrained one. The multiplier in the Lagrangian method automatically adjusts the weights during the training process, promoting cooperation between SP and ACP.
\end{abstract}
\begin{IEEEImpStatement}
The Current policies trained by Multi-Agent Reinforcement Learning (MARL) predominantly rely on meticulously designed structured environments, which considerably constrain the agents' generalization capabilities across multitasking and cross-task skill reuse. In this paper, we design a novel masked autoencoders for MARL to coordinate the skills of different agents and learn generalizable and consistent skill semantics when each agent only receives partial observations. Experimental results demonstrate that our proposed MA2RL framework significantly enhances both the asymptotic performance and generalization capabilities of the generalizable models. Specifically, MA2RL introduces masked autoencoders tailored for MARL, aimed at enhancing generalizable models. The framework holds promise for inspiring further explorations into the generalization of multi-agent reinforcement learning.
\end{IEEEImpStatement}


% Note that keywords are not normally used for peerreview papers.
\begin{IEEEkeywords}
Multi-Agent reinforcement learning, generalization, self-supervised learning.
\end{IEEEkeywords}


\IEEEpeerreviewmaketitle
% 
% 
The widespread integration of communication networks and smart devices in modern control systems has increased the vulnerability of industrial systems to online cyber-attacks, e.g., Industroyer, Blackenergy, etc \citep{osti_1505628}.
% Modern control systems have seen a large push to include communication networks and smart devices to increase performance, made possible by improvements in communication device cost and energy consumption. This trend has been coupled with the usage of open-standard communication protocols among industrial control systems, making them vulnerable to online cyber-attacks such as Industroyer, Blackenergy, etc \citep{osti_1505628}. 
To counter this, methods have been developed to improve security by achieving attack detection, mitigation, and monitoring, among others \citep{sandberg2022secure}. This paper focuses on active attack diagnosis to mitigate stealthy attacks. 
%
%\subsection{Literature review}

Active diagnosis techniques rely on the inclusion of additional moduli to control systems
% inclusion within the control system of additional moduli 
to alter the behavior of the system compared to information known by the attacker. 
For instance, the concept of additive watermarking was introduced in \cite{mo2015physical}, where noise signals of known mean and variance are added at the plant and compensated for it at the controller. 
This compensation, however, is not exact, causing some performance degradation. Thus, trade-offs between performance and detectability  are necessary \citep{zhu2023detection}.
% A later work \citep{zhu2023detection} designs the watermark signal by trading performance for detection. Thus, although additive watermarking serves as a good detection scheme, they endure performance losses even in the nominal case. 

In encrypted control \citep{darup2021encrypted}, the sensor data is encrypted, sent to the controller, and then operated on directly. Encrypted input signals are sent back to the plant for decryption. Although encryption is widespread in IT security, in control systems it presents some concerns, such as the introduction of time delays \citep{stabile2024verifiable}, while it may present inherent weaknesses \citep{alisic2023model}.
% they are not preferred as they introduce time delays \citep{stabile2024verifiable} which can cause instability, and some encryption schemes can be very weak  \citep{alisic2023model}. 

In moving target defense \citep{griffioen2020moving}, the plant is augmented with fictitious dynamics, known to the controller. The plant output is transmitted to the controller along with the fictitious states over a network under attack. 
The additional measurements then aide in the detection of attacks. 
This comes at the cost of higher communication bandwidth needs, which increases rapidly with the dimension of the augmented systems.
% Since the dynamics of the fictitious dynamics are exactly known to the controller, the attack is detected easily. However, when the scale of the system increases, the communication bandwidth used by moving the target defense approach increases rapidly. 

Other recently proposed works include two-way coding \citep{fang2019two}, a weak encryuption technique, and dynamic masking \citep{abdalmoaty2023privacy}, which enhances privacy as well as security, have been shown to be effective against zero-dynamics attacks.
% Two-way coding \citep{fang2019two} and dynamic masking \citep{abdalmoaty2023privacy} are other recently proposed approaches. Two-way coding is another form of weak encryption technique whilst dynamic masking proposes an architecture that enhances both privacy and security. These schemes are shown to be effective against zero dynamics attacks but remain to be studied for other classes of attacks. 
% Recent extensions include \citep{mukherjee2021secure,ramos2024privacy}.
% Some other works which are related are \citep{mukherjee2021secure}, an extension of \cite{fang2019two}. The work \citep{ramos2024privacy} is an extension of moving target defense for multi-agent systems. 
Furthermore, filtering techniques for attack detection are proposed by \cite{murguia2020security,hashemi2022codesign,escudero2023safety}, while not focusing on stealthy attacks.
% The works \citep{murguia2020security,hashemi2022codesign,escudero2023safety} develop filtering techniques to guarantee safety, without being focused on stealthy covert attacks.

Multiplicative watermarking (mWM) has been proposed by the authors as a diagnosis technique \citep{ferrari2020switching}. mWM consists of a pair of filters on each communication channel between the plant and its controller; the scheme is affine to weak encryption, whereby ``encoding'' and ``decoding'' are done by changing signals' dynamic characteristics through inverse pairs of filters. This enables original signals to be recovered exactly, and thus does not lead to performance degradation.
% A multiplicative watermark is an affine to a weak encryption technique, through which the signal is ``encoded'' by a filter, changing its dynamic behavior. The use of inverse pairs means that the original signal can be recovered, through ``decoding'' via an inverse filter. As such, differently to techniques based on additive watermarking, no performance is lost due to the injection of noise, and there are no bandwidth limitations.

%\subsection{Contributions}
One of the critical features of multiplicative watermarking is that to detect stealthy attacks, the mWM filter parameters must be switched over time. In this paper, an algorithm to optimally design the mWM parameters after a switching event is presented, enhancing detection performance, without changing the switching time.
% This is done without changing the switching time, which is taken as given.

\textcolor{black}{
To formalize the filter design problem, we suppose the defender is interested in optimal performance against adversaries injecting covert attacks with matched system parameters \citep{smith2015covert}, including the mWM parameters prior to the switch. This scenario represents a worst case where malicious agents can take full control of the system while remaining undetected.
Thus, the attack strategy is explicitly included within the formulation of the closed-loop system, and the mWM filters are chosen by solving an optimization problem minimizing the attack-energy-constrained output-to-output gain (AEC-OOG) \citep{anand2023risk}, a variation of the output-to-output gain proposed in  \cite{teixeira2015strategic}.
}
The main contributions of this paper are:
% We consider an adversary injecting a covert attack with matched system parameters \citep{smith2015covert}, i.e., an attacker with full knowledge of the control system parameters, including those of the mWM filters before the switch. This scenario is taken as a worst case, as it has been shown that this class of attacks can be made stealthy. To quantitatively define a cost, the output-to-output gain (OOG) \citep{teixeira2015strategic} is leveraged,
% a metric introduced to evaluate the impact of an additive attack in a control system. %Specifically, OOG evaluates the worst-case performance loss that an attacker injecting an undetectable attack can obtain. 
% Here, the maximum performance loss caused by a stealthy adversary with limited energy is taken, the attack-energy-constrained OOG (AEC-OOG) \citep{anand2023risk}. The main contributions of this paper are:
\begin{enumerate}
%[label=\alph*.]
\item The problem of optimally designing the switching mWM filters is formulated as an optimization problem, with the AEC-OOG is taken as the objective;%where the AEC-OOG is taken as the impact metric; 
\item The worst-case scenario of a covert attack with exact knowledge of plant and mWM filter parameters is embedded within the design problem;
% The optimization problem is defined to incorporate the worst-case scenario of a covert attack with exact knowledge of plant and mWM filter parameters;
\item The feasibility of the optimization problem is shown to be dependent only on stability conditions; 
\item A solution scheme is proposed to promote randomization of the mWM filter parameters such that an eavesdropping adversary cannot remain stealthy.
\end{enumerate} 

This builds on the results of \cite{ferrari2020switching}, where the focus was on the design of the switching protocols, rather than the parameters themselves.
Compared to previous work \citep{gallo2021design}, this paper introduces an optimization problem which is always feasible (thanks to the use of AEC-OOG in the objective), while also considering a more sophisticated class of covert attacks, where the presence of watermark is known to the adversary. 
Moreover, this paper poses a different objective than \citep{zhang2023hybrid}; indeed, while \citep{zhang2023hybrid} provided a design strategy to ensure certain privacy properties, in this paper we address the problem of optimal parameter design following a switching event.


%\subsection{Organization}
The rest of the paper is organized as follows. 
After formulating the problem in Section~\ref{sec:PF}, we propose our design algorithm in Section~\ref{sec:main}, and analyze its properties. It is then evaluated through a numerical example in Section~\ref{sec:NE}, and concluding remarks are given Section~\ref{sec:Con}.
% We provide the problem background in Section~\ref{sec:PF}. We formulate the design problem in Section~\ref{sec:main}, together with an analysis of its properties. The proposed algorithm is evaluated through a numerical example in Section \ref{sec:NE}. Concluding remarks are offered in Section \ref{sec:Con}.
\section{Related Works}
\subsection{Computational Quasi-Conformal Mapping}
Computational quasi-conformal mapping is a powerful tool to control the geometric variation and topology in image science \cite{lam2014landmark} and surface processing \cite{levy2002least,gu2004genus}. Benefitting from the Beltrami representation, the mapping between two different domains can preserve good geometric properties like bijectivity and smoothness, through controlling the Beltrami coefficients with such representation of mappings. Driven by the motivation to preserve different geometric information, ways of parameterization methods are proposed~\cite{gu2003global}. Such convenient representations are also popular and succeed in the computational fabrication community~\cite{Soliman:2018:OCS,Crane:2013:RFC,panetta2019x}. With the capability to handle large deformations, the quasi-conformal method also succeeds in registration for images~\cite{lam2014landmark} and surfaces~\cite{choi2015fast} and segmentation with topology- and convexity prior~\cite{zhang2021topology,zhang2024qis}. In \cite{zhang2022nondeterministic,zhang2022new}, quasi-conformality is used for deformation analysis with uncertainties to study medical images for disease analysis. 


\subsection{Deformable Convolution}

Deformable convolution has been proposed as a solution to the limitations of the traditional convolution operation in Convolutional Neural Networks (CNNs). Jeon \etal proposed the Active Convolution (AC) \cite{jeon2017active}, which integrates a trainable attention mechanism into the convolution operation to adaptively select informative features for different input instances. Another related approach is the Spatial Transformer Network (STN) \cite{jaderberg2015spatial}, which introduces a learnable transformation module that can warp the input feature map based on a set of learnable parameters. Zhang \etal \cite{zhang2024learning,zhang2023deformation} extend it with a Relu-Jacobian regularization to make the produced mapping bijective. By introducing an explicit spatial transformation module, the STN allows the network to learn spatial transformations that can better align the input with the task at hand, leading to improved performance in tasks such as digit recognition and image classification. 

Building on the STN, Dai \etal proposed the Deformable Convolution (DCN) \cite{dai2017deformable}, which extends the idea of spatial transformation to the convolution operation itself, by introducing learnable offsets for each position in the convolutional kernel. This allows the DCN to dynamically adjust the sampling locations of the convolution kernel for each input instance, leading to improved performance on tasks such as object detection and semantic segmentation. However, the original DCN has limitations in handling large deformations and invariance to occlusion. To address these limitations, researchers have proposed several variations, such as the Deformable Convolution v2 (DCNv2) \cite{zhu2019deformable}, which introduces additional deformable offsets for the intermediate feature maps, and the Deformable RoI Pooling (DRoIPool) \cite{dai2017deformable}, which extends the DCN to the task of region-based object detection. However, Luo \etal found that the contribution of each pixel is not equal to the final results in DCN \cite{luo2016understanding}. These findings suggest the need for further improvements in the deformable convolution operation to address its limitations and maximize its performance.

\subsection{Geometric Learning}
In the field of geometric modelling, Bronstein \etal introduced manifold convolution with geodesic patch operators, demonstrating its success in various applications~\cite{bronstein2017geometric, masci2015geodesic}. Similarly, Boscaini \etal utilized an anisotropic heat kernel to define the convolution window, further contributing to the field~\cite{boscaini2016learning}. Other convolution definitions have also succeeded in registration tasks~\cite{bouritsas2019neural, gong2019spiralnet++}. Additionally, the MeshCNN framework by Hanocka \etal is noteworthy, as it redefined convolution using edges rather than vertices, offering a natural and straightforward approach to the concept~\cite{hanocka2019meshcnn}. Schonsheck \etal propose \cite{schonsheck2022parallel} Parallel Transport Convolution to enhance the translation invariance and allow the construction of compactly supported filters in manifold neural networks.
\section{Mathematical Background}
\subsection{Quasi-Conformal Geometry}

\begin{definition}[Quasi-conformal map]
A quasi-conformal map is a map $f: \mathbb{C} \rightarrow \mathbb{C}$ that satisfies the Beltrami equation
\begin{equation}
\frac{\partial f}{\partial \bar{z}}=\mu(z) \frac{\partial f}{\partial z}
\label{eq:beleq}
\end{equation}
for some complex-valued function named as Beltrami coefficient $\mu$ satisfying $\|\mu\|_{\infty}<1$ and $\frac{\partial f}{\partial z}$ is non-vanishing almost everywhere. The complex partial derivatives are given by
\begin{equation}
\frac{\partial f}{\partial z}:=\frac{1}{2}\left(\frac{\partial f}{\partial x}-i \frac{\partial f}{\partial y}\right) 
\quad \text{ and } \quad 
\frac{\partial f}{\partial \bar{z}}:=\frac{1}{2}\left(\frac{\partial f}{\partial x}+i \frac{\partial f}{\partial y}\right).
\end{equation}

\begin{figure}
    \centering
    \includegraphics[width=0.5\textwidth]{im/conformality.PNG}
    \caption{Illustration of how the Beltrami coefficient measures the conformality distortion of a quasi-conformal map}
    \label{fig:qcmap}
\end{figure}

\end{definition}
$\mu$ is the Beltrami representation, which is also called the Beltrami coefficient, of the quasi-conformal map $f$. It's worthy to mention that $\mu$ is a measure of non-conformality. Particularly, for a point $p$, the associated quasi-conformal map $f$ is conformal around a small neighbourhood of $p$ if $\mu(p)=0$. In this case, Equation \ref{eq:beleq} becomes the Cauchy-Riemann equation. This can also illustrate that conformality analysis of a quasi-conformal map $f$ can be simplified into the analysis of its associated Beltrami coefficient $\mu$. Infinitesimally, such a map $f$ can be rewritten as follows in a local neighbourhood around a point $p$:
\begin{equation}
\begin{aligned}
f(z) &=f(p)+f_{z}(p) z+f_{\bar{z}}(p) \bar{z} \\
&=f(p)+f_{z}(p)(z+\mu(p) \bar{z}).
\end{aligned}
\end{equation}
This further enhances our discussion before that $f$ is conformal when $\mu(p) = 0$. To explain the equation above, $f(p)$ is a translation, while $f_z(p)$ is a dilation. Since both of them are conformal, all the non-conformality of $f$ is brought by $D(z)=z+\mu(p) \bar{z}$. Hence, the Beltrami coefficient $\mu$ actually encodes the conformality of $f$. Analyzing quasi-conformal $f$ is equivalent to that for its associated Beltrami coefficient $\mu$. To be detail, the angle of maximal magnification is $\arg (\mu(p)) / 2$ with magnifying factor $1 + |\mu(p)|$; the maximal shrinking is the orthogonal angle $(\arg (\mu(p))-\pi) / 2$ with shrinking factor $1 - |\mu(p)|$. 

The maximal quasi-conformal dilation of $f$ is given by
\begin{equation}
K=\frac{1+\|\mu\|_{\infty}}{1-\|\mu\|_{\infty}}.
\end{equation}
Figure \ref{fig:qcmap} illustrates the geometry of a quasi-conformal map.

Another important relationship between a map and its Beltrami coefficients is the diffeomorphism property. By a norm constraint on $\mu$, the bijectivity of $f$ can be preserved which is explained by the following theory.

\begin{theorem}
If $f: \mathbb{C} \rightarrow \mathbb{C}$ is a $C^{1}$ map. Define 
\begin{equation}
\mu=\frac{\partial f}{\partial \bar{z}} / \frac{\partial f}{\partial z}.
\end{equation}
If $\mu$ satisfies $\left\|\mu_{f}\right\|_{\infty}<1$, then $f$ is bijective.
\end{theorem}

\begin{theorem}[Measurable Riemann mapping theorem \cite{gardiner2000quasiconformal}]
\label{them:RiemannMapping}
Suppose $\mu: \mathbb{C}\rightarrow\mathbb{C}$ is Lebesgue measurable satisfying $\|\mu\|_{\infty}<1$, then there exists a quasi-conformal mapping $f:\mathbb{C}\rightarrow \mathbb{C}$ in the Sobolev space $W^{1,2}$ that satisfies the Beltrami equation in the distribution sense. Furthermore, assuming that the mapping is stationary at $0, 1$ and $\infty$, then the associated quasi-conformal mapping $f$ is uniquely determined.
\end{theorem}

\begin{figure}
    \centering
    \includegraphics[width=0.8\textwidth]{im/beltramidifferential.pdf}
    \caption{Illustration of quasi-conformal mapping between Riemann surfaces.}
    \label{fig:bcdifferential}
\end{figure}

The Beltrami coefficient of a composition of quasi-conformal maps is related to the Beltrami coefficients of the original maps. Suppose $f: \Omega \rightarrow f(\Omega)$ and $g: f(\Omega) \rightarrow \mathbb{C}$ are two quasi-conformal maps with Beltrami coefficients $\mu_f$ and $\mu_g$ correspondingly. The Beltrami coefficient of the composition map $g \circ f$ is given by
$$
\mu_{g \circ f}=\frac{\mu_f+\frac{\overline{f_z}}{f_z}\left(\mu_g \circ f\right)}{1+\frac{\overline{f_z}}{f_z} \overline{\mu_f}\left(\mu_g \circ f\right)} .
$$

Quasi-conformal maps can also be defined between two Riemann surfaces. In this case, the Beltrami differential is used. A Beltrami differential $\mu(z) \frac{d\overline{z}}{d z}$ on a Riemann surface $S$ is an assignment to each chart ($U_\alpha, \phi_\alpha$) of an $L_{\infty}$ complex-valued function $\mu_\alpha$, defined on local parameter $z_\alpha$ such that
$$
\mu_\alpha \frac{d \overline{z_\alpha}}{d z_\alpha}=\mu_\beta \frac{d \overline{z_\beta}}{d z_\beta},
$$
on the domain which is also covered by another chart $\left(U_\beta, \phi_\beta\right)$. Here, $\frac{d z_\beta}{d z_\alpha}=\frac{d}{d z_\alpha} \phi_{\alpha \beta}$ and $\phi_{\alpha \beta}=\phi_\beta \circ \phi_\alpha$. An orientation preserving diffeomorphism $f: M \rightarrow N$ is called quasi-conformal associated with $\mu(z) \frac{d z}{d z}$ if for any chart ($U_\alpha, \phi_\alpha$) on $M$ and any chart $\left(U_\beta, \psi_\beta\right)$ on $N$, the mapping $f_{\alpha \beta}:=\psi_\beta \circ f \circ f_\alpha^{-1}$ is quasi-conformal associated with $\mu_\alpha \frac{d \overline{z}}{d z_\alpha}$. Readers are referred to \cite{gardiner2000quasiconformal,lehto1973quasiconformal} for more details about quasi-conformal theories.
\section{Adaptive Convolution on Riemannian Manifolds}

Convolution is a fundamental mathematical operator in mathematics, physics, and engineering. It combines two functions to illustrate how the characteristics of one function are modified by the other. The necessity of defining convolution stems from its applications in various contexts. In this section, we will define convolution on Riemann surfaces, starting with an investigation of convolution on general manifolds and progressing to the definition of convolution on simply connected surfaces through Quasi-conformal Convolution (QCC), which is applicable to many real-world situations.

\subsection{Convolution on Riemannian $n$-manifold}

\subsubsection{Convolution on Manifold}
Before providing a definition of convolution on manifolds, we first examine the standard convolution operation in Euclidean space, as outlined in the following definition.
% In an Euclidean space, the convolution operation can be given as the following definition.

\begin{definition}[Convolution]
For two functions $h, k : \mathbb{R}^n \to \mathbb{R}$, the convolution of $h$ and $k$ is defined as:
\begin{equation}
(h \ast k)(x) = \int_{\mathbb{R}^n} h(y) k(x - y) \, dy,
\end{equation}
where:
\begin{itemize}
    \item $x \in \mathbb{R}^n$ is the point at which the convolution is evaluated,
    \item $y \in \mathbb{R}^n$ is the integration variable,
    \item $k(x - y)$ translates $k$ to align it with $h$.
\end{itemize}
\end{definition}

Defining convolution on manifolds or surfaces is more complex than in Euclidean space due to the absence of a global linear structure on manifolds. In the Euclidean case, convolution involves translating the kernel function $k$ using the displacement vector $x - y$. However, this approach does not directly apply to manifolds. A suitable notion of displacement must be established before performing convolution on manifolds.
%Defining convolution on manifolds or surfaces is more complex than in Euclidean space due to the absence of a global linear structure on manifolds.  In the Euclidean case, convolution involves translating the function $k$ by $x-y$, a shift-invariant operation that does not directly apply to manifolds. Consequently, convolution on an $n$-manifold $\mathcal{M}$ can only be defined between a function on $\mathcal{M}$ and a kernel function on $\mathbb{R}^n$. Since the displacement vector $x-y$ is not well-defined when $x$ and $y$ are points on the manifold, it becomes essential to establish a suitable notion of displacement before performing convolution on manifolds.

\begin{definition}[Displacement function and displacement vector]
    Let $\mathcal{M}$ be a Riemannian $n$-manifold, and $U \subseteq \mathcal{M}$ be a subset of $\mathcal{M}$. A function $d: U\times U \to \mathbb{R}^n$ is a displacement function on $U$ if it satisfies the following properties:
    \begin{enumerate}
        \item For all $p, q \in U$, $d(p,q) = 0$ if and only if $p=q$.
        \item For all $p, q, r \in U$, $d(p,r) = d(p,q) + d(q,r)$.
    \end{enumerate}
    Then, the vector $d(p, q)$ is referred to as the displacement vector from $p$ to $q$.
    
    Moreover, if the functions $d(\cdot, q_0)$ and $d(p_0, \cdot)$ are orientation-preserving homeomorphisms from $U$ to subsets of $\mathbb{R}^n$ depending on $p_0$ and $q_0$ for all fixed $p_0, q_0 \in U$, then the displacement function $d$ is said to be regular.
\end{definition}

% \begin{theorem}
%     Suppose $d: \mathcal{M}\times\mathcal{M} \to \mathbb{R}^n$ is a displacement function, then $d$ satisfies:
%     \begin{enumerate}
%         \item For all $p \in \mathcal{M}$, $d(p,p) = 0$.
%         \item For all $p, q \in \mathcal{M}$, $d(p,q) = -d(q,p)$.
%         \item 
%     \end{enumerate}
% \end{theorem}

The above displacement function is introduced to replace the standard expression $ x-y $ in $ \mathbb{R}^n $. This displacement function closely mimics the fundamental property of translation symmetry \cite{schonsheck2022parallel}. According to our definition, the displacement between any two distinct points on a manifold is always nonzero, and the sum of vectors along a path aligns precisely with the vector directly connecting the path’s endpoints. By adhering to these properties, the kernel can be effectively translated across different points on the manifold.

With the displacement function $d$ on $\mathcal{M}$, we are now ready to give a general definition of manifold convolution.

\begin{definition}[Manifold convolution]\label{mani_conv}
Let $\mathcal{M}$ be a Riemannian $n$-manifold with a metric $g$, and let $h: \mathcal{M} \to \mathbb{R}$ be a manifold function with a kernel function $k: \mathbb{R}^{n}\to \mathbb{R}$. The convolution of $h$ and $k$ on $\mathcal{M}$ is defined as:

\begin{equation}\label{eq:mani_conv}
(h \ast_{\mathcal{M},d,g} k)(p) = \int_{\mathcal{M}} h(q) k(d(p, q)) \, dq,
\end{equation}

where:
\begin{itemize}
    \item $p, q \in \mathcal{M}$,
    \item $d: \mathcal{M}\times\mathcal{M} \to \mathbb{R}^n $ is a global displacement function on $\mathcal{M}$.
    % \item $d\nu(q)$ is the volume form on $\mathcal{M}$ derived from the metric $g$.
\end{itemize}
For simplicity, we denote $*_{\mathcal{M}, d,g}$ as $*_{d,g}$.
Moreover, $*_{d,g}$ is said to be regular if the displacement function $d$ is regular.

\end{definition}

%In Euclidean space, convolution relies on translating a kernel function $ k(x) $ into $ k(x-y) $. However, on a manifold, translation is not well-defined due to curvature. Therefore, a displacement function is introduced to replace the standard expression $ x-y $ in $ \mathbb{R}^n $. This displacement function should closely mimic the fundamental property of translation symmetry \cite{schonsheck2022parallel}. According to our definition, the displacement between any two distinct points on a manifold is always nonzero, and the sum of vectors along a path aligns precisely with the vector directly connecting the path’s endpoints. By adhering to these properties, the kernel can be effectively translated across different points on the manifold.

% These observations highlight the close relationship between the displacement function $d$ and volume form $d\nu$ derived from the Riemannian metric $g_{\mu\nu}$, as formalized in Definition \ref{mani_conv}. Ideally, $d\nu$ should be directly induced by the displacement function $d$, ensuring metric consistency throughout the integral. This relationship and its implications will be further explored in the next subsection.

Note that the manifold convolution above is not commutative as the manifold function $h$ is defined on $\mathcal{M}$ while kernel function $k$ is defined on $\mathbb{R}^n$. This lack of commutativity does not hinder the definition of the convolution operation for deep learning tasks on Riemann surfaces. 

% Before we proceed to the next section, let's make another remark about the volume form $d\nu$ and the Riemannian metric $g$ to make the discussion later more convenient.

Before we proceed to the next section, we shall emphasize that the convolution operator $*_{d,g}$ does depend on the Riemannian metric $g$ of $\mathcal{M}$ due to the standard definition of integration on manifolds used in Equation \ref{eq:mani_conv}. More details are discussed in the following remark.
\begin{remark}\label{rm:volume_form}
    Let $\mathcal{M}$ be a Riemannian $n$-manifold and let $h: \mathcal{M} \to \mathbb{R}$ be a manifold function. 
    Suppose $\mathcal{M}$ is covered by a collection of coordinate charts $\{(U_\alpha, \phi_\alpha)\}$, and let $\{\psi_\alpha\}$ be a partition of unity subordinate to the cover $\{U_\alpha\}$. In local coordinates, the {\it volume form $d\nu$ induced by $g$} is given by:
\begin{equation}
d\nu =  \sqrt{\det(g(x))} \, dx^1 \wedge dx^2 \wedge \cdots dx^n,
\end{equation}
\noindent where $g(x)=(g_{ij}(x))_{1\leq i,j\leq n}$. The manifold integral of $h$ is then defined as
    % \begin{equation}
    % \int_\mathcal{M} h \, d\nu = \int_\Omega h(\phi(x)) \sqrt{\det(g(x))} \, dx.
    % \end{equation}
    \begin{equation}
    \int_\mathcal{M} h \, d\nu = \sum_\alpha \int_{U_\alpha} \psi_\alpha(x) h(\phi_\alpha^{-1}(x)) \sqrt{\det(g(x))} \, dx.
    \end{equation}
  \end{remark}

%\begin{remark}
 %   Let $\mathcal{M}$ be a Riemannian $n$-manifold, and let $h: \mathcal{M} \to \mathbb{R}$ be a manifold function. For a diffeomorphism $\phi : \Omega \to \mathcal{M}$, where $\Omega \in \mathbb{R}^{n}$, the manifold integral could be defined as    
 %   \begin{equation}
  %  \int_\mathcal{M} h \, d\nu = \int_U h(\phi(x)) \sqrt{\det(g(x))} \, dx,    
   % \end{equation}
    %where $g=(g_{ij})$ and:  
    %\begin{equation}
    %g_{ij} = \langle \frac{\partial \phi}{\partial x_i}, \frac{\partial \phi}{\partial x_j} \rangle,
    %\end{equation}
    %we say the volume form $d\nu$ is derived by metric $g=(g_{ij})$ via $\phi$.
%\end{remark}

\subsubsection{Convolution on Manifold via Parameterization}

The concept of the convolution operation on a manifold is not well-established, primarily due to the curvature of the manifold. In the Euclidean domain, the plain convolution operator involves shifting the kernel function. This shifting is straightforward in Euclidean space, where the geometry is flat and the displacement from one point to another is well-defined. In this subsection, we introduce the idea of defining convolution on a Riemann surface on its 2D parametric flat domain.
%When defining convolution on a manifold, one is confronted with the challenges introduced by the manifold's curvature and the lack of a proper manifold metric definition. The typical convolution operation, which relies on shifting and integrating a kernel over a space, is straightforward in Euclidean spaces, where the geometry is flat and distances are well-defined. Therefore, it is natural to define a convolution on the parameterization domain.

\begin{definition}[Parametrized manifold convolution]
    Let $\mathcal{M}$ be a Riemannian $n$-manifold, and let $h: \mathcal{M} \to \mathbb{R}$ be a manifold function with a kernel function $k: \mathbb{R}^{n}\to \mathbb{R}$. Suppose there exists a bijective parametrization $\phi:\Omega \to \mathcal{M}$, where $\Omega \subset \mathbb{R}^n$. The parametrized manifold convolution of $h$ and $k$ on $\mathcal{M}$ is defined as:
    \begin{equation}
        (h *_\phi k)(p) = \int_\Omega h(\phi(y))k(\phi^{-1}(p)-y)dy.
    \end{equation}
    
    % \begin{align*}
    %     (h *_\phi g)(p) &= \int_\Omega h(\phi(y))g(\phi^{-1}(p)-y)dy\\
    %     &= \int_\mathcal{M} h(q)g(\phi^{-1}(p)-\phi^{-1}(q))d\phi^{-1}(q)
    % \end{align*}
    
    % $$(h *_\phi g)(p) = \int_\Omega h(\phi(y))g(\phi^{-1}(p)-y)dy$$
    % \int_\mathcal{M} h(q)g(\phi^{-1}(p) - \phi^{-1}(q))d\phi{-1}(q),$$
    % ($\phi$ doesn't need to be bijective here(?))
    \label{them:parameterconv}
\end{definition}

The definition above simplifies the computation of manifold convolution and provides a more intuitive viewpoint by performing convolution on an Euclidean domain. To explore the relationship between parameterized manifold convolution and manifold convolution, the following lemma is essential to demonstrate that a displacement function on a manifold gives rise to a surface parameterization, enabling us to define the parameterized manifold convolution.

% This allows us to consider a substantial subset of manifold convolutions, where the displacement function $d$ and metric $g$ are properly related, as parametrized manifold convolution, or vice versa, whichever is more intuitive.


% \begin{theorem}
%     Given a $n$-manifold $\mathcal{M}$, then there exists a displacement function $d:\mathcal{M} \times \mathcal{M} \to \mathbb{R}$ if and only if there exists a bijective parametrization $\phi: \Omega \to \mathcal{M}$ such that $$d(\phi(x),\phi(y)) = x-y$$ for all $x, y \in \Omega$, where $\Omega \subset \mathbb{R}^n$.
% \end{theorem}

\begin{lemma}\label{disp2phi}
     Let $\mathcal{M}$ be a Riemannian $n$-manifold. If there exists a displacement function $d:\mathcal{M} \times \mathcal{M} \to \mathbb{R}^n$, then there exists a bijective parametrization $\phi: \Omega \to \mathcal{M}$ such that
     \begin{equation}
         d(p,q) = \phi^{-1}(p)-\phi^{-1}(q)
     \end{equation}
     for all $p, q \in \mathcal{M}$, where $\Omega \subset \mathbb{R}^n$.

    Conversely, if there exists a bijective parametrization $\phi: \Omega \to \mathcal{M}$, where $\Omega \subset \mathbb{R}^n$, then a function $d:\mathcal{M} \times \mathcal{M} \to \mathbb{R}^n$ defined by 
    \begin{equation}
         d(p,q) = \phi^{-1}(p)-\phi^{-1}(q)
    \end{equation}
    for all $p, q \in \mathcal{M}$, is a displacement function on $\mathcal{M}$.
\end{lemma}

\begin{proof}
    $(\Rightarrow)$ Suppose there exists a displacement function $d:\mathcal{M} \times \mathcal{M} \to \mathbb{R}$. Pick any $p_0 \in \mathcal{M}$ and let $\Omega = \{d(p, p_0): p \in \mathcal{M}\} \subset \mathbb{R}^n$. Define $\tilde{\phi}: \mathcal{M} \to \Omega$ by $\tilde{\phi}(p) = d(p, p_0)$. Then
    \begin{equation}
        d(p, q) = d(p, p_0) - d(q, p_0) = \tilde{\phi}(p)-\tilde{\phi}(q).
    \end{equation}
    Note that this equation shows that $\tilde{\phi}$ is injective as $d(p,q)=0$ implies $p=q$, and the surjectivity of $\tilde{\phi}$ is guaranteed by the definition of $\Omega$, $\tilde{\phi}$ is therefore bijective. 
    % Suppose $\tilde{\phi}(p) = \tilde{\phi}(q)$ for some $p, q \in \mathcal{M}$, then
    % $$d(p, q) = d(p, p_0) - d(q, p_0) = \tilde{\phi}(p) - \tilde{\phi}(q) = 0,$$
    % which follows that $p=q$, and so $\tilde{\phi}$ is injective. Note that the surjectivity of $\tilde{\phi}$ is guaranteed by the definition of $\Omega$, $\tilde{\phi}$ is therefore bijective. For all $p, q \in \mathcal{M}$, we can check that
    % $$d(p, q) = d(p, p_0) - d(q, p_0) = \tilde{\phi}(p)-\tilde{\phi}(q).$$
    Thus $\phi = \tilde{\phi}^{-1}$ is the desired bijective parametrization.
    % $(\Leftarrow)$ It is a direct consequence of the definition of displacement functions.
    % $(\Leftarrow)$ Suppose there exists a bijective parametrization $\phi: \Omega \to \mathcal{M}$ such that $d(\phi(x),\phi(y)) = x-y$ for all $x, y \in \Omega$, where $\Omega \subset \mathbb{R}^n$. Define $d:\mathcal{M} \times \mathcal{M} \to \mathbb{R}$ by
    % $$d(p, q) = \phi^{-1}(p) - \phi^{-1}(q),$$
    % then we can
    
    $(\Leftarrow)$ Check that for all $p, q, r \in \mathcal{M}$,
    $$d(p, q) = 0 \iff \phi^{-1}(p) = \phi^{-1}(q) \iff p=q$$
    and
    $$d(p, r) = \phi^{-1}(p) - \phi^{-1}(r) = (\phi^{-1}(p) - \phi^{-1}(q)) + (\phi^{-1}(q) - \phi^{-1}(r)) = d(p, q) + d(q, r).$$
    Therefore, $d$ is a displacement function for $\mathcal{M}$.
    % , which obviously satisfies $d(\phi(x),\phi(y)) = x-y$ for all $x, y \in \Omega$.
\end{proof}

% (We need to choose one of the three theorems below)
% \begin{theorem}[original]
%     Let $\mathcal{M}$ be a Riemannian $n$-manifold and $*_\mathcal{M}$ be an operator. Then $*_\mathcal{M} = *_d$ for some displacement function $d$ on $\mathcal{M}$ if and only if $*_\mathcal{M} = *_\mathcal{\phi}$ for some bijective parametrization $\phi: \Omega \to \mathcal{M}$, where $\Omega \subset \mathbb{R}^n$.
% \end{theorem}

% \begin{proof}
%     ($\Rightarrow$) Suppose $*_\mathcal{M} = *_d$ for some displacement function $d$ on $\mathcal{M}$. By Lemma \ref{disp2phi}, there exists a bijective parametrization $\phi: \Omega \to \mathcal{M}$ such that $$d(p,q) = \phi^{-1}(p)-\phi^{-1}(q)$$ for all $p, q \in \mathcal{M}$, where $\Omega \subset \mathbb{R}^n$. Then for any manifold function $h:\mathcal{M} \to \mathbb{R}$ and kernel function $g: \mathbb{R}^n \to \mathbb{R}$, we have 
%     \begin{align*}
%         (h *_\phi g)(p) &= \int_\mathcal{M} h(q)g(\phi^{-1}(p)-\phi^{-1}(q))d\phi^{-1}(q) \\
%         &= \int_\mathcal{M} h(q)g(d(p,q))d\phi^{-1}(q) \\
%         &= (h *_d g)(p)
%     \end{align*}
%     where $\nu(q)$ is induced by $\nu = L^1 \circ \phi^{-1}$.

%     ($\Leftarrow$) By Lemma \ref{disp2phi}, the function $d(p,q) = \phi^{-1}(p)-\phi^{-1}(q)$ is a displacement function. Therefore, the result follows similarly.
% \end{proof}

\begin{theorem}\label{equiv_conv}
    Let $\mathcal{M}$ be a Riemannian $n$-manifold and $d$ be a displacement function on $\mathcal{M}$. Then there exists a bijective parametrization $\phi: \Omega \to \mathcal{M}$, where $\Omega \subset \mathbb{R}^n$, along with a metric $g$ of $\mathcal{M}$, such that $*_{d, g} = *_\phi$. Conversely, for any bijective parametrization $\phi: \Omega \to \mathcal{M}$, there exists a displacement function $d$ on $\mathcal{M}$ and a metric $g$ of $\mathcal{M}$ such that $*_{d, g} = *_\phi$.
\end{theorem}

\begin{proof}
    ($\Rightarrow$) Suppose $*_\mathcal{M} = *_{d,g}$ for some displacement function $d$ on $\mathcal{M}$. By Lemma \ref{disp2phi}, there exists a bijective parametrization $\phi: \Omega \to \mathcal{M}$ such that $$d(p,q) = \phi^{-1}(p)-\phi^{-1}(q)$$ for all $p, q \in \mathcal{M}$, where $\Omega \subset \mathbb{R}^n$. Consider $\phi^{-1}$ as the coordinate chart function and define the Riemannian metric $g = (\phi^{-1})^*g_{\mathbb{R}^n}$ as a pullback metric of $\mathcal{M}$, where $g_{\mathbb{R}^n}$ is the standard Euclidean metric. The volume form $d\nu$ can be obtained by Remark \ref{rm:volume_form}. 
    
    Since the distance $dg(p,q) = |\phi^{-1}(p)-\phi^{-1}(q)|$ for all $p,q \in \mathcal{M}$, $\phi^{-1}$ is an isometric mapping with respect to the metric $g$ and $\det(D\phi^{-1}) = 1$, therefore $dy = |\det(D\phi^{-1})| \, dq = dq$ for $\phi(y) = q$. For any manifold function $h:\mathcal{M} \to \mathbb{R}$, kernel function $k: \mathbb{R}^n \to \mathbb{R}$ and $p \in \mathcal{M}$, we now have 
    \begin{equation}
        \begin{aligned}
            (h *_\phi k)(p) &= \int_\Omega h(\phi(y))k(\phi^{-1}(p)-y)dy \\
            &= \int_\mathcal{M} h(q)k(\phi^{-1}(p)-\phi^{-1}(q))dq \\
            &= \int_\mathcal{M} h(q)k(d(p,q))dq \\
            &= (h *_{d,g} k)(p).
        \end{aligned}
    \end{equation}
    Therefore $*_{d, g} = *_\phi$.

    ($\Leftarrow$) By Lemma \ref{disp2phi}, the function $d(p,q) = \phi^{-1}(p)-\phi^{-1}(q)$ is a displacement function. Therefore, the result follows similarly.
\end{proof}


% **********************************NEED DISCUSSION HERE******************************************
% \begin{remark}
%     The parameterization $\phi$ is an isometric mapping with respect to the volume form $d\nu$, which is directly induced by $\phi$ in the construction.
% \end{remark}


% \begin{theorem}
%     Let $\mathcal{M}$ be a Riemannian $n$-manifold. An operator $*_\mathcal{M}$ is a generalized manifold convolution with respect to some displacement function $d$ on $\mathcal{M}$ if and only if it is a parametrized manifold convolution with respect to some bijective parametrization $\phi: \Omega \to \mathcal{M}$, where $\Omega \subset \mathbb{R}^n$.
% \end{theorem}


% \begin{theorem}
%     Let $\mathcal{M}$ be a Riemannian $n$-manifold and $*_\mathcal{M}$ be a manifold convolution on $\mathcal{M}$. Then there exists a bijective parametrization $\phi: \Omega \to \mathcal{M}$, where $\Omega \subset \mathbb{R}^n$, such that $*_\phi = *_\mathcal{M}$. (This only includes a single direction.)
% \end{theorem}
% \begin{proof}
%     ...
% \end{proof}

Theorem \ref{equiv_conv} establishes that the space of manifold convolutions, equipped with a displacement function $d$ and an associated metric $g$, is equivalent to the space of parameterized manifold convolutions. Hence, the set of all parameterized manifold convolutions encompasses a substantial portion of manifold convolutions, suggesting that many spatially defined manifold convolutions can be effectively represented within this parameterized framework. To conclude this section, we will investigate the regularity properties of both types of convolutions.
%Theorem \ref{equiv_conv} establishes that the space of manifold convolutions, equipped with a displacement function $d$ and an associated volume form $d\nu$ derived from a suitable metric, is equivalent to the space of parametrized manifold convolutions. Nonetheless, the set of all parametrized manifold convolutions encompasses a substantial portion of manifold convolutions, indicating that many manifold convolutions which are spatially defined can be effectively represented within this parametrized framework. To conclude this section, we investigate the regularity properties of both types of convolutions.

%Theorem \ref{equiv_conv} establishes that the space of manifold convolutions, equipped with a displacement function $d$ and a volume form $d\nu$ derived from a suitable metric, is equivalent to the space of parametrized manifold convolutions. Nonetheless, the set of all parametrized manifold convolutions represents a significant subset of manifold convolutions that are geometrically defined, in contrast to those that are not, such as spectral convolutions \cite{bruna2014spectral}. To conclude this section, we investigate the regularity properties of both types of convolutions. 


\begin{corollary}
    Let $\mathcal{M}$ be a Riemannian $n$-manifold equipped with a displacement function $d$ and a metric $g$, is parametrized by a function $\phi$ such that $*_{d,g} = *_\phi$. Then the following statements are equivalent:
    \begin{enumerate}
        \item $d$ is regular.
        \item $\phi$ is an orientation-preserving homeomorphism.
        \item $*_{d,g} = *_\phi$ is regular.
    \end{enumerate}
    \label{them:manifoldparametrized}
\end{corollary}

% \begin{corollary}
%     Suppose a Riemannian $n$-manifold $\mathcal{M}$ is parametrized by a diffeomorphic function $\phi$. Then the manifold convolution $*_\phi$ is regular.
%     % in the sense that there exists an equivalent convolution operator $*_\mathcal{M}$, such that the corresponding displacement function $d$ is regular.
% \end{corollary}
\begin{proof}
    Note that (1) and (3) are equivalent by definition, and (2) implies (1) as $*_{d,g} = *_\phi$ implies $d(p,q) = \phi^{-1}(p) - \phi^{-1}(q)$, which is regular if $\phi$ is an orientation-preserving homeomorphism.
    
    (1) $\Rightarrow$ (2): Suppose $d$ is regular, then using the construction of $\phi$ in the previous theorem, we immediately see $\phi$ is an orientation-preserving homeomorphism. Assume $*_{d,g} = *_{\phi_1}$ for some bijective parametrization $\phi_1: \Omega_1 \to \mathcal{M}$, where $\Omega_1 \subset \mathbb{R}^n$, then for all $p, q \in \mathcal{M}$,
    \begin{equation}
    \begin{aligned}
        &\phi^{-1}(p) - \phi^{-1}(q) = \phi_1^{-1}(p) - \phi_1^{-1}(q) \\
        \Rightarrow \quad &\phi^{-1}(p) - \phi_1^{-1}(p) = \phi^{-1}(q) - \phi_1^{-1}(q) \\
        \Rightarrow \quad &\phi^{-1} - \phi_1^{-1} \equiv c \qquad \text{for some constant vector } c \in \mathbb{R}^n.
    \end{aligned}
    \end{equation}
    Hence $\phi_1$ is also an orientation-preserving homeomorphism. 
\end{proof}

\subsection{Convolution on Riemann Surfaces}
In this work, we focus our problems on simply connected open surfaces embedded in $\mathbb{R}^3$, now we will move on to describe how we can define convolution on Riemann surfaces.

\subsubsection{Conformal Convolution}
A natural and useful approach to produce a parametrized manifold convolution is to employ a conformal parametrization of the manifold. A conformal parametrization $\phi$ is a map from a domain $\Omega \subset \mathbb{R}^2$ to the manifold $\mathcal{M} \subset \mathbb{R}^3$ that preserves angles. By mapping the curved surface to a flat Euclidean space, we can take advantage of the well-established theory of Euclidean convolutions. 

To define the convolution of a manifold function $h : \mathcal{M} \to \mathbb{R}$ on a $2$-manifold $\mathcal{M}$ and a kernel function $k : \mathbb{R}^2 \to \mathbb{R}$, we begin by pulling the function back to the parameter domain $\Omega$ using the conformal parametrization $\phi$. This transforms the problem of manifold convolution into a more manageable Euclidean convolution problem. Specifically, for each function $h$ defined on the surface and $k$ defined on $\mathbb{R}^2$, we define the pullback function $\tilde{h}$ on the flat domain $\Omega$ as follows:
\begin{equation}
\tilde{h}:=\phi^* h = h \circ \phi.
\end{equation}
% *********************************************************************************
% \enoch{(A more standard notation should be $\phi^* h = h \circ \phi$. How about we write $\tilde{h}:=\phi^* h = h \circ \phi$?)\\}
% *********************************************************************************
The pullback functions $\tilde{h}$ are now defined on the Euclidean space $\Omega \subset \mathbb{R}^2$, where the convolution operation can be performed using the standard Euclidean formulation.

The convolution of $\tilde{h}$ and $k$ in the Euclidean domain $\Omega$ is then defined as:
\begin{equation}
(\tilde{h} \ast k)(x) = \int_{\Omega} \tilde{h}(y) k(x - y) \, dy,
\end{equation}
where the integral is taken over the domain $\Omega \subset \mathbb{R}^2$. This is the classical convolution in Euclidean space, which is computationally efficient and well-understood.

Once the convolution is computed in the Euclidean domain, the result must be mapped back to the manifold $\mathcal{M}$. This is done by applying the inverse of the conformal parametrization $\phi^{-1}$, which maps the Euclidean result back onto the manifold's coordinates. The convolution on the manifold is then given by:
\begin{equation}
(h \ast_\phi k)(p) = (\tilde{h} \ast k)(\phi^{-1}(p)),
\end{equation}
where $p \in \mathcal{M}$ is a point on the manifold, and $\phi^{-1}(p)$ returns the corresponding point in the parameter domain $\Omega$.

Then, the manifold convolution is reduced to a Euclidean convolution performed in the parametrized space $\Omega$, followed by a pullback to the manifold using the inverse of the conformal map. The final formal definition is then written as:

\begin{definition}[Conformal Convolution]
Let $\mathcal{M} \subset \mathbb{R}^3$ be a 2-manifold, and let $\phi: \Omega \to \mathcal{M}$ be a conformal parametrization, where $\Omega \subset \mathbb{R}^2$ is a domain in the Euclidean plane. Let $h: \mathcal{M} \to \mathbb{R}$ be a manifold function with a kernel function $k: \mathbb{R}^{n}\to \mathbb{R}$. The conformal convolution of $h$ and $k$ is defined as:
\begin{equation}
\begin{aligned}
(h \ast_\phi k)(p) &= \int_\mathcal{M} h(q) k(\phi^{-1}(p) - \phi^{-1}(q)) dq \\
&= \int_{\Omega} \tilde{h}(y) k(x - y) dy \\
&= (\tilde{h} \ast k)(\phi^{-1}(p)),
\end{aligned}
\end{equation}
where:
\begin{itemize}
    \item $p,q \in \mathcal{M}$,
    \item $x=\phi^{-1}(p), y=\phi^{-1}(q) \in \Omega$,
    \item $\tilde{h} = h \circ \phi$ is the pullback function,
    \item $g = (\phi^{-1})^* g_{\mathbb{R}^2}$ is the Riemannian metric of $\mathcal{M}$.
\end{itemize}
\label{def:conformalconv}
\end{definition}

It is also important to note that, according to the Riemann Mapping Theorem (Theorem \ref{them:RiemannMapping}), the conformal parameterization is unique when three points are fixed. In practice, by mapping the surface to a disk or rectangle while fixing the boundary points, the parametrization becomes determined. The conformal convolution is a special case of parametrized convolution, and thus a manifold convolution by taking $d(p,q) \coloneqq   \phi^{-1}(p) - \phi^{-1}(q)$ and $d\nu$ derived from metric $g$  via $\phi^{-1}$, according to Theorem \ref{equiv_conv}. 

Although conformal parametrization retains nice geometric properties once the surface is mapped to the 2D domain, there is no evidence that it is the best parametrization to define a parametrized convolution. As a special case of parametrized convolution, conformal convolution is too restrictive for advanced usage, such as when implemented into deep learning tasks. Therefore, a much more flexible convolution, namely Quasi-Conformal Convolution, will be introduced in the next subsection.

\subsubsection{Quasi-Conformal Convolution}

The proposed Quasi-Conformal Convolution (QCC) is to define convolution operation on manifolds using quasi-conformal mappings. Quasi-conformal theory offers a mathematically robust framework for studying deformations between surfaces while preserving local geometric structures. By leveraging this theory, QCC extends the convolution operation to non-Euclidean domains such as manifolds, enabling deep learning methods to process irregular and geometrically distorted data.

Here, similar to how we define parametrized convolution, we have the following definition for Quasi-conformal Convolution.

\begin{definition}[Quasi-conformal Convolution]
Let $\mathcal{M} \subset \mathbb{R}^3$ be a 2-manifold, and let $\phi: \Omega \to \mathcal{M}$ be a conformal parametrization, where $\Omega \subset \mathbb{R}^2$ is a domain in the Euclidean plane. Let $h: \mathcal{M} \to \mathbb{R}$ be a manifold function with a kernel function $k: \mathbb{R}^{n}\to \mathbb{R}$.  Let $f: \Omega \to \Omega$ be a quasi-conformal mapping. The quasi-conformal convolution of $h$ and $k$ with respect to $\phi$ and $f$ is defined as:

\begin{equation}
\begin{aligned}
(h \ast_{\phi, f} k)(p)
&= \int_{\mathcal{M}}h(q)k(f\circ\phi^{-1}(p) - f\circ\phi^{-1}(q)) dq\\
&= \int_{\Omega} h\circ\phi(y) k(f(x) - f(y)) \, df(y) = \int_{\Omega} \tilde{h}(y)k(f(x) - f(y)) \, df(y)\\
&= \int_{\Omega} \tilde{h}\circ f^{-1}(y')k(x' - y') \, dy' = \int_{\Omega} h^\#(y')k(x' - y') \, dy'\\
&= h^\# \ast k(f\circ\phi^{-1}(p))
\label{eq:parametrizedQCC}
\end{aligned}
\end{equation}

where:
\begin{itemize}
    \item $p,q \in \mathcal{M}$ is a point on the manifold,
    \item $p = \phi(x), q = \phi(y) \in \Omega$,
    \item $x'=f(x), y'=f(y) \in \Omega$,
    \item $f:\Omega\to\Omega$ is a quasi-conformal mapping on the parametrized domain $\Omega$,
    \item $\tilde{h}=h\circ\phi$ is the pullback function,
    \item $h^\#=\tilde{h}\circ f^{-1}$ is the transformed pullback function,
    \item $g = (f \circ \phi^{-1})^* g_{\mathbb{R}^2}$ is the Riemannian metric of $\mathcal{M}$.
\end{itemize}
\end{definition}

Recall that, according to the Riemann Mapping Theorem (Theorem \ref{them:RiemannMapping}), the conformal parameterization is unique when three points are fixed which is generally ensured through the fix the boundaries. Therefore, the subscript $\phi$ of the operation $\ast_{\phi, f}$ can be omitted into $\ast_{f}$.

\begin{theorem}
    Let $\mathcal{M} \subset \mathbb{R}^3$ be a 2-manifold and $*_\mathcal{M}$ be a parametrized manifold convolution. Then $*_\mathcal{M}$ is regular if and only if $*_\mathcal{M}$ is a quasi-conformal convolution.
\end{theorem}
\begin{proof}
    The mapping $\phi: \Omega \to \mathcal{M}$ is quasi-conformal if and only if it is an orientation-preserving homeomorphism, which is equivalent to saying that the parametrized manifold convolution $*_\phi$ is regular.
\end{proof}

\begin{remark}
    Quasi-conformal Convolution is a parametrized manifold convolution with parametrization $\phi \circ f^{-1}$.
\end{remark}

The theorem and remark above show that we can generalize any regular parametrized manifold convolution on Riemann surfaces into quasi-conformal convolution, through which a substantial subset of manifold convolution could be represented. As different Quasi-conformal parameterizations would yield distinct convolution operators on the Riemann surface, one can find the best convolution operator for a specific task by optimizing the corresponding Quasi-conformal parameterization. Within this framework, we can identify the most effective convolution operator on a Riemann surface by optimizing the Quasi-conformal parameterization. 

By representing parametrized manifold convolutions through quasi-conformal mappings, learning the convolution operations in neural networks becomes feasible. In the next section, we will develop a deep neural network framework to learn the optimal Quasi-conformal parameterization associated with the best convolution operator for a given task. Quasi-conformal mappings preserve local geometric structures while allowing controlled deformations. This characteristic ensures that the convolution operation aligns with the intrinsic geometry of the manifold, enabling more robust and effective feature extraction when implementing quasi-conformal convolution into deep neural networks.

\begin{figure*}
	\centering
	\includegraphics[width = \linewidth]{figure/AgentArena.pdf}
	\caption{\textbf{Stock Trading Workflow in \textit{Agent Trading Arena}.} 
	\textbf{Top:} Workflow of a trading day, including preparation, trading, and post-trading reflection. Agents discuss insights in the chat pool, analyze market trends, execute trades, and refine strategies based on performance.  
	\textbf{Bottom:} Example of agents' interactions in the chat pool and dynamic strategy updates.}
	\label{fig:AgentArena}
	\vspace{-3pt}
\end{figure*}

\section{Proposed Method}

% 核心部分visual representation,

To mitigate the influence of human prior knowledge and memory, we designed a closed-loop economic system~\citep{guo2024economics} called the \textit{Agent Trading Arena}, a zero-sum game simulating complex, quantitative real-world scenarios. The simulation workflow is illustrated in \autoref{fig:AgentArena} and further detailed in \autoref{appendix_arena}. In the \textit{Agent Trading Arena}, agents can invest in assets, earn dividends from holding assets, and pay daily expenses using virtual currency. The agent with the highest total return wins the game.

\subsection{Agent Trading Arena}

\paragraph{Structure of Agent Trading Arena.} 

To eliminate external knowledge biases, asset prices are determined by a bid-ask system, reflecting the prices at which buyers and sellers are willing to transact. The system evolves solely based on agents' actions and interactions, without external influences. This design ensures that the outcomes of agents' actions are not immediately apparent but unfold gradually, influenced by other agents' decisions.

To encourage active participation, a dividend mechanism is introduced. There are two primary sources of income in this system: capital gains from asset price differentials and dividends from holding assets. Dividends for each asset are distributed according to a predefined ratio, serving as an implicit anchor for asset prices. Agents holding more low-cost assets receive higher dividends. To prevent passive asset holding until the end of the game, agents must pay a daily capital cost proportional to their total wealth. These expenses are offset by asset dividends, and only agents with sufficient low-cost assets can cover costs. Under the pressure of significant daily expenses, agents must act swiftly and strategically, triggering frequent trades and price fluctuations to stimulate market activity. This dynamic mechanism ensures fairness in the zero-sum game while preventing agents from relying on fixed strategies to find optimal solutions.

\vspace{-3pt}

\paragraph{Agents Learn and Compete in Arena.}

The zero-sum game structure is crucial to eliminating the possibility of a universally optimal strategy. In fixed scenarios with a static optimal solution, agents could rely on predefined rules or memory-based approaches, bypassing adaptive decision-making. The zero-sum game ensures that there is no universally correct solution, with outcomes evolving dynamically based on agent interactions and competition. This design forces agents to continually adapt, learn from feedback, and develop context-dependent strategies, promoting deeper environmental exploration and preventing reliance on static or memory-driven solutions.

In the \textit{Agent Trading Arena}, agents are unaware of implicit rules, except for the objective to maximize their virtual wealth throughout the simulation. To win this zero-sum game, agents must effectively learn from experience, decipher hidden game rules, and develop strategies to counter competitors. This requires the ability to comprehend numerical feedback, formulate enduring strategies, and make informed decisions. Unlike other mathematical reasoning problems, the results of their actions unfold gradually and dynamically. Moreover, agents are easily misled by erroneous information from competitors, hindering their ability to discern strategic cues from competitors' textual data. Importantly, agents remain unaware of these implicit rules, so applying real-world knowledge does not benefit their performance. Therefore, agents must rely on experiential learning to decipher the hidden game rules and ultimately achieve victory.

\subsection{Types of Numerical Data Input}

\paragraph{Limitations of Textual Numerical Data.}

In the \textit{Agent Trading Arena}, the generated stock data is stored in numerical format. When used directly as input to an LLM, the models often struggle to interpret numerical data accurately or make sound decisions. To mitigate this, we convert the data into textual formats~\citep{numerical_text, long_text}, enhancing semantic features and clarifying output requirements to improve the models' understanding. During interactions, the LLMs process stock prices, trading volumes, and market indices presented as textual numerical data.

\begin{figure*}
	\centering
	\includegraphics[width = \linewidth]{figure/v_t.pdf}
	\caption{\textbf{Textual and Visual Representations of Corresponding Inputs and Outputs.} The left images display the agent’s Buy and Sell trading records, daily trade prices, and K-line charts for three stocks. The output from visual inputs (bottom right) captures overall stock trends and long-term behavior, while the output from textual inputs (top right) focuses on specific current prices.}
	\label{textual_visualized}
	\vspace{-3pt}
\end{figure*}

However, this textual approach reveals significant limitations. While the data is presented clearly, LLMs tend to focus excessively on specific values rather than identifying long-term trends or global patterns. They also struggle with understanding correlative relations and percentage changes, limiting their ability to assess differences and identify connections between data points. When analyzing time-series data with complex patterns, LLMs often fixate on individual data points, overlooking overarching relations. This issue is evident in the analysis output in the top-right corner of \autoref{textual_visualized}, where LLMs' focus on individual values impedes their ability to generalize, reducing their capacity to extract meaningful global insights.

Additionally, LLMs often overemphasize recent data while undervaluing historical information, even when prompted to consider its importance. This prevents them from effectively integrating past data and recognizing long-term patterns, complicating their understanding of numerical relations and trends. These challenges highlight the need for improved mechanisms to process numerical relations, identify global trends, and derive deeper insights from textual numerical data.

\vspace{-3pt}

\paragraph{Potential of Visual Numerical Data.}

Since textual numerical data often leads LLMs to focus on local details while neglecting broader relations, we investigated whether visual representations, such as scatter plots, line charts, and bar charts, could help LLMs better understand overall trends, similar to human reasoning. Thus, we transition from textual numerical data inputs to visualized formats ~\citep{storyllava}. As demonstrated in the bottom-right corner of \autoref{textual_visualized}, visual representations enable LLMs to more effectively grasp global trends, patterns, and relations that are often difficult to discern from textual numerical data alone.

These findings highlight the advantages of structured, visual numerical data, indicating that this format allows LLMs to more intuitively and comprehensively understand complex data, better capturing overall fluctuations, whereas text tends to focus on local details. By combining visualization and textual representations, LLMs not only overcome the challenges of relations in time-series data but also demonstrate better performance in identifying long-term trends and global patterns, while still attending to local details.

\subsection{Reflection Module}

We propose a strategy distillation method, illustrated in \autoref{fig:reflection}, that delivers real-time feedback to LLMs by analyzing both descriptive textual and visual numerical data. This enables the generation of new strategies and optimization of action plans. The approach allows agents to evaluate their results, refine strategies, and adapt continuously based on feedback. The process begins with assessing the day’s trajectory memory and associated strategies using an evaluation function. The strategic generation process leverages contrastive analysis of peak and nadir performers from the evaluation phase, creating bidirectional learning signals that inform subsequent iterations. This iterative cycle ensures continuous strategy evolution, fostering sustained improvement in decision-making.

\begin{figure}[t]
	\centering
	\includegraphics[width = \linewidth]{figure/reflection.pdf}
	\caption{\textbf{Design of the Reflection Module.} The process evaluates daily trajectory memory and strategies (top right), then generates new strategies (center) based on evaluation, environmental feedback (bottom right), and feedback from the 5 top- and bottom-performing strategies. Stock visualization (bottom left) enhances reflection, driving continuous improvement.}
	%The process evaluates daily trajectory memory and strategies, generating new strategies based on positive and negative feedback from the top- and bottom-performing strategies. Stock visualizations (bottom left) further enhance the reflection process, reinforcing continuous strategy refinement.}
	\label{fig:reflection}
	\vspace{-3pt}
\end{figure}

% We propose a strategy distillation method, illustrated in \autoref{fig:reflection}, that provides real-time feedback to LLMs by analyzing both descriptive textual and visualized numerical data. This enables the generation of new strategies and the optimization of action plans. The approach allows agents to assess their results, refine strategies, and continuously adapt based on feedback. The process begins by evaluating the day's trajectory memory and associated strategies using an evaluation function. From this assessment, new strategies are generated by selecting the top-performing and lowest-performing strategies, offering both positive and negative feedback. This iterative cycle ensures continuous strategy evolution, driving sustained improvement in decision-making.

The reflection module plays a crucial role in refining strategies by offering real-time feedback. It analyzes both descriptive textual and visual numerical data to generate new strategies and optimize action plans. Within the \textit{Agent Trading Arena}, the reflection module is triggered regularly to consolidate daily trading records and evaluate the effectiveness of strategies, refining both successful and unsuccessful experiences to guide future decisions. Ineffective strategies are stored in a strategy library for future reference, allowing agents to review and learn from past experiences. Further details can be found in \autoref{appendix_arena}.


\newpage
\section{Experiment}\label{sec-experiment}
\subsection{Experimental Setup}
We briefly introduce experimental settings to verify our proposed MoR, including Datasets \& Baselines, Implementation Details, and Evaluation Metrics. More details are in Appendix~\ref{app-expr-setting}.

\textbf{Datasets \& Baselines:} We use three TG-KBs from STaRK~\cite{wu2024stark} covering three knowledge domains, including E-commerce Products (Amazon), Academic Papers (MAG), and Biomedicine (Prime). We compare our MoR with baselines established by~\citet{wu2024stark} and categorize them into textual/structural/hybrid-based ones. More recent state-of-the-art hybird retrieval approaches fro TG-KBs such as KAR~\cite{xia2024knowledge} and MFAR$^{*}$~\cite{li2024multi} are also compared.


\textbf{Implementation Details:} 
To enhance the planning capability of our planning module, we fine-tune the Llama 3.2 (3B) on 1000 sampled queries with their corresponding ground-truth planning graphs, serving as the textual graph generator. In the absence of ground-truths, we synthesize them using LLMs. For the Prime dataset, we empirically find that directly prompting LLMs can hardly generate accurate planning graphs due to the lack of biomedical domain knowledge~\cite{Shen2024TagLLMRG}. Therefore, we adopt an alternative approach. First, we instruct LLMs to extract triplets from each query and then construct the planning graphs by merging triplets with shared entities. 
During mixed traversal, textual matching can be implemented using any lexical or semantic methods. For this study, we employ BM25 for Amazon and MAG and fine-tune a contriever to complement the biomedical knowledge for Prime.
To initialize the structural traversal, we employ textual matching to locate the top 5 nodes that are most relevant to the query as seeds. Additionally, at each layer, we incorporate the top 10 nodes retrieved via textual matching and append them to the current candidate set for the next round of traversal. Notably, due to the uncertainty of LLMs, the generated planning graphs can be invalid. In this case, we will directly conduct textual matching to retrieve candidates. For our ablations without reranker, we employ Ada-002~\cite{wu2024stark} with cosine similarity as the scorer to rank candidates for evaluating performance.

\textbf{Evaluation Metrics:}
We follow~\citet{wu2024stark} for evaluation by reporting Hit@1 (H@1), Hit@5 (H@5), Recall@20 (R@20), and mean reciprocal rank MRR to evaluate in the full spectrum. 


 

\newpage
\subsection{Overall Retrieval Performance}
We compare MoR with other baselines on three TG-KBs in Table~\ref{tab-merged}. Generally, hybrid methods, AvaTAR, KAR, MFAR$^{*}$, and our MoR, achieve better performance than purely textual or structural methods owing to their ability to integrate both structural and textual knowledge. 
Among all baselines, our proposed MoR achieves the overall best performance with a substantial margin on average, with the first ranking on MAG and the second ranking on Amazon/Prime datasets. This demonstrates the effectiveness of our proposed mixture of structural and textual knowledge retrieval. 
Textual retrieval performs better on Amazon than on MAG, suggesting that Amazon queries rely more on textual knowledge. In contrast, its weaker performance on MAG is due to MAG's lower textual richness and stronger structural signals. This disparity aligns with the distribution analysis presented by~\citet{wu2024stark} and supports our hypothesis that queries in different TG-KB datasets require varying desires for textual and structural knowledge. Meanwhile, structural retrieval methods such as conventional knowledge graph-based ones perform poorly because they are designed for graphs with minimal textual information compared to TG-KBs.
Different from Amazon and MAG, all existing methods without supervised tuning (e.g., Ada-002) exhibit significantly lower performance on Prime. This is due to the extreme domain expertise required in biology, where word-count-based, pre-trained textual similarity-based, and even more powerful LLMs are all poorly applicable here. Through fine-tuning, MFAR$^{*}$ and our proposed MoR generally achieve better performance, demonstrating the necessity of domain-specific knowledge for answering queries in knowledge-intensive domains. 




\newpage
\subsection{Ablation Study}
After verifying the superiority of MoR, we conduct ablation studies to assess its different components, including module and feature ablation.

\subsubsection{Module Ablation}


To assess the contribution of each module in MoR, namely, Text Matching-based Retrieval, Neighborhood-Fetching-based Structural Retrieval, and Reranker, we conduct a series of ablation experiments. First, we remove the Reranker, resulting in the variant MoR$_{\text{w/o R}}$. On top of that, we further separately eliminate Text Retrieval and Structural Retrieval, yielding MoR$_{\text{w/o RT}}$ and MoR$_{\text{w/o RS}}$, respectively.
As shown in Table~\ref{tab-merged}, the complete MoR framework consistently achieves the highest performance across all datasets, demonstrating the synergistic effect of the Textual Retriever, Structural Retriever, and Reranker.
After removing Reranker, MoR$_{\text{w/o R}}$ exhibits a consistent performance drop across all datasets and evaluation metrics. This underscores the importance of the Reranker in refining retrieval by filtering noisy candidates from the intermediate reasoning stage. 
Eliminating Text Retrieval, i.e., MoR$_{\text{w/o RT}}$, leads to a notable performance drop on Amazon but an unexpected improvement on MAG. This suggests that while textual knowledge benefits Amazon, it introduces misleading hard negatives that compromise the ranking method (e.g., Ada-002) for MAG. Conversely, removing Structural Retrieval, MoR$_{\text{w/o RS}}$, results in a slight performance decrease further on MAG, reinforcing the importance of structural knowledge in MAG-related queries.
%
These results underscore the Reranker's crucial role in adaptively harmonizing, balancing, and selecting knowledge from both structural and textual retrieval experts.






\begin{table}[t!]
\small
\setlength\tabcolsep{4.5pt}
\centering
\begin{tabular}{l|ccc|cccc}
\toprule
\textbf{Dataset} &\textbf{TF} & \textbf{SF} & \textbf{TI} & \textbf{H@1} & \textbf{H@5} & \textbf{R@20} & \textbf{MRR} \\ \midrule
\multirow{7}{*}{\textbf{MAG}} 
& \cmark & \xmark & \xmark & 48.96 & 73.02 & 72.44 & 59.79 \\
&      \xmark            & \cmark       &         \xmark         & 18.79 & 41.91 & 52.85 & 29.84 \\
&        \xmark          &         \xmark         & \cmark       & 18.16 & 41.53 & 52.78 & 29.31 \\
\cline{2-8}
& \cmark       & \cmark       &    \xmark              & 58.04 & 77.14 & 74.42 & 66.75 \\
& \cmark       &        \xmark          & \cmark       & \underline{58.16} & \underline{77.59} & \underline{74.96} & \underline{66.85} \\
&          \xmark        & \cmark       & \cmark       & 17.93 & 38.01 & 46.79 & 27.48 \\
\cline{2-8}
& \cmark       & \cmark       & \cmark       & \textbf{58.19} & \textbf{78.34} & \textbf{75.01} & \textbf{67.14} \\ \midrule
\multirow{7}{*}{\textbf{Amazon}}    
& \cmark       &      \xmark            &       \xmark           & \underline{51.21} & \underline{74.05} & \underline{59.79} & \underline{61.27} \\
&        \xmark          & \cmark       &      \xmark            & 8.09  & 24.48 & 25.62 & 16.94 \\
&         \xmark         &      \xmark            & \cmark       & 5.84  & 16.62 & 12.94 & 11.57 \\
\cline{2-8}
& \cmark       & \cmark       &      \xmark            & 50.91 & 73.38 & 59.58 & 61.15 \\
& \cmark       &         \xmark         & \cmark       & 51.09 & 73.56 & 59.61 & 61.14 \\
&            \xmark      & \cmark       & \cmark       & 8.09  & 24.48 & 25.62 & 16.94 \\
\cline{2-8}
& \cmark       & \cmark       & \cmark       & \textbf{52.19} & \textbf{74.65} & \textbf{59.92} & \textbf{62.24} \\ \bottomrule
\end{tabular}
\caption{Ablation study investigating the importance of three features, Textual Fingerprint (\textbf{TF}), Structural Fingerprint (\textbf{SF}), and Traversal Identifier (\textbf{TI}), of the traversal trajectories used in our Structure-aware Reranker.}
\label{tab-feature-ablation}
\vspace{-2ex}
\end{table}



\subsubsection{Feature Ablation}
The above ablation study highlights the crucial role of Structure-aware Reranker in adaptively integrating structural and textual knowledge. To further analyze the contributions of its three key features, \textbf{Textual Fingerprint (TF)}, \textbf{Structural Fingerprint (SF)}, and \textbf{Traversal Identifier (TI)} defined in Section~\ref{sec-organizing}, we conduct a feature ablation analysis and report retrieval performance across different feature configurations in Table~\ref{tab-feature-ablation}.
%Overall and individual performance
Overall, using three features together yields the best performance on both MAG and Amazon, highlighting their synergistic effect. Individually, TF contributes the most and outperforms SF and TI on both datasets. 
The reason is that based on the definition in Section~\ref{sec-organizing}, TF directly captures the relevance between the query and the retrieved nodes along the trajectory, whereas SF and TI primarily characterize the structural patterns and retrieval types, serving more as complementary factors. Therefore, equipping TF with these complementary factors (i.e., SF or TI) yields around 10\% additional gains on MAG. This is because SF and TI help the reranker selectively emphasize the relevance scores given by TF for certain nodes along the path. However, this boost is not observed on Amazon. We hypothesize that the textual knowledge needed there is predominantly derived from the final node on each path, making the structural cues provided by SF and TI less beneficial and even prone to overfitting. A deeper analysis to further justify this hypothesis is in Section~\ref{sec-further}. Overall, these findings underscore the varying importance of structural features in ranking across datasets.



\begin{table}[t!]
\small
\setlength\tabcolsep{4pt}
\centering
\begin{tabular}{l|ccc|ccc}
\toprule
\multirow{2}{*}{\textbf{Feature}} & \multicolumn{3}{c|}{\textbf{MAG}} & \multicolumn{3}{c}{\textbf{Amazon}} \\

 & H@1 & R@20 & MRR & H@1 & R@20 & MRR \\
\midrule
Last Node & 49.91 & 73.49 & 59.92 & 50.36 & 59.62 & 61.05   \\
Full Path & \textbf{58.19} & \textbf{75.01} & \textbf{67.14} & \textbf{52.19} & \textbf{59.92} & \textbf{62.24}   \\
\bottomrule
\end{tabular}
\caption{Comparing reranking performance using last node in the retrieved trajectory and the whole trajectory.}
\label{tab-Reranker-ablation}
\vspace{-2ex}
\end{table}

\begin{figure}[t!]
    \centering
    \includegraphics[width=0.49\textwidth, height = 0.22\textwidth]{figures/query-pattern-20250215.png}
    \vspace{-4.5ex}
    \caption{Imbalance number of queries and performance of different retrievers across different logical structures.}
    \label{fig-analysis}
    \vspace{-3ex}
\end{figure}





\subsection{Further Analysis}\label{sec-further}
This section understands MoR’s behavior by examining three questions, each of which enriches our insight into MoR’s functionality and offers novel perspectives inspiring future query retrieval research.

\textbf{Do structure signals affect reranking?}
To assess the impact of trajectory information on the Reranker's decision-making, we introduce a node-based Reranker that constructs trajectory features using only TF/SF/TI of the last node. In Table~\ref{tab-Reranker-ablation}, the path-based Reranker outperforms the node-based variant, especially on MAG. This highlights the critical role of trajectory features/structural knowledge in reranking. The minor performance boost on Amazon after switching to the full path trajectory indicates its textual knowledge preference over the last node rather than the whole trajectory.


\textbf{How does MoR perform on different logical structures?}
Figure~\ref{fig-analysis} shows the average performance of MoR on each query group categorized by their logical structures, where "Others" refer to queries with undefined logical structures in~\citet{wu2024stark} MoR consistently outperforms structural and textual retrievers across different logical structures. Among all queries, MoR performs the worst on "P → P" queries due to the ambiguity, although well-known, uniquely caused by repeated product entities from multi-step traversal.
The average-performing ``Others" group underscores the utility of diverse planning strategies for the same query.
Lastly, the skewed query distribution and retrieval performance across planning patterns reflect the varying nature of real-world planning needs. We hope these insights inspire research on data-centric reasoning designs and error control of planning.


\begin{figure}[t!]
    \centering
    \includegraphics[width=0.5\textwidth]{figures/heatmap-20250215.pdf}
    \vspace{-3ex}
    \caption{Saliency map visualization of query attention over three entities along the retrieved paths}
    \label{fig-map}
    \vspace{-2ex}
\end{figure}

\textbf{Does MoR indeed adaptively leverage the trajectory knowledge?} To understand how our proposed reranker prioritizes candidates in the Top-K results, we visualize the saliency map by computing the gradient of ranking scores with respect to the textual fingerprint (TF) of three nodes along the traversed path, which quantifies their importance for answering a given query. Figure~\ref{fig-map} illustrates this by analyzing trajectories for 100 ground-truth candidates across 100 queries on the Amazon and MAG datasets. Each dimension corresponds to a traversed node, with the final one representing the candidate itself. 
While the saliency score is concentrated in the last dimension for Amazon, 
MAG exhibits a more evenly distributed saliency pattern, where multiple nodes along the path contribute significantly to ranking score computation. This suggests that structural knowledge is more critical for answering queries in MAG, aligning with the previously observed lower performance of purely textual retrieval on MAG in Table~\ref{tab-merged}. Further case studies explain why the reranker attends different nodes for different queries. In Figure~\ref{fig-map}(a), the reranker favors the last two dimensions as the rich textual restriction (i.e., "Northwest Company..." and "NFL Seattle...") aids in identifying the correct node at the corresponding reasoning step, as discussed in Section~\ref{sec-reasoning}. The correct nodes, having higher similarity scores with the query, help guide the retrieval process toward the ground truth.
Conversely, in Figure~\ref{fig-map}(b),
since the first node ("University of Lausanne") helps narrow the search space and the last node ("frameless...") further filter candidates, both nodes have high saliency scores. Overall, our findings demonstrate that the reranker dynamically adapts its reliance on structural and textual knowledge depending on the dataset and query. 

Software development is increasingly conceived as a collaboration activity between developers and AIs. Indeed, IDEs already implement features to enable interactive development, with AI suggesting implementations that are reused by developers.

Although multiple studies show this interaction can be successful, there is still limited understanding of how the models must be configured and used in the context of code generation tasks. This study addresses this gap, systematically investigating the impact of several key parameters, including the repeated submission of a prompt to accommodate for the non-deterministic nature of the models.

Our study reveals several key findings about the usage of ChatGPT. In particular, we discovered how creativity, although up to a limited extent, is useful to increase the range of methods whose code can be generated correctly. A major role is played by parameter top-p, which is commonly underrated, and instead has a major impact on the correctness of the results, with lower values producing better results. Finally, prompts should be submitted multiple times, with $5$ repetitions combined with a temperature of $1.2$ resulting in an effective configuration in our experiments.  

Future work concerns two main research directions. One is about replicating this experiment with other AI assistants, to validate our findings in multiple contexts. The second research direction concerns finding strategies to deal with the need to submit the same prompt multiple times to obtain a useful result, and thus developing approaches able to select or merge multiple responses automatically. 
% \appendix
% \section{An example appendix} 
\section*{Acknowledgments}
This work was supported by HKRGC GRF (Project ID: 14306721), and Hong Kong Centre for Cerebro- Cardiovascular Health Engineering (COCHE).

\bibliographystyle{siamplain}
\bibliography{references}

\end{document}
