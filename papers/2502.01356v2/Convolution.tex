\section{Adaptive Convolution on Riemannian Manifolds}

Convolution is a fundamental mathematical operator in mathematics, physics, and engineering. It combines two functions to illustrate how the characteristics of one function are modified by the other. The necessity of defining convolution stems from its applications in various contexts. In this section, we will define convolution on Riemann surfaces, starting with an investigation of convolution on general manifolds and progressing to the definition of convolution on simply connected surfaces through Quasi-conformal Convolution (QCC), which is applicable to many real-world situations.

\subsection{Convolution on Riemannian $n$-manifold}

\subsubsection{Convolution on Manifold}
Before providing a definition of convolution on manifolds, we first examine the standard convolution operation in Euclidean space, as outlined in the following definition.
% In an Euclidean space, the convolution operation can be given as the following definition.

\begin{definition}[Convolution]
For two functions $h, k : \mathbb{R}^n \to \mathbb{R}$, the convolution of $h$ and $k$ is defined as:
\begin{equation}
(h \ast k)(x) = \int_{\mathbb{R}^n} h(y) k(x - y) \, dy,
\end{equation}
where:
\begin{itemize}
    \item $x \in \mathbb{R}^n$ is the point at which the convolution is evaluated,
    \item $y \in \mathbb{R}^n$ is the integration variable,
    \item $k(x - y)$ translates $k$ to align it with $h$.
\end{itemize}
\end{definition}

Defining convolution on manifolds or surfaces is more complex than in Euclidean space due to the absence of a global linear structure on manifolds. In the Euclidean case, convolution involves translating the kernel function $k$ using the displacement vector $x - y$. However, this approach does not directly apply to manifolds. A suitable notion of displacement must be established before performing convolution on manifolds.
%Defining convolution on manifolds or surfaces is more complex than in Euclidean space due to the absence of a global linear structure on manifolds.  In the Euclidean case, convolution involves translating the function $k$ by $x-y$, a shift-invariant operation that does not directly apply to manifolds. Consequently, convolution on an $n$-manifold $\mathcal{M}$ can only be defined between a function on $\mathcal{M}$ and a kernel function on $\mathbb{R}^n$. Since the displacement vector $x-y$ is not well-defined when $x$ and $y$ are points on the manifold, it becomes essential to establish a suitable notion of displacement before performing convolution on manifolds.

\begin{definition}[Displacement function and displacement vector]
    Let $\mathcal{M}$ be a Riemannian $n$-manifold, and $U \subseteq \mathcal{M}$ be a subset of $\mathcal{M}$. A function $d: U\times U \to \mathbb{R}^n$ is a displacement function on $U$ if it satisfies the following properties:
    \begin{enumerate}
        \item For all $p, q \in U$, $d(p,q) = 0$ if and only if $p=q$.
        \item For all $p, q, r \in U$, $d(p,r) = d(p,q) + d(q,r)$.
    \end{enumerate}
    Then, the vector $d(p, q)$ is referred to as the displacement vector from $p$ to $q$.
    
    Moreover, if the functions $d(\cdot, q_0)$ and $d(p_0, \cdot)$ are orientation-preserving homeomorphisms from $U$ to subsets of $\mathbb{R}^n$ depending on $p_0$ and $q_0$ for all fixed $p_0, q_0 \in U$, then the displacement function $d$ is said to be regular.
\end{definition}

% \begin{theorem}
%     Suppose $d: \mathcal{M}\times\mathcal{M} \to \mathbb{R}^n$ is a displacement function, then $d$ satisfies:
%     \begin{enumerate}
%         \item For all $p \in \mathcal{M}$, $d(p,p) = 0$.
%         \item For all $p, q \in \mathcal{M}$, $d(p,q) = -d(q,p)$.
%         \item 
%     \end{enumerate}
% \end{theorem}

The above displacement function is introduced to replace the standard expression $ x-y $ in $ \mathbb{R}^n $. This displacement function closely mimics the fundamental property of translation symmetry \cite{schonsheck2022parallel}. According to our definition, the displacement between any two distinct points on a manifold is always nonzero, and the sum of vectors along a path aligns precisely with the vector directly connecting the path’s endpoints. By adhering to these properties, the kernel can be effectively translated across different points on the manifold.

With the displacement function $d$ on $\mathcal{M}$, we are now ready to give a general definition of manifold convolution.

\begin{definition}[Manifold convolution]\label{mani_conv}
Let $\mathcal{M}$ be a Riemannian $n$-manifold with a metric $g$, and let $h: \mathcal{M} \to \mathbb{R}$ be a manifold function with a kernel function $k: \mathbb{R}^{n}\to \mathbb{R}$. The convolution of $h$ and $k$ on $\mathcal{M}$ is defined as:

\begin{equation}\label{eq:mani_conv}
(h \ast_{\mathcal{M},d,g} k)(p) = \int_{\mathcal{M}} h(q) k(d(p, q)) \, dq,
\end{equation}

where:
\begin{itemize}
    \item $p, q \in \mathcal{M}$,
    \item $d: \mathcal{M}\times\mathcal{M} \to \mathbb{R}^n $ is a global displacement function on $\mathcal{M}$.
    % \item $d\nu(q)$ is the volume form on $\mathcal{M}$ derived from the metric $g$.
\end{itemize}
For simplicity, we denote $*_{\mathcal{M}, d,g}$ as $*_{d,g}$.
Moreover, $*_{d,g}$ is said to be regular if the displacement function $d$ is regular.

\end{definition}

%In Euclidean space, convolution relies on translating a kernel function $ k(x) $ into $ k(x-y) $. However, on a manifold, translation is not well-defined due to curvature. Therefore, a displacement function is introduced to replace the standard expression $ x-y $ in $ \mathbb{R}^n $. This displacement function should closely mimic the fundamental property of translation symmetry \cite{schonsheck2022parallel}. According to our definition, the displacement between any two distinct points on a manifold is always nonzero, and the sum of vectors along a path aligns precisely with the vector directly connecting the path’s endpoints. By adhering to these properties, the kernel can be effectively translated across different points on the manifold.

% These observations highlight the close relationship between the displacement function $d$ and volume form $d\nu$ derived from the Riemannian metric $g_{\mu\nu}$, as formalized in Definition \ref{mani_conv}. Ideally, $d\nu$ should be directly induced by the displacement function $d$, ensuring metric consistency throughout the integral. This relationship and its implications will be further explored in the next subsection.

Note that the manifold convolution above is not commutative as the manifold function $h$ is defined on $\mathcal{M}$ while kernel function $k$ is defined on $\mathbb{R}^n$. This lack of commutativity does not hinder the definition of the convolution operation for deep learning tasks on Riemann surfaces. 

% Before we proceed to the next section, let's make another remark about the volume form $d\nu$ and the Riemannian metric $g$ to make the discussion later more convenient.

Before we proceed to the next section, we shall emphasize that the convolution operator $*_{d,g}$ does depend on the Riemannian metric $g$ of $\mathcal{M}$ due to the standard definition of integration on manifolds used in Equation \ref{eq:mani_conv}. More details are discussed in the following remark.
\begin{remark}\label{rm:volume_form}
    Let $\mathcal{M}$ be a Riemannian $n$-manifold and let $h: \mathcal{M} \to \mathbb{R}$ be a manifold function. 
    Suppose $\mathcal{M}$ is covered by a collection of coordinate charts $\{(U_\alpha, \phi_\alpha)\}$, and let $\{\psi_\alpha\}$ be a partition of unity subordinate to the cover $\{U_\alpha\}$. In local coordinates, the {\it volume form $d\nu$ induced by $g$} is given by:
\begin{equation}
d\nu =  \sqrt{\det(g(x))} \, dx^1 \wedge dx^2 \wedge \cdots dx^n,
\end{equation}
\noindent where $g(x)=(g_{ij}(x))_{1\leq i,j\leq n}$. The manifold integral of $h$ is then defined as
    % \begin{equation}
    % \int_\mathcal{M} h \, d\nu = \int_\Omega h(\phi(x)) \sqrt{\det(g(x))} \, dx.
    % \end{equation}
    \begin{equation}
    \int_\mathcal{M} h \, d\nu = \sum_\alpha \int_{U_\alpha} \psi_\alpha(x) h(\phi_\alpha^{-1}(x)) \sqrt{\det(g(x))} \, dx.
    \end{equation}
  \end{remark}

%\begin{remark}
 %   Let $\mathcal{M}$ be a Riemannian $n$-manifold, and let $h: \mathcal{M} \to \mathbb{R}$ be a manifold function. For a diffeomorphism $\phi : \Omega \to \mathcal{M}$, where $\Omega \in \mathbb{R}^{n}$, the manifold integral could be defined as    
 %   \begin{equation}
  %  \int_\mathcal{M} h \, d\nu = \int_U h(\phi(x)) \sqrt{\det(g(x))} \, dx,    
   % \end{equation}
    %where $g=(g_{ij})$ and:  
    %\begin{equation}
    %g_{ij} = \langle \frac{\partial \phi}{\partial x_i}, \frac{\partial \phi}{\partial x_j} \rangle,
    %\end{equation}
    %we say the volume form $d\nu$ is derived by metric $g=(g_{ij})$ via $\phi$.
%\end{remark}

\subsubsection{Convolution on Manifold via Parameterization}

The concept of the convolution operation on a manifold is not well-established, primarily due to the curvature of the manifold. In the Euclidean domain, the plain convolution operator involves shifting the kernel function. This shifting is straightforward in Euclidean space, where the geometry is flat and the displacement from one point to another is well-defined. In this subsection, we introduce the idea of defining convolution on a Riemann surface on its 2D parametric flat domain.
%When defining convolution on a manifold, one is confronted with the challenges introduced by the manifold's curvature and the lack of a proper manifold metric definition. The typical convolution operation, which relies on shifting and integrating a kernel over a space, is straightforward in Euclidean spaces, where the geometry is flat and distances are well-defined. Therefore, it is natural to define a convolution on the parameterization domain.

\begin{definition}[Parametrized manifold convolution]
    Let $\mathcal{M}$ be a Riemannian $n$-manifold, and let $h: \mathcal{M} \to \mathbb{R}$ be a manifold function with a kernel function $k: \mathbb{R}^{n}\to \mathbb{R}$. Suppose there exists a bijective parametrization $\phi:\Omega \to \mathcal{M}$, where $\Omega \subset \mathbb{R}^n$. The parametrized manifold convolution of $h$ and $k$ on $\mathcal{M}$ is defined as:
    \begin{equation}
        (h *_\phi k)(p) = \int_\Omega h(\phi(y))k(\phi^{-1}(p)-y)dy.
    \end{equation}
    
    % \begin{align*}
    %     (h *_\phi g)(p) &= \int_\Omega h(\phi(y))g(\phi^{-1}(p)-y)dy\\
    %     &= \int_\mathcal{M} h(q)g(\phi^{-1}(p)-\phi^{-1}(q))d\phi^{-1}(q)
    % \end{align*}
    
    % $$(h *_\phi g)(p) = \int_\Omega h(\phi(y))g(\phi^{-1}(p)-y)dy$$
    % \int_\mathcal{M} h(q)g(\phi^{-1}(p) - \phi^{-1}(q))d\phi{-1}(q),$$
    % ($\phi$ doesn't need to be bijective here(?))
    \label{them:parameterconv}
\end{definition}

The definition above simplifies the computation of manifold convolution and provides a more intuitive viewpoint by performing convolution on an Euclidean domain. To explore the relationship between parameterized manifold convolution and manifold convolution, the following lemma is essential to demonstrate that a displacement function on a manifold gives rise to a surface parameterization, enabling us to define the parameterized manifold convolution.

% This allows us to consider a substantial subset of manifold convolutions, where the displacement function $d$ and metric $g$ are properly related, as parametrized manifold convolution, or vice versa, whichever is more intuitive.


% \begin{theorem}
%     Given a $n$-manifold $\mathcal{M}$, then there exists a displacement function $d:\mathcal{M} \times \mathcal{M} \to \mathbb{R}$ if and only if there exists a bijective parametrization $\phi: \Omega \to \mathcal{M}$ such that $$d(\phi(x),\phi(y)) = x-y$$ for all $x, y \in \Omega$, where $\Omega \subset \mathbb{R}^n$.
% \end{theorem}

\begin{lemma}\label{disp2phi}
     Let $\mathcal{M}$ be a Riemannian $n$-manifold. If there exists a displacement function $d:\mathcal{M} \times \mathcal{M} \to \mathbb{R}^n$, then there exists a bijective parametrization $\phi: \Omega \to \mathcal{M}$ such that
     \begin{equation}
         d(p,q) = \phi^{-1}(p)-\phi^{-1}(q)
     \end{equation}
     for all $p, q \in \mathcal{M}$, where $\Omega \subset \mathbb{R}^n$.

    Conversely, if there exists a bijective parametrization $\phi: \Omega \to \mathcal{M}$, where $\Omega \subset \mathbb{R}^n$, then a function $d:\mathcal{M} \times \mathcal{M} \to \mathbb{R}^n$ defined by 
    \begin{equation}
         d(p,q) = \phi^{-1}(p)-\phi^{-1}(q)
    \end{equation}
    for all $p, q \in \mathcal{M}$, is a displacement function on $\mathcal{M}$.
\end{lemma}

\begin{proof}
    $(\Rightarrow)$ Suppose there exists a displacement function $d:\mathcal{M} \times \mathcal{M} \to \mathbb{R}$. Pick any $p_0 \in \mathcal{M}$ and let $\Omega = \{d(p, p_0): p \in \mathcal{M}\} \subset \mathbb{R}^n$. Define $\tilde{\phi}: \mathcal{M} \to \Omega$ by $\tilde{\phi}(p) = d(p, p_0)$. Then
    \begin{equation}
        d(p, q) = d(p, p_0) - d(q, p_0) = \tilde{\phi}(p)-\tilde{\phi}(q).
    \end{equation}
    Note that this equation shows that $\tilde{\phi}$ is injective as $d(p,q)=0$ implies $p=q$, and the surjectivity of $\tilde{\phi}$ is guaranteed by the definition of $\Omega$, $\tilde{\phi}$ is therefore bijective. 
    % Suppose $\tilde{\phi}(p) = \tilde{\phi}(q)$ for some $p, q \in \mathcal{M}$, then
    % $$d(p, q) = d(p, p_0) - d(q, p_0) = \tilde{\phi}(p) - \tilde{\phi}(q) = 0,$$
    % which follows that $p=q$, and so $\tilde{\phi}$ is injective. Note that the surjectivity of $\tilde{\phi}$ is guaranteed by the definition of $\Omega$, $\tilde{\phi}$ is therefore bijective. For all $p, q \in \mathcal{M}$, we can check that
    % $$d(p, q) = d(p, p_0) - d(q, p_0) = \tilde{\phi}(p)-\tilde{\phi}(q).$$
    Thus $\phi = \tilde{\phi}^{-1}$ is the desired bijective parametrization.
    % $(\Leftarrow)$ It is a direct consequence of the definition of displacement functions.
    % $(\Leftarrow)$ Suppose there exists a bijective parametrization $\phi: \Omega \to \mathcal{M}$ such that $d(\phi(x),\phi(y)) = x-y$ for all $x, y \in \Omega$, where $\Omega \subset \mathbb{R}^n$. Define $d:\mathcal{M} \times \mathcal{M} \to \mathbb{R}$ by
    % $$d(p, q) = \phi^{-1}(p) - \phi^{-1}(q),$$
    % then we can
    
    $(\Leftarrow)$ Check that for all $p, q, r \in \mathcal{M}$,
    $$d(p, q) = 0 \iff \phi^{-1}(p) = \phi^{-1}(q) \iff p=q$$
    and
    $$d(p, r) = \phi^{-1}(p) - \phi^{-1}(r) = (\phi^{-1}(p) - \phi^{-1}(q)) + (\phi^{-1}(q) - \phi^{-1}(r)) = d(p, q) + d(q, r).$$
    Therefore, $d$ is a displacement function for $\mathcal{M}$.
    % , which obviously satisfies $d(\phi(x),\phi(y)) = x-y$ for all $x, y \in \Omega$.
\end{proof}

% (We need to choose one of the three theorems below)
% \begin{theorem}[original]
%     Let $\mathcal{M}$ be a Riemannian $n$-manifold and $*_\mathcal{M}$ be an operator. Then $*_\mathcal{M} = *_d$ for some displacement function $d$ on $\mathcal{M}$ if and only if $*_\mathcal{M} = *_\mathcal{\phi}$ for some bijective parametrization $\phi: \Omega \to \mathcal{M}$, where $\Omega \subset \mathbb{R}^n$.
% \end{theorem}

% \begin{proof}
%     ($\Rightarrow$) Suppose $*_\mathcal{M} = *_d$ for some displacement function $d$ on $\mathcal{M}$. By Lemma \ref{disp2phi}, there exists a bijective parametrization $\phi: \Omega \to \mathcal{M}$ such that $$d(p,q) = \phi^{-1}(p)-\phi^{-1}(q)$$ for all $p, q \in \mathcal{M}$, where $\Omega \subset \mathbb{R}^n$. Then for any manifold function $h:\mathcal{M} \to \mathbb{R}$ and kernel function $g: \mathbb{R}^n \to \mathbb{R}$, we have 
%     \begin{align*}
%         (h *_\phi g)(p) &= \int_\mathcal{M} h(q)g(\phi^{-1}(p)-\phi^{-1}(q))d\phi^{-1}(q) \\
%         &= \int_\mathcal{M} h(q)g(d(p,q))d\phi^{-1}(q) \\
%         &= (h *_d g)(p)
%     \end{align*}
%     where $\nu(q)$ is induced by $\nu = L^1 \circ \phi^{-1}$.

%     ($\Leftarrow$) By Lemma \ref{disp2phi}, the function $d(p,q) = \phi^{-1}(p)-\phi^{-1}(q)$ is a displacement function. Therefore, the result follows similarly.
% \end{proof}

\begin{theorem}\label{equiv_conv}
    Let $\mathcal{M}$ be a Riemannian $n$-manifold and $d$ be a displacement function on $\mathcal{M}$. Then there exists a bijective parametrization $\phi: \Omega \to \mathcal{M}$, where $\Omega \subset \mathbb{R}^n$, along with a metric $g$ of $\mathcal{M}$, such that $*_{d, g} = *_\phi$. Conversely, for any bijective parametrization $\phi: \Omega \to \mathcal{M}$, there exists a displacement function $d$ on $\mathcal{M}$ and a metric $g$ of $\mathcal{M}$ such that $*_{d, g} = *_\phi$.
\end{theorem}

\begin{proof}
    ($\Rightarrow$) Suppose $*_\mathcal{M} = *_{d,g}$ for some displacement function $d$ on $\mathcal{M}$. By Lemma \ref{disp2phi}, there exists a bijective parametrization $\phi: \Omega \to \mathcal{M}$ such that $$d(p,q) = \phi^{-1}(p)-\phi^{-1}(q)$$ for all $p, q \in \mathcal{M}$, where $\Omega \subset \mathbb{R}^n$. Consider $\phi^{-1}$ as the coordinate chart function and define the Riemannian metric $g = (\phi^{-1})^*g_{\mathbb{R}^n}$ as a pullback metric of $\mathcal{M}$, where $g_{\mathbb{R}^n}$ is the standard Euclidean metric. The volume form $d\nu$ can be obtained by Remark \ref{rm:volume_form}. 
    
    Since the distance $dg(p,q) = |\phi^{-1}(p)-\phi^{-1}(q)|$ for all $p,q \in \mathcal{M}$, $\phi^{-1}$ is an isometric mapping with respect to the metric $g$ and $\det(D\phi^{-1}) = 1$, therefore $dy = |\det(D\phi^{-1})| \, dq = dq$ for $\phi(y) = q$. For any manifold function $h:\mathcal{M} \to \mathbb{R}$, kernel function $k: \mathbb{R}^n \to \mathbb{R}$ and $p \in \mathcal{M}$, we now have 
    \begin{equation}
        \begin{aligned}
            (h *_\phi k)(p) &= \int_\Omega h(\phi(y))k(\phi^{-1}(p)-y)dy \\
            &= \int_\mathcal{M} h(q)k(\phi^{-1}(p)-\phi^{-1}(q))dq \\
            &= \int_\mathcal{M} h(q)k(d(p,q))dq \\
            &= (h *_{d,g} k)(p).
        \end{aligned}
    \end{equation}
    Therefore $*_{d, g} = *_\phi$.

    ($\Leftarrow$) By Lemma \ref{disp2phi}, the function $d(p,q) = \phi^{-1}(p)-\phi^{-1}(q)$ is a displacement function. Therefore, the result follows similarly.
\end{proof}


% **********************************NEED DISCUSSION HERE******************************************
% \begin{remark}
%     The parameterization $\phi$ is an isometric mapping with respect to the volume form $d\nu$, which is directly induced by $\phi$ in the construction.
% \end{remark}


% \begin{theorem}
%     Let $\mathcal{M}$ be a Riemannian $n$-manifold. An operator $*_\mathcal{M}$ is a generalized manifold convolution with respect to some displacement function $d$ on $\mathcal{M}$ if and only if it is a parametrized manifold convolution with respect to some bijective parametrization $\phi: \Omega \to \mathcal{M}$, where $\Omega \subset \mathbb{R}^n$.
% \end{theorem}


% \begin{theorem}
%     Let $\mathcal{M}$ be a Riemannian $n$-manifold and $*_\mathcal{M}$ be a manifold convolution on $\mathcal{M}$. Then there exists a bijective parametrization $\phi: \Omega \to \mathcal{M}$, where $\Omega \subset \mathbb{R}^n$, such that $*_\phi = *_\mathcal{M}$. (This only includes a single direction.)
% \end{theorem}
% \begin{proof}
%     ...
% \end{proof}

Theorem \ref{equiv_conv} establishes that the space of manifold convolutions, equipped with a displacement function $d$ and an associated metric $g$, is equivalent to the space of parameterized manifold convolutions. Hence, the set of all parameterized manifold convolutions encompasses a substantial portion of manifold convolutions, suggesting that many spatially defined manifold convolutions can be effectively represented within this parameterized framework. To conclude this section, we will investigate the regularity properties of both types of convolutions.
%Theorem \ref{equiv_conv} establishes that the space of manifold convolutions, equipped with a displacement function $d$ and an associated volume form $d\nu$ derived from a suitable metric, is equivalent to the space of parametrized manifold convolutions. Nonetheless, the set of all parametrized manifold convolutions encompasses a substantial portion of manifold convolutions, indicating that many manifold convolutions which are spatially defined can be effectively represented within this parametrized framework. To conclude this section, we investigate the regularity properties of both types of convolutions.

%Theorem \ref{equiv_conv} establishes that the space of manifold convolutions, equipped with a displacement function $d$ and a volume form $d\nu$ derived from a suitable metric, is equivalent to the space of parametrized manifold convolutions. Nonetheless, the set of all parametrized manifold convolutions represents a significant subset of manifold convolutions that are geometrically defined, in contrast to those that are not, such as spectral convolutions \cite{bruna2014spectral}. To conclude this section, we investigate the regularity properties of both types of convolutions. 


\begin{corollary}
    Let $\mathcal{M}$ be a Riemannian $n$-manifold equipped with a displacement function $d$ and a metric $g$, is parametrized by a function $\phi$ such that $*_{d,g} = *_\phi$. Then the following statements are equivalent:
    \begin{enumerate}
        \item $d$ is regular.
        \item $\phi$ is an orientation-preserving homeomorphism.
        \item $*_{d,g} = *_\phi$ is regular.
    \end{enumerate}
    \label{them:manifoldparametrized}
\end{corollary}

% \begin{corollary}
%     Suppose a Riemannian $n$-manifold $\mathcal{M}$ is parametrized by a diffeomorphic function $\phi$. Then the manifold convolution $*_\phi$ is regular.
%     % in the sense that there exists an equivalent convolution operator $*_\mathcal{M}$, such that the corresponding displacement function $d$ is regular.
% \end{corollary}
\begin{proof}
    Note that (1) and (3) are equivalent by definition, and (2) implies (1) as $*_{d,g} = *_\phi$ implies $d(p,q) = \phi^{-1}(p) - \phi^{-1}(q)$, which is regular if $\phi$ is an orientation-preserving homeomorphism.
    
    (1) $\Rightarrow$ (2): Suppose $d$ is regular, then using the construction of $\phi$ in the previous theorem, we immediately see $\phi$ is an orientation-preserving homeomorphism. Assume $*_{d,g} = *_{\phi_1}$ for some bijective parametrization $\phi_1: \Omega_1 \to \mathcal{M}$, where $\Omega_1 \subset \mathbb{R}^n$, then for all $p, q \in \mathcal{M}$,
    \begin{equation}
    \begin{aligned}
        &\phi^{-1}(p) - \phi^{-1}(q) = \phi_1^{-1}(p) - \phi_1^{-1}(q) \\
        \Rightarrow \quad &\phi^{-1}(p) - \phi_1^{-1}(p) = \phi^{-1}(q) - \phi_1^{-1}(q) \\
        \Rightarrow \quad &\phi^{-1} - \phi_1^{-1} \equiv c \qquad \text{for some constant vector } c \in \mathbb{R}^n.
    \end{aligned}
    \end{equation}
    Hence $\phi_1$ is also an orientation-preserving homeomorphism. 
\end{proof}

\subsection{Convolution on Riemann Surfaces}
In this work, we focus our problems on simply connected open surfaces embedded in $\mathbb{R}^3$, now we will move on to describe how we can define convolution on Riemann surfaces.

\subsubsection{Conformal Convolution}
A natural and useful approach to produce a parametrized manifold convolution is to employ a conformal parametrization of the manifold. A conformal parametrization $\phi$ is a map from a domain $\Omega \subset \mathbb{R}^2$ to the manifold $\mathcal{M} \subset \mathbb{R}^3$ that preserves angles. By mapping the curved surface to a flat Euclidean space, we can take advantage of the well-established theory of Euclidean convolutions. 

To define the convolution of a manifold function $h : \mathcal{M} \to \mathbb{R}$ on a $2$-manifold $\mathcal{M}$ and a kernel function $k : \mathbb{R}^2 \to \mathbb{R}$, we begin by pulling the function back to the parameter domain $\Omega$ using the conformal parametrization $\phi$. This transforms the problem of manifold convolution into a more manageable Euclidean convolution problem. Specifically, for each function $h$ defined on the surface and $k$ defined on $\mathbb{R}^2$, we define the pullback function $\tilde{h}$ on the flat domain $\Omega$ as follows:
\begin{equation}
\tilde{h}:=\phi^* h = h \circ \phi.
\end{equation}
% *********************************************************************************
% \enoch{(A more standard notation should be $\phi^* h = h \circ \phi$. How about we write $\tilde{h}:=\phi^* h = h \circ \phi$?)\\}
% *********************************************************************************
The pullback functions $\tilde{h}$ are now defined on the Euclidean space $\Omega \subset \mathbb{R}^2$, where the convolution operation can be performed using the standard Euclidean formulation.

The convolution of $\tilde{h}$ and $k$ in the Euclidean domain $\Omega$ is then defined as:
\begin{equation}
(\tilde{h} \ast k)(x) = \int_{\Omega} \tilde{h}(y) k(x - y) \, dy,
\end{equation}
where the integral is taken over the domain $\Omega \subset \mathbb{R}^2$. This is the classical convolution in Euclidean space, which is computationally efficient and well-understood.

Once the convolution is computed in the Euclidean domain, the result must be mapped back to the manifold $\mathcal{M}$. This is done by applying the inverse of the conformal parametrization $\phi^{-1}$, which maps the Euclidean result back onto the manifold's coordinates. The convolution on the manifold is then given by:
\begin{equation}
(h \ast_\phi k)(p) = (\tilde{h} \ast k)(\phi^{-1}(p)),
\end{equation}
where $p \in \mathcal{M}$ is a point on the manifold, and $\phi^{-1}(p)$ returns the corresponding point in the parameter domain $\Omega$.

Then, the manifold convolution is reduced to a Euclidean convolution performed in the parametrized space $\Omega$, followed by a pullback to the manifold using the inverse of the conformal map. The final formal definition is then written as:

\begin{definition}[Conformal Convolution]
Let $\mathcal{M} \subset \mathbb{R}^3$ be a 2-manifold, and let $\phi: \Omega \to \mathcal{M}$ be a conformal parametrization, where $\Omega \subset \mathbb{R}^2$ is a domain in the Euclidean plane. Let $h: \mathcal{M} \to \mathbb{R}$ be a manifold function with a kernel function $k: \mathbb{R}^{n}\to \mathbb{R}$. The conformal convolution of $h$ and $k$ is defined as:
\begin{equation}
\begin{aligned}
(h \ast_\phi k)(p) &= \int_\mathcal{M} h(q) k(\phi^{-1}(p) - \phi^{-1}(q)) dq \\
&= \int_{\Omega} \tilde{h}(y) k(x - y) dy \\
&= (\tilde{h} \ast k)(\phi^{-1}(p)),
\end{aligned}
\end{equation}
where:
\begin{itemize}
    \item $p,q \in \mathcal{M}$,
    \item $x=\phi^{-1}(p), y=\phi^{-1}(q) \in \Omega$,
    \item $\tilde{h} = h \circ \phi$ is the pullback function,
    \item $g = (\phi^{-1})^* g_{\mathbb{R}^2}$ is the Riemannian metric of $\mathcal{M}$.
\end{itemize}
\label{def:conformalconv}
\end{definition}

It is also important to note that, according to the Riemann Mapping Theorem (Theorem \ref{them:RiemannMapping}), the conformal parameterization is unique when three points are fixed. In practice, by mapping the surface to a disk or rectangle while fixing the boundary points, the parametrization becomes determined. The conformal convolution is a special case of parametrized convolution, and thus a manifold convolution by taking $d(p,q) \coloneqq   \phi^{-1}(p) - \phi^{-1}(q)$ and $d\nu$ derived from metric $g$  via $\phi^{-1}$, according to Theorem \ref{equiv_conv}. 

Although conformal parametrization retains nice geometric properties once the surface is mapped to the 2D domain, there is no evidence that it is the best parametrization to define a parametrized convolution. As a special case of parametrized convolution, conformal convolution is too restrictive for advanced usage, such as when implemented into deep learning tasks. Therefore, a much more flexible convolution, namely Quasi-Conformal Convolution, will be introduced in the next subsection.

\subsubsection{Quasi-Conformal Convolution}

The proposed Quasi-Conformal Convolution (QCC) is to define convolution operation on manifolds using quasi-conformal mappings. Quasi-conformal theory offers a mathematically robust framework for studying deformations between surfaces while preserving local geometric structures. By leveraging this theory, QCC extends the convolution operation to non-Euclidean domains such as manifolds, enabling deep learning methods to process irregular and geometrically distorted data.

Here, similar to how we define parametrized convolution, we have the following definition for Quasi-conformal Convolution.

\begin{definition}[Quasi-conformal Convolution]
Let $\mathcal{M} \subset \mathbb{R}^3$ be a 2-manifold, and let $\phi: \Omega \to \mathcal{M}$ be a conformal parametrization, where $\Omega \subset \mathbb{R}^2$ is a domain in the Euclidean plane. Let $h: \mathcal{M} \to \mathbb{R}$ be a manifold function with a kernel function $k: \mathbb{R}^{n}\to \mathbb{R}$.  Let $f: \Omega \to \Omega$ be a quasi-conformal mapping. The quasi-conformal convolution of $h$ and $k$ with respect to $\phi$ and $f$ is defined as:

\begin{equation}
\begin{aligned}
(h \ast_{\phi, f} k)(p)
&= \int_{\mathcal{M}}h(q)k(f\circ\phi^{-1}(p) - f\circ\phi^{-1}(q)) dq\\
&= \int_{\Omega} h\circ\phi(y) k(f(x) - f(y)) \, df(y) = \int_{\Omega} \tilde{h}(y)k(f(x) - f(y)) \, df(y)\\
&= \int_{\Omega} \tilde{h}\circ f^{-1}(y')k(x' - y') \, dy' = \int_{\Omega} h^\#(y')k(x' - y') \, dy'\\
&= h^\# \ast k(f\circ\phi^{-1}(p))
\label{eq:parametrizedQCC}
\end{aligned}
\end{equation}

where:
\begin{itemize}
    \item $p,q \in \mathcal{M}$ is a point on the manifold,
    \item $p = \phi(x), q = \phi(y) \in \Omega$,
    \item $x'=f(x), y'=f(y) \in \Omega$,
    \item $f:\Omega\to\Omega$ is a quasi-conformal mapping on the parametrized domain $\Omega$,
    \item $\tilde{h}=h\circ\phi$ is the pullback function,
    \item $h^\#=\tilde{h}\circ f^{-1}$ is the transformed pullback function,
    \item $g = (f \circ \phi^{-1})^* g_{\mathbb{R}^2}$ is the Riemannian metric of $\mathcal{M}$.
\end{itemize}
\end{definition}

Recall that, according to the Riemann Mapping Theorem (Theorem \ref{them:RiemannMapping}), the conformal parameterization is unique when three points are fixed which is generally ensured through the fix the boundaries. Therefore, the subscript $\phi$ of the operation $\ast_{\phi, f}$ can be omitted into $\ast_{f}$.

\begin{theorem}
    Let $\mathcal{M} \subset \mathbb{R}^3$ be a 2-manifold and $*_\mathcal{M}$ be a parametrized manifold convolution. Then $*_\mathcal{M}$ is regular if and only if $*_\mathcal{M}$ is a quasi-conformal convolution.
\end{theorem}
\begin{proof}
    The mapping $\phi: \Omega \to \mathcal{M}$ is quasi-conformal if and only if it is an orientation-preserving homeomorphism, which is equivalent to saying that the parametrized manifold convolution $*_\phi$ is regular.
\end{proof}

\begin{remark}
    Quasi-conformal Convolution is a parametrized manifold convolution with parametrization $\phi \circ f^{-1}$.
\end{remark}

The theorem and remark above show that we can generalize any regular parametrized manifold convolution on Riemann surfaces into quasi-conformal convolution, through which a substantial subset of manifold convolution could be represented. As different Quasi-conformal parameterizations would yield distinct convolution operators on the Riemann surface, one can find the best convolution operator for a specific task by optimizing the corresponding Quasi-conformal parameterization. Within this framework, we can identify the most effective convolution operator on a Riemann surface by optimizing the Quasi-conformal parameterization. 

By representing parametrized manifold convolutions through quasi-conformal mappings, learning the convolution operations in neural networks becomes feasible. In the next section, we will develop a deep neural network framework to learn the optimal Quasi-conformal parameterization associated with the best convolution operator for a given task. Quasi-conformal mappings preserve local geometric structures while allowing controlled deformations. This characteristic ensures that the convolution operation aligns with the intrinsic geometry of the manifold, enabling more robust and effective feature extraction when implementing quasi-conformal convolution into deep neural networks.