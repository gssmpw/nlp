\begin{abstract}
Deep learning on non-Euclidean domains is important for analyzing complex geometric data that lacks common coordinate systems and familiar Euclidean properties. A central challenge in this field is to define convolution on domains, which inherently possess irregular and non-Euclidean structures.
%
In this work, we introduce Quasi-conformal Convolution (QCC), a novel framework for defining convolution on Riemann surfaces using quasi-conformal theories. Each QCC operator is linked to a specific quasi-conformal mapping, enabling the adjustment of the convolution operation through manipulation of this mapping. By utilizing trainable estimator modules that produce Quasi-conformal mappings, QCC facilitates adaptive and learnable convolution operators that can be dynamically adjusted according to the underlying data structured on Riemann surfaces. QCC unifies a broad range of spatially defined convolutions, facilitating the learning of tailored convolution operators on each underlying surface optimized for specific tasks. Building on this foundation, we develop the Quasi-Conformal Convolutional Neural Network (QCCNN) to address a variety of tasks related to geometric data.
%
We validate the efficacy of QCCNN through the classification of images defined on curvilinear Riemann surfaces, demonstrating superior performance in this context. Additionally, we explore its potential in medical applications, including craniofacial analysis using 3D facial data and lesion segmentation on 3D human faces, achieving enhanced accuracy and reliability.
%
\end{abstract}
\begin{keywords}
  Quasi-Conformal Geometry, Deformable Convolution, Geometric Learning, Manifold Learning
\end{keywords}

\begin{AMS}
  53Z50, 68T45, 68U05, 65D18  
\end{AMS}