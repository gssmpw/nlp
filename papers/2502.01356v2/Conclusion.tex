\section{Conclusion}

In this work, we introduce a novel framework for defining convolution on Riemann surfaces based on quasi-conformal theory, termed Quasi-conformal Convolution (QCC). This approach generalizes the conventional definition of convolution, enabling the convolution of functions defined on Riemann surfaces with a given kernel. By leveraging different quasi-conformal mappings, QCC dynamically adjusts convolution operations to suit specific tasks and data.

QCC facilitates adaptive and effective convolutional operations on Riemann surfaces through a dedicated module that generates data-responsive quasi-conformal mappings. This allows the definition of convolution to be dynamic and tailored to the input data. Building on this foundation, we develop the Quasi-conformal Convolutional Neural Network (QCCNN), which is designed to handle tasks involving geometric data on Riemann surfaces. By adapting the convolution process to the underlying Riemann surface structure and training data, QCCNN overcomes the challenges of defining optimal convolution operations for complex geometries, enabling the convolution definition to be learned and refined during training.

We also establish a comprehensive theoretical foundation for convolution on Riemann surfaces and the proposed QCC. This not only enhances our understanding of QCC but also provides valuable insights for constructing models tailored to specific tasks.

Our experiments demonstrate the effectiveness of QCCNN across a range of applications, including Riemann surface image classification, craniofacial analysis using 3D facial data, and facial lesion segmentation. In each case, QCC outperforms existing methods in terms of accuracy and reliability. The proposed QCC framework offers significant potential for advancing deep learning applications in fields involving complex, non-Euclidean geometric structures.