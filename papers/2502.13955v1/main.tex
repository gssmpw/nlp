%
%  $Description: Author guidelines and sample document in LaTeX 2.09$ 
%
%  $Author: ienne $
%  $Date: 1995/09/15 15:20:59 $
%  $Revision: 1.4 $
%
\pdfoutput=1
\documentclass[runningheads]{llncs}
%\usepackage{amsthm}
\usepackage{physics}
\usepackage{amssymb}
\usepackage{listings}
\usepackage{graphicx}
\usepackage{capt-of}% or \usepackage{caption}
\usepackage[all]{xy}
\usepackage{tikz}
\usepackage{multirow}
\usetikzlibrary{shapes,arrows,automata,shadows,positioning,fit}
\usepackage{paralist} % for inline itemize
\usepackage{xspace}
\usepackage{alltt}
\usepackage{enumerate}
\usepackage[inline,shortlabels]{enumitem}
\usepackage{mathtools}
\usepackage{bm}
\usepackage{courier}
\usepackage{color}
\usepackage{caption}
\usepackage{stmaryrd}
\usepackage{adjustbox}
\usepackage{wrapfig}
\usepackage{todonotes}
\usepackage{pgfplots}
\usepackage{amstext} % for \text macro
\usepackage{array}   % for \newcolumntype macro
\usepackage[colorlinks=true,citecolor=blue,linkcolor=blue,urlcolor=blue]{hyperref}
\captionsetup{justification=Centering,
              skip=0.5\baselineskip}
\newcolumntype{L}{>{$}l<{$}} % math-mode version of "l" column type
\usetikzlibrary{decorations.pathreplacing,calc}

\usepackage[algoruled,vlined,linesnumbered]{algorithm2e}
\SetKwInOut{Input}{input}
\SetKwInput{KwInput}{Input}                % Set the Input
\SetKwInput{KwOutput}{Output}              % set the Output
%\usepackage{algorithmic}
\usetikzlibrary{decorations.pathreplacing,calc}
%\usepackage[colorlinks=true,citecolor=blue,linkcolor=blue,urlcolor=blue]{hyperref}

\newcommand{\tikzmark}[1]{\tikz[overlay,remember picture] \node (#1) {};}
\newcommand{\myvec}[1]{\ensuremath{\begin{pmatrix}#1\end{pmatrix}}}

%\usepackage[linesnumbered,vlined,boxed]{algorithm2e}
%% Save the class definition of \subparagraph


\newcommand{\quotes}[1]{``#1''}
\newcommand{\CTL}{\textsf{CTL}}
\newcommand{\LTL}{\textsf{LTL}}
\newcommand{\LTLX}{\textsf{LTL$\setminus$X}}
\newcommand{\FORL}{\textsf{FORL}}
\newcommand{\JAVA}{\textsf{JAVA}}
\newcommand{\Sketch}{\textsf{Sketch}}
\newcommand{\PSketch}{\textsf{PSketch}}
\newcommand{\AlloyStar}{\textsf{Alloy}$^*$}
\newcommand{\Alloy}{\textsf{Alloy}}
\newcommand{\KodKod}{\textsf{KodKod}}
\newcommand{\NuSMV}{\textsf{NuSMV}}
\newcommand{\psketch}{\textsf{PSketch}}
\newcommand{\Party}{\textsf{Party}}

\newcommand{\spec}{Spec}
\newcommand{\tree}{\mathcal{T}}
\newcommand{\modelsss}{\models^{*}}

\newcommand{\E}{\mathbf{E}}
\newcommand{\A}{\mathbf{A}}
\newcommand{\X}{\bigcirc}
\newcommand{\Until}{\mathbf{U}}
\newcommand{\Always}{\Box}
\newcommand{\WUntil}{\mathbf{W}}
\newcommand{\Future}{\Diamond}
\newcommand{\CNF}{\mathsf{CNF}}
\newcommand{\NOT}{\mathsf{NOT}}
\newcommand\proj{\mathord{\uparrow}}
\newcommand{\formula}{\text{Formula}}
\newcommand{\expr}{\text{expr}}
\newcommand{\decl}{\text{decl}}
%\newcommand{\var}{\text{var}}
\newcommand{\typexpr}{\text{typeexpr}}
\newcommand{\probl}{\text{problem}}
\newcommand{\all}{\textsf{all}}
\newcommand{\some}{\textsf{some}}
\newcommand{\conj}{\textsf{and}}
\newcommand{\disj}{\textsf{or}}
\newcommand{\equival}{\textsf{iff}}
\newcommand{\imp}{\textsf{implies}}
\newcommand{\type}{\text{type}}
\newcommand{\inc}{\textsf{in}}
\newcommand{\nega}{\textsf{not}}
\newcommand{\Variable}{\text{Var}}
\newcommand{\variable}{\text{var}}
\newcommand{\traces}[1]{\mathit{Traces}(#1)}
\newcommand{\fairtraces}[1]{\mathit{FairTraces}(#1)}
\newcommand{\Path}[1]{\mathit{Path}(#1)}
\newcommand{\post}[2]{\mathit{Post}(#1,#2)}
\newcommand{\tcpost}[2]{\mathit{Post}^*(#1,#2)}
\newcommand{\lts}[1]{#1 = \langle S, \textit{Act}, \rightarrow, I, \textit{AP}, L \rangle}
\newcommand{\nonamelts}{\langle S, \textit{Act}, \rightarrow, I, \textit{AP}, L \rangle}
\newcommand{\indexedlts}[2]{#1 = \langle S^{#2}, \textit{Act}^{#2}, \rightarrow^{#2}, s^{#2}_0, \textit{AP}^{#2}, L^{#2}\rangle}
\newcommand{\refin}[1]{\mathit{Ref}(#1)}
\newcommand{\processspec}[1]{#1 = \langle \langle \mathit{Sh},  \mathit{Loc}, \mathit{Act}\rangle, \Phi \rangle}
\newcommand{\nonameprocessspec}{\langle \langle \mathit{Sh},  \mathit{Loc}, \mathit{Act}\rangle, \Phi \rangle}
\newcommand{\specif}[1]{#1 = \langle \{ P^i \}_{i \in \mathcal{I}}, \phi \rangle}
\newcommand{\nonamespecif}{\langle \{ P^i \}_{i \in \mathcal{I}}, \phi \rangle}



%\theoremstyle{definition}
%\newtheorem{example}{Example}[section]
%\theoremstyle{definition}
%\newtheorem{definition}{Definition}[section]
%\theoremstyle{definition}
%\newtheorem{theorem}{Theorem}[section]
\newcommand{\lstfont}[1]{\color{#1}\scriptsize\ttfamily}

\newcounter{nalg} % defines algorithm counter for chapter-level
\renewcommand{\thenalg}{\arabic{nalg}} %defines appearance of the algorithm counter
\DeclareCaptionLabelFormat{algocaption}{Algorithm \thenalg} % defines a new caption label as Algorithm x.y

\lstdefinestyle{Spec}{
    %language=Spec,
    showstringspaces=false,
    backgroundcolor=\color{white},
    basicstyle=\lstfont{black},
    identifierstyle=\lstfont{black},
    keywordstyle=\color{black}\bfseries,%\lstfont{magenta!40},
    numberstyle=\lstfont{black},
    stringstyle=\lstfont{cyan},
    commentstyle=\lstfont{red},
    emph={
        action, process,
        spec, invariant, main, 
    },
    emphstyle=\color{black}\bfseries,
    breaklines=true
}

\lstdefinestyle{Unity}{
    %language=Alloy,
    mathescape=true,
    showstringspaces=false,
    backgroundcolor=\color{white},
    basicstyle=\lstfont{black},
    identifierstyle=\lstfont{black},
    keywordstyle=\color{black}\bfseries\em,
    numberstyle=\lstfont{black},
    stringstyle=\lstfont{cyan},
    commentstyle=\lstfont{red},
    emph={
        var, Program, Process, begin, end, initial
    },
    emphstyle={\lstfont{black}\bfseries},
    breaklines=true
}

\lstnewenvironment{algo}[1][] %defines the algorithm listing environment
{   
    \refstepcounter{nalg} %increments algorithm number
    \setcounter{lstlisting}{\value{nalg}}
    \captionsetup{labelformat=algocaption,labelsep=colon} %defines the caption setup for: it ises label format as the declared caption label above and makes label and caption text to be separated by a ':'
    \lstset{ %this is the stype
        mathescape=true,
        frame=tB,
        numbers=left, 
        numberstyle=\tiny,
        %basicstyle=\scriptsize, 
        basicstyle= \scriptsize\ttfamily,%\lstfont{black},
        keywordstyle=\color{black}\bfseries\em,
        %keywordstyle=\lstfont{blue},
        keywords={,input, output, return, datatype, function, in, if, else, foreach, while, begin, end, endif, endwhile, endfor, procedure, then, for, do, all, some, such, that,} %add the keywords you want, or load a language as Rubens explains in his comment above.
        numbers=left,
        xleftmargin=.04\textwidth,
        emphstyle={\bfseries},%{\lstfont{blue}},
        morecomment=[l]{//},  % l is for line comment
        #1 % this is to add specific settings to an usage of this environment (for instnce, the caption and referable label)
    }
}
{}




\addtolength{\textfloatsep}{-0.2in}
%\documentstyle[times,art10,twocolumn,latex8]{article}

%------------------------------------------------------------------------- 
% take the % away on next line to produce the final camera-ready version 
%\pagestyle{empty}

%------------------------------------------------------------------------- 


%\title{Synthesis of Synchronization  Skeletons of Concurrent Programs using SAT Solving and Symbolic Model Checking} 
\title{Bounded Synthesis of Synchronized Distributed Models from Lightweight Specifications} 
\titlerunning{Bounded Synthesis of Synchronized Distributed Models...}
\author{Pablo F.  Castro\inst{1,2},
Luciano Putruele\inst{1,2} \\
Renzo Degiovanni\inst{3},
Nazareno Aguirre\inst{1,2} 
}
%\institute{}
%\authorrunning{Castro et al.}
\institute{Departamento de Computaci\'on, Universidad Nacional de R\'io Cuarto, Argentina \and
Consejo Nacional de Investigaciones Cient\'ificas y T\'ecnicas (CONICET), Argentina \and
Luxembourg Institute of Science and Technology, Luxembourg}

\begin{document}

\maketitle
%\thispagestyle{empty}
\begin{abstract}
We present an approach to automatically synthesize synchronized  models from lightweight formal specifications.  Our approach takes as input a specification of a distributed system along with a global linear time constraint, which must be fulfilled by the interaction of the system's components.  It produces executable models for the component specifications (in the style of Promela language) whose concurrent execution satisfies the global constraint.  The component specifications consist of  a collection of actions described by means of pre and post conditions together with first-order relational formulas prescribing their behavior.  We use the \emph{Alloy Analyzer} to encode the component specifications and enumerate their potential implementations up to some bound,  whose concurrent composition is model checked against the global property.  Even though this approach is sound and complete up to the selected bound,  it is impractical as the number of candidate implementations grows exponentially.  
To address this, we propose an algorithm that uses batches of counterexamples to prune the solution space, it has two main phases:  \emph{exploration}, 
the algorithm collects a batch of counterexamples, and \emph{exploitation},   where this knowledge is used to speed up the search.
The approach is sound,  while  its completeness depends on the batches used.
We present a prototype tool,  describe some  experiments,  and compare it with  related approaches.

%We assume  multi-threaded shared-variables model of computation. Our approach takes as input a collection of specifications for a distributed system, expressed with pre/post-conditions for their actions and first-order (local) constraints, together with a global temporal constraint, to be achieved by the process interaction. We encode the process specifications into Alloy, and use Alloy Analyzer to enumerate potential implementations of the (local) processes, whose concurrent composition is model checked against the global property. If successful, a distributed algorithm is obtained; otherwise, the obtained counterexamples are used to guide the SAT-based search, by incrementally refining the local process implementations. We show that our approach is sound and complete, for a given bound on the size of the process implementations. We also present a prototype of our approach and its application to well-known case studies of distributed algorithms. 
%%%%%VERSION ANTERIOR
%In this paper, we present an approach to automatically synthesize synchronization code for distributed programs, assuming a multithread shared-variables model of computation. Our method combines SAT solving over specifications written in First-Order Relational Logic, and symbolic  model checking. Intuitively, our approach starts by using SAT solving to enumerate potential implementations of the (local) processes, and checking if there are executions of their concurrent composition that falsify the required global properties, this is done until a valid synchronization code is obtained. This latter step is  performed  via a symbolic model checker, the obtained counterexamples are used to incrementally refine the SAT-based search for local process implementations. We  developed a prototype of our approach and applied it to well-known case studies of distributed algorithms.	
%%%%%%%%%%	 
	 
%   Developing \emph{correct} concurrent programs is known to be a very challenging task, this is mainly due to subtle misbehaviours that may arise in concurrent settings, such as deadlocks and race conditions. Many of these situations are due to an incorrect use of synchronization primitives such as locks and semaphores. Furthermore, this situation is even worse for distributed algorithms, i.e., concurrent programs whose components act independently to achieve a common goal, which usually lack from a centralized control.
	 

%	terms of the set of actions that each process is allowed to execute, together with a collection of global properties, expressed in temporal logic, that the concurrent execution of the processes must satisfy. Intuitively, our approach starts by using SAT solving to enumerate potential implementations of the (local) processes, and checking whether their concurrent composition satisfies the required global properties, until a valid synchronization skeleton is obtained. This latter step is performed using symbolic model checking, and the obtained counterexamples are used to incrementally refine the SAT-based search for local process implementations. We show that our approach is correct and (bounded-)complete, under user predefined bounds for process size (in terms of their maximum number of states)

%Concurrency enables software efficiency improvements by exploiting modern multi-processor hardware, but developing \emph{correct} concurrent programs is known to be a very challenging task, due to subtle misbehaviours that arise in concurrent settings, such as deadlocks and race conditions. Many of these situations are due to an incorrect use of synchronization primitives such as locks and semaphores, which can often be decoupled from the actual computations being carried out by the concurrent processes, and avoided through the use of appropriate \emph{synchronization skeletons}. In this paper, we present an approach to automatically synthesize synchronization skeletons of concurrent programs, that combines SAT solving and symbolic model checking. Our approach takes as input a set of process specifications, described in terms of the set of actions that each process is allowed to execute, together with a collection of global properties, expressed in temporal logic, that the concurrent execution of the processes must satisfy. Intuitively, our approach starts by using SAT solving to enumerate potential implementations of the (local) processes, and checking whether their concurrent composition satisfies the required global properties, until a valid synchronization skeleton is obtained. This latter step is performed using symbolic model checking, and the obtained counterexamples are used to incrementally refine the SAT-based search for local process implementations. We show that our approach is correct and (bounded-)complete, under user predefined bounds for process size (in terms of their maximum number of states). Moreover, we assess our approach and show that it can effectively produce synchronization skeletons for a number of case studies commonly found in the literature.
\end{abstract}

\section{Introduction}


\begin{figure}[t]
\centering
\includegraphics[width=0.6\columnwidth]{figures/evaluation_desiderata_V5.pdf}
\vspace{-0.5cm}
\caption{\systemName is a platform for conducting realistic evaluations of code LLMs, collecting human preferences of coding models with real users, real tasks, and in realistic environments, aimed at addressing the limitations of existing evaluations.
}
\label{fig:motivation}
\end{figure}

\begin{figure*}[t]
\centering
\includegraphics[width=\textwidth]{figures/system_design_v2.png}
\caption{We introduce \systemName, a VSCode extension to collect human preferences of code directly in a developer's IDE. \systemName enables developers to use code completions from various models. The system comprises a) the interface in the user's IDE which presents paired completions to users (left), b) a sampling strategy that picks model pairs to reduce latency (right, top), and c) a prompting scheme that allows diverse LLMs to perform code completions with high fidelity.
Users can select between the top completion (green box) using \texttt{tab} or the bottom completion (blue box) using \texttt{shift+tab}.}
\label{fig:overview}
\end{figure*}

As model capabilities improve, large language models (LLMs) are increasingly integrated into user environments and workflows.
For example, software developers code with AI in integrated developer environments (IDEs)~\citep{peng2023impact}, doctors rely on notes generated through ambient listening~\citep{oberst2024science}, and lawyers consider case evidence identified by electronic discovery systems~\citep{yang2024beyond}.
Increasing deployment of models in productivity tools demands evaluation that more closely reflects real-world circumstances~\citep{hutchinson2022evaluation, saxon2024benchmarks, kapoor2024ai}.
While newer benchmarks and live platforms incorporate human feedback to capture real-world usage, they almost exclusively focus on evaluating LLMs in chat conversations~\citep{zheng2023judging,dubois2023alpacafarm,chiang2024chatbot, kirk2024the}.
Model evaluation must move beyond chat-based interactions and into specialized user environments.



 

In this work, we focus on evaluating LLM-based coding assistants. 
Despite the popularity of these tools---millions of developers use Github Copilot~\citep{Copilot}---existing
evaluations of the coding capabilities of new models exhibit multiple limitations (Figure~\ref{fig:motivation}, bottom).
Traditional ML benchmarks evaluate LLM capabilities by measuring how well a model can complete static, interview-style coding tasks~\citep{chen2021evaluating,austin2021program,jain2024livecodebench, white2024livebench} and lack \emph{real users}. 
User studies recruit real users to evaluate the effectiveness of LLMs as coding assistants, but are often limited to simple programming tasks as opposed to \emph{real tasks}~\citep{vaithilingam2022expectation,ross2023programmer, mozannar2024realhumaneval}.
Recent efforts to collect human feedback such as Chatbot Arena~\citep{chiang2024chatbot} are still removed from a \emph{realistic environment}, resulting in users and data that deviate from typical software development processes.
We introduce \systemName to address these limitations (Figure~\ref{fig:motivation}, top), and we describe our three main contributions below.


\textbf{We deploy \systemName in-the-wild to collect human preferences on code.} 
\systemName is a Visual Studio Code extension, collecting preferences directly in a developer's IDE within their actual workflow (Figure~\ref{fig:overview}).
\systemName provides developers with code completions, akin to the type of support provided by Github Copilot~\citep{Copilot}. 
Over the past 3 months, \systemName has served over~\completions suggestions from 10 state-of-the-art LLMs, 
gathering \sampleCount~votes from \userCount~users.
To collect user preferences,
\systemName presents a novel interface that shows users paired code completions from two different LLMs, which are determined based on a sampling strategy that aims to 
mitigate latency while preserving coverage across model comparisons.
Additionally, we devise a prompting scheme that allows a diverse set of models to perform code completions with high fidelity.
See Section~\ref{sec:system} and Section~\ref{sec:deployment} for details about system design and deployment respectively.



\textbf{We construct a leaderboard of user preferences and find notable differences from existing static benchmarks and human preference leaderboards.}
In general, we observe that smaller models seem to overperform in static benchmarks compared to our leaderboard, while performance among larger models is mixed (Section~\ref{sec:leaderboard_calculation}).
We attribute these differences to the fact that \systemName is exposed to users and tasks that differ drastically from code evaluations in the past. 
Our data spans 103 programming languages and 24 natural languages as well as a variety of real-world applications and code structures, while static benchmarks tend to focus on a specific programming and natural language and task (e.g. coding competition problems).
Additionally, while all of \systemName interactions contain code contexts and the majority involve infilling tasks, a much smaller fraction of Chatbot Arena's coding tasks contain code context, with infilling tasks appearing even more rarely. 
We analyze our data in depth in Section~\ref{subsec:comparison}.



\textbf{We derive new insights into user preferences of code by analyzing \systemName's diverse and distinct data distribution.}
We compare user preferences across different stratifications of input data (e.g., common versus rare languages) and observe which affect observed preferences most (Section~\ref{sec:analysis}).
For example, while user preferences stay relatively consistent across various programming languages, they differ drastically between different task categories (e.g. frontend/backend versus algorithm design).
We also observe variations in user preference due to different features related to code structure 
(e.g., context length and completion patterns).
We open-source \systemName and release a curated subset of code contexts.
Altogether, our results highlight the necessity of model evaluation in realistic and domain-specific settings.





\section{Related Work}\label{sec:related_works}
\gls{bp} estimation from \gls{ecg} and \gls{ppg} waveforms has received significant attention due to its potential for continuous, unobtrusive monitoring. Earlier work relied on classical machine learning with handcrafted features, but deep learning methods have since emerged as more robust alternatives. Convolutional or recurrent architectures designed for \gls{ecg}/\gls{ppg} have shown strong performance, including ResUNet with self-attention~\cite{Jamil}, U-Net variants~\cite{Mahmud_2022}, and hybrid \gls{cnn}--\gls{rnn} models~\cite{Paviglianiti2021ACO}. These architectures often outperform traditional feature-engineering approaches, particularly when both \gls{ecg} and \gls{ppg} signals are used~\cite{Paviglianiti2021ACO}.

Nevertheless, many existing methods train solely on \gls{ecg}/\gls{ppg} data, which, while plentiful~\cite{mimiciii,vitaldb,ptb-xl}, often exhibit significant variability in signal quality and patient-specific characteristics. This variability poses challenges for achieving robust generalization across populations. Recent work has explored transfer learning to overcome these issues; for example, Yang \emph{et~al.}~\cite{yang2023cross} studied the transfer of \gls{eeg} knowledge to \gls{ecg} classification tasks, achieving improved performance and reduced training costs. Joshi \emph{et~al.}~\cite{joshi2021deep} also explored the transfer of \gls{eeg} knowledge using a deep knowledge distillation framework to enhance single-lead \gls{ecg}-based sleep staging. However, these studies have largely focused on within-modality or narrow domain adaptations, leaving open the broader question of whether an \gls{eeg}-based foundation model can serve as a versatile starting point for generalized biosignal analysis.

\gls{eeg} has become an attractive candidate for pre-training large models not only because of the availability of large-scale \gls{eeg} repositories~\cite{TUEG} but also due to its rich multi-channel, temporal, and spectral dynamics~\cite{jiang2024large}. While many time-series modalities (for example, voice) also exhibit rich temporal structure, \gls{eeg}, \gls{ecg}, and \gls{ppg} share common physiological origins and similar noise characteristics, which facilitate the transfer of temporal pattern recognition capabilities. In other words, our hypothesis is that the underlying statistical properties and multi-dimensional dynamics in \gls{eeg} make it particularly well-suited for learning robust representations that can be effectively adapted to \gls{ecg}/\gls{ppg} tasks. Our work is the first to validate the feasibility of fine-tuning a transformer-based model initially trained on EEG (CEReBrO~\cite{CEReBrO}) for arterial \gls{bp} estimation using \gls{ecg} and \gls{ppg} data.

Beyond accuracy, real-world deployment of \gls{bp} estimation models calls for efficient inference. Traditional deep networks can be computationally expensive, motivating recent interest in quantization and other compression techniques~\cite{nagel2021whitepaperneuralnetwork}. Few studies have combined large-scale pre-training with post-training quantization for \gls{bp} monitoring. Hence, our method integrates these two aspects: leveraging a potent \gls{eeg}-based foundation model and applying quantization for a compact, high-accuracy cuffless \gls{bp} solution.
\section{Motivation}
\begin{figure*}
    \centering
    \setlength{\abovecaptionskip}{0cm}
    \includegraphics[width=1.0\linewidth]{Figure/motivation_sample.pdf}
    \caption{(a), (b), and (c) show real code snippets from an early Airpush version, a later Airpush version, and the Hiddad adware family. Airpush's core behavior includes: (1) get ad data from a specific URL, (2) asynchronous execution to avoid user interruptio, and (3) push ads continuously through the background service. (a) and (b) demonstrate that both Airpush versions share invariant behaviors, with similar API calls and permissions despite implementation differences. Hiddad, while skipping step (2) for simpler ad display, shares steps (1) and (3) with Airpush, especially the newer version.}
    \label{fig:motivation_sample}
\end{figure*}

\subsection{Invariance in Malware Evolution}
% Malware families commonly evolve to circumvent new detection techniques and security measures, resulting in constant changes in their code implementations or API calls. These changes cause the feature space, extracted from applications, to gradually deviate from the initial decision boundaries of malware detectors, thereby significantly degrading detection performance. However, we have identified that during the evolution of malware, core malicious behaviors and execution logic exhibit a certain degree of invariance. This invariance is reflected in the training set through specific intents, permissions, and function calls, which are captured by feature extraction techniques. We categorize this invariance into two types: intra-family invariance and inter-family invariance. 
Malware families commonly evolve to circumvent new detection techniques and security measures, resulting in constant changes in their code implementations or API calls to bypass detection. This leads to drifts in the feature space and a decline in detection accuracy. Yet, we argue that during the evolution of malware, core malicious behaviors and execution patterns remain partially invariant, captured in training data through intents, permissions, and function calls. We define these invariances as intra- and inter-family invariance.
\begin{itemize}
    \item Intra-family invariance: While versions within a malware family may vary in implementation, their core malicious intent remains relatively stable.
    \item Inter-family invariance: Certain malicious behavior patterns are consistent across different malware families. As malware trends shift, even new families emerging after detector training may share malicious intents with families in the training set.
\end{itemize}
To illustrate invariant malicious behaviors in drift scenarios, we select APKs from Androzoo\footnote{https://androzoo.uni.lu} and decompile them using JADX\footnote{https://github.com/skylot/jadx} to obtain .java files. Our analysis focuses on core malicious behaviors in the source code. For intra-family invariance, we use versions of the Airpush family, known for intrusive ad delivery, from different periods. For inter-family invariance, we examine the Hiddad family, which shares aggressive ad delivery and tracking tactics but uses broader permissions, increasing privacy risks. Figure.\ref{fig:motivation_sample} shows code snippets with colored boxes highlighting invariant behaviors across samples. While Airpush uses asynchronous task requests, Hiddad relies on background services and scheduled tasks to evade detection.

% To highlight the invariance of malicious behaviors in drift scenarios, we selected real APK files from Androzoo~\footnote{https://androzoo.uni.lu} platform and decompiled them using the jadx tool~\footnote{https://github.com/skylot/jadx} to obtain their corresponding .java files. The invariance analysis is conducted on the core malicious behaviors represented in the source code. For intra-family invariance, we used the long-standing Airpush malware family as an example, selecting versions from different periods, which is known for its intrusive ad delivery. For inter-family invariance, we chose the Hiddad adware family, which emerged later. Airpush and Hiddad rely on aggressive ad delivery and user tracking, but Hiddad uses broader permissions, violating more privacy. Figure.\ref{fig:motivation_sample} presents code snippets from these malware families, with colored boxes highlighting the invariant malicious behaviors shared across different family samples. Since Hiddad prioritizes ad delivery via background services and scheduled tasks to avoid suspicion and detection, it omits the asynchronous task requests used in step two by Airpush.

Figure~\ref{fig:motivation_sample}(a)\footnote{MD5: 17950748f9d37bed2f660daa7a6e7439} and (b)\footnote{MD5: ccc833ad11c7c648d1ba4538fe5c0445} show core code from this family in 2014 and later years, respectively. The 2014 version uses \verb|NotifyService| and \verb|TimerTask| to notify users every 24 hours, maintaining ad exposure. The later version, adapting to Android 8.0’s restrictions, triggers \verb|NotifyService| via \verb|BroadcastReceiver| with \verb|WAKE_LOCK| to sustain background activity. In Drebin’s~\cite{Arpdrebin} feature space, these invariant behaviors are captured through features like \verb|android_app_NotificationManager;notify|, \verb|permission_READ_PHONE_STATE| and so on. Both implementations also use \verb|HttpURLConnection| for remote communication, asynchronously downloading ads and tracking user activity, and sharing Drebin features such as \verb|java/net/HttpURLConnection| and \verb|android_permission_INTERNET|.

Similarly, Figure.~\ref{fig:motivation_sample}(c)\footnote{MD5: 84573e568185e25c1916f8fc575a5222} shows a real sample from the Hiddad family, which uses HTTP connections for ad delivery, along with \verb|AnalyticsServer| and \verb|WAKE_LOCK| for continuous background services. Permissions like \verb|android_permission_WAKE_LOCK| and API calls such as \verb|getSystemService| reflect shared, cross-family invariant behaviors, whose learning would enhance model detection across variants.

Capturing the core malicious behaviors of Airpush aids in detecting both new Airpush variants and the Hiddad family, as they share similar malicious intents. These stable behaviors form consistent indicators in the feature space. However, detectors with high validation performance often fail to adapt to such variants, underscoring the need to investigate root causes and develop a drift-robust malware detector.

% Based on this analysis, learning features that represent the core malicious behaviors of the Airpush family not only aids in detecting new Airpush variants but also helps in identifying the Hiddad family. However, malware detectors trained on historical data often fail to effectively detect these malicious samples. Even if the model achieves near-perfect performance on the validation set, its performance deteriorates significantly over time, prompting us to further investigate the root causes and build a drift-robust malware detector.

\begin{center}
\fcolorbox{black}{gray!10}{\parbox{.9\linewidth}{\textit{\textbf{Take Away}: The feature space of training samples contains invariance within and among malware families to be learned.}}}
\end{center}

% The analysis suggests that learning invariant features from the Airpush family not only aids in detecting newer versions of Airpush but also improves the detection of Hiddad. However, in practice, detectors trained on these features fail to effectively identify such samples, with performance degrading over time despite near-perfect validation set results. This motivates further our exploration of an ideal malware detector that can learn these invariant behaviours and remain robust in its discrimination throughout malware evolution.

% We present several representative pseudo-code implementations of malicious software to demonstrate invariance visually. For intra-family invariance, we selected Rootkit, a common malware family present in both the training and testing phases. This family exploits system vulnerabilities to obtain root privileges, enabling high-privilege operations such as modifying system files or the kernel, and intercepting system calls to mask its behavior, showcasing a deep confrontation with the operating system’s security mechanisms. Rootkit execution can be simplified into three main steps: file hiding, root privilege acquisition, and system call hooking. With the introduction of stricter permission management and security measures in Android 6.0, this family had to rely on more complex kernel-level attacks to maintain stealth. Figure X illustrates two simplified executions of Rootkit before and after this update, with (a) representing the earlier version. While (b) introduces more complex system calls like openat and fork for file and process hiding, it retains the core semantics from (a). Both versions use the \textit{interceptSystemCall()} function to intercept system calls. The earlier version intercepted \textit{readdir} to hide malicious files, whereas later versions achieved more sophisticated file hiding by intercepting \textit{openat}. Additionally, both versions implement system-level privilege escalation. The only difference in the new version is that it intercepts the fork system call, returning an error code to hide the malicious process \textit{com.malicious.app}. 

% Privilege escalation and process hiding are also commonly used by other malware families. Figure (c) shows a pseudo-code from the Spyware family, which specializes in stealing sensitive user information while maintaining stealth. Thus, in terms of execution logic, Spyware shares root privilege escalation and process hiding with Rootkit, but additionally calls \textit{TelephonyManager} to collect SIM card serial numbers and send them to a remote server. This is understandable, as despite the diversity of malware families, core malicious behaviors can be categorized into a limited number of types\cite{malradar}. Therefore, we conclude that stable patterns indicative of malicious behavior exist in the training samples. While new functionalities will inevitably emerge, once these patterns are learned, the malware detector will exhibit some robustness against drift.








% \subsection{The Contribution of Features to Detectors}
% \subsection{Create Ideal Drift-robust Malware Detector}
\subsection{Failure of Learning Invariance}
% \subsubsection{Vanilla Malware Detector}
Let $f_r \in \mathcal{R}$ be a sample in the data space with label $y \in \mathcal{Y} = {0, 1}$, where 0 represents benign software and 1 represents malware. The input feature vector $x \in \mathcal{X}$ includes features $\mathcal{F}$ extracted from $f_r$ according to predefined rules. The goal of learning-based malware detection is to train a model $\mathcal{M}$ based on $\mathcal{F}$, mapping these features into a latent space $\mathcal{H}$ and passing them to a classifier for prediction. The process is formally described as follows:

\begin{equation}
\arg \min _{\theta} R_{erm}\left(\mathcal{F}\right)
\end{equation}
where $\theta$ is the model parameter to be optimized and $R_{erm}(\mathcal{F})$ represents the expected loss based on features space $\mathcal{F}$, defined as:
\begin{equation}
R_{erm}\left(\mathcal{F}\right)=\mathbb{E}[\ell(\hat{y}, y)].
\end{equation}
$\ell$ is a loss function. By minimizing the loss function, $\mathcal{M}$ achieves the lowest overall malware detection error. 

% Let $r \in \mathcal{R}$ represent a sample in the original data space, where the corresponding label is denoted as $y \in \mathcal{Y} = \{0, 1\}$, with 0 indicating benign software and 1 indicating malware. The input feature vector $x \in \mathcal{X}$ comprises features $\mathcal{F}$ extracted from $r$ according to specific rules. The objective of learning-based malware detection schemes is to learn a model $\mathcal{M}$ based on the feature set $\mathcal{F}$, which maps the features into a latent space $\mathcal{H}$ and feeds them into a classifier to generate predictions. The process is formally described as follows:

% However, as malware evolves, the model's performance gradually degrades. This motivates us to explore an ideal malware detector that can consistently maintain strong discriminative power throughout malware evolution.

% To explore the robustness of various features in response to malware evolution and their contribution to the performance of the detector, we define two key properties of the features used for training: stability and discriminability. Therefore, the aforementioned objective encourages the model to learn discriminative features that can minimize the loss function. However, as we know, this objective may fail during the evolution of malware. 

% This optimization objective encourages the model to learn discriminative features that minimize the loss function. However, the aforementioned objective function may fail during the evolution of malware. Therefore, to explore the robustness of various features in response to malware evolution and their contribution to the performance of the detector, we define two key properties of the features used for training: stability and discriminability. Intuitively, an ideal drift-robust malware detector would be designed to learn features that are both stable and discriminative. 


\subsubsection{Stability and Discriminability of Features}
\label{active ratio}
To investigate the drift robustness in malware evolution from the feature perspective, we introduce two key properties of features: stability and discriminability. Stability refers to a feature's ability to maintain consistent relevance across different distributions, while discriminability reflects a feature's capacity to distinguish different categories effectively. Typically, feature analysis relies on model performance and architecture, which may introduce bias in defining these feature properties. Therefore, we propose a modelless formal definition, making it applicable across various model architectures. 

Let $f_j$ represent the $j$-th feature in the feature set $\mathcal{F}$, and $S$ denote the set of all samples. To capture the behavior of feature $f_j$ under different conditions, we compute its active ratio over a subset $S^{\prime} \subseteq S$, representing how frequently or to what extent the feature is ``active'' within that subset. Specifically, for a binary feature space, feature $f_j$ takes values 0 or 1 (indicating the absence or presence of the feature, respectively), the active ratio of $f_j$ in the subset $S^{\prime}$ is defined as the proportion of samples where $f_j$ is present, which is defined as Eq.~\ref{active ratio}:
\begin{equation}
\label{active ratio}
r\left(f_j, S^{\prime}\right)=\frac{1}{\left|S^{\prime}\right|} \sum_{s \in S^{\prime}} f_j(s) 
\end{equation}
The ratio measures how frequently the feature is activated within the subset $S^{\prime}$ relative to the total number of samples in the subset. At this point, we can define the stability and discriminability of features.

\begin{myDef} 
\textbf{Stable Feature}: A feature $f_j$ is defined as stable if, for any sufficiently large subset of samples $S^{\prime} \subseteq S$, the active ratio $r\left(f_j, S^{\prime}\right)$ remains within an $\epsilon$-bound of the overall active ratio $r\left(f_j, S\right)$ across the entire sample set, regardless of variations in sample size or composition. Formally, $f_j$ is stable if:
\begin{equation}
\forall S^{\prime} \subseteq S,\left|S^{\prime}\right| \geq n_0, \quad\left|r\left(f_j, S^{\prime}\right)-r\left(f_j, S\right)\right| \leq \epsilon    
\end{equation}
where $\epsilon>0$ is a small constant, and $n_0$ represents a minimum threshold for the size of $S^{\prime}$ to ensure the stability condition holds.
\end{myDef}

When we consider discriminability, there is a need to focus on the category to which the sample belongs. Thus, let $C=\left\{C_1, C_2, \ldots, C_k\right\}$ be a set of $k$ classes, and $S_k \subseteq S$ be the subset of samples belonging to class $C_k$. The active ratio of feature $f_j$ in class $C_k$ is given by:
\begin{equation}
  r\left(f_j, S_k\right)=\frac{1}{\left|S_k\right|} \sum_{s \in S_k} f_j(s)  
\end{equation}

\begin{myDef}
\textbf{Discriminative Feature}: A feature $f_j$ is discriminative if its active ratio differs significantly between at least two classes, $C_p$ and $C_q$. Specifically, there exists a threshold $\delta > 0$ such that:
\begin{equation}
  \exists C_p, C_q \in C, p \neq q, \quad\left|r\left(f_j, S_p\right)-r\left(f_j, S_q\right)\right| \geq \delta  
\end{equation}
\end{myDef}
Furthermore, the discriminative nature of the feature should be independent of the relative class sizes, meaning that the difference in activation should remain consistent despite variations in the proportion of samples in different classes. Mathematically, for any subset $\tilde{S}_p \subseteq S_p$ and $\tilde{S}_q \subseteq S_q$, where $\left|\tilde{S}_p\right| \neq\left|S_p\right|$ or $\left|\tilde{S}_q\right| \neq\left|S_q\right|$, the discriminative property still holds:
\begin{equation}
    \left|r\left(f_j, \tilde{S}_p\right)-r\left(f_j, \tilde{S}_q\right)\right| \geq \delta
\end{equation}

% \begin{figure}
%     \centering
%     % \setlength{\abovecaptionskip}{0.5cm}
%     \includegraphics[width=\linewidth]{Figure/feature_diff_10.pdf}
%     \caption{Discriminative change of Top 10 discriminative features in the training set during the test phase}
%     \label{fig:f1_family}
% \end{figure}
\begin{figure*}[htbp]
    \centering
    \begin{subfigure}[t]{0.48\textwidth}  
        \centering
        \includegraphics[width=1.0\textwidth]{Figure/feature_diff_10.pdf}
        \caption{}
        \label{fig:diff}
    \end{subfigure}
    \hfill
    \begin{subfigure}[t]{0.48\textwidth}
        \centering
        \includegraphics[width=1.0\textwidth]{Figure/feature_importance_10.pdf}
        \caption{}
        \label{fig:importance}
    \end{subfigure}
    
    \caption{(a) and (b) illustrate changes in the Discriminability of the top 10 discriminative training features and the top 10 important testing features, respectively. ``Discriminability'' is defined as the absolute difference in active ratios between benign and malicious samples. The grey dotted line indicates the start of the testing phase, with preceding values representing each feature's discriminability across months in the training set.}
    \label{fig:feature_discrimination}
\end{figure*}


\subsubsection{Failure Due to Learning Unstable Discriminative Features}
\label{motivation: failure}
The high test set performance of the malware detector within the same period suggests that it effectively learns discriminative features to distinguish benign software from malware. However, our analysis reveals that performance degradation over time is mainly due to the model's inability to capture stable discriminative features from the training set. To illustrate this, we sample 110,723 benign and 20,790 malware applications from the Androzoo\footnote{https://androzoo.uni.lu} platform (2014-2021). Applications are sorted by release date, with 2014 samples used for training and subsequent data divided into 30 equally spaced test intervals. We extract DREBIN~\cite{Arpdrebin} features, covering nine behavioral categories such as hardware components, permissions, and restrict API calls, and select the top 10 discriminative features based on active ratio differences to track over time.

The model configuration follows DeepDrebin~\cite{Grossedeepdrebin}, a three-layer fully connected neural network with 200 neurons per layer. We evaluate performance in each interval using macro-F1 scores. As shown in Figure~\ref{fig:feature_discrimination}, although the top 10 discriminative features maintain stable active ratios, the detector’s performance consistently declines. We further examine feature importance over time using Integrated Gradients (IG) with added binary noise, averaging results across five runs to ensure robustness, as recommended by Warnecke et al.~\cite{IG_explain}.

Figure~\ref{fig:feature_discrimination} presents the top 10 discriminative (a) and important features (b) identified by the model and their active ratio changes. While stable, highly discriminative features from the training set persist through the test phase, the ERM-based detector often relied on unstable features whose discriminative power fluctuated over time. This reliance leads to inconsistent model performance, stabilizing only when the feature discriminative power remains steady. Thus, we attribute the failure of ERM-based malware detectors in drift scenarios to their over-reliance on these unstable features and under-learning of already existing stable discriminative features, limiting its generalization to new samples.

% Figure.\ref{fig:importance} presents the top 10 important features identified by the model and their active ratio changes. The results show that while the highly discriminative features in the training set remained relatively stable during the test phase, the detector trained under empirical risk minimization (ERM) tended to rely on transient discriminative features. The discriminative power of certain features that the model focused on significantly fluctuated during the test phase, and does not seem to be significant enough in the training set. Moreover, when the discriminative power of features remained stable, the model’s performance also stabilized. Therefore, we attribute the failure of ERM-based malware detectors in drift scenarios to their over-reliance on these unstable features and under-learning of already existing stable discriminative features, limiting its generalization to new samples.

Moreover, we observe that highly discriminative features are often associated with high-permission operations and indicate potential malicious activity. For instance, features like \verb|api_calls::java/lang/Runtime;->exec| and \verb|GET_TASKS| are rarely used in legitimate applications. This aligns with malware invariance over time, where core malicious intents remain stable even as implementation details evolve.

\begin{center}
\fcolorbox{black}{gray!10}{\parbox{.9\linewidth}{\textit{\textbf{Take Away}: There are stable and highly discriminative features representing invariance in the training samples, yet current malware detectors fail to learn these features leading to decaying models' performance.}}}
\end{center}


\subsection{Create Model to Learn Invariance}
\label{learn_invariant_feature}
Our discussion highlights the importance of learning stable, discriminative features for drift-robust malware detection. ERM captures features correlated with the target variable, including both stable and unstable information~\cite{understanding}. When unstable information is highly correlated with the target, the model tends to rely on it. Thus, the key challenge is to isolate and enhance stable features, aligning with the goals of invariant learning outlined in Section~\ref{invariant_learning}.
% Our preceding discussion emphasized the importance of learning stable discriminative features for building drift-robust malware detectors. The goal of Empirical Risk Minimization (ERM) is to capture features closely related to the target variable, including both stable and unstable information, and the model is more inclined to rely on transient information when it is more relevant to the target~\cite{understanding}. The main challenge is therefore to isolate the stable component, which is consistent with the invariant learning goal described in Section~\ref{invariant_learning}. 

However, applying invariant learning methods is challenging. Its effectiveness presupposes firstly that the environment segmentation can expose the unstable information that the model needs to forget~\cite{environment_label, env_label}. In malware detection, it is uncertain which application variants will trigger distribution changes. Effective invariant learning requires the encoder to produce rich and diverse representations that provide valuable information for the invariant predictor~\cite{yang2024invariant}. Without high-quality representations, invariant learning may fail. This is also reflected in Figure~\ref{fig:feature_discrimination}, where even in the training phase, the learnt features are still deficient in discriminating between goodware and malware, and hard to fully represent the execution purpose of malware, relying instead on easily confusing features.

Thus, given arbitrary malware detectors, our intuition is to use time-aware environment segmentation to naturally expose the instability in malware distribution drift. Within each environment, ERM assumptions guide associations with the target variable, while the encoder provides both stable and unstable features for the invariant predictor. By minimizing invariant risk, unstable elements are filtered, thereby enhancing the detector's generalization capability.

% Thus, given arbitrary malware detectors, our scheme aims to further extend the learning capability of the encoder to provide more learnable information to the invariant predictor by relying only on the time-aware environment segmentation. Minimizing the invariant risk, in turn, filters out instability in the feature representation, thus further enhancing the generalization ability of the malware detector.

\begin{center}
\fcolorbox{black}{gray!10}{\parbox{.9\linewidth}{\textit{\textbf{Take Away}: Invariant learning helps to learn temporal stable features, but it is necessary to ensure that the training set can expose unstable information and the encoder can learn rich and good representations.}}}
\end{center}



%!TEX root=main.tex
\section{System Specifications}

In this section, we provide a more formal and detailed definition of specifications, as introduced in the previous section.  In the following definitions we  mainly use the first-order logic with transitive closure described in Section~2.  We start by giving a precise definition of the notion of component specification.

\begin{definition} A \emph{process (or component) specification} $\mathit{PS}$ is a tuple $\langle \langle \mathit{Sh},  \mathit{Loc},$ $\mathit{Act}\rangle, \Phi \rangle$ where $\mathit{Sh}$, $\mathit{Loc}$, $\mathit{Act}$ are finite and mutually disjoint sets, and $\Phi$ is a finite set of first-order (relational) formulas over the vocabulary defined by $\mathit{Sh} \cup \mathit{Loc} \cup \mathit{Act}$.
\end{definition}
Intuitively,  \textit{Sh} are the shared variables used by the process,  \textit{Loc} are the process's local variables, and \textit{Act} are the process's actions. A process specification defines a collection of LTSs satisfying the requirements in the specification. Given a process specification $\textit{PS}=\langle \langle \mathit{Sh},  \mathit{Loc}, \mathit{Act}\rangle, \Phi \rangle$ and an LTS $T=\langle S, \mathit{Act}, \rightarrow, I, \mathit{Sh}\cup\mathit{Loc} , L\rangle$, we write $T \vDash PS$ iff $T \vDash \phi$ for every $\phi \in \Phi$.
 
A specification of a distributed system is a collection of process specifications with the same shared variables, and an additional global temporal requirement.
\begin{definition} A system \emph{specification} $\mathcal{S}$ is a tuple $\langle \{ \textit{PS}^i \}_{i \in \mathcal{I}},$ $ \phi \rangle$, where $\mathcal{I}$ is a finite index set, each $\textit{PS}^i = \langle \langle \mathit{Sh}, \mathit{Loc}^i, \mathit{Act}^i\rangle,$ $\Phi^i \rangle$ is a \emph{local} process specification, and $\phi$ is a \emph{global} requirement expressed by an {\LTL} formula over the vocabulary $\mathit{Sh} \cup \coprod_{i \in \mathcal{I}} \mathit{Loc}^i$.
\end{definition}

By simply considering that a system specification $\mathcal{S}$ is a collection of component specifications, we do not provide any specific semantics to shared variables, compared to process local variables. As a consequence,  locally correct implementations may often lead to invalid system implementations (system implementations violating the constraints), since when putting the process implementations together,  any local assumptions on shared variables may simply not hold. This can be solved by forcing local process specifications to assume unrestricted behavior of the environment, which is formalized as follows: 
 
 \begin{definition}\label{def:system-spec}
Let $\mathcal{L} = \{\ell_0, \dots, \ell_m \}$ be a set of elements (called \emph{locks}). An $\mathcal{L}$-\emph{synchronized specification} is a system specification $\mathcal{S} = \langle \{\textit{PS}^i\}_{i \in \mathcal{I}}, \phi \rangle$ with \emph{shared} variables $\{\mathit{av}_{\ell_0}, \dots,$ $\mathit{av}_{\ell_m}\}  \subseteq \mathit{Sh}$, and where each process specification $\textit{PS}^i = \langle \langle \mathit{Sh}, \mathit{Loc}^i, \mathit{Act}^i\rangle, \Phi^i \rangle$ has local variables  $\{ \mathit{own}_{\ell_0}, \dots,$ $\mathit{own}_{\ell_m} \} \subseteq \mathit{Loc}^i$, and actions $\{\textit{ch}_{\ell_0},\dots,\textit{ch}_{\ell_m}\} \cup \{\mathit{ch}_g \mid g \in \mathit{Sh}\} \subseteq \mathit{Act}^i$. Furthermore, the following formulas belong to each $\Phi^i$:
\begin{enumerate}[(a)]%[font=\normalfont]
	\item\label{system-spec-formula1}	 $\bigwedge_{\ell \in \mathcal{L}}(\forall s : \textit{own}_\ell(s) \Rightarrow   \neg \mathit{av}_\ell(s))$,
	\item\label{system-spec-formula2} $\bigwedge_{\ell \in \mathcal{L}}(\forall s  : \neg \textit{own}_\ell(s) \equiv (\exists s'  : s \overset{\textit{ch}_\ell}{\rightarrow} s'))$,
	\item\label{system-spec-formula3} $\bigwedge_{\ell \in \mathcal{L}}(\forall s,s'  : s \overset{\textit{ch}_\ell}{\rightarrow}{s'} \Rightarrow (\textit{av}_\ell(s) \equiv  \neg \textit{av}_\ell(s'))$,
	\item\label{system-spec-formula4} $\bigwedge_{\ell \in \mathcal{L}}\bigwedge_{v  \in (\textit{Sh}\cup\mathit{Loc} \setminus \{\mathit{own}_\ell, \mathit{av}_\ell\})}(\forall  s,s'  : s \overset{\textit{ch}_\ell}{\rightarrow} s' \Rightarrow (v(s) \equiv v(s'))$,
	\item\label{system-spec-formula5}   $\bigwedge_{g \in \textit{Sh}\setminus\{\mathit{av}_{\ell_0}, \dots,\mathit{av}_{\ell_m}\} }(\forall s: (\exists s': s \overset{ch_g}{\rightarrow} s' \wedge g(s')) \wedge  (\exists s': s \overset{ch_g}{\rightarrow} s' \wedge \neg g(s')))$,
	\item\label{system-spec-formula6} 	$\bigwedge_{g \in \mathit{Sh}}\bigwedge_{g'  \in (\textit{Sh}\cup\mathit{Loc} \setminus \{g\})}(\forall  s,s'  : s \overset{\textit{ch}_g}{\rightarrow} s' \Rightarrow (g'(s) \equiv g'(s'))$.
% these are the formulas added in the Alloy spec:
%			all s:philMeta.nodes | Own_left[philMeta, s] implies (not Av_left[philMeta, s])
%			 all s:philMeta.nodes | (not Own_left[philMeta,s]) iff (some philMeta.ACTchange_left[s])
%			 all s:philMeta.nodes | all s':philMeta.ACTchange_left[s] | Av_left[philMeta,s] iff (not Av_left[philMeta, s']) 
%			all s:philMeta.nodes | all s':(philMeta.env[s] - philMeta.ACTchange_left[s]) | Av_left[philMeta,s] iff Av_left[philMeta, s']
\end{enumerate}
\end{definition}
Intuitively, locks are special kinds of shared variables used for synchronization.  Actions $\{\textit{ch}_{\ell_0},\dots,\textit{ch}_{\ell_m}\} \cup \{\mathit{ch}_g \mid g \in \mathit{Sh}\}$ 
are used to model environment actions, for instance, $\textit{ch}_{\ell_0}$ represents an action of the environment that changes the state of lock $\ell_0$.  Variables $\mathit{av}_i$ and
$\mathit{own}_i$ are used to indicate that a lock is free,  or owned by the current component.  Formula \ref{system-spec-formula1} expresses that, if a component owns a lock, then it is not available.
Formula \ref{system-spec-formula2}  expresses that, if a lock is not owned by the current component, then it can be changed by the environment.  Formula \ref{system-spec-formula3} says that, if the environment changes lock $\ell_i$, then the variable 
$\mathit{av}_i$ changes accordingly.  Formula \ref{system-spec-formula4} states that changes in the locks do not affect other variables.  Formula \ref{system-spec-formula5} expresses that shared variables can be changed by the environment. Finally, formula \ref{system-spec-formula6} states that the changes in one shared variable does not affect the other variables.
From now on, we write $s \dashrightarrow s'$ if $s \xrightarrow{\textit{ch}_\ell} s'$ or $s \xrightarrow{\textit{ch}_g} s'$, for some lock $\ell$ or shared variable $g$, respectively.

In the same way as component specifications can be put together, we also consider the asynchronous composition of LTSs.
\begin{definition} Let $\mathcal{S} = \langle \{\textit{PS}^i\}_{i \in [0,n]}, \phi \rangle$ be an $\mathcal{L}$-\emph{synchronized specification}, and 
let $T^0,\dots,T^n$ be LTSs, such that for each $T^i = \langle \mathit{S}^i, \mathit{Act}^i, \rightarrow^i,I^i, \mathit{Sh}\cup \mathit{Loc}^i, L^i \rangle$ we have $\mathit{Sh}\cap \mathit{Loc}^i = \emptyset$ and $T^i \vDash \mathit{PS}^i$. We define the LTS $T^0 \parallel \dots \parallel T^n = \langle S, \coprod_{i \in [0,n]} \mathit{Act}^i, \rightarrow, I, \mathit{Sh}\cup \coprod_{i \in [0,n]}\mathit{Loc}^i, L\rangle$   as follows:
\begin{itemize}	
	\item $S = \{s \mid s \in \prod_{i \in [0,n]}S^i \wedge (\forall g \in \mathit{Sh} : \forall i,j \in [0,n] : L^i(s \proj i)(g) = L^j(s \proj j)(g)) \}$,
	%\item $\mathit{Act} =  \coprod_{i \in [0,k]} \mathit{Act}^i$,
	\item $\rightarrow  = \{ s \xrightarrow{a} s' \mid \exists i \in [0,n] : (s \proj i \xrightarrow{a}\mathrel{\vphantom{\to}^i}  s' \proj i) \wedge (\forall j \neq i : (s \proj j \dashrightarrow^j s' \proj j) \vee (s \proj j  =  s' \proj j) \}$,
	\item $I  =\{ s \mid \forall i \in [0,n]: s{\uparrow}i \in I^i\}$,	
	%\item $\mathit{AP} = \mathit{Sh} \cup	\bigcup_{i \in [0,n]}\mathit{AP}^i$,
	%\item $L(s)(x) =  L_i(s \proj i)(x) \mbox{ if  $x \in Loc_i$ for some $i \in [0,n]$ }$,
	%\item $L(s)(x)  =  L_0(s \proj 0)(x)$, otherwise.
	\item  $L(s)(x)=  \begin{cases*}
      					L^i(s \proj i)(x) & if  $x \in Loc^i$ for some $i \in [0,n]$, \\
     					L^0(s \proj 0)(x)      & otherwise.
   				  \end{cases*}$
\end{itemize}
\end{definition}
The semantics of distributed specifications is straightforwardly defined using asynchronous product, i.e., the combination of LTSs that produces all interleavings of their corresponding actions.
\begin{definition} Given a system specification $\mathcal{S} = \langle \{\mathit{PS}^i\}_{i \in [0,n]},$ $\phi \rangle$ a \emph{model} or \emph{implementation} of $\mathcal{S}$ is a collection of LTSs $T^0,\dots,T^n$ such that $T^i \vDash \mathit{PS}^i$, for every $i \in [0,n]$, and $T^0 \parallel \dots \parallel T^n \vDash \phi$.
\end{definition}
An interesting point about the definition above, which in particular holds thanks to the formulas that give semantics to shared variables, is that linear-time temporal properties (without the next operator) local to the process implementations can be promoted to global implementations, i.e., to asynchronous products they participate in, provided the asynchronous product is \emph{strongly fair} (strong fairness assumption is necessary to guarantee the promotion of liveness properties). That is, LTSs satisfying \ref{system-spec-formula1}-\ref{system-spec-formula6} preserve their (local) temporal properties under \emph{any} environment that guarantees strong fairness:
\begin{theorem}\label{theorem:stuttering-equiv} Let $\langle \{ \mathit{PS}^i \}_{i \in [0,n]}, \phi \rangle$ be a distributed specification, and $T^0, \dots, T^n$ LTSs such that $T^i \vDash \mathit{PS}^i$ (for every $i \in [0,n]$). Given an {\LTLX} formula $\psi$, if $T^i \vDash \psi$  (for any $i \in [0,n]$), then $T^0 \parallel \dots \parallel T^n \vDash_f \psi$.
\end{theorem}
	
From an LTS $T$, we can obtain a specification that characterizes it (up to isomorphism), using existentially quantified variables for identifying the states, and formulas for describing the transitions. Moreover,we can obtain a specification that characterizes \emph{all} LTSs that can be obtained by removing some (local) transitions from $T$. Intuitively,  this specification captures refinements of $T$. 
\begin{definition}\label{def:ref-spec} Let $\mathit{PS}= \langle \langle \mathit{Sh},  \mathit{Loc}, \mathit{Act}\rangle, \Phi \rangle$ be a component specification, and let $T=\langle S, \mathit{Act}, \rightarrow, I, (\mathit{Sh} \cup \mathit{Loc}), L\rangle$ be an LTS, such that $S=\{s_0,\dots,s_n\}$, $\mathit{Sh} \cup \mathit{Loc}=\{p_0,\dots,p_m\}$, $\mathit{Act}=\{a_0,\dots,a_k\}$ and $T \vDash \mathit{PS}$. The process specification $\refin{\mathit{PS},T}$ is the tuple $\langle \langle \mathit{Sh}, \mathit{Loc}, \mathit{Act}\rangle, \Phi \cup \{ \exists s_0,\dots,s_n: \phi^T\} \rangle$ where $\phi^T$ is the following formula:
 \[\displaystyle
\begin{array}{l}(\bigwedge_{0\leq i< j \leq n} s_i \neq s_j) \wedge I(s_0) 
															\wedge \bigwedge_{j \in [0,n]} \bigwedge_{i \in [0,m]} \{  p_i(s_j) \mid p_i \in L(s_j) \}  \\
															\wedge  \bigwedge_{j \in [0,n]} \bigwedge_{i \in [0,m]} \{ \neg p_i(s_j) \mid p_i \notin L(s_j) \} \\
															\wedge \bigwedge_{j,j' \in [0,n]} \bigwedge_{i \in [0,k]}  \{ \neg a_i(s_j,s_{j'}) \mid \neg (s_j \xrightarrow{a_i} s_{j'}) \}\\
															\wedge \bigwedge_{j,j'\in [0,n]} \bigwedge_{i \in [0,k]} \{ a_i(s_j,s_{j'}) \mid 
															s_j \xrightarrow{a_i},s_{j'} \wedge a_i \in \{\textit{ch}_g \mid g \in \mathit{Sh} \} \}   
																													
\end{array}				
\]
\end{definition}
Formula $\exists s_0,\dots,s_n: \phi^T$ identifies each state with a variable, describes the properties of each state, enumerates the transitions labeled with environment actions, and rules out the introduction of local transitions not present in $T$. Summing up, models of $\refin{\mathit{PS}, T}$ may remove some local transitions present in $T$ but still satisfy specification $\mathit{PS}$. It is direct to see that $T \vDash \refin{\mathit{PS},T}$. Furthermore, $\refin{\mathit{PS},T}$ preserves all the safety properties of $T$ (cf. \cite{Katoen08} for a formal definition of safety).
\begin{theorem}\label{theorem:preserve-safety} Let $\phi$ be an {\LTL} safety formula, $\mathit{PS}$ a process specification, and $T$ an LTS such that $T \vDash \mathit{PS}$. Then:
$
	T \vDash \phi \text{ implies } \refin{\mathit{PS},T} \vDash \phi. 
$
\end{theorem}

The following notation will be useful in later sections, to refer to specifications complemented with additional formulas.

%\begin{figure*}[h!]
%\begin{minipage}[b]{0.60\linewidth}
%\[
% [a] \left( \begin{array}{l} 
% 			 state = [s] \\
%			 \wedge \bigwedge \{x = x(s) \mid x \in \mathit{Sh} \cup \mathit{Loc} \}\\ 
%			\wedge \bigwedge \{ \ell = i \mid \ell \in \mathcal{L} : \mathit{own}_\ell \in L(s)\}	 \\ 
%			 \wedge \bigwedge \{ \ell = \bot \mid \ell \in \mathcal{L} : \mathit{av}_\ell \in L(s)\}  \end{array} \right) \rightarrow \left( \begin{array}{l} 
% 																					  			  \{x {:=} x(s')  \mid x \in \mathit{Loc}\cup \mathit{Sh}\} \\ 
%																					 			  \cup \{\mathit{state} {:=} [s'] \} \\
%																								  \cup \{ \ell := i \mid \ell \in \mathcal{L} : \mathit{own}_\ell \in L(s)\}  \\
%																								  \cup \{ \ell := \bot \mid \ell \in \mathcal{L} : \mathit{av}_\ell \in L(s)\}
%																					  \end{array} \right)
%\]
%\end{minipage}
%\caption{Interpretation of LTSs as Guarded-Command programs}\label{fig:guarded-command}
%\end{figure*}


\begin{definition} Let $\mathit{PS}= \langle \langle \mathit{Sh},  \mathit{Loc}, \mathit{Act}\rangle, \Phi \rangle$ be a process specification, and $T=\langle S, \mathit{Act}, \rightarrow, I, (\mathit{Sh} \cup \mathit{Loc}), L\rangle$ an LTS such that $S=\{s_0,\dots,s_n\}$, and $T \vDash \mathit{PS}$. Given a formula $\psi$ with free variables $s_0,\dots,s_n$, we define:
\[
	\refin{\mathit{PS},T} \oplus \psi = \langle \langle \mathit{Sh}, \mathit{Loc}, \mathit{Act}\rangle, \Phi \cup \{ \exists s_0,\dots,s_n: \phi^T \wedge \psi\} \rangle
\]	
\end{definition}
\section{Obtaining Guarded-Command Programs.}
    In this section we describe how we obtain the guarded-command programs from LTSs.  It is worth noting that the programs we synthesize  use locks for achieving synchronization. When a guard checks for a lock availability and this is not available  the process may continue executing other branches (i.e., our locks are not blocking). However, note that a process could get blocked when all its guards are false, thus other synchronization mechanisms such as blocking locks, semaphores and condition variables can be expressed by these programs. 
    
	Given LTSs $T^i = \langle \mathit{S}^i, \mathit{Act}^i, \rightarrow^i,I^i, \mathit{Sh}\cup \mathit{Loc}^i, L^i \rangle$ for $i \in [0,n]$, we can abstract away the environmental transitions and define a corresponding program in  guarded-command notation,  denoted $\text{Prog}(T^0  \parallel \dots \parallel T^{n})$, as follows. The shared variables are those in $\textit{Sh}$ plus an additional shared variable $\ell$ for each lock, with domain $[0,n-1]\cup\{\bot\}$ (where $\bot$ is a value indicating that the lock is free). Additionally, for each $T^i$ we define a corresponding process. To do so, we introduce for each $s \in S^i$  the equivalence class: $[ s ]  = \{ s' \in S^i \mid (s \dashrightarrow^* s') \vee (s'  \dashrightarrow^* s)\}$. 
That is, it is the set of states connected to $s$ via environmental transitions. It is direct to see that it is already an equivalence class. The collection of all equivalence classes
is denoted $S^i /_{\dashrightarrow^*}$.The local variables of the process are those in $\textit{Loc}^i$ plus a fresh variable $\textit{st}_i$, with domain $S^i/_{\dashrightarrow^*}$ (for indicating the current state of the process). 

Finally, given states $s,s' \in S^i$ with $s \xrightarrow{a} s' \in \rightarrow^i$ and  $[s] \neq [s']$, we consider the following  guarded command:
\[
 [a] \left( \begin{array}{l} 
 			 state = [s] \\
			 \wedge \bigwedge \{x = x(s) \mid x \in \mathit{Sh} \cup \mathit{Loc} \}\\ 
			\wedge \bigwedge \{ \ell = i \mid \ell \in \mathcal{L} : \mathit{own}_\ell \in L(s)\}	 \\ 
			 \wedge \bigwedge \{ \ell = \bot \mid \ell \in \mathcal{L} : \mathit{av}_\ell \in L(s)\}  \end{array} \right) \rightarrow \left( \begin{array}{l} 
 																					  			  \{x {:=} x(s')  \mid x \in \mathit{Loc}\cup \mathit{Sh}\} \\ 
																					 			  \cup \{\mathit{state} {:=} [s'] \} \\
																								  \cup \{ \ell := i \mid \ell \in \mathcal{L} : \mathit{own}_\ell \in L(s)\}  \\
																								  \cup \{ \ell := \bot \mid \ell \in \mathcal{L} : \mathit{av}_\ell \in L(s)\}
																					  \end{array} \right)
\]

%	Note that this definition may result in programs with redundant branches, which can be simplified in different ways. 
We can prove that our translation from transition systems to programs is correct. That is, the executions of the program satisfy the same temporal properties as the 
asynchronous product $T^0 \parallel \dots \parallel T^{n}$.
\begin{theorem}\label{th:proppreservation} Given LTSs $T^i$ and \textsf{LTL} property $\phi$, then we have that:
$
	T^0 \parallel \dots \parallel T^{n} \vDash \phi \Leftrightarrow Prog(T^0 \parallel \dots \parallel T^{n}) \vDash \phi
$
\end{theorem}
\begin{figure}[t!]
\begin{minipage}[b]{0.30\linewidth}
\centering
\includegraphics[scale=0.6]{Figs/mutex.pdf}\label{fig:mutex-lts}
%\caption{LTS  for Mutex}\label{fig:mutex-lts}
%\end{figure}
\end{minipage}
%\begin{figure}
\hspace{0.7cm}
\begin{minipage}[b]{0.70\linewidth}
\centering
\begin{lstlisting}[style=Unity]
Program Mutex
 var m:Lock;
  Process $P_i$ with $i \in \{0,1\}$
   var $\text{try}_i, \text{ncs}_i, \text{cs}_i$:bit
   var $\text{st}_i$:$\{\text{S0},\text{S1},\text{S2},\text{S3},\text{S4},\text{S5}\}$
   initial: $\text{ncs}_i\wedge \neg \text{cs}_i \wedge \neg \text{try}_i$ 
   begin
    [enterTry]$\text{st}_i$=S5 $\rightarrow$ $\text{st}_i$:=S3,$\text{try}_i$:=1,$\text{ncs}_i$:=0
    [getLock]$\text{st}_i$=S3$\wedge$ m=$\bot$ $\rightarrow$ $\text{st}_i$:=S1,$\text{try}_i$:=1,m:=i
    [enterCS] $\text{st}_i$=S1$\rightarrow$ $\text{st}_i$:=S0,$\text{cs}_i$:=1,$\text{try}_i$:=0
    [enterNCS]$\text{st}_i$=S0$\rightarrow$st:=S5,$\text{ncs}_i$:=1,$\text{cs}_i$:=0 m:=$\bot$
   end
end
\end{lstlisting}
\end{minipage}
\caption{LTS and corresponding program for mutex}\label{fig:mutex}
\end{figure}
\begin{example}[Mutex]  
\label{ex:mutex-ts}
Consider a system composed of two processes (it can straightforwardly be generalized to $n$ processes) both with non-critical, waiting and critical sections. The global property  is mutual exclusion: the two processes cannot be in their critical sections simultaneously. We consider one lock $m$,  actions: $\mathit{enterNCS}$ (the process enters to the non-critical section), $\mathit{enterCS}$ (the process enters to the critical section), $\mathit{getLock}$ (the process acquires the lock), $\mathit{enterTry}$ (the process goes to the try state), and the corresponding propositions $\mathit{ncs}, \mathit{cs}, \mathit{try}, \mathit{own}_m, \mathit{av}_m$.  

The transition system and the program corresponding to this example are shown in Fig.~\ref{fig:mutex}. It is interesting to observe that,  the  conditions in Definition \ref{def:system-spec} allow one to use the locks as synchronization mechanisms, if the component $i$ owns the lock $m$, then the other processes cannot go into their critical sections since the lock will be not available for them. Also,  it is worth noting that we consider one state in the program for each equivalence class of states in the LTS.
\end{example}

%
%	We can prove that our translation from transition systems to programs is correct. That is, the executions of the program satisfy the same temporal properties as the 
%asynchronous product $P_0 \parallel \dots \parallel P_n$.
%\begin{theorem}\label{th:proppreservation} Given a program $P_0 \parallel \dots \parallel P_n$ and \textsf{LTL} property $\phi$, then we have that:
%$
%	P_0 \parallel \dots \parallel P_n \vDash \phi \Leftrightarrow Prog(P_0 \parallel \dots \parallel P_n) \vDash \phi
%$
%\end{theorem}
%such that, for every $\pi \in Tr(P_0 \parallel \dots \parallel P_n)$ and $i \geq 0$ we have that: $\pi[i] \vDash \phi$ iff $f(\pi)[i] \vDash \theta(\phi)$, being $\phi$ any boolean formula. 
% In order to prove this, we need to consider a translation (named $\theta$) between boolean formulas built up from the transition structure and the  program's boolean expressions.
%It is defined recursively as follows: $\theta(own_\ell) = (\ell = i)$, $\theta(av_\ell) = (\ell = \bot)$, $\theta(p) = p$ (for any $p \in Loc \cup Sh$), and $\theta(\varphi \vee \psi) = \theta(\varphi) \vee \theta(\psi)$, $\theta(\varphi \wedge \psi) = \theta(\varphi) \wedge \theta(\psi)$, $\theta(\neg \varphi) = \neg \theta(\varphi)$. Then, we can prove the following theorem:
%\begin{theorem}\label{th:proppreservation} Given a program $P_0 \parallel \dots \parallel P_n$, we have a one-to-one function $f: Tr(P_0 \parallel \dots \parallel P_n) \rightarrow Tr(Prog(P_0 \parallel \dots \parallel P_n))$,
%such that, for every $\pi \in Tr(P_0 \parallel \dots \parallel P_n)$ and $i \geq 0$ we have that: $\pi[i] \vDash \phi$ iff $f(\pi)[i] \vDash \theta(\phi)$, being $\phi$ any boolean formula. 
%\begin{proof} 
%	Let us define for each $\pi \in Tr(P_0 \parallel \dots \parallel P_k)$ a corresponding execution $f(\pi) \in Tr(Prog(P_0 \parallel \dots \parallel P_k))$. The definition of $f(\pi)$ is by induction.
%$\pi[0]$ corresponds to the initial state of the program, by definition the two satisfy the same boolean formulas. Assume that $f(\pi[0]) \dots f(\pi[n])$ are defined, and consider $\pi[n+1]$. We know that there is a transition $\pi[i] \xrightarrow{a} \pi[i+1]$, then by definition of the asynchronous product we have a process $P_j$ with a transition
%$\pi[i] \proj j \xrightarrow{a} \pi[i+1] \proj j$, if $[\pi[i]] = [\pi[i+1]]$ then we define $f(\pi[i+1]) = f(\pi[i])$, otherwise we have a corresponding guarded command $B \rightarrow C$ in the process 
%corresponding to $P_j$ such that, by induction, $f(\pi[i]) \vDash B$  and we define $f(\pi[i+1])$ as the state obtained after executing $C$, which by definition of the program satisfies the same
%boolean properties as $\pi[i+1]$. In a similar way we can define the inverse of $f$.
%\end{proof}
%\end{theorem}
\vspace{-0.5cm}



%Operator $\oplus$ allows us to add formulas to $\refin{\mathit{PS},T}$ capturing refinements of $T$.


%!TEX root=main.tex
\section{Synthesis Algorithm}\label{sec:algo}
In this section, we detail the algorithm we propose for synthesizing synchronized process models from specifications.
This algorithm takes as input a specification $\mathcal{S} = \langle \{ \textit{PS}^i \}_{i \in \mathcal{I}}, \phi \rangle$ for a distributed system, and attempts to synthesize  local implementations  $\{ T^i \}_{i \in \mathcal{I}}$, that satisfy the corresponding local specification (i.e., $T^i \models \mathit{PS}^i$, for every $i \in \mathcal{I}$); and whose  asynchronous parallel composition satisfies the global temporal requirement $\phi$, i.e., $\parallel_{i \in \mathcal{I}} T^i \models \phi$. The algorithm also uses a bound  $k$ to limit the maximum number of states for the synthesized LTSs.  When an implementation is found, it is returned in the {\NuSMV} language; if not, the algorithm deems the specification as unsatisfiable within the bound $k$. 

First, we present Alg.~\ref{alg:the_alg},  a basic  version of the synthesis procedure. 
In line \ref{alg1-for-loop}, it inspects all instances of the current specification.  Line \ref{alg1-base-case} handles the base case (the last specification): a model checker is called to verify the selected instances; if successful the obtained solution is returned; otherwise, another instance is chosen. Line \ref{alg1-recursive-call} addresses the recursive case, where a different instance is examined, and the algorithm continues recursively with the next specification.
As shown in Section \ref{sec:examples}  this algorithm can only cope with small problems. 
{\scriptsize
\begin{algorithm*}[t]
\caption{A Simple Search Algorithm}
\label{alg:the_alg}
%\SetAlgoLined
\SetKwProg{Fn}{Function}{}{end}
\SetKwFunction{Synt}{Synt}%
\SetFuncSty{textbf} 
\KwData{$i$ (process number)
$\textit{PS}^0,\dots,\textit{PS}^n$ (process specifications), $\phi$ (global property), $k$ (instance bound)}
\Fn{\Synt{$i,\textit{PS}^0,\dots,\textit{PS}^n, k $}}{
\KwResult{$r_0, \dots, r_n$ with $r_i \vDash \textit{PS}^i$ and
$r_0 \parallel \dots \parallel r_n \vDash \phi$ or $\emptyset$ otherwise}
 \For{all instances $M_i$ of $\textit{PS}^i$}{  \label{alg1-for-loop}
            $r_i \gets M_i$\;
            \lIf{$i=n \wedge r_0 \parallel \dots \parallel r_n \vDash \phi$}{ \Return{$\{r_0,\dots,r_n $\}} } \label{alg1-base-case}
                  %  \lIf{$r_0 \parallel \dots \parallel r_n \vDash \phi$}{ \Return{$\{r_0,\dots,r_n $\}} } \label{alg1-model-check}
                %}
             %{
             \If{$i < n$}{
                $\textit{found} \gets \Synt(i+1,\textit{PS}^0, \dots, \textit{PS}^n, k)$\; \label{alg1-recursive-call}
                \lIf{$\textit{found} \neq \emptyset$}{\Return{$\textit{found}$}}
            %
            }
   }
\Return{$\emptyset$}\;
}
\end{algorithm*}
}
One main issue with Alg.~\ref{alg:the_alg} is that a wrong choice for the initial instance of the first specifications  (in lexicographic order, e.g., $\textit{PS}^0$) could result in the algorithm
getting stuck while searching for instances of the last specifications (e.g.,  $\textit{PS}^n$) before coming back to the initial specifications to try with different instances of them.  

We improve Alg.~\ref{alg:the_alg} using two main techniques: first, we force the synthesizer to start from better instances (see below); second, we use  counterexamples to speed up the search.
For the first point we add to the specifications formulas that ensure that the obtained instances contain a large number of transitions, which can be pruned in later stages.  Specifically, for every action $\textit{act}$ in a specification $\textit{PS}^i$ we add the formula:
\[
    \forall s \in S : \textit{Pre}_{\textit{act}}(s) \Rightarrow \exists s' \in S : \textit{act}(s,s')
\]
which states that, for every state where the precondition of the action ($\textit{Pre}_{\textit{act}}$) holds,  an execution of $\textit{act}$ can be observed.  This ensures that the obtained model has a good amount of transitions.  We use this for the second technique we apply,  namely, we use counterexamples to prune action executions that may violate the global formulas.  This approach is obviously sound but not complete, because a wrong selection of initial instances may imply that no solution will be found. However, in Section \ref{sec:examples} we show that this procedure is able to deal with an interesting set of examples.  
Furthermore, to avoid that the algorithm gets stuck while inspecting instances of the last specifications without comebacking to try new instances of the first components, we introduce the use of batches.
More specifically,  we use a sequence of bounds $b_0 \dots b_n$
aimed at performing a bounded backtracking. Furthermore,   the counterexamples collected in each batch are utilized for speeding up the process.  Roughly speaking,  we first execute the algorithm inspecting only $b_0$ instances of each specification; if no solution is found,  we use the counterexamples collected  but now with $b_1$ batches,  and so on. 

Some words are useful about the way in which our algorithm uses counterexamples for improving the search. Given the LTSs $\{ T^i \}_{i \in \mathcal{I}}$ and the global {\LTLX} property $\phi$, a \emph{counterexample} of $\parallel_{i \in \mathcal{I}} T^i \models \phi$ is an execution $\pi$ of $\parallel_{i \in \mathcal{I}} T^i$ such that $\pi \nvDash \phi$.  Our modified algorithm takes a counterexample $\pi$ generated by a model checker, and \emph{projects} this global execution to local executions of the participating processes. This information is then used to \emph{refine} the local process specifications to get rid of the projected counterexamples, and explore new implementations for the processes participating in the removed counterexample.

In finite transition structures, counterexamples can be represented by finite paths, called \emph{lasso traces} (a finite sequence of states, such that the last state has a loop to some previous state) \cite{Biere+1999}.  Noting that any finite path can be identified with a formula in \emph{conjunctive normal form} yields the following definition.

\begin{definition}Given an LTS $\lts{T}$ and a finite path $\pi = s_0, s_1, \dots ,s_{m} \in \Path{T}$, we define the formula:
$
\CNF(\pi) = \bigwedge_{0 \leq j < m}(\bigvee_{a \in \mathit{Act}}(a(s_j,$ $s_{j+1})),
$
where $s_0,\dots,s_m$ are free variables.
\end{definition}

The idea behind this definition is that paths can be captured by means of formulas. Let us denote by  $\llbracket \CNF(\pi) \rrbracket$ the set of clauses of formula $\CNF(\pi)$. For instance, $\refin{\mathit{PS},T} \oplus \CNF(\pi)$ captures the refinements of $T$ satisfying specification $\mathit{PS}$ and preserving path $\pi$. Similarly, we can define a formula that removes a counterexample from the instances of a specification.

\begin{definition}\label{def:not} Let $\lts{T}$ be an LTS and $\pi = s_0, s_1, \dots, s_m \in \Path{T}$ a path such that $s_i \neq s_{i+1}$ for some $i$. We define the following formula over $T$:
$
\NOT(\pi) = \bigvee_{0 \leq i < m} (\bigwedge_{a \in \mathit{Act}} \neg  a(s_{i},s_{i+1})  \wedge \ (s_i \neq s_{i+1})),
$
where $s_0,\dots,s_m$ are free variables.
\end{definition}

$\refin{\mathit{PS},T} \oplus \NOT(\pi)$ identifies the refinements of $T$ that do not have path $\pi$, as proved by the following theorem.

\begin{theorem}\label{theorem:cex-entails} Let $\mathit{PS}$ and $T$ be a specification and an LTS, respectively, such that $T \vDash \mathit{PS}$, and $\pi = s_0,\dots, s_m \in \Path{T}$ , with $s_i \neq s_{i+1}$ for some $i$. Then:
$
	\refin{\mathit{PS},T} \oplus \NOT(\pi)  \nvDash \CNF(\pi).
$
\end{theorem}
	
Given LTSs $\{ T^i \}_{i \in \mathcal{I}}$ and a finite path $\pi = s_0, \dots, s_m \in \Path{\parallel_{i \in \mathcal{I}} T^i}$, we denote by $\pi {\uparrow} i$ the path projected to an execution of process $T^i$, defined as  $\pi {\uparrow} i = s_0 {\uparrow} i, \dots, s_m {\uparrow} i$. Projecting a global execution may introduce stuttering steps in local executions. The projections of a global execution form a tuple  $\langle \pi{\uparrow}i \rangle_{i \in \mathcal{I}}$, that will be used to refine the instances obtained via the satisfiability solver. 

Using these definitions we introduce our updated algorithm.  Alg.~\ref{alg:improved_alg} implements the ideas previously discussed.  The Algorithm consists of a starting function (\textbf{StartSearch}) that sets the starting models of each specification (line \ref{alg2-initial-models}), as explained above these initial models are obtained by enriching the specification with specific formulas.  Line \ref{alg2-counterexamples-init}
initializes each specification with a tailored specifications $\textit{Ref}(\textit{PS}^i,M_i) \oplus \bigwedge_{\pi \in \textit{cexs}} \NOT(\pi{\uparrow}i)$.  Intuitively, these specifications consider all the refinements of the initial LTSs together with a set of counterexamples that can be used for refining them.  Line \ref{alg2-call-batchsynth} calls to the auxiliary function \textbf{BatchSynt}, which has similar behavior to Alg.~\ref{alg:the_alg}, but it takes into account the bounds for each batch  (line \ref{alg2-batch-loop}).
%One main issue in Algorithm \ref{} is that the procedure could stay exploring the instances of the last specification,  which could prevent the algorithm from backtracking and exploring another component.  For dealing with this, we use batches of runs,  intuitively, we impose a bound in the number of times that the loop of line \ref{} is executed.  This is achieved by using a sequence of bounds $b_0, b_1, \dots$ for $b_i \in \mathbb{N}$.  The first time the loop of line \ref{} is performed $b_0$ times, and then it backtracks, the second time it is executed $b_1$ times, and so on. 
%This sequence of bounds could be sequence of integers. However, since we use the aforementioned loop to collect counterexamples if the bound are small only few ocunter examples are collected for each batch.  Hence, we need a balance between how fast we want to backtrack and how many counterexamples we collect in the loop.  In this section we use the exponential progression: $2,4,8,\dots$. intuitively, we collect few counterexamples first,  enforcing early backtracking,  but in later iterations the number of counterexamples starts to increasing in an exponential way. In Section \ref{} we make a comparison with other possible sequence of bounds.
{\scriptsize
\begin{algorithm*}[t]
\caption{Batches Algorithm}
\label{alg:improved_alg}
%\SetAlgoLined
\SetKwComment{Comment}{/* }{ */}
\SetKwProg{Fn}{Function}{}{end}
\SetKwFunction{StartSearch}{StartSearch}%
\SetKwFunction{BatchSynt}{BatchSynt}%
\SetFuncSty{textbf} 
\KwData{$\textit{PS}^0,\dots,\textit{PS}^n$ (process specifications), $\phi$ (global property), $k$ (instance size bound), $b_0,\dots,b_m$ (sequence of batch bounds), $cexs$ (global set of counterexamples)}
\Fn{\StartSearch{$\textit{PS}^0,\dots,\textit{PS}^n$, k,  $b_0,\dots,b_m$}}{
\KwResult{$r_0, \dots, r_n$ with $r_i \vDash \textit{PS}^i$ and
$r_0 \parallel \dots \parallel r_n \vDash \phi$ or $\emptyset$ otherwise}
$cexs \gets \emptyset$\;
 \lForEach{ i = 0 \dots n}{$M_i \gets \text{initial instance of } \textit{PS}^i \textit{of size } k$} \label{alg2-initial-models}
\For{$b = b_0,\dots,b_m$}{ \label{alg2-batch-loop}
    \lForEach{$i=0\dots n$}{  $\textit{PS}^i \gets \textit{Ref}(\textit{PS}^i,M_i) \oplus \bigwedge_{\pi \in \textit{cexs}} \NOT(\pi{\uparrow}i),$ }  \label{alg2-counterexamples-init}
    found $\gets$ {\BatchSynt}($0$, $\textit{PS}^0,\dots,\textit{PS}^n$, $\phi$,$b$)\; \label{alg2-call-batchsynth}
    \lIf{$\textit{found} \neq \emptyset$}{\Return{found}}
}
\Return{$\emptyset$}\;
}
\Fn{\BatchSynt{$i,\textit{PS}^0,\dots,\textit{PS}^n, \phi, b$}}{
%\KwData{$i$ (process number), $\textit{PS}^0,\dots,\textit{PS}^n$ (process specifications), $\phi$ (global property), $k$ (instance size bound), $b$ (batch bound)}
\KwResult{$r_0, \dots, r_n$ with $r_i \vDash \textit{PS}^i$ and
$r_0 \parallel \dots \parallel r_n \vDash \phi$ or $\emptyset$ otherwise}
$j \gets 0$\;
 \While{$\textit{PS}^i$ has unprocessed instances and $j < b$}{  \label{alg2-for-loop}
            $r_i \gets \text{next instance of } \textit{PS}^i$\;
            \If{$i=n$}{
                 \lIf{$r_0 \parallel \dots \parallel r_n \vDash \phi$}{\Return{$\{r_0,\dots,r_n \}$}}    \label{alg2-base-case}
                 $cexs \gets \text{ process counterexamples}$\;
              %   $   \lIf{$r_0 \parallel \dots \parallel r_n \vDash \phi$}{ \Return{$\{r_0,\dots,r_n $\}} } \label{alg1-model-check}
               %$ 
              }
             \If{$i<n$}{
                $\textit{found} \gets {\BatchSynt}(i+1,\textit{PS}^0,\dots,\textit{PS}^n,\phi,b)$\; \label{alg2-recursive-call}
                \lIf{$\textit{found} \neq \emptyset$}{\Return{$\textit{found}$}}
            }
    $j \gets j+1$ \; 
}
\Return{$\emptyset$}\;
}

\end{algorithm*}
}

\iffalse
Fig.~\ref{fig:algorithm} provides an overview of our approach. First, given the input specification $\mathcal{S} = \langle \{ \textit{PS}^i \}_{i \in \mathcal{I}}, \phi \rangle$ and the bound $k$, it checks whether the specifications are satisfiable or not. In the positive case, the SAT solver returns the local implementations $\{ T^i \}_{i \in \mathcal{I}}$ for the process specifications. Next, the approach relies on model checking to verify whether $\parallel_{i \in \mathcal{I}} T^i \models \phi$. If the answer is positive, the algorithm terminates returning the implementation of the processes, whose composition satisfies the global temporal requirement. Otherwise, the model checker returns a counterexample, i.e., an execution of the distributed system that violates the global property $\phi$. Finally, the produced counterexample is used for extracting information useful for improving the search, and helping the satisfiability procedure to produce new local implementations. Our prototype tool uses the Alloy Analyzer~\cite{AlloyBook} for satisfiability checking and {\NuSMV} \cite{Cimatti+2002} to model check LTSs against properties. 

\begin{figure}[t!]
\centering
\includegraphics[scale=0.25]{Figs/algorithm.pdf} % second figure itself
\caption{An Overview of the Algorithm}
\label{fig:algorithm}
\end{figure}

Let us describe how counterexamples are used for pruning the search (see step 3 in Fig.~\ref{fig:algorithm}). Given the LTSs $\{ T^i \}_{i \in \mathcal{I}}$ and the global {\LTLX} property $\phi$, a \emph{counterexample} of $\parallel_{i \in \mathcal{I}} T^i \models \phi$ is an execution $\pi$ of $\parallel_{i \in \mathcal{I}} T^i$ such that $\pi \nvDash \phi$. The refinement phase of our algorithm takes a counterexample $c$ generated by a model checker, and \emph{projects} this global execution to local executions of the participating processes. This information is then used to \emph{refine} the local process specifications to get rid of the projected counterexamples, and explore new implementations for the processes participating in the removed counterexample.

In finite transition structures, counterexamples can be represented by finite paths, called \emph{lasso traces} (a finite sequence of states, such that the last state has a loop to some previous state) \cite{Biere+1999}. Furthermore, any finite path can be identified with a formula in \emph{conjunctive normal form}. 

\begin{definition}Given an LTS $\lts{T}$ and a finite path $\pi = s_0, s_1, \dots ,s_{m} \in \Path{T}$, we define the formula:
$
\CNF(\pi) = \bigwedge_{0 \leq j < m}(\bigvee_{a \in \mathit{Act}}(a(s_j,$ $s_{j+1})),
$
where $s_0,\dots,s_m$ are free variables.
\end{definition}

The idea behind this definition is that paths can be captured by means of formulas. Let us denote by  $\llbracket \CNF(\pi) \rrbracket$ the set of clauses of formula $\CNF(\pi)$. For instance, $\refin{\mathit{PS},T} \oplus \CNF(\pi)$ captures the refinements of $T$ satisfying specification $\mathit{PS}$ and preserving path $\pi$. Similarly, we can define a formula that removes a counterexample from the instances of a specification.

\begin{definition}\label{def:not} Let $\lts{T}$ be an LTS and $c = s_0, s_1, \dots, s_m \in \Path{T}$ a path such that $s_i \neq s_{i+1}$ for some $i$. We define the following formula over $T$:
$
\NOT(c) = \bigvee_{0 \leq i < m} (\bigwedge_{a \in \mathit{Act}} \neg  a(s_{i},s_{i+1})  \wedge \ (s_i \neq s_{i+1})),
$
where $s_0,\dots,s_m$ are free variables.
\end{definition}

$\refin{\mathit{PS},T} \oplus \NOT(c)$ identifies the refinements of $T$ that do not have path $c$, as proved by the following theorem.

\begin{theorem}\label{theorem:cex-entails} Let $\mathit{PS}$ and $T$ be a specification and an LTS, respectively, such that $T \vDash \mathit{PS}$, and $c = s_0,\dots, s_m \in \Path{T}$ , with $s_i \neq s_{i+1}$ for some $i$. Then:
$
	\refin{\mathit{PS},T} \oplus \NOT(c)  \nvDash \CNF(c).
$
\end{theorem}
	
Given LTSs $\{ T^i \}_{i \in \mathcal{I}}$ and a finite path $\pi = s_0, \dots, s_m \in \Path{\parallel_{i \in \mathcal{I}} T^i}$, we denote by $\pi {\uparrow} i$ the path projected to an execution of process $T^i$, defined as  $\pi {\uparrow} i = s_0 {\uparrow} i, \dots, s_m {\uparrow} i$. Projecting a global execution may introduce stuttering steps in local executions. The projections of a global execution form a tuple  $\langle \pi{\uparrow}i \rangle_{i \in \mathcal{I}}$, that will be used to refine the instances obtained via the satisfiability solver. 

\begin{algorithm*}[t]
\SetAlgoLined
\SetKwProg{Fn}{Function}{}{end}
\SetKwFunction{Synt}{Synt}%
\Fn{\Synt{$i,\textit{PS}^0,\dots,\textit{PS}^n, k $}}{
\KwData{$i$ (process number)
$\textit{PS}^0,\dots,\textit{PS}^n$ (process specifications), $\phi$ (global property), $k$ (instance bound)}
\KwResult{$r_0, \dots, r_n$ with $r_i \vDash \textit{PS}^i$ and
$r_0 \parallel \dots \parallel r_n \vDash \phi$ or $\emptyset$ otherwise}
$\textit{Q}_i \gets \textit{new stack}$\; \label{Q:creation}
 \For{all instance $M_i$ of $\textit{PS}^i$}{  \label{for-loop}
    $\textit{sol} \gets \textit{get solver for } \textit{Ref}(\textit{PS}^i,M_i)$\; \label{solver-creation}
    $\textit{push}(Q_i, (\textit{sol},\emptyset))$\; \label{push-sol}
    \While{$Q_i \textit{ is not empty}$}{ \label{while-loop}
        $(\textit{solver}_i,\textit{cexs}_i) \gets \textit{pop}(Q_i)$\; \label{solver-popped}
        \While{$\textit{solver}_i \textit{ has instances}$}{ \label{solver-loop}
            $r_i \gets \textit{next instance of solver}$\;
            \eIf{$i=n$}{ \label{base-case}
                \eIf{$r_0 \parallel \dots \parallel r_n \vDash \phi$}{ \Return{$\{r_0,\dots,r_n $\}} \label{model-check}
                }{
                    $\textit{cex} \gets \textit{found counterexample}$\; \label{found-cex}
                    \For{$m \gets 0$ to $n$}{ \label{cex-dealing-starts}
                        \If{$\forall c \in \textit{cexs}_m: c * \textit{cex} \vee \textit{cex} \sqsubseteq c $}{ \label{cex-treatment}
                            $(\textit{oldSol}, \textit{oldCexs}) \gets (\textit{sol}_m,\textit{cexs}_m)$\; \label{sol-stored}
                            push($Q_m$, $(\textit{oldSol},\textit{oldCexs}))$\; \label{solver-pushed}
                            $\displaystyle \textit{sol}_m \gets \textit{solver for } \textit{Ref}(\textit{PS}^i,M_i) \oplus \bigwedge_{c \in \textit{cexs}_m} \NOT(c)$\; \label{new-solver-creation}
                            $\textit{cexs}_m \gets \textit{cexs}_m \cup \{\textit{cex} \}$\;  \label{cex-dealing-ends}
                        }
                        }
                    }
                }
            {
                $\textit{found} \gets \textit{Synt}(i+1,\textit{PS}^0, \dots, \textit{PS}^n, k)$\; \label{recursive-call}
                \lIf{$\textit{found} \neq \emptyset$}{\Return{$\textit{found}$}}
            
            }
            }
        }
    }
\Return{$\emptyset$}\;
}
\end{algorithm*}

$\refin{\mathit{PS},T} \oplus \NOT(c)$ identifies the refinements of $T$ that do not have path $c$, as proved by the following theorem.

\begin{theorem}\label{theorem:cex-entails} Let $\mathit{PS}$ and $T$ be a specification and an LTS, respectively, such that $T \vDash \mathit{PS}$, and $c = s_0,\dots, s_m \in \Path{T}$ , with $s_i \neq s_{i+1}$ for some $i$. Then:
$
	\refin{\mathit{PS},T} \oplus \NOT(c)  \nvDash \CNF(c).
$
\end{theorem}
	
Given LTSs $\{ T^i \}_{i \in \mathcal{I}}$ and a finite path $\pi = s_0, \dots, s_m \in \Path{\parallel_{i \in \mathcal{I}} T^i}$, we denote by $\pi {\uparrow} i$ the path projected to an execution of process $T^i$, defined as  $\pi {\uparrow} i = s_0 {\uparrow} i, \dots, s_m {\uparrow} i$. Projecting a global execution may introduce stuttering steps in local executions. The projections of a global execution form a tuple  $\langle \pi{\uparrow}i \rangle_{i \in \mathcal{I}}$, that will be used to refine the instances obtained via the satisfiability solver. 

	The synthesis algorithm is described in Alg.1. The algorithm uses the Alloy model finder (aka, Alloy solver) to get instances of the  specifications,  
This procedure takes as input an index ($i$) indicating the specification being processed, the collection of process specifications ($\textit{PS}^0,\dots,\textit{PS}^n$), the global property  ($\phi$), a bound on the number of states of the processes to be synthesized ($k$). It performs a backtracking search over all the process instances for each specification. For a given $i$, a queue $Q_i$ stores the current model finder and the set of the found counterexamples. The loop of line \ref{while-loop} inspects all the instances of specification number $i$. For each instance $M_i$ obtained, a model finder is created for the specification $\textit{Ref}(\textit{PS}^i, M_i)$ (capturing the refinements of $\textit{PS}^i$), and it is pushed onto the stack, together with an empty collection of
counterexamples. Then, the loop of line \ref{while-loop} iterates over the elements of $Q_i$, the main idea here is that the model finders on the top of the stack work over strengthened specifications, and so these Alloy solvers have less instances to find than the ones placed  above in the stack.  Also, note that the Alloy solver in the bottom of the stack considers no counterexamples and so it can be used to get all the instances of the original specification. The queue $Q_i$ guarantees that all the model finders will be eventually used, thus no instance of the specification will be skipped.
	In line \ref{solver-popped} the current Alloy solver and the accompanying counterexamples are stored in variables $(\textit{solver}_i, \textit{cexs}_i)$, afterwards all
	the instances of $\textit{solver}_i,$ are inspected. If $i=n$ (line \ref{base-case}), this is the base case and the algorithm checks whether the instances obtained for each process ($r_0,\dots,r_n$) satisfy the global property, returning a valid program, or adding the found counterexample to the set of counterexamples (lines \ref{cex-dealing-stars}-\ref{cex-dealing-ends}), before continuing with the search. Our approach uses counterexamples to improve the search in the following way: when a new counterexample is found, the procedure checks (for every process $m$) if the projection of the counterexample to process $m$ is disjoint or refines the counterexamples being under consideration for process $m$ (line \ref{cex-treatment}), in such a case, it adds the projected counterexample to the corresponding collection, and  it creates a new model finder for specification $\refin{\mathit{PS}^i,M_i}\oplus (\bigwedge_{c \in \mathit{cexs}_i} \NOT(c))$ (line \ref{new-solver-creation}), and the stack is updated (line \ref{}). If  $i \neq n$, this is no the base case, and the procedure calls itself recursively (line \ref{recursive-call}).




%
% for the current specification $M_i$
%
%
%
% \ref{for-loop} inspects all the instances of specification number $i$.
%For each instance $M_i$ obtained, a model finder is obtained for $\textit{Ref}(\textit{PS}^i,M_i)$ and then
%
%
%A queue $Q_i$ (for every instance $i$) is used to store the model finders and counterexamples used during the different phases of the search. For each instance $i$, $Q_i$
%
%
%
%A collection $\mathit{gcexs}$ is used to keep track of the found counterexamples. The loop of line $4$ inspects all the instances of specification number $i$. For each instance $M[i]$ obtained, we use two auxiliary sets: $\mathit{inCexs}[i]$ keeps track of the projections of the counterexamples found when dealing with the current instance (it behaves in a stack-like way); and $\mathit{outCexs}[i]$, used to save the counterexamples that were (unsuccessfully) taken into account during the iterations. Then, all the instances of $\refin{\mathit{PS}^i,M[i]}\oplus (\bigwedge_{c \in \mathit{inCexs}[i]} \NOT(c))$ are inspected in the loop of line $7$. In the first iteration, $\bigwedge_{c \in \mathit{inCexs}[i]} \NOT(c) = \textit{true}$, since $\mathit{inCexs}[i]$ is empty. In line $17$, the procedure calls itself recursively until the last specification is reached ($i=n$). At this point, the algorithm checks if the instances obtained for each process ($r[0],\dots,r[n]$) satisfy the global property, returning a valid program, or adding the found counterexample to the set of counterexamples, before continuing with the search. 
%	 
%Our approach uses counterexamples to improve the search in two ways: \emph{(i)} if a disjoint counterexample is found, we restart the search adding the new counterexample to $\mathit{inCexs[i]}$ for every $i$ (call to $\mathit{UpdateCexs}$ in line $13$). If a counterexample refining some counterexample in $\mathit{gcexs}$ is found (line $14$), the latter is changed by the former, i.e., we refine the set of counterexamples (represented by call to $\mathit{RefineCex}$). More detailed algorithmic descriptions of these counterexample processing mechanisms are deferred to the Appendix (Algs. 2 and 3), due to space restrictions.  

%A particular case arises when the found counterexample is only disjoint with the current one (line $2$ in function \textit{UpdateCex}), in which case we add the corresponding counterexample to $\mathit{inCexs}_n$ and the search is resumed. For the other specifications (that is, if $i<n$), if no program was found then either new counterexamples were added to $\mathit{inCexs_i}$, in which case the Alloy solver will search for instances over a refined specification (line $7$); or no further counterexamples were added to $\mathit{inCexs}_i$ and the stack is popped. The stack is used to strengthen the specification, so that the instance finding process can be more efficient. 
	  
The soundness of this algorithm is direct, since a model checker is used to verify the output. Termination is guaranteed by the fact that both,  the number of instances of specifications, and the number of possible counterexamples are finite, thus the loops of lines \ref{solver-loop} and \ref{while-loop} terminate. Bounded completeness follows from the queue policy used in the algorithm, in the worst case all the instances for the specifications are inspected

\begin{theorem}\label{theorem:cex-completeness} Alg.1 terminates, is sound and (bounded-)complete.
\end{theorem}
\fi

	
	

\section{The flag trick in action}\label{sec:examples}
In this section, we provide some applications of the flag trick to several learning problems. We choose to focus on subspace recovery, trace ratio and spectral clustering problems. Other ones, like domain adaptation, matrix completion and subspace tracking are developed or mentioned in the last subsection but not experimented for conciseness.

\subsection{Outline and experimental setting}
For each application, we first present the learning problem as an optimization on Grassmannians. Second, we formulate the associated flag learning problem by applying the flag trick (Definition~\ref{def:flag_trick}). Third, we optimize the problem on flag manifolds with the steepest descent method (Algorithm~\ref{alg:GD})---more advanced algorithms are also derived in the appendix. 
Finally, we perform various nestedness and ensemble learning experiments via Algorithm~\ref{alg:flag_trick} on both synthetic and real datasets.

The general methodology to compare Grassmann-based methods to flag-based methods is the following one. For each experiment, we first choose a flag signature $~{q_{1\rightarrow d} := (q_1, \dots, q_d)}$, then we run independent optimization algorithms on $\Gr(p, q_1), \dots, \Gr(p, q_d)$~\eqref{eq:subspace_problem} and finally we compare the optimal subspaces $\S_k^* \in \Gr(p, q_k)$ to the optimal flag of subspaces $\Sf^* \in \Fl(p, q_{1\rightarrow d})$ obtained via the flag trick~\eqref{eq:flag_problem}. 
To show the nestedness issue in Grassmann-based methods, we compute the subspace distances $\Theta(\S_k^*, \S_{k+1}^*)_{k=1\dots d-1}$, where $\Theta$ is the generalized Grassmann distance of~\citet[Eq.~(14)]{ye_schubert_2016}. It consists in the $\ell_2$ norm of the principal angles, which can be obtained from the singular value decomposition (SVD) of the inner-products matrices ${U_k}\T {U_{k+1}}$, where $U_k \in \St(p, q_k)$ is an orthonormal basis of $\S_k^*$.

Regarding the implementation of the steepest descent algorithm on flag manifolds (Algorithm~\ref{alg:GD}), we develop a new class of manifolds in \href{https://pymanopt.org/}{PyManOpt}~\citep{boumal_manopt_2014,townsend_pymanopt_2016}, and run their \href{https://github.com/pymanopt/pymanopt/blob/master/src/pymanopt/optimizers/steepest_descent.py}{SteepestDescent} algorithm. Our implementation of the \texttt{Flag} class is based on the Stiefel representation of flag manifolds, detailed in \autoref{sec:flags}, with the retraction being the polar retraction. For the computation of the gradient, we use automatic differentiation with the \texttt{\href{https://github.com/HIPS/autograd}{autograd}} package. We could derive the gradients by hand from the expressions we get, but we use automatic differentiation as strongly suggested in PyManOpt's \href{https://pymanopt.org/docs/stable/quickstart.html}{documentation}.
Finally, the real datasets and the machine learning methods used in the experiments can be found in \href{https://scikit-learn.org/stable/}{scikit-learn}~\citep{pedregosa_scikit-learn_2011}.

\section{Robust subspace recovery: extensions and proofs}\label{app:RSR}

\subsection{An IRLS algorithm for robust subspace recovery}
Iteratively reweighted least squares (IRLS) is a ubiquitous method to solve optimization problems involving $L^p$-norms. Motivated by the computation of the geometric median~\citep{weiszfeld_sur_1937}, IRLS is highly used to find robust maximum likelihood estimates of non-Gaussian probabilistic models (typically those containing outliers) and finds application in robust regression~\citep{huber_robust_1964}, sparse recovery~\citep{daubechies_iteratively_2010} etc.

The recent fast median subspace (FMS) algorithm~\citep{lerman_fast_2018}, achieving state-of-the-art results in RSR uses an IRLS scheme to optimize the Least Absolute Deviation (LAD)~\eqref{eq:RSR_Gr}.
The idea is to first rewrite the LAD as 
\begin{equation}
	\sum_{i=1}^n \norm{x_i - \Pi_{\S} x_i}_2 = \sum_{i=1}^n w_i(\S) \norm{x_i - \Pi_{\S} x_i}_2^2,
\end{equation}
with $w_i(\S) = \frac{1}{\norm{x_i - \Pi_{\S} x_i}_2}$, and then successively compute the weights $w_i$ and update the subspace according to the weighted objective.
More precisely, the FMS algorithm creates a sequence of subspaces $\S^1, \dots, \S^m$ such that 
\begin{equation}\label{eq:IRLS_FMF}
    \S^{t+1} = \argmin{\S \in \Gr(p, q)} \sum_{i=1}^n w_i(\S^t) \norm{x_i - \Pi_\S x_i}_2^2.
\end{equation}
This weighted least-squares problem enjoys a closed-form solution which relates to the eigenvalue decomposition of the weighted covariance matrix $\sum_{i=1}^n w_i(\S^t) x_i {x_i}\T$~\citep[Chapter~3.3]{vidal_generalized_2016}.

We wish to derive an IRLS algorithm for the flag-tricked version of the LAD minimization problem~\eqref{eq:RSR_Fl}.
In order to stay close in mind to the recent work of \citet{peng_convergence_2023} who proved convergence of a general class of IRLS algorithms under some mild assumptions, we first rewrite~\eqref{eq:RSR_Fl} as
\begin{equation}~\label{eq:RSR_Fl_IRLS}
    \argmin{\S_{1:d} \in\Fl(p, \qf)} \sum_{i=1}^n \rho(r(\S_{1:d}, x_i)),
\end{equation}
where $r(\S_{1:d}, x) = \norm{x - \Pi_{\Sf} x}_2$ is the \textit{residual} and $\rho(r) = |r|$ is the \textit{outlier-robust} loss function.
Following~\citet{peng_convergence_2023}, the IRLS scheme associated with~\eqref{eq:RSR_Fl_IRLS} is:
\begin{equation}
\begin{cases}
w_i^{t+1} = \rho'(r(\S_{1:d}^t, x_i)) /  r(\S_{1:d}^t, x_i) = 1 / \norm{x_i - \Pi_{\Sf} x_i}_2,\\
(\S_{1:d})^{t+1} = \argmin{\S_{1:d} \in\Fl(p, \qf)} \sum_{i=1}^n w_i^{t+1} \norm{x_i - \Pi_{\Sf} x_i}_2^2.
\end{cases}
\end{equation}
We now show that the second step enjoys a closed-form solution.
\begin{theorem}\label{thm:IRLS_FMF}
The RLS problem
\begin{equation}
    \argmin{\S_{1:d} \in\Fl(p, \qf)} \sum_{i=1}^n w_i \norm{x_i - \Pi_{\Sf} x_i}_2^2
\end{equation}
has a closed-form solution $\S_{1:d}^* \in\Fl(p, \qf)$, which is given by the eigenvalue decomposition of the weighted sample covariance matrix $S_w = \sum_{i=1}^n w_i x_i {x_i}\T = \sum_{j=1}^p \ell_j v_j {v_j}\T$, i.e.
\begin{equation}
    \S_k^* = \operatorname{Span}(v_1, \dots, v_{q_k}) \quad (k=1\twodots d).
\end{equation}
\end{theorem}
\begin{proof}
One has
\begin{equation}
	\sum_{i=1}^n w_i \norm{x_i - \Pi_{\Sf} x_i}_2^2 = \tr{(I - \Pi_{\Sf})^2 \lrp{\sum_{i=1}^n w_i x_i {x_i}\T}}.
\end{equation}
Therefore, we are exactly in the same case as in \autoref{thm:flag_trick}, if we replace $X X\T$ with the reweighted covariance matrix $\sum_{i=1}^n w_i x_i {x_i}\T$. This does not change the result, so we conclude with the end of the proof of \autoref{thm:flag_trick} (which itself relies on~\citet{szwagier_curse_2024}).
\end{proof}
Hence, one gets an IRLS scheme for the LAD minimization problem. 
One can modify the robust loss function $\rho(r) = |r|$ by a Huber-like loss function to avoid weight explosion. Indeed, one can show that the weight $w_i := 1 / \norm{x_i - \Pi_{\Sf} x_i}_2$ goes to infinity when the first subspace $\S_1$ of the flag gets close to $x_i$ .
Therefore in practice, we take 
\begin{equation}
    \rho(r) = 
        \begin{cases}
            r^2 / (2 p \delta) & \text{if } |r| <= p\delta,\\
            r - p \delta / 2 & \text{if } |r| > p\delta.
        \end{cases}
\end{equation}
This yields
\begin{equation}
    w_i = 1 / \max\lrp{p\delta,  1 / \norm{x_i - \Pi_{\Sf} x_i}_2}.
\end{equation}
The final proposed scheme is given in Algorithm~\ref{alg:FMF}, named \textit{fast median flag} (FMF), in reference to the fast median subspace algorithm of~\citet{lerman_fast_2018}.
\begin{algorithm}
\caption{Fast median flag}\label{alg:FMF}
\begin{algorithmic}
\Require $X\in \R^{p\times n}$ (data), $\quad q_1 < \dots < q_d$ (signature), $\quad t_{max}$ (max number of iterations), $\quad \eta$ (convergence threshold), $\quad \varepsilon$ (Huber-like saturation parameter)
\Ensure
$U \in \St(p, q)$
\State $t \gets 0, \quad \Delta \gets \infty, \quad U^0 \gets \operatorname{SVD}(X, q)$
\While{$\Delta > \eta$ and $t < t_{max}$}
    \State $t \gets t+1$
    \State $r_i \gets \norm{x_i - \Pi_{\Sf} x_i}_2$
    \State $y_i \gets {x_i} / {\max(\sqrt{r_i}, \varepsilon)}$
    \State $U^t \gets \operatorname{SVD(Y, q)}$
    \State $\Delta \gets \sqrt{\sum_{k=1}^{d} \Theta(U^t_{q_k}, U^{t-1}_{q_k})^2}$
\EndWhile
\end{algorithmic}
\end{algorithm}
We can easily check that FMF is a direct generalization of FMS for Grassmannians (i.e. when $d=1$).


\begin{remark}
This is far beyond the scope of the paper, but we believe that the convergence result of~\citet[Theorem~1]{peng_convergence_2023} could be generalized to the FMF algorithm, due to the compactness of flag manifolds and the expression of the residual function $r$.
\end{remark}

\subsection{Proof of Proposition~\ref{prop:RSR}}
Let $\Sf \in \Fl(p, \qf)$ and $U_{1:d+1} := [U_1|U_2|\dots|U_d|U_{d+1}] \in \O(p)$ be an orthogonal representative of $\Sf$. One has:
\begin{align}
	\norm{x_i - \Pi_{\S_{1:d}} x_i}_2 &= \sqrt{{(x_i - \Pi_{\S_{1:d}} x_i)}\T (x_i - \Pi_{\S_{1:d}} x_i)},\\
	 &= \sqrt{{x_i}\T {(I_p - \Pi_{\S_{1:d}})}^2 x_i},\\
	 &= \sqrt{{x_i}\T {\lrp{I_p - \frac1d \sum_{k=1}^d\Pi_{\S_k}}}^2 x_i},\\
 	 &= \sqrt{\frac1{d^2} {x_i}\T {\lrp{\sum_{k=1}^d (I_p - \Pi_{\S_k})}}^2 x_i},\\
 	 % &= \sqrt{\frac1{d^2} {x_i}\T U_{1:d+1} \diag{0, 1, \dots, d-1, d}^2 {U_{1:d+1}}\T  x_i},\\
 	 &= \sqrt{\frac1{d^2} {x_i}\T \lrp{\sum_{k=1}^{d+1} (k-1) U_k {U_k}\T}^2  x_i},\\
 	 &= \sqrt{\frac1{d^2} {x_i}\T \lrp{\sum_{k=1}^{d+1} (k-1)^2 U_k {U_k}\T}  x_i},\\
 	 &= \sqrt{\sum_{k=1}^{d+1} \lrp{\frac {k-1} {d}}^2 {x_i}\T \lrp{ U_k {U_k}\T}  x_i},\\
  	 \norm{x_i - \Pi_{\S_{1:d}} x_i}_2 &= \sqrt{\sum_{k=1}^{d+1} \lrp{\frac {k-1} {d}}^2 \norm{{U_k}\T x_i}_2^2},
\end{align}
which concludes the proof.
\subsection{The flag trick for trace ratio problems}\label{subsec:TR}
Trace ratio problems are ubiquitous in machine learning~\citep{ngo_trace_2012}. They write as:
\begin{equation}\label{eq:TR_St}
\argmax{U \in \St(p, q)} \frac{\tr{U\T A U}}{\tr{U\T B U}},
\end{equation}
where $A, B \in \R^{p\times p}$ are positive semi-definite matrices, with $\operatorname{rank}(B) > p - q$.

A famous example of TR problem is Fisher's linear discriminant analysis (LDA)~\citep{fisher_use_1936,belhumeur_eigenfaces_1997}.
It is common in machine learning to project the data onto a low-dimensional subspace before fitting a classifier, in order to circumvent the curse of dimensionality. It is well known that performing an unsupervised dimension reduction method like PCA comes with the risks of mixing up the classes, since the directions of maximal variance are not necessarily the most discriminating ones~\citep{chang_using_1983}. The goal of LDA is to use the knowledge of the data labels to learn a linear subspace that does not mix the classes.
Let $~{X := [x_1|\dots|x_n] \in \R^{p\times n}}$ be a dataset with labels $Y := [y_1|\dots|y_n] \in {[1, C]}^n$. Let $\mu = \frac{1}{n} \sum_{i=1}^n x_i$ be the dataset mean and $\mu_c = \frac{1}{\#\{i : y_i=c\}}\sum_{i : y_i=c} x_i$ be the class-wise means. 
The idea of LDA is to search for a subspace $\S \in \Gr(p, q)$ that simultaneously maximizes the projected \textit{between-class variance} $\sum_{c=1}^C \|\Pi_\S \mu_c - \Pi_\S \mu\|_2^2$ and minimizes the projected \textit{within-class variance} $\sum_{c=1}^C \sum_{i : y_i = c} \|\Pi_\S x_i - \Pi_\S \mu_c\|_2^2$. This can be reformulated as a trace ratio problem~\eqref{eq:TR_St}, with $A = \sum_{c=1}^C (\mu_c - \mu) (\mu_c - \mu)\T$ and $B = \sum_{c=1}^C \sum_{i : y_i = c} (x_i - \mu_c) (x_i - \mu_c)\T$.


More generally, a large family of dimension reduction methods can be reformulated as a TR problem. The seminal work of~\citet{yan_graph_2007} shows that many dimension reduction and manifold learning objective functions can be written as a trace ratio involving Laplacian matrices of attraction and repulsion graphs. Intuitively, those graphs determine which points should be close in the latent space and which ones should be far apart.
Other methods involving a ratio of traces are \textit{multi-view learning}~\citep{wang_trace_2023}, \textit{partial least squares} (PLS)~\citep{geladi_partial_1986,barker_partial_2003} and \textit{canonical correlation analysis} (CCA)~\citep{hardoon_canonical_2004}, although these methods are originally \textit{sequential} problems (cf. footnote~\ref{footnote:sequential}) and not \textit{subspace} problems.

Classical Newton-like algorithms for solving the TR problem~\eqref{eq:TR_St} come from the seminal works of~\citet{guo_generalized_2003, wang_trace_2007, jia_trace_2009}.
The interest of optimizing a trace-ratio instead of a ratio-trace (of the form $\tr{(U\T B U)^{-1}(U\T A U)}$), that enjoys an explicit solution given by a generalized eigenvalue decomposition, is also tackled in those papers. The \textit{repulsion Laplaceans}~\citep{kokiopoulou_enhanced_2009} instead propose to solve a regularized version $\tr{U\T B U} - \rho \tr{U\T A U}$, which enjoys a closed-form, but has a hyperparameter $\rho$, which is directly optimized in the classical Newton-like algorithms for trace ratio problems.

\subsubsection{Application of the flag trick to trace ratio problems}
The trace ratio problem~\eqref{eq:TR_St} can be straightforwardly reformulated as an optimization problem on Grassmannians, due to the orthogonal invariance of the objective function:
\begin{equation}\label{eq:TR_Gr}
\argmax{\S \in \Gr(p, q)} \frac{\tr{\Pi_\S A}}{\tr{\Pi_\S B}}.
\end{equation}
The following proposition applies the flag trick to the TR problem~\eqref{eq:TR_Gr}.
\begin{proposition}[Flag trick for TR]\label{prop:TR}
The flag trick applied to the TR problem~\eqref{eq:TR_Gr} reads
\begin{equation}\label{eq:TR_Fl}
	\argmax{\S_{1:d} \in \Fl(p, q_{1:d})} \frac{\tr{\Pi_{\S_{1:d}} A}}{\tr{\Pi_{\S_{1:d}} B}}.
\end{equation}
and is equivalent to the following optimization problem:
\begin{equation}\label{eq:TR_Fl_equiv}
\argmax{U_{1:d} \in \St(p, q)} \frac{\sum_{k=1}^{d} (d - (k-1)) \tr{{U_k}\T A {U_k}}}{\sum_{l=1}^{d} (d - (l-1)) \tr{{U_{l}}\T B {U_{l}}}}.
\end{equation}
\end{proposition}
\begin{proof}
The proof is given in Appendix (\autoref{app:TR}).
\end{proof}
Equation~\eqref{eq:TR_Fl_equiv} tells us several things. First, the subspaces $~{\operatorname{Span}(U_1) \perp \dots \perp \operatorname{Span}(U_d)}$ are weighted decreasingly, which means that they have less and less importance with respect to the TR objective.
Second, we can see that the nested trace ratio problem~\eqref{eq:TR_Fl} somewhat maximizes the numerator $\tr{\Pi_{\S_{1:d}} A}$ while minimizing the denominator $\tr{\Pi_{\S_{1:d}} B}$. Both subproblems have an explicit solution corresponding to our nested PCA Theorem~\ref{thm:flag_trick}. Hence, one can naturally initialize the steepest descent algorithm with the $q$ highest eigenvalues of $A$ or the $q$ lowest eigenvalues of $B$ depending on the application.
For instance, for LDA, initializing Algorithm~\ref{alg:GD} with the highest eigenvalues of $A$ would spread the classes far apart, while initializing it with the lowest eigenvalues of $B$ would concentrate the classes, which seems less desirable since we do not want the classes to concentrate at the same point.

\subsubsection{Nestedness experiments for trace ratio problems}
For all the experiments of this subsection, we consider the particular TR problem of LDA, although many other applications (\textit{marginal Fisher analysis}~\citep{yan_graph_2007}, \textit{local discriminant embedding}~\citep{chen_local_2005} etc.) could be investigated similarly.

First, we consider a synthetic dataset with five clusters.
The ambient dimension is $p = 3$ and the intrinsic dimensions that we try are $q_{1:2} = (1, 2)$.
We adopt a preprocessing strategy similar to~\citet{ngo_trace_2012}: we first center the data, then run a PCA to reduce the dimension to $n - C$ (if $n - C < p$), then construct the LDA scatter matrices $A$ and $B$, then add a diagonal covariance regularization of $10^{-5}$ times their trace and finally normalize them to have unit trace.
We run Algorithm~\ref{alg:GD} on Grassmann manifolds to solve the TR maximization problem~\eqref{eq:TR_Gr}, successively for $q_1 = 1$ and $q_2 = 2$. Then we plot the projections of the data points onto the optimal subspaces. We compare them to the nested projections onto the optimal flag output by running Algorithm~\ref{alg:GD} on $\Fl(3, (1, 2))$ to solve~\eqref{eq:TR_Fl}. The results are shown in Figure~\ref{fig:TR_nested}.
\begin{figure}
	\centering
    \includegraphics[width=.9\linewidth]{Fig/FT_exp_TR_synthetic.pdf}
    \caption{
    Illustration of the nestedness issue in linear discriminant analysis (trace ratio problem). Given a dataset with five clusters, we plot its projection onto the optimal 1D subspace and 2D subspace obtained by solving the associated Grassmannian optimization problem~\eqref{eq:TR_Gr} or flag optimization problem~\eqref{eq:TR_Fl}. 
    We can see that the Grassmann representations are not nested, while the flag representations are nested and well capture the distribution of clusters. In this example, it is less the nestedness than the \textit{rotation} of the optimal axes inside the 2D subspace that is critical to the analysis of the Grassmann-based method.
    }
	\label{fig:TR_nested}
\end{figure}
\begin{figure}
	\centering
    \includegraphics[width=.9\linewidth]{Fig/FT_exp_TR_digits.pdf}
    \caption{
    Illustration of the nestedness issue in linear discriminant analysis (trace ratio problem) on the digits dataset. For $q_k \in \qf := (1, 2, \dots, 63)$, we solve the Grassmannian optimization problem~\eqref{eq:TR_Gr} on $\Gr(64, q_k)$ and plot the subspace angles $\Theta(\S_k^*, \S_{k+1}^*)$ (left) and explained variances ${\operatorname{tr}(\Pi_{\S_k^*} X X\T)} / {\operatorname{tr}(X X\T)}$ (right) as a function of $k$. We compare those quantities to the ones obtained by solving the flag optimization problem~\eqref{eq:TR_Fl}. 
    We can see that the Grassmann-based method is highly non-nested and even yields an extremely paradoxical non-increasing explained variance (cf. red circle on the right).
    }
	\label{fig:TR_nested_digits}
\end{figure}
We can see that the Grassmann representations are non-nested while their flag counterparts perfectly capture the filtration of subspaces that best and best approximates the distribution while discriminating the classes. Even if the colors make us realize that the issue in this experiment for LDA  is not much about the non-nestedness but rather about the rotation of the principal axes within the 2D subspace, we still have an important issue of consistency.

Second, we consider the (full) handwritten digits dataset~\citep{alpaydin_optical_1998}. It contains $8 \times 8$ pixels images of handwritten digits, from $0$ to $9$, almost uniformly class-balanced. One has $n = 1797$, $p=64$ and $C = 10$.
We run a steepest descent algorithm to solve the trace ratio problem~\eqref{eq:TR_Fl}. We choose the full signature $q_{1:63} = (1, 2, \dots, 63)$ and compare the output flag to the individual subspaces output by running optimization on $\Gr(p, q_k)$ for $q_k \in q_{1:d}$.
We plot the subspace angles $\Theta(\S_k^*, \S_{k+1}^*)$ and the explained variance ${\operatorname{tr}(\Pi_{\S_k^*} X X\T)} / {\operatorname{tr}(X X\T)}$ as a function of the $k$. The results are illustrated in \autoref{fig:TR_nested_digits}.
We see that the subspace angles are always positive and even very large sometimes with the LDA. Worst, the explained variance is not monotonous. This implies that we sometimes \textit{loose} some information when \textit{increasing} the dimension, which is extremely paradoxical.

Third, we perform some classification experiments on the optimal subspaces for different datasets. For different datasets, we run the optimization problems on $\Fl(p, q_{1:d})$, then project the data onto the different subspaces in $\S_{1:d}^*$ and run a nearest neighbors classifier with $5$ neighbors.
The predictions are then ensembled (cf. Algorithm~\ref{alg:flag_trick}) by weighted averaging, either with uniform weights or with weights minimizing the average cross-entropy:
\begin{equation}\label{eq:soft_voting}
	w_1^*, \dots, w_d^* = \argmin{\substack{w_k \geq 0 \\ \sum_{k=1}^d w_k = 1}} - \frac 1 {n C} \sum_{i=1}^n \sum_{c=1}^C y_{ic} \ln\lrp{\sum_{k=1}^d w_k y_{kic}^*},
\end{equation}
where $y_{kic}^* \in [0, 1]$ is the predicted probability that $x_i \in \R^p$ belongs to class $c \in \{1 \dots C\}$, by the classifier $g_k^*$ that is trained on $Z_k := {U_k^*}\T X \in \R^{q_k \times n}$. One can show that the latter is a convex problem, which we optimize using the \href{https://www.cvxpy.org/index.html}{cvxpy} Python package~\citep{diamond2016cvxpy}.
We report the results in \autoref{tab:TR_classif}.
\begin{table}
  \caption{Classification results for the TR problem on real datasets. For each method (Gr: Grassmann optimization~\eqref{eq:TR_Gr}, Fl: flag optimization~\eqref{eq:TR_Gr}, Fl-U: flag optimization + uniform soft voting, Fl-W: flag optimization + optimal soft voting~\eqref{eq:soft_voting}), we give the cross-entropy of the projected-predictions with respect to the true labels.}
  \label{tab:TR_classif}
  \centering
  \begin{tabular}{ccccccccc}
    \toprule
    dataset & $n$ & $p$ & $q_{1:d}$ & Gr & Fl & Fl-U & Fl-W & weights\\
    \midrule
    breast & $569$ & $30$ & $(1, 2, 5)$ & $0.0986$ & $0.0978$ & $0.0942$ & $0.0915$ & $(0.754, 0, 0.246)$\\
    iris & $150$ & $4$ & $(1, 2, 3)$ & $0.0372$ & $0.0441$ & $0.0410$ & $0.0368$ & $(0.985, 0, 0.015)$\\
    wine & $178$ & $13$ & $(1, 2, 5)$ & $0.0897$ & $0.0800$ & $0.1503$ & $0.0677$ & $(0, 1, 0)$\\
    digits & $1797$ & $64$ & $(1, 2, 5, 10)$ & $0.4507$ & $0.4419$ & $0.5645$ & $0.4374$ & $(0, 0, 0.239, 0.761)$\\
    \bottomrule
  \end{tabular}
\end{table}
The first example tells us that the optimal $5D$ subspace obtained by Grassmann optimization less discriminates the classes than the $5D$ subspace from the optimal flag. This may show that the flag takes into account some lower dimensional variability that enables to better discriminate the classes. We can also see that the uniform averaging of the predictors at different dimensions improves the classification. Finally, the optimal weights improve even more the classification and tell that the best discrimination happens by taking a soft blend of classifier at dimensions $1$ and $5$. Similar kinds of analyses can be made for the other examples.

\subsubsection{Discussion on TR optimization and kernelization}
\paragraph{A Newton algorithm}
In all the experiments of this paper, we use a steepest descent method on flag manifolds (Algorithm~\ref{alg:GD}) to solve the flag problems.
However, for the specific problem of TR~\eqref{eq:TR_Fl}, we believe that more adapted algorithms should be derived to take into account the specific form of the objective function, which is classically solved via a Newton-Lanczos method~\citep{ngo_trace_2012}. 
To that extent, we develop in the appendix (\autoref{app:TR}) an extension of the baseline Newton-Lanczos algorithm for the flag-tricked problem~\eqref{eq:TR_Fl}.
In short, the latter can be reformulated as a penalized optimization problem of the form $\operatorname{argmax}_{\Sf\in\Fl(p, \qf)} {\sum_{k=1}^d \tr{\Pi_{\S_k} (A - \rho B)}}$, where $\rho$ is updated iteratively according to a Newton scheme. Once again, our central Theorem~\ref{thm:flag_trick} enables to get explicit expressions for the iterations, which results without much difficulties in a Newton method, that is known to be much more efficient than first-order methods like the steepest descent.

\paragraph{A non-linearization via the kernel trick}
The classical trace ratio problems look for \textit{linear} embeddings of the data.
However, in most cases, the data follow a \textit{nonlinear} distribution, which may cause linear dimension reduction methods to fail. The \textit{kernel trick}~\citep{hofmann_kernel_2008} is a well-known method to embed nonlinear data into a linear space and fit linear machine learning methods.
As a consequence, we propose in appendix (\autoref{app:TR}) a kernelization of the trace ratio problem~\eqref{eq:TR_Fl} in the same fashion as the one of the seminal graph embedding work~\citep{yan_graph_2007}.
This is expected to yield much better embedding and classification results.
\section{Spectral clustering: extensions and proofs}\label{app:SSC}


\subsection{Proof of Proposition~\ref{prop:SSC}}
Let $\Sf \in \Fl(p, \qf)$ and $U_{1:d+1} := [U_1|U_2|\dots|U_d|U_{d+1}] \in \O(p)$ be an orthogonal representative of $\Sf$. One has:
\begin{align}
	\langle \Pi_{\S_{1:d}}, L\rangle_F + \beta \norm{\Pi_{\S_{1:d}}}_1 &= \left\langle \frac1d\sum_{k=1}^d \Pi_{\S_k}, L\right\rangle_F + \beta \norm{\frac1d\sum_{k=1}^d \Pi_{\S_k}}_1,\\
	&= \frac1d \lrp{\left\langle \sum_{k=1}^{d+1} (d - (k-1)) U_k {U_k}\T, L\right\rangle_F + \beta \norm{\sum_{k=1}^{d+1} (d - (k-1)) U_k {U_k}\T }_1},\\
	&= \frac1d \lrp{\sum_{k=1}^{d+1} (d - (k-1)) \left\langle U_k {U_k}\T, L\right\rangle_F + \beta \norm{\sum_{k=1}^{d+1} (d - (k-1)) U_k {U_k}\T }_1},\\
	\langle \Pi_{\S_{1:d}}, L\rangle_F + \beta \norm{\Pi_{\S_{1:d}}}_1 &= \frac1d \lrp{\sum_{k=1}^{d+1} (d - (k-1)) \tr{{U_k}\T L U_k} + \beta \norm{\sum_{k=1}^{d+1} (d - (k-1)) U_k {U_k}\T }_1},
\end{align}
which concludes the proof.
\subsection{The flag trick for other machine learning problems}
Subspace learning finds many applications beyond robust subspace recovery, trace ratio and spectral clustering problems, as evoked in~\autoref{sec:intro}. The goal of this subsection is to provide a few more examples in brief, without experiments.


\subsubsection{Domain adaptation}
In machine learning, it is often assumed that the training and test datasets follow the same distribution. However, some \textit{domain shift} issues---where training and test distributions are different---might arise, notably if the test data has been acquired from a different source (for instance a professional camera and a phone camera) or if the training data has been acquired a long time ago. \textit{Domain adaptation} is an area of machine learning that deals with domain shifts, usually by matching the training and test distributions---often referred to as \textit{source} and \textit{target} distributions---before fitting a classical model~\citep{farahani_brief_2021}. 
A large body of works (called ``subspace-based'') learn some intermediary subspaces between the source and target data, and perform the inference for the projected target data on these subspaces. The \textit{sampling geodesic flow}~\citep{gopalan_domain_2011} first performs a geodesic interpolation on Grassmannians between the source and target subspaces, then projects both datasets on (a discrete subset of) the interpolated subspaces, which results in a new representation of the data distributions, that can then be given as an input to a machine learning model. The higher the number of intermediary subspaces, the better the approximation, but the larger the dimension of the representation.
The celebrated \textit{geodesic flow kernel}~\citep{boqing_gong_geodesic_2012} circumvents this issue by integrating the projected data onto the continuum of interpolated subspaces. This yields an inner product between infinite-dimensional embeddings that can be computed explicitly and incorporated in a kernel method for learning. The \textit{domain invariant projection}~\citep{baktashmotlagh_unsupervised_2013} learns a \textit{domain-invariant} subspace that minimizes the maximum mean discrepancy (MMD)~\citep{gretton_kernel_2012} between the projected source $X_s := [x_{s1}|\dots|x_{s n_s}] \in \R^{p\times n_s}$ and target distributions $X_t := [x_{t1}|\dots|x_{t n_t}] \in \R^{p\times n_t}$:
\begin{equation}
	\argmin{U \in \St(p, q)} \operatorname{MMD}^2(U\T X_{s}, U\T X_{t}),
\end{equation}
where 
\begin{equation}
	\operatorname{MMD} (X, Y) = \norm{\frac 1 n \sum_{i=1}^n \phi (x_i) - \frac 1 m \sum_{i=1}^m \phi (y_i)}_\mathcal{H}.
\end{equation}
This can be rewritten, using the Gaussian kernel function $\phi(x)\colon y \mapsto \exp\lrp{-\frac{x\T y}{2\sigma^2}}$, as
\begin{multline}\label{eq:DIP}
	\argmin{\S \in \Gr(p, q)} 
	\frac 1 {n_s^2} \sum_{i,j=1}^{n_s} \exp\lrp{-\frac{(x_{si} - x_{sj})\T \Pi_\S (x_{si} - x_{sj})}{2 \sigma^2}}\\
	+ \frac 1 {n_t^2} \sum_{i,j=1}^{n_t} \exp\lrp{-\frac{(x_{ti} - x_{tj})\T \Pi_\S (x_{ti} - x_{tj})}{2 \sigma^2}}\\
	- \frac 2 {n_s n_t} \sum_{i=1}^{n_s} \sum_{j=1}^{n_t} \exp\lrp{-\frac{(x_{si} - x_{tj})\T \Pi_\S (x_{si} - x_{tj})}{2 \sigma^2}}.
\end{multline}
The flag trick applied to the domain invariant projection problem~\eqref{eq:DIP} yields:
\begin{multline}
	\argmin{\S_{1:d} \in \Fl(p, q_{1:d})} 
	\frac 1 {n_s^2} \sum_{i,j=1}^{n_s} \exp\lrp{-\frac{(x_{si} - x_{sj})\T \Pi_{\S_{1:d}} (x_{si} - x_{sj})}{2 \sigma^2}}\\
	+ \frac 1 {n_t^2} \sum_{i,j=1}^{n_t} \exp\lrp{-\frac{(x_{ti} - x_{tj})\T \Pi_{\S_{1:d}} (x_{ti} - x_{tj})}{2 \sigma^2}}\\
	- \frac 2 {n_s n_t} \sum_{i=1}^{n_s} \sum_{j=1}^{n_t} \exp\lrp{-\frac{(x_{si} - x_{tj})\T \Pi_{\S_{1:d}} (x_{si} - x_{tj})}{2 \sigma^2}},
\end{multline}
and can be rewritten as:
\begin{multline}
	\argmin{U_{1:d} \in \St(p, q)}
	\frac 1 {{n_s}^2} \sum_{i,j=1}^{n_s} \exp\lrp{-\sum_{k=1}^d \frac{d+1-k}{d} \frac{\norm{{U_k}\T (x_{si} - x_{sj})}_2^2}{2 \sigma^2}}\\
	+ \frac 1 {{n_t}^2} \sum_{i,j=1}^{n_t} \exp\lrp{-\sum_{k=1}^d \frac{d+1-k}{d} \frac{\norm{{U_k}\T (x_{ti} - x_{tj})}_2^2}{2 \sigma^2}}\\
	- \frac 2 {{n_s} {n_t}} \sum_{i=1}^{n_s} \sum_{j=1}^{n_t} \exp\lrp{-\sum_{k=1}^d \frac{d+1-k}{d} \frac{\norm{{U_k}\T (x_{si} - x_{tj})}_2^2}{2 \sigma^2}}.
\end{multline}
Some experiments similar to the ones of~\citet{baktashmotlagh_unsupervised_2013} can be performed. For instance, one can consider the benchmark visual object recognition dataset of~\citet{saenko_adapting_2010}, learn nested domain invariant projections, fit some support vector machines to the projected source samples at increasing dimensions, and then perform soft-voting ensembling by learning the optimal weights on the target data according to Equation~\eqref{eq:soft_voting}.

\subsubsection{Low-rank decomposition}
Many machine learning methods involve finding low-rank representations of a data matrix. 

This is the case of \textit{matrix completion}~\citep{candes_exact_2012} problems where one looks for a low-rank representation of an incomplete data matrix by minimizing the discrepancy with the observed entries, and which finds many applications including the well-known \href{https://en.wikipedia.org/wiki/Netflix_Prize}{Netflix problem}. Although its most-known formulation is as a convex relaxation, it can also be formulated as an optimization problem on Grassmann manifolds~\citep{keshavan_matrix_2010,boumal_rtrmc_2011} to avoid optimizing the nuclear norm in the full space which can be of high dimension. The intuition is that a low-dimensional point can be described by the subspace it belongs to and its coordinates within this subspace. More precisely, the SVD-based low-rank factorization $M = UW$, with $M \in \R^{p \times n}$, $U \in \St(p, q)$ and $W \in \R^{q \times n}$ is orthogonally-invariant---in the sense that for any $R\in\O(q)$, one has $(UR) (R\T W) = U W$. One could therefore apply the flag trick to such problems, with the intuition that we would try low-rank matrix decompositions at different dimensions. The application of the flag trick would however not be as straightforward as in the previous problems since the subspace-projection matrices $\Pi_\S := U U\T$ do not appear explicitly, and since the coefficient matrix $W$ also depends on the dimension $q$.

Many other low-rank problems can be formulated as a Grassmannian optimization. \textit{Robust PCA}~\citep{candes_robust_2011} looks for a low rank + sparse corruption factorization of a data matrix. \textit{Subspace Tracking}~\citep{balzano_online_2010} incrementally updates a subspace from streaming and highly-incomplete observations via small steps on Grassmann manifolds.

\subsubsection{Linear dimensionality reduction}
Finally, many other general dimension reduction algorithms---referred to as \textit{linear dimensionality reduction methods}~\citep{cunningham_linear_2015}---involve optimization on Grassmannians. For instance, linear dimensionality reduction encompasses the already-discussed PCA and LDA, but also many other problems like \textit{multi-dimensional scaling}~\citep{torgerson_multidimensional_1952}, \textit{slow feature analysis}~\citep{wiskott_slow_2002}, \textit{locality preserving projections}~\citep{he_locality_2003} and \textit{factor analysis}~\citep{spearman_general_1904}.
\subsubsection{Conditioned Diffusion Models}

By operating the data in latent space instead of pixel space, conditioned diffusion models have gained promising development \cite{rombach2022latentDiff}. MM-Diffusion \cite{ruan2023mmdi} designed for joint audio and video generation took advantage of coupled denoising autoencoders to generate aligned audio-video pairs from Gaussian noise. Extending the scalability of diffusion models, diffusion Transformers treat all inputs, including time, conditions, and noisy image patches, as tokens, leveraging the Transformer architecture to process these inputs \cite{bao2023ViTDiff}. In DiT \cite{peebles2023DiT}, William et al. emphasized the potential for diffusion models to benefit from Transformer architectures, where conditions were tokenized along with image tokens to achieve in-context conditioning. 

\subsubsection{Diffusion Models in Robotics}

Recently, a probabilistic multimodal action representation was proposed by Cheng Chi et al. \cite{chi2023diffusionpolicy}, where the robot action generation is considered as a conditional diffusion denoising process. Leveraging the diffusion policy, Ze et al. \cite{ze20243d} conditioned the diffusion policy on compact 3D representations and robot poses to generate coherent action sequences. Furthermore, GR-MG combined a progress-guided goal image generation model with a multimodal goal-conditioned policy, enabling the robot to predict actions based on both text instructions and generated goal images \cite{li2025grmg}. BESO used score-based diffusion models to learn goal-conditioned policies from large, uncurated datasets without rewards. Score-based diffusion models progressively add noise to the data and then reverse this process to generate new samples, making them suitable for capturing the multimodal nature of play data \cite{reuss2023md}. RDT-1B employed a scalable Transformer backbone combined with diffusion models to capture the complexity and multimodality of bimanual actions, leveraging diffusion models as a foundation model to effectively represent the multimodality inherent in bimanual manipulation tasks \cite{liu2024rdt-1b}. NoMaD exploited the diffusion model to handle both goal-directed navigation and task-agnostic exploration in unfamiliar environments, using goal masking to condition the policy on an optional goal image, allowing the model to dynamically switch between exploratory and goal-oriented behaviors \cite{sridhar2023nomad}. The aforementioned insights grounded the significant advancements of diffusion models in robotic tasks.

\subsubsection{VLM-based Autonomous Driving}

End-to-end autonomous driving introduces policy learning from sensor data input, resulting in a data-driven motion planning paradigm \cite{chen2024vadv2}. As part of the development of VLMs, they have shown significant promise in unifying multimodal data for specific downstream tasks, notably improving end-to-end autonomous driving systems\cite{ma2024dolphins}. DriveMM can process single images, multiview images, single videos, and multiview videos, and perform tasks such as object detection, motion prediction, and decision making, handling multiple tasks and data types in autonomous driving \cite{huang2024drivemm}. HE-Drive aims to create a human-like driving experience by generating trajectories that are both temporally consistent and comfortable. It integrates a sparse perception module, a diffusion-based motion planner, and a trajectory scorer guided by a Vision Language Model to achieve this goal \cite{wang2024hedrive}. Based on current perspectives, a differentiable end-to-end autonomous driving paradigm that directly leverages the capabilities of VLM and a multimodal action representation should be developed. 








\chapter{\textcolor{black}{Conclusion}}
\label{ch: Conclusion}
\thispagestyle{plain}

In this thesis, the potential of \gls{sc} and \gls{goc} paradigms within modern digital networks has been explored and exploited. The rapid proliferation of data driven technologies such as the \gls{iot}, autonomous vehicles and smart cities has underscored the limitations of traditional bit-centric communication systems. These systems, grounded in Shannon's information theory, focus primarily on the accurate transmission of raw data without considering the contextual significance of the information being conveyed. This fundamental mismatch between data production and communication infrastructure capabilities has necessitated the exploration of more efficient and intelligent communication frameworks.

\cref{ch: SEMCOM} discussed how the core of this thesis focused on integrating of \gls{sc} principles with generative models and their potential applications in the context of edge computing. By focusing on the conveyance of relevant meaning rather than exact data reproduction, \gls{sc} reduces unnecessary bandwidth consumption and inefficiencies. In all those cases where it is possible and reasonable to discuss the semantics, then the faithful representation of the original data is unnecessary as long as the meaning has been conveyed. This paradigm also aligns with \gls{goc}, where the transmitted data is tailored to meet specific objectives, further reducing the communication overhead. The goal of the communication can either be the classical syntactic data transmission or the semantic preservation of the data. By focusing on the goal of the communication, it is possible to transmit only the most pertinent information, thereby reducing the load in communication networks and optimizing resource utilization.

In \cref{ch: SPIC}, the \gls{spic} framework was introduced as a novel method for semantic-aware image compression. The framework demonstrated the potential for high-fidelity image reconstruction from compressed semantic representations. The proposed modular transmitter-receiver architecture is based on a doubly conditioned \gls{ddpm} model, the \gls{semcore}, specifically designed to perform \gls{sr} under the conditioning of the \gls{ssm}. By doing so the reconstructed images preserve their semantic features at a fraction of the \gls{bpp} compared to classical methods such as \gls{bpg} and \gls{jpeg2000}.

Furthermore, the enhancement introduced by \gls{cspic} addressed a critical aspect in image reconstruction: the accurate representation of small and detailed objects. Without requiring extensive retraining of the underlying \gls{semcore} model, \gls{cspic} improved the preservation of important semantic classes, such as traffic signs.  The modular design at the core of the \gls{spic} and \gls{cspic} showcased the flexibility and adaptability of the system in different contexts.

The integration of \gls{sc} principles continued in \cref{ch: SQGAN}, where the \gls{sqgan} model was proposed. This architecture employed vector quantization in tandem with a semantic-aware masking mechanism, enabling selective transmission of semantically important regions of the image and the \gls{ssm}. By prioritizing critical semantic classes and utilizing techniques such as Semantic Relevant Classes Enhancement or the Semantic-Aware discriminator, the model excelled at maintaining high reconstruction quality even at very low bit rates, further emphasizing the efficiency gains of the proposed approach.

Finally, in \cref{ch: Goal_oriented}, the thesis was extended to include the \gls{goc} for resource allocation in \glspl{en}. By adopting the \gls{ib} principle to perform \gls{goc} was developed a framework to dynamically adjust compression and transmission parameters based on network conditions and resource constraints. This dynamic adaptation was crucial in balancing compression efficiency with semantic preservation, optimizing the use of computational and communication resources in edge networks.

By leveraging the \gls{sqgan} within the \gls{en}, the research demonstrated the synergy between \gls{sc} and \gls{goc}. Real-time network conditions informed adjustments to the masking process, enabling the edge network to operate autonomously and efficiently. This approach validated the potential of \gls{sgoc} to enhance resource utilization in modern network infrastructures.



% In this thesis, the potential of \gls{sc} and \gls{goc} paradigms within modern digital networks has been explored and exploited. The rapid proliferation of data driven technologies such as the \gls{iot}, autonomous vehicles, and smart cities has underscored the limitations of traditional bit-centric communication systems. These systems, grounded in Shannon's information theory, focus primarily on the accurate transmission of raw data without considering the contextual significance of the information being conveyed. This fundamental mismatch between data production and communication infrastructure capabilities has necessitated the exploration of more efficient and intelligent communication frameworks.

% As explained in \cref{ch: SEMCOM} at the core of this thesis lies the integration of \gls{sc} principles with generative models, particularly within the context of edge computing. \gls{sc}, which emphasizes the conveyance of meaning rather than mere symbol reconstruction, offers a pathway to significantly reduce bandwidth usage and enhance the efficiency of data transmission. This approach aligns seamlessly with the objectives of \gls{goc}, which prioritizes the transmission of information that is directly relevant to achieving specific goals. By focusing on the semantic content of the data, it becomes possible to transmit only the most pertinent information, thereby reducing the load in  communication networks and optimizing resource utilization.

% In \cref{ch: SPIC} the development and implementation of the \gls{spic} framework marked a significant stride in bridging \gls{sc} with practical image compression techniques. By leveraging diffusion models, \glspl{ddpm} were employed to reconstruct high-resolution images from compressed semantic representations. This modular approach, consisting of a transmitter and receiver architecture, facilitated the efficient encoding and decoding of both the low-resolution original image and the associated \gls{ssm}. The \gls{spic} framework demonstrated the capability to maintain high levels of semantic preservation while achieving substantial compression rates, thereby showcasing its potential as a viable alternative to classical image compression algorithms such as \gls{bpg} and \gls{jpeg2000}.

% Building upon the foundational work of \gls{spic}, the introduction of the \gls{cspic} further refined the approach by addressing the reconstruction of small and detailed objects within images. This enhancement was achieved without necessitating additional fine-tuning or retraining of the underlying \gls{semcore} model, thereby exploiting the framework's modularity and flexibility. The \gls{cspic} model underscored the importance of preserving critical semantic classes, ensuring that essential details (i.e. "traffic signs") are preserved. These level of semantic preservation was evaluated by the Traffic signs classification accuracy presented in \sref{sec: GM evaluation metrics}.

% In \cref{ch: SQGAN} the \gls{sqgan} model represented a novel integration of vector quantization and \gls{sc} principles. The \gls{sqgan} architecture incorporated a \gls{samm} to selectively transmit semantically relevant regions of the data. This selective encoding process significantly reduced redundancy and enhanced communication efficiency, particularly at extremely low \gls{bpp} values. The introduction of the \gls{samm} and the \gls{spe} facilitated the prioritization of latent vectors associated with critical semantic classes, thereby improving the overall reconstruction quality of important objects within images. Additionally, the designed Semantic Relevant Classes Enhancement data augmentation technique and the Semantic Aware Discriminator further refined the model's ability to preserve critical semantic information.

% In \cref{ch: Goal_oriented} asignificant contribution of this research was the exploration of goal-oriented resource allocation within \glspl{en}. By leveraging the \gls{ib} principle, the thesis addressed the challenge of dynamically adjusting compression parameters to balance the trade-off between compression efficiency and semantic preservation. The application of stochastic optimization techniques facilitated the optimal allocation of computational and communication resources, ensuring that the \gls{en} operates efficiently under varying network conditions and resource constraints. This integration of \gls{goc} principles with resource optimization strategies underscored the importance of adaptive and intelligent network management in modern communication infrastructures.

% Additionally, by employing the \gls{sqgan} model within the \gls{en} framework, the research demonstrated the potential of \gls{sgoc}. The integration of the \gls{sqgan} model within the \gls{en} architecture enabled the dynamic adjustment of the masking fractions based on real-time network conditions and resource availability. This approach ensured that the \gls{en} could autonomously optimize its operations, thereby enhancing communication efficiency and resource utilization in a goal-oriented fashion with focus on \gls{sc}.

% Throughout the research, the importance of modular and flexible framework design was emphasized. The proposed models, \gls{spic}, \gls{cspic}, and \gls{sqgan}, were designed to be easily integrated into existing communication systems without the need for extensive modifications. This design philosophy ensures that the advancements in \gls{sc} can be readily adopted in practical applications, facilitating the transition from traditional to intelligent communication paradigms.

% The comparative analysis of the proposed models against classical compression algorithms highlighted the superiority of semantic-aware approaches in preserving critical information at low bit rates. While traditional algorithms excel in minimizing pixel-level distortions, they fall short in maintaining the semantic integrity of the data. In contrast, the proposed semantic and goal-oriented models demonstrated enhanced performance in preserving meaningful content, thereby offering a more effective solution for applications where semantic accuracy is crucial.



\bibliographystyle{splncs}
\bibliography{myBib}



%% Bibliography
%\bibliography{bibfile}


%% Appendix
\appendix
%\section{Appendix}
\subsection{Lloyd-Max Algorithm}
\label{subsec:Lloyd-Max}
For a given quantization bitwidth $B$ and an operand $\bm{X}$, the Lloyd-Max algorithm finds $2^B$ quantization levels $\{\hat{x}_i\}_{i=1}^{2^B}$ such that quantizing $\bm{X}$ by rounding each scalar in $\bm{X}$ to the nearest quantization level minimizes the quantization MSE. 

The algorithm starts with an initial guess of quantization levels and then iteratively computes quantization thresholds $\{\tau_i\}_{i=1}^{2^B-1}$ and updates quantization levels $\{\hat{x}_i\}_{i=1}^{2^B}$. Specifically, at iteration $n$, thresholds are set to the midpoints of the previous iteration's levels:
\begin{align*}
    \tau_i^{(n)}=\frac{\hat{x}_i^{(n-1)}+\hat{x}_{i+1}^{(n-1)}}2 \text{ for } i=1\ldots 2^B-1
\end{align*}
Subsequently, the quantization levels are re-computed as conditional means of the data regions defined by the new thresholds:
\begin{align*}
    \hat{x}_i^{(n)}=\mathbb{E}\left[ \bm{X} \big| \bm{X}\in [\tau_{i-1}^{(n)},\tau_i^{(n)}] \right] \text{ for } i=1\ldots 2^B
\end{align*}
where to satisfy boundary conditions we have $\tau_0=-\infty$ and $\tau_{2^B}=\infty$. The algorithm iterates the above steps until convergence.

Figure \ref{fig:lm_quant} compares the quantization levels of a $7$-bit floating point (E3M3) quantizer (left) to a $7$-bit Lloyd-Max quantizer (right) when quantizing a layer of weights from the GPT3-126M model at a per-tensor granularity. As shown, the Lloyd-Max quantizer achieves substantially lower quantization MSE. Further, Table \ref{tab:FP7_vs_LM7} shows the superior perplexity achieved by Lloyd-Max quantizers for bitwidths of $7$, $6$ and $5$. The difference between the quantizers is clear at 5 bits, where per-tensor FP quantization incurs a drastic and unacceptable increase in perplexity, while Lloyd-Max quantization incurs a much smaller increase. Nevertheless, we note that even the optimal Lloyd-Max quantizer incurs a notable ($\sim 1.5$) increase in perplexity due to the coarse granularity of quantization. 

\begin{figure}[h]
  \centering
  \includegraphics[width=0.7\linewidth]{sections/figures/LM7_FP7.pdf}
  \caption{\small Quantization levels and the corresponding quantization MSE of Floating Point (left) vs Lloyd-Max (right) Quantizers for a layer of weights in the GPT3-126M model.}
  \label{fig:lm_quant}
\end{figure}

\begin{table}[h]\scriptsize
\begin{center}
\caption{\label{tab:FP7_vs_LM7} \small Comparing perplexity (lower is better) achieved by floating point quantizers and Lloyd-Max quantizers on a GPT3-126M model for the Wikitext-103 dataset.}
\begin{tabular}{c|cc|c}
\hline
 \multirow{2}{*}{\textbf{Bitwidth}} & \multicolumn{2}{|c|}{\textbf{Floating-Point Quantizer}} & \textbf{Lloyd-Max Quantizer} \\
 & Best Format & Wikitext-103 Perplexity & Wikitext-103 Perplexity \\
\hline
7 & E3M3 & 18.32 & 18.27 \\
6 & E3M2 & 19.07 & 18.51 \\
5 & E4M0 & 43.89 & 19.71 \\
\hline
\end{tabular}
\end{center}
\end{table}

\subsection{Proof of Local Optimality of LO-BCQ}
\label{subsec:lobcq_opt_proof}
For a given block $\bm{b}_j$, the quantization MSE during LO-BCQ can be empirically evaluated as $\frac{1}{L_b}\lVert \bm{b}_j- \bm{\hat{b}}_j\rVert^2_2$ where $\bm{\hat{b}}_j$ is computed from equation (\ref{eq:clustered_quantization_definition}) as $C_{f(\bm{b}_j)}(\bm{b}_j)$. Further, for a given block cluster $\mathcal{B}_i$, we compute the quantization MSE as $\frac{1}{|\mathcal{B}_{i}|}\sum_{\bm{b} \in \mathcal{B}_{i}} \frac{1}{L_b}\lVert \bm{b}- C_i^{(n)}(\bm{b})\rVert^2_2$. Therefore, at the end of iteration $n$, we evaluate the overall quantization MSE $J^{(n)}$ for a given operand $\bm{X}$ composed of $N_c$ block clusters as:
\begin{align*}
    \label{eq:mse_iter_n}
    J^{(n)} = \frac{1}{N_c} \sum_{i=1}^{N_c} \frac{1}{|\mathcal{B}_{i}^{(n)}|}\sum_{\bm{v} \in \mathcal{B}_{i}^{(n)}} \frac{1}{L_b}\lVert \bm{b}- B_i^{(n)}(\bm{b})\rVert^2_2
\end{align*}

At the end of iteration $n$, the codebooks are updated from $\mathcal{C}^{(n-1)}$ to $\mathcal{C}^{(n)}$. However, the mapping of a given vector $\bm{b}_j$ to quantizers $\mathcal{C}^{(n)}$ remains as  $f^{(n)}(\bm{b}_j)$. At the next iteration, during the vector clustering step, $f^{(n+1)}(\bm{b}_j)$ finds new mapping of $\bm{b}_j$ to updated codebooks $\mathcal{C}^{(n)}$ such that the quantization MSE over the candidate codebooks is minimized. Therefore, we obtain the following result for $\bm{b}_j$:
\begin{align*}
\frac{1}{L_b}\lVert \bm{b}_j - C_{f^{(n+1)}(\bm{b}_j)}^{(n)}(\bm{b}_j)\rVert^2_2 \le \frac{1}{L_b}\lVert \bm{b}_j - C_{f^{(n)}(\bm{b}_j)}^{(n)}(\bm{b}_j)\rVert^2_2
\end{align*}

That is, quantizing $\bm{b}_j$ at the end of the block clustering step of iteration $n+1$ results in lower quantization MSE compared to quantizing at the end of iteration $n$. Since this is true for all $\bm{b} \in \bm{X}$, we assert the following:
\begin{equation}
\begin{split}
\label{eq:mse_ineq_1}
    \tilde{J}^{(n+1)} &= \frac{1}{N_c} \sum_{i=1}^{N_c} \frac{1}{|\mathcal{B}_{i}^{(n+1)}|}\sum_{\bm{b} \in \mathcal{B}_{i}^{(n+1)}} \frac{1}{L_b}\lVert \bm{b} - C_i^{(n)}(b)\rVert^2_2 \le J^{(n)}
\end{split}
\end{equation}
where $\tilde{J}^{(n+1)}$ is the the quantization MSE after the vector clustering step at iteration $n+1$.

Next, during the codebook update step (\ref{eq:quantizers_update}) at iteration $n+1$, the per-cluster codebooks $\mathcal{C}^{(n)}$ are updated to $\mathcal{C}^{(n+1)}$ by invoking the Lloyd-Max algorithm \citep{Lloyd}. We know that for any given value distribution, the Lloyd-Max algorithm minimizes the quantization MSE. Therefore, for a given vector cluster $\mathcal{B}_i$ we obtain the following result:

\begin{equation}
    \frac{1}{|\mathcal{B}_{i}^{(n+1)}|}\sum_{\bm{b} \in \mathcal{B}_{i}^{(n+1)}} \frac{1}{L_b}\lVert \bm{b}- C_i^{(n+1)}(\bm{b})\rVert^2_2 \le \frac{1}{|\mathcal{B}_{i}^{(n+1)}|}\sum_{\bm{b} \in \mathcal{B}_{i}^{(n+1)}} \frac{1}{L_b}\lVert \bm{b}- C_i^{(n)}(\bm{b})\rVert^2_2
\end{equation}

The above equation states that quantizing the given block cluster $\mathcal{B}_i$ after updating the associated codebook from $C_i^{(n)}$ to $C_i^{(n+1)}$ results in lower quantization MSE. Since this is true for all the block clusters, we derive the following result: 
\begin{equation}
\begin{split}
\label{eq:mse_ineq_2}
     J^{(n+1)} &= \frac{1}{N_c} \sum_{i=1}^{N_c} \frac{1}{|\mathcal{B}_{i}^{(n+1)}|}\sum_{\bm{b} \in \mathcal{B}_{i}^{(n+1)}} \frac{1}{L_b}\lVert \bm{b}- C_i^{(n+1)}(\bm{b})\rVert^2_2  \le \tilde{J}^{(n+1)}   
\end{split}
\end{equation}

Following (\ref{eq:mse_ineq_1}) and (\ref{eq:mse_ineq_2}), we find that the quantization MSE is non-increasing for each iteration, that is, $J^{(1)} \ge J^{(2)} \ge J^{(3)} \ge \ldots \ge J^{(M)}$ where $M$ is the maximum number of iterations. 
%Therefore, we can say that if the algorithm converges, then it must be that it has converged to a local minimum. 
\hfill $\blacksquare$


\begin{figure}
    \begin{center}
    \includegraphics[width=0.5\textwidth]{sections//figures/mse_vs_iter.pdf}
    \end{center}
    \caption{\small NMSE vs iterations during LO-BCQ compared to other block quantization proposals}
    \label{fig:nmse_vs_iter}
\end{figure}

Figure \ref{fig:nmse_vs_iter} shows the empirical convergence of LO-BCQ across several block lengths and number of codebooks. Also, the MSE achieved by LO-BCQ is compared to baselines such as MXFP and VSQ. As shown, LO-BCQ converges to a lower MSE than the baselines. Further, we achieve better convergence for larger number of codebooks ($N_c$) and for a smaller block length ($L_b$), both of which increase the bitwidth of BCQ (see Eq \ref{eq:bitwidth_bcq}).


\subsection{Additional Accuracy Results}
%Table \ref{tab:lobcq_config} lists the various LOBCQ configurations and their corresponding bitwidths.
\begin{table}
\setlength{\tabcolsep}{4.75pt}
\begin{center}
\caption{\label{tab:lobcq_config} Various LO-BCQ configurations and their bitwidths.}
\begin{tabular}{|c||c|c|c|c||c|c||c|} 
\hline
 & \multicolumn{4}{|c||}{$L_b=8$} & \multicolumn{2}{|c||}{$L_b=4$} & $L_b=2$ \\
 \hline
 \backslashbox{$L_A$\kern-1em}{\kern-1em$N_c$} & 2 & 4 & 8 & 16 & 2 & 4 & 2 \\
 \hline
 64 & 4.25 & 4.375 & 4.5 & 4.625 & 4.375 & 4.625 & 4.625\\
 \hline
 32 & 4.375 & 4.5 & 4.625& 4.75 & 4.5 & 4.75 & 4.75 \\
 \hline
 16 & 4.625 & 4.75& 4.875 & 5 & 4.75 & 5 & 5 \\
 \hline
\end{tabular}
\end{center}
\end{table}

%\subsection{Perplexity achieved by various LO-BCQ configurations on Wikitext-103 dataset}

\begin{table} \centering
\begin{tabular}{|c||c|c|c|c||c|c||c|} 
\hline
 $L_b \rightarrow$& \multicolumn{4}{c||}{8} & \multicolumn{2}{c||}{4} & 2\\
 \hline
 \backslashbox{$L_A$\kern-1em}{\kern-1em$N_c$} & 2 & 4 & 8 & 16 & 2 & 4 & 2  \\
 %$N_c \rightarrow$ & 2 & 4 & 8 & 16 & 2 & 4 & 2 \\
 \hline
 \hline
 \multicolumn{8}{c}{GPT3-1.3B (FP32 PPL = 9.98)} \\ 
 \hline
 \hline
 64 & 10.40 & 10.23 & 10.17 & 10.15 &  10.28 & 10.18 & 10.19 \\
 \hline
 32 & 10.25 & 10.20 & 10.15 & 10.12 &  10.23 & 10.17 & 10.17 \\
 \hline
 16 & 10.22 & 10.16 & 10.10 & 10.09 &  10.21 & 10.14 & 10.16 \\
 \hline
  \hline
 \multicolumn{8}{c}{GPT3-8B (FP32 PPL = 7.38)} \\ 
 \hline
 \hline
 64 & 7.61 & 7.52 & 7.48 &  7.47 &  7.55 &  7.49 & 7.50 \\
 \hline
 32 & 7.52 & 7.50 & 7.46 &  7.45 &  7.52 &  7.48 & 7.48  \\
 \hline
 16 & 7.51 & 7.48 & 7.44 &  7.44 &  7.51 &  7.49 & 7.47  \\
 \hline
\end{tabular}
\caption{\label{tab:ppl_gpt3_abalation} Wikitext-103 perplexity across GPT3-1.3B and 8B models.}
\end{table}

\begin{table} \centering
\begin{tabular}{|c||c|c|c|c||} 
\hline
 $L_b \rightarrow$& \multicolumn{4}{c||}{8}\\
 \hline
 \backslashbox{$L_A$\kern-1em}{\kern-1em$N_c$} & 2 & 4 & 8 & 16 \\
 %$N_c \rightarrow$ & 2 & 4 & 8 & 16 & 2 & 4 & 2 \\
 \hline
 \hline
 \multicolumn{5}{|c|}{Llama2-7B (FP32 PPL = 5.06)} \\ 
 \hline
 \hline
 64 & 5.31 & 5.26 & 5.19 & 5.18  \\
 \hline
 32 & 5.23 & 5.25 & 5.18 & 5.15  \\
 \hline
 16 & 5.23 & 5.19 & 5.16 & 5.14  \\
 \hline
 \multicolumn{5}{|c|}{Nemotron4-15B (FP32 PPL = 5.87)} \\ 
 \hline
 \hline
 64  & 6.3 & 6.20 & 6.13 & 6.08  \\
 \hline
 32  & 6.24 & 6.12 & 6.07 & 6.03  \\
 \hline
 16  & 6.12 & 6.14 & 6.04 & 6.02  \\
 \hline
 \multicolumn{5}{|c|}{Nemotron4-340B (FP32 PPL = 3.48)} \\ 
 \hline
 \hline
 64 & 3.67 & 3.62 & 3.60 & 3.59 \\
 \hline
 32 & 3.63 & 3.61 & 3.59 & 3.56 \\
 \hline
 16 & 3.61 & 3.58 & 3.57 & 3.55 \\
 \hline
\end{tabular}
\caption{\label{tab:ppl_llama7B_nemo15B} Wikitext-103 perplexity compared to FP32 baseline in Llama2-7B and Nemotron4-15B, 340B models}
\end{table}

%\subsection{Perplexity achieved by various LO-BCQ configurations on MMLU dataset}


\begin{table} \centering
\begin{tabular}{|c||c|c|c|c||c|c|c|c|} 
\hline
 $L_b \rightarrow$& \multicolumn{4}{c||}{8} & \multicolumn{4}{c||}{8}\\
 \hline
 \backslashbox{$L_A$\kern-1em}{\kern-1em$N_c$} & 2 & 4 & 8 & 16 & 2 & 4 & 8 & 16  \\
 %$N_c \rightarrow$ & 2 & 4 & 8 & 16 & 2 & 4 & 2 \\
 \hline
 \hline
 \multicolumn{5}{|c|}{Llama2-7B (FP32 Accuracy = 45.8\%)} & \multicolumn{4}{|c|}{Llama2-70B (FP32 Accuracy = 69.12\%)} \\ 
 \hline
 \hline
 64 & 43.9 & 43.4 & 43.9 & 44.9 & 68.07 & 68.27 & 68.17 & 68.75 \\
 \hline
 32 & 44.5 & 43.8 & 44.9 & 44.5 & 68.37 & 68.51 & 68.35 & 68.27  \\
 \hline
 16 & 43.9 & 42.7 & 44.9 & 45 & 68.12 & 68.77 & 68.31 & 68.59  \\
 \hline
 \hline
 \multicolumn{5}{|c|}{GPT3-22B (FP32 Accuracy = 38.75\%)} & \multicolumn{4}{|c|}{Nemotron4-15B (FP32 Accuracy = 64.3\%)} \\ 
 \hline
 \hline
 64 & 36.71 & 38.85 & 38.13 & 38.92 & 63.17 & 62.36 & 63.72 & 64.09 \\
 \hline
 32 & 37.95 & 38.69 & 39.45 & 38.34 & 64.05 & 62.30 & 63.8 & 64.33  \\
 \hline
 16 & 38.88 & 38.80 & 38.31 & 38.92 & 63.22 & 63.51 & 63.93 & 64.43  \\
 \hline
\end{tabular}
\caption{\label{tab:mmlu_abalation} Accuracy on MMLU dataset across GPT3-22B, Llama2-7B, 70B and Nemotron4-15B models.}
\end{table}


%\subsection{Perplexity achieved by various LO-BCQ configurations on LM evaluation harness}

\begin{table} \centering
\begin{tabular}{|c||c|c|c|c||c|c|c|c|} 
\hline
 $L_b \rightarrow$& \multicolumn{4}{c||}{8} & \multicolumn{4}{c||}{8}\\
 \hline
 \backslashbox{$L_A$\kern-1em}{\kern-1em$N_c$} & 2 & 4 & 8 & 16 & 2 & 4 & 8 & 16  \\
 %$N_c \rightarrow$ & 2 & 4 & 8 & 16 & 2 & 4 & 2 \\
 \hline
 \hline
 \multicolumn{5}{|c|}{Race (FP32 Accuracy = 37.51\%)} & \multicolumn{4}{|c|}{Boolq (FP32 Accuracy = 64.62\%)} \\ 
 \hline
 \hline
 64 & 36.94 & 37.13 & 36.27 & 37.13 & 63.73 & 62.26 & 63.49 & 63.36 \\
 \hline
 32 & 37.03 & 36.36 & 36.08 & 37.03 & 62.54 & 63.51 & 63.49 & 63.55  \\
 \hline
 16 & 37.03 & 37.03 & 36.46 & 37.03 & 61.1 & 63.79 & 63.58 & 63.33  \\
 \hline
 \hline
 \multicolumn{5}{|c|}{Winogrande (FP32 Accuracy = 58.01\%)} & \multicolumn{4}{|c|}{Piqa (FP32 Accuracy = 74.21\%)} \\ 
 \hline
 \hline
 64 & 58.17 & 57.22 & 57.85 & 58.33 & 73.01 & 73.07 & 73.07 & 72.80 \\
 \hline
 32 & 59.12 & 58.09 & 57.85 & 58.41 & 73.01 & 73.94 & 72.74 & 73.18  \\
 \hline
 16 & 57.93 & 58.88 & 57.93 & 58.56 & 73.94 & 72.80 & 73.01 & 73.94  \\
 \hline
\end{tabular}
\caption{\label{tab:mmlu_abalation} Accuracy on LM evaluation harness tasks on GPT3-1.3B model.}
\end{table}

\begin{table} \centering
\begin{tabular}{|c||c|c|c|c||c|c|c|c|} 
\hline
 $L_b \rightarrow$& \multicolumn{4}{c||}{8} & \multicolumn{4}{c||}{8}\\
 \hline
 \backslashbox{$L_A$\kern-1em}{\kern-1em$N_c$} & 2 & 4 & 8 & 16 & 2 & 4 & 8 & 16  \\
 %$N_c \rightarrow$ & 2 & 4 & 8 & 16 & 2 & 4 & 2 \\
 \hline
 \hline
 \multicolumn{5}{|c|}{Race (FP32 Accuracy = 41.34\%)} & \multicolumn{4}{|c|}{Boolq (FP32 Accuracy = 68.32\%)} \\ 
 \hline
 \hline
 64 & 40.48 & 40.10 & 39.43 & 39.90 & 69.20 & 68.41 & 69.45 & 68.56 \\
 \hline
 32 & 39.52 & 39.52 & 40.77 & 39.62 & 68.32 & 67.43 & 68.17 & 69.30  \\
 \hline
 16 & 39.81 & 39.71 & 39.90 & 40.38 & 68.10 & 66.33 & 69.51 & 69.42  \\
 \hline
 \hline
 \multicolumn{5}{|c|}{Winogrande (FP32 Accuracy = 67.88\%)} & \multicolumn{4}{|c|}{Piqa (FP32 Accuracy = 78.78\%)} \\ 
 \hline
 \hline
 64 & 66.85 & 66.61 & 67.72 & 67.88 & 77.31 & 77.42 & 77.75 & 77.64 \\
 \hline
 32 & 67.25 & 67.72 & 67.72 & 67.00 & 77.31 & 77.04 & 77.80 & 77.37  \\
 \hline
 16 & 68.11 & 68.90 & 67.88 & 67.48 & 77.37 & 78.13 & 78.13 & 77.69  \\
 \hline
\end{tabular}
\caption{\label{tab:mmlu_abalation} Accuracy on LM evaluation harness tasks on GPT3-8B model.}
\end{table}

\begin{table} \centering
\begin{tabular}{|c||c|c|c|c||c|c|c|c|} 
\hline
 $L_b \rightarrow$& \multicolumn{4}{c||}{8} & \multicolumn{4}{c||}{8}\\
 \hline
 \backslashbox{$L_A$\kern-1em}{\kern-1em$N_c$} & 2 & 4 & 8 & 16 & 2 & 4 & 8 & 16  \\
 %$N_c \rightarrow$ & 2 & 4 & 8 & 16 & 2 & 4 & 2 \\
 \hline
 \hline
 \multicolumn{5}{|c|}{Race (FP32 Accuracy = 40.67\%)} & \multicolumn{4}{|c|}{Boolq (FP32 Accuracy = 76.54\%)} \\ 
 \hline
 \hline
 64 & 40.48 & 40.10 & 39.43 & 39.90 & 75.41 & 75.11 & 77.09 & 75.66 \\
 \hline
 32 & 39.52 & 39.52 & 40.77 & 39.62 & 76.02 & 76.02 & 75.96 & 75.35  \\
 \hline
 16 & 39.81 & 39.71 & 39.90 & 40.38 & 75.05 & 73.82 & 75.72 & 76.09  \\
 \hline
 \hline
 \multicolumn{5}{|c|}{Winogrande (FP32 Accuracy = 70.64\%)} & \multicolumn{4}{|c|}{Piqa (FP32 Accuracy = 79.16\%)} \\ 
 \hline
 \hline
 64 & 69.14 & 70.17 & 70.17 & 70.56 & 78.24 & 79.00 & 78.62 & 78.73 \\
 \hline
 32 & 70.96 & 69.69 & 71.27 & 69.30 & 78.56 & 79.49 & 79.16 & 78.89  \\
 \hline
 16 & 71.03 & 69.53 & 69.69 & 70.40 & 78.13 & 79.16 & 79.00 & 79.00  \\
 \hline
\end{tabular}
\caption{\label{tab:mmlu_abalation} Accuracy on LM evaluation harness tasks on GPT3-22B model.}
\end{table}

\begin{table} \centering
\begin{tabular}{|c||c|c|c|c||c|c|c|c|} 
\hline
 $L_b \rightarrow$& \multicolumn{4}{c||}{8} & \multicolumn{4}{c||}{8}\\
 \hline
 \backslashbox{$L_A$\kern-1em}{\kern-1em$N_c$} & 2 & 4 & 8 & 16 & 2 & 4 & 8 & 16  \\
 %$N_c \rightarrow$ & 2 & 4 & 8 & 16 & 2 & 4 & 2 \\
 \hline
 \hline
 \multicolumn{5}{|c|}{Race (FP32 Accuracy = 44.4\%)} & \multicolumn{4}{|c|}{Boolq (FP32 Accuracy = 79.29\%)} \\ 
 \hline
 \hline
 64 & 42.49 & 42.51 & 42.58 & 43.45 & 77.58 & 77.37 & 77.43 & 78.1 \\
 \hline
 32 & 43.35 & 42.49 & 43.64 & 43.73 & 77.86 & 75.32 & 77.28 & 77.86  \\
 \hline
 16 & 44.21 & 44.21 & 43.64 & 42.97 & 78.65 & 77 & 76.94 & 77.98  \\
 \hline
 \hline
 \multicolumn{5}{|c|}{Winogrande (FP32 Accuracy = 69.38\%)} & \multicolumn{4}{|c|}{Piqa (FP32 Accuracy = 78.07\%)} \\ 
 \hline
 \hline
 64 & 68.9 & 68.43 & 69.77 & 68.19 & 77.09 & 76.82 & 77.09 & 77.86 \\
 \hline
 32 & 69.38 & 68.51 & 68.82 & 68.90 & 78.07 & 76.71 & 78.07 & 77.86  \\
 \hline
 16 & 69.53 & 67.09 & 69.38 & 68.90 & 77.37 & 77.8 & 77.91 & 77.69  \\
 \hline
\end{tabular}
\caption{\label{tab:mmlu_abalation} Accuracy on LM evaluation harness tasks on Llama2-7B model.}
\end{table}

\begin{table} \centering
\begin{tabular}{|c||c|c|c|c||c|c|c|c|} 
\hline
 $L_b \rightarrow$& \multicolumn{4}{c||}{8} & \multicolumn{4}{c||}{8}\\
 \hline
 \backslashbox{$L_A$\kern-1em}{\kern-1em$N_c$} & 2 & 4 & 8 & 16 & 2 & 4 & 8 & 16  \\
 %$N_c \rightarrow$ & 2 & 4 & 8 & 16 & 2 & 4 & 2 \\
 \hline
 \hline
 \multicolumn{5}{|c|}{Race (FP32 Accuracy = 48.8\%)} & \multicolumn{4}{|c|}{Boolq (FP32 Accuracy = 85.23\%)} \\ 
 \hline
 \hline
 64 & 49.00 & 49.00 & 49.28 & 48.71 & 82.82 & 84.28 & 84.03 & 84.25 \\
 \hline
 32 & 49.57 & 48.52 & 48.33 & 49.28 & 83.85 & 84.46 & 84.31 & 84.93  \\
 \hline
 16 & 49.85 & 49.09 & 49.28 & 48.99 & 85.11 & 84.46 & 84.61 & 83.94  \\
 \hline
 \hline
 \multicolumn{5}{|c|}{Winogrande (FP32 Accuracy = 79.95\%)} & \multicolumn{4}{|c|}{Piqa (FP32 Accuracy = 81.56\%)} \\ 
 \hline
 \hline
 64 & 78.77 & 78.45 & 78.37 & 79.16 & 81.45 & 80.69 & 81.45 & 81.5 \\
 \hline
 32 & 78.45 & 79.01 & 78.69 & 80.66 & 81.56 & 80.58 & 81.18 & 81.34  \\
 \hline
 16 & 79.95 & 79.56 & 79.79 & 79.72 & 81.28 & 81.66 & 81.28 & 80.96  \\
 \hline
\end{tabular}
\caption{\label{tab:mmlu_abalation} Accuracy on LM evaluation harness tasks on Llama2-70B model.}
\end{table}

%\section{MSE Studies}
%\textcolor{red}{TODO}


\subsection{Number Formats and Quantization Method}
\label{subsec:numFormats_quantMethod}
\subsubsection{Integer Format}
An $n$-bit signed integer (INT) is typically represented with a 2s-complement format \citep{yao2022zeroquant,xiao2023smoothquant,dai2021vsq}, where the most significant bit denotes the sign.

\subsubsection{Floating Point Format}
An $n$-bit signed floating point (FP) number $x$ comprises of a 1-bit sign ($x_{\mathrm{sign}}$), $B_m$-bit mantissa ($x_{\mathrm{mant}}$) and $B_e$-bit exponent ($x_{\mathrm{exp}}$) such that $B_m+B_e=n-1$. The associated constant exponent bias ($E_{\mathrm{bias}}$) is computed as $(2^{{B_e}-1}-1)$. We denote this format as $E_{B_e}M_{B_m}$.  

\subsubsection{Quantization Scheme}
\label{subsec:quant_method}
A quantization scheme dictates how a given unquantized tensor is converted to its quantized representation. We consider FP formats for the purpose of illustration. Given an unquantized tensor $\bm{X}$ and an FP format $E_{B_e}M_{B_m}$, we first, we compute the quantization scale factor $s_X$ that maps the maximum absolute value of $\bm{X}$ to the maximum quantization level of the $E_{B_e}M_{B_m}$ format as follows:
\begin{align}
\label{eq:sf}
    s_X = \frac{\mathrm{max}(|\bm{X}|)}{\mathrm{max}(E_{B_e}M_{B_m})}
\end{align}
In the above equation, $|\cdot|$ denotes the absolute value function.

Next, we scale $\bm{X}$ by $s_X$ and quantize it to $\hat{\bm{X}}$ by rounding it to the nearest quantization level of $E_{B_e}M_{B_m}$ as:

\begin{align}
\label{eq:tensor_quant}
    \hat{\bm{X}} = \text{round-to-nearest}\left(\frac{\bm{X}}{s_X}, E_{B_e}M_{B_m}\right)
\end{align}

We perform dynamic max-scaled quantization \citep{wu2020integer}, where the scale factor $s$ for activations is dynamically computed during runtime.

\subsection{Vector Scaled Quantization}
\begin{wrapfigure}{r}{0.35\linewidth}
  \centering
  \includegraphics[width=\linewidth]{sections/figures/vsquant.jpg}
  \caption{\small Vectorwise decomposition for per-vector scaled quantization (VSQ \citep{dai2021vsq}).}
  \label{fig:vsquant}
\end{wrapfigure}
During VSQ \citep{dai2021vsq}, the operand tensors are decomposed into 1D vectors in a hardware friendly manner as shown in Figure \ref{fig:vsquant}. Since the decomposed tensors are used as operands in matrix multiplications during inference, it is beneficial to perform this decomposition along the reduction dimension of the multiplication. The vectorwise quantization is performed similar to tensorwise quantization described in Equations \ref{eq:sf} and \ref{eq:tensor_quant}, where a scale factor $s_v$ is required for each vector $\bm{v}$ that maps the maximum absolute value of that vector to the maximum quantization level. While smaller vector lengths can lead to larger accuracy gains, the associated memory and computational overheads due to the per-vector scale factors increases. To alleviate these overheads, VSQ \citep{dai2021vsq} proposed a second level quantization of the per-vector scale factors to unsigned integers, while MX \citep{rouhani2023shared} quantizes them to integer powers of 2 (denoted as $2^{INT}$).

\subsubsection{MX Format}
The MX format proposed in \citep{rouhani2023microscaling} introduces the concept of sub-block shifting. For every two scalar elements of $b$-bits each, there is a shared exponent bit. The value of this exponent bit is determined through an empirical analysis that targets minimizing quantization MSE. We note that the FP format $E_{1}M_{b}$ is strictly better than MX from an accuracy perspective since it allocates a dedicated exponent bit to each scalar as opposed to sharing it across two scalars. Therefore, we conservatively bound the accuracy of a $b+2$-bit signed MX format with that of a $E_{1}M_{b}$ format in our comparisons. For instance, we use E1M2 format as a proxy for MX4.

\begin{figure}
    \centering
    \includegraphics[width=1\linewidth]{sections//figures/BlockFormats.pdf}
    \caption{\small Comparing LO-BCQ to MX format.}
    \label{fig:block_formats}
\end{figure}

Figure \ref{fig:block_formats} compares our $4$-bit LO-BCQ block format to MX \citep{rouhani2023microscaling}. As shown, both LO-BCQ and MX decompose a given operand tensor into block arrays and each block array into blocks. Similar to MX, we find that per-block quantization ($L_b < L_A$) leads to better accuracy due to increased flexibility. While MX achieves this through per-block $1$-bit micro-scales, we associate a dedicated codebook to each block through a per-block codebook selector. Further, MX quantizes the per-block array scale-factor to E8M0 format without per-tensor scaling. In contrast during LO-BCQ, we find that per-tensor scaling combined with quantization of per-block array scale-factor to E4M3 format results in superior inference accuracy across models. 

%Text of appendix \ldots

\end{document}
