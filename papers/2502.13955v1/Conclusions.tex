%!TEX root=main.tex
\section{Final Remarks}
\label{sec:conclusions}

We presented a correct-by-construction bounded approach for the synthesis of distributed algorithms. The approach iteratively generates candidate process implementations using the Alloy tool and manipulates the found counterexamples to extract information that helps to refine  the search. In contrast to other approaches, our synthesis algorithm performs a local reasoning, to avoid the explicit construction of the global state space, and applies to general temporal properties, including safety and liveness. We developed a prototype tool that effectively solved common case studies of distributed algorithms. As noted in Section~\ref{sec:examples} the number of shared variables may affect the scalability of our algorithm. In view of this, we plan to extend our approach in several ways. e.g.,  exploring other heuristics to improve the exploration of the instance space.  One benefit of the proposed approach is the versatility provided by the Alloy notation, which allows us to model different kinds of distributed systems, as illustrated with the token ring systems in Section~\ref{sec:examples}.

%: first, we plan to add symmetry-breaking formulas to the generated Alloy code, which may improve the generation of instances; second, we will investigate the possibility of adding environment assumption to our specifications, to simplify the SAT phase; and third, we plan to explore other heuristics to improve the exploration of the instance space.


%and we compared it with {\Sketch}, a state-of-the-art synthesis tool. As noted in Section~6, the number of shared variables affects the scalability of our algorithm. In view of this, we plan to extend our approach in several ways: first, we plan to add symmetry-breaking formulas to the generated Alloy code, which may improve the generation of instances; second, we will investigate the possibility of adding environment assumption to our specifications, to simplify the SAT phase; and third, we plan to use heuristics to improve the exploration of the instance space.
