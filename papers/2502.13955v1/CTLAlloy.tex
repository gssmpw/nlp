\section{Translating CTL to Alloy}\label{sec:ctl-alloy}

In this section we show how \ctl{} can be embedded into Alloy. As shown above, LTSs can be captured in Alloy as in Fig.~\ref{fig:lts-alloy}. Any finite LTS is an instance of that specification. One may also capture a \emph{specific} LTS, simply as an extension of the specification in Fig.~\ref{fig:lts-alloy} that states exactly how each of the LTS fields is formed, via a \emph{fact}. A simple example of this is shown in Fig.~\ref{fig:lts-alloy2}, where an LTS $\ell$ is shown together with its characterization in Alloy. Let us call this model $\mathcal{A}(\ell)$, %(or just $\ell$ when no confusion is possible) 
pointing out that this is the Alloy version of the structure $\ell$. 
%Summing up,  any LTS can be captured in two (equivalent) ways: as an instance of the specification of Fig.~\ref{fig:lts-alloy}, or as an extension of this specification. 
It is also important to notice that, in the specification in Fig.~\ref{fig:lts-alloy2}, we may replace the formula \verb"succs=s"$_0$\verb"->s"$_1$\verb"+s"$_1$-\verb">s"$_0$ by some weaker one, e.g., \verb"succs in s"$_0$-\verb">s"$_1$\verb"+s"$_1$\verb"->s"$_0$. As a result, instances of the specification will be \emph{substructures} of $\ell$, where not all transitions in $\ell$ must necessarily be present. We call this specification $Ref(\ell)$, and if $\ell'$ is an instance of $Ref(\ell)$, then we say that $\ell'$ refines $\ell$ (formally $\ell' \subseteq \ell$). Intuitively, in a refinement some transitions have been disabled. % execution was removed from the model. Note that we can refine a LTS  by reducing its non-determinism.

\begin{figure}[t]
    \centering
    \begin{minipage}{0.5\linewidth}
        \centering
         \includegraphics[scale=0.5]{lts.pdf}
   	% \caption{An LTS}
   % \label{fig:lts}
    \end{minipage}%
    \begin{minipage}{0.5\linewidth} 
        \centering
        \begin{lstlisting}[style=Alloy,basicstyle=\tiny,mathescape=true]
	abstract sig Node { }
        one sig s$_0$, s$_1$ extends Node { }
	abstract sig Prop { }
        one sig p, q extends Prop { }
        one sig $\ell${
		nodes : set Node,
		val : nodes -> Prop,
		succs : nodes -> nodes,
		init : nodes
	}{
		nodes = s$_0$ + s$_1$
		val = s$_0$->p + s$_1$->q
		succs = s$_0$->s$_1$ + s$_1$->s$_0$
		init = s$_0$
	}
	\end{lstlisting}
    \end{minipage}
    \caption{The structure $\ell$ and its Alloy specification}\label{fig:lts-alloy2}
\end{figure}

Let us now describe how \ctl{} formulas are translated into Alloy. First, notice that Alloy is expressive enough to capture \ctl{}: it extends first-order logic with transitive closure, and as shown in \cite{ImmermanVardi97}, this formalism can express the \ctl{} operators. Our translation is however slightly different from that in \cite{ImmermanVardi97}, since ours is tailored ot better fit Alloy's syntax. We show the translation of formulas $\E\X(\phi), \E(\phi \U \psi), \E \G(\phi)$;the rest of \ctl{} can be expressed using this minimal set of operators \cite{Katoen08}. We start with the translation of $\E \G \phi$, expressed as an Alloy predicate:
\begin{lstlisting}[style=Alloy,basicstyle=\tiny,mathescape=true]
pred EG$_{\phi}$[m:LTS, s:Node]{
	(some s':(^(succs$_{\phi}$[m]))[s] | s' in (^(succs$_{\phi}$[modelName]))[s'] and $\phi$[m,s']
}
\end{lstlisting}
This formula is true (in finite LTSs) when we have a cycle along which $\phi$ is true. In this formula, \verb"succs"$_{\phi}$ is obtained from \verb"succs" restricting its domain to those nodes satisfying $\phi$, which can be captured as follows:
\begin{lstlisting}[style=Alloy,basicstyle=\tiny,mathescape=true]
fun Succs$_{\phi}$[m:LTS]:Node -> Node{
	{ n:m.nodes, n':m.nodes | (n->n' in m.succs) and  $\phi$[m, n] }
}
\end{lstlisting}
Formula $\E(\phi \U \psi)$ can be translated as follows:
\begin{lstlisting}[style=Alloy,basicstyle=\tiny,mathescape=true]
pred EU$_{\phi, \psi}$[m:LTS, s:Node]{ (some s':(*(succs$_{\phi}$[m]))[s] | $\psi$[m, s'] }
\end{lstlisting}
Translating formula $\E\X \phi$ is straightforward. Expressing the standard boolean operators is also direct, resorting to Alloy's boolean operators. Given a formula $\phi$, we denote by $\mathcal{A}(\phi)$ its translation to Alloy. Note that the translated formula is an Alloy predicate, taking as parameters a state and an LTS. We will say that $\ell, s \vDash^k_{\mathcal{A}} \phi$ when predicate $\mathcal{A}(\phi)$ holds for a scope of $k$ and state $s$ in specification $\mathcal{A}(\ell)$. This can be automatically checked using the Alloy Analyzer. 

Let us remark that in this paper we restrict ourselves to finite LTS. This is not an important restriction, since \ctl{} satisfies the small model property \cite{Emerson90}; thus, if a formula is satisfiable, then it is satisfiable in a finite LTS.

The following result shows that our translation is sound.
\begin{theorem}\label{theorem:ctl-alloy} Given an LTS $\ell$, a state $s$ and a \ctl formula $\phi$ we have that:
$$
	\ell,s \vDash \phi \mbox{ iff } \ell, s \vDash^k_{\mathcal{A}} \mathcal{A}(\phi) \mbox{ for some $k$ }
$$
\end{theorem}
The proof is by induction. The interested reader is referred to the Appendix. 
%When $\ell, s \vDash^k_{\mathcal{A}} \mathcal{A}(\phi)$ and $s = \ell.init$ then we just write $\ell \vDash^k_{\mathcal{A}} \mathcal{A}(\phi)$. 
 
Interestingly, this result implies that we can use the Alloy Analyzer to check for the satisfiability of a \ctl{} formula: thanks to the fact that \ctl{} has the small model property, checking a formula $\phi$ with a scope of $2^{|\phi|}$ will decide its satisfiability, although of course such a large scope may make the analysis infeasible. 
 
