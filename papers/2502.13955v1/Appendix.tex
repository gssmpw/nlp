\section*{Appendix}
%	For proving Theorem \ref{theorem:ctl-alloy} we assume the standard semantics of Alloy, given by means of relations and the first-order quantifiers.
%\\ \\
%\noindent \textbf{Proof of Theorem \ref{theorem:ctl-alloy}:} For the proof we use the semantics for Alloy presented 
%in \cite{Jackson02}, for each expression $t$, $X[t]e \subseteq \mbox{atom} \times \mbox{atom}$ denotes its semantics as a binary relation (where $\mbox{atom}$ is the universe of atoms). First, let us note that for any LTS $M$, the translation to Alloy fixes only one instance, determining the nodes and the relation of the structure by means of facts, thus the environment is  specified by the facts in the specification. For the sake of simplicity, when no confusion is possible, we write $e$ instead of $\mathcal{A}(e)$ for any expression $e$.
%
%	 The proof is by induction in formula $\phi$, the base case is direct: $M, s \vDash p$, iff $p \in X[M.val[s]]e$ which is equivalent to $M,s \vDash_{\mathcal{A}} p$.
%For the inductive case we	 prove  formula $\E \G \phi$, the rest is similar. 
% 	For the right implication assume $M, s \vDash \E \G \phi$, this means that there is a cycle in $M$ $c_0, c_1, \dots, c_k, c_0$ such that $M, c_i \vDash \phi$ for each $i$ and furthermore
%$c_0$ is reachable from $s$. By induction, we have that $ \mathcal{A}(M), \mathcal{A}(c_i) \vDash \mathcal{A}(\phi)$. Now, let us note that, if $s_i R s_j$ then
%$(X[s_i]e, X[s_j]e)) \in X[M.succs_{\phi}]e$, since we have the text  $\dots$ \verb"s"$_i$ \verb"->" \verb"s"$_j$ $\dots = succs$ in the specification. Thus we have 
%that $X[c_0]e \in (X[M.succs_\phi]e)^+$ (being $R^+$ the transitive closure of $R$), furthermore,  $X[c_0]e$ is reachable from $X[s]e$ and then we have that
%$EG_\phi[M,s]$ (see Section \ref{sec:ctl-alloy}) holds, and then $M, s \vDash_{\mathcal{A}} \E\G \phi$. For the rest of the temporal operators the proof is similar.
%\\ \\
\noindent \textbf{Proof of Theorem  \ref{theorem:stuttering-equiv}. (Sketch)} We prove that for any trace $\pi' \in \traces{T^0 \parallel \dots \parallel T^k}$ starting at state $s$
we have that the projection of this trace to process $T^i$ is stutter equivalent to some trace $\pi \in \traces{T^i}$ starting at state $(s \proj i)$. 
Since stutter equivalence preserves $\LTLX$ properties \cite{PeledWilke1997},  the result follows. 

	Let $\pi' \in \traces{T^0 \parallel \dots \parallel T^k}$ starting a state $s$, then we prove that the trace $\pi \proj i = \pi[0] \proj i \rightarrow \pi[1] \proj i \rightarrow \dots$ is stutter equivalent to a trace 
$\pi \in \traces{T^i}$ starting at state $(s \proj i )$. $\pi$ is just defined by removing all the transitions in $\pi[j] \proj i \rightarrow \pi[j+1] \proj i$  that
do not correspond to transitions in $T^i$, it is simple to see that the removed transitions are stuttering steps, that is $L^i(\pi[j] \proj i)  = L^i(\pi[j+1] \proj i)$, otherwise if $L^i(\pi[j] \proj i)  \neq L^i(\pi[j+1] \proj i)$
the transition must correspond to the transition of another process (say $T^j$); so, $L^i(\pi[j] \proj i)$ and $L^i(\pi[j+1] \proj i)$ can only differ on their valuation of shared variables (they must coincide on the valuation of $\textit{Loc}^i$), but by the conditions \ref{system-spec-formula1}-\ref{system-spec-formula6} we have a matching transition in $T^i$ which is a contradiction. Also note that removing these transitions keeps the trace infinite, since $\pi$ is fair.
	Since we have removed only stuttering steps, the resulting execution is stutter equivalent to $\pi \proj i$. Now, given any {\LTLX} property $\phi$ then suppose
that $T^i \vDash \phi$ and $T^0 \parallel \dots \parallel T^k \nvDash  \phi$, that is, we have some $\pi \in \traces{T^0 \parallel \dots \parallel T^k}$ starting at $s$ such that
$\pi \nvDash \phi$ but since all the propositional variables in $\phi$ are in $Sh \cup Loc^i$ we have that $\pi \proj i \nvDash \phi$, and by the property described above,
we have a $\pi' \in \traces{T^i}(s \proj i)$ such that $\pi' \nvDash \phi$ which is a contradiction, and so $T^0 \parallel \dots \parallel T^k \vDash  \phi$.

\noindent  \textbf{Proof of Theorem \ref{theorem:preserve-safety}.  (Sketch)} % put reference to the theorem
	Observe that for any $\pi=s_0,s_1,\dots \in \traces{T'}$ such that $T' \vDash \refin{T,S}$ we have some $\pi' \in \traces{T}$ with $L(\pi) = L(\pi')$. Now assume that there is
some safety property such that $T' \nvDash \phi$ for some $T' \vDash \refin{T}{S}$ thus since $\phi$ is  safety property, we have a finite path $L(s_0),\dots,L(s_k)$ in which $\phi$
is not true, but then there is some $\pi'$ with the same prefix thus $T \nvDash \phi$ which is a contradiction.	
	
\noindent \textbf{Proof of Theorem 3. (Sketch)} It follows by the definition of $\NOT{(c)}$, note that $\NOT{(c)}  \equiv \neg \CNF{(c)}$, then 
if $T \vDash \refin{T, S}\oplus \NOT{(c)}$ then $T \nvDash \CNF{(c)}$, and the proof follows.

\noindent \textbf{Proof of Theorem 4 (Sketch)} Similar to Theorem 3.

%\noindent \textbf{Proof of Theorem 5.}
%For termination, the loop of line $4$ finalizes because the number of instances of size $k$ for any specification is finite. The loop of line $7$ is also executed a finite number of times: if $i=n$, the theory $\refin{S,M[i]}\oplus (\bigwedge_{c \in \mathit{inCexs}[i]} \NOT(c))$ has a finite number of instances, but note that the set $\mathit{inCexs}[i]$ could be increased (line $15$), but since first-order logic is a monotonic logic,  the number of instances is decremented when more counterexamples are considered, that is, eventually either we will reach an inconsistent specification,  or all the instances will be processed, in both cases the algorithm will exit the loop. For $i<n$,  the set $\mathit{inCexs}[i]$ is updated in a similar way, but when an inconsistent specification is generated (line $28$) we start popping  counterexamples from the stack. In this case, we go back to a specification that has been dealt with before, and then we will get the next instance that was not still processed for this theory, at some point all the instances will be generated and the loop will terminate. Note also that counterexamples are not added twice to $\mathit{inCexs}[i]$ (for any $i$) since already used counterexamples for process $i$ are stored in $\mathit{outCexs}[i]$. 
%
%For the bounded completeness, suppose that there are instances $r_0, \dots, r_n$ of size $k$ such that $r_0 \parallel \dots \parallel r_n \vDash \phi$. Note that in line $5$ all the instances for each $i$ are generated, afterwards the algorithm starts inspecting the refinements of each instance, including itself. As remarked above, the loop of line $7$ will terminate because it eventually starts to pop elements out from the stack, and therefore the next instance of specification $i$ will be obtained. Suppose that $r_0, \dots, r_n$ is not found by the algorithm, this could happens since it added
%a projected part of a counterexample to a specification $\refin{\mathit{PS}^i,T}$ (which could done in some of the auxiliary procedures to speed up the process) that rules out instance $r_i$ but this projected counterexample will be popped out from $\mathit{inCexs[i]}$ (line 21), and therefore
%at some point $\mathit{inCexs[i]}$ will be empty and therefore $r_i$ must be considered (note that a some point the tool will inspect specification $\refin{\mathit{PS}^i, r_i}$), otherwise the SAT solver is unsound which contradicts our assumptions.
%Summing up, (in the worst case) all the combinations of instances of the specifications will be checked by the model checker, and then the algorithm will find the sequence $r_0,\dots,r_n$. 
%
