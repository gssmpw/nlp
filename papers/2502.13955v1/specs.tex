%!TEX root=main.tex
\section{System Specifications}

In this section, we provide a more formal and detailed definition of specifications, as introduced in the previous section.  In the following definitions we  mainly use the first-order logic with transitive closure described in Section~2.  We start by giving a precise definition of the notion of component specification.

\begin{definition} A \emph{process (or component) specification} $\mathit{PS}$ is a tuple $\langle \langle \mathit{Sh},  \mathit{Loc},$ $\mathit{Act}\rangle, \Phi \rangle$ where $\mathit{Sh}$, $\mathit{Loc}$, $\mathit{Act}$ are finite and mutually disjoint sets, and $\Phi$ is a finite set of first-order (relational) formulas over the vocabulary defined by $\mathit{Sh} \cup \mathit{Loc} \cup \mathit{Act}$.
\end{definition}
Intuitively,  \textit{Sh} are the shared variables used by the process,  \textit{Loc} are the process's local variables, and \textit{Act} are the process's actions. A process specification defines a collection of LTSs satisfying the requirements in the specification. Given a process specification $\textit{PS}=\langle \langle \mathit{Sh},  \mathit{Loc}, \mathit{Act}\rangle, \Phi \rangle$ and an LTS $T=\langle S, \mathit{Act}, \rightarrow, I, \mathit{Sh}\cup\mathit{Loc} , L\rangle$, we write $T \vDash PS$ iff $T \vDash \phi$ for every $\phi \in \Phi$.
 
A specification of a distributed system is a collection of process specifications with the same shared variables, and an additional global temporal requirement.
\begin{definition} A system \emph{specification} $\mathcal{S}$ is a tuple $\langle \{ \textit{PS}^i \}_{i \in \mathcal{I}},$ $ \phi \rangle$, where $\mathcal{I}$ is a finite index set, each $\textit{PS}^i = \langle \langle \mathit{Sh}, \mathit{Loc}^i, \mathit{Act}^i\rangle,$ $\Phi^i \rangle$ is a \emph{local} process specification, and $\phi$ is a \emph{global} requirement expressed by an {\LTL} formula over the vocabulary $\mathit{Sh} \cup \coprod_{i \in \mathcal{I}} \mathit{Loc}^i$.
\end{definition}

By simply considering that a system specification $\mathcal{S}$ is a collection of component specifications, we do not provide any specific semantics to shared variables, compared to process local variables. As a consequence,  locally correct implementations may often lead to invalid system implementations (system implementations violating the constraints), since when putting the process implementations together,  any local assumptions on shared variables may simply not hold. This can be solved by forcing local process specifications to assume unrestricted behavior of the environment, which is formalized as follows: 
 
 \begin{definition}\label{def:system-spec}
Let $\mathcal{L} = \{\ell_0, \dots, \ell_m \}$ be a set of elements (called \emph{locks}). An $\mathcal{L}$-\emph{synchronized specification} is a system specification $\mathcal{S} = \langle \{\textit{PS}^i\}_{i \in \mathcal{I}}, \phi \rangle$ with \emph{shared} variables $\{\mathit{av}_{\ell_0}, \dots,$ $\mathit{av}_{\ell_m}\}  \subseteq \mathit{Sh}$, and where each process specification $\textit{PS}^i = \langle \langle \mathit{Sh}, \mathit{Loc}^i, \mathit{Act}^i\rangle, \Phi^i \rangle$ has local variables  $\{ \mathit{own}_{\ell_0}, \dots,$ $\mathit{own}_{\ell_m} \} \subseteq \mathit{Loc}^i$, and actions $\{\textit{ch}_{\ell_0},\dots,\textit{ch}_{\ell_m}\} \cup \{\mathit{ch}_g \mid g \in \mathit{Sh}\} \subseteq \mathit{Act}^i$. Furthermore, the following formulas belong to each $\Phi^i$:
\begin{enumerate}[(a)]%[font=\normalfont]
	\item\label{system-spec-formula1}	 $\bigwedge_{\ell \in \mathcal{L}}(\forall s : \textit{own}_\ell(s) \Rightarrow   \neg \mathit{av}_\ell(s))$,
	\item\label{system-spec-formula2} $\bigwedge_{\ell \in \mathcal{L}}(\forall s  : \neg \textit{own}_\ell(s) \equiv (\exists s'  : s \overset{\textit{ch}_\ell}{\rightarrow} s'))$,
	\item\label{system-spec-formula3} $\bigwedge_{\ell \in \mathcal{L}}(\forall s,s'  : s \overset{\textit{ch}_\ell}{\rightarrow}{s'} \Rightarrow (\textit{av}_\ell(s) \equiv  \neg \textit{av}_\ell(s'))$,
	\item\label{system-spec-formula4} $\bigwedge_{\ell \in \mathcal{L}}\bigwedge_{v  \in (\textit{Sh}\cup\mathit{Loc} \setminus \{\mathit{own}_\ell, \mathit{av}_\ell\})}(\forall  s,s'  : s \overset{\textit{ch}_\ell}{\rightarrow} s' \Rightarrow (v(s) \equiv v(s'))$,
	\item\label{system-spec-formula5}   $\bigwedge_{g \in \textit{Sh}\setminus\{\mathit{av}_{\ell_0}, \dots,\mathit{av}_{\ell_m}\} }(\forall s: (\exists s': s \overset{ch_g}{\rightarrow} s' \wedge g(s')) \wedge  (\exists s': s \overset{ch_g}{\rightarrow} s' \wedge \neg g(s')))$,
	\item\label{system-spec-formula6} 	$\bigwedge_{g \in \mathit{Sh}}\bigwedge_{g'  \in (\textit{Sh}\cup\mathit{Loc} \setminus \{g\})}(\forall  s,s'  : s \overset{\textit{ch}_g}{\rightarrow} s' \Rightarrow (g'(s) \equiv g'(s'))$.
% these are the formulas added in the Alloy spec:
%			all s:philMeta.nodes | Own_left[philMeta, s] implies (not Av_left[philMeta, s])
%			 all s:philMeta.nodes | (not Own_left[philMeta,s]) iff (some philMeta.ACTchange_left[s])
%			 all s:philMeta.nodes | all s':philMeta.ACTchange_left[s] | Av_left[philMeta,s] iff (not Av_left[philMeta, s']) 
%			all s:philMeta.nodes | all s':(philMeta.env[s] - philMeta.ACTchange_left[s]) | Av_left[philMeta,s] iff Av_left[philMeta, s']
\end{enumerate}
\end{definition}
Intuitively, locks are special kinds of shared variables used for synchronization.  Actions $\{\textit{ch}_{\ell_0},\dots,\textit{ch}_{\ell_m}\} \cup \{\mathit{ch}_g \mid g \in \mathit{Sh}\}$ 
are used to model environment actions, for instance, $\textit{ch}_{\ell_0}$ represents an action of the environment that changes the state of lock $\ell_0$.  Variables $\mathit{av}_i$ and
$\mathit{own}_i$ are used to indicate that a lock is free,  or owned by the current component.  Formula \ref{system-spec-formula1} expresses that, if a component owns a lock, then it is not available.
Formula \ref{system-spec-formula2}  expresses that, if a lock is not owned by the current component, then it can be changed by the environment.  Formula \ref{system-spec-formula3} says that, if the environment changes lock $\ell_i$, then the variable 
$\mathit{av}_i$ changes accordingly.  Formula \ref{system-spec-formula4} states that changes in the locks do not affect other variables.  Formula \ref{system-spec-formula5} expresses that shared variables can be changed by the environment. Finally, formula \ref{system-spec-formula6} states that the changes in one shared variable does not affect the other variables.
From now on, we write $s \dashrightarrow s'$ if $s \xrightarrow{\textit{ch}_\ell} s'$ or $s \xrightarrow{\textit{ch}_g} s'$, for some lock $\ell$ or shared variable $g$, respectively.

In the same way as component specifications can be put together, we also consider the asynchronous composition of LTSs.
\begin{definition} Let $\mathcal{S} = \langle \{\textit{PS}^i\}_{i \in [0,n]}, \phi \rangle$ be an $\mathcal{L}$-\emph{synchronized specification}, and 
let $T^0,\dots,T^n$ be LTSs, such that for each $T^i = \langle \mathit{S}^i, \mathit{Act}^i, \rightarrow^i,I^i, \mathit{Sh}\cup \mathit{Loc}^i, L^i \rangle$ we have $\mathit{Sh}\cap \mathit{Loc}^i = \emptyset$ and $T^i \vDash \mathit{PS}^i$. We define the LTS $T^0 \parallel \dots \parallel T^n = \langle S, \coprod_{i \in [0,n]} \mathit{Act}^i, \rightarrow, I, \mathit{Sh}\cup \coprod_{i \in [0,n]}\mathit{Loc}^i, L\rangle$   as follows:
\begin{itemize}	
	\item $S = \{s \mid s \in \prod_{i \in [0,n]}S^i \wedge (\forall g \in \mathit{Sh} : \forall i,j \in [0,n] : L^i(s \proj i)(g) = L^j(s \proj j)(g)) \}$,
	%\item $\mathit{Act} =  \coprod_{i \in [0,k]} \mathit{Act}^i$,
	\item $\rightarrow  = \{ s \xrightarrow{a} s' \mid \exists i \in [0,n] : (s \proj i \xrightarrow{a}\mathrel{\vphantom{\to}^i}  s' \proj i) \wedge (\forall j \neq i : (s \proj j \dashrightarrow^j s' \proj j) \vee (s \proj j  =  s' \proj j) \}$,
	\item $I  =\{ s \mid \forall i \in [0,n]: s{\uparrow}i \in I^i\}$,	
	%\item $\mathit{AP} = \mathit{Sh} \cup	\bigcup_{i \in [0,n]}\mathit{AP}^i$,
	%\item $L(s)(x) =  L_i(s \proj i)(x) \mbox{ if  $x \in Loc_i$ for some $i \in [0,n]$ }$,
	%\item $L(s)(x)  =  L_0(s \proj 0)(x)$, otherwise.
	\item  $L(s)(x)=  \begin{cases*}
      					L^i(s \proj i)(x) & if  $x \in Loc^i$ for some $i \in [0,n]$, \\
     					L^0(s \proj 0)(x)      & otherwise.
   				  \end{cases*}$
\end{itemize}
\end{definition}
The semantics of distributed specifications is straightforwardly defined using asynchronous product, i.e., the combination of LTSs that produces all interleavings of their corresponding actions.
\begin{definition} Given a system specification $\mathcal{S} = \langle \{\mathit{PS}^i\}_{i \in [0,n]},$ $\phi \rangle$ a \emph{model} or \emph{implementation} of $\mathcal{S}$ is a collection of LTSs $T^0,\dots,T^n$ such that $T^i \vDash \mathit{PS}^i$, for every $i \in [0,n]$, and $T^0 \parallel \dots \parallel T^n \vDash \phi$.
\end{definition}
An interesting point about the definition above, which in particular holds thanks to the formulas that give semantics to shared variables, is that linear-time temporal properties (without the next operator) local to the process implementations can be promoted to global implementations, i.e., to asynchronous products they participate in, provided the asynchronous product is \emph{strongly fair} (strong fairness assumption is necessary to guarantee the promotion of liveness properties). That is, LTSs satisfying \ref{system-spec-formula1}-\ref{system-spec-formula6} preserve their (local) temporal properties under \emph{any} environment that guarantees strong fairness:
\begin{theorem}\label{theorem:stuttering-equiv} Let $\langle \{ \mathit{PS}^i \}_{i \in [0,n]}, \phi \rangle$ be a distributed specification, and $T^0, \dots, T^n$ LTSs such that $T^i \vDash \mathit{PS}^i$ (for every $i \in [0,n]$). Given an {\LTLX} formula $\psi$, if $T^i \vDash \psi$  (for any $i \in [0,n]$), then $T^0 \parallel \dots \parallel T^n \vDash_f \psi$.
\end{theorem}
	
From an LTS $T$, we can obtain a specification that characterizes it (up to isomorphism), using existentially quantified variables for identifying the states, and formulas for describing the transitions. Moreover,we can obtain a specification that characterizes \emph{all} LTSs that can be obtained by removing some (local) transitions from $T$. Intuitively,  this specification captures refinements of $T$. 
\begin{definition}\label{def:ref-spec} Let $\mathit{PS}= \langle \langle \mathit{Sh},  \mathit{Loc}, \mathit{Act}\rangle, \Phi \rangle$ be a component specification, and let $T=\langle S, \mathit{Act}, \rightarrow, I, (\mathit{Sh} \cup \mathit{Loc}), L\rangle$ be an LTS, such that $S=\{s_0,\dots,s_n\}$, $\mathit{Sh} \cup \mathit{Loc}=\{p_0,\dots,p_m\}$, $\mathit{Act}=\{a_0,\dots,a_k\}$ and $T \vDash \mathit{PS}$. The process specification $\refin{\mathit{PS},T}$ is the tuple $\langle \langle \mathit{Sh}, \mathit{Loc}, \mathit{Act}\rangle, \Phi \cup \{ \exists s_0,\dots,s_n: \phi^T\} \rangle$ where $\phi^T$ is the following formula:
 \[\displaystyle
\begin{array}{l}(\bigwedge_{0\leq i< j \leq n} s_i \neq s_j) \wedge I(s_0) 
															\wedge \bigwedge_{j \in [0,n]} \bigwedge_{i \in [0,m]} \{  p_i(s_j) \mid p_i \in L(s_j) \}  \\
															\wedge  \bigwedge_{j \in [0,n]} \bigwedge_{i \in [0,m]} \{ \neg p_i(s_j) \mid p_i \notin L(s_j) \} \\
															\wedge \bigwedge_{j,j' \in [0,n]} \bigwedge_{i \in [0,k]}  \{ \neg a_i(s_j,s_{j'}) \mid \neg (s_j \xrightarrow{a_i} s_{j'}) \}\\
															\wedge \bigwedge_{j,j'\in [0,n]} \bigwedge_{i \in [0,k]} \{ a_i(s_j,s_{j'}) \mid 
															s_j \xrightarrow{a_i},s_{j'} \wedge a_i \in \{\textit{ch}_g \mid g \in \mathit{Sh} \} \}   
																													
\end{array}				
\]
\end{definition}
Formula $\exists s_0,\dots,s_n: \phi^T$ identifies each state with a variable, describes the properties of each state, enumerates the transitions labeled with environment actions, and rules out the introduction of local transitions not present in $T$. Summing up, models of $\refin{\mathit{PS}, T}$ may remove some local transitions present in $T$ but still satisfy specification $\mathit{PS}$. It is direct to see that $T \vDash \refin{\mathit{PS},T}$. Furthermore, $\refin{\mathit{PS},T}$ preserves all the safety properties of $T$ (cf. \cite{Katoen08} for a formal definition of safety).
\begin{theorem}\label{theorem:preserve-safety} Let $\phi$ be an {\LTL} safety formula, $\mathit{PS}$ a process specification, and $T$ an LTS such that $T \vDash \mathit{PS}$. Then:
$
	T \vDash \phi \text{ implies } \refin{\mathit{PS},T} \vDash \phi. 
$
\end{theorem}

The following notation will be useful in later sections, to refer to specifications complemented with additional formulas.

%\begin{figure*}[h!]
%\begin{minipage}[b]{0.60\linewidth}
%\[
% [a] \left( \begin{array}{l} 
% 			 state = [s] \\
%			 \wedge \bigwedge \{x = x(s) \mid x \in \mathit{Sh} \cup \mathit{Loc} \}\\ 
%			\wedge \bigwedge \{ \ell = i \mid \ell \in \mathcal{L} : \mathit{own}_\ell \in L(s)\}	 \\ 
%			 \wedge \bigwedge \{ \ell = \bot \mid \ell \in \mathcal{L} : \mathit{av}_\ell \in L(s)\}  \end{array} \right) \rightarrow \left( \begin{array}{l} 
% 																					  			  \{x {:=} x(s')  \mid x \in \mathit{Loc}\cup \mathit{Sh}\} \\ 
%																					 			  \cup \{\mathit{state} {:=} [s'] \} \\
%																								  \cup \{ \ell := i \mid \ell \in \mathcal{L} : \mathit{own}_\ell \in L(s)\}  \\
%																								  \cup \{ \ell := \bot \mid \ell \in \mathcal{L} : \mathit{av}_\ell \in L(s)\}
%																					  \end{array} \right)
%\]
%\end{minipage}
%\caption{Interpretation of LTSs as Guarded-Command programs}\label{fig:guarded-command}
%\end{figure*}


\begin{definition} Let $\mathit{PS}= \langle \langle \mathit{Sh},  \mathit{Loc}, \mathit{Act}\rangle, \Phi \rangle$ be a process specification, and $T=\langle S, \mathit{Act}, \rightarrow, I, (\mathit{Sh} \cup \mathit{Loc}), L\rangle$ an LTS such that $S=\{s_0,\dots,s_n\}$, and $T \vDash \mathit{PS}$. Given a formula $\psi$ with free variables $s_0,\dots,s_n$, we define:
\[
	\refin{\mathit{PS},T} \oplus \psi = \langle \langle \mathit{Sh}, \mathit{Loc}, \mathit{Act}\rangle, \Phi \cup \{ \exists s_0,\dots,s_n: \phi^T \wedge \psi\} \rangle
\]	
\end{definition}
\section{Obtaining Guarded-Command Programs.}
    In this section we describe how we obtain the guarded-command programs from LTSs.  It is worth noting that the programs we synthesize  use locks for achieving synchronization. When a guard checks for a lock availability and this is not available  the process may continue executing other branches (i.e., our locks are not blocking). However, note that a process could get blocked when all its guards are false, thus other synchronization mechanisms such as blocking locks, semaphores and condition variables can be expressed by these programs. 
    
	Given LTSs $T^i = \langle \mathit{S}^i, \mathit{Act}^i, \rightarrow^i,I^i, \mathit{Sh}\cup \mathit{Loc}^i, L^i \rangle$ for $i \in [0,n]$, we can abstract away the environmental transitions and define a corresponding program in  guarded-command notation,  denoted $\text{Prog}(T^0  \parallel \dots \parallel T^{n})$, as follows. The shared variables are those in $\textit{Sh}$ plus an additional shared variable $\ell$ for each lock, with domain $[0,n-1]\cup\{\bot\}$ (where $\bot$ is a value indicating that the lock is free). Additionally, for each $T^i$ we define a corresponding process. To do so, we introduce for each $s \in S^i$  the equivalence class: $[ s ]  = \{ s' \in S^i \mid (s \dashrightarrow^* s') \vee (s'  \dashrightarrow^* s)\}$. 
That is, it is the set of states connected to $s$ via environmental transitions. It is direct to see that it is already an equivalence class. The collection of all equivalence classes
is denoted $S^i /_{\dashrightarrow^*}$.The local variables of the process are those in $\textit{Loc}^i$ plus a fresh variable $\textit{st}_i$, with domain $S^i/_{\dashrightarrow^*}$ (for indicating the current state of the process). 

Finally, given states $s,s' \in S^i$ with $s \xrightarrow{a} s' \in \rightarrow^i$ and  $[s] \neq [s']$, we consider the following  guarded command:
\[
 [a] \left( \begin{array}{l} 
 			 state = [s] \\
			 \wedge \bigwedge \{x = x(s) \mid x \in \mathit{Sh} \cup \mathit{Loc} \}\\ 
			\wedge \bigwedge \{ \ell = i \mid \ell \in \mathcal{L} : \mathit{own}_\ell \in L(s)\}	 \\ 
			 \wedge \bigwedge \{ \ell = \bot \mid \ell \in \mathcal{L} : \mathit{av}_\ell \in L(s)\}  \end{array} \right) \rightarrow \left( \begin{array}{l} 
 																					  			  \{x {:=} x(s')  \mid x \in \mathit{Loc}\cup \mathit{Sh}\} \\ 
																					 			  \cup \{\mathit{state} {:=} [s'] \} \\
																								  \cup \{ \ell := i \mid \ell \in \mathcal{L} : \mathit{own}_\ell \in L(s)\}  \\
																								  \cup \{ \ell := \bot \mid \ell \in \mathcal{L} : \mathit{av}_\ell \in L(s)\}
																					  \end{array} \right)
\]

%	Note that this definition may result in programs with redundant branches, which can be simplified in different ways. 
We can prove that our translation from transition systems to programs is correct. That is, the executions of the program satisfy the same temporal properties as the 
asynchronous product $T^0 \parallel \dots \parallel T^{n}$.
\begin{theorem}\label{th:proppreservation} Given LTSs $T^i$ and \textsf{LTL} property $\phi$, then we have that:
$
	T^0 \parallel \dots \parallel T^{n} \vDash \phi \Leftrightarrow Prog(T^0 \parallel \dots \parallel T^{n}) \vDash \phi
$
\end{theorem}
\begin{figure}[t!]
\begin{minipage}[b]{0.30\linewidth}
\centering
\includegraphics[scale=0.6]{Figs/mutex.pdf}\label{fig:mutex-lts}
%\caption{LTS  for Mutex}\label{fig:mutex-lts}
%\end{figure}
\end{minipage}
%\begin{figure}
\hspace{0.7cm}
\begin{minipage}[b]{0.70\linewidth}
\centering
\begin{lstlisting}[style=Unity]
Program Mutex
 var m:Lock;
  Process $P_i$ with $i \in \{0,1\}$
   var $\text{try}_i, \text{ncs}_i, \text{cs}_i$:bit
   var $\text{st}_i$:$\{\text{S0},\text{S1},\text{S2},\text{S3},\text{S4},\text{S5}\}$
   initial: $\text{ncs}_i\wedge \neg \text{cs}_i \wedge \neg \text{try}_i$ 
   begin
    [enterTry]$\text{st}_i$=S5 $\rightarrow$ $\text{st}_i$:=S3,$\text{try}_i$:=1,$\text{ncs}_i$:=0
    [getLock]$\text{st}_i$=S3$\wedge$ m=$\bot$ $\rightarrow$ $\text{st}_i$:=S1,$\text{try}_i$:=1,m:=i
    [enterCS] $\text{st}_i$=S1$\rightarrow$ $\text{st}_i$:=S0,$\text{cs}_i$:=1,$\text{try}_i$:=0
    [enterNCS]$\text{st}_i$=S0$\rightarrow$st:=S5,$\text{ncs}_i$:=1,$\text{cs}_i$:=0 m:=$\bot$
   end
end
\end{lstlisting}
\end{minipage}
\caption{LTS and corresponding program for mutex}\label{fig:mutex}
\end{figure}
\begin{example}[Mutex]  
\label{ex:mutex-ts}
Consider a system composed of two processes (it can straightforwardly be generalized to $n$ processes) both with non-critical, waiting and critical sections. The global property  is mutual exclusion: the two processes cannot be in their critical sections simultaneously. We consider one lock $m$,  actions: $\mathit{enterNCS}$ (the process enters to the non-critical section), $\mathit{enterCS}$ (the process enters to the critical section), $\mathit{getLock}$ (the process acquires the lock), $\mathit{enterTry}$ (the process goes to the try state), and the corresponding propositions $\mathit{ncs}, \mathit{cs}, \mathit{try}, \mathit{own}_m, \mathit{av}_m$.  

The transition system and the program corresponding to this example are shown in Fig.~\ref{fig:mutex}. It is interesting to observe that,  the  conditions in Definition \ref{def:system-spec} allow one to use the locks as synchronization mechanisms, if the component $i$ owns the lock $m$, then the other processes cannot go into their critical sections since the lock will be not available for them. Also,  it is worth noting that we consider one state in the program for each equivalence class of states in the LTS.
\end{example}

%
%	We can prove that our translation from transition systems to programs is correct. That is, the executions of the program satisfy the same temporal properties as the 
%asynchronous product $P_0 \parallel \dots \parallel P_n$.
%\begin{theorem}\label{th:proppreservation} Given a program $P_0 \parallel \dots \parallel P_n$ and \textsf{LTL} property $\phi$, then we have that:
%$
%	P_0 \parallel \dots \parallel P_n \vDash \phi \Leftrightarrow Prog(P_0 \parallel \dots \parallel P_n) \vDash \phi
%$
%\end{theorem}
%such that, for every $\pi \in Tr(P_0 \parallel \dots \parallel P_n)$ and $i \geq 0$ we have that: $\pi[i] \vDash \phi$ iff $f(\pi)[i] \vDash \theta(\phi)$, being $\phi$ any boolean formula. 
% In order to prove this, we need to consider a translation (named $\theta$) between boolean formulas built up from the transition structure and the  program's boolean expressions.
%It is defined recursively as follows: $\theta(own_\ell) = (\ell = i)$, $\theta(av_\ell) = (\ell = \bot)$, $\theta(p) = p$ (for any $p \in Loc \cup Sh$), and $\theta(\varphi \vee \psi) = \theta(\varphi) \vee \theta(\psi)$, $\theta(\varphi \wedge \psi) = \theta(\varphi) \wedge \theta(\psi)$, $\theta(\neg \varphi) = \neg \theta(\varphi)$. Then, we can prove the following theorem:
%\begin{theorem}\label{th:proppreservation} Given a program $P_0 \parallel \dots \parallel P_n$, we have a one-to-one function $f: Tr(P_0 \parallel \dots \parallel P_n) \rightarrow Tr(Prog(P_0 \parallel \dots \parallel P_n))$,
%such that, for every $\pi \in Tr(P_0 \parallel \dots \parallel P_n)$ and $i \geq 0$ we have that: $\pi[i] \vDash \phi$ iff $f(\pi)[i] \vDash \theta(\phi)$, being $\phi$ any boolean formula. 
%\begin{proof} 
%	Let us define for each $\pi \in Tr(P_0 \parallel \dots \parallel P_k)$ a corresponding execution $f(\pi) \in Tr(Prog(P_0 \parallel \dots \parallel P_k))$. The definition of $f(\pi)$ is by induction.
%$\pi[0]$ corresponds to the initial state of the program, by definition the two satisfy the same boolean formulas. Assume that $f(\pi[0]) \dots f(\pi[n])$ are defined, and consider $\pi[n+1]$. We know that there is a transition $\pi[i] \xrightarrow{a} \pi[i+1]$, then by definition of the asynchronous product we have a process $P_j$ with a transition
%$\pi[i] \proj j \xrightarrow{a} \pi[i+1] \proj j$, if $[\pi[i]] = [\pi[i+1]]$ then we define $f(\pi[i+1]) = f(\pi[i])$, otherwise we have a corresponding guarded command $B \rightarrow C$ in the process 
%corresponding to $P_j$ such that, by induction, $f(\pi[i]) \vDash B$  and we define $f(\pi[i+1])$ as the state obtained after executing $C$, which by definition of the program satisfies the same
%boolean properties as $\pi[i+1]$. In a similar way we can define the inverse of $f$.
%\end{proof}
%\end{theorem}
\vspace{-0.5cm}



%Operator $\oplus$ allows us to add formulas to $\refin{\mathit{PS},T}$ capturing refinements of $T$.

