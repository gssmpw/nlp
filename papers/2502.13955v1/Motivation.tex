%!TEX root=main.tex
\section{Motivating Example}\label{sec:motivating-example}
%
\sloppy We motivate and illustrate our technique with a well-known example of  a distributed system, Dijkstra's \emph{dining philosophers} \cite{Dijkstra71}.   There are $n$ philosophers sitting around a table and sharing $n$ forks.  Each philosopher has exactly two forks available, one to each side, shared with the adjacent  philosophers. On the table, there is a bowl of pasta, and the philosophers alternate between thinking and eating.  
\begin{wrapfigure}[17]{r}{.40\textwidth}
 \centering
 \vspace{-0.7cm}
 \includegraphics[scale=0.3]{Figs/philsEx.pdf} % first figure itself
  \caption{LTS $T^i$ modelling philosopher $i$}\label{fig:philsEx}
\end{wrapfigure}
Each philosopher needs both forks to eat, and the access to a fork is (of course) mutually exclusive between its two users.  We want to design a  distributed protocol for this problem  that guarantees deadlock freedom. %, as well as overall \emph{fairness}, i.e., every philosopher is allowed to eat infinitely many times. 
This problem admits many  solutions. For instance,  a way to avoid deadlock is implementing a policy where each philosopher picks up a fork only when her both forks are free.  Another possible solution is obtained  if  each philosopher first takes her left fork and then her right one,  excepting  one philosopher that takes the forks in the opposite order. These are distributed protocols, i.e., they do not depend on a centralised architecture. These are the kind of solutions we expect our technique to synthesize. 
\begin{wrapfigure}[25]{hr}{.25\textwidth}
    \vspace{-0.7cm}
    \includegraphics[scale=0.45]{Figs/TracePhil.pdf} % second figure itself
    \caption{A Path in $T^0 \parallel T^1 \parallel T^2$ leading to deadlock}\label{fig:philsTrace}
\end{wrapfigure}
Let us describe how this problem is specified in our setting.  The behavior of  philosopher's $i$ (for $0 \leq i < n$) is modeled  via the following set $\mathit{PS}^i$ of formulas:
\begin{enumerate}[(1)]
\item \label{phils-form-1} $\forall s : I(s) \Rightarrow \mathit{thk}^i(s) \wedge \neg \mathit{ownLeft}^i(s) \wedge \neg \mathit{ownRight}^i(s)$,
\item \label{phils-form-2}  $\forall s,s' : \neg \mathit{eat}^i(s) \Rightarrow \neg \textit{getThk}^i(s,s')$,
\item \label{phils-form-3} $\forall s,s' : \textit{getThk}^i(s,s') \wedge \mathit{eat}^i(s) \Rightarrow \mathit{thk}^i(s') \wedge \neg \mathit{ownRight}^i(s') \wedge \neg \mathit{ownLeft}^{i }(s') $,
\item  \label{phils-form-4} $\forall s,s': \textit{getHgr}^i(s,s') \Rightarrow \mathit{hgr}^i(s')$,
\item \label{phils-form-5} $\forall s, s':  \neg (\mathit{hgr}^i(s) \land \mathit{ownRight}^i(s) \land \mathit{ownLeft}^i(s)) \Rightarrow \neg \mathit{getEat}^i(s,s')$,
\item \label{phils-form-6} $\forall s,s': \textit{getEat}^i(s,s') \wedge \mathit{hgr}^i(s) \wedge \mathit{ownRight}^i(s) \wedge \mathit{ownLeft}^i(s) \Rightarrow \mathit{eat}^i(s')$,
\item \label{phils-form-7} $\forall s,s': \mathit{getRight}^i(s,s') \wedge \mathit{avFork}_{i+1}(s) \wedge \mathit{hgr}^i(s) \Rightarrow \mathit{ownRight}^i(s')$,
\item  \label{phils-form-8}$\forall s,s': \mathit{getLeft}^i(s,s') \wedge \mathit{avFork}_{i}(s) \wedge \mathit{hgr}^i(s) \Rightarrow \mathit{ownLeft}^i(s')$,
\item \label{phils-form-9}$\forall s \in S : I(s) \Rightarrow (\exists s' \in S: \tcpost{s}{s'} \wedge \mathit{eat}^i(s'))$,
\item \label{phils-form-10}$\forall s :  \mathit{avFork}_i(s)  \Rightarrow (\exists s' : \mathit{chFork_{i}}(s,s') \wedge \neg \mathit{avFork}_i(s'))$,
\item \label{phils-form-11}$\forall s :  \mathit{avFork}_{i+1}(s)  \Rightarrow (\exists s' : \mathit{chFork_{i+1}}(s,s') \wedge \neg \mathit{avFork}_{i+1}(s'))$.
\end{enumerate}

Thus, $\mathit{PS}^i$ describes the \emph{local} specification of philosopher $i$. These formulas may originate from a specification in a domain specific language, referring to actions (e.g., $\mathit{getEat}$), their corresponding guards (e.g., formula \ref{phils-form-5}) and effects (e.g., formula \ref{phils-form-6}). We do not deal with the design of a specification language in this paper, and directly refer to the set of formulas capturing the components behaviors. Notice how the local view of the system state is captured. Each philosopher uses Boolean variables $\mathit{ownLeft}^i$ and $\mathit{ownRight}^i$ for signaling the acquisition of the right and left fork, respectively. Variables $\mathit{avFork}_{i}$ and $\mathit{avFork}_{i+1}$ are used to indicate whether a fork is busy or not (being $+$ the addition modulo $n$).  The actions for philosopher $i$ are: $\mathit{getThk}^i$ (the philosopher goes to the thinking state), $\mathit{getHgr}^i$ (the philosopher gets hungry), $\mathit{getEat}^i$ (the philosopher goes to the eating state). Philosophers also have actions to obtain the forks: $\mathit{getRight}^i$ (obtain the right fork if available), and $\mathit{getLeft}^{i}$ (obtain the left fork of available). %The action definitions are relatively straightforward, as shown above. Note that
Variables $\mathit{avFork}_{i}$ and $\mathit{avFork}_{i+1}$ are shared with other philosophers. 
%\begin{figure}[t!]
%    \centering
%    \begin{minipage}{0.45\textwidth}
%        \centering
%        \includegraphics[scale=0.3,width=0.9\textwidth]{Figs/philsEx.pdf} % first figure itself
%        \caption{LTS $T^i$ modelling philosopher $i$}\label{fig:philsEx}
%    \end{minipage}\hfill
%    \begin{minipage}{0.45\textwidth}
%        \centering
%        \includegraphics[scale=0.4]{Figs/TracePhil.pdf} % second figure itself
%        \caption{A Path in $T^0 \parallel T^1 \parallel T^2$ leading to deadlock}\label{fig:philsTrace}
%    \end{minipage}
%\end{figure}
%
Formula \ref{phils-form-9} in $\mathit{PS}^i$ is a reachability (local) constraint, requiring that philosopher $i$ can  reach the eating state from the initial state. In this example, the forks act as locks: they are shared variables, and thus can be changed by the ``environment'' (from a local philosopher's perspective). This is captured through actions $\mathit{chFork}_{i}$ and $\mathit{chFork}_{i+1}$ that model the acquisition of the locks by the environment. Constraints \ref{phils-form-10}-\ref{phils-form-11} establish that, if a fork is free, it can be grabbed by another philosopher. 

The inputs for our technique include a \emph{global} temporal requirement, constraining the components interaction.  For instance,  in the case  $n=3$ we will have the local specifications $\mathit{PS}^0, \mathit{PS}^1$, and $\mathit{PS}^2$ together with the  {\LTL}  formula:
\begin{multline*}
   \Box \neg (\mathit{ownRight}^0 \wedge \mathit{ownRight}^1 \wedge \mathit{ownRight}^2) 
   \wedge \neg (\mathit{ownLeft}^0 \wedge \mathit{ownLeft}^1 \wedge \mathit{ownLeft}^2),
\end{multline*}
which states that  the system is deadlock-free.  

Note that, for each specification $\mathit{PS}^i$, one can employ a model finder to build an LTS $T^i$ that satisfies specification $\mathit{PS}^i$. Given the expressiveness of the logic, this can only be done up to a given bound on the size of the LTSs (the logic is undecidable). We use the Alloy Analyzer~\cite{AlloyBook}, a bounded model finder for relational logic (which subsumes first-order logic with transitive closure), for this task. Fig.~\ref{fig:philsEx} shows an LTS satisfying $\mathit{PS}^i$, obtained in this way. The dotted arrows represent environment transitions (which, once components are composed, will become actions performed by other philosophers). The global system behavior is obtained by the (asynchronous) parallel composition of the obtained LTSs, denoted  $T^0 \parallel T^1 \parallel T^2$. Now, we can verify whether the global system satisfies the global properties. In our example, if the LTSs $T^0$, $T^1$ and $T^2$ are as shown in Fig.~\ref{fig:philsEx}, then the global system will contain an execution leading to deadlock. 
The trace in Fig.~\ref{fig:philsTrace}, in which the three philosophers get hungry in turns, and then take their corresponding right fork, leads to deadlock. %Since all the forks have been taken, no one of the philosophers is able to pick the left hand side fork to eat. Thus, the system reach a deadlock situation, violating the global property. 

Our approach exploits counterexamples to global properties to guide the search for local implementations.  For instance,  we force at least one of the local implementations to avoid the consecutive execution of $\textit{getHgr}^i$ and $\mathit{getRight}^i$. Assuming that under this new constraint we obtain an LTS $T'^{0}$ for philosopher $0$, that takes the left fork before taking the right one, then the global system $T'^{0} \parallel T^1 \parallel T^2$ is deadlock-free and satisfies the global temporal requirement.


%An example of execution in 	$T_1 \parallel T_2 \parallel T_3$ is show:
%\[
%	(s^0_0, s^1_0, s^2_0) \rightarrow (s^0_4,s^1_0,s^2_0) \rightarrow (s^0_4,s^1_4,s^2_0)  \rightarrow (s^0_4,s^1_4,s^2_4)  \rightarrow (s^0_8,s^1_1,s^2_7) \rightarrow
%\]
