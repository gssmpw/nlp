\section{Evaluation}
\label{sec:eval}
In this section, we provide a detailed evaluation of our system against various baselines, on two European Portuguese benchmarks. 
The code for training and evaluation, as well as the trained checkpoints of our models, is available in our repository\footnote{\url{https://github.com/hmosousa/tradutor}}.

\subsection{Metrics}

For a comprehensive evaluation, we include both classical and embedding-based metrics.

\paragraph{N-gram based metrics} Classical machine translation metrics~\cite{Popovic2015,Papineni2002,Lin2004,Banerjee2005} usually focus on n-gram overlap between the reference translation and the translation generated by the system. Although by design this metrics fail to recognize semantic similarity beyond the lexical level, they remain widely adopted in academic research. In this work, we incorporate the two most common metrics: BLEU~\cite{Papineni2002} and ROUGE~\cite{Lin2004}.


%\textbf{Embedding-based:} These metrics assess the semantic similarity between the reference and generated text by comparing their embeddings, and show a higher correlation with human judgments.~\cite{Servan2016,Zhang2020,Tattar2017,Lo2019}.  In this research, we include BERTScore~\cite{Zhang2020}.\\

\paragraph{Learnable metrics}
These methods focus on directly learning human judgment through training. 
For this family, we include COMET~\cite{Rei2020}, which leverages a pre-trained multilingual model. Although COMET can function as a reference-less metric, in this work, we report results exclusively for its direct assessment variant.

%BLEURT~\cite{Sellam2020}, a metric based on the BERT model, specifically trained to mimic human judgment. We also include 


\paragraph{Language variety metric}
To assess if the text produced by our translation system is indeed in European Portuguese, we use a Portuguese language variant classifier~\cite{Sousa2025}, which distinguishes between Brazilian and European Portuguese.
After translation of the benchmark data, we employ the classifier to label all generated texts and to compute the percentage of documents that are labeled as European Portuguese. 
Since the classifier might contain intrinsic errors and bias, we also compute the percentage of documents labeled as European Portuguese in the reference translations. 
This step is crucial for handling cases where the translated text for the two variants might be identical. 
In such scenarios, the classifier might incorrectly classify the variety as the wrong one due to the lack of distinguishing features in the text. 
By comparing the results of our system with the reference translations, we can correct for this potential bias and obtain a more accurate assessment of how well our system produces European Portuguese. 
We refer to the ratio of these two percentages as the VID score, which serves as a measure of the system's effectiveness in generating European Portuguese text.

\subsection{Test Benchmarks}
As a low-resource language variant, the number of benchmarks that include European Portuguese is limited. 
In this study, we use two high-quality publicly available datasets that feature this variant:
\begin{itemize}
    \item \textbf{FRMT}: This dataset is specifically designed to contain regional variants of Portuguese and Chinese~\cite{Riley2023}, containing human translations of sentences from English Wikipedia articles that were manually translated to European and Brazilian Portuguese. 
    
    \item \textbf{NTrex}: The dataset consists of high-quality translations by speakers who are bilingual in English and in one of the 128 target languages, including 123 documents and 1,997 sentences for each language~\cite{Federmann2022}.
\end{itemize}

\subsection{Baselines}
We compare our models to three sets of baselines:

\paragraph{Closed Baselines}
We include industry-standard systems for Portuguese translation, including Google Translate and DeepL. Recently, Google Translate introduced a model specifically designed for European Portuguese, referred to as Google$_{pt}$, which we include in our evaluation alongside the original model that does not distinguish between Portuguese varieties, referred to as Google$_{br}$.

\paragraph{Open Baselines}
For open-source systems, we evaluate ArgosTranslate\footnote{\url{https://github.com/argosopentech/argos-translate}}, which uses OpenNMT~\cite{Klein2017} as its backend. Although Portuguese is listed as a supported language, the specific variety is not indicated. We also consider the Opus-MT project~\cite{Tiedemann2020}, another open-source system that provides a model for translating from English to Portuguese. However, like ArgosTranslate, this system does not differentiate between Portuguese varieties.

\paragraph{Zero-shot}
Additionally, we assess the zero-shot capabilities of language models without applying our task-specific fine-tuning to demonstrate the effectiveness of the fine-tuning process.

\subsection{Implementation Details}

All models were trained and evaluated on a server with six A-100 GPUs, each with 40GB of memory. The batch size and training duration varied depending on the memory requirements of each model. In our repository, we provide training and evaluation scripts compatible with the two libraries used to train the language models: \texttt{torchtune}\footnote{\url{https://github.com/pytorch/torchtune/}} and \texttt{transformers}\footnote{\url{https://huggingface.co/}}. While the \texttt{transformers} library was chosen for its practicality, we found it limiting when training larger models. In that scenario, \texttt{torchtune} presented as a reliable alternative with significantly better memory management.
Training runs we executed with early stopping -- using the test set the DSL-TL corpus as validation set -- with patience of 3,000 steps. As a result, the number of training steps varied across models. All LoRA variants were trained with an alpha of 128 and a rank of 256. Detailed training configurations can be found in our repository.

The parameter setup is as follows:

\begin{itemize}
    \item \textbf{Phi-3:} For both the LoRA and full fine-tuning of Phi-3 models, we used a batch size of 512, a learning rate of 2e-5, a weight decay of 0.1, and a warm-up of 1,000 steps.
    \item \textbf{Gemma-2:} For both variants, the learning rate was set to 2e-5 with a weight decay of 0.1. The full fine-tuned model was trained with a 1,000-step warm-up and a batch size of 512, while the LoRA variant had 500 warm-up steps and a batch size of 256.
    \item \textbf{LLaMA-3:} Both variants were trained with a batch size of 256 and a learning rate of 2e-5. The LoRA variant additionally includes a warm-up of 100 steps and a weight decay of 0.1 on the learning rate.
\end{itemize}

\subsection{Results}
\label{sec:results}

\begin{table*}[t]
\centering
\fontsize{11pt}{11pt}\selectfont
\begin{tabular}{lllllllllllll}
\toprule
\multicolumn{1}{c}{\textbf{task}} & \multicolumn{2}{c}{\textbf{Mir}} & \multicolumn{2}{c}{\textbf{Lai}} & \multicolumn{2}{c}{\textbf{Ziegen.}} & \multicolumn{2}{c}{\textbf{Cao}} & \multicolumn{2}{c}{\textbf{Alva-Man.}} & \multicolumn{1}{c}{\textbf{avg.}} & \textbf{\begin{tabular}[c]{@{}l@{}}avg.\\ rank\end{tabular}} \\
\multicolumn{1}{c}{\textbf{metrics}} & \multicolumn{1}{c}{\textbf{cor.}} & \multicolumn{1}{c}{\textbf{p-v.}} & \multicolumn{1}{c}{\textbf{cor.}} & \multicolumn{1}{c}{\textbf{p-v.}} & \multicolumn{1}{c}{\textbf{cor.}} & \multicolumn{1}{c}{\textbf{p-v.}} & \multicolumn{1}{c}{\textbf{cor.}} & \multicolumn{1}{c}{\textbf{p-v.}} & \multicolumn{1}{c}{\textbf{cor.}} & \multicolumn{1}{c}{\textbf{p-v.}} &  &  \\ \midrule
\textbf{S-Bleu} & 0.50 & 0.0 & 0.47 & 0.0 & 0.59 & 0.0 & 0.58 & 0.0 & 0.68 & 0.0 & 0.57 & 5.8 \\
\textbf{R-Bleu} & -- & -- & 0.27 & 0.0 & 0.30 & 0.0 & -- & -- & -- & -- & - &  \\
\textbf{S-Meteor} & 0.49 & 0.0 & 0.48 & 0.0 & 0.61 & 0.0 & 0.57 & 0.0 & 0.64 & 0.0 & 0.56 & 6.1 \\
\textbf{R-Meteor} & -- & -- & 0.34 & 0.0 & 0.26 & 0.0 & -- & -- & -- & -- & - &  \\
\textbf{S-Bertscore} & \textbf{0.53} & 0.0 & {\ul 0.80} & 0.0 & \textbf{0.70} & 0.0 & {\ul 0.66} & 0.0 & {\ul0.78} & 0.0 & \textbf{0.69} & \textbf{1.7} \\
\textbf{R-Bertscore} & -- & -- & 0.51 & 0.0 & 0.38 & 0.0 & -- & -- & -- & -- & - &  \\
\textbf{S-Bleurt} & {\ul 0.52} & 0.0 & {\ul 0.80} & 0.0 & 0.60 & 0.0 & \textbf{0.70} & 0.0 & \textbf{0.80} & 0.0 & {\ul 0.68} & {\ul 2.3} \\
\textbf{R-Bleurt} & -- & -- & 0.59 & 0.0 & -0.05 & 0.13 & -- & -- & -- & -- & - &  \\
\textbf{S-Cosine} & 0.51 & 0.0 & 0.69 & 0.0 & {\ul 0.62} & 0.0 & 0.61 & 0.0 & 0.65 & 0.0 & 0.62 & 4.4 \\
\textbf{R-Cosine} & -- & -- & 0.40 & 0.0 & 0.29 & 0.0 & -- & -- & -- & -- & - & \\ \midrule
\textbf{QuestEval} & 0.23 & 0.0 & 0.25 & 0.0 & 0.49 & 0.0 & 0.47 & 0.0 & 0.62 & 0.0 & 0.41 & 9.0 \\
\textbf{LLaMa3} & 0.36 & 0.0 & \textbf{0.84} & 0.0 & {\ul{0.62}} & 0.0 & 0.61 & 0.0 &  0.76 & 0.0 & 0.64 & 3.6 \\
\textbf{our (3b)} & 0.49 & 0.0 & 0.73 & 0.0 & 0.54 & 0.0 & 0.53 & 0.0 & 0.7 & 0.0 & 0.60 & 5.8 \\
\textbf{our (8b)} & 0.48 & 0.0 & 0.73 & 0.0 & 0.52 & 0.0 & 0.53 & 0.0 & 0.7 & 0.0 & 0.59 & 6.3 \\  \bottomrule
\end{tabular}
\caption{Pearson correlation on human evaluation on system output. `R-': reference-based. `S-': source-based.}
\label{tab:sys}
\end{table*}



\begin{table}%[]
\centering
\fontsize{11pt}{11pt}\selectfont
\begin{tabular}{llllll}
\toprule
\multicolumn{1}{c}{\textbf{task}} & \multicolumn{1}{c}{\textbf{Lai}} & \multicolumn{1}{c}{\textbf{Zei.}} & \multicolumn{1}{c}{\textbf{Scia.}} & \textbf{} & \textbf{} \\ 
\multicolumn{1}{c}{\textbf{metrics}} & \multicolumn{1}{c}{\textbf{cor.}} & \multicolumn{1}{c}{\textbf{cor.}} & \multicolumn{1}{c}{\textbf{cor.}} & \textbf{avg.} & \textbf{\begin{tabular}[c]{@{}l@{}}avg.\\ rank\end{tabular}} \\ \midrule
\textbf{S-Bleu} & 0.40 & 0.40 & 0.19* & 0.33 & 7.67 \\
\textbf{S-Meteor} & 0.41 & 0.42 & 0.16* & 0.33 & 7.33 \\
\textbf{S-BertS.} & {\ul0.58} & 0.47 & 0.31 & 0.45 & 3.67 \\
\textbf{S-Bleurt} & 0.45 & {\ul 0.54} & {\ul 0.37} & 0.45 & {\ul 3.33} \\
\textbf{S-Cosine} & 0.56 & 0.52 & 0.3 & {\ul 0.46} & {\ul 3.33} \\ \midrule
\textbf{QuestE.} & 0.27 & 0.35 & 0.06* & 0.23 & 9.00 \\
\textbf{LlaMA3} & \textbf{0.6} & \textbf{0.67} & \textbf{0.51} & \textbf{0.59} & \textbf{1.0} \\
\textbf{Our (3b)} & 0.51 & 0.49 & 0.23* & 0.39 & 4.83 \\
\textbf{Our (8b)} & 0.52 & 0.49 & 0.22* & 0.43 & 4.83 \\ \bottomrule
\end{tabular}
\caption{Pearson correlation on human ratings on reference output. *not significant; we cannot reject the null hypothesis of zero correlation}
\label{tab:ref}
\end{table}


\begin{table*}%[]
\centering
\fontsize{11pt}{11pt}\selectfont
\begin{tabular}{lllllllll}
\toprule
\textbf{task} & \multicolumn{1}{c}{\textbf{ALL}} & \multicolumn{1}{c}{\textbf{sentiment}} & \multicolumn{1}{c}{\textbf{detoxify}} & \multicolumn{1}{c}{\textbf{catchy}} & \multicolumn{1}{c}{\textbf{polite}} & \multicolumn{1}{c}{\textbf{persuasive}} & \multicolumn{1}{c}{\textbf{formal}} & \textbf{\begin{tabular}[c]{@{}l@{}}avg. \\ rank\end{tabular}} \\
\textbf{metrics} & \multicolumn{1}{c}{\textbf{cor.}} & \multicolumn{1}{c}{\textbf{cor.}} & \multicolumn{1}{c}{\textbf{cor.}} & \multicolumn{1}{c}{\textbf{cor.}} & \multicolumn{1}{c}{\textbf{cor.}} & \multicolumn{1}{c}{\textbf{cor.}} & \multicolumn{1}{c}{\textbf{cor.}} &  \\ \midrule
\textbf{S-Bleu} & -0.17 & -0.82 & -0.45 & -0.12* & -0.1* & -0.05 & -0.21 & 8.42 \\
\textbf{R-Bleu} & - & -0.5 & -0.45 &  &  &  &  &  \\
\textbf{S-Meteor} & -0.07* & -0.55 & -0.4 & -0.01* & 0.1* & -0.16 & -0.04* & 7.67 \\
\textbf{R-Meteor} & - & -0.17* & -0.39 & - & - & - & - & - \\
\textbf{S-BertScore} & 0.11 & -0.38 & -0.07* & -0.17* & 0.28 & 0.12 & 0.25 & 6.0 \\
\textbf{R-BertScore} & - & -0.02* & -0.21* & - & - & - & - & - \\
\textbf{S-Bleurt} & 0.29 & 0.05* & 0.45 & 0.06* & 0.29 & 0.23 & 0.46 & 4.2 \\
\textbf{R-Bleurt} & - &  0.21 & 0.38 & - & - & - & - & - \\
\textbf{S-Cosine} & 0.01* & -0.5 & -0.13* & -0.19* & 0.05* & -0.05* & 0.15* & 7.42 \\
\textbf{R-Cosine} & - & -0.11* & -0.16* & - & - & - & - & - \\ \midrule
\textbf{QuestEval} & 0.21 & {\ul{0.29}} & 0.23 & 0.37 & 0.19* & 0.35 & 0.14* & 4.67 \\
\textbf{LlaMA3} & \textbf{0.82} & \textbf{0.80} & \textbf{0.72} & \textbf{0.84} & \textbf{0.84} & \textbf{0.90} & \textbf{0.88} & \textbf{1.00} \\
\textbf{Our (3b)} & 0.47 & -0.11* & 0.37 & 0.61 & 0.53 & 0.54 & 0.66 & 3.5 \\
\textbf{Our (8b)} & {\ul{0.57}} & 0.09* & {\ul 0.49} & {\ul 0.72} & {\ul 0.64} & {\ul 0.62} & {\ul 0.67} & {\ul 2.17} \\ \bottomrule
\end{tabular}
\caption{Pearson correlation on human ratings on our constructed test set. 'R-': reference-based. 'S-': source-based. *not significant; we cannot reject the null hypothesis of zero correlation}
\label{tab:con}
\end{table*}

\section{Results}
We benchmark the different metrics on the different datasets using correlation to human judgement. For content preservation, we show results split on data with system output, reference output and our constructed test set: we show that the data source for evaluation leads to different conclusions on the metrics. In addition, we examine whether the metrics can rank style transfer systems similar to humans. On style strength, we likewise show correlations between human judgment and zero-shot evaluation approaches. When applicable, we summarize results by reporting the average correlation. And the average ranking of the metric per dataset (by ranking which metric obtains the highest correlation to human judgement per dataset). 

\subsection{Content preservation}
\paragraph{How do data sources affect the conclusion on best metric?}
The conclusions about the metrics' performance change radically depending on whether we use system output data, reference output, or our constructed test set. Ideally, a good metric correlates highly with humans on any data source. Ideally, for meta-evaluation, a metric should correlate consistently across all data sources, but the following shows that the correlations indicate different things, and the conclusion on the best metric should be drawn carefully.

Looking at the metrics correlations with humans on the data source with system output (Table~\ref{tab:sys}), we see a relatively high correlation for many of the metrics on many tasks. The overall best metrics are S-BertScore and S-BLEURT (avg+avg rank). We see no notable difference in our method of using the 3B or 8B model as the backbone.

Examining the average correlations based on data with reference output (Table~\ref{tab:ref}), now the zero-shoot prompting with LlaMA3 70B is the best-performing approach ($0.59$ avg). Tied for second place are source-based cosine embedding ($0.46$ avg), BLEURT ($0.45$ avg) and BertScore ($0.45$ avg). Our method follows on a 5. place: here, the 8b version (($0.43$ avg)) shows a bit stronger results than 3b ($0.39$ avg). The fact that the conclusions change, whether looking at reference or system output, confirms the observations made by \citet{scialom-etal-2021-questeval} on simplicity transfer.   

Now consider the results on our test set (Table~\ref{tab:con}): Several metrics show low or no correlation; we even see a significantly negative correlation for some metrics on ALL (BLEU) and for specific subparts of our test set for BLEU, Meteor, BertScore, Cosine. On the other end, LlaMA3 70B is again performing best, showing strong results ($0.82$ in ALL). The runner-up is now our 8B method, with a gap to the 3B version ($0.57$ vs $0.47$ in ALL). Note our method still shows zero correlation for the sentiment task. After, ranks BLEURT ($0.29$), QuestEval ($0.21$), BertScore ($0.11$), Cosine ($0.01$).  

On our test set, we find that some metrics that correlate relatively well on the other datasets, now exhibit low correlation. Hence, with our test set, we can now support the logical reasoning with data evidence: Evaluation of content preservation for style transfer needs to take the style shift into account. This conclusion could not be drawn using the existing data sources: We hypothesise that for the data with system-based output, successful output happens to be very similar to the source sentence and vice versa, and reference-based output might not contain server mistakes as they are gold references. Thus, none of the existing data sources tests the limits of the metrics.  


\paragraph{How do reference-based metrics compare to source-based ones?} Reference-based metrics show a lower correlation than the source-based counterpart for all metrics on both datasets with ratings on references (Table~\ref{tab:sys}). As discussed previously, reference-based metrics for style transfer have the drawback that many different good solutions on a rewrite might exist and not only one similar to a reference.


\paragraph{How well can the metrics rank the performance of style transfer methods?}
We compare the metrics' ability to judge the best style transfer methods w.r.t. the human annotations: Several of the data sources contain samples from different style transfer systems. In order to use metrics to assess the quality of the style transfer system, metrics should correctly find the best-performing system. Hence, we evaluate whether the metrics for content preservation provide the same system ranking as human evaluators. We take the mean of the score for every output on each system and the mean of the human annotations; we compare the systems using the Kendall's Tau correlation. 

We find only the evaluation using the dataset Mir, Lai, and Ziegen to result in significant correlations, probably because of sparsity in a number of system tests (App.~\ref{app:dataset}). Our method (8b) is the only metric providing a perfect ranking of the style transfer system on the Lai data, and Llama3 70B the only one on the Ziegen data. Results in App.~\ref{app:results}. 


\subsection{Style strength results}
%Evaluating style strengths is a challenging task. 
Llama3 70B shows better overall results than our method. However, our method scores higher than Llama3 70B on 2 out of 6 datasets, but it also exhibits zero correlation on one task (Table~\ref{tab:styleresults}).%More work i s needed on evaluating style strengths. 
 
\begin{table}%[]
\fontsize{11pt}{11pt}\selectfont
\begin{tabular}{lccc}
\toprule
\multicolumn{1}{c}{\textbf{}} & \textbf{LlaMA3} & \textbf{Our (3b)} & \textbf{Our (8b)} \\ \midrule
\textbf{Mir} & 0.46 & 0.54 & \textbf{0.57} \\
\textbf{Lai} & \textbf{0.57} & 0.18 & 0.19 \\
\textbf{Ziegen.} & 0.25 & 0.27 & \textbf{0.32} \\
\textbf{Alva-M.} & \textbf{0.59} & 0.03* & 0.02* \\
\textbf{Scialom} & \textbf{0.62} & 0.45 & 0.44 \\
\textbf{\begin{tabular}[c]{@{}l@{}}Our Test\end{tabular}} & \textbf{0.63} & 0.46 & 0.48 \\ \bottomrule
\end{tabular}
\caption{Style strength: Pearson correlation to human ratings. *not significant; we cannot reject the null hypothesis of zero corelation}
\label{tab:styleresults}
\end{table}

\subsection{Ablation}
We conduct several runs of the methods using LLMs with variations in instructions/prompts (App.~\ref{app:method}). We observe that the lower the correlation on a task, the higher the variation between the different runs. For our method, we only observe low variance between the runs.
None of the variations leads to different conclusions of the meta-evaluation. Results in App.~\ref{app:results}.
%\begin{table*}[h]
\setlength{\tabcolsep}{5pt} % Default value: 6pt
\renewcommand{\arraystretch}{1.2} % Default value: 1
\resizebox{\textwidth}{!}{%
\begin{tabular}{lll}
\toprule
   \textbf{English} to \textbf{Portuguese (Reference)} & \textbf{Translations} & \textbf{COMET Diff}  \\
\midrule
 \textbf{En:} Her casting was announced in November 2011.  & \textbf{Deepl:} O seu elenco foi anunciado em novembro de 2011.  & -0.1295 \\
 \textbf{Pt:} As suas provas de casting foram anunciadas em novembro de 2011. &   \textbf{Llama-3:} O casting dela foi anunciado em novembro de 2011. & \\ 
\midrule
 \textbf{En:} Usually double-breasted suits have one hole on each lapel (with a flower just on the left),  & \textbf{Deepl:} Normalmente, os fatos trespassados têm um buraco em cada lapela (com uma flor apenas& -0.1313 \\
     while single-breasted suits have just one on the left.  &  à esquerda), enquanto os fatos trespassados têm apenas um à esquerda.& \\ 
 \textbf{Pt:} Normalmente, os fatos em jaquetão têm uma casa em cada lapela (com uma flor &   \textbf{Llama-3:} Geralmente os fatos de colarinho duplo têm um buraco em cada lapela (com flor   & \\ 
 no lado esquerdo), enquanto os fatos direitos têm apenas uma à esquerda. & só na esquerda), enquanto os fatos de colarinho simples têm apenas um na esquerda.& \\ 
\midrule
   \textbf{En:} In the developed world, cups are often distributed for promotional purposes.  & \textbf{Deepl:} No mundo desenvolvido, as chávenas são frequentemente distribuídas para fins promocionais.  & -0.1347 \\
   \textbf{Pt:} Nos países desenvolvidos, as taças costumam ser distribuídas por motivos promocionais. &\textbf{Llama-3:} No mundo desenvolvido, muitas vezes distribuem-se taças para efeitos promocionais. & \\
   \midrule
    \textbf{En:} Random people organized a charity and paid for the grave to remain.  & \textbf{Deepl:} Pessoas aleatórias organizaram uma ação de caridade e pagaram para que a  & 0.2650 \\
    \textbf{Pt:} Um grupo de pessoas aleatórias organizou um peditório e pagou pela sepultura . & sepultura permanecesse.  & \\
    para que esta se mantivesse& \textbf{Llama-3:} Pessoas aleatórias organizaram uma caridade e pagaram para a campa ficar. & \\
      \midrule
     \textbf{En:} It was taken by the Moorish Almoravids in 1111.  & \textbf{Deepl:} Foi tomada pelos mouros almorávidas em 1111. & 0.2705 \\
     \textbf{Pt:}  Em 1111, foi ocupada por almorávidas mouros. & \textbf{Llama-3:} Foi tomada pelos mouriscos Almorávidas em 1111. \\
      \midrule
       \textbf{En:}Vincent it stretches a further 50 km to the north.  & \textbf{Deepl:} Vincent estende-se por mais 50 km para norte. & 0.2710 \\
       \textbf{Pt:}  Vincente, estende-se 50 km a norte. &\textbf{Llama-3:} Vincent estica mais 50 km para norte.  & \\
\bottomrule
\end{tabular}%
}

\caption{Case study of sample translations of our best performing model, Llama-3, and the strongest baseline DeepL.} 
\label{tab:parameters}
\end{table*}

The result of our evaluation is shown in Table~\ref{tab:results}, where the best overall values are marked with an underline, and the most effective open-source systems are marked in bold. 
The general trend on both datasets is quite similar and high-performing models maintain a stable performance across both benchmarks. 
Yet, values for NTrex are slightly below FRMT for all systems, indicating a harder benchmark.
In the following, we describe our main findings and highlight avenues for future research.

\paragraph{LoRA models:} These variants effectively learn the vocabulary of European Portuguese, but their overall translation quality remains subpar. 
This discrepancy is evident as they score highly on the VID metric, nevertheless, the BLEU, ROUGE-L, and COMET metrics suggest that they struggle with generating high-quality text. 
Upon closer inspection of the generated translations, we found that both the Phi-3 and LLaMA-3 models, when trained with LoRA, tend to enter a repetition loop, where the same token is generated repeatedly until the process is interrupted. 
This suggests that for a medium size language model translation is a complex task, and simply adding adapter parameters is insufficient to fully capture its nuances. 
This issue, combined with the fact that the early stopping criteria were reached quickly during training (training loss had plateaued while the evaluation was increasing), suggests that the models may require increased capacity (by adjusting the alpha and rank parameters) to better learn from the training data.

\paragraph{Full fine-tuning (FFT):}
The fully fine-tuned models sacrifice their mastery of European Portuguese nuances in favor of high-quality, coherent translations. 
These models yield more moderate scores for the VID metric while achieving significantly higher text quality metrics. 
In particular, the fine-tuned LLaMA-3 model beats all open source software on all metrics and produces results comparable to Google$_{br}$ in terms of BLEU and ROUGE-L on both benchmarks, only falling short on the COMET metric. 

It is important to note that the COMET metric, as a learnable metric, is subject to the same biases that affect any trainable neural model. Since the training data does not differentiate between European and Brazilian Portuguese and most open-source resources are skewed toward Brazilian Portuguese, this bias may have influenced the scores produced by this metric.
The slight difference between Google$_{br}$ and fine-tuned LLaMA-3 models suggests a potential bias in the COMET score toward Brazilian Portuguese.

The overall performance of the fully fine-tuned models has a direct correlation with model size, where our largest model, LLaMA-3, with 8 billion parameters outperforms the smaller models of Phi-3 (3.8 billion parameters) and Gemma-2 (2 billion parameters). 
This behavior is typical for large language models and indicates that increasing their size can lead to better results. This indicates that by applying this methodology to even larger models, it may be possible to achieve results that are on par with industry-standard systems. % However, our study focuses on smaller language models and data generation through back translation, rather than merely scaling up the model size.


Since our focus is on translation specific to European Portuguese, it is important to examine the VID scores across both benchmarks.
Our best model, LLaMA-3, once again beats all open source baselines and achieves scores significantly higher than Google$_{br}$ and is comparable to Google$_{pt}$ and DeepL. This suggests that the proposed methodology is effective in achieving the targeted goal of producing text specific to a language variety. 
Even our smaller-size models, perform comparable to open-source baselines on translation metrics, dramatically improving on the VID score. 

It is true that in terms of text quality metrics, the LLaMA-3 model still lags behind the European Portuguese-specific industry models, namely Google$_{pt}$ and DeepL. However, it is important to emphasize that our goal was not to beat the specialized model from industry, but to propose a computationally efficient, adaptable, and resource-efficient method for adapting small language models to translate specific language varieties. Surpassing the current open-source software and achieving a score close to industry-level models, which benefit from dedicated teams of experts and annotators for each language, is a significant accomplishment.
