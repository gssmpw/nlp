\section{Conclusion \& Future Work}
\label{sec:conculsion}
In this paper, we present a methodology for creating a parallel corpus and training a translation model tailored to a low-resource language variety. Specifically, we developed and open-sourced the largest European Portuguese-English parallel corpus, along with European Portuguese-specific translation models. Our extensive evaluation demonstrates the effectiveness of our approach and the fidelity of the generated translations to the desired language variety. Thanks to the proposed methodology we managed to achieve performance on par with industry-level translation systems with minimal resources and limited computation.

In future work, it would be interesting to investigate the impact of different generation configurations on the translations produced by our model. In this study, we used greedy decoding, but other generation techniques, such as beam search, could yield better results. Another promising direction is to explore prompt optimization both before and after model training. This approach has been shown to improve outcomes in other studies~\cite{Soylu2024} and might enhance our system's performance as well. Finally, we plan to conduct a human evaluation with linguists to identify areas where our model falls short compared to other systems.
