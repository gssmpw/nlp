\section{Related Works}
\label{sec:relatedworks}
% \{NOTES TO HELP WRITING THIS SECTION:
% - Admittance Control for Human-Robot Interaction Using an Industrial Robot Equipped with a F/T Sensor____
% - Soft skin____
% - Everything about pHRI, taking account of the injury risk____ and collision detection____
% First, safety issued should be addressed by identifying performance oriented
% solutions. In other words, the approach should change from considering safety as a requirement that limits performance, but rather performances should be optimized subject to the constraint of safety.____
% \}

Several pHRI frameworks have been introduced in recent years, with early research primarily focused on collision detection and reaction mechanisms to prevent injuries caused by robots in industrial settings____. Over time, more advanced approaches have emerged, integrating risk metrics into robotic control systems to account for injury risks during human-robot interactions____. Additionally, innovations such as precise but heavy compliance achieved through admittance control on industrial robots____ and the use of soft skins with embedded force sensors____ have made robots more adaptable to close-contact interactions. However, despite these advancements, the performance of these systems has often fallen short of industry requirements. This highlights the need to shift the approach toward performance-oriented solutions. Instead of viewing safety as a limiting factor, the framework presented in this paper positions safety as a constraint within performance optimization, as suggested in____, ensuring both high safety standards and enhanced performance in industrial pHRI.