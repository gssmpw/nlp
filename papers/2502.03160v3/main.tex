
% \documentclass[manuscript,screen,review]{acmart}
\documentclass[manuscript]{acmart}

\AtBeginDocument{%
  \providecommand\BibTeX{{%
    Bib\TeX}}}

\setcopyright{acmlicensed}
\copyrightyear{2025}
\acmYear{2025}
\acmDOI{XXXXXXX.XXXXXXX}

%% These commands are for a PROCEEDINGS abstract or paper.
% \acmConference[ISSTA '25]{Make sure to enter the correct
%   conference title from your rights confirmation emai}{June Wed 25 - Sat 28,
%   2025}{Trondheim, Norway}

%%
%%  Uncomment \acmBooktitle if the title of the proceedings is different
%%  from ``Proceedings of ...''!
%%
% \acmBooktitle{Woodstock '18: ACM Symposium on Neural Gaze Detection,
%  June 03--05, 2018, Woodstock, NY}
% \acmISBN{978-1-4503-XXXX-X/18/06}

% \def\BibTeX{{\rm B\kern-.05em{\sc i\kern-.025em b}\kern-.08em
%     T\kern-.1667em\lower.7ex\hbox{E}\kern-.125emX}}
    
% \settopmatter{printacmref=false} 
% \renewcommand\footnotetextcopyrightpermission[1]{} 
\usepackage{graphics}
\usepackage{tabularx}
\usepackage{xspace}
\usepackage{xcolor}
\usepackage{multirow}
\usepackage{booktabs}
% \usepackage{cite}
\let\Bbbk\relax
\usepackage{amsmath,amssymb,amsfonts}
\usepackage{textcomp}
\usepackage{url}
\usepackage[colorinlistoftodos,prependcaption,textsize=tiny]{todonotes}
\usepackage{enumitem}
\usepackage{graphicx}
\usepackage{adjustbox} 
\usepackage{tabularx}  % 引入tabularx宏包
\usepackage{tcolorbox}  % 引入tcolorbox宏包
\usepackage{lipsum}     % 引入lipsum宏包,用于生成示例文本
\usepackage{hyperref}  % 在导言区导入 hyperref 宏包
\usepackage{subfigure}
\usepackage{booktabs}
\usepackage{balance}
\usepackage{bm}

\usepackage{algorithm}
\usepackage{algorithmic}
\usepackage{multirow}
\usepackage{soul}

\newcommand{\etal}{{\em et al.}\xspace}
\newcommand{\ie}{{\em i.e.},\xspace}
\newcommand{\eg}{{\em e.g.},\xspace}
\newcommand{\modify}[1]{\textcolor{red}{#1}}
\newcommand{\xl}[1]{\textcolor{blue}{XL: #1}}
\newcommand{\methodname}{{AL-Bench}\xspace}
\newcommand{\zhu}[1]{\textcolor{blue}{#1}}



\begin{document}
%%
%% The "author" command and its associated commands are used to define
%% the authors and their affiliations.
%% Of note is the shared affiliation of the first two authors, and the
%% "authornote" and "authornotemark" commands
%% used to denote shared contribution to the research.
\author{Boyin Tan}
\email{BoyinTan@link.cuhk.edu.cn}
\affiliation{
  \institution{The Chinese University of Hong Kong, Shenzhen}
  \country{China}
}


\author{Junjielong Xu}
\email{junjielongxu@link.cuhk.edu.cn}
\authornote{Junjielong Xu and Pinjia He are corresponding authors.}
\affiliation{
  \institution{The Chinese University of Hong Kong, Shenzhen}
  \country{China}
}

\author{Zhouruixing Zhu}
\email{zhouruixingzhu@link.cuhk.edu.cn}
\affiliation{
  \institution{The Chinese University of Hong Kong, Shenzhen}
  \country{China}
}

\author{Pinjia He}
\email{hepinjia@cuhk.edu.cn}
\authornotemark[1]
\affiliation{
  \institution{The Chinese University of Hong Kong, Shenzhen}
  \country{China}
}



\begin{abstract}
Logging, the practice of inserting log statements into source code, is critical for improving software reliability. 
Recently, language model-based techniques have been developed to automate log statement generation based on input code. 
% These tools show promising results in their own evaluation. 
% However, current evaluation practices in log statement generation face significant challenges. The lack of a unified, large-scale dataset forces studies to rely on ad-hoc data, hindering consistency and reproducibility.
While these tools show promising results in prior studies, the fairness of their {results} comparisons is not guaranteed due to the use of ad hoc datasets.
% {Additionally, existing assessments solely based on metrics like code similarity fail to reflect real-world effectiveness.}
In addition, {existing evaluation approaches exclusively dependent on code similarity metrics fail to capture the impact of code diff on runtime logging behavior, as minor code modifications can induce program uncompilable and substantial discrepancies in log output semantics.}
% These limitations underscore the need for a comprehensive public benchmark to standardize evaluation.
To enhance the consistency and reproducibility of logging evaluation, we introduce \methodname, a comprehensive benchmark designed specifically for automatic logging tools. \methodname includes a large-scale, high-quality, diverse dataset collected from 10 widely recognized projects with varying logging requirements. Moreover, it introduces a novel dynamic evaluation methodology to provide a run-time perspective of logging quality in addition to the traditional static evaluation at source code level.
Specifically, \methodname not only evaluates the similarity between the oracle and predicted log statements in source code, but also evaluates the difference between the log files printed by both log statements {during runtime}.
% Different from the existing evaluations that focus only on components of log statements like code similarity, \methodname assesses both the compilability of the code with inserted log statements and the effectiveness of the logs generated by them during runtime, which we believe can better reflect the effectiveness of logging techniques in practice.
%%% 在静态评估上有啥点数的区别吗。是不是也得提一句?
%%% “所有现有方法在AL-Bench上的PA,LA,MA这三个传统静态评估指标上平均下降了X\%, Y\%, Z\%.
\methodname reveals significant limitations in existing static evaluation, as all logging tools show average accuracy drops of 37.49\%, 23.43\%, and 15.80\% in predicting log position, level, and message compared to their reported results.
Furthermore, with dynamic evaluation, \methodname reveals that 20.1\%-83.6\% of these generated log statements are unable to compile. 
% In addition, even the best-performing tool only achieves 0.213 cosine similarity between the runtime logs produced by the generated log statements and the ground-truth log statements.
Moreover, the best-performing tool achieves only 21.32\% cosine similarity between the log files of the oracle and generated log statements.
% The results reveal substantial opportunities to further enhance the development of automatic logging tools.
These results underscore substantial opportunities to advance the development of automatic logging tools.
We believe this work establishes a foundational step in furthering this research direction.

\end{abstract}

% attains a 57.54\% accuracy in logging position, 62.72\% in level accuracy,
% 17.66\% in dynamic variable accuracy
% a BLEU score of 20.20 for text accuracy

% are limited by the use of low-quality data and incomplete assessments.
% More specifically,  (1) low-quality evaluation data compromise effectiveness in assessing the performance, (2)  the evaluation does not verify if the generated code is compilable, and (3) there is no assessment of runtime logs generated by predicted log statements. 
\title{\textit{\methodname}: A Benchmark for Automatic Logging}

%%
%% The code below is generated by the tool at http://dl.acm.org/ccs.cfm.
%% Please copy and paste the code instead of the example below.
%%
\begin{CCSXML}
<ccs2012>
   <concept>
       <concept_id>10011007</concept_id>
       <concept_desc>Software and its engineering</concept_desc>
       <concept_significance>500</concept_significance>
       </concept>
 </ccs2012>
\end{CCSXML}

\ccsdesc[500]{Software and its engineering}

%%
%% Keywords. The author(s) should pick words that accurately describe
%% the work being presented. Separate the keywords with commas.
\keywords{Software Maintenance, Logging, Benchmark}

\maketitle
\tcbset{
    colback=gray!10,    % 背景颜色
    colframe=black,     % 边框颜色
    boxrule=0.5mm,      % 边框宽度
    arc=3mm,            % 边角的圆弧大小
    auto outer arc,
    width=\textwidth,   % 宽度为页面宽度
    left=5pt,          % 增加左侧边距
    right=5pt,          % 右侧保持较小间距
    boxsep=5pt,         % 内边距
    before=\vskip10pt,  % 上方间距
    after=\vskip10pt,   % 下方间距
    center    
}
% \onecolumn


\section*{Cover Letter}

\vspace{10mm}

\setcounter{table}{0}
\setcounter{figure}{0}

\renewcommand{\thetable}{C\arabic{table}}  
\renewcommand{\thefigure}{C\arabic{figure}}
\renewcommand{\thesubsection}{C\arabic{subsection}}
		
Dear ACs and reviewers for AAAI-22,

We would like to submit our paper entitled ``Leave Before You Leave: Data-Free Restoration of Pruned Neural Networks Without Fine-Tuning,'' for your consideration in AAAI-22 conference. In this version, we have made the following revisions to best appreciate all the previous comments raised by reviewers of NeurIPS-21.

\begin{enumerate}
    \item We have confirmed by new experimental results that LBYL is practically useful even when fine-tuning with data is available. Please refer to Table \ref{tab:finetune_scratch} and Figure \ref{fig:Fine tuning and From Scratch}.
    \item We have shown by supplementary experimental results on MobileNet-V2 that LBYL is also effective in tiny architectures. Please refer to Table \ref{tab:mobilenet}.
    \item We further improved our presentation quality by revising figures more descriptive, presenting a notation table in Appendix, and correcting all the typos and ambiguous expressions. Please particularly refer to Figure \ref{fig:framework} and Table \ref{tab:notations}.
    \item All the additional experimental results have been shown in either manuscript or Appendix.
\end{enumerate}


The followings are the detailed responses to each of the reviewers' comments.

\subsection*{Reviewer fQob}

\smalltitle{Comment 1}
\textit{By LBYL, the pruned network can recover the outputs of the original model with minimal approximation. But how to explain why LBYL can lead to even better performance than the original model.}

\smalltitle{Response 1}
As our goal is to approximate the layer-wise output after pruning, it would be impossible for a restored model to beat the original model. This question is probably due to the result of ResNet-50 on CIFAR-100 in Table A1 of the previous version of Appendix, where there was a typo in the table such that the accuracy of the original ResNet-50 should be 78.82, not 73.27. We have corrected this typo in this version.

\smalltitle{Comment 2}
\textit{The authors do not provide how to tackle the residual issue. How to make sure that the output is still lossless when residual connections are presented.}

\smalltitle{Response 2}
For a CNN with skip connections, we have used the common pruning scheme that prunes only first two conv layers within each residual block so that the dimension of each block's output remains the same after pruning. Furthermore, since our restoration method is only applied to each pruned layer, the output of each block would properly be restored regardless of residual connections.


\smalltitle{Comment 3}
\textit{It seems that the LBYL works well in models with much redundancy. How about the performance of trunk pruning in efficient models such as mobilenet.}

\smalltitle{Response 3}
Computationally efficient architectures such as MobileNet are already highly optimized in terms of model size, and hence pruning these compact models would lead to severe drop in accuracy. Furthermore, these tiny architectures commonly have depth-wise convolution layers, which makes us difficult to apply our restoration method in a straightforward way. Nevertheless, we have conducted a series of experiments using MobileNet-v2 using the scheme that prunes only the first layer of each block. This scheme can be seen as a naive adaptation of our method for MobileNet-v2. Table \ref{tab:mobilenet} summarizes the corresponding experimental results, where we can clearly observe that LBYL outperforms NM in most cases and is quite effective to recover the performance of pruned networks.


\smalltitle{Comment 4}
\textit{It would be nice to provide an algorithm table for a better understanding of the method.}

\smalltitle{Response 4}
Due to page limits, we have added a notation table in Appendix. Please refer to Table \ref{tab:notations}, which would further help to understand our methodology.



%%% Limitations And Societal Impact:
%%% The authors do not address the limitations and potential negative societal impact of their work. The authors can discuss how the proposed pruning method affects the development of edge AI.

% \smalltitle{Limitations and Societal Impact}
% The potential negative societal impact of our work can be to make less accurate yet lightweight models highly prevalent in the era of edge AI. This is because our method is extremely simple to implement and easy to apply to any heavyweight models even without using any data and fine-tuning.


%--> it is difficult to prune the computation efficient models such as mobile net without fine-tuning process becacues they are developed with considering the computation. However we experment the mobilenet-V2 using the scheme that prune first layer for each block and calculate the scale, compensating in the last layer. Actually, this compensation method is not fully match with our proposed method due to depthwise convolutional layer in block. but we can loose approximate the output.






\subsection*{Reviewer VxV3}

% Thank you for the very detailed and structured reviews.

%%% I don't think these implications are true. implications: The nonexistence of data implies the following two facts. First, we have to use a pruning criterion exploiting only the values of filters themselves such as L1-norm, Secondly, since we cannot make appropriate changes in the remaining filters by fine-tuning.....
%There are techniques like conserving synaptic flow... just because there's no training data doesn't mean that synthetic inputs can't be fed to the network.

\smalltitle{Comment 1}
\textit{I'd like to have seen comparisons to other data-free approaches, even if they require fine-tuning, since there can be time for fine-tuning, even if there's no data available.}


\smalltitle{Response 1}
This work is intended not to use any data regardless of whether it is real or synthetic. Nevertherless, we provide the comparison between NM and LBYL on the accuracy after fine-tuning with real data. As shown in Table \ref{tab:finetune_scratch} and Figure \ref{fig:Fine tuning and From Scratch}, we observe that LBYL converges much faster than NM in the middle of fine-tuning, and thereby achieves the best accuracy after 20 epochs of fine-tuning.





% In the CRC version, we can further clarify this issue. However, we would still like to point out the following two facts in response to your detailed comments.

% 1. \textit{conserving synaptic flow} [27] is a weight pruning (a.k.a. unstructured pruning) method, and hence it cannot be directly compared to any structured pruning methods like filter pruning that this paper focuses on. 

% 2. To the best of our knowledge, there is no existing work on fine-tuning with synthetic data a compressed model whose filters are pruned. In general, however, we agree that it would be valuable to study whether it is effective to use synthetic data generated by approaches like \textit{Deep Inversion} [2] in order to fine-tune a pruned network even though it will require heavy computation as mentioned by the reviewer.


% %%% I'd like to have seen comparisons to other data-free approaches, even if they require fine-tuning, since there can be time for fine tuning, even if there's no data available.

% Unfortunately, we have time limit for rebuttal, and therefore we cannot implement such a fine-tuning method via synthetic data for the comparison. Alternatively, we provide the comparison between NM and LBYL on the accuracy after fine-tuning with real data. As shown table below, we observe that LBYL converges much faster than NM in the middle of fine-tuning, and thereby achieves the best accuracy after 20 epochs of fine-tuning.

% \begin{table}[h]
% \centering
% \scriptsize
% \begin{tabular}{c||c|c|c} \Xhline{2\arrayrulewidth}
% Pruning ratio& LBYL& NM& Prune \\ \Xhline{2\arrayrulewidth}
% 10 \% & \textbf{78.14} $\rightarrow$ \textbf{78.3}  & 77.28 $\rightarrow$ 77.57 & 75.14 $\rightarrow$ 77.67\\\hline
% 20 \% & \textbf{76.15} $\rightarrow$ \textbf{77.22} &  72.73 $\rightarrow$ 75.62 &  63.39 $\rightarrow$ 75.4\\\hline
% 30 \% & \textbf{73.29} $\rightarrow$ \textbf{75.99} & 64.47 $\rightarrow$ 73.03 &  39.96 $\rightarrow$ 73.18\\\hline
% 40 \% & \textbf{65.21} $\rightarrow$ \textbf{74.13} & 46.4 $\rightarrow$ 69.71 &  15.32 $\rightarrow$ 70.12\\\hline
% 50 \% & \textbf{52.61} $\rightarrow$ \textbf{71.15} & 25.98 $\rightarrow$ 65.03 &  5.22 $\rightarrow$ 65.72\\\Xhline{2\arrayrulewidth}
% \end{tabular}%
% \vspace{2mm}
% \caption{Fine-tuned results of ResNet-50 on CIFAR100 using L2-norm criterion, where all the pruned models are fine-tuned for 20 epochs.}
% \label{cov:tab:fine-tuning}
% \end{table}


%%% While the writing is mostly understandable with few typos, I still may not understand how the technique works. If layer i is pruned, and the remaining filters share its duties, is the output of layer i the original size, or half-sized due to half the neurons being pruned? Relatedly, does the next layer downstream see a reduced-size input, and correspondingly have its own input dimension pruned?

\smalltitle{Comment 2}
\textit{While the writing is mostly understandable with few typos, I still may not understand how the technique works. If layer i is pruned, and the remaining filters share its duties, is the output of layer i the original size, or half-sized due to half the neurons being pruned? Relatedly, does the next layer downstream see a reduced-size input, and correspondingly have its own input dimension pruned?}

\smalltitle{Response 2}
The reviewer's understanding is indeed correct. If half of filters in layer $i$ are pruned, the output of layer $i$ is half-sized. This leads to the reduction of the number of input channels in layer $(i+1)$, what the reviewer calls the input dimension of the next layer. More specifically, if $j$-th filter of $\ell$-th layer is pruned, its corresponding channel of filters in $(\ell+1)$-th layer is consequently removed. However, it does not imply that the output size of $(\ell+1)$-th layer is also reduced, but its value is \textit{damaged} by the loss of $j$-th input channel.

In this version, we have added the following description to the caption of Figure \ref{fig:framework}:
$s, s',$ and $s^*$ are the coefficients that quantify how much each preserved filter should carry the information of the pruned filter.

We have double-checked the paper and revised all the typos therein as well.


\smalltitle{Comment 3}
\textit{You can always show top-1 or top-5 accuracy on a data set, but in the absence of such a data set, how will the deployer know when to stop pruning? There's a big accuracy gap between 10\% and 20\% pruning on the ImageNet results.}

\smalltitle{Response 3}
First of all, if there is no available data, we have no choice but to compress the target network in a data-free manner anyway, which is equally applied to both LBYL and NM. A possible way to determine when to stop pruning is to keep track of how the values of our two data-free error terms (i.e., RE and BE) are changing. As shown in Figure \ref{fig:error_components}, we can roughly guess how good the final performance would be by observing both RE and BE, which strongly contribute the reconstruction error. 

Also, our method can still be useful when we are given a target model size yet the final accuracy is not the most critical factor (e.g., recommendation app running in mobile devices).

\smalltitle{Comment 4}
\textit{This technique also seems like it would be useful as a pre-processing step before performing fine-tuning, potentially encouraging accuracy recovery or reducing the amount of fine-tuning required.}

\smalltitle{Response 4}
We agree with the reviewer that our method can also be seen as a preprocessing step before fine-tuning. As shown in Table \ref{tab:finetune_scratch} and Figure \ref{fig:Fine tuning and From Scratch}, we found that LBYL is indeed effective for fast convergence during the fine-tuning phase. 





% As you said, the accuracy of restored model decreases dramatically as pruning ratio increases. If we have a table of restored model's performances according to pruning ratio in experimental setting and the situation where our method is applied is similar with the experimental setting, the performance can be guessed through this table.

% But, when this is not the case, we suggest a naive solution to know when to stop pruning. A naive solution is to employ random noisy images(e.g., gaussian random samples) and then measure weighted average reconstruction error(WARE) between original model and pruned model. Increasing the pruning ratio and Checking the WARE, we may be able to avoid drastic loss. but this is just a rough method.

% On the other hand, our proposed methods can be used practically when users need to compress the original model and performance loss is relatively not critical.(e.g., recommendation system) For example, Choosing whether to update or not the smartphone is relatively robust to poor performance of model.


%--> As you said, it is true that accuracy of restored model in pruning ratio 30$\%$ is dropped with large margin compared to original model. Therefore we suggest a naive solution to stop whether prune more or not without using any data. A naive solution is to employ random images and then measure weighted average reconstruction error(WARE) between pre-trained model and pruned model. Based on WARE, we can choose whether to prune more or not.

%our proposed methods can practically be used when users need to compress the pre-trained model without considering performance such as recommendation system. for example, choosing whether to update or not does smartphone not affects smartphone's performance much because our method can correctly increase the performance after pruning. 




%%% Q4) Figure 1 has elements labeled 's' - these should be explained in the caption.

% Thanks for your reminder. In figure 1, $s_{i}$s mean coefficients by which a pruned filter is approximated using remained filters in l-th layer. and this is same with the coefficients by which a pruned channel is approximated using remained channels after CONV.

% Similarly, $s^{'}_{i}$s and $s^{*}_{i}$s are the coefficients by which a pruned channel is approximated using remained channels after BN and ReLU, respectively.




% \begin{table}[]
% \centering
% \normalsize
% \begin{tabular}{c||c|c|c|c} \Xhline{2\arrayrulewidth}
% Pruning ratio& OURS& NM& Prune & From Scratch (80epochs) \\ \Xhline{2\arrayrulewidth}
% 10 \% & \textbf{78.3}   & 77.57  & 77.67  & 48.22 (72.07)\\\hline
% 20 \% & \textbf{77.22}  &  75.62  &  75.4   & 47.85 (72.01)\\\hline
% 30 \% & \textbf{75.99}  & 73.03  &  73.18  & 46.7 (72.24)\\\hline
% 40 \% & \textbf{74.13}  & 69.71  &  70.12  & 46.53 (72.88)\\\hline
% 50 \% & \textbf{71.15}  & 65.03  &  65.72  & 44 (69.33)\\\Xhline{2\arrayrulewidth}
% \end{tabular}%
% \vspace{2mm}
% \caption{Fine-tuned results of ResNet-50 on CIFAR100 using L2-norm criterion. We fine-tune pruned and restored models for 20 epochs to compare methods after fine-tuning, where we use a Nesterov SGD optimizer with 0.9 momentum and the initial learning rate to 0.00001 divided by 10 at 5, 10 and 15 epochs. In case of from scratch, we train the randomly initialized model for 20 epochs where we use a Nesterov SGD optimizer with 0.9 momentum and the initial learning rate to 0.1}
% \label{tab:params:Fine-tuned results of ResNet-50 on CIFAR100 using L2-norm criterion}
% \end{table}




% show the comparison between our method and other data-free approaches that uses synthetic data for fine-tuning. Alternatively, we provide the comparison result between NM and our method LBYL after fine-tuning. the result is shown below. Table \ref{tab:param:ResNet50-CIFAR100 using l2-norm} presents that 





% You mentioned about "conserving synaptic flow"[27], but as far as I know, that method is introduced for weight pruning. It is possible to prune convolutional filters using weight pruning methods but pruning filters with weight pruning can't guarantee performance of compressed model because many kinds of weight pruning methods rely on fine-tuning. On the contrary, filter pruning methods try to reduce the fine-tuning as mentioned in related works in our paper. Therefore as mentioned in our paper, we focused on the filter pruning not weight pruning. 

% However, I think it will be a valuable study that uses fine-tuning via synthetic data generated by pre-trained generating models although it may require a lot of computation time as you mentioned. and If that method is applicable in filter pruning, our second sentence "since we cannot make appropriate changes in the remaining filters by fine tuning" should be changed as you mentioned.










%--> Thank for your guidance, In figure 1, $s_{i}$ means coefficient considering the linearity between preserved filters and the pruned filter and $s^{'}_{i}$ means coefficient considering both linearity and a batch normalization layer and $s^{*}_{i}$ means optimized scales considering both linearity and layers such as batch normalization and activation layer.

% \smalltitle{Q5}
% %%% Q5) Typos noticed

% Thanks for reminder. We will correct all typo you said in the final CRC version. 




% \begin{table}[]
% \centering
% \normalsize
% \begin{tabular}{c||c|c|c} \Xhline{2\arrayrulewidth}
% Pruning ratio& OURS& NM& Prune \\ \Xhline{2\arrayrulewidth}
% 10 \% & \textbf{78.14}   & 77.28  & 75.14  \\\hline
% 20 \% & \textbf{76.15}  &  72.73  &  63.39 \\\hline
% 30 \% & \textbf{73.29}  & 64.47  &  39.96  \\\hline
% 40 \% & \textbf{65.21}  & 46.4  &  15.32  \\\hline
% 50 \% & \textbf{52.61}  & 25.98  &  5.22  \\\Xhline{2\arrayrulewidth}
% \end{tabular}%
% \vspace{2mm}
% \caption{Recovery results of ResNet-50 on CIFAR-100 using L2-norm criterion}
% \label{tab:param:ResNet50-CIFAR100 using l2-norm}
% % \end{table}
% % \begin{table}[]
% \centering
% \normalsize
% \begin{tabular}{c||c|c|c|c} \Xhline{2\arrayrulewidth}
% Pruning ratio& OURS& NM& Prune & From Scratch (80epochs) \\ \Xhline{2\arrayrulewidth}
% 10 \% & \textbf{78.3}   & 77.57  & 77.67  & 48.22 (72.07)\\\hline
% 20 \% & \textbf{77.22}  &  75.62  &  75.4   & 47.85 (72.01)\\\hline
% 30 \% & \textbf{75.99}  & 73.03  &  73.18  & 46.7 (72.24)\\\hline
% 40 \% & \textbf{74.13}  & 69.71  &  70.12  & 46.53 (72.88)\\\hline
% 50 \% & \textbf{71.15}  & 65.03  &  65.72  & 44 (69.33)\\\Xhline{2\arrayrulewidth}
% \end{tabular}%
% \vspace{2mm}
% \caption{Fine-tuned results of ResNet-50 on CIFAR100 using L2-norm criterion. We fine-tune pruned and restored models for 20 epochs to compare methods after fine-tuning, where we use a Nesterov SGD optimizer with 0.9 momentum and the initial learning rate to 0.00001 divided by 10 at 5, 10 and 15 epochs. In case of from scratch, we train the randomly initialized model for 20 epochs where we use a Nesterov SGD optimizer with 0.9 momentum and the initial learning rate to 0.1}
% \label{tab:params:Fine-tuned results of ResNet-50 on CIFAR100 using L2-norm criterion}
% \end{table}

\subsection*{Reviewer AsvP}

% Thank you for the positive feedback and insightful comments that we have not thought about.

\smalltitle{Comment 1}
\textit{Would like to see analysis on how merged filters are changed to accommodate the information from pruned filters. How much information in the un-pruned kernel is lost as a result of the process? Does classic kernel behavior in terms of learning specific features hold after the kernel merging process?}

\smalltitle{Response 1}
Even though it is not trivial to quantify how much information in the preserved filter is lost as a result of the restoration process, we can claim that such a side effect is not much in LBYL, compared to NM. This is because LBYL minimizes the amount of those changes in remaining filters by making as many filters as possible to participate in the restoration process. On the other hand, NM forces each pruned filter to deliver its information to only one remaining filter, which can dramatically change the role of the remaining filter. As shown the table below, the average and maximum of absolute scaling factors (i.e., coefficients) of LBYL are much smaller than those of NM. More specifically, each un-pruned filter in LBYL is multiplied by only 0.00005 on the average, and consequently the information loss of the filter cannot be significant.
\begin{table}[h]
\centering
\scriptsize
\begin{tabular}{c||c|c}\Xhline{2\arrayrulewidth}
Statistic& LBYL & NM \\ \Xhline{2\arrayrulewidth}
Mean & 0.00005 & 0.00045 \\ \hline
Max & 0.03977 & 0.34971 \\ \hline
Min & 0 & 0 \\ \hline
\end{tabular}%
\label{cov:tab:scale}
\vspace{2mm}
\caption{Comparison on absolute scale coefficients between LBYL and NM}
\end{table}



\smalltitle{Comment 2}
\textit{One limitation of this method is that it looks at each layer individually when recovering pruned kernel information -- it might be interesting to see if this method could not be generalized for compatible cross-layer information recovery.}

\smalltitle{Response 2} 
Although our strategy is to minimize the layer-wise output error in this paper, it would be a promising future work to incorporate cross-layer information as well to further optimize the restoration process.



%that the large scales can lead to distortion of remained filter's property.


%--> Original property on remained filters is changed to capture features that the pruned filter captured. However, the amount of changes in remained filters is very small than NM because we assume linearity between pruned filter and preserved filters. That means our method give a appropriate information on pruned filter to the remained filters compared to NM that give a scale based on norm ratio between pruned and remained filter. In our github, we will show that scale on pruned filter is very small than NM that the large scales can lead to distortion of remained filter's property.


% \begin{figure}[!htbp]
% \centering
%   \includegraphics[width=14cm]{Figure_OURS}
%   \hfil
% \caption{OURS scale}
% \label{Figure_OURS}
% \end{figure} 

% \begin{figure}[!htbp]
% \centering
%   \includegraphics[width=14cm]{Figure_NM}
%   \hfil
% \caption{NM scale}
% \label{Figure_NM}
% \end{figure} 



\subsection*{Reviewer VcLn }

\smalltitle{Comment 1}
\textit{Understanding the entire idea took me a long time due to the constant switching between layers, notation (A vs Z) etc ... . I think this can be done in a much cleaner way.}

\smalltitle{Response 1}
We would like to point out that most reviewers other than this one commonly said our manuscript is well written and easy to understand.

Having said that, we have provided the table of notations frequently used throughout the paper in Appendix, which would help to understand how our method works. We have attached our notations in Figure 1 as well in order to clarify what each notation actually means in Figure \ref{fig:framework}. For example, $\mathbf{Z}^{(\ell)}$ are illustrated by blue squares, and $\mathbf{A}^{(\ell)}$ are represented by green squares in Figure 1.



\smalltitle{Comment 2}
\textit{While the experiments show it works better than a method based on a similar idea, it is not clear to me whether the approach works any better than training a smaller network to start with. (in essence training a smaller model from scratch). It would also be interesting to see how it compares against fine-tuning, or whether this approach can benefit from fine tuning.}

\smalltitle{Response 2}
Although our method is motivated by a data-free scenario and therefore is compared to another data-free pruning method, we totally agree with the reviewer that it is essential to test whether our method is also effective when the data becomes available later on. Therefore, we additionally present new experimental results in Table \ref{tab:finetune_scratch} on the final accuracy of each pruned or restored model after fine-tuning along with that of the same-sized architecture trained from scratch. In all the cases, our LBYL method outperforms the competitors particularly when the pruning ratio increases. In addition, as shown in Figure \ref{fig:Fine tuning and From Scratch}, LBYL shows the fastest convergence speed during the fine-tuning process. Both of these observations justify the fact that our LBYL method is useful in practice even when the data is available.


% Our approach is essentially based on data-free assumption. so our method may have benefit compared to the smaller network that requires fine-tuning in the sense that our method can be used when the training data is not available or the training data is available but the computing resource is constrained.(e.g., mobile device...)

% Even when the training data and fine-tuning process are available, our method can have benefit compared to the smaller network which is trained from scratch in the sense that the restored model converge much more faster than From Scratch model shown in the table \ref{tab:params:Fine-tuned results of ResNet-50 on CIFAR100 using L2-norm criterion}.

% Our method can be used when the model compression is needed but training data is not available or heavy computation is burdensome. But as reviewer 2 mentioned, you may wonder when to stop the pruning in real world because the accuracy of restored model decreases dramatically as pruning ratio increases. In this case, 
% If we have a table of restored model's performances according to pruning ratio in experimental setting and the situation where our method is applied is similar with the experimental setting, the performance can be guessed through this table.

% But, when this is not the case, we suggest a naive solution to know when to stop pruning. A naive solution is to employ random noisy images(e.g., gaussian random samples) and then measure weighted average reconstruction error(WARE) between original model and pruned model. Increasing the pruning ratio and checking the WARE, we may be able to avoid drastic loss. but this is just a rough method.

% On the other hand, our proposed methods can be used practically when users need to compress the original model and performance loss is relatively not critical.(e.g., recommendation system) For example, Choosing whether to update or not the smartphone is relatively robust to poor performance of model.

% Or, As reviewer 2 mentioned and shown in the table \ref{tab:params:Fine-tuned results of ResNet-50 on CIFAR100 using L2-norm criterion}, our method can also be used as the pre-processing step before performing fine-tuning. We found the converge time is reduced by comparison with NM and From Scratch.

% \kim{continued}
%--> Our method shows better performance compared to randomly initialized model with same fine-tuning epochs. However, if we train predefined small model from scratch during a lot of epochs, result will show similar performance regardless of restored methods because a few works[1,2] show that randomly initialized small models reaches the similar performance regardless of pruning methods. 
%Actually, performing the fine-tuning process assumes that there are sufficient data and computation such as GPU. Unfortunately, in practice, these assumptions are not always satisfactory. Therefore, our proposed method can be used in the following situations. Our method can be used in resource-constrained devices such as edge device and also pre-processing step before fine-tuning as the reviewer3 said too.

%[1] rethinking of network pruning
%[2] pruning from scratch 








\section{introduction}
\label{sec:intro}
As software grows in size and complexity, logging has become increasingly essential to ensuring software reliability~\cite{he2021survey, Chen2021ASO}. 
Logging means writing log statements into the source code, which generate runtime logs that record valuable information for a range of downstream tasks such as anomaly detection \cite{Zhang2019RobustLA, Lou2010MiningIF, He2016ExperienceRS, Du2017DeepLogAD, Zhang2019RobustLA}, fault diagnosis \cite{Zou2016UiLogIL,xu2023hue, xu2024divlog}, root cause analysis \cite{Amar2019MiningHT, He2018IdentifyingIS, Lin2016LogCB,xu2025openrca}, and program analysis \cite{Ding2015Log2AC, Shang2013AssistingDO, xu2024aligning}. 
The effectiveness of these downstream tasks heavily relies on the quality of the software logs~\cite{He2018CharacterizingTN}. Therefore, appropriate logging is essential to capture critical behaviors during software operation~\cite{YuanUsenixA1}.
\begin{figure*}[ht]
    \centering
    \includegraphics[width=0.9\textwidth]{img/logging_task.pdf}
    \caption{An example of logging statement generation. Logging statement generation can be separated as three parts: determining the position, selecting the level, and specifying the message.}
    \label{fig:logging_example}
\end{figure*}

% Numerous logging tools have been developed to assist software developers by automatically suggesting log statements based on provided code snippets~\cite{Mastropaolo2022UsingDL, Liu2021WhichVS, Xu2024UniLogAL, xie2024fastlog}.
To help software developers implement effective and efficient logging practices, researchers have been focusing on exploring methodologies for automatic logging in recent years.
As illustrated in Figure \ref{fig:logging_example}, automatic logging typically includes three steps: (1) determining the position, (2) deciding the verbosity level, and (3) specifying the message to be recorded.
% Leveraging the advanced text generation capabilities of large language models (LLMs) \cite{Floridi2020GPT3IN, raffel2020exploring}, \textit{LANCE}~\cite{Mastropaolo2022UsingDL} is the first end-to-end tool to integrate positioning, level selection, and message generation.
To address the automatic logging tasks, numerous logging methods have been proposed thanks to the advanced development of the Large Language Model (LLM).
For example, \textit{LANCE}~\cite{Mastropaolo2022UsingDL} was introduced as the first end-to-end logging method, seamlessly integrating the recommendation of log statement position, verbosity level, and message content.
Building on this foundation,
\textit{UniLog} \cite{Xu2024UniLogAL} employed a warm-up and in-context learning (ICL) strategy to enhance performance. 
\textit{FastLog} \cite{xie2024fastlog} improved the generation efficiency while maintaining precision. 
\textit{LEONID} \cite{Mastropaolo2023LogSG}, based on \textit{LANCE}, combined with deep learning and information retrieval technologies to enhance performance. \textit{SCLogger}~\cite{Li2024GoSC} adapted static analysis to extend the context for the code snippet.
% These studies typically evaluate performance using ad-hoc data splitting from the entire dataset, focusing on metrics such as the accuracy of log statement components, including position, level, and message.
% Machine translation metrics, such as BLEU~\cite{Papineni2002BleuAM} and ROUGE~\cite{Lin2004ROUGEAP}, are also employed to evaluate the quality of generated log messages. Although these evaluations offer valuable insights into logging tool performance, the use of low-quality data and incomplete assessments undermines the reliability of the results.
%%% TODO:陈述事实。(下面这句我随便写的。后面你看情况再作修改。
% However, we find that they all face certain limitations in the method evaluation phases:

% 然而,当前的logging研究面临了一个最大挑战:缺乏一套通用的评估数据和评估方法,使得不同的logging方法之间可以公平对比。这具体体现在两个方面:1. 每个logging method的test set实际上是从其自身training set之间抽取出来的。其ad-hoc的性质导致不同方法在实验中的表现都是来自于不同的数据集,而method之间的有效性无法公平对比 2. 每个logging method都不会使用完全一致的同一套metric做评估。这就导致每个method可能会在不同的维度上体现出更明显的优势性。出于对这一现状的认知,我们认为提出一套comprehensive的benchmark是有必要的。然而,由于当前可用的数据和评估策略都有flaw,因此我们无法trival的从现有数据和方法中构建benchmark,而这我们将在下方讨论。

However, current logging research faces a significant challenge: \textbf{The absence of a standardized evaluation dataset and methodology hinders fair comparisons between different logging methods}. This issue manifests in two key aspects:
(1) The test sets of different logging methods are often derived from their own training data, leading to inconsistent experimental performance and hindering fair comparisons.
(2) The evaluation metrics used for different logging methods are not fully consistent, resulting in varying advantages across different dimensions.
Given these challenges, we argue that \textit{establishing a comprehensive logging benchmark is essential}. However, it is infeasible to naively construct a benchmark via \textit{ensembling} existing evaluation data and metrics due to two major limitations of existing evaluation datasets and methods:


%% 修改策略:多搞几个bad pattern。说明人家有垃圾数据,我们没有垃圾数据(我们自己筛掉了)。此外,人家数据限制512,我们没有这种妥协于小LM的下策。
% \textbf{First, loose standards collection and inappropriate clean strategy compromise the quality of evaluation data and effectiveness in assessing the performance.}
\textbf{First, existing evaluation datasets lack the quality required for reliable assessment.}
For example, 18.45\% cases (2218/12020) in LANCE's test set and 30.36\% cases (2197/7237) in FastLog's test set involve numerous bad patterns of log messages. 
As illustrated in Figure~\ref{fig:low_quality} these anomalies predominantly fall into the following types: (a) duplicated content, (b) empty string, (c) duplicated empty and special characters, (d) contents that mismatch the log levels, and (e) explicit type casting.
These issues distort evaluation outcomes, as tools producing such flawed log statements tend to appear better on metrics despite having poor logging quality.
Moreover, these datasets are restricted to cases with very short contents ($\leq$ 512 tokens), a limitation imposed by their use of foundation models (e.g., T5, PLBART), which support a maximum input of 512 tokens. This constraint not only reflects a compromise to accommodate their models but also inhibits the evaluation of realistic, function-level logging capabilities, as many industrial functions exceed this token limit. Consequently, the datasets fail to meet the requirements for assessing logging quality in real-world scenarios.
% Previous evaluation datasets were typically created by splitting the entire dataset.
% The data selection rules of the entire dataset are commonly loose standards to ensure sufficient data for training. These criteria fail to ensure the quality and consistency of the data.
% Moreover, to accommodate the limitations of the tools, they filtered out all instances exceeding 512 tokens, ignoring the long code snippets that are commonly encountered in real-world development environments.
% This approach undermines the overall effectiveness and real-world applicability of the evaluation results, as it does not accurately reflect the true complexity of software projects. 

%% 修改策略:合并到第三点。
% \textbf{Second, the current evaluation method does not verify whether the generated log statements are compilable.} Generating the compilable log statements is the first requirement when applying automatic logging tools in practice. To relieve developers from the heavy effort required to design and maintain log statements~\cite{Chen2019ExtractingAS, Chen2017CharacterizingAD}, the basic requirement is to ensure that our tool can be seamlessly integrated into the DevOps process without introducing additional errors that require extra debugging effort from developers. Current evaluation methods merely focus on whether each component of log statements (\ie position, verbosity level, message) matches the ground truth but cannot assess whether the generated log statements might introduce compilation errors such as wrong code format or use undefined variables. Evaluating the compilability of predicted log statements reflects the effectiveness and reliability of logging tools, which are essential for practical use.

%% 修改策略:1. imply一下“similarity-based” quality是唯一解。 2. 不要强调2k个dynamic来自hadoop。
% \textbf{Third, the evaluation method cannot evaluate the quality of runtime logs generated by the predicted log statements}. Current evaluation methods assess the performance of tools based on the correctness of individual components of log statements.
% However, current metrics struggle to accurately reflect the quality of runtime logs in the real execution environment. Even a slight shift of log statement can lead to the miss of essential runtime details, such as a several-line shift from the ground truth or a mismatch in verbosity level. For instance, a minor difference in verbosity levels (e.g., debug vs. info) can cause critical information missed due to log level threshold settings in the source code.
% Therefore, we need to evaluate the quality of log statements in a real execution environment, with the goal of obtaining appropriate logs in specific scenarios, rather than merely focusing on the correctness of individual components or relying on statistical metrics to reflect the performance of logging tools.
\textbf{Second, current evaluation methods lack a fine-grained view for comprehensive assessment.}
Existing evaluations primarily focus on static code similarity between candidate log statements generated by logging methods and oracle log statements written by developers.
However, the code compilability is entirely neglected and can not be measured by static evaluation.
Our preliminary analysis revealed that 20\%-80\% of log statements generated by current methods fail to compile, a figure that has not been previously reported.
Furthermore, such static similarity fails to capture logging performance during program execution.
For instance, placing a log statement inside or outside a \textit{loop} may differ by only one line in code but result in vastly different numbers of log entries at runtime. Conversely, moving a log statement above or below a \textit{comment} may have no effect on the runtime logs despite a one-line difference in the code. Such runtime differences are overlooked in static evaluations.
These limitations highlight fundamental flaws in the current evaluation approach.
    \begin{figure}[t]
    \centering
    \includegraphics[width=0.9\linewidth]{img/low_quality_data_example.pdf}
    \caption{Bad patterns in existing datasets (using instances from LANCE dataset as examples): (1) \textbf{Duplicated Variable} records the same information multiple times, costing redundant overhead for both printing and recording logs in runtime. (2) \textbf{Empty String} provides insufficient context, hindering the effectiveness of debugging through printed logs. (3) \textbf{Unpredictable Character} contains numerous meaningless special tokens, making both the logging methods hard to predict and their printed logs difficult to parse for downstream analysis~\cite {gojko2006logging}. (4) \textbf{Wrong Verbosity Level} often misleads developers for debugging and fault localization~\cite{Chen2017CharacterizingAD}. (5) \textbf{Explicit Cast} couple logs to variable type casting, might cause runtime type conversion errors and system crash~\cite{Chen2017CharacterizingAD}.
    }
    \label{fig:low_quality}
    \end{figure}
    % 或者: (5) \textbf{Explicit Cast} couple logs to variable type casting, increasing the risk of runtime errors~\cite{Chen2017CharacterizingAD}.

    % \modify{
    % Better give statistics (even if some bad patterns rare)}
%%% 建议表格做以下修改:
% 1. duplicated variables -> duplicated content (应该也有静态文本是重复的内容)。例子上最好增加一个文本重复的情况。 -> 在一条 log 中暂时没发现静态文本存在重复的情况
% 2. empty content in string -> empty string -> 这个不完全是 empty string,还存在一个 string 中有一段连续的 empty content
% 3. special characters -> unpredictable / meaningless content -> 这个得强调是 special characters,因为这部分被定义为 bad pattern 的理由是会影响下游 log parse
% 4. mismatch level -> content mismatching log level -> 最后两个都是 <Characterizing and Detecting Anti-patterns in the  Logging Code> 这篇 paper 里面的,我改成原文里的名字了。
% 5. 逻辑耦合?
%%% 此外,还建议:
% 1. 标题改成“bad patterns in log contents”
% 2. caption里面解释一下每种bad pattern的影响, 比如duplicate logs会导致运维人员没办法根据log level来对应上确定的程度位置(比如在做程序修复/debug的时候常用的操作);空字符串则彻底没有任何意义;unpredictable characters也和空字符串类似,我们可以就直接说人类也预测不出来,因此对评估模型的logging能力而言没有意义;mismatch level这种和duplicate log接近,也是让开发者/运维人很为难。

% 1. 重复记录变量 -> 


% 具体来说,当前的评估指标仅仅从candidate log statement generated by logging methods和oracle log statement written by developers的静态代码相似度来评估logging quality。然而静态代码相似度并不能体现log statement在程序执行时的performance。例如把一个log statement生成在循环体内部或外部可能只相差一行,但执行时产生的log数量可能会有数量级上的差距。但是把log statement生成在一个注释语句的上方或下方,差距也只有一行,但是却可能产生的日志不会有任何差异。这种运行时产生日志的差异是无法从静态评估中得到的。在我们的初步评估中,我们发现现有方法产生的日志中,有20\%甚至都无法通过编译并生成日志文件——而这种可编译性的评估在当前的logging research中是完全被忽视的。这一切都说明当前的评估是flaw的。

%%% 由于评估数据来自大量GitHub仓库导致我们无法直接从真实的部署场景中使用日志做一些well-designed下游任务来直接评估日志语句的好坏,researcher只能暂时妥协于使用similarity-based strategy来作为proxy,来评估日志语句的好坏。我们的信念应当是:oracle log被长久维护且广泛认可,那么接近于它的生成的log应该是好的(即similarity-based)。但是,当前只从静态代码的角度评估similarity是非常不全面的。

%%% TODO: 我们的insight?对similarity-based的考虑?都没写。

% 为了在解决这两个问题的前提下构建一个全新的comprehensive logging benchmark,我们的insight是:1. 更热门且迭代更频繁的仓库的日志语句的质量相比大多数不知名代码仓库的质量会更高,减少各类bad-pattern存在的可能性。2. 通过代码执行的方式直接评估生成的日志文件的质量可以提供一种run-time的视角,enabling fine-grained logging quality evaluation。基于这种insight,我们提出了\methodname.

  % \textbf{Our Work.} 
To address these two challenges, our key insights are: (1) \textit{\textbf{Log statements from the most widely used and frequently updated repositories tend to be of higher quality with fewer bad logging patterns.}} (2) \textit{\textbf{Log files printed in runtime can serve as a proxy for evaluating log statements quality with a more comprehensive view of program logging performance.}}
Based on these insights, we introduce \methodname, the first unified logging benchmark involving a large-scale evaluation dataset and a comprehensive evaluation method for assessing both the static log statements and runtime log files printed via code execution.
Our dataset comprises 21,804 instances mined from the 10 most popular and actively maintained GitHub projects~\cite{GitHub}, each with at least 10,000 stars, and 500 log-related issues spanning diverse domains and application scenarios.
Furthermore, \methodname introduces \textit{dynamic evaluation} in addition to traditional static evaluation for log statements.
Specifically, it not only applies a suite of static metrics to quantify the quality of generated log statements but also reintegrates them into real project code, compiles, and executes them.
This approach enables a realistic assessment beyond the traditional static approach, highlighting major limitations in existing logging methods:  even the best-performing logging method fails to compile in 20.1\% of cases, and the logs printed in runtime show only 21.32\% cosine similarity to the oracle logs.
% Through its rigorous and standardized evaluation approach, \methodname bridges the gap between real-world logging requirements and prior assessments, highlighting substantial opportunities to further advance the development of automatic logging tools.
By providing a unified, standardized, and comprehensive evaluation dataset and methodology, we believe \methodname establishes a foundational step toward advancing research in automatic logging.


To sum up, our contributions are shown as follows:
\begin{enumerate}
    \item We collected a high-quality, diverse, and large-scale dataset comprising 21,804 instances from 10 popular, high-quality GitHub~\cite{GitHub} projects, spanning various domains with differing logging requirements.
    \item We propose a dynamic evaluation approach that reintegrates the generated log statements into projects to evaluate their printed runtime logs as a more fine-grained proxy for illustrating logging performance.
    \item We conducted a comprehensive evaluation of popular automatic logging tools and revealed the key limitations based on the analysis of the evaluation results. 
    \item All the data and code for \methodname are publicly available\footnote{\url{https://github.com/shuaijiumei/logging-benchmark-scripts}}, providing valuable resources for both developers and researchers to advance the field of automatic logging.
\end{enumerate}

  % \textit{Paper Structure:} The rest of the paper is organized as follows.
  % The section \ref{sec:moti} introduces existing related work and motivates our work by analyzing the existing dataset, and evaluation methodology. Section \ref{sec:method} introduces \methodname. Section \ref{sec:eval} we conduct a comprehensive evaluation for recent automatic logging tools via \methodname. Section~\ref{sec: vadility} discusses the potential threats to validity. Finally, Section~\ref{sec: conclusion} gives a conclusion for this paper.
  

% \section{background and motivation}
\section{Background}
\label{sec:moti}
    % This section introduces the overview of the current research on automatic logging tools, followed by the shortcomings in the evaluation work, and explains the motivation for \methodname.

% \subsection{Log Statement Generation}
\subsection{Related Work of Automatic Logging}
Logging, the process of generating informative log messages with appropriate verbosity levels at strategically placed locations within code, has long been recognized as a critical challenge in software engineering~\cite{he2021survey, Chen2019ExtractingAS, Chen2021ASO, Chen2021ExperienceRD}. Over the years, substantial research efforts have aimed to support developers in crafting more effective logging statements, which in turn enhance software maintenance and testing~\cite{Jia2018SMARTLOGPE, Zhao2017Log20FA, Zhu2015LearningTL}. Early studies in this domain often addressed isolated subproblems, typically operating under stringent assumptions that limit the applicability of their findings in real-world scenarios. For example, Li \etal~\cite{Li2021DeepLVSL} proposed \textit{DeepLV} to predict the appropriate logging level by taking surrounding code features into a neural network. Liu \etal~\cite{Liu2022TeLLLL} proposed \textit{Tell} to further adapted flow graphs to help the suggestions of verbosity levels. Zhu \etal~\cite{Zhu2015LearningTL} proposed \textit{LogAdvisor} and Yao \etal~\cite{Yao2019Log4PerfSA} proposed \textit{Log4Perf} to assist developers to add new log statements in a specific position. Ding \etal proposed \textit{LoGenText}~\cite{Ding2022LoGenTextAG} and \textit{LoGenText-Plus}~\cite{Ding2023LoGenTextPlusIN} to advise developers what should be logged, and Liu \etal \cite{Liu2021WhichVS} proposed tools for deciding which variables should be logged. However, none of them can generate a complete log statement. 

%%% 从intro中搬运替换过来的内容。
Recently, with the advanced capabilities of large language models (LLMs)~\cite{Floridi2020GPT3IN, raffel2020exploring}, numerous LLM-based logging methods have been proposed. 
Specifically, \textit{LANCE}~\cite{Mastropaolo2022UsingDL} was introduced as the first end-to-end logging method, seamlessly integrating the recommendation of log statement position, verbosity level, and message content.
Building on this foundation, \textit{UniLog} \cite{Xu2024UniLogAL} employed a warm-up and in-context learning (ICL) strategy to enhance performance. 
\textit{FastLog} \cite{xie2024fastlog} improved the generation efficiency while maintaining precision. 
\textit{LEONID} \cite{Mastropaolo2023LogSG}, based on \textit{LANCE}, combined with deep learning and information retrieval technologies to enhance performance. \textit{SCLogger}~\cite{Li2024GoSC} adapted static analysis to extend the context for the code snippet.

% Recently, Mastropaolo \etal \cite{Mastropaolo2022UsingDL} proposed the first end-to-end tool \textit{LANCE} to generate complete log statements based on T5~\cite{Raffel2019ExploringTL}. Following \textit{LANCE}, Xu \etal~\cite{Xu2024UniLogAL} proposed \textit{UniLog} to adapt ICL and warm-up strategy to enhance the LLM ability for generating log statements. Xie \etal~\cite{xie2024fastlog} proposed \textit{FastLog} increase the generation time while keeping the accuracy, and Mastropaolo \etal~\cite{Mastropaolo2023LogSG} further proposed \textit{LEONID} with a combination of Deep Learning (DL) and Information Retrieval (IR) achieving a better performance. 
    
% While end-to-end automatic logging tools have demonstrated promising results in their respective evaluations, our analysis reveals notable issues in both the datasets used and the evaluation methods employed.


\subsection{Philosophy of Logging Evaluation}\label{sec:phy}
%%% 这个section需要讨论一下我们对logging的评估的认知和信念。不要重复提到之前方法的问题了。intro已经详细讨论过了。
% \subsection{Limitations of Evaluation Methodology}
%%% TODO: 介绍logging评估的哲学。从downstream task说开去。

%%% 如何评价logging质量,一直是automatic logging领域一个关键但是又难以解决的问题。在理想情况下,高质量的logging可以意味着其产生的日志能够充分记录软件系统的关键运行状态信息,同时又不至于过度logging以导致占用过量计算资源和存储资源的开销。因此从直觉上来说,最直接的logging evaluation方法是依靠日志在这些下游任务上的表现来评估logging质量。然而,当我们想要量化这种logging质量的时候,就会出现许多的挑战。首先,日志的下游任务非常多。除了常见的log-based debugging、anomaly detection、root cause analysis等等任务以外,也可以单纯用于记录软件运行时产生的事件序列以监控系统运行状态,或者其他开发者认为可以用日志完成的任务等等。由于我们没法保证我们能够覆盖日志的所有可能的应用场景,因此我们构建由特定的下游任务作为代理,也只能评估在这些task-specific的logging表现,而无法声称这种方法可以评估general-purpose logging能力。此外,即使像anomaly detection等传统log analysis任务理论上可以通过故障注入的方式来评估logging有效性,但是对单纯的log-based monitoring而言,我们没法自动化地量化评估log是否提供了开发者所需的信息记录。另外,实际上要对多样化的软件构建传统log analysis任务如anomaly detection的故障数据集也是non-trival的,因为对于不同的软件系统而言,存在的故障类型是各不一样的,因此没法scalable的实现注入。最后,考虑到打印log本身也是会占据计算和存储资源开销的。为了避免产生大量日志并加重后续log compression或log reduction等任务的负担,在实际logging practice中,开发者也会对logging的effectiveness和efficiency做一个trade-off。因此,哪怕我们能够非常全面地评估logging在任何任务上的有效性并构建一套benchmark,我们也不能说明这样的logging是好的——很可能为了解决一些平时不常见的问题,日常打印log的开销会成倍增加,而这是不划算的。至于到底对于什么样的trade-off是合适的,一般不同的软件系统上开发者都有一套ad-hoc的认知,而无法通过指标量化的形式确定“什么样的logging是合适的”。综上所述,要通过采用下游任务作为proxy来体现general-purpose logging quality的好坏是无论从理论还是实践上来说都challenging的。

Logging quality evaluation has always been a critical yet challenging problem in automatic logging~\cite{Bogatinovski2022QuLogDA, Li2019DLFinderCA, Chen2017CharacterizingAD, Chen2021ASO, he2021survey}. In an ideal scenario, high-quality logs should accurately capture key runtime state information of a software system while avoiding excessive logging that may lead to unnecessary consumption of computational and storage resources. Intuitively, the most direct approach to assessing logging quality is to evaluate how well the logs support specific \textbf{downstream tasks}~\cite{YuanUsenixA1}. However, using downstream tasks as proxies to measure general-purpose logging quality in a scalable way is both theoretically and practically problematic for several reasons:

\begin{enumerate}
    \item First, it is very challenging to cover all potential downstream tasks for logs exhaustively. There are countless log-based downstream tasks: beyond the tasks with clear task formulations (e.g., debugging, anomaly detection, root cause analysis), logs are also used for a wide range of tasks that lack a universal formulation (e.g., recording sequences of runtime events to monitor system status), often driven by the unique needs of each system’s developer~\cite{Zhang2019RobustLA, xu2024divlog, Amar2019MiningHT}. Even if we only choose a fixed set of tasks with clear formulations to evaluate logging quality, these proxies can only assess \textit{task-specific aspects} rather than provide a comprehensive measure of \textit{overall} logging quality. This limitation makes them unreliable indicators of \textit{general-purpose} logging quality.
    \item  Second, it is not practical to build testbeds for these downstream tasks across all software systems. While it might be theoretically possible to create a testbed for evaluating logging quality in common tasks like anomaly detection, assessing tasks like whether logs capture key information for event monitoring needs is challenging, as it depends heavily on subjective developer requirements. Moreover, even creating fault datasets to support log-based anomaly detection is also challenging since different software systems have unique fault characteristics, making scalable fault injection across diverse environments infeasible~\cite{zhao_identifying_2021}.
    % \item Another key consideration is the trade-off between logging effectiveness and efficiency. Since logging incurs computational and storage costs, excessive logging can introduce significant overhead and complicate related processes such as log compression and reduction. Consequently, practitioners must carefully balance these trade-offs to optimize real-world logging practices based on their ad-hoc empirical experience on certain software application scenarios. For instance, overemphasizing solutions for some rare or edge-case scenarios could substantially increase routine logging costs without proportional benefits. Thus, such a process is still challenging for us to design an appropriate trade-off mechanism to evaluate whether a logging strategy is appropriate for each specific software.
    \item Third, the trade-off between logging effectiveness and efficiency is another unmeasurable aspect of logging quality. Massive logging offers detailed runtime data for fault diagnosis but increases storage costs and resource usage, while sparse logging reduces expenses but risks missing critical information. System designers balance these tradeoffs primarily through operational experience from maintaining and debugging specific systems. However, such experiences are inherently context-dependent, varying across systems due to divergent functional, architectural, and operational priorities~\cite{rong2023developers}. Consequently, it remains challenging to establish a universal trade-off mechanism for assessing the appropriateness of logging strategies for each specific software.
\end{enumerate}

% However, it is both theoretically and practically problematic to use downstream tasks as proxies to measure general-purpose logging quality for several reasons.


% 1. 下游任务太多,选其中几个无法为 general purpose logging benchmark
% 2. 其中一些下游任务构建 testbed 过于麻烦
% 3. 我们要评估 trade off , 但每一个软件系统的 trade off strategy 是不一样的,甚至在同一软件的不同部分的 trade off strategy 也是不一样的。 所有我们没办法去设定一定 standard 的 标准来评估这个 trade off,必须 case by case。

% 我们要评估 trade off , 但每一个软件系统的 trade off strategy 是不一样的,甚至在同一软件的不同部分的 trade off strategy 也是不一样的。 所有我们没办法去设定一定 standard 的 标准来评估这个 trade off,必须 case by case。 我理解到的是这个意思。 但感觉有点没说清楚。 首先我们这三点都是为了说明为什么用下游任务作为 evaluation proxy 是不可行的, 第一点说 下游任务多,不好选 第二点说有些下游任务难,不好构建。 这两点感觉都很清晰,并且 make sense。 但第三点感觉读第一遍有点没有强调好重点。 
% 我感觉逻辑应该是 effectiveness and efficiency trade off 是现代软件设计中很重要的一环。过多的 log 会提供更详尽的运行时状态,便于开发者们定位问题,但随之而来的是高额的存储成本和处理资源消耗; 过少的 log 虽然能节约成本,但是或许会漏掉许多关键信息。 对于每个系统的设计开发者来说,如何平衡 effectiveness and efficiency 主要依赖的是他们长期以来对运维和修复系统漏洞积累的专属于这个系统的宝贵经验。 对我们来说设计一套通用的 trade off strategy 来 suit 每一个系统是不可能的。


%%% 因此,作为一种妥协和替代方案,当前logging research通常采用了基于相似度比较大logging评估方式。其核心思路是,既然我们难以直接根据下游任务评估日志语句的质量,我们可以把那些本来具有较好质量的日志语句作为参考,通过构建benchmark来评估logging method能否生成和这些高质量的参考相似的日志语句。这种方式使得logging evaluation可以以scalable的方式完成,因为大量开源软件库中已经有相当大数量log statement可以作为reference了。然而,这种similarity-based evaluation strategy也有其固有的弱点,即,无法正确评价那些比reference log statement质量更好的log statement。那些quality比reference更好的predicted log statement很可能因为和reference不够相似从而获得更低的similarity score。要尽量缓解这一局限带来的影响,就必须确保作为reference的log statement质量足够高。这也就对数据的收集提出了挑战。要确保reference的质量,我们认为那些最热门的project中长期保持不变、没有被修改过的log statement是高质量的、经过时间检验的,因此也是最适合作为reference的。此外,当前的similarity-based evaluation仅仅停留在静态代码层面的评估。然而,静态代码上非常小的差异可能导致其产生的日志文件内容和规模上都有十分大的改变。因此,我们认为在评估静态代码相似度的前提下,也要额外评估生成的日志文件的相似度,这样才能提供一个全面的视野来评估质量。这两个insight drives us to propose our first logging benchmark: \methodname.

As an alternative, logging research often adopts a \textbf{similarity-based} approach to evaluate logging quality.
The core idea is that, since assessing log quality via downstream tasks is challenging, we can use high-quality log statements as references to evaluate new ones.
Thus, by constructing benchmarks, we can check if a logging method generates log statements similar to these reference ones, making evaluation scalable with log statements from open-source software.

However, this similarity-based approach does have limitations.
It may not accurately assess log statements of \textit{higher quality} than the reference, as they could receive lower similarity scores simply for not matching closely enough. To mitigate this, reference log statements must be of sufficiently high quality, which poses a challenge in data collection. We argue that log statements from the most \textit{widely-used}, \textit{time-tested} projects are the best reference candidates. Additionally, current evaluations focus solely on static code, but small changes in code can lead to significant differences in log content. Therefore, evaluating both static code and the generated log files provides a more comprehensive measure of logging quality. These insights inform the development of our first logging benchmark: \methodname.

\section{\methodname}
\label{sec:method}
\methodname is a novel benchmark designed to evaluate automatic logging tools in codebases systematically. The benchmark comprises 21,804 code snippets and 39,600 log statements extracted from 10 most popular open-source repositories, selected for their active maintenance, diverse application domains, and representation of modern software engineering practices. \methodname employs a dual evaluation framework: \textbf{Static Evaluation} evaluates the \textit{similarity of log statements} (\eg \textit{position}, \textit{level}, \textit{variable}, \textit{message}) between predicted and reference log statements, while \textbf{Dynamic Evaluation} assesses code \textit{compileability} and the \textit{similarity of logs} printed by those log statements during runtime execution. Both evaluation methods will be described later in this section.







% 思考一下,我们的数据集应该有哪些 feature 才能说明数据集质量高?
% 1. 数据本身质量高
% 2. 数据集足够 diversity

% 这段内容依然缺少了两个关键arguement:
% 1. repository是按rank筛选的the best ones。
% 2. log statement是tested by time。长期不动的。默认是开发者觉得高质量不用管的。
% 最开始的思路是, log statement 的数量足够多 -> 开发者对 日志代码质量的重视程度高 -> 所以单个的质量放心


\subsection{Dataset Construction}
\label{sec:method:static_dataset_building} 
As discussed in Sec.~\ref{sec:intro}, the quality of the evaluation dataset is crucial for assessing the performance of tools. 
% The strictest existing rules for dataset collection—500 commits, 10 contributors, and 10 stars—are inadequate to ensure the quality of the evaluation dataset. 
% Drawing inspiration from the use of GitHub repository stars as an effective metric for identifying high-quality code datasets ~\cite{Jiang2024ASO}, and considering the unique characteristics of log statements, 
% Drawing on evidence that GitHub repository stars effectively signal code quality ~\cite{Jiang2024ASO}, we propose dataset inclusion criteria tailored to log statement analysis: repositories must have at least 10,000 stars and 500 log-related issues to qualify for inclusion in our dataset. 
% Besides those requirements, we consider the amount of log statements are bigger, the developpers more care about the quality of log statements. So we rank the repos by the log statement number, in the same time, we consider the diverse log requirements are also important. So in the rank list, we choose 10 diversity projects.
Building on empirical evidence that GitHub repository stars correlate with code quality~\cite{Jiang2024ASO}, we determine dataset inclusion criteria for log statement analysis by requiring repositories to have $\geq$ 10,000 stars and $\geq$ 500 log-related issues to ensure community validation and logging relevance. Candidate repositories are then ranked by total log statement count to prioritize those with extensive logging practices.
% From the top of this ranked list, we manually curate a final dataset of 10 repositories, applying two qualitative criteria: (i) diverse logging needs: including projects with conflicting priorities, such as performance-critical systems (minimal logs), trace-intensive applications (detailed logs), and privacy-sensitive services (anonymized logs). and (ii) active maintenance: measured by frequent commits (>50/month), high issue resolution rate (>70\%), and recent updates (within 6 months). This dual-phase approach balances the quantitative scale with domain representativeness.
From the top of this ranked list, we manually curate a final dataset of 10 repositories to ensure that the set covers different priorities in the logging effectiveness-efficiency trade-off, including performance-critical systems (requiring minimal logs) and trace-intensive applications (requiring detailed logs). After that, we ensure that all of these repositories are actively maintained, with frequent commits (>50/month), a high issue resolution rate (>70\%), and recent updates (within the past 6 months). This dual-phase approach balances the quantitative scale with domain representativeness.


As shown in Table~\ref{table:dataset}, our final dataset includes projects with a total of 21,804 code snippets and 39,600 log statements, covering a wide range of logging needs and practices.
The dataset spans multiple domains, including database management, task scheduling, distributed storage, messaging systems, and IoT platforms.
% Each domain presents \textit{unique requirements} for log statements.
%% 这些仓库都因为其特定场景的需求,导致其有完全不同的logging的目标(即具体用log来做哪种下游任务),同时也有不同的effectiveness-efficiency trade-off,就如同Sec.3.3所示。
These repositories have \textit{\textbf{distinct logging objectives (targeted downstream tasks)}} based on the specific needs of their respective scenarios and exhibit \textit{\textbf{unique effectiveness-efficiency trade-offs (more logging vs less logging)}}, as shown in Sec.~\ref{sec:phy}.
% For example, database management systems such as DBeaver and Doris prioritize minimizing the impact of logging on high performance. Task scheduling systems, including DolphinScheduler, rely on logging to trace task dependencies and monitor runtime statistics. Similarly, distributed systems like Hadoop and Zookeeper require robust logging practices to address challenges in distributed coordination, fault tolerance, and scalability, which differ from the requirements of other command logs. Messaging systems such as Kafka and Pulsar have adapted logging practices to trace message flows, ensure reliable message delivery, and debug asynchronous communication. Meanwhile, IoT platforms like ThingsBoard utilize logging to manage device connectivity, monitor data streams, and enable real-time system oversight. Identity and access management systems such as Keycloak prioritize protecting sensitive information in logs to prevent privacy breaches.
% \textbf{\textit{Logging requirements diverge sharply across domains}}: 
For example, database systems like DBeaver minimize logging overhead to sustain performance, while task schedulers such as DolphinScheduler emphasize dependency-tracing logs for runtime monitoring. Distributed systems (e.g., Hadoop) enforce rigorous logging for fault tolerance and distributed coordination, contrasting with messaging platforms like Kafka, which track message flows to ensure delivery reliability. IoT solutions (e.g., ThingsBoard) leverage logs for real-time device connectivity oversight, whereas security-focused systems like Keycloak prioritize log anonymization to safeguard sensitive data.
% 这些不同的日志需求充分展现了日志代码的多样性和 xxx, 充分证明了我们无法设定一个通用的标准来评估日志代码的质量。 选择这些充满多样性,并且受到社区认可,经受住时间考验的项目来作为评估日志质量的 testbed,无疑 xxx


These diverse logging requirements not only demonstrate the heterogeneity and inherent contradictions in logging practices (e.g., performance-sensitive scenarios demand minimal logging overhead, while debugging scenarios necessitate comprehensive event records) but also substantiate the impossibility of establishing universal standards for assessing log code quality.
By documenting real-world logging practices from community-vetted, widely-used projects across domains, we establish a foundation for evaluating logging strategies in their native environments. This domain-rooted dataset supports comprehensive analysis of automated logging techniques across diverse operational scenarios.
    
% In addition to emphasizing data quality, we also addressed the potential risk of data contamination. Since all our data were extracted from public GitHub repositories, which may have been used for training pre-trained models, we implemented precautions to minimize this risk. Specifically, we collected the latest version of each project to reduce the likelihood of it being included in any model’s pre-training data. Furthermore, we wrapped the code snippets in a common class and standardized the formatting using Google-Java-Format~\cite{Google-java-format}. This approach altered the format of the code to prevent pre-trained models from recognizing the same information and structure. These strategies have been demonstrated as effective in recent studies~\cite{Xu2024UniLogAL, Tian2024DebugBenchED}.
% Although we adopted effective strategies, data contamination cannot be entirely avoided~\cite{Cao2024ConcernedWD}. However, our methods minimize this risk and have been proven effective in previous work, providing a solid foundation for analysis and evaluation. In the future, we plan to regularly update our dataset to ensure that the evaluation data remains current. After completing the necessary appeal actions, we finalized our static evaluation dataset.

 \begin{table}[htbp]
\caption{Details of \methodname dataset.}
\label{table:dataset}
\centering
\scalebox{0.8}{
\begin{tabular}{cccc}
    \toprule
    % \textbf{Dataset} & \textbf{Domain}  & \textbf{Code snippets with log statements} & \textbf{Total log statements}  \\
    \textbf{Dataset} & \textbf{Domain}  & \textbf{\# Code {w/ LogStmt}} & \textbf{\# LogStmt}  \\
    \midrule
    Dbeaver & Database Management & 1,178 & 1,707 \\
    Dolphinscheduler & Task Scheduling & 883 & 1,855 \\
    Doris & High Performance Database & 1,635 & 2,860 \\
    Flink & Data Processing & 1,916 & 3,150 \\
    Hadoop & Distributed Storage & 8,351 & 14,987 \\
    Kafka & Messaging Systems & 1,667 & 3,308 \\
    Keycloak & Identity and Access Management & 491 & 926 \\
    Pulsar & Messaging Systems & 3,279 & 6,347 \\
    Thingsboard & IoT Platform & 1,544 & 2,588 \\
    Zookeeper & Distributed Coordination & 860 & 1,872 \\     
    \midrule
    \textbf{Total} & - & \textbf{21,804} & \textbf{39,600} \\
    \bottomrule
\end{tabular}
}
\end{table}


\subsection{Static Evaluation}
\label{sec:statci_evaluation}
% \subsubsection{Static Evaluation Method.}
% The static evaluation focuses on log statement components—position, verbosity, and message, with details provided in Section~\ref{sec: Motivation: Evaluation Method}.
\subsubsection{Task Formulation}
We structure each entry in our dataset as a tuple \(<Code_{w/o~ LogStmt}, LogPos, LogStmt>\). As illustrated in Figure~\ref{fig:static_evaluation_task_formulation},  \(Code_{w/o~ LogStmt}\) represents the input code context with one log statement deliberately removed, \(LogPos\) indicates the ground truth position for the missing log statement, and \(LogStmt\) represents the exact log statement to be predicted.
 This process yielded 39,600 high-quality instances, providing a robust foundation for evaluating log generation models.

 \subsubsection{Metrics}
 \label{sec:method:metrics}
% We adopted five metrics for static evaluation. In addition to the previously established metrics, 
% To advance log-quality evaluation beyond conventional metrics, we introduce \textbf{Dynamic Expression Accuracy (DEA)} and \textbf{Static Text Similarity (STS)}. DEA assesses whether generated logs preserve the structural integrity of composite runtime expressions (\eg \(retries\_left * timeout\)), ensuring they encapsulate the same operational logic as ground-truth rather than merely replicating isolated variables. 


% STS evaluates the semantic alignment of static template messages (e.g., "Connection refused by host") using complementary metrics: BLEU and ROUGE. These metrics collectively enhance the fidelity of log information assessment under code-only evaluation constraints by jointly validating runtime semantic consistency and static descriptive intent. 

We introduce six static evaluation metrics to assess the similarity between the generated and reference log statements.

\textit{\textbf{Metric 1: Position Accuracy (PA):}} PA focuses on the precise line of log statements within the source code. The correct placement of log statements helps accurately trace the execution flow and diagnose issues. For Position Accuracy, we rigorously compare the predicted positions of log statements with their actual positions in the source code. This metric is calculated by taking the number of correctly positioned log statements \(P_c\) and dividing it by the total number of log statements \(N_a\) to obtain the accuracy value: \(PA = \frac{P_c}{N_a}\).

\textit{\textbf{Metric 2: Level Accuracy (LA):}}  
LA evaluates the exact match between predicted and reference log levels, which are essential for prioritizing operational events in DevOps pipelines. Common log levels carry distinct semantic implications, like "\textit{info}," referring to the normal information of runtime behavior.  "\textit{Warning}" indicates potential problems that might not immediately cause disruption but could lead to future issues if not resolved. "\textit{Error}" refers to runtime anomalies or issues that need to be addressed.
LA is calculated as the ratio of correctly predicted levels $L_c$ to the total log statements \(LA = \frac{L_c}{N_a}\).


\textbf{\textit{Metric 3: Average Level Distance (ALD):}} ALD further quantifies the \textit{severity deviation} of mispredicted levels from the reference level. We assign ordinal values to log levels: 
\textit{trace: 0, debug: 1, info: 2, warn: 3, error:4, fatal: 5}, and compute the absolute difference between predicted level $L_p^{(i)}$ and oracle level $L_a^{(i)}$ for each log. ALD is the mean deviation across all instances: $ ALD = \frac{1}{N_a} \sum_{i=1}^{N_a} \left| L_p^{(i)} - L_a^{(i)} \right| $.


\textit{\textbf{Metric 4: Message Accuracy (MA):}}
MA evaluates how accurately the predicted log messages match the oracle, which is essential for providing meaningful and relevant information during runtime. The content of log messages helps developers understand the system’s behavior, and inaccuracies in message generation can lead to confusion or missed insights during debugging. For Message Accuracy, we compare the predicted log messages to the actual messages in the source code. This metric is calculated by determining the number of log messages that are fully identical to the ground truth \(M_c\) and dividing it by the total number of log messages \(N_a\), yielding the accuracy value: \(MA = \frac{M_c}{N_a}\).

\textit{\textbf{Metric 5: Dynamic Expression Accuracy (DEA)}:} DEA evaluates whether generated logs preserve the structural integrity of runtime expressions, including both individual variables and composite logic (\eg ternary operators, arithmetic). For example, in the log template:\textit{("The server run on the ports, \{\}", args.status ? localPort : remotePort)}, the conditional expression \textit{(args.status ? localPort : remotePort)} is treated as a single semantic unit.
We aim to use this metric to ensure that the dynamic information recorded in logs remains consistent. This metric is calculated by taking the number of the exactly matched \(DP_c\) and dividing it by the total number of log statements \(N_a\) to obtain the accuracy value: \(DEA = \frac{DP_C}{N_a}\).

 \textit{\textbf{Metric 6: Static Text Similarity (STS) (with BLEU or ROUGE):}} 
 STS focuses on the static part of the log message. Unlike the dynamic variable, the static part always records the same information in log files, which will not vary due to the runtime behavior of software. For example, in the log message \textit{("The server is running, \{\}", status)}, "The server is running, \{\}" is regarded as the static part.
 Since this part primarily consists of natural language content, we use BLEU~\cite{Papineni2002BleuAM} and ROUGE~\cite{Lin2004ROUGEAP} metrics to evaluate the quality of the static text.
 In our implementation, we use the DM variant of BLEU~\cite{Chen2014ASC, Shi2021OnTE}, \ie the sentence-level BLEU without any smoothing method — coupled with ROUGE-L to holistically assess both lexical precision and long-sequence coherence. Specifically, ROUGE-L focuses on the longest common subsequence that effectively captures key operational patterns (e.g., error codes or API call chains) in multi-line logs, while BLEU's n-gram overlap measurement complements it by evaluating template fidelity at the token level.


 \begin{figure}
    \centering
    % \includegraphics[width=1\linewidth]
    \includegraphics[scale=0.7]
    {img/static_evaluation_task_formulation.pdf}
    \caption{\(Code_{w/o\ \ LogStmt}\) indicates the code without one log statement, the \(LogPos\) means the position of this log statement, \(LogStmt\) is the log statement itself. Those three structure the evaluation tuple.}
\label{fig:static_evaluation_task_formulation}
\end{figure}
%  The calculation methods for BLEU-DM are detailed below as shown in Eq.~\ref{eq:BLEU}.
%     \begin{equation}
%     \label{eq:BLEU}
% \text{BLEU} = \frac{1}{N} \sum_{i \in N} (\text{BP} \cdot \exp(\sum_{n=1}^{4} w_n \log p_n))
% \end{equation}
% where \(w_n\) denotes the weight of each n-gram and \(p_n\) represents the precision of each n-gram, which can be calculated as Eq.~\ref{eq:pn}
%     \begin{equation}
%     \label{eq:pn}
% \text{\(p_n\)} = \frac{\#\text{n-grams in the reference}}{\#\text{n-grams in the candidate}} 
% \end{equation}
% \(BP\) is the penalty value parameter for short candidate texts, and its calculation method is shown in Eq.~\ref{eq:bp}
%     \begin{equation}
%     \label{eq:bp}
% BP = \begin{cases} 
% 1 & \text{if } c \geq r \\ 
% e^{(1-r/c)} & \text{if } c \leq r 
% \end{cases}
% \end{equation}
%  where \(r\) is the length of reference and \(c\) is the length of the candidate text.

\subsection{Dynamic Evaluation}
\begin{figure*}[t]
    \centering
    \includegraphics[width=1\textwidth]{img/evaluation_based_on_execution.pdf}
    \caption{The general workflow of dynamic evaluation. First, compile the project and run the unit test to obtain ground truth logs. Then, replace log statements with predictions, re-run the test to generate new logs, and finally analyze the results.} 
    % \modify{Move to next page.}
    \label{fig:dynamic_evaluation_workflow}
\end{figure*}
\label{sec:method:dynamic_evaluation}
\subsubsection{Task Formulation}
In dynamic evaluation, we provide the runtime perspective of evaluating the automatic logging tools. 
Two key evaluation metrics are employed: (1) compilability—ensuring the code with predicted log statements compiles without errors, and (2) log file similarity—measuring the alignment between log files generated by the predicted log statements and the oracle through textual similarity analysis. These criteria jointly validate both the functional correctness of log integration and the relevance of logged content.
% The compilability of the code and the similarity of log files generated by the log statements with oracle are the two primary criteria.
% 这种方法在可控并且不需要消耗大量计算资源的情况下,模拟了软件运行时的状态,提供给我们宝贵的运行时日志。

As illustrated in Figure~\ref{fig:dynamic_evaluation_workflow}, the core innovation of our methodology lies in utilizing unit tests to emulate runtime log generation under controlled conditions. By executing these tests, we synthetically replicate software runtime states with minimal computational overhead, thereby enabling systematic collection of runtime logs that mirror real-world execution patterns for downstream analysis.
% As illustrated in Figure~\ref{fig:dynamic_evaluation_workflow}, the novelty of our approach is to leverage unit tests to simulate runtime log generation in a controlled environment. This approach validates both the compilability of code with inserted logging statements and generates runtime logs, enabling direct comparison against expected log outputs.
% We define evaluation quadruple as \(<Code_{input}, Code_{output}, Log_{output}>\), where \(Code_{input}\) is the code snippets without log statements covered by unit test, \(Code_{output}\) is the code snippets with predicted log statement, and \(Log_{output}\) is the logs produced by those log statements in \(Code_{output}\) via unit tests. If the \(Code_{output}\) is not compiliable, \(Log_{output}\) is null.
We define task formulation as a tuple \(<Code_{w/o\ \ LogStmt}, Code_{w/\ \ LogStmt}, Logs>\), where \(Code_{w/o\ \ LogStmt}\) is the code snippets without log statements covered by unit test, \(Code_{w/\ \ LogStmt}\) is the code snippets with predicted log statement, and \(Logs\) is the logs produced by those log statements in \(Code_{w/\ \ LogStmt}\) via unit tests. If the \(Code_{w/\ \ LogStmt}\) is not compiliable, \(Logs\) is null.

To build the dynamic evaluation pair based on our dataset, we begin by compiling the project to ensure all dependencies are resolved and the project is ready for execution. Next, we systematically identify all available unit tests within the project. For each unit test, we execute it individually while employing the Jacoco Plugin~\cite{Jacoco} to trace code coverage, specifically identifying whether the unit test covers any log statements in the codebase. Simultaneously, we use the SureFire Plugin~\cite{surefire} to capture the logs generated during the execution of the unit tests.
By correlating Jacoco coverage data with SureFire logs, we can match specific code snippets containing log statements to the corresponding unit tests that cover them, along with the runtime logs they generate.
We finally built 2,238 instances for dynamic evaluation.

% To construct a dynamic evaluation dataset, it is essential to select high-quality projects with comprehensive unit tests, as these tests provide a realistic simulation of diverse production environments. Each instance of dynamic evaluation requires recompiling the project and executing the test to collect logs, making the process highly time-consuming. To balance this intensive time requirement with the need for sufficient dataset diversity and quantity, we employed Hadoop as our dynamic evaluation platform. 




% \subsubsection{Dynamic Evaluation Method}
% Different from static evaluation, dynamic evaluation focuses on compiling the code and runtime-generated logs, addressing static evaluation’s inability to verify code compilability and runtime logs. 
% To directly assess runtime logs, we generate them using unit tests, which are widely used in software development to verify code functionality in isolated scenarios. Unit tests are readily available in most projects and they are designed to test the functionality and behaviors of code when facing different situations. They offer a natural method for simulating realistic situations, allowing generating logs without the need for complex runtime environments. 
 % Figure~\ref{fig:dynamic_evaluation_workflow} demonstrates the general workflow of dynamic evaluation. The process begins with compiling the source code and executing the unit tests to obtain logs from the original log statements. Using tools like Jacoco~\cite{Jacoco} and SureFire~\cite{surefire}, we collect the logs and remove the log statements covered by unit tests in the source code.
% Next, we input the modified source code into an automatic logging tool to generate predicted log statements. Then we inserted the predicted log statements back into the source code, replacing the original log statements. After recompiling the modified code, we rerun the unit tests to capture the logs produced by the inserted predicted log statements. The whole process provides us with two sets of logs—those generated by the original log statements and those generated by the predicted log statements.
% Finally, we evaluate the effectiveness of the predicted log statements using two key metrics: Compilation Success Rate and Log File Similarity. Compilation Success Rate ensures that the predicted log statements do not introduce compilation errors, while Log File Similarity measures the similarity between the logs generated by the predicted log statements and those generated by the original log statements.
% In the following sections, we will detail how we built the dataset for dynamic evaluation and introduce the specific metrics used to measure the performance of the automatic logging tools.


% \subsubsection{Dataset Construction}
% To build the dynamic evaluation dataset, we begin by compiling the entire project to ensure all dependencies are resolved and the project is ready for execution. Next, we systematically identify all available unit tests within the project. For each unit test, we execute it individually while employing the Jacoco Plugin~\cite{Jacoco} to trace code coverage, specifically identifying whether the unit test interacts with or covers any log statements in the codebase. Simultaneously, we use the SureFire Plugin~\cite{surefire} to capture the logs generated during the execution of the unit tests.

% By correlating Jacoco coverage data with SureFire logs, we can match specific code snippets containing log statements to the corresponding unit tests that cover them, along with the runtime logs they generate. This process enables us to construct a comprehensive dataset consisting of triples: the code snippet, the unit test that triggers it, and the recorded logs. These triples are critical for dynamic evaluation, as they provide the ground truth for assessing the quality and effectiveness of predicted log statements in actual runtime scenarios.

% To construct a dynamic evaluation dataset, it is essential to select high-quality projects with comprehensive unit tests, as these tests provide a realistic simulation of diverse production environments. Each instance of dynamic evaluation requires recompiling the project and executing the test to collect logs, making the process highly time-consuming. To balance this intensive time requirement with the need for sufficient dataset diversity and quantity, we employed Hadoop as our dynamic evaluation platform. This approach allowed us to build a dataset of 2,238 instances that balances diversity with sufficient size. Additionally, we open-source the entire suite of tools used in this evaluation process, enabling researchers and organizations to easily deploy and customize their own dynamic evaluation datasets.

\subsubsection{Metrics}
\label{sect:method_dynamic_metrics}
% In dynamic evaluation, we proposed four metrics to assess the performance of logging tools: \textbf{\textit{Compilation Success Rate}}, \textbf{\textit{Log Similarity}}, \textbf{\textit{False Positive Log Generation Rate}}, and \textbf{\textit{False Negative Log Generation Rate}}, which we will introduce below.
We propose four dynamic evaluation metrics to assess the similarity of logs printed by the generated and reference log statements in runtime execution.

\textit{\textbf{Metric 7: Compilation Success Rate (CSR):}} CSR measures the compilability of the predicted log statements. Due to issues such as undefined variables in the predictions or missing/outdated dependencies in the project environment, not all code snippets with predicted log statements can successfully compile.  We recorded the successfully compiled code snippet number as \(C_s\) and the evaluation pair number as \(C_a\). The metric is then calculated as: \(CSR = \frac{C_s}{C_a}\).
 
 \textit{\textbf{Metric 8: Log File Similarity (LFS) (with COS, BLEU, ROUGE):}}
 LFS evaluates how closely logs generated by predicted statements match those produced by ground truth statements.
 % To eliminate unnecessary differences, we remove log headers (e.g., timestamps), retaining only log content.
 We remove log headers (e.g., timestamps) and only assess the main message body of each log entry to eliminate unnecessary differences.
 For a comprehensive assessment, we apply multiple similarity measures, including Cosine Similarity (COS)~\cite{Salton1975AVS}, BLEU~\cite{Papineni2002BleuAM}, and ROUGE~\cite{Lin2004ROUGEAP}. Cosine Similarity, commonly used in text analysis, calculates the cosine of the angle between two TF-IDF~\cite{SprckJones2021ASI} vectors, yielding 1 for identical vectors and 0 for orthogonal ones. Using TF-IDF, we down-weight frequent terms, emphasizing distinctive content in logs. This method effectively captures the similarity between meaningful log content, filtering out redundant information for a more accurate relevance measure.
% The formula for Cosine Similarity is shown in Eq.~\ref{eq:cosine_similarity}.
% \begin{equation}
%     \label{eq:cosine_similarity}
%     \text{Cosine Similarity} = \frac{A \cdot B}{||A|| \, ||B||} = \frac{\sum_{i=1}^{n} A_i B_i}{\sqrt{\sum_{i=1}^{n} A_i^2} \, \sqrt{\sum_{i=1}^{n} B_i^2}}
% \end{equation}
ROUGE, on the other hand, focuses on recall by comparing n-grams between the predicted and reference logs. It evaluates how much of the reference content is preserved in the prediction. The most commonly used variant is ROUGE-N, which calculates the overlap of n-grams between two texts. 

% The formula for ROUGE-N is shown in Eq.~\ref{eq:rouge}
% \begin{equation}
%     \label{eq:rouge}
%     \text{ROUGE-N} = \frac{\sum_{\text{match} \in \text{n-gram}} \text{Count}(\text{match})}{\sum_{\text{n-gram} \in \text{reference}} \text{Count}(\text{n-gram})}
% \end{equation}

 \textit{\textbf{Metric 9: False Positive Log Generation Rate (FPLR):}}
FPLR measures the proportion of predicted log statements that generate logs during unit test execution when the ground truth log statements would not have produced any logs. It helps assess whether the predicted log statements introduce unnecessary or redundant logs in scenarios where no log should be generated. The number of false positive instances is recorded as \(FP\), and the total number of predictions is \(P\). the metric is calculated as: \(FRLR=\frac{FP}{P}\).

 \textit{\textbf{Metric 10: False Negative Log Generation Rate (FNLR):}}
FNLR evaluates the proportion of predicted log statements that fail to generate logs during unit test execution when the ground truth log statements should have produced logs. It highlights instances where the predicted logs miss important events or information. The number of false negative instances is recorded as \(FN\), and the total number of predictions is \(P\). The metric is calculated as: \(FNLR = \frac{FN}{P}\).

% In summary, the evaluation of log statements in this paper leverages a comprehensive set of metrics designed to assess various aspects of both static and dynamic evaluations. These include syntactic correctness through Compilation Success Rate (CSR), practical effectiveness via runtime analysis such as False Positive Log Generation Rate (FPLR) and False Negative Log Generation Rate (FNLR), and content similarity through Cosine Similarity, BLEU, and ROUGE. Together, these metrics provide a thorough assessment of log statement quality, capturing both surface-level correctness and deeper runtime effectiveness. This multi-faceted evaluation framework ensures that automatic logging tools are tested not only for their technical accuracy but also for their real-world utility in generating meaningful and actionable logs.

\section{EXPERIMENTS}
\label{sec:eval}
In this section, we use \methodname to evaluate existing end-to-end automatic logging tools and analyze the evaluation results to find insights to guide the following work. From the perspective of empirical software engineering, we set three research questions:
\begin{itemize}
    % \item \textbf{RQ1:} How accurately can automatic logging tools predict logging statements in static evaluation?
    % \item \textbf{RQ2:} What percentage of source code remains compilable after automated logging statement insertion?
    % \item \textbf{RQ3:} How similar do the generated log statements print logs in dynamic evaluation?
    \item \textbf{RQ1:} How similar are automatically generated log statements to human-written logging practices?
    \item \textbf{RQ2:} How compilable are the log statements generated by automatic logging tools?
    \item \textbf{RQ3:} How consistent are runtime logs from auto-logged programs with oracle-generated logs?
    %%% 当前logging methods在\methodname上的
\end{itemize}

% examines how well logging tools predict log statements using the \methodname static evaluation dataset and corresponding static evaluation method. 
    Specifically, RQ1 aims to evaluate the performance of the tools in generating log statements that closely resemble the ground truth. RQ2 focuses on assessing the compilability of the predicted log statements, determining whether these predictions can be seamlessly integrated into the code without causing compilation errors. Finally, RQ3 evaluates the runtime logs produced by the generated log statements. We examine the similarity between the predicted logs and the expected logs. We begin by introducing the automatic logging tools selected for evaluation. Then we provide a detailed analysis of these tools by answering the three research questions. This comprehensive evaluation allows us to draw insights into the strengths and limitations of current approaches for automatic logging.

\subsection{Automatic Logging Tools}
Automatic logging is a hot topic, leading to the development of many tools for determining specific parts of log statements and end-to-end automatic logging tools in recent years. In this paper, we focus on end-to-end logging tools. We reached out to the authors of popular end-to-end automatic logging tools for assistance in rebuilding these tools. Because SCLogger~\cite{Li2024GoSC} is still under construction,
we ultimately selected four methods for evaluation: \textit{LANCE~\cite{Mastropaolo2022UsingDL}, LEONID~\cite{Mastropaolo2023LogSG}, FastLog~\cite{xie2024fastlog}, UniLog~\cite{Xu2024UniLogAL}}. We detailed the method in the following.

\textit{\textbf{LANCE:}} \(LANCE\)~\cite{Mastropaolo2022UsingDL} is the first model designed to generate and insert complete log statements in code. It takes a method requiring a log statement and outputs a meaningful log message with an appropriate logging level in the correct position. Built on the Text-To-Text Transfer Transformer (T5) model~\cite{Raffel2019ExploringTL}, \(LANCE\) is trained specifically for injecting proper logging statements.

\textit{\textbf{LEONID}:} \(LEONID\)~\cite{Mastropaolo2023LogSG} is the updated version of \(LANCE\). With a combination of DL and Information Retrieval (IR), \(LEONID\) achieved a better performance. \(LEONID\) provided two versions, \(LEONID_S\) is for single log statement generation, and \(LEONID_M\) is for multiple log statements generation. Since \(LEONID_M\) can generate more than one log statement at a time, it introduces ambiguity in determining the correct correspondence between the generated and expected log statements when more than one log statements are generated by static evaluation~\cite{Mastropaolo2023LogSG}. Therefore, we only applied \(LEONID_S\) for static evaluation.

\textit{\textbf{UniLog:}} \(UniLog\)~\cite{Xu2024UniLogAL} is the first attempt to adapt Warm-up and In-context-learning strategy to enhance the model's ability to generate log statements.
Due to limitations in assessing the original \textit{UniLog}, we reproduced it using two backbone models: CodeLlama-7B~\cite{Rozire2023CodeLO} and DeepSeek-V3~\cite{DeepSeekAI2024DeepSeekV3TR}. We applied the warmup process exclusively to the CodeLlama-7B backbone model while employing the ICL strategy to construct prompts for both models.
The data are sourced from \textit{LANCE}~\cite{Mastropaolo2022UsingDL} to warm up and generate ICL content. The effectiveness of In-Context Learning often depends on whether the examples are in-distribution or out-of-distribution relative to the evaluation data. Since \textit{LANCE’s} data distribution differs from our evaluation data, this may affect \textit{UniLog’s} performance. We will use \(\bm{UniLog_{{cl}}}\) to represent the version based on CodeLlama-7B, \(\bm{UniLog_{{ds}}}\) to represent the version based on DeepSeek-V3.

\textit{\textbf{FastLog:}} \textit{FastLog}~\cite{xie2024fastlog} defines the logging task in two steps: finding the position and generating and inserting a complete log statement into the source code. This approach avoids rewriting the source code, a key limitation of \(LANCE\). They utilized PLBART~\cite{Ahmad2021UnifiedPF} as the base model to fine-tune two separate models: one for predicting insertion position, the other for generating log statements. With the heuristic rule, log statements only appear after certain special characters, \textit{FastLog} enhances efficiency while maintaining accuracy in generating log statements.


 % \begin{table*}[htbp]
% \caption{Static evaluation on the complete logging task was conducted. The values in the \textbf{Original} lines indicate the reported results from the respective methods' previous evaluations. \textbf{Now} lines indicate the evaluation results in \methodname. \textbf{PA} means position accuracy, \textbf{LA} means level accuracy, \textbf{MA} means message accuracy, 
% \textbf{ALD} means average level distance, \textbf{DEA} means dynamic expression accuracy, and \textbf{STS} means static text similarity. The best performance evaluated by \methodname across different metrics is highlighted.}
% \label{table:static_evaluation_results}
% \centering
% \setlength{\tabcolsep}{8pt} % 调整列间距
% \scalebox{0.9}{ % 适当缩放表格
% \begin{tabular}{l|c|c|c|c|c|c|c|c|c|c}
%     \toprule
%     \multirow{2}{*}{\textbf{Method}} & \multicolumn{2}{|c|}{\textbf{PA}} & \multicolumn{2}{|c|}{\textbf{LA}} & \multicolumn{2}{|c|}{\textbf{MA}} & \multirow{2}{*}{\textbf{ALD}}& \multirow{2}{*}{\textbf{DEA}} & \multicolumn{2}{|c}{\textbf{STS}}\\

%     % \multirow{2}{*}{\textbf{BLEU}} & \multirow{2}{*}{\textbf{ROUGE-L}}
    
%     \cmidrule(lr){2-3} \cmidrule(lr){4-5} \cmidrule(lr){6-7} \cmidrule(lr){10-11}
%     & \textbf{Now} & \textbf{Original} & \textbf{Now} & \textbf{Original} & \textbf{Now} & \textbf{Original} & & & \textbf{BLEU-4} & \textbf{ROUGE-L} \\
%     \midrule
%     \textit{FastLog} & \(\textbf{58.39}\)& \(58.84\) & \(\textbf{62.15}\)& \(59.75\) & \(\textbf{6.93}\)& \(4.52\) & 0.63& \textbf{18.00} &\textbf{20.14}&29.32\\
%     \(UniLog_{{ds}}\) & \(37.11\) & \(76.90\) & \(60.66\) & \(72.30\) & \(5.23\) & \(22.40\) & \textbf{0.61} & 16.20 &11.62&26.37\\
%     \(UniLog_{{cl}}\) & \(23.49\) & \(76.90\) & \(50.97\) & \(72.30\) & \(2.71\) & \(22.40\) & 0.79 & 16.60 &8.79&\textbf{29.88}\\
%     \textit{LANCE} & \(35.97\) & \(65.40\) & \(37.70\) & \(66.24\) & \(3.11\) & \(16.90\) & 2.11 & 15.25 &6.7&15.64\\
%     \(LEONID_S\) & \(11.26\) & \(76.45\) & \(17.90\) & \(73.53\) & \(1.96\) & \(31.55\) & 3.78 & 8.51 &3.45&6.69\\
%     \bottomrule

% \end{tabular}
% }
% \end{table*}


\begin{table*}[htbp]
\caption{Static evaluation on the complete logging task was conducted.
% \textbf{PA} means position accuracy, \textbf{LA} means level accuracy, \textbf{MA} means message accuracy, 
% \textbf{ALD} means average level distance (smaller is better), \textbf{DEA} means dynamic expression accuracy, and
Note that lower \textbf{ALD} means better log level quality, and
\textbf{BLEU-4 \& ROUGE-L} refer to two kinds of \textbf{STS} metrics. The best results are highlighted.}
\label{table:static_evaluation_results}
\centering
\setlength{\tabcolsep}{8pt} % 调整列间距
\scalebox{1.0}{ % 适当缩放表格
\begin{tabular}{l|c|c|c|c|c|c|c}
    \toprule
    \multirow{1}{*}{\textbf{Method}} & \multirow{1}{*}{\textbf{PA}} & \multirow{1}{*}{\textbf{LA}} & \multirow{1}{*}{\textbf{MA}} & \multirow{1}{*}{\textbf{ALD}}& \multirow{1}{*}{\textbf{DEA}} & \multirow{1}{*}{\textbf{BLEU-4}}&\multirow{1}{*}{\textbf{ROUGE-L}}\\
    
    \midrule
    \textit{FastLog} & \(\textbf{58.39}\) & \(\textbf{62.15}\) & \(\textbf{6.93}\)& 0.63& \textbf{18.00} &\textbf{20.14}&29.32\\
    \(UniLog_{{ds}}\) & \(37.11\) & \(60.66\) & \(5.23\) & \textbf{0.61} & 16.20 &11.62&26.37\\
    \(UniLog_{{cl}}\) & \(23.49\) & \(50.97\) & \(2.71\) & 0.79 & 16.60 &8.79&\textbf{29.88}\\
    \textit{LANCE} & \(35.97\) & \(37.70\) & \(3.11\)  & 2.11 & 15.25 &6.7&15.64\\
    \(LEONID_S\) & \(11.26\) & \(17.90\) & \(1.96\) & 3.78 & 8.51 &3.45&6.69\\
    \bottomrule

\end{tabular}
}
\end{table*}
\subsection{RQ1: How similar are automatically generated log statements to human-written logging practices?}
\label{rq1}
\begin{table*}[htbp]
\caption{The performance of tools when facing different length input data. \textbf{Short} means data shorter than 512 tokens, \textbf{Long} means data longer than 512 tokens, and $\Delta$ means the difference in logging tools performance when facing the \textbf{Short} data and the \textbf{Long} data.}
\label{table:static_len_compare}
\centering
\setlength{\tabcolsep}{12pt} % 调整列间距
\scalebox{0.85}{ % 适当缩放表格
\begin{tabular}{l|c|cclcccc}
    \toprule
\textbf{Method} & \textbf{Dataset} & \textbf{PA} & \textbf{LA}  & \textbf{MA}&\textbf{ALD} & \textbf{DEA} & \textbf{BLEU-4} &\textbf{ROUGE-L}\\
    \midrule
\multirow{3}{*}{\textit{FastLog}} & Short& 62.27& 61.79& 6.81&0.61& 19.65& 19.71& 29.43\\
 & Long& 45.04& 63.37& 7.37&0.68& 12.35& 21.63& 28.9\\
 & $\Delta$ & $\downarrow 17.23$& $\uparrow 1.58$ & $\uparrow 1.58$&$\uparrow 0.56$& $\downarrow 7.30$& $\uparrow 1.92$& $\downarrow 0.47$\\
     \midrule
 \multirow{3}{*}{\(UniLog_d{}_s\)} & Short& 41.19& 59.00& 4.11&0.63& 16.21& 9.93&24.51\\
 & Long& 23.10& 66.38& 9.06&0.54& 16.19& 17.40&32.77\\
 & $\Delta$ & $\downarrow 18.09$& $\uparrow7.38$& $\uparrow4.95$&$\downarrow 0.09$& $\downarrow 0.03$& $\uparrow7.47$&$\uparrow8.26$\\
     \midrule
 \multirow{3}{*}{\(UniLog_c{}_l\)} & Short& 27.78& 51.07& 2.64&0.77& 17.53& 8.86&30.24\\
 & Long& 8.73& 50.59& 2.94&0.82& 13.39& 8.44&24.58\\
 & $\Delta$ & $\downarrow 19.05$ & $\downarrow 0.48$& $\uparrow 0.30$&$\uparrow 0.09$& $\downarrow 4.14$& $\downarrow 0.42$&$\downarrow 5.66$\\
     \midrule
 \multirow{3}{*}{\textit{LANCE}} & Short& 46.44& 48.67& 4.01&1.56& 19.69& 8.65&20.19\\
 & Long& 0& 0&0& 0& 0& 0&0\\
 & $\Delta$ & - & -  &-& - & - & -  &-\\
     \midrule
 \multirow{3}{*}{\(LEONID_S\)} & Short& 14.54& 23.11& 2.53&3.43& 10.99& 4.45&8.64\\
 & Long& 0& 0&0& 0&  0& 0&0\\
 & $\Delta$ & - & -  &-& - & - & -  &-\\
     \bottomrule
\end{tabular}
}
\end{table*}
% Following previous studies~\cite{Mastropaolo2022UsingDL, Xu2024UniLogAL,xie2024fastlog,Mastropaolo2023LogSG}, we first processed our dataset to create evaluation pairs \(<M_s, M_t>\) with \(M_s\) representing the input provided to the model (\ie \(M_s\) with one removed log statement) and \(M_t\) being the expected output (\ie \(M_t\) is the removed log statement). After processing, we got 39,600 instances. 
We evaluated four automatic logging tools using \methodname, leveraging 39,600 static evaluation pairs.
The evaluation results are presented in Table~\ref{table:static_evaluation_results}. 
% The PA, LA, and MA metrics align with prior evaluation frameworks from the selected logging tools, with values directly sourced from their original studies. Our analysis reveals substantial disparities in effectiveness across the four evaluated tools, underscoring variability in their logging capabilities.
All logging tools show average accuracy drops of 37.49\%, 23.43\%, and 15.80\% in predicting log position, level, and message compared to their reported results. Most strikingly, \(LEONID_S\) collapses from 31.55 → 1.96 in MA (\(\downarrow93.8\%\)) and \(UniLog_{cl}\) plummets from 76.90 → 23.49 in PA (\(\downarrow69.5\%\)). Those drops expose severe overfitting to narrow, synthetic datasets in prior studies.
When evaluated on \methodname, state-of-the-art tools achieved limited scores of 18.00 (DEA), 20.14 (BLEU-4), and 29.88 (ROUGE-L) in STS. These results highlight persistent limitations in generating semantically coherent natural language descriptions and reliably identifying contextually relevant dynamic content for logging.

% Despite these shortcomings, \textit{FastLog} shows only minor deviations from previous studies, maintaining stable performance across all metrics. In contrast, the other three tools, particularly \textit{LEONID}, experience a significant decline in accuracy across all evaluated metrics.

% These factors collectively contribute to the observed instability in logging tool performance across different projects.  As demonstrated in our results, the variability in tool performance across different projects underscores the necessity of diverse evaluation data to capture the full spectrum of a tool’s capabilities and limitations.

To investigate the performance discrepancy relative to prior studies, we partitioned the dataset into two subsets: instances with sub-512 token lengths and those exceeding 512 tokens. This division enables us to determine whether instance length contributes to the performance discrepancies observed among the logging tools, providing further insight into how different data characteristics influence tool effectiveness.
After dividing the data, we obtained two groups: long data containing 8,925 instances and short data with 30,675 instances. The evaluation results in Table~\ref{table:static_len_compare} show a clear difference in tool performance based on the length of the instances.
Notably, \textit{LANCE} and \textit{LEONID} struggled with instances longer than 512 tokens, failing to generate syntax-correct code and, in some cases, producing incomplete code. This explains why their scores for these cases are reported as zero, highlighting the input length limitations of both tools and their inability to handle longer, more complex instances effectively.
\(UniLog_{cl}\) and \textit{FastLog} show a considerable drop in PA and DEA when handling longer data, indicating that they struggle to predict log positions and select dynamic variables in more complex instances. \(UniLog_{ds}\) exhibits a significant decline in positional accuracy but only marginal degradation in dynamic variable identification. Notably, it demonstrates stronger performance on metrics such as MA and STS when processing longer input sequences. This observation implies that enhancing a model’s architectural foundation and providing richer contextual data can improve its capacity to generate syntactically and semantically coherent logging statements through greater contextual awareness.

Although \(UniLog_{ds}\) is equipped with a more powerful backbone, it still struggles to choose the appropriate position for the log statement. As widely demonstrated that the LLMs are not good at counting numbers~\cite{Ahn2024LargeLM}, making LLM generate the exact line number might not be a wise choice. Furthermore, by simply analyzing the control flow graph of the code, we might be able to exclude positions where logging is not feasible to leave fewer choices.
    \begin{figure}[t]
    \centering
    \includegraphics[width=1.0\textwidth]{img/combined_chart.pdf}
    \caption{Performance of logging tools among different projects. The performance of each tool varies considerably across different projects and the trends of all methods across all projects generally remain consistent. }
    % \modify{clarify why only BLEU (no ROUGE)}}
     % \vspace{-5pt}  % 减少图像下方的空白
         \label{fig:project_diveristy}
    \end{figure}


To evaluate tool stability, we analyzed performance consistency across individual projects. As shown in Figure~\ref{fig:project_diveristy}, tool performance exhibits significant variability across projects. For instance, in Flink, the state-of-the-art tool achieves an MA score of 13.68, contrasting sharply with the overall project average of 6.93. Similarly, in Keycloak, the state-of-the-art DEA reaches 43.84 compared to the aggregate average of 16.60. These stark discrepancies underscore inconsistencies in the reliability and generalizability of logging tools when applied to projects with distinct requirements and contexts.


\begin{tcolorbox}
    \textbf{Answer to RQ1.} 
    All logging tools show average accuracy drops of 37.49\%, 23.43\%, and 15.80\% in
predicting log position, level, and message compared to their reported results.
    All logging tools exhibit persistent limitations in generating semantically coherent log descriptions and reliably identifying contextually relevant dynamic variables. Moreover, they fail to achieve consistent performance reliability across diverse logging requirements and project-specific contexts.
    % Since LLMs struggle with counting, it might be better to avoid outputting exact line numbers and instead add tags in the source code. Additionally, analyzing the control graph to exclude impossible positions might be useful to enhance log position determination.
\end{tcolorbox}

\begin{table}[h]
    \centering
    \begin{minipage}{0.45\textwidth}
\caption{Compilation Failure Rates Across Methods.}
\label{table:compile_failed_rate}
\centering
\setlength\tabcolsep{12pt}  % 设置列间距为12pt
\renewcommand{\arraystretch}{1.1}  % 设置行间距为1.2倍
\scalebox{1.1} {  % 将表格宽度设为页面宽度
\begin{tabular}{c|c}
    \toprule
% \textbf{Methods} & \textbf{Compilation Success Rates} \\     
\textbf{Methods} & \textbf{CSR} \\  
\midrule
\textit{\textbf{FastLog}} & \textbf{79.9\%} \\ 
\(UniLog_{{cl}}\) & 70.3\% \\ 
\(UniLog_{{ds}}\) & 60.2\% \\ 
\textit{LANCE} & 49.4\% \\ 
\(LEONID_M\) & 25.0\% \\
\(LEONID_S\) & 16.4\% \\ 
\bottomrule
\end{tabular}
}
    \end{minipage}
    \hfill
    \begin{minipage}{0.5\textwidth}
\caption{Compilation Failure Reasons Analysis.}
\label{table:compile_failed_reason}
\centering
\setlength\tabcolsep{12pt}  % 设置列间距为12pt
\renewcommand{\arraystretch}{1.1}  % 设置行间距为1.2倍
\scalebox{1} {  % 将表格宽度设为页面宽度
\begin{tabular}{c|c}
    \toprule
\textbf{Failed Reason} & \textbf{\# Failed} \\     \midrule
Using Wrong Logging Name & 56 \\ 
Using Undefined Method & 21 \\ 
Using Undefined Variables & 15 \\ 
Incompatible Types & 3 \\
Unreachable Statements & 1 \\ 
Other & 4 \\     \midrule
Total & 100 \\
\bottomrule
\end{tabular}
}
    \end{minipage}
\end{table}

\subsection{RQ2: How compilable are the log statements generated by automatic logging tools?}
\label{rq2}
To evaluate the tools’ ability to generate compilable log statements, we replace existing log statements with predicted ones and recompile the project to check for successful compilation. As shown in Table~\ref{table:compile_failed_rate},
the best-performing tool, \textit{FastLog}, achieves a 79.9\% Compilation Success Rate, followed by \(UniLog_cl\) at 70.3\%, \(UniLog_ds\) at 60.2\%, \textit{LANCE} at 49.4\%, \(LEONID_M\) and \(LEONID_S\) at 25.0\% and 16.4\%, respectively.
The results reveal a key limitation of \textit{LANCE} and \textit{LEONID} that they regenerate the entire code snippet, which increases the risk of unintended code changes and can potentially lead to compilation errors. In contrast, \textit{FastLog} and \textit{UniLog} focus solely on generating the new log statement, minimizing the risk of errors by limiting codebase modifications. Although \textit{FastLog} is one of the best tools available, it still leads to significant instances (20.1\%) of compile failure. 

% When log statements fail to compile, they can prevent the system from functioning as intended and complicate debugging or error tracking. To ensure the reliability of automatic logging tools in real-world environments, it's crucial that the log statements they generate integrate smoothly into the existing codebase.

% Given the importance of reliable compilation,

To understand the reason for the compilation failures reason, we conducted a manual review of the compilation failures to identify the specific causes. 
% This analysis will provide further insight into the key weaknesses of these tools and potential areas for improvement. 
We randomly selected 100 failed instances from the best tool, \textit{FastLog}, and the first two authors cross-checked the causes. The analysis results are presented in Table~\ref{table:compile_failed_reason}. The most common failure, occurring in 56 instances, is due to using the wrong logging name~(\ie \textit{Logger}, \textit{LOG}), indicating that incorrect or non-existent log functions are being invoked. Using undefined methods accounts for 21 failures, followed by using undefined variables with 15 failures. Less frequent issues include incompatible types (3), and unreachable statements (1). Others mean generating the wrong syntax code, which we will not analyze.

The majority of failures (totaling 92 instances) involve undefined references to methods (21 instances), variables (15 instances), or logging names (56 instances). These undefined reference failures are primarily due to the limited context provided by existing function-level logging tools, lacking critical details on valid variables, methods, libraries, and packages relevant to the target function. Current tools are designed for function-level input, highlighting the need for logging tools to integrate better context awareness and validation checks to ensure compatibility with the existing codebase. The less frequent errors are also noticeable. For instance, the issue of unreachable statements points to a weakness in that tools lack an understanding of the code’s control flow, resulting in log statements placed in non-executable paths. Collectively, these errors highlight the need for automated logging tools to adopt enhanced context-aware strategies, improving the accuracy and contextual relevance of generated logs.


% Figure~\ref{fig:unreachable_statements} shows an example of unreachable statements, the code contains a \textit{while (true)} loop with a conditional check for \textit{shutdown}. If the \textit{shutdown} condition is true, the program will execute the return statement, which will immediately exit the method, preventing any further code from being executed. As a result, the log statement at the bottom \textit{LOG.info("Started CacheReplicationMonitor.")} becomes unreachable because the return statement ensures that the method terminates before reaching this log statement. This example shows that not all locations are ideal for log statements. By carefully analyzing the control graph, we can narrow down suitable positions, helping us make better choices for log placement.


% \textbf{ \textit{Findings:}} The low compilation success rates suggest that automatic logging tools should incorporate more advanced context-awareness mechanisms. Given that current tools are designed for function-level input, we could either extend them to class-level or use static analysis to enable variable scope tracking, control flow analysis, and accurate type-checking. Especially, by analyzing the control graph, we can exclude unsuitable positions, helping us decide more effectively where to log.
\begin{tcolorbox}
    \textbf{Answer to RQ2.} \textsc{FastLog} achieved the best performance in generating compilable log statements, yet over 20\% of the generated log statements still failed to be compiled. According to our analysis of compilation failure reasons, these failures primarily stem from a lack of critical contexts corresponding to the target function, e.g., valid variables, methods, libraries, packages, execution paths, and type information. To improve the reliability of automatic logging tools, it is crucial that they incorporate mechanisms to gather and utilize this additional context during the log statement generation process.
\end{tcolorbox}
 
\subsection{RQ3: How consistent are runtime logs from auto-logged programs with oracle-generated logs?}
\label{rq3}
To answer RQ3, we compare the semantic similarity of logs generated by predicted log statements and by the oracle.
It is important to highlight that, aside from \(LEONID_M\), all other tools operate under the strong assumption that the given code snippet requires exactly one log statement. 
% To highlight this inappropriate assumption, we only allow the tools one chance to predict the log statement, even in cases where multiple log statements might be needed. 
To highlight the inappropriateness of this assumption, we only allow the tools one chance to predict the log statement, even in cases where multiple log statements are needed. 
Our experimental design emulates real-world logging scenarios by abandoning the assumption that single log statements universally suffice, thereby exposing a critical limitation in existing tools' capacity to determine contextually appropriate logging density for a given code context.

\begin{table*}[htbp]
\caption{ The semantic similarity between logs printed by the predicted log statements and the ground truth. }
\label{table:log_similarity}
\centering
\setlength{\tabcolsep}{6pt} % 调整列间距
\scalebox{0.9}{ % 适当缩放表格
\begin{tabular}{lcccccccccc}
    \toprule
    \textbf{Method} & \textbf{COS(\%)} & \textbf{BLEU-1} & \textbf{BLEU-2}& \textbf{BLEU-3}& \textbf{BLEU-4}& \textbf{ROUGE-1} & \textbf{ROUGE-2} & \textbf{ROUGE-L}\\
    \midrule
 \(FastLog\)       & \textbf{21.32}                 & \textbf{24.83}  &\ \textbf{19.62}  &\ \textbf{17.49}        & \textbf{15.96}            & \textbf{24.43}    &\ \textbf{18.74}      & \textbf{23.84}           \\ 
 \(UniLog_{{cl}}\)         & 17.38                  & 19.02    &\ 15.62  &\ 14.34        & 13.42          & 19.17     &\ 15.37       & 18.75           \\ 
\(UniLog_{{ds}}\) & 13.04                 & 14.67  &\ 12.72  &\ 12.02          & 11.56          & 14.00      &\ 12.23      & 13.87           \\ 
\(LANCE\)          & 9.93                   & 11.34   &\ 9.21  &\ 8.50         & 7.97           & 11.23      &\ 8.94      & 11.03           \\ 
 \(LEONID_M\)      & 7.19                  & 7.95    &\ 6.51  &\ 5.87         & 5.30           & 8.10      &\ 6.43       & 7.94            \\ 
 \(LEONID_S\)         & 4.45                 & 4.91     &\ 3.82  &\ 2.95        & 2.96           & 5.10        &\ 3.86     & 5.01            \\
\bottomrule
\end{tabular}
}
\end{table*}

Table~\ref{table:log_similarity} compares the semantic similarity between logs from source and predicted log statements.
The results indicate that logs from the predicted statements significantly deviate from the ground truth, with consistently low scores across metrics like Cosine Similarity, BLEU, and ROUGE.
 Limiting each code snippet to a single log statement often sharply compromises log similarity, especially when multiple statements are needed. This limitation is particularly clear when comparing \(LEONID_S\) and \(LEONID_M\). Although \(LEONID\) overall performs poorly, \(LEONID_M\) stands out as the only tool capable of generating multiple log statements, which enables it to outperform \(LEONID_S\) in similarity scores. This difference underscores the importance of tools being able to determine the appropriate number of log statements for accurate and effective logging, rather than assuming a single statement suffices.
% 新增的表格
\begin{table*}[htbp]
\centering
\begin{minipage}{0.48\textwidth}
\centering
\caption{False Positive and False Negative Logging Rates across Methods. \textbf{FPLR} means the predicted log statement record logs when it should not, and \textbf{FNLR} means the predicted log statement does not record logs when it should.}
\label{tab:table_FPLG_FNLG}
\setlength{\tabcolsep}{8pt} % 调整列间距
\begin{tabular}{lcc}
    \toprule
    \textbf{Method} & \textbf{FPLR} & \textbf{FNLR} \\
    \midrule
        \textit{FastLog} & 9.28\% & 18.28\% \\
        \(UniLog_{{cl}}\)  &	6.52\% &	30.59\% \\
        \(UniLog_{{ds}}\)  &	3.21\% &	 22.88\%\\
    \textit{LANCE}   & 5.71\% & 19.29\% \\
    \(LEONID_S\) &	8.15\% & 8.69\% \\
        \(LEONID_M\)  &	7.32\% & 11.60\% \\
    \bottomrule
\end{tabular}
\end{minipage}\hfill
\begin{minipage}{0.48\textwidth}
\centering
\caption{\textbf{FPLR} and \textbf{FNLR} Reason Analysis. For FNLR, the major reason is beyond the execution path, and for FPLR, the major reason is lower verbosity level.}
\label{tab:table_FPLG_FNLG_Reason}
\centering
\setlength\tabcolsep{12pt}  % 设置列间距为12pt
\renewcommand{\arraystretch}{1.1}  % 设置行间距为1.2倍
\scalebox{0.8} {  % 将表格宽度设为页面宽度
\begin{tabular}{c|c|c}
    \toprule
\textbf{Situation} & \textbf{Reason} & \textbf{\# False} \\     \midrule
\multirow{3}{*}{\textbf{FNLR}}& Beyond Execution Path & 35  \\ 
& Lower Verbosity Level & 24\\ 
& Wrong Code Format & 4 \\ \midrule
\multirow{2}{*}{\textbf{FPLR}} & Higher Verbosity Level & 30 \\ 
& Beyond Execution Path & 3 \\    \midrule
Total & -  & 100 \\
\bottomrule
\end{tabular}
}
\end{minipage}
\end{table*}


To understand the low semantic similarity scores, we examined the log generation process and found many instances where predicted logs were either redundant or missed key information present in the original logs. We quantify this issue using two metrics: FPLR and FNLR, as reported in Table~\ref{tab:table_FPLG_FNLG}. For example, \textit{FastLog} reports a 9.28\% FPLR, meaning logs record redundant information in 9.28\% of cases, and an 18.28\% FNLR, indicating expected information in logs was missing in 18.28\% of cases. We sampled 100 examples from FastLog, the state-of-the-art (SOTA) model, and manually analyzed the reasons for mismatches with the original logs. The results are presented in Table~\ref{tab:table_FPLG_FNLG_Reason}. We found that the primary reasons for failures in FNLR and FPLR differ significantly. For FNLR, the most common issue was the predicted log statements being beyond the execution path (35 cases), followed by lower verbosity levels (24 cases), and a smaller number caused by wrong code format (4 cases). In contrast, for FPLR, the main problem was higher verbosity levels (30 cases), with a few instances of log statements being beyond the execution path (3 cases). Overall, verbosity mismatches and execution path discrepancies were the dominant contributors, highlighting challenges in aligning predicted logs with actual logging requirements. The factors leading to FPLR and FNLR underscore a critical issue: while static metrics offer valuable insights into the quality of generated log statements, the actual logs are shaped by numerous contextual factors. Without a thorough understanding of the execution context, it is not possible to comprehensively evaluate the quality of log statements. Even minor discrepancies can cause significant deviations between generated logs and source logs. For instance, while the predicted log statement may capture the key information required to reflect system behavior, its effectiveness can be compromised if it is not positioned along a critical execution path or if its verbosity level is mismatched. In such cases, the log statement may fail to record essential information when key events occur. This issue majorly arises from the tools’ lack of awareness of verbosity thresholds and the control graph of the code, which limits their ability to adjust verbosity and determine appropriate log positions based on the context or execution requirements. These two limitations highlight that current logging tools lack the adaptability and context-awareness needed for effective real-world applications.

\begin{tcolorbox}
    \textbf{Answer to RQ3.} The best-predicted log statements by \textit{FastLog} achieve only 21.32\% cosine similarity with the original logs. Many predictions record redundant information, while others miss key details. The missing key information is primarily due to the prediction being placed beyond the execution path during important events, while setting a higher verbosity level in the log statements leads to redundancy. This result highlights that automatic logging tools still have significant room for improvement.
\end{tcolorbox}





\section{Conclusion}
In this paper, we present \dsns, a challenging VQA benchmark that requires visual knowledge as evidence to answer questions. Different from previous works that include images as part of queries and let models retrieve textual knowledge, our \ds is formed with text-only queries and enforces text-to-image retrieval for challenging visual knowledge. We evaluate 8 popular MLLMs under various multimodal RAG settings. The results indicate that images can serve as knowledge sources for answering visual knowledge intensive questions, but the models struggle to effectively extract visual knowledge from a single clue image. Surprisingly, the current models are able to reason across clue and non-clue images for better understanding of the questions. As one of the pioneer works in visual-knowledge centric multimodal RAG, our benchmark can help promote the research and development of MLLMs in extracting fine-grained visual knowledge and enhance cross-modal retrieval in a more realistic setting.

\section*{Limitations}

While our \ds  addresses the lack of visual knowledge intensive QA benchmarks, it has several design constraints. First, our query generation and manual rewriting process focus on single-hop questions only. Although a small subset of queries might involve simple comparisons within the same image, such as checking the colour of a bird’s upper beak vs. lower beak, there are no truly multi-hop queries requiring reasoning across multiple images. Designing such multi-image or even cross-species queries would demand significantly more expert knowledge, and we leave this as an avenue for future work.

Secondly, \ds focuses exclusively on organisms, which raises questions about the broader generalizability of our findings. We chose the domain of organisms because they allow for factual queries regarding visual features that remain consistent across multiple individuals of a species, yet are present in only a minority of images. Other commonly used domains in knowledge-intensive VQA such as buildings, artworks, news event, etc., do not always meet these criteria, as answers may not hold universally for all instances or there exists only one instance, and the required visual features are often too prevalent or too trivial to foster a genuinely challenging retrieval scenario.

Lastly, the overall scale of \ds is limited compared to larger benchmarks like InfoSeek and Encyclopedic-VQA. Their heuristic-driven methods for automatically generating queries and answers do not readily apply to our specialized setting, which hinges on expert-informed checks to ensure the presence (or absence) of specific visual features. Despite these constraints, \ds represents an important step toward truly visual knowledge intensive QA, and we hope it will inspire further research into more complex, multi-hop, and cross-domain tasks.

\section*{Ethical Considerations}
\paragraph{Licensing}
We do not own, nor do we redistribute the images of iNaturalist 2021 dataset. \ds only provides the annotations of queries and answers, as well as image labels on whether it is clue to query, linked to image GUID in iNatrualist 2021 dataset. All images in iNat21 dataset are shared under one of the following Creative Commons licenses, in particular: CC BY 4.0, CC BY-NC 4.0, CC BY-NC-ND 4.0, CC BY-NC-SA 4.0, CC0 1.0, CC BY-ND 4.0, CC BY-SA 4.0. Any usage of the images is subject to iNaturalist and iNaturalist 2021 dataset's term of use. We strictly adhere to the intended non-commercial research use granted by the iNaturalist community and iNaturalist 2021 dataset, and we share our annotation in \ds under CC BY-NC 4.0 license. The authors assume no responsibility for any non-compliant usage or legal/ethical issues related to the original iNaturalist 2021 dataset.

\paragraph{Potential Risks}
The annotation in this dataset, i.e. the queries, answers and image labels, are intended for research purposes only. Our dataset’s queries, answers, and image labels are provided by volunteer annotators who are not professional taxonomists or biologists. We make every effort to ensure correctness, such as proactively consulting publicly available expertised resources of biology and taxonomy over internet, and annotators are undergraduate students who possess sufficient English proficiency to utilize such publicly available resources for fact validation. Though, we cannot guarantee the absence of factual errors. We make no guarantees regarding completeness or correctness for real-world decision-making.

Our annotation does not contain any personally identifiable information, nor does it reference private data, locations, or individuals. Our annotation does not reveal exact geographical coordinates or sensitive ecological information that might pose risks to endangered species or protected habitats. For other potential risks regarding images in iNaturalist 2021 dataset, please refer to the original paper and iNaturalist website (www.inaturalist.org).

%%
%% The next two lines define the bibliography style to be used, and
%% the bibliography file.
\bibliographystyle{ACM-Reference-Format}
\bibliography{ref}

\end{document}


