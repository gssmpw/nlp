%%%%%%%% ICML 2025 EXAMPLE LATEX SUBMISSION FILE %%%%%%%%%%%%%%%%%

\documentclass[accepted]{article}

% Recommended, but optional, packages for figures and better typesetting:
\usepackage{microtype}
\usepackage{graphicx}
\usepackage{subfigure}
\usepackage{booktabs} % for professional tables

% hyperref makes hyperlinks in the resulting PDF.
% If your build breaks (sometimes temporarily if a hyperlink spans a page)
% please comment out the following usepackage line and replace
% \usepackage{icml2025} with \usepackage[nohyperref]{icml2025} above.
\usepackage{hyperref}


% Attempt to make hyperref and algorithmic work together better:
\newcommand{\theHalgorithm}{\arabic{algorithm}}

% Use the following line for the initial blind version submitted for review:
\usepackage{icml2025}

% If accepted, instead use the following line for the camera-ready submission:
% \usepackage[accepted]{icml2025}

% For theorems and such
\usepackage{amsmath}
\usepackage{amssymb}
\usepackage{mathtools}
\usepackage{amsthm}
\usepackage{makecell}

% Custom packages
\usepackage{amsfonts}
\usepackage{multirow}
\usepackage{ulem}
\newcommand{\note}[1]{{\color{red} #1}}
\newcommand{\notedel}[1]{{\color{red} \sout{#1}}}
\newcommand{\noteadd}[1]{{\color{orange} #1}}
\newcommand{\purple}[1]{{\color{purple} #1}}
\def\sysname{Lumina-Video}

% if you use cleveref..
\usepackage[capitalize,noabbrev]{cleveref}

%%%%%%%%%%%%%%%%%%%%%%%%%%%%%%%%
% THEOREMS
%%%%%%%%%%%%%%%%%%%%%%%%%%%%%%%%
\theoremstyle{plain}
\newtheorem{theorem}{Theorem}[section]
\newtheorem{proposition}[theorem]{Proposition}
\newtheorem{lemma}[theorem]{Lemma}
\newtheorem{corollary}[theorem]{Corollary}
\theoremstyle{definition}
\newtheorem{definition}[theorem]{Definition}
\newtheorem{assumption}[theorem]{Assumption}
\theoremstyle{remark}
\newtheorem{remark}[theorem]{Remark}

\usepackage[textsize=tiny]{todonotes}

\icmltitlerunning{Lumina-Video: Efficient and Flexible Video Generation with Multi-scale Next-DiT}

\begin{document}

\twocolumn[
\icmltitle{Lumina-Video: Efficient and Flexible Video Generation \\ with Multi-scale Next-DiT}

\icmlsetsymbol{equal}{$\textbf{*}$} %{*}
\icmlsetsymbol{corresp}{$\P$} %{$\spadesuit$}%{†}
\icmlsetsymbol{lead}{$\blacklozenge$} %{‡}
\icmlsetsymbol{sep}{\!\!,\!} 

\begin{icmlauthorlist}
\icmlauthor{Dongyang Liu}{equal,CUHK,AILAB}
\icmlauthor{Shicheng Li}{equal,AILAB}
\icmlauthor{Yutong Liu}{equal,AILAB}
\icmlauthor{Zhen Li}{equal,CUHK}
\icmlauthor{Kai Wang}{equal,AILAB}
\icmlauthor{Xinyue Li}{equal,AILAB}
\icmlauthor{Qi Qin}{AILAB}
\icmlauthor{Yufei Liu}{AILAB}
\icmlauthor{Yi Xin}{AILAB}
\icmlauthor{Zhongyu Li}{AILAB,NKU}
\icmlauthor{Bin Fu}{AILAB}
\icmlauthor{Chenyang Si}{AILAB}
\icmlauthor{Yuewen Cao}{AILAB}
\icmlauthor{Conghui He}{AILAB}
\icmlauthor{Ziwei Liu}{AILAB}
\icmlauthor{Yu Qiao}{AILAB}
\icmlauthor{Qibin Hou}{NKU}
\icmlauthor{Hongsheng Li}{corresp,CUHK,AILAB}
\icmlauthor{Peng Gao}{corresp,lead,AILAB}
\end{icmlauthorlist}

% \icmlaffiliation{CUHK}{The Chinese University of Hong Kong, Hong Kong, China}
% \icmlaffiliation{AILAB}{Shanghai AI Laboratory, Shanghai, China}
% \icmlaffiliation{NKU}{Nankai University, Tianjin, China}

\icmlaffiliation{CUHK}{The Chinese University of Hong Kong}
\icmlaffiliation{AILAB}{Shanghai AI Laboratory}
\icmlaffiliation{NKU}{Nankai University}

\icmlcorrespondingauthor{Peng Gao}{gaopeng@pjlab.org.cn}
\icmlcorrespondingauthor{Hongsheng Li}{hsli@ee.cuhk.edu.hk}

% \icmlaffiliation{yyy}{Department of XXX, University of YYY, Location, Country}
% \icmlaffiliation{comp}{Company Name, Location, Country}
% \icmlaffiliation{sch}{School of ZZZ, Institute of WWW, Location, Country}

% You may provide any keywords that you
% find helpful for describing your paper; these are used to populate
% the "keywords" metadata in the PDF but will not be shown in the document
\icmlkeywords{Machine Learning, ICML}

\vskip 0.3in
]
    
% this must go after the closing bracket ] following \twocolumn[ ...

% This command actually creates the footnote in the first column
% listing the affiliations and the copyright notice.
% The command takes one argument, which is text to display at the start of the footnote.
% The \icmlEqualContribution command is standard text for equal contribution.
% Remove it (just {}) if you do not need this facility.

%\printAffiliationsAndNotice{}  % leave blank if no need to mention equal contribution
% \printAffiliationsAndNotice{\icmlEqualContribution} % otherwise use the standard text.

\printAffiliationsAndNotice{%
    %\icmlEqualContribution
    \textsuperscript{\textbf{*}}Equal Contribution
    \textsuperscript{$\P$}Corresponding Authors
    \textsuperscript{$\blacklozenge$}Project Lead}


\begin{abstract}


The choice of representation for geographic location significantly impacts the accuracy of models for a broad range of geospatial tasks, including fine-grained species classification, population density estimation, and biome classification. Recent works like SatCLIP and GeoCLIP learn such representations by contrastively aligning geolocation with co-located images. While these methods work exceptionally well, in this paper, we posit that the current training strategies fail to fully capture the important visual features. We provide an information theoretic perspective on why the resulting embeddings from these methods discard crucial visual information that is important for many downstream tasks. To solve this problem, we propose a novel retrieval-augmented strategy called RANGE. We build our method on the intuition that the visual features of a location can be estimated by combining the visual features from multiple similar-looking locations. We evaluate our method across a wide variety of tasks. Our results show that RANGE outperforms the existing state-of-the-art models with significant margins in most tasks. We show gains of up to 13.1\% on classification tasks and 0.145 $R^2$ on regression tasks. All our code and models will be made available at: \href{https://github.com/mvrl/RANGE}{https://github.com/mvrl/RANGE}.

\end{abstract}


\section{Introduction}
Backdoor attacks pose a concealed yet profound security risk to machine learning (ML) models, for which the adversaries can inject a stealth backdoor into the model during training, enabling them to illicitly control the model's output upon encountering predefined inputs. These attacks can even occur without the knowledge of developers or end-users, thereby undermining the trust in ML systems. As ML becomes more deeply embedded in critical sectors like finance, healthcare, and autonomous driving \citep{he2016deep, liu2020computing, tournier2019mrtrix3, adjabi2020past}, the potential damage from backdoor attacks grows, underscoring the emergency for developing robust defense mechanisms against backdoor attacks.

To address the threat of backdoor attacks, researchers have developed a variety of strategies \cite{liu2018fine,wu2021adversarial,wang2019neural,zeng2022adversarial,zhu2023neural,Zhu_2023_ICCV, wei2024shared,wei2024d3}, aimed at purifying backdoors within victim models. These methods are designed to integrate with current deployment workflows seamlessly and have demonstrated significant success in mitigating the effects of backdoor triggers \cite{wubackdoorbench, wu2023defenses, wu2024backdoorbench,dunnett2024countering}.  However, most state-of-the-art (SOTA) backdoor purification methods operate under the assumption that a small clean dataset, often referred to as \textbf{auxiliary dataset}, is available for purification. Such an assumption poses practical challenges, especially in scenarios where data is scarce. To tackle this challenge, efforts have been made to reduce the size of the required auxiliary dataset~\cite{chai2022oneshot,li2023reconstructive, Zhu_2023_ICCV} and even explore dataset-free purification techniques~\cite{zheng2022data,hong2023revisiting,lin2024fusing}. Although these approaches offer some improvements, recent evaluations \cite{dunnett2024countering, wu2024backdoorbench} continue to highlight the importance of sufficient auxiliary data for achieving robust defenses against backdoor attacks.

While significant progress has been made in reducing the size of auxiliary datasets, an equally critical yet underexplored question remains: \emph{how does the nature of the auxiliary dataset affect purification effectiveness?} In  real-world  applications, auxiliary datasets can vary widely, encompassing in-distribution data, synthetic data, or external data from different sources. Understanding how each type of auxiliary dataset influences the purification effectiveness is vital for selecting or constructing the most suitable auxiliary dataset and the corresponding technique. For instance, when multiple datasets are available, understanding how different datasets contribute to purification can guide defenders in selecting or crafting the most appropriate dataset. Conversely, when only limited auxiliary data is accessible, knowing which purification technique works best under those constraints is critical. Therefore, there is an urgent need for a thorough investigation into the impact of auxiliary datasets on purification effectiveness to guide defenders in  enhancing the security of ML systems. 

In this paper, we systematically investigate the critical role of auxiliary datasets in backdoor purification, aiming to bridge the gap between idealized and practical purification scenarios.  Specifically, we first construct a diverse set of auxiliary datasets to emulate real-world conditions, as summarized in Table~\ref{overall}. These datasets include in-distribution data, synthetic data, and external data from other sources. Through an evaluation of SOTA backdoor purification methods across these datasets, we uncover several critical insights: \textbf{1)} In-distribution datasets, particularly those carefully filtered from the original training data of the victim model, effectively preserve the model’s utility for its intended tasks but may fall short in eliminating backdoors. \textbf{2)} Incorporating OOD datasets can help the model forget backdoors but also bring the risk of forgetting critical learned knowledge, significantly degrading its overall performance. Building on these findings, we propose Guided Input Calibration (GIC), a novel technique that enhances backdoor purification by adaptively transforming auxiliary data to better align with the victim model’s learned representations. By leveraging the victim model itself to guide this transformation, GIC optimizes the purification process, striking a balance between preserving model utility and mitigating backdoor threats. Extensive experiments demonstrate that GIC significantly improves the effectiveness of backdoor purification across diverse auxiliary datasets, providing a practical and robust defense solution.

Our main contributions are threefold:
\textbf{1) Impact analysis of auxiliary datasets:} We take the \textbf{first step}  in systematically investigating how different types of auxiliary datasets influence backdoor purification effectiveness. Our findings provide novel insights and serve as a foundation for future research on optimizing dataset selection and construction for enhanced backdoor defense.
%
\textbf{2) Compilation and evaluation of diverse auxiliary datasets:}  We have compiled and rigorously evaluated a diverse set of auxiliary datasets using SOTA purification methods, making our datasets and code publicly available to facilitate and support future research on practical backdoor defense strategies.
%
\textbf{3) Introduction of GIC:} We introduce GIC, the \textbf{first} dedicated solution designed to align auxiliary datasets with the model’s learned representations, significantly enhancing backdoor mitigation across various dataset types. Our approach sets a new benchmark for practical and effective backdoor defense.



\section{Related Work}

\subsection{Large 3D Reconstruction Models}
Recently, generalized feed-forward models for 3D reconstruction from sparse input views have garnered considerable attention due to their applicability in heavily under-constrained scenarios. The Large Reconstruction Model (LRM)~\cite{hong2023lrm} uses a transformer-based encoder-decoder pipeline to infer a NeRF reconstruction from just a single image. Newer iterations have shifted the focus towards generating 3D Gaussian representations from four input images~\cite{tang2025lgm, xu2024grm, zhang2025gslrm, charatan2024pixelsplat, chen2025mvsplat, liu2025mvsgaussian}, showing remarkable novel view synthesis results. The paradigm of transformer-based sparse 3D reconstruction has also successfully been applied to lifting monocular videos to 4D~\cite{ren2024l4gm}. \\
Yet, none of the existing works in the domain have studied the use-case of inferring \textit{animatable} 3D representations from sparse input images, which is the focus of our work. To this end, we build on top of the Large Gaussian Reconstruction Model (GRM)~\cite{xu2024grm}.

\subsection{3D-aware Portrait Animation}
A different line of work focuses on animating portraits in a 3D-aware manner.
MegaPortraits~\cite{drobyshev2022megaportraits} builds a 3D Volume given a source and driving image, and renders the animated source actor via orthographic projection with subsequent 2D neural rendering.
3D morphable models (3DMMs)~\cite{blanz19993dmm} are extensively used to obtain more interpretable control over the portrait animation. For example, StyleRig~\cite{tewari2020stylerig} demonstrates how a 3DMM can be used to control the data generated from a pre-trained StyleGAN~\cite{karras2019stylegan} network. ROME~\cite{khakhulin2022rome} predicts vertex offsets and texture of a FLAME~\cite{li2017flame} mesh from the input image.
A TriPlane representation is inferred and animated via FLAME~\cite{li2017flame} in multiple methods like Portrait4D~\cite{deng2024portrait4d}, Portrait4D-v2~\cite{deng2024portrait4dv2}, and GPAvatar~\cite{chu2024gpavatar}.
Others, such as VOODOO 3D~\cite{tran2024voodoo3d} and VOODOO XP~\cite{tran2024voodooxp}, learn their own expression encoder to drive the source person in a more detailed manner. \\
All of the aforementioned methods require nothing more than a single image of a person to animate it. This allows them to train on large monocular video datasets to infer a very generic motion prior that even translates to paintings or cartoon characters. However, due to their task formulation, these methods mostly focus on image synthesis from a frontal camera, often trading 3D consistency for better image quality by using 2D screen-space neural renderers. In contrast, our work aims to produce a truthful and complete 3D avatar representation from the input images that can be viewed from any angle.  

\subsection{Photo-realistic 3D Face Models}
The increasing availability of large-scale multi-view face datasets~\cite{kirschstein2023nersemble, ava256, pan2024renderme360, yang2020facescape} has enabled building photo-realistic 3D face models that learn a detailed prior over both geometry and appearance of human faces. HeadNeRF~\cite{hong2022headnerf} conditions a Neural Radiance Field (NeRF)~\cite{mildenhall2021nerf} on identity, expression, albedo, and illumination codes. VRMM~\cite{yang2024vrmm} builds a high-quality and relightable 3D face model using volumetric primitives~\cite{lombardi2021mvp}. One2Avatar~\cite{yu2024one2avatar} extends a 3DMM by anchoring a radiance field to its surface. More recently, GPHM~\cite{xu2025gphm} and HeadGAP~\cite{zheng2024headgap} have adopted 3D Gaussians to build a photo-realistic 3D face model. \\
Photo-realistic 3D face models learn a powerful prior over human facial appearance and geometry, which can be fitted to a single or multiple images of a person, effectively inferring a 3D head avatar. However, the fitting procedure itself is non-trivial and often requires expensive test-time optimization, impeding casual use-cases on consumer-grade devices. While this limitation may be circumvented by learning a generalized encoder that maps images into the 3D face model's latent space, another fundamental limitation remains. Even with more multi-view face datasets being published, the number of available training subjects rarely exceeds the thousands, making it hard to truly learn the full distibution of human facial appearance. Instead, our approach avoids generalizing over the identity axis by conditioning on some images of a person, and only generalizes over the expression axis for which plenty of data is available. 

A similar motivation has inspired recent work on codec avatars where a generalized network infers an animatable 3D representation given a registered mesh of a person~\cite{cao2022authentic, li2024uravatar}.
The resulting avatars exhibit excellent quality at the cost of several minutes of video capture per subject and expensive test-time optimization.
For example, URAvatar~\cite{li2024uravatar} finetunes their network on the given video recording for 3 hours on 8 A100 GPUs, making inference on consumer-grade devices impossible. In contrast, our approach directly regresses the final 3D head avatar from just four input images without the need for expensive test-time fine-tuning.


\section{Methodology}

\subsection{Problem Definition}

Given a multivariate time series input $X \in \mathbb{R}^{C  \times T}$, multivariate time series forecasting tasks are designed to predict its future $F$ time steps $\hat{Y}\in \mathbb{R}^{C \times F}$ using past $T$ steps. $C $ is the number of variates or channels.

\subsection{Preliminary Analysis}

This section presents why RevIN~\citep{Kim_revin,liu2022non}, High-pass, and Low-pass filters fail to address the Mid-Frequency Spectrum Gap. Let the input univariate time series be $ x(t) $ with length $ T $ and target $ y(t) $ with length $ F $. 

\begin{definition}[Frequency Spectral Energy]\label{def:energy}
The Fourier transform of $x(t)$, $X(f)$, and its spectral energy $E_X(f)$ is given by:
\vspace{-0.2cm}
\begin{align}
X(f) = \sum_{t=0}^{T-1} x(t) e^{-i 2 \pi f t / {T-1}}, \quad &f = 0, 1, \dots, T-1\notag\\
E_X(f) = |X(f)|^2.
\end{align}
\vspace{-0.2cm}
\end{definition}

\textbf{Impact of RevIN on Frequency Spectrum \quad}
\begin{definition}[Reversible Instance Normalization]\label{def:RevIN}
Given a \textbf{forecast model} $ f: \mathbb{R}^T \rightarrow \mathbb{R}^F $ that generates a forecast $ \hat{y}(t) $ from a given input $x(t)$, RevIN is defined as:
\vspace{-0.2cm}
\begin{align}
&\hat{x}(t) = \frac{x(t) - \mu}{\sigma},\quad t = 0, 1, \dots, T-1\notag\\
&\hat{y}(t) = f(\hat{x}(t)), \quad \hat{y}(t)_{rev}= \hat{y}(t) \cdot \sigma + \mu,\notag\\
&\mu = \frac{1}{T} \sum_{t=0}^{T-1} x(t), \quad \sigma = \sqrt{\frac{1}{T} \sum_{t=0}^{T-1} (x(t) - \mu)^2}.
\end{align}
\vspace{-0.2cm}
\end{definition}

\begin{theorem} [Frequency Spectrum after RevIN] \label{theorem:RevIN}
\vspace{-0.2cm}
The spectral energy of $\hat{x}(t)$ (transformed using RevIN):
\begin{align}
E_{\hat{X}}(0)=0,& \quad f=0, \notag\\
E_{\hat{X}}(f) = \left( \frac{1}{\sigma} \right)^2 |X(f)|^2,&\quad f = 1,2,\dots, T-1 . 
\end{align}
\vspace{-0.2cm}
\end{theorem}
The proof is in Appendix~\ref{app:RevIN}. Theorem~\ref{theorem:RevIN} suggests that RevIN scales the absolute spectral energy by $ \sigma^2 $ but does not affect its relative distribution except $E_{\hat{X}}(0)=0$. Thus, RevIN preserves the relative spectral energy distribution and leaves the Mid-Frequency Spectrum Gap unresolved. \textit{However, our experiments still employ RevIN to ensure a fair comparison with other baselines.}
\begin{figure*}[h]
  \centering
  \includegraphics[width=1.\linewidth]{Faker/source/assets/jpg/ReFocus.jpg}
  \caption{General structure of \textbf{ReFocus}. `Adaptive Mid-Frequency Energy Optimizer (AMEO)' enhances mid-frequency components modeling, and `Energy-based Key-Frequency Picking Block' (EKPB) effectively captures shared Key-Frequency across channels}
  \label{fig:refocus}
\end{figure*}

\begin{figure*}[h]
  \centering
  \includegraphics[width=0.7\linewidth]{Faker/source/assets/jpg/ket.jpg}
  \caption{General process of the \textbf{Key-Frequency Enhanced Training strategy (KET)}, where spectral information from other channels is randomly introduced into each channel, to enhance the extraction of the shared Key-Frequency.}
  \label{fig:reshuffle}
\end{figure*}
\textbf{Impact of High- and Low-pass filter \quad}
We still define $\hat{x}(t)$ to be the filtered (processed) signal, obtained by applying a filter $H(f)$ (High/Low-pass filter). The filter $ H(f) $ is 1 in the passband (High/Low frequency) and 0 in the stopband (Middle frequency). So $E_{\hat{X}}(f)=0,\quad E_{\hat{X}}\leq E_X(f)$ for middle frequencies, which creates even larger gap.

\subsection{Overall Structure of The Proposed ReFocus}

In this section, we elucidate the overall architecture of \textbf{ReFocus}, depicted in Figure \ref{fig:refocus}. We define frequency domain projection as $D1\rightarrow D2$ representing a projection from dimension $D1$ to $D2$ in the frequency domain~\citep{xu2024fits}. Initially, we apply \textbf{AMEO} to the input $X \in \mathbb{R}^{C \times T}$, yielding the processed spectrum $ X_{am} \in \mathbb{R}^{C  \times T} $. Next, we use a projection $T\rightarrow D$ to transform $ X_{am}$ into the Variate Embedding $ X_{em} \in \mathbb{R}^{C  \times D}$~\citep{LiuiTransformer}. Then, $X_{em}$ go through $N$ \textbf{EKPB} to generate representation $H_{N+1}$, which is projected to obtain final prediction $\hat{Y}$. 

\textbf{Adaptive Mid-Frequency Energy Optimizer \quad}
Building upon the \textbf{Preliminary Analysis}, we propose a convolution- and residual learning-based solution to address the Mid-Frequency Spectrum Gap, which we denoted as AMEO. 
\begin{definition}[Adaptive Mid-Frequency Energy Optimizer]\label{def:AMEO}
AMEO is defined as:
\begin{align}
&\hat{x}(t) = x(t)-\frac{\beta}{K}\sum_{k=0}^{K-1} \tilde{x}(t+K-1-k),\notag\\
&\tilde{x}(t) =\notag\\
&\begin{cases}
x(t-(\frac{K}{2}+1)), \quad \text{if } \frac{K}{2}+1 \leq t < T+\frac{K}{2}+1, \\
0,  \quad\text{if } 0 \leq t < \frac{K}{2}+1 \text{ or } T+\frac{K}{2}+1 \leq t < T+K.
\end{cases}
\end{align}
\vspace{-0.2cm}
\end{definition}

It is equivalent to $x=x-\beta \cdot Conv(x)$. $Conv$ is a 1D convolution (Zero-padding at both ends, stride $s=1$, kernel size $K$, with values initialized as $ \frac{1}{K} $). $\beta \in \mathbb{R}^{1}$ is a hyperparameter.

\begin{theorem} [Frequency Spectrum after AMEO] \label{theorem:AMEO}
The spectral energy of $\hat{x}(t)$ obtained using AMEO:
\begin{align}
E_{\hat{X}}(f) =|X(f)|^2 \left\{1 - \beta \cdot \underbrace{\frac{1}{K} \sum_{k=0}^{K-1} e^{i 2 \pi f (\frac{3K}{2}-k -2) / {T-1}}}_{G(f)}\right\}^2
\end{align}
\vspace{-0.2cm}
\end{theorem}

The proof is in Appendix~\ref{app:AMEO}. We have $E_{\hat{X}}(f) =|X(f)|^2(1-\beta  \cdot G(f))^2$. Generally, $ G(f) $ behaves as a decay function, gradually reducing its value from \textbf{One} to \textbf{Zero}. Such \textbf{decay behavior} makes AMEO relatively enhances mid-frequency components, thus addressing the Mid-Frequency Spectrum Gap.

\textbf{Energy-based Key-Frequency Picking Block \quad} In each \textbf{EKPB}, the input $ H_i \in \mathbb{R}^{C  \times D} (H_1=X_{em}) $ is first processed through an MLP to generate $ H_i^k \in \mathbb{R}^{C  \times Q}$. Then, FFT is applied to get $ H_i^f \in \mathbb{R}^{C  \times (Q/2+1)}$. For $ H_i^f$, we calculate its energy, denoted as $ H_i^e \in \mathbb{R}^{C  \times (Q/2+1)}$. A cross-channel softmax is then applied to $ H_i^e$ per frequency to obtain a probability distribution $ H_i^{soft} \in \mathbb{R}^{C  \times (Q/2+1)}$. Using $H_i^{soft}$, we select values from $ H_i^f$ across channels for each frequency, resulting in $K^f_i \in \mathbb{R}^{1  \times (Q/2+1)}$, which represents the Shared Key-Frequency across all channels. Then iFFT is performed on $K^f_i$ to get $K_i\in \mathbb{R}^{1  \times Q}$, followed by projection $Q\rightarrow D$ and repeating (C times) to get $\hat{K}_i \in \mathbb{R}^{C  \times D}$. This $\hat{K}_i$ is point-wisely added to $\hat{H_i}\in \mathbb{R}^{C  \times D}$ , which is the projection of $ H_i$ using projection $D\rightarrow D$. Then, an MLP and $Add\&Norm$ is applied to the result $HK\in \mathbb{R}^{C  \times D}$ to fuse inter-series dependencies information, and another MLP and $Add\&Norm$ is used to capture intra-series variations~\citep{LiuiTransformer}. The output of each \textbf{EKPB} is $\hat{O_i} \in \mathbb{R}^{C  \times D}$, where $H_{i+1}=\hat{O_i}$.

\subsection{Key-Frequency Enhanced Training strategy}

In real-world time series, certain channels often exhibit spectral dependencies, which may not be fully captured in the training set, and the specific channels with such dependencies are also unknown~\citep{geweke1984freqchannel,Zhao2024freqchannel}. So this work borrows insight from recent advancement of mix-up in time series~\citep{zhou2023mixup,ansari2024mixup}, randomly introducing spectral information from other channels into each channel, to enhance the extraction of the shared Key-Frequency, as in Figure~\ref{fig:reshuffle}. Given a multivariate time series input $X \in \mathbb{R}^{C \times T}$ and its ground-truth $Y \in \mathbb{R}^{C \times F}$, we generate a pseudo sample pair: 

\begin{align}
X' = iFFT(FFT(X) +\alpha \cdot FFT(X[\text{perm},:]))&,  \notag\\ 
Y' = iFFT(FFT(Y) +\alpha \cdot FFT(Y[\text{perm},:]))&.
\end{align}

$\alpha \in \mathbb{R}^{C \times 1}$ is a weight vector sampled from a normal distribution, $\text{perm}$ is a reshuffled channel index. Since $FFT$ and $iFFT$ are linear operations, this mix-up process can be equivalently simplified in the \textbf{Time Domain}:
\begin{align}
X' = X +\alpha \cdot X[\text{perm},:]&,  \notag\\
Y' = Y +\alpha \cdot Y[\text{perm},:]&
 \end{align}
We alternate training between real and synthetic data to preserve the spectral dependencies in real samples. This combines the advantages of data augmentation, such as improved generalization, while mitigating potential drawbacks like over-smoothing and training instability~\citep{ryu2024tf,alkhalifah2022tf}.












\section{Experiments}

\subsection{Setups}
\subsubsection{Implementation Details}
We apply our FDS method to two types of 3DGS: 
the original 3DGS, and 2DGS~\citep{huang20242d}. 
%
The number of iterations in our optimization 
process is 35,000.
We follow the default training configuration 
and apply our FDS method after 15,000 iterations,
then we add normal consistency loss for both
3DGS and 2DGS after 25000 iterations.
%
The weight for FDS, $\lambda_{fds}$, is set to 0.015,
the $\sigma$ is set to 23,
and the weight for normal consistency is set to 0.15
for all experiments. 
We removed the depth distortion loss in 2DGS 
because we found that it degrades its results in indoor scenes.
%
The Gaussian point cloud is initialized using Colmap
for all datasets.
%
%
We tested the impact of 
using Sea Raft~\citep{wang2025sea} and 
Raft\citep{teed2020raft} on FDS performance.
%
Due to the blurriness of the ScanNet dataset, 
additional prior constraints are required.
Thus, we incorporate normal prior supervision 
on the rendered normals 
in ScanNet (V2) dataset by default.
The normal prior is predicted by the Stable Normal 
model~\citep{ye2024stablenormal}
across all types of 3DGS.
%
The entire framework is implemented in 
PyTorch~\citep{paszke2019pytorch}, 
and all experiments are conducted on 
a single NVIDIA 4090D GPU.

\begin{figure}[t] \centering
    \makebox[0.16\textwidth]{\scriptsize Input}
    \makebox[0.16\textwidth]{\scriptsize 3DGS}
    \makebox[0.16\textwidth]{\scriptsize 2DGS}
    \makebox[0.16\textwidth]{\scriptsize 3DGS + FDS}
    \makebox[0.16\textwidth]{\scriptsize 2DGS + FDS}
    \makebox[0.16\textwidth]{\scriptsize GT (Depth)}

    \includegraphics[width=0.16\textwidth]{figure/fig3_img/compare3/gt_rgb/frame_00522.jpg}
    \includegraphics[width=0.16\textwidth]{figure/fig3_img/compare3/3DGS/frame_00522.jpg}
    \includegraphics[width=0.16\textwidth]{figure/fig3_img/compare3/2DGS/frame_00522.jpg}
    \includegraphics[width=0.16\textwidth]{figure/fig3_img/compare3/3DGS+FDS/frame_00522.jpg}
    \includegraphics[width=0.16\textwidth]{figure/fig3_img/compare3/2DGS+FDS/frame_00522.jpg}
    \includegraphics[width=0.16\textwidth]{figure/fig3_img/compare3/gt_depth/frame_00522.jpg} \\

    % \includegraphics[width=0.16\textwidth]{figure/fig3_img/compare1/gt_rgb/frame_00137.jpg}
    % \includegraphics[width=0.16\textwidth]{figure/fig3_img/compare1/3DGS/frame_00137.jpg}
    % \includegraphics[width=0.16\textwidth]{figure/fig3_img/compare1/2DGS/frame_00137.jpg}
    % \includegraphics[width=0.16\textwidth]{figure/fig3_img/compare1/3DGS+FDS/frame_00137.jpg}
    % \includegraphics[width=0.16\textwidth]{figure/fig3_img/compare1/2DGS+FDS/frame_00137.jpg}
    % \includegraphics[width=0.16\textwidth]{figure/fig3_img/compare1/gt_depth/frame_00137.jpg} \\

     \includegraphics[width=0.16\textwidth]{figure/fig3_img/compare2/gt_rgb/frame_00262.jpg}
    \includegraphics[width=0.16\textwidth]{figure/fig3_img/compare2/3DGS/frame_00262.jpg}
    \includegraphics[width=0.16\textwidth]{figure/fig3_img/compare2/2DGS/frame_00262.jpg}
    \includegraphics[width=0.16\textwidth]{figure/fig3_img/compare2/3DGS+FDS/frame_00262.jpg}
    \includegraphics[width=0.16\textwidth]{figure/fig3_img/compare2/2DGS+FDS/frame_00262.jpg}
    \includegraphics[width=0.16\textwidth]{figure/fig3_img/compare2/gt_depth/frame_00262.jpg} \\

    \includegraphics[width=0.16\textwidth]{figure/fig3_img/compare4/gt_rgb/frame00000.png}
    \includegraphics[width=0.16\textwidth]{figure/fig3_img/compare4/3DGS/frame00000.png}
    \includegraphics[width=0.16\textwidth]{figure/fig3_img/compare4/2DGS/frame00000.png}
    \includegraphics[width=0.16\textwidth]{figure/fig3_img/compare4/3DGS+FDS/frame00000.png}
    \includegraphics[width=0.16\textwidth]{figure/fig3_img/compare4/2DGS+FDS/frame00000.png}
    \includegraphics[width=0.16\textwidth]{figure/fig3_img/compare4/gt_depth/frame00000.png} \\

    \includegraphics[width=0.16\textwidth]{figure/fig3_img/compare5/gt_rgb/frame00080.png}
    \includegraphics[width=0.16\textwidth]{figure/fig3_img/compare5/3DGS/frame00080.png}
    \includegraphics[width=0.16\textwidth]{figure/fig3_img/compare5/2DGS/frame00080.png}
    \includegraphics[width=0.16\textwidth]{figure/fig3_img/compare5/3DGS+FDS/frame00080.png}
    \includegraphics[width=0.16\textwidth]{figure/fig3_img/compare5/2DGS+FDS/frame00080.png}
    \includegraphics[width=0.16\textwidth]{figure/fig3_img/compare5/gt_depth/frame00080.png} \\



    \caption{\textbf{Comparison of depth reconstruction on Mushroom and ScanNet datasets.} The original
    3DGS or 2DGS model equipped with FDS can remove unwanted floaters and reconstruct
    geometry more preciously.}
    \label{fig:compare}
\end{figure}


\subsubsection{Datasets and Metrics}

We evaluate our method for 3D reconstruction 
and novel view synthesis tasks on
\textbf{Mushroom}~\citep{ren2024mushroom},
\textbf{ScanNet (v2)}~\citep{dai2017scannet}, and 
\textbf{Replica}~\citep{replica19arxiv}
datasets,
which feature challenging indoor scenes with both 
sparse and dense image sampling.
%
The Mushroom dataset is an indoor dataset 
with sparse image sampling and two distinct 
camera trajectories. 
%
We train our model on the training split of 
the long capture sequence and evaluate 
novel view synthesis on the test split 
of the long capture sequences.
%
Five scenes are selected to evaluate our FDS, 
including "coffee room", "honka", "kokko", 
"sauna", and "vr room". 
%
ScanNet(V2)~\citep{dai2017scannet}  consists of 1,613 indoor scenes
with annotated camera poses and depth maps. 
%
We select 5 scenes from the ScanNet (V2) dataset, 
uniformly sampling one-tenth of the views,
following the approach in ~\citep{guo2022manhattan}.
To further improve the geometry rendering quality of 3DGS, 
%
Replica~\citep{replica19arxiv} contains small-scale 
real-world indoor scans. 
We evaluate our FDS on five scenes from 
Replica: office0, office1, office2, office3 and office4,
selecting one-tenth of the views for training.
%
The results for Replica are provided in the 
supplementary materials.
To evaluate the rendering quality and geometry 
of 3DGS, we report PSNR, SSIM, and LPIPS for 
rendering quality, along with Absolute Relative Distance 
(Abs Rel) as a depth quality metrics.
%
Additionally, for mesh evaluation, 
we use metrics including Accuracy, Completion, 
Chamfer-L1 distance, Normal Consistency, 
and F-scores.




\subsection{Results}
\subsubsection{Depth rendering and novel view synthesis}
The comparison results on Mushroom and 
ScanNet are presented in \tabref{tab:mushroom} 
and \tabref{tab:scannet}, respectively. 
%
Due to the sparsity of sampling 
in the Mushroom dataset,
challenges are posed for both GOF~\citep{yu2024gaussian} 
and PGSR~\citep{chen2024pgsr}, 
leading to their relative poor performance 
on the Mushroom dataset.
%
Our approach achieves the best performance 
with the FDS method applied during the training process.
The FDS significantly enhances the 
geometric quality of 3DGS on the Mushroom dataset, 
improving the "abs rel" metric by more than 50\%.
%
We found that Sea Raft~\citep{wang2025sea}
outperforms Raft~\citep{teed2020raft} on FDS, 
indicating that a better optical flow model 
can lead to more significant improvements.
%
Additionally, the render quality of RGB 
images shows a slight improvement, 
by 0.58 in 3DGS and 0.50 in 2DGS, 
benefiting from the incorporation of cross-view consistency in FDS. 
%
On the Mushroom
dataset, adding the FDS loss increases 
the training time by half an hour, which maintains the same
level as baseline.
%
Similarly, our method shows a notable improvement on the ScanNet dataset as well using Sea Raft~\citep{wang2025sea} Model. The "abs rel" metric in 2DGS is improved nearly 50\%. This demonstrates the robustness and effectiveness of the FDS method across different datasets.
%


% \begin{wraptable}{r}{0.6\linewidth} \centering
% \caption{\textbf{Ablation study on geometry priors.}} 
%         \label{tab:analysis_prior}
%         \resizebox{\textwidth}{!}{
\begin{tabular}{c| c c c c c | c c c c}

    \hline
     Method &  Acc$\downarrow$ & Comp $\downarrow$ & C-L1 $\downarrow$ & NC $\uparrow$ & F-Score $\uparrow$ &  Abs Rel $\downarrow$ &  PSNR $\uparrow$  & SSIM  $\uparrow$ & LPIPS $\downarrow$ \\ \hline
    2DGS&   0.1078&  0.0850&  0.0964&  0.7835&  0.5170&  0.1002&  23.56&  0.8166& 0.2730\\
    2DGS+Depth&   0.0862&  0.0702&  0.0782&  0.8153&  0.5965&  0.0672&  23.92&  0.8227& 0.2619 \\
    2DGS+MVDepth&   0.2065&  0.0917&  0.1491&  0.7832&  0.3178&  0.0792&  23.74&  0.8193& 0.2692 \\
    2DGS+Normal&   0.0939&  0.0637&  0.0788&  \textbf{0.8359}&  0.5782&  0.0768&  23.78&  0.8197& 0.2676 \\
    2DGS+FDS &  \textbf{0.0615} & \textbf{ 0.0534}& \textbf{0.0574}& 0.8151& \textbf{0.6974}&  \textbf{0.0561}&  \textbf{24.06}&  \textbf{0.8271}&\textbf{0.2610} \\ \hline
    2DGS+Depth+FDS &  0.0561 &  0.0519& 0.0540& 0.8295& 0.7282&  0.0454&  \textbf{24.22}& \textbf{0.8291}&\textbf{0.2570} \\
    2DGS+Normal+FDS &  \textbf{0.0529} & \textbf{ 0.0450}& \textbf{0.0490}& \textbf{0.8477}& \textbf{0.7430}&  \textbf{0.0443}&  24.10&  0.8283& 0.2590 \\
    2DGS+Depth+Normal &  0.0695 & 0.0513& 0.0604& 0.8540&0.6723&  0.0523&  24.09&  0.8264&0.2575\\ \hline
    2DGS+Depth+Normal+FDS &  \textbf{0.0506} & \textbf{0.0423}& \textbf{0.0464}& \textbf{0.8598}&\textbf{0.7613}&  \textbf{0.0403}&  \textbf{24.22}& 
    \textbf{0.8300}&\textbf{0.0403}\\
    
\bottomrule
\end{tabular}
}
% \end{wraptable}



The qualitative comparisons on the Mushroom and ScanNet dataset 
are illustrated in \figref{fig:compare}. 
%
%
As seen in the first row of \figref{fig:compare}, 
both the original 3DGS and 2DGS suffer from overfitting, 
leading to corrupted geometry generation. 
%
Our FDS effectively mitigates this issue by 
supervising the matching relationship between 
the input and sampled views, 
helping to recover the geometry.
%
FDS also improves the refinement of geometric details, 
as shown in other rows. 
By incorporating the matching prior through FDS, 
the quality of the rendered depth is significantly improved.
%

\begin{table}[t] \centering
\begin{minipage}[t]{0.96\linewidth}
        \captionof{table}{\textbf{3D Reconstruction 
        and novel view synthesis results on Mushroom dataset. * 
        Represents that FDS uses the Raft model.
        }}
        \label{tab:mushroom}
        \resizebox{\textwidth}{!}{
\begin{tabular}{c| c c c c c | c c c c c}
    \hline
     Method &  Acc$\downarrow$ & Comp $\downarrow$ & C-L1 $\downarrow$ & NC $\uparrow$ & F-Score $\uparrow$ &  Abs Rel $\downarrow$ &  PSNR $\uparrow$  & SSIM  $\uparrow$ & LPIPS $\downarrow$ & Time  $\downarrow$ \\ \hline

    % DN-splatter &   &  &  &  &  &  &  &  & \\
    GOF &  0.1812 & 0.1093 & 0.1453 & 0.6292 & 0.3665 & 0.2380  & 21.37  &  0.7762  & 0.3132  & $\approx$1.4h\\ 
    PGSR &  0.0971 & 0.1420 & 0.1196 & 0.7193 & 0.5105 & 0.1723  & 22.13  & 0.7773  & 0.2918  & $\approx$1.2h \\ \hline
    3DGS &   0.1167 &  0.1033&  0.1100&  0.7954&  0.3739&  0.1214&  24.18&  0.8392& 0.2511 &$\approx$0.8h \\
    3DGS + FDS$^*$ & 0.0569  & 0.0676 & 0.0623 & 0.8105 & 0.6573 & 0.0603 & 24.72  & 0.8489 & 0.2379 &$\approx$1.3h \\
    3DGS + FDS & \textbf{0.0527}  & \textbf{0.0565} & \textbf{0.0546} & \textbf{0.8178} & \textbf{0.6958} & \textbf{0.0568} & \textbf{24.76}  & \textbf{0.8486} & \textbf{0.2381} &$\approx$1.3h \\ \hline
    2DGS&   0.1078&  0.0850&  0.0964&  0.7835&  0.5170&  0.1002&  23.56&  0.8166& 0.2730 &$\approx$0.8h\\
    2DGS + FDS$^*$ &  0.0689 &  0.0646& 0.0667& 0.8042& 0.6582& 0.0589& 23.98&  0.8255&0.2621 &$\approx$1.3h\\
    2DGS + FDS &  \textbf{0.0615} & \textbf{ 0.0534}& \textbf{0.0574}& \textbf{0.8151}& \textbf{0.6974}&  \textbf{0.0561}&  \textbf{24.06}&  \textbf{0.8271}&\textbf{0.2610} &$\approx$1.3h \\ \hline
\end{tabular}
}
\end{minipage}\hfill
\end{table}

\begin{table}[t] \centering
\begin{minipage}[t]{0.96\linewidth}
        \captionof{table}{\textbf{3D Reconstruction 
        and novel view synthesis results on ScanNet dataset.}}
        \label{tab:scannet}
        \resizebox{\textwidth}{!}{
\begin{tabular}{c| c c c c c | c c c c }
    \hline
     Method &  Acc $\downarrow$ & Comp $\downarrow$ & C-L1 $\downarrow$ & NC $\uparrow$ & F-Score $\uparrow$ &  Abs Rel $\downarrow$ &  PSNR $\uparrow$  & SSIM  $\uparrow$ & LPIPS $\downarrow$ \\ \hline
    GOF & 1.8671  & 0.0805 & 0.9738 & 0.5622 & 0.2526 & 0.1597  & 21.55  & 0.7575  & 0.3881 \\
    PGSR &  0.2928 & 0.5103 & 0.4015 & 0.5567 & 0.1926 & 0.1661  & 21.71 & 0.7699  & 0.3899 \\ \hline

    3DGS &  0.4867 & 0.1211 & 0.3039 & 0.7342& 0.3059 & 0.1227 & 22.19& 0.7837 & 0.3907\\
    3DGS + FDS &  \textbf{0.2458} & \textbf{0.0787} & \textbf{0.1622} & \textbf{0.7831} & 
    \textbf{0.4482} & \textbf{0.0573} & \textbf{22.83} & \textbf{0.7911} & \textbf{0.3826} \\ \hline
    2DGS &  0.2658 & 0.0845 & 0.1752 & 0.7504& 0.4464 & 0.0831 & 22.59& 0.7881 & 0.3854\\
    2DGS + FDS &  \textbf{0.1457} & \textbf{0.0679} & \textbf{0.1068} & \textbf{0.7883} & 
    \textbf{0.5459} & \textbf{0.0432} & \textbf{22.91} & \textbf{0.7928} & \textbf{0.3800} \\ \hline
\end{tabular}
}
\end{minipage}\hfill
\end{table}


\begin{table}[t] \centering
\begin{minipage}[t]{0.96\linewidth}
        \captionof{table}{\textbf{Ablation study on geometry priors.}}
        \label{tab:analysis_prior}
        \resizebox{\textwidth}{!}{
\begin{tabular}{c| c c c c c | c c c c}

    \hline
     Method &  Acc$\downarrow$ & Comp $\downarrow$ & C-L1 $\downarrow$ & NC $\uparrow$ & F-Score $\uparrow$ &  Abs Rel $\downarrow$ &  PSNR $\uparrow$  & SSIM  $\uparrow$ & LPIPS $\downarrow$ \\ \hline
    2DGS&   0.1078&  0.0850&  0.0964&  0.7835&  0.5170&  0.1002&  23.56&  0.8166& 0.2730\\
    2DGS+Depth&   0.0862&  0.0702&  0.0782&  0.8153&  0.5965&  0.0672&  23.92&  0.8227& 0.2619 \\
    2DGS+MVDepth&   0.2065&  0.0917&  0.1491&  0.7832&  0.3178&  0.0792&  23.74&  0.8193& 0.2692 \\
    2DGS+Normal&   0.0939&  0.0637&  0.0788&  \textbf{0.8359}&  0.5782&  0.0768&  23.78&  0.8197& 0.2676 \\
    2DGS+FDS &  \textbf{0.0615} & \textbf{ 0.0534}& \textbf{0.0574}& 0.8151& \textbf{0.6974}&  \textbf{0.0561}&  \textbf{24.06}&  \textbf{0.8271}&\textbf{0.2610} \\ \hline
    2DGS+Depth+FDS &  0.0561 &  0.0519& 0.0540& 0.8295& 0.7282&  0.0454&  \textbf{24.22}& \textbf{0.8291}&\textbf{0.2570} \\
    2DGS+Normal+FDS &  \textbf{0.0529} & \textbf{ 0.0450}& \textbf{0.0490}& \textbf{0.8477}& \textbf{0.7430}&  \textbf{0.0443}&  24.10&  0.8283& 0.2590 \\
    2DGS+Depth+Normal &  0.0695 & 0.0513& 0.0604& 0.8540&0.6723&  0.0523&  24.09&  0.8264&0.2575\\ \hline
    2DGS+Depth+Normal+FDS &  \textbf{0.0506} & \textbf{0.0423}& \textbf{0.0464}& \textbf{0.8598}&\textbf{0.7613}&  \textbf{0.0403}&  \textbf{24.22}& 
    \textbf{0.8300}&\textbf{0.0403}\\
    
\bottomrule
\end{tabular}
}
\end{minipage}\hfill
\end{table}




\subsubsection{Mesh extraction}
To further demonstrate the improvement in geometry quality, 
we applied methods used in ~\citep{turkulainen2024dnsplatter} 
to extract meshes from the input views of optimized 3DGS. 
The comparison results are presented  
in \tabref{tab:mushroom}. 
With the integration of FDS, the mesh quality is significantly enhanced compared to the baseline, featuring fewer floaters and more well-defined shapes.
 %
% Following the incorporation of FDS, the reconstruction 
% results exhibit fewer floaters and more well-defined 
% shapes in the meshes. 
% Visualized comparisons
% are provided in the supplementary material.

% \begin{figure}[t] \centering
%     \makebox[0.19\textwidth]{\scriptsize GT}
%     \makebox[0.19\textwidth]{\scriptsize 3DGS}
%     \makebox[0.19\textwidth]{\scriptsize 3DGS+FDS}
%     \makebox[0.19\textwidth]{\scriptsize 2DGS}
%     \makebox[0.19\textwidth]{\scriptsize 2DGS+FDS} \\

%     \includegraphics[width=0.19\textwidth]{figure/fig4_img/compare1/gt02.png}
%     \includegraphics[width=0.19\textwidth]{figure/fig4_img/compare1/baseline06.png}
%     \includegraphics[width=0.19\textwidth]{figure/fig4_img/compare1/baseline_fds05.png}
%     \includegraphics[width=0.19\textwidth]{figure/fig4_img/compare1/2dgs04.png}
%     \includegraphics[width=0.19\textwidth]{figure/fig4_img/compare1/2dgs_fds03.png} \\

%     \includegraphics[width=0.19\textwidth]{figure/fig4_img/compare2/gt00.png}
%     \includegraphics[width=0.19\textwidth]{figure/fig4_img/compare2/baseline02.png}
%     \includegraphics[width=0.19\textwidth]{figure/fig4_img/compare2/baseline_fds01.png}
%     \includegraphics[width=0.19\textwidth]{figure/fig4_img/compare2/2dgs04.png}
%     \includegraphics[width=0.19\textwidth]{figure/fig4_img/compare2/2dgs_fds03.png} \\
      
%     \includegraphics[width=0.19\textwidth]{figure/fig4_img/compare3/gt05.png}
%     \includegraphics[width=0.19\textwidth]{figure/fig4_img/compare3/3dgs03.png}
%     \includegraphics[width=0.19\textwidth]{figure/fig4_img/compare3/3dgs_fds04.png}
%     \includegraphics[width=0.19\textwidth]{figure/fig4_img/compare3/2dgs02.png}
%     \includegraphics[width=0.19\textwidth]{figure/fig4_img/compare3/2dgs_fds01.png} \\

%     \caption{\textbf{Qualitative comparison of extracted mesh 
%     on Mushroom and ScanNet datasets.}}
%     \label{fig:mesh}
% \end{figure}












\subsection{Ablation study}


\textbf{Ablation study on geometry priors:} 
To highlight the advantage of incorporating matching priors, 
we incorporated various types of priors generated by different 
models into 2DGS. These include a monocular depth estimation
model (Depth Anything v2)~\citep{yang2024depth}, a two-view depth estimation 
model (Unimatch)~\citep{xu2023unifying}, 
and a monocular normal estimation model (DSINE)~\citep{bae2024rethinking}.
We adapt the scale and shift-invariant loss in Midas~\citep{birkl2023midas} for
monocular depth supervision and L1 loss for two-view depth supervison.
%
We use Sea Raft~\citep{wang2025sea} as our default optical flow model.
%
The comparison results on Mushroom dataset 
are shown in ~\tabref{tab:analysis_prior}.
We observe that the normal prior provides accurate shape information, 
enhancing the geometric quality of the radiance field. 
%
% In contrast, the monocular depth prior slightly increases 
% the 'Abs Rel' due to its ambiguous scale and inaccurate depth ordering.
% Moreover, the performance of monocular depth estimation 
% in the sauna scene is particularly poor, 
% primarily due to the presence of numerous reflective 
% surfaces and textureless walls, which limits the accuracy of monocular depth estimation.
%
The multi-view depth prior, hindered by the limited feature overlap 
between input views, fails to offer reliable geometric 
information. We test average "Abs Rel" of multi-view depth prior
, and the result is 0.19, which performs worse than the "Abs Rel" results 
rendered by original 2DGS.
From the results, it can be seen that depth order information provided by monocular depth improves
reconstruction accuracy. Meanwhile, our FDS achieves the best performance among all the priors, 
and by integrating all
three components, we obtained the optimal results.
%
%
\begin{figure}[t] \centering
    \makebox[0.16\textwidth]{\scriptsize RF (16000 iters)}
    \makebox[0.16\textwidth]{\scriptsize RF* (20000 iters)}
    \makebox[0.16\textwidth]{\scriptsize RF (20000 iters)  }
    \makebox[0.16\textwidth]{\scriptsize PF (16000 iters)}
    \makebox[0.16\textwidth]{\scriptsize PF (20000 iters)}


    % \includegraphics[width=0.16\textwidth]{figure/fig5_img/compare1/16000.png}
    % \includegraphics[width=0.16\textwidth]{figure/fig5_img/compare1/20000_wo_flow_loss.png}
    % \includegraphics[width=0.16\textwidth]{figure/fig5_img/compare1/20000.png}
    % \includegraphics[width=0.16\textwidth]{figure/fig5_img/compare1/16000_prior.png}
    % \includegraphics[width=0.16\textwidth]{figure/fig5_img/compare1/20000_prior.png}\\

    % \includegraphics[width=0.16\textwidth]{figure/fig5_img/compare2/16000.png}
    % \includegraphics[width=0.16\textwidth]{figure/fig5_img/compare2/20000_wo_flow_loss.png}
    % \includegraphics[width=0.16\textwidth]{figure/fig5_img/compare2/20000.png}
    % \includegraphics[width=0.16\textwidth]{figure/fig5_img/compare2/16000_prior.png}
    % \includegraphics[width=0.16\textwidth]{figure/fig5_img/compare2/20000_prior.png}\\

    \includegraphics[width=0.16\textwidth]{figure/fig5_img/compare3/16000.png}
    \includegraphics[width=0.16\textwidth]{figure/fig5_img/compare3/20000_wo_flow_loss.png}
    \includegraphics[width=0.16\textwidth]{figure/fig5_img/compare3/20000.png}
    \includegraphics[width=0.16\textwidth]{figure/fig5_img/compare3/16000_prior.png}
    \includegraphics[width=0.16\textwidth]{figure/fig5_img/compare3/20000_prior.png}\\
    
    \includegraphics[width=0.16\textwidth]{figure/fig5_img/compare4/16000.png}
    \includegraphics[width=0.16\textwidth]{figure/fig5_img/compare4/20000_wo_flow_loss.png}
    \includegraphics[width=0.16\textwidth]{figure/fig5_img/compare4/20000.png}
    \includegraphics[width=0.16\textwidth]{figure/fig5_img/compare4/16000_prior.png}
    \includegraphics[width=0.16\textwidth]{figure/fig5_img/compare4/20000_prior.png}\\

    \includegraphics[width=0.30\textwidth]{figure/fig5_img/bar.png}

    \caption{\textbf{The error map of Radiance Flow and Prior Flow.} RF: Radiance Flow, PF: Prior Flow, * means that there is no FDS loss supervision during optimization.}
    \label{fig:error_map}
\end{figure}




\textbf{Ablation study on FDS: }
In this section, we present the design of our FDS 
method through an ablation study on the 
Mushroom dataset to validate its effectiveness.
%
The optional configurations of FDS are outlined in ~\tabref{tab:ablation_fds}.
Our base model is the 2DGS equipped with FDS,
and its results are shown 
in the first row. The goal of this analysis 
is to evaluate the impact 
of various strategies on FDS sampling and loss design.
%
We observe that when we 
replace $I_i$ in \eqref{equ:mflow} with $C_i$, 
as shown in the second row, the geometric quality 
of 2DGS deteriorates. Using $I_i$ instead of $C_i$ 
help us to remove the floaters in $\bm{C^s}$, which are also 
remained in $\bm{C^i}$.
We also experiment with modifying the FDS loss. For example, 
in the third row, we use the neighbor 
input view as the sampling view, and replace the 
render result of neighbor view with ground truth image of its input view.
%
Due to the significant movement between images, the Prior Flow fails to accurately 
match the pixel between them, leading to a further degradation in geometric quality.
%
Finally, we attempt to fix the sampling view 
and found that this severely damaged the geometric quality, 
indicating that random sampling is essential for the stability 
of the mean error in the Prior flow.



\begin{table}[t] \centering

\begin{minipage}[t]{1.0\linewidth}
        \captionof{table}{\textbf{Ablation study on FDS strategies.}}
        \label{tab:ablation_fds}
        \resizebox{\textwidth}{!}{
\begin{tabular}{c|c|c|c|c|c|c|c}
    \hline
    \multicolumn{2}{c|}{$\mathcal{M}_{\theta}(X, \bm{C^s})$} & \multicolumn{3}{c|}{Loss} & \multicolumn{3}{c}{Metric}  \\
    \hline
    $X=C^i$ & $X=I^i$  & Input view & Sampled view     & Fixed Sampled view        & Abs Rel $\downarrow$ & F-score $\uparrow$ & NC $\uparrow$ \\
    \hline
    & \ding{51} &     &\ding{51}    &    &    \textbf{0.0561}        &  \textbf{0.6974}         & \textbf{0.8151}\\
    \hline
     \ding{51} &           &     &\ding{51}    &    &    0.0839        &  0.6242         &0.8030\\
     &  \ding{51} &   \ding{51}  &    &    &    0.0877       & 0.6091        & 0.7614 \\
      &  \ding{51} &    &    & \ding{51}    &    0.0724           & 0.6312          & 0.8015 \\
\bottomrule
\end{tabular}
}
\end{minipage}
\end{table}




\begin{figure}[htbp] \centering
    \makebox[0.22\textwidth]{}
    \makebox[0.22\textwidth]{}
    \makebox[0.22\textwidth]{}
    \makebox[0.22\textwidth]{}
    \\

    \includegraphics[width=0.22\textwidth]{figure/fig6_img/l1/rgb/frame00096.png}
    \includegraphics[width=0.22\textwidth]{figure/fig6_img/l1/render_rgb/frame00096.png}
    \includegraphics[width=0.22\textwidth]{figure/fig6_img/l1/render_depth/frame00096.png}
    \includegraphics[width=0.22\textwidth]{figure/fig6_img/l1/depth/frame00096.png}

    % \includegraphics[width=0.22\textwidth]{figure/fig6_img/l2/rgb/frame00112.png}
    % \includegraphics[width=0.22\textwidth]{figure/fig6_img/l2/render_rgb/frame00112.png}
    % \includegraphics[width=0.22\textwidth]{figure/fig6_img/l2/render_depth/frame00112.png}
    % \includegraphics[width=0.22\textwidth]{figure/fig6_img/l2/depth/frame00112.png}

    \caption{\textbf{Limitation of FDS.} }
    \label{fig:limitation}
\end{figure}


% \begin{figure}[t] \centering
%     \makebox[0.48\textwidth]{}
%     \makebox[0.48\textwidth]{}
%     \\
%     \includegraphics[width=0.48\textwidth]{figure/loss_Ignatius.pdf}
%     \includegraphics[width=0.48\textwidth]{figure/loss_family.pdf}
%     \caption{\textbf{Comparison the photometric error of Radiance Flow and Prior Flow:} 
%     We add FDS method after 2k iteration during training.
%     The results show
%     that:  1) The Prior Flow is more precise and 
%     robust than Radiance Flow during the radiance 
%     optimization; 2) After adding the FDS loss 
%     which utilize Prior 
%     flow to supervise the Radiance Flow at 2k iterations, 
%     both flow are more accurate, which lead to
%     a mutually reinforcing effects.(TODO fix it)} 
%     \label{fig:flowcompare}
% \end{figure}






\textbf{Interpretive Experiments: }
To demonstrate the mutual refinement of two flows in our FDS, 
For each view, we sample the unobserved 
views multiple times to compute the mean error 
of both Radiance Flow and Prior Flow. 
We use Raft~\citep{teed2020raft} as our default optical flow model
for visualization.
The ground truth flow is calculated based on 
~\eref{equ:flow_pose} and ~\eref{equ:flow} 
utilizing ground truth depth in dataset.
We introduce our FDS loss after 16000 iterations during 
optimization of 2DGS.
The error maps are shown in ~\figref{fig:error_map}.
Our analysis reveals that Radiance Flow tends to 
exhibit significant geometric errors, 
whereas Prior Flow can more accurately estimate the geometry,
effectively disregarding errors introduced by floating Gaussian points. 

%





\subsection{Limitation and further work}

Firstly, our FDS faces challenges in scenes with 
significant lighting variations between different 
views, as shown in the lamp of first row in ~\figref{fig:limitation}. 
%
Incorporating exposure compensation into FDS could help address this issue. 
%
 Additionally, our method struggles with 
 reflective surfaces and motion blur,
 leading to incorrect matching. 
 %
 In the future, we plan to explore the potential 
 of FDS in monocular video reconstruction tasks, 
 using only a single input image at each time step.
 


\section{Conclusions}
In this paper, we propose Flow Distillation Sampling (FDS), which
leverages the matching prior between input views and 
sampled unobserved views from the pretrained optical flow model, to improve the geometry quality
of Gaussian radiance field. 
Our method can be applied to different approaches (3DGS and 2DGS) to enhance the geometric rendering quality of the corresponding neural radiance fields.
We apply our method to the 3DGS-based framework, 
and the geometry is enhanced on the Mushroom, ScanNet, and Replica datasets.

\section*{Acknowledgements} This work was supported by 
National Key R\&D Program of China (2023YFB3209702), 
the National Natural Science Foundation of 
China (62441204, 62472213), and Gusu 
Innovation \& Entrepreneurship Leading Talents Program (ZXL2024361)
\section{Conclusion}
We introduce a novel approach, \algo, to reduce human feedback requirements in preference-based reinforcement learning by leveraging vision-language models. While VLMs encode rich world knowledge, their direct application as reward models is hindered by alignment issues and noisy predictions. To address this, we develop a synergistic framework where limited human feedback is used to adapt VLMs, improving their reliability in preference labeling. Further, we incorporate a selective sampling strategy to mitigate noise and prioritize informative human annotations.

Our experiments demonstrate that this method significantly improves feedback efficiency, achieving comparable or superior task performance with up to 50\% fewer human annotations. Moreover, we show that an adapted VLM can generalize across similar tasks, further reducing the need for new human feedback by 75\%. These results highlight the potential of integrating VLMs into preference-based RL, offering a scalable solution to reducing human supervision while maintaining high task success rates. 

\section*{Impact Statement}
This work advances embodied AI by significantly reducing the human feedback required for training agents. This reduction is particularly valuable in robotic applications where obtaining human demonstrations and feedback is challenging or impractical, such as assistive robotic arms for individuals with mobility impairments. By minimizing the feedback requirements, our approach enables users to more efficiently customize and teach new skills to robotic agents based on their specific needs and preferences. The broader impact of this work extends to healthcare, assistive technology, and human-robot interaction. One possible risk is that the bias from human feedback can propagate to the VLM and subsequently to the policy. This can be mitigated by personalization of agents in case of household application or standardization of feedback for industrial applications. 


% In the unusual situation where you want a paper to appear in the
% references without citing it in the main text, use \nocite
\nocite{langley00}

\bibliography{example_paper}
\bibliographystyle{icml2025}


%%%%%%%%%%%%%%%%%%%%%%%%%%%%%%%%%%%%%%%%%%%%%%%%%%%%%%%%%%%%%%%%%%%%%%%%%%%%%%%
%%%%%%%%%%%%%%%%%%%%%%%%%%%%%%%%%%%%%%%%%%%%%%%%%%%%%%%%%%%%%%%%%%%%%%%%%%%%%%%
% APPENDIX
%%%%%%%%%%%%%%%%%%%%%%%%%%%%%%%%%%%%%%%%%%%%%%%%%%%%%%%%%%%%%%%%%%%%%%%%%%%%%%%
%%%%%%%%%%%%%%%%%%%%%%%%%%%%%%%%%%%%%%%%%%%%%%%%%%%%%%%%%%%%%%%%%%%%%%%%%%%%%%%
\newpage
\appendix
\onecolumn

%\title{Generating 3D \hl{Small} Binding Molecules Using Shape-Conditioned Diffusion Models with Guidance}
%\date{\vspace{-5ex}}

%\author{
%	Ziqi Chen\textsuperscript{\rm 1}, 
%	Bo Peng\textsuperscript{\rm 1}, 
%	Tianhua Zhai\textsuperscript{\rm 2},
%	Xia Ning\textsuperscript{\rm 1,3,4 \Letter}
%}
%\newcommand{\Address}{
%	\textsuperscript{\rm 1}Computer Science and Engineering, The Ohio Sate University, Columbus, OH 43210.
%	\textsuperscript{\rm 2}Perelman School of Medicine, University of Pennsylvania, Philadelphia, PA 19104.
%	\textsuperscript{\rm 3}Translational Data Analytics Institute, The Ohio Sate University, Columbus, OH 43210.
%	\textsuperscript{\rm 4}Biomedical Informatics, The Ohio Sate University, Columbus, OH 43210.
%	\textsuperscript{\Letter}ning.104@osu.edu
%}

%\newcommand\affiliation[1]{%
%	\begingroup
%	\renewcommand\thefootnote{}\footnote{#1}%
%	\addtocounter{footnote}{-1}%
%	\endgroup
%}



\setcounter{secnumdepth}{2} %May be changed to 1 or 2 if section numbers are desired.

\setcounter{section}{0}
\renewcommand{\thesection}{S\arabic{section}}

\setcounter{table}{0}
\renewcommand{\thetable}{S\arabic{table}}

\setcounter{figure}{0}
\renewcommand{\thefigure}{S\arabic{figure}}

\setcounter{algorithm}{0}
\renewcommand{\thealgorithm}{S\arabic{algorithm}}

\setcounter{equation}{0}
\renewcommand{\theequation}{S\arabic{equation}}


\begin{center}
	\begin{minipage}{0.95\linewidth}
		\centering
		\LARGE 
	Generating 3D Binding Molecules Using Shape-Conditioned Diffusion Models with Guidance (Supplementary Information)
	\end{minipage}
\end{center}
\vspace{10pt}

%%%%%%%%%%%%%%%%%%%%%%%%%%%%%%%%%%%%%%%%%%%%%
\section{Parameters for Reproducibility}
\label{supp:experiments:parameters}
%%%%%%%%%%%%%%%%%%%%%%%%%%%%%%%%%%%%%%%%%%%%%

We implemented both \SE and \methoddiff using Python-3.7.16, PyTorch-1.11.0, PyTorch-scatter-2.0.9, Numpy-1.21.5, Scikit-learn-1.0.2.
%
We trained the models using a Tesla V100 GPU with 32GB memory and a CPU with 80GB memory on Red Hat Enterprise 7.7.
%
%We released the code, data, and the trained model at Google Drive~\footnote{\url{https://drive.google.com/drive/folders/146cpjuwenKGTd6Zh4sYBy-Wv6BMfGwe4?usp=sharing}} (will release to the public on github once the manuscript is accepted).

%===================================================================
\subsection{Parameters of \SE}
%===================================================================


In \SE, we tuned the dimension of all the hidden layers including VN-DGCNN layers
(Eq.~\ref{eqn:shape_embed}), MLP layers (Eq.~\ref{eqn:se:decoder}) and
VN-In layer (Eq.~\ref{eqn:se:decoder}), and the dimension $d_p$ of generated shape latent embeddings $\shapehiddenmat$ with the grid-search algorithm in the 
parameter space presented in Table~\ref{tbl:hyper_se}.
%
We determined the optimal hyper-parameters according to the mean squared errors of the predictions of signed distances for 1,000 validation molecules that are selected as described in Section ``Data'' 
in the main manuscript.
%
The optimal dimension of all the hidden layers is 256, and the optimal dimension $d_p$ of shape latent embedding \shapehiddenmat is 128.
%
The optimal number of points $|\pc|$ in the point cloud \pc is 512.
%
We sampled 1,024 query points in $\mathcal{Z}$ for each molecule shape.
%
We constructed graphs from point clouds, which are employed to learn $\shapehiddenmat$ with VN-DGCNN layer (Eq.~\ref{eqn:shape_embed}), using the $k$-nearest neighbors based on Euclidean distance with $k=20$.
%
We set the number of VN-DGCNN layers as 4.
%
We set the number of MLP layers in the decoder (Eq.~\ref{eqn:se:decoder}) as 2.
%
We set the number of VN-In layers as 1.

%
We optimized the \SE model via Adam~\cite{adam} with its parameters (0.950, 0.999), %betas (0.95, 0.999), 
learning rate 0.001, and batch size 16.
%
We evaluated the validation loss every 2,000 training steps.
%
We scheduled to decay the learning rate with a factor of 0.6 and a minimum learning rate of 1e-6 if 
the validation loss does not decrease in 5 consecutive evaluations.
%
The optimal \SE model has 28.3K learnable parameters. 
%
We trained the \SE model %for at most 80 hours 
with $\sim$156,000 training steps.
%
The training took 80 hours with our GPUs.
%
The trained \SE model achieved the minimum validation loss at 152,000 steps.


\begin{table*}[!h]
  \centering
      \caption{{Hyper-Parameter Space for \SE Optimization}}
  \label{tbl:hyper_se}
  \begin{threeparttable}
 \begin{scriptsize}
      \begin{tabular}{
%	@{\hspace{2pt}}l@{\hspace{2pt}}
	@{\hspace{2pt}}l@{\hspace{5pt}} 
	@{\hspace{2pt}}r@{\hspace{2pt}}         
	}
        \toprule
        %Notation &
          Hyper-parameters &  Space\\
        \midrule
        %$t_a$    & 
         %hidden layer dimension         & \{16, 32, 64, 128\} \\
         %atom/node embedding dimension &  \{16, 32, 64, 128\} \\
         %$\latent^{\add}$/$\latent^{\delete}$ dimension        & \{8, 16, 32, 64\} \\
         hidden layer dimension            & \{128, 256\}\\
         dimension $d_p$ of \shapehiddenmat        &  \{64, 128\} \\
         \#points in \pc        & \{512, 1,024\} \\
         \#query points in $\mathcal{Z}$                & 1,024 \\%1024 \\%\bo{\{1024\}}\\
         \#nearest neighbors              & 20          \\
         \#VN-DGCNN layers (Eq~\ref{eqn:shape_embed})               & 4            \\
         \#MLP layers in Eq~\ref{eqn:se:decoder} & 4           \\
        \bottomrule
      \end{tabular}
%  	\begin{tablenotes}[normal,flushleft]
%  		\begin{footnotesize}
%  	
%  	\item In this table, hidden dimension represents the dimension of hidden layers and 
%  	atom/node embeddings; latent dimension represents the dimension of latent embedding \latent.
%  	\par
%  \end{footnotesize}
%  
%\end{tablenotes}
%      \begin{tablenotes}
%      \item 
%      \par
%      \end{tablenotes}
\end{scriptsize}
  \end{threeparttable}
\end{table*}

%
\begin{table*}[!h]
  \centering
      \caption{{Hyper-Parameter Space for \methoddiff Optimization}}
  \label{tbl:hyper_diff}
  \begin{threeparttable}
 \begin{scriptsize}
      \begin{tabular}{
%	@{\hspace{2pt}}l@{\hspace{2pt}}
	@{\hspace{2pt}}l@{\hspace{5pt}} 
	@{\hspace{2pt}}r@{\hspace{2pt}}         
	}
        \toprule
        %Notation &
          Hyper-parameters &  Space\\
        \midrule
        %$t_a$    & 
         %hidden layer dimension         & \{16, 32, 64, 128\} \\
         %atom/node embedding dimension &  \{16, 32, 64, 128\} \\
         %$\latent^{\add}$/$\latent^{\delete}$ dimension        & \{8, 16, 32, 64\} \\
         scalar hidden layer dimension         & 128 \\
         vector hidden layer dimension         & 32 \\
         weight of atom type loss $\xi$ (Eq.~\ref{eqn:loss})  & 100           \\
         threshold of step weight $\delta$ (Eq.~\ref{eqn:diff:obj:pos}) & 10 \\
         \#atom features $K$                   & 15 \\
         \#layers $L$ in \molpred             & 8 \\
         %\# \eqgnn/\invgnn layers     &  8 \\
         %\# heads {$n_h$} in $\text{MHA}^{\mathtt{x}}/\text{MHA}^{\mathtt{v}}$                               & 16 \\
         \#nearest neighbors {$N$}  (Eq.~\ref{eqn:geometric_embedding} and \ref{eqn:attention})            & 8          \\
         {\#diffusion steps $T$}                  & 1,000 \\
        \bottomrule
      \end{tabular}
%  	\begin{tablenotes}[normal,flushleft]
%  		\begin{footnotesize}
%  	
%  	\item In this table, hidden dimension represents the dimension of hidden layers and 
%  	atom/node embeddings; latent dimension represents the dimension of latent embedding \latent.
%  	\par
%  \end{footnotesize}
%  
%\end{tablenotes}
%      \begin{tablenotes}
%      \item 
%      \par
%      \end{tablenotes}
\end{scriptsize}
  \end{threeparttable}

\end{table*}


%===================================================================
\subsection{Parameters of \methoddiff}
%===================================================================

Table~\ref{tbl:hyper_diff} presents the parameters used to train \methoddiff.
%
In \methoddiff, we set the hidden dimensions of all the MLP layers and the scalar hidden layers in GVPs (Eq.~\ref{eqn:pred:gvp} and Eq.~\ref{eqn:mess:gvp}) as 128. %, including all the MLP layers in \methoddiff and the scalar dimension of GVP layers in Eq.~\ref{eqn:pred:gvp} and Eq.~\ref{eqn:mess:gvp}. %, and MLP layer (Eq.~\ref{eqn:diff:graph:atompred}) as 128.
%
We set the dimensions of all the vector hidden layers in GVPs as 32.
%
We set the number of layers $L$ in \molpred as 8.
%and the number of layers in graph neural networks as 8.
%
Both two GVP modules in Eq.~\ref{eqn:pred:gvp} and Eq.~\ref{eqn:mess:gvp} consist of three GVP layers. %, which consisa GVP modset the number of layer of GVP modules %is a multi-head attention layer ($\text{MHA}^{\mathtt{x}}$ or $\text{MHA}^{\mathtt{h}}$) with 16 heads.
% 
We set the number of VN-MLP layers in Eq.~\ref{eqn:shaper} as 1 and the number of MLP layers as 2 for all the involved MLP functions.
%

We constructed graphs from atoms in molecules, which are employed to learn the scalar embeddings and vector embeddings for atoms %predict atom coordinates and features  
(Eq.~\ref{eqn:geometric_embedding} and \ref{eqn:attention}), using the $N$-nearest neighbors based on Euclidean distance with $N=8$. 
%
We used $K=15$ atom features in total, indicating the atom types and its aromaticity.
%
These atom features include 10 non-aromatic atoms (i.e., ``H'', ``C'', ``N'', ``O'', ``F'', ``P'', ``S'', ``Cl'', ``Br'', ``I''), 
and 5 aromatic atoms (i.e., ``C'', ``N'', ``O'', ``P'', ``S'').
%
We set the number of diffusion steps $T$ as 1,000.
%
We set the weight $\xi$ of atom type loss (Eq.~\ref{eqn:loss}) as $100$ to balance the values of atom type loss and atom coordinate loss.
%
We set the threshold $\delta$ (Eq.~\ref{eqn:diff:obj:pos}) as 10.
%
The parameters $\beta_t^{\mathtt{x}}$ and $\beta_t^{\mathtt{v}}$ of variance scheduling in the forward diffusion process of \methoddiff are discussed in 
Supplementary Section~\ref{supp:forward:variance}.
%
%Please note that as in \squid, we did not perform extensive hyperparameter optimization for \methoddiff.
%
Following \squid, we did not perform extensive hyperparameter tunning for \methoddiff given that the used 
hyperparameters have enabled good performance.

%
We optimized the \methoddiff model via Adam~\cite{adam} with its parameters (0.950, 0.999), learning rate 0.001, and batch size 32.
%
We evaluated the validation loss every 2,000 training steps.
%
We scheduled to decay the learning rate with a factor of 0.6 and a minimum learning rate of 1e-5 if 
the validation loss does not decrease in 10 consecutive evaluations.
%
The \methoddiff model has 7.8M learnable parameters. 
%
We trained the \methoddiff model %for at most 60 hours 
with $\sim$770,000 training steps.
%
The training took 70 hours with our GPUs.
%
The trained \methoddiff achieved the minimum validation loss at 758,000 steps.

During inference, %the sampling, 
following Adams and Coley~\cite{adams2023equivariant}, we set the variance $\phi$ 
of atom-centered Gaussians as 0.049, which is used to build a set of points for shape guidance in Section ``\method with Shape Guidance'' 
in the main manuscript.
%
We determined the number of atoms in the generated molecule using the atom number distribution of training molecules that have surface shape sizes similar to the condition molecule.
%
The optimal distance threshold $\gamma$ is 0.2, and the optimal stop step $S$ for shape guidance is 300.
%
With shape guidance, each time we updated the atom position (Eq.~\ref{eqn:shape_guidance}), we randomly sampled the weight $\sigma$ from $[0.2, 0.8]$. %\bo{(XXX)}.
%
Moreover, when using pocket guidance as mentioned in Section ``\method with Pocket Guidance'' in the main manuscript, each time we updated the atom position (Eq.~\ref{eqn:pocket_guidance}), we randomly sampled the weight $\epsilon$ from $[0, 0.5]$. 
%
For each condition molecule, it took around 40 seconds on average to generate 50 molecule candidates with our GPUs.



%%%%%%%%%%%%%%%%%%%%%%%%%%%%%%%%%%%%%%%%%%%%%%
\section{Performance of \decompdiff with Protein Pocket Prior}
\label{supp:app:decompdiff}
%%%%%%%%%%%%%%%%%%%%%%%%%%%%%%%%%%%%%%%%%%%%%%

In this section, we demonstrate that \decompdiff with protein pocket prior, referred to as \decompdiffbeta, shows very limited performance in generating drug-like and synthesizable molecules compared to all the other methods, including \methodwithpguide and \methodwithsandpguide.
%
We evaluate the performance of \decompdiffbeta in terms of binding affinities, drug-likeness, and diversity.
%
We compare \decompdiffbeta with \methodwithpguide and \methodwithsandpguide and report the results in Table~\ref{tbl:comparison_results_decompdiff}.
%
Note that the results of \methodwithpguide and \methodwithsandpguide here are consistent with those in Table~\ref{tbl:overall_results_docking2} in the main manuscript.
%
As shown in Table~\ref{tbl:comparison_results_decompdiff}, while \decompdiffbeta achieves high binding affinities in Vina M and Vina D, it substantially underperforms \methodwithpguide and \methodwithsandpguide in QED and SA.
%
Particularly, \decompdiffbeta shows a QED score of 0.36, while \methodwithpguide substantially outperforms \decompdiffbeta in QED (0.77) with 113.9\% improvement.
%
\decompdiffbeta also substantially underperforms \methodwithpguide in terms of SA scores (0.55 vs 0.76).
%
These results demonstrate the limited capacity of \decompdiffbeta in generating drug-like and synthesizable molecules.
%
As a result, the generated molecules from \decompdiffbeta can have considerably lower utility compared to other methods.
%
Considering these limitations of \decompdiffbeta, we exclude it from the baselines for comparison.

\begin{table*}[!h]
	\centering
		\caption{Comparison on PMG among \methodwithpguide, \methodwithsandpguide and \decompdiffbeta}
	\label{tbl:comparison_results_decompdiff}
\begin{threeparttable}
	\begin{scriptsize}
	\begin{tabular}{
		@{\hspace{2pt}}l@{\hspace{2pt}}
		%
		%@{\hspace{2pt}}l@{\hspace{2pt}}
		%
		@{\hspace{2pt}}r@{\hspace{2pt}}
		@{\hspace{2pt}}r@{\hspace{2pt}}
		%
		@{\hspace{6pt}}r@{\hspace{6pt}}
		%
		@{\hspace{2pt}}r@{\hspace{2pt}}
		@{\hspace{2pt}}r@{\hspace{2pt}}
		%
		@{\hspace{5pt}}r@{\hspace{5pt}}
		%
		@{\hspace{2pt}}r@{\hspace{2pt}}
		@{\hspace{2pt}}r@{\hspace{2pt}}
		%
		@{\hspace{5pt}}r@{\hspace{5pt}}
		%
		@{\hspace{2pt}}r@{\hspace{2pt}}
	         @{\hspace{2pt}}r@{\hspace{2pt}}
		%
		@{\hspace{5pt}}r@{\hspace{5pt}}
		%
		@{\hspace{2pt}}r@{\hspace{2pt}}
		@{\hspace{2pt}}r@{\hspace{2pt}}
		%
		@{\hspace{5pt}}r@{\hspace{5pt}}
		%
		@{\hspace{2pt}}r@{\hspace{2pt}}
		@{\hspace{2pt}}r@{\hspace{2pt}}
		%
		@{\hspace{5pt}}r@{\hspace{5pt}}
		%
		@{\hspace{2pt}}r@{\hspace{2pt}}
		@{\hspace{2pt}}r@{\hspace{2pt}}
		%
		@{\hspace{5pt}}r@{\hspace{5pt}}
		%
		@{\hspace{2pt}}r@{\hspace{2pt}}
		%@{\hspace{2pt}}r@{\hspace{2pt}}
		%@{\hspace{2pt}}r@{\hspace{2pt}}
		}
		\toprule
		\multirow{2}{*}{method} & \multicolumn{2}{c}{Vina S$\downarrow$} & & \multicolumn{2}{c}{Vina M$\downarrow$} & & \multicolumn{2}{c}{Vina D$\downarrow$} & & \multicolumn{2}{c}{{HA\%$\uparrow$}}  & & \multicolumn{2}{c}{QED$\uparrow$} & & \multicolumn{2}{c}{SA$\uparrow$} & & \multicolumn{2}{c}{Div$\uparrow$} & %& \multirow{2}{*}{SR\%$\uparrow$} & 
		& \multirow{2}{*}{time$\downarrow$} \\
	    \cmidrule{2-3}\cmidrule{5-6} \cmidrule{8-9} \cmidrule{11-12} \cmidrule{14-15} \cmidrule{17-18} \cmidrule{20-21}
		& Avg. & Med. &  & Avg. & Med. &  & Avg. & Med. & & Avg. & Med.  & & Avg. & Med.  & & Avg. & Med.  & & Avg. & Med.  & & \\ %& & \\
		%\multirow{2}{*}{method} & \multirow{2}{*}{\#c\%} &  \multirow{2}{*}{\#u\%} &  \multirow{2}{*}{QED} & \multicolumn{3}{c}{$\nmax=50$} & & \multicolumn{2}{c}{$\nmax=1$}\\
		%\cmidrule(r){5-7} \cmidrule(r){8-10} 
		%& & & & \avgshapesim(std) & \avggraphsim(std  &  \diversity(std  & & \avgshapesim(std) & \avggraphsim(std \\
		\midrule
		%Reference                          & -5.32 & -5.66 & & -5.78 & -5.76 & & -6.63 & -6.67 & & - & - & & 0.53 & 0.49 & & 0.77 & 0.77 & & - & - & %& 23.1 & & - \\
		%\midrule
		%\multirow{4}{*}{PM} 
		%& \AR & -5.06 & -4.99 & &  -5.59 & -5.29 & &  -6.16 & -6.05 & &  37.69 & 31.00 & &  0.50 & 0.49 & &  0.66 & 0.65 & & - & - & %& 7.0 & 
		%& 7,789 \\
		%& \pockettwomol   & -4.50 & -4.21 & &  -5.70 & -5.27 & &  -6.43 & -6.25 & &  48.00 & 51.00 & &  0.58 & 0.58 & &  \textbf{0.77} & \textbf{0.78} & &  0.69 & 0.71 &  %& 24.9 & 
		%& 2,544 \\
		%& \targetdiff     & -4.88 & \underline{-5.82} & &  -6.20 & \underline{-6.36} & &  \textbf{-7.37} & \underline{-7.51} & &  57.57 & 58.27 & &  0.50 & 0.51 & &  0.60 & 0.59 & &  0.72 & 0.71 & % & 10.4 & 
		%& 1,252 \\
		 \decompdiffbeta             & -4.72 & -4.86 & & \textbf{-6.84} & \textbf{-6.91} & & \textbf{-8.85} & \textbf{-8.90} & &  {72.16} & {72.16} & &  0.36 & 0.36 & &  0.55 & 0.55 & & 0.59 & 0.59 & & 3,549 \\ 
		%-4.76 & -6.18 & &  \textbf{-6.86} & \textbf{-7.52} & &  \textbf{-8.85} & \textbf{-8.96} & &  \textbf{72.7} & \textbf{89.8} & &  0.36 & 0.34 & &  0.55 & 0.57 & & 0.59 & 0.59 & & 15.4 \\
		%& \decompdiffref  & -4.58 & -4.77 & &  -5.47 & -5.51 & &  -6.43 & -6.56 & &  47.76 & 48.66 & &  0.56 & 0.56 & &  0.70 & 0.69  & &  0.72 & 0.72 &  %& 15.2 & 
		%& 1,859 \\
		%\midrule
		%\multirow{2}{*}{PC}
		\methodwithpguide       &  \underline{-5.53} & \underline{-5.64} & & {-6.37} & -6.33 & &  \underline{-7.19} & \underline{-7.52} & &  \underline{78.75} & \textbf{94.00} & &  \textbf{0.77} & \textbf{0.80} & &  \textbf{0.76} & \textbf{0.76} & & 0.63 & 0.66 & & 462 \\
		\methodwithsandpguide   & \textbf{-5.81} & \textbf{-5.96} & &  \underline{-6.50} & \underline{-6.58} & & -7.16 & {-7.51} & &  \textbf{79.92} & \underline{93.00} & &  \underline{0.76} & \underline{0.79} & &  \underline{0.75} & \underline{0.74} & & 0.64 & 0.66 & & 561\\
		\bottomrule
	\end{tabular}%
	\begin{tablenotes}[normal,flushleft]
		\begin{footnotesize}
	\item 
\!\!Columns represent: {``Vina S'': the binding affinities between the initially generated poses of molecules and the protein pockets; 
		``Vina M'': the binding affinities between the poses after local structure minimization and the protein pockets;
		``Vina D'': the binding affinities between the poses determined by AutoDock Vina~\cite{Eberhardt2021} and the protein targets;
		``QED'': the drug-likeness score;
		``SA'': the synthesizability score;
		``Div'': the diversity among generated molecules;
		``time'': the time cost to generate molecules.}
		
		\par
		\par
		\end{footnotesize}
	\end{tablenotes}
	\end{scriptsize}
\end{threeparttable}
  \vspace{-10pt}    
\end{table*}



%===================================================================
\section{{Additional Experimental Results on SMG}}
\label{supp:app:results}
%===================================================================

%-------------------------------------------------------------------------------------------------------------------------------------
\subsection{Comparison on Shape and Graph Similarity}
\label{supp:app:results:overall_shape}
%-------------------------------------------------------------------------------------------------------------------------------------

%\ziqi{Outline for this section:
%	\begin{itemize}
%		\item \method can consistently generate molecules with novel structures (low graph similarity) and similar shapes (high shape similarity), such that these molecules have comparable binding capacity with the condition molecules, and potentially better properties as will be shown in Table~\ref{tbl:overall_results_quality_10}.
%	\end{itemize}
%}

\begin{table*}[!h]
	\centering
		\caption{Similarity Comparison on SMG}
	\label{tbl:overall_sim}
\begin{threeparttable}
	\begin{scriptsize}
	\begin{tabular}{
		@{\hspace{0pt}}l@{\hspace{8pt}}
		%
		@{\hspace{8pt}}l@{\hspace{8pt}}
		%
		@{\hspace{8pt}}c@{\hspace{8pt}}
		@{\hspace{8pt}}c@{\hspace{8pt}}
		%
	    	@{\hspace{0pt}}c@{\hspace{0pt}}
		%
		@{\hspace{8pt}}c@{\hspace{8pt}}
		@{\hspace{8pt}}c@{\hspace{8pt}}
		%
		%@{\hspace{8pt}}r@{\hspace{8pt}}
		}
		\toprule
		$\delta_g$  & method          & \avgshapesim$\uparrow$(std) & \avggraphsim$\downarrow$(std) & & \maxshapesim$\uparrow$(std) & \maxgraphsim$\downarrow$(std)       \\ %& \#n\%$\uparrow$  \\ 
		\midrule
		%\multirow{5}{0.079\linewidth}%{\hspace{0pt}0.1} & \dataset   & 0.0             & 0.628(0.139)          & 0.567(0.068)          & 0.078(0.010)          &  & 0.588(0.086)          & 0.081(0.013)          & 4.7              \\
		%&  \squid($\lambda$=0.3) & 0.0             & 0.320(0.000)          & 0.420(0.163)          & \textbf{0.056}(0.032) &  & 0.461(0.170)          & \textbf{0.065}(0.033) & 1.4              \\
		%& \squid($\lambda$=1.0) & 0.0             & 0.414(0.177)          & 0.483(0.184)          & \underline{0.064}(0.030)  &  & 0.531(0.182)          & \underline{0.073}(0.029)  & 2.4              \\
		%& \method               & \underline{1.6}     & \textbf{0.857}(0.034) & \underline{0.773}(0.045)  & 0.086(0.011)          &  & \underline{0.791}(0.053)  & 0.087(0.012)          & \underline{5.1}      \\
		%& \methodwithsguide      & \textbf{3.7}    & \underline{0.833}(0.062)  & \textbf{0.812}(0.037) & 0.088(0.009)          &  & \textbf{0.835}(0.047) & 0.089(0.010)          & \textbf{6.2}     \\ 
		%\cmidrule{2-10}
		%& improv\% & - & 36.5 & 43.2 & -53.6 &  & 42.0 & -33.8 & 31.9  \\
		%\midrule
		\multirow{6}{0.059\linewidth}{\hspace{0pt}0.3} & \dataset             & 0.745(0.037)          & \textbf{0.211}(0.026) &  & 0.815(0.039)          & \textbf{0.215}(0.047)      \\ %    & \textbf{100.0}   \\
			& \squid($\lambda$=0.3) & 0.709(0.076)          & 0.237(0.033)          &  & 0.841(0.070)          & 0.253(0.038)        \\ %  & 45.5             \\
		    & \squid($\lambda$=1.0) & 0.695(0.064)          & \underline{0.216}(0.034)  &  & 0.841(0.056)          & 0.231(0.047)        \\ %  & 84.3             \\
			& \method               & \underline{0.770}(0.039)  & 0.217(0.031)          &  & \underline{0.858}(0.038)  & \underline{0.220}(0.046)  \\ %& \underline{87.1}     \\
			& \methodwithsguide     & \textbf{0.823}(0.029) & 0.217(0.032)          &  & \textbf{0.900}(0.028) & 0.223(0.048)  \\ % & 86.0             \\ 
		%\cmidrule{2-7}
		%& improv\% & 10.5 & -2.8 &  & 7.0 & -2.3  \\ % & %-12.9  \\
		\midrule
		\multirow{6}{0.059\linewidth}{\hspace{0pt}0.5} & \dataset & 0.750(0.037)          & \textbf{0.225}(0.037) &  & 0.819(0.039)          & \textbf{0.236}(0.070)          \\ %& \textbf{100.0}   \\
			& \squid($\lambda$=0.3)  & 0.728(0.072)          & 0.301(0.054)          &  & \underline{0.888}(0.061)  & 0.355(0.088)          \\ %& 85.9             \\
			& \squid($\lambda$=1.0)  & 0.699(0.063)          & 0.233(0.043)          &  & 0.850(0.057)          & 0.263(0.080)          \\ %& \underline{99.5}     \\
			& \method               & \underline{0.771}(0.039)  & \underline{0.229}(0.043)  &  & 0.862(0.036)          & \textbf{0.236}(0.065) \\ %& 99.2             \\
			& \methodwithsguide    & \textbf{0.824}(0.029) & \underline{0.229}(0.044)  &  & \textbf{0.903}(0.027) & \underline{0.242}(0.069)  \\ %& 99.0             \\ 
		%\cmidrule{2-7}
		%& improv\% & 9.9 & -1.8 &  & 1.7 & 0.0 \\ %& -0.8  \\
		\midrule
		\multirow{6}{0.059\linewidth}{\hspace{0pt}0.7} 
		& \dataset &  0.750(0.037) & \textbf{0.226}(0.038) & & 0.819(0.039) & \underline{0.240}(0.081) \\ %& \textbf{100.0} \\
		%& \dataset & 12.3            & 0.736(0.076)          & 0.768(0.037)          & \textbf{0.228}(0.042) &  & 0.819(0.039)          & \underline{0.242}(0.085)  & \textbf{100.0}   \\
			& \squid($\lambda$=0.3) &  0.735(0.074)          & 0.328(0.070)          &  & \underline{0.900}(0.062)  & 0.435(0.143)          \\ %& 95.4             \\
			& \squid($\lambda$=1.0) &  0.699(0.064)          & 0.234(0.045)          &  & 0.851(0.057)          & 0.268(0.090)          \\ %& \underline{99.9}     \\
			& \method               &  \underline{0.771}(0.039)  & \underline{0.229}(0.043)  &  & 0.862(0.036)          & \textbf{0.237}(0.066) \\ %& 99.3             \\
			& \methodwithsguide     &  \textbf{0.824}(0.029) & 0.230(0.045)          &  & \textbf{0.903}(0.027) & 0.244(0.074)          \\ %& 99.2             \\ 
		%\cmidrule{2-7}
		%& improv\% & 9.9 & -1.3 &  & 0.3 & 1.3 \\%& -0.7  \\
		\midrule
		\multirow{6}{0.059\linewidth}{\hspace{0pt}1.0} 
		& \dataset & 0.750(0.037)          & \textbf{0.226}(0.038) &  & 0.819(0.039)          & \underline{0.242}(0.085)  \\%& \textbf{100.0}  \\
		& \squid($\lambda$=0.3) & 0.740(0.076)          & 0.349(0.088)          &  & \textbf{0.909}(0.065) & 0.547(0.245)       \\ %   & \textbf{100.0}  \\
		& \squid($\lambda$=1.0) & 0.699(0.064)          & 0.235(0.045)          &  & 0.851(0.057)          & 0.271(0.097)          \\ %& \textbf{100.0}   \\
		& \method               & \underline{0.771}(0.039)  & \underline{0.229}(0.043)  &  & 0.862(0.036)          & \textbf{0.237}(0.066) \\ %& \underline{99.3}  \\
		& \methodwithsguide      & \textbf{0.824}(0.029) & 0.230(0.045)          &  & \underline{0.903}(0.027)  & 0.244(0.076)          \\ %& 99.2            \\
		%\cmidrule{2-7}
		%& improv\% &  9.9               & -1.3              &  & -0.7              & -2.1           \\ %       & -0.7 \\
		\bottomrule
	\end{tabular}%
	\begin{tablenotes}[normal,flushleft]
		\begin{footnotesize}
	\item 
\!\!Columns represent: ``$\delta_g$'': the graph similarity constraint; 
%``\#d\%'': the percentage of molecules that satisfy the graph similarity constraint and are with high \shapesim ($\shapesim>=0.8$);
%``\diversity'': the diversity among the generated molecules;
``\avgshapesim/\avggraphsim'': the average of shape or graph similarities between the condition molecules and generated molecules with $\graphsim<=\delta_g$;
``\maxshapesim'': the maximum of shape similarities between the condition molecules and generated molecules with $\graphsim<=\delta_g$;
``\maxgraphsim'': the graph similarities between the condition molecules and the molecules with the maximum shape similarities and $\graphsim<=\delta_g$;
%``\#n\%'': the percentage of molecules that satisfy the graph similarity constraint ($\graphsim<=\delta_g$).
%
``$\uparrow$'' represents higher values are better, and ``$\downarrow$'' represents lower values are better.
%
 Best values are in \textbf{bold}, and second-best values are \underline{underlined}. 
\par
		\par
		\end{footnotesize}
	\end{tablenotes}
\end{scriptsize}
\end{threeparttable}
  \vspace{-10pt}    
\end{table*}
%\label{tbl:overall_sim}


{We evaluate the shape similarity \shapesim and graph similarity \graphsim of molecules generated from}
%Table~\ref{tbl:overall_sim} presents the comparison of shape-conditioned molecule generation among 
\dataset, \squid, \method and \methodwithsguide under different graph similarity constraints  ($\delta_g$=1.0, 0.7, 0.5, 0.3). 
%
%During the evaluation, for each molecule in the test set, all the methods are employed to generate or identify 50 molecules with similar shapes.
%
We calculate evaluation metrics using all the generated molecules satisfying the graph similarity constraints.
%
Particularly, when $\delta_g$=1.0, we do not filter out any molecules based on the constraints and directly calculate metrics on all the generated molecules.
%
When $\delta_g$=0.7, 0.5 or 0.3, we consider only generated molecules with similarities lower than $\delta_g$.
%
Based on \shapesim and \graphsim as described in Section ``Evaluation Metrics'' in the main manuscript,
we calculate the following metrics using the subset of molecules with \graphsim lower than $\delta_g$, from a set of 50 generated molecules for each test molecule and report the average of  these metrics across all test molecules:
%
(1) \avgshapesim\ measures the average \shapesim across each subset of generated molecules with $\graphsim$ lower than $\delta_g$; %per test molecule, with the overall average calculated across all test molecules; }%the 50 generated molecules for each test molecule, averaged across all test molecules;
(2) \avggraphsim\ calculates the average \graphsim for each set; %, with these means averaged across all test molecules}; %} 50 molecules, %\bo{@Ziqi rephrase}, with results averaged on the test set;\ziqi{with the average computed over the test set; }
(3) \maxshapesim\ determines the maximum \shapesim within each set; %, with these maxima averaged across all test molecules; }%\hl{among 50 molecules}, averaged across all test molecules;
(4) \maxgraphsim\ measures the \graphsim of the molecule with maximum \shapesim in each set. %, averaged across all test molecules; }%\hl{among 50 molecules}, averaged across all test molecules;

%
As shown in Table~\ref{tbl:overall_sim}, \method and \methodwithsguide demonstrate outstanding performance in terms of the average shape similarities (\avgshapesim) and the average graph similarities (\avggraphsim) among generated molecules.
%
%\ziqi{
%Table~\ref{tbl:overall} also shows that \method and \methodwithsguide consistently outperform all the baseline methods in average shape similarities (\avgshapesim) and only slightly underperform 
%the best baseline \dataset in average graph similarities (\avggraphsim).
%}
%
Specifically, when $\delta_g$=0.3, \methodwithsguide achieves a substantial 10.5\% improvement in \avgshapesim\ over the best baseline \dataset. 
%
In terms of \avggraphsim, \methodwithsguide also achieves highly comparable performance with \dataset (0.217 vs 0.211, in \avggraphsim, lower values indicate better performance).
%
%This trend remains consistent across various $\delta_g$ values.
This trend remains consistent when applying various similarity constraints (i.e., $\delta_g$) as shown in Table~\ref{tbl:overall_sim}.


Similarly, \method and \methodwithsguide demonstrate superior performance in terms of the average maximum shape similarity across generated molecules for all test molecules (\maxshapesim), as well as the average graph similarity of the molecules with the maximum shape similarities (\maxgraphsim). %maximum shape similarities of generated molecules (\maxshapesim) and the average graph similarities of molecules with the maximum shape similarities (\maxgraphsim). %\bo{\maxgraphsim is misleading... how about $\text{avgMSim}_\text{g}$}
%
%\bo{
%in terms of the maximum shape similarities (\maxshapesim) and the maximum graph similarities (\maxgraphsim) among all the generated molecules.
%@Ziqi are the metrics maximum values or the average of maximum values?
%}
%
Specifically, at \maxshapesim, Table~\ref{tbl:overall_sim} shows that \methodwithsguide outperforms the best baseline \squid ($\lambda$=0.3) when $\delta_g$=0.3, 0.5, and 0.7, and only underperforms
it by 0.7\% when $\delta$=1.0.
%
We also note that the molecules generated by {\methodwithsguide} with the maximum shape similarities have substantially lower graph similarities ({\maxgraphsim}) compared to those generated by {\squid} ({$\lambda$}=0.3).
%\hl{We also note that the molecules with the maximum shape similarities generated by {\methodwithsguide} are with significantly lower graph similarities ({\maxgraphsim}) than those generated by {\squid} ({$\lambda$}=0.3).}
%
%\bo{@Ziqi please rephrase the language}
%
%\bo{
%@Ziqi the conclusion is not obvious. You may want to remind the meaning of \maxshapesim and \maxgraphsim here, and based on what performance you say this.
%}
%
%\bo{\st{This also underscores the ability of {\methodwithsguide} in generating molecules with similar shapes to condition molecules and novel graph structures.}}
%
As evidenced by these results, \methodwithsguide features strong capacities of generating molecules with similar shapes yet novel graph structures compared to the condition molecule, facilitating the discovery of promising drug candidates.
%

\begin{comment}
\ziqi{replace \#n\% with the percentage of novel molecules that do not exist in the dataset and update the discussion accordingly}
%\ziqi{
Table~\ref{tbl:overall_sim} also presents \bo{\#n\%}, the percentage of molecules generated by each method %\st{(\#n\%)} 
with graph similarities lower than the constraint $\delta_g$. 
%
%\bo{
%Table~\ref{tbl:overall_sim} also presents \#n\%, the percentage of generated molecules with graph similarities lower than the constraint $\delta_g$, of different methods. 
%}
%
As shown in Table~\ref{tbl:overall_sim},  when a restricted constraint (i.e., $\delta_g$=0.3) is applied, \method and \methodwithsguide could still generate a sufficient number of molecules satisfying the constraint.
%
Particularly, when $\delta_g$=0.3, \method outperforms \squid with $\lambda$=0.3 by XXX and \squid with $\lambda$=1.0 by XXX.
% achieve the second and the third in \#n\% and only underperform the best baseline \dataset.
%
This demonstrates the ability of \method in generating molecules with novel structures. 
%
When $\delta_g$=0.5, 0.7 and 1.0, both methods generate over 99.0\% of molecules satisfying the similarity constraint $\delta_g$.
%
%Note that \dataset is guaranteed to identify at least 50 molecules satisfying the $\delta_g$ by searching within a training dataset of diverse molecules.
%
Note that \dataset is a search algorithm that always first identifies the molecules satisfying $\delta_g$ and then selects the top-50 molecules of the highest shape similarities among them. 
%
Due to the diverse molecules in %\hl{the subset} \bo{@Ziqi why do you want to stress subset?} of 
the training set, \dataset can always identify at least 50 molecules under different $\delta_g$ and thus achieve 100\% in \#n\%.
%
%\bo{
%Note that \dataset is a search algorithm that always generate molecules XXX
%@Ziqi
%We need to discuss here. For \dataset, \#n\% in this table does not look aligned with that in Fig 1 if the highlighted defination is correct...
%}
%
%Thus, \dataset achieves 100.0\% in \#n\% under different $\delta_g$.
%
It is also worth noting that when $\delta_g$=1.0, \#n\% reflects the validity among all the generated molecules. 
%
As shown in Table~\ref{tbl:overall_sim}, \method and \methodwithsguide are able to generate 99.3\% and 99.2\% valid molecules.
%
This demonstrates their ability to effectively capture the underlying chemical rules in a purely data-driven manner without relying on any prior knowledge (e.g., fragments) as \squid does.
%
%\bo{
%@Ziqi I feel this metric is redundant with the avg graph similarity when constraint is 1.0. Generally, if the avg similarity is small. You have more mols satisfying the requirement right?
%}
\end{comment}

Table~\ref{tbl:overall_sim} also shows that by incorporating shape guidance, \methodwithsguide
%\bo{
%@Ziqi where does this come from...
%}
substantially outperforms \method in both \avgshapesim and \maxshapesim, while maintaining comparable graph similarities (i.e., \avggraphsim\ and \maxgraphsim).
%
Particularly, when $\delta_g$=0.3, \methodwithsguide 
establishes a considerable improvement of 6.9\% and 4.9\%
%\bo{\st{achieves 6.9\% and 4.9\% improvements}} 
over \method in \avgshapesim and \maxshapesim, respectively. 
%
%\hl{In the meanwhile}, 
%\bo{@Ziqi it is not the right word...}
Meanwhile, \methodwithsguide achieves the same \avggraphsim with \method and only slightly underperforms \method in \maxgraphsim (0.223 vs 0.220).
%\bo{
%XXX also achieves XXX
%}
%it maintains the same \avggraphsim\ with \method and only slightly underperforms \method in \maxgraphsim (0.223 vs 0.220).
%
%Compared with \method, \methodwithsguide consistently generates molecules with higher shape similarities while maintaining comparable graph similarities.
%
%\bo{
%@Ziqi you may want to highlight the utility of "generating molecules with higher shape similarities while maintaining comparable graph similarities" in real drug discovery applications.
%
%
%\bo{
%@Ziqi You did not present the details of method yet...
%}
%
%\methodwithsguide leverages additional shape guidance to push the predicted atoms to the shape of condition molecules \bo{and XXX (@Ziqi boosts the shape similarities XXX)} , as will be discussed in Section ``\method with Shape Guidance'' later.
%
The superior performance of \methodwithsguide suggests that the incorporation of shape guidance effectively boosts the shape similarities of generated molecules without compromising graph similarities.
%
%This capability could be crucial in drug discovery, 
%\bo{@Ziqi it is a strong statement. Need citations here}, 
%as it enables the discovery of drug candidates that are both more potentially effective due to the improved shape similarities and novel induced by low graph similarities.
%as it could enable the identification of candidates with similar binding patterns %with the condition molecule (i.e., high shape similarities) 
%(i.e., high shape similarities) and graph structures distinct from the condition molecules (i.e., low graph similarities).
%\bo{\st{and enjoys novel structures (i.e., low graph similarities) with potentially better properties. } \ziqi{change enjoys}}
%\bo{
%and enjoys potentially better properties (i.e., low graph similarities). \ziqi{this looks weird to me... need to discuss}
%}
%\st{potentially better properties (i.e., low graph similarities).}}

%-------------------------------------------------------------------------------------------------------------------------------------
\subsection{Comparison on Validity and Novelty}
\label{supp:app:results:valid_novel}
%-------------------------------------------------------------------------------------------------------------------------------------

We evaluate the ability of \method and \squid to generate molecules with valid and novel 2D molecular graphs.
%
We calculate the percentages of the valid and novel molecules among all the generated molecules.
%
As shown in Table~\ref{tbl:validity_novelty}, both \method and \methodwithsguide outperform \squid with $\lambda$=0.3 and $\lambda$=1.0 in generating novel molecules.
%
Particularly, almost all valid molecules generated by \method and \methodwithsguide are novel (99.8\% and 99.9\% at \#n\%), while the best baseline \squid with $\lambda$=0.3 achieves 98.4\% in novelty.
%
In terms of the percentage of valid and novel molecules among all the generated ones (\#v\&n\%), \method and \methodwithsguide again outperform \squid with $\lambda$=0.3 and $\lambda$=1.0.
%
We also note that at \#v\%,  \method (99.1\%) and \methodwithsguide (99.2\%) slightly underperform \squid with $\lambda$=0.3 and $\lambda$=1.0 (100.0\%) in generating valid molecules.
%
\squid guarantees the validity of generated molecules by incorporating valence rules into the generation process and ensuring it to avoid fragments that violate these rules.
%
Conversely, \method and \methodwithsguide use a purely data-driven approach to learn the generation of valid molecules.
%
These results suggest that, even without integrating valence rules, \method and \methodwithsguide can still achieve a remarkably high percentage of valid and novel generated molecules.

\begin{table*}
	\centering
		\caption{Comparison on Validity and Novelty between \method and \squid}
	\label{tbl:validity_novelty}
	\begin{scriptsize}
\begin{threeparttable}
%	\setlength\tabcolsep{0pt}
	\begin{tabular}{
		@{\hspace{3pt}}l@{\hspace{10pt}}
		%
		@{\hspace{10pt}}r@{\hspace{10pt}}
		%
		@{\hspace{10pt}}r@{\hspace{10pt}}
		%
		@{\hspace{10pt}}r@{\hspace{3pt}}
		}
		\toprule
		method & \#v\% & \#n\% & \#v\&n\% \\
		\midrule
		\squid ($\lambda$=0.3) & \textbf{100.0} & 96.7 & 96.7 \\
		\squid ($\lambda$=1.0) & \textbf{100.0} & 98.4 & 98.4 \\
		\method & 99.1 & 99.8 & 98.9 \\
		\methodwithsguide & 99.2 & \textbf{99.9} & \textbf{99.1} \\
		\bottomrule
	\end{tabular}%
	%
	\begin{tablenotes}[normal,flushleft]
		\begin{footnotesize}
	\item 
\!\!Columns represent: ``\#v\%'': the percentage of generated molecules that are valid;
		``\#n\%'': the percentage of valid molecules that are novel;
		``\#v\&n\%'': the percentage of generated molecules that are valid and novel.
		Best values are in \textbf{bold}. 
		\par
		\end{footnotesize}
	\end{tablenotes}
\end{threeparttable}
\end{scriptsize}
\end{table*}


%-------------------------------------------------------------------------------------------------------------------------------------
\subsection{Additional Quality Comparison between Desirable Molecules Generated by \method and \squid}
\label{supp:app:results:quality_desirable}
%-------------------------------------------------------------------------------------------------------------------------------------

\begin{table*}[!h]
	\centering
		\caption{Comparison on Quality of Generated Desirable Molecules between \method and \squid ($\delta_g$=0.5)}
	\label{tbl:overall_results_quality_05}
	\begin{scriptsize}
\begin{threeparttable}
	\begin{tabular}{
		@{\hspace{0pt}}l@{\hspace{16pt}}
		@{\hspace{0pt}}l@{\hspace{2pt}}
		%
		@{\hspace{6pt}}c@{\hspace{6pt}}
		%
		%@{\hspace{3pt}}c@{\hspace{3pt}}
		@{\hspace{3pt}}c@{\hspace{3pt}}
		@{\hspace{3pt}}c@{\hspace{3pt}}
		@{\hspace{3pt}}c@{\hspace{3pt}}
		@{\hspace{3pt}}c@{\hspace{3pt}}
		%
		%
		}
		\toprule
		group & metric & 
        %& \dataset 
        & \squid ($\lambda$=0.3) & \squid ($\lambda$=1.0)  &  \method & \methodwithsguide  \\
		%\multirow{2}{*}{method} & \multirow{2}{*}{\#c\%} &  \multirow{2}{*}{\#u\%} &  \multirow{2}{*}{QED} & \multicolumn{3}{c}{$\nmax=50$} & & \multicolumn{2}{c}{$\nmax=1$}\\
		%\cmidrule(r){5-7} \cmidrule(r){8-10} 
		%& & & & \avgshapesim(std) & \avggraphsim(std  &  \diversity(std  & & \avgshapesim(std) & \avggraphsim(std \\
		\midrule
		\multirow{2}{*}{stability}
		& atom stability ($\uparrow$) & 
        %& 0.990 
        & \textbf{0.996} & 0.995 & 0.992 & 0.989     \\
		& mol stability ($\uparrow$) & 
        %& 0.875 
        & \textbf{0.948} & 0.947 & 0.886 & 0.839    \\
		%\midrule
		%\multirow{3}{*}{Drug-likeness} 
		%& QED ($\uparrow$) & 
        %& \textbf{0.805} 
        %& 0.766 & 0.760 & 0.755 & 0.751    \\
	%	& SA ($\uparrow$) & 
        %& \textbf{0.874} 
        %& 0.814 & 0.813 & 0.699 & 0.692    \\
	%	& Lipinski ($\uparrow$) & 
        %& \textbf{4.999} 
        %& 4.979 & 4.980 & 4.967 & 4.975    \\
		\midrule
		\multirow{4}{*}{3D structures} 
		& RMSD ($\downarrow$) & 
        %& \textbf{0.419} 
        & 0.907 & 0.906 & 0.897 & \textbf{0.881}    \\
		& JS. bond lengths ($\downarrow$) & 
        %& \textbf{0.286} 
        & 0.457 & 0.477 & 0.436 & \textbf{0.428}    \\
		& JS. bond angles ($\downarrow$) & 
        %& \textbf{0.078} 
        & 0.269 & 0.289 & \textbf{0.186} & 0.200    \\
		& JS. dihedral angles ($\downarrow$) & 
        %& \textbf{0.151} 
        & 0.199 & 0.209 & \textbf{0.168} & 0.170    \\
		\midrule
		\multirow{5}{*}{2D structures} 
		& JS. \#bonds per atoms ($\downarrow$) & 
        %& 0.325 
        & 0.291 & 0.331 & \textbf{0.176} & 0.181    \\
		& JS. basic bond types ($\downarrow$) & 
        %& \textbf{0.055} 
        & \textbf{0.071} & 0.083 & 0.181 & 0.191    \\
		%& JS. freq. bond types ($\downarrow$) & 
        %& \textbf{0.089} 
        %& 0.123 & 0.130 & 0.245 & 0.254    \\
		%& JS. freq. bond pairs ($\downarrow$) & 
        %& \textbf{0.078} 
        %& 0.085 & 0.089 & 0.209 & 0.221    \\
		%& JS. freq. bond triplets ($\downarrow$) & 
        %& \textbf{0.089} 
        %& 0.097 & 0.114 & 0.211 & 0.223    \\
		%\midrule
		%\multirow{3}{*}{Rings} 
		& JS. \#rings ($\downarrow$) & 
        %& 0.142 
        & 0.280 & 0.330 & \textbf{0.043} & 0.049    \\
		& JS. \#n-sized rings ($\downarrow$) & 
        %& \textbf{0.055} 
        & \textbf{0.077} & 0.091 & 0.099 & 0.112    \\
		& \#Intersecting rings ($\uparrow$) & 
        %& \textbf{6} 
        & \textbf{6} & 5 & 4 & 5    \\
		%\method (+bt)            & 100.0 & 98.0 & 100.0 & 0.742 & 0.772 (0.040) & 0.211 (0.033) & & 0.862 (0.036) & 0.211 (0.033) & 0.743 (0.043) \\
		%\methodwithguide (+bt)    & 99.8 & 98.0 & 100.0 & 0.736 & 0.814 (0.031) & 0.193 (0.042) & & 0.895 (0.029) & 0.193 (0.042) & 0.745 (0.045) \\
		%
		\bottomrule
	\end{tabular}%
	\begin{tablenotes}[normal,flushleft]
		\begin{footnotesize}
	\item 
\!\!Rows represent:  {``atom stability'': the proportion of stable atoms that have the correct valency; 
		``molecule stability'': the proportion of generated molecules with all atoms stable;
		%``QED'': the drug-likeness score;
		%``SA'': the synthesizability score;
		%``Lipinski'': the Lipinski 
		``RMSD'': the root mean square deviation (RMSD) between the generated 3D structures of molecules and their optimal conformations; % identified via energy minimization;
		``JS. bond lengths/bond angles/dihedral angles'': the Jensen-Shannon (JS) divergences of bond lengths, bond angles and dihedral angles;
		``JS. \#bonds per atom/basic bond types/\#rings/\#n-sized rings'': the JS divergences of bond counts per atom, basic bond types, counts of all rings, and counts of n-sized rings;
		%``JS. \#rings/\#n-sized rings'': the JS divergences of the total counts of rings and the counts of n-sized rings;
		``\#Intersecting rings'': the number of rings observed in the top-10 frequent rings of both generated and real molecules. } \par
		\par
		\end{footnotesize}
	\end{tablenotes}
\end{threeparttable}
\end{scriptsize}
\end{table*}

%\label{tbl:overall_quality05}

\begin{table*}[!h]
	\centering
		\caption{Comparison on Quality of Generated Desirable Molecules between \method and \squid ($\delta_g$=0.7)}
	\label{tbl:overall_results_quality_07}
	\begin{scriptsize}
\begin{threeparttable}
	\begin{tabular}{
		@{\hspace{0pt}}l@{\hspace{14pt}}
		@{\hspace{0pt}}l@{\hspace{2pt}}
		%
		@{\hspace{4pt}}c@{\hspace{4pt}}
		%
		%@{\hspace{3pt}}c@{\hspace{3pt}}
		@{\hspace{3pt}}c@{\hspace{3pt}}
		@{\hspace{3pt}}c@{\hspace{3pt}}
		@{\hspace{3pt}}c@{\hspace{3pt}}
		@{\hspace{3pt}}c@{\hspace{3pt}}
		%
		%
		}
		\toprule
		group & metric & 
        %& \dataset 
        & \squid ($\lambda$=0.3) & \squid ($\lambda$=1.0)  &  \method & \methodwithsguide  \\
		%\multirow{2}{*}{method} & \multirow{2}{*}{\#c\%} &  \multirow{2}{*}{\#u\%} &  \multirow{2}{*}{QED} & \multicolumn{3}{c}{$\nmax=50$} & & \multicolumn{2}{c}{$\nmax=1$}\\
		%\cmidrule(r){5-7} \cmidrule(r){8-10} 
		%& & & & \avgshapesim(std) & \avggraphsim(std  &  \diversity(std  & & \avgshapesim(std) & \avggraphsim(std \\
		\midrule
		\multirow{2}{*}{stability} 
		& atom stability ($\uparrow$) & 
        %&  0.990 
        & \textbf{0.995} & 0.995 & 0.992 & 0.988 \\
		& molecule stability ($\uparrow$) & 
        %& 0.876 
        & 0.944 & \textbf{0.947} & 0.885 & 0.839 \\
		\midrule
		%\multirow{3}{*}{Drug-likeness} 
		%& QED ($\uparrow$) & 
        %& \textbf{0.805} 
        %& 0.766 & 0.760 & 0.755 & 0.751    \\
	%	& SA ($\uparrow$) & 
        %& \textbf{0.874} 
        %& 0.814 & 0.813 & 0.699 & 0.692    \\
	%	& Lipinski ($\uparrow$) & 
        %& \textbf{4.999} 
        %& 4.979 & 4.980 & 4.967 & 4.975    \\
	%	\midrule
		\multirow{4}{*}{3D structures} 
		& RMSD ($\downarrow$) & 
        %& \textbf{0.420} 
        & 0.897 & 0.906 & 0.897 & \textbf{0.881}    \\
		& JS. bond lengths ($\downarrow$) & 
        %& \textbf{0.286} 
        & 0.457 & 0.477 & 0.436 & \textbf{0.428}    \\
		& JS. bond angles ($\downarrow$) & 
        %& \textbf{0.078} 
        & 0.269 & 0.289 & \textbf{0.186} & 0.200    \\
		& JS. dihedral angles ($\downarrow$) & 
        %& \textbf{0.151} 
        & 0.199 & 0.209 & \textbf{0.168} & 0.170    \\
		\midrule
		\multirow{5}{*}{2D structures} 
		& JS. \#bonds per atoms ($\downarrow$) & 
        %& 0.325 
        & 0.285 & 0.329 & \textbf{0.176} & 0.181    \\
		& JS. basic bond types ($\downarrow$) & 
        %& \textbf{0.055} 
        & \textbf{0.067} & 0.083 & 0.181 & 0.191    \\
	%	& JS. freq. bond types ($\downarrow$) & 
        %& \textbf{0.089} 
        %& 0.123 & 0.130 & 0.245 & 0.254    \\
	%	& JS. freq. bond pairs ($\downarrow$) & 
        %& \textbf{0.078} 
        %& 0.085 & 0.089 & 0.209 & 0.221    \\
	%	& JS. freq. bond triplets ($\downarrow$) & 
        %& \textbf{0.089} 
        %& 0.097 & 0.114 & 0.211 & 0.223    \\
	%	\midrule
	%	\multirow{3}{*}{Rings} 
		& JS. \#rings ($\downarrow$) & 
        %& 0.143 
        & 0.273 & 0.328 & \textbf{0.043} & 0.049    \\
		& JS. \#n-sized rings ($\downarrow$) & 
        %& \textbf{0.055} 
        & \textbf{0.076} & 0.091 & 0.099 & 0.112    \\
		& \#Intersecting rings ($\uparrow$) & 
        %& \textbf{6} 
        & \textbf{6} & 5 & 4 & 5    \\
		%\method (+bt)            & 100.0 & 98.0 & 100.0 & 0.742 & 0.772 (0.040) & 0.211 (0.033) & & 0.862 (0.036) & 0.211 (0.033) & 0.743 (0.043) \\
		%\methodwithguide (+bt)    & 99.8 & 98.0 & 100.0 & 0.736 & 0.814 (0.031) & 0.193 (0.042) & & 0.895 (0.029) & 0.193 (0.042) & 0.745 (0.045) \\
		%
		\bottomrule
	\end{tabular}%
	\begin{tablenotes}[normal,flushleft]
		\begin{footnotesize}
	\item 
\!\!Rows represent:  {``atom stability'': the proportion of stable atoms that have the correct valency; 
		``molecule stability'': the proportion of generated molecules with all atoms stable;
		%``QED'': the drug-likeness score;
		%``SA'': the synthesizability score;
		%``Lipinski'': the Lipinski 
		``RMSD'': the root mean square deviation (RMSD) between the generated 3D structures of molecules and their optimal conformations; % identified via energy minimization;
		``JS. bond lengths/bond angles/dihedral angles'': the Jensen-Shannon (JS) divergences of bond lengths, bond angles and dihedral angles;
		``JS. \#bonds per atom/basic bond types/\#rings/\#n-sized rings'': the JS divergences of bond counts per atom, basic bond types, counts of all rings, and counts of n-sized rings;
		%``JS. \#rings/\#n-sized rings'': the JS divergences of the total counts of rings and the counts of n-sized rings;
		``\#Intersecting rings'': the number of rings observed in the top-10 frequent rings of both generated and real molecules. } \par
		\par
		\end{footnotesize}
	\end{tablenotes}
\end{threeparttable}
\end{scriptsize}
\end{table*}

%\label{tbl:overall_quality07}

\begin{table*}[!h]
	\centering
		\caption{Comparison on Quality of Generated Desirable Molecules between \method and \squid ($\delta_g$=1.0)}
	\label{tbl:overall_results_quality_10}
	\begin{scriptsize}
\begin{threeparttable}
	\begin{tabular}{
		@{\hspace{0pt}}l@{\hspace{14pt}}
		@{\hspace{0pt}}l@{\hspace{2pt}}
		%
		@{\hspace{4pt}}c@{\hspace{4pt}}
		%
		%@{\hspace{3pt}}c@{\hspace{3pt}}
		@{\hspace{3pt}}c@{\hspace{3pt}}
		@{\hspace{3pt}}c@{\hspace{3pt}}
		@{\hspace{3pt}}c@{\hspace{3pt}}
		@{\hspace{3pt}}c@{\hspace{3pt}}
		%
		%
		}
		\toprule
		group & metric & 
        %& \dataset 
        & \squid ($\lambda$=0.3) & \squid ($\lambda$=1.0)  &  \method & \methodwithsguide \\
		%\multirow{2}{*}{method} & \multirow{2}{*}{\#c\%} &  \multirow{2}{*}{\#u\%} &  \multirow{2}{*}{QED} & \multicolumn{3}{c}{$\nmax=50$} & & \multicolumn{2}{c}{$\nmax=1$}\\
		%\cmidrule(r){5-7} \cmidrule(r){8-10} 
		%& & & & \avgshapesim(std) & \avggraphsim(std  &  \diversity(std  & & \avgshapesim(std) & \avggraphsim(std \\
		\midrule
		\multirow{2}{*}{stability}
		& atom stability ($\uparrow$) & 
        %& 0.990 
        & \textbf{0.995} & \textbf{0.995} & 0.992 & 0.988     \\
		& mol stability ($\uparrow$) & 
        %& 0.876 
        & 0.942 & \textbf{0.947} & 0.885 & 0.839    \\
		\midrule
	%	\multirow{3}{*}{Drug-likeness} 
	%	& QED ($\uparrow$) & 
        %& \textbf{0.805} 
        %& \textbf{0.766} & 0.760 & 0.755 & 0.751    \\
	%	& SA ($\uparrow$) & 
        %& \textbf{0.874} 
        %& \textbf{0.813} & \textbf{0.813} & 0.699 & 0.692    \\
	%	& Lipinski ($\uparrow$) & 
        %& \textbf{4.999} 
        %& 4.979 & \textbf{4.980} & 4.967 & 4.975    \\
	%	\midrule
		\multirow{4}{*}{3D structures} 
		& RMSD ($\downarrow$) & 
        %& \textbf{0.420} 
        & 0.898 & 0.906 & 0.897 & \textbf{0.881}    \\
		& JS. bond lengths ($\downarrow$) & 
        %& \textbf{0.286} 
        & 0.457 & 0.477 & 0.436 & \textbf{0.428}    \\
		& JS. bond angles ($\downarrow$) & 
        %& \textbf{0.078} 
        & 0.269 & 0.289 & \textbf{0.186} & 0.200   \\
		& JS. dihedral angles ($\downarrow$) & 
        %& \textbf{0.151} 
        & 0.199 & 0.209 & \textbf{0.168} & 0.170    \\
		\midrule
		\multirow{5}{*}{2D structures} 
		& JS. \#bonds per atoms ($\downarrow$) & 
        %& 0.325 
        & 0.280 & 0.330 & \textbf{0.176} & 0.181    \\
		& JS. basic bond types ($\downarrow$) & 
        %& \textbf{0.055} 
        & \textbf{0.066} & 0.083 & 0.181 & 0.191   \\
	%	& JS. freq. bond types ($\downarrow$) & 
        %& \textbf{0.089} 
        %& \textbf{0.123} & 0.130 & 0.245 & 0.254    \\
	%	& JS. freq. bond pairs ($\downarrow$) & 
        %& \textbf{0.078} 
        %& \textbf{0.085} & 0.089 & 0.209 & 0.221    \\
	%	& JS. freq. bond triplets ($\downarrow$) & 
        %& \textbf{0.089} 
        %& \textbf{0.097} & 0.114 & 0.211 & 0.223    \\
		%\midrule
		%\multirow{3}{*}{Rings} 
		& JS. \#rings ($\downarrow$) & 
        %& 0.143 
        & 0.269 & 0.328 & \textbf{0.043} & 0.049    \\
		& JS. \#n-sized rings ($\downarrow$) & 
        %& \textbf{0.055} 
        & \textbf{0.075} & 0.091 & 0.099 & 0.112    \\
		& \#Intersecting rings ($\uparrow$) & 
        %& \textbf{6} 
        & \textbf{6} & 5 & 4 & 5    \\
		%\method (+bt)            & 100.0 & 98.0 & 100.0 & 0.742 & 0.772 (0.040) & 0.211 (0.033) & & 0.862 (0.036) & 0.211 (0.033) & 0.743 (0.043) \\
		%\methodwithguide (+bt)    & 99.8 & 98.0 & 100.0 & 0.736 & 0.814 (0.031) & 0.193 (0.042) & & 0.895 (0.029) & 0.193 (0.042) & 0.745 (0.045) \\
		%
		\bottomrule
	\end{tabular}%
	\begin{tablenotes}[normal,flushleft]
		\begin{footnotesize}
	\item 
\!\!Rows represent:  {``atom stability'': the proportion of stable atoms that have the correct valency; 
		``molecule stability'': the proportion of generated molecules with all atoms stable;
		%``QED'': the drug-likeness score;
		%``SA'': the synthesizability score;
		%``Lipinski'': the Lipinski 
		``RMSD'': the root mean square deviation (RMSD) between the generated 3D structures of molecules and their optimal conformations; % identified via energy minimization;
		``JS. bond lengths/bond angles/dihedral angles'': the Jensen-Shannon (JS) divergences of bond lengths, bond angles and dihedral angles;
		``JS. \#bonds per atom/basic bond types/\#rings/\#n-sized rings'': the JS divergences of bond counts per atom, basic bond types, counts of all rings, and counts of n-sized rings;
		%``JS. \#rings/\#n-sized rings'': the JS divergences of the total counts of rings and the counts of n-sized rings;
		``\#Intersecting rings'': the number of rings observed in the top-10 frequent rings of both generated and real molecules. } \par
		\par
		\end{footnotesize}
	\end{tablenotes}
\end{threeparttable}
\end{scriptsize}
\end{table*}

%\label{tbl:overall_quality10}

Similar to Table~\ref{tbl:overall_results_quality_desired} in the main manuscript, we present the performance comparison on the quality of desirable molecules generated by different methods under different graph similarity constraints $\delta_g$=0.5, 0.7 and 1.0, as detailed in Table~\ref{tbl:overall_results_quality_05}, Table~\ref{tbl:overall_results_quality_07}, and Table~\ref{tbl:overall_results_quality_10}, respectively.
%
Overall, these tables show that under varying graph similarity constraints, \method and \methodwithsguide can always generate desirable molecules with comparable quality to baselines in terms of stability, 3D structures, and 2D structures.
%
These results demonstrate the strong effectiveness of \method and \methodwithsguide in generating high-quality desirable molecules with stable and realistic structures in both 2D and 3D.
%
This enables the high utility of \method and \methodwithsguide in discovering promising drug candidates.


\begin{comment}
The results across these tables demonstrate similar observations with those under $\delta_g$=0.3 in Table~\ref{tbl:overall_results_quality_desired}.
%
For stability, when $\delta_g$=0.5, 0.7 or 1.0, \method and \methodwithsguide achieve comparable performance or fall slightly behind \squid ($\lambda$=0.3) and \squid ($\lambda$=1.0) in atom stability and molecule stability.
%
For example, when $\delta_g$=0.5, as shown in Table~\ref{tbl:overall_results_quality_05}, \method achieves similar performance with the best baseline \squid ($\lambda$=0.3) in atom stability (0.992 for \method vs 0.996 for \squid with $\lambda$=0.3).
%
\method underperforms \squid ($\lambda$=0.3) in terms of molecule stability.
%
For 3D structures, \method and \methodwithsguide also consistently generate molecules with more realistic 3D structures compared to \squid.
%
Particularly, \methodwithsguide achieves the best performance in RMSD and JS of bond lengths across $\delta_g$=0.5, 0.7 and 1.0.
%
In JS of dihedral angles, \method achieves the best performance among all the methods.
%
\method and \methodwithsguide underperform \squid in JS of bond angles, primarily because \squid constrains the bond angles in the generated molecules.
%
For 2D structures, \method and \methodwithsguide again achieve the best performance 
\end{comment}

%===================================================================
\section{Additional Experimental Results on PMG}
\label{supp:app:results_PMG}
%===================================================================

%\label{tbl:comparison_results_decompdiff}


%-------------------------------------------------------------------------------------------------------------------------------------
%\subsection{{Additional Comparison for PMG}}
%\label{supp:app:results:docking}
%-------------------------------------------------------------------------------------------------------------------------------------

In this section, we present the results of \methodwithpguide and \methodwithsandpguide when generating 100 molecules. 
%
Please note that both \methodwithpguide and \methodwithsandpguide show remarkable efficiency over the PMG baselines.
%
\methodwithpguide and \methodwithsandpguide generate 100 molecules in 48 and 58 seconds on average, respectively, while the most efficient baseline \targetdiff requires 1,252 seconds.
%
We report the performance of \methodwithpguide and \methodwithsandpguide against state-of-the-art PMG baselines in Table~\ref{tbl:overall_results_docking_100}.


%
According to Table~\ref{tbl:overall_results_docking_100}, \methodwithpguide and \methodwithsandpguide achieve comparable performance with the PMG baselines in generating molecules with high binding affinities.
%
Particularly, in terms of Vina S, \methodwithsandpguide achieves very comparable performance (-4.56 kcal/mol) to the third-best baseline \decompdiff (-4.58 kcal/mol) in average Vina S; it also achieves the third-best performance (-4.82 kcal/mol) among all the methods and slightly underperforms the second-best baseline \AR (-4.99 kcal/mol) in median Vina S
%
\methodwithsandpguide also achieves very close average Vina M (-5.53 kcal/mol) with the third-best baseline \AR (-5.59 kcal/mol) and the third-best performance (-5.47 kcal/mol) in median Vina M.
%
Notably, for Vina D, \methodwithpguide and \methodwithsandpguide achieve the second and third performance among all the methods.
%
In terms of the average percentage of generated molecules with Vina D higher than those of known ligands (i.e., HA), \methodwithpguide (58.52\%) and \methodwithsandpguide (58.28\%) outperform the best baseline \targetdiff (57.57\%).
%
These results signify the high utility of \methodwithpguide and \methodwithsandpguide in generating molecules that effectively bind with protein targets and have better binding affinities than known ligands.

In addition to binding affinities, \methodwithpguide and \methodwithsandpguide also demonstrate similar performance compared to the baselines in metrics related to drug-likeness and diversity.
%
For drug-likeness, both \methodwithpguide and \methodwithsandpguide achieve the best (0.67) and the second-best (0.66) QED scores.
%
They also achieve the third and fourth performance in SA scores.
%
In terms of the diversity among generated molecules,  \methodwithpguide and \methodwithsandpguide slightly underperform the baselines, possibly due to the design that generates molecules with similar shapes to the ligands.
%
These results highlight the strong ability of \methodwithpguide and \methodwithsandpguide in efficiently generating effective binding molecules with favorable drug-likeness and diversity.
%
This ability enables them to potentially serve as promising tools to facilitate effective and efficient drug development.

\begin{table*}[!h]
	\centering
		\caption{Additional Comparison on PMG When All Methods Generate 100 Molecules}
	\label{tbl:overall_results_docking_100}
\begin{threeparttable}
	\begin{scriptsize}
	\begin{tabular}{
		@{\hspace{2pt}}l@{\hspace{2pt}}
		%
		@{\hspace{2pt}}r@{\hspace{2pt}}
		%
		@{\hspace{2pt}}r@{\hspace{2pt}}
		@{\hspace{2pt}}r@{\hspace{2pt}}
		%
		@{\hspace{6pt}}r@{\hspace{6pt}}
		%
		@{\hspace{2pt}}r@{\hspace{2pt}}
		@{\hspace{2pt}}r@{\hspace{2pt}}
		%
		@{\hspace{5pt}}r@{\hspace{5pt}}
		%
		@{\hspace{2pt}}r@{\hspace{2pt}}
		@{\hspace{2pt}}r@{\hspace{2pt}}
		%
		@{\hspace{5pt}}r@{\hspace{5pt}}
		%
		@{\hspace{2pt}}r@{\hspace{2pt}}
	         @{\hspace{2pt}}r@{\hspace{2pt}}
		%
		@{\hspace{5pt}}r@{\hspace{5pt}}
		%
		@{\hspace{2pt}}r@{\hspace{2pt}}
		@{\hspace{2pt}}r@{\hspace{2pt}}
		%
		@{\hspace{5pt}}r@{\hspace{5pt}}
		%
		@{\hspace{2pt}}r@{\hspace{2pt}}
		@{\hspace{2pt}}r@{\hspace{2pt}}
		%
		@{\hspace{5pt}}r@{\hspace{5pt}}
		%
		@{\hspace{2pt}}r@{\hspace{2pt}}
		@{\hspace{2pt}}r@{\hspace{2pt}}
		%
		@{\hspace{5pt}}r@{\hspace{5pt}}
		%
		@{\hspace{2pt}}r@{\hspace{2pt}}
		%@{\hspace{2pt}}r@{\hspace{2pt}}
		%@{\hspace{2pt}}r@{\hspace{2pt}}
		}
		\toprule
		\multirow{2}{*}{method} & \multicolumn{2}{c}{Vina S$\downarrow$} & & \multicolumn{2}{c}{Vina M$\downarrow$} & & \multicolumn{2}{c}{Vina D$\downarrow$} & & \multicolumn{2}{c}{{HA\%$\uparrow$}}  & & \multicolumn{2}{c}{QED$\uparrow$} & & \multicolumn{2}{c}{SA$\uparrow$} & & \multicolumn{2}{c}{Div$\uparrow$} & %& \multirow{2}{*}{SR\%$\uparrow$} & 
		& \multirow{2}{*}{time$\downarrow$} \\
	    \cmidrule{2-3}\cmidrule{5-6} \cmidrule{8-9} \cmidrule{11-12} \cmidrule{14-15} \cmidrule{17-18} \cmidrule{20-21}
		 & Avg. & Med. &  & Avg. & Med. &  & Avg. & Med. & & Avg. & Med.  & & Avg. & Med.  & & Avg. & Med.  & & Avg. & Med.  & & \\ %& & \\
		%\multirow{2}{*}{method} & \multirow{2}{*}{\#c\%} &  \multirow{2}{*}{\#u\%} &  \multirow{2}{*}{QED} & \multicolumn{3}{c}{$\nmax=50$} & & \multicolumn{2}{c}{$\nmax=1$}\\
		%\cmidrule(r){5-7} \cmidrule(r){8-10} 
		%& & & & \avgshapesim(std) & \avggraphsim(std  &  \diversity(std  & & \avgshapesim(std) & \avggraphsim(std \\
		\midrule
		Reference                          & -5.32 & -5.66 & & -5.78 & -5.76 & & -6.63 & -6.67 & & - & - & & 0.53 & 0.49 & & 0.77 & 0.77 & & - & - & %& 23.1 & 
		& - \\
		\midrule
		\AR & \textbf{-5.06} & -4.99 & &  -5.59 & -5.29 & &  -6.16 & -6.05 & &  37.69 & 31.00 & &  0.50 & 0.49 & &  0.66 & 0.65 & & 0.70 & 0.70 & %& 7.0 & 
		& 7,789 \\
		\pockettwomol   & -4.50 & -4.21 & &  -5.70 & -5.27 & &  -6.43 & -6.25 & &  48.00 & 51.00 & &  0.58 & 0.58 & &  \textbf{0.77} & \textbf{0.78} & &  0.69 & 0.71 &  %& 24.9 & 
		& 2,150 \\
		\targetdiff     & -4.88 & \textbf{-5.82} & &  \textbf{-6.20} & \textbf{-6.36} & &  \textbf{-7.37} & \textbf{-7.51} & &  57.57 & 58.27 & &  0.50 & 0.51 & &  0.60 & 0.59 & &  \textbf{0.72} & 0.71 & % & 10.4 & 
		& 1,252 \\
		%& \decompdiffbeta                    & 63.03 & %-4.72 & -4.86 & & \textbf{-6.84} & \textbf{-6.91} & & \textbf{-8.85} & \textbf{-8.90} & &  \textbf{72.16} & \textbf{72.16} & &  0.36 & 0.36 & &  0.55 & 0.55 & & 0.59 & 0.59 & & 14.9 \\ 
		%-4.76 & -6.18 & &  \textbf{-6.86} & \textbf{-7.52} & &  \textbf{-8.85} & \textbf{-8.96} & &  \textbf{72.7} & \textbf{89.8} & &  0.36 & 0.34 & &  0.55 & 0.57 & & 0.59 & 0.59 & & 15.4 \\
		\decompdiffref  & -4.58 & -4.77 & &  -5.47 & -5.51 & &  -6.43 & -6.56 & &  47.76 & 48.66 & &  0.56 & 0.56 & &  0.70 & 0.69  & &  \textbf{0.72} & \textbf{0.72} &  %& 15.2 & 
		& 1,859 \\
		\midrule
		%\method & 14.04 & 9.74 & &  -2.80 & -3.87 & &  -6.32 & -6.41 & &  42.37 & 40.40 & &  0.70 & 0.71 & &  0.73 & 0.72 & & 0.71 & 0.74 & & 42 \\
		%\methodwithsguide & 1.04 & -0.33 & &  -4.23 & -4.39 & &  -6.31 & -6.46 & &  46.18 & 44.00 & &  0.69 & 0.71 & &  0.72 & 0.71 & & 0.70 & 0.73 & 53 \\
		\methodwithpguide      & -4.15 & -4.59 & &  -5.41 & -5.34 & &  -6.49 & -6.74 & &  \textbf{58.52} & 59.00 & &  \textbf{0.67} & \textbf{0.69} & &  0.68 & 0.68 & & 0.67 & 0.70 & %& 28.0 & 
		& 48 \\
		\methodwithsandpguide  & -4.56 & -4.82 & &  -5.53 & -5.47 & &  -6.60 & -6.78 & &  58.28 & \textbf{60.00} & &  0.66 & 0.68 & &  0.67 & 0.66 & & 0.68 & 0.71 &
		& 58 \\
		\bottomrule
	\end{tabular}%
	\begin{tablenotes}[normal,flushleft]
		\begin{footnotesize}
	\item 
\!\!Columns represent: {``Vina S'': the binding affinities between the initially generated poses of molecules and the protein pockets; 
		``Vina M'': the binding affinities between the poses after local structure minimization and the protein pockets;
		``Vina D'': the binding affinities between the poses determined by AutoDock Vina~\cite{Eberhardt2021} and the protein pockets;
		``HA'': the percentage of generated molecules with Vina D higher than those of condition molecules;
		``QED'': the drug-likeness score;
		``SA'': the synthesizability score;
		``Div'': the diversity among generated molecules;
		``time'': the time cost to generate molecules.}
		\par
		\par
		\end{footnotesize}
	\end{tablenotes}
	\end{scriptsize}
\end{threeparttable}
\end{table*}


%\label{tbl:overall_results_docking_100}

%-------------------------------------------------------------------------------------------------------------------------------------
%\subsection{{Comparison of Pocket Guidance}}
%\label{supp:app:results:docking}
%-------------------------------------------------------------------------------------------------------------------------------------


\begin{comment}
%-------------------------------------------------------------------------------------------------------------------------------------
\subsection{\ziqi{Simiarity Comparison for Pocket-based Molecule Generation}}
%-------------------------------------------------------------------------------------------------------------------------------------


\begin{table*}[t!]
	\centering
	\caption{{Overall Comparison on Similarity for Pocket-based Molecule Generation}}
	\label{tbl:docking_results_similarity}
	\begin{small}
		\begin{threeparttable}
			\begin{tabular}{
					@{\hspace{0pt}}l@{\hspace{5pt}}
					%
					@{\hspace{3pt}}l@{\hspace{3pt}}
					%
					@{\hspace{3pt}}r@{\hspace{8pt}}
					@{\hspace{3pt}}c@{\hspace{3pt}}
					%
					@{\hspace{3pt}}c@{\hspace{3pt}}
					@{\hspace{3pt}}c@{\hspace{3pt}}
					%
					@{\hspace{0pt}}c@{\hspace{0pt}}
					%
					@{\hspace{3pt}}c@{\hspace{3pt}}
					@{\hspace{3pt}}c@{\hspace{3pt}}
					%
					@{\hspace{3pt}}r@{\hspace{3pt}}
				}
				\toprule
				$\delta_g$  & method          & \#d\%$\uparrow$ & $\diversity_d$$\uparrow$(std) & \avgshapesim$\uparrow$(std) & \avggraphsim$\downarrow$(std) & & \maxshapesim$\uparrow$(std) & \maxgraphsim$\downarrow$(std)       & \#n\%$\uparrow$  \\ 
				\midrule
				%\multirow{6}{0.059\linewidth}{\hspace{0pt}0.1} 
				%& \AR   & 4.4 & 0.781(0.076) & 0.511(0.197) & \textbf{0.056}(0.020) & & 0.619(0.222) & 0.074(0.024) & 21.4  \\
				%& \pockettwomol & 6.6 & 0.795(0.099) & 0.519(0.216) & 0.063(0.020) & & 0.608(0.236) & 0.076(0.022) & \textbf{24.1}  \\
				%& \targetdiff & 2.0 & 0.872(0.041) & 0.619(0.110) & 0.068(0.018) & & 0.721(0.146) & 0.075(0.023) & 17.7  \\
				%& \decompdiffbeta & 0.0 & - & 0.374(0.138) & 0.059(0.031) & & 0.414(0.141) & \textbf{0.058}(0.032) & 9.8  \\
				%& \decompdiffref & 8.5 & 0.805(0.096) & 0.810(0.070) & 0.076(0.018) & & 0.861(0.085) & 0.076(0.020) & 11.3  \\
				%& \methodwithpguide   &  9.9 & \textbf{0.876}(0.041) & 0.795(0.058) & 0.073(0.015) & & 0.869(0.073) & 0.076(0.020) & 17.7  \\
				%& \methodwithsandpguide & \textbf{11.9} & 0.872(0.036) & \textbf{0.813}(0.052) & 0.075(0.014) & & \textbf{0.874}(0.069) & 0.080(0.014) & 17.0  \\
				%\cmidrule{2-10}
				%& improv\% & 40.4$^*$ & 8.8$^*$ & 0.4 & -30.4$^*$ &  & 1.6 & -30.0$^*$ & -26.3$^*$  \\
				%\midrule
				\multirow{7}{0.059\linewidth}{\hspace{0pt}1.0} 
				& \AR & 14.6 & 0.681(0.163) & 0.644(0.119) & 0.236(0.123) & & 0.780(0.110) & 0.284(0.177) & 95.8  \\
				& \pockettwomol & 18.6 & 0.711(0.152) & 0.654(0.131) &   \textbf{0.217}(0.129) & & 0.778(0.121) &   \textbf{0.243}(0.137) &  \textbf{98.3}  \\
				& \targetdiff & 7.1 &  \textbf{0.785}(0.085) & 0.622(0.083) & 0.238(0.122) & & 0.790(0.102) & 0.274(0.158) & 90.4  \\
				%& \decompdiffbeta & 0.1 & 0.589(0.030) & 0.494(0.124) & 0.263(0.143) & & 0.567(0.143) & 0.275(0.162) & 67.7  \\
				& \decompdiffref & 37.3 & 0.721(0.108) & 0.770(0.087) & 0.282(0.130) & & \textbf{0.878}(0.059) & 0.343(0.174) & 83.7  \\
				& \methodwithpguide   &  27.4 & 0.757(0.134) & 0.747(0.078) & 0.265(0.165) & & 0.841(0.081) & 0.272(0.168) & 98.1  \\
				& \methodwithsandpguide &\textbf{45.2} & 0.724(0.142) &   \textbf{0.789}(0.063) & 0.265(0.162) & & 0.876(0.062) & 0.264(0.159) & 97.8  \\
				\cmidrule{2-10}
				& Improv\%  & 21.2$^*$ & -3.6 & 2.5$^*$ & -21.7$^*$ &  & -0.1 & -8.4$^*$ & -0.2  \\
				\midrule
				\multirow{7}{0.059\linewidth}{\hspace{0pt}0.7} 
				& \AR   & 14.5 & 0.692(0.151) & 0.644(0.119) & 0.233(0.116) & & 0.779(0.110) & 0.266(0.140) & 94.9  \\
				& \pockettwomol & 18.6 & 0.711(0.152) & 0.654(0.131) & \textbf{0.217}(0.129) & & 0.778(0.121) & \textbf{0.243}(0.137) & \textbf{98.2}  \\
				& \targetdiff & 7.1 & \textbf{0.786}(0.084) & 0.622(0.083) & 0.238(0.121) & & 0.790(0.101) & 0.270(0.151) & 90.3  \\
				%& \decompdiffbeta & 0.1 & 0.589(0.030) & 0.494(0.124) & 0.263(0.142) & &0.567(0.143) & 0.273(0.156) & 67.6  \\
				& \decompdiffref & 36.2 & 0.721(0.113) & 0.770(0.086) & 0.273(0.123) & & \textbf{0.876}(0.059) & 0.325(0.139) & 82.3  \\
				& \methodwithpguide   &  27.4 & 0.757(0.134) & 0.746(0.078) & 0.263(0.160) & & 0.841(0.081) & 0.271(0.164) & 96.8  \\
				& \methodwithsandpguide      & \textbf{45.0} & 0.732(0.129) & \textbf{0.789}(0.063) & 0.262(0.157) & & \textbf{0.876}(0.063) & 0.262(0.153) & 96.2  \\
				\cmidrule{2-10}
				& Improv\%  & 24.3$^*$ & -3.6 & 2.5$^*$ & -20.8$^*$ &  & 0.0 & -7.6$^*$ & -1.5  \\
				\midrule
				\multirow{7}{0.059\linewidth}{\hspace{0pt}0.5} 
				& \AR   & 14.1 & 0.687(0.160) & 0.639(0.124) & 0.218(0.097) & & 0.778(0.110) & 0.260(0.130) & 89.8  \\
				& \pockettwomol & 18.5 & 0.711(0.152) & 0.649(0.134) & \textbf{0.209}(0.114) & & 0.777(0.121) & \textbf{0.240}(0.131) & \textbf{93.2}  \\
				& \targetdiff & 7.1 & \textbf{0.786}(0.084) & 0.621(0.083) & 0.230(0.111) & & 0.788(0.105) & 0.254(0.127) & 86.5  \\
				%&\decompdiffbeta & 0.1 & 0.595(0.025) & 0.494(0.124) & 0.254(0.129) & & 0.565(0.142) & 0.259(0.138) & 63.9  \\
				& \decompdiffref & 34.7 & 0.730(0.105) & 0.769(0.086) & 0.261(0.109) & & 0.874(0.080) & 0.301(0.117) & 77.3   \\
				& \methodwithpguide  &  27.2 & 0.765(0.123) & 0.749(0.075) & 0.245(0.135) & & 0.840(0.082) & 0.252(0.137) & 88.6  \\
				& \methodwithsandpguide & \textbf{44.3} & 0.738(0.122) & \textbf{0.791}(0.059) & 0.247(0.132) &  & \textbf{0.875}(0.065) & 0.249(0.130) & 88.8  \\
				\cmidrule{2-10}
				& Improv\%   & 27.8$^*$ & -2.7 & 2.9$^*$ & -17.6$^*$ &  & 0.2 & -3.4 & -4.7$^*$  \\
				\midrule
				\multirow{7}{0.059\linewidth}{\hspace{0pt}0.3} 
				& \AR   & 12.2 & 0.704(0.146) & 0.614(0.146) & 0.164(0.059) & & 0.751(0.138) & 0.206(0.059) & 66.4  \\
				& \pockettwomol & 17.1 & 0.731(0.129) & 0.617(0.163) & \textbf{0.155}(0.056) & & 0.740(0.159) & \textbf{0.190}(0.076) & \textbf{71.0}  \\
				& \targetdiff & 6.2 & \textbf{0.809}(0.061) & 0.619(0.087) & 0.181(0.068) & & 0.768(0.119) & 0.196(0.076) & 61.7  \\				
                %& \decompdiffbeta & 0.0 & - & 0.489(0.124) & 0.195(0.080) & & 0.547(0.139) & 0.203(0.087) & 42.0  \\
				& \decompdiffref & 27.7 & 0.775(0.081) & 0.767(0.086) & 0.202(0.062) & & 0.854(0.093) & 0.216(0.068) & 52.6  \\
				& \methodwithpguide   &  24.4 & 0.805(0.084) & 0.763(0.066) & 0.180(0.074) & & 0.847(0.080) & \textbf{0.190}(0.059) & 61.4  \\
				& \methodwithsandpguide & \textbf{36.3} & 0.789(0.081) & \textbf{0.800}(0.056) & 0.181(0.071) & &\textbf{0.878}(0.067) & \textbf{0.190}(0.078) & 61.8  \\
				\cmidrule{2-10}
				& improv\% & 31.1$^*$ & 3.9$^*$ & 4.3$^*$ & -16.5$^*$ &  & 2.8$^*$ & 0.0 & -12.9$^*$  \\
				\bottomrule
			\end{tabular}%
			\begin{tablenotes}[normal,flushleft]
				\begin{footnotesize}
					\item 
					\!\!Columns represent: \ziqi{``$\delta_g$'': the graph similarity constraint; ``\#n\%'': the percentage of molecules that satisfy the graph similarity constraint ($\graphsim<=\delta_g$);
						``\#d\%'': the percentage of molecules that satisfy the graph similarity constraint and are with high \shapesim ($\shapesim>=0.8$);
						``\avgshapesim/\avggraphsim'': the average of shape or graph similarities between the condition molecules and generated molecules with $\graphsim<=\delta_g$;
						``\maxshapesim'': the maximum of shape similarities between the condition molecules and generated molecules with $\graphsim<=\delta_g$;
						``\maxgraphsim'': the graph similarities between the condition molecules and the molecules with the maximum shape similarities and $\graphsim<=\delta_g$;
						``\diversity'': the diversity among the generated molecules.
						%
						``$\uparrow$'' represents higher values are better, and ``$\downarrow$'' represents lower values are better.
						%
						Best values are in \textbf{bold}, and second-best values are \underline{underlined}. 
					} 
					%\todo{double-check the significance value}
					\par
					\par
				\end{footnotesize}
			\end{tablenotes}
		\end{threeparttable}
	\end{small}
	\vspace{-10pt}    
\end{table*}
%\label{tbl:docking_results_similarity}

\bo{@Ziqi you may want to check my edits for the discussion in Table 1 first.
%
If the pocket if known, do you still care about the shape similarity in real applications?
}

\ziqi{Table~\ref{tbl:docking_results_similarity} presents the overall comparison on similarity-based metrics between \methodwithpguide, \methodwithsandpguide and other baselines under different graph similarity constraints  ($\delta_g$=1.0, 0.7, 0.5, 0.3), similar to Table~\ref{tbl:overall}. 
%
As shown in Table~\ref{tbl:docking_results_similarity}, regarding desirable molecules,  \methodwithsandpguide consistently outperforms all the baseline methods in the likelihood of generating desirable molecules (i.e., $\#d\%$).
%
For example, when $\delta_g$=1.0, at $\#d\%$, \methodwithsandpguide (45.2\%) demonstrates significant improvement of $21.2\%$ compared to the best baseline \decompdiff (37.3\%).
%
In terms of $\diversity_d$, \methodwithpguide and \methodwithsandpguide also achieve the second and the third best performance. 
%
Note that the best baseline \targetdiff in $\diversity_d$ achieves the least percentage of desirable molecules (7.1\%), substantially lower than \methodwithpguide and \methodwithsandpguide.
%
This makes its diversity among desirable molecules incomparable with other methods. 
%
When $\delta_g$=0.7, 0.5, and 0.3, \methodwithsandpguide also establishes a significant improvement of 24.3\%, 27.8\%, and 31.1\% compared to the best baseline method \decompdiff.
%
It is also worth noting that the state-of-the-art baseline \decompdiff underperforms \methodwithpguide and \methodwithsandpguide in binding affinities as shown in Table~\ref{tbl:overall_results_docking}, even though it outperforms \methodwithpguide in \#d\%.
%
\methodwithpguide and \methodwithsandpguide also achieve the second and the third best performance in $\diversity_d$ at $\delta_g$=0.7, 0.5, and 0.3. 
%
The superior performance of \methodwithpguide and \methodwithsandpguide in $\#d\%$ at small $\delta_g$ indicates their strong capacity in generating desirable molecules of novel graph structures, thereby facilitating the discovery of novel drug candidates.
%
}

\ziqi{Apart from the desirable molecules, \methodwithpguide and \methodwithsandpguide also demonstrate outstanding performance in terms of the average shape similarities (\avgshapesim) and the average graph similarities (\avggraphsim).
%
Specifically, when $\delta_g$=1.0, \methodwithsandpguide achieves a significant 2.5\% improvement in \avgshapesim\ over the best baseline \decompdiff. 
%
In terms of \avggraphsim, \methodwithsandpguide also achieves higher performance than the baseline \decompdiff of the highest \avgshapesim (0.265 vs 0.282).
%
Please note that all the baseline methods except \decompdiff achieve substantially lower performance in \avgshapesim than \methodwithpguide and \methodwithsandpguide, even though these methods achieve higher \avggraphsim values.
%
This trend remains consistent when applying various similarity constraints (i.e., $\delta_g$) as shown in Table~\ref{tbl:overall_results_docking}.
}

\ziqi{Similarly, \methodwithpguide and \methodwithsandpguide also achieve superior performance in \maxshapesim and \maxgraphsim.
%
Specifically, when $\delta_g$=1.0, for \maxshapesim, \methodwithsandpguide achieves highly comparable performance in \maxshapesim\ compared to the best baseline \decompdiff (0.876 vs 0.878).
%
We also note that \methodwithsandpguide achieves lower \maxgraphsim\ than the \decompdiff with 23.0\% difference. 
%
When $\delta_g$ gets smaller from 0.7 to 0.3, \methodwithsandpguide maintains a high \maxshapesim value around 0.876, while the best baseline \decompdiff has \maxshapesim decreased from 0.878 to 0.854.
%
This demonstrates the superior ability of \methodwithsandpguide in generating molecules with similar shapes and novel structures.
%
}

\ziqi{
In terms of \#n\%, when $\delta_g$=1.0, the percentage of molecules with \graphsim below $\delta_g$ can be interpreted as the percentage of valid molecules among all the generated molecules. 
%
As shown in Table~\ref{tbl:docking_results_similarity}, \methodwithpguide and \methodwithsandpguide are able to generate 98.1\% and 97.8\% of valid molecules, slightly below the best baseline \pockettwomol (98.3\%). 
%
When $\delta_g$=0.7, 0.5, or 0.3, all the methods, including \methodwithpguide and \methodwithsandpguide, can consistently find a sufficient number of novel molecules that meet the graph similarity constraints.
%
The only exception is \decompdiff, which substantially underperforms all the other methods in \#n\%.
}
\end{comment}

%%%%%%%%%%%%%%%%%%%%%%%%%%%%%%%%%%%%%%%%%%%%%
\section{Properties of Molecules in Case Studies for Targets}
\label{supp:app:results:properties}
%%%%%%%%%%%%%%%%%%%%%%%%%%%%%%%%%%%%%%%%%%%%%

%-------------------------------------------------------------------------------------------------------------------------------------
\subsection{Drug Properties of Generated Molecules}
\label{supp:app:results:properties:drug}
%-------------------------------------------------------------------------------------------------------------------------------------

Table~\ref{tbl:drug_property} presents the drug properties of three generated molecules: NL-001, NL-002, and NL-003.
%
As shown in Table~\ref{tbl:drug_property}, each of these molecules has a favorable profile, making them promising drug candidates. 
%
{As discussed in Section ``Case Studies for Targets'' in the main manuscript, all three molecules have high binding affinities in terms of Vina S, Vina M and Vina D, and favorable QED and SA values.
%
In addition, all of them meet the Lipinski's rule of five criteria~\cite{Lipinski1997}.}
%
In terms of physicochemical properties, all these properties of NL-001, NL-002 and NL-003, including number of rotatable bonds, molecule weight, LogP value, number of hydrogen bond doners and acceptors, and molecule charges, fall within the desired range of drug molecules. 
%
This indicates that these molecules could potentially have good solubility and membrane permeability, essential qualities for effective drug absorption.

These generated molecules also demonstrate promising safety profiles based on the predictions from ICM~\cite{Neves2012}.
%
In terms of drug-induced liver injury prediction scores, all three molecules have low scores (0.188 to 0.376), indicating a minimal risk of hepatotoxicity. 
%
NL-001 and NL-002 fall under `Ambiguous/Less concern' for liver injury, while NL-003 is categorized under 'No concern' for liver injury. 
%
Moreover, all these molecules have low toxicity scores (0.000 to 0.236). 
%
NL-002 and NL-003 do not have any known toxicity-inducing functional groups. 
%
NL-001 and NL-003 are also predicted not to include any known bad groups that lead to inappropriate features.
%
These attributes highlight the potential of NL-001, NL-002, and NL-003 as promising treatments for cancers and Alzheimer’s disease.

%\begin{table*}
	\centering
		\caption{Drug Properties of Generated Molecules}
	\label{tbl:binding_drug_mols}
	\begin{scriptsize}
\begin{threeparttable}
	\begin{tabular}{
		@{\hspace{6pt}}r@{\hspace{6pt}}
		@{\hspace{6pt}}r@{\hspace{6pt}}
		@{\hspace{6pt}}r@{\hspace{6pt}}
		@{\hspace{6pt}}r@{\hspace{6pt}}
		@{\hspace{6pt}}r@{\hspace{6pt}}
		@{\hspace{6pt}}r@{\hspace{6pt}}
		@{\hspace{6pt}}r@{\hspace{6pt}}
		@{\hspace{6pt}}r@{\hspace{6pt}}
		@{\hspace{6pt}}r@{\hspace{6pt}}
		%
		}
		\toprule
Target & Molecule & Vina S & Vina M & Vina D & QED   & SA   & Logp  & Lipinski \\
\midrule
\multirow{3}{*}{CDK6} & NL-001 & -6.817      & -7.251    & -8.319     & 0.834 & 0.72 & 1.313 & 5        \\
& NL-002 & -6.970       & -7.605    & -8.986     & 0.851 & 0.74 & 3.196 & 5        \\
\cmidrule{2-9}
& 4AU & 0.736       & -5.939    & -7.592     & 0.773 & 0.79 & 2.104 & 5        \\
\midrule
\multirow{2}{*}{NEP} & NL-003 & -11.953     & -12.165   & -12.308    & 0.772 & 0.57 & 2.944 & 5        \\
\cmidrule{2-9}
& BIR & -9.399      & -9.505    & -9.561     & 0.463 & 0.73 & 2.677 & 5        \\
		\bottomrule
	\end{tabular}%
	\begin{tablenotes}[normal,flushleft]
		\begin{footnotesize}
	\item Columns represent: {``Target'': the names of protein targets;
		``Molecule'': the names of generated molecules and known ligands;
		``Vina S'': the binding affinities between the initially generated poses of molecules and the protein pockets; 
		``Vina M'': the binding affinities between the poses after local structure minimization and the protein pockets;
		``Vina D'': the binding affinities between the poses determined by AutoDock Vina~\cite{Eberhardt2021} and the protein pockets;
		``HA'': the percentage of generated molecules with Vina D higher than those of condition molecules;
		``QED'': the drug-likeness score;
		``SA'': the synthesizability score;
		``Div'': the diversity among generated molecules;
		``time'': the time cost to generate molecules.}
\!\! \par
		\par
		\end{footnotesize}
	\end{tablenotes}
\end{threeparttable}
\end{scriptsize}
  \vspace{-10pt}    
\end{table*}

%\label{tbl:binding_drug_mols}

\begin{table*}
	\centering
		\caption{Drug Properties of Generated Molecules}
	\label{tbl:drug_property}
	\begin{scriptsize}
\begin{threeparttable}
	\begin{tabular}{
		@{\hspace{0pt}}p{0.23\linewidth}@{\hspace{5pt}}
		%
		@{\hspace{1pt}}r@{\hspace{2pt}}
		@{\hspace{2pt}}r@{\hspace{6pt}}
		@{\hspace{6pt}}r@{\hspace{6pt}}
		%
		}
		\toprule
		Property Name & NL-001 & NL-002 & NL-003 \\
		\midrule
Vina S & -6.817 &  -6.970 & -11.953 \\
Vina M & -7.251 & -7.605 & -12.165 \\
Vina D & -8.319 & -8.986 & -12.308 \\
QED    & 0.834  & 0.851  & 0.772 \\
SA       & 0.72    & 0.74    & 0.57    \\
Lipinski & 5 & 5 & 5 \\
%bbbScore          & 3.386                                                                                        & 4.240                                                                                        & 3.892      \\
%drugLikeness      & -0.081                                                                                       & -0.442                                                                                       & -0.325     \\
%molLogP1          & 1.698                                                                                        & 2.685                                                                                        & 2.382      \\
\#rotatable bonds          & 3                                                                                        & 2                                                                                        & 2      \\
molecule weight         & 267.112                                                                                      & 270.117                                                                                      & 390.206    \\
molecule LogP           & 1.698                                                                                        & 2.685                                                                                        & 2.382     \\
\#hydrogen bond doners           & 1                                                                                        & 1                                                                                        & 2      \\
\#hydrogen bond acceptors           & 5                                                                                       & 3                                                                                        & 5      \\
\#molecule charges   & 1                                                                                        & 0                                                                                        & 0      \\
drug-induced liver injury predScore    & 0.227                                                                                        & 0.376                                                                                        & 0.188      \\
drug-induced liver injury predConcern  & Ambiguous/Less concern                                                                       & Ambiguous/Less concern                                                                       & No concern \\
drug-induced liver injury predLabel    & Warnings/Precautions/Adverse reactions & Warnings/Precautions/Adverse reactions & No match   \\
drug-induced liver injury predSeverity & 2                                                                                        & 3                                                                                        & 2      \\
%molSynth1         & 0.254                                                                                        & 0.220                                                                                        & 0.201      \\
%toxicity class         & 0.480                                                                                        & 0.480                                                                                        & 0.450      \\
toxicity names         & hydrazone                                                                                    &   -                                                                                           &   -         \\
toxicity score         & 0.236                                                                                        & 0.000                                                                                        & 0.000      \\
bad groups         & -                                                                                             & Tetrahydroisoquinoline:   allergies                                                          &   -         \\
%MolCovalent       &                                                                                              &                                                                                              &            \\
%MolProdrug        &                                                                                              &                                                                                              &            \\
		\bottomrule
	\end{tabular}%
	\begin{tablenotes}[normal,flushleft]
		\begin{footnotesize}
	\item ``-'': no results found by algorithms
\!\! \par
		\par
		\end{footnotesize}
	\end{tablenotes}
\end{threeparttable}
\end{scriptsize}
  \vspace{-10pt}    
\end{table*}

%\label{tbl:drug_property}

%-------------------------------------------------------------------------------------------------------------------------------------
\subsection{Comparison on ADMET Profiles between Generated Molecules and Approved Drugs}
\label{supp:app:results:properties:admet}
%-------------------------------------------------------------------------------------------------------------------------------------

\paragraph{Generated Molecules for CDK6}
%
Table~\ref{tbl:admet_cdk6} presents the comparison on ADMET profiles between two generated molecules for CDK6 and the approved CDK6 inhibitors, including Abemaciclib~\cite{Patnaik2016}, Palbociclib~\cite{Lu2015}, and Ribociclib~\cite{Tripathy2017}.
%
As shown in Table~\ref{tbl:admet_cdk6}, the generated molecules, NL-001 and NL-002, exhibit comparable ADMET profiles with those of the approved CDK6 inhibitors. 
%
Importantly, both molecules demonstrate good potential in most crucial properties, including Ames mutagenesis, favorable oral toxicity, carcinogenicity, estrogen receptor binding, high intestinal absorption and favorable oral bioavailability.
%
Although the generated molecules are predicted as positive in hepatotoxicity and mitochondrial toxicity, all the approved drugs are also predicted as positive in these two toxicity.
%
This result suggests that these issues might stem from the limited prediction accuracy rather than being specific to our generated molecules.
%
Notably, NL-001 displays a potentially better plasma protein binding score compared to other molecules, which may improve its distribution within the body. 
%
Overall, these results indicate that NL-001 and NL-002 could be promising candidates for further drug development.


\begin{table*}
	\centering
		\caption{Comparison on ADMET Profiles among Generated Molecules and Approved Drugs Targeting CDK6}
	\label{tbl:admet_cdk6}
	\begin{scriptsize}
\begin{threeparttable}
	\begin{tabular}{
		%@{\hspace{0pt}}p{0.23\linewidth}@{\hspace{5pt}}
		%
		@{\hspace{6pt}}l@{\hspace{5pt}}
		@{\hspace{6pt}}r@{\hspace{6pt}}
		@{\hspace{6pt}}r@{\hspace{6pt}}
		@{\hspace{6pt}}r@{\hspace{6pt}}
		@{\hspace{6pt}}r@{\hspace{6pt}}
		@{\hspace{6pt}}r@{\hspace{6pt}}
		%
		%
		@{\hspace{6pt}}r@{\hspace{6pt}}
		%@{\hspace{6pt}}r@{\hspace{6pt}}
		%
		}
		\toprule
		\multirow{2}{*}{Property name} & \multicolumn{2}{c}{Generated molecules} & & \multicolumn{3}{c}{FDA-approved drugs} \\
		\cmidrule{2-3}\cmidrule{5-7}
		 & NL--001 & NL--002 & & Abemaciclib & Palbociclib & Ribociclib \\
		\midrule
\rowcolor[HTML]{D2EAD9}Ames   mutagenesis                             & --   &  --  & & + &  --  & --  \\
\rowcolor[HTML]{D2EAD9}Acute oral toxicity (c)                           & III & III & &  III          & III          & III         \\
Androgen receptor binding                         & +                          & +            &              & +            & +            & +             \\
Aromatase binding                                 & +                          & +            &              & +            & +            & +            \\
Avian toxicity                                    & --                          & --          &                & --            & --            & --            \\
Blood brain barrier                               & +                          & +            &              & +            & +            & +            \\
BRCP inhibitior                                   & --                          & --          &                & --            & --            & --            \\
Biodegradation                                    & --                          & --          &                & --            & --            & --           \\
BSEP inhibitior            & +                          & +            &              & +            & +            & +        \\
Caco-2                                            & +                          & +            &              & --            & --            & --            \\
\rowcolor[HTML]{D2EAD9}Carcinogenicity (binary)                          & --                          & --             &             & --            & --            & --          \\
\rowcolor[HTML]{D2EAD9}Carcinogenicity (trinary)                         & Non-required               & Non-required   &            & Non-required & Non-required & Non-required  \\
Crustacea aquatic toxicity & --                          & --            &              & --            & --            & --            \\
 CYP1A2 inhibition                                 & +                          & +            &              & --            & --            & +             \\
CYP2C19 inhibition                                & --                          & +            &              & +            & --            & +            \\
CYP2C8 inhibition                                 & --                          & --           &               & +            & +            & +            \\
CYP2C9 inhibition                                 & --                          & --           &               & --            & --            & +             \\
CYP2C9 substrate                                  & --                          & --           &               & --            & --            & --            \\
CYP2D6 inhibition                                 & --                          & --           &               & --            & --            & --            \\
CYP2D6 substrate                                  & --                          & --           &               & --            & --            & --            \\
CYP3A4 inhibition                                 & --                          & +            &              & --            & --            & --            \\
CYP3A4 substrate                                  & +                          & --            &              & +            & +            & +            \\
\rowcolor[HTML]{D2EAD9}CYP inhibitory promiscuity                        & +                          & +                    &      & +            & --            & +            \\
Eye corrosion                                     & --                          & --           &               & --            & --            & --            \\
Eye irritation                                    & --                          & --           &               & --            & --            & --             \\
\rowcolor[HTML]{D8E7FF}Estrogen receptor binding                         & +                          & +                    &      & +            & +            & +            \\
Fish aquatic toxicity                             & --                          & +            &              & +            & --            & --            \\
Glucocorticoid receptor   binding                 & +                          & +             &             & +            & +            & +            \\
Honey bee toxicity                                & --                          & --           &               & --            & --            & --            \\
\rowcolor[HTML]{D2EAD9}Hepatotoxicity                                    & +                          & +            &              & +            & +            & +             \\
Human ether-a-go-go-related gene inhibition     & +                          & +               &           & +            & --            & --           \\
\rowcolor[HTML]{D8E7FF}Human intestinal absorption                       & +                          & +             &             & +            & +            & +    \\
\rowcolor[HTML]{D8E7FF}Human oral bioavailability                        & +                          & +              &            & +            & +            & +     \\
\rowcolor[HTML]{D2EAD9}MATE1 inhibitior                                  & --                          & --              &            & --            & --            & --    \\
\rowcolor[HTML]{D2EAD9}Mitochondrial toxicity                            & +                          & +                &          & +            & +            & +    \\
Micronuclear                                      & +                          & +                          & +            & +            & +           \\
\rowcolor[HTML]{D2EAD9}Nephrotoxicity                                    & --                          & --             &             & --            & --            & --             \\
Acute oral toxicity                               & 2.325                      & 1.874    &     & 1.870        & 3.072        & 3.138        \\
\rowcolor[HTML]{D8E7FF}OATP1B1 inhibitior                                & +                          & +              &            & +            & +            & +             \\
\rowcolor[HTML]{D8E7FF}OATP1B3 inhibitior                                & +                          & +              &            & +            & +            & +             \\
\rowcolor[HTML]{D2EAD9}OATP2B1 inhibitior                                & --                          & --             &             & --            & --            & --             \\
OCT1 inhibitior                                   & --                          & --        &                  & +            & --            & +             \\
OCT2 inhibitior                                   & --                          & --        &                  & --            & --            & +             \\
P-glycoprotein inhibitior                         & --                          & --        &                  & +            & +            & +     \\
P-glycoprotein substrate                          & --                          & --        &                  & +            & +            & +     \\
PPAR gamma                                        & +                          & +          &                & +            & +            & +      \\
\rowcolor[HTML]{D8E7FF}Plasma protein binding                            & 0.359        & 0.745     &    & 0.865        & 0.872        & 0.636       \\
Reproductive toxicity                             & +                          & +          &                & +            & +            & +           \\
Respiratory toxicity                              & +                          & +          &                & +            & +            & +         \\
Skin corrosion                                    & --                          & --        &                  & --            & --            & --           \\
Skin irritation                                   & --                          & --        &                  & --            & --            & --         \\
Skin sensitisation                                & --                          & --        &                  & --            & --            & --          \\
Subcellular localzation                           & Mitochondria               & Mitochondria  &             & Lysosomes    & Mitochondria & Mitochondria \\
Tetrahymena pyriformis                            & 0.398                      & 0.903         &             & 1.033        & 1.958        & 1.606         \\
Thyroid receptor binding                          & +                          & +             &             & +            & +            & +           \\
UGT catelyzed                                     & --                          & --           &               & --            & --            & --           \\
\rowcolor[HTML]{D8E7FF}Water solubility                                  & -3.050                     & -3.078              &       & -3.942       & -3.288       & -2.673     \\
		\bottomrule
	\end{tabular}%
	\begin{tablenotes}[normal,flushleft]
		\begin{footnotesize}
	\item Blue cells highlight crucial properties where a negative outcome (``--'') is desired; for acute oral toxicity (c), a higher category (e.g., ``III'') is desired; and for carcinogenicity (trinary), ``Non-required'' is desired.
	%
	Green cells highlight crucial properties where a positive result (``+'') is desired; for plasma protein binding, a lower value is desired; and for water solubility, values higher than -4 are desired~\cite{logs}.
\!\! \par
		\par
		\end{footnotesize}
	\end{tablenotes}
\end{threeparttable}
\end{scriptsize}
  \vspace{--10pt}    
\end{table*}

%\label{tbl:admet_cdk6}

\paragraph{Generated Molecule for NEP}
%
Table~\ref{tbl:admet_nep} presents the comparison on ADMET profiles between a generated molecule for NEP targeting Alzheimer's disease and three approved drugs, Donepezil, Galantamine, and Rivastigmine, for Alzheimer's disease~\cite{Hansen2008}.
%
Overall, NL-003 exhibits a comparable ADMET profile with the three approved drugs.  
%
Notably, same as other approved drugs, NL-003 is predicted to be able to penetrate the blood brain barrier, a crucial property for Alzheimer's disease.
%  
In addition, it demonstrates a promising safety profile in terms of Ames mutagenesis, favorable oral toxicity, carcinogenicity, estrogen receptor binding, high intestinal absorption, nephrotoxicity and so on.
%
These results suggest that NL-003 could be promising candidates for the drug development of Alzheimer's disease.

\begin{table*}
	\centering
		\caption{Comparison on ADMET Profiles among Generated Molecule Targeting NEP and Approved Drugs for Alzhimer's Disease}
	\label{tbl:admet_nep}
	\begin{scriptsize}
\begin{threeparttable}
	\begin{tabular}{
		@{\hspace{6pt}}l@{\hspace{5pt}}
		%
		@{\hspace{6pt}}r@{\hspace{6pt}}
		@{\hspace{6pt}}r@{\hspace{6pt}}
		@{\hspace{6pt}}r@{\hspace{6pt}}
		@{\hspace{6pt}}r@{\hspace{6pt}}
		@{\hspace{6pt}}r@{\hspace{6pt}}
		%
		%
		%@{\hspace{6pt}}r@{\hspace{6pt}}
		%
		}
		\toprule
		\multirow{2}{*}{Property name} & Generated molecule & & \multicolumn{3}{c}{FDA-approved drugs} \\
\cmidrule{2-2}\cmidrule{4-6}
			& NL--003 & & Donepezil	& Galantamine & Rivastigmine \\
		\midrule
\rowcolor[HTML]{D2EAD9} 
Ames   mutagenesis                            & --                      &              & --                                    & --                                 & --                     \\
\rowcolor[HTML]{D2EAD9}Acute oral toxicity (c)                       & III           &                       & III                                  & III                               & II                      \\
Androgen receptor binding                     & +      &      & +            & --         & --         \\
Aromatase binding                             & --     &       & +            & --         & --        \\
Avian toxicity                                & --     &                               & --                                    & --                                 & --                        \\
\rowcolor[HTML]{D8E7FF} 
Blood brain barrier                           & +      &                              & +                                    & +                                 & +                        \\
BRCP inhibitior                               & --     &       & --            & --         & --         \\
Biodegradation                                & --     &                               & --                                    & --                                 & --                        \\
BSEP inhibitior                               & +      &      & +            & --         & --         \\
Caco-2                                        & +      &      & +            & +         & +         \\
\rowcolor[HTML]{D2EAD9} 
Carcinogenicity (binary)                      & --     &                               & --                                    & --                                 & --                        \\
\rowcolor[HTML]{D2EAD9} 
Carcinogenicity (trinary)                     & Non-required    &                     & Non-required                         & Non-required                      & Non-required             \\
Crustacea aquatic toxicity                    & +               &                     & +                                    & +                                 & --                        \\
CYP1A2 inhibition                             & +               &                     & +                                    & --                                 & --                        \\
CYP2C19 inhibition                            & +               &                     & --                                    & --                                 & --                        \\
CYP2C8 inhibition                             & +               &                     & --                                    & --                                 & --                        \\
CYP2C9 inhibition                             & --              &                      & --                                    & --                                 & --                        \\
CYP2C9 substrate                              & --              &                      & --                                    & --                                 & --                        \\
CYP2D6 inhibition                             & --              &                      & +                                    & --                                 & --                        \\
CYP2D6 substrate                              & --              &                      & +                                    & +                                 & +                        \\
CYP3A4 inhibition                             & --              &                      & --                                    & --                                 & --                        \\
CYP3A4 substrate                              & +               &                     & +                                    & +                                 & --                        \\
\rowcolor[HTML]{D2EAD9} 
CYP inhibitory promiscuity                    & +               &                     & +                                    & --                                 & --                        \\
Eye corrosion                                 & --     &       & --            & --         & --         \\
Eye irritation                                & --     &       & --            & --         & --         \\
Estrogen receptor binding                     & +      &      & +            & --         & --         \\
Fish aquatic toxicity                         & --     &                               & +                                    & +                                 & +                        \\
Glucocorticoid receptor binding             & --      &      & +            & --         & --         \\
Honey bee toxicity                            & --    &                                & --                                    & --                                 & --                        \\
\rowcolor[HTML]{D2EAD9} 
Hepatotoxicity                                & +     &                               & +                                    & --                                 & --                        \\
Human ether-a-go-go-related gene inhibition & +       &     & +            & --         & --         \\
\rowcolor[HTML]{D8E7FF} 
Human intestinal absorption                   & +     &                               & +                                    & +                                 & +                        \\
\rowcolor[HTML]{D8E7FF} 
Human oral bioavailability                    & --    &                                & +                                    & +                                 & +                        \\
\rowcolor[HTML]{D2EAD9} 
MATE1 inhibitior                              & --    &                                & --                                    & --                                 & --                        \\
\rowcolor[HTML]{D2EAD9} 
Mitochondrial toxicity                        & +     &                               & +                                    & +                                 & +                        \\
Micronuclear                                  & +     &       & --            & --         & +         \\
\rowcolor[HTML]{D2EAD9} 
Nephrotoxicity                                & --    &                                & --                                    & --                                 & --                        \\
Acute oral toxicity                           & 2.704  &      & 2.098        & 2.767     & 2.726     \\
\rowcolor[HTML]{D8E7FF} 
OATP1B1 inhibitior                            & +      &                              & +                                    & +                                 & +                        \\
\rowcolor[HTML]{D8E7FF} 
OATP1B3 inhibitior                            & +      &                              & +                                    & +                                 & +                        \\
\rowcolor[HTML]{D2EAD9} 
OATP2B1 inhibitior                            & --     &                               & --                                    & --                                 & --                        \\
OCT1 inhibitior                               & +      &      & +            & --         & --         \\
OCT2 inhibitior                               & --     &       & +            & --         & --         \\
P-glycoprotein inhibitior                     & +      &      & +            & --         & --         \\
\rowcolor[HTML]{D8E7FF} 
P-glycoprotein substrate                      & +      &                              & +                                    & +                                 & --                        \\
PPAR gamma                                    & +      &      & --            & --         & --         \\
\rowcolor[HTML]{D8E7FF} 
Plasma protein binding                        & 0.227   &                             & 0.883                                & 0.230                             & 0.606                    \\
Reproductive toxicity                         & +       &     & +            & +         & +         \\
Respiratory toxicity                          & +       &     & +            & +         & +         \\
Skin corrosion                                & --      &      & --            & --         & --         \\
Skin irritation                               & --      &      & --            & --         & --         \\
Skin sensitisation                            & --      &      & --            & --         & --         \\
Subcellular localzation                       & Mitochondria & &Mitochondria & Lysosomes & Mitochondria  \\
Tetrahymena pyriformis                        & 0.053           &                     & 0.979                                & 0.563                             & 0.702                        \\
Thyroid receptor binding                      & +       &     & +            & +         & --             \\
UGT catelyzed                                 & --      &      & --            & +         & --             \\
\rowcolor[HTML]{D8E7FF} 
Water solubility                              & -3.586   &                            & -2.425                               & -2.530                            & -3.062                       \\
		\bottomrule
	\end{tabular}%
	\begin{tablenotes}[normal,flushleft]
		\begin{footnotesize}
	\item Blue cells highlight crucial properties where a negative outcome (``--'') is desired; for acute oral toxicity (c), a higher category (e.g., ``III'') is desired; and for carcinogenicity (trinary), ``Non-required'' is desired.
	%
	Green cells highlight crucial properties where a positive result (``+'') is desired; for plasma protein binding, a lower value is desired; and for water solubility, values higher than -4 are desired~\cite{logs}.
\!\! \par
		\par
		\end{footnotesize}
	\end{tablenotes}
\end{threeparttable}
\end{scriptsize}
  \vspace{--10pt}    
\end{table*}

%\label{tbl:admet_nep}

\clearpage
%%%%%%%%%%%%%%%%%%%%%%%%%%%%%%%%%%%%%%%%%%%%%
\section{Algorithms}
\label{supp:algorithms}
%%%%%%%%%%%%%%%%%%%%%%%%%%%%%%%%%%%%%%%%%%%%%

Algorithm~\ref{alg:shapemol} describes the molecule generation process of \method.
%
Given a known ligand \molx, \method generates a novel molecule \moly that has a similar shape to \molx and thus potentially similar binding activity.
%
\method can also take the protein pocket \pocket as input and adjust the atoms of generated molecules for optimal fit and improved binding affinities.
%
Specifically, \method first calculates the shape embedding \shapehiddenmat for \molx using the shape encoder \SEE described in Algorithm~\ref{alg:see_shaperep}.
%
Based on \shapehiddenmat, \method then generates a novel molecule with a similar shape to \molx using the diffusion-based generative model \methoddiff as in Algorithm~\ref{alg:diffgen}.
%
During generation, \method can use shape guidance to directly modify the shape of \moly to closely resemble the shape of \molx.
%
When the protein pocket \pocket is available, \method can also use pocket guidance to ensure that \moly is specifically tailored to closely fit within \pocket.

\begin{algorithm}[!h]
    \caption{\method}
    \label{alg:shapemol}
         %\hspace*{\algorithmicindent} 
	\textbf{Required Input: $\molx$} \\
 	%\hspace*{\algorithmicindent} 
	\textbf{Optional Input: $\pocket$} 
    \begin{algorithmic}[1]
        \FullLineComment{calculate a shape embedding with Algorithm~\ref{alg:see_shaperep}}
        \State $\shapehiddenmat$, $\pc$ = $\SEE(\molx)$
        \FullLineComment{generate a molecule conditioned on the shape embedding with Algorithm~\ref{alg:diffgen}}
         \If{\pocket is not available}
        \State $\moly = \diffgenerative(\shapehiddenmat, \molx)$
        \Else
        \State $\moly = \diffgenerative(\shapehiddenmat, \molx, \pocket)$
        \EndIf
        \State \Return \moly
    \end{algorithmic}
\end{algorithm}
%\label{alg:shapemol}

\begin{algorithm}[!h]
    \caption{\SEE for shape embedding calculation}
    \label{alg:see_shaperep}
    \textbf{Required Input: $\molx$}
    \begin{algorithmic}[1]
        %\Require $\molx$
        \FullLineComment{sample a point cloud over the molecule surface shape}
        \State $\pc$ = $\text{samplePointCloud}(\molx)$
        \FullLineComment{encode the point cloud into a latent embedding (Equation~\ref{eqn:shape_embed})}
        \State $\shapehiddenmat = \SEE(\pc)$
        \FullLineComment{move the center of \pc to zero}
        \State $\pc = \pc - \text{center}(\pc)$
        \State \Return \shapehiddenmat, \pc
    \end{algorithmic}
\end{algorithm}
%\label{alg:see_shaperep}

\begin{algorithm}[!h]
    \caption{\diffgenerative for molecule generation}
    \label{alg:diffgen}
    	\textbf{Required Input: $\molx$, \shapehiddenmat} \\
 	%\hspace*{\algorithmicindent} 
	\textbf{Optional Input: $\pocket$} 
    \begin{algorithmic}[1]
        \FullLineComment{sample the number of atoms in the generated molecule}
        \State $n = \text{sampleAtomNum}(\molx)$
        \FullLineComment{sample initial positions and types of $n$ atoms}
        \State $\{\pos_T\}^n = \mathcal{N}(0, I)$
        \State $\{\atomfeat_T\}^n = C(K, \frac{1}{K})$
        \FullLineComment{generate a molecule by denoising $\{(\pos_T, \atomfeat_T)\}^n$ to $\{(\pos_0, \atomfeat_0)\}^n$}
        \For{$t = T$ to $1$}
            \IndentLineComment{predict the molecule without noise using the shape-conditioned molecule prediction module \molpred}{1.5}
            \State $(\tilde{\pos}_{0,t}, \tilde{\atomfeat}_{0,t}) = \molpred(\pos_t, \atomfeat_t, \shapehiddenmat)$
            \If{use shape guidance and $t > s$}
                \State $\tilde{\pos}_{0,t} = \shapeguide(\tilde{\pos}_{0,t}, \molx)$
                %\State $\tilde{\pos}_{0,t} = \pos^*_{0,t}$
            \EndIf
            \IndentLineComment{sample $(\pos_{t-1}, \atomfeat_{t-1})$ from $(\pos_t, \atomfeat_t)$ and $(\tilde{\pos}_{0,t}, \tilde{\atomfeat}_{0,t})$}{1.5}
            \State $\pos_{t-1} = P(\pos_{t-1}|\pos_t, \tilde{\pos}_{o,t})$
            \State $\atomfeat_{t-1} = P(\atomfeat_{t-1}|\atomfeat_t, \tilde{\atomfeat}_{o,t})$
            \If{use pocket guidance and $\pocket$ is available}
                \State $\pos_{t-1} = \pocketguide(\pos_{t-1}, \pocket)$
                %\State $\pos_{t-1} = \pos_{t-1}^*$
            \EndIf  
        \EndFor
        \State \Return $\moly = (\pos_0, \atomfeat_0)$
    \end{algorithmic}
\end{algorithm}
%\label{alg:diffgen}

%\input{algorithms/train_SE}
%\label{alg:train_se}

%\begin{algorithm}[!h]
    \caption{Training Procedure of \methoddiff}
    \label{alg:diffgen}
    \begin{algorithmic}[1]
        \Require $\shapehiddenmat, \molx, \pocket$
        \FullLineComment{sample the number of atoms in the generated molecule}
    \end{algorithmic}
\end{algorithm}
%\label{alg:train_diff}

%---------------------------------------------------------------------------------------------------------------------
\section{{Equivariance and Invariance}}
\label{supp:ei}
%---------------------------------------------------------------------------------------------------------------------

%.................................................................................................
\subsection{Equivariance}
\label{supp:ei:equivariance}
%.................................................................................................

{Equivariance refers to the property of a function $f(\pos)$ %\bo{is it the property of the function or embedding (x)?} 
that any translation and rotation transformation from the special Euclidean group SE(3)~\cite{Atz2021} applied to a geometric object
$\pos\in\mathbb{R}^3$ is mirrored in the output of $f(\pos)$, accordingly.
%
This property ensures $f(\pos)$ to learn a consistent representation of an object's geometric information, regardless of its orientation or location in 3D space.
%
%As a result, it provides $f(\pos)$ better generalization capabilities~\cite{Jonas20a}.
%
Formally, given any translation transformation $\mathbf{t}\in\mathbb{R}^3$ and rotation transformation $\mathbf{R}\in\mathbb{R}^{3\times3}$ ($\mathbf{R}^{\mathsf{T}}\mathbf{R}=\mathbb{I}$), %\xia{change the font types for $^{\mathsf{T}}$ and $\mathbb{I}$ in the entire manuscript}), 
$f(\pos)$ is equivariant with respect to these transformations %$g$ (\bo{where is $g$...})
if it satisfies
\begin{equation}
f(\mathbf{R}\pos+\mathbf{t}) = \mathbf{R}f(\pos) + \mathbf{t}. %\ \text{where}\ \hiddenpos = f(\pos).
\end{equation}
%
%where $\hiddenpos=f(\pos)$ is the output of $\pos$. 
%
In \method, both \SE and \methoddiff are developed to guarantee equivariance in capturing the geometric features of objects regardless of any translation or rotation transformations, as will be detailed in the following sections.
}

%.................................................................................................
\subsection{Invariance}
\label{supp:ei:invariance}
%.................................................................................................

%In contrast to equivariance, 
Invariance refers to the property of a function that its output {$f(\pos)$} remains constant under any translation and rotation transformations of the input $\pos$. %a geometric object's feature $\pos$.
%
This property enables $f(\pos)$ to accurately capture %a geometric object's 
the inherent features (e.g., atom features for 3D molecules) that are invariant of its orientation or position in 3D space.
%
Formally, $f(\pos)$ is invariant under any translation $\mathbf{t}$ and  rotation $\mathbf{R}$ if it satisfies
%
\begin{equation}
f(\mathbf{R}\pos+\mathbf{t}) = f(\pos).
\end{equation}
%
In \method, both \SE and \methoddiff capture the inherent features of objects in an invariant way, regardless of any translation or rotation transformations, as will be detailed in the following sections.

%%%%%%%%%%%%%%%%%%%%%%%%%%%%%%%%%%%%%%%%%%%%%
\section{Point Cloud Construction}
\label{supp:point_clouds}
%%%%%%%%%%%%%%%%%%%%%%%%%%%%%%%%%%%%%%%%%%%%%

In \method, we represented molecular surface shapes using point clouds (\pc).
%
$\pc$
serves as input to \SE, from which we derive shape latent embeddings.
%
To generate $\pc$, %\bo{\st{create this}}, \bo{generate $\pc$}
we initially generated a molecular surface mesh using the algorithm from the Open Drug Discovery Toolkit~\cite{Wjcikowski2015oddt}.
%
Following this, we uniformly sampled points on the mesh surface with probability proportional to the face area, %\xia{how to uniformly?}, ensuring the sampling is done proportionally to the face area, with
using the algorithm from PyTorch3D~\cite{ravi2020pytorch3d}.
%
This point cloud $\pc$ is then centralized by setting the center of its points to zero.
%
%

%%%%%%%%%%%%%%%%%%%%%%%%%%%%%%%%%%%%%%%%%%%%%
\section{Query Point Sampling}
\label{supp:training:shapeemb}
%%%%%%%%%%%%%%%%%%%%%%%%%%%%%%%%%%%%%%%%%%%%%

As described in Section ``Shape Decoder (\SED)'', the signed distances of query points $z_q$ to molecule surface shape $\pc$ are used to optimize \SE.
%
In this section, we present how to sample these points $z_q$ in 3D space.
%
Particularly, we first determined the bounding box around the molecular surface shape, using the maximum and minimum \mbox{($x$, $y$, $z$)-axis} coordinates for points in our point cloud \pc,
denoted as $(x_\text{min}, y_\text{min}, z_\text{min})$ and $(x_\text{max}, y_\text{max}, z_\text{max})$.
%
We extended this box slightly by defining its corners as \mbox{$(x_\text{min}-1, y_\text{min}-1, z_\text{min}-1)$} and \mbox{$(x_\text{max}+1, y_\text{max}+1, z_\text{max}+1)$}.
%
For sampling $|\mathcal{Z}|$ query points, we wanted an even distribution of points inside and outside the molecule surface shape.
%
%\ziqi{Typically, within this bounding box, molecules occupy only a small portion of volume, which makes it more likely to sample
%points outside the molecule surface shape.}
%
When a bounding box is defined around the molecule surface shape, there could be a lot of empty spaces within the box that the molecule does not occupy due to 
its complex and irregular shape.
%
This could lead to that fewer points within the molecule surface shape could be sampled within the box.
%
Therefore, we started by randomly sampling $3k$ points within our bounding box to ensure that there are sufficient points within the surface.
%
We then determined whether each point lies within the molecular surface, using an algorithm from Trimesh~\footnote{https://trimsh.org/} based on the molecule surface mesh.
%
If there are $n_w$ points found within the surface, we selected $n=\min(n_w, k/2)$ points from these points, 
and randomly choose the remaining 
%\bo{what do you mean by remaining? If all the 3k sampled points are inside the surface, you get no points left.} 
$k-n$ points 
from those outside the surface.
%
For each query point, we determined its signed distance to the molecule surface by its closest distance to points in \pc with a sign indicating whether it is inside the surface.

%%%%%%%%%%%%%%%%%%%%%%%%%%%%%%%%%%%%%%%%%%%%%
\section{Forward Diffusion (\diffnoise)}
\label{supp:forward}
%%%%%%%%%%%%%%%%%%%%%%%%%%%%%%%%%%%%%%%%%%%%%

%===================================================================
\subsection{{Forward Process}}
\label{supp:forward:forward}
%===================================================================

Formally, for atom positions, the probability of $\pos_t$ sampled given $\pos_{t-1}$, denoted as $q(\pos_t|\pos_{t-1})$, is defined as follows,
%\xia{revise the representation, should be $\beta^x_t$ -- note the space} as follows,
%
\begin{equation}
q(\pos_t|\pos_{t-1}) = \mathcal{N}(\pos_t|\sqrt{1-\beta^{\mathtt{x}}_t}\pos_{t-1}, \beta^{\mathtt{x}}_t\mathbb{I}), 
\label{eqn:noiseposinter}
\end{equation}
%
%\xia{should be a comma after the equation. you also missed it. }
%\st{in which} 
where %\hl{$\pos_0$ denotes the original atom position;} \xia{no $\pos_0$ in the equation...}
%$\mathbf{I}$ denotes the identity matrix;
$\mathcal{N}(\cdot)$ is a Gaussian distribution of $\pos_t$ with mean $\sqrt{1-\beta_t^{\mathtt{x}}}\pos_{t-1}$ and covariance $\beta_t^{\mathtt{x}}\mathbf{I}$.
%\xia{what is $\mathcal{N}$? what is $q$? you abused $q$. need to be crystal clear... }
%\bo{Should be $\sim$ not $=$ in the equation}
%
Following Hoogeboom \etal~\cite{hoogeboom2021catdiff}, 
%the forward process for the discrete atom feature $\atomfeat_t\in\mathbb{R}^K$ adds 
%categorical noise into $\atomfeat_{t-1}$ according to a variance schedule $\beta_t^v\in (0, 1)$. %as follows, %\hl{$\betav_t\in (0, 1)$} as follows,
%\xia{presentation...check across the entire manuscript... } as follows,
%
%\ziqi{Formally, 
for atom features, the probability of $\atomfeat_t$ across $K$ classes given $\atomfeat_{t-1}$ is defined as follows,
%
\begin{equation}
q(\atomfeat_t|\atomfeat_{t-1}) = \mathcal{C}(\atomfeat_t|(1-\beta^{\mathtt{v}}_t) \atomfeat_{t-1}+\beta^{\mathtt{v}}_t\mathbf{1}/K),
\label{eqn:noisetypeinter}
\end{equation}
%
where %\hl{$\atomfeat_0$ denotes the original atom positions}; 
$\mathcal{C}$ is a categorical distribution of $\atomfeat_t$ derived from the %by 
noising $\atomfeat_{t-1}$ with a uniform noise $\beta^{\mathtt{v}}_t\mathbf{1}/K$ across $K$ classes.
%adding an uniform noise $\beta^v_t$ to $\atomfeat_{t-1}$ across K classes.
%\xia{there is always a comma or period after the equations. Equations are part of a sentence. you always missed it. }
%\xia{what is $\mathcal{C}$? what does $q$ mean? it is abused. }

Since the above distributions form Markov chains, %} \xia{grammar!}, 
the probability of any $\pos_t$ or $\atomfeat_t$ can be derived from $\pos_0$ or $\atomfeat_0$:
%samples $\mol_0$ as follows,
%
\begin{eqnarray}
%\begin{aligned}
& q(\pos_t|\pos_{0}) & = \mathcal{N}(\pos_t|\sqrt{\cumalpha^{\mathtt{x}}_t}\pos_0, (1-\cumalpha^{\mathtt{x}}_t)\mathbb{I}), \label{eqn:noisepos}\\
& q(\atomfeat_t|\atomfeat_0)  & = \mathcal{C}(\atomfeat_t|\cumalpha^{\mathtt{v}}_t\atomfeat_0 + (1-\cumalpha^{\mathtt{v}}_t)\mathbf{1}/K), \label{eqn:noisetype}\\
& \text{where }\cumalpha^{\mathtt{u}}_t & = \prod\nolimits_{\tau=1}^{t}\alpha^{\mathtt{u}}_\tau, \ \alpha^{\mathtt{u}}_\tau=1 - \beta^{\mathtt{u}}_\tau, \ {\mathtt{u}}={\mathtt{x}} \text{ or } {\mathtt{v}}.\;\;\;\label{eqn:noiseschedule}
%\end{aligned}
\label{eqn:pos_prior}
\end{eqnarray}
%\xia{always punctuations after equations!!! also use ``eqnarray" instead of ``equation" + ``aligned" for multiple equations, each
%with a separate reference numbering...}
%\st{in which}, 
%where \ziqi{$\cumalpha^u_t = \prod_{\tau=1}^{t}\alpha^u_\tau$ and $\alpha^u_\tau=1 - \beta^u_\tau$ ($u$=$x$ or $v$)}.
%\xia{no such notations in the above equations; also subscript $s$ is abused with shape};
%$K$ is the number of categories for atom features.
%
%The details about noise schedules $\beta^x_t$ and $\beta^v_t$ are available in Supplementary Section \ref{XXX}. \ziqi{add trend}
%
Note that $\bar{\alpha}^{\mathtt{u}}_t$ ($\mathtt{u}={\mathtt{x}}\text{ or }{\mathtt{v}}$)
%($u$=$x$ or $v$) 
is monotonically decreasing from 1 to 0 over $t=[1,T]$. %\xia{=???}. 
%
As $t\rightarrow 1$, $\cumalpha^{\mathtt{x}}_t$ and $\cumalpha^{\mathtt{v}}_t$ are close to 1, leading to that $\pos_t$ or $\atomfeat_t$ approximates 
%the original data 
$\pos_0$ or $\atomfeat_0$.
%
Conversely, as  $t\rightarrow T$, $\cumalpha^{\mathtt{x}}_t$ and $\cumalpha^{\mathtt{v}}_t$ are close to 0,
leading to that $q(\pos_T|\pos_{0})$ %\st{$\rightarrow \mathcal{N}(\mathbf{0}, \mathbf{I})$} 
resembles  {$\mathcal{N}(\mathbf{0}, \mathbb{I})$} 
and $q(\atomfeat_T|\atomfeat_0)$ %\st{$\rightarrow \mathcal{C}(\mathbf{I}/K)$} 
resembles {$\mathcal{C}(\mathbf{1}/K)$}.

Using Bayes theorem, the ground-truth Normal posterior of atom positions $p(\pos_{t-1}|\pos_t, \pos_0)$ can be calculated in a
closed form~\cite{ho2020ddpm} as below,
%
\begin{eqnarray}
& p(\pos_{t-1}|\pos_t, \pos_0) = \mathcal{N}(\pos_{t-1}|\mu(\pos_t, \pos_0), \tilde{\beta}^\mathtt{x}_t\mathbb{I}), \label{eqn:gt_pos_posterior_1}\\
&\!\!\!\!\!\!\!\!\!\!\!\mu(\pos_t, \pos_0)\!=\!\frac{\sqrt{\bar{\alpha}^{\mathtt{x}}_{t-1}}\beta^{\mathtt{x}}_t}{1-\bar{\alpha}^{\mathtt{x}}_t}\pos_0\!+\!\frac{\sqrt{\alpha^{\mathtt{x}}_t}(1-\bar{\alpha}^{\mathtt{x}}_{t-1})}{1-\bar{\alpha}^{\mathtt{x}}_t}\pos_t, 
\tilde{\beta}^\mathtt{x}_t\!=\!\frac{1-\bar{\alpha}^{\mathtt{x}}_{t-1}}{1-\bar{\alpha}^{\mathtt{x}}_{t}}\beta^{\mathtt{x}}_t.\;\;\;
\end{eqnarray}
%
%\xia{Ziqi, please double check the above two equations!}
Similarly, the ground-truth categorical posterior of atom features $p(\atomfeat_{t-1}|\atomfeat_{t}, \atomfeat_0)$ can be calculated~\cite{hoogeboom2021catdiff} as below,
%
\begin{eqnarray}
& p(\atomfeat_{t-1}|\atomfeat_{t}, \atomfeat_0) = \mathcal{C}(\atomfeat_{t-1}|\mathbf{c}(\atomfeat_t, \atomfeat_0)), \label{eqn:gt_atomfeat_posterior_1}\\
& \mathbf{c}(\atomfeat_t, \atomfeat_0) = \tilde{\mathbf{c}}/{\sum_{k=1}^K \tilde{c}_k}, \label{eqn:gt_atomfeat_posterior_2} \\
& \tilde{\mathbf{c}} = [\alpha^{\mathtt{v}}_t\atomfeat_t + \frac{1 - \alpha^{\mathtt{v}}_t}{K}]\odot[\bar{\alpha}^{\mathtt{v}}_{t-1}\atomfeat_{0}+\frac{1-\bar{\alpha}^{\mathtt{v}}_{t-1}}{K}], 
\label{eqn:gt_atomfeat_posterior_3}
%\label{eqn:atomfeat_posterior}
\end{eqnarray}
%
%\xia{Ziqi: please double check the above equations!}
%
where $\tilde{c}_k$ denotes the likelihood of $k$-th class across $K$ classes in $\tilde{\mathbf{c}}$; 
$\odot$ denotes the element-wise product operation;
$\tilde{\mathbf{c}}$ is calculated using $\atomfeat_t$ and $\atomfeat_{0}$ and normalized into $\mathbf{c}(\atomfeat_t, \atomfeat_0)$ so as to represent
probabilities. %\xia{is this correct? is $\tilde{c}_k$ always greater than 0?}
%\xia{how is it calculated?}.
%\ziqi{the likelihood distribution $\tilde{c}$ is calculated by $p(\atomfeat_t|\atomfeat_{t-1})p(\atomfeat_{t-1}|\atomfeat_0)$, according to 
%Equation~\ref{eqn:noisetypeinter} and \ref{eqn:noisetype}.
%\xia{need to write the key idea of the above calculation...}
%
The proof of the above equations is available in Supplementary Section~\ref{supp:forward:proof}.

%===================================================================
\subsection{Variance Scheduling in \diffnoise}
\label{supp:forward:variance}
%===================================================================

Following Guan \etal~\cite{guan2023targetdiff}, we used a sigmoid $\beta$ schedule for the variance schedule $\beta_t^{\mathtt{x}}$ of atom coordinates as below,

\begin{equation}
\beta_t^{\mathtt{x}} = \text{sigmoid}(w_1(2 t / T - 1)) (w_2 - w_3) + w_3
\end{equation}
in which $w_i$($i$=1,2, or 3) are hyperparameters; $T$ is the maximum step.
%
We set $w_1=6$, $w_2=1.e-7$ and $w_3=0.01$.
%
For atom types, we used a cosine $\beta$ schedule~\cite{nichol2021} for $\beta_t^{\mathtt{v}}$ as below,

\begin{equation}
\begin{aligned}
& \bar{\alpha}_t^{\mathtt{v}} = \frac{f(t)}{f(0)}, f(t) = \cos(\frac{t/T+s}{1+s} \cdot \frac{\pi}{2})^2\\
& \beta_t^{\mathtt{v}} = 1 - \alpha_t^{\mathtt{v}} = 1 - \frac{\bar{\alpha}_t^{\mathtt{v}} }{\bar{\alpha}_{t-1}^{\mathtt{v}} }
\end{aligned}
\end{equation}
in which $s$ is a hyperparameter and set as 0.01.

As shown in Section ``Forward Diffusion Process'', the values of $\beta_t^{\mathtt{x}}$ and $\beta_t^{\mathtt{v}}$ should be 
sufficiently small to ensure the smoothness of forward diffusion process. In the meanwhile, their corresponding $\bar{\alpha}_t$
values should decrease from 1 to 0 over $t=[1,T]$.
%
Figure~\ref{fig:schedule} shows the values of $\beta_t$ and $\bar{\alpha}_t$ for atom coordinates and atom types with our hyperparameters.
%
Please note that the value of $\beta_{t}^{\mathtt{x}}$ is less than 0.1 for 990 out of 1,000 steps. %\bo{\st{, though it increases when $t$ is close to 1,000}}.
%
This guarantees the smoothness of the forward diffusion process.
%\bo{add $\beta_t^{\mathtt{x}}$ and $\beta_t^{\mathtt{v}}$ in the legend of the figure...}
%\bo{$\beta_t^{\mathtt{v}}$ does not look small when $t$ is close to 1000...}

\begin{figure}
	\begin{subfigure}[t]{.45\linewidth}
		\centering
		\includegraphics[width=.7\linewidth]{figures/var_schedule_beta.pdf}
	\end{subfigure}
	%
	\hfill
	\begin{subfigure}[t]{.45\linewidth}
		\centering
		\includegraphics[width=.7\linewidth]{figures/var_schedule_alpha.pdf}
	\end{subfigure}
	\caption{Schedule}
	\label{fig:schedule}
\end{figure}

%===================================================================
\subsection{Derivation of Forward Diffusion Process}
\label{supp:forward:proof}
%===================================================================

In \method, a Gaussian noise and a categorical noise are added to continuous atom position and discrete atom features, respectively.
%
Here, we briefly describe the derivation of posterior equations (i.e., Eq.~\ref{eqn:gt_pos_posterior_1}, and   \ref{eqn:gt_atomfeat_posterior_1}) for atom positions and atom types in our work.
%
We refer readers to Ho \etal~\cite{ho2020ddpm} and Kong \etal~\cite{kong2021diffwave} %\bo{add XXX~\etal here...} \cite{ho2020ddpm,kong2021diffwave} 	
for a detailed description of diffusion process for continuous variables and Hoogeboom \etal~\cite{hoogeboom2021catdiff} for
%\bo{add XXX~\etal here...} \cite{hoogeboom2021catdiff} for
the description of diffusion process for discrete variables.

For continuous atom positions, as shown in Kong \etal~\cite{kong2021diffwave}, according to Bayes theorem, given $q(\pos_t|\pos_{t-1})$ defined in Eq.~\ref{eqn:noiseposinter} and 
$q(\pos_t|\pos_0)$ defined in Eq.~\ref{eqn:noisepos}, the posterior $q(\pos_{t-1}|\pos_{t}, \pos_0)$ is derived as below (superscript $\mathtt{x}$ is omitted for brevity),

\begin{equation}
\begin{aligned}
& q(\pos_{t-1}|\pos_{t}, \pos_0)  = \frac{q(\pos_t|\pos_{t-1}, \pos_0)q(\pos_{t-1}|\pos_0)}{q(\pos_t|\pos_0)} \\
& =  \frac{\mathcal{N}(\pos_t|\sqrt{1-\beta_t}\pos_{t-1}, \beta_{t}\mathbf{I}) \mathcal{N}(\pos_{t-1}|\sqrt{\bar{\alpha}_{t-1}}\pos_{0}, (1-\bar{\alpha}_{t-1})\mathbf{I}) }{ \mathcal{N}(\pos_{t}|\sqrt{\bar{\alpha}_t}\pos_{0}, (1-\bar{\alpha}_t)\mathbf{I})}\\
& =  (2\pi{\beta_t})^{-\frac{3}{2}} (2\pi{(1-\bar{\alpha}_{t-1})})^{-\frac{3}{2}} (2\pi(1-\bar{\alpha}_t))^{\frac{3}{2}} \times \exp( \\
& -\frac{\|\pos_t - \sqrt{\alpha}_t\pos_{t-1}\|^2}{2\beta_t} -\frac{\|\pos_{t-1} - \sqrt{\bar{\alpha}}_{t-1}\pos_{0} \|^2}{2(1-\bar{\alpha}_{t-1})} \\
& + \frac{\|\pos_t - \sqrt{\bar{\alpha}_t}\pos_0\|^2}{2(1-\bar{\alpha}_t)}) \\
& = (2\pi\tilde{\beta}_t)^{-\frac{3}{2}} \exp(-\frac{1}{2\tilde{\beta}_t}\|\pos_{t-1}-\frac{\sqrt{\bar{\alpha}_{t-1}}\beta_t}{1-\bar{\alpha}_t}\pos_0 \\
& - \frac{\sqrt{\alpha_t}(1-\bar{\alpha}_{t-1})}{1-\bar{\alpha}_t}\pos_{t}\|^2) \\
& \text{where}\ \tilde{\beta}_t = \frac{1-\bar{\alpha}_{t-1}}{1-\bar{\alpha}_t}\beta_t.
\end{aligned}
\end{equation}
%\bo{marked part does not look right to me.}
%\bo{How to you derive from the second equation to the third one?}

Therefore, the posterior of atom positions is derived as below,

\begin{equation}
q(\pos_{t-1}|\pos_{t}, \pos_0)\!\!=\!\!\mathcal{N}(\pos_{t-1}|\frac{\sqrt{\bar{\alpha}_{t-1}}\beta_t}{1-\bar{\alpha}_t}\pos_0 + \frac{\sqrt{\alpha_t}(1-\bar{\alpha}_{t-1})}{1-\bar{\alpha}_t}\pos_{t}, \tilde{\beta}_t\mathbf{I}).
\end{equation}

For discrete atom features, as shown in Hoogeboom \etal~\cite{hoogeboom2021catdiff} and Guan \etal~\cite{guan2023targetdiff},
according to Bayes theorem, the posterior $q(\atomfeat_{t-1}|\atomfeat_{t}, \atomfeat_0)$ is derived as below (supperscript $\mathtt{v}$ is omitted for brevity),

\begin{equation}
\begin{aligned}
& q(\atomfeat_{t-1}|\atomfeat_{t}, \atomfeat_0) =  \frac{q(\atomfeat_t|\atomfeat_{t-1}, \atomfeat_0)q(\atomfeat_{t-1}|\atomfeat_0)}{\sum_{\scriptsize{\atomfeat}_{t-1}}q(\atomfeat_t|\atomfeat_{t-1}, \atomfeat_0)q(\atomfeat_{t-1}|\atomfeat_0)} \\
%& = \frac{\mathcal{C}(\atomfeat_t|(1-\beta_t)\atomfeat_{t-1} + \beta_t\frac{\mathbf{1}}{K}) \mathcal{C}(\atomfeat_{t-1}|\bar{\alpha}_{t-1}\atomfeat_0+(1-\bar{\alpha}_{t-1})\frac{\mathbf{1}}{K})} \\
\end{aligned}
\end{equation}

For $q(\atomfeat_t|\atomfeat_{t-1}, \atomfeat_0)$, we have % $\atomfeat_t=\atomfeat_{t-1}$ with probability $1-\beta_t+\beta_t / K$, and $\atomfeat_t \neq \atomfeat_{t-1}$
%with probability $\beta_t / K$.
%
%Therefore, this function can be symmetric, that is, 
%
\begin{equation}
\begin{aligned}
q(\atomfeat_t|\atomfeat_{t-1}, \atomfeat_0) & = \mathcal{C}(\atomfeat_t|(1-\beta_t)\atomfeat_{t-1} + \beta_t/{K})\\
& = \begin{cases}
1-\beta_t+\beta_t/K,\!&\text{when}\ \atomfeat_{t} = \atomfeat_{t-1},\\
\beta_t / K,\! &\text{when}\ \atomfeat_{t} \neq \atomfeat_{t-1},
\end{cases}\\
& = \mathcal{C}(\atomfeat_{t-1}|(1-\beta_t)\atomfeat_{t} + \beta_t/{K}).
\end{aligned}
%\mathcal{C}(\atomfeat_{t-1}|(1-\beta_{t})\atomfeat_{t} + \beta_t\frac{\mathbf{1}}{K}).
\end{equation}
%
Therefore, we have
%\bo{why it can be symmetric}
%
\begin{equation}
\begin{aligned}
& q(\atomfeat_t|\atomfeat_{t-1}, \atomfeat_0)q(\atomfeat_{t-1}|\atomfeat_0) \\
& = \mathcal{C}(\atomfeat_{t-1}|(1-\beta_t)\atomfeat_{t} + \beta_t\frac{\mathbf{1}}{K}) \mathcal{C}(\atomfeat_{t-1}|\bar{\alpha}_{t-1}\atomfeat_0+(1-\bar{\alpha}_{t-1})\frac{\mathbf{1}}{K}) \\
& = [\alpha_t\atomfeat_t + \frac{1 - \alpha_t}{K}]\odot[\bar{\alpha}_{t-1}\atomfeat_{0}+\frac{1-\bar{\alpha}_{t-1}}{K}].
\end{aligned}
\end{equation}
%
%\bo{what is $\tilde{\mathbf{c}}$...}
Therefore, with $q(\atomfeat_t|\atomfeat_{t-1}, \atomfeat_0)q(\atomfeat_{t-1}|\atomfeat_0) = \tilde{\mathbf{c}}$, the posterior is as below,

\begin{equation}
q(\atomfeat_{t-1}|\atomfeat_{t}, \atomfeat_0) = \mathcal{C}(\atomfeat_{t-1}|\mathbf{c}(\atomfeat_t, \atomfeat_0)) = \frac{\tilde{\mathbf{c}}}{\sum_{k}^K\tilde{c}_k}.
\end{equation}

%%%%%%%%%%%%%%%%%%%%%%%%%%%%%%%%%%%%%%%%%%%%%
\section{{Backward Generative Process} (\diffgenerative)}
\label{supp:backward}
%%%%%%%%%%%%%%%%%%%%%%%%%%%%%%%%%%%%%%%%%%%%%

Following Ho \etal~\cite{ho2020ddpm}, with $\tilde{\pos}_{0,t}$, the probability of $\pos_{t-1}$ denoised from $\pos_t$, denoted as $p(\pos_{t-1}|\pos_t)$,
can be estimated %\hl{parameterized} \xia{???} 
by the approximated posterior $p_{\boldsymbol{\Theta}}(\pos_{t-1}|\pos_t, \tilde{\pos}_{0,t})$ as below,
%
\begin{equation}
\begin{aligned}
p(\pos_{t-1}|\pos_t) & \approx p_{\boldsymbol{\Theta}}(\pos_{t-1}|\pos_t, \tilde{\pos}_{0,t}) \\
& = \mathcal{N}(\pos_{t-1}|\mu_{\boldsymbol{\Theta}}(\pos_t, \tilde{\pos}_{0,t}),\tilde{\beta}_t^{\mathtt{x}}\mathbb{I}),
\end{aligned}
\label{eqn:aprox_pos_posterior}
\end{equation}
%
where ${\boldsymbol{\Theta}}$ is the learnable parameter; $\mu_{\boldsymbol{\Theta}}(\pos_t, \tilde{\pos}_{0,t})$ is an estimate %estimation
of $\mu(\pos_t, \pos_{0})$ by replacing $\pos_0$ with its estimate $\tilde{\pos}_{0,t}$ 
in Equation~{\ref{eqn:gt_pos_posterior_1}}.
%
Similarly, with $\tilde{\atomfeat}_{0,t}$, the probability of $\atomfeat_{t-1}$ denoised from $\atomfeat_t$, denoted as $p(\atomfeat_{t-1}|\atomfeat_t)$, 
can be estimated %\hl{parameterized} 
by the approximated posterior $p_{\boldsymbol{\Theta}}(\atomfeat_{t-1}|\atomfeat_t, \tilde{\atomfeat}_{0,t})$ as below,
%
\begin{equation}
\begin{aligned}
p(\atomfeat_{t-1}|\atomfeat_t)\approx p_{\boldsymbol{\Theta}}(\atomfeat_{t-1}|\atomfeat_{t}, \tilde{\atomfeat}_{0,t}) 
=\mathcal{C}(\atomfeat_{t-1}|\mathbf{c}_{\boldsymbol{\Theta}}(\atomfeat_t, \tilde{\atomfeat}_{0,t})),\!\!\!\!
\end{aligned}
\label{eqn:aprox_atomfeat_posterior}
\end{equation}
%
where $\mathbf{c}_{\boldsymbol{\Theta}}(\atomfeat_t, \tilde{\atomfeat}_{0,t})$ is an estimate of $\mathbf{c}(\atomfeat_t, \atomfeat_0)$
by replacing $\atomfeat_0$  
with its estimate $\tilde{\atomfeat}_{0,t}$ in Equation~\ref{eqn:gt_atomfeat_posterior_1}.



%===================================================================
\section{\method Loss Function Derivation}
\label{supp:training:loss}
%===================================================================

In this section, we demonstrate that a step weight $w_t^{\mathtt{x}}$ based on the signal-to-noise ratio $\lambda_t$ should be 
included into the atom position loss (Eq.~\ref{eqn:diff:obj:pos}).
%
In the diffusion process for continuous variables, the optimization problem is defined 
as below~\cite{ho2020ddpm},
%
\begin{equation*}
\begin{aligned}
& \arg\min_{\boldsymbol{\Theta}}KL(q(\pos_{t-1}|\pos_t, \pos_0)|p_{\boldsymbol{\Theta}}(\pos_{t-1}|\pos_t, \tilde{\pos}_{0,t})) \\
& = \arg\min_{\boldsymbol{\Theta}} \frac{\bar{\alpha}_{t-1}(1-\alpha_t)}{2(1-\bar{\alpha}_{t-1})(1-\bar{\alpha}_{t})}\|\tilde{\pos}_{0, t}-\pos_0\|^2 \\
& = \arg\min_{\boldsymbol{\Theta}} \frac{1-\alpha_t}{2(1-\bar{\alpha}_{t-1})\alpha_{t}} \|\tilde{\boldsymbol{\epsilon}}_{0,t}-\boldsymbol{\epsilon}_0\|^2,
\end{aligned}
\end{equation*}
where $\boldsymbol{\epsilon}_0 = \frac{\pos_t - \sqrt{\bar{\alpha}_t}\pos_0}{\sqrt{1-\bar{\alpha}_t}}$ is the ground-truth noise variable sampled from $\mathcal{N}(\mathbf{0}, \mathbf{1})$ and is used to sample $\pos_t$ from $\mathcal{N}(\pos_t|\sqrt{\cumalpha_t}\pos_0, (1-\cumalpha_t)\mathbf{I})$ in Eq.~\ref{eqn:noisetype};
$\tilde{\boldsymbol{\epsilon}}_0 = \frac{\pos_t - \sqrt{\bar{\alpha}_t}\tilde{\pos}_{0, t}}{\sqrt{1-\bar{\alpha}_t}}$ is the predicted noise variable. 

%A simplified training objective is proposed by Ho \etal~\cite{ho2020ddpm} as below,
Ho \etal~\cite{ho2020ddpm} further simplified the above objective as below and
demonstrated that the simplified one can achieve better performance:
%
\begin{equation}
\begin{aligned}
& \arg\min_{\boldsymbol{\Theta}} \|\tilde{\boldsymbol{\epsilon}}_{0,t}-\boldsymbol{\epsilon}_0\|^2 \\
& = \arg\min_{\boldsymbol{\Theta}} \frac{\bar{\alpha}_t}{1-\bar{\alpha}_t}\|\tilde{\pos}_{0,t}-\pos_0\|^2,
\end{aligned}
\label{eqn:supp:losspos}
\end{equation}
%
where $\lambda_t=\frac{\bar{\alpha}_t}{1-\bar{\alpha}_t}$ is the signal-to-noise ratio.
%
While previous work~\cite{guan2023targetdiff} applies uniform step weights across
different steps, we demonstrate that a step weight should be included into the atom position loss according to Eq.~\ref{eqn:supp:losspos}.
%
However, the value of $\lambda_t$ could be very large when $\bar{\alpha}_t$ is close to 1 as $t$ approaches 1.
%
Therefore, we clip the value of $\lambda_t$ with threshold $\delta$ in Eq.~\ref{eqn:diff:obj:pos}.




\end{document}


% This document was modified from the file originally made available by
% Pat Langley and Andrea Danyluk for ICML-2K. This version was created
% by Iain Murray in 2018, and modified by Alexandre Bouchard in
% 2019 and 2021 and by Csaba Szepesvari, Gang Niu and Sivan Sabato in 2022.
% Modified again in 2023 and 2024 by Sivan Sabato and Jonathan Scarlett.
% Previous contributors include Dan Roy, Lise Getoor and Tobias
% Scheffer, which was slightly modified from the 2010 version by
% Thorsten Joachims & Johannes Fuernkranz, slightly modified from the
% 2009 version by Kiri Wagstaff and Sam Roweis's 2008 version, which is
% slightly modified from Prasad Tadepalli's 2007 version which is a
% lightly changed version of the previous year's version by Andrew
% Moore, which was in turn edited from those of Kristian Kersting and
% Codrina Lauth. Alex Smola contributed to the algorithmic style files.
