%%%%%%%%%%%%%%%%%%%%%%%%%%%%%%%%%%%%%%%%%%
% BoukeTech Template
% LaTeX Journal Article Template
% Version 1.0 (October 6, 2024)
%
% This template is created and maintained by:
% Dr. Mohamed Aly Bouke (bouke@ieee.org)
%
% License:
% This template is open-source and distributed under the terms of 
% the CC BY-NC-SA 4.0 License (https://creativecommons.org/licenses/by-nc-sa/4.0/).
%
% Description:
% The BoukeTech Template is a modern LaTeX template specifically 
% designed for academic journal articles. It includes customizable 
% features for improved typography, ORCID linking, and abstract 
% styling, as well as built-in support for multi-column content 
% and advanced reference management using biblatex. The design 
% is optimized for clarity and readability, ensuring your paper 
% meets professional publication standards.
%
% NOTE: The bibliography must be compiled using the biber engine.
%
%%%%%%%%%%%%%%%%%%%%%%%%%%%%%%%%%%%%%%%%%%
%----------------------------------------------------------------------------------------
%	PACKAGES AND OTHER DOCUMENT CONFIGURATIONS
%----------------------------------------------------------------------------------------




\documentclass[10pt, twocolumn]{article}
\usepackage[utf8]{inputenc}
\usepackage[backend=biber,style=numeric,sorting=none]{biblatex}
\addbibresource{references.bib}
% Add these commands to suppress TeX-related metadata
%\pdfinfoomitdate=1
%\pdftrailerid{}
%\pdfsuppressptexinfo=1

% Include packages and configurations from separate files
%%%%%%%%%%%%%%%%%%%%%%%%%%%%%%%%%%%%%%%%%%%%%%%%%%%%%%%%%%%%%%%%%%%%%%%%%%%%%%

%% Beautiful mathematics
\usepackage{amsmath, amssymb, amsfonts} 
\usepackage{nicefrac}
\usepackage{mathtools}
\usepackage{bm, bbm}
\usepackage[scr=boondoxo,scrscaled=1.05]{mathalfa}

%% References in the correct format 
%\usepackage[square,numbers]{natbib}
%\def\bibfont{\footnotesize} % fix to have the same font size as without natbib

\usepackage[sort, compress, space]{cite}            


%% Enumerate nicely 
\usepackage{enumitem}

%% Different color comments and commenting large parts of the text
\usepackage{xcolor}
\usepackage{comment}
\usepackage{soul}

%% Hyper references
\usepackage{hyperref}
\usepackage{cleveref}
%\usepackage[numbers]{natbib}

\usepackage{tikz}
%\usepackage{thm-restate}
%% Appendix package
%\usepackage{appendix}

%% Random text to test spacing 
\usepackage{blindtext}

\usepackage{afterpage}

\usepackage{algorithm, algorithmic}    



\usepackage{dsfont}

\usepackage{tikz}
\usepackage{graphicx}
\usepackage{tikzscale}
\usepackage{pgfplots}
\pgfplotsset{compat=newest}
\usepackage{xfrac}

\usepackage{thm-restate}

%\usepackage{subcaption}

\usepackage{balance}

\usepackage{cite}
\usepackage{amsmath,amssymb,amsfonts}
\usepackage{balance}
\usepackage{algorithmic}
\usepackage{graphicx}
\usepackage{textcomp}
\usepackage{xcolor}
\usepackage{amsmath}
\usepackage{amssymb}
\usepackage[mathscr]{euscript}
\usepackage{comment}
\usepackage{xcolor}
\usepackage{enumitem} 
\usepackage{amsthm}

       % Packages used in the document
% Define custom colors
\definecolor{orcid}{HTML}{A6CE39} % ORCID green color
\definecolor{primary}{RGB}{40, 116, 166} % Custom primary color
\definecolor{secondary}{RGB}{46, 204, 113} % Custom secondary color
\definecolor{abstractborder}{RGB}{40, 116, 166} % Color for the abstract border
\definecolor{abstractbg}{RGB}{240, 248, 255} % Light background color for the abstract
\definecolor{headerline}{RGB}{192, 192, 192} % Light gray color for header/footer line

% Geometry for margin sizes
\geometry{a4paper, margin=1in}

% Custom Variables
\newcommand{\papertitle}{Data Dams: A Novel Framework for Regulating and Managing Data Flow in Large-Scale Systems}   % Title of the paper


% Custom Variables (Authors and Affiliations)

% Authors
\newcommand{\authorone}{Mohamed Aly Bouke}                  % First author
\newcommand{\authoroneORCID}{https://orcid.org/0000-0003-3264-601X}

\newcommand{\authortwo}{Azizol Abdullah}                   % Second author
\newcommand{\authortwoORCID}{https://orcid.org/0000-0001-8321-9259,}

\newcommand{\authorthree}{Korhan Cengiz}                   % Third author
\newcommand{\authorthreeORCID}{https://orcid.org/0000-0001-6594-8861}

\newcommand{\authorfour}{Nikola Ivković}                   % Fourth author
\newcommand{\authorfourORCID}{https://orcid.org/0000-0003-1730-2518}

\newcommand{\authorfive}{Ivan Mihaljević}                   % fiveth author
\newcommand{\authorfiveORCID}{https://orcid.org/0009-0000-7105-8312}

\newcommand{\authorsix}{Mudathir Ahmed Mohamud}                   % sixtth author
\newcommand{\authorsixORCID}{https://orcid.org/0000-0001-8626-8347}

\newcommand{\authorseven}{Ahmed Kowrina
}                   % sixtth author
\newcommand{\authorsevenORCID}{https://orcid.org/0009-0009-5903-7231}

% Affiliations
\newcommand{\affila}{Department of Communication Technology and Network, Faculty of Computer Science and Information Technology, Universiti Putra Malaysia, Serdang 43400, Malaysia.}
\newcommand{\affilb}{Department of Electrical-Electronics Engineering, Biruni University, Istanbul, Turkey.}
\newcommand{\affilc}{Faculty of Organization and Informatics, University of Zagreb, Pavlinska 2, 42000 Varaždin, Croatia.}
\newcommand{\affild}{Faculty of Computing, SIMAD University, Mogadishu 252, Somalia.}
\newcommand{\affile}{Department of Computer Science, Faculty of Computer Science and Information Technology, Universiti Putra Malaysia, 43400 Serdang, Selangor, Malaysia}
\newcommand{\affilf}{Département d'Informatique et Mathématiques, Faculté des Sciences et Techniques, Université de Nouakchott Al Aasriya, 880, Nouakchott, Mauritanie}



\newcommand{\correspondingauthor}{bouke@ieee.org} % Corresponding author's email
\newcommand{\journalname}{ArXiv.org e-Print archive} % Journal name for header
\newcommand{\doi}{https://doi.org/xxxxxxx}  % DOI


\newcommand{\keywords}{Data Flow Management, Big Data Optimization, Dynamic Data Regulation, Queuing Theory in Data Systems, Cloud Computing Efficiency.} % Keywords for the paper
\newcommand{\citationstyle}{IEEE} % Define citation style (Set to IEEE or APA)


% New footer content: General academic statement
\newcommand{\academicstatement}{Published under an TBD license. All rights reserved.}

% Header/Footer settings with decorative lines
\pagestyle{fancy}
\fancyhf{}
\fancyhead[L]{\textcolor{primary}{\journalname} \textcolor{black}{| DOI: \href{\doi}{\doi}}} % Journal name and DOI in the left header
\usepackage{lastpage} % Add this package to reference the total page count

\fancyhead[R]{\textcolor{primary}{\thepage/\pageref{LastPage}}} % Modify this line to show "1/x"

\fancyfoot[C]{\academicstatement} % Footer contains the general academic statement
\fancyhead[C]{\tikz[baseline]{\draw[headerline,thick](0,-1em)--(2,0);}} % Decorative line in header

% Title and section formatting with improved typography
\titleformat{\section}
  {\color{primary}\normalfont\Large\bfseries}
  {\thesection}{1em}{}
  
\titleformat{\subsection}
  {\color{primary}\normalfont\large\bfseries}
  {\thesubsection}{1em}{}

% Hyperlink customization
\hypersetup{
    colorlinks=true,
    linkcolor=primary,
    urlcolor=primary,
    citecolor=blue
}

% Title formatting with enhanced typography and spacing
\makeatletter
\renewcommand\maketitle{
  \begin{center}
    {\huge\bfseries\textcolor{primary}{\papertitle}\par\vspace{0.5em}} % Large and bold title
    \vspace{0.3em}
    {\large

    %\iffalse

       \authorone\textsuperscript{1,*}\href{\authoroneORCID}{\raisebox{1.5pt}{\textcolor{orcid}{\faOrcid}}}, 
    \authortwo\textsuperscript{1}\href{\authortwoORCID}{\raisebox{1.5pt}{\textcolor{orcid}{\faOrcid}}}, 
    \authorthree\textsuperscript{2}\href{\authorthreeORCID}{\raisebox{1.5pt}{\textcolor{orcid}{\faOrcid}}}, 
    \authorfour\textsuperscript{3}\href{\authorfourORCID}{\raisebox{1.5pt}{\textcolor{orcid}{\faOrcid}}}, 
       \authorfive\textsuperscript{3}\href{\authorfiveORCID}{\raisebox{1.5pt}{\textcolor{orcid}{\faOrcid}}},
       \authorsix\textsuperscript{4,5}\href{\authorsixORCID}{\raisebox{1.5pt}{\textcolor{orcid}{\faOrcid}}},
       \authorseven\textsuperscript{6}\href{\authorsevenORCID}{\raisebox{1.5pt}{\textcolor{orcid}{\faOrcid}}}
      \par}
    \vspace{0.3em}

      \textsuperscript{1}\affila \par
    \textsuperscript{2}\affilb \par
    \textsuperscript{3}\affilc \par
   \textsuperscript{4}\affild \par
   \textsuperscript{5}\affile \par
   \textsuperscript{6}\affilf \par
    \vspace{0.3em}
    {\small
      \textbf{\textcolor{black}{*Corresponding author:}} \href{mailto:\correspondingauthor}{\textcolor{black}{\correspondingauthor}}
      \par}
      
     %\fi
       
    % Short blue decorative line under everything
    \vspace{1.0em}
    \begin{center}
      \noindent\textcolor{primary}{\rule{0.4\textwidth}{0.6pt}} % A short blue line
    \end{center}
  \end{center}
}
\makeatother


% Custom abstract environment: Styled abstract box with vertical line and shaded background
\renewenvironment{abstract}
    {\noindent\textbf{\textcolor{primary}{Abstract:}}\par
     \begin{tcolorbox}[colframe=abstractborder, colback=abstractbg, sharp corners, boxrule=0.5mm, left=6pt, top=6pt, bottom=6pt, width=\dimexpr\linewidth\relax, arc=2mm]
     \textit\ignorespaces}
    {\end{tcolorbox}\par\vspace{1.5em}}

% Check if two-column layout is used
\newcommand{\checktwocolumn}[2]{\ifthenelse{\boolean{@twocolumn}}{#1}{#2}}

  % Customizations like colors, header/footer settings

% Document starts here
\begin{document}

% Conditional one/two-column layout for title, authors, and abstract
\checktwocolumn{
    % If two-column, use this layout
    \twocolumn[{
        \maketitle

        % Abstract
        \begin{abstract}
Retrieval-Augmented Generation (RAG) is often used with Large Language Models (LLMs) to infuse domain knowledge or user-specific information. In RAG, given a user query, a retriever extracts chunks of relevant text from a knowledge base. These chunks are sent to an LLM as part of the input prompt. Typically, any given chunk is repeatedly retrieved across user questions. However, currently, for every question, attention-layers in LLMs fully compute the key values (KVs) repeatedly for the input chunks, as state-of-the-art methods cannot reuse KV-caches when chunks appear at arbitrary locations with arbitrary contexts. Naive reuse leads to output quality degradation.  This leads to potentially redundant computations on expensive GPUs and increases latency. In this work, we propose \sys, a system for managing and reusing precomputed KVs corresponding to the text chunks (we call \textit{chunk-caches}) in RAG-based systems. We present how to identify \hl{\textit{chunk-caches} that are reusable}, how to efficiently perform a small fraction of recomputation to \textit{fix} the cache to maintain output quality, and how to efficiently store and evict \textit{chunk-caches} in the hardware for maximizing reuse while masking any overheads. With real production workloads as well as synthetic datasets, we show that \sys reduces redundant computation by \textbf{51\%} over SOTA prefix-caching and \textbf{75\%} over full recomputation.
\hl{Additionally, with continuous batching on a real production workload, we get a \textbf{1.6$\times$} speedup in throughput and a \textbf{2$\times$} reduction in end-to-end response latency over prefix-caching while maintaining quality, for both the \llama-3-8B and \llama-3-70B models. 
}
\end{abstract}




   % Abstract section

       
    }]
}{
    % If one-column, use this layout
    \maketitle

    % Abstract
    \begin{abstract}
Retrieval-Augmented Generation (RAG) is often used with Large Language Models (LLMs) to infuse domain knowledge or user-specific information. In RAG, given a user query, a retriever extracts chunks of relevant text from a knowledge base. These chunks are sent to an LLM as part of the input prompt. Typically, any given chunk is repeatedly retrieved across user questions. However, currently, for every question, attention-layers in LLMs fully compute the key values (KVs) repeatedly for the input chunks, as state-of-the-art methods cannot reuse KV-caches when chunks appear at arbitrary locations with arbitrary contexts. Naive reuse leads to output quality degradation.  This leads to potentially redundant computations on expensive GPUs and increases latency. In this work, we propose \sys, a system for managing and reusing precomputed KVs corresponding to the text chunks (we call \textit{chunk-caches}) in RAG-based systems. We present how to identify \hl{\textit{chunk-caches} that are reusable}, how to efficiently perform a small fraction of recomputation to \textit{fix} the cache to maintain output quality, and how to efficiently store and evict \textit{chunk-caches} in the hardware for maximizing reuse while masking any overheads. With real production workloads as well as synthetic datasets, we show that \sys reduces redundant computation by \textbf{51\%} over SOTA prefix-caching and \textbf{75\%} over full recomputation.
\hl{Additionally, with continuous batching on a real production workload, we get a \textbf{1.6$\times$} speedup in throughput and a \textbf{2$\times$} reduction in end-to-end response latency over prefix-caching while maintaining quality, for both the \llama-3-8B and \llama-3-70B models. 
}
\end{abstract}




   % Abstract section

   
}

% Body content sections
\setlength{\parindent}{0pt}

\documentclass[../main.tex]{subfiles}
\graphicspath{{../images/}}
\makeatletter
\def\input@path{{../images/}}
\makeatother
\begin{document}
\section{Introduction}
\begin{figure}
\centering
\begin{tikzpicture}
\node[inner sep=0pt] (ws) at (0, 0) {
\includegraphics[height=.4\textwidth, trim={10cm 0 10cm 0},clip]{world_space.png}};
\node[inner sep=0pt] (cs) at (6,0) {\includegraphics[height=.4\textwidth, trim={10cm 1cm 10cm 4cm},clip]{conf_space.png}};
\end{tikzpicture}
\vspace{-5pt}
\label{fig:pbrm_intro}
\caption{\textbf{Left}: Shows world space obstacles as grey spheres. Robots start and goal configuration is colored red and green, respectively. Configurations along the computed path are colored transparent blue. \textbf{Right:} Mapped world space scenario to configuration space. Obstacle region is the grey mesh. Red spheres are collision-free regions computed by the neural SCDF. The optimized shortest path in the convex corridor is the blue curve.}
\vspace{-25pt}
\end{figure}
Motion planning is the problem of finding a collision-free trajectory that connects a given start and goal configuration. The planning takes place in the configuration space of the robot. For single body robots, like mobile robots or drones, the configuration space and the world space are usually the same. This simplifies the planning, since explicit obstacle representations are available which enables geometrical tools like separating hyperplanes, smallest distance to obstacles etc., to be used when designing motion planning algorithms. For multi-body robots like manipulators, the situation is completely different. The world space obstacles are usually mapped to non-convex regions, and to make the problem even harder, the mapping is usually not known. Forming explicit representations of the obstacle region in the configuration space is usually too expensive or intractable. Despite all of this, sampling based planners are used with great success, which mainly is due to their use of implicit representations of the obstacle region. The basic idea is to construct a graph in the configuration space that covers and connects the collision-free region. From this graph, a path can be extracted that connects a given start and goal configuration. The approach is computationally expensive, since the graph is constructed with the smallest geometrical building block available, points, which represents a collision-check. Furthermore, the extracted paths from the graph are non-smooth and jagged due to the stochastic nature of the approach. This adds an additional post-processing step to the process, where the paths are shortcutted and smoothened, before the path can be used for tracking. Clearly a lot of time is invested to form this graph and produce smooth paths. Thus, if the obstacles start to move, then all of this work is done in no use, since all points that make up this graph need to be re-verified, which is simply too time consuming to be done in real time.
\\\\
In this work, we want to address the existing drawbacks of the sampling based planners. Our main contribution is an improved motion planner where each vertex in the graph covers a collision-free region in the form of a sphere instead of a point and where the edges are formed with neighboring intersecting spheres. This representation has the advantage of instead of returning piecewise linear paths, returning a sequence of overlapping spheres, i.e. a convex corridor, that connects a given start and goal configuration, illustrated in Figure \ref{fig:pbrm_intro}. This convex corridor allows us to use convex optimization to produce smooth trajectories, instead of computationally expensive post-processing methods. The representation further allows us to estimate the coverage of the collision-free space, which gives us awareness and feedback in the offline roadmap construction phase. Finally, our representation is simple to adapt to moving obstacles, simply requery for the new radii and recheck for intersections. 
\\\\
The spherical collision-free regions are formed using a signed distance function (SDF), which is a function that returns the smallest distance from an arbitrary point to the boundary of an obstacle. As the name implies, the distance is signed, thus if the point is inside the obstacle it is negative otherwise positive. If the distance is positive, a sphere with radius equal to the distance is guaranteed to cover a collision-free region. Using an SDF in motion planning is not new, but what is novel about our approach is that we express the distance in the configuration space instead of the world space and by doing so allows us to form these convex collision-free regions. We refer to the resulting SDF as a signed configuration distance function (SCDF). Computing an SCDF analytically is non-trivial, our approach is therefore to parameterize the SCDF with a deep neural network and learn the mapping by supervised learning. Our resulting neural SCDF can compute distances for different parameter values of obstacle shapes and we also show how multiple distances can be combined, thus making our approach flexible.
\section{Related work}
Motion planning algorithms can roughly be divided into three families, grid-based, sampling based and optimization based methods. Grid-based methods (GBM) discretize the planning space from which a graph is then compiled. A standard search method is A$^\star$ \citep{a_star}, which is classified as an \textit{informed} search method, since it employs a heuristic function to speed up the search. A$^\star$ guarantees to return an optimal path at the level of discretization used. GBMs usually discretize the planning space by a regular lattice and this limits the GBMs to problems with low dimensionality due to the curse of dimensionality. Thus, GBMs are usually limited to single-body robots where the degrees of freedom (DOF) are low. To overcome the inherent scaling problem with the GBMs, stochastic methods are usually used for multi-body robots. These methods are termed as sampling-based methods (SBM) and core members within this family are the rapidly-exploring random trees (RRT) \citep{rrt} and the probabilistic roadmap (PRM) \citep{prm}. RRT grows a tree from the start configuration and explores the collision-free region in a rapid way until it is able to connect to the goal region. RRT is usually improved by bi-directional planning \citep{rrt_connect}, i.e. an additional tree is grown from the goal configuration and the trees are tested for connection after any tree has been expanded. RRT is a single-query method, thus it searches for a path from scratch each time it is queried. Contrary to this, PRM is a multi-query method, which solves for multiple queries without starting from scratch. PRM does this by creating a roadmap (graph) that covers the collision-free space as an offline step. The graph is then used to solve for multiple queries. PRMs are used in cases where the environment does not change since the extra offline step is too computationally costly and needs to be re-done if the environment is changed. In our work, we address this inherent issue by using a different roadmap representation. Our vertices in the graph cover a collision-free region in the form of spheres and we form the edges by checking for intersecting spheres. If something in the environment changes, we recompute the spheres radii and recheck the intersections, without relying on collision detection. We use a trained neural network to compute the sphere radius, therefore querying for the radius can be done fast, hence our representation enables the PRM for dynamic environments.
\\\\
In the recent decades, optimization based methods (OBM) \citep{chomp, schulman, itomp, stomp} have been introduced as an alternative to SBM for multi-body robots. Like the SBM, the OBMs scale well to higher dimensional problems and produce smoother motion. It is common to use a SDF in the optimization since it is a smooth function, thus enabling gradient-based methods. However, the standard way of expressing the SDF is in world space. The distance therefore needs to be mapped to the configuration space by the forward kinematics. This mapping makes the optimization problem a non-linear program (NLP), which is computationally expensive to solve. Recently, a different approach has been proposed. In \cite{mp_gcs} motion planning is formulated as a convex optimization problem by using the graph of convex sets framework \citep{gcs}. The underlying idea is to decompose the collision-free space into intersecting convex sets from which a convex optimization problem is formulated. In cases where an explicit representation of the obstacles in the configuration space exists, like for single-body robots, creating collision-free convex regions can be done fast \citep{iris}. For multi-body robots, this is non-trivial. Existing work does this successfully \citep{iris_nlp, iris_c} by an optimization based approach, but the methods are still too time consuming to be used in the presence of moving obstacles. Our approach is instead to use deep learning to learn an SDF expressed in the configuration space. With this, we can query for shortest distances to the collision boundary, which allows us to expand spherical regions which are collision-free. Our approach is fast and therefore enables our suggested roadmap planner to be used in dynamic environments.
\\\\
Recent research has focused on learning collision detection \citep{fk_kernel_distance, diffco, graphdistnet} by predicting the signed distance between the robot links and the surrounding obstacles in the world space. The learned SDF is used in trajectory optimization but since the distance is expressed in the world space, the problem becomes an NLP and therefore takes a long time to solve. We take a novel approach and suggest to instead express the signed distance in the configuration space. This allows us to improve the PRM at the same time as it enables convex optimization for trajectory optimization, which runs faster and is more reliable than NLP solvers. In \cite{cspf} a learned signed distance function in the configuration space is proposed similar to our approach. However, their approach is restricted to point cloud representations, while we propose to represent the obstacles as parameterized geometric shapes, e.g. spheres. Furthermore, we also show how to use our learned SCDF to improve an existing roadmap planner.
\section{Problem formulation}
A robot is located in the world space, $\W \subset \R^3 $. The unique location of the robot is given by its configuration $\q \in \C$, where $\C$ is the configuration space. The set of points covered by the robots bodies at a certain configuration is expressed as $\B(\q) \subset \W$. The robot is surrounded by $\NrObst$ obstacles $\O = \bigcup_{i=1}^{\NrObst} \O_i$, where  $\O_i \subset \W$. The representation of the obstacle in the configuration space is the set $\C\O_i = \{\q \in \C \: |\: \B(\q) \cap \O_i \neq \emptyset \}$. The obstacle space is formed as $\Co = \bigcup_{i=1}^{\NrObst} \C \O_i$. The complement is referred to as the free space, $\Cf = \C \setminus \Co$. The path planning problem is a tuple, ($\Cf$, $\qStart$, $\qGoal$), where we want to connect a query pair, consisting of a start, $\qStart$, and goal configuration, $\qGoal$, with a geometric path, $\q(s): [0, 1] \mapsto \Cf$, such that $\q(0)=\qStart$ and $\q(1)=\qGoal$, or report correctly when such a path does not exist.
\end{document}
    % Introduction section
% Literature Review section
\section{Background}

The exponential growth of data, driven by the IoT, social media, cloud computing, and digital transformation, has led to a significant increase in the volume of data that organizations need to manage. Traditional data storage models, such as \textit{data lakes} and \textit{data warehouses}, provide scalable storage but are increasingly insufficient for managing the dynamic flow and real-time processing of data \cite{galliers2014strategic,hsu2015data,kleppmann2017designing}.

Several critical challenges in modern data management have been identified:
\begin{itemize}
    \item \textbf{Data overflow}: Systems often struggle to process or store data quickly enough, leading to information loss or delayed processing \cite{tabesh2019implementing,zhang2015memory}.
    \item \textbf{Latency issues}: Big data systems frequently experience delays when handling large data volumes, particularly when inflow and outflow rates are imbalanced \cite{clapp2015quantifying,tian2015latency}.
    \item \textbf{Inefficient access and redundancy}: The unstructured nature of data lakes leads to inefficiencies and difficulties in retrieving relevant data when needed, as redundant data copies may coexist across different systems \cite{gupta2018practical,azzabi2024data}.
\end{itemize}

Despite attempts by cloud storage systems (e.g., AWS, Azure) and distributed computing frameworks (e.g., Hadoop, Spark) to address these issues, most systems focus primarily on storage and retrieval, with limited mechanisms for regulating dynamic data flow \cite{hashem2015rise, khalid2021comparative,barika2019orchestrating}. The need for real-time processing and adaptive control continues to present a significant challenge \cite{nambiar2022overview}.

\subsection{Existing Data Management Models}

\subsubsection{Data Lakes and Data Warehouses}

Data lakes are popular solutions for storing vast amounts of unstructured and semi-structured data. However, their unstructured nature makes it difficult to control data access efficiently, leading to latency in processing and delays in real-time analytics \cite{nambiar2022overview}. While data warehouses offer more structure and are more suited for analytical queries, they still lack mechanisms to regulate dynamic data inflow and outflow \cite{bai2023data}.

\subsubsection{Stream Processing and Real-time Analytics}

Platforms such as \textit{Apache Kafka}, \textit{Apache Flink}, and \textit{Apache Spark Streaming} enable continuous ingestion and real-time data processing. While these platforms excel in specific real-time tasks, they do not inherently provide a mechanism for regulating the flow of large-scale distributed data. Instead, they rely on developers to implement custom control mechanisms, which increases system complexity \cite{fernandes2020big,bai2023data,saxena2017practical}.

\subsection{Control Theory and Queuing Models}

Control theory has been applied successfully in network flow control and data packet management, dynamically adjusting data transfer rates to prevent congestion. For example, in telecommunications, protocols such as \textit{TCP/IP} adjust data transfer rates to prevent overloading the network \cite{collis2004issues}.

Similarly, \textit{queuing theory} has been applied to model data arrival and service times, helping optimize throughput and minimize latency. The \textit{M/M/1 queue model}, which assumes a single server and exponential inter-arrival and service times, has been widely used to optimize data processing in cloud systems \cite{guo2014dynamic}. However, such models are not widely applied to big data storage systems, where the challenge is to manage large, unpredictable data streams across distributed systems.

\subsection{Emerging Trends and Research Gaps}

Despite advances in big data frameworks, a gap remains in controlling data flow between systems. Most platforms rely on static retrieval mechanisms or ad hoc stream processing techniques, which are inefficient during peak loads and fail to prevent overflow or data loss. Studies exploring data governance and access control focus mainly on security, leaving out proactive flow regulation \cite{georgiadis2021enterprise}.

\textit{Edge computing} has emerged as a solution to address latency and overflow by processing data closer to the source. By reducing the amount of data transferred to centralized data centers, edge computing alleviates some of the strain on cloud systems. However, regulating data flow across distributed nodes remains an unresolved challenge. While edge computing reduces response times, a mechanism is still required to dynamically regulate data transmission based on system load and bandwidth availability \cite{ullah2018information}.


As data becomes more sensitive, especially in fields like healthcare and finance, there is an increasing need for security mechanisms that control the flow of sensitive information. Most current frameworks handle security through encryption and access control but do not provide proactive regulation of sensitive data flow \cite{yang2020data,raparthi2021privacy,josphineleela2023big}.

\subsection{Positioning the Data Dam Framework}

The \textit{Data Dam} framework proposed in this paper directly addresses the shortcomings of existing data management systems by providing a dynamic control mechanism for managing data inflow, storage, and outflow. Drawing from concepts in control theory, queuing models, and real-time analytics, Data Dams offer a unified framework that balances system load, minimizes overflow, and enhances data security.

This framework introduces a proactive approach to managing data flows, which is absent in traditional data lakes, warehouses, and real-time streaming platforms. By regulating data flows through dynamic control mechanisms—akin to how physical dams regulate water flow—Data Dams ensure that data is stored, processed, and transmitted efficiently without overwhelming system resources.The integration of \textit{machine learning} algorithms to predict and adjust flow patterns offers a significant advantage over static control methods.

While current data management models provide robust storage and real-time processing capabilities, they lack the dynamic flow regulation necessary to optimize large-scale data environments. The \textit{Data Dam} model fills this gap by offering a novel mechanism that dynamically adjusts data flow based on system capacity, security needs, and processing demands. This review highlights the need for such a framework and demonstrates how Data Dams can revolutionize data flow management in distributed systems.

              % Literature Review section
\section{Method}

\subsection{Overview \& Setup}

Our framework consists of a large, highly capable model \textbf{\bigmodel} and a smaller, resource-efficient model \textbf{\smallmodel}. We assume that $S \in \mathbb{N}$ and $L \in \mathbb{N}$ represent the parameter count of each model with $S \ll L$. Both models can either function as classifiers (i.e., $\mathcal{M}: \mathbb{R}^D \rightarrow [C]$ with $\mathbb{R}^D$ denoting the input space and $C$ the number of total classes), or (multi-modal) sequence models (i.e., $\mathcal{M}: \mathbb{R}^D \rightarrow [V]^{T}$ where $V$ is the vocabulary and $T$ is the sequence length). We include experiments on all of these model classes in Section~\ref{sec:experiments}. Furthermore, we do not require a shared model family to be deployed on both \smallmodel and \bigmodel; for example, \smallmodel could be a custom convolutional neural network optimized for efficient inference and \bigmodel a vision transformer~\citep{dosovitskiy2020image}. The primary objective is to design a deferral mechanism that enables \smallmodel to decide when to return its predictions without the assistance of \bigmodel and when to instead defer to it.

\looseness=-1
Deferral decisions are made using signals derived from the small model \smallmodel as this approach is typically more cost-effective than employing a separate routing mechanism~\citep{teerapittayanon2016branchynet}. Approaches that involve querying the large model \bigmodel to assist in making deferral decisions at test time are excluded from our setup. Such methods --- common in domains like LLMs --- are counterproductive to our goal since querying \bigmodel defeats the purpose of making a deferral decision in the first place?. Examples of these inapplicable methods include collaborative LLM frameworks~\citep{mielke2022reducing} and techniques that rely on semantic entropy for uncertainty estimation~\citep{kuhn2023semantic}. As part of our setup, we assume that \smallmodel is strictly less capable than \bigmodel --- a realistic scenario in practice supported by scaling laws~\citep{kaplan2020scaling}. Under this assumption, mistakes made by \bigmodel are also made by \smallmodel; however, \smallmodel may make additional errors that \bigmodel would avoid. This reflects the general observation that larger models tend to outperform smaller models across a wide range of tasks.

As discussed in Section~\ref{sec:related-word}, the choice of deferral strategy often depends on the level of access available to \smallmodel. We assume white box access with full access to \smallmodel's internals. As such, deferral mechanisms can be directly integrated into the model's architecture and parameters. This involves fine-tuning \smallmodel to predict deferral decisions or to incorporate rejection mechanisms within its predictive process. Our work falls into this category as it proposes a new loss function to fine-tune \smallmodel. 

Our goal is to train a small model that can effectively distinguish between correct and incorrect predictions. While many past works have considered the question of whether it is possible to find proxy measures for prediction correctness, the central question we ask is:
\begin{center}
\textbf{Can we \emph{optimize} the small model \smallmodel to separate correct from incorrect predictions?}
\end{center}
\noindent We show that this is indeed achievable through a carefully designed fine-tuning stage that does not require any architectural modifications. This ensures that the ability to separate correct from incorrect decisions is integrated seamlessly into \smallmodel's existing structure.


\subsection{Confidence-Tuning for Deferral}

\begin{figure}
    \centering
    \resizebox{\linewidth}{!}{
    \begin{figure}[h]
\begin{center}
   \includegraphics[width=0.99\linewidth]{figs/pdf/loss.pdf}
\end{center}
   \caption{
    Training loss of VAR \textit{vs.} FlexVAR. FlexVAR demonstrates a faster convergence rate. We report the results with trained 40 epochs ($\sim$ 70K iterations). 
   }
\label{fig:loss}
\end{figure}

    }
    \vspace{-15pt}
    \caption{\textbf{Overview of \loss}: We want correctly predicted samples maintain their current prediction by ensuring that cross entropy is decreased (top, green). At the same time, we want incorrectly predicted samples to yield a uniform confidence across all classes, leading to a low overall confidence score (bottom, red).}
    \label{fig:opt_goal}
\end{figure}

\textbf{Stage 1: Standard Training.} We begin with a \smallmodel that has already been trained on the tasks it is intended to perform upon deployment. However, due to its limited capacity, \smallmodel cannot achieve the performance levels of \bigmodel. Importantly, we make no assumptions about the training process of \smallmodel—whether it was trained from scratch without supervision from an external model or through a distillation approach.

\sloppy
\textbf{Stage 2: Correctness-Aware Finetuning with \loss.} Next, we introduce a correctness-aware loss, dubbed \loss, to fine-tune \smallmodel for improved confidence calibration. Specifically, the model is trained to make correct predictions with high confidence while reducing the confidence of incorrect predictions (see Figure~\ref{fig:opt_goal}). This loss can either rely on true labels or utilize the outputs of \bigmodel with soft probabilities as targets. 


For a standard classification model, the calibration loss is defined as the following hybrid loss
\begin{align}
\mathcal{L} &= \alpha \mathcal{L}_\text{corr} + (1 - \alpha) \mathcal{L}_\text{incorr} \\
\mathcal{L}_\text{corr} &= \frac{1}{N} \sum_{i=1}^{N} \mathds{1}\{ y_i = \hat{y}_i \} \text{CE}(p_i(\mathbf{x}_i), y_i) \\
\mathcal{L}_\text{incorr} &= \frac{1}{N} \sum_{i=1}^{N} \mathds{1}\{ y_i \neq \hat{y}_i \} \text{KL}\left(p_i(\mathbf{x}_i) \parallel \mathcal{U}\right)
\end{align}
where  \( y_i \) and \( \hat{y}_i \) are the true and predicted labels for $\mathbf{x}_i$, respectively, \( p_i \) is the predicted probability distribution of \smallmodel over classes, \( \mathcal{U} \) represents the uniform distribution over all classes, \( N \) denotes the number samples in the current batch, \( \alpha \in (0, 1) \) is a tunable hyperparameter controlling the emphasis between correct and incorrect predictions, and the cross-entropy function and KL divergence are defined as \( \text{CE}(p, y) = -\sum_{c} y_c \log p_c \) and \( \text{KL}(p \parallel q) = \sum_{c} p_c \log ( \frac{p_c}{q_c}) \), respectively. We note that a similar loss has previously been proposed in Outlier Exposure (OE)~\citep{hendrycks2018deep} for out-of-distribution (OOD) sample detection. Here, the goal is to make sure that OOD examples are assigned low confidence scores by tuning the confidence on a auxiliary outlier dataset. However, to the best of our knowledge, this idea has not previously been used to improve deferral performance of a smaller model in a cascading chain.

We emphasize that the trade-off parameter $\alpha$ plays a critical role as part of this optimization setup as it directly influences model utility and deferral performance. A lower value of \(\alpha\) emphasizes reducing confidence in incorrect predictions by pushing them closer to the uniform distribution, making the model more cautious in regions where it may make mistakes. Conversely, a higher value of \(\alpha\) encourages the model to increase its confidence on correct predictions, sharpening its decision boundaries and enhancing accuracy where it is already performing well. Thus, \(\alpha\) serves as a crucial hyperparameter that balances the trade-off between improving calibration by mitigating overconfidence in errors and reinforcing confidence in accurate classifications. By appropriately tuning \(\alpha\), practitioners can control the model’s behavior to achieve a desired balance between reliability in uncertain regions and decisiveness in confident predictions, tailored to the specific requirements of their application.

We further generalize this loss to token-based models (e.g., LMs and VLMs), formulated as
\ifarxiv
\small
\fi
\begin{align}
    \mathcal{L}_\text{corr} & = \frac{1}{N} \sum_{i=1}^{N} \sum_{t=1}^{T} \mathds{1}\{ y_{i,t} = \hat{y}_{i,t} \} \text{CE}(p_{i,t}(\mathbf{x}_i), y_{i,t}) \\
    \mathcal{L}_\text{incorr} & = \frac{1}{N} \sum_{i=1}^{N} \sum_{t=1}^{T} \mathds{1}\{ y_{i,t} \neq \hat{y}_{i,t} \} \text{KL}\left(p_{i,t}(\mathbf{x}_i) \parallel \mathcal{U}\right)
\end{align}
\normalsize
where \( y_{i,t} \) and \( \hat{y}_{i,t} \) denote the true and predicted tokens at position \( t \) for sample \( i \), \( p_{i,t} \) is the predicted token distribution at position \( t \) for sample \( i \), and \( T \) is the sequence length for the token-based model. The token-level loss ensures that correct token predictions are made confidently while incorrect tokens are assigned smaller confidences.

\sloppy
\textbf{Stage 3: Confidence Computation \& Thresholding.} After fine-tuning \smallmodel with \loss, we apply standard confidence- and entropy-based techniques for model uncertainty to obtain a deferral signal. We use the selective prediction framework to determine whether a query point~$\mathbf{x} \in \mathbb{R}^D$ should be accepted by \smallmodel or routed to \bigmodel. Selective prediction alters the model inference stage by introducing a deferral state through a \textit{gating mechanism}~\citep{yaniv2010riskcoveragecurve}. At its core, this mechanism relies on a deferral function $g:\mathbb{R}^D \rightarrow \mathbb{R}$ which determines if \smallmodel should output a prediction for a sample~$\mathbf{x}$ or defer to \bigmodel. Given a targeted acceptance threshold $\tau$, the resulting predictive model can be summarized as:
\begin{equation}
\label{eq:deferral}
    (\mathcal{M}_S,\mathcal{M}_L,g)(\mathbf{x}) = \begin{cases}
  \mathcal{M}_S(\mathbf{x})  & g(\mathbf{x}) \geq \tau \\
  \mathcal{M}_L(\mathbf{x}) & \text{otherwise.}
\end{cases}
\end{equation}

\emph{Classification Models (Max Softmax).} Let \(\mathcal{M}_S\) produce a categorical distribution
\(\{p(y=c \mid \mathbf{x})\}_{c=1}^C\) over \(C\) classes. 
Then we define the gating function as
\begin{align}
g_{\text{CL}}(\mathbf{x}) \;=\; \max_{1 \,\le\, c \,\le\, C}\;p\bigl(y = c \,\big\vert\, \mathbf{x}\bigr).
\end{align}

\emph{Token-based Models (Negative Predictive Entropy).} 
Let \(\mathcal{M}_S\) produce a sequence of categorical distributions 
\(\{p(y_t = c \mid \mathbf{x})\}_{c=1}^C\) for each token index \(t \in T\). Then we define the gating function as
\begin{equation}
\footnotesize
g_{\text{NENT}}(\mathbf{x}) 
= \; \frac{1}{T} \sum_{t=1}^{T} \sum_{c=1}^{C} 
    p\bigl(y_t = c \,\big\vert\, \mathbf{x}\bigr)\,\log p\bigl(y_t = c \,\big\vert\, \mathbf{x}\bigr),
\end{equation}
where \(y_t \in [C]\) is the predicted token at time step \(t\), \(p(y_t=c \mid \mathbf{x})\) is the (conditional) probability of token \(k\) at step \(t\), and \(T\) is the total number of token positions for the sequence. Across both model classes, higher values of either $g_{\text{CL}}$ or $g_{\text{NENT}}$ indicate higher confidence in the predicted class or sequence generation, respectively.         % Methods section
%\begin{table}[ht!]
\centering
\caption{\textbf{Super Resolution Performance Results.} Our proposed WGAN EEG Spatial Upsampling method significantly outperforms a baseline of Bicubic Interpolation commonly used in EEG upsampling pipelines.}
\label{tab:results}
\resizebox{0.8\linewidth}{!}{%
\begin{tabular}{@{}cccccc@{}}
\toprule
\multirow{2}{*}{\textbf{Dataset}} & \multirow{2}{*}{\textbf{Scale}} & \multicolumn{2}{c}{\textbf{Bicubic}} & \multicolumn{2}{c}{\textbf{WGAN}} \\ \cmidrule(l){3-6} 
                      &   & \textbf{MSE} & \textbf{MAE} & \textbf{MSE}    & \textbf{MAE}   \\
\toprule
\multirow{2}{*}{Val}  & 2 & 3.71E7       & 3.89E3       & \textbf{2.01E3} & \textbf{24.38} \\
                      & 4 & 7.23E7       & 6.42E3       & \textbf{8.53E3} & \textbf{63.83} \\
\midrule
\multirow{2}{*}{Test} & 2 & 3.75E7       & 3.91E3       & \textbf{2.06E3} & \textbf{24.66} \\
                      & 4 & 7.30E7       & 6.45E3       & \textbf{8.68E3} & \textbf{64.39} \\
\bottomrule
\end{tabular}%
}
\end{table}         % Results section
This work identifies signal collapse as a critical bottleneck in one-shot neural network pruning. Performance loss in pruned networks is due to \textbf{signal collapse} in addition to the removal of critical parameters. We propose \textbf{REFLOW} (\textbf{Re}storing \textbf{F}low of \textbf{Low}-variance signals), a simple yet effective method that mitigates signal collapse without computationally expensive weight updates. By focusing on signal preservation, REFLOW highlights the importance of mitigating signal collapse in sparse networks and enables magnitude pruning to match or surpass state-of-the-art one-shot pruning methods such as CHITA, CBS, and WF.

REFLOW consistently achieves state-of-the-art accuracy across diverse architectures, restoring ResNeXt-101 from under 4.1\% to 78.9\% top-1 accuracy at 80\% sparsity on ImageNet. Its lightweight design makes it a practical solution for both research and deployment, delivering high-quality sparse models without the overhead of traditional approaches. These findings challenge the traditional emphasis on weight selection strategies and underscore the critical role of signal propagation for achieving high-quality sparse networks in the context of one-shot pruning.


      % Discussion section
\section*{Conclusion}
This paper aims to enhance our understanding of the computational complexity of computing various Shapley value variants. We found that for various ML models --- including decision trees, regression tree ensembles, weighted automata, and linear regression --- both local and global interventional and baseline SHAP can be computed in polynomial time under HMM modeled distributions. This extends popular algorithms, such as TreeSHAP, beyond their empirical distributional scope. We also establish strict complexity gaps between the various SHAP variants (baseline, interventional, and conditional) and prove the intractability of computing SHAP for tree ensembles and neural networks in simplified scenarios. Overall, we present SHAP as a versatile framework whose complexity depends on four key factors: \begin{inparaenum}[(i)] \item model type, \item SHAP variant, \item distribution modeling approach, \item and local vs. global explanations\end{inparaenum}. We believe this perspective provides deeper insight into the computational complexity of SHAP, paving the way for future work.




%We believe that our framework provides a more intricate understanding of SHAP computation complexity across different models, distributions, and variants, paving the way for further research.

Our work opens promising directions for future research. First, expanding our computational analysis to other SHAP-related metrics, such as asymmetric SHAP~\citep{frye20} and SAGE~\citep{covert2020understanding}, would be valuable. Additionally, we aim to explore more expressive distribution classes and relaxed assumptions beyond those in Section \ref{sec:tractable} while maintaining tractable SHAP computation. Finally, when exact computation is intractable (Section \ref{sec:intractable}), investigating the approximability of SHAP metrics through approximation and parameterized complexity theory~\citep{downey2012parameterized} is an important direction.

%Our work opens several promising avenues for future research on the computational properties of explainable AI methods, with a particular focus on SHAP. First, it would be interesting to broaden the computational analysis conducted in this work to include other popular SHAP-related metrics in the literature, such as asymmetric SHAP \cite{frye20} and SAGE \cite{covert2020understanding}. Also, in the future, we aim to explore more expressive distribution classes and relaxed distributional assumptions—extending beyond those examined in Section \ref{sec:tractable} —that still yield tractable SHAP computation. Finally, when exact computation proves intractable (Section \ref{sec:intractable}), it is worthwhile to theoretically investigate the question of the approximability of computing the SHAP metrics across various configurations, through the lens of approximation and parametrized complexity theory \cite{arora2009computational}.

%This paper aims to deepen our understanding of the computational complexity involved in obtaining different Shapley value variants. We found that for a variety of ML models, including decision trees, tree ensembles for regression, weighted automata, and linear regression models — computing both local and global interventional and baseline SHAP can be done in polynomial time when distributions are modeled by HMMs. This extends the distributional scope of popular algorithms like TreeSHAP, which is limited to empirical distributions. Additionally, we demonstrate a strict complexity gap between SHAP variants, showing that interventional and baseline SHAP can be strictly easier to compute than conditional SHAP. Despite these positive results, we uncovered intractability for various SHAP variants in neural networks and tree ensembles. Finally, we provided generalized complexity relations across SHAP variants. We believe that our framework offers a deeper understanding of the complexity involved in computing SHAP across various variants, models, distributions, as well as in both local and global computations, laying the groundwork for future research.      % Conclusion section

% Declaration and other sections
\sectopic{Data Availability. }
We have made the code and data used in this paper publicly available in an online annex~\cite{annex}.
% \sectopic{Declaration. }
\section*{Declaration}
\textbf{Conflict of Interest.} The authors declared that they have no conflict of interest.
% \newline
% \textbf{Ethics declaration:} not applicable before references.

% References Section
\printbibliography[heading=bibintoc, title={References}]

\end{document}
