% Abstract section
\begin{abstract}In the era of big data, managing dynamic data flows efficiently is crucial as traditional storage models struggle with real-time regulation and risk overflow. This paper introduces Data Dams, a novel framework designed to optimize data inflow, storage, and outflow by dynamically adjusting flow rates to prevent congestion while maximizing resource utilization. Inspired by physical dam mechanisms, the framework employs intelligent sluice controls and predictive analytics to regulate data flow based on system conditions such as bandwidth availability, processing capacity, and security constraints. Simulation results demonstrate that the Data Dam significantly reduces average storage levels (371.68 vs. 426.27 units) and increases total outflow (7999.99 vs. 7748.76 units) compared to static baseline models. By ensuring stable and adaptive outflow rates under fluctuating data loads, this approach enhances system efficiency, mitigates overflow risks, and outperforms existing static flow control strategies. The proposed framework presents a scalable solution for dynamic data management in large-scale distributed systems, paving the way for more resilient and efficient real-time processing architectures.\par
\noindent\textbf{Keywords:} \keywords 
    % Placeholder for abstract. Replace with your abstract.
\end{abstract}
