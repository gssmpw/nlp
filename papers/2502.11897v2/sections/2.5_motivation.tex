% 视频本身有不同片段的信息不同的特性,我们可以观察到,不同的视频以及同一视频的不同片段中,内容变化的特性可能存在显著差异。现有研究[1-3]表明,视频内容的信息密度呈现显著的非均匀性特征。如图2所示,在体育赛事视频中,高速运动片段(如足球射门)的帧间差异可达低速片段(球员站立)的5-8倍;在自然纪录片中,动物捕食动作的时序复杂度比静态景观片段高出60%以上。问题是,在原始视频被压缩后的latent空间,是否也存在这样的?

% 从Nyquist-Shannon sampling theorem角度分析Latent空间的frame

% 我们对不同视频进行了分析,发现了VAE Encode转换后的latent空间,不同视频片段信号的 f_max 差异很大,这证明了在latent空间也存在降低帧率的可能性。

% 在视频处理和压缩领域,传统上对帧率的优化分析集中在原始视频信号的时间域特性上。视频的不同片段由于内容复杂度的不同,其时间信号的频率成分存在显著差异。现有研究[1-3]表明,视频内容的信息密度呈现显著的非均匀性特征。例如,在体育赛事视频中,高速运动片段(如足球射门)的帧间差异可达低速片段(如球员站立)的 5-8 倍;在自然纪录片中,动物捕食动作的时序复杂度比静态景观片段高出 60% 以上(见图 2)。这表明,对于原始视频信号,不同片段的时间频率分布决定了帧率的需求。

% 然而,随着深度学习模型在视频生成中的广泛应用,视频信号被映射到一个潜在空间(Latent Space)中。潜在空间中信号的时间特性如何分布?是否也呈现类似于原始视频信号的频率变化?是否可以在潜在空间中动态调整帧率?这些问题尚未被深入研究。

% 原始视频信号的时间频率分析
% 原始视频信号可以表示为随时间变化的连续函数 x(t),其频谱表示为 X(f)。对于一个带限信号,其频谱的最高频率为 f_max,即:

% 采样过程以帧率 F_s 为采样频率,将连续信号离散化得到离散帧信号:

% 根据 Nyquist-Shannon 采样定理,当帧率满足以下条件时:

% 采样信号可以完整重建原始信号,无混叠现象。对于原始视频,不同片段的 f_max 差异性决定了其帧率需求的可变性。

% 潜在空间信号的时间频率特性
% 设原始视频信号为连续时间函数 I(t),通过编码器 E 映射到Latent空间,得到隐变量序列:

% 由于 E 的复杂性,信号在潜在空间中的频谱分布 Z(f) 可能与原始空间 X(f) 的分布存在显著差异。理论上,潜在空间信号的频率分布可以表示为:

% 如果 f_enc() 是平滑的非线性变换,其频率响应会在某些范围内压缩或放大原始信号的频率成分。例如,某些高频分量可能被编码器过滤掉,而低频分量可能被保留甚至增强。因此,潜在空间信号的最高频率 f_max' 通常满足:

% g(·) 是由编码器频率响应决定的函数。通过对多个视频片段的分析(图8),我们发现潜在空间信号的频率分布依然呈现显著的非均匀性。高速运动视频片段(如快速摄像机移动或物体快速运动)的潜在信号通常包含更高的时间频率分量,而静态或低速运动片段的潜在信号则集中在较低频率范围内。这表明潜在空间中的时间信号仍然具有潜在的可变帧率优化空间。

% 潜在空间中可变帧率的理论分析
% 根据 Nyquist-Shannon 采样定理,潜在空间中信号的帧率需求由其最高时间频率 f_max' 决定。为了在潜在空间中实现可变帧率,我们需要对每个片段的信号频率进行估计,并动态调整帧率 F_s',满足以下条件:

% 在潜在空间中,根据每个片段的 f_max',动态调整采样帧率 F_s'。假设潜在空间信号的时序复杂度可以分为 N 个类别,每个类别的最高频率分别为 f_{\text{max},1}', f_{\text{max},2}', \dots, f_{\text{max},N}',则帧率分配策略为:

\section{Motivation}

In the field of video processing and compression, traditional optimization of frame rates has primarily focused on the temporal characteristics of raw video signals. Different segments of a video exhibit significant variations in the frequency components of their temporal signals due to differences in content complexity. Existing studies [1-3] have demonstrated that the information density of video content exhibits pronounced non-uniformity. For instance, in sports videos, the inter-frame difference in high-motion segments (e.g., a soccer goal shot) can be 5-8 times greater than that in low-motion segments (e.g., players standing still). Similarly, in nature documentaries, the temporal complexity of animal hunting actions is over 60\% higher than that of static landscape scenes (see Figure 2). These observations indicate that the temporal frequency distribution of different segments in raw video signals determines the required frame rates.

However, with the widespread adoption of deep learning models in video generation, video signals are now often mapped into a latent space. This raises several critical questions: How are the temporal characteristics of signals distributed in the latent space? Do they exhibit frequency variations similar to those of raw video signals? Is it possible to dynamically adjust frame rates in the latent space? These questions remain largely unexplored in the literature.

\subsection{Temporal Frequency Analysis of Video Signals}

Original video signals can be represented as continuous functions of time, denoted as \( x(t) \). The frequency spectrum of \( x(t) \) can be expressed as \( X(f) \) using the Fourier transform. For a band-limited signal, the frequency spectrum \( X(f) \) is nonzero only within a finite range. Specifically, the highest frequency component of the signal is denoted as \( f_{\text{max}} \), which satisfies:

\[
X(f) = 0, \quad \forall |f| > f_{\text{max}}.
\]

The sampling process is performed at a frame rate \( F_s \), which acts as the sampling frequency. The continuous signal \( x(t) \) is thus discretized into a sequence of frames:

\[
x[n] = x(nT), \quad T = \frac{1}{F_s},
\]

where \( T \) is the sampling interval. According to the Nyquist-Shannon sampling theorem, the sampling frequency \( F_s \) must satisfy the following condition to ensure the accurate reconstruction of the original signal without aliasing:

\[
F_s \geq 2 f_{\text{max}}.
\]

This implies that the frame rate must be at least twice the highest frequency component of the signal. For original video signals, the temporal frequency \( f_{\text{max}} \) varies across different segments of the video. This variation determines the frame rate requirements for different video segments, providing the basis for adaptive frame rate optimization.

\subsection{Temporal Frequency Analysis of Latent Space}

Let the original video signal be represented as a continuous-time function \( x(t) \). Through an encoder \( \mathcal{E} \), the signal is mapped into a latent space, resulting in a sequence of latent variables \( z(t) \):

\[
z(t) =\mathcal{E}(x(t)).
\]

Due to the complexity of the encoder \( \mathcal{E} \), the frequency spectrum of the signal in the latent space, denoted as \( Z(f) \), may differ significantly from the frequency spectrum of the original signal \( X(f) \). Theoretically, the frequency distribution in the latent space can be expressed as:

\[
Z(f) = \int_{-\infty}^{\infty} z(t) e^{-j 2 \pi f t} \, dt.
\]

If the encoder \( E \) (or equivalently, \( f_{\text{enc}} \)) is a smooth nonlinear transformation, its frequency response may compress or amplify certain frequency components of the original signal. For instance, high-frequency components may be filtered out by the encoder, while low-frequency components may be preserved or enhanced. Consequently, the highest frequency in the latent space, \( f_{\text{max}}' \), typically satisfies:

\[
f_{\text{max}}' \leq g(f_{\text{max}}),
\]

where \( g(\cdot) \) is a function determined by the frequency response of the encoder.

Through the analysis of multiple video segments (as illustrated in Figure 8), we observe that the frequency distribution of latent space signals exhibits significant non-uniformity. For example:
- High-speed video segments, such as rapid camera movements or fast-moving objects, often contain higher temporal frequency components in the latent space.
- Static or low-speed segments, on the other hand, exhibit frequency components concentrated in the lower range.

This indicates that the temporal signals in the latent space retain the potential for frame rate optimization, similar to the original video signal.

% \subsection{Theoretical Analysis of Variable Frame Rates in Latent Space}

According to the Nyquist-Shannon sampling theorem, the frame rate \( F_s' \) required in the latent space is determined by the highest temporal frequency \( f_{\text{max}}' \) of the latent signal. To implement variable frame rates in the latent space, it is necessary to estimate the temporal frequency of each segment and dynamically adjust \( F_s' \) to satisfy the following condition:

\[
F_s' \geq 2 f_{\text{max}}'.
\]

% \subsection{Dynamic Frame Rate Adjustment in Latent Space}

In the latent space, the frame rate \( F_s' \) can be dynamically adjusted based on the maximum temporal frequency \( f_{\text{max}}' \) of each video segment. 

% Suppose the temporal complexity of latent signals can be categorized into \( N \) classes, where each class has a corresponding maximum frequency \( f_{\text{max},1}', f_{\text{max},2}', \dots, f_{\text{max},N}' \). The frame rate allocation can then be expressed as:

% \[
% F_s' = 
% \begin{cases} 
% 2 f_{\text{max},1}' & \text{if the segment belongs to class 1}, \\
% 2 f_{\text{max},2}' & \text{if the segment belongs to class 2}, \\
% \vdots & \vdots \\
% 2 f_{\text{max},N}' & \text{if the segment belongs to class \( N \)}.
% \end{cases}
% \]

% This adaptive strategy ensures that the latent space signals are sampled without loss of information while minimizing the required frame rate.

% \subsection{Potential Benefits of Latent Space Optimization}

% By implementing variable frame rates in the latent space, significant computational and storage savings can be achieved while maintaining signal fidelity. Our experiments demonstrate that:
% 1. The latent space signals of high-speed video segments exhibit \( f_{\text{max}}' \) values that are 3-5 times higher than those of static segments.
% 2. Dynamically adjusting \( F_s' \) according to \( f_{\text{max}}' \) reduces the total frame count by 40\%-60\% without noticeable loss of quality.

% These findings highlight the feasibility and potential advantages of frame rate optimization in the latent space, paving the way for more efficient video compression and generation pipelines.
