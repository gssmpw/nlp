%%%%%%%% ICML 2025 EXAMPLE LATEX SUBMISSION FILE %%%%%%%%%%%%%%%%%

\documentclass{article}

% Recommended, but optional, packages for figures and better typesetting:
\usepackage{microtype}
\usepackage[table,dvipsnames]{xcolor}
\usepackage{colortbl}
\usepackage{graphicx}
\usepackage{subcaption}
% \usepackage{subfigure}
\usepackage{booktabs} % for professional tables
\usepackage{amsmath,amssymb} % define this before the line numbering.
\usepackage{color}

\usepackage{booktabs}
\usepackage{multirow}

% hyperref makes hyperlinks in the resulting PDF.
% If your build breaks (sometimes temporarily if a hyperlink spans a page)
% please comment out the following usepackage line and replace
% \usepackage{icml2025} with \usepackage[nohyperref]{icml2025} above.
\usepackage{hyperref}


% Attempt to make hyperref and algorithmic work together better:
\newcommand{\theHalgorithm}{\arabic{algorithm}}

% Use the following line for the initial blind version submitted for review:
% \usepackage{icml2025}

% If accepted, instead use the following line for the camera-ready submission:
\usepackage[accepted]{icml2025}

% For theorems and such
\usepackage{amsmath}
\usepackage{amssymb}
\usepackage{mathtools}
\usepackage{amsthm}

\usepackage{pifont}% http://ctan.org/pkg/pifont
\usepackage{enumitem}
\let\oldding\ding% Store old \ding in \oldding
\renewcommand{\ding}[2][1]{\scalebox{#1}{\oldding{#2}}}

% if you use cleveref..
\usepackage[capitalize,noabbrev]{cleveref}

%%%%%%%%%%%%%%%%%%%%%%%%%%%%%%%%
% THEOREMS
%%%%%%%%%%%%%%%%%%%%%%%%%%%%%%%%
\theoremstyle{plain}
\newtheorem{theorem}{Theorem}[section]
\newtheorem{proposition}[theorem]{Proposition}
\newtheorem{lemma}[theorem]{Lemma}
\newtheorem{corollary}[theorem]{Corollary}
\theoremstyle{definition}
\newtheorem{definition}[theorem]{Definition}
\newtheorem{assumption}[theorem]{Assumption}
\theoremstyle{remark}
\newtheorem{remark}[theorem]{Remark}

% Todonotes is useful during development; simply uncomment the next line
%    and comment out the line below the next line to turn off comments
%\usepackage[disable,textsize=tiny]{todonotes}
\usepackage[textsize=tiny]{todonotes}

\newcommand{\dlfr}[0]{\texttt{DLFR-VAE}}
\newcommand{\syz}[1]{\textcolor{blue}{SYZ:}}

% The \icmltitle you define below is probably too long as a header.
% Therefore, a short form for the running title is supplied here:
\icmltitlerunning{\dlfr: Dynamic Latent Frame Rate VAE for Video Generation}

\begin{document}

\twocolumn[
\icmltitle{\dlfr: Dynamic Latent Frame Rate VAE for Video Generation}

% It is OKAY to include author information, even for blind
% submissions: the style file will automatically remove it for you
% unless you've provided the [accepted] option to the icml2025
% package.

% List of affiliations: The first argument should be a (short)
% identifier you will use later to specify author affiliations
% Academic affiliations should list Department, University, City, Region, Country
% Industry affiliations should list Company, City, Region, Country

% You can specify symbols, otherwise they are numbered in order.
% Ideally, you should not use this facility. Affiliations will be numbered
% in order of appearance and this is the preferred way.
\icmlsetsymbol{equal}{*}

\begin{icmlauthorlist}
\icmlauthor{Zhihang Yuan}{equal,yyy,comp}
\icmlauthor{Siyuan Wang}{equal,yyy}
\icmlauthor{Rui Xie}{yyy,comp}
\icmlauthor{Hanling Zhang}{yyy,comp}
\icmlauthor{Tongcheng Fang}{yyy,comp}
\icmlauthor{Yuzhang Shang}{iit}
\icmlauthor{Shengen Yan}{comp}
%\icmlauthor{}{sch}
\icmlauthor{Guohao Dai}{shjt,comp}
\icmlauthor{Yu Wang}{yyy}
%\icmlauthor{}{sch}
%\icmlauthor{}{sch}
\end{icmlauthorlist}

\icmlaffiliation{yyy}{Tsinghua University}
\icmlaffiliation{comp}{Infinigence AI}
\icmlaffiliation{iit}{Illinois Tech}
\icmlaffiliation{shjt}{Shanghai Jiao Tong University}

\icmlcorrespondingauthor{Yu Wang}{yu-wang@tsinghua.edu.cn}
\icmlcorrespondingauthor{Yuzhang Shang}{yshang4@hawk.iit.edu}

% You may provide any keywords that you
% find helpful for describing your paper; these are used to populate
% the "keywords" metadata in the PDF but will not be shown in the document
\icmlkeywords{Machine Learning, ICML}

\vskip 0.3in
]

% this must go after the closing bracket ] following \twocolumn[ ...

% This command actually creates the footnote in the first column
% listing the affiliations and the copyright notice.
% The command takes one argument, which is text to display at the start of the footnote.
% The \icmlEqualContribution command is standard text for equal contribution.
% Remove it (just {}) if you do not need this facility.

%\printAffiliationsAndNotice{}  % leave blank if no need to mention equal contribution
% \printAffiliationsAndNotice{\icmlEqualContribution} % otherwise use the standard text.
% \icmlEqualContribution

\renewcommand{\thefootnote}{\fnsymbol{footnote}}
\footnotetext[1]{Equal contribution. $^1$Tsinghua University $^2$Infinigence AI
 $^3$Illinois Tech $^4$Shanghai Jiao Tong University.}
\renewcommand{\thefootnote}{\arabic{footnote}}


\begin{abstract}
Building a virtual cell capable of accurately simulating cellular behaviors in silico has long been a dream in computational biology. We introduce \emph{CellFlow}, an image-generative model that simulates cellular morphology changes induced by chemical and genetic perturbations using flow matching. Unlike prior methods, \emph{CellFlow} models distribution-wise transformations from unperturbed to perturbed cell states, effectively distinguishing actual perturbation effects from experimental artifacts such as batch effects—a major challenge in biological data. Evaluated on chemical (BBBC021), genetic (RxRx1), and combined perturbation (JUMP) datasets, \emph{CellFlow} generates biologically meaningful cell images that faithfully capture perturbation-specific morphological changes, achieving a 35\% improvement in FID scores and a 12\% increase in mode-of-action prediction accuracy over existing methods. Additionally, \emph{CellFlow} enables continuous interpolation between cellular states, providing a potential tool for studying perturbation dynamics. These capabilities mark a significant step toward realizing virtual cell modeling for biomedical research.
\end{abstract}


\section{Introduction}

Video generation has garnered significant attention owing to its transformative potential across a wide range of applications, such media content creation~\citep{polyak2024movie}, advertising~\citep{zhang2024virbo,bacher2021advert}, video games~\citep{yang2024playable,valevski2024diffusion, oasis2024}, and world model simulators~\citep{ha2018world, videoworldsimulators2024, agarwal2025cosmos}. Benefiting from advanced generative algorithms~\citep{goodfellow2014generative, ho2020denoising, liu2023flow, lipman2023flow}, scalable model architectures~\citep{vaswani2017attention, peebles2023scalable}, vast amounts of internet-sourced data~\citep{chen2024panda, nan2024openvid, ju2024miradata}, and ongoing expansion of computing capabilities~\citep{nvidia2022h100, nvidia2023dgxgh200, nvidia2024h200nvl}, remarkable advancements have been achieved in the field of video generation~\citep{ho2022video, ho2022imagen, singer2023makeavideo, blattmann2023align, videoworldsimulators2024, kuaishou2024klingai, yang2024cogvideox, jin2024pyramidal, polyak2024movie, kong2024hunyuanvideo, ji2024prompt}.


In this work, we present \textbf{\ours}, a family of rectified flow~\citep{lipman2023flow, liu2023flow} transformer models designed for joint image and video generation, establishing a pathway toward industry-grade performance. This report centers on four key components: data curation, model architecture design, flow formulation, and training infrastructure optimization—each rigorously refined to meet the demands of high-quality, large-scale video generation.


\begin{figure}[ht]
    \centering
    \begin{subfigure}[b]{0.82\linewidth}
        \centering
        \includegraphics[width=\linewidth]{figures/t2i_1024.pdf}
        \caption{Text-to-Image Samples}\label{fig:main-demo-t2i}
    \end{subfigure}
    \vfill
    \begin{subfigure}[b]{0.82\linewidth}
        \centering
        \includegraphics[width=\linewidth]{figures/t2v_samples.pdf}
        \caption{Text-to-Video Samples}\label{fig:main-demo-t2v}
    \end{subfigure}
\caption{\textbf{Generated samples from \ours.} Key components are highlighted in \textcolor{red}{\textbf{RED}}.}\label{fig:main-demo}
\end{figure}


First, we present a comprehensive data processing pipeline designed to construct large-scale, high-quality image and video-text datasets. The pipeline integrates multiple advanced techniques, including video and image filtering based on aesthetic scores, OCR-driven content analysis, and subjective evaluations, to ensure exceptional visual and contextual quality. Furthermore, we employ multimodal large language models~(MLLMs)~\citep{yuan2025tarsier2} to generate dense and contextually aligned captions, which are subsequently refined using an additional large language model~(LLM)~\citep{yang2024qwen2} to enhance their accuracy, fluency, and descriptive richness. As a result, we have curated a robust training dataset comprising approximately 36M video-text pairs and 160M image-text pairs, which are proven sufficient for training industry-level generative models.

Secondly, we take a pioneering step by applying rectified flow formulation~\citep{lipman2023flow} for joint image and video generation, implemented through the \ours model family, which comprises Transformer architectures with 2B and 8B parameters. At its core, the \ours framework employs a 3D joint image-video variational autoencoder (VAE) to compress image and video inputs into a shared latent space, facilitating unified representation. This shared latent space is coupled with a full-attention~\citep{vaswani2017attention} mechanism, enabling seamless joint training of image and video. This architecture delivers high-quality, coherent outputs across both images and videos, establishing a unified framework for visual generation tasks.


Furthermore, to support the training of \ours at scale, we have developed a robust infrastructure tailored for large-scale model training. Our approach incorporates advanced parallelism strategies~\citep{jacobs2023deepspeed, pytorch_fsdp} to manage memory efficiently during long-context training. Additionally, we employ ByteCheckpoint~\citep{wan2024bytecheckpoint} for high-performance checkpointing and integrate fault-tolerant mechanisms from MegaScale~\citep{jiang2024megascale} to ensure stability and scalability across large GPU clusters. These optimizations enable \ours to handle the computational and data challenges of generative modeling with exceptional efficiency and reliability.


We evaluate \ours on both text-to-image and text-to-video benchmarks to highlight its competitive advantages. For text-to-image generation, \ours-T2I demonstrates strong performance across multiple benchmarks, including T2I-CompBench~\citep{huang2023t2i-compbench}, GenEval~\citep{ghosh2024geneval}, and DPG-Bench~\citep{hu2024ella_dbgbench}, excelling in both visual quality and text-image alignment. In text-to-video benchmarks, \ours-T2V achieves state-of-the-art performance on the UCF-101~\citep{ucf101} zero-shot generation task. Additionally, \ours-T2V attains an impressive score of \textbf{84.85} on VBench~\citep{huang2024vbench}, securing the top position on the leaderboard (as of 2025-01-25) and surpassing several leading commercial text-to-video models. Qualitative results, illustrated in \Cref{fig:main-demo}, further demonstrate the superior quality of the generated media samples. These findings underscore \ours's effectiveness in multi-modal generation and its potential as a high-performing solution for both research and commercial applications.

\section{Related Work}

\subsection{Large 3D Reconstruction Models}
Recently, generalized feed-forward models for 3D reconstruction from sparse input views have garnered considerable attention due to their applicability in heavily under-constrained scenarios. The Large Reconstruction Model (LRM)~\cite{hong2023lrm} uses a transformer-based encoder-decoder pipeline to infer a NeRF reconstruction from just a single image. Newer iterations have shifted the focus towards generating 3D Gaussian representations from four input images~\cite{tang2025lgm, xu2024grm, zhang2025gslrm, charatan2024pixelsplat, chen2025mvsplat, liu2025mvsgaussian}, showing remarkable novel view synthesis results. The paradigm of transformer-based sparse 3D reconstruction has also successfully been applied to lifting monocular videos to 4D~\cite{ren2024l4gm}. \\
Yet, none of the existing works in the domain have studied the use-case of inferring \textit{animatable} 3D representations from sparse input images, which is the focus of our work. To this end, we build on top of the Large Gaussian Reconstruction Model (GRM)~\cite{xu2024grm}.

\subsection{3D-aware Portrait Animation}
A different line of work focuses on animating portraits in a 3D-aware manner.
MegaPortraits~\cite{drobyshev2022megaportraits} builds a 3D Volume given a source and driving image, and renders the animated source actor via orthographic projection with subsequent 2D neural rendering.
3D morphable models (3DMMs)~\cite{blanz19993dmm} are extensively used to obtain more interpretable control over the portrait animation. For example, StyleRig~\cite{tewari2020stylerig} demonstrates how a 3DMM can be used to control the data generated from a pre-trained StyleGAN~\cite{karras2019stylegan} network. ROME~\cite{khakhulin2022rome} predicts vertex offsets and texture of a FLAME~\cite{li2017flame} mesh from the input image.
A TriPlane representation is inferred and animated via FLAME~\cite{li2017flame} in multiple methods like Portrait4D~\cite{deng2024portrait4d}, Portrait4D-v2~\cite{deng2024portrait4dv2}, and GPAvatar~\cite{chu2024gpavatar}.
Others, such as VOODOO 3D~\cite{tran2024voodoo3d} and VOODOO XP~\cite{tran2024voodooxp}, learn their own expression encoder to drive the source person in a more detailed manner. \\
All of the aforementioned methods require nothing more than a single image of a person to animate it. This allows them to train on large monocular video datasets to infer a very generic motion prior that even translates to paintings or cartoon characters. However, due to their task formulation, these methods mostly focus on image synthesis from a frontal camera, often trading 3D consistency for better image quality by using 2D screen-space neural renderers. In contrast, our work aims to produce a truthful and complete 3D avatar representation from the input images that can be viewed from any angle.  

\subsection{Photo-realistic 3D Face Models}
The increasing availability of large-scale multi-view face datasets~\cite{kirschstein2023nersemble, ava256, pan2024renderme360, yang2020facescape} has enabled building photo-realistic 3D face models that learn a detailed prior over both geometry and appearance of human faces. HeadNeRF~\cite{hong2022headnerf} conditions a Neural Radiance Field (NeRF)~\cite{mildenhall2021nerf} on identity, expression, albedo, and illumination codes. VRMM~\cite{yang2024vrmm} builds a high-quality and relightable 3D face model using volumetric primitives~\cite{lombardi2021mvp}. One2Avatar~\cite{yu2024one2avatar} extends a 3DMM by anchoring a radiance field to its surface. More recently, GPHM~\cite{xu2025gphm} and HeadGAP~\cite{zheng2024headgap} have adopted 3D Gaussians to build a photo-realistic 3D face model. \\
Photo-realistic 3D face models learn a powerful prior over human facial appearance and geometry, which can be fitted to a single or multiple images of a person, effectively inferring a 3D head avatar. However, the fitting procedure itself is non-trivial and often requires expensive test-time optimization, impeding casual use-cases on consumer-grade devices. While this limitation may be circumvented by learning a generalized encoder that maps images into the 3D face model's latent space, another fundamental limitation remains. Even with more multi-view face datasets being published, the number of available training subjects rarely exceeds the thousands, making it hard to truly learn the full distibution of human facial appearance. Instead, our approach avoids generalizing over the identity axis by conditioning on some images of a person, and only generalizes over the expression axis for which plenty of data is available. 

A similar motivation has inspired recent work on codec avatars where a generalized network infers an animatable 3D representation given a registered mesh of a person~\cite{cao2022authentic, li2024uravatar}.
The resulting avatars exhibit excellent quality at the cost of several minutes of video capture per subject and expensive test-time optimization.
For example, URAvatar~\cite{li2024uravatar} finetunes their network on the given video recording for 3 hours on 8 A100 GPUs, making inference on consumer-grade devices impossible. In contrast, our approach directly regresses the final 3D head avatar from just four input images without the need for expensive test-time fine-tuning.



% % 视频本身有不同片段的信息不同的特性,我们可以观察到,不同的视频以及同一视频的不同片段中,内容变化的特性可能存在显著差异。现有研究[1-3]表明,视频内容的信息密度呈现显著的非均匀性特征。如图2所示,在体育赛事视频中,高速运动片段(如足球射门)的帧间差异可达低速片段(球员站立)的5-8倍;在自然纪录片中,动物捕食动作的时序复杂度比静态景观片段高出60%以上。问题是,在原始视频被压缩后的latent空间,是否也存在这样的?

% 从Nyquist-Shannon sampling theorem角度分析Latent空间的frame

% 我们对不同视频进行了分析,发现了VAE Encode转换后的latent空间,不同视频片段信号的 f_max 差异很大,这证明了在latent空间也存在降低帧率的可能性。

% 在视频处理和压缩领域,传统上对帧率的优化分析集中在原始视频信号的时间域特性上。视频的不同片段由于内容复杂度的不同,其时间信号的频率成分存在显著差异。现有研究[1-3]表明,视频内容的信息密度呈现显著的非均匀性特征。例如,在体育赛事视频中,高速运动片段(如足球射门)的帧间差异可达低速片段(如球员站立)的 5-8 倍;在自然纪录片中,动物捕食动作的时序复杂度比静态景观片段高出 60% 以上(见图 2)。这表明,对于原始视频信号,不同片段的时间频率分布决定了帧率的需求。

% 然而,随着深度学习模型在视频生成中的广泛应用,视频信号被映射到一个潜在空间(Latent Space)中。潜在空间中信号的时间特性如何分布?是否也呈现类似于原始视频信号的频率变化?是否可以在潜在空间中动态调整帧率?这些问题尚未被深入研究。

% 原始视频信号的时间频率分析
% 原始视频信号可以表示为随时间变化的连续函数 x(t),其频谱表示为 X(f)。对于一个带限信号,其频谱的最高频率为 f_max,即:

% 采样过程以帧率 F_s 为采样频率,将连续信号离散化得到离散帧信号:

% 根据 Nyquist-Shannon 采样定理,当帧率满足以下条件时:

% 采样信号可以完整重建原始信号,无混叠现象。对于原始视频,不同片段的 f_max 差异性决定了其帧率需求的可变性。

% 潜在空间信号的时间频率特性
% 设原始视频信号为连续时间函数 I(t),通过编码器 E 映射到Latent空间,得到隐变量序列:

% 由于 E 的复杂性,信号在潜在空间中的频谱分布 Z(f) 可能与原始空间 X(f) 的分布存在显著差异。理论上,潜在空间信号的频率分布可以表示为:

% 如果 f_enc() 是平滑的非线性变换,其频率响应会在某些范围内压缩或放大原始信号的频率成分。例如,某些高频分量可能被编码器过滤掉,而低频分量可能被保留甚至增强。因此,潜在空间信号的最高频率 f_max' 通常满足:

% g(·) 是由编码器频率响应决定的函数。通过对多个视频片段的分析(图8),我们发现潜在空间信号的频率分布依然呈现显著的非均匀性。高速运动视频片段(如快速摄像机移动或物体快速运动)的潜在信号通常包含更高的时间频率分量,而静态或低速运动片段的潜在信号则集中在较低频率范围内。这表明潜在空间中的时间信号仍然具有潜在的可变帧率优化空间。

% 潜在空间中可变帧率的理论分析
% 根据 Nyquist-Shannon 采样定理,潜在空间中信号的帧率需求由其最高时间频率 f_max' 决定。为了在潜在空间中实现可变帧率,我们需要对每个片段的信号频率进行估计,并动态调整帧率 F_s',满足以下条件:

% 在潜在空间中,根据每个片段的 f_max',动态调整采样帧率 F_s'。假设潜在空间信号的时序复杂度可以分为 N 个类别,每个类别的最高频率分别为 f_{\text{max},1}', f_{\text{max},2}', \dots, f_{\text{max},N}',则帧率分配策略为:

\section{Motivation}

In the field of video processing and compression, traditional optimization of frame rates has primarily focused on the temporal characteristics of raw video signals. Different segments of a video exhibit significant variations in the frequency components of their temporal signals due to differences in content complexity. Existing studies [1-3] have demonstrated that the information density of video content exhibits pronounced non-uniformity. For instance, in sports videos, the inter-frame difference in high-motion segments (e.g., a soccer goal shot) can be 5-8 times greater than that in low-motion segments (e.g., players standing still). Similarly, in nature documentaries, the temporal complexity of animal hunting actions is over 60\% higher than that of static landscape scenes (see Figure 2). These observations indicate that the temporal frequency distribution of different segments in raw video signals determines the required frame rates.

However, with the widespread adoption of deep learning models in video generation, video signals are now often mapped into a latent space. This raises several critical questions: How are the temporal characteristics of signals distributed in the latent space? Do they exhibit frequency variations similar to those of raw video signals? Is it possible to dynamically adjust frame rates in the latent space? These questions remain largely unexplored in the literature.

\subsection{Temporal Frequency Analysis of Video Signals}

Original video signals can be represented as continuous functions of time, denoted as \( x(t) \). The frequency spectrum of \( x(t) \) can be expressed as \( X(f) \) using the Fourier transform. For a band-limited signal, the frequency spectrum \( X(f) \) is nonzero only within a finite range. Specifically, the highest frequency component of the signal is denoted as \( f_{\text{max}} \), which satisfies:

\[
X(f) = 0, \quad \forall |f| > f_{\text{max}}.
\]

The sampling process is performed at a frame rate \( F_s \), which acts as the sampling frequency. The continuous signal \( x(t) \) is thus discretized into a sequence of frames:

\[
x[n] = x(nT), \quad T = \frac{1}{F_s},
\]

where \( T \) is the sampling interval. According to the Nyquist-Shannon sampling theorem, the sampling frequency \( F_s \) must satisfy the following condition to ensure the accurate reconstruction of the original signal without aliasing:

\[
F_s \geq 2 f_{\text{max}}.
\]

This implies that the frame rate must be at least twice the highest frequency component of the signal. For original video signals, the temporal frequency \( f_{\text{max}} \) varies across different segments of the video. This variation determines the frame rate requirements for different video segments, providing the basis for adaptive frame rate optimization.

\subsection{Temporal Frequency Analysis of Latent Space}

Let the original video signal be represented as a continuous-time function \( x(t) \). Through an encoder \( \mathcal{E} \), the signal is mapped into a latent space, resulting in a sequence of latent variables \( z(t) \):

\[
z(t) =\mathcal{E}(x(t)).
\]

Due to the complexity of the encoder \( \mathcal{E} \), the frequency spectrum of the signal in the latent space, denoted as \( Z(f) \), may differ significantly from the frequency spectrum of the original signal \( X(f) \). Theoretically, the frequency distribution in the latent space can be expressed as:

\[
Z(f) = \int_{-\infty}^{\infty} z(t) e^{-j 2 \pi f t} \, dt.
\]

If the encoder \( E \) (or equivalently, \( f_{\text{enc}} \)) is a smooth nonlinear transformation, its frequency response may compress or amplify certain frequency components of the original signal. For instance, high-frequency components may be filtered out by the encoder, while low-frequency components may be preserved or enhanced. Consequently, the highest frequency in the latent space, \( f_{\text{max}}' \), typically satisfies:

\[
f_{\text{max}}' \leq g(f_{\text{max}}),
\]

where \( g(\cdot) \) is a function determined by the frequency response of the encoder.

Through the analysis of multiple video segments (as illustrated in Figure 8), we observe that the frequency distribution of latent space signals exhibits significant non-uniformity. For example:
- High-speed video segments, such as rapid camera movements or fast-moving objects, often contain higher temporal frequency components in the latent space.
- Static or low-speed segments, on the other hand, exhibit frequency components concentrated in the lower range.

This indicates that the temporal signals in the latent space retain the potential for frame rate optimization, similar to the original video signal.

% \subsection{Theoretical Analysis of Variable Frame Rates in Latent Space}

According to the Nyquist-Shannon sampling theorem, the frame rate \( F_s' \) required in the latent space is determined by the highest temporal frequency \( f_{\text{max}}' \) of the latent signal. To implement variable frame rates in the latent space, it is necessary to estimate the temporal frequency of each segment and dynamically adjust \( F_s' \) to satisfy the following condition:

\[
F_s' \geq 2 f_{\text{max}}'.
\]

% \subsection{Dynamic Frame Rate Adjustment in Latent Space}

In the latent space, the frame rate \( F_s' \) can be dynamically adjusted based on the maximum temporal frequency \( f_{\text{max}}' \) of each video segment. 

% Suppose the temporal complexity of latent signals can be categorized into \( N \) classes, where each class has a corresponding maximum frequency \( f_{\text{max},1}', f_{\text{max},2}', \dots, f_{\text{max},N}' \). The frame rate allocation can then be expressed as:

% \[
% F_s' = 
% \begin{cases} 
% 2 f_{\text{max},1}' & \text{if the segment belongs to class 1}, \\
% 2 f_{\text{max},2}' & \text{if the segment belongs to class 2}, \\
% \vdots & \vdots \\
% 2 f_{\text{max},N}' & \text{if the segment belongs to class \( N \)}.
% \end{cases}
% \]

% This adaptive strategy ensures that the latent space signals are sampled without loss of information while minimizing the required frame rate.

% \subsection{Potential Benefits of Latent Space Optimization}

% By implementing variable frame rates in the latent space, significant computational and storage savings can be achieved while maintaining signal fidelity. Our experiments demonstrate that:
% 1. The latent space signals of high-speed video segments exhibit \( f_{\text{max}}' \) values that are 3-5 times higher than those of static segments.
% 2. Dynamically adjusting \( F_s' \) according to \( f_{\text{max}}' \) reduces the total frame count by 40\%-60\% without noticeable loss of quality.

% These findings highlight the feasibility and potential advantages of frame rate optimization in the latent space, paving the way for more efficient video compression and generation pipelines.

% \section{Study Design}
% robot: aliengo 
% We used the Unitree AlienGo quadruped robot. 
% See Appendix 1 in AlienGo Software Guide PDF
% Weight = 25kg, size (L,W,H) = (0.55, 0.35, 06) m when standing, (0.55, 0.35, 0.31) m when walking
% Handle is 0.4 m or 0.5 m. I'll need to check it to see which type it is.
We gathered input from primary stakeholders of the robot dog guide, divided into three subgroups: BVI individuals who have owned a dog guide, BVI individuals who were not dog guide owners, and sighted individuals with generally low degrees of familiarity with dog guides. While the main focus of this study was on the BVI participants, we elected to include survey responses from sighted participants given the importance of social acceptance of the robot by the general public, which could reflect upon the BVI users themselves and affect their interactions with the general population \cite{kayukawa2022perceive}. 

The need-finding processes consisted of two stages. During Stage 1, we conducted in-depth interviews with BVI participants, querying their experiences in using conventional assistive technologies and dog guides. During Stage 2, a large-scale survey was distributed to both BVI and sighted participants. 

This study was approved by the University’s Institutional Review Board (IRB), and all processes were conducted after obtaining the participants' consent.

\subsection{Stage 1: Interviews}
We recruited nine BVI participants (\textbf{Table}~\ref{tab:bvi-info}) for in-depth interviews, which lasted 45-90 minutes for current or former dog guide owners (DO) and 30-60 minutes for participants without dog guides (NDO). Group DO consisted of five participants, while Group NDO consisted of four participants.
% The interview participants were divided into two groups. Group DO (Dog guide Owner) consisted of five participants who were current or former dog guide owners and Group NDO (Non Dog guide Owner) consisted of three participants who were not dog guide owners. 
All participants were familiar with using white canes as a mobility aid. 

We recruited participants in both groups, DO and NDO, to gather data from those with substantial experience with dog guides, offering potentially more practical insights, and from those without prior experience, providing a perspective that may be less constrained and more open to novel approaches. 

We asked about the participants' overall impressions of a robot dog guide, expectations regarding its potential benefits and challenges compared to a conventional dog guide, their desired methods of giving commands and communicating with the robot dog guide, essential functionalities that the robot dog guide should offer, and their preferences for various aspects of the robot dog guide's form factors. 
For Group DO, we also included questions that asked about the participants' experiences with conventional dog guides. 

% We obtained permission to record the conversations for our records while simultaneously taking notes during the interviews. The interviews lasted 30-60 minutes for NDO participants and 45-90 minutes for DO participants. 

\subsection{Stage 2: Large-Scale Surveys} 
After gathering sufficient initial results from the interviews, we created an online survey for distributing to a larger pool of participants. The survey platform used was Qualtrics. 

\subsubsection{Survey Participants}
The survey had 100 participants divided into two primary groups. Group BVI consisted of 42 blind or visually impaired participants, and Group ST consisted of 58 sighted participants. \textbf{Table}~\ref{tab:survey-demographics} shows the demographic information of the survey participants. 

\subsubsection{Question Differentiation} 
Based on their responses to initial qualifying questions, survey participants were sorted into three subgroups: DO, NDO, and ST. Each participant was assigned one of three different versions of the survey. The surveys for BVI participants mirrored the interview categories (overall impressions, communication methods, functionalities, and form factors), but with a more quantitative approach rather than the open-ended questions used in interviews. The DO version included additional questions pertaining to their prior experience with dog guides. The ST version revolved around the participants' prior interactions with and feelings toward dog guides and dogs in general, their thoughts on a robot dog guide, and broad opinions on the aesthetic component of the robot's design. 



\section{Method}
In this section, we present \dlfr, a training-free solution for dynamic frame rate control in the latent space. We first establish our theoretical foundation by analyzing the temporal frequency of video signals in both pixel and latent spaces (Sec.\ref{subsec:motivation}). Based on this analysis, we propose our dynamic latent frame rate space (Sec.\ref{subsec:dflr_space}). To realize this design, we propose two key technical components: \raisebox{-1.1pt}{\ding[1.1]{182\relax}} a Dynamic Latent Frame Rate Scheduler that determines optimal frame rates based on content complexity (Sec.\ref{subsec:dflr_Scheduler}), and \raisebox{-1.1pt}{\ding[1.1]{183\relax}} a training-free adaptation mechanism that enables pretrained VAE architectures to process variable frame rate features (Sec.\ref{subsec:dflr_vae}).
Beyond these components, we provide a more straightforward explanation for why our simple yet effective approach can work (Sec.\ref{subsec:discuss}).



\subsection{Motivation}
\label{subsec:motivation}
Traditional frame rate optimization in video processing and compression has predominantly focused on raw video signals \cite{song2001rate,mackin2015study}. Previous studies in video content analysis have demonstrated that video information density exhibits strong temporal non-uniformity. This non-uniformity manifests as significant variations in temporal frequency across different video segments \cite{menon2022vca,papakonstantinou2023content}. 
For instance, in our analysis of the BVI-HFR dataset (Fig.~\ref{fig:video_examples}), fast-motion sequences like ``books'' exhibit 5--8× higher temporal frequency magnitude compared to static scenes like ``flowers'' (see Fig.\ref{fig:analysis_on_video_seg} Top). 

However, with the growing adoption of deep learning for video generation, videos are increasingly mapped into a latent space via an encoder~\cite{fan2024fluid,kuaishou2024,zheng2024open,kong2024hunyuanvideo}. This raises crucial questions: \textit{(i) How do temporal characteristics transfer into the latent space? (ii) Does the latent space preserve the frequency variations observed in the original pixel domain? (iii) Can we apply dynamic frame rates within the latent space?} Although these questions are highly relevant, all of them remain underexplored.




\paragraph{Frequency Analysis of Signals.}

A continuous-time signal can be denoted as $x(t)$. Its frequency spectrum $X(f)$ is obtained via the Fourier transform, and for a  band-limited signal, it is nonzero only up to a maximum frequency $f_{\max}$:
\begin{equation}
X(f) = 0, \quad \forall|f| > f_{\max}.
\end{equation}
Sampling the continuous signal at sampling frequency $F_s$ discretizes $x(t)$ into frames:
\begin{equation}
x[n] = x(nT), \quad T = \frac{1}{F_s}.
\end{equation}
According to the Nyquist-Shannon sampling theorem~\cite{Nyquist_theory,shannon_noise}, $F_s$ must satisfy
\begin{equation}
F_s \geq 2f_{\max}.
\end{equation}
to prevent aliasing. As $f_{\max}$ varies across different segments of the signal, different segments naturally require different frame rates, motivating adaptive frame rate strategies.

\paragraph{Temporal Frequency Analysis of Latent Space.}
Let the video luminance signal be $x(t)$, and its encoder mapping be $\mathcal{E}$. In the latent space, the signal becomes $z(t) = \mathcal{E}(x(t))$ with a  corresponding frequency spectrum $Z(f)$:
\begin{equation}
Z(f) = \int_{-\infty}^{\infty} z(t)e^{-j2\pi ft} dt.
\end{equation}
Note that $\mathcal{E}$ is a complex nonlinear transformation, it can alter the amplitude, phase, and frequency characteristics of a signal or generate new frequency components, leading to changes in the video signal's shape and spectrum.
Therefore, we first analyze the signal in the latent space.
% Hence, the highest frequency in the latent domain, $f'_{\max}$, usually satisfies:
% \begin{equation}
% f'_{\max} \leq g(f_{\max}),
% \end{equation}
% where $g(\cdot)$ denotes the encoder's frequency response function~\cite{smith1997scientist}.

% \begin{figure}[!t]
%     \centering
%     \includegraphics[width=0.99\linewidth]{figures/frequency_analyse.pdf}
%     \caption{Frequency analysis of original video and latent space in temporal dimension. The video segments are 1080p, 60Hz, and 10-second length in BVI-HFR dataset~\cite{mackin2018study}.}
%     \vspace{-0.2in}
%     \label{fig:analysis_on_video_seg}
% \end{figure}


Through empirical analysis on multiple video segments (see Fig.\ref{fig:analysis_on_video_seg}), we observe pronounced frequency variability in both the original video domain and latent domain. High-speed segments—such as rapid camera pans or fast-moving objects—often retain higher temporal frequencies, whereas low-motion segments focus on lower-frequency components. This suggests that adaptive frame rate optimization remains viable in latent space, just as it is in raw pixel space.

In line with the Nyquist-Shannon theorem, when the latent-space sampling rate $F'_s$ meet
\begin{equation}
F'_s \geq 2f'_{\max},
\end{equation}
aliasing can be prevented. The $f'_{\max}$ is its maximum frequency in latent space. Hence, segment-wisely estimating $f'_{\max}$ and adjusting $F'_s$ provides an opportunity to implement a variable frame rate in latent space. 

\begin{figure*}[!tbhp] % h:here 当前位置 % b bottom % t top % p 浮动
    \centering
    \includegraphics[width=0.98\textwidth]{figures/Dynamic_VAE.pdf} %ims/xx.png
    \vspace{-0.1in}
    \caption{\textbf{Architecture overview of the Dynamic Latent Frame Rate (DLFR) VAE.} The input video is first divided into segments. The dynamic encoder processes these segments through a series of 3D convolution layers interspersed with dynamic downsample operations (Eq.\ref{eq:encoder} in Sec.\ref{subsec:dflr_vae}), where the execution of downsample is determined by the schedule (Sec.\ref{subsec:dflr_Scheduler}). The resulting latent representations maintain varying temporal resolutions according to segment complexity (Sec.\ref{subsec:dflr_space}). The dynamic decoder then reconstructs the video through corresponding upsampling operations (Eq.\ref{eq:decoder} in Sec.\ref{subsec:dflr_vae}), restoring the original frame rate while preserving temporal consistency. Each segment can be processed at different frame rates, enabling content-adaptive temporal compression in latent space.}
    \vspace{-0.1in}
    \label{fig:dynamic_vae}
\end{figure*}

\subsection{Dynamic Frame Rate Latent Space}
\label{subsec:dflr_space}
Building on the above analysis, we propose a dynamic frame rate latent space wherein each video segment can have a distinct frame rate, allocated based on its temporal complexity. Specifically, suppose an input video is divided into $M$ segments $\{S_1,\ldots,S_M\}$, each comprising $N$ frames. For segment $S_i$, its latent representation is $z_i(t)$. The frequency spectrum $Z_i(f)$ is defined as
\begin{equation}
Z_i(f) = \int_{t_i}^{t_i+NT} z_i(t)e^{-j2\pi ft} dt,
\end{equation}
where $T = 1/F_s$ is the sampling interval in the raw video domain.



From empirical observations, certain high-frequency components in $Z_i(f)$ have negligible amplitude and minimal impact on overall fidelity. We thus define an effective maximum frequency $f_{\text{eff},i}$ for each segment, identifying the point where the amplitude remains above a threshold $\epsilon$:
\begin{equation}
f_{\text{eff},i} = \max\{f \mid |Z_i(f)| \geq \epsilon\}.
\end{equation}

By the Nyquist-Shannon principle~\cite{ash2012information}, the corresponding latent-space frame rate for segment $S_i$ can be lowered to
\begin{equation}
F'_{s,i} = 2f_{\text{eff},i}.
\end{equation}
This adaptive sampling ensures each segment maintains only the minimum frame rate necessary to preserve perceptually significant temporal details. To maintain temporal consistency across segment boundaries, we implement a
smooth transition mechanism that gradually adjusts frame
rates between adjacent segments.


\subsection{DLFR Scheduler} 
\label{subsec:dflr_Scheduler} 
While theoretically sound, computing exact frequency spectra for real-time video processing presents significant computational challenges. We address this through a practical approximation strategy that maintains the benefits of dynamic frame rates while ensuring computational efficiency. Our approach discretizes the continuous space of temporal complexities into $N$ distinct levels.
Each level $k \in \{1,\ldots,N\}$ is associated with an effective frequency $f'_{\text{eff},k}$~\footnote{For example, we use \{1, 2, 4\}Hz for the 16 FPS video, which have \{16x, 8x, 4x\} temporal downsample ratio.}. % TODO: the number ablation
If segment $S_i$ falls into complexity class $k$, its latent frame rate becomes
\begin{equation}
F'_{s,i} = 2f'_{\text{eff},k}.
\end{equation}
Formally, we express this as:
\begin{equation}
F'_{s,i} = \sum_{k=1}^N (2f'_{\text{eff},k}) \mathbb{I}_{C_k}(S_i),
\end{equation}
where $\mathbb{I}_{C_k}(S_i)$ is an indicator function that is 1 if $S_i$ belongs to class $C_k$ and 0 otherwise.

\begin{figure}[!tb] % h:here 当前位置 % b bottom % t top % p 浮动
    \centering
    \includegraphics[width=0.9\linewidth]{figures/frequency_analyse_eff.pdf} %ims/xx.png
    \caption{Content complexity experiment on HunyuanVideo VAE. The upper figure illustrates the relationship between content complexity and effective frequency, with  $\epsilon=1.8$ used in this experiment. The lower figure demonstrates the alignment between content complexity and reconstruction LPIPS, indicating a strong correlation.}
    \label{im_content_complexity}
\end{figure}

Directly evaluating $\mathbb{I}_{C_k}(S_i)$ from raw or latent signals can still be challenging. 
Instead, we use a practical content complexity metric $C(S_i)$ as a proxy, which considers the SSIM of adjacent frames in a raw video segment: 
\begin{equation}
C(S_i) = \frac{1}{N}\sum_{j=1}^{N-1} (1-\text{SSIM}(x[j],x[j+1])),
\end{equation}
This metric can efficiently distinguish high-motion segments from low-motion ones without explicitly analyzing the latent frequency spectrum. The scheduling logic then maps each segment's metric value to an appropriate complexity class $k$, and hence to a frame rate $F'_{s,i}$. 
As illustrated in Figure~\ref{im_content_complexity}, the content complexity metric exhibits a high correlation with both the effective frequency and the VAE reconstruction performance.
We use this metric and thresholds $Th$ to determine the frame rate:
\begin{equation} 
\label{eq_complexity_threshold}
\mathbb{I}_{C_k}(S_i) = 
\begin{cases} 1, & \text{if}~~~~Th^{down}_{i}<C(S_i)\le Th^{up}_{i} \\
0, & \text{else}. \end{cases} 
\end{equation} 


% \[
% \mathbb{I}_{C_k}(S_i) = 
% \begin{cases} 
% 1 & \text{if } S_i \in C_k, \\
% 0 & \text{otherwise}.
% \end{cases}
% \]




% we cannot get the real the temporal space of input video is quantized, 
% 如何评价,提出Content Information

% 具体而言,我们引入了一个动态帧率Scheduler(Dynamic Frame Rate Encoder Scheduler)。该模块首先评估视频每个chunk的信息密度,并动态分配对应的潜空间帧率。

\subsection{Transform Static VAE to Dynamic VAE}
\label{subsec:dflr_vae}

% 如何从原始视频生成动态帧率的潜空间,以及如何从动态潜空间还原回原始视频。我们注意到:以往的视频VAE模型通过在大规模包含各种动作与节奏的视频数据上训练,已经具备了将视频压缩到固定帧率潜空间并还原回原始视频的能力。这一特性启发我们对现有的静态帧率VAE模型进行改造,扩展出动态帧率的潜空间生成能力。我们需要一个运行的过程中动态根据输入scheduler的变化而变化的模型(Encoder模型的输出维度动态变化,Decoder模型的输入维度动态变化),因此我们需要将静态模型转换为一个动态模型。

% 具体来说,我们在VAE的Encoder中插入一个dynamic down下采样器我们在VAE模型中的合适位置根据分配的帧率对VAE中间特征进行动态下采样,从而得到包含不同帧率的chunk特征。在该module之后的VAE模型以相同方式推理具有不同的chunk帧率的features,最终输出的结果就是在dynamic frame rate的latent space。在解码过程中,我们在Deocoder模型中的合适位置根据编码时确定的帧率对中间特征进行上采样,将所有chunk特征恢复到一致的帧率,最终重建出原始视频。我们将其称为DLFR-VAE(Dynamic Latent Frame Rate VAE)


To convert a pretrained static-frame VAE into a dynamic-frame version, we exploit the existing capacity of modern video VAEs, which have learned to compress videos into fixed-frame latent spaces~\cite{chen2024deep,xing2024large,zhu2023designing}. As illustrated in Fig.~\ref{fig:dynamic_vae}, our approach introduces two key modifications to the pretrained VAE: a dynamic downsampling module in the VAE's encoder and a corresponding upsampling module in the decoder. This design allows us to leverage the robust compression capabilities of pretrained VAEs while enabling variable frame rate processing without requiring additional training.

\paragraph{Encoder Modification.} Let the input video be $\{x_1,x_2,\ldots,x_T\}$. A pretrained video VAE encoder $E$ typically processes this input into a latent representation $z$. To support variable frame rates, we introduce a dynamic downsampler at a strategically chosen point in the encoder. Given a frame-rate schedule $\{F'_{s,1},F'_{s,2},\ldots,F'_{s,M}\}$, the downsampler transforms encoder features $h_i$ for each segment $S_i$ into a reduced-rate feature $h'_i$:
\begin{equation}
h'_i = \text{Downsample}(h_i, F'_{s,i}).
\label{eq:encoder}
\end{equation}
These reduced-rate features are then passed through the remaining encoder layers, denoted $E_{\text{post}}$, to yield segment-wise latent codes $\{z_1,\ldots,z_M\}$.

\paragraph{Decoder Modification.} Decoding requires reversing the frame rate changes. A dynamic upsampler is inserted at the corresponding decoder stage. For each segment's latent code $z_i$, the initial decoder layers $D_{\text{pre}}$ produce intermediate features $h''_i$:
\begin{equation}
h''_i = D_{\text{pre}}(z_i).
\end{equation}
The upsampler then restores the original frame rate $F_s$:
\begin{equation}
h'''_i = \text{Upsample}(h''_i, F_s),
\label{eq:decoder}
\end{equation}
after which the remaining decoder layers $D_{\text{post}}$ reconstruct the final segment $\hat{S}_i$. The overall video reconstruction $\hat{V}$ is formed by concatenating $\{\hat{S}_1,\ldots,\hat{S}_M\}$.

Crucially, these modifications allow the pretrained encoder and decoder weights to remain largely unchanged, except for the newly inserted downsampling and upsampling operators. Consequently, \dlfr\ can be deployed as a training-free extension on top of mainstream video VAEs, seamlessly enabling dynamic latent frame rate control.

\subsection{Discussion on \dlfr}
\label{subsec:discuss}
In addition to our information-theoretic formulation, we offer an intuitive explanation for why our \textbf{simple but effective} \dlfr\ can compress latent space with minimal reconstruction loss.
At its core, \dlfr\ dynamically downsamples the pretrained VAE encoder and, in turn, upsamples its decoder—effectively achieving content-dependent spatial-temporal compression without additional training as shown in Fig.\ref{fig:dynamic_vae}.

Pretrained video VAEs~\cite{chen2024deep,xing2024large,zhu2023designing} are typically trained on large-scale datasets that include diverse motion types, ranging from slow to fast, and from sparse to dense. Many training pipelines also involve augmentation techniques (e.g., temporal interpolation) that effectively expose the VAE to slower versions of the same content. Consequently, the VAE develops an internal capacity to represent video content at different temporal scales, but its default latent space is merely configured to operate at a fixed frame rate. \dlfr~can thus be seen as a mechanism that ``reactivates'' this dormant flexibility. By strategically downsampling and upsampling in the encoder and decoder, respectively, \textit{we allow the VAE to adapt to each segment’s temporal complexity, leveraging the latent representational power that was already learned but not previously utilized for frame rate variation.}


% \clearpage
\section{Experiments}
\label{sec:exp}
\subsection{Experimental settings}

\noindent\textbf{Benchmark.}  We conduct experiments on two established 3D occupancy benchmarks: (i) nuScenes~\cite{nuScenes}, which provides instance-level annotations with manually labeled 3D bounding boxes (position/size/orientation) for dynamic objects, and (ii) Occ3D~\cite{Occ3D}, which generates voxel-level occupancy labels (0.4m resolution) through automated LiDAR point cloud aggregation and mesh reconstruction, including occlusion states. Both benchmarks share identical scene configurations of 1,050 driving scenes, each containing up to 40 timestamped frames. Every frame includes six synchronized camera views (front, front-left, front-right, back, back-left, back-right) at 1600$\times$900 resolution. In our experiments, we extend single-frame baselines~\cite{MonoScene,surroundOcc,viewformer} by aggregating features from $N$ historical keyframes. Additionally, we extract unlabeled intermediate frames from the ``sweeps'' folder~\cite{nuScenes} to provide implicit motion cues, enabling self-supervised temporal consistency learning.

\noindent\textbf{Implementation details.} For the nuScenes benchmark~\cite{nuScenes}, we follow the parameter settings of SurroundOcc~\cite{surroundOcc}, using $Cam=6$, $p=6$, $v=32$,$L=116$, and $W=200$. For the Occ3D benchmark~\cite{Occ3D}, we adopt ViewFormer's~\cite{viewformer} standard setup with $Cam=6$, $p=6$, $v=32$, $L=32$, and $W=88$. The output of the occupancy result on both benchmarks is formatted into a vector with dimensions $[200, 200, 16]$. In this vector, the first two dimensions (200 and 200) represent the length and width, while the third (16) indicates the height. The occupancy result covers a range from -50 meters to 50 meters in both width and length, and the vertical height varies from -5 meters to 3 meters. Each voxel corresponds to a cube measuring 0.5 meters on each side. Occupied voxels are categorized into one of 17~\cite{nuScenes,surroundOcc} and 18~\cite{Occ3D} semantic classes.
More details on implementation can be found in the supplementary material.

\subsection{Evaluation Metrics}

To validate the temporal consistency and occupancy accuracy of moving and static objects, objects are divided into two general classes~\cite{Cam4docc}: General Moving Objects (GMO) and General Static Objects (GSO). Detailed classification classes are introduced in the supplementary material.

\noindent\textbf{Occupancy Accuracy Metric.} To ensure rigorous evaluation across different benchmarks, we employ both Intersection over Union (IoU) and Mean Intersection over Union (mIoU) metrics. These metrics are widely adopted in 3D semantic occupancy prediction tasks~\cite{PASCAL, Microsoft_COCO, Cityscapes_dataset, Mask_R_CNN}. The mIoU are calculated separately for three category groups: All classes, GMO classes, and GSO classes.

\noindent\textbf{Temporal Consistency Metric.} \label{para:consistency_metric} To evaluate the effect achieved by integrating \ours\ with baseline models, we propose a temporal consistency metric. We aim to detect and measure changes in a scene from one frame to the next. This metric reflects the stability of prediction results, which directly impacts the user's visual experience. Let $\sigma_{i,n}^{(x,y,z)}$ denote the semantic label of the $n$-th voxel point (with coordinates $(x,y,z)$) in frame $i$, and define the indicator function $\delta(e_1,e_2) = \mathbb{I}(e_1 \neq e_2)$. 

In the occupancy results of frames $i$ and $j$, voxels at corresponding positions may undergo changes, which are categorized into two types: ``Static Object Change"~(SOC) and ``Moving Object Change"~(MOC). The definitions of these changes are as table \ref{tab:moc-soc}.

\begin{wrapfigure}[7]{l}{80mm}
\centering
% \setlength{\tabcolsep}{8pt}
\captionsetup{type=table}
    \begin{tabular}{c|c}
    \toprule
    \textbf{Type} & \textbf{Condition} \\  
    \midrule
    MOC & $\sigma_{i,n}^{(x,y,z)} \in \text{GMO} \lor \sigma_{j,n}^{(x,y,z)} \in \text{GMO}$ \\
    % \midrule
    SOC & $\sigma_{i,n}^{(x,y,z)} \wedge \sigma_{j,n}^{(x,y,z)} \in \text{GSO}$ \\  
    \bottomrule
    \end{tabular}
    \vspace{8pt}
    \caption{Definition of MOC and SOC. $N_{mc}$/$N_{sc}$ denote the number of MOC/SOC voxels, respectively.}
    \label{tab:moc-soc}
    % \vspace{-10pt}
\end{wrapfigure}

Based on these definitions, we can define disparity metrics~($\Delta_{m}$/$\Delta_{s}$) to quantify temporal inconsistencies across frames~($i$ and $j$). The process is defined as:
\begin{equation}
\begin{dcases}
\Delta_{m}(i,j) = \dfrac{1}{N_{mc}} \sum\limits_{n=1}^{N_{mc}} \delta\left(\sigma_{i,n}^{(x,y,z)}, \sigma_{j,n}^{(x,y,z)}\right) \\
\Delta_{s}(i,j) = \dfrac{1}{N_{sc}} \sum\limits_{n=1}^{N_{sc}} \delta\left(\sigma_{i,n}^{(x,y,z)}, \sigma_{j,n}^{(x,y,z)}\right).
\end{dcases}
\label{eq:disparity}
\end{equation}

The temporal consistency metrics -- $S_m$ (moving) and $S_s$ (static) -- are derived through aggregation of $\Delta_{m}$ and $\Delta_{s}$ across sequential frames. Formally, we have:
\begin{equation}
S_{m/s} = 1 - \dfrac{1}{M-1} \sum\limits_{k=1}^{M-1} \Delta_{m/s}(k,k+1),
\label{eq:consistency_scores}
\end{equation}
where $M$ is the scene's total frame count. Final metrics $\overline{S_m}$/$\overline{S_s}$ average across all scenes. A higher temporal consistency score indicates that the predictions within the scene are smoother and more consistent over time.

\begin{table}[t]
    % \centering
    \footnotesize 
    \begin{minipage}[t]{0.48\textwidth}
        \centering
        % \captionsetup{justification=centering, singlelinecheck=false}
        \setlength{\tabcolsep}{2pt}
        \renewcommand{\arraystretch}{1.25}
        \begin{tabular}{r|cccc|cc}
            \toprule
            \multicolumn{1}{r|}{\multirow{2}{*}[-0.4em]{Method}} & \multicolumn{1}{c|}{\multirow{2}{*}[-0.4em]{IoU~$\uparrow$}} & \multicolumn{3}{c|}{mIoU~$\uparrow$} & \multicolumn{1}{c}{\multirow{2}{*}[-0.4em]{$\overline{S_m}\uparrow$}} & \multicolumn{1}{c}{\multirow{2}{*}[-0.4em]{$\overline{S_s}\uparrow$}} \\ \cmidrule(lr){3-5}
            \multicolumn{1}{c|}{} & \multicolumn{1}{c|}{} & All & GMO & GSO & \multicolumn{1}{c}{} & \multicolumn{1}{c}{} \\ 
            \midrule        
            Atlas~\cite{Atlas} & 28.66 & 15.00 & 12.64 & 17.35 & \text{--} & \text{--}  \\
            BEVFormer{~\cite{BEVFormer}} & 30.50 & 16.75 & 14.17 & 19.33 & \text{--} & \text{--}  \\
            TPVFormer~\cite{TPVFormer} & 30.86 & 17.10 & 14.04 & 20.15 & \text{--} & \text{--} \\
            BEVDet4D-Occ~\cite{bevdet4d} & 24.26 & 14.22 & 11.10 & 17.34 & \text{--} & \text{--} \\
            MonoScene~\cite{MonoScene} & 10.04 & 1.15 & 0.24 & 2.07 & 46.53 & 81.77 \\
            % Cam4DOcc~\cite{Cam4docc} & 23.92 & 7.12 & 4.71 & 10.17 & 60.34 & 91.15 \\
            SurroundOcc~\cite{surroundOcc} & 31.49 & 20.30  & \cellcolor{gray!20}18.39 & 22.20 & 58.33 & 91.71 \\
            \midrule
            \makecell[r]{MonoScene \\ \textbf{+\ours}} & \makecell{13.10\\\textbf{+3.06}} & \makecell{1.69\\\textbf{+0.54}}  & \makecell{0.34\\\textbf{+0.10}} & \makecell{3.04\\\textbf{+0.98}} & \makecell{54.21\\\textbf{+7.68}} & \makecell{83.84\\\textbf{+2.07}} \\
            \midrule
            \makecell[r]{SurroundOcc \\ \textbf{+\ours}} & \cellcolor{gray!20}\makecell{33.12 \\ \textbf{+1.63}} & \cellcolor{gray!20}\makecell{20.67\\\textbf{+0.37}}  & \makecell{18.26\\-0.13} & \cellcolor{gray!20}\makecell{23.08\\\textbf{+0.88}} & \cellcolor{gray!20}\makecell{60.64\\\textbf{+2.31}} & \cellcolor{gray!20}\makecell{92.54\\\textbf{+0.83}} \\
            \bottomrule
        \end{tabular}
        \vspace{2mm}
        \caption{Occupancy prediction accuracy on \textbf{nuScenes benchmark~\cite{nuScenes}}. For a fair comparison, we ensure that all models have uniform input data. The best performance is highlighted in gray.}
        \label{tab:main-res-a}
    \end{minipage}\hfill
    \begin{minipage}[t]{0.48\textwidth}
        \centering
        % \captionsetup{justification=centering, singlelinecheck=false}
        \setlength{\tabcolsep}{2pt}
        \begin{tabular}{r|cccc|cc}
            \toprule
            \multicolumn{1}{r|}{\multirow{2}{*}[-0.4em]{Method}} & \multicolumn{1}{c|}{\multirow{2}{*}[-0.4em]{IoU~$\uparrow$}} & \multicolumn{3}{c|}{mIoU~$\uparrow$} & \multicolumn{1}{c}{\multirow{2}{*}[-0.4em]{$\overline{S_m}\uparrow$}} & \multicolumn{1}{c}{\multirow{2}{*}[-0.4em]{$\overline{S_s}\uparrow$}} \\ \cmidrule(lr){3-5}
            \multicolumn{1}{c|}{} & \multicolumn{1}{c|}{} & All & GMO & GSO & \multicolumn{1}{c}{} & \multicolumn{1}{c}{} \\ 
            \midrule    
            MonoScene~\cite{MonoScene} & \text{--} & 6.06 & 5.36 & 6.68 & \text{--} & \text{--} \\
            OccFormer~\cite{OccFormer} & \text{--} & 21.93 & 21.78 & 22.06 & \text{--} & \text{--} \\
            % CTF-Occ~\cite{Occ3D} & 28.53 & 27.42 & 29.52 & \text{--} & \text{--} \\
            FB-OCC~\cite{fb_occ} & \text{--} & 39.11  & 33.74 & 43.88 & \text{--} & \text{--} \\
            SparseOcc~\cite{SparseOcc_Liu} & \text{--} & 30.10  & \text{--} & \text{--} & \text{--} & \text{--} \\
            BEVDet4D-Occ~\cite{bevdet4d} & \text{--} & 39.30  & 29.09 & 42.16 & \text{--} & \text{--} \\ 
            OPUS-L~\cite{opus} & \text{--} & 36.20  & 31.25 & 40.44 & \text{--} & \text{--} \\      
            SurroundOcc~\cite{surroundOcc} & 51.89 & 7.24  & 0.36 & 13.35 & 65.35 & 89.54 \\
            ViewFormer~\cite{viewformer} & 70.39 & 40.46  & 33.73 & 46.45 & 67.26 & 86.06 \\
            \midrule
            \makecell[r]{SurroundOcc \\ \textbf{+\ours}} & \makecell{52.13\\\textbf{+0.24}}& \makecell{10.33\\\textbf{+3.09}}  & \makecell{1.98\\\textbf{+1.62}} & \makecell{17.76\\\textbf{+4.41}} & \makecell{69.60\\\textbf{+4.25}} & \cellcolor{gray!20}\makecell{90.91\\\textbf{+1.37}} \\
            \midrule
            \makecell[r]{ViewFormer \\ \textbf{+\ours}} & \cellcolor{gray!20}\makecell{70.63\\\textbf{+0.24}} & \cellcolor{gray!20}\makecell{41.30\\\textbf{+0.84}}  & \cellcolor{gray!20}\makecell{34.33\\\textbf{+0.60}} & \cellcolor{gray!20}\makecell{47.50\\\textbf{+1.05}} & \cellcolor{gray!20}\makecell{70.13\\\textbf{+2.87}} & \makecell{87.10\\\textbf{+1.04}} \\
            \bottomrule
        \end{tabular}
        \vspace{2mm}
        \caption{Occupancy prediction accuracy on \textbf{Occ3D benchmark~\cite{Occ3D}}. For a fair comparison, we ensure that all models have uniform input data. The best performance is highlighted in gray.}
        \label{tab:main-res-b}
    \end{minipage}
    \vspace{-3mm}
    % \caption{Occupancy prediction accuracy on two benchmarks. The best performance is highlighted in gray.}
    \label{tab:main-res}
    \vspace{-5mm}
\end{table}

\subsection{Comparison Results}

\noindent\textbf{Occupancy accuracy on nuScenes.} We compare our method against several SOTA models, including Atlas~\cite{Atlas}, BEVFormer~\cite{BEVFormer}, TPVFormer~\cite{TPVFormer}, MonoScene~\cite{MonoScene}, and SurroundOcc~\cite{surroundOcc}. For a fair comparison, all methods are trained on the same ground truth and follow the same training procedure. By combining methods such as MonoScene~\cite{MonoScene} and SurroundOcc~\cite{surroundOcc} with \ours, we evaluate the effect of \ours\ in performance enhancement. The results presented in \cref{tab:main-res-a} show that our performance improvement is significant. Notably, the incorporation of \ours\ into SurroundOcc~\cite{surroundOcc} has led to improved metrics that surpass those of all other models listed in this table. The results are improved by 1.63\% and 0.37\% compared with SurroundOcc~\cite{surroundOcc} in IoU and mIoU~(All), respectively.

\noindent\textbf{Occupancy accuracy on Occ3D.} We also conduct experiments on Occ3D~\cite{Occ3D} in \cref{tab:main-res-b}. To validate \ours, we conducted two sets of experiments: First, integrating \ours\ with the 3D VONs~\cite{surroundOcc,viewformer} improved one of the original models'~\cite{viewformer} performance by 0.24\% in IoU and 0.84\% in mIoU. Second, \ours\ consistently outperforms existing history-aware VONs~\cite{opus,bevdet,SparseOcc_Liu,fb_occ} by over 2\% mIoU, demonstrating the efficacy of the \ours.


\noindent\textbf{Temporal Consistency.} The results of $\overline{S_m}$ and $\overline{S_s}$ shown in \cref{tab:main-res} indicate that the integration of \ours\ improved the temporal consistency of occupancy across all frames in all scenes for all models, demonstrating \ours' effectiveness. This enhancement can be attributed to the incorporation of previous keyframes from the dataset~\cite{nuScenes,Occ3D}, along with the addition of intermediate frames from the ``sweeps''~\cite{nuScenes} directory for the SFE and MFE modules. These elements provide critical historical information and motion clues for the model.

\subsection{Ablation study}

Our ablation experiments are all conducted on the nuScenes benchmark~\cite{nuScenes}. The results are presented in~\cref{tab:ablation-studies}.

\begin{wrapfigure}{l}{90mm}
\centering
\captionsetup{type=table}
    \begin{subtable}[t]{0.48\textwidth}
    \centering
    \footnotesize
    \begin{tabular}{c|ccc|cccc}
    \toprule 
    Idx. & Pre & Cur & Mid & IoU$\uparrow$ & mIoU$\uparrow$ & $\overline{S_m}\uparrow$ & $\overline{S_s}\uparrow$ \\
    \midrule
    \textbf{M0} & \ding{55} & \ding{55} & \ding{55} & 31.49 & 20.30 & 58.33 & 91.71 \\
    \textbf{M1} & \ding{55} & \ding{51} & \ding{51} & 33.04 & 20.04 & 60.59 & 92.25 \\
    \textbf{M2} & \ding{51} & \ding{55} & \ding{51} & 33.05 & 19.98 & 60.09 & 92.44 \\
    \textbf{M3} & \ding{51} & \ding{51} & \ding{55} & 32.88 & 20.10 & 60.24 & 92.24 \\
    \textbf{M4} & \ding{51} & \ding{51} & \ding{51} & 31.97 & 20.11 & 60.19 & 92.01 \\
    \textbf{M5} & \ding{51} & \ding{51} & \ding{51} & \textbf{33.12} & \textbf{20.67} & \textbf{60.64} & \textbf{92.54} \\
    \bottomrule
    \end{tabular}
    \vspace{1mm}
    \caption{Ablation study of \ours. \textbf{Cur}, \textbf{Pre} and \textbf{Mid} represent the $f_{cur}$, $f_{pre}$ and $f_{mid}$ input, respectively, in the MSI.}
    \label{tab:ablation-modules}
    \end{subtable}
% \hfill
    \vspace{2mm}
    
    \begin{subtable}[t]{0.48\textwidth}
    \centering
    \footnotesize
    \setlength{\tabcolsep}{7pt}
    \begin{tabular}{c|c|cccc}
    \toprule 
    Idx. & \makecell{Type of\\motion info.} & IoU~$\uparrow$ & mIoU~$\uparrow$ & $\overline{S_m}\uparrow$ & $\overline{S_s}\uparrow$ \\
    \midrule
    \textbf{I0} & -  & 31.49 & 20.30 & 58.33 & 91.71 \\
    \textbf{I1} & Raw Image  & 32.39 & 19.45 & 59.01 & 91.15 \\
    \textbf{I2} & Optical Flow  & 32.80 & 20.27 & 60.53 & 92.13 \\
    \textbf{I3} & Frame Diff.  & \textbf{33.12} & \textbf{20.67}  & \textbf{60.64} & \textbf{92.54} \\
    \bottomrule
    \end{tabular}
    \vspace{1mm}
    \caption{Effect of different types of motion information.}
    \label{tab:ablation-motion}
    \end{subtable}
    % \vspace{-2mm}
\caption{Ablation studies on \ours\ modules and motion information. Best results are \textbf{bolded}.}
\label{tab:ablation-studies}
\end{wrapfigure}

\noindent\textbf{Different combinations of \ours.} \label{para:aba-comb}\cref{tab:ablation-modules} presents the performance results of different combination of \ours's components for $N$=$1$. In \cref{tab:ablation-modules}, there are 6 different combinations: \textbf{M0} shows results from SurroundOcc~\cite{surroundOcc}, which represents the basic model without our method. \textbf{M1} means the model variant in which the part responsible for processing previous keyframes is removed, thereby excluding the input data \(F_{pre}^{1}\). \textbf{M2} refers to the model variant that omits the current feature \(F_{cur}\). \textbf{M3} indicates the model configuration that has \(F_{mid}^{1}\) removed. \textbf{M4} indicates that $I_{cur}$ is used to compute MHAM's query, while $I_{pre}$ and $I_{mid}$ are utilized to compute MHAM's key and value, which differs from the standard design. \textbf{M5} represents the full model with all components included.
% 验证了我们三种输入的必要性
\cref{tab:ablation-modules} clearly demonstrates that the removal of any single input from \ours\ module significantly reduces performance both in prediction accuracy and in temporal consistency. This validates the necessity of the three inputs. Furthermore, the comparison between \textbf{M4} and \textbf{M5} confirms that the cues provided by the previous keyframes and the intermediate frames are crucial for occupancy prediction.

\noindent\textbf{Impact of different types of motion information.} \label{para:motion-extracting} This experiment was conducted on the MFE module to investigate the effects of various types of motion information for $N=1$. The results are presented in \cref{tab:ablation-motion}. Specifically, \textbf{I0} served as the base model~\cite{surroundOcc} without using any motion information. \textbf{I1} employed raw intermediate frames as the input for the MFE. \textbf{I2} used optical flow~\cite{OpticalFlow} as the motion information input. \textbf{I3} used frame difference~\cite{Frame_difference} to capture motion information. It is clear that \textbf{I1} surpasses \textbf{I0} in terms of IoU metrics; however, it exhibits the lowest performance in mIoU, $\overline{S_m}$, and $\overline{S_s}$ metrics compared with \textbf{I1}, \textbf{I2}, and \textbf{I3}. This discrepancy is mainly because of the substantial amount of irrelevant information in the raw, intermediate frames, which complicates the extraction of motion features by the MFE. In addition, the results show that \textbf{I3} significantly outperforms \textbf{I2} in both IoU and mIoU metrics and slightly improves in $\overline{S_m}$ and $\overline{S_s}$ metrics. This indicates that frame difference more effectively captures sudden changes in a scene, such as the abrupt appearance of pedestrians or vehicles exiting intersections, while optical flow may experience delays in processing these sudden events. Furthermore, given the lightweight design of \ours, the frame difference method~\cite{Frame_difference} reduces data processing complexity by only processing simple differential data, thereby contributing to computing speed.

\begin{wrapfigure}[22]{l}{90mm}
    \centering
    \includegraphics[width=90mm]{assets/case_single_light.png}
    \caption{
    Several challenging scenarios are presented: pedestrians are partially occluded by vehicles in the first and second columns, and the road boundary appears visually obscure in the third column. Ours achieves more accurate predictions, while SOTA methods display significant artifacts.
    }
    \label{fig:case-extra}
\end{wrapfigure}

\noindent\textbf{Impact of different numbers of previous keyframes.}\label{para:ntrack} We conduct ablation experiments on $N$ to explore the performance of the model when $N$=$0$, $N$=$1$ and $N$=$2$. $N$=$0$ represents SurroundOcc~\cite{surroundOcc}, which does not use any previous keyframes. Detailed experiment results are documented in the supplementary material.

\begin{figure}[!t]
\centering
% \fbox{\rule{0pt}{2in} \rule{0.9\linewidth}{0pt}}
\includegraphics[width=\linewidth]{assets/case_big_light.png}
% \vspace{-7mm}
\caption{
Comparison under a T-junction scenario, where a pedestrian is partially and dynamically occluded in certain frames. Ours showcases robust predictions, with the pedestrian being consistently tracked, while SOTA methods show a flickering phenomenon.
}
\label{fig:case-study-big}
\vspace{-20pt}
\end{figure}

\subsection{Case analysis}
To visually evaluate the effectiveness of our method~(SurroundOcc+\ours), we compare it with the SOTA 3D VONs~\cite{surroundOcc} and the SOTA history-aware VONs~\cite{bevdet4d}.

\noindent\textbf{Temporal visualization case.} As shown in~\cref{fig:case-study-big} (Scene 277, Frames \#7-\#11), a pedestrian traversing the sidewalk parallel to the ego-motion trajectory is intermittently occluded by roadside vegetation. SurroundOcc~\cite{surroundOcc} exhibits severe instability in predictions (missing in Frames \#7/\#9), revealing fundamental limitations in temporal modeling. BEVDet4D-Occ~\cite{bevdet4d} alleviates this issue through data fusion but still suffers from occasional inconsistencies, such as detection dropout in Frame \#8. In contrast, our method completely eliminates flickering artifacts and maintains consistent detection across all occlusion states.

\noindent\textbf{Extra single frame visualization case.} ~\cref{fig:case-extra} highlights challenging scenarios: 
(i) Vehicle-pedestrian occlusion (Scene-0911 Frame \#15, Scene-0928 Frame \#14): Both SurroundOcc~\cite{surroundOcc} and BEVDet4D-Occ~\cite{bevdet4d} fail to recover the occluded pedestrian’s occupancy, while our method successfully localizes the target with precise geometry.
(ii) Curved road prediction (Scene-0923 Frame \#28): Our approach correctly anticipates the right-turn road geometry where baselines produce fragmented or erroneous occupancy, achieving superior shape consistency with real-world conditions.




\subsection{Overhead analysis}

For a fair comparison, all overhead analysis experiments are performed on a single NVIDIA L20 GPU.

\begin{figure}[htbp]
    \centering
    \begin{minipage}{0.48\textwidth}
        \footnotesize
        \begin{tabular}{r|ccc}
            \toprule
            Model & mIoU~$\uparrow$ & \makecell{Memory (MB)\\ Train~/~Test}~$\downarrow$ & Latency~$\downarrow$ \\
            \midrule
            FB-Occ~\cite{fb_occ} & 39.11 & 32,915~/~5,933 & 0.09s \\
            % SparseOcc~\cite{SparseOcc_Liu} & 30.10 & $>$49,140~/~7,147 & \cellcolor{Gray} 0.05s \\
            OPUS-L~\cite{opus} & 36.20 & OOM~/~10,579 & 0.16s \\
            OPUS-T~\cite{opus} & 33.20 & 48,532~/~6,711 & \textbf{0.03s} \\
            BEVDet4D-Occ~\cite{bevdet4d} & 39.30 & 22,833~/~4,689 & 0.26s \\
            \midrule
            ViewFormer+Ours & \textbf{41.30} & \textbf{16,619~/~4,687} & 0.12s \\
            \bottomrule
        \end{tabular}
        \vspace{2mm}
        \caption{Comparison of computational overhead. All models are benchmarked with ResNet-50 backbones. Our result (ViewFormer+\ours) in this table is measured for $N = 1$. OOM indicates out of CUDA memory. Best results are \textbf{bolded}.}
        \label{tab:efficiency}
    \end{minipage}\hfill
    \begin{minipage}{0.48\textwidth}
        \centering
        \includegraphics[width=\linewidth]{assets/bubble_0304.png}  % 替换成你的图片
        % \vspace{-8mm}
        \caption{Comparison of memory and latency overheads. Lower-left positions indicate superior performance with reduced memory consumption and faster inference. Large circles indicate better mIoU quality.}
        \label{fig:bubble}
    \end{minipage}
\end{figure}



As illustrated in \cref{tab:efficiency} and \cref{fig:bubble}, we conducted a comparative study to evaluate the computational overhead of our model against existing temporal methods~\cite{bevdet4d,opus,fb_occ}. The analysis focuses on GPU memory consumption during the training/testing phases and per-sample inference latency. The result shows that our method establishes an optimal accuracy-memory balance, achieving state-of-the-art mIoU while maintaining minimal GPU memory consumption alongside sustained computational efficiency that avoids runtime bottlenecks. For quantitative benchmarking, we compare two baseline frameworks:
\begin{itemize}
    \item ViewFormer on Occ3D: (i) Training memory: ViewFormer+\ours\ requires 16 GB of GPU memory, with the \ours\ module consuming only 0.22 GB, accounting for \textbf{1.4\%} of total usage; (ii) Inference latency: Full sample processing takes 0.1218s, where \ours\ contributes merely 0.0043s, accounting for \textbf{3.5\%} of total computation.
    \item SurroundOcc on nuScenes: (i) Training memory: SurroundOcc+\ours\ consumes 39 GB of GPU memory, with \ours\ occupying only 0.69 GB, which is \textbf{1.8\%} of total memory; (ii) Inference latency: Complete sample inference requires 0.9200s, while \ours\ takes 0.0065s, contributing to \textbf{0.7\%} of total latency.
\end{itemize}

These measurements confirm that our architecture introduces negligible computational overhead while delivering competitive performance.

\section{Conclusion}
We introduce a novel approach, \algo, to reduce human feedback requirements in preference-based reinforcement learning by leveraging vision-language models. While VLMs encode rich world knowledge, their direct application as reward models is hindered by alignment issues and noisy predictions. To address this, we develop a synergistic framework where limited human feedback is used to adapt VLMs, improving their reliability in preference labeling. Further, we incorporate a selective sampling strategy to mitigate noise and prioritize informative human annotations.

Our experiments demonstrate that this method significantly improves feedback efficiency, achieving comparable or superior task performance with up to 50\% fewer human annotations. Moreover, we show that an adapted VLM can generalize across similar tasks, further reducing the need for new human feedback by 75\%. These results highlight the potential of integrating VLMs into preference-based RL, offering a scalable solution to reducing human supervision while maintaining high task success rates. 

\section*{Impact Statement}
This work advances embodied AI by significantly reducing the human feedback required for training agents. This reduction is particularly valuable in robotic applications where obtaining human demonstrations and feedback is challenging or impractical, such as assistive robotic arms for individuals with mobility impairments. By minimizing the feedback requirements, our approach enables users to more efficiently customize and teach new skills to robotic agents based on their specific needs and preferences. The broader impact of this work extends to healthcare, assistive technology, and human-robot interaction. One possible risk is that the bias from human feedback can propagate to the VLM and subsequently to the policy. This can be mitigated by personalization of agents in case of household application or standardization of feedback for industrial applications. 

\clearpage
\bibliography{main}
\bibliographystyle{icml2025}


\clearpage
\clearpage
\renewcommand{\thefigure}{A\arabic{figure}}
\renewcommand{\thetable}{A\arabic{table}}
\renewcommand{\theequation}{A\arabic{equation}}
\setcounter{figure}{0}
\setcounter{table}{0}
\setcounter{equation}{0}

Our Appendix is organized as follows. First, we present the pseudocode for the key components of iGCT. We also include the proof for unit variance and boundary conditions in preconditioning iGCT's noiser. Next, we detail the training setups for our CIFAR-10 and ImageNet64 experiments. Additionally, we provide ablation studies on using guided synthesized images as data augmentation in image classification. Finally, we present more uncurated results comparing iGCT and CFG-EDM on inversion, editing and guidance, thoroughly of iGCT.

\vspace{-0.2cm}
\label{appendix:iGCT}
\section{Pseudocode for iGCT}
\vspace{-0.2cm}

iGCT is trained under a continuous-time scheduler similar to the one proposed by ECT \cite{ect}. Our noise sampling function follows a lognormal distribution, \(p(t) = \textit{LogNormal}(P_\textit{mean}, P_\textit{std})\), with \(P_\textit{mean}=-1.1\) and \( P_\textit{std}=2.0\). At training, the sampled noise is clamped at \(t_\text{min} = 0.002\) and \(t_\text{max} = 80.0\). Step function \(\Delta t (t)=\frac{t}{2^{\left\lfloor k/d \right\rfloor}}n(t)\), is used to compute the step size from the sampled noise \(t\), with \(k,d\) being the current training iteration and the number of iterations for halfing \(\Delta t\), and \(n(t) = 1 + 8 \sigma(-t)\) is a sigmoid adjusting function. 

In Guided Consistency Training, the guidance mask function determines whether the sampled noise \( t \) should be supervised for guidance training. With probability \( q(t) \in [0,1] \), the update is directed towards the target sample \( \boldsymbol{x}_0^{\text{tar}} \); otherwise, no guidance is applied. In practice, \( q(t) \) is higher in noisier regions and zero in low-noise regions, 
\begin{equation}
    q(t) = 0.9 \cdot \left( \text{clamp} \left( \frac{t - t_{\text{low}}}{t_{\text{high}} - t_{\text{low}}}, 0, 1 \right) \right)^2,
\end{equation}
where \( t_{\text{low}} = 11.0 \) and \( t_{\text{high}} = 14.3 \). For the range of guidance strength, we set \(w_\text{min} = 1\) and \(w_\text{max} = 15\). Guidance strengths are sampled uniformly at training, with \(w_\text{min} = 1\) means no guidance applied. 


\begin{algorithm}
\caption{Guided Consistency Training}
\label{alg:GCT}
\begin{algorithmic}[1]  % Adds line numbers
\setlength{\baselineskip}{0.9\baselineskip} % Adjust line spacing
\INPUT Dataset $\mathcal{D}$, weighting function $\lambda(t)$, noise sampling function $p(t)$, noise range $[t_\text{min}, t_\text{max}]$, step function $\Delta t(t)$, guidance mask function $q(t)$, guidance range $[w_\text{min}, w_\text{max}]$, denoiser $D_\theta$
\STATE \rule{0.96\textwidth}{0.45pt} 
\STATE Sample $(\boldsymbol{x}_0^{\text{src}}, c^{\text{src}}), (\boldsymbol{x}_0^{\text{tar}}, c^{\text{tar}}) \sim \mathcal{D}$ 
\STATE Sample noise $\boldsymbol{z} \sim \mathcal{N}(\boldsymbol{0},\mathbf{I})$, time step $t \sim p(t)$, and guidance weight $w \sim \mathcal{U}(w_\text{min}, w_\text{max})$
\STATE Clamp $t \leftarrow \text{clamp}(t,t_\text{min}, t_\text{max})$
\STATE Compute noisy sample: $\boldsymbol{x}_t = \boldsymbol{x}_0^{\text{src}} + t\boldsymbol{z}$
\STATE Sample $\rho \sim \mathcal{U}(0,1)$  
\vspace{0.3em}
\IF{$\rho > q(t)$}
    \STATE Compute step as normal CT: $\boldsymbol{x}_r = \boldsymbol{x}_t - \Delta t(t) \boldsymbol{z}$
    \STATE Set target class: $c \leftarrow c^{\text{src}}$
\ELSE
    \STATE Compute guided noise: $\boldsymbol{z}^* = (\boldsymbol{x}_t - \boldsymbol{x}_0^{\text{tar}}) / t$
    \STATE Compute guided step: $\boldsymbol{x}_r = \boldsymbol{x}_t - \Delta t(t) [w \boldsymbol{z}^* + (1-w)\boldsymbol{z}]$
    \STATE Set target class: $c \leftarrow c^{\text{tar}}$
\ENDIF
\vspace{0.3em} % Reduces extra vertical space before the loss line
\STATE Compute loss: 
\[
\mathcal{L}_\text{gct} = \lambda(t) \, d(D_{\theta}(\boldsymbol{x}_t, t, c, w), D_{{\theta}^-}(\boldsymbol{x}_r, r, c, w))
\]
\STATE Return $\mathcal{L}_\text{gct}$ 
\end{algorithmic}
\end{algorithm}



A \textit{noiser} trained under \textit{Inverse Consistency Training} maps an image to its latent noise in a single step. In contrast, DDIM Inversion requires multiple steps with a diffusion model to accurately produce an image's latent representation. Since the boundary signal is reversed, spreading from \( t_\text{max} \) down to \( t_\text{min} \), we design the importance weighting function \( \lambda'(t) \) to emphasize higher noise regions, defined as:
\begin{equation}
    \lambda'(t) = \frac{\Delta t (t)}{t_\text{max}},
\end{equation}
where the step size \( \Delta t (t) \) is proportional to the sampled noise level \(t\), and \( t_\text{max} \) is a constant that normalizes the scale of the inversion loss. The noise sampling function \( p(t) \) and the step function \( \Delta t (t) \) used in computing both \(\mathcal{L}_\text{gct}\) and \(\mathcal{L}_\text{ict}\) are the same.



\begin{algorithm}
\caption{Inverse Consistency Training}
\label{alg:iCT}
\begin{algorithmic}[1]  % Adds line numbers
\setlength{\baselineskip}{0.9\baselineskip} % Adjust line spacing
\INPUT Dataset $\mathcal{D}$, weighting function $\lambda'(t)$, noise sampling function $p(t)$, noise range $[t_\text{min}, t_\text{max}]$, step function $\Delta t(t)$, noiser $N_\varphi$
\STATE \rule{0.96\textwidth}{0.45pt} 
\STATE Sample $\boldsymbol{x}_0, c \sim \mathcal{D}$ 
\STATE Sample noise $\boldsymbol{z} \sim \mathcal{N}(\boldsymbol{0},\mathbf{I})$, time step $t \sim p(t)$
\STATE Clamp $t \leftarrow \text{clamp}(t,t_\text{min}, t_\text{max})$
\STATE Compute noisy sample: $\boldsymbol{x}_t = \boldsymbol{x}_0 + t\boldsymbol{z}$
\STATE Compute cleaner sample: $\boldsymbol{x}_r = \boldsymbol{x}_t - \Delta t(t) \boldsymbol{z}$
\vspace{0.3em} 
\STATE Compute loss: 
\[
\mathcal{L}_\text{ict} = \lambda'(t) \, d(N_{\varphi}(\boldsymbol{x}_r, r, c), D_{{\varphi}^-}(\boldsymbol{x}_t, t, c))
\]
\STATE Return $\mathcal{L}_\text{ict}$ 
\end{algorithmic}
\end{algorithm}

Together, iGCT jointly optimizes the two consistency objectives \(\mathcal{L}_\text{gct}, \mathcal{L}_\text{ict}\), and aligns the noiser and denoiser via a reconstruction loss, \(\mathcal{L}_\text{recon}\). To improve training efficiency, \(\mathcal{L}_\text{recon}\) is computed every \(i_\text{skip}\), reducing the computational cost of back-propagation through both the weights of the \textit{denoiser} \(\theta\) and the \textit{noiser} \(\varphi\). Alg. \ref{alg:iGCT} provides an overview of iGCT. 

\begin{algorithm}
\caption{iGCT}
\label{alg:iGCT}
\begin{algorithmic}[1]  % Adds line numbers
\setlength{\baselineskip}{0.9\baselineskip} % Adjust line spacing
\INPUT Dataset $\mathcal{D}$, learning rate $\eta$, weighting functions $\lambda'(t), \lambda(t), \lambda_{\text{recon}}$, noise sampling function $p(t)$, noise range $[t_\text{min}, t_\text{max}]$, step function $\Delta t(t)$, guidance mask function $q(t)$, guidance range $[w_\text{min}, w_\text{max}]$, reconstruction skip iters $i_\text{skip}$, models $N_\varphi, D_\theta$
\STATE \rule{0.9\textwidth}{0.45pt}  % Horizontal line to separate input from main algorithm
\STATE \textbf{Init:} $\theta, \varphi$, $\text{Iters} = 0$
\REPEAT
\STATE Do guided consistency training 
\[
\mathcal{L}_\text{gct}(\theta;\mathcal{D},\lambda(t),p(t),t_\text{min},t_\text{max},\Delta t(t),q(t),w_\text{min},w_\text{max})
\]
\STATE Do inverse consistency training
\[
\mathcal{L}_\text{ict}(\varphi;\mathcal{D},\lambda'(t),p(t),t_\text{min},t_\text{max},\Delta t(t))
\]
\IF{$(\text{Iters} \ \% \ i_\text{skip}) == 0$}
\STATE Compute reconstruction loss
\[
\mathcal{L}_\text{recon} = d(D_{\theta}(N_{\varphi}(\boldsymbol{x}_0,t_\text{min},c),t_\text{max},c,0), \boldsymbol{x}_0)
\]
\ELSE
\STATE \[
\mathcal{L}_\text{recon} = 0
\]
\ENDIF
\STATE Compute total loss: 
\[
\mathcal{L} = \mathcal{L}_\text{gct} + \mathcal{L}_\text{ict} + \lambda_{\text{recon}}\mathcal{L}_\text{recon}
\]
\STATE $\theta \leftarrow \theta - \eta \nabla_{\theta} \mathcal{L}, \ \varphi \leftarrow \varphi - \eta \nabla_{\varphi} \mathcal{L}$
\STATE $\text{Iters} = \text{Iters} + 1$
\UNTIL{$\Delta t \rightarrow dt$}
\end{algorithmic}
\end{algorithm}



\vspace{-0.3cm}
\section{Preconditioning for Noiser}
\label{appendix:unit-variance}
\vspace{-0.1cm}

We define 
\begin{equation}
    N_{\varphi}(\boldsymbol{x}_t, t, c) = c_\text{skip}(t) \, \boldsymbol{x}_t + c_\text{out}(t) \, F_{\varphi}(c_\text{in}(t) \, \boldsymbol{x}_t, t, c),
\end{equation}
where \( c_\text{in}(t) = \frac{1}{\sqrt{t^2 + \sigma_\text{data}^2}} \), \( c_\text{skip}(t) = 1 \), and \( c_\text{out}(t) = t_\text{max} - t \). This setup naturally serves as a boundary condition. Specifically:

\begin{itemize}
    \item When \( t = 0 \),
    \begin{equation}
        c_\text{out}(0) = t_\text{max} \gg c_\text{skip}(0) = 1,
    \end{equation}
    emphasizing that the model's noise prediction dominates the residual information given a relatively clean sample.

    \item When \( t = t_\text{max} \),
    \begin{equation}
        N_{\varphi}(\boldsymbol{x}_{t_\text{max}}, t_\text{max}, c) = \boldsymbol{x}_{t_\text{max}},
    \end{equation}
    satisfying the condition that \( N_{\varphi} \) outputs \( \boldsymbol{x}_{t_\text{max}} \) at the maximum time step.
\end{itemize}



We show that these preconditions ensure unit variance for the model’s input and target. First, \(\text{Var}_{\boldsymbol{x}_0, z}[\boldsymbol{x}_t] = \sigma_\text{data}^2 + t^2\), so setting \( c_\text{in}(t) = \frac{1}{\sqrt{\sigma_\text{data}^2 + t^2}} \) normalizes the input variance to 1. Second, we require the training target to have unit variance. Given the noise target for \( N_{\varphi} \) is \(\boldsymbol{x}_{t_\text{max}} = \boldsymbol{x}_0 + t_\text{max} z\), by moving of terms, the effective target for \( F_{\varphi} \) can be written as,
\begin{equation}
    \frac{\boldsymbol{x}_{t_\text{max}} - c_\text{skip}(t)\boldsymbol{x}_{t}}{c_\text{out}(t)}
\end{equation}
When \(c_\text{skip}(t) = 1\), \(c_\text{out}(t) = t_\text{max} - t \), we verify that target is unit variance,
\begin{align}
    &\text{Var}_{\boldsymbol{x}_0, \boldsymbol{z}} \left[ \frac{\boldsymbol{x}_{t_\text{max}} - c_\text{skip}(t) \, \boldsymbol{x}_{t}}{c_\text{out}(t)} \right] \\ \notag
    = \ &\text{Var}_{\boldsymbol{x}_0, \boldsymbol{z}} \left[ \frac{\boldsymbol{x}_0 + t_\text{max} \, \boldsymbol{z} - (\boldsymbol{x}_0 + t \, \boldsymbol{z})}{t_\text{max} - t} \right] \notag \\
    = \ &\text{Var}_{\boldsymbol{x}_0, \boldsymbol{z}} \left[ \frac{(t_\text{max} - t) \, \boldsymbol{z}}{t_\text{max} - t} \right] \notag \\
    = \ &\text{Var}_{\boldsymbol{x}_0, \boldsymbol{z}}[\boldsymbol{z}] \notag \\
    = \ &1. \notag
\end{align}

\vspace{-0.3cm}
\section{Baselines \& Training Details}
\label{appendix:bs-config}
\vspace{-0.1cm}

\begin{figure}[t!]  
    \centering
    \begin{subfigure}[b]{0.33\textwidth}
    \includegraphics[width=\textwidth]{fig/appendix/guidance_embed.pdf} 
        \caption{Guidance embedding.}
    \end{subfigure}
    \hfill
    \begin{subfigure}[b]{0.33\textwidth}
    \includegraphics[width=\textwidth]{fig/appendix/adm_arch.pdf} 
        \caption{NCSN++ architecture.}
    \end{subfigure}
    \hfill
    \begin{subfigure}[b]{0.33\textwidth}
    \includegraphics[width=\textwidth]{fig/appendix/ncsnpp_arch.pdf} 
        \caption{ADM architecture.}
    \end{subfigure}
    \hfill
    \caption{Design of guidance embedding, and conditioning under different network architectures.}
    \vspace{-1em}
    \label{fig:guidance_conditioning}
\end{figure}

For our diffusion model baseline, we follow \textit{EDM}'s official repository (\href{https://github.com/NVlabs/edm}{https://github.com/NVlabs/edm}) instructions for training and set \textit{label\_dropout} to 0.1 to optimize a CFG (classifier-free guided) DM. We will use this DM as the teacher model for our consistency model baseline via consistency distillation. 

The consistency model baseline \textit{Guided CD} is trained with a discrete-time schedule. We set the discretization steps \( N = 18 \) and use a Heun ODE solver to predict update directions based on the CFG EDM, as in \cite{song2023consistency}. Following \cite{luo2023latent}, we modify the model's architecture and iGCT's denoiser to accept guidance strength \(w\) by adding an extra linear layer. See the detailed architecture design for guidance conditioning of consistency model in Fig. \ref{fig:guidance_conditioning}. A range of guidance scales \(w \in [1,15]\) is uniformly sampled at training. Following \cite{song2023improved}, we replace LPIPS by Pseudo-Huber loss, with \(c=0.03 \) determining the breadth of the smoothing section between L1 and L2. See Table \ref{tab:training_configs} for a summary of the training configurations for our baseline models.


\begin{table}[t!]
\centering
\renewcommand{\arraystretch}{1.3} % Adjust vertical spacing
\small % Reduce text size
\caption{Summary of training configurations for baseline models.}
\begin{tabular}{lccc}
\toprule
\multirow{2}{*}{} & \multicolumn{2}{c}{\textbf{CIFAR-10}} & \textbf{ImageNet64}  \\
                  & EDM & Guided-CD & EDM \\
\midrule
\multicolumn{4}{l}{\textbf{\small Arch. config.}} \\
\hline
model arch.        & NCSN++ & NCSN++ & ADM     \\
channels mult.     & 2,2,2  & 2,2,2  & 1,2,3,4 \\
UNet size          & 56.4M  & 56.4M  & 295.9M  \\
\midrule
\multicolumn{4}{l}{\textbf{\small Training config.}} \\
\hline
lr             & 1e-3  & 4e-4  & 2e-4 \\
batch          & 512   & 512   & 4096 \\
dropout        & 0.13  & 0     & 0.1 \\
label dropout  & 0.1   & (n.a.) & 0.1 \\
loss           & L2    & Huber & L2    \\
training iterations & 390k  & 800k  & 800K \\
\bottomrule
\end{tabular}
\label{tab:training_configs}
\end{table}


\begin{table}[t!]
\centering
\renewcommand{\arraystretch}{1.3} % Adjust vertical spacing
\small % Reduce text size
\caption{Summary of training configurations for iGCT.}
\begin{tabular}{lcc}
\toprule
\multirow{2}{*}{} & \textbf{CIFAR-10} & \textbf{ImageNet64}  \\
                  & iGCT & iGCT \\
\midrule
\multicolumn{3}{l}{\textbf{\small Arch. config.}} \\
\hline
model arch.        & NCSN++ & ADM \\
channels mult.     & 2,2,2  & 1,2,2,3 \\
UNet size          & 56.4M  & 182.4M \\ 
Total size         & 112.9M & 364.8M \\ 
\midrule
\multicolumn{3}{l}{\textbf{\small Training config.}} \\
\hline
lr              & 1e-4 & 1e-4 \\
batch           & 1024 & 1024 \\
dropout            & 0.2 & 0.3 \\
loss               & Huber   & Huber \\
\(c\)                  & 0.03    &  0.06 \\
\(d\)                  & 40k     &  40k \\
\( P_\textit{mean} \) & -1.1 &  -1.1 \\
\( P_\textit{std} \) &  2.0  &  2.0  \\
\( \lambda_{\text{recon}} \) & 2e-5 & \parbox[t]{3.5cm}{\centering 2e-5, (\(\leq\) 180k)\\ 4e-5, (\(\leq\) 200k)\\ 6e-5, (\(\leq\) 260k) } \\  
\( i_{\text{skip}} \)        & 10 &  10 \\  
training iterations & 360k &  260k \\
\bottomrule
\end{tabular}
\label{tab:igct_training_configs}
\end{table}  

\begin{figure*}[t] 
    \centering
    \includegraphics[width=1.0\textwidth]{fig/appendix/inversion_collapse.pdf} 
    \caption{Inversion collapse observed during training on ImageNet64. The left image shows the input data. The middle image depicts the inversion collapse that occurred at iteration 220k, where leakage of signals in the noise latent can be visualized. The right image shows the inversion results at iteration 220k after appropriately increasing $\lambda_{\text{recon}}$ to 6e-5. The inversion images are generated by scaling the model's outputs by $1/80$, i.e., $ 1/t_\text{max}$, then clipping the values to the range [-3, 3] before denormalizing them to the range [0, 255]. }
    \vspace{-1.5em}
    \label{fig:inversion_collpase}
\end{figure*}

iGCT is trained with a continuous-time scheduler inspired by ECT \cite{ect}. To rigorously assess its independence from diffusion-based models, iGCT is trained from scratch rather than fine-tuned from a pre-trained diffusion model. Consequently, the training curriculum begins with an initial diffusion training stage, followed by consistency training with the step size halved every \(d\) iterations. In practice, we adopt the same noise sampling distribution \(p(t)\), same step function \(\Delta t (t) \), and same distance metric \( d(\cdot, \cdot) \) for both guided consistency training and inverse consistency training. 

For CIFAR-10, iGCT adopts the same UNet architecture as the baseline models. However, the overall model size is doubled, as iGCT comprises two UNets: one for the denoiser and one for the noiser. The Pseudo-Huber loss is employed as the distance metric, with a constant parameter \( c = 0.03 \). Consistency training is organized into nine stages, each comprising 400k iterations with the step size halved from the last stage. We found that training remains stable when the reconstruction weight \( \lambda_{\text{recon}} \) is fixed at \( 2 \times 10^{-5} \) throughout the entire training process.
 
For ImageNet64, iGCT employs a reduced ADM architecture \cite{dhariwal2021diffusionmodelsbeatgans} with smaller channel sizes to address computational constraints. A higher dropout rate and Pseudo-Huber loss with \( c = 0.06 \) is used, following prior works \cite{ect,song2023improved}. During our experiments, we observed that training on ImageNet64 is sensitive to the reconstruction weight. Keeping \(\lambda_{\text{recon}}\) fixed throughout training leads to inversion collapse, with significant signal leaked to the latent noise (see Fig. \ref{fig:inversion_collpase}). We found that increasing \(\lambda_{\text{recon}}\) to \( 4 \times 10^{-5} \) at iteration 1800 and to \( 6 \times 10^{-5} \) at iteration 2000 effectively stabilizes training and prevents collapse. This suggests that the reconstruction loss serves as a regularizer for iGCT. Additionally, we observed diminishing returns when training exceeded 240k iterations, leading us to stop at 260k iterations for our experiments. These findings indicate that alternative training strategies, such as framing iGCT as a multi-task learning problem \cite{kendall2018multi,liu2019loss}, and conducting a more sophisticated analysis of loss weighting, may be necessary to enhance stability and improve convergence. See Table \ref{tab:igct_training_configs} for a summary of the training configurations for iGCT.



\begin{table}[t]
\caption{Comparison of GPU hours across the methods used in our experiments on CIFAR-10.}
\centering
\begin{tabular}{|l|c|}
\hline
\textbf{Methods} & \textbf{A100 (40G) GPU hours} \\ \hline
CFG-EDM \cite{karras2022elucidating} & 312 \\ \hline
Guided-CD \cite{song2023consistency} & 3968 \\ \hline
iGCT (ours) & 2032 \\ \hline
\end{tabular}
\label{table:compute_resources}
\end{table}



\begin{figure*}[t!]  
    \centering
    \begin{subfigure}[b]{0.33\textwidth}
    \includegraphics[width=\textwidth]{fig/cls_exp_w1.png} 
        \caption{Accuracy on various ratios of augmented data, guidance scale w=1.}
    \end{subfigure}
    \begin{subfigure}[b]{0.33\textwidth}
    \includegraphics[width=\textwidth]{fig/cls_exp_w3.png} 
        \caption{Accuracy on various ratios of augmented data, guidance scale w=3.}
    \end{subfigure}
    \begin{subfigure}[b]{0.33\textwidth}
    \includegraphics[width=\textwidth]{fig/cls_exp_w5.png} 
        \caption{Accuracy on various ratios of augmented data, guidance scale w=5.}
    \end{subfigure}
    \begin{subfigure}[b]{0.33\textwidth}
    \includegraphics[width=\textwidth]{fig/cls_exp_w7.png} 
        \caption{Accuracy on various ratios of augmented data, guidance scale w=7.}
    \end{subfigure}
    \begin{subfigure}[b]{0.33\textwidth}
    \includegraphics[width=\textwidth]{fig/cls_exp_w9.png} 
        \caption{Accuracy on various ratios of augmented data, guidance scale w=9.}
    \end{subfigure}
    \caption{Comparison of synthesized methods, CFG-EDM vs iGCT, used for data augmentation in image classification. iGCT consistently improves accuracy. Conversely, augmentation data synthesized from CFG-EDM offers only limited gains.}
    \vspace{-1.5em}
    \label{fig:cls_results}
\end{figure*}


\vspace{-0.1cm}
\section{Application: Data Augmentation Under Different Guidance}
\vspace{-0.2cm}

In this section, we show the effectiveness of data augmentation with diffusion-based models, CFG-EDM and iGCT, across varying guidance scales for image classification on CIFAR-10 \cite{article}. High quality data augmentation has been shown to enhance classification performance \cite{yang2023imagedataaugmentationdeep}. Under high guidance, augmentation data generated from iGCT consistently improves accuracy. Conversely, augmentation data synthesized from CFG-EDM offers only limited gains. We describe the ratios of real to synthesized data, the classifier architecture, and the training setup in the following. 

\noindent{\bf Training Details.} We conduct classification experiments trained on six different mixtures of augmented data synthesized by iGCT and CFG-EDM: \(0\%\), \(20\%\), \(40\%\), \(80\%\), and \(100\%\). The ratio represents \(\textit{synthesized data} / \textit{real data}\). For example, \(0\%\) indicates that the training and validation sets contain only 50k of real samples from CIFAR-10, and \(20\%\) includes 50k real \textit{and} 10k synthesized samples. In terms of guidance scales, we choose \(w=1,3,5,7,9\) to synthesize the augmented data using iGCT and CFG-EDM. 
The augmented dataset is split 80/20 for training and validation. For testing, the model is evaluated on the CIFAR-10 test set with 10k samples and ground truth labels. 

The standard ResNet-18 \cite{he2015deepresiduallearningimage} is used to train on all different augmented datasets. All models are trained for 250 epochs, with batch size 64, using an Adam optimizer \cite{kingma2017adammethodstochasticoptimization}. For each augmentation dataset, we train the model six times under different seeds and report the average classification accuracy.

\noindent{\bf Results.} The classifier's accuracy, trained on augmented data synthesized by CFG-EDM and iGCT, is shown in Fig. \ref{fig:cls_results}. With \(w=1\) (no guidance), both iGCT and CFG-EDM provide comparable performance boosts. As guidance scale increases, iGCT shows more significant improvements than CFG-EDM. At high guidance and augmentation ratios, performance drops, but this effect occurs later for iGCT (e.g., at \(100\%\) augmentation and \(w=9\)), while CFG-EDM stops improving accuracy at \(w=7\). This experiment highlights the importance of high-quality data under high guidance, with iGCT outperforming CFG-EDM in data quality.

\section{Uncurated Results}
In this section, we present additional qualitative results to highlight the performance of our proposed iGCT method compared to the multi-step EDM baseline. These visualizations include both inversion and guidance tasks across the CIFAR-10 and ImageNet64 datasets. The results demonstrate iGCT's ability to maintain competitive quality with significantly fewer steps and minimal artifacts, showcasing the effectiveness of our approach.

\subsection{Inversion Results}
We provide additional visualization of the latent noise on both CIFAR-10 and ImageNet64 datasets. Fig. \ref{fig:CIFAR-10_inversion_reconstruction} and Fig. \ref{fig:im64_inversion_reconstruction} compare our 1-step iGCT with the multi-step EDM on inversion and reconstruction.  

\subsection{Editing Results}
In this section, we dump more uncurated editing results on ImageNet64's subgroups mentioned in Sec. \ref{sec:image-editing}. Fig. \ref{fig:im64_edit_1}--\ref{fig:im64_edit_4} illustrate a comparison between our 1-step iGCT and the multi-step EDM approach.

\subsection{Guidance Results}
In Section \ref{sec:guidance}, we demonstrated that iGCT provides a guidance solution without introducing the high-contrast artifacts commonly observed in CFG-based methods. Here, we present additional uncurated results on CIFAR-10 and ImageNet64. For CIFAR-10, iGCT achieves competitive performance compared to the baseline diffusion model, which requires multiple steps for generation. See Figs. \ref{fig:CIFAR-10_guided_1}--\ref{fig:CIFAR-10_guided_10}. For ImageNet64, although the visual quality of iGCT's generated images falls slightly short of expectations, this can be attributed to the smaller UNet architecture used—only 61\% of the baseline model size—and the need for a more robust training curriculum to prevent collapse, as discussed in Section \ref{appendix:bs-config}. Nonetheless, even at higher guidance levels, iGCT maintains style consistency, whereas CFG-based methods continue to suffer from pronounced high-contrast artifacts. See Figs. \ref{fig:im64_guided_1}--\ref{fig:im64_guided_4}.


\begin{figure*}[t]
    \centering
    \begin{subfigure}{0.48\textwidth}
        \centering
        \includegraphics[width=\linewidth]{fig/appendix/recon_c10_data.png}
        \caption{CIFAR-10: Original data}
    \end{subfigure}
    \begin{subfigure}{0.48\textwidth}
        \centering
        \includegraphics[width=\linewidth]{fig/appendix/recon_im64_data.png}
        \caption{ImageNet64: Original data}
    \end{subfigure}

    \begin{subfigure}{0.48\textwidth}
        \centering
        \includegraphics[width=\linewidth]{fig/appendix/inv_c10_edm.png}
    \end{subfigure}
    \begin{subfigure}{0.48\textwidth}
        \centering
        \includegraphics[width=\linewidth]{fig/appendix/inv_im64_edm.png}
    \end{subfigure}

    \begin{subfigure}{0.48\textwidth}
        \centering
        \includegraphics[width=\linewidth]{fig/appendix/recon_c10_edm.png}
        \caption{CIFAR-10: Inversion + reconstruction, EDM (18 NFE)}
    \end{subfigure}
    \begin{subfigure}{0.48\textwidth}
        \centering
        \includegraphics[width=\linewidth]{fig/appendix/recon_im64_edm.png}
        \caption{ImageNet64: Inversion + reconstruction, EDM (18 NFE)}
    \end{subfigure}

    \begin{subfigure}{0.48\textwidth}
        \centering
        \includegraphics[width=\linewidth]{fig/appendix/inv_c10_igct.png}
    \end{subfigure}
    \begin{subfigure}{0.48\textwidth}
        \centering
        \includegraphics[width=\linewidth]{fig/appendix/inv_im64_igct.png}
    \end{subfigure}

    \begin{subfigure}{0.48\textwidth}
        \centering
        \includegraphics[width=\linewidth]{fig/appendix/recon_c10_igct.png}
        \caption{CIFAR-10: Inversion + reconstruction, iGCT (1 NFE)}
    \end{subfigure}
    \begin{subfigure}{0.48\textwidth}
        \centering
        \includegraphics[width=\linewidth]{fig/appendix/recon_im64_igct.png}
        \caption{ImageNet64: Inversion + reconstruction, iGCT (1 NFE)}
    \end{subfigure}

    \caption{Comparison of inversion and reconstruction for CIFAR-10 (left) and ImageNet64 (right).}
    \label{fig:comparison_CIFAR-10_imagenet64}
\end{figure*}




\begin{figure*}[t]
    \centering

    % Left column: corgi -> golden retriever
    \begin{minipage}{0.48\textwidth}
        \centering
        \begin{subfigure}{0.48\textwidth}
            \includegraphics[width=\linewidth]{fig/appendix_edit_igct/src_corgi.png}
            \caption{Original: "corgi"}
        \end{subfigure}

        \begin{subfigure}{0.48\textwidth}
            \includegraphics[width=\linewidth]{fig/appendix_edit_edm/w=0_src_corgi_tar_golden_retriever.png}
            \caption{EDM (18 NFE), w=1}
        \end{subfigure}
        \begin{subfigure}{0.48\textwidth}
            \includegraphics[width=\linewidth]{fig/appendix_edit_edm/w=6_src_corgi_tar_golden_retriever.png}
            \caption{EDM (18 NFE), w=7}
        \end{subfigure}
        \begin{subfigure}{0.48\textwidth}
            \includegraphics[width=\linewidth]{fig/appendix_edit_igct/w=6_src_corgi_tar_golden_retriever.png}
            \caption{iGCT (1 NFE), w=7}
        \end{subfigure}
        \begin{subfigure}{0.48\textwidth}
            \includegraphics[width=\linewidth]{fig/appendix_edit_igct/w=0_src_corgi_tar_golden_retriever.png}
            \caption{iGCT (1 NFE), w=1}
        \end{subfigure}

        \caption{ImageNet64: "corgi" $\rightarrow$ "golden retriever"}
        \label{fig:im64_edit_1}
    \end{minipage}
    \hfill
    % Right column: zebra -> horse
    \begin{minipage}{0.48\textwidth}
        \centering
        \begin{subfigure}{0.48\textwidth}
            \includegraphics[width=\linewidth]{fig/appendix_edit_igct/src_zebra.png}
            \caption{Original: "zebra"}
        \end{subfigure}

        \begin{subfigure}{0.48\textwidth}
            \includegraphics[width=\linewidth]{fig/appendix_edit_edm/w=0_src_zebra_tar_horse.png}
            \caption{EDM (18 NFE), w=1}
        \end{subfigure}
        \begin{subfigure}{0.48\textwidth}
            \includegraphics[width=\linewidth]{fig/appendix_edit_edm/w=6_src_zebra_tar_horse.png}
            \caption{EDM (18 NFE), w=7}
        \end{subfigure}
        \begin{subfigure}{0.48\textwidth}
            \includegraphics[width=\linewidth]{fig/appendix_edit_igct/w=0_src_zebra_tar_horse.png}
            \caption{iGCT (1 NFE), w=1}
        \end{subfigure}
        \begin{subfigure}{0.48\textwidth}
            \includegraphics[width=\linewidth]{fig/appendix_edit_igct/w=6_src_zebra_tar_horse.png}
            \caption{iGCT (1 NFE), w=7}
        \end{subfigure}

        \caption{ImageNet64: "zebra" $\rightarrow$ "horse"}
        \label{fig:im64_edit_2}
    \end{minipage}

\end{figure*}

\begin{figure*}[t]
    \centering

    % Left column: broccoli -> cauliflower
    \begin{minipage}{0.48\textwidth}
        \centering
        \begin{subfigure}{0.48\textwidth}
            \includegraphics[width=\linewidth]{fig/appendix_edit_igct/src_broccoli.png}
            \caption{Original: "broccoli"}
        \end{subfigure}

        \begin{subfigure}{0.48\textwidth}
            \includegraphics[width=\linewidth]{fig/appendix_edit_edm/w=0_src_broccoli_tar_cauliflower.png}
            \caption{EDM (18 NFE), w=1}
        \end{subfigure}
        \begin{subfigure}{0.48\textwidth}
            \includegraphics[width=\linewidth]{fig/appendix_edit_edm/w=6_src_broccoli_tar_cauliflower.png}
            \caption{EDM (18 NFE), w=7}
        \end{subfigure}
        \begin{subfigure}{0.48\textwidth}
            \includegraphics[width=\linewidth]{fig/appendix_edit_igct/w=0_src_broccoli_tar_cauliflower.png}
            \caption{iGCT (1 NFE), w=1}
        \end{subfigure}
        \begin{subfigure}{0.48\textwidth}
            \includegraphics[width=\linewidth]{fig/appendix_edit_igct/w=6_src_broccoli_tar_cauliflower.png}
            \caption{iGCT (1 NFE), w=7}
        \end{subfigure}

        \caption{ImageNet64: "broccoli" $\rightarrow$ "cauliflower"}
        \label{fig:im64_edit_3}
    \end{minipage}
    \hfill
    % Right column: jaguar -> tiger
    \begin{minipage}{0.48\textwidth}
        \centering
        \begin{subfigure}{0.48\textwidth}
            \includegraphics[width=\linewidth]{fig/appendix_edit_igct/src_jaguar.png}
            \caption{Original: "jaguar"}
        \end{subfigure}

        \begin{subfigure}{0.48\textwidth}
            \includegraphics[width=\linewidth]{fig/appendix_edit_edm/w=0_src_jaguar_tar_tiger.png}
            \caption{EDM (18 NFE), w=1}
        \end{subfigure}
        \begin{subfigure}{0.48\textwidth}
            \includegraphics[width=\linewidth]{fig/appendix_edit_edm/w=6_src_jaguar_tar_tiger.png}
            \caption{EDM (18 NFE), w=7}
        \end{subfigure}
        \begin{subfigure}{0.48\textwidth}
            \includegraphics[width=\linewidth]{fig/appendix_edit_igct/w=0_src_jaguar_tar_tiger.png}
            \caption{iGCT (1 NFE), w=1}
        \end{subfigure}
        \begin{subfigure}{0.48\textwidth}
            \includegraphics[width=\linewidth]{fig/appendix_edit_igct/w=6_src_jaguar_tar_tiger.png}
            \caption{iGCT (1 NFE), w=7}
        \end{subfigure}

        \caption{ImageNet64: "jaguar" $\rightarrow$ "tiger"}
        \label{fig:im64_edit_4}
    \end{minipage}

\end{figure*}






\begin{figure*}[b]
    \centering
    % First image
    \begin{subfigure}{0.25\textwidth}
        \includegraphics[width=\linewidth]{fig/appendix_edm/0_0.0_middle_4x4_grid.png}
        \caption{CFG-EDM (18 NFE), w=1.0}
    \end{subfigure}
    \begin{subfigure}{0.25\textwidth}
        \includegraphics[width=\linewidth]{fig/appendix_edm/0_6.0_middle_4x4_grid.png}
        \caption{CFG-EDM (18 NFE), w=7.0}
    \end{subfigure}
    \begin{subfigure}{0.25\textwidth}
        \includegraphics[width=\linewidth]{fig/appendix_edm/0_12.0_middle_4x4_grid.png}
        \caption{CFG-EDM (18 NFE), w=13.0}
    \end{subfigure}
    \begin{subfigure}{0.25\textwidth}
        \includegraphics[width=\linewidth]{fig/appendix_igct/0_0.0_middle_4x4_grid.png}
        \caption{iGCT (1 NFE), w=1.0}
    \end{subfigure}
    \begin{subfigure}{0.25\textwidth}
        \includegraphics[width=\linewidth]{fig/appendix_igct/0_6.0_middle_4x4_grid.png}
        \caption{iGCT (1 NFE), w=7.0}
    \end{subfigure}
    % Third image
    \begin{subfigure}{0.25\textwidth}
        \includegraphics[width=\linewidth]{fig/appendix_igct/0_12.0_middle_4x4_grid.png}
        \caption{iGCT (1 NFE), w=13.0}
    \end{subfigure}
    \caption{CIFAR-10 "airplane"}
    \label{fig:CIFAR-10_guided_1}
\end{figure*}
\begin{figure*}[t]
    \centering
    % First image
    \begin{subfigure}{0.25\textwidth}
        \includegraphics[width=\linewidth]{fig/appendix_edm/1_0.0_middle_4x4_grid.png}
        \caption{CFG-EDM (18 NFE), w=1.0}
    \end{subfigure}
    \begin{subfigure}{0.25\textwidth}
        \includegraphics[width=\linewidth]{fig/appendix_edm/1_6.0_middle_4x4_grid.png}
        \caption{CFG-EDM (18 NFE), w=7.0}
    \end{subfigure}
    \begin{subfigure}{0.25\textwidth}
        \includegraphics[width=\linewidth]{fig/appendix_edm/1_12.0_middle_4x4_grid.png}
        \caption{CFG-EDM (18 NFE), w=13.0}
    \end{subfigure}
    \begin{subfigure}{0.25\textwidth}
        \includegraphics[width=\linewidth]{fig/appendix_igct/1_0.0_middle_4x4_grid.png}
        \caption{iGCT (1 NFE), w=1.0}
    \end{subfigure}
    % Second image
    \begin{subfigure}{0.25\textwidth}
        \includegraphics[width=\linewidth]{fig/appendix_igct/1_6.0_middle_4x4_grid.png}
        \caption{iGCT (1 NFE), w=7.0}
    \end{subfigure}
    % Third image
    \begin{subfigure}{0.25\textwidth}
        \includegraphics[width=\linewidth]{fig/appendix_igct/1_12.0_middle_4x4_grid.png}
        \caption{iGCT (1 NFE), w=13.0}
    \end{subfigure}
    \caption{CIFAR-10 "car"}
    \label{fig:CIFAR-10_guided_2}
\end{figure*}
\begin{figure*}[t]
    \centering
    % First image
    \begin{subfigure}{0.25\textwidth}
        \includegraphics[width=\linewidth]{fig/appendix_edm/2_0.0_middle_4x4_grid.png}
        \caption{CFG-EDM (18 NFE), w=1.0}
    \end{subfigure}
    \begin{subfigure}{0.25\textwidth}
        \includegraphics[width=\linewidth]{fig/appendix_edm/2_6.0_middle_4x4_grid.png}
        \caption{CFG-EDM (18 NFE), w=7.0}
    \end{subfigure}
    \begin{subfigure}{0.25\textwidth}
        \includegraphics[width=\linewidth]{fig/appendix_edm/2_12.0_middle_4x4_grid.png}
        \caption{CFG-EDM (18 NFE), w=13.0}
    \end{subfigure}
    \begin{subfigure}{0.25\textwidth}
        \includegraphics[width=\linewidth]{fig/appendix_igct/2_0.0_middle_4x4_grid.png}
        \caption{iGCT (1 NFE), w=1.0}
    \end{subfigure}
    % Second image
    \begin{subfigure}{0.25\textwidth}
        \includegraphics[width=\linewidth]{fig/appendix_igct/2_6.0_middle_4x4_grid.png}
        \caption{iGCT (1 NFE), w=7.0}
    \end{subfigure}
    % Third image
    \begin{subfigure}{0.25\textwidth}
        \includegraphics[width=\linewidth]{fig/appendix_igct/2_12.0_middle_4x4_grid.png}
        \caption{iGCT (1 NFE), w=13.0}
    \end{subfigure}
    \caption{CIFAR-10 "bird"}
    \label{fig:CIFAR-10_guided_3}
\end{figure*}
\begin{figure*}[t]
    \centering
    % First image
    \begin{subfigure}{0.25\textwidth}
        \includegraphics[width=\linewidth]{fig/appendix_edm/3_0.0_middle_4x4_grid.png}
        \caption{CFG-EDM (18 NFE), w=1.0}
    \end{subfigure}
    \begin{subfigure}{0.25\textwidth}
        \includegraphics[width=\linewidth]{fig/appendix_edm/3_6.0_middle_4x4_grid.png}
        \caption{CFG-EDM (18 NFE), w=7.0}
    \end{subfigure}
    \begin{subfigure}{0.25\textwidth}
        \includegraphics[width=\linewidth]{fig/appendix_edm/3_12.0_middle_4x4_grid.png}
        \caption{CFG-EDM (18 NFE), w=13.0}
    \end{subfigure}
    \begin{subfigure}{0.25\textwidth}
        \includegraphics[width=\linewidth]{fig/appendix_igct/3_0.0_middle_4x4_grid.png}
        \caption{iGCT (1 NFE), w=1.0}
    \end{subfigure}
    % Second image
    \begin{subfigure}{0.25\textwidth}
        \includegraphics[width=\linewidth]{fig/appendix_igct/3_6.0_middle_4x4_grid.png}
        \caption{iGCT (1 NFE), w=7.0}
    \end{subfigure}
    % Third image
    \begin{subfigure}{0.25\textwidth}
        \includegraphics[width=\linewidth]{fig/appendix_igct/3_12.0_middle_4x4_grid.png}
        \caption{iGCT (1 NFE), w=13.0}
    \end{subfigure}
    \caption{CIFAR-10 "cat"}
    \label{fig:CIFAR-10_guided_4}
\end{figure*}
\begin{figure*}[t]
    \centering
    % First image
    \begin{subfigure}{0.25\textwidth}
        \includegraphics[width=\linewidth]{fig/appendix_edm/4_0.0_middle_4x4_grid.png}
        \caption{CFG-EDM (18 NFE), w=1.0}
    \end{subfigure}
    \begin{subfigure}{0.25\textwidth}
        \includegraphics[width=\linewidth]{fig/appendix_edm/4_6.0_middle_4x4_grid.png}
        \caption{CFG-EDM (18 NFE), w=7.0}
    \end{subfigure}
    \begin{subfigure}{0.25\textwidth}
        \includegraphics[width=\linewidth]{fig/appendix_edm/4_12.0_middle_4x4_grid.png}
        \caption{CFG-EDM (18 NFE), w=13.0}
    \end{subfigure}
    \begin{subfigure}{0.25\textwidth}
        \includegraphics[width=\linewidth]{fig/appendix_igct/4_0.0_middle_4x4_grid.png}
        \caption{iGCT (1 NFE), w=1.0}
    \end{subfigure}
    % Second image
    \begin{subfigure}{0.25\textwidth}
        \includegraphics[width=\linewidth]{fig/appendix_igct/4_6.0_middle_4x4_grid.png}
        \caption{iGCT (1 NFE), w=7.0}
    \end{subfigure}
    % Third image
    \begin{subfigure}{0.25\textwidth}
        \includegraphics[width=\linewidth]{fig/appendix_igct/4_12.0_middle_4x4_grid.png}
        \caption{iGCT (1 NFE), w=13.0}
    \end{subfigure}
    \caption{CIFAR-10 "deer"}
    \label{fig:CIFAR-10_guided_5}
\end{figure*}
\begin{figure*}[t]
    \centering
    % First image
    \begin{subfigure}{0.25\textwidth}
        \includegraphics[width=\linewidth]{fig/appendix_edm/5_0.0_middle_4x4_grid.png}
        \caption{CFG-EDM (18 NFE), w=1.0}
    \end{subfigure}
    \begin{subfigure}{0.25\textwidth}
        \includegraphics[width=\linewidth]{fig/appendix_edm/5_6.0_middle_4x4_grid.png}
        \caption{CFG-EDM (18 NFE), w=7.0}
    \end{subfigure}
    \begin{subfigure}{0.25\textwidth}
        \includegraphics[width=\linewidth]{fig/appendix_edm/5_12.0_middle_4x4_grid.png}
        \caption{CFG-EDM (18 NFE), w=13.0}
    \end{subfigure}
    \begin{subfigure}{0.25\textwidth}
        \includegraphics[width=\linewidth]{fig/appendix_igct/5_0.0_middle_4x4_grid.png}
        \caption{iGCT (1 NFE), w=1.0}
    \end{subfigure}
    % Second image
    \begin{subfigure}{0.25\textwidth}
        \includegraphics[width=\linewidth]{fig/appendix_igct/5_6.0_middle_4x4_grid.png}
        \caption{iGCT (1 NFE), w=7.0}
    \end{subfigure}
    % Third image
    \begin{subfigure}{0.25\textwidth}
        \includegraphics[width=\linewidth]{fig/appendix_igct/5_12.0_middle_4x4_grid.png}
        \caption{iGCT (1 NFE), w=13.0}
    \end{subfigure}
    \caption{CIFAR-10 "dog"}
    \label{fig:CIFAR-10_guided_6}
\end{figure*}
\begin{figure*}[t]
    \centering
    % First image
    \begin{subfigure}{0.25\textwidth}
        \includegraphics[width=\linewidth]{fig/appendix_edm/6_0.0_middle_4x4_grid.png}
        \caption{CFG-EDM (18 NFE), w=1.0}
    \end{subfigure}
    \begin{subfigure}{0.25\textwidth}
        \includegraphics[width=\linewidth]{fig/appendix_edm/6_6.0_middle_4x4_grid.png}
        \caption{CFG-EDM (18 NFE), w=7.0}
    \end{subfigure}
    \begin{subfigure}{0.25\textwidth}
        \includegraphics[width=\linewidth]{fig/appendix_edm/6_12.0_middle_4x4_grid.png}
        \caption{CFG-EDM (18 NFE), w=13.0}
    \end{subfigure}
    \begin{subfigure}{0.25\textwidth}
        \includegraphics[width=\linewidth]{fig/appendix_igct/6_0.0_middle_4x4_grid.png}
        \caption{iGCT (1 NFE), w=1.0}
    \end{subfigure}
    % Second image
    \begin{subfigure}{0.25\textwidth}
        \includegraphics[width=\linewidth]{fig/appendix_igct/6_6.0_middle_4x4_grid.png}
        \caption{iGCT (1 NFE), w=7.0}
    \end{subfigure}
    % Third image
    \begin{subfigure}{0.25\textwidth}
        \includegraphics[width=\linewidth]{fig/appendix_igct/6_12.0_middle_4x4_grid.png}
        \caption{iGCT (1 NFE), w=13.0}
    \end{subfigure}
    \caption{CIFAR-10 "frog"}
    \label{fig:CIFAR-10_guided_7}
\end{figure*}
\begin{figure*}[t]
    \centering
    % First image
    \begin{subfigure}{0.25\textwidth}
        \includegraphics[width=\linewidth]{fig/appendix_edm/7_0.0_middle_4x4_grid.png}
        \caption{CFG-EDM (18 NFE), w=1.0}
    \end{subfigure}
    \begin{subfigure}{0.25\textwidth}
        \includegraphics[width=\linewidth]{fig/appendix_edm/7_6.0_middle_4x4_grid.png}
        \caption{CFG-EDM (18 NFE), w=7.0}
    \end{subfigure}
    \begin{subfigure}{0.25\textwidth}
        \includegraphics[width=\linewidth]{fig/appendix_edm/7_12.0_middle_4x4_grid.png}
        \caption{CFG-EDM (18 NFE), w=13.0}
    \end{subfigure}
    \begin{subfigure}{0.25\textwidth}
        \includegraphics[width=\linewidth]{fig/appendix_igct/7_0.0_middle_4x4_grid.png}
        \caption{iGCT (1 NFE), w=1.0}
    \end{subfigure}
    % Second image
    \begin{subfigure}{0.25\textwidth}
        \includegraphics[width=\linewidth]{fig/appendix_igct/7_6.0_middle_4x4_grid.png}
        \caption{iGCT (1 NFE), w=7.0}
    \end{subfigure}
    % Third image
    \begin{subfigure}{0.25\textwidth}
        \includegraphics[width=\linewidth]{fig/appendix_igct/7_12.0_middle_4x4_grid.png}
        \caption{iGCT (1 NFE), w=13.0}
    \end{subfigure}
    \caption{CIFAR-10 "horse"}
    \label{fig:CIFAR-10_guided_8}
\end{figure*}
\begin{figure*}[t]
    \centering
    % First image
    \begin{subfigure}{0.25\textwidth}
        \includegraphics[width=\linewidth]{fig/appendix_edm/8_0.0_middle_4x4_grid.png}
        \caption{CFG-EDM (18 NFE), w=1.0}
    \end{subfigure}
    \begin{subfigure}{0.25\textwidth}
        \includegraphics[width=\linewidth]{fig/appendix_edm/8_6.0_middle_4x4_grid.png}
        \caption{CFG-EDM (18 NFE), w=7.0}
    \end{subfigure}
    \begin{subfigure}{0.25\textwidth}
        \includegraphics[width=\linewidth]{fig/appendix_edm/8_12.0_middle_4x4_grid.png}
        \caption{CFG-EDM (18 NFE), w=13.0}
    \end{subfigure}
    \begin{subfigure}{0.25\textwidth}
        \includegraphics[width=\linewidth]{fig/appendix_igct/8_0.0_middle_4x4_grid.png}
        \caption{iGCT (1 NFE), w=1.0}
    \end{subfigure}
    % Second image
    \begin{subfigure}{0.25\textwidth}
        \includegraphics[width=\linewidth]{fig/appendix_igct/8_6.0_middle_4x4_grid.png}
        \caption{iGCT (1 NFE), w=7.0}
    \end{subfigure}
    % Third image
    \begin{subfigure}{0.25\textwidth}
        \includegraphics[width=\linewidth]{fig/appendix_igct/8_12.0_middle_4x4_grid.png}
        \caption{iGCT (1 NFE), w=13.0}
    \end{subfigure}
    \caption{CIFAR-10 "ship"}
    \label{fig:CIFAR-10_guided_9}
\end{figure*}
\begin{figure*}[t]
    \centering
    % First image
    \begin{subfigure}{0.25\textwidth}
        \includegraphics[width=\linewidth]{fig/appendix_edm/9_0.0_middle_4x4_grid.png}
        \caption{CFG-EDM (18 NFE), w=1.0}
    \end{subfigure}
    \begin{subfigure}{0.25\textwidth}
        \includegraphics[width=\linewidth]{fig/appendix_edm/9_6.0_middle_4x4_grid.png}
        \caption{CFG-EDM (18 NFE), w=7.0}
    \end{subfigure}
    \begin{subfigure}{0.25\textwidth}
        \includegraphics[width=\linewidth]{fig/appendix_edm/9_12.0_middle_4x4_grid.png}
        \caption{CFG-EDM (18 NFE), w=13.0}
    \end{subfigure}
    \begin{subfigure}{0.25\textwidth}
        \includegraphics[width=\linewidth]{fig/appendix_igct/9_0.0_middle_4x4_grid.png}
        \caption{iGCT (1 NFE), w=1.0}
    \end{subfigure}
    % Second image
    \begin{subfigure}{0.25\textwidth}
        \includegraphics[width=\linewidth]{fig/appendix_igct/9_6.0_middle_4x4_grid.png}
        \caption{iGCT (1 NFE), w=7.0}
    \end{subfigure}
    % Third image
    \begin{subfigure}{0.25\textwidth}
        \includegraphics[width=\linewidth]{fig/appendix_igct/9_12.0_middle_4x4_grid.png}
        \caption{iGCT (1 NFE), w=13.0}
    \end{subfigure}
    \caption{CIFAR-10 "truck"}
    \label{fig:CIFAR-10_guided_10}
\end{figure*}


\begin{figure*}[b]
    \centering
    % First image
    \begin{subfigure}{0.25\textwidth}
        \includegraphics[width=\linewidth]{fig/appendix_im64_edm/edm_class_291_w=0.0.png}
        \caption{CFG-EDM (18 NFE), w=1.0}
    \end{subfigure}
    \begin{subfigure}{0.25\textwidth}
        \includegraphics[width=\linewidth]{fig/appendix_im64_edm/edm_class_291_w=6.0.png}
        \caption{CFG-EDM (18 NFE), w=7.0}
    \end{subfigure}
    \begin{subfigure}{0.25\textwidth}
        \includegraphics[width=\linewidth]{fig/appendix_im64_edm/edm_class_291_w=12.0.png}
        \caption{CFG-EDM (18 NFE), w=13.0}
    \end{subfigure}
    \begin{subfigure}{0.25\textwidth}
        \includegraphics[width=\linewidth]{fig/appendix_im64_igct/class_291_w=0.0.png}
        \caption{iGCT (2 NFE), w=1.0}
    \end{subfigure}
    \begin{subfigure}{0.25\textwidth}
        \includegraphics[width=\linewidth]{fig/appendix_im64_igct/class_291_w=6.0.png}
        \caption{iGCT (2 NFE), w=7.0}
    \end{subfigure}
    % Third image
    \begin{subfigure}{0.25\textwidth}
        \includegraphics[width=\linewidth]{fig/appendix_im64_igct/class_291_w=12.0.png}
        \caption{iGCT (2 NFE), w=13.0}
    \end{subfigure}
    \caption{ImageNet64 "lion"}
    \label{fig:im64_guided_1}
\end{figure*}



\begin{figure*}[b]
    \centering
    % First image
    \begin{subfigure}{0.25\textwidth}
        \includegraphics[width=\linewidth]{fig/appendix_im64_edm/edm_class_292_w=0.0.png}
        \caption{CFG-EDM (18 NFE), w=1.0}
    \end{subfigure}
    \begin{subfigure}{0.25\textwidth}
        \includegraphics[width=\linewidth]{fig/appendix_im64_edm/edm_class_292_w=6.0.png}
        \caption{CFG-EDM (18 NFE), w=7.0}
    \end{subfigure}
    \begin{subfigure}{0.25\textwidth}
        \includegraphics[width=\linewidth]{fig/appendix_im64_edm/edm_class_292_w=12.0.png}
        \caption{CFG-EDM (18 NFE), w=13.0}
    \end{subfigure}
    \begin{subfigure}{0.25\textwidth}
        \includegraphics[width=\linewidth]{fig/appendix_im64_igct/class_292_w=0.0.png}
        \caption{iGCT (2 NFE), w=1.0}
    \end{subfigure}
    \begin{subfigure}{0.25\textwidth}
        \includegraphics[width=\linewidth]{fig/appendix_im64_igct/class_292_w=6.0.png}
        \caption{iGCT (2 NFE), w=7.0}
    \end{subfigure}
    % Third image
    \begin{subfigure}{0.25\textwidth}
        \includegraphics[width=\linewidth]{fig/appendix_im64_igct/class_292_w=12.0.png}
        \caption{iGCT (2 NFE), w=13.0}
    \end{subfigure}
    \caption{ImageNet64 "tiger"}
    \label{fig:im64_guided_2}
\end{figure*}


\begin{figure*}[b]
    \centering
    % First image
    \begin{subfigure}{0.25\textwidth}
        \includegraphics[width=\linewidth]{fig/appendix_im64_edm/edm_class_28_w=0.0.png}
        \caption{CFG-EDM (18 NFE), w=1.0}
    \end{subfigure}
    \begin{subfigure}{0.25\textwidth}
        \includegraphics[width=\linewidth]{fig/appendix_im64_edm/edm_class_28_w=6.0.png}
        \caption{CFG-EDM (18 NFE), w=7.0}
    \end{subfigure}
    \begin{subfigure}{0.25\textwidth}
        \includegraphics[width=\linewidth]{fig/appendix_im64_edm/edm_class_28_w=12.0.png}
        \caption{CFG-EDM (18 NFE), w=13.0}
    \end{subfigure}
    \begin{subfigure}{0.25\textwidth}
        \includegraphics[width=\linewidth]{fig/appendix_im64_igct/class_28_w=0.0.png}
        \caption{iGCT (2 NFE), w=1.0}
    \end{subfigure}
    \begin{subfigure}{0.25\textwidth}
        \includegraphics[width=\linewidth]{fig/appendix_im64_igct/class_28_w=6.0.png}
        \caption{iGCT (2 NFE), w=7.0}
    \end{subfigure}
    % Third image
    \begin{subfigure}{0.25\textwidth}
        \includegraphics[width=\linewidth]{fig/appendix_im64_igct/class_28_w=12.0.png}
        \caption{iGCT (2 NFE), w=13.0}
    \end{subfigure}
    \caption{ImageNet64 "salamander"}
    \label{fig:im64_guided_3}
\end{figure*}


\begin{figure*}[b]
    \centering
    % First image
    \begin{subfigure}{0.25\textwidth}
        \includegraphics[width=\linewidth]{fig/appendix_im64_edm/edm_class_407_w=0.0.png}
        \caption{CFG-EDM (18 NFE), w=1.0}
    \end{subfigure}
    \begin{subfigure}{0.25\textwidth}
        \includegraphics[width=\linewidth]{fig/appendix_im64_edm/edm_class_407_w=6.0.png}
        \caption{CFG-EDM (18 NFE), w=7.0}
    \end{subfigure}
    \begin{subfigure}{0.25\textwidth}
        \includegraphics[width=\linewidth]{fig/appendix_im64_edm/edm_class_407_w=12.0.png}
        \caption{CFG-EDM (18 NFE), w=13.0}
    \end{subfigure}
    \begin{subfigure}{0.25\textwidth}
        \includegraphics[width=\linewidth]{fig/appendix_im64_igct/class_407_w=0.0.png}
        \caption{iGCT (2 NFE), w=1.0}
    \end{subfigure}
    \begin{subfigure}{0.25\textwidth}
        \includegraphics[width=\linewidth]{fig/appendix_im64_igct/class_407_w=6.0.png}
        \caption{iGCT (2 NFE), w=7.0}
    \end{subfigure}
    % Third image
    \begin{subfigure}{0.25\textwidth}
        \includegraphics[width=\linewidth]{fig/appendix_im64_igct/class_407_w=12.0.png}
        \caption{iGCT (2 NFE), w=13.0}
    \end{subfigure}
    \caption{ImageNet64 "ambulance"}
    \label{fig:im64_guided_4}
\end{figure*}


\end{document}


% This document was modified from the file originally made available by
% Pat Langley and Andrea Danyluk for ICML-2K. This version was created
% by Iain Murray in 2018, and modified by Alexandre Bouchard in
% 2019 and 2021 and by Csaba Szepesvari, Gang Niu and Sivan Sabato in 2022.
% Modified again in 2023 and 2024 by Sivan Sabato and Jonathan Scarlett.
% Previous contributors include Dan Roy, Lise Getoor and Tobias
% Scheffer, which was slightly modified from the 2010 version by
% Thorsten Joachims & Johannes Fuernkranz, slightly modified from the
% 2009 version by Kiri Wagstaff and Sam Roweis's 2008 version, which is
% slightly modified from Prasad Tadepalli's 2007 version which is a
% lightly changed version of the previous year's version by Andrew
% Moore, which was in turn edited from those of Kristian Kersting and
% Codrina Lauth. Alex Smola contributed to the algorithmic style files.
