\section{Experimental Results \& Discussion}
Until now, we have discussed the theoretical aspects and assumptions of treatment effect estimation and uncertainty quantification in the context of A/B-testing.
To assess the practical utility of the aforementioned methods and their potential, we wish to empirically answer the following research questions:

\begin{description}
    \item[\textbf{RQ1}] \textit{Can the Kolmogorov-Smirnov test on resampled A/A-tests uncover outcomes $Y$ for which normal CIs are not appropriate?}
    \item[\textbf{RQ2}] \textit{Does the $D$-statistic provide additional information as a diagnostic over the number of observations alone?}
    \item[\textbf{RQ3}] \textit{Can we leverage other information about the distribution of $Y$?}
\end{description}

To provide empirical answers to these questions, we resort to a subet of a proprietary log of user activity data on a consumer-facing application. 
As our discussions and insights are general and agnostic to the use-cases, we expect our findings to translate to other applications.
The data consists of $|\mathcal{U}|\approx2$  million users, and approximately 17 million logged instances across 50 event types.

\paragraph{\textbf{RQ1}: Utility of the approach.}
For every possible outcome $Y$ to measure, we construct $n=5\,000$ synthetic A/A comparisons and collect $D$-statistics and Kolmogorov-Smirnov $p$-values w.r.t. the expected uniform distribution.
Figure~\ref{fig:1} visualises their distribution over event types.
As expected, the null hypothesis that the CLT has sufficiently kicked in cannot be refuted for the majority of outcomes $Y$, and normal CIs are appropriate to reflect the uncertainty in the ATE estimate.
Nevertheless, the procedure succeeds in highlighting several events that require further investigation.\footnote{Note that a direct interpretation of Kolmogorov-Smirnov $p$-values would require a multiple testing correction to be applied~\cite{Shaffer1995}. Even with a crude Bonferroni correction, the ATE($Y$) distribution of several events still violates normality significantly.}
\begin{figure}[!t]
    \centering
    \includegraphics[width=\linewidth]{img/SIRIP_Fig1.pdf}
    \caption{Visualising the Kolmogorov-Smirnov $D$-statistic and resulting $p$-value per user-event we measure. Whilst the majority of $p$-value distributions cannot be distinguished from uniform, we reject the null hypothesis for several.}
    \label{fig:1}
\end{figure}

\paragraph{\textbf{RQ2}: Considering event frequency.}
A natural question to consider is whether the sample size is the deciding factor in determining whether the sampling distribution of the ATE has approached normality sufficiently well enough for the $t$-test to be valid.
Since all comparisons use the full dataset (i.e. roughly 1 million users per A/A group), and the sample size is thus constant, this is clearly not the case.
Instead, we might then consider event frequency, as for rare events the majority of users will not contribute to the ATE.
Figure~\ref{fig:2} visualises the number of event observations on a logarithmic scale, ranging from approximately 200 to 8 million, in relation to the $D$-statistic on the y-axis. 
Whilst a clear correlation is visible (Spearman's $\rho\approx0.45$), there is no monotonic relationship.
This suggests that while event frequency is informative, the $D$-statistic brings additional diagnostic value when assessing $t$-test validity.

\begin{figure}[!t]
    \centering
    \includegraphics[width=\linewidth]{img/SIRIP_Fig2.pdf}
    \caption{Visualising the number of event observations overall to their $D$-statistic, with a log-linear trendline. Whilst rare events lead to an increase in distribution divergence, the relationship is not monotonic (Spearman's $\rho\approx0.45$).}
    \label{fig:2}
\end{figure}

\paragraph{\textbf{RQ3}: Exploring other summary statistics for $\mathsf{P}(Y)$.}
Aside from event frequency (i.e. $\mathbb{E}[Y]$), we might be interested in other moments of the outcome distribution $\mathsf{P}(Y)$.
\citet{Kohavi2014} and \cite{Kohavi2022} discuss the skewness as an important diagnostic for CLT appropriateness.
We consider the sample skewness of $Y$, and report it alongside histograms for four events in Figure~\ref{fig:3}.
We visualise, in order of the legend:
\begin{enumerate*}[label=(\roman*)]
  \item the most frequent event,
  \item the rarest event for which normality cannot be rejected,
  \item an event with similar frequency to (iv) but low $D$-statistic, and
  \item the most frequent event with $p$-value < $1e-4$.
\end{enumerate*}
Sample skewness estimates are shown in the legend, suggesting that a higher skewness implies slower CLT convergence.
Nevertheless, there is no monotonic relationship: Spearman's $\rho\approx0.43$ suggests a significant rank-correlation with the $D$-statistic, but confirms the independent informational value of the Kolmogorov-Smirnov test as a diagnostic tool.

\begin{figure}[!t]
    \centering
    \includegraphics[width=\linewidth]{img/SIRIP_Fig3.pdf}
    \caption{The empirical density function for various events intuitively shows that the sample skewness of the empirical event distribution per user is an indicator of the required sample size for the CLT to kick in, and the mean event distribution to approach normality (Spearman's $\rho\approx0.43$).}
    \label{fig:3}
\end{figure}