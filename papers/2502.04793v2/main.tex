%\documentclass[sigconf, screen, dvipsnames, anonymous, nonacm]{acmart}
%\documentclass[sigconf, screen, dvipsnames, authordraft]{acmart}
%\DocumentMetadata{}
\documentclass[sigconf, screen, dvipsnames, nonacm]{acmart}
%\documentclass[sigconf, screen, dvipsnames]{acmart}
%\documentclass[sigconf, screen, dvipsnames]{acmart}



% Package definitions
\usepackage{hyperref}
\usepackage[noend]{algorithmic}
\usepackage{algorithm}
\usepackage{multirow}
\usepackage{amsmath}
\usepackage{bbm}
\usepackage{booktabs}
\usepackage[inline]{enumitem} % Inline lists
\usepackage{subcaption}
\usepackage{bm} % Bold greek
\usepackage{xcolor}

\usepackage{lipsum}

\usepackage{wrapfig}

% Dashed lines in tables
\usepackage{arydshln}
\makeatletter
\def\adl@drawiv#1#2#3{
        \hskip.5\tabcolsep
        \xleaders#3{#2.5\@tempdimb #1{1}#2.5\@tempdimb}%
                #2\z@ plus1fil minus1fil\relax
        \hskip.5\tabcolsep}
\newcommand{\cdashlinelr}[1]{%
  \noalign{\vskip\aboverulesep
          \global\let\@dashdrawstore\adl@draw
          \global\let\ adl@draw\adl@drawiv}
  \cdashline{#1}
  \noalign{\global\let\adl@draw\@dashdrawstore
          \vskip\belowrulesep}}
\makeatother

% PGM in Tikz
\usepackage{tikz}
\usetikzlibrary{automata, arrows, bayesnet, bending}

\usepackage{tikzscale}

% Reduce space after floats (such as tables and figures)
\setlength{\textfloatsep}{1pt plus 0pt minus .5pt}
\setlength{\intextsep}{1pt plus 0pt minus 1.0pt}
\setlength{\abovecaptionskip}{2pt plus 0pt minus 1pt}
\setlength{\belowcaptionskip}{2pt plus 0pt minus 1pt}
% Reduce space around equations
\setlength{\abovedisplayskip}{2pt plus 1pt minus 1pt}
\setlength{\belowdisplayskip}{2pt plus 1pt minus 1pt}

% Argmax as mathematical operator
\DeclareMathOperator*{\argmax}{arg\,max}
\DeclareMathOperator*{\argmin}{arg\,min}
\DeclareMathOperator{\vect}{vec}

\newtheorem{prop}{Proposition}


%Conference
\copyrightyear{2025}
\acmYear{2025}
\setcopyright{rightsretained}

\begin{document}
\title{$t$-Testing the Waters}
\subtitle{Empirically Validating Assumptions for Reliable A/B-Testing}

\author{Olivier Jeunen}
\affiliation{
  \institution{aampe}
  \city{Antwerp}
  \country{Belgium}
}
\email{olivier@aampe.com}

\begin{abstract}
A/B-tests are a cornerstone of experimental design on the web, with wide-ranging applications and use-cases.
The statistical $t$-test comparing differences in means is the most commonly used method for assessing treatment effects, often justified through the Central Limit Theorem (CLT).
The CLT ascertains that, as the sample size grows, the sampling distribution of the Average Treatment Effect converges to normality, making the $t$-test valid for sufficiently large sample sizes.
When outcome measures are skewed or non-normal, quantifying what ``\emph{sufficiently large}'' entails is not straightforward.

To ensure that confidence intervals maintain proper coverage and that $p$-values accurately reflect the false positive rate, it is critical to validate this normality assumption.
We propose a practical method to test this, by analysing repeatedly resampled A/A-tests.
When the normality assumption holds, the resulting $p$-value distribution should be uniform, and this property can be tested using the Kolmogorov-Smirnov test.
This provides an efficient and effective way to empirically assess whether the $t$-test's assumptions are met, and the A/B-test is valid.
We demonstrate our methodology and highlight how it helps to identify scenarios prone to inflated Type-I errors.
Our approach provides a practical framework to ensure and improve the reliability and robustness of A/B-testing practices.
\end{abstract}

\maketitle

%具身智能体在复杂场景下 manipulation 的 performance robustness 和泛化能力始终是一个广受关注的研究方向。其中,visuomotor imitation learning 是具身智能体 Policy 的主流范式之一,它允许 agent 从高维视觉观察和机器人本体感知中 effectively 学习 manipulation skills。
%然而,增加场景的复杂度和 visual distraction,会导致在简单场景下表现良好的决策模型性能下降。实际上,不仅是 simple imitation learning policy,先进的多模态 foundation models such as GPT-4o 或 vision language action models (VLA),也不能很好地关注一张语义丰富的图片中的特定的局部问题。对于 robot control or 多模态大模型,其往往侧重于 action prediction, observation mapping or 多模态 alignment,而缺少直观的视觉感知增强。模型需要隐性地或遵循 high-level text instruction 从相关的视觉区域中获得面向任务语义的定位知识。
%To tackle this challenge problem, we introduce Imit Diff, a diffusion transformer imitation learning framework with dual resolution enhancement guided by fine-grained semantics information。具体来说,our work 有三个关键组成部分。
%1) Semanstic Injection. Imit Diff 通过 vision language models (VLM) 和 vision foundation models 的 pretrain knowledge 将面向任务的语义信息和高层文本指导转化为显式的 pixel-level 视觉定位标签,注入到 environment observation中。
%2) Dual Res Fusion。 我们构建了双分辨率图像观测流,使用双分辨率视觉编码器分别提取全局和细粒度视觉特征。多尺度视觉信息随后在 attention block 中进行融合,在保证计算 effiency 的前提下,为全局视觉观测引入多尺度细粒度信息,提升场景理解能力。
%3) Consistency policy on diffusion transformer。Diffusion based imitation policies 通常受到 denoise times 的困扰。我们建立了基于 consistency policy 的 DiT action head。Policy 的决策层可以通过 single step denoise 实现系统高频响应。额外地,受益于较快的 inference time,我们引入 temperal ensemble 改善预测动作的平滑性。
%我们设计了四个在 manipulation 精细度上具有挑战性的现实世界任务来评估 Imit Diff,并通过增加场景复杂度和 visual distraction 来测试模型的场景理解能力。额外地,我们设计了 visual distraction 和 category generalization 的 zero shot 实验来验证模型是否受益于 dual res enhancement framework and fine-grained semantics injection。实验结果表明,Imit Diff outperforms 现有的 strong baselines。
%In summary, the contributions of our work are three-fold:
%1) We propose Imit Diff, a DiT architecture imitation learning framework with dual res enhancement guied by fine-grained semantics information.
%2) 我们构建了 open-set vision foundation models pipeline 来获得显式视觉遮罩。该方法能够有效处理机器人控制场景的运动模糊、遮挡、物体丢失情况。并将其作为 fine-grained 语义信息引导 policy decision。
%3) 我们在DiT上实现了consistency policy,显著减少了模型推理时间。通过异步控制框架,实现了 open-set vision foundation models 工作流下的实时控制。
%The code will be publicly available soon。

\section{Introduction}


\label{Intro}
The performance robustness and generalization capabilities of embodied agents in complex manipulation scenarios have long been a focus of significant research interest \citep{ju2025robo, yuan2024learning}. Visuomotor imitation learning is one of the mainstream paradigms of robot manipulation policy \citep{chi2023diffusion, shridhar2023perceiver, ze2023gnfactor, florence2022implicit, hansen2022pre}. This approach enables agents to derive state estimation and decision-making capabilities from expert demonstrations that incorporate high-dimensional visual observations and robot proprioception \citep{ze20243d}.

However, as scene complexity and visual distractions increase, the performance of decision models that excel in simpler environments tends to degrade \citep{zheng2024instruction, liurobustness}. Not only do simple imitation learning policies face challenges, but even advanced multimodal foundation models, such as GPT-4o \citep{hurst2024gpt} or vision language action models (VLA) \citep{liu2024rdt, brohan2022rt, brohan2023rt, o2023open, kim2024openvla, wen2024diffusion}, struggle to accurately focus on specific details within semantically complex images. In fact, in robot control and embodied multimodal foundation models, the focus is often on action prediction, observation mapping, or multimodal alignment. Therefore, intuitive visual perception enhancement is typically lacking. Models can only acquire task-oriented semantic localization knowledge from relevant visual regions either implicitly or when guided by high-level text instructions \citep{reuss2023multimodal}.

To tackle this challenge problem, we introduce \textbf{Imit Diff}, a diffusion transformer imitation learning framework with dual resolution enhancement guided by fine-grained semantics information. Specifically, our work has three key components:

\begin{enumerate}

\item \textbf{Semanstic injection.} Imit Diff transforms task-oriented semantic information and high-level textual guidance into explicit pixel-level visual localization labels through the pretrain knowledge of vision language models (VLM) and vision foundation models, and injects them into the policy observation.

\item \textbf{Dual resolution (dual res) fusion.} We develop a dual res image observation stream and employed a dual res vision encoder to extract global and fine-grained visual features. The extracted multi-scale visual information is subsequently fused within an attention block, integrating fine-grained details into the global visual feature. This approach enhances scene understanding while maintaining computational efficiency.

\item \textbf{Consistency policy on diffusion transformer (DiT).} Diffusion-based imitation policies often suffer from inefficiencies due to the required denoising steps. To address this, we design a DiT \citep{peebles2023scalable} action head incorporating a consistency policy \citep{song2023consistency}, enabling the decision layer to achieve high-frequency system responses through single-step denoising. Furthermore, leveraging faster inference times, we introduce temperal ensemble to enhance the smoothness of predicted actions.

\end{enumerate}

We design four real-world tasks with challenging manipulation precision to evaluate Imit Diff and test the model's scene understanding capabilities by introducing increased scene complexity and visual distractions. Additionally, we conducted zero-shot experiments on visual distraction and category generalization to assess the benefits of the dual res enhancement framework and fine-grained semantic injection. Experimental results demonstrate that Imit Diff significantly outperforms existing strong baselines. 

In summary, the contributions of our work are three-fold:

\begin{enumerate}

\item We propose Imit Diff, a DiT architecture imitation learning framework with dual res enhancement guied by fine-grained semantics information.

\item We developed an open-set vision foundation model pipeline to generate explicit visual masks. This approach effectively addresses challenges such as motion blur, occlusion, and object loss in robot control scenarios, leveraging the generated masks as fine-grained semantic information to guide policy decisions.

\item We implemented a consistency policy on DiT, which significantly reduced the model inference time. Through the asynchronous control framework, we achieved real-time control under the workflow of open-set vision foundation models.

\end{enumerate}

The code will be made publicly available soon.

\begin{figure*}[htb!]
		\centering
		\begin{tabular}{c}
		    	\includegraphics[width=0.75\linewidth]{fig/framework.pdf}
		\end{tabular}
	\caption{Framework of the unified semantic and ID representation learning. Firstly, the model integrates both semantic tokens, learned through RQ-VAE, and ID tokens for the recommendation task. Secondly, cosine similarity is applied in the first two layers to decouple accumulated embeddings, while Euclidean distance is utilized in the final layer to effectively distinguish unique items. Finally, the overall model is optimized in an end-to-end manner, combining the recommendation loss, RQ-VAE quantization loss, and text reconstruction loss.}	\label{fig:framework}
\end{figure*}

\section{Unified Representation Learning}
In this section, as illustrated in Figure~\ref{fig:framework}, we introduce a unified semantic and ID representation learning framework. Our method is designed to fully exploit the complementary strengths of semantic and ID tokens, integrate cosine similarity and Euclidean distance, and jointly optimize both the quantization and recommendation tasks. The key components of the framework are described as follows:

\begin{itemize}[leftmargin=*]
    \item \textbf{Unified Semantic and ID Tokenization}: To balance capturing unique and shared item characteristics, we retain only a small proportion of ID token dimensions to represent the unique attributes of items. Meanwhile, the semantic tokens, learned through RQ-VAE, are employed to capture the shared, transferable characteristics across items. This hybrid approach reduces redundancy in the ID space while enhancing generalization.
    
    \item \textbf{Unified Cosine Similarity and Euclidean Distance}: We leverage the strengths of cosine similarity and Euclidean distance in different layers of our model. Specifically, cosine similarity is applied in the earlier layers to effectively decouple accumulated embeddings, while Euclidean distance is employed in the final layer to distinguish unique items. This design maximizes the benefits of both metrics during codebook searching, enhancing the accuracy of item representation.
    
    \item \textbf{End-to-End Joint Optimization}: Our framework is trained in an end-to-end manner, jointly optimizing three key objectives: (1) the recommendation loss to ensure accurate predictions, (2) the RQ-VAE loss for effective codebook assignment, and (3) the text reconstruction loss to maintain the quality of semantic representation. This joint optimization strategy ensures that all components of the model are fine-tuned for optimal performance in both quantization and recommendation tasks.
\end{itemize}


\subsection{Unified Semantic and ID Tokenization}
\begin{figure}[!htb]
		\centering
		\begin{tabular}{c}
		    	\includegraphics[width=0.65\linewidth]{fig/architecture.pdf}
		\end{tabular}
	\caption{Illustration of unified semantic and ID tokenization. Specifically, we replace ID tokens with low-dimension ID tokens and semantic tokens.}	\label{fig:token}
\end{figure}

While ID tokenization is effective at capturing unique, item-specific information, it tends to suffer from redundancy and poor generalization, particularly in cold-start scenarios. In contrast, semantic tokenization excels at generalization by capturing shared, transferable features but may introduce item duplication when similar items are mapped to the same token. Therefore, these two approaches are complementary, and combining their strengths can address their respective limitations.

To this end, we propose a unified tokenization strategy that integrates both ID and semantic tokenization. Given that the number of items \( m \) can be very large, we reduce the dimensionality of the ID embeddings by setting \( D \) smaller than the dimension \( D' \) used for semantic embeddings. As shown in Figure~\ref{fig:token}, our method replaces most dimensions of the ID token with the more generalizable semantic token to reduce redundancy while retaining the ability to capture unique item characteristics. Specifically, for each item \( i_t \) in the user’s interaction history, we concatenate the semantic embedding \( \hat{\boldsymbol{z}}_{i_t} \) and the reduced ID embedding \( \boldsymbol{e}_{i_t} \) to form a unified representation, defined as:
\(
\boldsymbol{s}_{i_t} = [\hat{\boldsymbol{z}}_{i_t}, \boldsymbol{e}_{i_t}],
\)
which results in a sequence of unified embeddings for user \( u \), denoted as:
\(
\hat{\mathcal{S}}_{u} = (\hat{\boldsymbol{s}}_{i_{1}}, \hat{\boldsymbol{s}}_{i_{2}}, \ldots, \hat{\boldsymbol{s}}_{i_{t}})
\)

By combining ID and semantic embeddings, the unified tokenization approach retains the unique characteristics of each item while leveraging the semantic embedding's ability to generalize across similar items. This hybrid representation aims to improve both the efficiency and accuracy of recommendation by reducing redundancy in the ID space and enhancing the model's capacity to generalize to cold-start items.


% As shown in Figure~\ref{fig:token}, in this section, we replace most dimensions of ID tokens with semantic tokens, aiming to reduce the redundancy and improve the generalization ability of representation learning.

% \paragraph{\textbf{Unified Tokenization}} As ID tokenization is good at capturing unique information but falls short in generation and is redundant,  
% semantic tokenization is good at generation but has duplicate problem. That is to say, they are complimentary to each other, and we further make combination of them here. Firstly, given that the number of items \( m \) can be very large, we set the dimension \( D \) typically smaller than the dimension \( D' \) of ID tokenization. Further, we concatenate the ID embedding and semantic embedding for each item together as $\boldsymbol{s}_{i_t}$ = [$\hat{\boldsymbol{{z}}}_{i_t}$, $\boldsymbol{{e}}_{i_t}$]. Then we obtain a sequence of unified embeddings $\hat{\mathcal{S}}_{u} = (\hat{\boldsymbol{s}}_{i_{1}}, \hat{\boldsymbol{s}}_{i_{2}}, \ldots, \hat{\boldsymbol{s}}_{i_{t}})$ for user $u$.
 % $\boldsymbol{M}^c \in \mathbb{R}^{L \times K \times D}$



% \begin{equation}
% \mathcal{C}_l:=\left\{\boldsymbol{e}_{k}\right\}_{k=1}^K
% \end{equation}

% \begin{equation}
% \boldsymbol{z}_i=\textbf{Encoder}({x}_{i})
% \end{equation}
%  \begin{equation}
%  \begin{aligned}
% k_l=\arg \min_k\left\|\boldsymbol{r}_{l}-\boldsymbol{e}^c_{l, k}\right\|, \boldsymbol{r}_{l + 1} = \boldsymbol{r}_l-\boldsymbol{e}^c_{l, k}, \boldsymbol{r}_0 = \boldsymbol{z}_t,\\
%  \hat{\boldsymbol{{z}}}_t = \sum_{l = 1}^{L} \boldsymbol{e}^c_{l, k_l}
% % \\
% % k_{1,t}=\arg \min_k\left\|\boldsymbol{z}_t-\boldsymbol{e}^c_{1, k}\right\|, r_{1, t} = r_{0, t}-\boldsymbol{e}^c_{1, k_{1, t}, 
%  \end{aligned}
% \end{equation}


\subsection{Unified Distance Function}\label{sec:unified_distance}
\begin{table}[!htb]
\centering
\begin{tabular}{|l|c|c|}
\hline
\textbf{Type}         & \textbf{Cosine} & \textbf{Euclidean} \\ \hline
First layer           & 97.66\%         & 5.86\%             \\ \hline
Second layer          & 98.44\%         & 100.00\%           \\ \hline
Third layer           & 97.66\%         & 100.00\%           \\ \hline
Total coverage        & 70.13\%         & 92.67\%            \\ \hline
\end{tabular}

\caption{Comparison of cosine similarity and Euclidean distance in terms of the percentage of activated codebook across three layers and total coverage of unique items. Cosine similarity shows a high percentage of activated codebooks in all layers but lower overall coverage of unique items. In contrast, Euclidean distance exhibits high coverage of unique items, but struggles with a significantly lower percentage of activated codebooks in the first layer.}
\label{tab:distance}
\end{table}

To enhance the accuracy of codebook selection in our framework, we aim to improve the distance function used for identifying the closest codebook in $k=\arg \min_k\left\|\boldsymbol{r}_{l}-\boldsymbol{e}^c_{k}\right\|$, as defined in Algorithm~\ref{alg:rq} of Appendix~\ref{sec:semantic_token}.

\begin{figure*}[t!]
		\centering
		\begin{tabular}{ccc}
		    	\includegraphics[width=0.31\linewidth]{fig/cos_categories_first.pdf} &  \includegraphics[width=0.31\linewidth]{fig/cos_categories_second.pdf} &
       \includegraphics[width=0.31\linewidth]{fig/cos_categories_third.pdf} 
		     \\ First Codebook & Second Codebook & Third Codebook
		\end{tabular}
    \caption{Visualization of the codebook selection using cosine similarity across three layers. This figure shows the count of items from various categories assigned to specific token indices, with a focus on the top-3 codebook indices that contain the highest number of items. The distinct distribution of items across different indices suggests that cosine similarity effectively captures category-specific information and helps in distinguishing between categories.}\label{fig:cosine}
\end{figure*} 

\begin{figure*}[!htb]
		\centering
		\begin{tabular}{ccc}
		    	\includegraphics[height=0.38\linewidth]{fig/elu_categories_first.pdf} &  \includegraphics[height=0.38\linewidth]{fig/elu_categories_second.pdf} &
       \includegraphics[height=0.38\linewidth]{fig/elu_categories_third.pdf} 
		     \\ First Codebook & Second Codebook & Third Codebook
		\end{tabular}
 \caption{Visualization of the codebook selection using Euclidean distance across three layers. The uniform distribution of items across categories in the first layer indicates that Euclidean distance struggles to effectively capture category-specific information at this stage, making it less capable of distinguishing between categories compared to later layers.}
\label{fig:elu}
\end{figure*} 

\paragraph{\textbf{Statistical Analysis}} 
Our initial analysis, summarized in Table~\ref{tab:distance}, reveals that cosine similarity activates a high percentage of the codebook but struggles to cover unique items effectively. In contrast, Euclidean distance provides high coverage of unique items but activates a much lower percentage of the codebook, with only 5.86\% activation in the first layer. The limited activation of Euclidean distance in the early layers may result from its difficulty in decoupling accumulated embeddings, as these embeddings tend to cluster tightly at the beginning. Cosine similarity, on the other hand, excels in decoupling these embeddings, possibly due to its ability to handle orthogonal relationships between embeddings. However, cosine similarity’s limited ability to distinguish between distinct embeddings may be attributed to the bounded angular range of 0 to 360$^{\circ}$, while Euclidean distance, grounded in the Cartesian coordinate system, provides a more precise measure for distinguishing embeddings based on distance in \( \mathbb{R} \).





\begin{figure*}[t!]
		\centering
		\begin{tabular}{ccc}
		    	\includegraphics[width=0.31\linewidth]{fig/mix_categories_first.pdf} &  \includegraphics[width=0.31\linewidth]{fig/mix_categories_second.pdf} &
       \includegraphics[width=0.31\linewidth]{fig/mix_categories_third.pdf} 
		     \\ First Codebook & Second Codebook & Third Codebook
		\end{tabular}
  \caption{Visualization of codebook selection using the hybrid approach that combines cosine similarity and Euclidean distance. The variation in the counts of items assigned to different codebook tokens across categories demonstrates the effectiveness of this combined method in capturing category-specific information. The integration of both distance measures enhances the ability of Euclidean distance to distinguish between different categories, leading to more accurate item categorization.}\label{fig:hybrid}
\end{figure*} 

\paragraph{\textbf{Visualized Analysis}} 
To further investigate the performance of cosine similarity and Euclidean distance in codebook selection, we visualized the counts of the top-learned codebooks across different categories using both methods, as shown in Figures~\ref{fig:cosine} and \ref{fig:elu}, respectively. These visualizations demonstrate that cosine similarity can effectively capture category-specific information across layers, while Euclidean distance struggles to do so in the first layer. Specifically, in the first layer, the codebook entries selected by Euclidean distance appear uniformly distributed across categories, indicating that it fails to differentiate between them.

Based on these observations, we propose the following assumption: \textit{Cosine similarity is more effective at minimizing interference within accumulated embeddings but less capable of distinguishing distinct embeddings, whereas Euclidean distance excels at distinguishing unique embeddings but struggles to decouple accumulated ones.}

\begin{table}[!htb]
\centering
\caption{Effectiveness of the hybrid approach combining cosine similarity and Euclidean distance. The integration of Euclidean distance into cosine similarity results in a 100\% activation of the codebook across layers, while also improving the coverage of unique items. This demonstrates the advantage of leveraging both distance measures for more comprehensive and accurate item representation.}
\label{tab:hybrid}
\begin{tabular}{ccc}
\toprule
\multirow{3}{*}{\begin{tabular}[c]{@{}c@{}}Activated\\ codebook\end{tabular}} & First layer & 100.00\% \\ \cline{2-3} 
                & Second layer               & 100.00\% \\ \cline{2-3} 
                & Third layer                & 100.00\% \\ \midrule
\multicolumn{2}{c}{Coverage of unique items} & 83.27\%  \\ \bottomrule
\end{tabular}
\end{table}
\paragraph{\textbf{Proposed Method and Experimental Validation}} 
Building on this assumption, we propose a unified approach that combines cosine similarity and Euclidean distance. In the initial layers, cosine similarity is employed to decouple accumulated embeddings, while Euclidean distance is applied in the final layer to better distinguish unique items. To validate the effectiveness of this hybrid approach, we visualize the codebook selection counts across categories in Figure~\ref{fig:hybrid}. The results show that the combination of cosine similarity and Euclidean distance successfully captures category-specific information. Moreover, as shown in Table~\ref{tab:hybrid}, the percentage of activated codebook entries reaches 100\%, and the coverage of unique items improves significantly compared to using cosine similarity alone.

\paragraph{\textbf{Limitations}} 
Despite the improvements, our proposed method still results in approximately 17\% duplicate items, as observed in Table~\ref{tab:hybrid}. This issue arises when sentence embeddings for certain items are too similar to be distinguished. While this challenge is difficult to completely eliminate, it can be mitigated by assigning a unique, low-dimensional ID token to each item, helping to further differentiate items with highly similar embeddings.







\subsection{End-to-end Joint Optimization}
After unified tokenization of input item sequence for given user $u$, we then can predict the probability of next item as below.
\begin{equation}
    \hat{y}_{u, t } = \Phi (\boldsymbol{s}_{i_1}, \boldsymbol{s}_{i_2}, \cdots \boldsymbol{s}_{i_{t - 1}})
\end{equation}
where $\Phi$ is the sequential recommendation model to predict the probability $\hat{y}_{u, t }$ of next item. Here $\Phi$ can be any type of sequential recommendation models and we use SASRec~\citep{sasrec} here.
Based on the popular logloss~\citep{sasrec,dcn}, we then can optimize the recommendation model as: 
\begin{equation}\label{eq:loss}
\mathcal{L}_{recom}=-\frac{1}{|\mathcal{R}|} \sum_{(u, \mathcal{I}_{u}) \in \mathcal{R}}\left(y_{u, t} \log \hat{y}_{u, t}+\left(1-y_{u, t}\right) \log \left(1-\hat{y}_{u, t}\right)\right)  + \lambda\|\Theta\|,
\end{equation}
where $\mathcal{R}$ represents the training set, $\Theta$ denotes the learnable model parameters, and $\lambda$ denotes the regularization hyper-parameter. Finally, we jointly optimize the loss of recommendation, the loss of RQ-VAE, and the loss of reconstruction for text embedding as $\mathcal{L} = \mathcal{L}_{recom} + \mathcal{L}_{rqvae} + \mathcal{L}_{recon}$ (please refer to the algorithm in Appendix~\ref{sec:semantic_token}).

% \section{Complexity}


\section{Experimental Results \& Discussion}
Until now, we have discussed the theoretical aspects and assumptions of treatment effect estimation and uncertainty quantification in the context of A/B-testing.
To assess the practical utility of the aforementioned methods and their potential, we wish to empirically answer the following research questions:

\begin{description}
    \item[\textbf{RQ1}] \textit{Can the Kolmogorov-Smirnov test on resampled A/A-tests uncover outcomes $Y$ for which normal CIs are not appropriate?}
    \item[\textbf{RQ2}] \textit{Does the $D$-statistic provide additional information as a diagnostic over the number of observations alone?}
    \item[\textbf{RQ3}] \textit{Can we leverage other information about the distribution of $Y$?}
\end{description}

To provide empirical answers to these questions, we resort to a subet of a proprietary log of user activity data on a consumer-facing application. 
As our discussions and insights are general and agnostic to the use-cases, we expect our findings to translate to other applications.
The data consists of $|\mathcal{U}|\approx2$  million users, and approximately 17 million logged instances across 50 event types.

\paragraph{\textbf{RQ1}: Utility of the approach.}
For every possible outcome $Y$ to measure, we construct $n=5\,000$ synthetic A/A comparisons and collect $D$-statistics and Kolmogorov-Smirnov $p$-values w.r.t. the expected uniform distribution.
Figure~\ref{fig:1} visualises their distribution over event types.
As expected, the null hypothesis that the CLT has sufficiently kicked in cannot be refuted for the majority of outcomes $Y$, and normal CIs are appropriate to reflect the uncertainty in the ATE estimate.
Nevertheless, the procedure succeeds in highlighting several events that require further investigation.\footnote{Note that a direct interpretation of Kolmogorov-Smirnov $p$-values would require a multiple testing correction to be applied~\cite{Shaffer1995}. Even with a crude Bonferroni correction, the ATE($Y$) distribution of several events still violates normality significantly.}
\begin{figure}[!t]
    \centering
    \includegraphics[width=\linewidth]{img/SIRIP_Fig1.pdf}
    \caption{Visualising the Kolmogorov-Smirnov $D$-statistic and resulting $p$-value per user-event we measure. Whilst the majority of $p$-value distributions cannot be distinguished from uniform, we reject the null hypothesis for several.}
    \label{fig:1}
\end{figure}

\paragraph{\textbf{RQ2}: Considering event frequency.}
A natural question to consider is whether the sample size is the deciding factor in determining whether the sampling distribution of the ATE has approached normality sufficiently well enough for the $t$-test to be valid.
Since all comparisons use the full dataset (i.e. roughly 1 million users per A/A group), and the sample size is thus constant, this is clearly not the case.
Instead, we might then consider event frequency, as for rare events the majority of users will not contribute to the ATE.
Figure~\ref{fig:2} visualises the number of event observations on a logarithmic scale, ranging from approximately 200 to 8 million, in relation to the $D$-statistic on the y-axis. 
Whilst a clear correlation is visible (Spearman's $\rho\approx0.45$), there is no monotonic relationship.
This suggests that while event frequency is informative, the $D$-statistic brings additional diagnostic value when assessing $t$-test validity.

\begin{figure}[!t]
    \centering
    \includegraphics[width=\linewidth]{img/SIRIP_Fig2.pdf}
    \caption{Visualising the number of event observations overall to their $D$-statistic, with a log-linear trendline. Whilst rare events lead to an increase in distribution divergence, the relationship is not monotonic (Spearman's $\rho\approx0.45$).}
    \label{fig:2}
\end{figure}

\paragraph{\textbf{RQ3}: Exploring other summary statistics for $\mathsf{P}(Y)$.}
Aside from event frequency (i.e. $\mathbb{E}[Y]$), we might be interested in other moments of the outcome distribution $\mathsf{P}(Y)$.
\citet{Kohavi2014} and \cite{Kohavi2022} discuss the skewness as an important diagnostic for CLT appropriateness.
We consider the sample skewness of $Y$, and report it alongside histograms for four events in Figure~\ref{fig:3}.
We visualise, in order of the legend:
\begin{enumerate*}[label=(\roman*)]
  \item the most frequent event,
  \item the rarest event for which normality cannot be rejected,
  \item an event with similar frequency to (iv) but low $D$-statistic, and
  \item the most frequent event with $p$-value < $1e-4$.
\end{enumerate*}
Sample skewness estimates are shown in the legend, suggesting that a higher skewness implies slower CLT convergence.
Nevertheless, there is no monotonic relationship: Spearman's $\rho\approx0.43$ suggests a significant rank-correlation with the $D$-statistic, but confirms the independent informational value of the Kolmogorov-Smirnov test as a diagnostic tool.

\begin{figure}[!t]
    \centering
    \includegraphics[width=\linewidth]{img/SIRIP_Fig3.pdf}
    \caption{The empirical density function for various events intuitively shows that the sample skewness of the empirical event distribution per user is an indicator of the required sample size for the CLT to kick in, and the mean event distribution to approach normality (Spearman's $\rho\approx0.43$).}
    \label{fig:3}
\end{figure}
\section{Conclusions \& Outlook}
A/B-tests are omnipresent in modern technology companies, often seen as the ``gold standard'' of experimentation practices.
The default estimand is typically the ATE on a metric of interest, and statistical uncertainty on the estimate is handled through standard methods. 
It is often forgotten that these methods rely on implicit assumptions that might be violated, invalidating the estimates.

Most commonly, we assume that the distribution of the ATE has converged to normality due to the CLT. 
If this is false, the CIs and $p$-values we use to assess A/B-testing outcomes become misleading.
In this work, we propose an efficient and effective manner to validate this assumption empirically, through the use of repeatedly resampled A/A-tests and a Kolmogorov-Smirnov test on the uniformity of the resulting $p$-value distribution.
We describe our approach, present empirical results on real-world data that both elucidate the methods and highlight that potential proxies (like sample size or skewness) are promising but imperfect.
This provides a practical framework to assess A/B-testing estimands and avoid situations where improper estimation methods are applied, in the hope that it will help the community to enforce statistical rigour and avoid inflated false positive risk going forward.

% \section*{Presenter Biography}
% \textbf{Olivier Jeunen} is Principal Research Scientist at Aampe with a PhD from the University of Antwerp, who has previously held positions at ShareChat, Amazon, Spotify, Meta and Criteo.
% His research focuses on applying ideas from causal and counterfactual inference to recommendation and advertising problems, which have led to 45\textsuperscript{+} peer reviewed contributions, two of which have been recognised with best paper awards. He is an active (Senior) Program Committee
% member in the community---which has led to four outstanding reviewer awards. 
% Olivier has been a workshop and conference organising committee member in various roles, recently co-chairing the Industry Tracks at ECIR '24 and RecSys '25.

\bibliographystyle{ACM-Reference-Format}
\bibliography{bibliography}

\end{document}