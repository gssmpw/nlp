%\documentclass[sigconf, screen, dvipsnames, anonymous, nonacm]{acmart}
%\documentclass[sigconf, screen, dvipsnames, authordraft]{acmart}
%\DocumentMetadata{}
\documentclass[sigconf, screen, dvipsnames, nonacm]{acmart}
%\documentclass[sigconf, screen, dvipsnames]{acmart}
%\documentclass[sigconf, screen, dvipsnames]{acmart}



% Package definitions
\usepackage{hyperref}
\usepackage[noend]{algorithmic}
\usepackage{algorithm}
\usepackage{multirow}
\usepackage{amsmath}
\usepackage{bbm}
\usepackage{booktabs}
\usepackage[inline]{enumitem} % Inline lists
\usepackage{subcaption}
\usepackage{bm} % Bold greek
\usepackage{xcolor}

\usepackage{lipsum}

\usepackage{wrapfig}

% Dashed lines in tables
\usepackage{arydshln}
\makeatletter
\def\adl@drawiv#1#2#3{
        \hskip.5\tabcolsep
        \xleaders#3{#2.5\@tempdimb #1{1}#2.5\@tempdimb}%
                #2\z@ plus1fil minus1fil\relax
        \hskip.5\tabcolsep}
\newcommand{\cdashlinelr}[1]{%
  \noalign{\vskip\aboverulesep
          \global\let\@dashdrawstore\adl@draw
          \global\let\ adl@draw\adl@drawiv}
  \cdashline{#1}
  \noalign{\global\let\adl@draw\@dashdrawstore
          \vskip\belowrulesep}}
\makeatother

% PGM in Tikz
\usepackage{tikz}
\usetikzlibrary{automata, arrows, bayesnet, bending}

\usepackage{tikzscale}

% Reduce space after floats (such as tables and figures)
\setlength{\textfloatsep}{1pt plus 0pt minus .5pt}
\setlength{\intextsep}{1pt plus 0pt minus 1.0pt}
\setlength{\abovecaptionskip}{2pt plus 0pt minus 1pt}
\setlength{\belowcaptionskip}{2pt plus 0pt minus 1pt}
% Reduce space around equations
\setlength{\abovedisplayskip}{2pt plus 1pt minus 1pt}
\setlength{\belowdisplayskip}{2pt plus 1pt minus 1pt}

% Argmax as mathematical operator
\DeclareMathOperator*{\argmax}{arg\,max}
\DeclareMathOperator*{\argmin}{arg\,min}
\DeclareMathOperator{\vect}{vec}

\newtheorem{prop}{Proposition}


%Conference
\copyrightyear{2025}
\acmYear{2025}
\setcopyright{rightsretained}

\begin{document}
\title{$t$-Testing the Waters}
\subtitle{Empirically Validating Assumptions for Reliable A/B-Testing}

\author{Olivier Jeunen}
\affiliation{
  \institution{aampe}
  \city{Antwerp}
  \country{Belgium}
}
\email{olivier@aampe.com}

\begin{abstract}
A/B-tests are a cornerstone of experimental design on the web, with wide-ranging applications and use-cases.
The statistical $t$-test comparing differences in means is the most commonly used method for assessing treatment effects, often justified through the Central Limit Theorem (CLT).
The CLT ascertains that, as the sample size grows, the sampling distribution of the Average Treatment Effect converges to normality, making the $t$-test valid for sufficiently large sample sizes.
When outcome measures are skewed or non-normal, quantifying what ``\emph{sufficiently large}'' entails is not straightforward.

To ensure that confidence intervals maintain proper coverage and that $p$-values accurately reflect the false positive rate, it is critical to validate this normality assumption.
We propose a practical method to test this, by analysing repeatedly resampled A/A-tests.
When the normality assumption holds, the resulting $p$-value distribution should be uniform, and this property can be tested using the Kolmogorov-Smirnov test.
This provides an efficient and effective way to empirically assess whether the $t$-test's assumptions are met, and the A/B-test is valid.
We demonstrate our methodology and highlight how it helps to identify scenarios prone to inflated Type-I errors.
Our approach provides a practical framework to ensure and improve the reliability and robustness of A/B-testing practices.
\end{abstract}

\maketitle

\section{Introduction}%

Decision-making is at the heart of artificial intelligence systems, enabling agents to navigate complex environments, achieve goals, and adapt to changing conditions. Traditional decision-making frameworks often rely on associations or statistical correlations between variables, which can lead to suboptimal outcomes when the underlying causal relationships are ignored \citep{pearl2009causal}. 
The rise of causal inference as a field has provided powerful frameworks and tools to address these challenges, such as structural causal models and potential outcomes frameworks \citep{rubin1978bayesian,pearl2000causality}. 
Unlike traditional methods, \textit{causal decision-making} focuses on identifying and leveraging cause-effect relationships, allowing agents to reason about the consequences of their actions, predict counterfactual scenarios, and optimize decisions in a principled way \citep{spirtes2000causation}. In recent years, numerous decision-making methods based on causal reasoning have been developed, finding applications in diverse fields such as recommender systems \citep{zhou2017large}, clinical trials \citep{durand2018contextual}, finance \citep{bai2024review}, and ride-sharing platforms \citep{wan2021pattern}. Despite these advancements, a fundamental question persists: 

\begin{center}
    \textit{When and why do we need causal modeling in decision-making?}
\end{center} 

% Numerous decision-making methods based on causal reasoning have been developed recently with wide applications 
% %Decision makings based on causal reasoning have been widely applied 
% in a variety of fields, including 
% recommender systems \citep{zhou2017large}, clinical trials \citep{durand2018contextual}, 
% finance \citep{bai2024review}, 
% ride-sharing platforms \citep{wan2021pattern}, and so on. 


 

% At the intersection of these fields, causal decision-making seeks to answer critical questions: How can agents make decisions when causal knowledge is incomplete? How do we integrate learning and reasoning about causality into real-world decision-making systems? What role do interventions, counterfactuals, and observational data play in guiding decisions? 

% Our review is structured as follows: 
 

This question is closely tied to the concept of counterfactual thinking—reasoning about what might have happened under alternative decisions or actions. Counterfactual analysis is crucial in domains where the outcomes of unchosen decisions are challenging, if not impossible, to observe. For instance, a business leader selecting one marketing strategy over another may never fully know the outcome of the unselected option \citep{rubin1974estimating, pearl2009causal}. Similarly, in econometrics, epidemiology, psychology, and social sciences, \textit{the inability to observe counterfactuals directly often necessitates causal approaches} \citep{morgan2015counterfactuals, imbens2015causal}. 
Conversely, non-causal analysis may suffice in scenarios where alternative outcomes are readily determinable. For example, a personal investor's actions may have negligible impact on stock market dynamics, enabling potential outcomes of alternate investment decisions to be inferred from existing stock price time series \citep{angrist2008mostly}. However, even in cases where counterfactual outcomes are theoretically calculable—such as in environments with known models like AlphaGo—exhaustively computing all possible outcomes is computationally infeasible \citep{silver2017mastering, silver2018general}. 
In such scenarios, causal modeling remains advantageous by offering \textit{structured ways to infer outcomes efficiently and make robust decisions}. 


%This perspective not only enhances the interpretability of decisions but also provides a principled framework for addressing uncertainty, guiding actions, and improving performance across a broad range of applications.

% Data-driven decision-making exists before the causal revolution. \textit{So when and why do we need causal modelling in decision-making?} 
% This is closely related to the presence of counterfactuals in many applications. 
% The counterfactual thinking involves considering what would have happened in an alternate scenario where a different decision or action was taken. 
% In many fields, including econometrics, epidemiology, psychology, and social sciences, accessing outcomes from unchosen decisions is often challenging if not impossible. 
% For example, a business leader who selects one marketing strategy over another may never know the outcome of the unselected option. 
% Conversely, non-causal analysis may be adequate in situations where potential outcomes of alternate actions are more readily determinable: for example, the investment of a personal investor may have minimal impact on the market, therefore her counterfactual investment decision's outcomes can still be calculated with the data of stock price time series. 
% However, it is important to note that even when counterfactuals are theoretically calculable, as in environments with known models like AlphaGo, computing all possible outcomes may not be feasible. 
% In such scenarios, a causal perspective  remains beneficial. 


 

% 1. significance of decision making
% 2. role of causal in decision making
% 3. refer to the https://jair.org/index.php/jair/article/view/13428/26917

% Decision makings based on causal reasoning have been widely applied in a variety of fields, including recommender systems \citep{zhou2017large}, clinical trials \citep{durand2018contextual}, 
% business economics scenarios \citep{shen2015portfolio}, 
% ride-sharing platforms \citep{wan2021pattern}, and so on. 
% However, most existing works primarily assume either sophisticated prior knowledge or strong causal models to conduct follow-up decision-making. To make effective and trustworthy decisions, it is critical to have a thorough understanding of the causal connections between actions, environments, and outcomes.

\begin{figure}[!t]
    \centering
    \includegraphics[width = .75\linewidth]{Figure/3Steps_V2.png}
    \caption{Workflow of the \acrlong{CDM}. $f_1$, $f_2$, and $f_3$ represent the impact sizes of the directed edges. Variables enclosed in solid circles are observed, while those in dashed circles are actionable.}\label{fig:cdm}
\end{figure}


Most existing works primarily assume either sophisticated prior knowledge or strong causal models to conduct follow-up decision-making. To make effective and trustworthy decisions, it is critical to have a thorough understanding of the causal relationships among actions, environments, and outcomes. This review synthesizes the current state of research in \acrfull{CDM}, providing an overview of foundational concepts, recent advancements, and practical applications. Specifically, this work discusses the connections of \textbf{three primary components of decision-making} through a causal lens: 1) discovering causal relationships through \textit{\acrfull{CSL}}, 2) understanding the impacts of these relationships through \textit{\acrfull{CEL}}, and 3) applying the knowledge gained from the first two aspects to decision making via \textit{\acrfull{CPL}}. 

Let $\boldsymbol{S}$ denote the state of the environment, which includes all relevant feature information about the environment the decision-makers interact with, $A$ the action taken, $\pi$ the action policy that determines which action to take, and $R$ the reward observed after taking action $A$. As illustrated in Figure \ref{fig:cdm}, \acrshort{CDM} typically begins with \acrshort{CSL}, which aims to uncover the unknown causal relationships among various variables of interest. Once the causal structure is established, \acrshort{CEL} is used to assess the impact of a specific action on the outcome rewards. To further explore more complex action policies and refine decision-making strategies, \acrshort{CPL} is employed to evaluate a given policy or identify an optimal policy. In practice, it is also common to move directly from \acrshort{CSL} to \acrshort{CPL} without conducting \acrshort{CEL}. Furthermore, \acrshort{CPL} has the potential to improve both \acrshort{CEL} and \acrshort{CSL} by facilitating the development of more effective experimental designs \citep{zhu2019causal,simchi2023multi} or adaptively refining causal structures \citep{sauter2024core}. %However, these are beyond the scope of this paper.

\begin{figure}[!t]
    \centering
    \includegraphics[width = .9\linewidth]{Figure/Table_of_Six_Scenarios_S.png}
    \caption{Common data dependence structures (paradigms) in \acrshort{CDM}. Detailed notations and explanations can be found in Section \ref{sec:paradigms}.}
    \label{Fig:paradigms}
\end{figure}
Building on this framework, decision-making problems discussed in the literature can be further categorized into \textbf{six paradigms}, as summarized in Figure \ref{Fig:paradigms}. These paradigms summarize the common assumptions about data dependencies frequently employed in practice. Paradigms 1-3 describe the data structures in offline learning settings, where data is collected according to an unknown and fixed behavior policy. In contrast, paradigms 4-6 capture the online learning settings, where policies dynamically adapt to newly collected data, enabling continuous policy improvement. These paradigms also reflect different assumptions about state dependencies. The simplest cases, paradigms 1 and 4, assume that all observations are independent, implying no long-term effects of actions on future observations. To account for sequental dependencies, the \acrfull{MDP} framework, summarized in paradigms 2 and 5, assumes Markovian state transition. Specifically, it assumes that given the current state-action pair $(S_t, A_t)$, the next state $S_{t+1}$ and reward $R_t$ are independent of all prior states $\{S_j\}_{j < t}$ and actions $\{A_j\}_{j < t}$. When such independence assumptions do not hold, paradigms 3 and 6 account for scenarios where all historical observations may impact state transitions and rewards. This includes but not limited to researches on \acrfull{POMDP} \citep{hausknecht2015deep, littman2009tutorial}, panel data analysis \citep{hsiao2007panel,hsiao2022analysis}, \acrfull{DTR} with finite stages \citep{chakraborty2014dynamic, chakraborty2013statistical}. 

Each \acrshort{CDM} task has been studied under different paradigms, with \acrshort{CSL} extensively explored within paradigm 1. \acrshort{CEL} and offline \acrshort{CPL} encompass paradigms 1-3, while online \acrshort{CPL} spans paradigms 4-6. By organizing the discussion around these three tasks and six paradigms, this review aims to provide a cohesive framework for understanding the field of \acrlong{CDM} across diverse tasks and data structures.

%Recognizing the importance of long-term effects in decision-making

%Further discussions on these paradigms and their connections to various causal decision-making problems are provided in Section \ref{sec:paradigms}.


\textbf{Contribution.} In this paper, we conduct a comprehensive survey of \acrshort{CDM}. 
Our contributions are as follows. 
\begin{itemize}
    \item We for the first time organize the related causal decision-making areas into three tasks and six paradigms, connecting previously disconnected areas (including economics, statistics, machine learning, and reinforcement learning) using a consistent language. For each paradigm and task, we provide a few taxonomies to establish a unified view of the recent literature.
    \item We provide a comprehensive overview of \acrshort{CDM}, covering all three major tasks and six classic problem structures, addressing gaps in existing reviews that either focus narrowly on specific tasks or paradigms or overlook the connection between decision-making and causality (detailed in Section \ref{sec::related_work}).
    %\item We outline three key challenges that emerge when utilizing CDM in practice. Moreover, we delve into a comprehensive discussion on the recent advancements and progress made in addressing these challenges. We also suggest six future directions for these problems.
    \item We provide real-world examples to illustrate the critical role of causality in decision-making and to reveal how \acrshort{CSL}, \acrshort{CEL} and \acrshort{CPL} are inherently interconnected in daily applications, often without explicit recognition.
    \item We are actively maintaining and expanding a GitHub repository and online book, providing detailed explanations of key methods reviewed in this paper, along with a code package and demos to support their implementation, with URL: \url{https://causaldm.github.io/Causal-Decision-Making}.
\end{itemize}
% Our review is structured as follows: 


%%%%%%%%%%%%%%%%%%%%%%%%%%%%%%%%%%
%  causal helps over "Correlational analysis"
%Correlational analysis, though widely used in various fields, has inherent limitations, particularly when it comes to decision-making. While it identifies relationships between variables, it fails to establish causality, often leading to misinterpretations and misguided decisions. For example, the positive correlation between ice cream sales and drowning incidents is a classic example of how correlational data can be misleading, as both are influenced by a third factor, temperature, rather than causing each other. Such spurious correlations, due to oversight of confounding variables, underscore the necessity of causal modeling in decision making. Causal models excel where correlational analysis falls short, offering predictive power and a deeper understanding of underlying mechanisms. They enable us to predict the outcomes of interventions, even under untested conditions, and provide insights into the processes leading to these outcomes, thereby informing more effective strategies. Moreover, causal models are good at generalizing findings across different contexts, a capability often limited in purely correlational studies. 

%  causal helps in causal RL 
%From another complementary angle, although causal concepts have traditionally not been explicitly incorporated in fields like online bandits \citep{lattimore2020bandit} and \acrfull{RL} \citep{sutton2018reinforcement}, much of the literature in these areas implicitly relies on basic assumptions outlined in Section \ref{sec:prelim_assump} to utilize observed data in place of potential outcomes in their analyses, and there is also a growing recognition of the significance of the causal perspective \citep{lattimore2016causal, zeng2023survey} in these areas. 
% \textbf{Read causal RL survey and summarize. } However, by integrating causal concepts and leverging existing methodologies, we open up possibilities for developing more robust models to remove spurious correlation and selection bias \citep{xu2023instrumental, forney2017counterfactual}, designing more sample-efficient \citep{sontakke2021causal, seitzer2021causal} and robust \citep{dimakopoulou2019balanced, ye2023doubly} algorithms, and improving the generalizability \citep{zhang2017transfer, eghbal2021learning}, explanability \citep{foerster2018counterfactual, herlau2022reinforcement}, and fairness \citep{zhang2018fairness,huang2022achieving,balakrishnan2022scales} of these methods. %, and safety \cite{hart2020counterfactual}

%


%\subsection{Paper Structure}
The remainder of this paper is organized as follows: Section \ref{sec::related_work} provides an overview of related survey papers. Section \ref{sec:preliminary} introduces the foundational concepts, assumptions, and notations that form the foundation for the subsequent discussions. In Section \ref{sec:3task6paradigm}, we offer a detailed introduction to the three key tasks and six learning paradigms in \acrshort{CDM}. Sections \ref{Sec:CSL} through \ref{sec:Online CPL} form the core of the paper, with each section dedicated to a specific topic within \acrshort{CDM}: \acrshort{CSL}, \acrshort{CEL}, Offline \acrshort{CPL}, and Online \acrshort{CPL}, respectively. Section \ref{sec:assump_violated} then explores extensions needed when standard causal assumptions are violated. To illustrate the practical application of the \acrshort{CDM} framework, Section \ref{sec:real_data} presents two real-world case studies. Finally, Section \ref{sec:conclusion} concludes the paper with a summary of our contributions and a discussion of additional research directions that are actively being explored.



\section{Problem Statement \& Methodology}
\subsection{Estimating a confidence interval for the ATE}
Our aim is to leverage online controlled experiments to assess the ATE of an intervention on some outcome of interest $Y$.
Without loss of generality, we assume that this random variable indicates a logged user-level event (e.g. a user opens the app, clicks, converts, renews, churns, et cetera).
We denote the intervention by superscript, for treatment $Y^{\rm T}$ and control $Y^{\rm C}$.
Our estimand is then, with an expectation over experiment randomisation units (i.e. users):
\begin{equation}
    \mathop{\rm ATE}\limits_{{\rm C} \to {\rm T}}(Y) = \mathbb{E}[Y^{\rm T}-Y^{\rm  C}].
\end{equation}

A straightforward estimator for the ATE is given by the difference in sample means.
For a set of users belonging to a \emph{group} (i.e. control C, treatment T, or a general A/B-testing group A) and their observed outcomes $Y$, we have:
\begin{equation}
    \mu_{\rm A}(Y) = \frac{1}{|\mathcal{U}_{\rm A}|} \sum_{i \in \mathcal{U}_{\rm A}} Y_i,\, \, \, \, \enskip \text{and} \, \, \, \, \enskip \widehat{\mathop{\rm ATE}\limits_{{\rm C} \to {\rm T}}}(Y)  =  \mu_{\rm T}(Y) - \mu_{\rm C}(Y).
\end{equation}

To quantify the uncertainty in the estimate, we wish to construct a confidence interval.
The CLT tells us that the distribution of $\mathbb{E}[Y]$ converges to a normal distribution, and hence, so does the distribution for the ATE.
This implies that we can compute:
\begin{align}
    \sigma^{2}_{\rm A}(Y) &= \frac{1}{|\mathcal{U}_{\rm A}|}\sum_{i \in \mathcal{U}_{\rm A}}(Y_i - \mu_{\rm A}(Y))^{2}, \\
   \text{SE}\left(\widehat{\mathop{\rm ATE}\limits_{{\rm C} \to {\rm T}}}(Y)\right) &= \sqrt{\frac{\sigma^{2}_{\rm C}(Y)}{|\mathcal{U}_{\rm C}|} + \frac{\sigma^{2}_{\rm T}(\rm Y)}{|\mathcal{U}_{\rm T}|}}.
\end{align}
A $100\cdot(1-\alpha)\%$ confidence interval can then be obtained as:
\begin{equation}
    \widehat{\mathop{\rm ATE}\limits_{{\rm C} \to {\rm T}}}(Y) \pm  \Phi^{-1}\left(1-\frac{\alpha}{2}\right)\cdot\text{SE}\left(\widehat{\mathop{\rm ATE}\limits_{{\rm C} \to {\rm T}}}(Y)\right),
\end{equation}
where the inverse cumulative distribution function for the standard normal distribution $\Phi^{-1}$ gives the critical value for confidence level $\alpha$.
It should include the ground truth ATE in $100\cdot(1-\alpha)\%$ of cases.
A confidence interval around the ATE is a crucial component to consider when properly interpreting A/B-testing results.

In  a statistical hypothesis testing framework, when zero is not contained by this interval, the null hypothesis is rejected and the result is deemed significant at level $\alpha$.
Alternatively, we can construct a two-tailed $p$-value as:
\begin{equation}\label{eq:pvalue}
    p = 2 \left(1- \Phi\left( \left| \underbrace{\frac{\widehat{\mathop{\rm ATE}\limits_{{\rm C} \to {\rm T}}}(Y)}{\text{SE}\left(\widehat{\mathop{\rm ATE}\limits_{{\rm C} \to {\rm T}}}(Y)\right)}}_{z\text{-score}} \right| \right)\right),
\end{equation}
and reject the null hypothesis when $p < \alpha$.
The $p$-value can be described as the probability of observing results at least as extreme as what is observed, given that the null hypothesis holds true.

The meaning that is ascribed to both confidence intervals and $p$-values relies heavily on the assumption that the distribution of the estimand has approached normality. 
Whilst the CLT guarantees this property to hold asymptotically, finite sample scenarios require us to empirically validate that the above procedure is appropriate. 

\subsection{Empirically validating confidence intervals}
Naturally, directly validating whether the obtained CI includes the true ATE would require knowledge of the latter, which is prohibitive.
Alternatively, A/A-tests allow us to emulate experiments where we know the true ATE by design (i.e. $0$, as the null hypothesis holds).
This enables us to estimate the empirical coverage of the obtained CIs, through repeated resampling of A/A-groups.
When the distribution of the ATE has approached normality, the distribution of the $p$-values that we obtain over resampled A/A-tests should resemble a uniform distribution.
The Kolmogorov-Smirnov test provides a rigorous statistical framework to flag cases where it does not.
This allows us to, for a set of users $\mathcal{U}$ and outcomes $Y$, assess how amenable the data is to reliable estimation of CIs on ATE($Y$) using the above-mentioned standard methods.

As such, we repeatedly resample groups ${\rm A}_{i}$,${\rm A}_{i}^\prime$ for $n$ iterations, obtain $n$ confidence intervals for $\widehat{\mathop{\rm ATE}\limits_{{\rm A}_i \to {\rm A}_{i}^{\prime}}}(Y)$ and obtain a set of $p$-values $\{p_{1},\ldots,p_{n}\}$.
Given the empirical Cumulative Distribution Function (eCDF) of $p$-values $F_{\rm emp}(p)$, we wish to assess how it deviates from the uniform distribution with CDF $F_{\rm uni}(p)=p$ for $p\in[0,1]$.
The test statistic leveraged by Kolmogorov-Smirnov, known as the $D$-statistic, measures the $\infty$-norm over the observed differences between the two CDFs as:
\begin{equation}
    D = \sup_{p\in[0,1]} \left|F_{\rm emp}(p) - F_{\rm uni}(p)\right|.
\end{equation}
Under the null hypothesis that the distributions are equivalent, $D$ follows a Kolmogorov distribution.
As such, we can obtain a $p$-value that is used to reject the null hypothesis that the $p$-values obtained from the A/A-tests are uniformly distributed, or, that we have sufficient samples to reliably estimate treatment effects on $Y$.
Note that this is equivalent to testing whether the $z$-scores in Equation~\ref{eq:pvalue} follow a standard normal distribution.
Arguments against the statistical hypothesis testing framework apply here as well.
We suggest to pay special attention to cases where the $D$-statistic is high, or conversely, the Kolmogorov Smirnov $p$-value is low, rather than assigning binary (non-)significant labels to metrics.

In cases where an outcome $Y$ is flagged through this procedure, CIs and $p$-values obtained through the $t$- or $z$-test should be considered unreliable.
We can rely on alternative non-parametric methods to estimate CIs on ATE($Y$) in those situations, e.g. based on permutation sampling or bootstrapping.
We note that other commonly used methods like the Mann-Whitney U-test formulate a different null hypothesis, and cannot be used as a drop-in replacement without careful consideration~\cite{Fay2010}.
Furthermore, as the above-mentioned alternatives typically come with a considerable computational cost and specialised engineering solutions, they are significantly less desirable as a default approach.
A deeper exploration of their applicability falls outside of the scope for this work, but provides an interesting avenue for future research.

\section{Experiments}
In our experiments, we evaluate the proposed method on three real-world benchmark datasets, focusing on the following key research questions (RQs): \textbf{RQ1}: Does the proposed unified representation learning method outperform state-of-the-art sequential recommendation models in terms of prediction accuracy? \textbf{RQ2}: What is the impact of our unified semantic and ID tokenization method on recommendation performance? Additionally, is the integration of cosine similarity and Euclidean distance effective in improving the final recommendation performance?
\textbf{RQ3}: To what extent can we reduce the dimensionality of ID tokens without compromising performance? Specifically, how does the model’s performance vary with different ID token dimensions?
\textbf{RQ4}: What patterns do the semantic and ID tokens learn, and how do these tokens contribute to the overall representation of items?

Additionally, in Appendix~\ref{appendix:codebooksize}, we explore the effects of varying the codebook size on the patterns learned by the semantic tokens.

\subsection{Experimental Setup}

\paragraph{\textbf{Datasets}} We evaluate the recommendation performance on Amazon product review datasets~\citep{he2016ups}. The statistics of these three benchmark datasets after applying 5-core filtering are presented in Appendix~\ref{appendix:data}.

\paragraph{\textbf{Evaluation Metrics}} We follow the approach used in prior work~\citep{zhou2020s3}, using Hit Ratio (HIT@k), Normalized Discounted Cumulative Gain (NDCG@k), and Mean Reciprocal Rank (MRR) as evaluation metrics, where $k$ is the number of top ranked items. Consistent with previous studies~\citep{zhou2020s3, dcn}, given a user behavior sequence, we use the last item for testing, the second-to-last item for validation, and the rest for training. Given the large item set, ranking against all possible items is computationally expensive. Therefore, following a commonly used approach~\citep{sasrec, man}, we evaluate the model by sampling 99 negative items along with the ground-truth item. All metrics are calculated based on the ranking of sampled and ground-truth items, and we present the mean scores across users.

\paragraph{\textbf{Baselines}}
To evaluate the pure impact of semantic tokenization, we compare our proposed method against several competitive recommendation baselines, including FM~\citep{fm}, GRU4Rec~\citep{gru4rec}, Caser~\citep{caser}, SASRec~\citep{sasrec}, BERT4Rec~\citep{bert4rec}, and HGN~\citep{hgn}. It is important to note that we do not compare our method with existing work~\citep{rajput2024recommender} that utilizes a different model architecture with a deeper network when incorporating RQ-VAE. The primary focus here is to examine the effects of semantic tokenization within the context of the same sequential recommendation model to ensure a fair and consistent comparison. Besides, we directly use the results of all baseline from prior work~\citep{zhou2020s3} and implement our method based on SASRec under its framework for a fair comparison. Besides, we show the detailed description of these baselines in Appendix~\ref{appendix:baseline}.

\paragraph{\textbf{Hyper-parameter Settings}} We directly use the results of all baseline from prior work~\citep{zhou2020s3} and implement our method based on its framework for a fair comparison. Besides, we set some new hyper-parameters of RQ-VAE following prior work~\citep{rajput2024recommender} with $L=3$ layers of codebook. We search the codebook size $K$ from 64 to 1024 and select 256 for both Beauty and Toys dataset, while 128 for Sports dataset. Besides, we set the dimension of codebook $D' = 64$ to align with the ID token only method. All other parameters like recommendation model layer and hidden size are set strictly the same as baselines. Additionally, we put more implementation details in Appendix~\ref{appendix:implementation}.

\subsection{Overall Performance}

\begin{table*}[t!]
\centering
\caption{Our method improves baseline significantly by 6\% to around 18\% on three benchmark datasets.}
\label{tab:overall}
% \tabcolsep=1mm
\begin{tabular}{cc|cccccc|c|c}
\toprule
Datasets & Metric  & FM     & GRU4Rec & Caser  & SASRec       & BERT4Rec     & HGN    & Ours            & Improv. \\ \midrule
\multirow{5}{*}{Beauty} & HIT@5 & 0.1461 & 0.3125 & 0.3032 & {\ul 0.3741} & 0.3640 & 0.3544 & \textbf{0.4201} & 12.30\% \\  
         & NDCG@5  & 0.0934 & 0.2268  & 0.2219 & {\ul 0.2848} & 0.2622       & 0.2656 & \textbf{0.3079} & 8.11\%  \\  
         & HIT@10   & 0.2311 & 0.4106  & 0.3942 & 0.4696       & {\ul 0.4739} & 0.4503 & \textbf{0.5318} & 12.22\% \\  
         & NDCG@10 & 0.1207 & 0.2584  & 0.2512 & {\ul 0.3156} & 0.2975       & 0.2965 & \textbf{0.3440} & 9.00\%  \\  
         & MRR     & 0.1096 & 0.2308  & 0.2263 & {\ul 0.2852} & 0.2614       & 0.2669 & \textbf{0.3025} & 6.07\%  \\ \midrule
\multirow{5}{*}{Sports} & HIT@5    & 0.1603 & 0.3055  & 0.2866 & {\ul 0.3466} & 0.3375       & 0.3349 & \textbf{0.3849} & 11.05\% \\
                        & NDCG@5  & 0.1048 & 0.2126  & 0.2020 & {\ul 0.2497} & 0.2341       & 0.2420 & \textbf{0.2717} & 8.81\%  \\
                        & HIT@10   & 0.2491 & 0.4299  & 0.4014 & 0.4622       & {\ul 0.4722} & 0.4551 & \textbf{0.5247} & 11.12\% \\
                        & NDCG@10 & 0.1334 & 0.2527  & 0.2390 & {\ul 0.2869} & 0.2775       & 0.2806 & \textbf{0.3168} & 10.42\% \\
                        & MRR     & 0.1202 & 0.2191  & 0.2100 & {\ul 0.2520} & 0.2378       & 0.2469 & \textbf{0.2722} & 8.02\%  \\ \midrule
\multirow{5}{*}{Toys}   & HIT@5 & 0.0978 & 0.2795 & 0.2614 & {\ul 0.3682} & 0.3344 & 0.3276 & \textbf{0.4340} & 17.87\% \\  
         & NDCG@5  & 0.0614 & 0.1919  & 0.1885 & {\ul 0.2820} & 0.2327       & 0.2423 & \textbf{0.3141} & 11.38\% \\  
         & HIT@10   & 0.1715 & 0.3896  & 0.3540 & {\ul 0.4663} & 0.4493       & 0.4211 & \textbf{0.5456} & 17.01\% \\  
         & NDCG@10 & 0.0850 & 0.2274  & 0.2183 & {\ul 0.3136} & 0.2698       & 0.2724 & \textbf{0.3501} & 11.64\% \\  
         & MRR     & 0.0819 & 0.1973  & 0.1967 & {\ul 0.2842} & 0.2338       & 0.2454 & \textbf{0.3064} & 7.81\%  \\ \bottomrule
\end{tabular}
\end{table*}
To compare the performance of our method with existing sequential recommenders, as shown in Table~\ref{tab:overall}, we evaluate them in three benchmark datasets under five metrics. From the table, we can have the following observation: \textbf{Our method achieves significant improvement.} The improvement of our method towards baselines ranges from 6.07\% to 17.87\%, which is very significant in sequential recommendation task~\citep{sasrec, zhou2020s3}. Besides, our method improves more on HIT metric than NDCG metric and MRR metric. This may be because semantic embedding is naturally less insensitive at ranking position due to duplicate tokenization, though we have added unique ID embedding.


\subsection{Ablation Study}
% Please add the following required packages to your document preamble:
% \usepackage{multirow}
% \usepackage[normalem]{ulem}
% \useunder{\uline}{\ul}{}
\begin{table*}[!htb]
% \tabcolsep=0.5mm
\small
\centering
\caption{Unified tokenization outperforms ID-only and semantic-only tokenizations with significant reduction of token size. Besides, the semantic tokenization outperforms ID tokenization in position-insensitive metric.}
\label{tab:token}
\begin{tabular}{cc|c|c|c|c|c|c|c}
\toprule
\multirow{2}{*}{Dataset} &
  \multirow{2}{*}{Method} &
  \multicolumn{3}{c|}{Metric} &
  \multicolumn{3}{c|}{Token Size} &
  \multirow{2}{*}{\begin{tabular}[c]{@{}c@{}}Token\\ Reduction\end{tabular}} \\ 
 &
   &
  HIT@10 &
  NDCG@10 &
  MRR &
  ID &
  Semantic &
  Total &
   \\ \midrule
\multirow{3}{*}{Beauty} &
  ID &
  0.4654 &
  {\ul 0.3121} &
  {\ul 0.282} &
  12,101 $\times$ 64 &
  0 &
  774,464 &
  \textbackslash{} \\  
 &
  Semantic &
  {\ul 0.4956} &
  0.2914 &
  0.2476 &
  0 &
  3 $\times$ 256 $\times$ 64 &
  49,152 &
  93.65\% \\  
 &
  Unified &
  \textbf{0.5318} &
  \textbf{0.344} &
  \textbf{0.3025} &
  12,101 $\times$ 8 &
  3 $\times$ 256 $\times$ 64 &
  145,960 &
  81.15\% \\ \midrule
\multirow{3}{*}{Sports} &
  ID &
  0.4582 &
  {\ul 0.2826} &
  {\ul 0.2482} &
  18,357 $\times$ 64 &
  0 &
  1,174,848 &
  \textbackslash{} \\  
 &
  Semantic &
  {\ul 0.4704} &
  0.2554 &
  0.2131 &
  0 &
  3 $\times$ 128 $\times$ 64 &
  24,576 &
  97.91\% \\  
 &
  Unified &
  \textbf{0.5247} &
  \textbf{0.3168} &
  \textbf{0.2722} &
  18,357 $\times$ 8 &
  3 $\times$ 128 $\times$ 64 &
  171,432 &
  85.41\% \\ \midrule
\multirow{3}{*}{Toys} &
  ID &
  0.4603 &
  {\ul 0.3092} &
  {\ul 0.2804} &
  11,924 $\times$ 64 &
  0 &
  763,136 &
  \textbackslash{} \\  
 &
  Semantic &
  {\ul 0.4644} &
  0.2741 &
  0.236 &
  0 &
  3 $\times$ 256 $\times$ 64 &
  49,152 &
  93.56\% \\  
 &
  Unified &
  \textbf{0.5456} &
  \textbf{0.3501} &
  \textbf{0.3064} &
  11,924 $\times$ 8 &
  3 $\times$ 256 $\times$ 64 &
  144,544 &
  81.06\% \\ \bottomrule
\end{tabular}
\end{table*}
To further study the performance of different tokenization methods, we compare our method with the ID tokenization only method and semantic tokenization only method as Table~\ref{tab:token}. From the table, we can have the following observations: (1) \textbf{Unified tokenization performs best with significant reduction of token.} In all these three benchmark datasets, our proposed method is significantly superior to solely ID tokenization and semantic tokenization methods. More importantly, compared with the traditional ID tokenization method, our method reduces by at least 80\% and even 85\% of tokens on Sports dataset. Here we reduce the tokens by replacing 56 dimensions of ID tokens with a small amount of semantic tokens, which supports our previous analysis that most ID tokens are redundant. (2) \textbf{Semantic tokenization outperforms ID tokenization in position-insensitive metric with significant reduction of token.} In three datasets, it is obvious that the semantic tokenization only method even outperforms ID tokenization only method on HIT metric with less than 10\% of tokens. This result also supports our previous analysis that semantic tokenization is effective at generalization and capturing high-level semantic information. However, semantic tokenization only method often performs poor at NDCG and MRR metrics which are sensitive to position. This is because the position of duplicate tokenized items from semantic tokenization only method are hard to distinguish in ranking.
           

% Please add the following required packages to your document preamble:
% \usepackage{multirow}
\begin{table*}[!htb]
% \tabcolsep=0.5mm
\centering
\caption{The integration of Euclidean distance into cosine similarity can improve the recommendation performance.}
\label{tab:abl_distance}
\begin{tabular}{c|ccc|ccc|ccc}
\toprule
\multirow{2}{*}{Method} & \multicolumn{3}{c|}{Beauty} & \multicolumn{3}{c|}{Sports} & \multicolumn{3}{c}{Toys}  \\  
                        & HIT@10  & NDCG@10 & MRR    & HIT@10  & NDCG@10 & MRR    & HIT@10 & NDCG@10 & MRR    \\ \midrule
Cosine       & 0.5212  & 0.3334  & 0.2921 & 0.5129  & 0.3081  & 0.2649 & 0.5252 & 0.3309  & 0.2879 \\ 
Unified & \textbf{0.5318} & \textbf{0.3440} & \textbf{0.3025} & \textbf{0.5247} & \textbf{0.3168} & \textbf{0.2722} & \textbf{0.5456} & \textbf{0.3501} & \textbf{0.3064} \\ \bottomrule
\end{tabular}
\end{table*}
Besides, we also compare our method with cosine similarity only method when searching the codebook of RQ-VAE, as shown in Table~\ref{tab:abl_distance}. From the table, we can observe that: \textbf{Our unified method outperforms cosine similarity.} The unified method which integrates cosine similarity with Euclidean distance proposed in Section~\ref{sec:unified_distance} outperforms the solely cosine similarity method on three benchmark datasets. This means our unified cosine similarity and Euclidean distance not only can improve the percentage of activated codebook and coverage of unique items, but also can really improve the final recommendation performance.


\subsection{Hyper-parameter Study}
\begin{figure*}[htb!]
		\centering
		\begin{tabular}{ccc}
		    	\includegraphics[width=0.24\linewidth]{fig/hit.pdf} &  \includegraphics[width=0.24\linewidth]{fig/NDCG.pdf} &
       \includegraphics[width=0.24\linewidth]{fig/mrr.pdf} 
		\end{tabular}
	\caption{The performance improvement shrinks when scaling up dimension of ID token, which means a small proportion of ID tokens is sufficient for capturing the item's unique characteristic.}	\label{fig:hyper_id}
\end{figure*} 
To further verify that we only need a small proportion of ID tokens, we further vary the ID dimension from $\{0, 4, 8, 16\}$ and study the performance under three key metrics as Figure~\ref{fig:hyper_id}. From the figure, we discvoer that: \textbf{The performance improvement shrinks when scaling up dimension of ID token.} It is obvious that the performance improvement becomes less and less with the growing of ID token dimension, and the performance even drops when dimension is greater than 8. This means a small proportion of ID tokens is sufficient for learning the unique information, and others are indeed redundant and can be saved.

\subsection{Token Visualization}
\begin{figure*}[htb!]
		\centering
		\begin{tabular}{cccc}
\includegraphics[width=0.2\linewidth]{fig/first_layerbeauty_4.png} &
       \includegraphics[width=0.2\linewidth]{fig/second_layerbeauty_4.png}  & \includegraphics[width=0.2\linewidth]{fig/third_layerbeauty_3.png}  & \includegraphics[width=0.2\linewidth]{fig/uniquebeauty_3.png}
		     \\ First Codebook & Second Codebook & Third Codebook & Unique Tokens
		\end{tabular}
	\caption{The patterns of codebooks are various across different layers and unique tokens are uniform for different items on Beauty dataset.}	\label{fig:vis_beauty}
\end{figure*} 




% \begin{figure}[htb!]
% 		\centering
% 		\begin{tabular}{c|c|c}
% \includegraphics[width=0.33\linewidth]{fig/uniquebeauty_3.png} &
%        \includegraphics[width=0.33\linewidth]{fig/uniqueSports_and_Outdoors.png}  & \includegraphics[width=0.33\linewidth]{fig/uniqueToys_and_Games.png}  
% 		     \\ (a) Beauty & (b) Sports & (c) Toys 
% 		\end{tabular}
% 	\caption{Visualization of ID tokens on three datasets.}	\label{fig:vis_id}
% \end{figure} 
To study the learned semantic and ID tokens, we further visualize these tokens on Beauty dataset using t-SNE, as shown in Figure~\ref{fig:vis_beauty}. Besides, we also visualize the tokens on Sports and Toys datasets in \ref{fig:vis_sport} and \ref{fig:vis_toys} of Appendix~\ref{sec:visual_token}. Here we label each semantic token with a unique color and thus these are totally $K$ types of color. In ID tokens, we also label them with $K$ types of color to show the distribution when they are assigned with one of the codebooks. Based on the visualized results, we can discover that: (1) \textbf{Semantic tokens vary across different layers.} It is obvious that the semantic tokens vary across different layer on all datasets, which means different layers of semantic codebooks can capture various shared patterns. With the combination of these shared patterns, we can better represent each item's semantic information. (2) \textbf{ID tokens distribute uniformly.} The unique ID tokens are uniform on all datasets. This means the ID token successfully capture the unique characteristic of each item and thus they will not accumulate together.
\section{Conclusions \& Outlook}
A/B-tests are omnipresent in modern technology companies, often seen as the ``gold standard'' of experimentation practices.
The default estimand is typically the ATE on a metric of interest, and statistical uncertainty on the estimate is handled through standard methods. 
It is often forgotten that these methods rely on implicit assumptions that might be violated, invalidating the estimates.

Most commonly, we assume that the distribution of the ATE has converged to normality due to the CLT. 
If this is false, the CIs and $p$-values we use to assess A/B-testing outcomes become misleading.
In this work, we propose an efficient and effective manner to validate this assumption empirically, through the use of repeatedly resampled A/A-tests and a Kolmogorov-Smirnov test on the uniformity of the resulting $p$-value distribution.
We describe our approach, present empirical results on real-world data that both elucidate the methods and highlight that potential proxies (like sample size or skewness) are promising but imperfect.
This provides a practical framework to assess A/B-testing estimands and avoid situations where improper estimation methods are applied, in the hope that it will help the community to enforce statistical rigour and avoid inflated false positive risk going forward.

% \section*{Presenter Biography}
% \textbf{Olivier Jeunen} is Principal Research Scientist at Aampe with a PhD from the University of Antwerp, who has previously held positions at ShareChat, Amazon, Spotify, Meta and Criteo.
% His research focuses on applying ideas from causal and counterfactual inference to recommendation and advertising problems, which have led to 45\textsuperscript{+} peer reviewed contributions, two of which have been recognised with best paper awards. He is an active (Senior) Program Committee
% member in the community---which has led to four outstanding reviewer awards. 
% Olivier has been a workshop and conference organising committee member in various roles, recently co-chairing the Industry Tracks at ECIR '24 and RecSys '25.

\bibliographystyle{ACM-Reference-Format}
\bibliography{bibliography}

\end{document}