% IEEEAerospace2012.cls requires the following packages: times, rawfonts, oldfont, geometry
\documentclass[twocolumn,letterpaper]{IEEEAerospaceCLS}  % only supports two-column, letterpaper format

% The next line gives some packages you may find useful for your paper--these are not required though.
%\usepackage[]{graphicx,float,latexsym,amssymb,amsfonts,amsmath,amstext,times,psfig}
% NOTE: The .cls file is now compatible with amsmath!!!

\usepackage[]{graphicx}    % We use this package in this document
\newcommand{\ignore}[1]{}  % {} empty inside = %% comment
\usepackage{hyperref}
\usepackage{tabularx,booktabs}
%\usepackage{subfigure}
\usepackage{subcaption}
\usepackage[noadjust]{cite}
\usepackage{url}
\usepackage{dblfloatfix}    % To enable figures at the bottom of page


\usepackage{lipsum}
\usepackage{fancyhdr}
\usepackage{amsmath,amssymb}
\usepackage{gensymb}

\usepackage{fancyhdr}
\usepackage{lipsum}

%Define footers
%\fancypagestyle{Whatever}{
%  \fancyhf{} %Clear header/footer
%  \renewcommand{\headrulewidth}{0pt} %No line in header area
%  \fancyfoot[C]{DISTRIBUTION STATEMENT A: Approved for public release. %Distribution is unlimited. \\ \thepage} %Centered footer
%}

%Set page style
%\pagestyle{Whatever} %Set page style.
%\renewcommand\chapterpagestyle{Whatever} %New command

%\fancypagestyle{plain}{}




\begin{document}
\title{Seasonal Station-Keeping of Short Duration High Altitude Balloons using Deep Reinforcement Learning}

\author{%
Tristan K. Schuler\\ 
U.S. Naval Research Laboratory\\
4555 Overlook Ave. S.W\\
Washington, D.C. 20375\\
tristan.k.schuler.civ@us.navy.mil
\and 
Chinthan Prasad\\
U.S. Naval Research Laboratory\\
4555 Overlook Ave. S.W\\
Washington, D.C. 20375\\
chinthan.b.prasad.civ@us.navy.mil
\and 
Georgiy Kiselev\\
U.S. Naval Research Laboratory (NREIP Intern)\\
4555 Overlook Ave. S.W\\
Washington, D.C. 20375\\
georgiy.a.kiselev.ctr@us.navy.mil
\and 
Donald Sofge\\ 
U.S. Naval Research Laboratory\\
4555 Overlook Ave. S.W\\
Washington, D.C. 20375\\
donald.a.sofge.civ@us.navy.mil
%%%% IMPORTANT: Use the correct copyright information--IEEE, Crown, or U.S. government. %%%%%
\thanks{U.S. Government work not protected by U.S. copyright}              % This creates the copyright info that is the correct 2023 data.
%\thanks{\footnotesize 978-1-6654-9032-0/23/$\$31.00$ \copyright2023 IEEE}
%\thanks{{U.S. Government work not protected by U.S. copyright}}         % Use this copyright notice only if you are employed by the U.S. Government.
%\thanks{{978-1-6654-9032-0/23/$\$31.00$ \copyright2023 Crown}}          % Use this copyright notice only if you are employed by a crown government (e.g., Canada, UK, Australia).
%\thanks{{978-1-6654-9032-0/23/$\$31.00$ \copyright2023 European Union}}    % Use this copyright notice is you are employed by the European Union.
}



\maketitle

%\thispagestyle{plain}
\fancypagestyle{firstpage}{
    \fancyhf{} % Clear all headers and footers
    \fancyfoot[C]{DISTRIBUTION STATEMENT A: Approved for public release, distribution is unlimited.} % Footer in the center
    %\renewcommand{\footrulewidth}{0.4pt} % Optional: Add a line above the footer
    \renewcommand{\headrulewidth}{0pt}
}
\pagestyle{plain}



\maketitle
\thispagestyle{firstpage}
\pagestyle{plain}

\begin{abstract} 
Station-Keeping short-duration high-altitude balloons (HABs) in a region of interest is a challenging path-planning problem due to partially observable, complex, and dynamic wind flows. Deep reinforcement learning is a popular strategy for solving the station-keeping problem. A custom simulation environment was developed to train and evaluate Deep Q-Learning (DQN) for short-duration HAB agents in the simulation. To train the agents on realistic winds, synthetic wind forecasts were generated from aggregated historical radiosonde data to apply horizontal kinematics to simulated agents. The synthetic forecasts were closely correlated with ECWMF ERA5 Reanalysis forecasts, providing a realistic simulated wind field and seasonal and altitudinal variances between the wind models. DQN HAB agents were then trained and evaluated across different seasonal months. To highlight differences and trends in months with vastly different wind fields, a Forecast Score algorithm was introduced to independently classify forecasts based on wind diversity, and trends between station-keeping success and the Forecast Score were evaluated across all seasons.
\end{abstract} 



\tableofcontents



\section{Introduction}
Implicit Neural Representations (INRs), which fit the target function using only input coordinates, have recently gained significant attention.
%
By leveraging the powerful fitting capability of Multilayer Perceptrons (MLPs), INRs can implicitly represent the target function without requiring their analytical expressions. 
%
The versatility of MLPs allows INRs to be applied in various fields, including inverse graphics~\citep{mildenhall2021nerf, barron2023zip, martin2021nerf}, image super-resolution~\citep{chen2021learning, yuan2022sobolev, gao2023implicit}, 
image generation~\citep{skorokhodov2021adversarial}, and more~\citep{chen2021nerv, strumpler2022implicit, shue20233d}.
%
\begin{figure}
    \includegraphics[width=0.5\textwidth]{Image/Fig2.pdf}
    \caption{As illustrated at the circled blue regions and green regions, it can be observed that even with well-chosen standard deviation/scale, as experimented in \autoref{figure:combined}, the results are still unsatisfactory. However, using our proposed method, the noise is significantly alleviated while further enhancing the high-frequency details.}
    \label{fig:var}
    \vspace{-10pt}
\end{figure}

\begin{figure*}[!ht]
    \centering
    \begin{minipage}[b]{0.25\textwidth}
        \centering
        \includegraphics[width=1.\textwidth]{Image/fig_cropped.pdf} % 替换为你的小图文件
        \label{figure:small_image}
        \vspace{-20pt}
    \end{minipage}%
    \hfill
    \begin{minipage}[b]{0.75\textwidth}
        \centering
        \includegraphics[width=1.\textwidth]{Image/psnr_trends_rff_pe_simplified.pdf} % 替换为你的大图文件
        \vspace{-20pt}
        \label{figure:large_image}
        
    \end{minipage}
    \caption{We test the performance of MLPs with Random Fourier Features (RFF) and MLPs with Positional Encoding (PE) on a 1024-resolution image to better distinguish between high- and low-frequency regions, as demonstrated on the left-hand side of this figure. We find that the performance of MLPs+RFF degrades rapidly with increasing standard deviation compared with MLPs+PE. Since positional encoding is deterministic, scale=512 can be considered to have standard deviation around 121.}
    \label{figure:combined}
    \vspace{-10pt}
\end{figure*}
Varying the sampling standard deviation/scale may lead to degradation results, as shown in \autoref{figure:combined}.
%
However, MLPs face a significant challenge known as the spectral bias, where low-frequency signals are typically favored during training~\citep{rahaman2019spectral}. 
A common solution is to map coordinates into the frequency domain using Fourier features, such as Random Fourier Features and Positional Encoding, which can be understood as manually set high-frequency correspondence prior to accelerating the learning of high-frequency targets.~\citep{tancik2020fourier}. 
This embeddings widely applied to the INRs for novel view synthesis~\citep{mildenhall2021nerf,barron2021mip}, dynamic scene reconstruction~\citep{pumarola2021d}, object tracking~\citep{wang2023tracking}, and medical imaging~\citep{corona2022mednerf}.
% \begin{figure}[!h]
%     \centering
%     \includegraphics[width=1.\textwidth]{Image/psnr_trends_rff_pe_simplified.pdf}
%     \caption{This figure shows the change of PSNR on the whole, low-frequency region, and high-frequency region of the image fitting by using two Fourier Features Embedding with varying scale of variance: (Right) Positional Encoding (PE) (Left) Random Fourier Features (RFF). Both PE and RFF will degrade the low-frequency regions of the target image when variance increases.}
%     \vspace{-20pt} 
%     \label{figure:stats}
% \end{figure}


Although many INRs' downstream application scenarios use this encoding type, it has certain limitations when applied to specific tasks.
%
It depends heavily on two key hyperparameters: the sampling standard deviation/scale (available sampling range of frequencies) and the number of samples.
%
Even with a proper choice of sampling standard deviation/scale, the output remains unsatisfactory, as shown in \autoref{fig:var}: Noisy low-frequency regions and degraded high-frequency regions persist with well chosen sampling standard deviation/scale with the grid-searched standard deviation/scale, which may potentially affect the performance of the downstream applications resulting in noisy or coarse output.
%
However, limited research has contributed to explaining the reason and finding a proper frequency embeddings for input~\citep{landgraf2022pins, yuce2022structured}.

In this paper, we aim to offer a potential explanation for the high-frequency noise and propose an effective solution to the inherent drawbacks of Fourier feature embeddings for INRs.
%
Firstly, we hypothesize that the noisy output arises from the interaction between Fourier feature embeddings and multi-layer perceptrons (MLPs). We argue that these two elements can enhance each other's representation capabilities when combined. However, this combination also introduces the inherent properties of the Fourier series into the MLPs.
%
To support our hypothesis, we propose a simple theorem stating that the unsampled frequency components of the embeddings establish a lower bound on the expected performance. This underpins our hypothesis, as the primary fitting error in finitely sampled Fourier series originates from these unsampled frequencies.

Inspired by the analysis of noisy output and the properties of Fourier series expansion, we propose an approach to address this issue by enabling INRs to adaptively filter out unnecessary high-frequency components in low-frequency regions while enriching the input frequencies of the embeddings if possible.
%
To achieve this, we employ bias-free (additive term-free) MLPs. These MLPs function as adaptive linear filters due to their strictly linear and scale-invariant properties~\citep{mohan2019robust}, which preserves the input pattern through each activation layer and potentially enhances the expressive capability of the embeddings.
%
Moreover, by viewing the learning rate of the proposed filter and INRs as a dynamically balancing problem, we introduce a custom line-search algorithm to adjust the learning rate during training. This algorithm tackles an optimization problem to approximate a global minimum solution. Integrating these approaches leads to significant performance improvements in both low-frequency and high-frequency regions, as demonstrated in the comparison shown in \autoref{fig:var}.
%
Finally, to evaluate the performance of the proposed method, we test it on various INRs tasks and compare it with state-of-the-art models, including BACON~\citep{lindell2022bacon}, SIREN~\citep{sitzmann2020implicit}, GAUSS~\citep{ramasinghe2022beyond} and WIRE~\citep{saragadam2023wire}. 
The experimental results prove that our approach enables MLPs to capture finer details via Fourier Features while effectively reducing high-frequency noise without causing oversmoothness.
%
To summarize, the following are the main contributions of this work:
\begin{itemize}
    \item From the perspective of Fourier features embeddings and MLPs, we hypothesize that the representation capacity of their combination is also the combination of their strengths and limitations. A simple lemma offers partial validation of this hypothesis.

    
    \item  We propose a method that employs a bias-free MLP as an adaptive linear filter to suppress unnecessary high frequencies. Additionally, a custom line-search algorithm is introduced to dynamically optimize the learning rate, achieving a balance between the filter and INRs modules.

    \item To validate our approach, we conduct extensive experiments across a variety of tasks, including image regression, 3D shape regression, and inverse graphics. These experiments demonstrate the effectiveness of our method in significantly reducing noisy outputs while avoiding the common issue of excessive smoothing.
\end{itemize}

\section{Station Keeping with DQN}\label{section:SimEnvforDQN}

To successfully perform station-keeping, HABs typically oscillate between altitudes with opposing winds, stay in calm wind regions ($<$ 2m/s) regions, or some combination of the two. Most altitude-controlled HAB platforms operate between 18-28 km, above the typical max altitude regulation for many countries (such as FL600 for the USA).  Luckily, this region of the atmosphere frequently has vertical wind diversity for maneuvering HABs, where different altitudes can have opposing winds. The probability of available diverse winds in a region of interest substantially differs geographically and seasonally. 
%cite david brown and Tristan Schuler papers

%Problem Formulation?
Throughout a flight, the HAB can adjust altitude any number of times, as well as leave and re-enter the region any number of times, resulting in many unique trajectories and solutions to the problem. We use deep reinforcement learning to approach the station-keeping problem as opposed to a more deterministic controller because real-world wind flows are extremely complex and dynamic, as well as often heavily differing from weather forecasts.  For a reinforcement learning approach to station keeping, we design the simulation environment in a game-like structure to learn a policy, introducing simple actions, observation states, and positive rewards for making good actions towards the station-keeping goal.  We implement the popular DQN deep reinforcement learning algorithm to train a policy that maximizes rewards based on state action pairs. Training the DQN policy involves running hundreds of thousands of individual simulations, further referred to as episodes, updating the neural network as new conditions are encountered, and building off of past experiences to determine optimal actions for any given state. For evaluating trained agents, we use a time within the region of 50 km (TWR50) metric to measure station-keeping success.

%\subsubsection{Station-Keeping Success Metric}\label{section:SimEnvforDQN_StationKeepingSuccessMetric}For evaluating trained agents, we use a time within region of 50 km (TWR50) metric to measure station-keeping success rather than the mean reward.  While rewards
%are necessary for training a DQN policy, the rewards are arbitrary values that vary between different reward functions and does not always necessarily equate to successful station-keeping.  For example, in a case where there are winds such that a HAB circled a station in a 60 km radius using the piecewise reward function, the balloon would have a high overall reward, but a TWR50 score of 0.

\subsection{Simulation Setup}\label{section:SimEnvforDQN_SimStructure}

%Our Python HAB simulation environment is wrapped in a custom gym environment, a standardized framework for testing RL algorithms. %Training agents with DQN is done using the popular StableBaselines3 codebase \cite{brockman2016openai, stable-baselines3}.  

In the simulation, we use $u$ and $v$ wind data from ERA5 Reanalysis as our forecast, and pre-generated synthetic winds (discussed in Section \ref{section:SynthForecastModeling}) as our ground truth for dictating HAB movement. ECWMF ERA5 Reanalysis forecasts provide a large collection of historical atmospheric variables through reanalysis, including wind vectors at 137 static pressure levels at a maximum horizontal resolution of 0.25\degree \cite{hersbach2018era5}. For operational balloon missions ERA5 would be replaced by a forward-looking forecast and Synthetic winds would be the true unknown winds. At the start of an episode, we prepare a HAB arena (150 x 150 x 10 km) with a subset of 3D flow fields derived from a random coordinate within the bounds of a pre-downloaded and processed master forecast (e.g., Southwestern United States).  The HAB is then initialized to a random altitude at the station coordinate. We standardized the episode length to be 20 hours long at a resolution of 1 minute. %, even though SHAB-Vs cannot fly at night, to improve the experience buffer. 
At each simulation step, the horizontal flow is applied according to the $u$ and $v$ wind components from the synthetic winds forecast, while the observation space is updated with relative wind profiles from the ERA5 forecast. 
Throughout an episode, intermediate rewards are appended to an overall score based on the current observation after executing an action; the DQN policy aims to maximize the overall reward by training the network to choose the best intermediate actions.
 Figure \ref{fig:simulator} shows a snapshot of a trained DQN agent station-keeping in the simulation environment GUI.  The simulator GUI shows the 3D trajectory, altitude profile, and corresponding ERA5 and synthetic forecasts for the simulated coordinates and timestamp.  


%Talk about seasonal differences for station keeping?
% Replacing a controller due to complex dymanic flows.  We can mention synthetic winds for making more robust, and realastic to errors. Goal, is to stay within TWR50, and we want to train and evaluate in such a way to find seasonal differenecs.  BASED ON PREVISOU WORK. Urban Sky, Tristan.  Seaonal exploration   

\subsection{Solar Balloon Dynamics}\label{section:SimEnvforDQN_SolarBalloonDynamics}

For training short-duration HABs with DQN, we selected Vented Solar High Altitude Balloons (SHAB-Vs) for simulating HAB dynamics. Solar balloons utilize a lightweight material that absorbs solar radiation, providing a free source of lift and eliminating the need for lighter than air gas or heat source. Solar balloons can fly as long as there is unobstructed sunlight; 12 hours in equatorial regions but up to 16+ hours at higher latitudes in the summer months. Researchers have logged hundreds of flight hours with the original heliotrope design as well as modified versions, lifting up to 3 kg payloads and conducting multi-hour stratospheric research \cite{bowman2020multihour, schuler2022solar, swaim2024performance, Lien2024EarthSHAB}.  Solar balloons have additionally been proposed as a feasible platform for exploring Venus' atmosphere due to the thick, highly reflective cloud layer \cite{schuler2022long}. Recent development of a novel mechanical vent for solar balloons allows SHAB-Vs to adjust altitude precisely between a typical altitude range of 15-23 km and demonstrated initial manual station-keeping in the Southwestern United States \cite{schuler2023altitude}.  




%Wand applied a simplification of the dynamics model

%\begin{equation}\label{horizontalMotion}
%    \begin{split}
%        x &= x_{0} + v_{x}\delta t \\
%        y &= y_{0} + v_{y}\delta t
%    \end{split}
%\end{equation}
 %The wind velocity vector ($v_x$, $v_y$) is applied from the pre-generated synthetic wind forecasts, discussed in Section \ref{}. %It is a function of the current location of the agent in the Geographic Coordinate System (GCS). At each time step, the Cartesian coordinate of the agent is converted into GCS latitude, longitude, and altitude. This GCS coordinate is used to query the synthetic wind data for the wind velocity vector in the Cartesian coordinate frame. 
 %At a macroscopic scale, these ascent and descent rates can apply to all heliotropes of comparable envelope volume and payload mass. \textcolor{red}{Do we need to elaborate on why we think this is true?}

 \subsection{Action Space}

At each simulation step, the SHAB-V agent can perform three discrete actions: ascend, maintain altitude, or descend. In heliotrope solar balloon flight experiments conducted in July-September 2021, Swain et al. showed that the average ascent and descent rates of non-vented heliotrope balloons are 1.8 $\pm$ 0.14 and 2.8 $\pm$ 0.3 m/s respectively \cite{swain2024heliotrope}. Additionally, SHAB-V flight experiments conducted by Schuler et al. implemented an altitude controller with standard deviation $\sigma$ = 1.25 m/s. We simulate the 3 discrete actions and reasonable errors by mapping each venting action to the normal distributions found in Table~\ref{tab:actionspace}. These distributions holistically represent the motion profile experienced via the onboard altitude controller.

% \begin{equation}\label{verticalMotionDistribution}
%     \begin{split}
%         A &\sim N(\mu = 1.80, \sigma = 0.14) \\
%         D &\sim N(\mu = 2.80, \sigma = 0.30) \\
%         S &\sim N(\mu = 0.00, \sigma = 1.25) \\
%     \end{split}
% \end{equation}

\begin{table}[h]
\renewcommand{\arraystretch}{1.2}
\caption{Action Space for Training}
\centering
\begin{tabular}{|c|c|c|}
\hline
\bfseries Action & \bfseries Sampling Distribution \\
\hline%\hline
Ascend & $1.80 \pm 0.14$ m/s\\
Descend & $2.80 \pm 0.30$ m/s\\
Stay & $0.00 \pm 1.25$ m/s\\
\hline
\end{tabular}
\label{tab:actionspace}
\end{table}

Yajima et al. showed that the lateral motion of HABs is a function of the relative velocity between the balloon and the wind and dictated by the composition of the balloon, payload mass, and varying atmospheric properties at different altitudes \cite{yajima2009ballooning}. Smaller radiosonde balloons (of mass 0.3 kg to 2 kg) require between 20 to 40 seconds to respond to a 2 m/s velocity step change \cite{yajima2009ballooning}. Over large distances and multi-hour timescales, this delay in matching lateral wind velocities creates small deviations when considering station-keeping trajectories that are constantly adapting to wind flow changes. Additionally, winds typically experience a gradual shift between altitudes of different directional flows, with turbulent regions occurring less often. With these conditions, we assume the lightweight SHAB-Vs match the wind instantaneously in our simulation environment. Horizontal wind velocities from pre-computed synthetic forecasts, discussed in Section \ref{section:SynthForecastModeling}, are applied to the SHAB-V over the course of a 60-second time step ($\delta t$). A 60-second time step allows for reasonable simulation training time and also renders the time required to reach zero relative velocity negligible.

%The value sampled from the corresponding normal distribution are then applied as instantaneous velocity $v_z$ via Eulerian integration in Eq. \ref{verticalMotion} at each simulation time step. Similar to the horizontal motion dynamics, the $\delta t$ of one minute provides a viable high-level approximation of the PID or other low-level control schema. 

%\begin{equation}\label{verticalMotion}
%    z = z_{0} + v_{z}\delta t
%\end{equation}


\begin{figure*}[t]
    \centering 
    \includegraphics[width=\textwidth]{img/Simulator.png} 
    \caption{Altitude Profile and 3D trajectory of a trained DQN HAB agent in simulation with ERA5 forecasts as observation, and synthetic winds for movement.}
    \label{fig:simulator} 
\end{figure*} 

\subsection{Observation Space}
\begin{table}[h]
\renewcommand{\arraystretch}{1.2}
\caption{Observation Space for Controller Input}
\centering
\begin{tabular}{|c|c|}
\hline
\bfseries Observation & \bfseries Range\\
\hline%\hline
HAB Altitude & $15 - 25$ km\\
Relative Distance & $0 - \infty$ km\\
Relative Bearing & $0 - 180\degree$\\
Wind Column & \{Altitude, Magnitude, \\
            & Relative Bearing\} * N \\
\hline
\end{tabular}
\label{tab:observationspace}
\end{table}

The observation space for training the DQN policy is a subset of the agent's full state and is shown in Table \ref{tab:observationspace}. This observation space does not include all simulator state information such as velocity, coordinates, or an entire forecast as DRL algorithms may take longer to train as a function of a more complex observation space. By converting coordinates to relative distance and bearing, we are able to reduce the dimensionality of the observation vector.
The relative distance and relative bearing are from the balloon to the central coordinate for station-keeping where the maximum relative bearing is 180\degree and would be a HAB traveling on a trajectory directly away from the central coordinate. The wind column is a triple that encodes relative wind velocities at the current coordinate to the central coordinate from an ERA5 forecast along with the true altitude of those coordinates. The number of levels included in the wind column remains static for the entire training set (for the altitude region of 15-25 km we include 7 levels between 20 and 150 hPa). Because ERA5 is a pressure-based forecast and we extrapolate the altitude values from the geopotential variable, altitude is not constant across a forecast and why we include it in the wind column triple.  For small forecast areas (only a few degrees of latitude and longitude across) the variance in altitude at the same pressure level is typically within $\pm$ 100 m.  By including true ERA5 altitudes in the wind column, the DRL algorithm is able to implicitly learn discrepancies between ERA5 and the synthetic winds in different seasons.
%^New sentence based on vconversation with Tony
In the future, we plan to experiment with expanding the number of altitude levels in the wind column vector based on a combination of ERA5 forecasts and previous actions, experimenting with relative altitude values, and including uncertainty estimates.

%\subsubsection{Forecasts for Training}\label{section:SimEnvforDQN_ForecastforTraining}
%The horizontal movement of the HAB agent is based on the synthetic winds discussed in Section \ref{SynthForecastModeling_SynthWindDataGeneration}, not the ERA5 based wind column in the observation space. The synthetic forecasts and ERA5 forecasts are assumed to already be downloaded and formatted correctly before initiating DRL simulations. 
%At the start of a training, a primary ERA5 and Synthetic forecast of the same large area (e.g., Southwestern United States) for the desired altitude range (15-25 km) and time frame (typically a single month) is loaded into RAM.  At the beginning of each episode, a smaller multi-dimensional forecast subset array is generated from a randomized central coordinate and includes a relative X and Y distance of forecast data; we used 150 km. This subset is used to produce the horizontal wind components as a function of altitude. The range of 150 km ensures that the HAB has sufficient spatial awareness, without requiring the use of extensive RAM for training. 

\subsection{Reward Functions}\label{section:SimEnvforDQN_RewardFunctions}

For short-duration station-keeping, we experimented with the distance-based reward proposed by Bellemare et al. in Eq. \ref{eq:google} as well as some variations \cite{bellemare2020autonomous}. For short-duration station-keeping missions we had the best performance with the piecewise reward function that introduces an inner radius as shown in Eq. \ref{piecewise_reard}.  For short-duration HAB station-keeping, if the winds have good station-keeping probabilities, they frequently do throughout the entire episode, whereas multi-day or multi-week flights can experience significantly different wind profiles over a longer time scale. The Euclidean Reward function in Eq. \ref{euclidan_reard} performed slightly worse than the piecewise reward function and about the same as Google's reward function in our simulation environment, and we believe would be the best candidate for navigating to waypoints as opposed to assuming the HAB is already in the station-keeping region. 
%^Need to explain that better

With SHAB-Vs, we ignore power as a resource constraint because the HABs can lift several kilogram payloads and can carry enough payload to continuously vent. Similarly, longer duration altitude controlled balloon platforms such as air ballasted super pressure HABs and vent and ballasted zero pressure HABs could also reduce or eliminate power constraints for short duration flights.  Unlike other HAB platforms, SHAB-Vs do not have a lighter-than-air gas as a resource constraint because the lift is generated via heating ambient atmospheric air.  Therefore, the altitude profile and horizontal motion are the most important dynamics to replicate.

Google Loon Reward Function:

\begin{equation}
r_{\mathrm{dist}}(\Delta) =
    \begin{cases}
        1.0 & \text{if } \Delta < \rho_{50km} \\
        c_{\mathrm{cliff}}2^{-(\Delta-\rho)/\tau)} & \text{otherwise }
    \end{cases}
    \label{eq:google}
\end{equation}

Piecewise Reward Function:

\begin{equation}
r_{\mathrm{dist}}(\Delta) =
    \begin{cases}
        2.0 & \text{if } \Delta \leq \rho_{25km} \\
        1.0 & \text{if }  \rho_{25km} < \Delta \leq \rho_{50km} \\
        c_{\mathrm{cliff}}2^{-(\Delta-\rho)/\tau)} & \text{otherwise }
    \end{cases}
    \label{piecewise_reard}
\end{equation}

Euclidean Reward Function:

\begin{equation}
r_{\mathrm{dist}}(\Delta) =
    \begin{cases}
        |\Delta| & \text{if } \Delta < \rho_{50km} \\
        c_{\mathrm{cliff}}2^{-(\Delta-\rho)/\tau)} & \text{otherwise }
    \end{cases}
    \label{euclidan_reard}
\end{equation}



\subsection{Training and Hyperparameter Tuning}\label{section:SimEnvforDQN_TrainingAndHyperparam}
%Do we show a reward + TWR training plot here for a few months?
%explain exploration rate. 
%Possibly explain most important hyperparameters
%cite optuna?

Figure \ref{fig:Learning} shows an example of one of the best-performing trained models for an HAB in the month of April trained for 150 million time steps and reaching an average TWR50 of 75\%. Training was only done on forecasts with decent wind diversity, so the final TWR50 is higher than the overall mean. The TWR50 and TWR75 closely follow the mean reward learning curve whereas the inner TWR25 is much lower. The mean reward started to plateau toward the end of training. However, all 3 of the TWR metrics are still increasing, suggesting longer training times could lead to improved station-keeping results.  

\begin{figure}[h]
    \centering 
    \includegraphics[width=.5\textwidth]{img/Learning.png}
    \caption{Learning curve over 1.5 million timesteps for a station-keeping DQN HAB agent} 
    \label{fig:Learning} 
\end{figure}


The best-trained models for short-duration HAB station-keeping are extremely sensitive to hyperparameters of the DQN algorithm as well as the geographic area and season where the agent was trained. We used Optuna, an open-source automated hyperparameter tuning framework, to train our initial models \cite{akiba2019optuna}. Overall, the learning rate and exploration rate had the most significant impacts on success. The best-performing models had learning rates between 1e-5 and 1e-4 and a high exploitation to exploration ratio (how often the agent takes random actions as opposed to picking actions from the trained policy.) For the exploration strategy, the best-performing models started with taking random actions approximately 25-50\% of the time at the beginning of training and linearly decayed for about a third of the overall training time to take random actions 10-20\% of the rest of the time, a more deterministic policy. 

%To avoid overfitting, we trained individually on the first month of each quarter but evaluated the monthly model on the entire region for all four months. In The Southwestern United states that are huge differences in winds between the summer and winter months.. 
%expand more here when we have better models

%Are we overfitting? 
%Should we evaluate on another year.  Should we combine all the months?

From preliminary investigations, we noticed that models trained on one region (e.g., the Southwestern United States) but evaluated on another region (e.g., an equatorial region) performed significantly worse than models trained on the other region.  The primary reason is that the HAB agent is exposed to wind flows never seen before when only training on a medium-sized region. Equatorial regions typically have significantly higher wind diversity in all dimensions (vertically, horizontally, and temporally) than the Southwestern United States, and also more deviations between ERA5 and synthetic forecasts.  Combining models, implementing curriculum learning, and other optimization strategies could help the trained DRL agents be more robust to much larger areas and timescales, but this is currently out of the scope of this paper for short-duration HABs. 
















\section{Synthetic Forecast Modeling}\label{section:SynthForecastModeling}
%Do we Call this something Else:?

To have autonomous HABs that are more robust to inaccurate weather forecasts, realistic deviations from available weather forecasts are necessary.  However, determining and applying these deviations is nontrivial; while adding simple noise to the ERA5 forecasts could be one approach, it is not very realistic for several reasons. Often, real winds at a particular altitude during flight can be up to 180\degree off from the forecast.  Additionally, winds at one altitude level do not necessarily linearly shift in speed and direction when there are significant changes between time steps. 
%Additionally, winds at a particular altitude level typically follow local trends horizontally and temporally. For example, if winds at a particular altitude level are blowing east at one timestamp, but then west 6 hours later, the inner timestamps do not necessarily linearly shift in speed and direction.  
Finally, and most importantly, winds at one altitude level are frequently decoupled from other altitude levels, hence why opposing winds and wind diversity are possible. %Therefore, while noise could be added to forecasts at mandatory pressure levels, creating realistic wind profiles between those mandatory levels is impossible to do with forecasts alone.  

%In order to station keep, HABs typically oscillate between altitudes with opposing winds, stay in calm winds ($<$ 2m/s) regions, or some combination of the two. Most altitude controlled HAB platforms operate between 18-28 km, above the typical max altitude regulation for many countries (such as FL600 for the USA).  Luckily, this region of the atmosphere frequently has vertical wind diversity for maneuvering HABs, where different altitudes can have opposing winds. The probability of available diverse winds in a region of interest substantially differs geographically and seasonally. This concept is explored further in subsequent sections.

%In order to transition simulated trained DQN HAB models to the real world, the simulator needs to closely match real-world HAB dynamics, weather conditions, and uncertainty with weather forecasts. 
Simulating realistic winds and uncertainty in forecasts is a major challenge due to a lack of in situ weather data in the upper troposphere and in the lower stratosphere. Popular Reanalysis forecasts such as ECWMF and NASA GEOS only report wind data at specific levels, that lack vertical resolution. NOAA's GFS forecast is similar but forward predicts forecasts rather than reanalysis;  we will most likely use GFS forecasts for real-world SHAB flights with trained models in the future. Since pressure falls exponentially from sea level to the upper atmosphere,  higher altitudes have decreased vertical resolution. Within the typical HAB operating altitude region, most forecasts only report between 5-10 pressure levels of wind data. 

%\subsection{Synthetic Wind Data Generation}\label{SynthForecastModeling_SynthWindDataGeneration}



\begin{figure}[!ht]
\centering
    \begin{subfigure}[t]{0.4\textwidth}
    \centering
    \includegraphics[width=\textwidth]{img/Radiosonde-Map.png}
    \caption{Map of Radiosonde Locations in the Southwestern United States} \label{fig1}
\end{subfigure}\hfill
\begin{subfigure}[t]{0.4\textwidth}
    \centering
    \includegraphics[width=\textwidth]{img/Radiosonde3D.png}
    \caption{3D wind vector visualization from unprocessed radiosonde measurements} \label{fig2}
\end{subfigure}\hfill
\begin{subfigure}[t]{0.4\textwidth}
    \centering
    \includegraphics[width=\textwidth]{img/Smoothed-Radiosonde-Level.png}
    \caption{Smoothed Synthetic Winds at altitude level 16 km} \label{fig2}
\end{subfigure}\hfill

\caption{Synthetic Wind Generation from Aggregated and Interpolated Radiosonde Data in the Southwestern United States on August
23, 2023 at 1200 UTC}
\label{fig:synthwinds}
\end{figure}

\begin{figure*}[b] % Figure* spans two columns, [t] places it at the top of the page
    \centering
    % Row 1
    \subfloat[ERA5 Model (Jan)]{\includegraphics[width=0.32\textwidth]{img/01_17_50_a.png}\label{fig:01_17_50_a}}%
    \hspace{0.01\textwidth} % Minimal horizontal space between figures
    \subfloat[Synthetic Model (Jan)]{\includegraphics[width=0.32\textwidth]{img/01_17_50_b.png}\label{fig:01_17_50_b}}%
    \hspace{0.01\textwidth} % Minimal horizontal space between figures
    \subfloat[Model Difference (Jan)]{\includegraphics[width=0.32\textwidth]{img/01_17_50_c.png}\label{fig:01_17_50_c}}
    
    \vspace{0.1cm} % Reduce vertical spacing between rows
    
    % Row 2
    \subfloat[ERA5 Model (Jul)]{\includegraphics[width=0.32\textwidth]{img/07_17_50_a.png}\label{fig:07_17_50_a}}%
    \hspace{0.01\textwidth} % Minimal horizontal space between figures
    \subfloat[Synthetic Model (Jul)]{\includegraphics[width=0.32\textwidth]{img/07_17_50_b.png}\label{fig:07_17_50_b}}%
    \hspace{0.01\textwidth} % Minimal horizontal space between figures
    \subfloat[Model Difference (Jul)]{\includegraphics[width=0.32\textwidth]{img/07_17_50_c.png}\label{fig:07_17_50_c}}
    
    \caption{ERA5 and Synthetic Model Variations on January 17 and July 17, 2023 at 000UTC at the 50 hPa Pressure Level}
    \label{fig:JanJulyVariation}
\end{figure*}




To generate realistic wind fields for higher vertical resolution and realistic deviations from ERA5 forecasts we generate synthetic wind fields based on aggregated and interpolated radiosonde data from a particular region. The international Radiosonde program is the largest source of high-resolution in-situ atmospheric meteorological measurements over 2000 radiosonde launch sites locations all over the world are launched twice a day. These radiosondes typically collect temperature, pressure, humidity, and wind velocity measurements from surface level to 25+ km; the data is then assimilated into weather forecasts. We collected our radiosonde data from the University of Wyoming's Upper Air Sounding Database \cite{UofWyUpperAir}. Throughout this work, we use the Southwestern United States region, shown in Figure \ref{fig:synthwinds}a for our station-keeping simulations. We have conducted several flight experiments of SHAB-Vs in this region and it contains an adequate number of radiosonde launch sites for generating synthetic forecasts. Figure \ref{fig:synthwinds}b shows all of the unprocessed wind velocity measurements recorded from radiosondes in the Southwestern United States region launched on August 23, 2023, at 1200 UTC. While radiosonde data has a very high vertical resolution compared to ERA5 forecasts,  they have poor horizontal and temporal resolution.  To account for this, we make several assumptions during aggregation and interpolation of the radiosonde data to generate synthetic forecasts.  
%Should we cite university of wyoming sounding database here? 


%Explanation of Synthetic Winds
First, we bin the radiosonde data from an individual station into 250 m regions, taking the nearest $u$ and $v$ wind components.  Typically, there are multiple wind velocities per bin, in which case we take the closest value instead of an average (ex. with radiosonde altitude readings of 15998 and 16103 meters, we would select 15998 m for the 16000 m bin). In the case of empty bins, the bin is filled with $u$ and $v$ wind components via linear interpolation from the nearest filled bins. Next, the radiosonde data is interpolated horizontally using a nearest neighbor method; this generates coarse planar fields of winds at each altitude level for the area of interest. Finally, we apply Gaussian smoothing to each level to add smooth variation between the sub-areas of wind. Figure \ref{fig:synthwinds}c shows the final smoothed plane of a synthetic wind field at 16 km. 

Unfortunately, the synthetic winds are typically only available in 12-hour increments, at 0000 and 1200 UTC.  In simulation, we reduce this time gap from 12 hours to 3 hours to increase variation and ideally make the trained DRL HAB agents more robust to situations with highly dynamic forecasts. Three hours is also the temporal resolution of forward-predicted GFS forecasts. %Due to diurnal effects as well as other processes that happen over 12 hours, this can significantly increase the occurrence of large shifts in winds. 
We hypothesize that if the trained DQN HAB agents are successful at station-keeping with large intraday wind shifts, they should also perform similarly or better on smoother changes.


\subsection{Comparing Synthetic Wind Forecasts with ERA5 Reanalysis Forecasts} \label{SynthForecastModeling_SynthCompareERA5}

%There are a few key distinctions between the ERA5 reanalysis models and the synthetic wind models generated from radiosonde data. For one, the vertical resolution of the ERA5 model is significantly less than that of our synthetic wind model. For example, the ERA5 model provides seven pressure levels: 20 hPa, 30 hPa, 50 hPa, 70 hPa, 100 hPa, 125 hPa, and 150 hPa. For simulation purposes, this vertical resolution is insufficient, and thus our synthetic wind model provides 46 pressure levels between 21.6 hPa and 121.1 hPa. 

When comparing the two forecasts, the synthetic winds forecasts are overall highly correlated to the ERA5 forecasts but also exhibit variance trends across certain pressure levels and seasons. Fig. \ref{fig:JanJulyVariation} shows the differences between the ERA5 and synthetic wind model for the SW USA region on January 17, 2023 and July 17, 2023. Fig. \ref{fig:01_17_50_c} shows the wind direction between an ERA5 and synthetic forecast at the 50 hPa pressure level on January 17, 2023, in the USA region. As shown visually, there is a significant difference in magnitude and direction with a mean angular difference of $61.6 \pm 47.2 \degree$ in January.  In this context, a large standard deviation is indicative of a non-uniform shift between the two models' wind direction. Similarly, a lower standard deviation may indicate that the model is inaccurate but has a uniform shift of some magnitude.
%What do you mean by this?
The same pressure level (50 hPa) on July 17, shown in Fig. \ref{fig:07_17_50_c} has a mean difference of $13.7 \pm 12.2 \degree$. When compared to the same region in January, the accuracy of the synthetic model in July is considerably better. 
This may be due to the increased turbulence in transition regions, and pressure levels between 50 hPa and 100 hPa, during the month of January. As a result, the number of inflection points can produce inaccuracies during the spatial interpolation. Subsequent sections analyze seasonal trends that help characterize these discrepancies. 



%\section{Methods}

\section{ Task Generalization Beyond i.i.d. Sampling and Parity Functions
}\label{sec:Discussion}
% Discussion: From Theory to Beyond
% \misha{what is beyond?}
% \amir{we mean two things: in the first subsection beyond i.i.d subsampling of parity tasks and in the second subsection beyond parity task}
% \misha{it has to be beyond something, otherwise it is not clear what it is about} \hz{this is suggested by GPT..., maybe can be interpreted as from theory to beyond theory. We can do explicit like Discussion: Beyond i.i.d. task sampling and the Parity Task}
% \misha{ why is "discussion" in the title?}\amir{Because it is a discussion, it is not like separate concrete explnation about why these thing happens or when they happen, they just discuss some interesting scenraios how it relates to our theory.   } \misha{it is not really a discussion -- there is a bunch of experiments}

In this section, we extend our experiments beyond i.i.d. task sampling and parity functions. We show an adversarial example where biased task selection substantially hinders task generalization for sparse parity problem. In addition, we demonstrate that exponential task scaling extends to a non-parity tasks including arithmetic and multi-step language translation.

% In this section, we extend our experiments beyond i.i.d. task sampling and parity functions. On the one hand, we find that biased task selection can significantly degrade task generalization; on the other hand, we show that exponential task scaling generalizes to broader scenarios.
% \misha{we should add a sentence or two giving more detail}


% 1. beyond i.i.d tasks sampling
% 2. beyond parity -> language, arithmetic -> task dependency + implicit bias of transformer (cannot implement this algorithm for arithmatic)



% In this section, we emphasize the challenge of quantifying the level of out-of-distribution (OOD) differences between training tasks and testing tasks, even for a simple parity task. To illustrate this, we present two scenarios where tasks differ between training and testing. For each scenario, we invite the reader to assess, before examining the experimental results, which cases might appear “more” OOD. All scenarios consider \( d = 10 \). \kaiyue{this sentence should be put into 5.1}






% for parity problem




% \begin{table*}[th!]
%     \centering
%     \caption{Generalization Results for Scenarios 1 and 2 for $d=10$.}
%     \begin{tabular}{|c|c|c|c|}
%         \hline
%         \textbf{Scenario} & \textbf{Type/Variation} & \textbf{Coordinates} & \textbf{Generalization accuracy} \\
%         \hline
%         \multirow{3}{*}{Generalization with Missing Pair} & Type 1 & \( c_1 = 4, c_2 = 6 \) & 47.8\%\\ 
%         & Type 2 & \( c_1 = 4, c_2 = 6 \) & 96.1\%\\ 
%         & Type 3 & \( c_1 = 4, c_2 = 6 \) & 99.5\%\\ 
%         \hline
%         \multirow{3}{*}{Generalization with Missing Pair} & Type 1 &  \( c_1 = 8, c_2 = 9 \) & 40.4\%\\ 
%         & Type 2 & \( c_1 = 8, c_2 = 9 \) & 84.6\% \\ 
%         & Type 3 & \( c_1 = 8, c_2 = 9 \) & 99.1\%\\ 
%         \hline
%         \multirow{1}{*}{Generalization with Missing Coordinate} & --- & \( c_1 = 5 \) & 45.6\% \\ 
%         \hline
%     \end{tabular}
%     \label{tab:generalization_results}
% \end{table*}

\subsection{Task Generalization Beyond i.i.d. Task Sampling }\label{sec: Experiment beyond iid sampling}

% \begin{table*}[ht!]
%     \centering
%     \caption{Generalization Results for Scenarios 1 and 2 for $d=10, k=3$.}
%     \begin{tabular}{|c|c|c|}
%         \hline
%         \textbf{Scenario}  & \textbf{Tasks excluded from training} & \textbf{Generalization accuracy} \\
%         \hline
%         \multirow{1}{*}{Generalization with Missing Pair}
%         & $\{4,6\} \subseteq \{s_1, s_2, s_3\}$ & 96.2\%\\ 
%         \hline
%         \multirow{1}{*}{Generalization with Missing Coordinate}
%         & \( s_2 = 5 \) & 45.6\% \\ 
%         \hline
%     \end{tabular}
%     \label{tab:generalization_results}
% \end{table*}




In previous sections, we focused on \textit{i.i.d. settings}, where the set of training tasks $\mathcal{F}_{train}$ were sampled uniformly at random from the entire class $\mathcal{F}$. Here, we explore scenarios that deliberately break this uniformity to examine the effect of task selection on out-of-distribution (OOD) generalization.\\

\textit{How does the selection of training tasks influence a model’s ability to generalize to unseen tasks? Can we predict which setups are more prone to failure?}\\

\noindent To investigate this, we consider two cases parity problems with \( d = 10 \) and \( k = 3 \), where each task is represented by its tuple of secret indices \( (s_1, s_2, s_3) \):

\begin{enumerate}[leftmargin=0.4 cm]
    \item \textbf{Generalization with a Missing Coordinate.} In this setup, we exclude all training tasks where the second coordinate takes the value \( s_2 = 5 \), such as \( (1,5,7) \). At test time, we evaluate whether the model can generalize to unseen tasks where \( s_2 = 5 \) appears.
    \item \textbf{Generalization with Missing Pair.} Here, we remove all training tasks that contain both \( 4 \) \textit{and} \( 6 \) in the tuple \( (s_1, s_2, s_3) \), such as \( (2,4,6) \) and \( (4,5,6) \). At test time, we assess whether the model can generalize to tasks where both \( 4 \) and \( 6 \) appear together.
\end{enumerate}

% \textbf{Before proceeding, consider the following question:} 
\noindent \textbf{If you had to guess.} Which scenario is more challenging for generalization to unseen tasks? We provide the experimental result in Table~\ref{tab:generalization_results}.

 % while the model struggles for one of them while as it generalizes almost perfectly in the other one. 

% in the first scenario, it generalizes almost perfectly in the second. This highlights how exposure to partial task structures can enhance generalization, even when certain combinations are entirely absent from the training set. 

In the first scenario, despite being trained on all tasks except those where \( s_2 = 5 \), which is of size $O(\d^T)$, the model struggles to generalize to these excluded cases, with prediction at chance level. This is intriguing as one may expect model to generalize across position. The failure  suggests that positional diversity plays a crucial role in the task generalization of Transformers. 

In contrast, in the second scenario, though the model has never seen tasks with both \( 4 \) \textit{and} \( 6 \) together, it has encountered individual instances where \( 4 \) appears in the second position (e.g., \( (1,4,5) \)) or where \( 6 \) appears in the third position (e.g., \( (2,3,6) \)). This exposure appears to facilitate generalization to test cases where both \( 4 \) \textit{and} \( 6 \) are present. 



\begin{table*}[t!]
    \centering
    \caption{Generalization Results for Scenarios 1 and 2 for $d=10, k=3$.}
    \resizebox{\textwidth}{!}{  % Scale to full width
        \begin{tabular}{|c|c|c|}
            \hline
            \textbf{Scenario}  & \textbf{Tasks excluded from training} & \textbf{Generalization accuracy} \\
            \hline
            Generalization with Missing Pair & $\{4,6\} \subseteq \{s_1, s_2, s_3\}$ & 96.2\%\\ 
            \hline
            Generalization with Missing Coordinate & \( s_2 = 5 \) & 45.6\% \\ 
            \hline
        \end{tabular}
    }
    \label{tab:generalization_results}
\end{table*}

As a result, when the training tasks are not i.i.d, an adversarial selection such as exclusion of specific positional configurations may lead to failure to unseen task generalization even though the size of $\mathcal{F}_{train}$ is exponentially large. 


% \paragraph{\textbf{Key Takeaways}}
% \begin{itemize}
%     \item Out-of-distribution generalization in the parity problem is highly sensitive to the diversity and positional coverage of training tasks.
%     \item Adversarial exclusion of specific pairs or positional configurations can lead to systematic failures, even when most tasks are observed during training.
% \end{itemize}




%################ previous veriosn down
% \textit{How does the choice of training tasks affect the ability of a model to generalize to unseen tasks? Can we predict which setups are likely to lead to failure?}

% To explore these questions, we crafted specific training and test task splits to investigate what makes one setup appear “more” OOD than another.

% \paragraph{Generalization with Missing Pair.}

% Imagine we have tasks constructed from subsets of \(k=3\) elements out of a larger set of \(d\) coordinates. What happens if certain pairs of coordinates are adversarially excluded during training? For example, suppose \(d=5\) and two specific coordinates, \(c_1 = 1\) and \(c_2 = 2\), are excluded. The remaining tasks are formed from subsets of the other coordinates. How would a model perform when tested on tasks involving the excluded pair \( (c_1, c_2) \)? 

% To probe this, we devised three variations in how training tasks are constructed:
%     \begin{enumerate}
%         \item \textbf{Type 1:} The training set includes all tasks except those containing both \( c_1 = 1 \) and \( c_2 = 2 \). 
%         For this example, the training set includes only $\{(3,4,5)\}$. The test set consists of all tasks containing the rest of tuples.

%         \item \textbf{Type 2:} Similar to Type 1, but the training set additionally includes half of the tasks containing either \( c_1 = 1 \) \textit{or} \( c_2 = 2 \) (but not both). 
%         For the example, the training set includes all tasks from Type 1 and adds tasks like \(\{(1, 3, 4), (2, 3, 5)\}\) (half of those containing \( c_1 = 1 \) or \( c_2 = 2 \)).

%         \item \textbf{Type 3:} Similar to Type 2, but the training set also includes half of the tasks containing both \( c_1 = 1 \) \textit{and} \( c_2 = 2 \). 
%         For the example, the training set includes all tasks from Type 2 and adds, for instance, \(\{(1, 2, 5)\}\) (half of the tasks containing both \( c_1 \) and \( c_2 \)).
%     \end{enumerate}

% By systematically increasing the diversity of training tasks in a controlled way, while ensuring no overlap between training and test configurations, we observe an improvement in OOD generalization. 

% % \textit{However, the question is this improvement similar across all coordinate pairs, or does it depend on the specific choices of \(c_1\) and \(c_2\) in the tasks?} 

% \textbf{Before proceeding, consider the following question:} Is the observed improvement consistent across all coordinate pairs, or does it depend on the specific choices of \(c_1\) and \(c_2\) in the tasks? 

% For instance, consider two cases for \(d = 10, k = 3\): (i) \(c_1 = 4, c_2 = 6\) and (ii) \(c_1 = 8, c_2 = 9\). Would you expect similar OOD generalization behavior for these two cases across the three training setups we discussed?



% \paragraph{Answer to the Question.} for both cases of \( c_1, c_2 \), we observe that generalization fails in Type 1, suggesting that the position of the tasks the model has been trained on significantly impacts its generalization capability. For Type 2, we find that \( c_1 = 4, c_2 = 6 \) performs significantly better than \( c_1 = 8, c_2 = 9 \). 

% Upon examining the tasks where the transformer fails for \( c_1 = 8, c_2 = 9 \), we see that the model only fails at tasks of the form \((*, 8, 9)\) while perfectly generalizing to the rest. This indicates that the model has never encountered the value \( 8 \) in the second position during training, which likely explains its failure to generalize. In contrast, for \( c_1 = 4, c_2 = 6 \), while the model has not seen tasks of the form \((*, 4, 6)\), it has encountered tasks where \( 4 \) appears in the second position, such as \((1, 4, 5)\), and tasks where \( 6 \) appears in the third position, such as \((2, 3, 6)\). This difference may explain why the model generalizes almost perfectly in Type 2 for \( c_1 = 4, c_2 = 6 \), but not for \( c_1 = 8, c_2 = 9 \).



% \paragraph{Generalization with Missing Coordinates.}
% Next, we investigate whether a model can generalize to tasks where a specific coordinate appears in an unseen position during training. For instance, consider \( c_1 = 5 \), and exclude all tasks where \( c_1 \) appears in the second position. Despite being trained on all other tasks, the model fails to generalize to these excluded cases, highlighting the importance of positional diversity in training tasks.



% \paragraph{Key Takeaways.}
% \begin{itemize}
%     \item OOD generalization depends heavily on the diversity and positional coverage of training tasks for the parity problem.
%     \item adversarial exclusion of specific pairs or positional configurations in the parity problem can lead to failure, even when the majority of tasks are observed during training.
% \end{itemize}


%################ previous veriosn up

% \paragraph{Key Takeaways} These findings highlight the complexity of OOD generalization, even in seemingly simple tasks like parity. They also underscore the importance of task design: the diversity of training tasks can significantly influence a model’s ability to generalize to unseen tasks. By better understanding these dynamics, we can design more robust training regimes that foster generalization across a wider range of scenarios.


% #############


% Upon examining the tasks where the transformer fails for \( c_1 = 8, c_2 = 9 \), we see that the model only fails at tasks of the form \((*, 8, 9)\) while perfectly generalizing to the rest. This indicates that the model has never encountered the value \( 8 \) in the second position during training, which likely explains its failure to generalize. In contrast, for \( c_1 = 4, c_2 = 6 \), while the model has not seen tasks of the form \((*, 4, 6)\), it has encountered tasks where \( 4 \) appears in the second position, such as \((1, 4, 5)\), and tasks where \( 6 \) appears in the third position, such as \((2, 3, 6)\). This difference may explain why the model generalizes almost perfectly in Type 2 for \( c_1 = 4, c_2 = 6 \), but not for \( c_1 = 8, c_2 = 9 \).

% we observe a striking pattern: generalization fails entirely in Type 1, regardless of the coordinate pair (\(c_1, c_2\)). However, in Type 2, generalization varies: \(c_1 = 4, c_2 = 6\) achieves 96\% accuracy, while \(c_1 = 8, c_2 = 9\) lags behind at 70\%. Why? Upon closer inspection, the model struggles specifically with tasks like \((*, 8, 9)\), where the combination \(c_1 = 8\) and \(c_2 = 9\) is entirely novel. In contrast, for \(c_1 = 4, c_2 = 6\), the model benefits from having seen tasks where \(4\) appears in the second position or \(6\) in the third. This suggests that positional exposure during training plays a key role in generalization.

% To test whether task structure influences generalization, we consider two variations:
% \begin{enumerate}
%     \item \textbf{Sorted Tuples:} Tasks are always sorted in ascending order.
%     \item \textbf{Unsorted Tuples:} Tasks can appear in any order.
% \end{enumerate}

% If the model struggles with generalizing to the excluded position, does introducing variability through unsorted tuples help mitigate this limitation?

% \paragraph{Discussion of Results}

% In \textbf{Generalization with Missing Pairs}, we observe a striking pattern: generalization fails entirely in Type 1, regardless of the coordinate pair (\(c_1, c_2\)). However, in Type 2, generalization varies: \(c_1 = 4, c_2 = 6\) achieves 96\% accuracy, while \(c_1 = 8, c_2 = 9\) lags behind at 70\%. Why? Upon closer inspection, the model struggles specifically with tasks like \((*, 8, 9)\), where the combination \(c_1 = 8\) and \(c_2 = 9\) is entirely novel. In contrast, for \(c_1 = 4, c_2 = 6\), the model benefits from having seen tasks where \(4\) appears in the second position or \(6\) in the third. This suggests that positional exposure during training plays a key role in generalization.

% In \textbf{Generalization with Missing Coordinates}, the results confirm this hypothesis. When \(c_1 = 5\) is excluded from the second position during training, the model fails to generalize to such tasks in the sorted case. However, allowing unsorted tuples introduces positional diversity, leading to near-perfect generalization. This raises an intriguing question: does the model inherently overfit to positional patterns, and can task variability help break this tendency?




% In this subsection, we show that the selection of training tasks can affect the quality of the unseen task generalization significantly in practice. To illustrate this, we present two scenarios where tasks differ between training and testing. For each scenario, we invite the reader to assess, before examining the experimental results, which cases might appear “more” OOD. 

% % \amir{add examples, }

% \kaiyue{I think the name of scenarios here are not very clear}
% \begin{itemize}
%     \item \textbf{Scenario 1:  Generalization Across Excluded Coordinate Pairs (\( k = 3 \))} \\
%     In this scenario, we select two coordinates \( c_1 \) and \( c_2 \) out of \( d \) and construct three types of training sets. 

%     Suppose \( d = 5 \), \( c_1 = 1 \), and \( c_2 = 2 \). The tuples are all possible subsets of \( \{1, 2, 3, 4, 5\} \) with \( k = 3 \):
%     \[
%     \begin{aligned}
%     \big\{ & (1, 2, 3), (1, 2, 4), (1, 2, 5), (1, 3, 4), (1, 3, 5), \\
%            & (1, 4, 5), (2, 3, 4), (2, 3, 5), (2, 4, 5), (3, 4, 5) \big\}.
%     \end{aligned}
%     \]

%     \begin{enumerate}
%         \item \textbf{Type 1:} The training set includes all tuples except those containing both \( c_1 = 1 \) and \( c_2 = 2 \). 
%         For this example, the training set includes only $\{(3,4,5)\}$ tuple. The test set consists of tuples containing the rest of tuples.

%         \item \textbf{Type 2:} Similar to Type 1, but the training set additionally includes half of the tuples containing either \( c_1 = 1 \) \textit{or} \( c_2 = 2 \) (but not both). 
%         For the example, the training set includes all tuples from Type 1 and adds tuples like \(\{(1, 3, 4), (2, 3, 5)\}\) (half of those containing \( c_1 = 1 \) or \( c_2 = 2 \)).

%         \item \textbf{Type 3:} Similar to Type 2, but the training set also includes half of the tuples containing both \( c_1 = 1 \) \textit{and} \( c_2 = 2 \). 
%         For the example, the training set includes all tuples from Type 2 and adds, for instance, \(\{(1, 2, 5)\}\) (half of the tuples containing both \( c_1 \) and \( c_2 \)).
%     \end{enumerate}

% % \begin{itemize}
% %     \item \textbf{Type 1:} The training set includes tuples \(\{1, 3, 4\}, \{2, 3, 4\}\) (excluding tuples with both \( c_1 \) and \( c_2 \): \(\{1, 2, 3\}, \{1, 2, 4\}\)). The test set contains the excluded tuples.
% %     \item \textbf{Type 2:} The training set includes all tuples in Type 1 plus half of the tuples containing either \( c_1 = 1 \) or \( c_2 = 2 \) (e.g., \(\{1, 2, 3\}\)).
% %     \item \textbf{Type 3:} The training set includes all tuples in Type 2 plus half of the tuples containing both \( c_1 = 1 \) and \( c_2 = 2 \) (e.g., \(\{1, 2, 4\}\)).
% % \end{itemize}
    
%     \item \textbf{Scenario 2: Scenario 2: Generalization Across a Fixed Coordinate (\( k = 3 \))} \\
%     In this scenario, we select one coordinate \( c_1 \) out of \( d \) (\( c_1 = 5 \)). The training set includes all task tuples except those where \( c_1 \) is the second coordinate of the tuple. For this scenario, we examine two variations:
%     \begin{enumerate}
%         \item \textbf{Sorted Tuples:} Task tuples are always sorted (e.g., \( (x_1, x_2, x_3) \) with \( x_1 \leq x_2 \leq x_3 \)).
%         \item \textbf{Unsorted Tuples:} Task tuples can appear in any order.
%     \end{enumerate}
% \end{itemize}




% \paragraph{Discussion of Results.} In the first scenario, for both cases of \( c_1, c_2 \), we observe that generalization fails in Type 1, suggesting that the position of the tasks the model has been trained on significantly impacts its generalization capability. For Type 2, we find that \( c_1 = 4, c_2 = 6 \) performs significantly better than \( c_1 = 8, c_2 = 9 \). 

% Upon examining the tasks where the transformer fails for \( c_1 = 8, c_2 = 9 \), we see that the model only fails at tasks of the form \((*, 8, 9)\) while perfectly generalizing to the rest. This indicates that the model has never encountered the value \( 8 \) in the second position during training, which likely explains its failure to generalize. In contrast, for \( c_1 = 4, c_2 = 6 \), while the model has not seen tasks of the form \((*, 4, 6)\), it has encountered tasks where \( 4 \) appears in the second position, such as \((1, 4, 5)\), and tasks where \( 6 \) appears in the third position, such as \((2, 3, 6)\). This difference may explain why the model generalizes almost perfectly in Type 2 for \( c_1 = 4, c_2 = 6 \), but not for \( c_1 = 8, c_2 = 9 \).

% This position-based explanation appears compelling, so in the second scenario, we focus on a single position to investigate further. Here, we find that the transformer fails to generalize to tasks where \( 5 \) appears in the second position, provided it has never seen any such tasks during training. However, when we allow for more task diversity in the unsorted case, the model achieves near-perfect generalization. 

% This raises an important question: does the transformer have a tendency to overfit to positional patterns, and does introducing more task variability, as in the unsorted case, prevent this overfitting and enable generalization to unseen positional configurations?

% These findings highlight that even in a simple task like parity, it is remarkably challenging to understand and quantify the sources and levels of OOD behavior. This motivates further investigation into the nuances of task design and its impact on model generalization.


\subsection{Task Generalization Beyond Parity Problems}

% \begin{figure}[t!]
%     \centering
%     \includegraphics[width=0.45\textwidth]{Figures/arithmetic_v1.png}
%     \vspace{-0.3cm}
%     \caption{Task generalization for arithmetic task with CoT, it has $\d =2$ and $T = d-1$ as the ambient dimension, hence $D\ln(DT) = 2\ln(2T)$. We show that the empirical scaling closely follows the theoretical scaling.}
%     \label{fig:arithmetic}
% \end{figure}



% \begin{wrapfigure}{r}{0.4\textwidth}  % 'r' for right, 'l' for left
%     \centering
%     \includegraphics[width=0.4\textwidth]{Figures/arithmetic_v1.png}
%     \vspace{-0.3cm}
%     \caption{Task generalization for the arithmetic task with CoT. It has $d =2$ and $T = d-1$ as the ambient dimension, hence $D\ln(DT) = 2\ln(2T)$. We show that the empirical scaling closely follows the theoretical scaling.}
%     \label{fig:arithmetic}
% \end{wrapfigure}

\subsubsection{Arithmetic Task}\label{subsec:arithmetic}











We introduce the family of \textit{Arithmetic} task that, like the sparse parity problem, operates on 
\( d \) binary inputs \( b_1, b_2, \dots, b_d \). The task involves computing a structured arithmetic expression over these inputs using a sequence of addition and multiplication operations.
\newcommand{\op}{\textrm{op}}

Formally, we define the function:
\[
\text{Arithmetic}_{S} \colon \{0,1\}^d \to \{0,1,\dots,d\},
\]
where \( S = (\op_1, \op_2, \dots, \op_{d-1}) \) is a sequence of \( d-1 \) operations, each \( \op_k \) chosen from \( \{+, \times\} \). The function evaluates the expression by applying the operations sequentially from left-to-right order: for example, if \( S = (+, \times, +) \), then the arithmetic function would compute:
\[
\text{Arithmetic}_{S}(b_1, b_2, b_3, b_4) = ((b_1 + b_2) \times b_3) + b_4.
\]
% Thus, the sequence of operations \( S \) defines how the binary inputs are combined to produce an integer output between \( 0 \) and \( d \).
% \[
% \text{Arithmetic}_{S} 
% (b_1,\,b_2,\,\dots,b_d)
% =
% \Bigl(\dots\bigl(\,(b_1 \;\op_1\; b_2)\;\op_2\; b_3\bigr)\,\dots\Bigr) 
% \;\op_{d-1}\; b_d.
% \]
% We now introduce an \emph{Arithmetic} task that, like the sparse parity problem, operates on $d$ binary inputs $b_1, b_2, \dots, b_d$. Specifically, we define an arithmetic function
% \[
% \text{Arithmetic}_{S}\colon \{0,1\}^d \;\to\; \{0,1,\dots,d\},
% \]
% where $S = (i_1, i_2, \dots, i_{d-1})$ is a sequence of $d-1$ operations, each $i_k \in \{+,\,\times\}$. The value of $\text{Arithmetic}_{S}$ is obtained by applying the prescribed addition and multiplication operations in order, namely:
% \[
% \text{Arithmetic}_{S}(b_1,\,b_2,\,\dots,b_d)
% \;=\;
% \Bigl(\dots\bigl(\,(b_1 \;i_1\; b_2)\;i_2\; b_3\bigr)\,\dots\Bigr) 
% \;i_{d-1}\; b_d.
% \]

% This is an example of our framework where $T = d-1$ and $|\Theta_t| = 2$ with total $2^d$ possible tasks. 




By introducing a step-by-step CoT, arithmetic class belongs to $ARC(2, d-1)$: this is because at every step, there is only $\d = |\Theta_t| = 2$ choices (either $+$ or $\times$) while the length is  $T = d-1$, resulting a total number of $2^{d-1}$ tasks. 


\begin{minipage}{0.5\textwidth}  % Left: Text
    Task generalization for the arithmetic task with CoT. It has $d =2$ and $T = d-1$ as the ambient dimension, hence $D\ln(DT) = 2\ln(2T)$. We show that the empirical scaling closely follows the theoretical scaling.
\end{minipage}
\hfill
\begin{minipage}{0.4\textwidth}  % Right: Image
    \centering
    \includegraphics[width=\textwidth]{Figures/arithmetic_v1.png}
    \refstepcounter{figure}  % Manually advances the figure counter
    \label{fig:arithmetic}  % Now this label correctly refers to the figure
\end{minipage}

Notably, when scaling with \( T \), we observe in the figure above that the task scaling closely follow the theoretical $O(D\log(DT))$ dependency. Given that the function class grows exponentially as \( 2^T \), it is truly remarkable that training on only a few hundred tasks enables generalization to an exponentially larger space—on the order of \( 2^{25} > 33 \) Million tasks. This exponential scaling highlights the efficiency of structured learning, where a modest number of training examples can yield vast generalization capability.





% Our theory suggests that only $\Tilde{O}(\ln(T))$ i.i.d training tasks is enough to generalize to the rest of unseen tasks. However, we show in Figure \ref{fig:arithmetic} that transformer is not able to match  that. The transformer out-of distribution generalization behavior is not consistent across different dimensions when we scale the number of training tasks with $\ln(T)$. \hongzhou{implicit bias, optimization, etc}
 






% \subsection{Task generalization Beyond parity problem}

% \subsection{Arithmetic} In this setting, we still use the set-up we introduced in \ref{subsec:parity_exmaple}, the input is still a set of $d$ binary variable, $b_1, b_2,\dots,b_d$ and ${Arithmatic_{S}}:\{0,1\}\rightarrow \{0, 1, \dots, d\}$, where $S = (i_1,i_2,\dots,i_{d-1})$ is a tuple of size $d-1$ where each coordinate is either add($+
% $) or multiplication ($\times$). The function is as following,

% \begin{align*}
%     Arithmatic_{S}(b_1, b_2,\dots,b_d) = (\dots(b1(i1)b2)(i3)b3\dots)(i{d-1})
% \end{align*}
    


\subsubsection{Multi-Step Language Translation Task}

 \begin{figure*}[h!]
    \centering
    \includegraphics[width=0.9\textwidth]{Figures/combined_plot_horiz.png}
    \vspace{-0.2cm}
    \caption{Task generalization for language translation task: $\d$ is the number of languages and $T$ is the length of steps.}
    \vspace{-2mm}
    \label{fig:language}
\end{figure*}
% \vspace{-2mm}

In this task, we study a sequential translation process across multiple languages~\cite{garg2022can}. Given a set of \( D \) languages, we construct a translation chain by randomly sampling a sequence of \( T \) languages \textbf{with replacement}:  \(L_1, L_2, \dots, L_T,\)
where each \( L_t \) is a sampled language. Starting with a word, we iteratively translate it through the sequence:
\vspace{-2mm}
\[
L_1 \to L_2 \to L_3 \to \dots \to L_T.
\]
For example, if the sampled sequence is EN → FR → DE → FR, translating the word "butterfly" follows:
\vspace{-1mm}
\[
\text{butterfly} \to \text{papillon} \to \text{schmetterling} \to \text{papillon}.
\]
This task follows an \textit{AutoRegressive Compositional} structure by itself, specifically \( ARC(D, T-1) \), where at each step, the conditional generation only depends on the target language, making \( D \) as the number of languages and the total number of possible tasks is \( D^{T-1} \). This example illustrates that autoregressive compositional structures naturally arise in real-world languages, even without explicit CoT. 

We examine task scaling along \( D \) (number of languages) and \( T \) (sequence length). As shown in Figure~\ref{fig:language}, empirical  \( D \)-scaling closely follows the theoretical \( O(D \ln D T) \). However, in the \( T \)-scaling case, we observe a linear dependency on \( T \) rather than the logarithmic dependency \(O(\ln T) \). A possible explanation is error accumulation across sequential steps—longer sequences require higher precision in intermediate steps to maintain accuracy. This contrasts with our theoretical analysis, which focuses on asymptotic scaling and does not explicitly account for compounding errors in finite-sample settings.

% We examine task scaling along \( D \) (number of languages) and \( T \) (sequence length). As shown in Figure~\ref{fig:language}, empirical scaling closely follows the theoretical \( O(D \ln D T) \) trend, with slight exceptions at $ T=10 \text{ and } 3$ in Panel B. One possible explanation for this deviation could be error accumulation across sequential steps—longer sequences require each intermediate translation to be approximated with higher precision to maintain test accuracy. This contrasts with our theoretical analysis, which primarily focuses on asymptotic scaling and does not explicitly account for compounding errors in finite-sample settings.

Despite this, the task scaling is still remarkable — training on a few hundred tasks enables generalization to   $4^{10} \approx 10^6$ tasks!






% , this case, we are in a regime where \( D \ll T \), we observe  that the task complexity empirically scales as \( T \log T \) rather than \( D \log T \). 


% the model generalizes to an exponentially larger space of \( 2^T \) unseen tasks. In case $T=25$, this is $2^{25} > 33$ Million tasks. This remarkable exponential generalization demonstrates the power of structured task composition in enabling efficient generalization.


% In the case of parity tasks, introducing CoT effectively decomposes the problem from \( ARC(D^T, 1) \) to \( ARC(D, T) \), significantly improving task generalization.

% Again, in the regime scaling $T$, we again observe a $T\log T$ dependency. Knowing that the function class is scaling as $D^T$, it is remarkable that training on a few hundreds tasks can generalize to $4^{10} \approx 1M$ tasks. 





% We further performed a preliminary investigation on a semi-synthetic word-level translation task to show that (1) task generalization via composition structure is feasible beyond parity and (2) understanding the fine-grained mechanism leading to this generalization is still challenging. 
% \noindent
% \noindent
% \begin{minipage}[t]{\columnwidth}
%     \centering
%     \textbf{\scriptsize In-context examples:}
%     \[
%     \begin{array}{rl}
%         \textbf{Input} & \hspace{1.5em} \textbf{Output} \\
%         \hline
%         \textcolor{blue}{car}   & \hspace{1.5em} \textcolor{red}{voiture \;,\; coche} \\
%         \textcolor{blue}{house} & \hspace{1.5em} \textcolor{red}{maison \;,\; casa} \\
%         \textcolor{blue}{dog}   & \hspace{1.5em} \textcolor{red}{chien \;,\; perro} 
%     \end{array}
%     \]
%     \textbf{\scriptsize Query:}
%     \[
%     \begin{array}{rl}
%         \textbf{Input} & \textbf{Output} \\
%         \hline
%         \textcolor{blue}{cat} & \hspace{1.5em} \textcolor{red}{?} \\
%     \end{array}
%     \]
% \end{minipage}



% \begin{figure}[h!]
%     \centering
%     \includegraphics[width=0.45\textwidth]{Figures/translation_scale_d.png}
%     \vspace{-0.2cm}
%     \caption{Task generalization behavior for word translation task.}
%     \label{fig:arithmetic}
% \end{figure}


\vspace{-1mm}
\section{Conclusions}
% \misha{is it conclusion of the section or of the whole paper?}    
% \amir{The whole paper. It is very short, do we need a separate section?}
% \misha{it should not be a subsection if it is the conclusion the whole thing. We can just remove it , it does not look informative} \hz{let's do it in a whole section, just to conclude and end the paper, even though it is not informative}
%     \kaiyue{Proposal: Talk about the implication of this result on theory development. For example, it calls for more fine-grained theoretical study in this space.  }

% \huaqing{Please feel free to edit it if you have better wording or suggestions.}

% In this work, we propose a theoretical framework to quantitatively investigate task generalization with compositional autoregressive tasks. We show that task to $D^T$ task is theoretically achievable by training on only $O (D\log DT)$ tasks, and empirically observe that transformers trained on parity problem indeed achieves such task generalization. However, for other tasks beyond parity, transformers seem to fail to achieve this bond. This calls for more fine-grained theoretical study the phenomenon of task generalization specific to transformer model. It may also be interesting to study task generalization beyond the setting of in-context learning. 
% \misha{what does this add?} \amir{It does not, i dont have any particular opinion to keep it. @Hongzhou if you want to add here?}\hz{While it may not introduce anything new, we are following a good practice to have a short conclusion. It provides a clear closing statement, reinforces key takeaways, and helps the reader leave with a well-framed understanding of our contributions. }
% In this work, we quantitatively investigate task generalization under autoregressive compositional structure. We demonstrate that task generalization to $D^T$ tasks is theoretically achievable by training on only $\tilde O(D)$ tasks. Empirically, we observe that transformers trained indeed achieve such exponential task generalization on problems such as parity, arithmetic and multi-step language translation. We believe our analysis opens up a new angle to understand the remarkable generalization ability of Transformer in practice. 

% However, for tasks beyond the parity problem, transformers appear to fail to reach this bound. This highlights the need for a more fine-grained theoretical exploration of task generalization, especially for transformer models. Additionally, it may be valuable to investigate task generalization beyond the scope of in-context learning.



In this work, we quantitatively investigated task generalization under the autoregressive compositional structure, demonstrating both theoretically and empirically that exponential task generalization to $D^T$ tasks can be achieved with training on only $\tilde{O}(D)$ tasks. %Our theoretical results establish a fundamental scaling law for task generalization, while our experiments validate these insights across problems such as parity, arithmetic, and multi-step language translation. The remarkable ability of transformers to generalize exponentially highlights the power of structured learning and provides a new perspective on how large language models extend their capabilities beyond seen tasks. 
We recap our key contributions  as follows:
\begin{itemize}
    \item \textbf{Theoretical Framework for Task Generalization.} We introduced the \emph{AutoRegressive Compositional} (ARC) framework to model structured task learning, demonstrating that a model trained on only $\tilde{O}(D)$ tasks can generalize to an exponentially large space of $D^T$ tasks.
    
    \item \textbf{Formal Sample Complexity Bound.} We established a fundamental scaling law that quantifies the number of tasks required for generalization, proving that exponential generalization is theoretically achievable with only a logarithmic increase in training samples.
    
    \item \textbf{Empirical Validation on Parity Functions.} We showed that Transformers struggle with standard in-context learning (ICL) on parity tasks but achieve exponential generalization when Chain-of-Thought (CoT) reasoning is introduced. Our results provide the first empirical demonstration of structured learning enabling efficient generalization in this setting.
    
    \item \textbf{Scaling Laws in Arithmetic and Language Translation.} Extending beyond parity functions, we demonstrated that the same compositional principles hold for arithmetic operations and multi-step language translation, confirming that structured learning significantly reduces the task complexity required for generalization.
    
    \item \textbf{Impact of Training Task Selection.} We analyzed how different task selection strategies affect generalization, showing that adversarially chosen training tasks can hinder generalization, while diverse training distributions promote robust learning across unseen tasks.
\end{itemize}



We introduce a framework for studying the role of compositionality in learning tasks and how this structure can significantly enhance generalization to unseen tasks. Additionally, we provide empirical evidence on learning tasks, such as the parity problem, demonstrating that transformers follow the scaling behavior predicted by our compositionality-based theory. Future research will  explore how these principles extend to real-world applications such as program synthesis, mathematical reasoning, and decision-making tasks. 


By establishing a principled framework for task generalization, our work advances the understanding of how models can learn efficiently beyond supervised training and adapt to new task distributions. We hope these insights will inspire further research into the mechanisms underlying task generalization and compositional generalization.

\section*{Acknowledgements}
We acknowledge support from the National Science Foundation (NSF) and the Simons Foundation for the Collaboration on the Theoretical Foundations of Deep Learning through awards DMS-2031883 and \#814639 as well as the  TILOS institute (NSF CCF-2112665) and the Office of Naval Research (ONR N000142412631). 
This work used the programs (1) XSEDE (Extreme science and engineering discovery environment)  which is supported by NSF grant numbers ACI-1548562, and (2) ACCESS (Advanced cyberinfrastructure coordination ecosystem: services \& support) which is supported by NSF grants numbers \#2138259, \#2138286, \#2138307, \#2137603, and \#2138296. Specifically, we used the resources from SDSC Expanse GPU compute nodes, and NCSA Delta system, via allocations TG-CIS220009. 

We present RiskHarvester, a risk-based tool to compute a security risk score based on the value of the asset and ease of attack on a database. We calculated the value of asset by identifying the sensitive data categories present in a database from the database keywords. We utilized data flow analysis, SQL, and Object Relational Mapper (ORM) parsing to identify the database keywords. To calculate the ease of attack, we utilized passive network analysis to retrieve the database host information. To evaluate RiskHarvester, we curated RiskBench, a benchmark of 1,791 database secret-asset pairs with sensitive data categories and host information manually retrieved from 188 GitHub repositories. RiskHarvester demonstrates precision of (95\%) and recall (90\%) in detecting database keywords for the value of asset and precision of (96\%) and recall (94\%) in detecting valid hosts for ease of attack. Finally, we conducted an online survey to understand whether developers prioritize secret removal based on security risk score. We found that 86\% of the developers prioritized the secrets for removal with descending security risk scores.

%\begin{itemize}
%    \item Simulator Development 
%    \item Physics Modeling () - matching 
%    \item Add in Noise and Uncertainty (Synth Winds)
%        \item compare that with historical LOOn data
%    \item Synthetic Wind Development
%    \item Forecast Classifcation
%    \item Baseline Controller?
%    \item forecast classifier vs twr
%    \item Google, Piecewise, Euclidian.  Maybe we tweak radiuses, cliffs?
%    \item Other SOTA RL algorithms?
%\end{itemize}





%%%%%%%%%%%%%%%%%%%%%%%%%%%%%%%%%%%%%%%%%%%%%%%%%%%%%%%%%%%%%%%%%%%%%%%%%%%%%%%%%%%%%%%%%%%%%%%%%%%%%%
%\acknowledgments
 


%%%%%%%%%%%%%%%%%%%%%%%%%%%%%%%%%%%%%%%%%%%%%%%%%%%%%%%%%%%%%%%%%%%%%%%%%%%%%%%%%%%%%%%%%%%%%%%%%%%%%%
\bibliographystyle{IEEEtran}
\bibliography{main}



\vskip -2.5\baselineskip plus -1fil
\begin{IEEEbiographynophoto}{Krithika Iyer} is a computing PhD candidate at the Kahlert School of Computing and 
the Scientific Computing and Imaging Institute at the University of Utah in the Image Analysis track. She received her B.E. in Electronics \& Telecommunication in 2015. Her research interests include medical image analysis, probabilistic modeling, deep learning, and statistical shape modeling.\end{IEEEbiographynophoto}
\vskip -2.5\baselineskip plus -1fil
\begin{IEEEbiographynophoto}{Mokshagna Sai Teja Karanam} is a computing PhD student at the Kahlert School of Computing and the Scientific Computing and Imaging Institute at the University of Utah in the Image Analysis track. He received his B.E. in Computer Science in 2020 and his Masters in 2024. His research interests include deep learning, computer vision and statistical shape modeling.\end{IEEEbiographynophoto}
\vskip -2.5\baselineskip plus -1fil
\begin{IEEEbiographynophoto}{Shireen Elhabian} is a faculty member at the Kahlert School of Computing and the Scientific Computing and Imaging Institute at the University of Utah. Her research focuses on enhancing diagnostic accuracy through deep learning, probabilistic modeling, and advanced computer vision algorithms. She has published over 100 peer-reviewed publications in prestigious journals and conferences, including IEEE-TMI, MedIA, ICLR, CVPR, ICCV, MICCAI, IPMI, AAAI. \end{IEEEbiographynophoto}
\vskip -3\baselineskip plus -1fil
%\newpage
\newpage
\subsection{Lloyd-Max Algorithm}
\label{subsec:Lloyd-Max}
For a given quantization bitwidth $B$ and an operand $\bm{X}$, the Lloyd-Max algorithm finds $2^B$ quantization levels $\{\hat{x}_i\}_{i=1}^{2^B}$ such that quantizing $\bm{X}$ by rounding each scalar in $\bm{X}$ to the nearest quantization level minimizes the quantization MSE. 

The algorithm starts with an initial guess of quantization levels and then iteratively computes quantization thresholds $\{\tau_i\}_{i=1}^{2^B-1}$ and updates quantization levels $\{\hat{x}_i\}_{i=1}^{2^B}$. Specifically, at iteration $n$, thresholds are set to the midpoints of the previous iteration's levels:
\begin{align*}
    \tau_i^{(n)}=\frac{\hat{x}_i^{(n-1)}+\hat{x}_{i+1}^{(n-1)}}2 \text{ for } i=1\ldots 2^B-1
\end{align*}
Subsequently, the quantization levels are re-computed as conditional means of the data regions defined by the new thresholds:
\begin{align*}
    \hat{x}_i^{(n)}=\mathbb{E}\left[ \bm{X} \big| \bm{X}\in [\tau_{i-1}^{(n)},\tau_i^{(n)}] \right] \text{ for } i=1\ldots 2^B
\end{align*}
where to satisfy boundary conditions we have $\tau_0=-\infty$ and $\tau_{2^B}=\infty$. The algorithm iterates the above steps until convergence.

Figure \ref{fig:lm_quant} compares the quantization levels of a $7$-bit floating point (E3M3) quantizer (left) to a $7$-bit Lloyd-Max quantizer (right) when quantizing a layer of weights from the GPT3-126M model at a per-tensor granularity. As shown, the Lloyd-Max quantizer achieves substantially lower quantization MSE. Further, Table \ref{tab:FP7_vs_LM7} shows the superior perplexity achieved by Lloyd-Max quantizers for bitwidths of $7$, $6$ and $5$. The difference between the quantizers is clear at 5 bits, where per-tensor FP quantization incurs a drastic and unacceptable increase in perplexity, while Lloyd-Max quantization incurs a much smaller increase. Nevertheless, we note that even the optimal Lloyd-Max quantizer incurs a notable ($\sim 1.5$) increase in perplexity due to the coarse granularity of quantization. 

\begin{figure}[h]
  \centering
  \includegraphics[width=0.7\linewidth]{sections/figures/LM7_FP7.pdf}
  \caption{\small Quantization levels and the corresponding quantization MSE of Floating Point (left) vs Lloyd-Max (right) Quantizers for a layer of weights in the GPT3-126M model.}
  \label{fig:lm_quant}
\end{figure}

\begin{table}[h]\scriptsize
\begin{center}
\caption{\label{tab:FP7_vs_LM7} \small Comparing perplexity (lower is better) achieved by floating point quantizers and Lloyd-Max quantizers on a GPT3-126M model for the Wikitext-103 dataset.}
\begin{tabular}{c|cc|c}
\hline
 \multirow{2}{*}{\textbf{Bitwidth}} & \multicolumn{2}{|c|}{\textbf{Floating-Point Quantizer}} & \textbf{Lloyd-Max Quantizer} \\
 & Best Format & Wikitext-103 Perplexity & Wikitext-103 Perplexity \\
\hline
7 & E3M3 & 18.32 & 18.27 \\
6 & E3M2 & 19.07 & 18.51 \\
5 & E4M0 & 43.89 & 19.71 \\
\hline
\end{tabular}
\end{center}
\end{table}

\subsection{Proof of Local Optimality of LO-BCQ}
\label{subsec:lobcq_opt_proof}
For a given block $\bm{b}_j$, the quantization MSE during LO-BCQ can be empirically evaluated as $\frac{1}{L_b}\lVert \bm{b}_j- \bm{\hat{b}}_j\rVert^2_2$ where $\bm{\hat{b}}_j$ is computed from equation (\ref{eq:clustered_quantization_definition}) as $C_{f(\bm{b}_j)}(\bm{b}_j)$. Further, for a given block cluster $\mathcal{B}_i$, we compute the quantization MSE as $\frac{1}{|\mathcal{B}_{i}|}\sum_{\bm{b} \in \mathcal{B}_{i}} \frac{1}{L_b}\lVert \bm{b}- C_i^{(n)}(\bm{b})\rVert^2_2$. Therefore, at the end of iteration $n$, we evaluate the overall quantization MSE $J^{(n)}$ for a given operand $\bm{X}$ composed of $N_c$ block clusters as:
\begin{align*}
    \label{eq:mse_iter_n}
    J^{(n)} = \frac{1}{N_c} \sum_{i=1}^{N_c} \frac{1}{|\mathcal{B}_{i}^{(n)}|}\sum_{\bm{v} \in \mathcal{B}_{i}^{(n)}} \frac{1}{L_b}\lVert \bm{b}- B_i^{(n)}(\bm{b})\rVert^2_2
\end{align*}

At the end of iteration $n$, the codebooks are updated from $\mathcal{C}^{(n-1)}$ to $\mathcal{C}^{(n)}$. However, the mapping of a given vector $\bm{b}_j$ to quantizers $\mathcal{C}^{(n)}$ remains as  $f^{(n)}(\bm{b}_j)$. At the next iteration, during the vector clustering step, $f^{(n+1)}(\bm{b}_j)$ finds new mapping of $\bm{b}_j$ to updated codebooks $\mathcal{C}^{(n)}$ such that the quantization MSE over the candidate codebooks is minimized. Therefore, we obtain the following result for $\bm{b}_j$:
\begin{align*}
\frac{1}{L_b}\lVert \bm{b}_j - C_{f^{(n+1)}(\bm{b}_j)}^{(n)}(\bm{b}_j)\rVert^2_2 \le \frac{1}{L_b}\lVert \bm{b}_j - C_{f^{(n)}(\bm{b}_j)}^{(n)}(\bm{b}_j)\rVert^2_2
\end{align*}

That is, quantizing $\bm{b}_j$ at the end of the block clustering step of iteration $n+1$ results in lower quantization MSE compared to quantizing at the end of iteration $n$. Since this is true for all $\bm{b} \in \bm{X}$, we assert the following:
\begin{equation}
\begin{split}
\label{eq:mse_ineq_1}
    \tilde{J}^{(n+1)} &= \frac{1}{N_c} \sum_{i=1}^{N_c} \frac{1}{|\mathcal{B}_{i}^{(n+1)}|}\sum_{\bm{b} \in \mathcal{B}_{i}^{(n+1)}} \frac{1}{L_b}\lVert \bm{b} - C_i^{(n)}(b)\rVert^2_2 \le J^{(n)}
\end{split}
\end{equation}
where $\tilde{J}^{(n+1)}$ is the the quantization MSE after the vector clustering step at iteration $n+1$.

Next, during the codebook update step (\ref{eq:quantizers_update}) at iteration $n+1$, the per-cluster codebooks $\mathcal{C}^{(n)}$ are updated to $\mathcal{C}^{(n+1)}$ by invoking the Lloyd-Max algorithm \citep{Lloyd}. We know that for any given value distribution, the Lloyd-Max algorithm minimizes the quantization MSE. Therefore, for a given vector cluster $\mathcal{B}_i$ we obtain the following result:

\begin{equation}
    \frac{1}{|\mathcal{B}_{i}^{(n+1)}|}\sum_{\bm{b} \in \mathcal{B}_{i}^{(n+1)}} \frac{1}{L_b}\lVert \bm{b}- C_i^{(n+1)}(\bm{b})\rVert^2_2 \le \frac{1}{|\mathcal{B}_{i}^{(n+1)}|}\sum_{\bm{b} \in \mathcal{B}_{i}^{(n+1)}} \frac{1}{L_b}\lVert \bm{b}- C_i^{(n)}(\bm{b})\rVert^2_2
\end{equation}

The above equation states that quantizing the given block cluster $\mathcal{B}_i$ after updating the associated codebook from $C_i^{(n)}$ to $C_i^{(n+1)}$ results in lower quantization MSE. Since this is true for all the block clusters, we derive the following result: 
\begin{equation}
\begin{split}
\label{eq:mse_ineq_2}
     J^{(n+1)} &= \frac{1}{N_c} \sum_{i=1}^{N_c} \frac{1}{|\mathcal{B}_{i}^{(n+1)}|}\sum_{\bm{b} \in \mathcal{B}_{i}^{(n+1)}} \frac{1}{L_b}\lVert \bm{b}- C_i^{(n+1)}(\bm{b})\rVert^2_2  \le \tilde{J}^{(n+1)}   
\end{split}
\end{equation}

Following (\ref{eq:mse_ineq_1}) and (\ref{eq:mse_ineq_2}), we find that the quantization MSE is non-increasing for each iteration, that is, $J^{(1)} \ge J^{(2)} \ge J^{(3)} \ge \ldots \ge J^{(M)}$ where $M$ is the maximum number of iterations. 
%Therefore, we can say that if the algorithm converges, then it must be that it has converged to a local minimum. 
\hfill $\blacksquare$


\begin{figure}
    \begin{center}
    \includegraphics[width=0.5\textwidth]{sections//figures/mse_vs_iter.pdf}
    \end{center}
    \caption{\small NMSE vs iterations during LO-BCQ compared to other block quantization proposals}
    \label{fig:nmse_vs_iter}
\end{figure}

Figure \ref{fig:nmse_vs_iter} shows the empirical convergence of LO-BCQ across several block lengths and number of codebooks. Also, the MSE achieved by LO-BCQ is compared to baselines such as MXFP and VSQ. As shown, LO-BCQ converges to a lower MSE than the baselines. Further, we achieve better convergence for larger number of codebooks ($N_c$) and for a smaller block length ($L_b$), both of which increase the bitwidth of BCQ (see Eq \ref{eq:bitwidth_bcq}).


\subsection{Additional Accuracy Results}
%Table \ref{tab:lobcq_config} lists the various LOBCQ configurations and their corresponding bitwidths.
\begin{table}
\setlength{\tabcolsep}{4.75pt}
\begin{center}
\caption{\label{tab:lobcq_config} Various LO-BCQ configurations and their bitwidths.}
\begin{tabular}{|c||c|c|c|c||c|c||c|} 
\hline
 & \multicolumn{4}{|c||}{$L_b=8$} & \multicolumn{2}{|c||}{$L_b=4$} & $L_b=2$ \\
 \hline
 \backslashbox{$L_A$\kern-1em}{\kern-1em$N_c$} & 2 & 4 & 8 & 16 & 2 & 4 & 2 \\
 \hline
 64 & 4.25 & 4.375 & 4.5 & 4.625 & 4.375 & 4.625 & 4.625\\
 \hline
 32 & 4.375 & 4.5 & 4.625& 4.75 & 4.5 & 4.75 & 4.75 \\
 \hline
 16 & 4.625 & 4.75& 4.875 & 5 & 4.75 & 5 & 5 \\
 \hline
\end{tabular}
\end{center}
\end{table}

%\subsection{Perplexity achieved by various LO-BCQ configurations on Wikitext-103 dataset}

\begin{table} \centering
\begin{tabular}{|c||c|c|c|c||c|c||c|} 
\hline
 $L_b \rightarrow$& \multicolumn{4}{c||}{8} & \multicolumn{2}{c||}{4} & 2\\
 \hline
 \backslashbox{$L_A$\kern-1em}{\kern-1em$N_c$} & 2 & 4 & 8 & 16 & 2 & 4 & 2  \\
 %$N_c \rightarrow$ & 2 & 4 & 8 & 16 & 2 & 4 & 2 \\
 \hline
 \hline
 \multicolumn{8}{c}{GPT3-1.3B (FP32 PPL = 9.98)} \\ 
 \hline
 \hline
 64 & 10.40 & 10.23 & 10.17 & 10.15 &  10.28 & 10.18 & 10.19 \\
 \hline
 32 & 10.25 & 10.20 & 10.15 & 10.12 &  10.23 & 10.17 & 10.17 \\
 \hline
 16 & 10.22 & 10.16 & 10.10 & 10.09 &  10.21 & 10.14 & 10.16 \\
 \hline
  \hline
 \multicolumn{8}{c}{GPT3-8B (FP32 PPL = 7.38)} \\ 
 \hline
 \hline
 64 & 7.61 & 7.52 & 7.48 &  7.47 &  7.55 &  7.49 & 7.50 \\
 \hline
 32 & 7.52 & 7.50 & 7.46 &  7.45 &  7.52 &  7.48 & 7.48  \\
 \hline
 16 & 7.51 & 7.48 & 7.44 &  7.44 &  7.51 &  7.49 & 7.47  \\
 \hline
\end{tabular}
\caption{\label{tab:ppl_gpt3_abalation} Wikitext-103 perplexity across GPT3-1.3B and 8B models.}
\end{table}

\begin{table} \centering
\begin{tabular}{|c||c|c|c|c||} 
\hline
 $L_b \rightarrow$& \multicolumn{4}{c||}{8}\\
 \hline
 \backslashbox{$L_A$\kern-1em}{\kern-1em$N_c$} & 2 & 4 & 8 & 16 \\
 %$N_c \rightarrow$ & 2 & 4 & 8 & 16 & 2 & 4 & 2 \\
 \hline
 \hline
 \multicolumn{5}{|c|}{Llama2-7B (FP32 PPL = 5.06)} \\ 
 \hline
 \hline
 64 & 5.31 & 5.26 & 5.19 & 5.18  \\
 \hline
 32 & 5.23 & 5.25 & 5.18 & 5.15  \\
 \hline
 16 & 5.23 & 5.19 & 5.16 & 5.14  \\
 \hline
 \multicolumn{5}{|c|}{Nemotron4-15B (FP32 PPL = 5.87)} \\ 
 \hline
 \hline
 64  & 6.3 & 6.20 & 6.13 & 6.08  \\
 \hline
 32  & 6.24 & 6.12 & 6.07 & 6.03  \\
 \hline
 16  & 6.12 & 6.14 & 6.04 & 6.02  \\
 \hline
 \multicolumn{5}{|c|}{Nemotron4-340B (FP32 PPL = 3.48)} \\ 
 \hline
 \hline
 64 & 3.67 & 3.62 & 3.60 & 3.59 \\
 \hline
 32 & 3.63 & 3.61 & 3.59 & 3.56 \\
 \hline
 16 & 3.61 & 3.58 & 3.57 & 3.55 \\
 \hline
\end{tabular}
\caption{\label{tab:ppl_llama7B_nemo15B} Wikitext-103 perplexity compared to FP32 baseline in Llama2-7B and Nemotron4-15B, 340B models}
\end{table}

%\subsection{Perplexity achieved by various LO-BCQ configurations on MMLU dataset}


\begin{table} \centering
\begin{tabular}{|c||c|c|c|c||c|c|c|c|} 
\hline
 $L_b \rightarrow$& \multicolumn{4}{c||}{8} & \multicolumn{4}{c||}{8}\\
 \hline
 \backslashbox{$L_A$\kern-1em}{\kern-1em$N_c$} & 2 & 4 & 8 & 16 & 2 & 4 & 8 & 16  \\
 %$N_c \rightarrow$ & 2 & 4 & 8 & 16 & 2 & 4 & 2 \\
 \hline
 \hline
 \multicolumn{5}{|c|}{Llama2-7B (FP32 Accuracy = 45.8\%)} & \multicolumn{4}{|c|}{Llama2-70B (FP32 Accuracy = 69.12\%)} \\ 
 \hline
 \hline
 64 & 43.9 & 43.4 & 43.9 & 44.9 & 68.07 & 68.27 & 68.17 & 68.75 \\
 \hline
 32 & 44.5 & 43.8 & 44.9 & 44.5 & 68.37 & 68.51 & 68.35 & 68.27  \\
 \hline
 16 & 43.9 & 42.7 & 44.9 & 45 & 68.12 & 68.77 & 68.31 & 68.59  \\
 \hline
 \hline
 \multicolumn{5}{|c|}{GPT3-22B (FP32 Accuracy = 38.75\%)} & \multicolumn{4}{|c|}{Nemotron4-15B (FP32 Accuracy = 64.3\%)} \\ 
 \hline
 \hline
 64 & 36.71 & 38.85 & 38.13 & 38.92 & 63.17 & 62.36 & 63.72 & 64.09 \\
 \hline
 32 & 37.95 & 38.69 & 39.45 & 38.34 & 64.05 & 62.30 & 63.8 & 64.33  \\
 \hline
 16 & 38.88 & 38.80 & 38.31 & 38.92 & 63.22 & 63.51 & 63.93 & 64.43  \\
 \hline
\end{tabular}
\caption{\label{tab:mmlu_abalation} Accuracy on MMLU dataset across GPT3-22B, Llama2-7B, 70B and Nemotron4-15B models.}
\end{table}


%\subsection{Perplexity achieved by various LO-BCQ configurations on LM evaluation harness}

\begin{table} \centering
\begin{tabular}{|c||c|c|c|c||c|c|c|c|} 
\hline
 $L_b \rightarrow$& \multicolumn{4}{c||}{8} & \multicolumn{4}{c||}{8}\\
 \hline
 \backslashbox{$L_A$\kern-1em}{\kern-1em$N_c$} & 2 & 4 & 8 & 16 & 2 & 4 & 8 & 16  \\
 %$N_c \rightarrow$ & 2 & 4 & 8 & 16 & 2 & 4 & 2 \\
 \hline
 \hline
 \multicolumn{5}{|c|}{Race (FP32 Accuracy = 37.51\%)} & \multicolumn{4}{|c|}{Boolq (FP32 Accuracy = 64.62\%)} \\ 
 \hline
 \hline
 64 & 36.94 & 37.13 & 36.27 & 37.13 & 63.73 & 62.26 & 63.49 & 63.36 \\
 \hline
 32 & 37.03 & 36.36 & 36.08 & 37.03 & 62.54 & 63.51 & 63.49 & 63.55  \\
 \hline
 16 & 37.03 & 37.03 & 36.46 & 37.03 & 61.1 & 63.79 & 63.58 & 63.33  \\
 \hline
 \hline
 \multicolumn{5}{|c|}{Winogrande (FP32 Accuracy = 58.01\%)} & \multicolumn{4}{|c|}{Piqa (FP32 Accuracy = 74.21\%)} \\ 
 \hline
 \hline
 64 & 58.17 & 57.22 & 57.85 & 58.33 & 73.01 & 73.07 & 73.07 & 72.80 \\
 \hline
 32 & 59.12 & 58.09 & 57.85 & 58.41 & 73.01 & 73.94 & 72.74 & 73.18  \\
 \hline
 16 & 57.93 & 58.88 & 57.93 & 58.56 & 73.94 & 72.80 & 73.01 & 73.94  \\
 \hline
\end{tabular}
\caption{\label{tab:mmlu_abalation} Accuracy on LM evaluation harness tasks on GPT3-1.3B model.}
\end{table}

\begin{table} \centering
\begin{tabular}{|c||c|c|c|c||c|c|c|c|} 
\hline
 $L_b \rightarrow$& \multicolumn{4}{c||}{8} & \multicolumn{4}{c||}{8}\\
 \hline
 \backslashbox{$L_A$\kern-1em}{\kern-1em$N_c$} & 2 & 4 & 8 & 16 & 2 & 4 & 8 & 16  \\
 %$N_c \rightarrow$ & 2 & 4 & 8 & 16 & 2 & 4 & 2 \\
 \hline
 \hline
 \multicolumn{5}{|c|}{Race (FP32 Accuracy = 41.34\%)} & \multicolumn{4}{|c|}{Boolq (FP32 Accuracy = 68.32\%)} \\ 
 \hline
 \hline
 64 & 40.48 & 40.10 & 39.43 & 39.90 & 69.20 & 68.41 & 69.45 & 68.56 \\
 \hline
 32 & 39.52 & 39.52 & 40.77 & 39.62 & 68.32 & 67.43 & 68.17 & 69.30  \\
 \hline
 16 & 39.81 & 39.71 & 39.90 & 40.38 & 68.10 & 66.33 & 69.51 & 69.42  \\
 \hline
 \hline
 \multicolumn{5}{|c|}{Winogrande (FP32 Accuracy = 67.88\%)} & \multicolumn{4}{|c|}{Piqa (FP32 Accuracy = 78.78\%)} \\ 
 \hline
 \hline
 64 & 66.85 & 66.61 & 67.72 & 67.88 & 77.31 & 77.42 & 77.75 & 77.64 \\
 \hline
 32 & 67.25 & 67.72 & 67.72 & 67.00 & 77.31 & 77.04 & 77.80 & 77.37  \\
 \hline
 16 & 68.11 & 68.90 & 67.88 & 67.48 & 77.37 & 78.13 & 78.13 & 77.69  \\
 \hline
\end{tabular}
\caption{\label{tab:mmlu_abalation} Accuracy on LM evaluation harness tasks on GPT3-8B model.}
\end{table}

\begin{table} \centering
\begin{tabular}{|c||c|c|c|c||c|c|c|c|} 
\hline
 $L_b \rightarrow$& \multicolumn{4}{c||}{8} & \multicolumn{4}{c||}{8}\\
 \hline
 \backslashbox{$L_A$\kern-1em}{\kern-1em$N_c$} & 2 & 4 & 8 & 16 & 2 & 4 & 8 & 16  \\
 %$N_c \rightarrow$ & 2 & 4 & 8 & 16 & 2 & 4 & 2 \\
 \hline
 \hline
 \multicolumn{5}{|c|}{Race (FP32 Accuracy = 40.67\%)} & \multicolumn{4}{|c|}{Boolq (FP32 Accuracy = 76.54\%)} \\ 
 \hline
 \hline
 64 & 40.48 & 40.10 & 39.43 & 39.90 & 75.41 & 75.11 & 77.09 & 75.66 \\
 \hline
 32 & 39.52 & 39.52 & 40.77 & 39.62 & 76.02 & 76.02 & 75.96 & 75.35  \\
 \hline
 16 & 39.81 & 39.71 & 39.90 & 40.38 & 75.05 & 73.82 & 75.72 & 76.09  \\
 \hline
 \hline
 \multicolumn{5}{|c|}{Winogrande (FP32 Accuracy = 70.64\%)} & \multicolumn{4}{|c|}{Piqa (FP32 Accuracy = 79.16\%)} \\ 
 \hline
 \hline
 64 & 69.14 & 70.17 & 70.17 & 70.56 & 78.24 & 79.00 & 78.62 & 78.73 \\
 \hline
 32 & 70.96 & 69.69 & 71.27 & 69.30 & 78.56 & 79.49 & 79.16 & 78.89  \\
 \hline
 16 & 71.03 & 69.53 & 69.69 & 70.40 & 78.13 & 79.16 & 79.00 & 79.00  \\
 \hline
\end{tabular}
\caption{\label{tab:mmlu_abalation} Accuracy on LM evaluation harness tasks on GPT3-22B model.}
\end{table}

\begin{table} \centering
\begin{tabular}{|c||c|c|c|c||c|c|c|c|} 
\hline
 $L_b \rightarrow$& \multicolumn{4}{c||}{8} & \multicolumn{4}{c||}{8}\\
 \hline
 \backslashbox{$L_A$\kern-1em}{\kern-1em$N_c$} & 2 & 4 & 8 & 16 & 2 & 4 & 8 & 16  \\
 %$N_c \rightarrow$ & 2 & 4 & 8 & 16 & 2 & 4 & 2 \\
 \hline
 \hline
 \multicolumn{5}{|c|}{Race (FP32 Accuracy = 44.4\%)} & \multicolumn{4}{|c|}{Boolq (FP32 Accuracy = 79.29\%)} \\ 
 \hline
 \hline
 64 & 42.49 & 42.51 & 42.58 & 43.45 & 77.58 & 77.37 & 77.43 & 78.1 \\
 \hline
 32 & 43.35 & 42.49 & 43.64 & 43.73 & 77.86 & 75.32 & 77.28 & 77.86  \\
 \hline
 16 & 44.21 & 44.21 & 43.64 & 42.97 & 78.65 & 77 & 76.94 & 77.98  \\
 \hline
 \hline
 \multicolumn{5}{|c|}{Winogrande (FP32 Accuracy = 69.38\%)} & \multicolumn{4}{|c|}{Piqa (FP32 Accuracy = 78.07\%)} \\ 
 \hline
 \hline
 64 & 68.9 & 68.43 & 69.77 & 68.19 & 77.09 & 76.82 & 77.09 & 77.86 \\
 \hline
 32 & 69.38 & 68.51 & 68.82 & 68.90 & 78.07 & 76.71 & 78.07 & 77.86  \\
 \hline
 16 & 69.53 & 67.09 & 69.38 & 68.90 & 77.37 & 77.8 & 77.91 & 77.69  \\
 \hline
\end{tabular}
\caption{\label{tab:mmlu_abalation} Accuracy on LM evaluation harness tasks on Llama2-7B model.}
\end{table}

\begin{table} \centering
\begin{tabular}{|c||c|c|c|c||c|c|c|c|} 
\hline
 $L_b \rightarrow$& \multicolumn{4}{c||}{8} & \multicolumn{4}{c||}{8}\\
 \hline
 \backslashbox{$L_A$\kern-1em}{\kern-1em$N_c$} & 2 & 4 & 8 & 16 & 2 & 4 & 8 & 16  \\
 %$N_c \rightarrow$ & 2 & 4 & 8 & 16 & 2 & 4 & 2 \\
 \hline
 \hline
 \multicolumn{5}{|c|}{Race (FP32 Accuracy = 48.8\%)} & \multicolumn{4}{|c|}{Boolq (FP32 Accuracy = 85.23\%)} \\ 
 \hline
 \hline
 64 & 49.00 & 49.00 & 49.28 & 48.71 & 82.82 & 84.28 & 84.03 & 84.25 \\
 \hline
 32 & 49.57 & 48.52 & 48.33 & 49.28 & 83.85 & 84.46 & 84.31 & 84.93  \\
 \hline
 16 & 49.85 & 49.09 & 49.28 & 48.99 & 85.11 & 84.46 & 84.61 & 83.94  \\
 \hline
 \hline
 \multicolumn{5}{|c|}{Winogrande (FP32 Accuracy = 79.95\%)} & \multicolumn{4}{|c|}{Piqa (FP32 Accuracy = 81.56\%)} \\ 
 \hline
 \hline
 64 & 78.77 & 78.45 & 78.37 & 79.16 & 81.45 & 80.69 & 81.45 & 81.5 \\
 \hline
 32 & 78.45 & 79.01 & 78.69 & 80.66 & 81.56 & 80.58 & 81.18 & 81.34  \\
 \hline
 16 & 79.95 & 79.56 & 79.79 & 79.72 & 81.28 & 81.66 & 81.28 & 80.96  \\
 \hline
\end{tabular}
\caption{\label{tab:mmlu_abalation} Accuracy on LM evaluation harness tasks on Llama2-70B model.}
\end{table}

%\section{MSE Studies}
%\textcolor{red}{TODO}


\subsection{Number Formats and Quantization Method}
\label{subsec:numFormats_quantMethod}
\subsubsection{Integer Format}
An $n$-bit signed integer (INT) is typically represented with a 2s-complement format \citep{yao2022zeroquant,xiao2023smoothquant,dai2021vsq}, where the most significant bit denotes the sign.

\subsubsection{Floating Point Format}
An $n$-bit signed floating point (FP) number $x$ comprises of a 1-bit sign ($x_{\mathrm{sign}}$), $B_m$-bit mantissa ($x_{\mathrm{mant}}$) and $B_e$-bit exponent ($x_{\mathrm{exp}}$) such that $B_m+B_e=n-1$. The associated constant exponent bias ($E_{\mathrm{bias}}$) is computed as $(2^{{B_e}-1}-1)$. We denote this format as $E_{B_e}M_{B_m}$.  

\subsubsection{Quantization Scheme}
\label{subsec:quant_method}
A quantization scheme dictates how a given unquantized tensor is converted to its quantized representation. We consider FP formats for the purpose of illustration. Given an unquantized tensor $\bm{X}$ and an FP format $E_{B_e}M_{B_m}$, we first, we compute the quantization scale factor $s_X$ that maps the maximum absolute value of $\bm{X}$ to the maximum quantization level of the $E_{B_e}M_{B_m}$ format as follows:
\begin{align}
\label{eq:sf}
    s_X = \frac{\mathrm{max}(|\bm{X}|)}{\mathrm{max}(E_{B_e}M_{B_m})}
\end{align}
In the above equation, $|\cdot|$ denotes the absolute value function.

Next, we scale $\bm{X}$ by $s_X$ and quantize it to $\hat{\bm{X}}$ by rounding it to the nearest quantization level of $E_{B_e}M_{B_m}$ as:

\begin{align}
\label{eq:tensor_quant}
    \hat{\bm{X}} = \text{round-to-nearest}\left(\frac{\bm{X}}{s_X}, E_{B_e}M_{B_m}\right)
\end{align}

We perform dynamic max-scaled quantization \citep{wu2020integer}, where the scale factor $s$ for activations is dynamically computed during runtime.

\subsection{Vector Scaled Quantization}
\begin{wrapfigure}{r}{0.35\linewidth}
  \centering
  \includegraphics[width=\linewidth]{sections/figures/vsquant.jpg}
  \caption{\small Vectorwise decomposition for per-vector scaled quantization (VSQ \citep{dai2021vsq}).}
  \label{fig:vsquant}
\end{wrapfigure}
During VSQ \citep{dai2021vsq}, the operand tensors are decomposed into 1D vectors in a hardware friendly manner as shown in Figure \ref{fig:vsquant}. Since the decomposed tensors are used as operands in matrix multiplications during inference, it is beneficial to perform this decomposition along the reduction dimension of the multiplication. The vectorwise quantization is performed similar to tensorwise quantization described in Equations \ref{eq:sf} and \ref{eq:tensor_quant}, where a scale factor $s_v$ is required for each vector $\bm{v}$ that maps the maximum absolute value of that vector to the maximum quantization level. While smaller vector lengths can lead to larger accuracy gains, the associated memory and computational overheads due to the per-vector scale factors increases. To alleviate these overheads, VSQ \citep{dai2021vsq} proposed a second level quantization of the per-vector scale factors to unsigned integers, while MX \citep{rouhani2023shared} quantizes them to integer powers of 2 (denoted as $2^{INT}$).

\subsubsection{MX Format}
The MX format proposed in \citep{rouhani2023microscaling} introduces the concept of sub-block shifting. For every two scalar elements of $b$-bits each, there is a shared exponent bit. The value of this exponent bit is determined through an empirical analysis that targets minimizing quantization MSE. We note that the FP format $E_{1}M_{b}$ is strictly better than MX from an accuracy perspective since it allocates a dedicated exponent bit to each scalar as opposed to sharing it across two scalars. Therefore, we conservatively bound the accuracy of a $b+2$-bit signed MX format with that of a $E_{1}M_{b}$ format in our comparisons. For instance, we use E1M2 format as a proxy for MX4.

\begin{figure}
    \centering
    \includegraphics[width=1\linewidth]{sections//figures/BlockFormats.pdf}
    \caption{\small Comparing LO-BCQ to MX format.}
    \label{fig:block_formats}
\end{figure}

Figure \ref{fig:block_formats} compares our $4$-bit LO-BCQ block format to MX \citep{rouhani2023microscaling}. As shown, both LO-BCQ and MX decompose a given operand tensor into block arrays and each block array into blocks. Similar to MX, we find that per-block quantization ($L_b < L_A$) leads to better accuracy due to increased flexibility. While MX achieves this through per-block $1$-bit micro-scales, we associate a dedicated codebook to each block through a per-block codebook selector. Further, MX quantizes the per-block array scale-factor to E8M0 format without per-tensor scaling. In contrast during LO-BCQ, we find that per-tensor scaling combined with quantization of per-block array scale-factor to E4M3 format results in superior inference accuracy across models. 



\end{document}
