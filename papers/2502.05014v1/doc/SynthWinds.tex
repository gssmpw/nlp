\section{Synthetic Forecast Modeling}\label{section:SynthForecastModeling}
%Do we Call this something Else:?

To have autonomous HABs that are more robust to inaccurate weather forecasts, realistic deviations from available weather forecasts are necessary.  However, determining and applying these deviations is nontrivial; while adding simple noise to the ERA5 forecasts could be one approach, it is not very realistic for several reasons. Often, real winds at a particular altitude during flight can be up to 180\degree off from the forecast.  Additionally, winds at one altitude level do not necessarily linearly shift in speed and direction when there are significant changes between time steps. 
%Additionally, winds at a particular altitude level typically follow local trends horizontally and temporally. For example, if winds at a particular altitude level are blowing east at one timestamp, but then west 6 hours later, the inner timestamps do not necessarily linearly shift in speed and direction.  
Finally, and most importantly, winds at one altitude level are frequently decoupled from other altitude levels, hence why opposing winds and wind diversity are possible. %Therefore, while noise could be added to forecasts at mandatory pressure levels, creating realistic wind profiles between those mandatory levels is impossible to do with forecasts alone.  

%In order to station keep, HABs typically oscillate between altitudes with opposing winds, stay in calm winds ($<$ 2m/s) regions, or some combination of the two. Most altitude controlled HAB platforms operate between 18-28 km, above the typical max altitude regulation for many countries (such as FL600 for the USA).  Luckily, this region of the atmosphere frequently has vertical wind diversity for maneuvering HABs, where different altitudes can have opposing winds. The probability of available diverse winds in a region of interest substantially differs geographically and seasonally. This concept is explored further in subsequent sections.

%In order to transition simulated trained DQN HAB models to the real world, the simulator needs to closely match real-world HAB dynamics, weather conditions, and uncertainty with weather forecasts. 
Simulating realistic winds and uncertainty in forecasts is a major challenge due to a lack of in situ weather data in the upper troposphere and in the lower stratosphere. Popular Reanalysis forecasts such as ECWMF and NASA GEOS only report wind data at specific levels, that lack vertical resolution. NOAA's GFS forecast is similar but forward predicts forecasts rather than reanalysis;  we will most likely use GFS forecasts for real-world SHAB flights with trained models in the future. Since pressure falls exponentially from sea level to the upper atmosphere,  higher altitudes have decreased vertical resolution. Within the typical HAB operating altitude region, most forecasts only report between 5-10 pressure levels of wind data. 

%\subsection{Synthetic Wind Data Generation}\label{SynthForecastModeling_SynthWindDataGeneration}



\begin{figure}[!ht]
\centering
    \begin{subfigure}[t]{0.4\textwidth}
    \centering
    \includegraphics[width=\textwidth]{img/Radiosonde-Map.png}
    \caption{Map of Radiosonde Locations in the Southwestern United States} \label{fig1}
\end{subfigure}\hfill
\begin{subfigure}[t]{0.4\textwidth}
    \centering
    \includegraphics[width=\textwidth]{img/Radiosonde3D.png}
    \caption{3D wind vector visualization from unprocessed radiosonde measurements} \label{fig2}
\end{subfigure}\hfill
\begin{subfigure}[t]{0.4\textwidth}
    \centering
    \includegraphics[width=\textwidth]{img/Smoothed-Radiosonde-Level.png}
    \caption{Smoothed Synthetic Winds at altitude level 16 km} \label{fig2}
\end{subfigure}\hfill

\caption{Synthetic Wind Generation from Aggregated and Interpolated Radiosonde Data in the Southwestern United States on August
23, 2023 at 1200 UTC}
\label{fig:synthwinds}
\end{figure}

\begin{figure*}[b] % Figure* spans two columns, [t] places it at the top of the page
    \centering
    % Row 1
    \subfloat[ERA5 Model (Jan)]{\includegraphics[width=0.32\textwidth]{img/01_17_50_a.png}\label{fig:01_17_50_a}}%
    \hspace{0.01\textwidth} % Minimal horizontal space between figures
    \subfloat[Synthetic Model (Jan)]{\includegraphics[width=0.32\textwidth]{img/01_17_50_b.png}\label{fig:01_17_50_b}}%
    \hspace{0.01\textwidth} % Minimal horizontal space between figures
    \subfloat[Model Difference (Jan)]{\includegraphics[width=0.32\textwidth]{img/01_17_50_c.png}\label{fig:01_17_50_c}}
    
    \vspace{0.1cm} % Reduce vertical spacing between rows
    
    % Row 2
    \subfloat[ERA5 Model (Jul)]{\includegraphics[width=0.32\textwidth]{img/07_17_50_a.png}\label{fig:07_17_50_a}}%
    \hspace{0.01\textwidth} % Minimal horizontal space between figures
    \subfloat[Synthetic Model (Jul)]{\includegraphics[width=0.32\textwidth]{img/07_17_50_b.png}\label{fig:07_17_50_b}}%
    \hspace{0.01\textwidth} % Minimal horizontal space between figures
    \subfloat[Model Difference (Jul)]{\includegraphics[width=0.32\textwidth]{img/07_17_50_c.png}\label{fig:07_17_50_c}}
    
    \caption{ERA5 and Synthetic Model Variations on January 17 and July 17, 2023 at 000UTC at the 50 hPa Pressure Level}
    \label{fig:JanJulyVariation}
\end{figure*}




To generate realistic wind fields for higher vertical resolution and realistic deviations from ERA5 forecasts we generate synthetic wind fields based on aggregated and interpolated radiosonde data from a particular region. The international Radiosonde program is the largest source of high-resolution in-situ atmospheric meteorological measurements over 2000 radiosonde launch sites locations all over the world are launched twice a day. These radiosondes typically collect temperature, pressure, humidity, and wind velocity measurements from surface level to 25+ km; the data is then assimilated into weather forecasts. We collected our radiosonde data from the University of Wyoming's Upper Air Sounding Database \cite{UofWyUpperAir}. Throughout this work, we use the Southwestern United States region, shown in Figure \ref{fig:synthwinds}a for our station-keeping simulations. We have conducted several flight experiments of SHAB-Vs in this region and it contains an adequate number of radiosonde launch sites for generating synthetic forecasts. Figure \ref{fig:synthwinds}b shows all of the unprocessed wind velocity measurements recorded from radiosondes in the Southwestern United States region launched on August 23, 2023, at 1200 UTC. While radiosonde data has a very high vertical resolution compared to ERA5 forecasts,  they have poor horizontal and temporal resolution.  To account for this, we make several assumptions during aggregation and interpolation of the radiosonde data to generate synthetic forecasts.  
%Should we cite university of wyoming sounding database here? 


%Explanation of Synthetic Winds
First, we bin the radiosonde data from an individual station into 250 m regions, taking the nearest $u$ and $v$ wind components.  Typically, there are multiple wind velocities per bin, in which case we take the closest value instead of an average (ex. with radiosonde altitude readings of 15998 and 16103 meters, we would select 15998 m for the 16000 m bin). In the case of empty bins, the bin is filled with $u$ and $v$ wind components via linear interpolation from the nearest filled bins. Next, the radiosonde data is interpolated horizontally using a nearest neighbor method; this generates coarse planar fields of winds at each altitude level for the area of interest. Finally, we apply Gaussian smoothing to each level to add smooth variation between the sub-areas of wind. Figure \ref{fig:synthwinds}c shows the final smoothed plane of a synthetic wind field at 16 km. 

Unfortunately, the synthetic winds are typically only available in 12-hour increments, at 0000 and 1200 UTC.  In simulation, we reduce this time gap from 12 hours to 3 hours to increase variation and ideally make the trained DRL HAB agents more robust to situations with highly dynamic forecasts. Three hours is also the temporal resolution of forward-predicted GFS forecasts. %Due to diurnal effects as well as other processes that happen over 12 hours, this can significantly increase the occurrence of large shifts in winds. 
We hypothesize that if the trained DQN HAB agents are successful at station-keeping with large intraday wind shifts, they should also perform similarly or better on smoother changes.


\subsection{Comparing Synthetic Wind Forecasts with ERA5 Reanalysis Forecasts} \label{SynthForecastModeling_SynthCompareERA5}

%There are a few key distinctions between the ERA5 reanalysis models and the synthetic wind models generated from radiosonde data. For one, the vertical resolution of the ERA5 model is significantly less than that of our synthetic wind model. For example, the ERA5 model provides seven pressure levels: 20 hPa, 30 hPa, 50 hPa, 70 hPa, 100 hPa, 125 hPa, and 150 hPa. For simulation purposes, this vertical resolution is insufficient, and thus our synthetic wind model provides 46 pressure levels between 21.6 hPa and 121.1 hPa. 

When comparing the two forecasts, the synthetic winds forecasts are overall highly correlated to the ERA5 forecasts but also exhibit variance trends across certain pressure levels and seasons. Fig. \ref{fig:JanJulyVariation} shows the differences between the ERA5 and synthetic wind model for the SW USA region on January 17, 2023 and July 17, 2023. Fig. \ref{fig:01_17_50_c} shows the wind direction between an ERA5 and synthetic forecast at the 50 hPa pressure level on January 17, 2023, in the USA region. As shown visually, there is a significant difference in magnitude and direction with a mean angular difference of $61.6 \pm 47.2 \degree$ in January.  In this context, a large standard deviation is indicative of a non-uniform shift between the two models' wind direction. Similarly, a lower standard deviation may indicate that the model is inaccurate but has a uniform shift of some magnitude.
%What do you mean by this?
The same pressure level (50 hPa) on July 17, shown in Fig. \ref{fig:07_17_50_c} has a mean difference of $13.7 \pm 12.2 \degree$. When compared to the same region in January, the accuracy of the synthetic model in July is considerably better. 
This may be due to the increased turbulence in transition regions, and pressure levels between 50 hPa and 100 hPa, during the month of January. As a result, the number of inflection points can produce inaccuracies during the spatial interpolation. Subsequent sections analyze seasonal trends that help characterize these discrepancies. 

