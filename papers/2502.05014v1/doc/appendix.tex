\onecolumn
\section*{Appendix}
\addcontentsline{toc}{section}{Appendix}

\begin{figure*}[!ht] % Use !ht to enforce placement closer to the command
    \centering
    % Row 1
    \begin{subfigure}[h]{\textwidth}
        \centering
        \includegraphics[width=\textwidth]{img/ModelVariationJanuary.png}
        \caption{Mean and Standard Deviation of Model Variation in January 2023}
        \label{fig:ModelVariationJanuary}
    \end{subfigure}
    
    \vspace{4mm} % Slightly reduce vertical space

    % Row 2
    \begin{subfigure}[h]{\textwidth}
        \centering
        \includegraphics[width=\textwidth]{img/ModelVariationApril.png}
        \caption{Mean and Standard Deviation of Model Variation in April 2023}
        \label{fig:ModelVariationApril}
    \end{subfigure}
    
    \vspace{4mm} % Slightly reduce vertical space

    % Row 3
    \begin{subfigure}[h]{\textwidth}
        \centering
        \includegraphics[width=\textwidth]{img/ModelVariationJuly.png}
        \caption{Mean and Standard Deviation of Model Variation in July 2023}
        \label{fig:ModelVariationJuly}
    \end{subfigure}
    
    \vspace{4mm} % Slightly reduce vertical space

    % Row 4
    \begin{subfigure}[h]{\textwidth}
        \centering
        \includegraphics[width=\textwidth]{img/ModelVariationOctober.png}
        \caption{Mean and Standard Deviation of Model Variation in October 2023}
        \label{fig:ModelVariationOctober}
    \end{subfigure}

    \caption{Mean \& Standard Deviation of Model Variation at Different Pressure Levels in Southwestern USA}
    \label{fig:ModelVariations}
\end{figure*}
