\section{Conclusions}

Altitude-controlled HABs such as SHAB-Vs can leverage opposing winds at various altitude levels to conduct short-duration station-keeping missions; the winds are constantly changing, and frequently unknown, making path planning very difficult. Recently, deep reinforcement learning has become a popular tool in robotics for path planning and obstacle avoidance in complex dynamic environments.  In this work, we developed a custom simulation environment for training and evaluating the station-keeping performance of short-duration HABs with deep reinforcement learning using Deep Q-Networks (DQN). A major limitation of training HABs for station keeping in simulation and transitioning to real-world flights is a lack of high-resolution wind data to include.  ERA5 reanalysis forecasts have historical hourly wind data at specific pressure levels from 1940 to the present but significantly lack vertical resolution, with typically less than 10 pressure levels included in the maneuverable altitude region, with up to several kilometer gaps between levels.  

We introduced a new strategy for generating synthetic wind forecasts from historical radiosonde data to create realistic deviations from the ERA5 forecast. From our initial analysis of forecasts in the Southwestern United States for several months in 2023, we show that overall, the synthetic wind forecasts tend to have a high correlation with ERA5 forecasts. Because the Synthetic Winds are aggregated from real balloon data, this leads us to assume that synthetic winds at altitude levels in between the mandatory pressure levels from ERA5 forecasts are realistic to real-world.  The biggest limitation with using Radiosondes for generating synthetic forecasts is the spare launch sides worldwide, as well as a lack of temporal resolution, with radiosondes typically only being launched twice a day.  

With DQN, we successfully trained short-duration HABs in simulation to station-keep and maintain time within a 50km region (TWR50) approximately 50\% of the time.  The best models for station-keeping short-duration HABS are highly dependent on when and where the HABs are launched.  The best months for station keeping have both high wind diversity and high wind velocity correlations between the synthetic forecasts and ERA5 forecasts.  To help predict station-keeping performance based on an ERA5 forecast, we introduced a forecast classification algorithm independent of the DQN algorithm that directionally bins winds at each altitude level and totals how many levels have opposing wind pairs, with higher totals leading to higher Forecast Scores.  Overall, The Forecast Score classification method shows higher Forecast Scores (typically over 60\%) have higher probabilities of successful TWR50 station-keeping performance.   Monthly evaluation of trained agents compared with Forecast Scores shows that probabilities of station-keeping success with high Forecast Scores vary, with April and July having higher probabilities than January and October. 

\subsection{Future Work}

In the future, we plan to explore more advanced interpolation, noise application, and smoothing strategies to increase the temporal resolution of the synthetic winds while remaining coupled to changes in the known ERA5 reanalysis forecasts. We also plan to compare synthetic forecasts and the performance of trained DQN agents on GFS forecasts as opposed to ERA5 forecasts, which are forward predicted, have lower temporal resolution than ERA5 forecasts, and most likely have lower correlation.  We plan to integrate the trained DQN algorithms on SHAB-Vs to evaluate station-keeping performance on real short-duration HABs.  Before conducting flight tests, we want to continue training agents on larger subsets of forecast data, instead of monthly forecasts, to be more robust to winds with high variance from seasonal trends and other unforeseen weather conditions.  To help transition from simulated agents to HABs, we also plan to develop an indoor testbed with miniature autonomous blimps and artificial wind fields using fans to evaluate and iterate our algorithms on easy-to-deploy lighter-than-air platforms.   


%\begin{itemize}
%    \item Comparing synth to real flight data
%    \item Doing real flights with SHAB-Vs
%    \item indoor simulations 
%    \item improve era5 vs synth (more analysis, realistic noise, etc.)
%    \item GFS predictions vs era5 vs synth comparison
%    \item Training more years (more robust) 
%    \item Training on GFS vs era5 in observation space (does this already exist?)
%\end{itemize}

