\section{Discussion}
\label{sec:discussion}
The following subsections compare our findings against prior works and suggest recommendations.

\subsection{Potential Explanations of Several  Key Findings}
The results of our RQ2 (Section~\ref{sec:rq2} suggest that project popularity, measured in terms of the `number of stars,' is associated with increased toxicity. A project's popularity may put pressure on contributors to deliver new features and maintain quality. However, a rapid development pace can cause stress, burnout, and toxicity. We also found toxicity increasing with project age. Our manual investigation of sample projects suggests staleness (i.e., lack of response to issues or PRs) may be a potential reason.

The results of RQ3 (Section~\ref{sec:rq3}) suggest that review duration and the number of required iterations are positively associated with toxicity, with stronger associations seen among higher PRF groups. {These results suggest that frustrations may grow between authors and reviewers due to multiple review iterations, particularly among projects with higher activity levels.}
We also found a positive association between code complexity and toxicity. {This result indicates that complex changes, which are difficult to understand and review, may cause confusion~\cite{ebert2019confusion} and are more likely to be associated with toxicity.}



\begin{table*}
    
    \caption{Comparison against prior empirical studies investigating anti-social behaviors among OSS project}
    \label{tab:comparison}
    \centering   
\resizebox{\textwidth}{!}{  
    \begin{tabular}{|p{1.8cm}|p{2cm}|p{2.5cm}|p{3.5cm}|p{1.5cm}|p{3.5cm}|} \hline

    
     \textbf{Study} & \textbf{Method}  & \textbf{Sample Size}   & \textbf{{Sampling Criteria}} &  \textbf{\# factors}& \textbf{Comparison with ours}\\ \hline
     
Raman \textit{et} al.~\cite{raman2020stress} & Quantitative & 872k issues from 30 popular projects. & i) Training dataset from  `too heated locked issues; ii) Poor performance of their classifier with 47\% F1-score. &  3 & One overlapping factor, which concurs with our finding.  \\  \hline



Ferreira \textit{et} al.~\cite{ferreira2021shut} &	Qualitative &	1,545 email threads from Linux. &	i) Only rejected patches, ii) Linux kernel maintainers are known to be harsh.	& 9	& Two overlapping factors, where finding for one contradicts, and the other one concurs. \\  \hline



Miller \textit{et} al.~\cite{miller2022did} &	Qualitative &	100 issue discussions.	& i) Small sample, ii) Only locked issue threads.	& 10	& Six overlapping factors, where findings for two contradict, and the remaining four concur.

\\  \hline

Egelman \textit{et} al.~\cite{egelman2020predicting} &	Opinion survey &	Surveyed 1,397 developers in Google.	& i) One organization, ii) lack of quantitative validation	& 5	& Three overlapping factors, where all concur with our finding. \\  \hline


Ours &	Mixed, mostly quantitative &	2,828 GitHub projects and over 100M comments. &	Limitations of ToxiCR~\cite{sarker2022automated} applies. &	32	&  - \\  \hline

    \end{tabular}
}
     %\vspace{-15pt}
\end{table*}

\subsection{Comparison with  Prior SE Studies} 

Our large-scale empirical investigation includes 32 attributes from four categories. Out of those, 11 were investigated in other contexts in prior studies. Therefore, Table~\ref{tab:comparison} compares sample size, projects, and the number of factors and overlaps with our studies against the others to illustrate the novelty and significance of this study. Similar to~\citet{miller2022did}, profanity is the prevalent toxicity in our randomly sampled dataset. They reported a high share (25\%) of entitled issue comments, which are demands to project maintainers as if they had a contractual relationship or obligation~\cite{miller2022did}. However, we found only 3.2\% such cases in our sample. Supporting their findings, our results indicate repeat offenders, toxicity increasing with project popularity, long-term project contributors being authors of toxicity, and gaming projects harboring more toxic cases~\cite{miller2022did}. They also reported toxic comments from new GitHub accounts~\cite{miller2022did}. Aligning with this finding, we noticed the likelihood of authoring toxic comments decreased with GitHub tenure.
However, contrary to their findings, we notice a lower likelihood of project newcomers authoring toxic comments.
Our results also concur with one finding by~\citet{raman2020stress}, as we found a lower likelihood of toxicity among corporate-sponsored projects.
During their manual investigation of the Linux kernel, \citet{ferreira2021shut} found uncivil comments during reviews of rejected codes. Aligning with their findings, we noticed lower odds of toxicity among accepted PRs. They also reported incivility among project maintainers' feedback~\cite{ferreira2021shut}. However, contrasting their findings, we noticed a lower likelihood of toxicity from project members.
\citet{egelman2020predicting} reported a higher likelihood of pushback on large code changes. Our result aligns with this finding, as we found that the odds of toxicity increase with code churn.
\citet{raman2020stress} reported incivility due to poor-quality code changes. While we did not measure code quality directly, we may use the number of review comments as an indication of code quality since each review comment indicates an issue identified by a reviewer. Aligning with their findings, our results indicate higher odds of toxicity with the number of review comments.


\vspace{3pt}
\noindent \textbf{Potential reasons behind some of our findings contradicting prior studies:}
While we do not have a concrete answer to why some of our results contradict prior studies, we hypothesize that sampling differences may be a major factor. Prior studies picked samples from contexts where antisocial behaviors are more likely to occur (e.g., locked issues, rejected patches, heated discussions). However, these cases are not very frequent. For example, only 3.1\% of PR-linked issues in our sample are locked.
We noticed the most differences (i.e., two) against~\citet{miller2022did}, an exploratory study investigating a sample of 100 locked issues. Although valid for a specific context, their {selected} cases may not represent broader trends across GitHub. For example, arrogant contributors forcefully demanding acceptance of their pull request is a reasonable cause to lock issue threads. Hence, they obtained 25\% entitlement, but such cases are significantly lower among non-locked issue threads. For the same reason, \citet{miller2022did} reported {entitled type} behavior from newcomers, but our analysis suggests that newcomers are less likelier to author toxic texts than long-term contributors.
We also noted a discrepancy compared to ~\citet{ferreira2021shut}, who drew their sample from rejected patches of the Linux kernel mailing list and reported uncivil behavior from maintainers. Several Linux kernel maintainers have been known for their blunt communication style for years~\cite{toxic-blog-linux1,toxic-blog-linux3}. Our findings suggest that what was reported from LKML may not be a broader trend across the OSS spectrum. These contradictions also highlight the need for a large-scale study with diverse samples to understand the landscape of toxicity better.






\subsection{Actionable Recommendations}
Due to our study design, we cannot claim causal relationships for the associations identified in this study. However, some of the following recommendations apply only if such relationships exist.

\vspace{4pt}
\noindent \textbf{ I. Project Maintainers:}  
Our results from RQ3 (i.e., table~\ref{tab:toxicity_context}) suggest that delays in fixing bugs or answering user queries may create unhappy users and toxic comments targeted toward maintainers.
As a project's popularity grows, maintainers should focus on improving bug resolution since our results also show that a higher bug resolution rate is negatively associated with toxicity. Even if an issue is delayed, maintainers should respond politely and suggest workarounds, if possible, to avoid toxic interactions.
From the RQ4 analysis, project tenure is positively associated with toxicity. Therefore, building a positive culture needs to start with project maintainers since they are likelier to be project members with the longest tenures~\cite{toxic-blog-linux3}. 
Supporting prior studies~\cite{miller2022did,belskie2023measuring,paul2018toxic,beres2021don}, we also found a proliferation of toxicity among gaming projects in RQ2. Therefore,  we recommend that maintainers of gaming projects adopt a Code of Conduct and its enforcement mechanism to build a diverse community. 




\vspace{4pt}
\noindent\textbf{II. Developers:}   We recommend developers avoid creating pull request contexts that are positively associated with toxicity.  For example, our results from RQ3 indicate that delayed pull requests are associated with toxicity. Therefore, reviewers should provide on-time reviews to avoid frustrating authors.
Similarly, large code changes are not only bug-prone~\cite{bosu2015characteristics} and difficult to review~\cite{thongtanunam2017review} but also likely to encounter toxicity. Therefore, when possible, creating pull requests with smaller changes is recommended. According to the findings from RQ3, pull requests with a large number of issues indicate poor quality codes and are more likely to receive harsh critiques. Therefore, developers should not create pull requests with changes that do not yet meet the quality standards for a project. A higher number of review iterations also frustrates authors and may cause toxicity. Hence, if possible, reviewers should request all required changes within a single cycle to avoid back and forth.
Complex changes are hard to review and are more likely to receive toxicity. Hence, authors should annotate such changes and include helpful descriptions to avoid confusion~\cite{ebert2019confusion} as well as toxicity.
 Even when frustrated or angry, developers should not use toxic languages since developers who use such languages are more likely to become victims (i.e., findings from RQ4). 
Finally, while contrary to prior evidence, we find that women and newcomers are less likely to be targets of toxicity in RQ4, we still recommend long-term contributors avoid such language if such persons are present in a discussion since toxicity not only dissuades newcomers from becoming a part of the communities~\cite{steinmacher2015social} but also disproportionately hurts minorities~\cite{gunawardena2022destructive}.

\vspace{4pt}
\noindent\textbf{III. Prospective joiners: } If a newcomer wants to avoid negative experiences associated with toxic cultures, we recommend they start with a corporate-sponsored OSS project that matches their expertise and interests. We also recommend such contributors avoid gaming or stale projects.

\vspace{4pt}
\noindent \textbf{IV. Researchers:}
We found variations among terminologies used for almost identical concepts among SE studies investing in anti-social behaviors. We also noticed conflicting opinions about whether a particular category should be considered anti-social. Since existing schemes are primarily based on the decisions of the respective researchers, they may not reflect the broader OSS community. Therefore, existing identification tools based on these schemes may not align with OSS developers' needs and would fail to achieve broader adoption. Moreover, recent research suggests whether a text should be considered toxic depends on various demographic characteristics~\cite{goyal2022your}. Hence, understanding the opinions of the broader OSS community and how their demographics influence perspectives of toxicity is essential to developing a custom mitigation strategy. 
