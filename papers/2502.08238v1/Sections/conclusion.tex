%\vspace{-10pt}
\section{Conclusion}


\label{sec:conclusion}
We conducted a large-scale mixed-method empirical study of 2,828 GitHub-based OSS projects to understand 
the nature of toxicities on GitHub and how various measurable characteristics of a project, a pull request's context, and participants associate with their prevalence.
We found profanity to be the dominant form of toxicity on GitHub, followed by trolling and insults. 
While a project's popularity is positively associated with the prevalence of toxicity, its issue resolution rate has the opposite association.
Corporate-sponsored projects are less toxic, but gaming projects are seven times more likely than non-gaming ones to have a high volume of toxicities. OSS developers who have authored toxic comments in the past are significantly more likely to repeat them and become toxicity targets.
Based on the results of this study and our experience conducting it, we provide recommendations to OSS contributors and researchers.


 

