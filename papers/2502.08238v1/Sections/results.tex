\section{Results}
\label{sec:results}


The following subsections detail the results of the four RQs.
%vspace{-4pt}


\subsection{RQ1: Nature of toxicity}
\label{sec:results_rq1}
\begin{table*}
    %\vspace{-10pt}
    \caption{The most common forms of toxicities with definitions within our sample of manually labeled 532 PR comments. We also showed the mapping from existing works.}
\label{tab:toxicity_types}
    \centering
\resizebox{\textwidth}{!}{  
    \begin{tabular}{|p{1.8cm}|p{3.6cm}|p{3.8cm}|p{4cm}|>{\raggedleft\arraybackslash}p{1cm}|r|} \hline

    
     \textbf{Type} & \textbf{Mapping}& \textbf{Definition}     & \textbf{Example} &  \textbf{Count$\ddag$}& \textbf{Ratio}\\ \hline
     
Profanity & Profanity~\cite{sarker2022automated}, Expletives~\cite{miller2022did}, Vulgarity~\cite{ferreira2021shut}  & A comment that includes profanity. & \textcolor{brown}{``You know, at some point, github fucked me over. In Visual Studio, this was just fine, wtf....''}  &  311 & $58.45\%$  \\  \hline

Trolling & Trolling~\cite{miller2022did, ferreira2021shut} & Using trolling with destructive discussions and those are more severe and provoke arguments. &  \textcolor{brown}{``$@$clusterfuck There's a difference between being in cryogenics and not in cryogenics you big nerd''} &96 & 18.04\%\\  \hline

Insult & Insult~\cite{miller2022did,sarker2022automated} & Disrespectful expression towards another person. &  \textcolor{brown}{``Acknowledge that the vote wasn't entirely singulo shitposters $>$ ARE YOU SCHIZOPHRENIC??  Jesus dude why are you even here still''}  &92 & $17.3\%$ \\  \hline



Self-deprecation & Self-deprecation~\cite{miller2022did} & If a demeaning word towards him/herself consists of severe language, it would be marked as self-deprecation.  &  \textcolor{brown}{``@ComicIronic Okay, I'll fix it when I fix my shitty code, which will  have to happen tomorrow''} &67 & 12.6\% \\  \hline

Object Directed Toxicity & {New} & Anger, frustration, or profanity directed
toward software, products, or artifacts.  & \textcolor{brown}{``the PR has fallen into conflict hell, I'll be closing this and re-opening some of its changes shornestly''} &  49 &9.21\% \\ \hline 

  
Entitled & Entitled~\cite{miller2022did} & When people demand due to the expectation related to contractual relationship or payment.  & \textcolor{brown}{``Again I didn't break it  Are you fucking stupid lol  $>$ merge your update into PR $>$ buckling doesn't work''} & 17& 3.2\% \\  \hline

Identity attack & Identity attack~\cite{sarker2022automated}, Inappropriate jokes about an employee~\cite{egelman2020predicting} &  Attacking the person's identity. & \textcolor{brown}{``Fuck those argentinians. Did you test it?''} &17 & 3.2\% \\  \hline

Threats & Threats~\cite{ferreira2021shut,sarker2022automated} & A behavior that is aggressive or threatening someone. & \textcolor{brown}{``Done - I can always revoke your access if you mess things up ;''} &12 & 2.25\% \\  \hline


Obscenity & Reference to sexual activities~\cite{sarker2022automated} &An extremely offensive comment that demeans women or LGBTQ+ people.  & \textcolor{brown}{``dude you need to spend less time on programming and more time with women''} &6& $1.1\%$ \\ \hline 

Arrogance &Arrogance~\cite{miller2022did} & Imposing the own view on others due to superiority. & \textcolor{brown}{``Araneus is shit and generic as hell I think steely is an acceptable name''} &5 & $<1\%$ \\  \hline



Flirtation &Flirtation~\cite{sarker2022automated} & A comment that represents flirting. &  \textcolor{brown}{``Frigging love you Niki. Seriously''} &5 & $<1\%$ \\ \hline 

\multicolumn{6}{l}{$\ddag$ -\textit{since a text can belong to multiple categories, the sum of the categories is greater than our sample size.}}




    \end{tabular}
}
\end{table*}


Table~\ref{tab:toxicity_types} shows the distributions of 11 forms of toxicities among our manually labeled dataset of 532 PR review comments.
Similar to ~\citet{miller2022did}'s investigation, we found profanity (i.e., severe language, swearing, cursing) as the most common form, with more than half of the sample ($\approx 58\%$) belonging to it. 
Toxic texts authored by OSS contributors frequently include profane words such as `shit', `fuck', `ass', `crap', `suck', and `damn'.
We found trolling to be the second most common form, with 18\%, followed by insult, self-deprecation, and object-directed toxicity. 
Identity attacks, insults, and threats, which are regarded as severe toxicities~\cite{goyal2022your}, were found among $\approx$22\% of the samples.
We also noticed over two-thirds of our samples $\approx$72\% belonging to multiple forms.  For example, \textcolor{brown}{“:laughing: Holy shit, you are fucking stupid. It is an extremely simple proc with a switch. Seriously, did you even look at the code? Next, you'll tell me all switches are copypaste.}” represents both profanity and insult. 
While toxicity on social media has higher occurrences of flirtation and obscenity~\cite{gunasekara2018review,goyal2022your}, we found $\approx$2\% such cases in our sample.

%\vspace{10pt}

\begin{boxedtext}
\textbf{Key finding 1:} \emph{Profanity is the dominant form of toxicity in GitHub PRs. Severe toxicities, such as insults, identity attacks, and threats, represent $\approx$ 22\% cases. Unlike other online mediums, flirtation and obscenity were less common in our sample. }
\end{boxedtext}

%\vspace{5pt}

\subsection{RQ2: Project characteristics}
\label{sec:rq2}
\begin{table}

    \caption{Results of our MLR model to identify associations of project characteristics with toxicity. We set the `Low toxic' group as the reference to compute odds ratios. Hence, $OR>1$ indicates a higher likelihood of a project transitioning to the `Medium' or `High'  toxic group with an increment of that factor and vice versa. }
    \label{tab:toxicity_projects}
    \centering
%\resizebox{\textwidth}{!}{    

\begin{tabular}{l|r|l|r|l}
\hline
\multicolumn{1}{c|}{\multirow{2}{*}{\textbf{Attribute}}} & \multicolumn{2}{c|}{\textbf{Medium toxic}} & \multicolumn{2}{c}{\textbf{High toxic}} \\ \cline{2-5} 
\multicolumn{1}{c|}{}   & \multicolumn{1}{c|}{\textbf{OR}}    & \multicolumn{1}{c|}{\textbf{$p$}}   & \multicolumn{1}{c|}{\textbf{OR}}                                 & \multicolumn{1}{c}{\textbf{$p$}}   \\ \hline

isCorporate  & 0.888 & \textbf{0.000}$^{***}$ & 0.47 & \textbf{0.000}$^{***}$ \\  \hline
member count   & 0.999 & 0.821 & 1.0001 & 0.754  \\  \hline
stars & 1.001 & \textbf{0.000}$^{***}$ & 1.001 & \textbf{0.000}$^{***}$ \\  \hline
issues/month & 1.001 & 0.234 & 0.998 & 0.061 \\  \hline
project age &  1.004 & \textbf{0.000}$^{***}$ & 1.009 & \textbf{0.000}$^{***}$ \\  \hline
commits/month & 1.0001 & 0.316 & 1.0001 & 0.447 \\  \hline
release/month   & 1.003 & 0.310 & 0.998 & 0.813  \\  \hline
bug resolution   & 0.204 & \textbf{0.000}$^{***}$ & 0.985 & \textbf{0.000}$^{***}$ \\  \hline

isGame   &  0.486 & \textbf{0.000 }$^{***}$ &  7.259 & \textbf{0.000}$^{***}$ \\  \hline

\multicolumn{5}{l}{*** , **, and *  represent statistical significance at $p <$ 0.001, }\\

\multicolumn{5}{l}{ $p <$ 0.01, and $p <$ 0.05 respectively.}

\end{tabular}
%}

\end{table}

During RQ2's model training
two factors (i.e., forks and pulls per month) were dropped due to multicollinearity and were not included in our MLR. We estimated the fit of our MLR with NagelKerke $R^2$ =0.224. Our Log likelihood test results suggest that this model significantly differs ($\chi^2$ =545.16, $p<0.001$) from a null model. 
In addition to modeling the probability of a specific result based on a group of independent variables, MLR also enables the assessment of the probability of transitioning to a different dependent category from the current one when a specific independent variable changes~\cite{bayaga2010multinomial}. Hence, we set the `Low toxic' projects as the reference group in MLR and compute the odds of a project moving to the `Medium toxic' or `High toxic' group if one of the independents changes by a unit.  
Table~\ref{tab:toxicity_projects} shows the result of our MLR model, with OR values for each factor. 
Our results suggest that projects with corporate sponsorship ($isCorporate$) are significantly less likely to belong to the `Medium toxic' or `High toxic' groups than the `Low toxic' group. HR rules, professional codes of conduct, and the potential for job loss may be the reasons. 
We also noticed a significantly higher level of toxicity among the popular projects ( i.e., stars). 
We found that the prevalence of toxicity significantly increased with project age. 
{Moreover, our analysis} found no significant association between toxicity and development activities (i.e., commits/month and releases/month) and project quality (i.e., issues/month). On the other hand, issue resolution rates (i.e., percentage of resolved issues) significantly reduce toxicity, as projects with higher rates are more likely to belong to the `Low toxic' group than the `Medium toxic' or `High toxic' group.
Supporting observations from prior studies~\cite{miller2022did}, we also noticed the significantly higher prevalence of toxicity among gaming projects, as a gaming project is seven times more likely to belong to the `High toxic' group than the `Low toxic' or `Medium toxic' group.




%\vspace{10pt}
\begin{boxedtext}
\textbf{Key finding  2:} \emph{ While popularity and staleness are positively associated with the prevalence of toxicity, issue resolution rate has the opposite association.
While corporate-sponsored projects are likelier to be 'low toxic,' gaming projects are likelier to belong to the opposite spectrum. 
}
\end{boxedtext}

%\vspace{5pt}




\subsection{RQ3: Pull request context}
\label{sec:rq3}
\begin{table}
    \caption{{Associations between pull request contexts and toxicity. Values represent the median odds ratio for each factor with 95\% confidence intervals inside brackets.}}
    \label{tab:toxicity_context}
    \centering
   
    \resizebox{.9\textwidth}{!}{    

    \begin{tabular}{|p{3 cm}|R{3 cm}|R{3 cm}|R{3 cm}|} \hline
    
     \textbf{Attribute}     & \multicolumn{1}{c|}{\textbf{PRF-L}} & \multicolumn{1}{c|}{\textbf{PRF-M}} & \multicolumn{1}{c|}{\textbf{PRF-H}} \\ 
     \hline
     
     %& OR with 95\%CI  &OR with 95\%CI  &OR with 95\%CI\\ \hline
     %Null Deviance & 23,942 &1,46,413  &17,65,331  \\  \hline

%Residual Deviance& 23,428* &1,46,003**  & 16,94,292*** \\  \hline \hline

isBugFix & 1.01[0.99, 1.05] & 1.02 [1.01, 1.03] & 1.06*** [1.06, 1.07]  \\  \hline

commit count & 1.00 [1.00, 1.01] & 0.99* [0.99, 0.99] & 0.99*** [0.99, 1] \\  \hline

code churn (log) &  1.11 [1.10, 1.12] & 1.13*** [1.13, 1.14] &  1.11*** [1.11, 1.11]\\  \hline

 review interval& 1.11*** [1.11, 1.12] & 1.17*** [1.17, 1.17] & 1.20***[ 1.20, 1.20]  \\  \hline
 
review comments  & 1.06*** [1.05, 1.07] & 1.06*** [1.05, 1.06] & 1.04*** [1.04, 1.04] \\  \hline

isAccepted & 0.77*** [0.75, 0.80] & 0.71*** [ 0.70, 0.73] & 0.93*** [0.93, 0.94] \\  \hline

num iter & 1.01 [0.99, 1.03] & 1.01*** [1.01, 1.02] & 1.01*** [1.01, 1.01] \\  \hline

change entropy (log) & 1.57 [1.51, 1.67] & 1.35*** [1.32, 1.37] & 1.26*** [1.25, 1.27]  \\  \hline

\multicolumn{4}{p{14cm}}{ *** , **, and *  represent statistical significance at $p <$ 0.001, $p <$ 0.01, and $p <$ 0.05 respectively.}




%AIC score & 23,842 & 1,46,017  & 16,94,306 \\  \hline


    \end{tabular}
}
   %\vspace{-10pt}
\end{table}

One of the nine pull request context factors (i.e., the number of changed files) was dropped due to multicollinearity. Table~\ref{tab:toxicity_context} shows median odds ratios with 95\% confidence intervals for the remaining eight factors based on our bootstrapped logistic regression models repeated over 100 times. All eight factors show significant associations for the PRF-H group (i.e., PR/month $ >32$). For this group, bug fix PRs, code churn, review interval, the number of review comments, the number of review iterations, and change entropy are positively associated with toxicity. On the other hand, the number of commits and acceptance decisions are negatively associated. Similarly, we noticed almost identical associations for the PRF-M group (i.e., 8 $<$ PR/month $ <32$), except $isBugFix$ does not have a statistically significant association. For the PRF-L group (i.e.,  PR/month $ <8$), only three factors have statistically significant associations with toxicity, where the review interval and the number of review comments have positive ones. In contrast, $isAccepted$ has a negative one. 



A positive correlation between toxicity and isBugFix for the PRF-H group indicates that discussions on approach to fix pending issues might become heated for highly active projects, which support prior studies on locked issues~\cite{ferreira2022heated,miller2022did}. However, such a trend is not seen among projects belonging to PRF-M and PRF-L. While the number of commits included in a PR is negatively associated with toxicity for both PRF-M and PRF-H groups, our results suggest contradicting associations between toxicity and commit size measured using code churn. Since we code churn and the number of commits included in a PR are positively correlated, we were surprised by this finding. {Our in-depth investigation suggests that PRs with large numbers of commits are often due to inter-branch clean-up or feature imports.}
Hence, such PRs are less likely to have discussions~\cite{thongtanunam2017review} and, therefore, are less likely to be toxic. Positive correlations between toxicity and review interval across all project groups suggest that delayed decisions frustrate the participants and, hence, are more likely to instigate toxicity. 
Similarly, the number of review comments for a PR, an indicator of issues identified by reviewers, is positively associated with toxicity. This result indicates that PRs with poor-quality code are more likely to be associated with toxicity.
Unsurprisingly, we found a negative correlation between accepted PRs and toxicity among all groups, which indicates that rejected PRs are significantly more likely to be associated with toxicity than accepted ones.
The number of iterations, which indicates the number of times an author must add additional commits based on reviewers' suggestions, is positively associated with toxicity for projects belonging to the PRF-M and PRF-H groups. 
Finally, change entropy, a proxy for change complexity, is positively associated with toxicity among projects belonging to PRF-M and PRF-H groups. 


%\vspace{12pt}
\begin{boxedtext}
\textbf{Key finding  3:} \emph{Accepted PRs are less likely to encounter toxicity. On the contrary,  code churn, review intervals, the number of review comments,  change entropy,  and the number of review iterations are positively associated with toxicity on GitHub. }
\end{boxedtext}



%\vspace{6pt}

\subsection{RQ4: Participants}
\label{sec:res-rq4}
\begin{table*}
    \caption{{Associations between characteristics of authors toxicity. Values represent the median odds ratio for each factor with 95\% confidence intervals inside brackets.}}
    \label{tab:toxicity_authors}
    \centering
    \resizebox{.9\textwidth}{!}{    

    %\begin{tabular}{|l|r|r|r|} \hline

   \begin{tabular}{|p{3 cm}|R{3 cm}|R{3 cm}|R{3 cm}|} \hline
    
     \textbf{Attribute}     & \multicolumn{1}{c|}{\textbf{PRF-L}} & \multicolumn{1}{c|}{\textbf{PRF-M}} & \multicolumn{1}{c|}{\textbf{PRF-H}} \\ 
     \hline
     
     


isWoman & 0.80** [ 0.71, 0.793] & 1 [0.96, 1.01] & 0.90*** [0.86, 0.87] \\  \hline

isNewComer & 1.09 [ 1.05, 1.14] & 0.88*** [0.86, 0.89] & 0.69*** [0.68, 0.69] 
 \\  \hline

isMember  & 0.89** [0.87, 0.92]  & 0.77*** [0.77, 0.88] & 0.64***  [0.64, 0.64] \\  \hline

 GitHub tenure&   0.99*** [ 0.99, 0.99] & 0.99*** [0.99, 0.99] & 0.99*** [0.99, 0.99]  \\  \hline
 
project tenure  & 1.01** [1.01, 1.01]   & 1.01 [0.99, 1.01] & 1.01***  [1.01, 1.01] \\  \hline

toxicity/month  & 1.06*** [1.05, 1.07] & 1.02*** [1.02, 1.02] & 1.01*** [1.01, 1.01]  
 \\  \hline






\multicolumn{4}{l}{ *** , **, and *  represent statistical significance at $p <$ 0.001, $p <$ 0.01, and $p <$ 0.05 respectively.}



%AIC score& 33,099 & 2,05,611  & 23,13,144 \\  \hline


    \end{tabular}
}
\end{table*}


%\vspace{3pt}



  
\begin{table*}
    \caption{{Associations between characteristics of targets and toxicity. Values represent the median odds ratio for each factor with 95\% confidence intervals inside brackets.}}
    \label{tab:toxicity_targets}
    \centering
\resizebox{.9\textwidth}{!}{    

    %\begin{tabular}{|l|r|r|r|} \hline

   \begin{tabular}{|p{3 cm}|R{3 cm}|R{3 cm}|R{3 cm}|} \hline
    
     \textbf{Attribute}     & \multicolumn{1}{c|}{\textbf{PRF-L}} & \multicolumn{1}{c|}{\textbf{PRF-M}} & \multicolumn{1}{c|}{\textbf{PRF-H}} \\ 
     \hline
     
     

     
    % Null Deviance & 33,236 &2,06,883  &23,93,070  \\  \hline

%Residual Deviance& 33,005** &2,05,597**  & 23,13,130** \\  \hline \hline
isWoman & 0.48*** [ 0.46, 0.50] & 0.95 [ 0.93, 0.98] & 0.86*** [ 0.85, 0.87] \\  \hline

isNewComer & 0.91 [0.88, 0.95] & 0.90***[0.89, 0.92] & 0.74*** [0.74, 0.75] \\  \hline

isMember  & 1.12* [1.09, 1.15]  & 1.04 [1.02, 1.05] & 0.84*** [0.84, 0.84] \\  \hline

 GitHub tenure & 1.01 [0.99, 1.01] & 1.01*** [1.01, 1.01] & 1.01 [1.01, 1.01]  \\  \hline
 
project tenure & 0.99* [0.99, 0.99] & 1.01***[ 1.01, 1.01] & 1.01*** [1.01, 1.01]\\  \hline

toxicity/month & 1.99*** [ 1.81, 2.19] & 1.46*** [1.36, 1.52] & 1.26*** [1.25, 1.26]  \\  \hline

\multicolumn{4}{l}{ *** , **, and *  represent statistical significance at $p <$ 0.001, $p <$ 0.01, and $p <$ 0.05 respectively.}



%AIC score& 33,099 & 2,05,611  & 23,13,144 \\  \hline


    \end{tabular}
}
\end{table*}




We train two types of regression models, one to investigate the characteristics of persons authoring toxic comments and the other with the targets. Similar to the RQ3, we train bootstrapped logistic regression models repeated over 100 times. Tables~\ref{tab:toxicity_authors} and ~\ref{tab:toxicity_targets} show the odds ratios of authors and targets for each factor with 95\% confidence intervals for the three PRF-based project groups. The results of Log-likelihood ratio tests ($lrtest$) suggest that these models are significantly better than Null models and, therefore, are suitable to provide inferential insights. However, these models have low $R^2$ (i.e., low explainability power). This result indicates that the characteristics of participants, i.e., the attributes used to fit these regression models, have a very low explanatory power for toxic occurrences. Regardless, our models indicate several participant characteristics having significant associations, which we detail in the following.




%\vspace{10pt}
\begin{boxedtext}
\textbf{Key finding  4:} \emph{
Although occurrences of toxic comments are significantly associated with several participant characteristics, these have low explanatory power for toxicity in OSS PR contexts.}
\end{boxedtext}

\vspace{10pt}

\noindent \textbf{Characteristics of authors of toxic comments:}
Our results suggest significantly lower odds of women authoring toxic comments among PRF-L and PRF-H groups. Similarly, newcomers have lower authoring odds among PRF-M and PRF-H groups.
Among all three groups, project members are significantly less likely to author toxic comments, and the likelihood of being such an author significantly decreases with GitHub tenure.
The likelihood of authoring toxic comments significantly increased with project tenure among PRF-L and PRF-H groups. 
Our results support ~\citet{miller2022did}'s observation that there are many repeat offenders since the likelihood of authoring toxic comments significantly increases with the prior frequency of such occurrences.





\vspace{4pt}
\noindent \textbf{Characteristics of targets of toxic comments:}
Contrary to our expectations, formed based on results~\cite{gunawardena2022destructive,raman2020stress,steinmacher2015social}, we did not find any significantly higher odds of women or newcomers being targets of toxicity on GitHub. We noticed the opposite among PRF-H and PRF-L. Similarly, newcomers have significantly lower odds of becoming targets among PRF-H and PRF-M.
Being a project member significantly increases the odds of being a target among PRF-L and PRF-M but reduces among PRF-H.
Project tenure increases the odds of being a target for PRF-M and PRF-H but reduces among PRF-L.
The age of a GitHub account is positively associated with being a target only for PRF-M. 
Finally, these results suggest a `quid pro quo,' i.e., prior frequent authoring of toxic comments significantly increases the odds of becoming a target. 


%\vspace{10pt}
 

\begin{boxedtext}
\textbf{Key finding  5:} \emph{Women and newcomers are less likely to be either authors or targets of toxic comments in GitHub PR comments. Developers who have authored toxic comments frequently in the past are significantly more likely to repeat and more likely to become toxicity targets. }
\end{boxedtext}

%\vspace{10pt}