{\small \textcolor{red}{ Warning: This paper contains examples of language that some people may find offensive or upsetting.} }

\section{Introduction}
\label{sec:intro}
In 2023, GitHub, the most popular Open Source Software (OSS) project hosting platform, hosted more than 284 million public repositories~\cite{octoverse-2023}.  As OSS communities continue to grow, so do the interactions among contributors during various software development activities, such as issue discussions and pull request reviews. 
While these interactions are crucial to facilitating collaborations among the contributors, a few may take negative turns and cause harm~\cite{GitHub_GitHub_Open_Source}. 
Recent studies have investigated the umbrella of antisocial interactions among OSS developers using various lenses, which include  `toxicity,'~\cite{miller2022did, sarker2022automated,sarker2020benchmark,raman2020stress,sarker-esem-2023}, `incivility'~\cite{ferreira2021shut}, `destructive criticism'~\cite{gunawardena2022destructive}, and `sexism and misogyny'~\cite{sultana2021rubric}. Although these lenses differ, they all share a common attribute: the potential to cause severe repercussions among the participants. Consequences of antisocial interactions include stress and burnout~\cite{raman2020stress}, negative feelings~\cite{egelman2020predicting,ferreira2021shut,gunawardena2022destructive}, pushbacks~\cite{murphy2022pushback}, turnovers of long-term contributors~\cite{toxic-blog-linux2, toxic-open-source-maintainer, perl-toxic-2,leaving-for-toxicity}, adding barriers to newcomers' onboarding~\cite{raman2020stress}, and hurting diversity, equity, and inclusion (DEI) by disproportionately affecting women and other underrepresented minorities~\cite{gunawardena2022destructive,albusays2021diversity,nafus2012patches,murphy2022pushback}. Moreover, the prevalence of antisocial behaviors present substantial challenges to the growth and sustainability of an OSS project.




Therefore, recent Software Engineering (SE) research has focused on characterizing {toxicity and other} antisocial behaviors and their consequences through surveys, interviews, and qualitative analyses~\cite{ferreira2021shut,miller2022did,egelman2020predicting,gunawardena2022destructive,ferreira2022heated}. 
In a sample of 100 GitHub issue comments, ~\citet{miller2022did} found entitlement, arrogance, insult, and trolling as the most common forms of toxicity. Destructive criticism is another anti-social behavior found in code reviews~\cite{gunawardena2022destructive}. Although destructive criticisms are rare, they may have severe repercussions, which include conflicts, demotivation, and even hindering the participation of minorities~\cite{gunawardena2022destructive, egelman2020predicting}. 
~\citet{ferreira2021shut}'s study of rejected patches in Linux kernel mailing lists reported frustration, name-calling, and impatience as the most prevalent forms of incivility. Another recent workplace investigation reported inappropriate communication style as the primary cause of incivility~\cite{rahman2024words}.  
However, many of these studies suffer from limitations such as small sample sizes or narrow focuses on specific projects~\cite{ferreira2022heated,miller2022did},  organizations~\cite{qiu2022detecting,egelman2020predicting}, or a small developer group~\cite{rahman2024words}, which raises questions regarding the external validity of these findings at different contexts.  We also lack a quantitative empirical investigation of how various measurable characteristics of project contexts and participants are associated with the prevalence of antisocial behavior, such as toxicity.  
Such an investigation is necessary to formulate context-aware toxicity mitigation strategies for the broader OSS ecosystem. 


In response to this need, we have conducted a large-scale empirical investigation of toxicity during  Pull Requests (PRs). 
We selected PR since it is a crucial mechanism to attract contributions from non-members and facilitate newcomers' onboarding among OSS projects~\cite{gousios2014exploratory}. PRs allow contributors to propose changes, which other community members then review PRs. Due to the interpersonal nature of PR interactions and the potential for dissatisfaction due to unfavorable decisions, PR interactions may raise conflicts and anti-social behaviors. We select the `toxicity' lens since it has been most widely investigated~\cite{raman2020stress,miller2022did,sarker-esem-2023,sarker2020benchmark} and has a reliable automated identification tool~\cite{sarker2022automated}, which is a prerequisite for a large-scale empirical investigation. 
 Following the toxicity investigation framework proposed by \citet{miller2022did}, we investigate four research questions to characterize: i) the nature of toxicity, ii) projects that had higher toxic communication than others, iii) contextual factors that are more likely to be associated with toxicity, and iv) the participants of toxic interactions, respectively. We briefly motivate each of the research questions as follows.
 

\vspace{2pt}
\noindent \textbf{RQ1:  [Nature]} \textit{What are the common forms of toxicity observed during GitHub Pull Requests?}

\textit{Motivation:} 
Understanding the nature of toxicity may help project maintainers improve guidelines and interventions to foster respectful and constructive interactions. Prior studies have proposed various categorizations of SE domain-specific toxicity~\cite{miller2022did, sarker2022automated} and incivility~\cite{ferreira2021shut}. However, due to sampling criteria used in those studies (e.g., only locked issues, small samples, or a specific project), two key insights remain under-explored: i) whether these studies missed additional forms of toxicity, and ii) how frequently various toxicity categories occur on GitHub. RQ1 aims to fill in these knowledge gaps.


\vspace{2pt}
\noindent  \textbf{RQ2: [Projects]} \textit{What are the characteristics of the project that are more likely to encounter toxicity?}

\textit{Motivation:}  Does toxicity vary across project sponsorship, age,  popularity, quality, domain, or community size?  The identification of these factors will help project management undertake context-specific mitigation strategies. Determining which projects are more likely to suffer from toxicity may allow a project's management to allocate resources and define mitigation strategies.  Moreover, this insight will help a prospective joiner select projects.
 


\vspace{2pt}
\noindent \textbf{RQ3: [PR Context]} \textit{Which pull requests are more likely to be associated with toxicity on GitHub?}

\textit{Motivation:}
Does toxicity occur during poor-quality changes, unfavorable decisions, large changes, or delayed decisions?
Understanding contextual factors is crucial to educating developers to avoid creating specific scenarios. 



\vspace{2pt}
\noindent \textbf{RQ4: [Participants] } \textit{What are the characteristics of participants associated with toxic comments?}


\textit{Motivation:}
Prior studies~\cite{miller2022did,ferreira2021shut} suggested that some participants are likelier to author toxic comments due to their communication style or cultural background. On the other hand, another study suggests that participants representing underrepresented groups or newcomers are more likely to be targets~\cite{murphy2022pushback}. RQ4 aims to identify personal characteristics associated with being authors or victims of toxicity. This insight will help project management prepare community guidelines to combat toxicity and protect vulnerable participants.


\vspace{2pt}
\noindent \textbf{Research method:} We conducted a large-scale mixed-method empirical study of 2,828 GitHub-based OSS projects randomly selected based on a stratified sampling strategy. Our sample includes 16 million PRs and 101.5 million PR comments.  Using ToxiCR~\cite{sarker2022automated}, a state-of-the-art SE domain-specific toxicity detector, we automatically classified each comment as toxic or non-toxic. Additionally, we manually analyzed a random sample of 600 comments to validate ToxiCR's performance and gain insights into the nature of toxicity within our dataset. With ToxiCR demonstrating a reliable performance, we trained multivariate regression models to explore the associations between toxicity and various attributes of projects, PR contexts, and participants. 


\vspace{2pt}
\noindent \textbf{Key findings:}
We found 11 forms of toxic comments among GitHub PR reviews, with object-directed toxicity being a new form unreported in prior studies. 
The results of our study suggest that profanity is the most frequent toxicity on GitHub, followed by trolling and insults. 
While a project's popularity is positively associated with the prevalence of toxicity, its issue resolution rate has the opposite association.
Corporate-sponsored projects are less toxic, but gaming projects are seven times more toxic than non-gaming ones. OSS contributors who have authored toxic comments in the past are significantly more likely to repeat such behavior. Moreover, such individuals are more likely to become targets of toxic texts.



\vspace{4pt}
\noindent \textbf{Contributions} The primary contributions of this study include:
\begin{itemize}
\item  An empirical investigation of various categories of toxic communication among GitHub {PRs}.
\item  A large-scale empirical investigation of factors associated with toxicity on GitHub.
\item Actionable recommendations to mitigate toxicity among OSS projects. 

\item  We publicly make our dataset and scripts available at: \hyperlink{https://doi.org/10.5281/zenodo.14802294}{10.5281/zenodo.14802294}~\cite{dataset}.
 
\end{itemize}



\vspace{2pt}
\noindent \textbf{Organization:} The remainder of the paper is organized as follows.
Section~\ref{sec:background} briefly overview closely related works.
Section~\ref{sec:research-method} details our research methodology.  
 Section~\ref{sec:results} presents the results of our empirical investigation.  
 Section~\ref{sec:discussion} and Section~\ref{sec:threats} discuss implications and threats to the validity of our findings, respectively. 
 Finally, Section~\ref{sec:conclusion} concludes the paper.