\resizebox{\linewidth}{!}{    
     \begin{tabular}{|p{1.9cm}|p{5.5cm}|p{11.2cm}|} \hline
    \textbf{Variable}     & \textbf{Definition} & \textbf{Rationale}  \\ \hline
    \rowcolor[gray]{.8}
\multicolumn{3}{l}{ \textbf{RQ2: Project Characteristics}}\\ \hline

PR /month* & Average number of PRs per month.  &  Indicates the volume of development activity. Active projects may have a higher probability of toxic interactions~\cite{miller2022did}.  \\  \hline

issues /month & Average number of issues per month. & A higher number of bugs indicates the lack of quality, which may cause frustration among users and developers~\cite{miller2022did,raman2020stress}.    \\  \hline

commits /month & Average number of commits per month. & Commit is another indication of the volume of development activity. High activity may cause burnouts~\cite{raman2020stress}, and lack may cause frustration among users~\cite{miller2022did}.    \\  \hline

release /month & Average number of releases per month. & Frequent releases may satisfy the customers to decrease toxicity, and vice versa~\cite{costa2018impact}.  \\  \hline

issue resolution rate & Percentage of issues resolved. & Users may become frustrated due to issues affecting them not being resolved~\cite{miller2022did}.   \\  \hline



isCorporate*  & Whether the project is sponsored by a corporation. & Corporate projects may have less toxicity than non-corporate ones due to the consequences of HR policy violations~\cite{raman2020stress}.   \\  \hline


project age* & %Number of months%
Total months since a project's creation. & Older projects showed more toxicity~\cite{raman2020stress}.    \\  \hline


member count & Number of users with write access. & Toxicity increases with community size due to diverse views and higher potential conflicts~\cite{basirati2020understanding}.    \\  \hline


isGame* & Whether the project is gaming or not. & Prior studies have found prevalence of toxicity among gaming communities~\cite{miller2022did,belskie2023measuring,paul2018toxic,beres2021don}.    \\  \hline



stars & Number of stars on GitHub project. &  Popularity shows users' interests. Scrutiny and expectations increase with popularity and therefore stress on developers~\cite{raman2020stress}.   \\  \hline

forks & Number of forks on GitHub project. & Fork is another measure of project popularity~\cite{zhou2020has}. \\  \hline

\rowcolor[gray]{.8}
\multicolumn{3}{l}{ \textbf{RQ3: PR Context}}\\ \hline

commit count & Number of commits in a PR. & A large number of commits increases review effort~\cite{thongtanunam2017review}, which may frustrate reviewers, cause delays, and frustrate the author.  \\  \hline



number of changed files & The number of files changes in each PR. & A higher number of file changes requires a longer review time~\cite{review_2} and comprehension difficulty.
  
\\  \hline


  

code churn (log)* & The total number of rewritten or deleted code. & A high number of changed lines increases the probability of defects in the code~\cite{nagappan2005use,nagappan2007using} and may link to more toxic comments.  \\  \hline


isAccepted* & Whether the code review is accepted or rejected. & Developers used more toxic comments in rejected codes/patches~\cite{ferreira2021shut}.     \\  \hline

isBugFix & Whether the code review is for fixing a bug or not. & Issue discussions may instigate toxicity when the resolution is not liked by affected parties~\cite{ferreira2022heated,miller2022did}.    \\  \hline

change entropy (log) & A measure of change complexity, which estimates how much dispersed a changeset is among multiple files~\cite{thongtanunam2017review}. & Complexity of code change affects review time and participation~\cite{thongtanunam2017review}. Moreover, unnecessary complexity may be a sign of a poor quality change, which may receive harsh critique~\cite{rahman2024words}. \\  \hline

review interval & Time difference from the start of the code review to the end. & Delayed code reviews are more likely to cause frustration for developers~\cite{egelman2020predicting,turzo2023makes}.     \\  \hline

number of iterations (num iter)  & Total number of iterations (i.e., number of times changes requested) in a PR. &  Higher number of iterations frustrates both developers and reviewers due to additional time~\cite{turzo2023makes}. Higher iteration also indicates the lack of common understandings~\cite{ebert2019confusion} and potential disagreements~\cite{murphy2022pushback}.   \\  \hline

review comments* & The total number of review comments from reviewers in a PR.  & %Each review comment may indicate a change suggestion. 
A higher number of review comments indicate significant concerns from the reviewers over its quality, which often causes toxicity~\cite{rahman2024words}.\\  \hline




\rowcolor[gray]{.8}
\multicolumn{3}{l}{ \textbf{RQ4: Participants}}\\ \hline

isWoman& Whether the person is a woman & Prior studies have found women and marginalized minorities as frequent victims of toxicity~\cite{raman2020stress, gunawardena2022destructive}.
    \\  \hline

 

isMember* & Whether the person is a project member or not. & Project members are the authors of many toxic comments in replying to the outside members' query~\cite{cohen2021contextualizing, miller2022did}.  \\  \hline



 isNewComer* & Whether the person is a newcomer to the current project. & Newcomers may get frustrated due to delays~\cite{steinmacher2013newcomers} and unfavorable decisions~\cite{ferreira2021shut}. \\  \hline 

 GitHub tenure* & Age of GitHub account, in terms of the {total} months at the time of an event. & \cite{miller2022did} reported toxic comments from accounts with no prior activity on GitHub. \\  \hline  

  project tenure & Tenure with the current project in terms of the {total} months. & Although long-term members of a project are more committed to maintaining a professional environment in a community, Miller \textit{et} al. found toxic comments from them~\cite{miller2022did}. Moreover,  they may be targets if their decisions are not liked by issue reporters~\cite{ferreira2022heated}.   \\  \hline 

    toxicity / month* & The total number of toxic comments a user posts per month. & ~\cite{miller2022did} found many repeat offenders, as many OSS developers have toxic communication styles~\cite{miller2022did,leaving-for-toxicity}.   \\  \hline






    \end{tabular}
    }
    
%