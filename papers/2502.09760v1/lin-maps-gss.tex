
\subsection{Galerkin projection of linear maps on \gss}

We will be interested in the following question: given linear maps $V^n: H_n \to H_n$ (for example the Jacobian matrices for the flow defined by Galerkin projections), does there exists a limit $V:H \to H$, under some assumptions, which  are reasonable in the context of dissipative PDEs.  To make sense of this question we treat map  $V^n:H_n \to H_n$  as a linear map from $H$ to $H$ by setting
$V^n(u)= \iota_n (V^n(P_n u)) \in H_n \subset H$.


Observe that in view of Theorem~\ref{thm:lin-gss} we can represent in basis $\{e_j\}$ each continuous linear map $V:H \to H$ by its matrix $\{V_{ij}\}_{i,j \in \mathbb{Z}_+}$. 
We also set $V_{*j}=Ve_j$, which can be seen as $j$-th column of matrix $V_{ij}$ representing $V$.



\begin{theorem}
\label{thm:Vn-weak-lim}
 Let $V^n:H \to H$, $n\in\mathbb Z_+$ be linear maps such that $\|V^n\| \leq K$ for some $K\geq 0$. Assume that
 \begin{itemize}
 \item for all $i,j \in \mathbb{Z}_+$ there exists $V_{ij}$, such that
 \begin{equation}
   \lim_{n\to \infty} V^n_{ij} = V_{ij} \label{eq:dfg-Vij-conv0}
 \end{equation}
 \item there exists a family of compact sets $W^j\subset H$, $j\in\mathbb Z_+$ such that for all $n$ there holds
 \begin{equation}
 V^n_{\ast j} \in W^j. \label{eq:Vnj-approribnd}
 \end{equation}
\end{itemize}

Then there exists a bounded linear map $V:H\to H$, such that
\begin{enumerate}
\item  for all $a \in H$ there holds $\lim_{n \to \infty} V^n(a)=Va$,
\item $\|V\| \leq K$,
\item  $\pi_iV(e_j)= V_{ij} e_i, \quad \forall i,j\in \mathbb{Z}_+$ and
\item for each $i$ and all $a \in H$ there holds $\sum_{j=1}^\infty |V_{ij}| \cdot |a_j| < \infty$.
\end{enumerate}
\end{theorem}
\textbf{Proof:}

Let us fix $a \in H$. We will show that $V^n a$ is a Cauchy sequence. For any $n,k,m,M \in \mathbb{Z}_+$ we have
\begin{multline*}
  \|V^{n+k}a - V^n a\|\\\leq
  \|V^{n+k} (P_m a) - V^{n} (P_m a) \| + \|V^{n+k} (I-P_m) a\| +  \|V^{n} (I-P_m) a\| \\
   \leq \|P_M V^{n+k} (P_m a) - P_MV^{n} (P_m a) \| + \|(I-P_M)V^{n+k} (P_m a) \| \\
    +  \|(I-P_M)V^{n} (P_m a) \|
    + \|V^{n+k} (I-P_m) a\| +  \|V^{n} (I-P_m) a\| \\
    \leq \sum_{i=1}^M \|e_i\| \sum_{j\leq m} |V^{n+k}_{ij}-V^n_{ij}| \cdot |a_j|   + \|(I-P_M)V^{n+k} (P_m a) \| \\
    +  \|(I-P_M)V^{n} (P_m a) \| + 2 K \|(I-P_m)a\|.
\end{multline*}
Let us fix $\epsilon >0$. Then by taking $m$ big enough by Lemma~\ref{lem:gcp-prop} (condition (\ref{eq:remto0})) we obtain
that $2 K \|(I-P_m)a\| \leq \epsilon$. Let us fix such value of $m$.


Now we look at  $\|(I-P_M)V^{\ell} (P_m a) \|$ for any $\ell$. Observe that  $V^{\ell} (P_m a)$ is linear combination
of first $m$ columns in $V^{\ell}$. Indeed,
\begin{equation*}
  V^{\ell}P_m a = \sum_{j \leq m} V^{\ell}_{\ast j} a_j.
\end{equation*}
From this and assumption (\ref{eq:Vnj-approribnd}) it follows that for any $\ell$ there holds
\begin{eqnarray*}
  (I-P_M)  V^{\ell}P_m a  \subset (I-P_M) \left( \sum_{j\leq m} W^j a_j \right)
\end{eqnarray*}
The set $\sum_{j\leq m} W^j a_j$ is compact, hence from Lemma~\ref{lem:compt-gss} it follows there exists $M$, such that $\|(I-P_M) \left( \sum_{j\leq m} W^j a_j \right)\| \leq \epsilon/2$. We fix such value of $M$.

From condition (\ref{eq:dfg-Vij-conv0}) it follows that $ \sum_{i=1}^M \|e_i\| \sum_{j\leq m} |V^{n+k}_{ij}-V^n_{ij}| \cdot |a_j| \leq \epsilon$ for $n$ large enough and any $k$.

Summarizing, we obtained that $\|V^{n+k}a - V^n a\|\leq 3 \epsilon$ for $n$ large enough and therefore
$V^n a$ converges to some $Va$, which defines a linear operator $V:H \to H$ satisfying $\|V\| \leq K.$

Since $\pi_i V^n e_j=e_i V^n_{ij}$, we see that $V_{ij}=\lim_{n \to \infty} V^n_{ij}$.

Finally, from Lemma~\ref{lem:formOnH} applied
to $\pi_i V$ it follows that  $\sum_{j=1}^\infty |V_{ij}| \cdot |a_j| < \infty$.

\qed


\subsection{Block decomposition}
\label{subsec:lmgssblk-decmp}
Fix $M>1$ and let $J_1,\dots,J_m$ be a partition of the set $\{1,\dots,M\}$. We set $J_{m+1}=\{M+1,M+2,\dots\}$.
For $\ell=1,\dots,m+1$ we set $H_{<\ell>}=P_{J_\ell}H$. This defines a decomposition
\begin{equation}
 H=\bigoplus_{\ell \leq m+1} H_{\langle \ell \rangle}.   \label{eq:H-decmp}
 \end{equation}
For $x \in H$ we have $x=\sum_{\ell \leq m+1} x_{\langle \ell \rangle}$, where $x_{\langle \ell \rangle}=P_{J_{\ell}}x$.

For an operator $A \in \mbox{Lin}(H,H)$ we define its blocks according to the above decomposition by setting
\begin{equation*}
  \BD{A}{k_1}{k_2} x= P_{J_{k_1}}Ax, \quad x\in H_{\langle k_2\rangle}.
\end{equation*}
With this notation we have
\begin{equation*}
   (A x)_{\langle k \rangle} = \sum_{\ell=1}^{m+1} \BD{A}{k}{\ell} x_{\langle \ell \rangle}.
\end{equation*}


% To  distinguish between $V_{ij}$, when $i,j$ are treated as indices of particular coordinates and the situation
%when they are supposed to denote blocks  for the block entries involving $Y$ we will write $V_{yy}$, $V_{iy}$, $V_{yj}$.
% In this way $V_{ij}$ for any $i,j$ will denote always a number in the matrix $V$.



\begin{theorem}
\label{thm:gcpn-abstr}
Let $(H,\|\cdot\|)$ be \gss and fix a block decomposition of $H$, as in (\ref{eq:H-decmp}). Assume that
\begin{itemize}
\item $V^n:H_n \to H_n$, $n>M$ is a family of linear maps, such that for all $i,j \in \mathbb{Z}_+$ there exist a finite limit
$V_{ij}:=\lim_{n\to \infty} V^n_{ij}$, %  \label{eq:dfg-Vij-bl}
\item there exists a family $W^j\subset H$, $j\in\mathbb Z_+$ of compact sets, such that for $n>M$ there holds
 %\begin{equation*}
 $V^n_{\ast j} \in W^j$,% \label{eq:Vnj-apprbnds-bl}
% \end{equation*}
\item there exists $B \in \mathbb{R}^{(m+1) \times (m+1)}$, such that
      \begin{eqnarray}
       \|\BD{V^n}{k}{\ell}\| &\leq& B_{k\ell}, \quad k,\ell \leq m+1,\ n>M.  \label{eq:dfgVijn-estm}
      \end{eqnarray}
\end{itemize}
Then  there exists a bounded linear operator $V: H \to H$ such that
\begin{enumerate}
\item  for all $a \in H$ there holds $\lim_{n \to \infty} V^n(a)=Va$,
\item  for each $i$ and all $a \in H$ there holds $\sum_{j=1}^\infty |V_{ij}| \cdot |a_j| < \infty$,
\item  $\pi_i V(a) =  \left(\sum_{j=1}^\infty  V_{ij} a_j\right) e_i$ and
\item we have the following bounds
       \begin{eqnarray}
        \|\BD{V}{k}{\ell}\| &\leq& B_{k\ell}, \quad k,\ell \leq m+1.  \label{eq:dfgVij-estm}
      \end{eqnarray}
\end{enumerate}
\end{theorem}
\noindent
\textbf{Proof:}
From (\ref{eq:dfgVijn-estm}) and condition \NC4 we have
\begin{eqnarray*}
  \|V^n a\|  \leq \sum_{k,\ell=1}^{m+1} \|\BD{V^n}{k}{\ell}\| \cdot \|a_{\langle \ell \rangle}\| \leq  \left(\sum_{k,\ell=1}^{m+1} B_{k\ell}\right) \|a\|
\end{eqnarray*}
for $a \in H$. This proves that $\|V^n\|$ are uniformly bounded. From Theorem~\ref{thm:Vn-weak-lim} it follows that for each $a \in H$ the sequence $V^n a$ converges to $Va$, $V:H \to H$ is continuous and for each $i$ there holds  $\sum_{j=1}^\infty |V_{ij}| \cdot |a_j| < \infty$.

There remains to prove (\ref{eq:dfgVij-estm}). From (\ref{eq:dfgVijn-estm}) we have
$$
\|\BD{V}{k}{\ell}\| = \sup_{\|a_\ell\|=1}\|\BD{V}{k}{\ell}a_{\langle \ell \rangle}\|\leq \sup_{n>M}\sup_{\|a_\ell\|=1}\|\BD{V^n}{k}{\ell}a_\ell\|\leq B_{k\ell},
$$
which completes the proof.

\qed


