\section{Introduction}
%\label{sec:intro}

The main motivation for this research is to provide a theoretical framework for construction of $\mathcal C^1$-like algorithm for a class of dissipative PDEs. By a $\mathcal C^1$ algorithm we understand computation that provides guaranteed bounds on forward trajectories of underlying semi-flow and bound on the solutions to the associated variational equations.

The method of self-consistent bounds introduced in \cite{ZM}  lead to design  of rigorous integrators for a class of dissipative PDEs  \cite{C,WZ,ZKSper,ZKS3}, which  proved to be very efficient in studying dynamics of evolutionary dissipative PDEs \cite{BKZ,WZ,CW,CZ,Z,Z1,ZKSper,ZKS3}.

The aim of this paper is threefold. First, we revise the method of self-consistent bounds and reformulate it in an abstract setting. The statements are proved under very general assumptions about norms used, sets on which the existence of local semi-flow is proved and, the most important, assumptions on the vector field that guarantee uniform convergence of the solutions of finite-dimensional approximations (Galerkin projections) to the solutions of the original infinite-dimensional system. We will later refer to these assumptions by the letters \textbf{N} for norms, \textbf{S} for sets and \textbf{C} for convergence conditions.

The second aim is to formulate and prove results about convergence of the solutions to associated variational equations for Galerkin projections to the solutions of (properly understood) variational equations of infinite-dimensional systems. Here we will need additional convergence conditions for variational equations, we refer to as \textbf{VC}.

Finally we will show, that convergence conditions \textbf{C} and \textbf{VC} are not very restrictive, and they are naturally satisfied in a class of vector fields including widely studied models such as Kuramoto-Sivashinsky  \cite{KT,S}, Burgers \cite{B}, Brusselator (reaction-diffusion system) and Navier-Stokes \cite{T} equations. The crucial fact which make the proposed framework applicable is so-called \emph{isolation property} of the vector field, which informally means that the vector field is pointing inwards for all high modes leading to dissipative character of the system.

To illustrate the isolation property we consider a class of  PDEs of the following form
\begin{equation}
  u_t = L u + N\left(u,Du,\dots,D^ru\right), \label{eq:genpde-intro}
\end{equation}
where $u \in \mathbb{R}^n$,  $x \in \mathbb{T}^d=\left(\mathbb{R}\mod 2\pi\right)^d$, $L$ is a linear operator, $N$ is a
polynomial and by $D^s u$ we denote $s^{\text{th}}$ order derivative of
$u$, i.e. the collection of all spatial (i.e. with respect to variable $x$) partial derivatives of $u$ of
order $s$. %The reason to consider polynomial and not more general functions $N$ is technical --- we need to compute
%the Fourier coefficients of $N\left(u,Du,\dots,D^ru\right)$. This can be achieved by taking suitable convolutions of Fourier expansions of $u$ and its spatial partial derivatives.

We require, that the operator $L$ is diagonal in the Fourier basis
$\{e^{ikx}\}_{k \in \mathbb{Z}^d}$,
\begin{equation*}
  L e^{ikx}= -\lambda_k e^{ikx},
\end{equation*}
with
\begin{eqnarray*}
 L_* |k|^p &\leq& \lambda_k \leq  L^* |k|^p, \qquad \text{for all $|k| > K$ and  $K,L_*,L^* \geq 0$}, \\
    p &>& r.
\end{eqnarray*}
The assumption $p>r$ makes the equation dissipative,  we have a smoothing effect.

%As example let us consider the one-dimensional Kuramoto-Sivashinsky PDE \cite{KT,S}, which is given by
%\begin{equation}\label{eq:KS}
%u_t = -\nu u_{xxxx} - u_{xx} + (u^2)_x, \qquad \nu>0,
%\end{equation}
%where $x \in \mathbb{R}$, $u(t,x) \in \mathbb{R}$ and we impose
%odd and periodic boundary conditions
%\begin{equation}
%u(t,x)=-u(t,-x), \qquad u(t,x) = u(t,x+ 2\pi). \label{eq:KSbc}
%\end{equation}
%Here $p=4$ and $r=1$. Other examples are Navier-Stokes on the torus with $p=2$ and $r=1$.

If the solutions are sufficiently smooth, the problem (\ref{eq:genpde-intro}) can be written as an infinite ladder of ordinary differential equations for the Fourier
coefficients in $u(t,x)=\sum_{k \in \mathbb{Z}^d} u_k(t) e^{i kx}$, as follows
\begin{equation}
  \frac{d u_k}{dt} = F_k(u)=-\lambda_k u_k + N_k\left(\{u_j\}_{j \in \mathbb{Z}^d}\right), \qquad \mbox{for all
  $k \in \mathbb{Z}^d$}. \label{eq:fueq}
\end{equation}



The crucial fact, which makes our approach  possible is the \emph{isolation property}, which will be defined in Section~\ref{sec:isolation}.
Here, informally we express it as follows.

\emph{Let}
\begin{equation*}
   W=\left\{ \{u_k\}_{k \in \mathbb{Z}^d}\,:\,  |u_k| \leq \frac{C}{|k|^s q^{|k|}}\right\}
\end{equation*}
\emph{where $q\geq 1$, $C>0$, $s\geq 0$, satisfy some inequalities \\
Then under some assumptions there  exists $K>0$, such that for $|k| > K$ there holds}
\begin{equation*}
 \mbox{if} \quad u \in W, \ |u_k|=\frac{C}{|k|^s q^{|k|}}, \qquad \mbox{then} \qquad  u_k \cdot F_k(u) <0.
\end{equation*}

When $q=1$ and $s\geq p+d+1$ this is established in   Lemma~\ref{lem:iso-mainVar} in Section~\ref{subsec:iso-main-var} (see also Theorem 3.1 in \cite{ZKS3}). In  \cite{WZ} we used sets of the type described above with $s=0$ and $q>1$. % The isolation property for Kuramoto-Sivaskinsky pde on the line on such sets is established in \cite[Rem. 19]{WZ}.

Projection of the set $W$ onto $k^{\mathrm{th}}$ mode is a closed disc (or interval) centred at zero and of radius
$r_k =\frac{C}{|k|^s q^{|k|}}$. Geometrically the isolation property means that if $|u_k|=r_k$ for some $k> K$ then the $k$-th component of the vector field is pointing inwards the set. As a consequence, the only way a trajectory may leave the set $W$ forward in time is by increasing some of leading modes $|u_k|$, $k\leq K$ above the threshold $r_k$. However, one has to be careful with what "inwards" means in the above statement because $W$ has empty interior as a compact subset in infinite dimensional space. On the other side it makes perfect sense for Galerkin projections.


This property is used   in  our approach to obtain a priori bounds for $u_k(h)$ for small $h>0$ and $|k|>K$, while the leading modes $u_k$ for $|k| \leq K$ are computed using  tools for rigorous integration of ODEs \cite{CAPDREVIEW,Lo,NJP} based on the interval arithmetics \cite{Mo}. Moreover, from the point of view of topological method, the isolation property is of crucial importance as it shows that on the tail we have the entry behaviour, which enable us to apply the finite dimensional tools from the dynamics like
the Conley index \cite{ZM}, the  fixed point index and  covering relations \cite{ZKS3}.

In the present work we focus on the issues related to the convergence of Galerkin projections of equations (\ref{eq:fueq}) and its variational equations.

The paper is organized as follows. In Section~\ref{sec:gss} we introduce the notion of good sequence space (\gss space) and  establish some of their properties.  \gss will be later the spaces in which our Fourier series belong. In Section~\ref{sec:estmlinEq} we recall the notion of the logarithmic norm of a matrix and prove a useful result regarding block decomposition for nonautonomous linear ODEs. This is a crucial result for efficient implementation of an algorithm based on the framework introduced in Section~\ref{sec:c0-conver} and Section~\ref{sec:c1-conver}. There we state and prove the main results about the convergence of solutions of finite-dimensional Galerkin projections to solutions of underlying PDE for the main system (Theorem~\ref{thm:limitLN}) and for variational system (Theorem~\ref{thm:c1conver}), respectively. In Section~\ref{sec:isolation} we formally define the isolation property and we prove that it implies the existence of a-priori bounds, which appear in the assumptions the convergence theorems. Finally, in Section~\ref{sec:dissipativePDEs} we show that the introduced framework applies to a class of dissipative DPEs, that is the assumptions on sets \textbf{S}, convergence conditions \textbf{C}, \textbf{VC} and the isolation property are satisfied for various choices of Banach spaces.


\subsection*{Notation}


By $\mathbb{Z}_+$ we denote the set of all positive integers.
Given two normed vector spaces $V,W$ by $\text{Lin}(V,W)$ we will denote the space of all bounded linear maps from $V$ to $W$.





%Given a matrix $A \in \mathbb{R}^{n \times n}$ and a norm $\|\cdot\|$ by $\mu(A)$ we denote the logarithmic norm of $A$ induced by norm $\|\cdot\|$.
