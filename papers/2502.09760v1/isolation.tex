\section{Isolation property}
\label{sec:isolation}

In convergence theorems discussed in Section~\ref{sec:c0-conver} and Section~\ref{sec:c1-conver} the crucial assumption was the existence of uniform a-priori bounds $W$ and $W_V$. It turns out that such sets can be found in algorithmic way using the isolation property of the vector field discussed in Section~\ref{sec:dissipativePDEs}. The existence of such a-priori bounds will be later used in construction of an algorithm for rigorous integration of the flow and associated variational equations.

%In Section~\ref{sec:dissipativePDEs} we will prove that the isolation property is satisfied for a wide class of vector fields and thus the framework introduced in %Section~\ref{sec:c0-conver} and Section~\ref{sec:c1-conver} is applicable.

We adopt a notation used in Sections~\ref{sec:c0-conver} and Section~\ref{sec:c1-conver}. In particular we have some Galerkin filtration, which is used to define $\pi_k$. Thus, for $x \in H$,
$\pi_k x$ is the $k$-th component of $x$ and it is a vector of dimension $\# (J_k \setminus J_{k-1})$. 



In what follows for  $W \subset H$ we set $W_k=\pi_k (W)$. We define the isolation property in the following way.
\begin{definition}
	Let $W\subset H$ be a set satisfying condition {\bf S} in a \gss space $H$. We say that  vector field $F:H\to H$ satisfies the \emph{isolation property} on the set $W$ if $F$ satisfies condition \textbf{C} on $W$ and there exists $K_0\in\mathbb N$ such that
	\begin{eqnarray*}
		\forall_{k\geq K_0}\, \exists_{T_k>0}\, \forall_{t\in(0,T_k]}\forall_{u\in W}
		\left(  u_k\in\mathrm{bd} W_k\Longrightarrow u_k + tF_k(u) \in\mathrm{int}W_k\right).
	\end{eqnarray*}	
\end{definition}
Geometrically the above condition means that the vector field $F$ is pointing inwards each component $W_k$ on all far coordinates of $W$. In particular, each component $W_k$, $k\geq K_0$ has nonempty interior.
\begin{lemma}\label{lem:isolation-for-projections}
	If $F:H\to H$ satisfies  the isolation property on $W\subset H$ then there is $N_0$ such that for any Galerkin projection $n>N_0$ there holds
	\begin{eqnarray*}
	\forall_{N_0\leq k\leq n}\, \exists_{T_k>0}\, \forall_{t\in(0,T_k]}\forall_{u\in P_nW}
	\left(  u_k\in\mathrm{bd} W_k\Longrightarrow u_k + tF_k^n(u) \in\mathrm{int}W_k\right).
\end{eqnarray*}	
\end{lemma}
\textbf{Proof:}
From \textbf{S1} there is $M\geq 1$ such that $P_nW\subset W$ for $n\geq M$. Let $K_0$ be the constant from  the isolation property for $F$. Put $N_0=\max\{K_0,M\}$. Now, let us fix $n>N_0$, $N_0\leq k\leq n$ and $u\in P_nW$ such that $u_k\in\mathrm{bd}W_k$. Let $T_k$ be the same constant as in the isolation property for $F$ and index $k$.

Since $k\leq n$ and $u\in P_nW\subset W$ we have $F^n_k(u) =F_k(u)$ and thus
\begin{equation*}
	u_k+tF_k^n(u) = u_k+tF_k(u)\in \mathrm{int}W_k
\end{equation*}
for $t\in(0,T_k]$.
\qed

\begin{definition}
	We say that a vector field $F:H\to H$ is isolating on the family of sets $\mathcal W\subset 2^H$ if $F$ satisfies the isolation property on each set $W\in\mathcal W$.
\end{definition}
Notice, that the constant $K_0$ for each set in the family $\mathcal W$ can be different.

The next theorem addresses the question about the existence of a-priori bounds, as in Theorem~\ref{thm:limitLN}.

\begin{theorem}\label{thm:isolationGivesAPB}
	Assume the vector field $F:H\to H$ is isolating on the family of sets $\mathcal W$ satisfying the following properties:
	\begin{itemize}
		\item there exists $E\in \mathcal W$, such that $0\in\mathrm{int}E_k$ for all $k\in\mathbb Z_+$;
		\item $\mathcal W$ is closed with respect to the addition
		\begin{eqnarray*}
			W_1,W_2\in \mathcal W &\Longrightarrow&  W_1+W_2:=\{w_1+w_2: w_1\in W_1,\, w_2\in W_2 \}\in\mathcal W.
		\end{eqnarray*}
	\end{itemize}		
	Then for any $Z\in\mathcal W$ there exists $T>0$ such that the assumptions of Theorem~\ref{thm:limitLN} are satisfied for the set $W=Z+E$.
\end{theorem}
\textbf{Proof:}
We have to prove (\ref{eq:appriori-bnds}) with $W=Z+E$. From Lemma~\ref{lem:isolation-for-projections} there is $N_0$ such that for $n>N_0$ and $N_0\leq k\leq n$ the vector field $F^n$ is pointing inwards $W_k$ on the boundary of $W_k$. Thus, for $z\in P_nZ$ the only way a trajectory $z^n(t)$ of the $n$-th Galerkin projection may escape the set $P_nW$ is through one of the leading coordinate $z_k$, $k< N_0$.

Since $W$ is compact, $F$ is continuous on $W$ and $F^n$ converge uniformly to $F$ on $W$ we have
\begin{equation*}
	\sup_{n>N_0}\sup_{k< N_0}\sup_{u\in W}\|F_k^n(u)\| < \infty.
\end{equation*}
Hence, there is $T>0$ such that for all $n>N_0$ and all $k< N_0$ (finite number of coefficients) there holds
\begin{equation*}
	[0,T]F^n_k(P_nW)\subset \mathrm{int}E_k.
\end{equation*}
From this we get
\begin{equation*}
	Z_k+[0,T]F^n_k(P_nW)\subset \mathrm{int}W_k, \quad k<N_0,
\end{equation*}
which proves that for $z\in P_nZ$ the solution of the $n$-th Galerkin projection $z^n(t)$ exists for $t\in[0,T]$ and $z^n([0,T])\subset P_nW$.
\qed

In the applications we keep in mind, the family $\mathcal W$ may consists of sets with polynomial, exponential or mixed decay of far coefficients, which are clearly closed with respect to addition and the existence of $E$ with $0\in E_k$ is also satisfied.

In a similar way we can address the question regarding the existence of a-priori bounds for variational system needed  in Theorem~\ref{thm:c1conver}.
\begin{theorem}
	Assume the vector field $F:H\to H$ is isolating on the family of sets $\mathcal W$ satisfying the following properties:
	\begin{itemize}
		\item there exists $E\in \mathcal W$, such that $0\in\mathrm{int}E_k$ for all $k\in\mathbb Z_+$;
		\item $\mathcal W$ is closed with respect to the addition
		\begin{eqnarray*}
			W_1,W_2\in \mathcal W &\Longrightarrow&  W_1+W_2:=\{w_1+w_2: w_1\in W_1,\, w_2\in W_2 \}\in\mathcal W;
		\end{eqnarray*}
		\item for all $W\in\mathcal W$ the vector field $F_V$ is isolating on $W\times E$ and satisfies condition \VL
	\end{itemize}		
	Then for any $Z\in\mathcal W$ there exists $T>0$ such that the assumptions of Theorem~\ref{thm:c1conver} are satisfied with $W=Z+E$ and $W_{V_j}=C_jE$, for some $C_j>0$, $j\in\mathbb Z_+$.
\end{theorem}

\textbf{Proof:}
From Theorem~\ref{thm:isolationGivesAPB} there is $T>0$ and $N_0$ such that the assumptions of Theorem~\ref{thm:limitLN} are satisfied, that is (\ref{eq:appriori-bnds}) holds true with $W=Z+E$.

From Lemma~\ref{lem:isolation-for-projections} and since $F_V$ is isolating on $W\times E$ it follows that there is, possibly larger $N_1$ such that the vector field $F_V^n$ is pointing inwards $W\times E$ for all $n>N_1$ and $N_1<k\leq n$. Take $Z_V:=\frac{1}{2}E$. Reasoning as in the proof of Theorem~\ref{thm:isolationGivesAPB} and shrinking $T$ if necessary we obtain that for all $n>N_1$ and $t\in[0,T]$ there holds
\begin{eqnarray*}
	x^n(t)\in W&\text{for}& x^n(0)\in P_nZ,\\
	C^n(t)\in E&\text{for}& C^n(0)\in P_nZ_V.
\end{eqnarray*}
Let us fix $j\in \mathbb Z_+$. Since each component of $E$ is a convex set containing zero we can find a constant $C_j$ such that $e_j\in C_jZ_V$. Put $W_{V_j}=C_jE$. Now, due to linearity of the variational equation, for $n>N_1$ and $j\leq n$ we have
\begin{eqnarray*}
	C^n(t)\in W_{V_j}&\text{for}& C^n(0)\in P_n(C_jZ_V).
\end{eqnarray*}
In particular, if $C^n(0)=e_j$ then $C^n(t)\in W_{V_j}$ for $t\in[0,T]$ and thus (\ref{eq:var-a-priori-bnds}) is satisfied.

\qed 