
\subsection{Isolation property}

\textbf{PZ: byc moze dac to do wstepu}


To illustrate the \emph{isolation property}
let us consider a class of dissipative PDEs of the following form
\begin{equation}
  u_t = L u + N\left(u,Du,\dots,D^ru\right), \label{eq:genpde}
\end{equation}
where $u \in \mathbb{R}^n$,  $x \in \mathbb{T}^d=\left(\mathbb{R}\mod 2\pi\right)^d$, $L$ is a linear operator, $N$ is a
polynomial and by $D^s u$ we denote $s^{\text{th}}$ order derivative of
$u$, i.e. the collection of all spatial partial derivatives of $u$ of
order $s$. The reason to consider polynomial and not more general functions $N$ is technical --- we need to compute
the Fourier coefficients of $N\left(u,Du,\dots,D^ru\right)$. This can be achieved by taking suitable convolutions of Fourier expansions of $u$ and its spatial partial derivatives.

We require, that the operator $L$ is diagonal in the Fourier basis
$\{e^{ikx}\}_{k \in \mathbb{Z}^d}$,
\begin{equation*}
  L e^{ikx}= -\lambda_k e^{ikx},
\end{equation*}
with
\begin{eqnarray*}
 L_* |k|^p &\leq& \lambda_k \leq  L^* |k|^p, \qquad \text{for all $|k| > K$ and  $K,L_*,L^* \geq 0$}, \\
    p &>& r.
\end{eqnarray*}

If the solutions are sufficiently smooth, the problem (\ref{eq:genpde}) can be written as an infinite ladder of ordinary differential equations for the Fourier
coefficients in $u(t,x)=\sum_{k \in \mathbb{Z}^d} u_k(t) e^{i kx}$, as follows
\begin{equation}
  \frac{d u_k}{dt} = F_k(u)=-\lambda_k u_k + N_k\left(\{u_j\}_{j \in \mathbb{Z}^d}\right), \qquad \mbox{for all
  $k \in \mathbb{Z}^d$}. \label{eq:fueq}
\end{equation}



The crucial fact, which makes our approach to the rigorous integration of (\ref{eq:fueq})  possible is the \emph{isolation property}, which reads:

\emph{Let}
\begin{equation*}
   W=\left\{ \{u_k\}_{k \in \mathbb{Z}^d}\,|\,  |u_k| \leq \frac{C}{|k|^s q^{|k|}}\right\}
\end{equation*}
\emph{where $q\geq 1$, $C>0$, $s>0$. \\
Then there  exists $K>0$, such that for $|k| > K$ there holds}
\begin{equation*}
 \mbox{if} \quad u \in W, \ |u_k|=\frac{C}{|k|^s q^{|k|}}, \qquad \mbox{then} \qquad  u_k \cdot F_k(u) <0.
\end{equation*}

For the proof of  the above fact see Theorem 3.1 and its proof in \cite{ZKS3}, where the case with $q=1$ was treated.

Projection of the set $W$ onto $k^{\mathrm{th}}$ mode is a closed disc (or interval) centred at zero and of radius
$r_k =\frac{C}{|k|^s q^{|k|}}$. Geometrically the isolation property means that if $|u_k|=r_k$ for some $k> K$ then the $k$-th component of the vector field is pointing inwards the set. As a consequence, the only way a trajectory may leave the set $W$ forward in time is by increasing some of leading modes $|u_k|$, $k\leq K$ above the threshold $r_k$. However, one has to be careful with what "inwards" means in the above statement because $W$ has empty interior as a compact subset in infinite dimensional space.


This property is used   in  our approach to obtain a priori bounds for $u_k(h)$ for small $h>0$ and $|k|>K$, while the leading modes $u_k$ for $|k| \leq K$ are computed using  tools for rigorous integration of ODEs \cite{CAPDREVIEW,Lo,NJP} based on the interval arithmetics \cite{Mo}, i.e. to obtain set $W$ as in Theorem~\ref{thm:limitLN}. Moreover, from the point of view of topological method, the isolation property is of crucial importance as it shows that on the tail we have the entry behaviour, which enable us to apply the finite dimensional tools from the dynamics.


In this work we use sets of the type described above with $s=0$ and $q>1$. The isolation property for KS equation on such sets is established in \cite[Rem. 19]{WZ}.
