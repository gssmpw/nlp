\section{$C^0$-convergence}
\label{sec:c0-conver}

%The following lemma will be used later in the discussion of convergence of Galerkin projections for dissipative PDEs.
%\begin{lemma}
%\label{lem:C4'=>C4}
%Assume $H$ is \gss .
%Assume that $W \subset H$ is compact, there exists $M$ such that $P_n(W) \subset W$ for $n\geq M$, and $f:W \to H^*$ is continuous. Then there exists $L$, such that
%\begin{equation}
%  \|f(P_n z)_{|H_n} \| \leq L, \quad n \geq M, z \in W.
%\end{equation}
%\end{lemma}
%\textbf{Proof:}
%From compactness of $W$ and continuity of $f$ on $W$ it follows that there exists $L \in \mathbb{R}$, such that
%\begin{equation*}
%  \|f(z)\| \leq L, \quad  z\in W.
%\end{equation*}
%From the above and since $P_n(W) \subset W$  for all $n \geq M$ it follows that
%\begin{equation*}
% \|f(P_n z)_{|H_n}\| \leq \|f(P_n z)\| \leq L, \quad  z\in W, n\geq M.
%\end{equation*}
%This finishes the proof.
%\qed

The aim of this section is to state and prove theorem about uniform convergence of local flows induced by Galerkin projections to the local semi-flow for the original infinite-dimensional system. The assumptions, called later the $\mathcal C^0$-convergence conditions, will be about local properties of the vector field, only. We will assume, that the vector field satisfies certain condition on some sets (rough enclosures) satisfying the following geometric properties.


In the context of $H$ being \gss, we define a special set of Galerkin projections as follows.
Let $\emptyset\neq J_1  \subsetneq  J_2  \subsetneq J_3 \dots  $ be a family of finite sets, such that $\bigcup_{n \in \mathbb{Z}_+} J_n = \mathbb Z_+$.
For $n \in \mathbb{Z}_+$  by $H_n$ we denote a subspace spanned by
$\{e_j\}_{j \in J_n}$. Put $P_n:=P_{J_n}$ and $Q_n=\mathrm{Id}-P_n$. By $\iota_n:H_n \to H$ we denote the embedding $H_n$ into $H$ and we define $\pi_k x = P_k x - P_{k-1}x$, i.e.
$\pi_k$ is the projection onto space spanned by $\{e_j\}_{j \in J_k \setminus J_{k-1}}$.
Observe that we abuse a bit the notation here, because in Section~\ref{sec:gss} $P_n$, $H_n$, $\pi_n$ and $\iota_n$ had slightly different meaning, however the results proven there apply also in the present context, it is enough to replace $n$-th direction spanned by $e_n$ by a finite dimensional space $P_{J_n\setminus J_{n-1}}H$ which is spanned
by $\{e_j\}_{j \in J_n \setminus J_{n-1}}$. We will call family $\{J_n\}_{n \in \mathbb{Z}_+}$ a \emph{Galerkin filtration of $H$}.  The trivial Galerkin filtration of $H$ is defined by
$J_n=\{j , j \leq n\}$. 



\begin{definition}
	Assume that $(H,\|\cdot\|)$ is \gss space. We say that $W\subset H$ satisfies condition $\mathbf{S}$ if
	\begin{description}
		\item[S1:] $W$ is convex and there exists $M \geq 1$, such that $P_n(W) \subset W$
		for $n \geq M$,
		\item[S2:] $W$ is compact.
	\end{description}
\end{definition}



For a vector field $F:\dom(F)\subset H\to H$ we define its $n$-th Galerkin projection $F^n:H_n\to H_n$ by
\begin{equation}\label{eq:GalerkinODE}
u'=F^n(u) :=P_n(F(i_n(u))).
\end{equation}

%By $\varphi^n$ we will denote the local flow induced by $F^n$ on $H_n$. Such a function can be extended formally to a mapping on $H$ by setting $\varphi^n(t,x) = \varphi^n(t,P_nx)$. Clearly this is not a dynamical system on $H$ but this extension will be useful in the context of studying the uniform convergence of $\varphi^n$ -- then we need a common domain for these functions.

%We assume that $F_i$ is a formal power series in variables $a_k$ and symbols $\frac{\partial^k F_i}{\partial a_{i_1} \dots \partial a_{i_k}}$ are defined by formal differentiation of $F_i$.

\begin{definition}
%\label{def:Fadmisible}
  We say that $F$ is admissible, if for all $n\in\mathbb Z_+$  the function $F^n$ is defined on $H_n$ and it is $\mathcal{C}^3$ as a function on $H_n$.
\end{definition}




%Consider system (\ref{eq:feqLN}), where $F$ is admissible. The goal of this section is to formulate results regarding the convergence of solutions to Galerkin projections  to the solution of full system (\ref{eq:feqLN}) .

Now we are in the position to present our $\mathcal C^0$ convergence conditions for infinite-dimensional vector fields.
\begin{definition}
%\label{def:C1C2C3}
Let $(H,\|\cdot\|)$ be \gss,  $W\subset H$ and $F:\mathrm{dom}(F)\subset H \to H$. We say that $F$ satisfies condition \textbf{C} on $W$ if $F$ is admissible, $W$ satisfies condition \textbf{S} and
\begin{description}
\item[C1:] $W \subset \dom(F)$ and function $F|_W:W \to H$ is continuous;
\item[C2:] there exists  $l \in \mathbb{R}$ such that for all $n\in\mathbb Z_+$ there holds
\begin{equation*}
	\sup_{x \in P_n W}\mu\left(D F^n (x) \right) \leq l.
\end{equation*}
\end{description}
\end{definition}
The main idea behind condition \textbf{C2} is to ensure that the logarithmic norms (see Section~\ref{sec:estmlinEq}) for all Galerkin projections are uniformly bounded.

Observe that conditions \textbf{S} and \textbf{C1} imply that $F \circ P_n$ converges uniformly to $F$ on $W$.



The following theorem is a generalization of  Theorem 13 in \cite{Z}. There   it is assumed that  $W$ is a trapping region and  $H=l_2$ was used, but  the main idea of the proof is the same.
\begin{theorem}
\label{thm:limitLN}
 Let $(H,\|\cdot\|)$ be \gss and assume that $F:\mathrm{dom}(F)\subset H \to H$ satisfies condition \textbf{C} on $W\subset H$. Let $Z\subset W$ and $T>0$ be such that for all $n>M_1$ there holds
  \begin{equation}
    \varphi^n(t,x)\in W\quad \text{for } x\in P_n(Z),\ t\in[0,T], \label{eq:appriori-bnds}
  \end{equation}
  where $\varphi^n$ is a local dynamical system on $H_n$ induced by the $n$-th Galerkin projection (\ref{eq:GalerkinODE}). Then there exists a continuous function $\varphi\colon[0,T]\times Z\to W$ satisfying the following properties.
    \begin{description}
    \item[1.]{\bf Uniform convergence:} The functions $\widehat{\varphi}^n:=\iota_n\circ \varphi^n\circ P_n$ converge uniformly to $\varphi$ on $[0,T]\times Z$.
    \item[2.] {\bf Existence and uniqueness within $W$:} For all $x\in Z$ the function $u(t):=\varphi(t,x)$ is a unique solution to the initial value problem $u'=F(u)$, $u(0)=x$ and satisfying $u(t)\in W$ for $t\in[0,T]$.
    \item[3.]  {\bf Lipschitz constant:} For $x,y\in Z$ and $t\in[0,T]$ there holds
      \begin{equation*}
         \|\varphi(t,x) - \varphi(t,y) \| \leq e^{lt}\|x - y\|.% \label{eq:LipConstant}
      \end{equation*}
    \item[4.] {\bf Semidynamical system.} The partial map $\varphi:[0,T] \times W \to W$ defines a semidynamical system on $W$, that is
        \begin{itemize}
          \item $\varphi$ is continuous;
          \item $\varphi(0,u)=u$;
          \item $\varphi(t,\varphi(s,u)) = \varphi(t+s,u)$
        \end{itemize}
    provided $\varphi(t+s,u)$ exists.
  \end{description}
\end{theorem}
\begin{remark}
Observe that the essential difficulty in application of the above theorem is to find set $W$ satisfying (\ref{eq:appriori-bnds}) and the convergence conditions. A systematic way to construct it is based on the \emph{isolation property} (see Section~\ref{sec:isolation}). We will prove there that it is always possible to find a-priori bounds and make one time step of a rigorous integration algorithm.
\end{remark}
\noindent
\textbf{Proof of Theorem~\ref{thm:limitLN}:}

Let us fix $k \geq n$ and set $x^n=\varphi^n(\cdot,u)$, $x^k=\varphi^k(\cdot,v)$ for some $u\in P_nZ$ and $v\in P_kZ$. We start from estimation on the difference $x^n(t)-x^k(t)$ for $t \in [0,T]$. We have
\begin{eqnarray*}
  (P_n x^k(t))'&=&P_n F(x^k(t)) \\
  &=&P_nF^n(P_nx^k(t)) + \left(P_n F(x^k(t)) - P_n F(P_nx^k(t)) \right).
\end{eqnarray*}
Hence we can treat the function $y(t)=P_n x^k(t)$ as a solution to a perturbed equation $y'(t)=P_nF^n(y(t))+\delta(t)$, where $\delta$ is uniformly bounded by
\begin{eqnarray*}
  \|\delta(t)\|=\|P_n F(x^k(t)) - P_n F(P_nx^k(t))\| \leq \max_{y \in W} \|P_n F(y) - P_nF(P_ny)\| =:\delta_n.
\end{eqnarray*}
Obviously $\delta_n \to 0$ for $n \to \infty $, because $F \circ P_n$ converges uniformly to $F$ on $W$ -- this follows immediately from compactness of $W$ (condition \textbf{S2}) and \textbf{C1}.

From \textbf{C2} there is a uniform bound $l$ on all logarithmic norms of $DF^n(W)$, $n\in\mathbb Z_+$. Hence, by Lemma \ref{lem:estmLogN} we obtain
\begin{equation}
  \|x^n(t) - P_n(x^k(t))\| \leq e^{lt}\|x^n(0) - P_nx^k(0)\| + \delta_n \kappa_l(t), \quad t \in [0,T]  \label{eq:xn-Pn}
\end{equation}
where $\kappa_l$ is defined as in (\ref{eq:kappa}).

\textbf{Convergence.} For $x\in Z$, $t\in[0,T]$ and $k\geq n$ from (\ref{eq:xn-Pn}) with $u=P_nx$ and $v=P_kx$ we have
\begin{multline*}
 \|\widehat\varphi^n(t,x) - \widehat\varphi^k(t,x)\| \leq  \\
 \|\varphi^n(t,P_nx) - P_n(\varphi^k(t,P_kx))\| + \|(I-P_n) \varphi^k(t,P_kx)\| \leq \\
    \delta_n \kappa_l(t) + \|(I-P_n) x_k(t)\| \leq
     \delta_n \kappa_l(t) + \|(I-P_n)W\|.
\end{multline*}
From Lemma~\ref{lem:compt-gss} we get $\lim_{n\to\infty}\|(I-P_n)W\| = 0$. The function $\kappa_l$ is bounded on $[0,T]$ and given that $\lim_{n\to\infty}\delta_n=0$ we see that $\widehat\varphi^n$ is a Cauchy sequence in $\mathcal{C}([0,T]\times Z,H)$. Hence, it converges uniformly to a continuous function $\varphi:[0,T]\times Z\to W$.

\textbf{Existence.}
The function $\widehat\varphi^n$ satisfies the following integral equation for $x\in Z$
\begin{equation*}
	\widehat \varphi^n(t,x)=\widehat \varphi^n(0,x) + \int_0^t \iota_n P_n F(\widehat\varphi^n(s,x))ds.
\end{equation*}
From the uniform continuity of $F$ on $W$ and already established convergence of $\widehat \varphi^n(t,x)$ it follows that after passing to the limit $n \to \infty$ there holds
\begin{equation*}
	\varphi(t,x)=\varphi(0,x) + \int_0^t F(\varphi(s,x))ds,
\end{equation*}
which implies that for $x\in Z$ the function $u(t) = \varphi(t,x)$ is a solution of $u'=F(u)$ with $\varphi(0,x)=x$.

\textbf{Uniqueness.}
Let $u:[0,T] \to W$ be a solution of $u'=F(u)$ with initial condition $u(0)=x$. We will show that the sequence $\widehat \varphi^n(\cdot,x)$ converges uniformly to $u$. Applying Lemma~\ref{lem:estmLogN} to $n$-th Galerkin projection and the function $P_n u(t)$ we obtain an estimate
\begin{equation*}
  \|\widehat \varphi^n(t,x) - P_n(u(t))\| \leq  \delta_n \kappa_l(t).
\end{equation*}
Since the tail $\|(I-P_n)u(t)\| \leq \|(I-P_n)W\|$ is by Lemma~\ref{lem:compt-gss} uniformly converging to zero as $n\to \infty $, we see that $\widehat\varphi^n(\cdot,x) \to u$ uniformly on $[0,T]$.


\textbf{Lipschitz constant on $W$}. From Lemma~\ref{lem:estmLogN} applied to $n$-dimensional Galerkin projection for different initial conditions we obtain
\begin{equation*}
\|\widehat\varphi^n(t,x)-\widehat \varphi^n(t,y)\| =  \|\varphi^n(t,P_nx)-\varphi^n(t,P_ny)\| \leq e^{lt}\|P_n x - P_n y\|,
\end{equation*}
which after passing to the limit $n\to\infty$ gives
\begin{equation*}
  \|\varphi(t,x)-\varphi(t,y)\| \leq e^{lt}\|x - y\|.
\end{equation*}

Assertion 4 follows easily from the previous ones. \qed
