%\documentclass[10pt]{amsart}
\documentclass{article}
\usepackage{amssymb}
\usepackage{amsmath}
\usepackage{graphicx}
\usepackage{xspace}
%\usepackage{refcheck}

\newtheorem{theorem}{Theorem}
\newtheorem{definition}{Definition}
\newtheorem{lemma}[theorem]{Lemma}
\newtheorem{rem}[theorem]{Remark}
\newtheorem{pro}[theorem]{Proposition}
\newtheorem{con}[theorem]{Conjecture}
\newtheorem{cor}[theorem]{Corollary}
\newtheorem{alg}{Algorithm}
\newtheorem{remark}[theorem]{Remark}

\def\rep#1{\mbox{$\langle#1\rangle$}}


\newcommand{\inte }{{\rm int}\,}
\newcommand{\diam }{\,{\rm diam}\,}
\newcommand{\cl }{{\rm cl}\,}
\newcommand{\sgn }{{\rm sgn}\,}
\newcommand{\bd }{\partial}
\newcommand{\im }{\mbox{Im} \,}
\newcommand{\ind }{\mbox{ind} \,}
\newcommand{\id }{\mbox{Id} \,}
\newcommand{\dom }{{\rm dom}\,}
\newcommand{\dist }{\,{\rm dist}\,}
\newcommand{\ltwoq}{l_{2,q}}
\newcommand{\cover}[1]{\stackrel{#1}{\Longrightarrow}}
\newcommand{\WqS}[2]{W_{#1,#2}}
\newcommand{\PM}{\mathcal P}
\newcommand{\gss}{GSS\xspace}
\newcommand{\VL}{{\rm\textbf{VC}}\xspace}
\newcommand{\NC}[1]{\textbf{N#1}}
\newcommand{\BD}[3]{#1_{\langle #2 \rangle \langle #3 \rangle}}

%\newcommand{\qed}{$\Box$}
\def\qed{{\hfill{\vrule height5pt width3pt depth0pt}\medskip}}
\def\comment#1{{}}


\pdfoptionpdfminorversion=7   % powinno wylaczyc niektore warningi: pdf inclusion

\begin{document}
\begin{center}
{\Large \bf  Self-consistent bounds method for dissipative PDEs}

 \vskip 0.5cm
{\large Daniel Wilczak} and {\large Piotr Zgliczy\'nski}\footnote{Work of D.W. and P.Z. was supported by National Science Center (NCN) of Poland under project No. UMO-2016/22/A/ST1/00077} % Maestro
 \vskip 0.2cm
  Jagiellonian University, Faculty of Mathematics and Computer Science, \\
  \L ojasiewicza 6, 30--348  Krak\'ow, Poland \\ \texttt{e-mail:\,\{Daniel.Wilczak,Piotr.Zgliczynski\}@uj.edu.pl}
\vskip 0.5cm

\today
\end{center}

\begin{abstract}
 We discuss the method of self-consistent bounds for dissipative PDEs with periodic boundary conditions. We prove convergence theorems for a class of dissipative PDEs, which constitute a theoretical basis of a general framework for construction of an algorithm that computes bounds for the solutions of the underlying PDE and its dependence on initial conditions. 
 
 We also show, that the classical examples of parabolic PDEs including Kuramoto-Sivashinsky equation and the Navier-Stokes on the torus fit into this framework. 
\end{abstract}


\input introscb.tex

\input gcp-norms.tex

\input lin-maps-gss.tex

\input lineqestm.tex

\input c0conver.tex


\input c1conver.tex

\input isolation.tex

\input convercond.tex

\input varconvcond.tex

%\input bndOneStep.tex


%\input decayVar.tex

\input ref.tex

\end{document}
