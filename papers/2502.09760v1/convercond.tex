\section{Dissipative PDEs on the torus }
\label{sec:dissipativePDEs}

The aim of this section is to prove that the framework introduced in Section~\ref{sec:c0-conver} and Section~\ref{sec:c1-conver} is applicable to a certain class of PDEs. We start with technical estimates and then we show that conditions \textbf{S}, \textbf{C} and \textbf{VC} as well as the isolation property are satisfied for this class on certain families of sets.

Consider
\begin{equation}
  u_t = L u + N\left(u,Du,\dots,D^ru\right),  \label{eq:genpde}
\end{equation}
where $u \in \mathbb{R}^n$,  $x \in \mathbb{T}^d=\left(\mathbb{R}\mod 2\pi\right)^d$, $L$ is a linear operator, $N$ is a
polynomial and by $D^s u$ we denote $s^{\text{th}}$ order derivative of
$u$, i.e. the collection of all spatial (i.e. with respect to variable $x$) partial derivatives of $u$ of
order $s$. %The reason to consider polynomial and not more general functions $N$ is technical --- we need to compute
%the Fourier coefficients of $N\left(u,Du,\dots,D^ru\right)$. This can be achieved by taking suitable convolutions of Fourier expansions of $u$ and its spatial partial derivatives.

We require, that the operator $L$ is diagonal in the Fourier basis
$\{e^{ikx}\}_{k \in \mathbb{Z}^d}$,
\begin{equation*}
  L e^{ikx}= -\lambda_k e^{ikx},
\end{equation*}
with
\begin{eqnarray}
 L_* |k|^p &\leq& \lambda_k \leq  L^* |k|^p, \qquad \text{for all $|k| > K$ and  $K,L_*,L^* \geq 0$}, \label{eq:lambdak} \\
    p &>& r.  \label{eq:p>r}
\end{eqnarray}


If $a(t,x)$ is a sufficiently regular solution of
(\ref{eq:genpde}), then  we can expand it in Fourier series
$a(t,x)=\sum_{k \in \mathbb{Z}^d} a_k(t)e^{\mathrm{i}k\cdot x}$, $a_k \in \mathbb{C}^n$ to obtain
an infinite ladder of ordinary differential equations for the
coefficients $a_k$
\begin{equation}
  \frac{d a_k}{dt}=-\lambda_k a_k + N_k(a), \quad k \in \mathbb{Z}^d,  \label{eq:fugenpde}
\end{equation}
where $N_k(a)$ is $k$-th Fourier coefficient of function
$N(a,Da,\dots,D^ra)$.




Observe that $a_k$'s might not be independent variables. For example, assumption $a(t,x) \in \mathbb{R}^n$ forces the following \emph{reality condition}
\begin{equation}
  a_{-k}=\overline{a}_{k}.  \label{eq:reality}
\end{equation}
In such situation we have to consider the subspace defined by condition (\ref{eq:reality}). This subspace is invariant for all Galerkin projections of (\ref{eq:genpde}) onto subspaces containing both $a_k$ and $a_{-k}$. 

Other constraints like oddness or evenness of $a(t,x)$ may cause the
change of set of basic functions to $\sin(kx)$ or $\cos(kx)$ or combinations thereof.

In any case we will have $u(t,x)=\sum_{k \in I} a_k(t) e_k(x)$, where $I$ is a countable set and $a_k \in \mathbb{C}^n$ or $a_k \in \mathbb{R}^n$, hence our sequence space $H$ will be build
of real and imaginary parts of components of $a_k$ which  after choosing a suitable norm fits in the framework discussed in previous sections.  



\subsection{Some examples}


\subsubsection{Kuramoto-Sivashinsky equation}
Let us consider the one-dimensional Kuramoto-Sivashinsky PDE \cite{KT,S}, which is given by
\begin{equation}\label{eq:KS}
u_t = -\nu u_{xxxx} - u_{xx} + (u^2)_x, \qquad \nu>0,
\end{equation}
where $x \in \mathbb{R}$, $u(t,x) \in \mathbb{R}$ and we impose periodic boundary conditions

In the Fourier basis we obtain  the following ladder of ordinary differential equations for complex coefficients $a_k$
\begin{equation}
  \dot{a}_k = k^2(1-\nu k^2) a_k + \mathrm{i}k \sum_{\ell \in \mathbb{Z}} a_k a_{k - \ell},  \label{eq:Ksper}
\end{equation}
and the reality constraint (\ref{eq:reality}). We can either treat (\ref{eq:Ksper}) as equation acting on sequence space indexed 
by $k \in \mathbb{Z}$ and consider invariant subspace defined by reality condition (\ref{eq:reality}) with Galerkin filtration $J_n=\{ k \in \mathbb{Z}, |k| \leq n \}$
or we can eliminate variable $a_k$ for $k<0$ and rewrite the convolution term in (\ref{eq:Ksper}) in terms of $a_k$ with nonnegative $k$'s.

If for (\ref{eq:KS}) we impose odd and periodic boundary conditions (\ref{eq:KSbc})
 \begin{equation}
u(t,x)=-u(t,-x), \qquad u(t,x) = u(t,x+ 2\pi), \label{eq:KSbc}
\end{equation}
then we can represent  $u(t,x)=\sum_{k \in \mathbb{Z}_+} -2a_k(t) \sin(kx)$,
where $a_k \in \mathbb{R}$ and equation (\ref{eq:KS})
becomes \cite{CCP,ZM}
\begin{equation*}
  \frac{d a_k}{dt}=k^2(1-\nu k^2) a_k - k \sum_{n=1}^{k-1} a_n
  a_{k-n} + 2k \sum_{n=1}^{\infty} a_n  a_{n+k}, \quad k=1,2,3\dots
 % \label{eq:fuKS}
\end{equation*}



Observe that conditions (\ref{eq:lambdak}) and
(\ref{eq:p>r}) are satisfied for the KS equation. Namely we have
$\lambda_k \sim \nu k^4$, $p=4$, $r=1$.


\subsubsection{Navier Stokes equations on the torus}
 The general $d$-dimensional Navier-Stokes system
(NSS) is written for $d$ unknown functions
$u(t,x)=(u_1(t,x),\dots,u_d(t,x))$ of $d$ variables
$x=(x_1,\dots,x_d)$ and time $t$, and the pressure $p(t,x)$.
\begin{eqnarray}
  \frac{\partial u_j}{\partial t} + \sum_{k=1}^d u_k \frac{\partial u_j}{\partial
  x_k}&=& \nu \triangle u_j - \frac{\partial p}{\partial x_j} +
  f^{(j)} \label{eq:NS} \\
  \mbox{div}\ u&=& \sum_{j=1}^d \frac{\partial u_j}{\partial x_j}=0 \label{eq:div}
\end{eqnarray}
The functions $f^{(j)}$ are the components of the external
forcing, $\nu >0$ is the viscosity.

Therefore we have $\lambda_k = \nu |k|^2$, hence  we $p=2$ and $r=1$.

We consider (\ref{eq:NS}),(\ref{eq:div}) on the torus $\mathbb{T}^d=\left({\mathbb{R}/2\pi}\right)^d$. This enables us to use Fourier
series. We write
\begin{equation}
  u(t,x)=\sum_{k \in \mathbb{Z}^d} u_k(t)e^{\mathrm{i}(k,x)}, \qquad
  p(t,x)=\sum_{k \in \mathbb{Z}^d} p_k(t)e^{\mathrm{i}(k,x)}
\end{equation}
Observe that $u_k(t) \in \mathbb{C}^d$, i.e. they are
$d$-dimensional vectors and $p_k(t) \in \mathbb{C}$. We 
assume that $f_0=0$ and $u_0=0$.


Then (see \cite{Z} for details) (\ref{eq:div}) is reduced to 
\begin{eqnarray}
    (u_k,k)=0 \quad k \in \mathbb{Z}^d,  \label{eq:NSdivfourier}
\end{eqnarray}
and on the space of functions satisfying (\ref{eq:div}) the pressure disappears and we obtain the following infinite ladder of differential equations
for $u_k$
\begin{equation}
  \frac{d u_k}{d t}=-\mathrm{i} \sum_{k_1}(u_{k_1}|k)\sqcap_k u_{k-k_1} - \nu k^2u_k + \sqcap_k f_k,
    \label{eq:NSgal1}
\end{equation}
where $f_k$ are components of the external forcing, $\sqcap_k$
denotes the operator of orthogonal projection onto the
$(d-1)$-dimensional plane orthogonal to $k$. 



Observe that the subspace defined by incompressibility condition (\ref{eq:NSdivfourier}) and reality condition (\ref{eq:reality})  is invariant under  (\ref{eq:NSgal1})
and also is invariant for  Galerkin projections of (\ref{eq:NSgal1}), where for all $k \in \mathbb{Z}^d$ all components of $u_k$ and $u_{-k}$ are both included or excluded
in the projection.  This  defines a Galerkin filtration. Alternatively, using some more specific boundary conditions  (see for example \cite{AK21,BB21}), we define a set independent modes $\{u_k\}_{k \in I}$  (maybe with respect to some other function basis consisting of $\sin(kx)$ etc)
and rewrite the convolution term in (\ref{eq:NSgal1}) using only $u_k, k \in I$.  The formulas will be a bit more complicated, but essentially still could be seen as some
convolutions.





\subsection{Preparatory remarks}
Consider equation (\ref{eq:genpde}). We assume that $N$ is a polynomial and by $D^s u$ we denote $s^{\text{th}}$ order derivative of $u$, i.e. the collection of all spatial (i.e. with respect to variable $x$) partial derivatives of $u$ of order $s$. The reason to consider polynomial and not more general functions $N$ is technical --- we need to compute the Fourier coefficients of $N\left(u,Du,\dots,D^ru\right)$. This can be achieved by taking suitable convolutions of Fourier expansions of $u$ and its spatial partial derivatives. For analytic $N$ the results are a bit more involved, i.e. we will have infinite series of convolutions. This still it is manageable but we omit it for the sake
of simplicity.


In what follows all considerations will be done assuming $u_k$ represent the Fourier expansion of $u$,
\begin{equation}
  u(t,x)=\sum_{k \in \mathbb{Z}^d} u_k(t) \exp(\mathrm{i} k \cdot x), \label{eq:u-furier}
\end{equation}
 and $u_k$ are independent variables. The estimates
and results obtained under this assumptions can be easily translated to the case when some constraints coming from the reality requirement (\ref{eq:reality}), boundary conditions
lead us to use $\sin(kx)$, $\cos(kx)$ or combinations  thereof as the basis for the expansion.  In such situation we have a natural imbedding into the previous case of the Fourier expansion with respect to
$\exp(\mathrm{i} kx)$ and all estimates are the same up to some constants.  


In the sequel for $k \in \mathbb{Z}^d$ we will use norm $|k|$ which should satisfy \newline
 $|(k_1,\dots,k_d)| \geq |k_j|$ for $j=1,\dots,d$. Moreover, $u_k$ can be vectors or complex numbers. This can be easily reformulated to fit into the framework discussed in previous section, we will use a countable set of indices  $(k,\ell,a)$, where $k \in I$, $\ell \in \{1,\dots,d\}$, $a \in \{0,1\}$, so that $u_{k,\ell,a}$ is real ($a=0$) or imaginary ($a=1$) part of $\ell$-th component of vector $u_k \in \mathbb{C}^d$. The Galerkin filtration could be chosen as follows 
 \begin{equation}
 J_n=\{(k,\ell,a), |k| \leq n, \ell \in \{1,\dots,d\}, a \in \{0,1\}\}.
 \end{equation} 


 After formally inserting the Fourier expansion   for $u,Du,\dots,D^r u$  in $N()$ we obtain a sum
 expressions of the following type for each monomial in $N$
\begin{equation*}
  \sum_{k_1+\dots + k_l=k} v_{k_1} \cdot  v_{k_2} \cdot
  \dots \cdot v_{k_l},
\end{equation*}
where  each of the variables $v_{k_j}$, $j=1,\dots,l$ is some Fourier coefficient of one of the components of $u$ or its partial derivatives of the order less than or equal to $r$.  This is a formal expansion, the questions of convergence and differentiability are treated in the following sections.

\subsection{Derivatives with respect to Fourier coefficients}

The goal of this section it to write  compact formulas for derivatives with respect of Fourier coefficients.

We will use the following notation to denote (partial) derivatives. For a function $f(x_1,\dots,x_n)$ and $\alpha \in \mathbb{N}^n$ we set  $D^\alpha f(x_1,\dots,x_n)=\frac{\partial^{|\alpha|}}{\partial x_1^{\alpha_1} \partial x_2^{\alpha_2}\dots \partial x_n^{\alpha_n}}$, where $|\alpha|=\sum_j \alpha_j$.
For $k \in \mathbb{Z}^n$ we define $k^\alpha= k_1^{\alpha_1} k_2^{\alpha_2} \cdot \dots \cdot k_n^{\alpha_n}$.

When considering $N(u,Du,\dots)$ it is convenient to name the arguments of $N$ by $u$, $D^\alpha u$, which we will use below. For example, for $N(u,u_x)=uu_x$ we will
have $\frac{\partial N}{\partial u}(u,u_x)=u_x$ and  $\frac{\partial N}{\partial u_x}(u,u_x)=u$. We will use this convention below.



We have
\begin{multline*}
   N(u+h,\dots,D^r(u+h))= N(u,\dots,D^ru)
    + \sum_{0\leq |\alpha| \leq r } \frac{\partial N}{\partial (D^\alpha u)}(u,\dots,D^ru)D^\alpha h \\
   + \sum_{\substack{\alpha_1,\alpha_2\\  0\leq |\alpha_1|\leq r,\\ 0 \leq |\alpha_2|\leq r}} \frac{1}{2} \frac{\partial^2 N}{\partial (D^{\alpha_1}u) \partial(D^{\alpha_2}u)}(u,\dots,D^ru) D^{\alpha_1}h D^{\alpha_2}h\\
   + O(\|h\|^3+\ldots+\|D^rh\|^3)
\end{multline*}
Therefore we  have
\begin{eqnarray}
  \frac{\partial N_k}{\partial u_j} = \sum_{|\alpha| \leq r}\left(\left(\frac{\partial N}{\partial (D^\alpha u)}(u,\dots,D^ru)\right)_{k-j}\mathrm{i}^{|\alpha|} j^\alpha\right)  \label{eq:derNk-f}
\end{eqnarray}
and
\begin{eqnarray}
  \frac{\partial^2 N_k}{\partial u_{j_1}  \partial u_{j_2}} = \left(\sum_{\substack{\alpha_1,\alpha_2\\  0\leq |\alpha_1|\leq r, \\ 0 \leq |\alpha_2|\leq r}}\left(\frac{\partial^2 N}{\partial(D^{\alpha_1}u)  \partial(D^{\alpha_2} u) }(u,\dots,D^ru)\right)_{k-j_1-j_2}\mathrm{i}^{|\alpha_1|+|\alpha_2|}j_1^{\alpha_1} j_2^{\alpha_2}\right). \label{eq:der2Nk-f}
\end{eqnarray}
Formulas (\ref{eq:derNk-f},\ref{eq:der2Nk-f}) after formally inserting Fourier expansions define us formal power series, which we denote by symbols $ \frac{\partial N_k}{\partial u_j}$, $\frac{\partial^2 N_k}{\partial u_{j_1}  \partial u_{j_2}}$ without actually claim that these functions represent partial derivatives. This fact is proven later for
arguments belong to some sets with good convergence properties.

\subsection{Estimates for sets with polynomial decay}




Throughout this section a polymial $N(u,\dots,D^r u)$ is fixed  and various estimates given below will obviously depend on this polynomial, but this will not be clearly
indicated.

In this subsection our goal is to prove the following bounds on the vector field induced in Fourier domain and its derivatives on set with polynomial decay.
\begin{theorem}
\label{thm:Dgen}
 Let $s > s_0=d+r$.
 If $|a_k| \leq C/|k|^s$, $|a_0| \leq C$, then  the formal series defining $N_k$, $\frac{\partial N_k}{\partial a_j}$, $\frac{\partial^2 N_k}{\partial a_{j_1} \partial a_{j_2}}$
 are absolutely convergent and
there exists $D=D(C,s)$, $D_1=D_1(C,s)$, $D_2=D_2(C,s)$
\begin{eqnarray}
  |N_k| &\leq& \frac{D}{|k|^{s-r}}, k \neq 0, \qquad \mbox{and} \qquad |N_0| \leq D,  \label{eq:Nk-estm} \\
  \left|\frac{\partial N_k}{\partial a_j}\right| &\leq& \frac{D_1 |j|^r}{|k-j|^{s-r}},  k \neq j, \qquad \mbox{and} \nonumber \\
     & &  \qquad \left|\frac{\partial N_k}{\partial a_j}\right|  \leq D_1|j|^r, k=j \label{eq:derNk-estm} \\
   \left|\frac{\partial^2 N_k}{\partial a_{j_1} \partial a_{j_2}}\right| &\leq& \frac{D_2 |j_1|^r |j_2|^r}{|k-j_1-j_2|^{s-r}},   k \neq j_1+j_2, \qquad \mbox{and} \nonumber \\
     & & \qquad   \left|\frac{\partial^2 N_k}{\partial a_{j_1} \partial a_{j_2}}\right|  \leq D_2 |j_1|^r |j_2|^r, k=j_1+j_2.  \label{eq:derNk2-estm}
\end{eqnarray}
\end{theorem}
Assertion (\ref{eq:Nk-estm}) has been proven as Lemma 3.1 in \cite{ZKS3}.
Before the proof of Theorem~\ref{thm:Dgen} we need to establish several short lemmas. Some of them are taken from Section 3.1 in \cite{ZKS3}, but we include
their short proofs for the sake of completeness.

\begin{lemma} \cite[Lemma 3.2]{ZKS3}
\label{lem:powineq}
  Let $\gamma > 1$. For any $a,b \geq 0$ the following inequality
  is satisfied
\begin{equation*}
  (a+b)^\gamma \leq 2^{\gamma-1}(a^\gamma + b^\gamma). %\label{eq:ineq-pow}
\end{equation*}
\end{lemma}
\textbf{Proof:} This is an easy consequence of the convexity of
function $x \mapsto x^\gamma$ for $\gamma >1$. Namely
\begin{eqnarray*}
  (a+b)^\gamma = 2^\gamma \left(\frac{a + b}{2}\right)^\gamma \leq 2^\gamma \left(\frac{a^\gamma +
  b^\gamma}{2}\right)= 2^{\gamma-1} (a^\gamma + b^\gamma).
\end{eqnarray*}
\qed






The following  lemma was proved in \cite{Sa}
\begin{lemma} \cite[Lemma 3.3]{ZKS3}
\label{lem:estmQ} Assume that $\gamma > d$.  Then there exists
$S_Q(d,\gamma) \in \mathbb{R}$ such that for any $k \in
\mathbb{Z}^d \setminus \{0\}$ holds
\begin{equation*}
  \sum_{k_1 \in \mathbb{Z}^d \setminus \{0, k \}} \frac{1}{|k_1|^\gamma |k-k_1|^\gamma} \leq
    \frac{S_Q(d,\gamma)}{|k|^{\gamma}}.
\end{equation*}
\end{lemma}
\textbf{Proof:} From the triangle inequality and
Lemma~\ref{lem:powineq} we have
\begin{eqnarray*}
  \frac{|i|^\gamma}{|k-i|^\gamma |k|^\gamma} &\leq&
  \frac{\left( |k-i| + |k| \right)^\gamma}{|k-i|^\gamma |k|^\gamma} \\
 & & \leq  \frac{ 2^{\gamma -1}(|k-i|^\gamma + |k|^\gamma)}{|k-i|^\gamma
  |k|^\gamma} =
  2^{\gamma -1} \left( \frac{1}{|k|^\gamma} + \frac{1}{|k-i|^\gamma}
  \right).
\end{eqnarray*}
Hence
\begin{eqnarray*}
 \sum_{k \in {\mathbb Z}^d\setminus \{0,i\}} \frac{1}{|k|^\gamma |i -
  k|^\gamma} \leq  \sum_{k \in {\mathbb Z}^d\setminus \{0,i\}} \frac{2^{\gamma-1}}{|i|^\gamma}
  \left( \frac{1}{|k|^\gamma} +\frac{1}{ |i -
  k|^\gamma}\right) <  \\ \frac{2^{\gamma}}{|i|^\gamma} \sum_{k \in {\mathbb Z}^d \setminus \{0\}}
  \frac{1}{|k|^\gamma}.
\end{eqnarray*}
 \qed

Now we want to include  also the vectors of zero length in the sum
appearing in Lemma~\ref{lem:estmQ}.  To make
expression of some formulas less cumbersome in this subsection for
$0=\{0\}^d \in \mathbb{Z}^d$ we redefine its norm by setting
$|0|=1$.
\begin{lemma} \cite[Lemma 3.4]{ZKS3}
\label{lem:estmQ2} Assume that $\gamma > d$.  Then there exists
$C_2(d,\gamma) \in \mathbb{R}$ such that for any $k \in
\mathbb{Z}^d$ holds
\begin{equation*}
  \sum_{\substack{k_1,k_2 \in \mathbb{Z}^d\\ k_1+k_2=k}} \frac{1}{|k_1|^\gamma |k_2|^\gamma} \leq
    \frac{C_2(d,\gamma)}{|k|^{\gamma}}.
\end{equation*}
\end{lemma}
\textbf{Proof:} Consider two cases $k=0$ and $k \neq 0$.

If $k=0$, then there exists $\widetilde{C}(d,\gamma) \in
\mathbb{R}$ such that
\begin{eqnarray*}
  \sum_{\substack{k_1,k_2 \in \mathbb{Z}^d\\ k_1+k_2=k}} \frac{1}{|k_1|^\gamma |k_2|^\gamma} =
  1 + \sum_{k_1 \in  \mathbb{Z}^d\setminus \{0\}}
  \frac{1}{|k_1|^{2\gamma}} =\widetilde{C}(d,\gamma).
\end{eqnarray*}
If $k \neq 0$, then from Lemma~\ref{lem:estmQ} it follows that
\begin{eqnarray*}
 \sum_{\substack{k_1,k_2 \in \mathbb{Z}^d\\ k_1+k_2=k}} \frac{1}{|k_1|^\gamma
 |k_2|^\gamma}= \frac{2}{|k|^\gamma} +
 \sum_{\substack{k_1,k_2 \in \mathbb{Z}^d \setminus \{0\}\\ k_1+k_2=k}} \frac{1}{|k_1|^\gamma |k_2|^\gamma}
    \leq
    \frac{S_Q(d,\gamma) +   2}{|k|^\gamma}.
\end{eqnarray*}
Hence the assertion holds for
$C_2(d,\gamma)=\max(\widetilde{C}(d,\gamma),S_Q(d,\gamma) + 2 )$.
\qed

\begin{lemma} \cite[Lemma 3.5]{ZKS3}
\label{lem:estmQn} Assume $\gamma > d$. For any $n \in
\mathbb{Z}_+$, $n > 1$  there exists $C_n(d,\gamma) \in
\mathbb{R}$ such that for any $k \in \mathbb{Z}^d$ holds
\begin{equation*}
  \sum_{k_1,k_2,\dots,k_n \in \mathbb{Z}^d, \sum_{i=1}^n k_i=k}
  \frac{1}{|k_1|^\gamma |k_2|^\gamma \cdot \dots \cdot |k_n|^\gamma} \leq
    \frac{C_n(d,\gamma)}{|k|^{\gamma}}.
\end{equation*}
\end{lemma}
\textbf{Proof:} By induction. Case $n=2$ is contained in
Lemma~\ref{lem:estmQ2}. Assume now that the assertion holds for
$n$. We have
\begin{eqnarray*}
  \sum_{k_1,k_2,\dots,k_{n+1} \in \mathbb{Z}^d, \sum_{i=1}^{n+1} k_i=k}
  \frac{1}{|k_1|^\gamma |k_2|^\gamma \cdot \dots \cdot |k_{n+1}|^\gamma}
   = \\ \sum_{k_{n+1} \in \mathbb{Z}^d} \left( \frac{1}{|k_{n+1}|^\gamma}
    \sum_{k_1,k_2,\dots,k_n \in \mathbb{Z}^d, \sum_{i=1}^n k_i=k-k_{n+1}}
  \frac{1}{|k_1|^\gamma |k_2|^\gamma \cdot \dots \cdot |k_n|^\gamma}
  \right) \leq \\
 \sum_{k_{n+1} \in \mathbb{Z}^d}  \frac{1}{|k_{n+1}|^\gamma} \cdot
 \frac{C_n(d,\gamma)}{|k-k_{n+1}|^\gamma} \leq \frac{C_2(d,\gamma)
 C_n(d,\gamma)}{|k|^\gamma}.
\end{eqnarray*}
\qed

\noindent \textbf{Proof of Theorem~\ref{thm:Dgen}:} For the proof it
is enough to assume that $N$ is a monomial. After formally
inserting the Fourier expansion  for $u,Du,\dots,D^r u$ we obtain
the expression of the following type
\begin{equation}
  N_k(u)=\sum_{k_1+\dots + k_l=k} v_{k_1} \cdot  v_{k_2} \cdot
  \dots \cdot v_{k_l}, \label{eq:Nkmon}
\end{equation}
where  each of the variables $v_{k_i}$, $i=1,\dots,l$ is some
Fourier coefficient of one the components of $u$ or its partial
derivatives of the order less than or equal to $r$.

Observe that for the Fourier coefficients of partial derivatives
up to order $r$ we have the following estimates
\begin{equation}
  \left| \frac{\partial^{\beta_1 + \dots + \beta_l} u}{\partial x_1^{\beta_1}\dots \partial x_d^{\beta_l}
  }\right| \leq \frac{C}{|k|^{s - (\beta_1 + \dots + \beta_l)}}
  \leq \frac{C}{|k|^{s - r}}.  \label{eq:betaestm}
\end{equation}
From conditions (\ref{eq:Nkmon}) and (\ref{eq:betaestm}), and
Lemma~\ref{lem:estmQn} we obtain
\begin{equation*}
 |N_k(u)| \leq \sum_{k_1+\dots + k_n=k} \frac{C^n}{|k_1|^{s-r} \cdot \dots \cdot |k_n|^{s-r}}
   \leq \frac{C^n C_n(d,s-r)}{|k|^{s-r}}
\end{equation*}
This establishes (\ref{eq:Nk-estm}).

To prove (\ref{eq:derNk-estm}) we use (\ref{eq:derNk-f}). From previous reasoning applied to polynomials $\frac{\partial N}{\partial (D^\alpha u)}(u,\dots,D^ru)$ we immediately
obtain
\begin{eqnarray*}
 \left|\left(\frac{\partial N}{\partial (D^\alpha u)}(u,\dots,D^ru)\right)_{k-j}\right| \leq \frac{D_1}{|k-j|^{r-s}},
\end{eqnarray*}
which together with bound $|\alpha| \leq r$ gives (\ref{eq:derNk-estm}).

The proof of (\ref{eq:derNk2-estm}) is analogous. \qed



Later we will use the following lemma.
\begin{lemma}
\label{lem:sumForDer}
Assume that $s > 2r + d$, then there exists $C(d,s,r)$ such that
\begin{equation*}
   \sum_{k \in \mathbb{Z}^d, k \neq i}\frac{|k|^r}{|i-k|^{s-r}} \leq  C(d,s,r)(1+|i|^r).
\end{equation*}
\end{lemma}
\textbf{Proof:}
\begin{multline*}
    \sum_{k \in \mathbb{Z}^d, k \neq i}\frac{|k|^r}{|i-k|^{s-r}} \leq \sum_{k \in \mathbb{Z}^d, k \neq i}\frac{(|i-k| + |i|)^r}{|i-k|^{s-r}} \\
\leq \sum_{k \in \mathbb{Z}^d, k \neq i}\frac{2^{r-1}|i-k|^r + 2^{r-1}|i|^r}{|i-k|^{s-r}}\\
= 2^{r-1} \left(\sum_{k \in \mathbb{Z}^d, k \neq i}\frac{1}{|i-k|^{s-2r}} + \sum_{k \in \mathbb{Z}^d, k \neq i}\frac{|i|^r}{|i-k|^{s-r}}\right) \leq
C(d,s,r)(1 + |i|^r).
\end{multline*}
Observe that the two infinite sums are convergent under assumption $s> 2r +d$.
\qed

\subsection{Conditions S1 and S2}
\label{subsec:ch-concond}
We will be interested in the following candidates for space $H$:
\begin{itemize}
\item $c_0$, $\|x\|=\sup_{k\in\mathbb Z^d} |x_k|$,
\item $l_p$, $\|x\|=\left(\sum_{k\in\mathbb Z^d}  |x_k|^p\right)^{1/p}$ with $p \geq 1$,
\item $\mathcal{W}^{m,p}$, $\|x\|=\left( \sum_{i=0}^m  \sum_{k\in\mathbb Z^d}  |k|^{ip} |x_k|^p \right)^{1/p}$ with $p \geq 1$ and $m \geq 0$.
\end{itemize}
It is immediate that the space listed above are \gss spaces.

We consider sets of the form
\begin{eqnarray*}
  W_{P}(C,s) &=&\left\{ |x_k| \leq \frac{C}{|k|^s}, k \neq 0,  |x_0| \leq C \right\}, s >0, \\
  W_{exp}(q,S)&=&\left\{ |x_k| \leq \frac{S}{q^{|k|}} \right\}, \quad q>1.
\end{eqnarray*}
Clearly they satisfy condition \textbf{S1}.

\begin{remark}
	Observe, that if $W_{exp}(q,S)\subset W_P(C,s)\subset H$ then it is a closed subset and thus it inherits {\bf S2}, {\bf C1} and {\bf C2} from $W_P(C,s)$. Thus, in what follows we will focus on sets with polynomial decay, only.
\end{remark}

We will show that condition \textbf{S1} on sets $W_{exp}(q,S)$, $W_P(C,s)$ when considered in suitable spaces.
\begin{theorem}
\label{thm:polexp-comp}
The following statements hold true.
\begin{itemize}
 \item $W_P(C,s)$ is a compact subset of  $c_0$ for any $d$ and $s>1$.
 \item $W_P(C,s)$ is a compact subset of $l_p$ and $\mathcal{W}^{m,p}$ provided $(s-m)p >d$ (with $m=0$ for space  $l_p$).
 \item  $W_{exp}(q,S)$ is a compact subset of  $c_0$, $l_p$, $\mathcal{W}^{m,p}$ for any $d$, $m$, $p$ and $q>1$.
\end{itemize}
\end{theorem}
\textbf{Proof:}
We  use Lemma~\ref{lem:compt-gss} as a criterion for compactness. It is immediate that the sets $ W_{P}(C,s)$ and $ W_{exp}(q,S)$ are closed in all spaces considered (if contained in them).

The case of $c_0$ space is obvious. In other cases it is enough that to observe that
\begin{equation*}
  \sum_{|k| \geq n} |k|^a |x_k|^b \sim \int_{n}^\infty r^{d-1} r^a x(r)^{b}dr,
\end{equation*}
where $x(r)=\frac{C}{r^s}$ in case od $W_P$ and $x(r)=S q^{-r}$ in the case od $W_{exp}$. From this we obtain our assertion for $W_{exp}$ for all $d$, $p$ and $m$.

In the case of $W_P$ the largest exponent under integral must be less than $-1$ to obtain the convergence, which gives
\begin{equation*}
  d-1 + mp - sp < -1.
\end{equation*}
\qed



\subsection{Conditions \textbf{C1,C2}}






\subsubsection{Condition C1 for set with polynomial decay}

\begin{theorem}
\label{thm:existsCi} Consider (\ref{eq:fugenpde}). Assume that
conditions (\ref{eq:lambdak}) and
(\ref{eq:p>r}) hold.

Let $W=W_{P}(C,s)$. Let $H$ be one of space listed at the beginning of Section~\ref{subsec:ch-concond}.

Then
\begin{itemize}
  \item if $s > p+d$, then $W \subset c_0$ and $F:W \to c_0$ is continuous
  \item if $(s-p - \ell)q > d$, then   $W \subset \mathcal{W}^{\ell,q}$   and
  $F:W \to \mathcal{W}^{\ell,q}$ is continuous (with $\ell=0$ for space  $l_q$)
\end{itemize}


\end{theorem}
\textbf{Proof:}

The first question is whether  $W \subset \dom{F}$.
Consider $u \in W$. From Theorem~\ref{thm:Dgen} and condition (\ref{eq:lambdak}) it follows
that $F_k(u)$ is defined and for $|k|>K$ holds
\begin{equation}
  |F_k(u)| \leq L^* C |k|^{p-s} + D |k|^{r-s} \leq
  \frac{D_2}{|k|^{s-p}} \label{eq:fkbd}.
\end{equation}
for some constants $D$ and $D_2$.
From Theorem~\ref{thm:polexp-comp} and our assumptions it follows $W$ is compact in various spaces listed in the assertion and $F(W)$ is also in the listed space
and is contained in some compact space (due to Lemma~\ref{lem:compt-gss} and decay estimates (\ref{eq:fkbd})). Moreover for all  $u \in W$
\begin{equation*}
 \lim_{n \to \infty} P_n F(u)=F(u).
\end{equation*}
uniformly on $W$.

Hence to prove the continuity of $F:W \to H$ it is enough to prove that $F_k:W \to H_k$ is continuous.

Let us fix index $k$ and assume   $u^n,u^* \in W$, for $n
\in \mathbb{N}$ and $u^n \to u^*$ for $n \to \infty$. We have
(compare the proof of Theorem~\ref{thm:Dgen})
\begin{equation*}
  F_k(u)=\lambda_k u_k + N_k(u)=\lambda_k u_k + \sum_{i \in J} N_{k,i}(u),
\end{equation*}
where $J$ is some set of multindices and for each $i \in J$,
$N_{k,i}$ is monomial depending on the finite number of $u_{l}$,
i.e.
\begin{displaymath}
  N_{k,i}= a u_{k_1}\cdot u_{k_2}  \cdot \dots u_{k_l}, \quad
  \mbox{for some $a \in \mathbb{C}$ and $k_1+\dots+k_l=k$ }
\end{displaymath}

The term $\lambda_k u_k$ is continuous, hence it is enough to
consider $N_k$, only. Let us fix $\epsilon > 0$. From
Theorem~\ref{thm:Dgen} it follows that there exists a finite set $S
\subset J$, such that
\begin{equation}
  \sum_{i \in J \setminus S} |N_{k,i}(u)| < \epsilon/3, \qquad
    \mbox{ for all $u \in W$}. \label{eq:Ntailestm}
\end{equation}
There exists $L$, such that for all $i \in S$ monomials
$N_{k,i}(u)$ depend in fact on the variables $u_{l}$ for $|l| \leq
L$, hence $\sum_{i \in S} N_{k,i}(u)$ is continuous on $W$. Therefore there exists $n_0$, such that
\begin{equation}
  \left| \sum_{i \in S} N_{k,i}(u^n) -  \sum_{i \in S} N_{k,i}(u^*)
  \right|< \epsilon/3.  \label{eq:Nmainestm}
\end{equation}
From (\ref{eq:Nmainestm}) and (\ref{eq:Ntailestm}) we obtain for
$n > n_0$
\begin{eqnarray*}
  |N_k(u^n) - N_k(u^*)| \leq   \left| \sum_{i \in S} N_{k,i}(u^n) -  \sum_{i \in S} N_{k,i}(u^*)
  \right| +  \\ \sum_{i \in J \setminus S} |N_{k,i}(u^n)| +
     \sum_{i \in J \setminus S} |N_{k,i}(u^*)| < \epsilon.
\end{eqnarray*}
Hence $\lim_{n \to \infty} N_k(u^n)=N_k(u^*)$.
 \qed




\subsubsection{Condition C2 for sets with polynomial decay}
\label{sssec:condDpolDec}
We will treat only the spaces $l_2$, $c_0$ and $l_1$ because for those spaces we have nice and relatively compact formulas for logarithmic norms.

Namely, we have for $A \in \mathbb{R}^{n \times n}$
\begin{eqnarray*}
  \mu_2 (A) = \max (\lambda \in Sp(A+A^t)/2), \\
  \mu_1 (A) = \max_{i=1,\dots,n} \left(a_{ii} + \sum_{k,k \neq i} |a_{ik}| \right), \\
  \mu_\infty(A) = \max_{k=1,\dots,n} \left(a_{kk} + \sum_{i,k \neq i} |a_{ik}| \right).
\end{eqnarray*}
\begin{lemma}
The same assumptions and notation as in Lemma~\ref{thm:existsCi}, but we restrict $H$ to one of the following spaces $c_0$, $l_2$, $l_1$. Assume additionally that $s>2r+d$. Then $F$ satisfies
condition {\bf C2} on $W$.
\end{lemma}
\textbf{Proof:}
First we deal with $H=c_0$. In this case the logarithmic norm is given by $\mu_\infty$.
From Theorem~\ref{thm:Dgen} and Lemma~\ref{lem:sumForDer} we have on $W$
\begin{eqnarray*}
S(row)_k:=\sum_{k,k \neq i} \left|\frac{\partial F^n_{i}}{\partial a_k} \right| \leq  \sum_{k \in \mathbb{Z}^d, k \neq i}\frac{D_1|k|^r}{|i-k|^{s-r}} \leq \tilde{C}_1(1 + |i|^r).
\end{eqnarray*}
Since  by (\ref{eq:lambdak}) $a_{kk}   \leq -L_* |k|^p$, hence since $p>r$ (by (\ref{eq:p>r})) we obtain \\
 $S(row)_k + a_{kk} \to -\infty$ for $|k| \to \infty$.
Therefore there exists $l \in \mathbb{R}$, such that
$\mu_{\infty}(Df^n(P_nz))< l$ for all $n$ and $z \in W$.

Now we assume  $H=l_1$ and the logarithmic norm is $\mu_1$. We have
\begin{eqnarray*}
S(col)_i:=\sum_{i,k \neq i} \left|\frac{\partial F^n_{i}}{\partial a_k} \right| \leq  \sum_{i \in \mathbb{Z}^d, k \neq i}\frac{D_1|k|^r}{|i-k|^{s-r}} \leq  |k|^r \tilde{C}_2.
\end{eqnarray*}
We conclude exactly as in the case of $\mu_\infty$.

When $H=l_2$, then we use Gershogorin Theorem \cite{G} to estimate the spectrum $\frac{1}{2}\left(DF^n + \left(DF^n\right)^T\right)$. Observe the radius of $k$-th Gershgorin circle
will be bounded from above as follows
\begin{equation*}
  R_i \leq (S(col)_i + S(row)_i)/2 \leq \tilde{C}|i|^r.
\end{equation*}
We conclude as in the previous cases.
\qed
