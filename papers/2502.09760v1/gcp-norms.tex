\section{Sequence spaces with good norms}
\label{sec:gss}

The goal of this section is to define the sequence spaces on which we will work later in this paper. We will be dealing with sequence spaces $H \subset \mathbb{R}^{\mathbb{Z}_+}$ equipped with some norm $\|\cdot\|$. Put $e_i=(\delta_{ij})_{j\in\mathbb Z_+}$. By $e_j^* \in H^*$ we denote the dual form, that is $e^*_j(e_i)=\delta_{ij}$.

By $\pi_k$ we denote the projection onto $k$-th direction, i.e. $\pi_k\left(\sum w_j e_j\right)=w_ke_k$. For a nonempty set $J \subset \mathbb{Z}_+$ by $P_J$ we denote a projection defined by $P_J(w)=\sum_{i\in J}w_i e_i$.

%Let $\emptyset\neq J_1  \subsetneq  J_2  \subsetneq J_3 \dots  $ be a family of finite sets, such that $\bigcup_{n \in \mathbb{Z}_+} J_n = \mathbb Z_+$.
%For $n \in \mathbb{Z}_+$  by $H_n$ we denote a subspace spanned by
%$\{e_j\}_{j \in J_n}$. Put $P_n:=P_{J_n}$ and $Q_n=\mathrm{Id}-P_n$. By $\iota_n:H_n \to H$ we denote the embedding $H_n$ into $H$.

For $n \in \mathbb{Z}_+$  by $H_n$ we denote a subspace spanned by
$\{e_j\}_{j \leq n}$. Put $P_n:=P_{\{j \leq n\}}$ and $Q_n=\mathrm{Id}-P_n$. By $\iota_n:H_n \to H$ we denote the embedding $H_n$ into $H$.

We define our standing assumptions \NC1--\NC5 on the space $(H,\|\cdot\|)$.

\begin{definition}
%\label{def:gss-norm}
We will say that $H \subset \mathbb{R}^{\mathbb{Z}_+}$ with norm $\| \cdot \|$ is \emph{\gss} (good sequence space) if the following conditions are satisfied.
\begin{description}
\item[\NC 1] $H$ is a Banach space.
\item[\NC 2] $\forall w \in H\ w=\sum_i w_i e_i$.
\item[\NC 3] For all  $w \in H$ and for any $\alpha \in \{-1,1\}^{\mathbb{Z}_+}$ there holds $w^\alpha=\sum_i \alpha_i w_i e_i \in H$
and $\|w\|=\|w^\alpha\|$.
\item[\NC 4] $\|P_J w\| \leq \|w\|,  \quad \forall w \in H, \, \forall J \subset \mathbb{Z}_+$.
\item[\NC 5]  there exists  constant $G$, such that for all $w \in H$ and for all $i$ $|w_i| \leq G \|w\|$.
\end{description}
\end{definition}

\textbf{Examples:} $l_2$, $l_1$, $c_0$ (sequences converging to $0$) with the norm $\|\cdot\|_\infty$. However $l_\infty$ is not a \gss space, because \NC2 is not satisfied.
The space defined by convergence of $\sum_i |w_i|/2^i$ does not satisfy \NC 5.


\begin{lemma}
\label{lem:gcp-prop}
If $(H,\|\cdot\|)$  is \gss, then  for any $J \subset \mathbb{Z}_+$ the projection $P_J$ is continuous and for any $w \in H$ there holds
\begin{eqnarray}
    \|w\|&=&\sup_{n \in \mathbb{Z}_+} \|P_n w\|, \label{eq:nsupGalproj} \\
    \|(I-P_n) w\| &\to& 0, \quad \mbox{for $n \to \infty$}.  \label{eq:remto0}
\end{eqnarray}
\end{lemma}
\textbf{Proof:}
Continuity of $P_J$ follows immediately from \NC4. From \NC2 and \NC4 we have
$$\|w\|=\lim_{n\to\infty}\|P_n w^n\|\leq \sup_n\|P_nw\|\leq \|w\|,$$
which proves (\ref{eq:nsupGalproj}).
From \NC4 the sequence $\|P_nw\|$ is non-decreasing, hence $\lim_{n\to\infty}\|P_nw\|=\sup_{n}\|P_nw\|=\|w\|$. In consequence,
$$\|(I-P_n) w\|\leq \|w\|-\|P_nw\|\to 0.$$
\qed


\begin{lemma}
\label{lem:compt-gss}
  Assume that $H$ is \gss. Then $W \subset H$ is compact, iff it is closed, bounded and \emph{$W$ has uniform bounds on the tail}, i.e. for every $\epsilon >0$ there exists $N$, such that
  for all $n \geq N$ and for all $w \in W$ there holds $\|(I-P_n)w\| < \epsilon$.
\end{lemma}
\textbf{Proof:}
Implication $\Rightarrow$. Since $W$ is compact, it is also bounded and closed.  To prove the existence of uniform bound for the tail let us fix $\epsilon >0$
and define an open covering $\{U_n\}_{n \in \mathbb{Z}_+}$ of $W$ by setting $U_n=\{w \in H: \|(I-P_n)w\| < \epsilon\}$. It is indeed a covering due to (\ref{eq:remto0}) in Lemma~\ref{lem:gcp-prop}.
Observe that $U_n \subset U_{n+M}$ for any $n,M \in \mathbb{Z}_+$ (this a consequence of assumption \NC4).
From compactness of $W$ it follows that there exists $N$, such that $W \subset U_N \subset U_{N+M}$ for any $M \in \mathbb{Z}_+$.


Implication $\Leftarrow$.  Fix a sequence $c^k \in W$. We would like to prove that it has a convergent subsequence. From the boundedness of $W$ and the diagonal argument it follows that we can find a subsequence of $d^m=c^{k_m}$, such that for each $m \in \mathbb{Z}_+$ the sequence $(d^m_n)_{m\in\mathbb Z_+}$ is convergent to some $\bar{d}_n$. Therefore, for each $n$ the sequence $(P_nd^m)_{m\in\mathbb Z_+}$ satisfies the Cauchy condition. We will show that $d^m$ also satisfies the Cauchy condition.

Let us fix $\epsilon >0$. For $j\in{\mathbb Z_+}$ we have
\begin{eqnarray*}
  \|d^m - d^{m+j}\| \leq \|P_n d^m - P_n d^{m+j}\| + \|(I-P_n) d^m\| + \|(I-P_n) d^{m+j}\|.
\end{eqnarray*}
From the uniform bound on tail in $W$ we find $N$ such that $\|(I-P_N)w\| < \epsilon$ for $w \in W$, in particular for all $d^m$. Given fixed $N$ from the Cauchy condition for the sequence $(P_N d^m)_{m\in\mathbb Z_+}$ it follows that there exists $M$, such that for all $m \geq M$ and all $j \in \mathbb{Z}_+$ there holds $\|P_N d^m - P_N d^{m+j}\| < \epsilon$.

Hence we obtain for $m \geq M$ and all $j \in \mathbb{Z}_+$
 \begin{eqnarray*}
  \|d^m - d^{m+j}\| \leq 3 \epsilon.
\end{eqnarray*}
Therefore the sequence $d^m$ satisfies the Cauchy condition. Since $H$ is complete (assumption \NC1) there exists $d \in H$, such that $d^m \to d$. Since $W$
is closed, we conclude that $d \in W$.
\qed

From the proof of the above lemma it is  easy to infer the following result.
\begin{lemma}
\label{lem:convW-gss}
  Assume that $H$ is \gss  and $W \subset H$ is compact. Then, a sequence $c^k$ in $W$ is convergent to $c \in W$, iff for all $j\in \mathbb{Z}_+$ $\lim_{k \to \infty} c^k_j=c_j$.
\end{lemma}

\subsection{Linear forms on \gss}

The following lemma gives a characterization of continuous linear forms on \gss.
\begin{lemma}
\label{lem:formOnH}
Assume that $H$ is \gss and let $f:H \to \mathbb{R}$ be bounded linear functional. Put $f_j=f(e_j)$. Then, for every $a \in H$ there holds
\begin{eqnarray}
   \sum_{j \in \mathbb{Z}_+} |f_j| \cdot  |a_j| \leq \|f\| \cdot \|a\|  \label{eq:fkak-absconver}&\qquad \text{and}\\
   f(a)=\sum_{j \in \mathbb{Z}_+} f_j a_j.   \label{eq:funcLinonH}
\end{eqnarray}
\end{lemma}
\textbf{Proof:} Put $K=\|f\|$.
   Let $\bar{a} \in H$ be such that $|a_j|=|\bar{a}_j|$ and $f_j \bar{a}_j \geq 0$ for all $j$. From \NC3 we have $\bar{a} \in H$ and $\|a\|=\|\bar{a}\|$.  Since each term $f_j \bar{a}_j$ is nonnegative, using \NC4 we obtain
\begin{eqnarray*}
 K\|a\| = K \|\bar{a}\| \geq   K \|P_n \bar{a}\|  \geq    f(P_n \bar{a}) =   \sum_{j \leq n}  f_j  \cdot  \bar{a}_j =  \sum_{j \leq n}  |f_j|  \cdot  |a_j|
\end{eqnarray*}
hence we obtained (\ref{eq:fkak-absconver}).

Equation (\ref{eq:funcLinonH}) is a consequence of the continuity of $f$ and the assumption \NC2.

\qed

\subsection{Bounded linear self-maps on \gss}

%Let $H$ be a \gss space.  The goal of this section is to give a characterization of $\mbox{Lin}(H,H)$.

\begin{theorem}
\label{thm:lin-gss}
Let $H$ be \gss space.
Assume that $V:H \to H$ is bounded linear map. Then there exists a collection of real numbers $\{V_{ij}\}_{i,j \in \mathbb{Z}_+^2}$, such that for each $a \in H$ there holds
\begin{eqnarray*}
  \sum_{j} |V_{ij}| \cdot |a_j| &\leq& \|\pi_i V\| \cdot \|a\|,  \quad i \in \mathbb{Z}_+ \\
  Va &=& \sum_{i} \left(\sum_{j}V_{ij}a_j\right)e_i.
\end{eqnarray*}
\end{theorem}
\textbf{Proof:}
We apply \NC2 and Lemma~\ref{lem:formOnH} to each $\pi_i V$, which are bounded linear forms.
\qed

In view of the above theorem we will often represent $V \in \mbox{Lin}(H,H)$ as a matrix  $V=\{V_{ij}\} \in \mathbb{R}^{\mathbb{Z}_+ \times \mathbb{Z}_+}$ and by $V_{\ast j}$ we will denote its $j$-th column.
