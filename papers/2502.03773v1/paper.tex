%%%%%%%% ICML 2025 EXAMPLE LATEX SUBMISSION FILE %%%%%%%%%%%%%%%%%

\documentclass{article}

% Recommended, but optional, packages for figures and better typesetting:
\usepackage{microtype}
\usepackage{graphicx}
\usepackage{subfigure}
\usepackage{booktabs} % for professional tables

% hyperref makes hyperlinks in the resulting PDF.
% If your build breaks (sometimes temporarily if a hyperlink spans a page)
% please comment out the following usepackage line and replace
% \usepackage{icml2025} with \usepackage[nohyperref]{icml2025} above.
\usepackage{hyperref}

\usepackage{url}
\usepackage{comment}
\usepackage{amsthm}
\usepackage{tcolorbox}
\usepackage{amsmath}
\usepackage{paralist}
\usepackage{varwidth}
\usepackage{float}

% Attempt to make hyperref and algorithmic work together better:
\newcommand{\theHalgorithm}{\arabic{algorithm}}

% Use the following line for the initial blind version submitted for review:
%\usepackage{icml2025}

% If accepted, instead use the following line for the camera-ready submission:
\usepackage[accepted]{icml2025}

% For theorems and such
\usepackage{amsmath}
\usepackage{amssymb}
\usepackage{mathtools}
\usepackage{amsthm}

% if you use cleveref..
\usepackage[capitalize,noabbrev]{cleveref}

%%%%%%%%%%%%%%%%%%%%%%%%%%%%%%%%
% THEOREMS
%%%%%%%%%%%%%%%%%%%%%%%%%%%%%%%%
\theoremstyle{plain}
\newtheorem{theorem}{Theorem}[section]
\newtheorem{proposition}[theorem]{Proposition}
\newtheorem{lemma}[theorem]{Lemma}
\newtheorem{corollary}[theorem]{Corollary}
\theoremstyle{definition}
\newtheorem{definition}[theorem]{Definition}
\newtheorem{assumption}[theorem]{Assumption}
\theoremstyle{remark}
\newtheorem{remark}[theorem]{Remark}

% Todonotes is useful during development; simply uncomment the next line
%    and comment out the line below the next line to turn off comments
%\usepackage[disable,textsize=tiny]{todonotes}
\usepackage[textsize=tiny]{todonotes}


% The \icmltitle you define below is probably too long as a header.
% Therefore, a short form for the running title is supplied here:


\icmltitlerunning{ExpProof : Operationalizing Explanations for Confidential Models with ZKPs}
\begin{document}
\usepackage{bbm}
\usepackage{graphicx}
\usepackage{amsmath,amssymb,amsthm,amsfonts}

\usepackage{paralist}
\usepackage{bm}
\usepackage{xspace}
\usepackage{url}
\usepackage{prettyref}
\usepackage{boxedminipage}
\usepackage{wrapfig}
\usepackage{ifthen}
\usepackage{color}
\usepackage{xspace}

\newcommand{\ii}{{\sc Indicator-Instance}\xspace}
\newcommand{\midd}{{\sf mid}}


\usepackage{amsmath,amsthm,amsfonts,amssymb}
\usepackage{mathtools}
\usepackage{graphicx}


% \usepackage{fullpage}

\usepackage{nicefrac}

\newtheorem{inftheorem}{Informal Theorem}
\newtheorem{claim}{Claim}
\newtheorem*{definition*}{Definition}
\newtheorem{example}{Example}

\DeclareMathOperator*{\argmax}{arg\,max}
\DeclareMathOperator*{\argmin}{arg\,min}
\usepackage{subcaption}

\newtheorem{problem}{Problem}
\usepackage[utf8]{inputenc}
\newcommand{\rank}{\mathsf{rank}}
\newcommand{\tr}{\mathsf{Tr}}
\newcommand{\tv}{\mathsf{TV}}
\newcommand{\opt}{\mathsf{OPT}}
\newcommand{\rr}{\textsc{R}\space}
\newcommand{\alg}{\textsf{Alg}\space}
\newcommand{\sd}{\textsf{sd}_\lambda}
\newcommand{\lblq}{\mathfrak{lq} (X_1)}
\newcommand{\diag}{\textsf{diag}}
\newcommand{\sign}{\textsf{sgn}}
\newcommand{\BC}{\texttt{BC} }
\newcommand{\MM}{\texttt{MM} }
\newcommand{\Nexp}{N_{\mathrm{exp}}}
\newcommand{\Nrep}{N_{\mathrm{replay}}}
\newcommand{\Drep}{D_{\mathrm{replay}}}
\newcommand{\Nsim}{N_{\mathrm{sim}}}
\newcommand{\piBC}{\pi^{\texttt{BC}}}
\newcommand{\piRE}{\pi^{\texttt{RE}}}
\newcommand{\piEMM}{\pi^{\texttt{MM}}}
\newcommand{\mmd}{\texttt{Mimic-MD} }
\newcommand{\RE}{\texttt{RE} }
\newcommand{\dem}{\pi^E}
\newcommand{\Rlint}{\mathcal{R}_{\mathrm{lin,t}}}
\newcommand{\Rlipt}{\mathcal{R}_{\mathrm{lip,t}}}
\newcommand{\Rlin}{\mathcal{R}_{\mathrm{lin}}}
\newcommand{\Rlip}{\mathcal{R}_{\mathrm{lip}}}
\newcommand{\Rmax}{R_{\mathrm{max}}}
\newcommand{\Rall}{\mathcal{R}_{\mathrm{all}}}
\newcommand{\Rdet}{\mathcal{R}_{\mathrm{det}}}
\newcommand{\Fmax}{F_{\mathrm{max}}}
\newcommand{\Nmax}{\mathcal{N}_{\mathrm{max}}}
\newcommand{\piref}{\pi^{\mathrm{ref}}}
\newcommand{\green}{\text{\color{green!75!black} green}\;}
\newcommand{\thetaBC}{\widehat{\theta}^{\textsf{BC}}}
\newcommand{\ent}{\mathcal{E}_{\Theta,n,\delta}}
\newcommand{\eNt}{\mathcal{E}_{\Theta_t,\Nexp,\delta}}
\newcommand{\eNtH}{\mathcal{E}_{\Theta_t,\Nexp,\delta/H}}

\newcommand{\eref}[1]{(\ref{#1})}
\newcommand{\sref}[1]{Sec. \ref{#1}}
\newcommand{\dr}{\widehat{d}_{\mathrm{replay}}}
\newcommand{\figref}[1]{Fig. \ref{#1}}

\usepackage{xcolor}
\definecolor{expert}{HTML}{008000}
\definecolor{error}{HTML}{f96565}
\newcommand{\GKS}[1]{{\textcolor{violet}{\textbf{GKS: #1}}}}
\newcommand{\Q}[1]{{\textcolor{red}{\textbf{Question #1}}}}
\newcommand{\ZSW}[1]{{\textcolor{orange}{\textbf{ZSW: #1}}}}
\newcommand{\JAB}[1]{{\textcolor{teal}{\textbf{JAB: #1}}}}
\newcommand{\jab}[1]{{\textcolor{teal}{\textbf{JAB: #1}}}}
\newcommand{\SAN}[1]{{\textcolor{blue}{\textbf{SC: #1}}}}
\newcommand{\scnote}[1]{\SAN{#1}}
\newcommand{\norm}[1]{\left\lVert #1 \right\rVert}

\usepackage{color-edits}
\addauthor{sw}{blue}

\usepackage{thmtools}
\usepackage{thm-restate}

\usepackage{tikz}
\usetikzlibrary{arrows,calc} 
\newcommand{\tikzAngleOfLine}{\tikz@AngleOfLine}
\def\tikz@AngleOfLine(#1)(#2)#3{%
\pgfmathanglebetweenpoints{%
\pgfpointanchor{#1}{center}}{%
\pgfpointanchor{#2}{center}}
\pgfmathsetmacro{#3}{\pgfmathresult}%
}

\declaretheoremstyle[
    headfont=\normalfont\bfseries, 
    bodyfont = \normalfont\itshape]{mystyle} 
\declaretheorem[name=Theorem,style=mystyle,numberwithin=section]{thm}

% \usepackage{algorithm}
% \usepackage{algorithmic}
\usepackage[linesnumbered,algoruled,boxed,lined,noend]{algorithm2e}

\usepackage{listings}
\usepackage{amsmath}
\usepackage{amsthm}
\usepackage{tikz}
\usepackage{caption}
\usepackage{mdwmath}
\usepackage{multirow}
\usepackage{mdwtab}
\usepackage{eqparbox}
\usepackage{multicol}
\usepackage{amsfonts}
\usepackage{tikz}
\usepackage{multirow,bigstrut,threeparttable}
\usepackage{amsthm}
\usepackage{bbm}
\usepackage{epstopdf}
\usepackage{mdwmath}
\usepackage{mdwtab}
\usepackage{eqparbox}
\usetikzlibrary{topaths,calc}
\usepackage{latexsym}
\usepackage{cite}
\usepackage{amssymb}
\usepackage{bm}
\usepackage{amssymb}
\usepackage{graphicx}
\usepackage{mathrsfs}
\usepackage{epsfig}
\usepackage{psfrag}
\usepackage{setspace}
\usepackage[%dvips,
            CJKbookmarks=true,
            bookmarksnumbered=true,
            bookmarksopen=true,
%						bookmarks=false,
            colorlinks=true,
            citecolor=red,
            linkcolor=blue,
            anchorcolor=red,
            urlcolor=blue
            ]{hyperref}
%\usepackage{algorithm}
\usepackage[linesnumbered,algoruled,boxed,lined]{algorithm2e}
\usepackage{algpseudocode}
\usepackage{stfloats}
\RequirePackage[numbers]{natbib}

\usepackage{comment}
\usepackage{mathtools}
\usepackage{blkarray}
\usepackage{multirow,bigdelim,dcolumn,booktabs}

\usepackage{xparse}
\usepackage{tikz}
\usetikzlibrary{calc}
\usetikzlibrary{decorations.pathreplacing,matrix,positioning}

\usepackage[T1]{fontenc}
\usepackage[utf8]{inputenc}
\usepackage{mathtools}
\usepackage{blkarray, bigstrut}
\usepackage{gauss}

\newenvironment{mygmatrix}{\def\mathstrut{\vphantom{\big(}}\gmatrix}{\endgmatrix}

\newcommand{\tikzmark}[1]{\tikz[overlay,remember picture] \node (#1) {};}

%% Adapted form https://tex.stackexchange.com/questions/206898/braces-for-cases-in-tabular-environment/207704#207704
\newcommand*{\BraceAmplitude}{0.4em}%
\newcommand*{\VerticalOffset}{0.5ex}%  
\newcommand*{\HorizontalOffset}{0.0em}% 
\newcommand*{\blocktextwid}{3.0cm}%
\NewDocumentCommand{\InsertLeftBrace}{%
	O{} % #1 = draw options
	O{\HorizontalOffset,\VerticalOffset} % #2 = optional brace shift options
	O{\blocktextwid} % #3 = optional text width
	m   % #4 = top tikzmark
	m   % #5 = bottom tikzmark
	m   % #6 = node text
}{%
	\begin{tikzpicture}[overlay,remember picture]
	\coordinate (Brace Top)    at ($(#4.north) + (#2)$);
	\coordinate (Brace Bottom) at ($(#5.south) + (#2)$);
	\draw [decoration={brace, amplitude=\BraceAmplitude}, decorate, thick, draw=black, #1]
	(Brace Bottom) -- (Brace Top) 
	node [pos=0.5, anchor=east, align=left, text width=#3, color=black, xshift=\BraceAmplitude] {#6};
	\end{tikzpicture}%
}%
\NewDocumentCommand{\InsertRightBrace}{%
	O{} % #1 = draw options
	O{\HorizontalOffset,\VerticalOffset} % #2 = optional brace shift options
	O{\blocktextwid} % #3 = optional text width
	m   % #4 = top tikzmark
	m   % #5 = bottom tikzmark
	m   % #6 = node text
}{%
	\begin{tikzpicture}[overlay,remember picture]
	\coordinate (Brace Top)    at ($(#4.north) + (#2)$);
	\coordinate (Brace Bottom) at ($(#5.south) + (#2)$);
	\draw [decoration={brace, amplitude=\BraceAmplitude}, decorate, thick, draw=black, #1]
	(Brace Top) -- (Brace Bottom) 
	node [pos=0.5, anchor=west, align=left, text width=#3, color=black, xshift=\BraceAmplitude] {#6};
	\end{tikzpicture}%
}%
\NewDocumentCommand{\InsertTopBrace}{%
	O{} % #1 = draw options
	O{\HorizontalOffset,\VerticalOffset} % #2 = optional brace shift options
	O{\blocktextwid} % #3 = optional text width
	m   % #4 = top tikzmark
	m   % #5 = bottom tikzmark
	m   % #6 = node text
}{%
	\begin{tikzpicture}[overlay,remember picture]
	\coordinate (Brace Top)    at ($(#4.west) + (#2)$);
	\coordinate (Brace Bottom) at ($(#5.east) + (#2)$);
	\draw [decoration={brace, amplitude=\BraceAmplitude}, decorate, thick, draw=black, #1]
	(Brace Top) -- (Brace Bottom) 
	node [pos=0.5, anchor=south, align=left, text width=#3, color=black, xshift=\BraceAmplitude] {#6};
	\end{tikzpicture}%
}%

\usetikzlibrary{patterns}

\definecolor{cof}{RGB}{219,144,71}
\definecolor{pur}{RGB}{186,146,162}
\definecolor{greeo}{RGB}{91,173,69}
\definecolor{greet}{RGB}{52,111,72}

% provide arXiv number if available:
% \arxiv{cs.IT/1502.00326}

% put your definitions there:

%\newtheorem{remark}{Remark} \def\remref#1{Remark~\ref{#1}}
%\newtheorem{conjecture}{Conjecture} \def\remref#1{Remark~\ref{#1}}
%\newtheorem{example}{Example}

%\theorembodyfont{\itshape}
%\newtheorem{theorem}{Theorem}
%\newtheorem{proposition}{Proposition}
%\newtheorem{lemma}{Lemma} \def\lemref#1{Lemma~\ref{#1}}
%\newtheorem{corollary}{Corollary}


%\theorembodyfont{\rmfamily}
%\newtheorem{definition}{Definition}
%\numberwithin{equation}{section}
% \theoremstyle{plain}
% \newtheorem{theorem}{Theorem}
% \newtheorem{Example}{Example}
% \newtheorem{lemma}{Lemma}
% \newtheorem{remark}{Remark}
% \newtheorem{corollary}{Corollary}
% \newtheorem{definition}{Definition}
% \newtheorem{conjecture}{Conjecture}
% \newtheorem{question}{Question}
% \newtheorem*{induction}{Induction Hypothesis}
% \newtheorem*{folklore}{Folklore}
% \newtheorem{assumption}{Assumption}

\def \by {\bar{y}}
\def \bx {\bar{x}}
\def \bh {\bar{h}}
\def \bz {\bar{z}}
\def \cF {\mathcal{F}}
\def \bP {\mathbb{P}}
\def \bE {\mathbb{E}}
\def \bR {\mathbb{R}}
\def \bF {\mathbb{F}}
\def \cG {\mathcal{G}}
\def \cM {\mathcal{M}}
\def \cB {\mathcal{B}}
\def \cN {\mathcal{N}}
\def \var {\mathsf{Var}}
\def\1{\mathbbm{1}}
\def \FF {\mathbb{F}}


\newenvironment{keywords}
{\bgroup\leftskip 20pt\rightskip 20pt \small\noindent{\bfseries
Keywords:} \ignorespaces}%
{\par\egroup\vskip 0.25ex}
\newlength\aftertitskip     \newlength\beforetitskip
\newlength\interauthorskip  \newlength\aftermaketitskip















%%%%%%%%%%%%%%%%%%%%%%%%%%%% by Wu %%%%%%%%%%%%%%%%%%%%%%%%%%%%
\usepackage{xspace}

\newcommand{\Lip}{\mathrm{Lip}}
\newcommand{\stepa}[1]{\overset{\rm (a)}{#1}}
\newcommand{\stepb}[1]{\overset{\rm (b)}{#1}}
\newcommand{\stepc}[1]{\overset{\rm (c)}{#1}}
\newcommand{\stepd}[1]{\overset{\rm (d)}{#1}}
\newcommand{\stepe}[1]{\overset{\rm (e)}{#1}}
\newcommand{\stepf}[1]{\overset{\rm (f)}{#1}}


\newcommand{\floor}[1]{{\left\lfloor {#1} \right \rfloor}}
\newcommand{\ceil}[1]{{\left\lceil {#1} \right \rceil}}

\newcommand{\blambda}{\bar{\lambda}}
\newcommand{\reals}{\mathbb{R}}
\newcommand{\naturals}{\mathbb{N}}
\newcommand{\integers}{\mathbb{Z}}
\newcommand{\Expect}{\mathbb{E}}
\newcommand{\expect}[1]{\mathbb{E}\left[#1\right]}
\newcommand{\Prob}{\mathbb{P}}
\newcommand{\prob}[1]{\mathbb{P}\left[#1\right]}
\newcommand{\pprob}[1]{\mathbb{P}[#1]}
\newcommand{\intd}{{\rm d}}
\newcommand{\TV}{{\sf TV}}
\newcommand{\LC}{{\sf LC}}
\newcommand{\PW}{{\sf PW}}
\newcommand{\htheta}{\hat{\theta}}
\newcommand{\eexp}{{\rm e}}
\newcommand{\expects}[2]{\mathbb{E}_{#2}\left[ #1 \right]}
\newcommand{\diff}{{\rm d}}
\newcommand{\eg}{e.g.\xspace}
\newcommand{\ie}{i.e.\xspace}
\newcommand{\iid}{i.i.d.\xspace}
\newcommand{\fracp}[2]{\frac{\partial #1}{\partial #2}}
\newcommand{\fracpk}[3]{\frac{\partial^{#3} #1}{\partial #2^{#3}}}
\newcommand{\fracd}[2]{\frac{\diff #1}{\diff #2}}
\newcommand{\fracdk}[3]{\frac{\diff^{#3} #1}{\diff #2^{#3}}}
\newcommand{\renyi}{R\'enyi\xspace}
\newcommand{\lpnorm}[1]{\left\|{#1} \right\|_{p}}
\newcommand{\linf}[1]{\left\|{#1} \right\|_{\infty}}
\newcommand{\lnorm}[2]{\left\|{#1} \right\|_{{#2}}}
\newcommand{\Lploc}[1]{L^{#1}_{\rm loc}}
\newcommand{\hellinger}{d_{\rm H}}
\newcommand{\Fnorm}[1]{\lnorm{#1}{\rm F}}
%% parenthesis
\newcommand{\pth}[1]{\left( #1 \right)}
\newcommand{\qth}[1]{\left[ #1 \right]}
\newcommand{\sth}[1]{\left\{ #1 \right\}}
\newcommand{\bpth}[1]{\Bigg( #1 \Bigg)}
\newcommand{\bqth}[1]{\Bigg[ #1 \Bigg]}
\newcommand{\bsth}[1]{\Bigg\{ #1 \Bigg\}}
\newcommand{\xxx}{\textbf{xxx}\xspace}
\newcommand{\toprob}{{\xrightarrow{\Prob}}}
\newcommand{\tolp}[1]{{\xrightarrow{L^{#1}}}}
\newcommand{\toas}{{\xrightarrow{{\rm a.s.}}}}
\newcommand{\toae}{{\xrightarrow{{\rm a.e.}}}}
\newcommand{\todistr}{{\xrightarrow{{\rm D}}}}
\newcommand{\eqdistr}{{\stackrel{\rm D}{=}}}
\newcommand{\iiddistr}{{\stackrel{\text{\iid}}{\sim}}}
%\newcommand{\var}{\mathsf{var}}
\newcommand\indep{\protect\mathpalette{\protect\independenT}{\perp}}
\def\independenT#1#2{\mathrel{\rlap{$#1#2$}\mkern2mu{#1#2}}}
\newcommand{\Bern}{\text{Bern}}
\newcommand{\Poi}{\mathsf{Poi}}
\newcommand{\iprod}[2]{\left \langle #1, #2 \right\rangle}
\newcommand{\Iprod}[2]{\langle #1, #2 \rangle}
\newcommand{\indc}[1]{{\mathbf{1}_{\left\{{#1}\right\}}}}
\newcommand{\Indc}{\mathbf{1}}
\newcommand{\regoff}[1]{\textsf{Reg}_{\mathcal{F}}^{\text{off}} (#1)}
\newcommand{\regon}[1]{\textsf{Reg}_{\mathcal{F}}^{\text{on}} (#1)}

\definecolor{myblue}{rgb}{.8, .8, 1}
\definecolor{mathblue}{rgb}{0.2472, 0.24, 0.6} % mathematica's Color[1, 1--3]
\definecolor{mathred}{rgb}{0.6, 0.24, 0.442893}
\definecolor{mathyellow}{rgb}{0.6, 0.547014, 0.24}


\newcommand{\red}{\color{red}}
\newcommand{\blue}{\color{blue}}
\newcommand{\nb}[1]{{\sf\blue[#1]}}
\newcommand{\nbr}[1]{{\sf\red[#1]}}

\newcommand{\tmu}{{\tilde{\mu}}}
\newcommand{\tf}{{\tilde{f}}}
\newcommand{\tp}{\tilde{p}}
\newcommand{\tilh}{{\tilde{h}}}
\newcommand{\tu}{{\tilde{u}}}
\newcommand{\tx}{{\tilde{x}}}
\newcommand{\ty}{{\tilde{y}}}
\newcommand{\tz}{{\tilde{z}}}
\newcommand{\tA}{{\tilde{A}}}
\newcommand{\tB}{{\tilde{B}}}
\newcommand{\tC}{{\tilde{C}}}
\newcommand{\tD}{{\tilde{D}}}
\newcommand{\tE}{{\tilde{E}}}
\newcommand{\tF}{{\tilde{F}}}
\newcommand{\tG}{{\tilde{G}}}
\newcommand{\tH}{{\tilde{H}}}
\newcommand{\tI}{{\tilde{I}}}
\newcommand{\tJ}{{\tilde{J}}}
\newcommand{\tK}{{\tilde{K}}}
\newcommand{\tL}{{\tilde{L}}}
\newcommand{\tM}{{\tilde{M}}}
\newcommand{\tN}{{\tilde{N}}}
\newcommand{\tO}{{\tilde{O}}}
\newcommand{\tP}{{\tilde{P}}}
\newcommand{\tQ}{{\tilde{Q}}}
\newcommand{\tR}{{\tilde{R}}}
\newcommand{\tS}{{\tilde{S}}}
\newcommand{\tT}{{\tilde{T}}}
\newcommand{\tU}{{\tilde{U}}}
\newcommand{\tV}{{\tilde{V}}}
\newcommand{\tW}{{\tilde{W}}}
\newcommand{\tX}{{\tilde{X}}}
\newcommand{\tY}{{\tilde{Y}}}
\newcommand{\tZ}{{\tilde{Z}}}

\newcommand{\sfa}{{\mathsf{a}}}
\newcommand{\sfb}{{\mathsf{b}}}
\newcommand{\sfc}{{\mathsf{c}}}
\newcommand{\sfd}{{\mathsf{d}}}
\newcommand{\sfe}{{\mathsf{e}}}
\newcommand{\sff}{{\mathsf{f}}}
\newcommand{\sfg}{{\mathsf{g}}}
\newcommand{\sfh}{{\mathsf{h}}}
\newcommand{\sfi}{{\mathsf{i}}}
\newcommand{\sfj}{{\mathsf{j}}}
\newcommand{\sfk}{{\mathsf{k}}}
\newcommand{\sfl}{{\mathsf{l}}}
\newcommand{\sfm}{{\mathsf{m}}}
\newcommand{\sfn}{{\mathsf{n}}}
\newcommand{\sfo}{{\mathsf{o}}}
\newcommand{\sfp}{{\mathsf{p}}}
\newcommand{\sfq}{{\mathsf{q}}}
\newcommand{\sfr}{{\mathsf{r}}}
\newcommand{\sfs}{{\mathsf{s}}}
\newcommand{\sft}{{\mathsf{t}}}
\newcommand{\sfu}{{\mathsf{u}}}
\newcommand{\sfv}{{\mathsf{v}}}
\newcommand{\sfw}{{\mathsf{w}}}
\newcommand{\sfx}{{\mathsf{x}}}
\newcommand{\sfy}{{\mathsf{y}}}
\newcommand{\sfz}{{\mathsf{z}}}
\newcommand{\sfA}{{\mathsf{A}}}
\newcommand{\sfB}{{\mathsf{B}}}
\newcommand{\sfC}{{\mathsf{C}}}
\newcommand{\sfD}{{\mathsf{D}}}
\newcommand{\sfE}{{\mathsf{E}}}
\newcommand{\sfF}{{\mathsf{F}}}
\newcommand{\sfG}{{\mathsf{G}}}
\newcommand{\sfH}{{\mathsf{H}}}
\newcommand{\sfI}{{\mathsf{I}}}
\newcommand{\sfJ}{{\mathsf{J}}}
\newcommand{\sfK}{{\mathsf{K}}}
\newcommand{\sfL}{{\mathsf{L}}}
\newcommand{\sfM}{{\mathsf{M}}}
\newcommand{\sfN}{{\mathsf{N}}}
\newcommand{\sfO}{{\mathsf{O}}}
\newcommand{\sfP}{{\mathsf{P}}}
\newcommand{\sfQ}{{\mathsf{Q}}}
\newcommand{\sfR}{{\mathsf{R}}}
\newcommand{\sfS}{{\mathsf{S}}}
\newcommand{\sfT}{{\mathsf{T}}}
\newcommand{\sfU}{{\mathsf{U}}}
\newcommand{\sfV}{{\mathsf{V}}}
\newcommand{\sfW}{{\mathsf{W}}}
\newcommand{\sfX}{{\mathsf{X}}}
\newcommand{\sfY}{{\mathsf{Y}}}
\newcommand{\sfZ}{{\mathsf{Z}}}


\newcommand{\calA}{{\mathcal{A}}}
\newcommand{\calB}{{\mathcal{B}}}
\newcommand{\calC}{{\mathcal{C}}}
\newcommand{\calD}{{\mathcal{D}}}
\newcommand{\calE}{{\mathcal{E}}}
\newcommand{\calF}{{\mathcal{F}}}
\newcommand{\calG}{{\mathcal{G}}}
\newcommand{\calH}{{\mathcal{H}}}
\newcommand{\calI}{{\mathcal{I}}}
\newcommand{\calJ}{{\mathcal{J}}}
\newcommand{\calK}{{\mathcal{K}}}
\newcommand{\calL}{{\mathcal{L}}}
\newcommand{\calM}{{\mathcal{M}}}
\newcommand{\calN}{{\mathcal{N}}}
\newcommand{\calO}{{\mathcal{O}}}
\newcommand{\calP}{{\mathcal{P}}}
\newcommand{\calQ}{{\mathcal{Q}}}
\newcommand{\calR}{{\mathcal{R}}}
\newcommand{\calS}{{\mathcal{S}}}
\newcommand{\calT}{{\mathcal{T}}}
\newcommand{\calU}{{\mathcal{U}}}
\newcommand{\calV}{{\mathcal{V}}}
\newcommand{\calW}{{\mathcal{W}}}
\newcommand{\calX}{{\mathcal{X}}}
\newcommand{\calY}{{\mathcal{Y}}}
\newcommand{\calZ}{{\mathcal{Z}}}

\newcommand{\bara}{{\bar{a}}}
\newcommand{\barb}{{\bar{b}}}
\newcommand{\barc}{{\bar{c}}}
\newcommand{\bard}{{\bar{d}}}
\newcommand{\bare}{{\bar{e}}}
\newcommand{\barf}{{\bar{f}}}
\newcommand{\barg}{{\bar{g}}}
\newcommand{\barh}{{\bar{h}}}
\newcommand{\bari}{{\bar{i}}}
\newcommand{\barj}{{\bar{j}}}
\newcommand{\bark}{{\bar{k}}}
\newcommand{\barl}{{\bar{l}}}
\newcommand{\barm}{{\bar{m}}}
\newcommand{\barn}{{\bar{n}}}
\newcommand{\baro}{{\bar{o}}}
\newcommand{\barp}{{\bar{p}}}
\newcommand{\barq}{{\bar{q}}}
\newcommand{\barr}{{\bar{r}}}
\newcommand{\bars}{{\bar{s}}}
\newcommand{\bart}{{\bar{t}}}
\newcommand{\baru}{{\bar{u}}}
\newcommand{\barv}{{\bar{v}}}
\newcommand{\barw}{{\bar{w}}}
\newcommand{\barx}{{\bar{x}}}
\newcommand{\bary}{{\bar{y}}}
\newcommand{\barz}{{\bar{z}}}
\newcommand{\barA}{{\bar{A}}}
\newcommand{\barB}{{\bar{B}}}
\newcommand{\barC}{{\bar{C}}}
\newcommand{\barD}{{\bar{D}}}
\newcommand{\barE}{{\bar{E}}}
\newcommand{\barF}{{\bar{F}}}
\newcommand{\barG}{{\bar{G}}}
\newcommand{\barH}{{\bar{H}}}
\newcommand{\barI}{{\bar{I}}}
\newcommand{\barJ}{{\bar{J}}}
\newcommand{\barK}{{\bar{K}}}
\newcommand{\barL}{{\bar{L}}}
\newcommand{\barM}{{\bar{M}}}
\newcommand{\barN}{{\bar{N}}}
\newcommand{\barO}{{\bar{O}}}
\newcommand{\barP}{{\bar{P}}}
\newcommand{\barQ}{{\bar{Q}}}
\newcommand{\barR}{{\bar{R}}}
\newcommand{\barS}{{\bar{S}}}
\newcommand{\barT}{{\bar{T}}}
\newcommand{\barU}{{\bar{U}}}
\newcommand{\barV}{{\bar{V}}}
\newcommand{\barW}{{\bar{W}}}
\newcommand{\barX}{{\bar{X}}}
\newcommand{\barY}{{\bar{Y}}}
\newcommand{\barZ}{{\bar{Z}}}

\newcommand{\hX}{\hat{X}}
\newcommand{\Ent}{\mathsf{Ent}}
\newcommand{\awarm}{{A_{\text{warm}}}}
\newcommand{\thetaLS}{{\widehat{\theta}^{\text{\rm LS}}}}

\newcommand{\jiao}[1]{\langle{#1}\rangle}
\newcommand{\gaht}{\textsc{GoodActionHypTest}\;}
\newcommand{\iaht}{\textsc{InitialActionHypTest}\;}
\newcommand{\true}{\textsf{True}\;}
\newcommand{\false}{\textsf{False}\;}

% \usepackage[capitalize,noabbrev]{cleveref}
% \crefname{lemma}{Lemma}{Lemmas}
% \Crefname{lemma}{Lemma}{Lemmas}
% \crefname{thm}{Theorem}{Theorems}
% \Crefname{thm}{Theorem}{Theorems}
% \Crefname{assumption}{Assumption}{Assumptions}
% \Crefname{inftheorem}{Informal Theorem}{Informal Theorems}
% \crefformat{equation}{(#2#1#3)}

% % if you use cleveref..
% \usepackage[capitalize,noabbrev]{cleveref}
% \crefname{lemma}{Lemma}{Lemmas}
% \crefname{proposition}{Proposition}{Propositions}
% \crefname{remark}{Remark}{Remarks}
% \crefname{corollary}{Corollary}{Corollaries}
% \crefname{definition}{Definition}{Definitions}
% \crefname{conjecture}{Conjecture}{Conjectures}
% \crefname{figure}{Fig.}{Figures}

\twocolumn[
\icmltitle{\textit{ExpProof} : Operationalizing Explanations for Confidential Models with ZKPs}
%Provable Explanations for Confidential Models
% It is OKAY to include author information, even for blind
% submissions: the style file will automatically remove it for you
% unless you've provided the [accepted] option to the icml2025
% package.

% List of affiliations: The first argument should be a (short)
% identifier you will use later to specify author affiliations
% Academic affiliations should list Department, University, City, Region, Country
% Industry affiliations should list Company, City, Region, Country

% You can specify symbols, otherwise they are numbered in order.
% Ideally, you should not use this facility. Affiliations will be numbered
% in order of appearance and this is the preferred way.
\icmlsetsymbol{equal}{*}

\begin{icmlauthorlist}
\icmlauthor{Chhavi Yadav}{equal,yyy}
\icmlauthor{Evan Monroe Laufer}{equal,comp}
\icmlauthor{Dan Boneh}{comp}
\icmlauthor{Kamalika Chaudhuri}{yyy}

%\icmlauthor{}{sch}
%\icmlauthor{}{sch}
\end{icmlauthorlist}

\icmlaffiliation{yyy}{ UC San Diego}
\icmlaffiliation{comp}{Stanfard University}
%\icmlaffiliation{sch}{School of ZZZ, Institute of WWW, Location, Country}

\icmlcorrespondingauthor{Chhavi Yadav}{cyadav@ucsd.edu}
\icmlcorrespondingauthor{Evan Monroe Laufer}{emlaufer@stanford.edu}

% You may provide any keywords that you
% find helpful for describing your paper; these are used to populate
% the "keywords" metadata in the PDF but will not be shown in the document
\icmlkeywords{Machine Learning, ICML}

\vskip 0.3in
]

% this must go after the closing bracket ] following \twocolumn[ ...

% This command actually creates the footnote in the first column
% listing the affiliations and the copyright notice.
% The command takes one argument, which is text to display at the start of the footnote.
% The \icmlEqualContribution command is standard text for equal contribution.
% Remove it (just {}) if you do not need this facility.

%\printAffiliationsAndNotice{}  % leave blank if no need to mention equal contribution
\printAffiliationsAndNotice{\icmlEqualContribution} % otherwise use the standard text.

\begin{abstract}
% \begin{figure*}[!htp]
%     \centering
%     \includegraphics[width=\textwidth]{icml2024/figures/intro.pdf}
%     \caption{SayAnything performs audio-driven lip synchronization through video editing, demonstrating zero-shot generalization to in-the-wild and various style domains without fine-tuning. Our fusion scheme eliminates the dependency on additional supervision signals like SyncNet for lip synchronization. More video results are available in the supplementary materials.}
%     \label{fig:teaser}
% \end{figure*}

\begin{abstract}
Recent advances in diffusion models have led to significant progress in audio-driven lip synchronization. However, existing methods typically rely on constrained audio-visual alignment priors or multi-stage learning of intermediate representations to force lip motion synthesis. This leads to complex training pipelines and limited motion naturalness. In this paper, we present SayAnything, a conditional video diffusion framework that directly synthesizes lip movements from audio input while preserving speaker identity. Specifically, we propose three specialized modules, including an identity preservation module, an audio guidance module, and an editing control module. Our novel design effectively balances different condition signals in the latent space, enabling precise control over appearance, motion, and region-specific generation without requiring additional supervision signals or intermediate representations. Extensive experiments demonstrate that SayAnything generates highly realistic videos with improved lip-teeth coherence, enabling unseen characters to \textbf{say anything} while effectively generalizing to animated characters.
\end{abstract}
\vspace{-1cm}
\end{abstract}

\section{Introduction}


\begin{figure}[t]
\centering
\includegraphics[width=0.6\columnwidth]{figures/evaluation_desiderata_V5.pdf}
\vspace{-0.5cm}
\caption{\systemName is a platform for conducting realistic evaluations of code LLMs, collecting human preferences of coding models with real users, real tasks, and in realistic environments, aimed at addressing the limitations of existing evaluations.
}
\label{fig:motivation}
\end{figure}

\begin{figure*}[t]
\centering
\includegraphics[width=\textwidth]{figures/system_design_v2.png}
\caption{We introduce \systemName, a VSCode extension to collect human preferences of code directly in a developer's IDE. \systemName enables developers to use code completions from various models. The system comprises a) the interface in the user's IDE which presents paired completions to users (left), b) a sampling strategy that picks model pairs to reduce latency (right, top), and c) a prompting scheme that allows diverse LLMs to perform code completions with high fidelity.
Users can select between the top completion (green box) using \texttt{tab} or the bottom completion (blue box) using \texttt{shift+tab}.}
\label{fig:overview}
\end{figure*}

As model capabilities improve, large language models (LLMs) are increasingly integrated into user environments and workflows.
For example, software developers code with AI in integrated developer environments (IDEs)~\citep{peng2023impact}, doctors rely on notes generated through ambient listening~\citep{oberst2024science}, and lawyers consider case evidence identified by electronic discovery systems~\citep{yang2024beyond}.
Increasing deployment of models in productivity tools demands evaluation that more closely reflects real-world circumstances~\citep{hutchinson2022evaluation, saxon2024benchmarks, kapoor2024ai}.
While newer benchmarks and live platforms incorporate human feedback to capture real-world usage, they almost exclusively focus on evaluating LLMs in chat conversations~\citep{zheng2023judging,dubois2023alpacafarm,chiang2024chatbot, kirk2024the}.
Model evaluation must move beyond chat-based interactions and into specialized user environments.



 

In this work, we focus on evaluating LLM-based coding assistants. 
Despite the popularity of these tools---millions of developers use Github Copilot~\citep{Copilot}---existing
evaluations of the coding capabilities of new models exhibit multiple limitations (Figure~\ref{fig:motivation}, bottom).
Traditional ML benchmarks evaluate LLM capabilities by measuring how well a model can complete static, interview-style coding tasks~\citep{chen2021evaluating,austin2021program,jain2024livecodebench, white2024livebench} and lack \emph{real users}. 
User studies recruit real users to evaluate the effectiveness of LLMs as coding assistants, but are often limited to simple programming tasks as opposed to \emph{real tasks}~\citep{vaithilingam2022expectation,ross2023programmer, mozannar2024realhumaneval}.
Recent efforts to collect human feedback such as Chatbot Arena~\citep{chiang2024chatbot} are still removed from a \emph{realistic environment}, resulting in users and data that deviate from typical software development processes.
We introduce \systemName to address these limitations (Figure~\ref{fig:motivation}, top), and we describe our three main contributions below.


\textbf{We deploy \systemName in-the-wild to collect human preferences on code.} 
\systemName is a Visual Studio Code extension, collecting preferences directly in a developer's IDE within their actual workflow (Figure~\ref{fig:overview}).
\systemName provides developers with code completions, akin to the type of support provided by Github Copilot~\citep{Copilot}. 
Over the past 3 months, \systemName has served over~\completions suggestions from 10 state-of-the-art LLMs, 
gathering \sampleCount~votes from \userCount~users.
To collect user preferences,
\systemName presents a novel interface that shows users paired code completions from two different LLMs, which are determined based on a sampling strategy that aims to 
mitigate latency while preserving coverage across model comparisons.
Additionally, we devise a prompting scheme that allows a diverse set of models to perform code completions with high fidelity.
See Section~\ref{sec:system} and Section~\ref{sec:deployment} for details about system design and deployment respectively.



\textbf{We construct a leaderboard of user preferences and find notable differences from existing static benchmarks and human preference leaderboards.}
In general, we observe that smaller models seem to overperform in static benchmarks compared to our leaderboard, while performance among larger models is mixed (Section~\ref{sec:leaderboard_calculation}).
We attribute these differences to the fact that \systemName is exposed to users and tasks that differ drastically from code evaluations in the past. 
Our data spans 103 programming languages and 24 natural languages as well as a variety of real-world applications and code structures, while static benchmarks tend to focus on a specific programming and natural language and task (e.g. coding competition problems).
Additionally, while all of \systemName interactions contain code contexts and the majority involve infilling tasks, a much smaller fraction of Chatbot Arena's coding tasks contain code context, with infilling tasks appearing even more rarely. 
We analyze our data in depth in Section~\ref{subsec:comparison}.



\textbf{We derive new insights into user preferences of code by analyzing \systemName's diverse and distinct data distribution.}
We compare user preferences across different stratifications of input data (e.g., common versus rare languages) and observe which affect observed preferences most (Section~\ref{sec:analysis}).
For example, while user preferences stay relatively consistent across various programming languages, they differ drastically between different task categories (e.g. frontend/backend versus algorithm design).
We also observe variations in user preference due to different features related to code structure 
(e.g., context length and completion patterns).
We open-source \systemName and release a curated subset of code contexts.
Altogether, our results highlight the necessity of model evaluation in realistic and domain-specific settings.





\section{Preliminaries} \label{sec:prelims}
Before diving into the technical results, we state the basic graph notations used throughout the paper and recap the new non-standard definitions we have introduced throughout \Cref{sec:overview}.

\paragraph{Graphs.}
Throughout we consider directed simple graphs $G = (V, E)$, where $E \subseteq V^2$, with $n = |V|$ nodes and $m = |E|$ edges. The edges of the graph can be associated with some value: a length $\ell(e)$ or a capacity/cost $c(e)$, all of which we require to be positive. For any $U \subseteq V$, we write $\overline U = V \setminus U$. Let $G[U]$ be the subgraph induced by $U$. We denote with $\delta^{+}(U)$ the set of edges that have their starting point in $U$ and endpoint in~$\overline U$. We define $\delta^{-}(U)$ symmetrically. We also sometimes write $c(S) = \sum_{e \in S} c(e)$ (for a set of edges $S$) or $c(U, W) = \sum_{e \in E \cap (U \times W)} c(e)$ and $c(U) = c(U, U)$ (for sets of nodes $U, W$).

The distance between two nodes $v$ and $u$ is written $d_G(v,u)$ (throughout we consider only the \emph{length} functions to be relevant for distances). We may omit the subscript if it is clear from the context. The diameter of the graph is the maximum distance between any pair of nodes. For a subgraph $G'$ of $G$ we occasionally say that~$G'$ has \emph{weak diameter} $D$ if for all pairs of nodes $u, v$ in~$G'$, we have $d_G(u, v), d_G(v, u) \leq D$. A strongly connected component in a directed graph $G$ is a subgraph where for every pair of nodes $v,u$ there is a path from $v$ to $u$ and vise versa. Finally, for a radius $r \geq 0$ we write $B^+(v, r) = \set{x \in V : d_G(v, x) \leq r}$ and $B^-(v, r) = \set{y \in V : d_G(y, v) \leq r}$.


\paragraph{Polynomial Bounds.}
For graphs with edge lengths (or capacities), we assume that they are positive and the maximum edge length is bounded by $\poly(n)$. This is only for the sake of simplicity in \cref{sec:ldd-expander,sec:ldd-deterministic} (where in the more general case that all edge lengths are bounded by some threshold $W$ some logarithmic factors in $n$ become $\log (nW)$ instead), and is not necessary for our strongest LDD developed in \cref{sec:ldd-fast}.

\paragraph{Expander Graphs.}
Let $G = (V, E, \ell, c)$ be a directed graph with positive edge capacities $c$ and positive unit edge lengths $\ell$. We define the \emph{volume $\vol(U)$} by
\begin{equation*}
	\vol(U) = c(U, V) = \sum_{e \in E \cap (U \times V)} c(e),
\end{equation*}
and set $\minvol(U) = \min\set{\vol(U), \vol(\overline U)}$ where $\overline U = V \setminus U$. A node set $U$ naturally corresponds to a cut $(U, \overline U)$. The \emph{sparsity} (or \emph{conductance}) of $U$ is defined by
\begin{equation*}
	\phi(U) = \frac{c(U, \overline U)}{\minvol(U)}.
\end{equation*}
In the special cases that $U = \emptyset$ we set $\phi(U) = 1$ and in the special case that $U \neq \emptyset$ but $\vol(U) = 0$, we set $\phi(U) = 0$.
We say that $U$ is \emph{$\phi$-sparse} if $\phi(U) \leq \phi$. We say that a directed graph is a $\phi$-expander if it does not contain a $\phi$-sparse cut $U \subseteq V$. 
We define the \emph{lopsided sparsity} of $U$ as
\begin{equation*}
	\psi(U) = \frac{c(U, \overline U)}{\minvol(U) \cdot \log \frac{\vol(V)}{\minvol(U)}},
\end{equation*}
(with similar special cases), and we similarly say that $U$ is \emph{$\psi$-lopsided sparse} if $\psi(U) \leq \psi$. Finally, we call a graph a \emph{$\psi$-lopsided expander} if it does not contain a $\psi$-lopsided sparse cut $U \subseteq V$.



\section{Problem Setting \& Desiderata for Solution}\label{sec:probsol}



%By virtue of aiming to make models transparent,
%\cy{shorten this and put more focus on the formal setting onwards}
%Explanations are intended as a way to improve trust in ML models by virtue of making them transparent. Consequently, other than benign debugging applications, they are also seen as an answer to concerns regarding discrimination, recourse and correctness of predictions \cite{}. However, many of these use-cases involve parties with misaligned incentives, which leads to failure of explanations as a trust-enhancing tool. For instance, consider a bank which denies loan to an applicant based on an ML model's prediction and now has to return an explanation to the applicant justifying the model's prediction. Since the explanation can be used by the applicant to prove discrimination in the court of law, the bank is incentivized to return an \textit{incontestable} explanation rather than reveal the true working of the model.

%Such adversarial manipulations are exacerbated by conditions in which models are deployed, an important one being confidentiality of models. For instance, a bank which trains a loan prediction model on sensitive data will keep its model confidential due to IP and legal reasons. But confidentiality now enables the bank to potentially swap the model at its choice of customers without being caught. \cite{slack2020fooling} show that under confidentiality the model developer can use different models for in vs. out of distribution points and generate innocuous explanations even though the model was biased to begin with. Similar is the case for auditors, who determine the correctness of explanations usually with API access to the hidden model. In the absence of perpetual testing, nothing stops the bank from changing the model post-auditing even if the model and explanations were found to be unproblematic by the audit initially.

%Additionally, explanation algorithms themselves are often not fully deterministic and involve hyperparameters which can be tampered with while seeming completely benign. For instance, if the bank uses LIME explanations, it could learn the interpretable model in LIME with non-uniform and adversarial samples for minority groups such that the resulting explanations make discriminatory predictions look safe \cite{}. It could also pick different values of bandwidth parameter for minority groups as discussed in Sec. \ref{sec:advlime}. Adversarial interpretable models can also be learnt by solving an optimization problem while being good approximations to the original model as shown in \cite{shahin2022washing}.

%A plausible solution to some of these issues could be consistency checks. However, doing such checks is infeasible for individual customers as these involve collecting multiple pairs of explanations and predictions over different queries to get correct answers with high probability \cite{}. More importantly, here the onus of testing the correctness of explanations \textit{completely} lies on the customers -- this makes for a rather irrational and lopsided ask for multiple reasons including the fact that the customer is a layman in many cases and even when not, may not have the resources to test.

%\textit{Note that the above issues persist even with a perfectly faithful algorithm for generating explanations.}

To recall, explanations fail as a trust-enhancing tool in adversarial use-cases and can lead to a false sense of security while benefiting adversaries. Motivated by these problems, we investigate if a technical solution can be designed to operationalize explanations in adversarial settings.


\textbf{Formal Problem Setting.} Formally, a model owner confidentially holds a classification model $f$ which is not publicly released due to legal and IP reasons. A customer supplies an input $x$ to the model owner, who responds with a prediction $f(x)$ and an explanation $\Ef$  where $\E$ is the possibly-randomized algorithm generating the explanation. %This explanation can be verified by the customer. The customer is also guaranteed that the same model is used for everyone.

%The explanation algorithm $\E$ may be randomized.

\textbf{Solution Desiderata.} A technical solution to operationalize explanations in adversarial use-cases should provide the following guarantees.

\begin{enumerate}
    \item (Model Uniformity) the same model $f$ is used by the model owner for all inputs  : our solution is to use cryptographic commitments which force the model owner to commit to a model prior to receiving inputs,
    \item (Explanation Correctness) the explanation algorithm $\E$ is run correctly for generating explanations for all inputs : our solution is to use Zero-Knowledge Proofs, wherein the model owner supplies a cryptographic proof of correctness to be verified by the customer in a computationally feasible manner,
    \item (Model Consistency) the same model $f$ is used for inference and generating explanations : this is ensured by generating inference and explanations as a part of the same system and by using model commitments,
    \item (Model Confidentiality) the model $f$ is kept confidential in the sense that any technique for guaranteeing (1)-(3) does not leak anything else about the hidden model $f$ than is already leaked by predictions $f(x)$ and explanations $\Ef$ without using the technique : this comes as a by-product of using ZKPs and commitments (See Sec. \ref{app:subsec:secproof} for the formal theorem and proof),
    \item (Technique Reliability) the technique used for guaranteeing (1)-(4) is sound and complete (as in Sec.\ref{sec:prelims}): this comes as a by-product of using ZKPs and commitments (See Sec. \ref{app:subsec:secproof} for the formal theorem and proof).
\end{enumerate}

Our solution \name which provides the above guarantees will be discussed in Sec. \ref{sec:verifylime}.
\section{Variants of LIME}\label{sec:varlime}

%In this section, we propose different variants of LIME with the aim of identifying more ZKP-amenable designs by evaluating and comparing their overheads later on in Sec.\ref{sec:expts}.

Building zero-knowledge proofs of explanations requires the explanation algorithm to be implemented in a ZKP library\footnote{More precisely, arithmetic circuits for the explanation algorithm are implemented in the ZKP library.} which is known to introduce a significant computational overhead. Given this, a natural question that comes to mind is if there exist variants of LIME which provide similar quality of explanations but are more ZKP-amenable by design, meaning they introduce a smaller ZKP overhead?

\textbf{Standard LIME Variants.}~To create variants of standard LIME (Alg.\ref{alg:limeinclear}), we focus on the two steps which are carried out numerous times and hence create a computational bottleneck in the LIME algorithm -- sampling around input $x$ (Step 6 in Alg. \ref{alg:limeinclear}) and computing distance using exponential kernel (Step 7 in Alg. \ref{alg:limeinclear}). For sampling, we propose two options as found in the literature : gaussian (G) and uniform (U)  \cite{ribeiro2016should, garreau2020explaining, garreau2020looking}. For the kernel we propose to either use the exponential (E) kernel or no (N) kernel. These choices give rise to four variants of LIME, mentioned in Alg. \ref{alg:limevarinclear}. We address each variant by the intials in the brackets, for instance standard LIME with uniform sampling and no kernel is addressed as `LIME\_U+N'.

%propose some other variants of LIME and test if they can provide similar quality 

%\paragraph{Border LIME.} Traditionally LIME is created for the trusted setting where model developers are benign. However as mentioned earlier, the settings in which post-hoc explanations are meant to be used are adversarial in nature, which necessitates a version of LIME which is robust to adversarial manipulations. Next we briefly mention the threat model followed by the attack and defense for one such manipulation and additional concerns to robustify LIME.
%hold a non-malicious model and to set the parameters of LIME without adversarial incentives

%\textit{Threat Model.} The adversary in our case is the model developer. The adversary has access to the full training data and trains the model, but does not have access to the test inputs (which may come from a different distribution than the training distribution). Upon receiving a test input, it returns a prediction and an explanation. Parameters of the explanation algorithm are also set by the adversary. \cy{should refer to model developer as adversary or model developer?adversary should be mentioned at beginning or end?}

%The adversary here is the model developer itself who has access to the full training data and trains the model. Since explanations are provided by the adversary, it also gets to set the parameters of the explanation algorithm. However the adversary does not know the kind of inputs it will receive at test time. Having received an input from the user, the adversary returns a corresponding prediction and an explanation. %This adversary has its own incentives due to which it can create malicious explanations, for instance a bank not wanting to give loans to a minority group might give unfair loan application predictions and wishes to hide the unfairness with explanations.

%Now we will discuss some plausible adversarial manipulations and ways to prevent them with ZKPs and commitments.

%\textit{Attack1 and Defense.} The adversary can use different values for hyperparameters in LIME for different inputs to create seemingly innocuous explanations. For instance let us consider the bandwith parameter $\sigma$ of the similarity kernel in LIME, which controls the size of the neighborhood with large weights around the input point. A large value of the parameter means a large region around the input will be highly weighted and as a result the generated explanation tends to the global explanation with increasing sampling radius. This phenomena can be exploited by the adversary to hide discrimination \cy{shouldnt talk too much about fairness?} apparent in local explanations
%by outputting global explanations which may seem innocuous \cite{}. Our framework can rule out this kind of attack by (1) having the model developer cryptographically commit to the parameters of the explanation algorithm  apriori, which enforces that the same value of the parameters are used for all input points (this value can come from regulatory suggestions), or (2) having the customer (aka verifier) supply the value of the parameters to be used by the model developer.

%, but had it output the local explanation, the discrimination could have been apparent in the explanation 

%giving discriminatory predictions for a minority group in a part of the input space; it can hide its intent by setting a large value of the bandwidth parameter and 

%Additionally, the parameters of LIME can be set in an adversarial fashion. Consider the bandwidth parameter $\sigma$ of the similarity kernel which plays a key role in producing faithful explanations -- this is evidenced by many studies in the explanation literature which study its effect on the generated explanation and explore ways to set its value \cite{}. A small value of the bandwidth parameter results in a small neighborhood around the input point with high weights, thereby leading to unstable explanations (as all the sampled points are very similar to the input making it harder to learn a classifier). On the other hand, a large value of the parameter with a large neighborhood around the input implies that a global explanation is being learnt, rather than a local one, resulting in unfaithful explanations. While prior studies majorly look at the bandwidth parameter from the lens of faithfulness of generated explanations, we observe that an adversary can use this parameter to generate explanations that match its incentives. For instance, a model developer giving discriminatory predictions for a minority group in a part of the input space can hide its intent by setting a large value of the bandwidth parameter and outputting global explanation which may be seemingly innocuous, but had it output the local explanation, the discrimination could have been apparent in the explanation. To eliminate the potential for such attacks, we propose that either (1) the prover cryptographically commit to a value of the bandwidth parameter apriori, which enforces that the same value of the bandwidth parameter is used for all input points (this value can come from regulatory suggestions), or (2) the verifier supply a value of the bandwidth parameter to be used by the prover. Note that the same recommendations can apply to other parameters.

%Certain traditional considerations for LIME become more critical when viewed through the lens of adversarial manipulation. We describe these considerations below and give solutions to deal with them in our ZKP system.

\textbf{BorderLIME.} An important consideration for generating meaningful local explanations is that the sampled neighborhood should contain points from different classes \cite{laugel2018defining}. Any reasonable neighborhood for an input far off from the decision boundary will only contain samples from the same class, resulting in vacuous explanations.


To remedy the problem, \cite{laugel2017inverse, laugel2018defining} propose a \textit{radial} search algorithm, which finds the closest point to the input $x$ belonging to a different class, $x_{border}$, and then uses $x_{border}$ as the input to LIME (instead of original input $x$). Their algorithm incrementally grows (or shrinks) a search area radially from the input point and relies on random sampling within each `ring' (or sphere), looking for points with an opposite label. To cryptographically prove this algorithm, we would either have to reimplement the algorithm as-is or would have to give a probabilistic security guarantee (using a concentration inequality), both of which would require many classifier calls and thereby many proofs of inference, becoming inefficient in a ZKP system.


We transform their algorithm into a line search version, called BorderLIME, given in Alg. \ref{alg:robustLIME_highlevel} and \ref{alg:findclosestpoint}, using the notion of Stability Radius which is now fed as a parameter to the algorithm. The stability radius for an input $x$, $\delta_x$, is defined as the 
largest radius for which the model prediction remains unchanged within a ball of that radius around the input \( x \). The stability radius \( \delta \) is defined as the minimum stability radius across all inputs \( x \) sampled from the data distribution $D$. Formally,  
$
\delta = \inf_{x \sim \mathcal{D}} \delta_x, \quad \text{where} \quad \delta_x = \sup \{ r \geq 0 \mid f(x') = f(x), \forall x' \in \mathcal{B}(x, r) \}
$. Here \( \mathcal{B}(x, r) = \{ x' \mid \|x' - x\| \leq r \} \) denotes a ball of radius \( r \) centered at \( x \). Stability radius ensures that for any input from the data distribution, the model's prediction remains stable within at least a radius of \( \delta \).

Our algorithm samples $m$ directions and then starting from the original input $x$, takes $\delta$ steps until it finds a point with a different label along all these directions individually. The border point $x_{border}$ is that oppositely labeled point which is closest to the input $x$. Furthermore, unlike in the algorithm in \cite{laugel2017inverse}, our algorithm can exploit parallelization by searching along the different directions in parallel since these are independent.

%This algorithm uses the notion of Stability Radius $\delta$, such that the model prediction remains the same in a ball $\mathcal{B}(x, \delta)$ of radius $\delta$ around any input $x$ from the data distribution.  Formally, $\delta = \sup \{ r \geq 0 \mid f(x') = f(x), \forall x' \in \mathcal{B}(x, r) \}$ where \( \mathcal{B}(x, r) = \{ x' \mid \|x' - x\| \leq r \} \) denotes the ball of radius \( r \) centered at \( x \). largest radius for which the model prediction remains unchanged within a ball $\mathcal{B}(x, \delta)$ of radius $\delta$ around any input \( x \) from the data distribution. 

Determining the optimal value of the stability radius is an interesting research question, but it is not the focus of this work. We leave an in-depth exploration of this topic to future work while providing some high-level directions and suggestions next. Stability radius can (and perhaps should) be found \textit{offline} using techniques as proposed in \cite{jordan2019provable, yadav2024fairproofconfidentialcertifiable} or through an offline empirical evaluation on in-distribution points. A ZK proof for this radius can be generated one-time, in an offline manner and supplied by the model developer (for NNs see \cite{yadav2024fairproofconfidentialcertifiable}). It can also be  pre-committed to by the model developer (see Sec. \ref{sec:verifylime}).  

%If the stability radius is reasonable enough (meaning that the model is not too sensitive), the algorithm will only go through a few iterations. and finding a reasonable stability radius by starting with a value and reducing it iteratively; in a real-world setting if the test points are in-distribution, the stability radius found in such a way will work.


\begin{algorithm}[tbh]
\begin{algorithmic}[1]
 \caption{\textsc{BorderLIME}}
   \label{alg:robustLIME_highlevel}
    
    \STATE {\bfseries Input:} Input point $x$, Classifier $f$
    \STATE {\bfseries Parameters:} Number of points $n$ to be sampled around input point, Length of explanation $K$, Bandwidth parameter $\sigma$ for similarity kernel

    \STATE  {\bfseries Output:} Explanation $e$
    \STATE
    \STATE $x_{border}:=$\\\hspace{2em}\textsc{Find\_Closest\_Point\_With\_Opp\_Label}($x, f$) \hfill \textcolor{blue}{$\rhd$} See Alg. \ref{alg:findclosestpoint}
    \STATE $e :=$ \textsc{LIME}($x_{border}, f$) \textcolor{blue}{$\rhd$} Note that any variant of LIME can be used here
    \STATE Return Explanation $e$
\end{algorithmic}
\end{algorithm}

\begin{algorithm}[tbh]
\begin{algorithmic}[1]
 \caption{\textsc{Find\_Closest\_Point\_With\_Opp\_Label}}
   \label{alg:findclosestpoint}
    
    \STATE {\bfseries Input:} Input point $x$, Classifier $f$
    \STATE {\bfseries Parameters:} Number of random directions $m$, Stability radius $\delta$, Iteration Threshold $T$

    \STATE  {\bfseries Output:} Opposite label point $x_{border}$
    \STATE
    \STATE $\left\{\vec{u}_0, \vec{u}_1 \cdots \vec{u}_{m-1}\right\}:=$ Sample $m$ random directions
    \STATE Initialize $\textsc{$dist_0$} \cdots \textsc{$dist_{m-1}$}$ as $\inf$
    \FOR{$\vec{u}_i \in \left\{\vec{u}_0, \vec{u}_1 \cdots \vec{u}_{m-1}\right\}$}
    \STATE $x_{border_i} := x$
    \STATE $iter := 0$
    \WHILE{$f(x_{border_i}) == f(x)$ and $iter \leq T$}
    \STATE $x_{border_i} := x_{border_i}+ \delta \vec{u}_i$
    \STATE $iter := iter + 1$
    \ENDWHILE
    \IF{$f(x_{border_i}) != f(x)$}
    \STATE \textsc{$dist_i$} $:= \ell_2{(x, x_{border_i})}$
    \ENDIF
    \ENDFOR
    \STATE $x_{border}:= x_{border_i}$ \textrm{such that} $i:= \arg \min dist_i$
    \STATE Return $x_{border}$
\end{algorithmic}
\end{algorithm}

%\textit{Discussion on other kinds of attacks.} The model developer can train the model such that the model is very unstable in particular regions of the input space corresponding to minority classes, yet the LIME explanation will not be able to catch this manipulation. To remedy this problem, we propose to obtain a stability radius for the input point such the model prediction does not change in a ball of stability radius around the input point and check that the radius is above a threshold. A ZK proof for this radius can be supplied by the model developer along with the LIME ZK proof such as shown in \cite{yadav2024fairproofconfidentialcertifiable}.

%Lastly, adversarial manipulation can happen during the model training itself such that the model is trained to create innocuous explanations while giving biased predictions. Here usually a regularization term corresponding to the manipulated explanation is added to the loss function \cite{aivodji2019fairwashing, yadav2024influence}. Preventing such attacks requires ZK proof of training; however this is outside the scope of this paper and we refer an interested reader to \cite{garg2023experimenting, abbaszadeh2024zero}.

%This kind of manipulation requires other kinds of ZKP solutions. 


%\cy{have a section on limitations - our thing does test time doesnt prevent training time manipulations}Other kinds of manipulation where a term corresponding to the explanation is added to the loss function \cite{aivodji2019fairwashing, yadav2024influence} can be addressed with a ZK proof of training; however this is outside the scope of this paper and we refer an interested reader to \cite{garg2023experimenting, abbaszadeh2024zero}.

%However, this technique only gets the closest point in one direction -- the model makes its prediction due to the decision boundaries in multiple directions \cy{I dont think this argument is correct -- one direction might be enough to make the prediction}. Our resulting algorithm is called robustLIME and is given in Alg. \ref{alg:robustLIME}.

%\cy{COMMENTS on DAN's version of the algorithm: instead of picking the minimum distance $x_{border}$ Line 13 of Alg. \ref{alg:findclosestpoint} and line 5 of \ref{alg:robustLIME_highlevel}, dan's algorithm uses all the points that we came across along all directions to fit the linear model. This is wrong for 2 reasons. Firstly, this will be a highly imbalanced dataset with only m points of the opposite class (incase we wish to set m to number of dimensions, maximum number of dimensions in our datasets is 24, which might be too expensive) -- hence it essentially doesn't solve the problem that we had in the first place. Secondly, I don't think fitting a linear model across all directions is correct -- I think the linear model should be based on the closest point, not on all the directions. Do you have a pictorial example where the decision would be affected by more direction that the closest point one??}
\section{\name : Verification of Explanations}\label{sec:verifylime}
%\cy{give names to all versions of lime, 4.1 commitment, 4.2 Overview: verification- talk about the 2 versions, 4.3 verification key steps used in all algorithms}


Our system for operationalizing explanations in adversarial settings, \name, consists of two phases: (1) a One-time Commitment phase and (2) an Online verification phase which should be executed for every input.

\textbf{Commitment Phase.} To ensure model uniformity, the model owner cryptographically commits to a fixed set of model weights $\mathbf{W}$ belonging to the original model $f$, resulting in committed weights $\CW$. Architecture of model $f$ is assumed to be public. Additionally, model owner can also commit to the values of different parameters used in the explanation algorithm or these parameters can be public.%\cy{EVAN: what exactly about the architecture should be known? is the architecture public?} \el{Chhavi: yes, the architecture is completely public. The only thing that is hidden is the model weights. Granted, we can add that if there were methods to make zksnarks for model inference that hid the architecture (like a universal circuit of sorts, not sure if any papers exist on this), then our methods still are applicable)}



\textbf{Online Verification Phase.} This phase is executed every time a customer inputs a query. On receiving the query, the prover (bank) outputs a prediction, an explanation and a zero-knowledge proof of the explanation. Verifier (customer) validates the proof without looking at the model weights. If the proof passes verification, it means that the explanation is correctly computed with the committed model weights and explanation algorithm parameters.

To generate the explanation proof, a ZKP circuit which implements (a variant of) LIME is required. However since ZKPs can be computationally inefficient, instead of reimplementing the algorithm as-is in a ZKP library, we devise some smart strategies for verification, based on the fact that verification can be easier than redoing the computation. Since all the variants of LIME share some common functionalities, we next describe how the verification strategies for these functionalities. For more details on the verification for each variant, see Appendix Sec. \ref{app:sec:appexpproof}.

%has four modular functionalities, as described below.
%instead, leading to the ZKP version of LIME, called zkLIME App. Alg. \ref{}. 
%Note that we have different versions of LIME as proposed in Sec. \ref{sec:varlime} and one of these can be chosen but all of them share some similar steps.


\textit{1. Verifying Sampling (Alg.~\ref{alg:zk_check_poseidon}, \ref{alg:zk_uniform_sample}, \ref{alg:zk_gaussian_sample}).} We use the Poseidon~\cite{poseidon} hash function to generate random samples. As part of the setup phase, the prover commits to a random value $r_p$. When submitting an input for explanation, the
verifier sends another random value $r_v$. Prover generates uniformly sampled points using Poseidon with
a key $r_p + r_v$, which is uniformly random in the view of both the prover and the verifier. Then, during the proof generation phase, the prover proves that the sampled points are the correct outputs from Poseidon using \textit{ezkl}'s inbuilt efficient Poseidon verification circuit. We convert the uniform samples into Gaussian
samples using the inverse CDF, which is checked in the proof using a look-up table for the inverse CDF.
%; the prover uses this circuit along with private inputs (e.g. model weights) to generate a proof, while the verifier checks the proof’s validity without seeing the private inputs
%\cy{EVAN TO DO : how? using what?}. \el{This is using a poseidon circuit. We use one that ezkl uses already, not entirely sure on the details, but also not sure if we should discuss it. maybe I should just say that there is an efficient poseidon verification circuit? I guess to clarify more, poseidon is a hash function that was created expressly for this purpose: to use in SNARKs for efficient hash proofs}

\textit{2. Verifying Exponential Kernel (Alg.~\ref{alg:zk_exponential_kernel}).} ZKP libraries do not support many non-linear functions such as exponential, which is used for the similarity kernel in LIME (Step 5 of Alg.\ref{alg:limeinclear}). To resolve this problem, we implement a look-up table for the exponential function and prove that the exponential value is correct by comparing it with the value from the look-up table.

\textit{3. Verifying Inference.} Since LIME requires predictions for the sampled points in order to learn the linear explanation, we must verify that the predictions are correct. To generate proofs for correct predictions, we use \textit{ezkl}'s inbuilt inference verification circuit. %\cy{EVAN: Is this correct?}%, which is an efficient ZKP engine for doing inferences on deep learning models.

\textit{4. Verifying LASSO Solution (Alg.~\ref{alg:zk_lasso}).} ZKP libraries only accept integers and hence all floating points have to be quantized. Consequently, the LASSO solution for Step 7 of Alg. \ref{alg:limeinclear} is also quantized in a ZKP library, leading to small scale differences between the exact and quantized solutions. To verify optimality of the quantized LASSO solution, we use the standard concept of duality gap. For a primal objective $l$ and its dual objective $g$, to prove that the objective value from primal feasible $w$ is close to that from the primal optimal $w^*$, that is $l(w) - l(w^*) \leq \epsilon$, the duality gap should be smaller than $\epsilon$ as well, $l(w) - g(u,v) \leq \epsilon$ where $u,v$ are dual feasible. Since the primal and dual of LASSO have closed forms, as long we input any dual feasible values, we can verify that the quantized LASSO solution is close to the LASSO optimal. The prover provides the dual feasible as part of the witness to the proof. See App. Sec.\ref{app:subsec:lassoprimaldual} for closed forms of the primal and dual functions and for the technique to find dual feasible.

%For LASSO, the primal optimal $w^*$ and dual optimal $v^*$ for lasso are linked by the equation $y - Xw^* = \lambda v^*$. Therefore, given a primal feasible $w$ that is close to $w^*$, it is possible to generate a dual feasible $v$ close to $v^*$.\cy{what are y and x}\cy{EVAN : make changess to this}

The complete \name protocol can be found in Alg. \ref{alg:ExpProof}; its security guarantee is given as follows.

\begin{theorem}
(Informal) Given a model $f$ and an input point $x$, \name~returns prediction $f(x)$, LIME explanation $\mathcal{E}(f, x)$ and a ZK proof for the correct computation of the explanation, without leaking anything additional about the weights of model $f$ (in the sense described in Sec.\ref{sec:probsol}).
\end{theorem}

For the complete formal security theorem and proof, refer to App. Sec. \ref{app:subsec:secproof}. The proof follows from inherent properties of ZKPs.

%$min_{w,z} (z-\nu)^{\top} z+\nu^{\top} X w+\lambda\|w\|_1$
%where $z=Xw-y$


%Your techniques for verifying LIME : volume argument for closest point, sampling done in ZKP system, exp implemented in a table lookup, lasso solution : duality gap is small (), top-K -sort and give top k, sampled exactly n points, predictions are verified with EZKL, Theorem for ZKP like in fairproof, NP like what somesh mentioned last time?

%dual: to prove $f(w) - f(w^*) \leq \epsilon$, we have to prove $f(w) - g(u,v) \leq \epsilon$ for any feasible $u,v$. We know $w$ -- the one we use in ZK system, we know the closed form of f and g.

%closest point : if volume of opposite points in the ball around $x$ of radius $\eta$ is $5\%$ and 95\% points have same label and if we need 64-bit security then P(n sampled same labels)$\leq \frac{1}{2^{64}}$ or $(0.95)^n \leq \frac{1}{2^{64}}$ which gives $n=865$ which is not a big number. If opposite labels cover 1\% of the volume then $n=4414$ which is also okay. How do you quantify the volume?
\section{Experiments}
\label{sec:exps}

In this section, we present comprehensive experiment results to evaluate the effectiveness of \texttt{ProDistill} across various settings. Code is available at \url{https://github.com/JingXuTHU/Scalable_Model_Merging_with_Progressive_Layerwise_Distillation}.

\begin{table*}[t]
\setlength{\tabcolsep}{4pt}
\centering
\caption{\textbf{Performance of merging ViT-B-32 models across eight downstream vision tasks.} \texttt{ProDistill} consistently outperforms the baselines under different data availability. The results for Localize-and-Stich are directly taken from~\citet{he2024localize}.}
\label{tab:vitb32}   
\begin{tabular}{l|cccccccc|cc}
\toprule
\textbf{Method} &\textbf{SUN397}& \textbf{Cars}& \textbf{RESISC45}& \textbf{EuroSAT}& \textbf{SVHN}& \textbf{GTSRB}& \textbf{MNIST}& \textbf{DTD} &\textbf{Avg}  \\
\midrule
{Individual}  & 75.34 & 77.73 & 95.98 & 99.89 & 97.46 & 98.73 & 99.69 & 79.36 & 90.52 \\
Task Arithmetic & 55.32 & 54.98 & 66.68 & 78.89 & 80.21 & 69.68 & 97.34 & 50.37 & 69.18 \\
\midrule
RegMean& 67.47 & 66.63 & 81.75 & 93.33 & 86.68 & 79.92 & 97.30 & 60.16 & 79.15 \\
Fisher merging & 63.95 & 63.84 & 66.86 & 83.48 & 79.54 & 60.11 & 91.27 & 49.36 & 69.80 \\
Localize-and-Stich  & 67.20 & 68.30 & 81.80 & 89.40 & 87.90 & 86.60 & 94.80 & 62.90 & 79.90 \\
AdaMerging& 63.69 & 65.74 & 77.65 & 91.00 & 82.48 & 93.12 & 98.27 & 62.29 & 79.28 \\ 
\rowcolor{lightyellow}
\texttt{ProDistill}~(Ours)& \textbf{68.90} & \textbf{71.21} & \textbf{89.89} & \textbf{99.37} & \textbf{96.13} & \textbf{95.29} & \textbf{99.46} & \textbf{68.03} & \textbf{86.04} \\
\bottomrule
\end{tabular}
\end{table*}

\begin{figure*}
    \centering
    \includegraphics[width=1.0\linewidth]{figure/svhn_tsne11.jpg}
    \caption{\textbf{The t-SNE visualization of ViT-B-32 model trained by different merging algorithms, on the SVHN dataset.} The features given by \texttt{ProDistill}  are the most separated, resembling those of fine-tuned models.}
    \label{fig:tsne_svhn}
\end{figure*}

\subsection{Setup}
\label{sec:setup}
We consider three main experimental setups: 
(1) Merging Vision Transformers~\citep{dosovitskiy2020image} on image classification tasks; 
(2) Merging BERT~\citep{devlin2018bert} and RoBERTa~\citep{liu2019roberta} models on natural language understanding~(NLU) tasks; 
(3) Merging LLAMA2~\citep{touvron2023llama2} model on natural language generation~(NLG) tasks. 

\paragraph{Tasks and Models:}
For image classification tasks, we follow the setting in~\citet{ilharco2022editing} and use Vision Transformer~(ViT) models pre-trained on the ImageNet dataset and subsequently fine-tuned on 8 downstream datasets. 
For NLU and NLG tasks, we merge the BERT-base and RoBERTa-base models fine-tuned on 8 NLU tasks from the GLUE~\citep{wang2018glue} benchmark, and perform pairwise merging of WizardLM-13B, WizardMath-13B and llama-2-13b-code-alpaca models, following the setting in~\citep{yu2024language}. 
Detailed information on the models and datasets can be found in Appendix~\ref{apx:dataset}.

\paragraph{Baselines:}
For vision and NLU tasks, we compare \texttt{ProDistill} with a wide range of baselines, including 
Task Arithmetic~\citep{ilharco2022editing}, 
Fisher merging~\citep{matena2022merging},
RegMean~\citep{jin2022dataless}, 
AdaMerging~\citep{yang2023adamerging} and Localize-and-Stich~\citep{he2024localize}. 
All methods, except Task Arithmetic, require a few-shot unlabeled validation dataset, which is randomly sampled from the training set, with validation shot set to 64 per task.
For NLG tasks, we compare \texttt{ProDistill} with Task Arithmetic~\citep{ilharco2022editing}, TIES-Merging~\citep{yadav2024ties} and WIDEN~\citep{yu2024extend}, due to scale constraints.
A detailed discussion of the baselines and their implementations is provided in Appendix~\ref{apx:baselines} and~\ref{apx:impl}.



\subsection{Results on Merging ViT models}
Table~\ref{tab:vitb32} presents the performance of merging ViT-B-32 models across eight downstream vision tasks.
The results for ViT-B-16 and ViT-L-14 are provided in Appendix~\ref{apx: more results}.

Our method consistently outperforms all baselines, yielding significant improvements in average performance.
Specifically, \texttt{ProDistill} achieves an average performance of 86.04\%, surpassing the baselines by 6.14\%. Notably, it is only 4\% below the average performance of the individual fine-tuned models.

We also visualize the final-layer activations of the merged model using t-SNE~\citep{van2008visualizing}. The results are given in Figure~\ref{fig:tsne_svhn} and Appendix~\ref{apx: tsne}. The visualization shows that the features given by \texttt{ProDistill} are more separated compared to the baselines, closely resembling those of the fine-tuned models.




\subsection{Results on Merging Encoder-based Language Models}
Table~\ref{tab:roberta} summarizes the results of merging RoBERTa models fine-tuned on the NLU tasks.
The results of BERT models are deferred to Appendix~\ref{apx: more results}.
Similar to the vision tasks, \texttt{ProDistill} achieves significant performance improvements of 6.61\% on the NLU tasks, outperforming all baselines across nearly all tasks. 

Unlike vision tasks, the NLU tasks in the GLUE benchmark have small class numbers. For example, SUN387 dataset consists of 397 classes, while CoLA only has 2 classes.  This class size disparity limits the performance of methods that operate directly on the model output logits, such as AdaMerging and Fisher merging.
Our method, along with RegMean, performs particularly well, emphasizing the importance of leveraging internal feature embeddings for effective model merging.

\begin{table*}[t]
\centering
\setlength{\tabcolsep}{5pt}
\caption{\textbf{Performance of merging RoBERTa models on the NLU tasks.} \texttt{ProDistill} achieves superior performance across almost all tasks.}
\label{tab:roberta} 
\begin{tabular}{l|cccccccc|cc}
\toprule
\textbf{Method} & \textbf{CoLA} & \textbf{SST-2} & \textbf{MRPC} & \textbf{STS-B} & \textbf{QQP} & \textbf{MNLI} & \textbf{QNLI} & \textbf{RTE} & \textbf{Avg} \\

\midrule
Individual & 0.5458 & 0.9450 & 0.8858 & 0.9030 & 0.8999 & 0.8710 & 0.9244 & 0.7292 & 0.8380\\
Task Arithmetic  & 0.0804 & 0.8475 & 0.7865 & 0.4890 & 0.8133 & 0.7063 & 0.7558 & 0.6534 & 0.6415 \\
\midrule
RegMean & 0.3022 & 0.9255 & 0.8183 & 0.5152 & \textbf{0.8176} & 0.7089 & 0.8503 & 0.6462 & 0.6980 \\
Fisher merging& 0.1633 & 0.7064 & 0.7264 & 0.1274 & 0.6962 & 0.4968 & 0.5599 & 0.5776 & 0.5068 \\
Localize-and-Stich& 0.0464 & 0.8922 & 0.7916 & \textbf{0.7232} & 0.7821 & 0.5709 & 0.7703 & 0.5632 & 0.6425 \\
AdaMerging& 0.000 & 0.8532 & 0.7875 & 0.5483 & 0.8086 & 0.7039 & 0.7247 & 0.6390 & 0.6332 \\
\rowcolor{lightyellow}
\rowcolor{lightyellow}
\rowcolor{lightyellow}
\texttt{ProDistill}~(Ours)& \textbf{0.4442} & \textbf{0.9312} & \textbf{0.8464} & 0.6942 & 0.8134 & \textbf{0.7857} & \textbf{0.8900} & \textbf{0.7076} & \textbf{0.7641} \\
\bottomrule
\end{tabular}
\end{table*}


\subsection{Results on Merging Large Language Models}
We present the results of merging the WizardMath-13B and Llama-2-13B-Code-Alpaca models in~Table~\ref{tab: code_math}, with additional results provided in Appendix~\ref{apx: more results} and generation examples provided in Appendix~\ref{apx: llm example}. These findings demonstrate that our method effectively scales up to models with over 10B parameters, and achieves superior performance compared to baselines.

\begin{table*}[t]
\centering
\caption{\textbf{Performance of merging LLM models on Code and Math tasks.} Our method demonstrates an improved performance and a strong scalability. The results of TIES-Merging and WIDEN are directly taken from~\citet{yu2024extend}.}
\begin{tabular}{l|cccc|cc}
\toprule
\textbf{Method} & \textbf{GSM8K} & \textbf{MATH} & \textbf{HumanEval} & \textbf{MBPP} & \textbf{Avg} & \textbf{Norm Avg} \\

\midrule
WizardMath-13B & 0.6361 & 0.1456 & 0.0671 & 0.0800 & 0.2322 & 0.6430 \\
Llama-2-13b-code-alpaca & 0.000 & 0.000 & 0.2378 & 0.2760 & 0.1285 & 0.5000 \\
\midrule
Task Arithmetic & \textbf{0.6467} & \textbf{0.1462} & 0.0854 & 0.0840 & 0.2406 & 0.6711\\
TIES-Merging & 0.6323 & 0.1356 & 0.0976 & 0.2240 & 0.2723 & 0.7868\\
WIDEN & 0.6422 & 0.1358 & 0.0976 & 0.0980 & 0.2434 & 0.6769\\
\rowcolor{lightyellow}
\texttt{ProDistill} (Ours) & 0.6279 & 0.1424 & \textbf{0.1280} & \textbf{0.2239} & \textbf{0.2806} & \textbf{0.8288}\\
\bottomrule
\end{tabular}
\label{tab: code_math}
\end{table*}
\section{Discussion of Assumptions}\label{sec:discussion}
In this paper, we have made several assumptions for the sake of clarity and simplicity. In this section, we discuss the rationale behind these assumptions, the extent to which these assumptions hold in practice, and the consequences for our protocol when these assumptions hold.

\subsection{Assumptions on the Demand}

There are two simplifying assumptions we make about the demand. First, we assume the demand at any time is relatively small compared to the channel capacities. Second, we take the demand to be constant over time. We elaborate upon both these points below.

\paragraph{Small demands} The assumption that demands are small relative to channel capacities is made precise in \eqref{eq:large_capacity_assumption}. This assumption simplifies two major aspects of our protocol. First, it largely removes congestion from consideration. In \eqref{eq:primal_problem}, there is no constraint ensuring that total flow in both directions stays below capacity--this is always met. Consequently, there is no Lagrange multiplier for congestion and no congestion pricing; only imbalance penalties apply. In contrast, protocols in \cite{sivaraman2020high, varma2021throughput, wang2024fence} include congestion fees due to explicit congestion constraints. Second, the bound \eqref{eq:large_capacity_assumption} ensures that as long as channels remain balanced, the network can always meet demand, no matter how the demand is routed. Since channels can rebalance when necessary, they never drop transactions. This allows prices and flows to adjust as per the equations in \eqref{eq:algorithm}, which makes it easier to prove the protocol's convergence guarantees. This also preserves the key property that a channel's price remains proportional to net money flow through it.

In practice, payment channel networks are used most often for micro-payments, for which on-chain transactions are prohibitively expensive; large transactions typically take place directly on the blockchain. For example, according to \cite{river2023lightning}, the average channel capacity is roughly $0.1$ BTC ($5,000$ BTC distributed over $50,000$ channels), while the average transaction amount is less than $0.0004$ BTC ($44.7k$ satoshis). Thus, the small demand assumption is not too unrealistic. Additionally, the occasional large transaction can be treated as a sequence of smaller transactions by breaking it into packets and executing each packet serially (as done by \cite{sivaraman2020high}).
Lastly, a good path discovery process that favors large capacity channels over small capacity ones can help ensure that the bound in \eqref{eq:large_capacity_assumption} holds.

\paragraph{Constant demands} 
In this work, we assume that any transacting pair of nodes have a steady transaction demand between them (see Section \ref{sec:transaction_requests}). Making this assumption is necessary to obtain the kind of guarantees that we have presented in this paper. Unless the demand is steady, it is unreasonable to expect that the flows converge to a steady value. Weaker assumptions on the demand lead to weaker guarantees. For example, with the more general setting of stochastic, but i.i.d. demand between any two nodes, \cite{varma2021throughput} shows that the channel queue lengths are bounded in expectation. If the demand can be arbitrary, then it is very hard to get any meaningful performance guarantees; \cite{wang2024fence} shows that even for a single bidirectional channel, the competitive ratio is infinite. Indeed, because a PCN is a decentralized system and decisions must be made based on local information alone, it is difficult for the network to find the optimal detailed balance flow at every time step with a time-varying demand.  With a steady demand, the network can discover the optimal flows in a reasonably short time, as our work shows.

We view the constant demand assumption as an approximation for a more general demand process that could be piece-wise constant, stochastic, or both (see simulations in Figure \ref{fig:five_nodes_variable_demand}).
We believe it should be possible to merge ideas from our work and \cite{varma2021throughput} to provide guarantees in a setting with random demands with arbitrary means. We leave this for future work. In addition, our work suggests that a reasonable method of handling stochastic demands is to queue the transaction requests \textit{at the source node} itself. This queuing action should be viewed in conjunction with flow-control. Indeed, a temporarily high unidirectional demand would raise prices for the sender, incentivizing the sender to stop sending the transactions. If the sender queues the transactions, they can send them later when prices drop. This form of queuing does not require any overhaul of the basic PCN infrastructure and is therefore simpler to implement than per-channel queues as suggested by \cite{sivaraman2020high} and \cite{varma2021throughput}.

\subsection{The Incentive of Channels}
The actions of the channels as prescribed by the DEBT control protocol can be summarized as follows. Channels adjust their prices in proportion to the net flow through them. They rebalance themselves whenever necessary and execute any transaction request that has been made of them. We discuss both these aspects below.

\paragraph{On Prices}
In this work, the exclusive role of channel prices is to ensure that the flows through each channel remains balanced. In practice, it would be important to include other components in a channel's price/fee as well: a congestion price  and an incentive price. The congestion price, as suggested by \cite{varma2021throughput}, would depend on the total flow of transactions through the channel, and would incentivize nodes to balance the load over different paths. The incentive price, which is commonly used in practice \cite{river2023lightning}, is necessary to provide channels with an incentive to serve as an intermediary for different channels. In practice, we expect both these components to be smaller than the imbalance price. Consequently, we expect the behavior of our protocol to be similar to our theoretical results even with these additional prices.

A key aspect of our protocol is that channel fees are allowed to be negative. Although the original Lightning network whitepaper \cite{poon2016bitcoin} suggests that negative channel prices may be a good solution to promote rebalancing, the idea of negative prices in not very popular in the literature. To our knowledge, the only prior work with this feature is \cite{varma2021throughput}. Indeed, in papers such as \cite{van2021merchant} and \cite{wang2024fence}, the price function is explicitly modified such that the channel price is never negative. The results of our paper show the benefits of negative prices. For one, in steady state, equal flows in both directions ensure that a channel doesn't loose any money (the other price components mentioned above ensure that the channel will only gain money). More importantly, negative prices are important to ensure that the protocol selectively stifles acyclic flows while allowing circulations to flow. Indeed, in the example of Section \ref{sec:flow_control_example}, the flows between nodes $A$ and $C$ are left on only because the large positive price over one channel is canceled by the corresponding negative price over the other channel, leading to a net zero price.

Lastly, observe that in the DEBT control protocol, the price charged by a channel does not depend on its capacity. This is a natural consequence of the price being the Lagrange multiplier for the net-zero flow constraint, which also does not depend on the channel capacity. In contrast, in many other works, the imbalance price is normalized by the channel capacity \cite{ren2018optimal, lin2020funds, wang2024fence}; this is shown to work well in practice. The rationale for such a price structure is explained well in \cite{wang2024fence}, where this fee is derived with the aim of always maintaining some balance (liquidity) at each end of every channel. This is a reasonable aim if a channel is to never rebalance itself; the experiments of the aforementioned papers are conducted in such a regime. In this work, however, we allow the channels to rebalance themselves a few times in order to settle on a detailed balance flow. This is because our focus is on the long-term steady state performance of the protocol. This difference in perspective also shows up in how the price depends on the channel imbalance. \cite{lin2020funds} and \cite{wang2024fence} advocate for strictly convex prices whereas this work and \cite{varma2021throughput} propose linear prices.

\paragraph{On Rebalancing} 
Recall that the DEBT control protocol ensures that the flows in the network converge to a detailed balance flow, which can be sustained perpetually without any rebalancing. However, during the transient phase (before convergence), channels may have to perform on-chain rebalancing a few times. Since rebalancing is an expensive operation, it is worthwhile discussing methods by which channels can reduce the extent of rebalancing. One option for the channels to reduce the extent of rebalancing is to increase their capacity; however, this comes at the cost of locking in more capital. Each channel can decide for itself the optimum amount of capital to lock in. Another option, which we discuss in Section \ref{sec:five_node}, is for channels to increase the rate $\gamma$ at which they adjust prices. 

Ultimately, whether or not it is beneficial for a channel to rebalance depends on the time-horizon under consideration. Our protocol is based on the assumption that the demand remains steady for a long period of time. If this is indeed the case, it would be worthwhile for a channel to rebalance itself as it can make up this cost through the incentive fees gained from the flow of transactions through it in steady state. If a channel chooses not to rebalance itself, however, there is a risk of being trapped in a deadlock, which is suboptimal for not only the nodes but also the channel.

\section{Conclusion}
This work presents DEBT control: a protocol for payment channel networks that uses source routing and flow control based on channel prices. The protocol is derived by posing a network utility maximization problem and analyzing its dual minimization. It is shown that under steady demands, the protocol guides the network to an optimal, sustainable point. Simulations show its robustness to demand variations. The work demonstrates that simple protocols with strong theoretical guarantees are possible for PCNs and we hope it inspires further theoretical research in this direction.

\section{Related Work}
In the ML field ZKPs have been majorly used for verification of inferences made by models \cite{sun2024zkllm, chen2024zkml, kang2022scaling, PvCNN, sun2023zkdl, Zen, VI2, vCNN, ZKDT, Liu2021zkCNNZK, singh2022zero, fan2023validating}. A line of work also focuses on proving the training of ML models using ZKPs \cite{burkhalter2021rofl, huang2022zkmlaas, ruckel2022fairness, garg2023experimenting, abbaszadeh2024zero}. More recently they're also been used for verifying properties such as fairness \cite{yadav2024fairproofconfidentialcertifiable, confidant, Toreini2023VerifiableFP} and accuracy \cite{zhang2020zero} of confidential ML models.  Contrary to these and to the best of our knowledge, ours is the first work that identifies the need for proving explanations and provides ZKP based solutions for the same.

\section{Conclusion \& Future Work}
In this paper we take a step towards operationalizing explanations in adversarial contexts where the involved parties have misaligned interests. We propose a protocol \name using Commitments and Zero-Knowledge Proofs, which provides guarantees on the model used and correctness of explanations in the face of confidentiality requirements. We propose ZKP-efficient versions of the popular explainability algorithm LIME and demonstrate the feasibility of \name for Neural Networks \& Random Forests.

An interesting avenue for future work is the tailored design of explainability algorithms for high ZKP-efficiency and inherent robustness to adversarial manipulations. Another interesting avenue is finding other applications in ML where ZKPs can ensure verifiable computation and provide trust guarantees without revealing sensitive information.

% \section*{Impact Statement}

% This work takes a step towards operationalizing explanations in adversarial settings where the model is kept confidential from customers. With this work it can be guaranteed that the said committed model (1) is used for all inputs, (2) is used for generating predictions and explanations (3) cannot be swapped post-audits. It can also be guaranteed that the explanation is generated correctly using the said explanation algorithm. All of this is guaranteed while maintaining model confidentiality.

% While our protocol \name provides the above guarantees, in order to have complete trust guarantees we also require stable and faithful explanation algorithms which are robust to realistic adversarial attacks as mentioned in Sec.\ref{sec:discuss}. We mention existing solutions to tackle some of these issues in Sec.\ref{sec:discuss} and Sec.\ref{sec:varlime} and call for more research in these directions. These research directions though interesting and important are out of scope for our work.

\section*{Acknowledgements}
CY and KC would like to thank National Science Foundation NSF (CIF-2402817, CNS-1804829), SaTC-2241100,
CCF-2217058, ARO-MURI (W911NF2110317), and ONR under N00014-24-1-2304 for research support. DB and EL were partially supported by NSF, DARPA, and the Simons Foundation. Opinions, findings, and conclusions or recommendations expressed in this material are those of the authors and do not necessarily reflect the views of DARPA.

% In the unusual situation where you want a paper to appear in the
% references without citing it in the main text, use \nocite
%\nocite{langley00}


\bibliography{paper}
\bibliographystyle{icml2025}


%%%%%%%%%%%%%%%%%%%%%%%%%%%%%%%%%%%%%%%%%%%%%%%%%%%%%%%%%%%%%%%%%%%%%%%%%%%%%%%
%%%%%%%%%%%%%%%%%%%%%%%%%%%%%%%%%%%%%%%%%%%%%%%%%%%%%%%%%%%%%%%%%%%%%%%%%%%%%%%
% APPENDIX
%%%%%%%%%%%%%%%%%%%%%%%%%%%%%%%%%%%%%%%%%%%%%%%%%%%%%%%%%%%%%%%%%%%%%%%%%%%%%%%
%%%%%%%%%%%%%%%%%%%%%%%%%%%%%%%%%%%%%%%%%%%%%%%%%%%%%%%%%%%%%%%%%%%%%%%%%%%%%%%
\newpage
\appendix
\onecolumn
\subsection{Lloyd-Max Algorithm}
\label{subsec:Lloyd-Max}
For a given quantization bitwidth $B$ and an operand $\bm{X}$, the Lloyd-Max algorithm finds $2^B$ quantization levels $\{\hat{x}_i\}_{i=1}^{2^B}$ such that quantizing $\bm{X}$ by rounding each scalar in $\bm{X}$ to the nearest quantization level minimizes the quantization MSE. 

The algorithm starts with an initial guess of quantization levels and then iteratively computes quantization thresholds $\{\tau_i\}_{i=1}^{2^B-1}$ and updates quantization levels $\{\hat{x}_i\}_{i=1}^{2^B}$. Specifically, at iteration $n$, thresholds are set to the midpoints of the previous iteration's levels:
\begin{align*}
    \tau_i^{(n)}=\frac{\hat{x}_i^{(n-1)}+\hat{x}_{i+1}^{(n-1)}}2 \text{ for } i=1\ldots 2^B-1
\end{align*}
Subsequently, the quantization levels are re-computed as conditional means of the data regions defined by the new thresholds:
\begin{align*}
    \hat{x}_i^{(n)}=\mathbb{E}\left[ \bm{X} \big| \bm{X}\in [\tau_{i-1}^{(n)},\tau_i^{(n)}] \right] \text{ for } i=1\ldots 2^B
\end{align*}
where to satisfy boundary conditions we have $\tau_0=-\infty$ and $\tau_{2^B}=\infty$. The algorithm iterates the above steps until convergence.

Figure \ref{fig:lm_quant} compares the quantization levels of a $7$-bit floating point (E3M3) quantizer (left) to a $7$-bit Lloyd-Max quantizer (right) when quantizing a layer of weights from the GPT3-126M model at a per-tensor granularity. As shown, the Lloyd-Max quantizer achieves substantially lower quantization MSE. Further, Table \ref{tab:FP7_vs_LM7} shows the superior perplexity achieved by Lloyd-Max quantizers for bitwidths of $7$, $6$ and $5$. The difference between the quantizers is clear at 5 bits, where per-tensor FP quantization incurs a drastic and unacceptable increase in perplexity, while Lloyd-Max quantization incurs a much smaller increase. Nevertheless, we note that even the optimal Lloyd-Max quantizer incurs a notable ($\sim 1.5$) increase in perplexity due to the coarse granularity of quantization. 

\begin{figure}[h]
  \centering
  \includegraphics[width=0.7\linewidth]{sections/figures/LM7_FP7.pdf}
  \caption{\small Quantization levels and the corresponding quantization MSE of Floating Point (left) vs Lloyd-Max (right) Quantizers for a layer of weights in the GPT3-126M model.}
  \label{fig:lm_quant}
\end{figure}

\begin{table}[h]\scriptsize
\begin{center}
\caption{\label{tab:FP7_vs_LM7} \small Comparing perplexity (lower is better) achieved by floating point quantizers and Lloyd-Max quantizers on a GPT3-126M model for the Wikitext-103 dataset.}
\begin{tabular}{c|cc|c}
\hline
 \multirow{2}{*}{\textbf{Bitwidth}} & \multicolumn{2}{|c|}{\textbf{Floating-Point Quantizer}} & \textbf{Lloyd-Max Quantizer} \\
 & Best Format & Wikitext-103 Perplexity & Wikitext-103 Perplexity \\
\hline
7 & E3M3 & 18.32 & 18.27 \\
6 & E3M2 & 19.07 & 18.51 \\
5 & E4M0 & 43.89 & 19.71 \\
\hline
\end{tabular}
\end{center}
\end{table}

\subsection{Proof of Local Optimality of LO-BCQ}
\label{subsec:lobcq_opt_proof}
For a given block $\bm{b}_j$, the quantization MSE during LO-BCQ can be empirically evaluated as $\frac{1}{L_b}\lVert \bm{b}_j- \bm{\hat{b}}_j\rVert^2_2$ where $\bm{\hat{b}}_j$ is computed from equation (\ref{eq:clustered_quantization_definition}) as $C_{f(\bm{b}_j)}(\bm{b}_j)$. Further, for a given block cluster $\mathcal{B}_i$, we compute the quantization MSE as $\frac{1}{|\mathcal{B}_{i}|}\sum_{\bm{b} \in \mathcal{B}_{i}} \frac{1}{L_b}\lVert \bm{b}- C_i^{(n)}(\bm{b})\rVert^2_2$. Therefore, at the end of iteration $n$, we evaluate the overall quantization MSE $J^{(n)}$ for a given operand $\bm{X}$ composed of $N_c$ block clusters as:
\begin{align*}
    \label{eq:mse_iter_n}
    J^{(n)} = \frac{1}{N_c} \sum_{i=1}^{N_c} \frac{1}{|\mathcal{B}_{i}^{(n)}|}\sum_{\bm{v} \in \mathcal{B}_{i}^{(n)}} \frac{1}{L_b}\lVert \bm{b}- B_i^{(n)}(\bm{b})\rVert^2_2
\end{align*}

At the end of iteration $n$, the codebooks are updated from $\mathcal{C}^{(n-1)}$ to $\mathcal{C}^{(n)}$. However, the mapping of a given vector $\bm{b}_j$ to quantizers $\mathcal{C}^{(n)}$ remains as  $f^{(n)}(\bm{b}_j)$. At the next iteration, during the vector clustering step, $f^{(n+1)}(\bm{b}_j)$ finds new mapping of $\bm{b}_j$ to updated codebooks $\mathcal{C}^{(n)}$ such that the quantization MSE over the candidate codebooks is minimized. Therefore, we obtain the following result for $\bm{b}_j$:
\begin{align*}
\frac{1}{L_b}\lVert \bm{b}_j - C_{f^{(n+1)}(\bm{b}_j)}^{(n)}(\bm{b}_j)\rVert^2_2 \le \frac{1}{L_b}\lVert \bm{b}_j - C_{f^{(n)}(\bm{b}_j)}^{(n)}(\bm{b}_j)\rVert^2_2
\end{align*}

That is, quantizing $\bm{b}_j$ at the end of the block clustering step of iteration $n+1$ results in lower quantization MSE compared to quantizing at the end of iteration $n$. Since this is true for all $\bm{b} \in \bm{X}$, we assert the following:
\begin{equation}
\begin{split}
\label{eq:mse_ineq_1}
    \tilde{J}^{(n+1)} &= \frac{1}{N_c} \sum_{i=1}^{N_c} \frac{1}{|\mathcal{B}_{i}^{(n+1)}|}\sum_{\bm{b} \in \mathcal{B}_{i}^{(n+1)}} \frac{1}{L_b}\lVert \bm{b} - C_i^{(n)}(b)\rVert^2_2 \le J^{(n)}
\end{split}
\end{equation}
where $\tilde{J}^{(n+1)}$ is the the quantization MSE after the vector clustering step at iteration $n+1$.

Next, during the codebook update step (\ref{eq:quantizers_update}) at iteration $n+1$, the per-cluster codebooks $\mathcal{C}^{(n)}$ are updated to $\mathcal{C}^{(n+1)}$ by invoking the Lloyd-Max algorithm \citep{Lloyd}. We know that for any given value distribution, the Lloyd-Max algorithm minimizes the quantization MSE. Therefore, for a given vector cluster $\mathcal{B}_i$ we obtain the following result:

\begin{equation}
    \frac{1}{|\mathcal{B}_{i}^{(n+1)}|}\sum_{\bm{b} \in \mathcal{B}_{i}^{(n+1)}} \frac{1}{L_b}\lVert \bm{b}- C_i^{(n+1)}(\bm{b})\rVert^2_2 \le \frac{1}{|\mathcal{B}_{i}^{(n+1)}|}\sum_{\bm{b} \in \mathcal{B}_{i}^{(n+1)}} \frac{1}{L_b}\lVert \bm{b}- C_i^{(n)}(\bm{b})\rVert^2_2
\end{equation}

The above equation states that quantizing the given block cluster $\mathcal{B}_i$ after updating the associated codebook from $C_i^{(n)}$ to $C_i^{(n+1)}$ results in lower quantization MSE. Since this is true for all the block clusters, we derive the following result: 
\begin{equation}
\begin{split}
\label{eq:mse_ineq_2}
     J^{(n+1)} &= \frac{1}{N_c} \sum_{i=1}^{N_c} \frac{1}{|\mathcal{B}_{i}^{(n+1)}|}\sum_{\bm{b} \in \mathcal{B}_{i}^{(n+1)}} \frac{1}{L_b}\lVert \bm{b}- C_i^{(n+1)}(\bm{b})\rVert^2_2  \le \tilde{J}^{(n+1)}   
\end{split}
\end{equation}

Following (\ref{eq:mse_ineq_1}) and (\ref{eq:mse_ineq_2}), we find that the quantization MSE is non-increasing for each iteration, that is, $J^{(1)} \ge J^{(2)} \ge J^{(3)} \ge \ldots \ge J^{(M)}$ where $M$ is the maximum number of iterations. 
%Therefore, we can say that if the algorithm converges, then it must be that it has converged to a local minimum. 
\hfill $\blacksquare$


\begin{figure}
    \begin{center}
    \includegraphics[width=0.5\textwidth]{sections//figures/mse_vs_iter.pdf}
    \end{center}
    \caption{\small NMSE vs iterations during LO-BCQ compared to other block quantization proposals}
    \label{fig:nmse_vs_iter}
\end{figure}

Figure \ref{fig:nmse_vs_iter} shows the empirical convergence of LO-BCQ across several block lengths and number of codebooks. Also, the MSE achieved by LO-BCQ is compared to baselines such as MXFP and VSQ. As shown, LO-BCQ converges to a lower MSE than the baselines. Further, we achieve better convergence for larger number of codebooks ($N_c$) and for a smaller block length ($L_b$), both of which increase the bitwidth of BCQ (see Eq \ref{eq:bitwidth_bcq}).


\subsection{Additional Accuracy Results}
%Table \ref{tab:lobcq_config} lists the various LOBCQ configurations and their corresponding bitwidths.
\begin{table}
\setlength{\tabcolsep}{4.75pt}
\begin{center}
\caption{\label{tab:lobcq_config} Various LO-BCQ configurations and their bitwidths.}
\begin{tabular}{|c||c|c|c|c||c|c||c|} 
\hline
 & \multicolumn{4}{|c||}{$L_b=8$} & \multicolumn{2}{|c||}{$L_b=4$} & $L_b=2$ \\
 \hline
 \backslashbox{$L_A$\kern-1em}{\kern-1em$N_c$} & 2 & 4 & 8 & 16 & 2 & 4 & 2 \\
 \hline
 64 & 4.25 & 4.375 & 4.5 & 4.625 & 4.375 & 4.625 & 4.625\\
 \hline
 32 & 4.375 & 4.5 & 4.625& 4.75 & 4.5 & 4.75 & 4.75 \\
 \hline
 16 & 4.625 & 4.75& 4.875 & 5 & 4.75 & 5 & 5 \\
 \hline
\end{tabular}
\end{center}
\end{table}

%\subsection{Perplexity achieved by various LO-BCQ configurations on Wikitext-103 dataset}

\begin{table} \centering
\begin{tabular}{|c||c|c|c|c||c|c||c|} 
\hline
 $L_b \rightarrow$& \multicolumn{4}{c||}{8} & \multicolumn{2}{c||}{4} & 2\\
 \hline
 \backslashbox{$L_A$\kern-1em}{\kern-1em$N_c$} & 2 & 4 & 8 & 16 & 2 & 4 & 2  \\
 %$N_c \rightarrow$ & 2 & 4 & 8 & 16 & 2 & 4 & 2 \\
 \hline
 \hline
 \multicolumn{8}{c}{GPT3-1.3B (FP32 PPL = 9.98)} \\ 
 \hline
 \hline
 64 & 10.40 & 10.23 & 10.17 & 10.15 &  10.28 & 10.18 & 10.19 \\
 \hline
 32 & 10.25 & 10.20 & 10.15 & 10.12 &  10.23 & 10.17 & 10.17 \\
 \hline
 16 & 10.22 & 10.16 & 10.10 & 10.09 &  10.21 & 10.14 & 10.16 \\
 \hline
  \hline
 \multicolumn{8}{c}{GPT3-8B (FP32 PPL = 7.38)} \\ 
 \hline
 \hline
 64 & 7.61 & 7.52 & 7.48 &  7.47 &  7.55 &  7.49 & 7.50 \\
 \hline
 32 & 7.52 & 7.50 & 7.46 &  7.45 &  7.52 &  7.48 & 7.48  \\
 \hline
 16 & 7.51 & 7.48 & 7.44 &  7.44 &  7.51 &  7.49 & 7.47  \\
 \hline
\end{tabular}
\caption{\label{tab:ppl_gpt3_abalation} Wikitext-103 perplexity across GPT3-1.3B and 8B models.}
\end{table}

\begin{table} \centering
\begin{tabular}{|c||c|c|c|c||} 
\hline
 $L_b \rightarrow$& \multicolumn{4}{c||}{8}\\
 \hline
 \backslashbox{$L_A$\kern-1em}{\kern-1em$N_c$} & 2 & 4 & 8 & 16 \\
 %$N_c \rightarrow$ & 2 & 4 & 8 & 16 & 2 & 4 & 2 \\
 \hline
 \hline
 \multicolumn{5}{|c|}{Llama2-7B (FP32 PPL = 5.06)} \\ 
 \hline
 \hline
 64 & 5.31 & 5.26 & 5.19 & 5.18  \\
 \hline
 32 & 5.23 & 5.25 & 5.18 & 5.15  \\
 \hline
 16 & 5.23 & 5.19 & 5.16 & 5.14  \\
 \hline
 \multicolumn{5}{|c|}{Nemotron4-15B (FP32 PPL = 5.87)} \\ 
 \hline
 \hline
 64  & 6.3 & 6.20 & 6.13 & 6.08  \\
 \hline
 32  & 6.24 & 6.12 & 6.07 & 6.03  \\
 \hline
 16  & 6.12 & 6.14 & 6.04 & 6.02  \\
 \hline
 \multicolumn{5}{|c|}{Nemotron4-340B (FP32 PPL = 3.48)} \\ 
 \hline
 \hline
 64 & 3.67 & 3.62 & 3.60 & 3.59 \\
 \hline
 32 & 3.63 & 3.61 & 3.59 & 3.56 \\
 \hline
 16 & 3.61 & 3.58 & 3.57 & 3.55 \\
 \hline
\end{tabular}
\caption{\label{tab:ppl_llama7B_nemo15B} Wikitext-103 perplexity compared to FP32 baseline in Llama2-7B and Nemotron4-15B, 340B models}
\end{table}

%\subsection{Perplexity achieved by various LO-BCQ configurations on MMLU dataset}


\begin{table} \centering
\begin{tabular}{|c||c|c|c|c||c|c|c|c|} 
\hline
 $L_b \rightarrow$& \multicolumn{4}{c||}{8} & \multicolumn{4}{c||}{8}\\
 \hline
 \backslashbox{$L_A$\kern-1em}{\kern-1em$N_c$} & 2 & 4 & 8 & 16 & 2 & 4 & 8 & 16  \\
 %$N_c \rightarrow$ & 2 & 4 & 8 & 16 & 2 & 4 & 2 \\
 \hline
 \hline
 \multicolumn{5}{|c|}{Llama2-7B (FP32 Accuracy = 45.8\%)} & \multicolumn{4}{|c|}{Llama2-70B (FP32 Accuracy = 69.12\%)} \\ 
 \hline
 \hline
 64 & 43.9 & 43.4 & 43.9 & 44.9 & 68.07 & 68.27 & 68.17 & 68.75 \\
 \hline
 32 & 44.5 & 43.8 & 44.9 & 44.5 & 68.37 & 68.51 & 68.35 & 68.27  \\
 \hline
 16 & 43.9 & 42.7 & 44.9 & 45 & 68.12 & 68.77 & 68.31 & 68.59  \\
 \hline
 \hline
 \multicolumn{5}{|c|}{GPT3-22B (FP32 Accuracy = 38.75\%)} & \multicolumn{4}{|c|}{Nemotron4-15B (FP32 Accuracy = 64.3\%)} \\ 
 \hline
 \hline
 64 & 36.71 & 38.85 & 38.13 & 38.92 & 63.17 & 62.36 & 63.72 & 64.09 \\
 \hline
 32 & 37.95 & 38.69 & 39.45 & 38.34 & 64.05 & 62.30 & 63.8 & 64.33  \\
 \hline
 16 & 38.88 & 38.80 & 38.31 & 38.92 & 63.22 & 63.51 & 63.93 & 64.43  \\
 \hline
\end{tabular}
\caption{\label{tab:mmlu_abalation} Accuracy on MMLU dataset across GPT3-22B, Llama2-7B, 70B and Nemotron4-15B models.}
\end{table}


%\subsection{Perplexity achieved by various LO-BCQ configurations on LM evaluation harness}

\begin{table} \centering
\begin{tabular}{|c||c|c|c|c||c|c|c|c|} 
\hline
 $L_b \rightarrow$& \multicolumn{4}{c||}{8} & \multicolumn{4}{c||}{8}\\
 \hline
 \backslashbox{$L_A$\kern-1em}{\kern-1em$N_c$} & 2 & 4 & 8 & 16 & 2 & 4 & 8 & 16  \\
 %$N_c \rightarrow$ & 2 & 4 & 8 & 16 & 2 & 4 & 2 \\
 \hline
 \hline
 \multicolumn{5}{|c|}{Race (FP32 Accuracy = 37.51\%)} & \multicolumn{4}{|c|}{Boolq (FP32 Accuracy = 64.62\%)} \\ 
 \hline
 \hline
 64 & 36.94 & 37.13 & 36.27 & 37.13 & 63.73 & 62.26 & 63.49 & 63.36 \\
 \hline
 32 & 37.03 & 36.36 & 36.08 & 37.03 & 62.54 & 63.51 & 63.49 & 63.55  \\
 \hline
 16 & 37.03 & 37.03 & 36.46 & 37.03 & 61.1 & 63.79 & 63.58 & 63.33  \\
 \hline
 \hline
 \multicolumn{5}{|c|}{Winogrande (FP32 Accuracy = 58.01\%)} & \multicolumn{4}{|c|}{Piqa (FP32 Accuracy = 74.21\%)} \\ 
 \hline
 \hline
 64 & 58.17 & 57.22 & 57.85 & 58.33 & 73.01 & 73.07 & 73.07 & 72.80 \\
 \hline
 32 & 59.12 & 58.09 & 57.85 & 58.41 & 73.01 & 73.94 & 72.74 & 73.18  \\
 \hline
 16 & 57.93 & 58.88 & 57.93 & 58.56 & 73.94 & 72.80 & 73.01 & 73.94  \\
 \hline
\end{tabular}
\caption{\label{tab:mmlu_abalation} Accuracy on LM evaluation harness tasks on GPT3-1.3B model.}
\end{table}

\begin{table} \centering
\begin{tabular}{|c||c|c|c|c||c|c|c|c|} 
\hline
 $L_b \rightarrow$& \multicolumn{4}{c||}{8} & \multicolumn{4}{c||}{8}\\
 \hline
 \backslashbox{$L_A$\kern-1em}{\kern-1em$N_c$} & 2 & 4 & 8 & 16 & 2 & 4 & 8 & 16  \\
 %$N_c \rightarrow$ & 2 & 4 & 8 & 16 & 2 & 4 & 2 \\
 \hline
 \hline
 \multicolumn{5}{|c|}{Race (FP32 Accuracy = 41.34\%)} & \multicolumn{4}{|c|}{Boolq (FP32 Accuracy = 68.32\%)} \\ 
 \hline
 \hline
 64 & 40.48 & 40.10 & 39.43 & 39.90 & 69.20 & 68.41 & 69.45 & 68.56 \\
 \hline
 32 & 39.52 & 39.52 & 40.77 & 39.62 & 68.32 & 67.43 & 68.17 & 69.30  \\
 \hline
 16 & 39.81 & 39.71 & 39.90 & 40.38 & 68.10 & 66.33 & 69.51 & 69.42  \\
 \hline
 \hline
 \multicolumn{5}{|c|}{Winogrande (FP32 Accuracy = 67.88\%)} & \multicolumn{4}{|c|}{Piqa (FP32 Accuracy = 78.78\%)} \\ 
 \hline
 \hline
 64 & 66.85 & 66.61 & 67.72 & 67.88 & 77.31 & 77.42 & 77.75 & 77.64 \\
 \hline
 32 & 67.25 & 67.72 & 67.72 & 67.00 & 77.31 & 77.04 & 77.80 & 77.37  \\
 \hline
 16 & 68.11 & 68.90 & 67.88 & 67.48 & 77.37 & 78.13 & 78.13 & 77.69  \\
 \hline
\end{tabular}
\caption{\label{tab:mmlu_abalation} Accuracy on LM evaluation harness tasks on GPT3-8B model.}
\end{table}

\begin{table} \centering
\begin{tabular}{|c||c|c|c|c||c|c|c|c|} 
\hline
 $L_b \rightarrow$& \multicolumn{4}{c||}{8} & \multicolumn{4}{c||}{8}\\
 \hline
 \backslashbox{$L_A$\kern-1em}{\kern-1em$N_c$} & 2 & 4 & 8 & 16 & 2 & 4 & 8 & 16  \\
 %$N_c \rightarrow$ & 2 & 4 & 8 & 16 & 2 & 4 & 2 \\
 \hline
 \hline
 \multicolumn{5}{|c|}{Race (FP32 Accuracy = 40.67\%)} & \multicolumn{4}{|c|}{Boolq (FP32 Accuracy = 76.54\%)} \\ 
 \hline
 \hline
 64 & 40.48 & 40.10 & 39.43 & 39.90 & 75.41 & 75.11 & 77.09 & 75.66 \\
 \hline
 32 & 39.52 & 39.52 & 40.77 & 39.62 & 76.02 & 76.02 & 75.96 & 75.35  \\
 \hline
 16 & 39.81 & 39.71 & 39.90 & 40.38 & 75.05 & 73.82 & 75.72 & 76.09  \\
 \hline
 \hline
 \multicolumn{5}{|c|}{Winogrande (FP32 Accuracy = 70.64\%)} & \multicolumn{4}{|c|}{Piqa (FP32 Accuracy = 79.16\%)} \\ 
 \hline
 \hline
 64 & 69.14 & 70.17 & 70.17 & 70.56 & 78.24 & 79.00 & 78.62 & 78.73 \\
 \hline
 32 & 70.96 & 69.69 & 71.27 & 69.30 & 78.56 & 79.49 & 79.16 & 78.89  \\
 \hline
 16 & 71.03 & 69.53 & 69.69 & 70.40 & 78.13 & 79.16 & 79.00 & 79.00  \\
 \hline
\end{tabular}
\caption{\label{tab:mmlu_abalation} Accuracy on LM evaluation harness tasks on GPT3-22B model.}
\end{table}

\begin{table} \centering
\begin{tabular}{|c||c|c|c|c||c|c|c|c|} 
\hline
 $L_b \rightarrow$& \multicolumn{4}{c||}{8} & \multicolumn{4}{c||}{8}\\
 \hline
 \backslashbox{$L_A$\kern-1em}{\kern-1em$N_c$} & 2 & 4 & 8 & 16 & 2 & 4 & 8 & 16  \\
 %$N_c \rightarrow$ & 2 & 4 & 8 & 16 & 2 & 4 & 2 \\
 \hline
 \hline
 \multicolumn{5}{|c|}{Race (FP32 Accuracy = 44.4\%)} & \multicolumn{4}{|c|}{Boolq (FP32 Accuracy = 79.29\%)} \\ 
 \hline
 \hline
 64 & 42.49 & 42.51 & 42.58 & 43.45 & 77.58 & 77.37 & 77.43 & 78.1 \\
 \hline
 32 & 43.35 & 42.49 & 43.64 & 43.73 & 77.86 & 75.32 & 77.28 & 77.86  \\
 \hline
 16 & 44.21 & 44.21 & 43.64 & 42.97 & 78.65 & 77 & 76.94 & 77.98  \\
 \hline
 \hline
 \multicolumn{5}{|c|}{Winogrande (FP32 Accuracy = 69.38\%)} & \multicolumn{4}{|c|}{Piqa (FP32 Accuracy = 78.07\%)} \\ 
 \hline
 \hline
 64 & 68.9 & 68.43 & 69.77 & 68.19 & 77.09 & 76.82 & 77.09 & 77.86 \\
 \hline
 32 & 69.38 & 68.51 & 68.82 & 68.90 & 78.07 & 76.71 & 78.07 & 77.86  \\
 \hline
 16 & 69.53 & 67.09 & 69.38 & 68.90 & 77.37 & 77.8 & 77.91 & 77.69  \\
 \hline
\end{tabular}
\caption{\label{tab:mmlu_abalation} Accuracy on LM evaluation harness tasks on Llama2-7B model.}
\end{table}

\begin{table} \centering
\begin{tabular}{|c||c|c|c|c||c|c|c|c|} 
\hline
 $L_b \rightarrow$& \multicolumn{4}{c||}{8} & \multicolumn{4}{c||}{8}\\
 \hline
 \backslashbox{$L_A$\kern-1em}{\kern-1em$N_c$} & 2 & 4 & 8 & 16 & 2 & 4 & 8 & 16  \\
 %$N_c \rightarrow$ & 2 & 4 & 8 & 16 & 2 & 4 & 2 \\
 \hline
 \hline
 \multicolumn{5}{|c|}{Race (FP32 Accuracy = 48.8\%)} & \multicolumn{4}{|c|}{Boolq (FP32 Accuracy = 85.23\%)} \\ 
 \hline
 \hline
 64 & 49.00 & 49.00 & 49.28 & 48.71 & 82.82 & 84.28 & 84.03 & 84.25 \\
 \hline
 32 & 49.57 & 48.52 & 48.33 & 49.28 & 83.85 & 84.46 & 84.31 & 84.93  \\
 \hline
 16 & 49.85 & 49.09 & 49.28 & 48.99 & 85.11 & 84.46 & 84.61 & 83.94  \\
 \hline
 \hline
 \multicolumn{5}{|c|}{Winogrande (FP32 Accuracy = 79.95\%)} & \multicolumn{4}{|c|}{Piqa (FP32 Accuracy = 81.56\%)} \\ 
 \hline
 \hline
 64 & 78.77 & 78.45 & 78.37 & 79.16 & 81.45 & 80.69 & 81.45 & 81.5 \\
 \hline
 32 & 78.45 & 79.01 & 78.69 & 80.66 & 81.56 & 80.58 & 81.18 & 81.34  \\
 \hline
 16 & 79.95 & 79.56 & 79.79 & 79.72 & 81.28 & 81.66 & 81.28 & 80.96  \\
 \hline
\end{tabular}
\caption{\label{tab:mmlu_abalation} Accuracy on LM evaluation harness tasks on Llama2-70B model.}
\end{table}

%\section{MSE Studies}
%\textcolor{red}{TODO}


\subsection{Number Formats and Quantization Method}
\label{subsec:numFormats_quantMethod}
\subsubsection{Integer Format}
An $n$-bit signed integer (INT) is typically represented with a 2s-complement format \citep{yao2022zeroquant,xiao2023smoothquant,dai2021vsq}, where the most significant bit denotes the sign.

\subsubsection{Floating Point Format}
An $n$-bit signed floating point (FP) number $x$ comprises of a 1-bit sign ($x_{\mathrm{sign}}$), $B_m$-bit mantissa ($x_{\mathrm{mant}}$) and $B_e$-bit exponent ($x_{\mathrm{exp}}$) such that $B_m+B_e=n-1$. The associated constant exponent bias ($E_{\mathrm{bias}}$) is computed as $(2^{{B_e}-1}-1)$. We denote this format as $E_{B_e}M_{B_m}$.  

\subsubsection{Quantization Scheme}
\label{subsec:quant_method}
A quantization scheme dictates how a given unquantized tensor is converted to its quantized representation. We consider FP formats for the purpose of illustration. Given an unquantized tensor $\bm{X}$ and an FP format $E_{B_e}M_{B_m}$, we first, we compute the quantization scale factor $s_X$ that maps the maximum absolute value of $\bm{X}$ to the maximum quantization level of the $E_{B_e}M_{B_m}$ format as follows:
\begin{align}
\label{eq:sf}
    s_X = \frac{\mathrm{max}(|\bm{X}|)}{\mathrm{max}(E_{B_e}M_{B_m})}
\end{align}
In the above equation, $|\cdot|$ denotes the absolute value function.

Next, we scale $\bm{X}$ by $s_X$ and quantize it to $\hat{\bm{X}}$ by rounding it to the nearest quantization level of $E_{B_e}M_{B_m}$ as:

\begin{align}
\label{eq:tensor_quant}
    \hat{\bm{X}} = \text{round-to-nearest}\left(\frac{\bm{X}}{s_X}, E_{B_e}M_{B_m}\right)
\end{align}

We perform dynamic max-scaled quantization \citep{wu2020integer}, where the scale factor $s$ for activations is dynamically computed during runtime.

\subsection{Vector Scaled Quantization}
\begin{wrapfigure}{r}{0.35\linewidth}
  \centering
  \includegraphics[width=\linewidth]{sections/figures/vsquant.jpg}
  \caption{\small Vectorwise decomposition for per-vector scaled quantization (VSQ \citep{dai2021vsq}).}
  \label{fig:vsquant}
\end{wrapfigure}
During VSQ \citep{dai2021vsq}, the operand tensors are decomposed into 1D vectors in a hardware friendly manner as shown in Figure \ref{fig:vsquant}. Since the decomposed tensors are used as operands in matrix multiplications during inference, it is beneficial to perform this decomposition along the reduction dimension of the multiplication. The vectorwise quantization is performed similar to tensorwise quantization described in Equations \ref{eq:sf} and \ref{eq:tensor_quant}, where a scale factor $s_v$ is required for each vector $\bm{v}$ that maps the maximum absolute value of that vector to the maximum quantization level. While smaller vector lengths can lead to larger accuracy gains, the associated memory and computational overheads due to the per-vector scale factors increases. To alleviate these overheads, VSQ \citep{dai2021vsq} proposed a second level quantization of the per-vector scale factors to unsigned integers, while MX \citep{rouhani2023shared} quantizes them to integer powers of 2 (denoted as $2^{INT}$).

\subsubsection{MX Format}
The MX format proposed in \citep{rouhani2023microscaling} introduces the concept of sub-block shifting. For every two scalar elements of $b$-bits each, there is a shared exponent bit. The value of this exponent bit is determined through an empirical analysis that targets minimizing quantization MSE. We note that the FP format $E_{1}M_{b}$ is strictly better than MX from an accuracy perspective since it allocates a dedicated exponent bit to each scalar as opposed to sharing it across two scalars. Therefore, we conservatively bound the accuracy of a $b+2$-bit signed MX format with that of a $E_{1}M_{b}$ format in our comparisons. For instance, we use E1M2 format as a proxy for MX4.

\begin{figure}
    \centering
    \includegraphics[width=1\linewidth]{sections//figures/BlockFormats.pdf}
    \caption{\small Comparing LO-BCQ to MX format.}
    \label{fig:block_formats}
\end{figure}

Figure \ref{fig:block_formats} compares our $4$-bit LO-BCQ block format to MX \citep{rouhani2023microscaling}. As shown, both LO-BCQ and MX decompose a given operand tensor into block arrays and each block array into blocks. Similar to MX, we find that per-block quantization ($L_b < L_A$) leads to better accuracy due to increased flexibility. While MX achieves this through per-block $1$-bit micro-scales, we associate a dedicated codebook to each block through a per-block codebook selector. Further, MX quantizes the per-block array scale-factor to E8M0 format without per-tensor scaling. In contrast during LO-BCQ, we find that per-tensor scaling combined with quantization of per-block array scale-factor to E4M3 format results in superior inference accuracy across models. 



\end{document}


% This document was modified from the file originally made available by
% Pat Langley and Andrea Danyluk for ICML-2K. This version was created
% by Iain Murray in 2018, and modified by Alexandre Bouchard in
% 2019 and 2021 and by Csaba Szepesvari, Gang Niu and Sivan Sabato in 2022.
% Modified again in 2023 and 2024 by Sivan Sabato and Jonathan Scarlett.
% Previous contributors include Dan Roy, Lise Getoor and Tobias
% Scheffer, which was slightly modified from the 2010 version by
% Thorsten Joachims & Johannes Fuernkranz, slightly modified from the
% 2009 version by Kiri Wagstaff and Sam Roweis's 2008 version, which is
% slightly modified from Prasad Tadepalli's 2007 version which is a
% lightly changed version of the previous year's version by Andrew
% Moore, which was in turn edited from those of Kristian Kersting and
% Codrina Lauth. Alex Smola contributed to the algorithmic style files.
