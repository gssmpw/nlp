\section{Related Work}
\label{Sec:RelatedWork}
\textbf{Molecular docking.} As a cornerstone of drug discovery, molecular docking, often synonoymous with ligand-protein docking, focuses on the interactions between ligands and proteins. Traditional methods like Vina~\citep{trott:2010:Vina}, Smina~\citep{koes:2013:Smina}, Glide~\citep{friesner:2004:Glide}, Gnina~\citep{mcnutt:2021:Gnina} and Gold~\citep{jones:1997:Gold}, use physics-based scores to analyze these interaction, which, though effective, tend to be computationally intensive. Recent progress in geometric deep learning has sparked the development of deep learning-based docking strategies~\citep{crampon:2022:DockingSurvey}, which can be broadly categorized into regression-based and sampling-based methods. Regression-based methods like EquiBind~\citep{stark:2022:ICML:Equibind}, TankBind~\citep{lu:2022:NIPS:tankbind}, E3Bind~\citep{zhang:2023:ICLR:E3bind} and FABind~\citep{pei:2024:NIPS:fabind} leverage various geometric neural networks to directly predict binding structures. Conversely, sampling-based methods like DiffDock~\citep{corso:2022:diffdock}, manipulate the rotation, translation and torsion of ligands using diffusion models. These methods generally simplify the problem by assuming protein rigidity, neglecting the dynamic nature of protein in realistic docking scenarios.

\par
\textbf{Flexible molecular docking.} Recent methods in flexible docking, such as DynamicBind~\citep{lu:2024:NatCom:dynamicbind}, ReDock~\citep{huang:2024:ReDock}, PackDock~\citep{zhang:2024:PackDock} and NeuralPLexer~\citep{qiao:2024:NeuralPLexer:NMI}, are primarily based on diffusion models with sampling strategy. For example, DynamicBind~\citep{lu:2024:NatCom:dynamicbind} uses equivariant geometric diffusion networks to reconstruct the holo structures of both ligand and protein from their apo states. ReDock~\citep{huang:2024:ReDock} adopts the neural diffusion bridge model that employs energy-to-geometry mapping on geometric manifolds to predict protein-ligand binding structures. While these methods effectively enhance the docking performance, they suffer from the typical flaws associated with diffusion models and multi-round sampling strategy, i.e., low computational efficiency. This limitation impedes the scalability of these methods in assessing extensive volumes of potential, unknown molecule-protein interactions, which are crucial for advancing drug discovery. In this paper, we attempt to provide a regression-based solution aimed at achieving both fast inference and high docking accuracy.