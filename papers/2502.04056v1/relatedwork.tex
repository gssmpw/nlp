\section{Background and Related works}
\label{sec:background}

\subsection{Diffusion Models}
\label{ssec:DM}
Diffusion models operate by gradually introducing Gaussian noise to real data $x_0$ in a forward process and then learning a reverse process to denoise and generate high-quality images. For denoising diffusion probabilistic models (DDPMs) \cite{DM}, the forward process is defined as a Markov chain, represented by the following equation:

\begin{equation}
q(x_t | x_{t-1}) = \mathcal{N}(x_t; \sqrt{\alpha_t} x_{t-1}, \beta_t I),
\end{equation}
where $\alpha_t$ and $\beta_t$ are hyperparameters, with $\beta_t = 1 - \alpha_t$. 

\begin{figure}
    \centering
    \includegraphics[width=\columnwidth]{figure/Fig3.pdf}
    \caption{Maximum channel magnitudes after softmax are depicted for various timesteps during inference, revealing large variance across timesteps. This shows the necessity of handling timestep-dependent values effectively.
    } \label{fig3}
\end{figure}

In the reverse process, since directly modeling the true distribution $q(x_{t-1}|x_t)$ is infeasible, diffusion models employ variational inference to approximate it as a Gaussian distribution:

\begin{equation}
p_\theta(x_{t-1}|x_t) = \mathcal{N}(x_{t-1}; \mu_\theta(x_t, t), \Sigma_\theta(x_t, t)).
\end{equation}
The mean of the Gaussian can further be reparameterized using a noise prediction network $\epsilon_\theta(x_t, t)$ as follows.

\begin{equation}
\mu_\theta(x_t, t) = \frac{1}{\sqrt{\alpha_t}} \left( x_t - \frac{1 - \alpha_t}{\sqrt{1 - \bar{\alpha}_t}} \epsilon_\theta(x_t, t) \right),
\end{equation}
where $\bar{\alpha}_t = \prod_{s=1}^t \alpha_s$. The variance $\Sigma_\theta(x_t, t)$ can either be reparameterized or set to a fixed schedule $\sigma_t$. Under the fixed variance schedule, the distribution of $x_{t-1}$ is given by

\begin{equation}
x_{t-1} \sim \mathcal{N}(x_{t-1}; \mu_\theta(x_t, t), \sigma_t^2 I).
\end{equation}

In diffusion models, the noise at each time step $t$ is predicted from $x_t$ using a noise estimation model, which typically shares the same weights across all time steps.

\begin{figure}
    \centering
    \includegraphics[width=\columnwidth]{figure/Fig4_2.pdf}
    \caption{Illustration of the diffusion transformer (DiT)~\cite{DiT} with stacked transformer-based DiT blocks. Each block includes MHSA layers with softmax and PF layers with GELU activations, conditioned on class and timestep inputs.
    } \label{fig4}
\end{figure}

\subsection{Diffusion Transformers}
\label{ssec:DiT}
Despite considerable impact of the U-Net architecture on image generation models \cite{DM, ADM, LDM}, recent studies have shifted toward transformer-based approaches \cite{DMsurvey2}. Diffusion transformers (DiTs)~\cite{DiT} show state-of-the-art performance in image generation, while they can scale effectively in terms of data representation and model size.

DiTs are structured with $N$ transformer-based blocks that form the backbone of the denoising process, as depicted in Fig. \ref{fig4}. Each block includes two fundamental components: multi-head self-attention (MHSA) and pointwise feedforward (PF) layers. Both components are conditioned on class information and timestep inputs, ensuring the model effectively captures time-dependent features throughout the denoising process. 

MHSA mechanism primarily relies on linear projections and matrix multiplications (MatMul) of the query, key, and value matrices, allowing the model to capture contextual relationships among image patches. Each DiT block employs a softmax layer in the MHSA to normalize attention scores and effectively capture relative importance among tokens. This normalization is critical for the self-attention mechanism to function properly. For PF layers, two sequential linear transformations are applied, separated by a Gaussian Error Linear Unit (GELU) activation layer.

Although DiTs have shown remarkable efficiency in generating high-fidelity images, their significant computational demands present challenges for practical applications. To address this limitation, we propose a quantization framework designed for DiTs, substantially reducing memory usage and inference time. Notably, our approach achieves this efficiency without requiring re-training of the original model, making it a practical and scalable solution for deploying DiTs in resource-constrained environments.

\begin{figure*}
    \centering
    \includegraphics[width=\textwidth]{figure/Fig6_2.pdf}
    \caption{Illustration of the proposed TQ-DiT. Multi-Region Quantization (MRQ) handles skewed distributions in post-softmax and post-GELU layers within MHSA and PF. Hessian-guided Optimization (HO) with Time-Grouping Quantization(TGQ) addresses timestep-dependent activation variability in post-softmax layers.
    } \label{fig6}
\end{figure*}

\subsection{Model Quantization}
\label{ssec:DiT}

Quantization is employed for model compression to enhance the inference efficiency of deep learning models by converting full-precision tensors into $k$-bit integer representations \cite{PTQ4}. This conversion leads to significant improvements in computational efficiency and reductions in memory usage \cite{Qinference}.

For uniform quantization, the process can be mathematically expressed as

\begin{equation}
\hat{\mathbf{x}} = s \cdot \text{clip} \left( \lfloor\frac{\mathbf{x}}{s} \rceil + z, 0, 2^k-1 \right) - z,
\end{equation}
where $\lfloor \cdot \rceil$ denotes the rounding operation, $s = \frac{\max(\mathbf{x}) - \min(\mathbf{x})}{2^k-1}$ is the step size, and $z = -\lfloor \frac{\min(\mathbf{x})}{s} \rceil$ is the zero-point. Here, $k$ represents the bit-width of the quantization. This formula essentially maps floating-point values to a predefined set of fixed points (or grids).

For $k$-bit uniform asymmetric quantization, the set of quantization grids can be expressed as

\begin{equation}
\mathcal{Q}_k^\text{u} = s \times \{0, \dots, 2^k-1\} - z.
\end{equation}

The quantization function, denoted as $Q_k(\cdot \,;\Delta):\mathbb{R} \to \mathcal{Q}_k^\text{u}$, is often optimized to minimize the quantization error, defined by the deviation between the original and quantized grids. The optimization is formulated as

\begin{equation}
\min_{s, z} ||\hat{\mathbf{w}} - \mathbf{w}||_F^2 \quad \text{s.t.} \quad \hat{\mathbf{w}} \in \mathcal{Q}_k^\text{u},
\end{equation}
where $\mathbf{w}$ is the original parameter, and $\hat{\mathbf{w}}$ represents its quantized counterpart.

However, recent studies obtain that merely minimizing the quantization error in the parameter space does not always yield optimal task performance. Instead, task-aware approaches focus on minimizing the final task-specific loss function, such as cross-entropy or mean squared error, with quantized parameters. The task-aware approach can be expressed as

\begin{equation}
\label{quantization obj}
\min_{\Delta} \mathbb{E}[\mathcal{L}(\hat{\mathbf{w}})] \quad \text{s.t.} \quad \hat{\mathbf{w}} \in \mathcal{Q}_k^\text{u},
\end{equation}
where $\mathcal{L}(\cdot)$ denotes the task-specific loss function and $\Delta=\{s,z\}$ is quantization parameters. The task-aware quantization has demonstrated better preservation of model performance compared to conventional methods.

Among the various quantization techniques, PTQ has become popular for large-scale models due to its efficiency and ability to avoid resource-intensive re-training \cite{PTQ1, PTQD, PTQ4DiT}. PTQ utilizes a small calibration dataset to fine-tune the quantization parameters, enabling quantized models to achieve performance close to full-precision counterparts with minimal data and computation. It has been successfully applied to diverse architectures, including CNNs \cite{PTQ1, PTQ2}, language transformers \cite{QLLM, QLLM2}, vision transformers (ViTs) \cite{PTQ3}, and U-Net-based diffusion models \cite{PTQD, Q-Diffusion}.

A recent study extended PTQ to DiTs, introducing a technique that redistributes activations and weights based on their salience to mitigate quantization errors caused by outlier magnitudes \cite{PTQ4DiT}. However, this approach is limited by its reliance on salience-based redistribution, which requires extensive calibration time and a large-scale calibration dataset, imposing significant computational and resource burdens. Such inefficiencies are particularly problematic in real-world applications with limited computational resources, such as edge servers in distributed systems \cite{chainnet_edge_ai}, where efficient quantization strategies are critical for deployment. In comparison, our work proposes an alternative PTQ strategy that directly targets quantization errors across DiT, significantly reducing calibration overhead while maintaining generation quality. By addressing these inefficiencies, our study introduces a streamlined approach for effectively quantizing DiTs, enabling their deployment in resource-constrained environments without compromising performance in high-quality image generation tasks.