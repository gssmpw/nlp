\section{Conclusion}
% We address the previously unexplored challenge of class-incremental 3D instance segmentation in the presence of class imbalance. We propose an innovative approach that integrates a memory-efficient exemplar replay buffer, knowledge distillation, and a novel imbalance correction module. This framework mitigates the forgetting of less frequent classes during incremental learning by accounting for the frequency of object occurrences. To enable comprehensive evaluation, we design three incremental learning scenarios based on the ScanNet200 dataset, each comprising three phases that reflect real-world dynamics.
% % 
% Our experimental results demonstrate that the proposed framework significantly enhances the learning of new classes while reducing forgetting of previously learned ones. This work makes a valuable contribution to the incremental learning literature for 3D segmentation by introducing more realistic, real-world settings. The carefully designed scenarios and framework not only offer a strong baseline but also provide a clear benchmark for future research, laying a solid foundation for more advanced techniques in class-incremental learning.
We address the challenge of class-incremental 3D instance segmentation with class imbalance. We propose an innovative approach that integrates a memory-efficient exemplar replay buffer, knowledge distillation, and a novel imbalance correction module. This framework mitigates the forgetting of rare classes during incremental learning by accounting for the frequency of object occurrences. To enable comprehensive evaluation, we design three incremental learning scenarios, each comprising three phases that reflect real-world dynamics.
% 
Our experimental results demonstrate that the proposed framework significantly enhances the learning of new classes while reducing forgetting of previously learned ones. The carefully designed scenarios and framework not only offer a strong baseline but also provide a clear benchmark for future research, laying a foundation for more advanced techniques in class-incremental learning.