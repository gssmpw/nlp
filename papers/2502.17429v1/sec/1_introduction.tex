\section{Introduction}
\label{sec:intro}

\begin{figure*}[!htb]
  \centering
  \includegraphics[width=\linewidth]{fig/teaser.pdf}
  \caption{Class-incremental setup for 3D instance segmentation. As tasks progress through time, new classes are introduced incrementally. After each new task, the model should recognize both previously learned and newly introduced classes. For example, at Task-2, new classes such as \texttt{Pillow}, \texttt{Coffee Table}, and \texttt{Sofa Chair} are added, and the model is able to detect these three classes along with previous ones like \texttt{Table}, \texttt{Chair}, and \texttt{Couch}.}
  \label{fig:cl_setup}
\end{figure*}

% 3D instance segmentation plays a significant role in applications such as computer vision, graphics, robotics, and autonomous vehicles. 
3D instance segmentation is an essential task in computer vision that involves identifying and segmenting individual objects in the real physical space, playing a key role in applications across graphics, robotics, and autonomous systems. Its ability to provide precise object boundaries and class labels enhances scene understanding, facilitates object manipulation, and improves perception in dynamic environments.

In recent years, a variety of methods have been proposed, including top-down approaches \cite{wang2018sgpn, jia2021scaling, zhang2021point}, bottom-up approaches \cite{hou20193dsis, yang2019learning}, and transformer-based architectures \cite{Schult23ICRA}. These methods have shown impressive results in traditional setups, which assume that all object classes are available during training. However, this assumption limits its applicability in real-world scenarios where new categories gradually emerge over time, often exhibiting naturally imbalanced distributions. Thus, there is a need for class-incremental learning (CIL) frameworks that not only adapt to new classes but also preserve prior knowledge, especially for rare or less frequent categories, which are more prone to catastrophic forgetting.

Most existing research in class-incremental learning focuses on 2D image classification \cite{rebuffi2017icarl, li2017learning, aljundi2018memory, serra2018overcoming}, with some extensions to object detection \cite{liu2023continual, joseph2021towards, shmelkov2017incremental} and semantic segmentation \cite{cermelli2020modeling, cermelli2022incremental, douillard2021plop}. These methods employ strategies such as exemplar replay \cite{buzzega2020dark, rebuffi2017icarl, cha2021co2l, kamra2017deep}, regularization \cite{aljundi2018memory, li2017learning, serra2018overcoming}, and knowledge distillation \cite{douillard2020podnet, kang2022class} to preserve previously learned knowledge and mitigate catastrophic forgetting \cite{mccloskey1989catastrophic}. Few studies have applied CIL to point clouds; however, they mostly focus on object-level classification \cite{dong2021i3dol, liu2021l3doc, chowdhury2021learning}.
% At the scene level, works like \cite{Yang_2023_CVPR} have explored semantic segmentation, but they do not use state-of-the-art 3D segmentation models, limiting their applicability. 
At the scene level, some works have explored 3D semantic segmentation \cite{Yang_2023_CVPR} with incremental learning, but their performance is not as competitive as state-of-the-art methods that do not employ incremental learning, which limits their applicability.
Other methods tackle open-world incremental learning \cite{boudjoghra20243d} but rely heavily on large exemplar sets \cite{rolnick2019experience} and often neglect the challenge of class imbalance. 
% This highlights the need for a unified framework tackle both continual learning and class imbalance in real-world settings.

To address this, we propose \textbf{CLIMB-3D}, a unified framework that combines Exemplar Replay (ER), Knowledge Distillation (KD), and a novel Imbalance Correction (IC) module to tackle \textbf{C}ontinual \textbf{L}earning for \textbf{Imb}alance \textbf{3D} instance segmentation in indoor environments, as shown in \Cref{fig:cl_setup}. Our framework operates as follows: ER stores a subset of representative samples from previous stages, allowing for effective replay during new task learning. KD transfers knowledge by retaining a copy of the model from the previous task, thereby mitigating forgetting. The IC module is specifically designed to reduce forgetting of rare classes by leveraging the frequency of object occurrence. However, since we do not have access to the previous task data or statistics during incremental phases, we instead compile these statistics from the previous model used by KD to generate weight for earlier categories. These weights are used to favor both frequent and rare classes, ensuring a balanced learning and mitigating forgetting.

To evaluate CLIMB-3D in a realistic, incremental learning setup, we create three benchmark scenarios using the ScanNet200 dataset \cite{rozenberszki2022language}, which features 200 classes with natural class imbalances. These scenarios are designed to reflect real-world conditions where new categories emerge gradually and follow inherent class imbalances. These are based on \ct{1} frequency of object occurrence, \ct{2} semantic similarity between the object, and \ct{3} random grouping. 
% For comparison, we use ScanNetV2 with a prevision method in incremental setting for semantic segmentation, while for instance segmentation, we propose an exemplar replay baseline.
% Our experiments show that CLIMB-3D significantly improves performance by effectively mitigating forgetting of previous tasks compared to both semantic segmentation methods and the proposed instance segmentation baseline.
Our experiments demonstrate that CLIMB-3D significantly improves performance by effectively mitigating the forgetting of previous tasks compared to earlier exploration in class incremental 3D segmentation.

% In this way, our work introduces a novel problem setup and a method capable of solving it in a simplistic manner. Our contributions include:
In summary, our contributions are as follows:
\begin{enumerate}
% \item Proposed a new problem setting imbalance class incremental 3D segmentation and simple and effective method to solve this setting minimizing catastrophic forgetting and balancing the learning.
\item We propose a new problem setting for imbalanced class incremental 3D segmentation, along with a simple yet effective method to address this challenge by minimizing catastrophic forgetting and balancing the learning process.
% \item To benchmark this setting, we design a three scenarios aims to simulate real-world where objects emerge continuously with natural class imbalance.
\item To benchmark this setting, we design three scenarios aimed at simulating real-world conditions where objects emerge continuously with natural class imbalance.
\item Experimental results show that our proposed framework achieves state-of-the-art performance, with an increase of up to 16.76\% in mAP compared to the baseline.
% \item
\end{enumerate}