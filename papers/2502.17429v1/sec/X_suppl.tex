% \clearpage
% \setcounter{page}{1}
\maketitlesupplementary

\appendix 
\renewcommand{\thesection}{Appendix \Alph{section}}

In this supplementary material, we first demonstrate the performance gains on rare classes achieved by incorporating the IC module in \ref{app:rare_eval}. Next, we provide detailed split information for all scenarios, based on class names, in \ref{app:split_fullview}. Finally, we present a qualitative comparison between the baseline method and our proposed approach in \ref{app:qual_results}.
% \begin{table}[!ht]
    \centering
    \caption{Results for classes observed by the model 1–20 times during an epoch, evaluated on \fsplit~ for Phase 2, in terms of \mapft.}
    \resizebox{\columnwidth}{!}{%
    \setlength{\tabcolsep}{0.5em}
    \begin{tabular}{lrrr}
    \toprule
    \rowcolor{lightorange}   \textbf{Classes} & \textbf{Seen Count} & \textbf{ER+KD} & \textbf{ER+KD+IC} \\ 
    \midrule
        paper towel dispenser & 2 & 73.10 & 74.90 \\ 
        recycling bin & 3 & 55.80 & 60.50 \\ 
        ladder & 5 & 53.90 & 57.10 \\ 
        trash bin & 7 & 31.50 & 57.30 \\ 
        bulletin board & 8 & 23.30 & 38.20 \\ 
        shelf & 11 & 48.00 & 50.50 \\ 
        dresser & 12 & 44.00 & 55.80 \\ 
        copier & 12 & 93.30 & 94.50 \\ 
        object & 12 & 3.10 & 3.30 \\ 
        stairs & 13 & 51.70 & 67.70 \\ 
        bathtub & 16 & 80.30 & 86.60 \\ 
        oven & 16 & 1.50 & 3.30 \\ 
        divider & 18 & 36.40 & 45.00 \\ 
        column & 20 & 57.30 & 75.00 \\ 
    \midrule
        \rowcolor{lightgray} \textbf{Average} & - & 46.66 & \textbf{54.98} \\ 
    \bottomrule
    \end{tabular}
    }
    \label{tab:common_eval}
\end{table}



\section{Evaluation on Rare Categories}
\label{app:rare_eval}
The proposed imbalance correction (IC) module, as detailed in Section 4.2, is designed to address the performance gap for rare classes. To assess its impact, we compare its performance with the framework which has exemplar replay (ER) and knowledge distillation (KD). Specifically, we focus on its ability to improve performance for rare classes, which the model encounters infrequently compared to more common classes.

\begin{table}[!ht]
    \centering
    \caption{Results for classes observed by the model 1–20 times during an epoch, evaluated on \fsplit~ for Phase 2, in terms of \mapft.}
    \resizebox{\columnwidth}{!}{%
    \setlength{\tabcolsep}{0.5em}
    \begin{tabular}{lrrr}
    \toprule
    \rowcolor{lightorange}   \textbf{Classes} & \textbf{Seen Count} & \textbf{ER+KD} & \textbf{ER+KD+IC} \\ 
    \midrule
        paper towel dispenser & 2 & 73.10 & 74.90 \\ 
        recycling bin & 3 & 55.80 & 60.50 \\ 
        ladder & 5 & 53.90 & 57.10 \\ 
        trash bin & 7 & 31.50 & 57.30 \\ 
        bulletin board & 8 & 23.30 & 38.20 \\ 
        shelf & 11 & 48.00 & 50.50 \\ 
        dresser & 12 & 44.00 & 55.80 \\ 
        copier & 12 & 93.30 & 94.50 \\ 
        object & 12 & 3.10 & 3.30 \\ 
        stairs & 13 & 51.70 & 67.70 \\ 
        bathtub & 16 & 80.30 & 86.60 \\ 
        oven & 16 & 1.50 & 3.30 \\ 
        divider & 18 & 36.40 & 45.00 \\ 
        column & 20 & 57.30 & 75.00 \\ 
    \midrule
        \rowcolor{lightgray} \textbf{Average} & - & 46.66 & \textbf{54.98} \\ 
    \bottomrule
    \end{tabular}
    }
    \label{tab:common_eval}
\end{table}



The results, shown in \Cref{tab:common_eval} and \Cref{tab:tail_eval}, correspond to evaluations on \fsplit~ for \textit{Phase 2} and \textit{Phase 3}, respectively. In \textit{Phase 2}, we evaluate classes seen 1–20 times per epoch, while \textit{Phase 3} targets even less frequent classes, with observations limited to 1–10 times per epoch. 
% Note that \textit{Phase 2} utilizes a standard split of the ScanNet200 dataset, which includes both frequent and rare classes, justifying the 1–20 observation threshold.

As illustrated in \Cref{tab:common_eval}, the IC module substantially improves performance on rare classes in terms of \mapft~ in Phase 2 of \fsplit. For instance, classes like \texttt{recycling bin} and \texttt{trash bin}, seen only 3 and 7 times, respectively, shows significant improvement when the IC module is applied. Overall, the IC module provides an average boost of 8.32\%, highlighting its effectiveness in mitigating class imbalance.

\begin{table}[!ht]
    \centering
    \caption{Results for classes observed by the model 1–10 times during an epoch, evaluated on \fsplit~ for Phase 3, in terms of \mapft.}
    \resizebox{\columnwidth}{!}{%
    \setlength{\tabcolsep}{0.5em}
    \begin{tabular}{lrrr}
    \toprule
    \rowcolor{lightorange}   \textbf{Classes} & \textbf{Seen Count} & \textbf{ER+KD} & \textbf{ER+KD+IC} \\ 
    \midrule
        piano & 1 & 7.10 & 59.40 \\ 
        bucket & 1 & 21.10 & 31.50 \\ 
        laundry basket & 1 & 3.80 & 17.40 \\ 
        dresser & 2 & 55.00 & 55.40 \\ 
        paper towel dispenser & 2 & 32.50 & 35.50 \\ 
        cup & 2 & 24.70 & 30.30 \\ 
        bar & 2 & 35.40 & 39.50 \\ 
        divider & 2 & 28.60 & 42.40 \\ 
        case of water bottles & 2 & 0.00 & 1.70 \\ 
        shower & 3 & 0.00 & 45.50 \\ 
        mirror & 8 & 56.00 & 68.80 \\ 
        trash bin & 4 & 1.10 & 2.70 \\ 
        backpack & 5 & 74.50 & 76.70 \\ 
        copier & 5 & 94.00 & 96.80 \\ 
        bathroom counter & 3 & 3.90 & 20.30 \\ 
        ottoman & 4 & 32.60 & 36.20 \\ 
        storage bin & 3 & 5.10 & 10.50 \\ 
        dishwasher & 3 & 47.40 & 66.20 \\ 
        trash bin & 4 & 1.10 & 2.70 \\ 
        backpack & 5 & 74.50 & 76.70 \\ 
        copier & 5 & 94.00 & 96.80 \\ 
        sofa chair & 6 & 14.10 & 43.50 \\ 
        file cabinet & 6 & 49.20 & 57.60 \\ 
        tv stand & 7 & 67.70 & 68.60 \\ 
        mirror & 8 & 56.00 & 68.80 \\ 
        blackboard & 8 & 57.10 & 82.80 \\ 
        clothes dryer & 9 & 1.70 & 3.20 \\ 
        toaster & 9 & 0.10 & 25.90 \\ 
        wardrobe & 10 & 22.80 & 58.80 \\ 
        jacket & 10 & 1.20 & 4.10 \\ 
    \midrule
    \rowcolor{lightgray} \textbf{Average} & - & 32.08 & \textbf{44.21} \\ 
    \bottomrule
    \end{tabular}
    }
    \label{tab:tail_eval}
\end{table}
Similarly, \Cref{tab:tail_eval} presents results for \textit{Phase 3}, demonstrating significant gains for infrequent classes. For example, even though the classes such as \texttt{piano}, \texttt{bucket}, and \texttt{laundry basket} are observed only once, IC module improves the performance by 52.30\%, 10.40\%, and 13.60\%, respectively. The ER+KD module does not focus on rare classes like \texttt{shower} and \texttt{toaster} which results in low performance, but the IC module compensates for this imbalance by focusing on underrepresented categories. On average, the addition of the proposed IC module into the framework outperforms ER+KD by 12.13\%.

\begin{table*}[!ht]
    \centering
    \caption{Classes grouped by tasks for each proposed scenario on the ScanNet200 dataset labels. The three scenarios \fsplit, \fsplit, and \rsplit~ are each divided into three tasks: Task 1, Task 2, and Task 3.}
    \resizebox{\linewidth}{!}{%
    \setlength{\tabcolsep}{0.2em}
    \begin{tabular}{lll|lll|lll}
    \toprule
        % \rowcolor{lightorange} 
        \multicolumn{3}{c}{\fontfamily{lmtt}\fontseries{b}\selectfont{Split\_A}} & \multicolumn{3}{c}{\fontfamily{lmtt}\fontseries{b}\selectfont{Split\_B}} & \multicolumn{3}{c}{\fontfamily{lmtt}\fontseries{b}\selectfont{Split\_C}} \\ 
        \cmidrule(l{0.5cm}r{0.5cm}){1-3} \cmidrule(l{0.5cm}r{0.5cm}){4-6} \cmidrule(l{0.5cm}r{0.5cm}){7-9}
        % \rowcolor{lightorange} 
        \multicolumn{1}{l}{\textbf{Task 1}} & \multicolumn{1}{l}{\textbf{Task 2}} & \multicolumn{1}{l}{\textbf{Task 3}} & \multicolumn{1}{l}{\textbf{Task 1}} & \multicolumn{1}{l}{\textbf{Task 2}} & \multicolumn{1}{l}{\textbf{Task 3}} & \multicolumn{1}{l}{\textbf{Task 1}} & \multicolumn{1}{l}{\textbf{Task 2}} & \multicolumn{1}{l}{\textbf{Task 3}} \\ 
        % \cmidrule(lr){1-3} \cmidrule(lr){4-6} \cmidrule(lr){7-9} 
        \midrule
        chair & wall & pillow & tv stand & cushion & paper & broom & fan & rack \\ 
        table & floor & picture & curtain & end table & plate & towel & stove & music stand \\ 
        couch & door & book & blinds & dining table & soap dispenser & fireplace & tv & bed \\ 
        desk & cabinet & box & shower curtain & keyboard & bucket & blanket & dustpan & soap dish \\ 
        office chair & shelf & lamp & bookshelf & bag & clock & dining table & sink & closet door \\ 
        bed & window & towel & tv & toilet paper & guitar & shelf & toaster & basket \\ 
        sink & bookshelf & clothes & kitchen cabinet & printer & toilet paper holder & rail & doorframe & chair \\ 
        toilet & curtain & cushion & pillow & blanket & speaker & bathroom counter & wall & toilet paper \\ 
        monitor & kitchen cabinet & plant & lamp & microwave & cup & plunger & mattress & ball \\ 
        armchair & counter & bag & dresser & shoe & paper towel roll & bin & stand & monitor \\ 
        coffee table & ceiling & backpack & monitor & computer tower & bar & armchair & copier & bathroom cabinet \\ 
        refrigerator & whiteboard & toilet paper & object & bottle & toaster & trash bin & ironing board & shoe \\ 
        tv & shower curtain & blanket & ceiling & bin & ironing board & dishwasher & radiator & blackboard \\ 
        nightstand & closet & shoe & board & ottoman & soap dish & lamp & keyboard & vent \\ 
        dresser & computer tower & bottle & stove & bench & toilet paper dispenser & projector & toaster oven & bag \\ 
        stool & board & basket & closet wall & basket & fire extinguisher & potted plant & paper bag & paper \\ 
        bathtub & mirror & fan & couch & fan & ball & coat rack & structure & projector screen \\ 
        end table & shower & paper & office chair & laptop & hat & end table & picture & pillar \\ 
        dining table & blinds & person & kitchen counter & person & shower curtain rod & tissue box & purse & range hood \\ 
        keyboard & rack & plate & shower & paper towel dispenser & paper cutter & stairs & tray & coffee maker \\ 
        printer & blackboard & container & closet & oven & tray & fire extinguisher & couch & handicap bar \\ 
        tv stand & rail & soap dispenser & doorframe & rack & toaster oven & case of water bottles & telephone & pillow \\ 
        trash can & radiator & telephone & sofa chair & piano & mouse & water bottle & shower curtain rod & decoration \\ 
        stairs & wardrobe & bucket & mailbox & suitcase & toilet seat cover dispenser & ledge & trash can & printer \\ 
        microwave & column & clock & nightstand & rail & storage container & shower head & closet wall & object \\ 
        stove & ladder & stand & washing machine & container & scale & guitar case & cart & mirror \\ 
        bin & bathroom stall & light & picture & telephone & tissue box & kitchen cabinet & hat & ottoman \\ 
        ottoman & shower wall & pipe & book & stand & light switch & poster & paper cutter & water pitcher \\ 
        bench & mat & guitar & sink & light & crate & candle & storage organizer & refrigerator \\ 
        washing machine & windowsill & toilet paper holder & recycling bin & laundry basket & power outlet & bowl & vacuum cleaner & divider \\ 
        copier & bulletin board & speaker & table & pipe & sign & plate & mouse & toilet \\ 
        sofa chair & doorframe & bicycle & backpack & seat & projector & person & paper towel roll & washing machine \\ 
        file cabinet & shower curtain rod & cup & shower wall & column & candle & storage bin & laundry detergent & mat \\ 
        laptop & paper cutter & jacket & toilet & bicycle & plunger & microwave & calendar & scale \\ 
        paper towel dispenser & shower door & paper towel roll & copier & ladder & stuffed animal & office chair & wardrobe & dresser \\ 
        oven & pillar & machine & counter & jacket & headphones & clothes dryer & whiteboard & bookshelf \\ 
        piano & ledge & soap dish & stool & storage bin & broom & headphones & laundry basket & tv stand \\ 
        suitcase & light switch & fire extinguisher & refrigerator & coffee maker & guitar case & toilet seat cover dispenser & shower door & closet rod \\ 
        recycling bin & closet door & ball & window & dishwasher & dustpan & bathroom stall door & curtain & plant \\ 
        laundry basket & shower floor & hat & file cabinet & machine & hair dryer & speaker & folded chair & counter \\ 
        clothes dryer & projector screen & water cooler & chair & mat & water bottle & keyboard piano & suitcase & bench \\ 
        seat & divider & mouse & wall & windowsill & handicap bar & cushion & hair dryer & ceiling \\ 
        storage bin & closet wall & scale & plant & bulletin board & purse & table & mini fridge & piano \\ 
        coffee maker & bathroom stall door & power outlet & coffee table & fireplace & vent & nightstand & dumbbell & closet \\ 
        dishwasher & stair rail & decoration & stairs & mini fridge & shower floor & bathroom vanity & oven & cabinet \\ 
        bar & bathroom cabinet & sign & armchair & water cooler & water pitcher & laptop & luggage & cup \\ 
        toaster & closet rod & projector & cabinet & shower door & bowl & shower wall & bar & laundry hamper \\ 
        ironing board & structure & vacuum cleaner & bathroom vanity & pillar & paper bag & desk & pipe & light switch \\ 
        fireplace & coat rack & candle & bathroom stall & ledge & alarm clock & computer tower & bathroom stall & cd case \\ 
        kitchen counter & storage organizer & plunger & mirror & furniture & music stand & soap dispenser & blinds & backpack \\ 
        toilet paper dispenser & ~ & stuffed animal & blackboard & cart & laundry detergent & container & toilet paper dispenser & windowsill \\ 
        mini fridge & ~ & headphones & trash can & decoration & dumbbell & bicycle & coffee table & box \\ 
        tray & ~ & broom & stair rail & closet door & tube & light & dish rack & book \\ 
        toaster oven & ~ & guitar case & box & vacuum cleaner & cd case & clothes & guitar & mailbox \\ 
        toilet seat cover dispenser & ~ & hair dryer & towel & dish rack & closet rod & machine & seat & sofa chair \\ 
        furniture & ~ & water bottle & door & range hood & coffee kettle & furniture & clock & shower curtain \\ 
        cart & ~ & purse & clothes & projector screen & shower head & stair rail & alarm clock & bulletin board \\ 
        storage container & ~ & vent & whiteboard & divider & keyboard piano & toilet paper holder & board & crate \\ 
        tissue box & ~ & water pitcher & bed & bathroom counter & case of water bottles & floor & file cabinet & tube \\ 
        crate & ~ & bowl & floor & laundry hamper & coat rack & bucket & ceiling light & window \\ 
        dish rack & ~ & paper bag & bathtub & bathroom stall door & folded chair & stool & ladder & power outlet \\ 
        range hood & ~ & alarm clock & desk & ceiling light & fire alarm & door & paper towel dispenser & power strip \\ 
        dustpan & ~ & laundry detergent & wardrobe & trash bin & power strip & sign & shower floor & bathtub \\ 
        handicap bar & ~ & object & clothes dryer & bathroom cabinet & calendar & recycling bin & stuffed animal & column \\ 
        mailbox & ~ & ceiling light & radiator & structure & poster & shower & water cooler & fire alarm \\ 
        music stand & ~ & dumbbell & shelf & storage organizer & luggage & jacket & coffee kettle & storage container \\ 
        bathroom counter & ~ & tube & ~ & potted plant & ~ & bottle & kitchen counter & ~ \\ 
        bathroom vanity & ~ & cd case & ~ & mattress & ~ & ~ & ~ & ~ \\ 
        laundry hamper & ~ & coffee kettle & ~ & ~ & ~ & ~ & ~ & ~ \\ 
        trash bin & ~ & shower head & ~ & ~ & ~ & ~ & ~ & ~ \\ 
        keyboard piano & ~ & case of water bottles & ~ & ~ & ~ & ~ & ~ & ~ \\ 
        folded chair & ~ & fire alarm & ~ & ~ & ~ & ~ & ~ & ~ \\ 
        luggage & ~ & power strip & ~ & ~ & ~ & ~ & ~ & ~ \\ 
        mattress & ~ & calendar & ~ & ~ & ~ & ~ & ~ & ~ \\ 
        ~ & ~ & poster & ~ & ~ & ~ & ~ & ~ & ~ \\ 
        ~ & ~ & potted plant & ~ & ~ & ~ & ~ & ~ & ~ \\ 
    \bottomrule
    \end{tabular}
    }
    \label{tab:split_fullview}
\end{table*}
\section{Incremental Scenarios Phases}
\label{app:split_fullview}
\Cref{tab:split_fullview} presents the task splits for each proposed scenario introduced in Section 4.3 using the ScanNet200 dataset. The three scenarios, \fsplit, \ssplit, and \rsplit, are each divided into three tasks: Task 1, Task 2, and Task 3. Notably, the order of classes in these tasks is random. 


\section{Qualitative Results}
\label{app:qual_results}
In this section, we present a qualitative comparison of the proposed framework with the baseline method. \Cref{fig:fsplit_qual} illustrates the results on the \fsplit~evaluation after learning all tasks, comparing the performance of the baseline method and our proposed approach. As shown in the figure, our method demonstrates superior instance segmentation performance compared to the baseline. For example, in row 1, the baseline method fails to segment the \texttt{sink}, while in row 3, the \texttt{sofa} instance is missed. Overall, our framework consistently outperforms the baseline, with several missed instances by the baseline highlighted in red circles.

\begin{figure*}[!htb]
  \centering
  \includegraphics[width=\linewidth]{fig/freq_qual_grid_annot.pdf}
  \caption{Qualitative comparison of ground truth, the baseline method, and our proposed framework on the \fsplit~ evaluation after learning all tasks.}
  \label{fig:fsplit_qual}
\end{figure*}

In \Cref{fig:ssplit_qual}, we present the results on \ssplit, highlighting instances where the baseline method underperforms, marked with red circles. For example, in row 2, the baseline method incorrectly identifies the same sofa as separate instances. Similarly, in row 5, the washing machine is segmented into two instances by the baseline. In contrast, the proposed method delivers results that closely align with the ground truth, demonstrating its superior performance

\begin{figure*}[!htb]
  \centering
  \includegraphics[width=\linewidth]{fig/sema_qual_grid_annot.pdf}
  \caption{Qualitative comparison of ground truth, the baseline method, and our proposed framework on the \ssplit~ evaluation after learning all tasks.}
  \label{fig:ssplit_qual}
\end{figure*}

Similarly, \Cref{fig:rsplit_qual} highlights the results on \rsplit, where classes are encountered in random order. The comparison emphasizes the advantages of our method, as highlighted by red circles. The baseline method often misses instances or splits a single instance into multiple parts. In contrast, our approach consistently produces results that are closely aligned with the ground truth, further underscoring its effectiveness.

\begin{figure*}[!htb]
  \centering
  \includegraphics[width=\linewidth]{fig/rand_qual_grid_annot.pdf}
  \caption{Qualitative comparison of ground truth, the baseline method, and our proposed framework on the \rsplit~ evaluation after learning all tasks.}
  \label{fig:rsplit_qual}
\end{figure*}