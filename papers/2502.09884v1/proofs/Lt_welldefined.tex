\subsection{Size of Linear Transformation}
\begin{proposition}\label{prop:Lt_welldefined}
    The sequence $\{L_t\}$ is well-defined and converges to zero at rate $\mathcal{O}(\gamma_t/\alpha_t)$.
\end{proposition}
The recursive definition for $L_{t+1}$ in Eq. \eqref{eq:Lt} is equivalently expressed as 
\begin{align*}
    L_{t+1} (I - \gamma_t \Delta + \gamma_t A_{sf} L_t) 
    &= \left(I - \alpha_t A_{ff} - \gamma_t A_{ff}^{-1} A_{fs} A_{sf} \right) L_t + \gamma_t A_{ff}^{-1} A_{fs} \Delta 
    \\ & 
    = \left(I - \alpha_t A_{ff}\right) L_t + \gamma_t A_{ff}^{-1} A_{fs}(\Delta - A_{sf} L_t) .
\end{align*}
Denoting $B_t^{ss} = \Delta - A_{sf} L_t$ as before, 
\begin{equation}
    L_{t+1} (I - \gamma_t B_t^{ss}) = (I - \alpha_t A_{ff}) L_t + \gamma_t A_{ff}^{-1} A_{fs} B_t^{ss} .
\end{equation}
It was shown in Lemma 18 of \citep{kaledin2020finite} that $I - \gamma_t B_t^{ss}$ is non-singular and that $\lVert L_t \rVert$ is uniformly bounded.
Therefore, $ \max_{t \leq n} \prod_{j=t+1}^n \lVert (I - \gamma_t B_t^{ss})^{-1} \rVert$ is uniformly bounded by some constant $K_{ss}$ and we have
\begin{align*}
    \lVert L_{t+1} \rVert 
    &\leq \lVert I - \alpha_t A_{ff} \rVert \lVert L_t (I - \gamma_t B_t^{ss})^{-1} \rVert + \gamma_t \lVert A_{ff}^{-1} A_{fs} B_t^{ss} \rVert 
    \\ & \leq 
    \sqrt{1 - \alpha_t \frac{\mu_{ff}}{2}} \lVert (I - \gamma_t B_n^{ss})^{-1}\rVert \lVert L_t \rVert 
    + \gamma_t \lVert A_{ff}^{-1} A_{fs} B_n^{ss} \rVert 
    \\ & \leq \left(1 - \frac{\mu_{ff}}{4} \alpha_t\right) \lVert (I - \gamma_t B_t^{ss})^{-1} \rVert \lVert L_t \rVert + \gamma_t \lVert A_{ff}^{-1} A_{fs} B_t^{ss} \rVert 
\end{align*}
when $\alpha_t \leq \mu/\nu^2$, where we used that $\sqrt{1 - x} \leq 1 - x/2$ when $x \in [0, 1]$. 
By induction and Eq. \eqref{eq:induction_size_equation}, there exists a constant $K_L$ such that
\begin{align*}
    \frac{1}{K_{ss}}\lVert L_{n+1} \rVert 
    & \leq \prod_{t=1}^n \left(1 - \frac{\mu_{ff}}{4} \alpha_t \right) \lVert L_1 \rVert + \sum_{t=1}^n \gamma_t  \prod_{j=t+1}^n \left(1 - \frac{\mu_{ff}}{4} \alpha_j\right)  \lVert A_{ff}^{-1} A_{fs} B_t^{ss} \rVert 
    \\ & \leq 
    \prod_{t=1}^n \left(1 - \frac{\mu_{ff}}{4} \alpha_t \right) \lVert L_1 \rVert + K_L \frac{\gamma_n}{\alpha_n}
    .
\end{align*}



% \subsection{Remove after completing above; Proof of Proposition \ref{prop:Lt_welldefined}}




% {\color{red}
% Remove after finishing above:
% I need to merge this to blend in with this paper. 
% After parsing the requirements on my own, merge them into Assumption \ref{assumption:steps}.
% }
% \begin{proposition}\label{prop:Lt_welldefined}
%     {\color{red}Needs fixing/proving:}
%     When step sizes are chosen so that
%     \begin{equation}\label{eq:stepsize}
%         \gamma_1 \leq \min \frac{1}{2} \left\{ \frac{1}{\lVert Q_\Delta \rVert^2 \lVert \Delta\rVert^2_{Q_\Delta}}, \frac{\lVert Q_\Delta\rVert^2}{\lVert \Delta \rVert_{Q_\Delta} \lVert Q_\Delta\rVert^2 + 1} \right\}
%         , \quad 
%         \alpha_1 \leq \frac{1}{2 \lVert Q_{ff}\rVert^2 \lVert A_{ff}\rVert_{Q_ff}^2} ,
%     \end{equation}    
%     there exists a sequence of matrices $\{L_t\}$ with $L_0 = 0$ such that $I - \gamma_t (\Delta - A_{sf}L_t)$ is non-singular. 
%     Consequently, there exists a sequence satisfying the recursion \eqref{eq:Lt} for all $t >0$.
% \end{proposition}
% % \begin{customprop}{\ref{prop:Lt_welldefined}}
% %     When step sizes are chosen so that
% %     \begin{equation}\label{eq:stepsize1}
% %         % \gamma_0 \leq \min \frac{1}{2} \left\{ \frac{1}{\lVert Q_\Delta \rVert^2 \lVert \Delta\rVert^2_{Q_\Delta}}, \frac{\lVert Q_\Delta\rVert^2}{\lVert \Delta \rVert_{Q_\Delta} \lVert Q_\Delta\rVert^2 + 1} \right\}
% %         % , \quad 
% %         % \alpha_0 \leq \frac{1}{2 \lVert Q_{ff}\rVert^2 \lVert A_{ff}\rVert_{Q_ff}^2} ,
% %         \alpha_t \leq \frac{1}{2} \lVert Q_{ff} \rVert^{-2} \lVert A_{ff} \rVert^{-2}_{Q_{ff}} , \quad
% %         \gamma_t \leq \frac{1}{2} \left(\lVert \Delta \rVert_{Q_\Delta} + L_\infty \lVert A_{sf}\rVert_{Q_{ff}, Q_\Delta}\right)^{-1} ,
% %     \end{equation}    
% %     there exists a sequence of matrices $\{L_t\}$ with $L_0 = 0$ such that $I - \gamma_t (\Delta - A_{sf}L_t)$ is non-singular. 
% %     Consequently, there exists a sequence satisfying the recursion \eqref{eq:Lt} for all $t >0$.
% %     {\color{red}     There might be slight edits we need to make on the step sizes. 
% %     Must guarantee that all conditions used are met. }
% % \end{customprop}
% This will be proved with induction: (1) Show that $L_1$ is well-defined with bounded norm.
% (2) If $\lVert L_t \rVert_{Q_\Delta, Q_{ff}} \leq L_\infty$.
% {\color{red}To do: Step (1).
% }


% The second step is shown below.
% First observe that the recursion for $L_t$ can be written with a matrix identity on $L = L_t, L' = L_{t+1}$:
% \begin{align*}
%     L' (I-\gamma_t \Delta + \gamma_t A_{sf} L) 
%     &= \left(I - \alpha_t A_{ff} - \gamma_t A_{ff}^{-1} A_{fs} A_{sf}\right) L + \gamma_t A_{ff}^{-1} A_{fs} \Delta  
%     \\
%     &= (I - \alpha_t A_{ff}) L + \gamma_t A_{ff}^{-1} A_{fs} (\Delta - A_{sf} L)
%     .
%     \end{align*}
% Denoting $B^{ss}_t = \Delta - A_{sf} L_t$, we then have the expression
% \begin{equation}\label{eq:Ltidentity}
%     L' (I - \gamma_t B_t^{ss}) = (I - \alpha_t A_{ff})L + \gamma_t A_{ff}^{-1} A_{fs} B_t^{ss}.
% \end{equation}
% We will now see that there exists a unique solution to \eqref{eq:Ltidentity} if $\alpha_t, \gamma_t$ are sufficiently small.
% An upper bound on the norm of $L'$ can then be derived using that (Lemma 17, \citet{kaledin2020finite}) when $\alpha_t \leq \lVert Q_{ff} \rVert^2$, then
% \begin{equation}
%     \lVert I - \alpha_t A_{ff} \rVert_{Q_{ff}} \leq 1 - \alpha_t a_{ff} , \quad
%     a_{ff} = \frac{1}{2} \lVert Q_{ff} \rVert^{-2} .
% \end{equation}


% \begin{lemma}[Lemma 18, \citet{kaledin2020finite}]\label{lem:Ltsize}
%     Suppose $\lVert L \rVert_{Q_\Delta, Q_{22}} \leq L_\infty$ and consider the step sizes
%     \begin{align*}
%         \alpha_t \leq \frac{1}{2} \lVert Q_{ff} \rVert^{-2} \lVert A_{ff} \rVert^{-2}_{Q_{ff}} , \quad
%         \gamma_t \leq \frac{1}{2} \left(\lVert \Delta \rVert_{Q_\Delta} + L_\infty \lVert A_{sf}\rVert_{Q_{ff}, Q_\Delta}\right)^{-1} .
%     \end{align*}
%     Then there exists a unique solution $L'$ to \eqref{eq:Ltidentity} that satisfies
%     \begin{align*}
%         \lVert L' \rVert_{Q_\Delta, Q_{ff}} 
%             &\leq 
%         (1 - \alpha_t a_{ff}) \lVert L \rVert_{Q_{\Delta}, Q_{ff}} + \gamma_t C_D (L_\infty) , 
%         \\
%         C_D (L_\infty) 
%             &= 
%         2
%         \left(
%             \lVert A_{ff}^{-1} A_{fs} \rVert_{Q_\Delta, Q_{ff}} + L_\infty
%         \right) 
%         \left(
%             \lVert \Delta \rVert_{Q_\Delta} + L_\infty \lVert A_{sf} \rVert_{Q_{ff}, Q_\Delta} 
%         \right)
%     \end{align*}
%     Moreover, this implies that if $\gamma_t/\alpha_t \leq \epsilon a_{ff}/C_D(L_\infty)$, then $\lVert L'\rVert_{Q_\Delta, Q_{ff}} \leq L_\infty$.
% \end{lemma}
% Substituting back $L_t$ and $L_{t+1}$, we then have by induction that
% \begin{align*}
%     \lVert L_t\rVert_{Q_\Delta, Q_{ff}}
%         \leq 
%     C_D (L_\infty)
%     \sum_{j=0}^t \gamma_j \prod_{k=j+1}^t (1 - \alpha_k a_{ff} ) .
% \end{align*}
% The last term including summation is a recurring quantity which is extensively analyzed in Lemma 14 \citep{kaledin2020finite}, in this case (iv) asserts that for $\alpha_0 \leq a_{ff}^{-1} $,
% \begin{align*}
%     & \sum_{j=0}^t \gamma_j \prod_{k=j+1}^t (1 - \alpha_k a_{ff} ) \leq
%     \frac{2\gamma_t}{a_{ff} \alpha_t}  
%         \max \left\{
%              \xi \max \left\{1, \frac{a_{ff}}{4 a_\Delta}\right\},
%             2 \xi^3
%     \right\} , \\
%     & \xi = 1 + \max\left\{ \frac{a_{ff}}{8}\alpha_0, \frac{a_{\Delta}}{16} \gamma_0\right\} , \quad
%     a_{\Delta} = \frac{1}{2 \lVert Q_\Delta \rVert^2} .
% \end{align*}
% {\color{red}Lastly, we need to show that this upper bound is again bounded above by $L_\infty$, which I don't think is shown in \citep{kaledin2020finite}.}




% \textbf{Relating operator norm with weighted norm:}
% From the elementary inequalities
% \begin{align*}
%     \lambda_{\min}(Q) \lVert x \rVert^2 \leq x^T Q x \leq \lambda_{\max}(Q) \lVert x \rVert^2 ,
% \end{align*}
% we obtain that substituting $L x$ for $x$ above, the following is satisfied for any $Q \succ 0$:
% \begin{equation}
%     \frac{\lambda_{\min}(Q)}{\lambda_{\max}(Q)} \lVert L \rVert^2
%     \leq 
%     \lVert L \rVert_Q^2 = \sup_{x \neq 0} \frac{x^T L Q L x}{x^T Q x} 
%     \leq 
%     \frac{\lambda_{\max}(Q) }{\lambda_{\min}(Q)} \lVert L \rVert^2 .
% \end{equation}
% This is obtained by optimizing the numerator and denominator separately.
% From the lower bound, we have that for any positive definite $Q$
% \begin{equation}
%     \lVert L \rVert^2 \leq \kappa (Q) \lVert L \rVert_Q^2 . 
% \end{equation}
% More generally,
% \begin{equation}
%     \lVert L_t \rVert^2 \leq \frac{\lambda_{\max}(Q_{\Delta})}{\lambda_{\min}(Q_{ff})} \lVert L_t \rVert_{Q_\Delta, Q_{ff}}^2
% \end{equation}
% To relate this back to the properties of $\Delta$ and $A_{ff}$, we need to evaluate the minimum and maximum eigenvalues of a matrix $Q$ that solves the Lyapunov equation.
% \begin{proposition}[Theorem 10, \citet{lancaster1970explicit}]
%     Let $Q \succ 0$ be the unique solution to the Lypaunov equation
%     \begin{equation}
%         A Q + Q A^T = I 
%     \end{equation}
%     for a stable matrix $-A$.
%     Then,
%     \begin{align}
%         \frac{1}{\lvert \lambda_{\min}(A + A^T) \rvert} &\leq \lambda_{\min}(Q) \leq \frac{1}{2 \lvert \lambda_{\min}(A) \rvert} 
%         \\ 
%         \frac{1}{2 \lvert \lambda_{\max}(A)\rvert } &\leq \lambda_{\max}(Q) \leq \frac{1}{\lvert \lambda_{\max}(A + A^T)\rvert} . 
%     \end{align}
%     % {\color{red}But $A + A^T$ may not be Hurwitz? I may need to refine this. 
%     % Check details of this reference.
%     % }
% \end{proposition}
% {\color{blue}To do:
%     Using this, obtain a bound on the unweighed norm $\lVert L_t\rVert$ for all $t$ as a function of $\Delta$ and $A_{ff}$, in place of $Q_\Delta, Q_{ff}$.
% }
% \begin{lemma}\label{lem:unweighted_norm}
%     The unweighted norm $\lVert L_t \rVert$ is uniformly bounded for all time. {\color{red}Specify problem parameters.}
% \end{lemma}



% \subsection{Bound on $\Phi$: Lemma 12 \& 14 in \citep{kaledin2020finite}}
% Step 1: (E.g. Lemma 17 in \citep{kaledin2020finite}).
% For every $\alpha_t \leq a_{ff}$, {\color{red}Correct step size requirement.}
% \begin{equation}
%     \lVert I - \alpha_t A_{ff} \rVert_{Q_{ff}} \leq 1 - \alpha_t a_{ff} .
%     % \sum_{j=0}^{T-1} \alpha_j \sum_{}
% \end{equation}

% Step 2: Lemma 12 in \citep{kaledin2020finite}.

% Step 3: Relate back to unweighted norm. 

% Note that we also need to show that
% \begin{equation}
%     \frac{1}{T}\sum_{j=0}^{T-1} \lVert \Phi_{jT} \rVert^2 = \mathcal{O}(T^{-(1-\delta)})
% \end{equation}
% for some $\delta$ as in Appendix B \citep{srikant2024CLT}.
% Universal bound is not sufficient, since it would give a much worse rate of a constant above. 