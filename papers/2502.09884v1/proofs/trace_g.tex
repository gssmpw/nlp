

\subsection{Trace inequality for $g_t$}
{\color{red}Exercise verifying trace inequality.}
\begin{align*}
    g_t (C_t) &= 
    -\mathrm{Tr}\delta_t \left( A_{sf} C_t + \Delta^T \right) A_{sf}
    + 
    \alpha_t \mathrm{Tr} A_{ff} C_t \Delta^T  A_{sf}
    - \alpha_t \mathrm{Tr} A_{ff} C_t L_t^T A_{sf}^T A_{sf}
    \\ & 
    + \gamma_t \mathrm{Tr} \delta_t A_{sf} C_t \Delta^T A_{sf}
    + \mathrm{Tr} C_t L_t^T A_{sf}^T A_{sf}
    + \gamma_t \mathrm{Tr} \delta_t A_{sf} C_t L_t^T A_{sf}^T A_{sf}
    \\ &
    + \alpha_t \mathrm{Tr} \left(A_{ff} X_t A_{sf}^T 
    + \frac{\gamma_t}{\alpha_t} \left(\delta_t \Gamma_{ss} - \delta_t A_{sf} X_t A_{sf}^T \right) A_{sf}\right) .
\end{align*}


We can use the inequality that for arbitrary matrices $A, X$ of appropriate size, 
\begin{align*}
    \lvert \mathrm{Tr} A^T X \rvert \leq \lVert A \rVert \mathrm{Tr} \lvert X \rvert ,
\end{align*}
where $\lvert X \rvert = (X^T X)^{1/2}$ and $\lVert \cdot \rVert$ is the spectral norm.
Then we have that 
\begin{align*}
    \lvert g_t (C_t) \rvert \leq 
    \lVert \delta_t A_{sf} \rVert \mathrm{Tr} \lvert A_t C_t \rvert 
    + 
    \left(
        \alpha_t \lVert \Delta^T A_{sf} \rVert 
        - \alpha_t \lVert L_t^T A_{sf}^T \rVert 
    \right)
    \mathrm{Tr} \lvert A_{sf} A_{ff} C_t \rvert
    + \gamma_t \lVert \delta_t + \cdots .
\end{align*}
For the terms where there are more matrices inside, I think we can pull them out. 


Q. 
Can we instead write the recursion in terms of $\mathrm{Tr} \lvert C_t A_{sf} \rvert$?
Then, prove that 
\begin{align*}
    \mathrm{Tr} C_t A_{sf} \leq \mathrm{Tr} \lvert C_t A_{sf} \rvert .
\end{align*}
This is true because
\begin{align*}
    \mathrm{Tr} C_t A_{sf} \leq \lvert \mathrm{Tr} C_t A_{sf} \rvert \leq \mathrm{Tr} \lvert C_t A_{sf} \rvert
\end{align*}
(think of the singular values).
{\color{red}Actually, I think it will be easiest if we just prove the inequality for $\lvert g_t (C_t)$, since then the result is immediate.
Problem is, then we can't subsume the terms in $g_t (C_t)$ into the recursion; since we don't a-priori know that $C_t$ is uniformly bounded, we can't really say that $g_t (C_t)$ is bounded. 
}

% We want an inequality of the form
% \begin{align*}
%     \mathrm{Tr} A X \leq a \lvert \mathrm{Tr} X \rvert .
% \end{align*}
% Then go back and consider the two cases separately, when $\mathrm{Tr} C_t A_{sf}$ is positive vs. negative.
% One simple inequality is
% \begin{align*}
%     \lvert \mathrm{Tr} A X \rvert = \lvert \sum_{ij} A_{ij} X_{ij} \rvert \leq \max_{ij} \lvert X_{ij} \rvert \lVert A \rVert_{11}
% \end{align*}

% We can use the Frobenius inner product to express the trace of a product of two arbitrary rectaungular matrices $A, B$, which is subject to Cauchy-Schwartz inequality
% \begin{align*}
%     \mathrm{Tr} A^T B = \langle A, B \rangle_F \leq \lVert A \rVert_F \lVert B \rVert_F .
% \end{align*}
% % Using that $\lVert A \rVert_F = \sqrt{\sum_{i} \sigma_i^2 (A)} \leq \sum_{i} \sigma_i (A) = \sum_i \lvert \lambda_i (A)$, we then have, using the notation that for a matrix $A = U \Lambda V$, where $\Lambda$ is the singular value matrix of $A$, $\lvert A \rvert = U \lvert \Lambda \rvert V$ with $\lvert \cdot \rvert$ applied element-wise to a diagonal matrix:
% Therefore, we can express the trace of the product as a weighted multiple of the norm of singular values of $B$:
% \begin{align*}
%     \mathrm{Tr} A^T B \leq \lVert A \rVert_F \lVert \sigma_B \rVert_2 .
% \end{align*}


% For example if we want to obtain a bound on the first term in $g_t (C_t)$,
% \begin{align*}
%     \mathrm{Tr} (\delta_t A_{sf}) (C_t A_{sf}) \leq  \lVert \delta_t A_{sf} \rVert_F \lVert C_t A_{sf} \rVert_F
% \end{align*}