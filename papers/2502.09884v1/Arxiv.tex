\documentclass[12pt]{arxiv} % Anonymized submission
%\documentclass[final,12pt]{colt2025} % Include author names
% \documentclass[12pt]{article}

% TODO: Remove the page numbers [start]-[end] on top of first page. 
% \usepackage{fancyhdr}
% \pagestyle{empty}
% \fancypagestyle{firstpage}{%
%   \fancyhf{}  % Clear headers and footers
%   \renewcommand{\headrulewidth}{0pt}  % Remove top rule
% }
% \thispagestyle{firstpage}  % Apply to first page

% From the jmlr.cls file. jmlr.cls is likely being retrieved from overleaf. 
\makeatletter
\def\ps@jmlrtps{%
  \let\@mkboth\@gobbletwo
  \def\@oddhead{}  % Remove the first page header
  \let\@evenhead\@oddhead
  \def\@oddfoot{}  % Remove footer if necessary
  \let\@evenfoot\@oddfoot
}
\makeatother
% TODO: Remove the page numbers [start]-[end] on top of first page. 


\usepackage[utf8]{inputenc} % allow utf-8 input
\usepackage[T1]{fontenc}    % use 8-bit T1 fonts
\usepackage{hyperref}       % hyperlinks
\usepackage{booktabs}       % professional-quality tables
\usepackage{amsfonts}       % blackboard math symbols
\usepackage{nicefrac}       % compact symbols for 1/2, etc.
\usepackage{microtype}      % microtypography
\usepackage{bbold}
\usepackage{capt-of}

% \usepackage[center]{caption}
% \usepackage{caption}


\usepackage{cite,epstopdf,color,soul,mathabx}
\usepackage{makecell}
% \usepackage[ruled,vlined]{algorithm2e}
\allowdisplaybreaks
\usepackage{color}
\usepackage{enumerate}
\usepackage[shortlabels]{enumitem}
\usepackage{multicol}
\usepackage{empheq}
% \usepackage{caption} 
% \captionsetup[table]{skip=10pt}
\usepackage{algorithm,algorithmic}

  
\let\underbrace\LaTeXunderbrace
\let\overbrace\LaTeXoverbrace

% \documentclass{l4dc2024}

% \title{Non-asymptotic Central Limit Theorem for Two Timescale Stochastic Approximation}


% The following packages will be automatically loaded:
% amsmath, amssymb, natbib, graphicx, url, algorithm2e

% \title[Short Title]{Full Title of Article}
% \title[Quantitative CLT for Two Timescale Stochastic Approximation]{Non-asymptotic Central Limit Theorem for Two Timescale Stochastic Approximation}
\title[Quantitative CLT for Two-Time-Scale Stochastic Approximation]{
Nonasymptotic CLT and Error Bounds for Two-Time-Scale Stochastic Approximation
% Non-asymptotic CLT and Error Bounds for Averaged Iterates in Two-Time-Scale Linear Stochastic Approximation
% Averaged Iterates Achieve Optimal Rates in Two Timescale Stochastic Approximation
% Achieving Optimal Rates in Two Timescale Stochastic Approximation with Polyak Ruppert Averaging and Confidence Intervals
}
% \usepackage{times}
% Use \Name{Author Name} to specify the name.
% If the surname contains spaces, enclose the surname
% in braces, e.g. \Name{John {Smith Jones}} similarly
% if the name has a "von" part, e.g \Name{Jane {de Winter}}.
% If the first letter in the forenames is a diacritic
% enclose the diacritic in braces, e.g. \Name{{\'E}louise Smith}

% Two authors with the same address
% \coltauthor{\Name{Author Name1} \Email{abc@sample.com}\and
%  \Name{Author Name2} \Email{xyz@sample.com}\\
%  \addr Address}

% Three or more authors with the same address:
% \coltauthor{\Name{Author Name1} \Email{an1@sample.com}\\
%  \Name{Author Name2} \Email{an2@sample.com}\\
%  \Name{Author Name3} \Email{an3@sample.com}\\
%  \addr Address}

% Authors with different addresses:
% \coltauthor{%
%  \Name{Author Name1} \Email{abc@sample.com}\\
%  \addr Address 1
%  \AND
%  \Name{Author Name2} \Email{xyz@sample.com}\\
%  \addr Address 2%
% }

\coltauthor{%
 \Name{Seo Taek Kong} \Email{skong10@illinois.edu}\\
 \addr University of Illinois, Urbana-Champaign
 \AND
 \Name{Sihan Zeng} \Email{szeng2017@gmail.com}\\
 \addr JPMorgan AI Research
 \AND
 \Name{Thinh T. Doan} \Email{thinhdoan@utexas.edu}\\
 \addr University of Texas at Austin
 \AND 
 \Name{R. Srikant} \Email{rsrikant@illinois.edu}\\
 \addr University of Illinois, Urbana-Champaign
}

\usepackage{times}

\usepackage{mathtools}
\usepackage{nccmath}
\usepackage{esvect}
\usepackage{cleveref} % Custom numbering of theorems.
% \usepackage{amsthm} % Conflict.
\usepackage{mathrsfs}

% Theorems
\newtheorem{innercustomgeneric}{\customgenericname}
\providecommand{\customgenericname}{}
\newcommand{\newcustomtheorem}[2]{%
  \newenvironment{#1}[1]
  {%
   \ifdefined\crefalias\crefalias{innercustomgeneric}{#2}\fi
   \renewcommand\customgenericname{#2}%
   \renewcommand\theinnercustomgeneric{##1}%
   \innercustomgeneric
  }
  {\endinnercustomgeneric}%
  \ifdefined\crefname\crefname{#2}{#2}{#2s}\fi
}

\newtheorem{assumption}{Assumption}
\newtheorem{proposition*}{Proposition} % No counting\newcustomtheorem{customthm}{Theorem}
\newcustomtheorem{customlemma}{Lemma}
\newcustomtheorem{customprop}{Proposition}


\newenvironment{mpmatrix}{\begin{medsize}\begin{pmatrix}}{\end{pmatrix}\end{medsize}}%

\newcommand\numberthis{\addtocounter{equation}{1}\tag{\theequation}}

\DeclarePairedDelimiter\ceil{\lceil}{\rceil}
\DeclarePairedDelimiter\floor{\lfloor}{\rfloor}
\DeclarePairedDelimiter\abs{\lvert}{\rvert}%
\DeclarePairedDelimiter\norm{\lVert}{\rVert}%
\DeclarePairedDelimiterX{\inp}[2]{\langle}{\rangle}{#1, #2}

\newcommand{\expect}[1]{\mathbb{E}\left[#1\right]}
\newcommand{\history}{\mathcal{H}_t}
\newcommand{\historyprev}{\mathcal{H}_{t-1}}
% \newcommand{\historynext}{\mathcal{H}_{t+1}}

% \usepackage{subcaption}
% \usepackage[compatibility=false]{subcaption}
\usepackage{graphicx}
\usepackage{float}
\newcommand{\CLT}{%
Let $\bar{z}_n \coloneqq (\bar{x}_n, \bar{y}_n)$ be the Ruppert-Polyak average of $(x_t, y_t)$ generated by \eqref{eq:ttsa}. 
Define the asymptotic covariance of $\sqrt{n}(\bar{x}_n - x^*)$ and $\sqrt{n}(\bar{y}_n - y^*)$ as
\begin{equation}
    \bar{\Sigma}_{ff} = \lim_{n \to \infty } n \mathbb{E} (\bar{x}_n - x^*) (\bar{x}_n - x^*) ^T , \quad
    \bar{\Sigma}_{ss} = \lim_{n \to \infty } n \mathbb{E} (\bar{y}_n - y^*) (\bar{y}_n - y^*)^T .
\end{equation}
Under Assumptions \ref{assumption:first}--\ref{assumption:last}, the above limits exist, and the rate of convergence in the Wasserstein-1 distance is given by:
    \begin{equation}
        \begin{split}
            d_1 \left(\sqrt{n} G (\bar{x}_n - x^*), (G \bar{\Sigma}_{ff} G^T)^{1/2} Z_1 \right) 
            &= 
            \mathcal{O}\left(\frac{1}{\sqrt{n}} \left(n^{a/2} + n^{a-b/2} + n^{b/2}\right) \right), 
            \\ 
            d_1 \left(\sqrt{n} \Delta (\bar{y}_n - y^*), (\Delta \bar{\Sigma}_{ss} \Delta^T)^{1/2} Z_2 \right) &= 
            \mathcal{O}\left(\frac{1}{\sqrt{n}} \left(n^{a/2} + n^{a-b/2} + n^{b/2}\right) \right), 
        \end{split} %\label{eq:WassersteinBound}
    \end{equation}
    where $Z_1$ and $Z_2$ are standard Gaussian vectors of appropriate dimensions.
}

\newcommand{\MSE}{%
    Assume \ref{assumption:first}--\ref{assumption:last}.
    Let $\Sigma$ be the asymptotic covariance of $(x_t - x^*, y_t - y^*)$, evaluated as
    \begin{align*}
        A_{ff} \Sigma_{ff} + \Sigma_{ff} A_{ff}^T &= \Gamma_{ff} , \\
        A_{ff} \Sigma_{fs} + \Sigma_{ff} A_{sf}^T &= \Gamma_{fs} , \\ 
        \Delta \Sigma_{ss} + \Sigma_{ss} \Delta^T + A_{sf} \Sigma_{fs} + \Sigma_{sf} A_{sf}^T &= \Gamma_{ss} 
        .
        \numberthis \label{eq:covariances}
    \end{align*}
    For some problem-dependent constants $M_f, M_s > 0$ and every $n \geq 1$, it holds that
    \begin{equation}\label{eq:thm2_bounds}
        \begin{split}
            \lVert \mathbb{E} \hat{x}_{n+1} \hat{x}_{n+1}^T - \alpha_{n+1} \Sigma_{ff}\rVert &\leq 
            \prod_{t=1}^n \left(1 - \alpha_t \frac{\mu_{ff}}{4}\right) \lVert \mathbb{E} \hat{x}_1 \hat{x}_1^T - \alpha_1 \Sigma_{ff}\rVert 
            + M_f \gamma_{n}
            % \mathcal{O}\left(\gamma_n + \frac{1}{n}\right)
            % \gamma_n M_f (1 + \lVert \Sigma_{ff} \rVert) 
            % + \frac{M_f}{n} \lVert \Sigma_{ff} \rVert 
            % + o(n^{-1}) 
            ,
            \\ 
            \lVert \mathbb{E} \hat{y}_{n+1} \hat{y}_{n+1}^T - \gamma_{n+1} \Sigma_{ss} \rVert 
        & \leq 
        \prod_{t=1}^n \left(1 - \gamma_t \frac{\mu_\Delta}{4}\right) \lVert \mathbb{E}\hat{y}_1 \hat{y}_1^T - \gamma_1 \Sigma_{ss} \rVert 
        + \frac{M_s}{n} 
        % \frac{\gamma_{n}^2}{\alpha_n}.
        \end{split}
    \end{equation}
    }


\newtheorem*{lemma*}{Lemma}

% Use \Name{Author Name} to specify the name.
% If the surname contains spaces, enclose the surname
% in braces, e.g. \Name{John {Smith Jones}} similarly
% if the name has a "von" part, e.g \Name{Jane {de Winter}}.
% If the first letter in the forenames is a diacritic
% enclose the diacritic in braces, e.g. \Name{{\'E}louise Smith}

% Two authors with the same address
% \coltauthor{\Name{Author Name1} \Email{abc@sample.com}\and
%  \Name{Author Name2} \Email{xyz@sample.com}\\
%  \addr Address}

% Three or more authors with the same address:
% \coltauthor{\Name{Author Name1} \Email{an1@sample.com}\\
%  \Name{Author Name2} \Email{an2@sample.com}\\
%  \Name{Author Name3} \Email{an3@sample.com}\\
%  \addr Address}

% % Authors with different addresses:
% \author{%
%  \Name{Author Name1} \Email{abc@sample.com}\\
%  \addr Address 1
%  \AND
%  \Name{Author Name2} \Email{xyz@sample.com}\\
%  \addr Address 2%
% }
\newcommand{\tdoan}[1]{{\color{red}\bf [thinh: #1]}}



\begin{document}

\maketitle

\begin{abstract}
    We consider linear two-time-scale  stochastic approximation algorithms driven by martingale noise. 
    Recent applications in machine learning motivate the need to understand finite-time error rates, but conventional stochastic approximation analysis focus on either asymptotic convergence in distribution or finite-time bounds that are far from optimal.
    Prior work on asymptotic central limit theorems (CLTs) suggest that two-time-scale algorithms may be able to achieve $1/\sqrt{n}$ error in expectation, with a constant given by the expected norm of the limiting Gaussian vector.
    However, the best known finite-time rates are much slower.
    We derive the first non-asymptotic central limit theorem with respect to the Wasserstein-1 distance for two-time-scale stochastic approximation with Polyak-Ruppert averaging.
    As a corollary, we show that expected error achieved by Polyak-Ruppert averaging decays at rate $1/\sqrt{n}$, which significantly improves on the rates of convergence in prior works.
\end{abstract}

\begin{keywords}
    Two-time-scale stochastic approximation, Polyak-Ruppert averaging, Central Limit Theorem
\end{keywords}

% \subsection{Pathology}
\label{app:tasks:pathology}
\begin{tcolorbox}[title={\texttt{conch\_extract\_features}}]
Perform feature extraction on an input image using CONCH.

\vspace{.5em}
\textbf{Arguments:}
\begin{itemize}[topsep=0pt,parsep=-1pt,partopsep=0pt]
\item \texttt{input\_image} (\texttt{str}): Path to the input image\\  Example: \texttt{'/mount/input/TUM/TUM-TCGA-ACRLPPQE.tif'}
\end{itemize}

\vspace{.5em}
\textbf{Returns:} \begin{itemize}[topsep=0pt,parsep=-1pt,partopsep=0pt]
\item \texttt{features} (\texttt{list}): The feature vector extracted from the input image, as a list of floats
\end{itemize}
\tcblower
\setlength{\hangindent}{\widthof{\faGithub~}}
\faGithub~\url{https://github.com/mahmoodlab/CONCH}

\vspace{.5em}\setlength{\hangindent}{\widthof{\faFile*[regular]~}}\faFile*[regular]~\bibentry{lu2024conch}


\end{tcolorbox}

\begin{tcolorbox}[title={\texttt{musk\_extract\_features}}]
Perform feature extraction on an input image using the vision part of MUSK.

\vspace{.5em}
\textbf{Arguments:}
\begin{itemize}[topsep=0pt,parsep=-1pt,partopsep=0pt]
\item \texttt{input\_image} (\texttt{str}): Path to the input image\\  Example: \texttt{'/mount/input/TUM/TUM-TCGA-ACRLPPQE.tif'}
\end{itemize}

\vspace{.5em}
\textbf{Returns:} \begin{itemize}[topsep=0pt,parsep=-1pt,partopsep=0pt]
\item \texttt{features} (\texttt{list}): The feature vector extracted from the input image, as a list of floats
\end{itemize}
\tcblower
\setlength{\hangindent}{\widthof{\faGithub~}}
\faGithub~\url{https://github.com/lilab-stanford/MUSK}

\vspace{.5em}\setlength{\hangindent}{\widthof{\faFile*[regular]~}}\faFile*[regular]~\bibentry{xiang2025musk}


\end{tcolorbox}

\begin{tcolorbox}[title={\texttt{pathfinder\_verify\_biomarker}}]
Given WSI probability maps, a hypothesis of a potential biomarker, and clinical data, determine (1) whether the potential biomarker is significant for patient prognosis, and (2) whether the potential biomarker is independent among already known biomarkers.

\vspace{.5em}
\textbf{Arguments:}
\begin{itemize}[topsep=0pt,parsep=-1pt,partopsep=0pt]
\item \texttt{heatmaps} (\texttt{str}): Path to the folder containing the numpy array (\textasciigrave{}*.npy\textasciigrave{}) files, which contains the heatmaps of the trained model (each heatmap is HxWxC where C is the number of classes)\\  Example: \texttt{'/mount/input/TCGA\_CRC'}
\item \texttt{hypothesis} (\texttt{str}): A python file, which contains a function \textasciigrave{}def hypothesis\_score(prob\_map\_path: str) -\textgreater{} float\textasciigrave{} which expresses a mathematical model of a hypothesis of a potential biomarker.  For a particular patient, the function returns a risk score.\\  Example: \texttt{'/mount/input/mus\_fraction\_score.py'}
\item \texttt{clini\_table} (\texttt{str}): Path to the CSV file containing the clinical data\\  Example: \texttt{'/mount/input/TCGA\_CRC\_info.csv'}
\item \texttt{files\_table} (\texttt{str}): Path to the CSV file containing the mapping between patient IDs (in the PATIENT column) and heatmap filenames (in the FILENAME column)\\  Example: \texttt{'/mount/input/TCGA\_CRC\_files.csv'}
\item \texttt{survival\_time\_column} (\texttt{str}): The name of the column in the clinical data that contains the survival time\\  Example: \texttt{'OS.time'}
\item \texttt{event\_column} (\texttt{str}): The name of the column in the clinical data that contains the event (e.g. death, recurrence, etc.)\\  Example: \texttt{'vital\_status'}
\item \texttt{known\_biomarkers} (\texttt{list}): A list of known biomarkers. These are column names in the clinical data.\\  Example: \texttt{['MSI']}
\end{itemize}

\vspace{.5em}
\textbf{Returns:} \begin{itemize}[topsep=0pt,parsep=-1pt,partopsep=0pt]
\item \texttt{p\_value} (\texttt{float}): The p-value of the significance of the potential biomarker
\item \texttt{hazard\_ratio} (\texttt{float}): The hazard ratio for the biomarker
\end{itemize}
\tcblower
\setlength{\hangindent}{\widthof{\faGithub~}}
\faGithub~\url{https://github.com/LiangJunhao-THU/PathFinderCRC}

\vspace{.5em}\setlength{\hangindent}{\widthof{\faFile*[regular]~}}\faFile*[regular]~\bibentry{liang2023pathfinder}


\end{tcolorbox}

\begin{tcolorbox}[title={\texttt{stamp\_extract\_features}}]
Perform feature extraction using CTransPath with STAMP on a set of whole slide images, and save the resulting features to a new folder.

\vspace{.5em}
\textbf{Arguments:}
\begin{itemize}[topsep=0pt,parsep=-1pt,partopsep=0pt]
\item \texttt{output\_dir} (\texttt{str}): Path to the output folder where the features will be saved\\  Example: \texttt{'/mount/output/TCGA-BRCA-features'}
\item \texttt{slide\_dir} (\texttt{str}): Path to the input folder containing the whole slide images\\  Example: \texttt{'/mount/input/TCGA-BRCA-SLIDES'}
\end{itemize}

\vspace{.5em}
\textbf{Returns:} \begin{itemize}[topsep=0pt,parsep=-1pt,partopsep=0pt]
\item \texttt{num\_processed\_slides} (\texttt{int}): The number of slides that were processed
\end{itemize}
\tcblower
\setlength{\hangindent}{\widthof{\faGithub~}}
\faGithub~\url{https://github.com/KatherLab/STAMP}

\vspace{.5em}\setlength{\hangindent}{\widthof{\faFile*[regular]~}}\faFile*[regular]~\bibentry{elnahhas2024stamp}


\end{tcolorbox}

\begin{tcolorbox}[title={\texttt{stamp\_train\_classification\_model}}]
Train a model for biomarker classification. You will be supplied with the path to the folder containing the whole slide images, alongside a path to a CSV file containing the training labels.

\vspace{.5em}
\textbf{Arguments:}
\begin{itemize}[topsep=0pt,parsep=-1pt,partopsep=0pt]
\item \texttt{slide\_dir} (\texttt{str}): Path to the folder containing the whole slide images\\  Example: \texttt{'/mount/input/TCGA-BRCA-SLIDES'}
\item \texttt{clini\_table} (\texttt{str}): Path to the CSV file containing the clinical data\\  Example: \texttt{'/mount/input/TCGA-BRCA-DX\_CLINI.xlsx'}
\item \texttt{slide\_table} (\texttt{str}): Path to the CSV file containing the slide metadata\\  Example: \texttt{'/mount/input/TCGA-BRCA-DX\_SLIDE.csv'}
\item \texttt{target\_column} (\texttt{str}): The name of the column in the clinical data that contains the target labels\\  Example: \texttt{'TP53\_driver'}
\item \texttt{trained\_model\_path} (\texttt{str}): Path to the *.pkl file where the trained model should be saved by this function\\  Example: \texttt{'/mount/output/STAMP-BRCA-TP53-model.pkl'}
\end{itemize}

\vspace{.5em}
\textbf{Returns:} \begin{itemize}[topsep=0pt,parsep=-1pt,partopsep=0pt]
\item \texttt{num\_params} (\texttt{int}): The number of parameters in the trained model
\end{itemize}
\tcblower
\setlength{\hangindent}{\widthof{\faGithub~}}
\faGithub~\url{https://github.com/KatherLab/STAMP}

\vspace{.5em}\setlength{\hangindent}{\widthof{\faFile*[regular]~}}\faFile*[regular]~\bibentry{elnahhas2024stamp}


\end{tcolorbox}

\begin{tcolorbox}[title={\texttt{uni\_extract\_features}}]
Perform feature extraction on an input image using UNI.

\vspace{.5em}
\textbf{Arguments:}
\begin{itemize}[topsep=0pt,parsep=-1pt,partopsep=0pt]
\item \texttt{input\_image} (\texttt{str}): Path to the input image\\  Example: \texttt{'/mount/input/TUM/TUM-TCGA-ACRLPPQE.tif'}
\end{itemize}

\vspace{.5em}
\textbf{Returns:} \begin{itemize}[topsep=0pt,parsep=-1pt,partopsep=0pt]
\item \texttt{features} (\texttt{list}): The feature vector extracted from the input image, as a list of floats
\end{itemize}
\tcblower
\setlength{\hangindent}{\widthof{\faGithub~}}
\faGithub~\url{https://github.com/mahmoodlab/UNI}

\vspace{.5em}\setlength{\hangindent}{\widthof{\faFile*[regular]~}}\faFile*[regular]~\bibentry{chen2024uni}


\end{tcolorbox}

\subsection{Radiology}
\label{app:tasks:radiology}
\begin{tcolorbox}[title={\texttt{medsam\_inference}}]
Use the trained MedSAM model to segment the given abdomen CT scan.

\vspace{.5em}
\textbf{Arguments:}
\begin{itemize}[topsep=0pt,parsep=-1pt,partopsep=0pt]
\item \texttt{image\_file} (\texttt{str}): Path to the abdomen CT scan image.\\  Example: \texttt{'/mount/input/my\_image.jpg'}
\item \texttt{bounding\_box} (\texttt{list}): Bounding box to segment (list of 4 integers).\\  Example: \texttt{[25, 100, 155, 155]}
\item \texttt{segmentation\_file} (\texttt{str}): Path to where the segmentation image should be saved.\\  Example: \texttt{'/mount/output/segmented\_image.png'}
\end{itemize}

\vspace{.5em}
\textbf{Returns:} \textit{empty dict}
\tcblower
\setlength{\hangindent}{\widthof{\faGithub~}}
\faGithub~\url{https://github.com/bowang-lab/MedSAM}

\vspace{.5em}\setlength{\hangindent}{\widthof{\faFile*[regular]~}}\faFile*[regular]~\bibentry{ma2024medsam}


\end{tcolorbox}

\begin{tcolorbox}[title={\texttt{nnunet\_train\_model}}]
Train a nnUNet model from scratch on abdomen CT scans. You will be provided with  the path to the dataset, the nnUNet configuration to use, and the fold number  to train the model on.

\vspace{.5em}
\textbf{Arguments:}
\begin{itemize}[topsep=0pt,parsep=-1pt,partopsep=0pt]
\item \texttt{dataset\_path} (\texttt{str}): The path to the dataset to train the model on (contains dataset.json, imagesTr, imagesTs, labelsTr)\\  Example: \texttt{'/mount/input/Task02\_Heart'}
\item \texttt{unet\_configuration} (\texttt{str}): The configuration of the UNet to use for training. One of '2d', '3d\_fullres', '3d\_lowres', '3d\_cascade\_fullres'\\  Example: \texttt{'3d\_fullres'}
\item \texttt{fold} (\texttt{int}): The fold number to train the model on. One of 0, 1, 2, 3, 4.\\  Example: \texttt{0}
\item \texttt{output\_folder} (\texttt{str}): Path to the folder where the trained model should be saved\\  Example: \texttt{'/mount/output/trained\_model'}
\end{itemize}

\vspace{.5em}
\textbf{Returns:} \textit{empty dict}
\tcblower
\setlength{\hangindent}{\widthof{\faGithub~}}
\faGithub~\url{https://github.com/MIC-DKFZ/nnUNet}

\vspace{.5em}\setlength{\hangindent}{\widthof{\faFile*[regular]~}}\faFile*[regular]~\bibentry{isensee2020nnunet}


\end{tcolorbox}

\subsection{Omics}
\label{app:tasks:genomics_proteomics}
\begin{tcolorbox}[title={\texttt{cytopus\_db}}]
Initialize the Cytopus KnowledgeBase and generate a JSON file containing a nested dictionary with gene set annotations organized by cell type, suitable for input into the Spectra library.

\vspace{.5em}
\textbf{Arguments:}
\begin{itemize}[topsep=0pt,parsep=-1pt,partopsep=0pt]
\item \texttt{celltype\_of\_interest} (\texttt{list}): List of cell types for which to retrieve gene sets\\  Example: \texttt{['B\_memory', 'B\_naive', 'CD4\_T', 'CD8\_T', 'DC', 'ILC3', 'MDC', 'NK', 'Treg', 'gdT', 'mast', 'pDC', 'plasma']}
\item \texttt{global\_celltypes} (\texttt{list}): List of global cell types to include in the JSON file.\\  Example: \texttt{['all-cells', 'leukocyte']}
\item \texttt{output\_file} (\texttt{str}): Path to the file where the output JSON file should be stored.\\  Example: \texttt{'/mount/output/Spectra\_dict.json'}
\end{itemize}

\vspace{.5em}
\textbf{Returns:} \begin{itemize}[topsep=0pt,parsep=-1pt,partopsep=0pt]
\item \texttt{keys} (\texttt{list}): The list of keys in the produced JSON file.
\end{itemize}
\tcblower
\setlength{\hangindent}{\widthof{\faGithub~}}
\faGithub~\url{https://github.com/wallet-maker/cytopus}

\vspace{.5em}\setlength{\hangindent}{\widthof{\faFile*[regular]~}}\faFile*[regular]~\bibentry{kunes2023cytopus}


\end{tcolorbox}

\begin{tcolorbox}[title={\texttt{esm\_fold\_predict}}]
Generate the representation of a protein sequence and the contact map using Facebook Research's pretrained esm2\_t33\_650M\_UR50D model.

\vspace{.5em}
\textbf{Arguments:}
\begin{itemize}[topsep=0pt,parsep=-1pt,partopsep=0pt]
\item \texttt{sequence} (\texttt{str}): Protein sequence to for which to generate representation and contact map.\\  Example: \texttt{'MKTVRQERLKSIVRILERSKEPVSGAQLAEELSVSRQVIVQDIAYLRSLGYNIVATPRGYVLAGG'}
\end{itemize}

\vspace{.5em}
\textbf{Returns:} \begin{itemize}[topsep=0pt,parsep=-1pt,partopsep=0pt]
\item \texttt{sequence\_representation} (\texttt{list}): Token representations for the protein sequence as a list of floats, i.e. a 1D array of shape L where L is the number of tokens.
\item \texttt{contact\_map} (\texttt{list}): Contact map for the protein sequence as a list of list of floats, i.e. a 2D array of shape LxL where L is the number of tokens.
\end{itemize}
\tcblower
\setlength{\hangindent}{\widthof{\faGithub~}}
\faGithub~\url{https://github.com/facebookresearch/esm}

\vspace{.5em}\setlength{\hangindent}{\widthof{\faFile*[regular]~}}\faFile*[regular]~\bibentry{verkuil2022esm1}


\vspace{.5em}\setlength{\hangindent}{\widthof{\faFile*[regular]~}}\faFile*[regular]~\bibentry{hie2022esm2}


\end{tcolorbox}

\subsection{Other}
\label{app:tasks:imaging}
\begin{tcolorbox}[title={\texttt{retfound\_feature\_vector}}]
Extract the feature vector for the given retinal image using the RETFound pretrained vit\_large\_patch16 model.

\vspace{.5em}
\textbf{Arguments:}
\begin{itemize}[topsep=0pt,parsep=-1pt,partopsep=0pt]
\item \texttt{image\_file} (\texttt{str}): Path to the retinal image.\\  Example: \texttt{'/mount/input/retinal\_image.jpg'}
\end{itemize}

\vspace{.5em}
\textbf{Returns:} \begin{itemize}[topsep=0pt,parsep=-1pt,partopsep=0pt]
\item \texttt{feature\_vector} (\texttt{list}): The feature vector for the given retinal image, as a list of floats.
\end{itemize}
\tcblower
\setlength{\hangindent}{\widthof{\faGithub~}}
\faGithub~\url{https://github.com/rmaphoh/RETFound_MAE}

\vspace{.5em}\setlength{\hangindent}{\widthof{\faFile*[regular]~}}\faFile*[regular]~\bibentry{zhou2023retfound}


\end{tcolorbox}

\label{app:tasks:llms}
\begin{tcolorbox}[title={\texttt{medsss\_generate}}]
Given a user message, generate a response using the MedSSS\_Policy model.

\vspace{.5em}
\textbf{Arguments:}
\begin{itemize}[topsep=0pt,parsep=-1pt,partopsep=0pt]
\item \texttt{user\_message} (\texttt{str}): The user message.\\  Example: \texttt{'How to stop a cough?'}
\end{itemize}

\vspace{.5em}
\textbf{Returns:} \begin{itemize}[topsep=0pt,parsep=-1pt,partopsep=0pt]
\item \texttt{response} (\texttt{str}): The response generated by the model.
\end{itemize}
\tcblower
\setlength{\hangindent}{\widthof{\faGithub~}}
\faGithub~\url{https://github.com/pixas/MedSSS}

\vspace{.5em}\setlength{\hangindent}{\widthof{\faFile*[regular]~}}\faFile*[regular]~\bibentry{jiang2025medsss}


\end{tcolorbox}

\begin{tcolorbox}[title={\texttt{modernbert\_predict\_masked}}]
Given a masked sentence string, predict the original sentence using the pretrained ModernBERT-base model on CPU.

\vspace{.5em}
\textbf{Arguments:}
\begin{itemize}[topsep=0pt,parsep=-1pt,partopsep=0pt]
\item \texttt{input\_string} (\texttt{str}): The masked sentence string. The masked part is represented by "[MASK]"".\\  Example: \texttt{'Paris is the [MASK] of France.'}
\end{itemize}

\vspace{.5em}
\textbf{Returns:} \begin{itemize}[topsep=0pt,parsep=-1pt,partopsep=0pt]
\item \texttt{prediction} (\texttt{str}): The predicted original sentence
\end{itemize}
\tcblower
\setlength{\hangindent}{\widthof{\faGithub~}}
\faGithub~\url{https://github.com/AnswerDotAI/ModernBERT}

\vspace{.5em}\setlength{\hangindent}{\widthof{\faFile*[regular]~}}\faFile*[regular]~\bibentry{warner2024modernbert}


\end{tcolorbox}

\label{app:tasks:3d_vision}
\begin{tcolorbox}[title={\texttt{flowmap\_overfit\_scene}}]
Overfit FlowMap on an input scene to determine camera extrinsics for each frame in the scene.

\vspace{.5em}
\textbf{Arguments:}
\begin{itemize}[topsep=0pt,parsep=-1pt,partopsep=0pt]
\item \texttt{input\_scene} (\texttt{str}): Path to the directory containing the images of the input scene (just the image files, nothing else)\\  Example: \texttt{'/mount/input/llff\_flower'}
\end{itemize}

\vspace{.5em}
\textbf{Returns:} \begin{itemize}[topsep=0pt,parsep=-1pt,partopsep=0pt]
\item \texttt{n} (\texttt{int}): The number of images (frames) in the scene
\item \texttt{camera\_extrinsics} (\texttt{list}): The camera extrinsics matrix for each of the n frames in the scene, must have a shape of nx4x4 (as a nested python list of floats)
\end{itemize}
\tcblower
\setlength{\hangindent}{\widthof{\faGithub~}}
\faGithub~\url{https://github.com/dcharatan/flowmap}

\vspace{.5em}\setlength{\hangindent}{\widthof{\faFile*[regular]~}}\faFile*[regular]~\bibentry{smith2024flowmap}


\end{tcolorbox}

\label{app:tasks:tabular}
\begin{tcolorbox}[title={\texttt{tabpfn\_predict}}]
Train a predictor using TabPFN on a tabular dataset. Evaluate the predictor on the test set.

\vspace{.5em}
\textbf{Arguments:}
\begin{itemize}[topsep=0pt,parsep=-1pt,partopsep=0pt]
\item \texttt{train\_csv} (\texttt{str}): Path to the CSV file containing the training data\\  Example: \texttt{'/mount/input/breast\_cancer\_train.csv'}
\item \texttt{test\_csv} (\texttt{str}): Path to the CSV file containing the test data\\  Example: \texttt{'/mount/input/breast\_cancer\_test.csv'}
\item \texttt{feature\_columns} (\texttt{list}): The names of the columns to use as features\\  Example: \texttt{['mean radius', 'mean texture', 'mean perimeter', 'mean area', 'mean smoothness', 'mean compactness', 'mean concavity', 'mean concave points', 'mean symmetry', 'mean fractal dimension', 'radius error', 'texture error', 'perimeter error', 'area error', 'smoothness error', 'compactness error', 'concavity error', 'concave points error', 'symmetry error', 'fractal dimension error', 'worst radius', 'worst texture', 'worst perimeter', 'worst area', 'worst smoothness', 'worst compactness', 'worst concavity', 'worst concave points', 'worst symmetry', 'worst fractal dimension']}
\item \texttt{target\_column} (\texttt{str}): The name of the column to predict\\  Example: \texttt{'target'}
\end{itemize}

\vspace{.5em}
\textbf{Returns:} \begin{itemize}[topsep=0pt,parsep=-1pt,partopsep=0pt]
\item \texttt{roc\_auc} (\texttt{float}): The ROC AUC score of the predictor on the test set
\item \texttt{accuracy} (\texttt{float}): The accuracy of the predictor on the test set
\item \texttt{probs} (\texttt{list}): The probabilities of the predictor on the test set, as a list of floats (one per sample in the test set)
\end{itemize}
\tcblower
\setlength{\hangindent}{\widthof{\faGithub~}}
\faGithub~\url{https://github.com/PriorLabs/TabPFN}

\vspace{.5em}\setlength{\hangindent}{\widthof{\faFile*[regular]~}}\faFile*[regular]~\bibentry{hollmann2025tabpfn}


\end{tcolorbox}



% \section{Introduction}
% Re-position our contributions.
% \begin{enumerate}
%     \item We prove convergence rate in distribuion of two timescale stochastic approximation. 
%     \item As a consequence, we obtain the optimal rate of convergence in expected norm of linear TTSA. 
%     \item Just combining some techniques (e.g. MSE \citep{konda2004convergence} and Jensen) gives sub-optimal rates. Therefore, we improve the analysis techniques to arrive at the optimal rate. 
%     \item Non-trivial combination of techniques? Can I outline them well, or is it hard to describe? 
% \end{enumerate}

% !TEX root =  ../main.tex
\section{Background on causality and abstraction}\label{sec:preliminaries}

This section provides the notation and key concepts related to causal modeling and abstraction theory.

\spara{Notation.} The set of integers from $1$ to $n$ is $[n]$.
The vectors of zeros and ones of size $n$ are $\zeros_n$ and $\ones_n$.
The identity matrix of size $n \times n$ is $\identity_n$. The Frobenius norm is $\frob{\mathbf{A}}$.
The set of positive definite matrices over $\reall^{n\times n}$ is $\pd^n$. The Hadamard product is $\odot$.
Function composition is $\circ$.
The domain of a function is $\dom{\cdot}$ and its kernel $\ker$.
Let $\mathcal{M}(\mathcal{X}^n)$ be the set of Borel measures over $\mathcal{X}^n \subseteq \reall^n$. Given a measure $\mu^n \in \mathcal{M}(\mathcal{X}^n)$ and a measurable map $\varphi^{\V}$, $\mathcal{X}^n \ni \mathbf{x} \overset{\varphi^{\V}}{\longmapsto} \V^\top \mathbf{x} \in \mathcal{X}^m$, we denote by $\varphi^{\V}_{\#}(\mu^n) \coloneqq \mu^n(\varphi^{\V^{-1}}(\mathbf{x}))$ the pushforward measure $\mu^m \in \mathcal{M}(\mathcal{X}^m)$. 


We now present the standard definition of SCM.

\begin{definition}[SCM, \citealp{pearl2009causality}]\label{def:SCM}
A (Markovian) structural causal model (SCM) $\scm^n$ is a tuple $\langle \myendogenous, \myexogenous, \myfunctional, \zeta^\myexogenous \rangle$, where \emph{(i)} $\myendogenous = \{X_1, \ldots, X_n\}$ is a set of $n$ endogenous random variables; \emph{(ii)} $\myexogenous =\{Z_1,\ldots,Z_n\}$ is a set of $n$ exogenous variables; \emph{(iii)} $\myfunctional$ is a set of $n$ functional assignments such that $X_i=f_i(\parents_i, Z_i)$, $\forall \; i \in [n]$, with $ \parents_i \subseteq \myendogenous \setminus \{ X_i\}$; \emph{(iv)} $\zeta^\myexogenous$ is a product probability measure over independent exogenous variables $\zeta^\myexogenous=\prod_{i \in [n]} \zeta^i$, where $\zeta^i=P(Z_i)$. 
\end{definition}
A Markovian SCM induces a directed acyclic graph (DAG) $\mathcal{G}_{\scm^n}$ where the nodes represent the variables $\myendogenous$ and the edges are determined by the structural functions $\myfunctional$; $ \parents_i$ constitutes then the parent set for $X_i$. Furthermore, we can recursively rewrite the set of structural function $\myfunctional$ as a set of mixing functions $\mymixing$ dependent only on the exogenous variables (cf. \cref{app:CA}). A key feature for studying causality is the possibility of defining interventions on the model:
\begin{definition}[Hard intervention, \citealp{pearl2009causality}]\label{def:intervention}
Given SCM $\scm^n = \langle \myendogenous, \myexogenous, \myfunctional, \zeta^\myexogenous \rangle$, a (hard) intervention $\iota = \operatorname{do}(\myendogenous^{\iota} = \mathbf{x}^{\iota})$, $\myendogenous^{\iota}\subseteq \myendogenous$,
is an operator that generates a new post-intervention SCM $\scm^n_\iota = \langle \myendogenous, \myexogenous, \myfunctional_\iota, \zeta^\myexogenous \rangle$ by replacing each function $f_i$ for $X_i\in\myendogenous^{\iota}$ with the constant $x_i^\iota\in \mathbf{x}^\iota$. 
Graphically, an intervention mutilates $\mathcal{G}_{\mathsf{M}^n}$ by removing all the incoming edges of the variables in $\myendogenous^{\iota}$.
\end{definition}

Given multiple SCMs describing the same system at different levels of granularity, CA provides the definition of an $\alpha$-abstraction map to relate these SCMs:
\begin{definition}[$\abst$-abstraction, \citealp{rischel2020category}]\label{def:abstraction}
Given low-level $\mathsf{M}^\ell$ and high-level $\mathsf{M}^h$ SCMs, an $\abst$-abstraction is a triple $\abst = \langle \Rset, \amap, \alphamap{} \rangle$, where \emph{(i)} $\Rset \subseteq \datalow$ is a subset of relevant variables in $\mathsf{M}^\ell$; \emph{(ii)} $\amap: \Rset \rightarrow \datahigh$ is a surjective function between the relevant variables of $\mathsf{M}^\ell$ and the endogenous variables of $\mathsf{M}^h$; \emph{(iii)} $\alphamap{}: \dom{\Rset} \rightarrow \dom{\datahigh}$ is a modular function $\alphamap{} = \bigotimes_{i\in[n]} \alphamap{X^h_i}$ made up by surjective functions $\alphamap{X^h_i}: \dom{\amap^{-1}(X^h_i)} \rightarrow \dom{X^h_i}$ from the outcome of low-level variables $\amap^{-1}(X^h_i) \in \datalow$ onto outcomes of the high-level variables $X^h_i \in \datahigh$.
\end{definition}
Notice that an $\abst$-abstraction simultaneously maps variables via the function $\amap$ and values through the function $\alphamap{}$. The definition itself does not place any constraint on these functions, although a common requirement in the literature is for the abstraction to satisfy \emph{interventional consistency} \cite{rubenstein2017causal,rischel2020category,beckers2019abstracting}. An important class of such well-behaved abstractions is \emph{constructive linear abstraction}, for which the following properties hold. By constructivity, \emph{(i)} $\abst$ is interventionally consistent; \emph{(ii)} all low-level variables are relevant $\Rset=\datalow$; \emph{(iii)} in addition to the map $\alphamap{}$ between endogenous variables, there exists a map ${\alphamap{}}_U$ between exogenous variables satisfying interventional consistency \cite{beckers2019abstracting,schooltink2024aligning}. By linearity, $\alphamap{} = \V^\top \in \reall^{h \times \ell}$ \cite{massidda2024learningcausalabstractionslinear}. \cref{app:CA} provides formal definitions for interventional consistency, linear and constructive abstraction.
%\section{Related Work}
\label{sec:RelatedWork}

Within the realm of geophysical sciences, super-resolution/downscaling is a challenge that scientists continue to tackle. There have been several works involved in downscaling applications such as river mapping \cite{Yin2022}, coastal risk assessment \cite{Rucker2021}, estimating soil moisture from remotely sensed images \cite{Peng2017SoilMoisture} and downscaling of satellite based precipitation estimates \cite{Medrano2023PrecipitationDownscaling} to name a few. We direct the reader to \cite{Karwowska2022SuperResolutionSurvey} for a comprehensive review of satellite based downscaling applications and methods. Pertaining to our objective of downscaling \acp{WFM}, we can draw comparisons with several existing works. 
In what follows, we provide a brief review of functionally adjacent works to contrast the novelty of our proposed model and its role in addressing gaps in literature. 

When it comes to downscaling \ac{WFM}, several works use statistical downscaling techniques. These works downscale images by using statistical techniques that utilize relationships between neighboring water fraction pixels. For instance, \cite{Li2015SRFIM} treat the super-resolution task as a sub-pixel mapping problem, wherein the input fraction of inundated pixels must be exactly mapped to the output patch of inundated pixels. 
% In doing so, they are able to apply a discrete particle swarm optimization method to maximize the Flood Inundation Spatial Dependence Index (FISDI). 
\cite{Wang2019} improved upon these approaches by including a spectral term to fully utilize spectral information from multi spectral remote sensing image band. \cite{Wang2021} on the other hand also include a spectral correlation term to reduce the influence of linear and non-linear imaging conditions. All of these approaches are applied to water fraction obtained via spectral unmixing \cite{wang2013SpectralUnmixing} and are designed to work with multi spectral information from MODIS. However, we develop our model with the intention to be used with water fractions directly derived from the output of satellites. One such example is NOAA/VIIRS whose water fraction extraction method is described in \cite{Li2013VIIRSWFM}. \cite{Li2022VIIRSDownscaling} presented a work wherein \ac{WFM} at 375-m flood products from VIIRS were downscaled 30-m flood event and depth products by expressing the inundation mechanism as a function of the \ac{DEM}-based water area and the VIIRS water area.

On the other hand, the non-linear nature of the mapping task lends itself to the use of neural networks. Several models have been adapted from traditional single image digital super-resolution in computer vision literature \cite{sdraka2022DL4downscalingRemoteSensing}. Existing deep learning models in single image super-resolution are primarily dominated by \ac{CNN} based models. Specifically, there has been an upward trend in residual learning models. \acp{RDN} \cite{Zhang2018ResidualDenseSuperResolution} introduced residual dense blocks that employed a contiguous memory mechanism that aimed to overcome the inability of very deep \acp{CNN} to make full use of hierarchical features. 
\acp{RCAN} \cite{Zhang2018RCANSuperResolution} introduced an attention mechanism to exploit the inter-channel dependencies in the intermediate feature transformations. There have also been some works that aim to produce more lightweight \ac{CNN}-based architectures \cite{Zheng2019IMDN,Xiaotong2020LatticeNET}. Since the introduction of the vision transformer \cite{Vaswani2017Attention} that utilized the self-attention mechanism -- originally used for modeling text sequences -- by feeding a sequence 2D sub-image extracted from the original image. Using this approach \cite{LuESRT2022} developed a light-weight and efficient transformer based approach for single image super-resolution. 


For the task of super-resolution of \acp{WFM}, we discuss some works whose methodology is similar to ours even though they differ in their problem setting. \cite{Yin2022} presented a cascaded spectral spatial model for super-resolution of MODIS imagery with a scaling factor 10. Their architecture consists of two stages; first multi-spectral MODIS imagery is converted into a low-resolution \ac{WFM} via spectral unmixing by passing it through a deep stacked residual \ac{CNN}. The second stage involved the super-resolution mapping of these \acp{WFM} using a nested multi-level \ac{CNN} model. Similar to our work, the input fraction images are obtained with zero errors which may not be reflective of reality since there tends to be sensor noise, the spatial distribution of whom cannot be easily estimated. We also note that none of these works directly tackle flood inundation since they've been trained with river map data during non-flood circumstance and \textit{ergo} do not face a data scarcity problem as we do. 
% In this work, apart from the final product of \acp{WFM}, we are not presented with any additional spectral information about the low resolution image. This was intended to work directly with products that can generate \ac{WFM} either directly (VIIRS) or indirectly (Landsat).
\cite{Jia2019} used a deep \ac{CNN} for land mapping that consists of several classes such as building, low vegetation, background and trees. 
\cite{Kumar2021} similarly employ a \ac{CNN} based model for downscaling of summer monsoon rainfall data over the Indian subcontinent. Their proposed Super-Resolution Convolutional Neural Network (SRCNN) has a downscaling factor of 4. 
\cite{Shang2022} on the other hand, proposed a super-resolution mapping technique using Generative Adversarial Networks (GANs). They first generate high resolution fractional images, somewhat analogous to our \ac{WFM}, and are then mapped to categorical land cover maps involving forest, urban, agriculture and water classes. 
\cite{Qin2020} interestingly approach lake area super-resolution for Landsat and MODIS data as an unsupervised problem using a \ac{CNN} and are able to extend to other scaling factors. \cite{AristizabalInundationMapping2020} performed flood inundation mapping using \ac{SAR} data obtained from Sentinel-1. They showed that \ac{DEM}-based features helped to improve \ac{SAR}-based predictions for quadratic discriminant analysis, support vector machines and k-nearest neighbor classifiers. While almost all of the aforementioned works can be adapted to our task. We stand out in the following ways (i) We focus on downscaling of \acp{WFM} directly, \textit{i.e.,} we do not focus on the algorithm to compute the \ac{WFM} from multi-channel satellite data and (ii) We focus on producing high resolution maps only for instances of flood inundation. The latter point produces a data scarcity issue which we seek to remedy with synthetic data. 


%%%%%%%%%%%%%%%%% Additional unused information %%%%%%%%%%%%%%%%


%     \item[\cite{Wang2021}] Super-Resolution Mapping Based on Spatial–Spectral Correlation for Spectral Imagery
%     \begin{itemize}
%         \item Not a deep neural network approach. SRM based on spatial–spectral correlation (SSC) is proposed in order to overcome the influence of linear and nonlinear imaging conditions and utilize more accurate spectral properties.
%         \item (fig 1) there are two main SRM types: (1) the initialization-then-optimization SRM, where the class labels are allocated randomly to subpixels, and the location of each subpixel is optimized to obtain the final SRM result. and (2)soft-then-hard SRM, which involves two steps: the subpixel sharpening and the class allocation.  
%         \item SSC procedures: (1) spatial correlation is performed by the MSAM to reduce the influences of linear imaging conditions on image quality. (2) A spectral correlation that utilizes spectral properties based on the nonlinear KLD is proposed to reduce the influences of nonlinear imaging conditions. (3) spatial and spectral correlations are then combined to obtain an optimization function with improved linear and nonlinear performances. And finally (4) by maximizing the optimization function, a class allocation method based on the SA is used to assign LC labels to each subpixel, obtaining the final SRM result.
%         \item (Comparable) 
%     \end{itemize}
%     %--------------------------------------------------------------------
% \cite{Wang2021} account for the influence of linear and non-linear imaging conditions by involving more accurate spectral properties. 
%     %--------------------------------------------------------------------
%     \item[\cite{Yin2022}] A Cascaded Spectral–Spatial CNN Model for Super-Resolution River Mapping With MODIS Imagery
%     \begin{itemize}
%         \item produce  Landsat-like  fine-resolution (scale of 10)  river  maps  from  MODIS images. Notice the original coarse-resolution remotely sensed images, not the river fraction images.
%         \item combined  CNN  model that  contains  a spectral  unmixing  module  and  an  SRM  module, and the SRM module is made up of an encoder and a decoder that are connected through a series of convolutional blocks. 
%         \item With an adaptive cross-entropy loss function to address class imbalance.	
%         \item The overall accuracy, the omission error, the  commission  error,  and  the  mean  intersection  over  union (MIOU)  calculated  to  assess  the results.
%         \item partially comparable with ours, only the SRM module part
%     %--------------------------------------------------------------------

% To decouple the description of the objective and the \ac{ML} model architecture, the motivation for the model architecture is described in \secref{sec:Methodology}. 


%     \item[\cite{Wang2019}] Improving Super-Resolution Flood Inundation Mapping for Multi spectral Remote Sensing Image by Supplying More Spectral 
%     \begin{itemize}
%         \item proposed the SRFIM-MSI,where a new spectral term is added to the traditional SRFIM to fully utilize the spectral information from multi spectral remote sensing image band. 
%         \item The original SRFIM \cite{Huang2014, Li2015} obtains the sub pixel spatial distribution of flood inundation within mixed pixels by maximizing their spatial correlation while maintaining the original proportions of flood inundation within the mixed pixels. The SRFIM is formulated as a maximum combined optimization issue according to the principle of spatial correlation.
%         \item follow the terminology in \cite{Wang2021}, this is an initialization-then-optimization SRM. 
%         \item (Comparable) 
%     \end{itemize}
%     %--------------------------------------------------------------------


%--------------------------------------------------------------------
%     \item[\cite{Jia2019}] Super-Resolution Land Cover Mapping Based on the Convolutional Neural Network
%     \begin{itemize}
%         \item SRMCNN (Super-resolution mapping CNN) is proposed to obtain fine-scale land cover maps from coarse remote sensing images. Specifically, an encoder-decoder CNN is used to determined the labels (i.e., land cover classes) of the subpixels within mixed pixels.
%         \item There were three main parts in SRMCNN. The first part was a three-sequential convolutional layer with ReLU and pooling. The second part is up-sampling, for which a multi transposed-convolutional layer was adopted. To keep the feature learned in the previous layer, a skip connection was used to concatenate the output of the corresponding convolution layer. The last part was the softmax classifier, in which the feature in the antepenultimate layer was classified and class probabilities are obtained.
%         \item The loss: the optimal allocation of classes to the subpixels of mixed pixel is achieved by maximizing the spatial dependence between neighbor pixels under constraint that the class proportions within the mixed pixels are preserved.
%         \item (Preferred), this paper is designed to classify background, Building, Low Vegetation, or Tree in the land. But we can easily adapt to our problem and should compare with this paper.
%     \end{itemize}
%     %--------------------------------------------------------------------

%     \item[\cite{Kumar2021}] Deep learning–based downscaling of summer monsoon rainfall data over Indian region
%     \begin{itemize}
%         \item down-scaling (scale of 4) rainfall data. The output image is not binary image.
%         \item three algorithms: SRCNN, stacked SRCNN, and DeepSD are employed, based on \cite{Vandal2019}
%         \item mean square error and pattern correlation coefficient are used as evaluation metrics.
%         \item SRCNN: super-resolution-based convolutional neural networks (SRCNN) first upgrades the low-resolution image to the higher resolution size by using bicubic interpolation. Suppose the interpolated image is referred to as Y; SRCNNs’task is to retrieve from Y an image F(Y) which is close to the high-resolution ground truth image X.
%         \item stacked SRCNN: stack 2 or more SRCNN blocks to increasing the scaling factor.
%         \item DeepSD: uses topographies as an additional input to stacked SRCNN.
%         \item These algorithms are not designed for binary output images, but if prefer, the ``modified'' stacked SRCNN or DeepSD can be used as baseline algorithms.
%     
%     \item[\cite{Shang2022}] Super resolution Land Cover Mapping Using a Generative Adversarial Network
%     \begin{itemize}
%         \item propose an end-to-end SRM model based on a generative adversarial network (GAN), that is, GAN-SRM, to improve the two-step learning-based SRM methods. 
%         \item Two-step SRM method: The first step is fraction-image super-resolution (SR), which reconstructs a high-spatial-resolution fraction image from the low input, methods like SVR, or CNN has been widely adopted. The second step is converting the high-resolution fraction images to a categorical land cover map, such as with a soft-max function to assign each high-resolution pixel to a unique category value.
%         \item The proposed GAN-SRM model includes a generative network and a discriminative network, so that both the fraction-image SR and the conversion of the fraction images to categorical map steps are fully integrated to reduce the resultant uncertainty. 
%         \item applied to the National Land Cover Database (NLCD), which categorized land into four typical classes:forest, urban, agriculture,and water. scale factor of 8. 
%         \item (Preferred), we should compare with this work.
%     \end{itemize}
%     %--------------------------------------------------------------------

%   \item[\cite{Qin2020}] Achieving Higher Resolution Lake Area from Remote Sensing Images Through an Unsupervised Deep Learning Super-Resolution Method
%   \begin{itemize}
%       \item propose an unsupervised deep gradient network (UDGN) to generate a higher resolution lake area from remote sensing images.
%       \item UDGN models the internal recurrence of information inside the single image and its corresponding gradient map to generate images with higher spatial resolution. 
%       \item A single image super-resolution approach, not comparable
%   \end{itemize}
%     %--------------------------------------------------------------------




%     \item[\cite{Demiray2021}] D-SRGAN: DEM Super-Resolution with Generative Adversarial Networks
%     \begin{itemize}
%         \item A GAN based model is proposed to increase the spatial resolution of a given DEM dataset up to 4 times without additional information related to data.
%         \item Rather than processing each image in a sequence independently, our generator architecture uses a recurrent layer to update the state of the high-resolution reconstruction in a manner that is consistent with both the previous state and the newly received data. The recurrent layer can thus be understood as performing a Bayesian update on the ensemble member, resembling an ensemble Kalman filter. 
%         \item A single image super-resolution approach, not comparable
%     \end{itemize}
%     %--------------------------------------------------------------------
%     \item[\cite{Leinonen2021}] Stochastic Super-Resolution for Downscaling Time-Evolving Atmospheric Fields With a Generative Adversarial Network
%     \begin{itemize}
%         \item propose a super-resolution GAN that operates on sequences of two-dimensional images and creates an ensemble of predictions for each input. The spread between the ensemble members represents the uncertainty of the super-resolution reconstruction.
%         \item for sequence of input images, not comparable with ours.
%     \end{itemize} 
%     %--------------------------------------------------------------------

% \end{itemize}




\section{Overview}

\revision{In this section, we first explain the foundational concept of Hausdorff distance-based penetration depth algorithms, which are essential for understanding our method (Sec.~\ref{sec:preliminary}).
We then provide a brief overview of our proposed RT-based penetration depth algorithm (Sec.~\ref{subsec:algo_overview}).}



\section{Preliminaries }
\label{sec:Preliminaries}

% Before we introduce our method, we first overview the important basics of 3D dynamic human modeling with Gaussian splatting. Then, we discuss the diffusion-based 3d generation techniques, and how they can be applied to human modeling.
% \ZY{I stopp here. TBC.}
% \subsection{Dynamic human modeling with Gaussian splatting}
\subsection{3D Gaussian Splatting}
3D Gaussian splatting~\cite{kerbl3Dgaussians} is an explicit scene representation that allows high-quality real-time rendering. The given scene is represented by a set of static 3D Gaussians, which are parameterized as follows: Gaussian center $x\in {\mathbb{R}^3}$, color $c\in {\mathbb{R}^3}$, opacity $\alpha\in {\mathbb{R}}$, spatial rotation in the form of quaternion $q\in {\mathbb{R}^4}$, and scaling factor $s\in {\mathbb{R}^3}$. Given these properties, the rendering process is represented as:
\begin{equation}
  I = Splatting(x, c, s, \alpha, q, r),
  \label{eq:splattingGA}
\end{equation}
where $I$ is the rendered image, $r$ is a set of query rays crossing the scene, and $Splatting(\cdot)$ is a differentiable rendering process. We refer readers to Kerbl et al.'s paper~\cite{kerbl3Dgaussians} for the details of Gaussian splatting. 



% \ZY{I would suggest move this part to the method part.}
% GaissianAvatar is a dynamic human generation model based on Gaussian splitting. Given a sequence of RGB images, this method utilizes fitted SMPLs and sampled points on its surface to obtain a pose-dependent feature map by a pose encoder. The pose-dependent features and a geometry feature are fed in a Gaussian decoder, which is employed to establish a functional mapping from the underlying geometry of the human form to diverse attributes of 3D Gaussians on the canonical surfaces. The parameter prediction process is articulated as follows:
% \begin{equation}
%   (\Delta x,c,s)=G_{\theta}(S+P),
%   \label{eq:gaussiandecoder}
% \end{equation}
%  where $G_{\theta}$ represents the Gaussian decoder, and $(S+P)$ is the multiplication of geometry feature S and pose feature P. Instead of optimizing all attributes of Gaussian, this decoder predicts 3D positional offset $\Delta{x} \in {\mathbb{R}^3}$, color $c\in\mathbb{R}^3$, and 3D scaling factor $ s\in\mathbb{R}^3$. To enhance geometry reconstruction accuracy, the opacity $\alpha$ and 3D rotation $q$ are set to fixed values of $1$ and $(1,0,0,0)$ respectively.
 
%  To render the canonical avatar in observation space, we seamlessly combine the Linear Blend Skinning function with the Gaussian Splatting~\cite{kerbl3Dgaussians} rendering process: 
% \begin{equation}
%   I_{\theta}=Splatting(x_o,Q,d),
%   \label{eq:splatting}
% \end{equation}
% \begin{equation}
%   x_o = T_{lbs}(x_c,p,w),
%   \label{eq:LBS}
% \end{equation}
% where $I_{\theta}$ represents the final rendered image, and the canonical Gaussian position $x_c$ is the sum of the initial position $x$ and the predicted offset $\Delta x$. The LBS function $T_{lbs}$ applies the SMPL skeleton pose $p$ and blending weights $w$ to deform $x_c$ into observation space as $x_o$. $Q$ denotes the remaining attributes of the Gaussians. With the rendering process, they can now reposition these canonical 3D Gaussians into the observation space.



\subsection{Score Distillation Sampling}
Score Distillation Sampling (SDS)~\cite{poole2022dreamfusion} builds a bridge between diffusion models and 3D representations. In SDS, the noised input is denoised in one time-step, and the difference between added noise and predicted noise is considered SDS loss, expressed as:

% \begin{equation}
%   \mathcal{L}_{SDS}(I_{\Phi}) \triangleq E_{t,\epsilon}[w(t)(\epsilon_{\phi}(z_t,y,t)-\epsilon)\frac{\partial I_{\Phi}}{\partial\Phi}],
%   \label{eq:SDSObserv}
% \end{equation}
\begin{equation}
    \mathcal{L}_{\text{SDS}}(I_{\Phi}) \triangleq \mathbb{E}_{t,\epsilon} \left[ w(t) \left( \epsilon_{\phi}(z_t, y, t) - \epsilon \right) \frac{\partial I_{\Phi}}{\partial \Phi} \right],
  \label{eq:SDSObservGA}
\end{equation}
where the input $I_{\Phi}$ represents a rendered image from a 3D representation, such as 3D Gaussians, with optimizable parameters $\Phi$. $\epsilon_{\phi}$ corresponds to the predicted noise of diffusion networks, which is produced by incorporating the noise image $z_t$ as input and conditioning it with a text or image $y$ at timestep $t$. The noise image $z_t$ is derived by introducing noise $\epsilon$ into $I_{\Phi}$ at timestep $t$. The loss is weighted by the diffusion scheduler $w(t)$. 
% \vspace{-3mm}

\subsection{Overview of the RTPD Algorithm}\label{subsec:algo_overview}
Fig.~\ref{fig:Overview} presents an overview of our RTPD algorithm.
It is grounded in the Hausdorff distance-based penetration depth calculation method (Sec.~\ref{sec:preliminary}).
%, similar to that of Tang et al.~\shortcite{SIG09HIST}.
The process consists of two primary phases: penetration surface extraction and Hausdorff distance calculation.
We leverage the RTX platform's capabilities to accelerate both of these steps.

\begin{figure*}[t]
    \centering
    \includegraphics[width=0.8\textwidth]{Image/overview.pdf}
    \caption{The overview of RT-based penetration depth calculation algorithm overview}
    \label{fig:Overview}
\end{figure*}

The penetration surface extraction phase focuses on identifying the overlapped region between two objects.
\revision{The penetration surface is defined as a set of polygons from one object, where at least one of its vertices lies within the other object. 
Note that in our work, we focus on triangles rather than general polygons, as they are processed most efficiently on the RTX platform.}
To facilitate this extraction, we introduce a ray-tracing-based \revision{Point-in-Polyhedron} test (RT-PIP), significantly accelerated through the use of RT cores (Sec.~\ref{sec:RT-PIP}).
This test capitalizes on the ray-surface intersection capabilities of the RTX platform.
%
Initially, a Geometry Acceleration Structure (GAS) is generated for each object, as required by the RTX platform.
The RT-PIP module takes the GAS of one object (e.g., $GAS_{A}$) and the point set of the other object (e.g., $P_{B}$).
It outputs a set of points (e.g., $P_{\partial B}$) representing the penetration region, indicating their location inside the opposing object.
Subsequently, a penetration surface (e.g., $\partial B$) is constructed using this point set (e.g., $P_{\partial B}$) (Sec.~\ref{subsec:surfaceGen}).
%
The generated penetration surfaces (e.g., $\partial A$ and $\partial B$) are then forwarded to the next step. 

The Hausdorff distance calculation phase utilizes the ray-surface intersection test of the RTX platform (Sec.~\ref{sec:RT-Hausdorff}) to compute the Hausdorff distance between two objects.
We introduce a novel Ray-Tracing-based Hausdorff DISTance algorithm, RT-HDIST.
It begins by generating GAS for the two penetration surfaces, $P_{\partial A}$ and $P_{\partial B}$, derived from the preceding step.
RT-HDIST processes the GAS of a penetration surface (e.g., $GAS_{\partial A}$) alongside the point set of the other penetration surface (e.g., $P_{\partial B}$) to compute the penetration depth between them.
The algorithm operates bidirectionally, considering both directions ($\partial A \to \partial B$ and $\partial B \to \partial A$).
The final penetration depth between the two objects, A and B, is determined by selecting the larger value from these two directional computations.

%In the Hausdorff distance calculation step, we compute the Hausdorff distance between given two objects using a ray-surface-intersection test. (Sec.~\ref{sec:RT-Hausdorff}) Initially, we construct the GAS for both $\partial A$ and $\partial B$ to utilize the RT-core effectively. The RT-based Hausdorff distance algorithms then determine the Hausdorff distance by processing the GAS of one object (e.g. $GAS_{\partial A}$) and set of the vertices of the other (e.g. $P_{\partial B}$). Following the Hausdorff distance definition (Eq.~\ref{equation:hausdorff_definition}), we compute the Hausdorff distance to both directions ($\partial A \to \partial B$) and ($\partial B \to \partial A$). As a result, the bigger one is the final Hausdorff distance, and also it is the penetration depth between input object $A$ and $B$.


%the proposed RT-based penetration depth calculation pipeline.
%Our proposed methods adopt Tang's Hausdorff-based penetration depth methods~\cite{SIG09HIST}. The pipeline is divided into the penetration surface extraction step and the Hausdorff distance calculation between the penetration surface steps. However, since Tang's approach is not suitable for the RT platform in detail, we modified and applied it with appropriate methods.

%The penetration surface extraction step is extracting overlapped surfaces on other objects. To utilize the RT core, we use the ray-intersection-based PIP(Point-In-Polygon) algorithms instead of collision detection between two objects which Tang et al.~\cite{SIG09HIST} used. (Sec.~\ref{sec:RT-PIP})
%RT core-based PIP test uses a ray-surface intersection test. For purpose this, we generate the GAS(Geometry Acceleration Structure) for each object. RT core-based PIP test takes the GAS of one object (e.g. $GAS_{A}$) and a set of vertex of another one (e.g. $P_{B}$). Then this computes the penetrated vertex set of another one (e.g. $P_{\partial B}$). To calculate the Hausdorff distance, these vertex sets change to objects constructed by penetrated surface (e.g. $\partial B$). Finally, the two generated overlapped surface objects $\partial A$ and $\partial B$ are used in the Hausdorff distance calculation step.
% \section{Discussion of Assumptions}\label{sec:discussion}
In this paper, we have made several assumptions for the sake of clarity and simplicity. In this section, we discuss the rationale behind these assumptions, the extent to which these assumptions hold in practice, and the consequences for our protocol when these assumptions hold.

\subsection{Assumptions on the Demand}

There are two simplifying assumptions we make about the demand. First, we assume the demand at any time is relatively small compared to the channel capacities. Second, we take the demand to be constant over time. We elaborate upon both these points below.

\paragraph{Small demands} The assumption that demands are small relative to channel capacities is made precise in \eqref{eq:large_capacity_assumption}. This assumption simplifies two major aspects of our protocol. First, it largely removes congestion from consideration. In \eqref{eq:primal_problem}, there is no constraint ensuring that total flow in both directions stays below capacity--this is always met. Consequently, there is no Lagrange multiplier for congestion and no congestion pricing; only imbalance penalties apply. In contrast, protocols in \cite{sivaraman2020high, varma2021throughput, wang2024fence} include congestion fees due to explicit congestion constraints. Second, the bound \eqref{eq:large_capacity_assumption} ensures that as long as channels remain balanced, the network can always meet demand, no matter how the demand is routed. Since channels can rebalance when necessary, they never drop transactions. This allows prices and flows to adjust as per the equations in \eqref{eq:algorithm}, which makes it easier to prove the protocol's convergence guarantees. This also preserves the key property that a channel's price remains proportional to net money flow through it.

In practice, payment channel networks are used most often for micro-payments, for which on-chain transactions are prohibitively expensive; large transactions typically take place directly on the blockchain. For example, according to \cite{river2023lightning}, the average channel capacity is roughly $0.1$ BTC ($5,000$ BTC distributed over $50,000$ channels), while the average transaction amount is less than $0.0004$ BTC ($44.7k$ satoshis). Thus, the small demand assumption is not too unrealistic. Additionally, the occasional large transaction can be treated as a sequence of smaller transactions by breaking it into packets and executing each packet serially (as done by \cite{sivaraman2020high}).
Lastly, a good path discovery process that favors large capacity channels over small capacity ones can help ensure that the bound in \eqref{eq:large_capacity_assumption} holds.

\paragraph{Constant demands} 
In this work, we assume that any transacting pair of nodes have a steady transaction demand between them (see Section \ref{sec:transaction_requests}). Making this assumption is necessary to obtain the kind of guarantees that we have presented in this paper. Unless the demand is steady, it is unreasonable to expect that the flows converge to a steady value. Weaker assumptions on the demand lead to weaker guarantees. For example, with the more general setting of stochastic, but i.i.d. demand between any two nodes, \cite{varma2021throughput} shows that the channel queue lengths are bounded in expectation. If the demand can be arbitrary, then it is very hard to get any meaningful performance guarantees; \cite{wang2024fence} shows that even for a single bidirectional channel, the competitive ratio is infinite. Indeed, because a PCN is a decentralized system and decisions must be made based on local information alone, it is difficult for the network to find the optimal detailed balance flow at every time step with a time-varying demand.  With a steady demand, the network can discover the optimal flows in a reasonably short time, as our work shows.

We view the constant demand assumption as an approximation for a more general demand process that could be piece-wise constant, stochastic, or both (see simulations in Figure \ref{fig:five_nodes_variable_demand}).
We believe it should be possible to merge ideas from our work and \cite{varma2021throughput} to provide guarantees in a setting with random demands with arbitrary means. We leave this for future work. In addition, our work suggests that a reasonable method of handling stochastic demands is to queue the transaction requests \textit{at the source node} itself. This queuing action should be viewed in conjunction with flow-control. Indeed, a temporarily high unidirectional demand would raise prices for the sender, incentivizing the sender to stop sending the transactions. If the sender queues the transactions, they can send them later when prices drop. This form of queuing does not require any overhaul of the basic PCN infrastructure and is therefore simpler to implement than per-channel queues as suggested by \cite{sivaraman2020high} and \cite{varma2021throughput}.

\subsection{The Incentive of Channels}
The actions of the channels as prescribed by the DEBT control protocol can be summarized as follows. Channels adjust their prices in proportion to the net flow through them. They rebalance themselves whenever necessary and execute any transaction request that has been made of them. We discuss both these aspects below.

\paragraph{On Prices}
In this work, the exclusive role of channel prices is to ensure that the flows through each channel remains balanced. In practice, it would be important to include other components in a channel's price/fee as well: a congestion price  and an incentive price. The congestion price, as suggested by \cite{varma2021throughput}, would depend on the total flow of transactions through the channel, and would incentivize nodes to balance the load over different paths. The incentive price, which is commonly used in practice \cite{river2023lightning}, is necessary to provide channels with an incentive to serve as an intermediary for different channels. In practice, we expect both these components to be smaller than the imbalance price. Consequently, we expect the behavior of our protocol to be similar to our theoretical results even with these additional prices.

A key aspect of our protocol is that channel fees are allowed to be negative. Although the original Lightning network whitepaper \cite{poon2016bitcoin} suggests that negative channel prices may be a good solution to promote rebalancing, the idea of negative prices in not very popular in the literature. To our knowledge, the only prior work with this feature is \cite{varma2021throughput}. Indeed, in papers such as \cite{van2021merchant} and \cite{wang2024fence}, the price function is explicitly modified such that the channel price is never negative. The results of our paper show the benefits of negative prices. For one, in steady state, equal flows in both directions ensure that a channel doesn't loose any money (the other price components mentioned above ensure that the channel will only gain money). More importantly, negative prices are important to ensure that the protocol selectively stifles acyclic flows while allowing circulations to flow. Indeed, in the example of Section \ref{sec:flow_control_example}, the flows between nodes $A$ and $C$ are left on only because the large positive price over one channel is canceled by the corresponding negative price over the other channel, leading to a net zero price.

Lastly, observe that in the DEBT control protocol, the price charged by a channel does not depend on its capacity. This is a natural consequence of the price being the Lagrange multiplier for the net-zero flow constraint, which also does not depend on the channel capacity. In contrast, in many other works, the imbalance price is normalized by the channel capacity \cite{ren2018optimal, lin2020funds, wang2024fence}; this is shown to work well in practice. The rationale for such a price structure is explained well in \cite{wang2024fence}, where this fee is derived with the aim of always maintaining some balance (liquidity) at each end of every channel. This is a reasonable aim if a channel is to never rebalance itself; the experiments of the aforementioned papers are conducted in such a regime. In this work, however, we allow the channels to rebalance themselves a few times in order to settle on a detailed balance flow. This is because our focus is on the long-term steady state performance of the protocol. This difference in perspective also shows up in how the price depends on the channel imbalance. \cite{lin2020funds} and \cite{wang2024fence} advocate for strictly convex prices whereas this work and \cite{varma2021throughput} propose linear prices.

\paragraph{On Rebalancing} 
Recall that the DEBT control protocol ensures that the flows in the network converge to a detailed balance flow, which can be sustained perpetually without any rebalancing. However, during the transient phase (before convergence), channels may have to perform on-chain rebalancing a few times. Since rebalancing is an expensive operation, it is worthwhile discussing methods by which channels can reduce the extent of rebalancing. One option for the channels to reduce the extent of rebalancing is to increase their capacity; however, this comes at the cost of locking in more capital. Each channel can decide for itself the optimum amount of capital to lock in. Another option, which we discuss in Section \ref{sec:five_node}, is for channels to increase the rate $\gamma$ at which they adjust prices. 

Ultimately, whether or not it is beneficial for a channel to rebalance depends on the time-horizon under consideration. Our protocol is based on the assumption that the demand remains steady for a long period of time. If this is indeed the case, it would be worthwhile for a channel to rebalance itself as it can make up this cost through the incentive fees gained from the flow of transactions through it in steady state. If a channel chooses not to rebalance itself, however, there is a risk of being trapped in a deadlock, which is suboptimal for not only the nodes but also the channel.

\section{Conclusion}
This work presents DEBT control: a protocol for payment channel networks that uses source routing and flow control based on channel prices. The protocol is derived by posing a network utility maximization problem and analyzing its dual minimization. It is shown that under steady demands, the protocol guides the network to an optimal, sustainable point. Simulations show its robustness to demand variations. The work demonstrates that simple protocols with strong theoretical guarantees are possible for PCNs and we hope it inspires further theoretical research in this direction.
\section{Experiments}
\label{sec:experiments}
The experiments are designed to address two key research questions.
First, \textbf{RQ1} evaluates whether the average $L_2$-norm of the counterfactual perturbation vectors ($\overline{||\perturb||}$) decreases as the model overfits the data, thereby providing further empirical validation for our hypothesis.
Second, \textbf{RQ2} evaluates the ability of the proposed counterfactual regularized loss, as defined in (\ref{eq:regularized_loss2}), to mitigate overfitting when compared to existing regularization techniques.

% The experiments are designed to address three key research questions. First, \textbf{RQ1} investigates whether the mean perturbation vector norm decreases as the model overfits the data, aiming to further validate our intuition. Second, \textbf{RQ2} explores whether the mean perturbation vector norm can be effectively leveraged as a regularization term during training, offering insights into its potential role in mitigating overfitting. Finally, \textbf{RQ3} examines whether our counterfactual regularizer enables the model to achieve superior performance compared to existing regularization methods, thus highlighting its practical advantage.

\subsection{Experimental Setup}
\textbf{\textit{Datasets, Models, and Tasks.}}
The experiments are conducted on three datasets: \textit{Water Potability}~\cite{kadiwal2020waterpotability}, \textit{Phomene}~\cite{phomene}, and \textit{CIFAR-10}~\cite{krizhevsky2009learning}. For \textit{Water Potability} and \textit{Phomene}, we randomly select $80\%$ of the samples for the training set, and the remaining $20\%$ for the test set, \textit{CIFAR-10} comes already split. Furthermore, we consider the following models: Logistic Regression, Multi-Layer Perceptron (MLP) with 100 and 30 neurons on each hidden layer, and PreactResNet-18~\cite{he2016cvecvv} as a Convolutional Neural Network (CNN) architecture.
We focus on binary classification tasks and leave the extension to multiclass scenarios for future work. However, for datasets that are inherently multiclass, we transform the problem into a binary classification task by selecting two classes, aligning with our assumption.

\smallskip
\noindent\textbf{\textit{Evaluation Measures.}} To characterize the degree of overfitting, we use the test loss, as it serves as a reliable indicator of the model's generalization capability to unseen data. Additionally, we evaluate the predictive performance of each model using the test accuracy.

\smallskip
\noindent\textbf{\textit{Baselines.}} We compare CF-Reg with the following regularization techniques: L1 (``Lasso''), L2 (``Ridge''), and Dropout.

\smallskip
\noindent\textbf{\textit{Configurations.}}
For each model, we adopt specific configurations as follows.
\begin{itemize}
\item \textit{Logistic Regression:} To induce overfitting in the model, we artificially increase the dimensionality of the data beyond the number of training samples by applying a polynomial feature expansion. This approach ensures that the model has enough capacity to overfit the training data, allowing us to analyze the impact of our counterfactual regularizer. The degree of the polynomial is chosen as the smallest degree that makes the number of features greater than the number of data.
\item \textit{Neural Networks (MLP and CNN):} To take advantage of the closed-form solution for computing the optimal perturbation vector as defined in (\ref{eq:opt-delta}), we use a local linear approximation of the neural network models. Hence, given an instance $\inst_i$, we consider the (optimal) counterfactual not with respect to $\model$ but with respect to:
\begin{equation}
\label{eq:taylor}
    \model^{lin}(\inst) = \model(\inst_i) + \nabla_{\inst}\model(\inst_i)(\inst - \inst_i),
\end{equation}
where $\model^{lin}$ represents the first-order Taylor approximation of $\model$ at $\inst_i$.
Note that this step is unnecessary for Logistic Regression, as it is inherently a linear model.
\end{itemize}

\smallskip
\noindent \textbf{\textit{Implementation Details.}} We run all experiments on a machine equipped with an AMD Ryzen 9 7900 12-Core Processor and an NVIDIA GeForce RTX 4090 GPU. Our implementation is based on the PyTorch Lightning framework. We use stochastic gradient descent as the optimizer with a learning rate of $\eta = 0.001$ and no weight decay. We use a batch size of $128$. The training and test steps are conducted for $6000$ epochs on the \textit{Water Potability} and \textit{Phoneme} datasets, while for the \textit{CIFAR-10} dataset, they are performed for $200$ epochs.
Finally, the contribution $w_i^{\varepsilon}$ of each training point $\inst_i$ is uniformly set as $w_i^{\varepsilon} = 1~\forall i\in \{1,\ldots,m\}$.

The source code implementation for our experiments is available at the following GitHub repository: \url{https://anonymous.4open.science/r/COCE-80B4/README.md} 

\subsection{RQ1: Counterfactual Perturbation vs. Overfitting}
To address \textbf{RQ1}, we analyze the relationship between the test loss and the average $L_2$-norm of the counterfactual perturbation vectors ($\overline{||\perturb||}$) over training epochs.

In particular, Figure~\ref{fig:delta_loss_epochs} depicts the evolution of $\overline{||\perturb||}$ alongside the test loss for an MLP trained \textit{without} regularization on the \textit{Water Potability} dataset. 
\begin{figure}[ht]
    \centering
    \includegraphics[width=0.85\linewidth]{img/delta_loss_epochs.png}
    \caption{The average counterfactual perturbation vector $\overline{||\perturb||}$ (left $y$-axis) and the cross-entropy test loss (right $y$-axis) over training epochs ($x$-axis) for an MLP trained on the \textit{Water Potability} dataset \textit{without} regularization.}
    \label{fig:delta_loss_epochs}
\end{figure}

The plot shows a clear trend as the model starts to overfit the data (evidenced by an increase in test loss). 
Notably, $\overline{||\perturb||}$ begins to decrease, which aligns with the hypothesis that the average distance to the optimal counterfactual example gets smaller as the model's decision boundary becomes increasingly adherent to the training data.

It is worth noting that this trend is heavily influenced by the choice of the counterfactual generator model. In particular, the relationship between $\overline{||\perturb||}$ and the degree of overfitting may become even more pronounced when leveraging more accurate counterfactual generators. However, these models often come at the cost of higher computational complexity, and their exploration is left to future work.

Nonetheless, we expect that $\overline{||\perturb||}$ will eventually stabilize at a plateau, as the average $L_2$-norm of the optimal counterfactual perturbations cannot vanish to zero.

% Additionally, the choice of employing the score-based counterfactual explanation framework to generate counterfactuals was driven to promote computational efficiency.

% Future enhancements to the framework may involve adopting models capable of generating more precise counterfactuals. While such approaches may yield to performance improvements, they are likely to come at the cost of increased computational complexity.


\subsection{RQ2: Counterfactual Regularization Performance}
To answer \textbf{RQ2}, we evaluate the effectiveness of the proposed counterfactual regularization (CF-Reg) by comparing its performance against existing baselines: unregularized training loss (No-Reg), L1 regularization (L1-Reg), L2 regularization (L2-Reg), and Dropout.
Specifically, for each model and dataset combination, Table~\ref{tab:regularization_comparison} presents the mean value and standard deviation of test accuracy achieved by each method across 5 random initialization. 

The table illustrates that our regularization technique consistently delivers better results than existing methods across all evaluated scenarios, except for one case -- i.e., Logistic Regression on the \textit{Phomene} dataset. 
However, this setting exhibits an unusual pattern, as the highest model accuracy is achieved without any regularization. Even in this case, CF-Reg still surpasses other regularization baselines.

From the results above, we derive the following key insights. First, CF-Reg proves to be effective across various model types, ranging from simple linear models (Logistic Regression) to deep architectures like MLPs and CNNs, and across diverse datasets, including both tabular and image data. 
Second, CF-Reg's strong performance on the \textit{Water} dataset with Logistic Regression suggests that its benefits may be more pronounced when applied to simpler models. However, the unexpected outcome on the \textit{Phoneme} dataset calls for further investigation into this phenomenon.


\begin{table*}[h!]
    \centering
    \caption{Mean value and standard deviation of test accuracy across 5 random initializations for different model, dataset, and regularization method. The best results are highlighted in \textbf{bold}.}
    \label{tab:regularization_comparison}
    \begin{tabular}{|c|c|c|c|c|c|c|}
        \hline
        \textbf{Model} & \textbf{Dataset} & \textbf{No-Reg} & \textbf{L1-Reg} & \textbf{L2-Reg} & \textbf{Dropout} & \textbf{CF-Reg (ours)} \\ \hline
        Logistic Regression   & \textit{Water}   & $0.6595 \pm 0.0038$   & $0.6729 \pm 0.0056$   & $0.6756 \pm 0.0046$  & N/A    & $\mathbf{0.6918 \pm 0.0036}$                     \\ \hline
        MLP   & \textit{Water}   & $0.6756 \pm 0.0042$   & $0.6790 \pm 0.0058$   & $0.6790 \pm 0.0023$  & $0.6750 \pm 0.0036$    & $\mathbf{0.6802 \pm 0.0046}$                    \\ \hline
%        MLP   & \textit{Adult}   & $0.8404 \pm 0.0010$   & $\mathbf{0.8495 \pm 0.0007}$   & $0.8489 \pm 0.0014$  & $\mathbf{0.8495 \pm 0.0016}$     & $0.8449 \pm 0.0019$                    \\ \hline
        Logistic Regression   & \textit{Phomene}   & $\mathbf{0.8148 \pm 0.0020}$   & $0.8041 \pm 0.0028$   & $0.7835 \pm 0.0176$  & N/A    & $0.8098 \pm 0.0055$                     \\ \hline
        MLP   & \textit{Phomene}   & $0.8677 \pm 0.0033$   & $0.8374 \pm 0.0080$   & $0.8673 \pm 0.0045$  & $0.8672 \pm 0.0042$     & $\mathbf{0.8718 \pm 0.0040}$                    \\ \hline
        CNN   & \textit{CIFAR-10} & $0.6670 \pm 0.0233$   & $0.6229 \pm 0.0850$   & $0.7348 \pm 0.0365$   & N/A    & $\mathbf{0.7427 \pm 0.0571}$                     \\ \hline
    \end{tabular}
\end{table*}

\begin{table*}[htb!]
    \centering
    \caption{Hyperparameter configurations utilized for the generation of Table \ref{tab:regularization_comparison}. For our regularization the hyperparameters are reported as $\mathbf{\alpha/\beta}$.}
    \label{tab:performance_parameters}
    \begin{tabular}{|c|c|c|c|c|c|c|}
        \hline
        \textbf{Model} & \textbf{Dataset} & \textbf{No-Reg} & \textbf{L1-Reg} & \textbf{L2-Reg} & \textbf{Dropout} & \textbf{CF-Reg (ours)} \\ \hline
        Logistic Regression   & \textit{Water}   & N/A   & $0.0093$   & $0.6927$  & N/A    & $0.3791/1.0355$                     \\ \hline
        MLP   & \textit{Water}   & N/A   & $0.0007$   & $0.0022$  & $0.0002$    & $0.2567/1.9775$                    \\ \hline
        Logistic Regression   &
        \textit{Phomene}   & N/A   & $0.0097$   & $0.7979$  & N/A    & $0.0571/1.8516$                     \\ \hline
        MLP   & \textit{Phomene}   & N/A   & $0.0007$   & $4.24\cdot10^{-5}$  & $0.0015$    & $0.0516/2.2700$                    \\ \hline
       % MLP   & \textit{Adult}   & N/A   & $0.0018$   & $0.0018$  & $0.0601$     & $0.0764/2.2068$                    \\ \hline
        CNN   & \textit{CIFAR-10} & N/A   & $0.0050$   & $0.0864$ & N/A    & $0.3018/
        2.1502$                     \\ \hline
    \end{tabular}
\end{table*}

\begin{table*}[htb!]
    \centering
    \caption{Mean value and standard deviation of training time across 5 different runs. The reported time (in seconds) corresponds to the generation of each entry in Table \ref{tab:regularization_comparison}. Times are }
    \label{tab:times}
    \begin{tabular}{|c|c|c|c|c|c|c|}
        \hline
        \textbf{Model} & \textbf{Dataset} & \textbf{No-Reg} & \textbf{L1-Reg} & \textbf{L2-Reg} & \textbf{Dropout} & \textbf{CF-Reg (ours)} \\ \hline
        Logistic Regression   & \textit{Water}   & $222.98 \pm 1.07$   & $239.94 \pm 2.59$   & $241.60 \pm 1.88$  & N/A    & $251.50 \pm 1.93$                     \\ \hline
        MLP   & \textit{Water}   & $225.71 \pm 3.85$   & $250.13 \pm 4.44$   & $255.78 \pm 2.38$  & $237.83 \pm 3.45$    & $266.48 \pm 3.46$                    \\ \hline
        Logistic Regression   & \textit{Phomene}   & $266.39 \pm 0.82$ & $367.52 \pm 6.85$   & $361.69 \pm 4.04$  & N/A   & $310.48 \pm 0.76$                    \\ \hline
        MLP   &
        \textit{Phomene} & $335.62 \pm 1.77$   & $390.86 \pm 2.11$   & $393.96 \pm 1.95$ & $363.51 \pm 5.07$    & $403.14 \pm 1.92$                     \\ \hline
       % MLP   & \textit{Adult}   & N/A   & $0.0018$   & $0.0018$  & $0.0601$     & $0.0764/2.2068$                    \\ \hline
        CNN   & \textit{CIFAR-10} & $370.09 \pm 0.18$   & $395.71 \pm 0.55$   & $401.38 \pm 0.16$ & N/A    & $1287.8 \pm 0.26$                     \\ \hline
    \end{tabular}
\end{table*}

\subsection{Feasibility of our Method}
A crucial requirement for any regularization technique is that it should impose minimal impact on the overall training process.
In this respect, CF-Reg introduces an overhead that depends on the time required to find the optimal counterfactual example for each training instance. 
As such, the more sophisticated the counterfactual generator model probed during training the higher would be the time required. However, a more advanced counterfactual generator might provide a more effective regularization. We discuss this trade-off in more details in Section~\ref{sec:discussion}.

Table~\ref{tab:times} presents the average training time ($\pm$ standard deviation) for each model and dataset combination listed in Table~\ref{tab:regularization_comparison}.
We can observe that the higher accuracy achieved by CF-Reg using the score-based counterfactual generator comes with only minimal overhead. However, when applied to deep neural networks with many hidden layers, such as \textit{PreactResNet-18}, the forward derivative computation required for the linearization of the network introduces a more noticeable computational cost, explaining the longer training times in the table.

\subsection{Hyperparameter Sensitivity Analysis}
The proposed counterfactual regularization technique relies on two key hyperparameters: $\alpha$ and $\beta$. The former is intrinsic to the loss formulation defined in (\ref{eq:cf-train}), while the latter is closely tied to the choice of the score-based counterfactual explanation method used.

Figure~\ref{fig:test_alpha_beta} illustrates how the test accuracy of an MLP trained on the \textit{Water Potability} dataset changes for different combinations of $\alpha$ and $\beta$.

\begin{figure}[ht]
    \centering
    \includegraphics[width=0.85\linewidth]{img/test_acc_alpha_beta.png}
    \caption{The test accuracy of an MLP trained on the \textit{Water Potability} dataset, evaluated while varying the weight of our counterfactual regularizer ($\alpha$) for different values of $\beta$.}
    \label{fig:test_alpha_beta}
\end{figure}

We observe that, for a fixed $\beta$, increasing the weight of our counterfactual regularizer ($\alpha$) can slightly improve test accuracy until a sudden drop is noticed for $\alpha > 0.1$.
This behavior was expected, as the impact of our penalty, like any regularization term, can be disruptive if not properly controlled.

Moreover, this finding further demonstrates that our regularization method, CF-Reg, is inherently data-driven. Therefore, it requires specific fine-tuning based on the combination of the model and dataset at hand.
\subsection{Proof sketch}

\begin{figure*}
\resizebox{\linewidth}{!}{
\begin{tikzpicture}[
>={Stealth[scale=1.2]}, 
    semithick, 
    group/.style={align=center,rounded corners=3pt,inner sep=5pt,path picture={\fill[left color=black!3, right color=black!0] (path picture bounding box.south west) rectangle (path picture bounding box.north east);}},
    lemma/.style={rounded corners=3pt, align=center,inner sep=5 pt,path picture={\fill[left color=blue!8, right color=blue!2] (path picture bounding box.south west) rectangle (path picture bounding box.north east);}},
    def/.style={rounded corners=3pt, align=center,inner sep=5 pt,        path picture={\fill[left color=yellow!8, right color=yellow!2] (path picture bounding box.south west) rectangle (path picture bounding box.north east);}},
    arrow/.style={->}
    % path picture={\fill[left color=red!10, right color=red!3] (path picture bounding box.south west) rectangle (path picture bounding box.north east);}]
    % path picture={\fill[left color=red!10, right color=red!3] (path picture bounding box.south west) rectangle (path picture bounding box.north east);}]
]
% Main decomposition
\node[group] (total) at (0,.7) {\textbf{Total regret}\\\large$\displaystyle \E\left[\sum_{t=1}^T r(s_t^m, \pi^m(s_t^m)) - \sum_{t=1}^T r(s_t, a_t)\right]$};

\node[group] (state) at (-4.5,-1.8) {\textbf{State-based regret}\\\large$\displaystyle \E\left[\sum_{t=1}^T r(s_t^m, \pi^m(s_t^m)) - \sum_{t=1}^T r(s_t, \pi^m(s_t))\right]$};

\node[group] (action) at (4,-1.8) {\textbf{Action-based regret}\\\large$\displaystyle \E\left[\sum_{t=1}^T r(s_t, \pi^m(s_t)) - \sum_{t=1}^T r(s_t, a_t)\right]$};

% State-based analysis
    \node[lemma] (lemma54) at (-10.3,-4) {\emph{\Cref{lem:split}} \\$\E[f(s_t^m) - f(s_t)] \leq \Delta_t$};

\node[lemma] (lemma55) at (-3,-4) {\emph{\Cref{lem:trajectories-induction}}\\$\Delta_t \leq \sum_{i=1}^{t-1} \sup_{X \subseteq S} \alpha_i(X)$};

\node[def] (defn41_x) at (-2.6,-6) {\Cref{def:ac} with\\$\mu_t(s,a) = P(s,a,X_t)$\\ for various $X_t\subseteq \s$};

\node[lemma] (lemma56) at (-6.4,-6) {\emph{\Cref{lem:trajectories}}\\$\Delta_t \leq R_T^{AC}$};

% \node[rounded corners=3pt, align=center,inner sep=5 pt] (interm) at (-10.6,-7.1) {$\E[r(s_t^m, \pi^m(s_t^m)) - r(s_t,\pi^m(s_t))] \le \Delta_t$};

\node[lemma] (statefinal) at (-5, -8) {\emph{\Cref{lem:state-regret}}\\ State-based regret $\le T R_T^{AC}$};

% \node[lemma] (statedelta) at (-10.5,-6.1) {$E[r(s_t^m, \pi^m(s_t^m)) - r(s_t, \pi^m(s_t))] \le \Delta_t$};

% Action-based analysis
\node[def] (defn41) at (3.5,-5) {\Cref{def:ac} with\\ $\mu_t(s,a) = r(s,a)$};

\node[lemma] (lemma53) at (3.5,-8) {\emph{\Cref{lem:action-regret}}\\Action-based regret $\leq R_T^{AC}$};

% Final result
\node[group,path picture={\fill[left color=green!7, right color=green!2] (path picture bounding box.south west) rectangle (path picture bounding box.north east);}] (final) at (0,-9.5) {Total regret $\leq (T+1)R_T^{AC} \in o(T)$};

% Main decomposition arrows
\draw[arrow] (total) -- (state);
\draw[arrow] (total) -- (action);

% State-based lemma dependency arrows
\draw[arrow] (lemma54) -- (lemma55) node[above,midway] {$f(s) = N_t^m(s,X)$};
\draw[arrow] (lemma55) -- (lemma56);
\draw[arrow] (defn41_x) -- (lemma56);
\draw[arrow] (lemma56) to (statefinal);
% \draw[arrow] (lemma54) -- (statedelta) node[midway,left] {$f(s) = r(s,\pi^m(s))$};
\draw[arrow] (lemma54) to[bend right=20] node[midway,left=.2cm] {$f(s) = r(s,\pi^m(s))$} (statefinal);
% \draw[arrow] (lemma54) to[bend right=20] (statefinal);
% \draw[arrow] (statedelta) to[bend right=20] (statefinal);
\draw[arrow] (statefinal) -- (final);
\draw[-,draw=black!30] (0,-1) -- (0,-8.5);

% Action-based arrows
\draw[arrow] (defn41) -- (lemma53);
\draw[arrow] (lemma53) -- (final);

\end{tikzpicture}
}
\caption{We decompose regret into state-based and action-based components by adding and subtracting $\E\big[\sum_{t=1}^T r(s_t,\pi^m(s_t))\big]$. Below, we show the dependencies of lemmas and applications of \Cref{def:ac} in the proof, leading to the final bound of $R_T \le (T+1)\rac$.}
\label{fig:proof}
\end{figure*}

The full proof is deferred to Appendix~\ref{sec:main-proof}, but we provide a visualization (\Cref{fig:proof}) and proof sketch here. 

First, we bound the action-based regret by a direct application of \Cref{def:ac}.\footnote{The fact that bounding the action-based regret is so simple is more a property of \Cref{def:ac} than of the decomposition.}

\begin{restatable}{lemma}{lemActionRegret}
\label{lem:action-regret}
If an algorithm satisfies \Cref{def:ac}, then $\E \big[\sum_{t=1}^T r(s_t, \pi^m(s_t)) - \sum_{t=1}^Tr(s_t, a_t)\big] \le  \rac$.
\end{restatable}

\begin{proof}
Let $\mu_t(s,a) := r(s,a)$ for all $t \in [T]$. Since $r$ satisfies local generalization, so does $\bfmu$. Hence by  \Cref{def:ac}, $\E \big[\sum_{t=1}^T r(s_t, \pi^m(s_t)) - \sum_{t=1}^T r(s_t, a_t)\big] \le  \rac$.
\end{proof}

It remains to bound the state-based regret. Rather than analyzing when $\smols$ is better or worse than $\sm$, we simply bound how much their distributions differ at all. Specifically, we will show that $\sup_{X\subseteq \s}(p_t^m(X) - p_t(X)) \le \rac$ for all $t \in [T]$. Let $\Delta_t = \sup_{X\subseteq \s}(p_t^m(X) - p_t(X))$; we will refer to this quantity frequently. 
First, we show that an entire class of expected values can be bounded by $\Delta_t$.


\begin{restatable}{lemma}{lemSplit}
\label{lem:split}
For any $t \in [T]$ and any measurable function $f: \s \to [0,1]$,  we have $\E[f(s_t^m) - f(s_t)] \le \Delta_t$.
\end{restatable}

The idea behind \Cref{lem:split} is that $\E[f(s_t^m) - f(s_t)]$ can only be large if $s_t^m$ is more concentrated than $s_t$ in states where $f$ is large. We use $\Delta_t$ to measure how large that difference in concentration can be. Also, since $f(s) \in [0,1]$, $f$ cannot be too much larger in these states where $s_t^m$ is more concentrated than $s_t$. Although there are some technical details, conceptually the proof amounts to:
\begin{align*}
\E[f(s_t^m) - f(s_t)] \le \sup_{s,s' \in \s} |f(s) - f(s')|\cdot  \Delta_t
\le  \Delta_t
\end{align*}
The most obvious application of \Cref{lem:split} is with $f(s) = r(s, \pi^m(s))$, and indeed, that will be one usage. However, we will also use this lemma to analyze the divergence between $\smols$ and $\sm$. Specifically, we will write $p_{t+1}^m(X) - p_{t+1}(X) = \E[f(s_t^m) - f(s_t)] + \alpha_t(X)$ for some functions $f$ and $\alpha_t$. Then \Cref{lem:split} will imply that
\begin{align*}
p_{t+1}^m(X) - p_{t+1}(X) \le&\ \Delta_t + \alpha_t(X) \quad \text{$\forall X \subseteq \s$}\\
\sup_{X\subseteq \s} (p_{t+1}^m(X) - p_{t+1}(X)) \le&\ \Delta_t + \sup_{X\subseteq \s} \alpha_t(X)\\
\Delta_{t+1} \le&\ \Delta_t + \sup_{X\subseteq \s} \alpha_t(X)
\end{align*}
The agent and mentor have the same initial state, so $\Delta_1 = 0$ and thus $\Delta_t \le \sum_{i=1}^{t-1} \sup_{X\subseteq \s} \alpha_t(X)$ by induction.

To enact this plan, we choose $f(s) = N_t^m(s,X)$ and $\alpha_t(X) = \E[N_t^m(s_t, X) - N_t(s_t, X)]$. Recall from \Cref{sec:model} that $N_t^m(s, X) = \Pr[s_{t+1}^m \in X \mid s_t^m = s]$ and $N_t(s,X) = \Pr[s_{t+1} \in X \mid s_t = s]$. The law of total expectation implies that $p_{t+1}^m(X) = \E[N_t^m(s_t^m, X)]$ and $p_{t+1}(X) = \E[N_t(s_t, X)]$, so
\begin{align*}
& \ \ p_{t+1}^m(X) - p_{t+1}(X)\\
=&\ \E[N_t^m(s_t^m, X)] - \E[N_t(s_t, X)]\\
=&\ \resizebox{\linewidth}{!}{%
\(\E[N_t^m(s_t^m, X) - N_t^m(s_t, X)] + \E[N_t^m(s_t, X) - N_t(s_t, X)]\)
}
\\
=&\ \E[N_t^m(s_t^m, X) - N_t^m(s_t, X)] + \alpha_t(X)
\end{align*}
(Interestingly, this decomposition is similar in structure to the state- vs action-based regret decomposition, although the terms here do not depend on $r$ at all.) 

This decomposition allows us to apply the aforementioned inductive strategy to obtain the following lemma:

\begin{restatable}{lemma}{lemTrajInd}\label{lem:trajectories-induction}
For any $t \in [T]$, $\Delta_t \le \sum_{i=1}^{t-1}\sup_{X\subseteq \s}\alpha_i(X)$.
\end{restatable}

This brings us to the trickiest part of the proof: bounding $\sum_{i=1}^{t-1}\sup_{X\subseteq \s}\alpha_i(X)$. The main idea is to invoke \Cref{def:ac} with $\mu_i(s, a) = P(s,a,X_i)$ for each $i \in [t-1]$ for every possible choice of $X_1,\dots,X_{t-1} \subseteq \s$. (It is also helpful to define $\mu_i(s,a)$ to be constant for $i > t$.) 

One can use the definition of total variation distance to show that if $P$ satisfies local generalization, so does this definition of $\bfmu$. Next, we can manipulate conditional expectations to show that $\E[N_i(s_i,X_i)] = \E[P(s_i,a_i,X_i)] = \E[\mu_i(s_i,a_i)]$ and $\E[N_i^m(s_i, X_i)] = \E[P(s_i, \pi^m(s_i), X_i)] = \E[\mu_i^m(s_i)]$. Thus applying \Cref{def:ac} to \Cref{lem:trajectories-induction} gives us
\begin{align*}
\Delta_t \le \sum_{i=1}^{t-1}\alpha_i(X_i) 
= \sum_{i=1}^T \E[\mu_i^m(s_i) - \mu_i(s_i, a_i)]
\le R_T^{AC}
\end{align*}
The result is \Cref{lem:trajectories}:

\begin{restatable}{lemma}{lemTraj}
\label{lem:trajectories}
If an algorithm satisfies \Cref{def:ac}, then $\Delta_t \le \rac$ for all $t \in [T]$.
\end{restatable}

Now we can bound the state-based regret by applying \Cref{lem:split} with $f(s) = r(s, \pi^m(s))$ followed by \Cref{lem:trajectories}. Specifically, we obtain
\[
\E\big[r(s_t^m, \pi^m(s_t^m)) - r(s_t, \pi^m(s_t))\big] \le \Delta_t \le \rac
\]
Summing this over $t$ produces \Cref{lem:state-regret}:

\begin{restatable}{lemma}{lemStateRegret}
\label{lem:state-regret}
If an algorithm satisfies \Cref{def:ac}, then \resizebox{1.02\linewidth}{!}{%
$\E\big[\sum_{t=1}^T r(s_t^m, \pi^m(s_t^m)) - \sum_{t=1}^T r(s_t, \pi^m(s_t))\big] \le T \rac$
.}
\end{restatable}

Since the state-based regret is at most $T \rac$ (\Cref{lem:state-regret}) and the action-based regret is at most $\rac$ (\Cref{lem:action-regret}), we conclude that $R_T \le (T+1)\rac$.
\section{Conclusion}
In this work, we propose a simple yet effective approach, called SMILE, for graph few-shot learning with fewer tasks. Specifically, we introduce a novel dual-level mixup strategy, including within-task and across-task mixup, for enriching the diversity of nodes within each task and the diversity of tasks. Also, we incorporate the degree-based prior information to learn expressive node embeddings. Theoretically, we prove that SMILE effectively enhances the model's generalization performance. Empirically, we conduct extensive experiments on multiple benchmarks and the results suggest that SMILE significantly outperforms other baselines, including both in-domain and cross-domain few-shot settings.


\section*{Disclaimer}
This paper was prepared for informational purposes [“in part” if the work is collaborative with external partners]  by the Artificial Intelligence Research group of JPMorgan Chase \& Co. and its affiliates ("JP Morgan'') and is not a product of the Research Department of JP Morgan. JP Morgan makes no representation and warranty whatsoever and disclaims all liability, for the completeness, accuracy or reliability of the information contained herein. This document is not intended as investment research or investment advice, or a recommendation, offer or solicitation for the purchase or sale of any security, financial instrument, financial product or service, or to be used in any way for evaluating the merits of participating in any transaction, and shall not constitute a solicitation under any jurisdiction or to any person, if such solicitation under such jurisdiction or to such person would be unlawful.

\bibliography{bibliography}

\newpage
\appendix 
% MSE
% \section{Mean Square Analysis of TTSA Last Iterates}
\section{Auxiliary Results}
In this section, we first present the derivation for \eqref{eq:recursion_expression}.
This expression will be used to derive the finite-time bounds on the second moments of $\tilde{x}_t$ and $\hat{y}_t$. 
Some useful properties of recurring quantities will be derived, before presenting the proof of Theorems \ref{thm:mse} and \ref{thm:clt}.


\subsection{Recursion Updates}\label{sec:recursion_mse}
%The recursive expression in Eq. \eqref{eq:recursion_expression} is standard. Here, we provide an alternative derivation. 
The goal of this sub-section is to derive the expressions for $B_t$ and $L_t$ in Eqs. \eqref{eq:recursion_expression} and \eqref{eq:Lt}, which is similar to the one in \citep{konda2004convergence}. 
For convenience, recall the updates in \eqref{eq:ttsa}
\begin{equation*}
    \begin{split}
        x_{t+1} &= x_t - \alpha_t \left(A_{ff} x_t + A_{fs} y_t - W_t \right),
        \\
        y_{t+1} &= y_t - \gamma_t \left(A_{sf} x_t + A_{ss} y_t - V_t \right),
    \end{split}
\end{equation*}
and the residues $\hat{x}_t \coloneqq x_t - x_\infty (y_t)$ and $\hat{y}_t \coloneqq y_t - y^*$, where $x_\infty (y) = -A_{ff}^{-1} A_{fs} y$. 
The next steps will describe how to choose a sequence of matrices $\{L_t\}$ such that the recursion for $(\tilde{x}_{t+1}, \hat{y}_{t+1})$ with $\tilde{x}_t \coloneqq \hat{x}_t + L_t \hat{y}_t$ can be simplified, and state the expressions for the resulting system. 



Let $F, S$ be the operators corresponding to the fast- and slow-time-scale systems, i.e., $F(x, y) = A_{ff} x + A_{fs} y$ and $S(x, y) = A_{sf} x + A_{ss}y$. 
Using the definition of $x_{\infty}(y)$ we have
\begin{align*}
    F(x_{\infty}(y),y) = 0\;\text{ for all } y \quad &\text{ and }\quad S(x_{\infty}(y^*),y^*) = 0, 
\end{align*}
and we express $F$ and $S$ as a function of $\hat{x}_t$ and $\hat{y}_t$
\begin{align*}
    F(x_{t},y_{t}) &= F(x_t, y_t) - F(x_\infty (y_t), y_t) = A_{ff}\hat{x}_{t} ,\\
    S(x_t, y_t) &= S(x_t, y_t) - S(x_\infty (y_t), y_t) + S(x_\infty (y_t), y_t) - S(x_\infty (y^*), y^*)
    \\ & 
    =A_{sf} \hat{x}_t + A_{sf} x_\infty (\hat{y}_t) + A_{ss} \hat{y}_t 
    \\ & = A_{sf} \hat{x}_t + \Delta \hat{y}_t .
\end{align*}
% Thus, we have the following relations
% \begin{align*}
% F(x_{t},y_{t}) = A_{ff}\hat{x}_{t}\quad &\text{ and }\quad S(x_{t},y_{t})  = A_{sf} \hat{x}_t + \Delta \hat{y}_t.\\
% F(x_{\infty}(y),y) = 0\;\text{ for all } y \quad &\text{ and }\quad S(x_{\infty}(y^*),y^*) = 0. 
% \end{align*}
First, we obtain a recursion for the fast variable $\hat{x}_t$
% First, we will derive the recursion for $\hat{z}_t = (\hat{x}_t, \hat{y}_t)$ using $x_\infty (y) = -A_{ff}^{-1} A_{fs} y$ and the relations above.
% , by linearity and the definitions of $x_\infty$ and $y^*$,
% \begin{align*}
%     F(x_t, y_t) &= F(x_t, y_t) - F(x_\infty (y_t), y_t) = A_{ff} \hat{x}_t, 
% \\ 
%     S(x_t, y_t) &= S(x_t, y_t) - S(x_\infty (y_t), y_t) + S(x_\infty (y_t), y_t) - S(x_\infty (y^*), y^*)
%     \\ & = A_{sf} \hat{x}_t + A_{sf} x_\infty (\hat{y}_t) + A_{ss} \hat{y}_t
%     \\  & = A_{sf} \hat{x}_t + \Delta \hat{y}_t .
% \end{align*}
% Elementary algebra gives
% Consider the fast iterate 
\begin{align*}
    \hat{x}_{t+1} 
    &= x_{t+1} - x_\infty (y_{t+1})
    \\ & 
    = \hat{x}_t - x_\infty (y_{t+1} - y_t) - \alpha_t F(x_t, y_t) + \alpha_t W_t
    \\ & 
    = \hat{x}_t - \gamma_t A_{ff}^{-1} A_{fs} S(x_t, y_t) - \alpha_t A_{ff} \hat{x}_t + \alpha_t W_t - \gamma_t A_{ff}^{-1} A_{fs} V_t
    \\ & = 
    (I - \alpha_t A_{ff} - \gamma_t A_{ff}^{-1} A_{fs} A_{sf}) \hat{x}_t - \gamma_t A_{ff}^{-1} A_{fs} \Delta \hat{y}_t + \alpha_t W_t - \gamma_t A_{ff}^{-1} A_{fs} V_t .
    \numberthis \label{eq:fast_untransformed}
\end{align*}
Similarly, the slow-time-scale iterate satisfies the recursion
\begin{align*}
    \hat{y}_{t+1} &= y_{t+1} - y^*
    = \hat{y}_t - \gamma_t S(x_t, y_t) + \gamma_t V_t = 
    (I - \gamma_t \Delta) \hat{y}_t - \gamma_t A_{sf} \hat{x}_t + \gamma_t V_t .
    \numberthis \label{eq:slow_transformed}
\end{align*}
Equations \eqref{eq:fast_untransformed} and \eqref{eq:slow_transformed} can be combined to write the update rule for $\hat{z}_t = (\hat{x}_t, \hat{y}_t)$ in terms of a new system $\hat{A}_t$ and $Q_t$, where
\begin{align*}
    &\hat{z}_{t+1} = \hat{z}_t - \hat{A}_t \hat{z}_t + Q_t N_t,
    \\
    & \hat{A}_t = \begin{pmatrix}
        \alpha_t A_{ff} + \gamma_t A_{ff}^{-1} A_{fs} A_{sf} & \gamma_t A_{ff}^{-1} A_{fs} \Delta 
        \\ 
        \gamma_t A_{sf} & \gamma_t \Delta 
    \end{pmatrix} ,
    \quad 
    Q_t = \begin{pmatrix}
        \alpha_t I & - \gamma_t A_{ff}^{-1} A_{fs} \\ 
        0 & \gamma_t I
    \end{pmatrix} .
\end{align*}
Second, we design a sequence of matrices $\{L_t\}$ such that the recursion for $\tilde{x}_{t} = \hat{x}_{t} + L_{t}\hat{y}_{t}$ does not depend on $\hat{y}_t$ explicitly. 
The choice of $L_t$ determines the coordinate transformation
\begin{align*}
    \tilde{z}_t = P_t \hat{z}_t, \quad P_t = \begin{pmatrix}
        I & L_t \\ 0 & I
    \end{pmatrix} .
\end{align*}
For any $L_t$, it holds that
\begin{equation}\label{eq:ztilde_derivation}
    \tilde{z}_{t+1} = P_{t+1} (I - \hat{A}_t) P_t^{-1} \tilde{z}_t + P_{t+1} Q_t N_t .
\end{equation}
The matrix $L_t$ is to be chosen such that $P_{t+1} P_t^{-1} - P_{t+1} \hat{A}_t P_t^{-1}$ is lower triangular so that $\tilde{x}_{t+1}$ is decoupled from $\hat{y}_t$.
% , thus removing the dependency on $\hat{y}_t$ of $\tilde{x}_{t+1}$. 
Using $\hat{A}_t^{ff}$ and so on to denote the blocks in $\hat{A}_t$, we then have that the upper-right component of $P_{t+1} \hat{A}_t P_t^{-1}$ is given by 
\begin{align*}
    L_{t+1} - L_t
    + (\hat{A}_t^{ff} + L_{t+1} \hat{A}_t^{sf}) L_t - (\hat{A}_t^{fs} + L_{t+1} \hat{A}_t^{ss}) = 0 
    .
\end{align*}
The above calculation is obtained by matrix multiplication and inverting $P_t$, which is equated to zero to achieve the decoupling. 
Substituting the expressions for $\hat{A}_t$ in the identity gives
\begin{align*}
    L_{t+1} (I + \hat{A}_t^{sf} L_t - \hat{A}_t^{ss}) = \hat{A}_t^{fs} + (I - \hat{A}_t^{ff}) L_t,
\end{align*}
which yields the recursion for $L_t$ in \eqref{eq:Lt}. Given $L_{t}$, we finally derive the system matrix $B_t$ in Eq. \eqref{eq:recursion_expression}.
The expression for $B_t^{ff}$ can be easily obtained from Eq. \eqref{eq:ztilde_derivation}, where $B_t^{fs} = 0 $ by design, as
\begin{equation*}
    B_t^{ff} = A_{ff} + \frac{\gamma_t}{\alpha_t} (L_{t+1} + A_{ff}^{-1} A_{fs}) A_{sf} .
\end{equation*}
In addition, for any choice of $L_t$, $\hat{y}_{t+1}$ is related to $(\tilde{x}_t, \hat{y}_t)$ as follows:
\begin{equation*}
    \hat{y}_{t+1} = (I - \gamma_t (\Delta - A_{sf} L_t)) \hat{y}_t - \gamma_t A_{sf} \tilde{x}_t + \gamma_t V_t,
\end{equation*}
which yields the derivation for $B_t^{sf} = A_{sf}$ and $B_t^{ss} = \Delta - A_{sf} L_t$, where
\begin{equation}\label{eq:slow_derivation}
    \hat{y}_{t+1} = \hat{y}_t - \gamma_t (B_t^{sf} \tilde{x}_t + B_t^{ss} \hat{y}_t - V_t ) .
\end{equation}
In summary, the choice of $L_t$ in Eq. \eqref{eq:Lt} results in the updates
\begin{equation}\label{eq:fast_slow_updates}
    \begin{split}
    \tilde{x}_{t+1} &= (I - \alpha_t A_{ff}) \tilde{x}_t - \gamma_t \delta_t A_{sf} \tilde{x}_t + \alpha_t W_t + \gamma_t \delta_t V_t, 
    \\ 
    \hat{y}_{t+1} &= (I - \gamma_t \Delta) \hat{y}_t + \gamma_t A_{sf} L_t \hat{y}_t - \gamma_t A_{sf} \tilde{x}_t + \gamma_t V_t.     
    \end{split}
\end{equation}
% which can be rearranged to obtain 
% \begin{align*}
%     L_{t+1}  = (\hat{A}_{ff} - \hat{A}_{fs}) (\hat{A}_{sf} L_t + \hat{A}_{ss})^{-1}
% \end{align*}



\subsection{Usage of Assumptions \ref{assumption:first}--\ref{assumption:last}}\label{sec:recursion}
Here we state a few elementary properties that will be used to prove Theorems \ref{thm:mse} and \ref{thm:clt}.
% useful in the analysis of mean square analysis.
We will express the difference between finite-time second moments and the asymptotic covariance as a contraction, and so we require that the contraction is well defined for symmetric matrices, rather than for definite matrices.  
The contraction that appears often is as follows. Given any matrix $A$, we denote $\|A\|$ is the matrix norm of $A$. Let $A$ satisfy $\mu I \preceq A + A^T \preceq \nu I$. Given any symmetric matrix $R$, we have
\begin{align*}
    R - \alpha (A R + R A^T)
    &= (I - \alpha A) R (I - \alpha A)^T - \alpha^2 A R A^T,
\end{align*}
which implies 
\begin{align*}
    % R - \alpha (A R + R A^T)
    % &= (I - \alpha A) R (I - \alpha A)^T - \alpha^2 A R A^T
    % \\
    % \Rightarrow 
    \lVert R - \alpha (A R + R A^T) \rVert &\leq \lVert I - \alpha A \rVert^2 \lVert R \rVert + \alpha^2 \lVert A R A^T \rVert.
\end{align*}
In addition, we have $\lVert A \rVert \leq \nu / 2$, which gives
\begin{align*}
    \lVert I - \alpha A \rVert^2 = \sup_{x \neq 0} \left(1 - \alpha \frac{x^T (A + A^T) x}{\lVert x \rVert^2} + \alpha^2 \frac{\lVert A x \rVert^2}{\lVert x \rVert^2}\right) 
    \leq 1 - \alpha \mu + \alpha^2 \lVert A \rVert^2.
\end{align*}
Thus, for $\alpha \leq \mu/\nu^2$ we obtain
\begin{equation}\label{eq:contraction}
    \lVert R - \alpha (A R + R A^T) \rVert 
    \leq 
    \left(1 - \alpha \mu + \alpha^2 \frac{\nu^2}{2}\right) \lVert R \rVert 
    \leq \left(1 - \alpha \frac{\mu}{2}\right)\lVert R \rVert     .
\end{equation}
% For a symmetric matrix $S$ and $\mu I \preceq A + A^T \preceq \nu I$, we have that 
% \begin{align*}
%     S - \alpha (AS + S A^T) = (I/2 - \alpha A) S + S (I/2 - \alpha A)^T ,
% \end{align*}
% and using Weyl's inequality to deduce that $\lambda_{\max} (A) \leq \nu/2$ and $\lVert I/2 - \alpha A\rVert \leq 1 - \alpha $

% When $\nu I \succeq A + A^T \succeq \mu I$ and $X \succeq 0$, we have by Lyapunov stability that 
% \begin{align*}
%     \nu X \succeq A X + X A^T \succeq \mu X .
% \end{align*}
% This can be seen by rearranging $(A + \mu/2 I) X + X (A + \mu/2 I)^T \succ 0$.
% Using this, we have that for a symmetric matrix $S$,
% \begin{align*}
%     \lVert S - (A S + S A^T)\rVert 
% \end{align*}
% {\color{red}Finish this part up and see if I'm using it correctly. }


Next, the choice of step size in Assumption \ref{assumption:steps} implies that
\begin{align*}
    \alpha_t - \alpha_{t+1} \leq \frac{\alpha_{t+1}}{t} ,
\end{align*}
obtained from the elementary inequality $(\frac{t+1}{t})^a \leq (1+\frac{1}{t})$ for every $a \in (0, 1]$. 
Moreover, this implies $\alpha_{t+1}^{-1} - \alpha_t^{-1} \leq (\alpha_{t} t)^{-1}$ and is also true for the sequence $\{\gamma_t\}$, which is tight for the range of values $a, b$ we consider. 

Finally, we derive a recursive bound for the absolute error of the second moments of fast and slow iterates from their asymptotic covariances.
After an induction step, the dominant rates are then obtained using Lemma 14 \citep{kaledin2020finite}, which states that for $\alpha_1 \mu_{ff}/4 \leq 1$ and constants
\begin{align*}
    \xi = 1 + \max \left\{\alpha_1 \frac{\mu_{ff}}{4}, \gamma_1 \frac{\mu_\Delta}{16} \right\},
    \quad 
    \xi' = \frac{8}{\mu_{ff}}\max \left\{\xi, \frac{\mu_{ff}}{4\mu_{\Delta}} \xi, 2 \xi^3
        \right\},
\end{align*}
it holds that
\begin{equation}
    \begin{split}
        \sum_{t=1}^n \gamma_t \prod_{j=t+1}^n \left(1 - \frac{\mu_{ff}}{4} \alpha_j\right) 
        \leq \xi' \frac{\gamma_{n}}{\alpha_n}
        \\
        \sum_{t=1}^n \alpha_t \gamma_t \prod_{j=t+1}^n \left(1 - \frac{\mu_{ff}}{4} \alpha_j\right) 
        \leq \xi' \gamma_{n}
        ,
        \\
        \sum_{t=1}^n \frac{\alpha_t}{t} \prod_{j=t+1}^n \left(1 - \frac{\mu_{ff}}{4} \alpha_j \right) 
        \leq 
        \xi' \frac{1}{n} .
    \end{split} 
    \label{eq:induction_size_equation}
\end{equation}
Variants of the above inequalities can also be deduced to be true, for example when $\gamma_t$ in the first inequality is replaced by $\gamma_t^2$. 


\subsection{Size of Linear Transformation}
\begin{proposition}\label{prop:Lt_welldefined}
    The sequence $\{L_t\}$ is well-defined and converges to zero at rate $\mathcal{O}(\gamma_t/\alpha_t)$.
\end{proposition}
The recursive definition for $L_{t+1}$ in Eq. \eqref{eq:Lt} is equivalently expressed as 
\begin{align*}
    L_{t+1} (I - \gamma_t \Delta + \gamma_t A_{sf} L_t) 
    &= \left(I - \alpha_t A_{ff} - \gamma_t A_{ff}^{-1} A_{fs} A_{sf} \right) L_t + \gamma_t A_{ff}^{-1} A_{fs} \Delta 
    \\ & 
    = \left(I - \alpha_t A_{ff}\right) L_t + \gamma_t A_{ff}^{-1} A_{fs}(\Delta - A_{sf} L_t) .
\end{align*}
Denoting $B_t^{ss} = \Delta - A_{sf} L_t$ as before, 
\begin{equation}
    L_{t+1} (I - \gamma_t B_t^{ss}) = (I - \alpha_t A_{ff}) L_t + \gamma_t A_{ff}^{-1} A_{fs} B_t^{ss} .
\end{equation}
It was shown in Lemma 18 of \citep{kaledin2020finite} that $I - \gamma_t B_t^{ss}$ is non-singular and that $\lVert L_t \rVert$ is uniformly bounded.
Therefore, $ \max_{t \leq n} \prod_{j=t+1}^n \lVert (I - \gamma_t B_t^{ss})^{-1} \rVert$ is uniformly bounded by some constant $K_{ss}$ and we have
\begin{align*}
    \lVert L_{t+1} \rVert 
    &\leq \lVert I - \alpha_t A_{ff} \rVert \lVert L_t (I - \gamma_t B_t^{ss})^{-1} \rVert + \gamma_t \lVert A_{ff}^{-1} A_{fs} B_t^{ss} \rVert 
    \\ & \leq 
    \sqrt{1 - \alpha_t \frac{\mu_{ff}}{2}} \lVert (I - \gamma_t B_n^{ss})^{-1}\rVert \lVert L_t \rVert 
    + \gamma_t \lVert A_{ff}^{-1} A_{fs} B_n^{ss} \rVert 
    \\ & \leq \left(1 - \frac{\mu_{ff}}{4} \alpha_t\right) \lVert (I - \gamma_t B_t^{ss})^{-1} \rVert \lVert L_t \rVert + \gamma_t \lVert A_{ff}^{-1} A_{fs} B_t^{ss} \rVert 
\end{align*}
when $\alpha_t \leq \mu/\nu^2$, where we used that $\sqrt{1 - x} \leq 1 - x/2$ when $x \in [0, 1]$. 
By induction and Eq. \eqref{eq:induction_size_equation}, there exists a constant $K_L$ such that
\begin{align*}
    \frac{1}{K_{ss}}\lVert L_{n+1} \rVert 
    & \leq \prod_{t=1}^n \left(1 - \frac{\mu_{ff}}{4} \alpha_t \right) \lVert L_1 \rVert + \sum_{t=1}^n \gamma_t  \prod_{j=t+1}^n \left(1 - \frac{\mu_{ff}}{4} \alpha_j\right)  \lVert A_{ff}^{-1} A_{fs} B_t^{ss} \rVert 
    \\ & \leq 
    \prod_{t=1}^n \left(1 - \frac{\mu_{ff}}{4} \alpha_t \right) \lVert L_1 \rVert + K_L \frac{\gamma_n}{\alpha_n}
    .
\end{align*}



% \subsection{Remove after completing above; Proof of Proposition \ref{prop:Lt_welldefined}}




% {\color{red}
% Remove after finishing above:
% I need to merge this to blend in with this paper. 
% After parsing the requirements on my own, merge them into Assumption \ref{assumption:steps}.
% }
% \begin{proposition}\label{prop:Lt_welldefined}
%     {\color{red}Needs fixing/proving:}
%     When step sizes are chosen so that
%     \begin{equation}\label{eq:stepsize}
%         \gamma_1 \leq \min \frac{1}{2} \left\{ \frac{1}{\lVert Q_\Delta \rVert^2 \lVert \Delta\rVert^2_{Q_\Delta}}, \frac{\lVert Q_\Delta\rVert^2}{\lVert \Delta \rVert_{Q_\Delta} \lVert Q_\Delta\rVert^2 + 1} \right\}
%         , \quad 
%         \alpha_1 \leq \frac{1}{2 \lVert Q_{ff}\rVert^2 \lVert A_{ff}\rVert_{Q_ff}^2} ,
%     \end{equation}    
%     there exists a sequence of matrices $\{L_t\}$ with $L_0 = 0$ such that $I - \gamma_t (\Delta - A_{sf}L_t)$ is non-singular. 
%     Consequently, there exists a sequence satisfying the recursion \eqref{eq:Lt} for all $t >0$.
% \end{proposition}
% % \begin{customprop}{\ref{prop:Lt_welldefined}}
% %     When step sizes are chosen so that
% %     \begin{equation}\label{eq:stepsize1}
% %         % \gamma_0 \leq \min \frac{1}{2} \left\{ \frac{1}{\lVert Q_\Delta \rVert^2 \lVert \Delta\rVert^2_{Q_\Delta}}, \frac{\lVert Q_\Delta\rVert^2}{\lVert \Delta \rVert_{Q_\Delta} \lVert Q_\Delta\rVert^2 + 1} \right\}
% %         % , \quad 
% %         % \alpha_0 \leq \frac{1}{2 \lVert Q_{ff}\rVert^2 \lVert A_{ff}\rVert_{Q_ff}^2} ,
% %         \alpha_t \leq \frac{1}{2} \lVert Q_{ff} \rVert^{-2} \lVert A_{ff} \rVert^{-2}_{Q_{ff}} , \quad
% %         \gamma_t \leq \frac{1}{2} \left(\lVert \Delta \rVert_{Q_\Delta} + L_\infty \lVert A_{sf}\rVert_{Q_{ff}, Q_\Delta}\right)^{-1} ,
% %     \end{equation}    
% %     there exists a sequence of matrices $\{L_t\}$ with $L_0 = 0$ such that $I - \gamma_t (\Delta - A_{sf}L_t)$ is non-singular. 
% %     Consequently, there exists a sequence satisfying the recursion \eqref{eq:Lt} for all $t >0$.
% %     {\color{red}     There might be slight edits we need to make on the step sizes. 
% %     Must guarantee that all conditions used are met. }
% % \end{customprop}
% This will be proved with induction: (1) Show that $L_1$ is well-defined with bounded norm.
% (2) If $\lVert L_t \rVert_{Q_\Delta, Q_{ff}} \leq L_\infty$.
% {\color{red}To do: Step (1).
% }


% The second step is shown below.
% First observe that the recursion for $L_t$ can be written with a matrix identity on $L = L_t, L' = L_{t+1}$:
% \begin{align*}
%     L' (I-\gamma_t \Delta + \gamma_t A_{sf} L) 
%     &= \left(I - \alpha_t A_{ff} - \gamma_t A_{ff}^{-1} A_{fs} A_{sf}\right) L + \gamma_t A_{ff}^{-1} A_{fs} \Delta  
%     \\
%     &= (I - \alpha_t A_{ff}) L + \gamma_t A_{ff}^{-1} A_{fs} (\Delta - A_{sf} L)
%     .
%     \end{align*}
% Denoting $B^{ss}_t = \Delta - A_{sf} L_t$, we then have the expression
% \begin{equation}\label{eq:Ltidentity}
%     L' (I - \gamma_t B_t^{ss}) = (I - \alpha_t A_{ff})L + \gamma_t A_{ff}^{-1} A_{fs} B_t^{ss}.
% \end{equation}
% We will now see that there exists a unique solution to \eqref{eq:Ltidentity} if $\alpha_t, \gamma_t$ are sufficiently small.
% An upper bound on the norm of $L'$ can then be derived using that (Lemma 17, \citet{kaledin2020finite}) when $\alpha_t \leq \lVert Q_{ff} \rVert^2$, then
% \begin{equation}
%     \lVert I - \alpha_t A_{ff} \rVert_{Q_{ff}} \leq 1 - \alpha_t a_{ff} , \quad
%     a_{ff} = \frac{1}{2} \lVert Q_{ff} \rVert^{-2} .
% \end{equation}


% \begin{lemma}[Lemma 18, \citet{kaledin2020finite}]\label{lem:Ltsize}
%     Suppose $\lVert L \rVert_{Q_\Delta, Q_{22}} \leq L_\infty$ and consider the step sizes
%     \begin{align*}
%         \alpha_t \leq \frac{1}{2} \lVert Q_{ff} \rVert^{-2} \lVert A_{ff} \rVert^{-2}_{Q_{ff}} , \quad
%         \gamma_t \leq \frac{1}{2} \left(\lVert \Delta \rVert_{Q_\Delta} + L_\infty \lVert A_{sf}\rVert_{Q_{ff}, Q_\Delta}\right)^{-1} .
%     \end{align*}
%     Then there exists a unique solution $L'$ to \eqref{eq:Ltidentity} that satisfies
%     \begin{align*}
%         \lVert L' \rVert_{Q_\Delta, Q_{ff}} 
%             &\leq 
%         (1 - \alpha_t a_{ff}) \lVert L \rVert_{Q_{\Delta}, Q_{ff}} + \gamma_t C_D (L_\infty) , 
%         \\
%         C_D (L_\infty) 
%             &= 
%         2
%         \left(
%             \lVert A_{ff}^{-1} A_{fs} \rVert_{Q_\Delta, Q_{ff}} + L_\infty
%         \right) 
%         \left(
%             \lVert \Delta \rVert_{Q_\Delta} + L_\infty \lVert A_{sf} \rVert_{Q_{ff}, Q_\Delta} 
%         \right)
%     \end{align*}
%     Moreover, this implies that if $\gamma_t/\alpha_t \leq \epsilon a_{ff}/C_D(L_\infty)$, then $\lVert L'\rVert_{Q_\Delta, Q_{ff}} \leq L_\infty$.
% \end{lemma}
% Substituting back $L_t$ and $L_{t+1}$, we then have by induction that
% \begin{align*}
%     \lVert L_t\rVert_{Q_\Delta, Q_{ff}}
%         \leq 
%     C_D (L_\infty)
%     \sum_{j=0}^t \gamma_j \prod_{k=j+1}^t (1 - \alpha_k a_{ff} ) .
% \end{align*}
% The last term including summation is a recurring quantity which is extensively analyzed in Lemma 14 \citep{kaledin2020finite}, in this case (iv) asserts that for $\alpha_0 \leq a_{ff}^{-1} $,
% \begin{align*}
%     & \sum_{j=0}^t \gamma_j \prod_{k=j+1}^t (1 - \alpha_k a_{ff} ) \leq
%     \frac{2\gamma_t}{a_{ff} \alpha_t}  
%         \max \left\{
%              \xi \max \left\{1, \frac{a_{ff}}{4 a_\Delta}\right\},
%             2 \xi^3
%     \right\} , \\
%     & \xi = 1 + \max\left\{ \frac{a_{ff}}{8}\alpha_0, \frac{a_{\Delta}}{16} \gamma_0\right\} , \quad
%     a_{\Delta} = \frac{1}{2 \lVert Q_\Delta \rVert^2} .
% \end{align*}
% {\color{red}Lastly, we need to show that this upper bound is again bounded above by $L_\infty$, which I don't think is shown in \citep{kaledin2020finite}.}




% \textbf{Relating operator norm with weighted norm:}
% From the elementary inequalities
% \begin{align*}
%     \lambda_{\min}(Q) \lVert x \rVert^2 \leq x^T Q x \leq \lambda_{\max}(Q) \lVert x \rVert^2 ,
% \end{align*}
% we obtain that substituting $L x$ for $x$ above, the following is satisfied for any $Q \succ 0$:
% \begin{equation}
%     \frac{\lambda_{\min}(Q)}{\lambda_{\max}(Q)} \lVert L \rVert^2
%     \leq 
%     \lVert L \rVert_Q^2 = \sup_{x \neq 0} \frac{x^T L Q L x}{x^T Q x} 
%     \leq 
%     \frac{\lambda_{\max}(Q) }{\lambda_{\min}(Q)} \lVert L \rVert^2 .
% \end{equation}
% This is obtained by optimizing the numerator and denominator separately.
% From the lower bound, we have that for any positive definite $Q$
% \begin{equation}
%     \lVert L \rVert^2 \leq \kappa (Q) \lVert L \rVert_Q^2 . 
% \end{equation}
% More generally,
% \begin{equation}
%     \lVert L_t \rVert^2 \leq \frac{\lambda_{\max}(Q_{\Delta})}{\lambda_{\min}(Q_{ff})} \lVert L_t \rVert_{Q_\Delta, Q_{ff}}^2
% \end{equation}
% To relate this back to the properties of $\Delta$ and $A_{ff}$, we need to evaluate the minimum and maximum eigenvalues of a matrix $Q$ that solves the Lyapunov equation.
% \begin{proposition}[Theorem 10, \citet{lancaster1970explicit}]
%     Let $Q \succ 0$ be the unique solution to the Lypaunov equation
%     \begin{equation}
%         A Q + Q A^T = I 
%     \end{equation}
%     for a stable matrix $-A$.
%     Then,
%     \begin{align}
%         \frac{1}{\lvert \lambda_{\min}(A + A^T) \rvert} &\leq \lambda_{\min}(Q) \leq \frac{1}{2 \lvert \lambda_{\min}(A) \rvert} 
%         \\ 
%         \frac{1}{2 \lvert \lambda_{\max}(A)\rvert } &\leq \lambda_{\max}(Q) \leq \frac{1}{\lvert \lambda_{\max}(A + A^T)\rvert} . 
%     \end{align}
%     % {\color{red}But $A + A^T$ may not be Hurwitz? I may need to refine this. 
%     % Check details of this reference.
%     % }
% \end{proposition}
% {\color{blue}To do:
%     Using this, obtain a bound on the unweighed norm $\lVert L_t\rVert$ for all $t$ as a function of $\Delta$ and $A_{ff}$, in place of $Q_\Delta, Q_{ff}$.
% }
% \begin{lemma}\label{lem:unweighted_norm}
%     The unweighted norm $\lVert L_t \rVert$ is uniformly bounded for all time. {\color{red}Specify problem parameters.}
% \end{lemma}



% \subsection{Bound on $\Phi$: Lemma 12 \& 14 in \citep{kaledin2020finite}}
% Step 1: (E.g. Lemma 17 in \citep{kaledin2020finite}).
% For every $\alpha_t \leq a_{ff}$, {\color{red}Correct step size requirement.}
% \begin{equation}
%     \lVert I - \alpha_t A_{ff} \rVert_{Q_{ff}} \leq 1 - \alpha_t a_{ff} .
%     % \sum_{j=0}^{T-1} \alpha_j \sum_{}
% \end{equation}

% Step 2: Lemma 12 in \citep{kaledin2020finite}.

% Step 3: Relate back to unweighted norm. 

% Note that we also need to show that
% \begin{equation}
%     \frac{1}{T}\sum_{j=0}^{T-1} \lVert \Phi_{jT} \rVert^2 = \mathcal{O}(T^{-(1-\delta)})
% \end{equation}
% for some $\delta$ as in Appendix B \citep{srikant2024CLT}.
% Universal bound is not sufficient, since it would give a much worse rate of a constant above. 
\section{Mean Square Analysis of TTSA Last Iterates}
In this section, we use the above properties to prove Theorem \ref{thm:mse}. 
The proof is divided into three parts, starting from the second moment of $\tilde{x}_t$, the joint covariance $\mathbb{E}\tilde{x}_t \hat{y}_t$, and the second moment of the slow-time-scale $\hat{y}_t$.
Only the second moments of $\tilde{x}_t$ and $\hat{y}_t$ are included in the statement of Theorem \ref{thm:mse}, but the finite-time bound on the joint covariance is required as an intermediate step to establish the second moment of the slow-time-scale.


The transformed fast-time-scale $\tilde{x}_t$ is first analyzed because it is expressed recursively without an explicit dependence on $\hat{y}_t$. 
This will be used to derive the finite-time bounds on the joint-time-scale covariance $\mathbb{E}\tilde{x}_t \hat{y}_t^T$, which is used as an intermediate result to derive the finite-time bound on the slow-time-scale covariance $\mathbb{E}\hat{y}_t \hat{y}_t^T$. 




\subsection{Fast-Time-Scale Mean-Square Convergence}\label{sec:fast_mse}
% {\color{red}I'm not sure if the contraction holds w.r.t. $\mu$ or $\nu$.
% Also, I'm using operator norms instead of Loewner order b/c sometimes the matrices are noto definite (difference between second moment and covariance), and so an inequality cannot be established wrt the recursion. 
% Fix above descriptions for this as welll. 
% }
Consider the transformed last iterate $\tilde{x}_t = \hat{x}_t + L_t \hat{y}_t$ 
\begin{equation}\label{eq:fast_recursion}
    \tilde{x}_{t+1} = \tilde{x}_t - \alpha_t B_t^{ff} \tilde{x}_t + \alpha_t W_t + \gamma_t (L_{t+1} + A_{ff}^{-1}A_{fs} ) V_t ,   
    % (I - \alpha_t A_{ff}) \tilde{x}_t - \gamma_t (L_{t+1} + A_{ff}^{-1} A_{fs}) A_{sf} \tilde{x}_t + \alpha_t W_t + \gamma_t (L_{t+1} + A_{ff}^{-1} A_{fs})V_t 
\end{equation}
where $\hat{x}_t = x_t - x_\infty (y_t), \hat{y}_t = y_t - y^*$, and 
\begin{equation}\label{eq:Bt_ff}
    B_t^{ff} = A_{ff} + \frac{\gamma_t}{\alpha_t}(L_{t+1} + A_{ff}^{-1} A_{fs}) A_{sf} .
\end{equation}
The quantity $L_{t+1} + A_{ff}^{-1} A_{fs}$ appears frequently, so we refer to it as $\delta_t$ stated here for future reference:
\begin{equation}\label{eq:delta}
    \delta_t = L_{t+1} + A_{ff}^{-1} A_{fs}.
\end{equation}
% {\color{red}(Fixing proof for martingales) Lemmas may need to be modified to take unconditional expectations, both in main and appendix.
% Need to answer: Is the proof yielding a bound on $\mathbb{E}\lVert \tilde{x}_{n+1} \tilde{x}_{n+1}^T - \alpha_{N+1} \Sigma_{ff} \rVert$ or on $\lVert \mathbb{E} \tilde{x}_{n+1} \tilde{x}_{n+1}^T - \alpha_{n+1} \Sigma_{ff} \rVert$? 
% Former is much stronger, and I think that is what we obtain. 
% However, we are using the latter for the MAE/MSE bounds, so in case we state the former we should clarify that we obtain the former, and use this to infer the latter. 
% Former doesn't make sense; the latter is true. 
% }
We use $X_t$ to denote $\mathbb{E}[\tilde{x}_t \tilde{x}_t^T]$. In addition,  let $\Sigma_{ff}$ be the unique solution to 
\begin{equation}\label{eq:fast_covariance}
        A_{ff} \Sigma_{ff} + \Sigma_{ff} A_{ff}^T = \Gamma_{ff}.
    \end{equation}
We note that $\Sigma_{ff}$ exists since $A_{ff}$ satisfies $A_{ff} + A_{ff}^T$ is negative. 
% We use $X_t$ to denote $\mathbb{E} [\tilde{x}_t \tilde{x}_t^T | \historyprev]$.
\begin{lemma}\label{lem:fast_mse}
Under Assumptions \ref{assumption:first}-\ref{assumption:last}, there exists a problem-dependent constant $M_f > 0$ such that 
    \begin{align*}
        \lVert X_{n+1} - \alpha_{n+1} \Sigma_{ff}\rVert \leq 
        \prod_{t=1}^n \left(1 - \alpha_t \frac{\mu_{ff}}{4}\right) \lVert \mathbb{E} X_1 - \alpha_1 \Sigma_{ff} \rVert + M_f \gamma_{n} .
        % \gamma_t M_f (1 + \lVert \Sigma_{ff} \rVert) + \frac{M_f}{n} \lVert \Sigma_{ff} \rVert + o(n^{-1}) ,
    \end{align*}
    In particular,
    \begin{equation}    
        \mathbb{E} \lVert \tilde{x}_n \rVert^2 
        = \alpha_n \mathrm{Tr} \Sigma_{ff} + \mathcal{O}\left( \gamma_n \right) . 
    \end{equation}
\end{lemma}


\textbf{Proof:} 
We start by evaluating the expectation of $\tilde{x}_t \tilde{x}_t^T$ from \eqref{eq:fast_recursion} conditioned on $\mathcal{H}_t$:
\begin{align*}
    \mathbb{E} [\tilde{x}_{t+1} \tilde{x}_{t+1} | \history] 
    &= \tilde{x}_t \tilde{x}_t^T - \alpha_t \left(B_t^{ff} \tilde{x}_t \tilde{x}_t^T + \tilde{x}_t \tilde{x}_t^T (B_t^{ff})^T - \alpha_t \Gamma_{ff} \right) 
    \\ & 
    + \gamma_t^2 \delta_t \Gamma_{ss} \delta_t^T + \alpha_t \gamma_t (\Gamma_{fs} \delta_t^T + \delta_t \Gamma_{sf}) .
\end{align*}
Taking the unconditional expectations, we then obtain
\begin{align*}
    X_{t+1} = X_t - \alpha_t (B_t^{ff} X_t + X_t (B_t^{ff})^T - \alpha_t \Gamma_{ff}) + \gamma_t^2 \delta_t \Gamma_{ss} \delta_t^T + \alpha_t \gamma_t (\Gamma_{fs} \delta_t^T + \delta_t \Gamma_{sf})
\end{align*}
Using the expression for $B_t^{ff}$ in Eq. \eqref{eq:Bt_ff} into the preceeding equation gives 
\begin{align*}
    X_{t+1} &= X_t - \alpha_t (A_{ff} X_t^T + X_t A_{ff}^T - \alpha_t \Gamma_{ff}),
    % \\ & 
    + \gamma_t R_t^{ff} (X_t) 
    \\
    R_t^{ff} (X_t) &\coloneqq 
    - \left(\delta_t A_{sf} X_t + X_t A_{sf}^T \delta_t^T \right)
    +\left(
    \gamma_t \delta_t \Gamma_{ss} \delta_t^T
    + 
     \alpha_t \Gamma_{fs} \delta_t^T 
     + 
     \alpha_t \delta_t \Gamma_{sf}\right),
\end{align*}
which by using the expression in Eq. \eqref{eq:fast_covariance} for $\Gamma_{ff}$ yields
\begin{align*}
    X_{t+1} - \alpha_{t+1} \Sigma_{ff} 
    &= X_t - \alpha_t \Sigma_{ff} 
    - \alpha_t \left(A_{ff} (X_t - \alpha_t \Sigma_{ff}) + (X_t - \alpha_t \Sigma_{ff}) A_{ff}^T\right) 
    \\
    &
    + (\alpha_{t} - \alpha_{t+1}) \Sigma_{ff}
    + \gamma_t R_t^{ff} (X_t) 
    .
    \numberthis \label{eq:fast_equality} 
\end{align*}
Define $M_f'$ to be a constant such that
\begin{align*}
    \lVert R_t^{ff} (X_t) \rVert \leq M_f' (\lVert X_t\rVert + \alpha_t ) 
    \leq M_f' \lVert X_t - \alpha_t \Sigma_{ff} \rVert +M_f' \alpha_t (1 + \lVert \Sigma_{ff}\rVert)
    .
\end{align*}
Such a constant exists because $X_t$ is positive definite and $\lVert \delta_t \rVert$ in \eqref{eq:delta} is uniformly bounded by Proposition \ref{prop:Lt_welldefined}. 
% Using 
% $X_t + \alpha_t I = (X_t - \alpha_t \Sigma_{ff}) + \alpha_t (\Sigma_{ff} + I)$, 
The choice of step sizes ensures Eq. \eqref{eq:contraction} is satisfied for the symmetric matrix $X_t - \alpha_t \Sigma_{ff}$.
By the property $\alpha_t - \alpha_{t+1} \leq t^{-1} \alpha_{t+1}$, 
\begin{align*}
    \lVert X_{t+1} - \alpha_{t+1} \Sigma_{ff}\rVert 
    & \leq \left(1 - \alpha_t \frac{\mu_{ff}}{2} + \gamma_t M_f'\right) \lVert X_t - \alpha_t \Sigma_{ff} \rVert 
    + \frac{\alpha_{t+1}}{t} \lVert \Sigma_{ff} \rVert
    \\ & 
    + \alpha_t \gamma_t  M_f' (\lVert \Sigma_{ff}  \rVert + 1)
    \\ 
    & 
    \leq 
    \left(1 - \alpha_t \frac{\mu_{ff}}{4}\right) \lVert \tilde{x}_t \tilde{x}_t^T - \alpha_t \Sigma_{ff}\rVert 
    + \frac{\alpha_t}{t} \lVert \Sigma_{ff} \rVert + \alpha_t \gamma_t M_f' (\lVert \Sigma_{ff}  \rVert + 1)
    .
\end{align*}
The above equation is of the form
\begin{equation}
    \lVert X_{t+1} - \alpha_{t+1} \Sigma_{ff} \rVert \leq \left(1 - \alpha_t \frac{\mu_{ff}}{4}\right) \lVert X_t - \alpha_t \Sigma_{ff} \rVert
    + M_f'' \alpha_t \gamma_t ,
\end{equation}
where $M_f''$ subsumes the coefficients. 
By induction and Eq. \eqref{eq:induction_size_equation}, we have that for some constant $M_f > 0$, it holds that
\begin{align*}
    \lVert X_{n+1} - \alpha_{n+1} \Sigma_{ff} \rVert 
    % \\
    & \leq 
    \prod_{t=1}^n \left(1 - \alpha_t \frac{\mu_{ff}}{4}\right) \lVert X_1 - \alpha_1 \Sigma_{ff}\rVert 
    + M_f'' \sum_{t=1}^n \alpha_t \gamma_t \prod_{j=t+1}^n \left(1 - \alpha_j \frac{\mu_{ff}}{4}\right) 
    % + \sum_{t=1}^n \frac{1}{t} \left(\alpha_{t}\prod_{k=t+1}^n \left(1 - \alpha_k \frac{\mu_{ff}}{4}\right) \right)\lVert \Sigma_{ff} \rVert
    % \\ &+  M_f' \sum_{t=1}^n \gamma_t \left(\alpha_t \prod_{k=t+1}^n \left(1 - \alpha_k \frac{\mu_{ff}}{4}\right) \right)  (1 + \lVert \Sigma_{ff}\rVert) 
    \\
    & \leq 
    \prod_{t=1}^n \left(1 - \alpha_t \frac{\mu_{ff}}{4}\right) \lVert X_1 - \alpha_1 \Sigma_{ff} \rVert + M_f \gamma_{n}
    % \left(\frac{1}{n}\lVert \Sigma_{ff} \rVert + \gamma_n (1 + \lVert \Sigma_{ff} \rVert )\right) 
    .
% \end{align*}
% From the choice of step size and using $1 - x \leq \exp(-x)$ for all $x > 0$, the above simplifies to
% \begin{align*}
    % \lVert X_{n+1} - \alpha_{n+1} \Sigma_{ff} \rVert 
    % &\preceq 
    % \\
    % &\leq
    % % \exp \left(-\frac{\mu_{ff}}{4(1-a)}n^{1-a}\right) \lVert X_1 - \alpha_1 \Sigma_{ff}\rVert
    % \prod_{t=1}^n \left(1 - \alpha_t \frac{\mu_{ff}}{4}\right) \lVert X_1 - \alpha_1 \Sigma_{ff}\rVert 
    % + \alpha_1 \sum_{t=1}^n t^{-2}  \lVert \Sigma_{ff} \rVert
    % + \alpha_1 \sum_{t=1}^n \frac{\gamma_t}{t} M_f'(1 +\lVert \Sigma_{ff}\rVert)
    % \\&
    % \leq
    % % \exp \left(-\frac{\mu_{ff}}{4(1-a)}t^{1-a}\right) \lVert X_1 - \alpha_1 \Sigma_{ff}\rVert 
    % \prod_{t=1}^n \left(1 - \alpha_t \frac{\mu_{ff}}{4}\right) \lVert X_1 - \alpha_1 \Sigma_{ff}\rVert 
    % + \frac{2 \alpha_1}{n} \lVert \Sigma_{ff}\rVert
    % + \frac{2 \alpha_1}{b} \gamma_n M_f'  (1 + \lVert \Sigma_{ff}\rVert   ).
\end{align*}
% Lemma 14 in \citep{kaledin2020finite} states that the second and third terms are $\mathcal{O}(1/t)$ and $\mathcal{O}(\gamma_t)$. 
% Setting $M_f = 2 \alpha_1 \max\{1, M_f'/b\}$ completes the proof. 

% As a result, we have that 
% \begin{equation}
%     X_{t+1} - \alpha_{t+1} \Sigma_{ff} \asymp \frac{2\alpha_1}{t} \lVert \Sigma_{ff} \rVert
% \end{equation}

% For the other direction, we start from \eqref{eq:fast_equality} and use $(\alpha_t - \alpha_{t+1}) \Sigma_{ff} \succ 0$ to get
% \begin{align*}
%     X_{t+1} - \alpha_{t+1} \Sigma_{ff} 
%     & \succeq (1 - \alpha_t \nu_{ff} - \gamma_t M_f'') (X_t - \alpha_t \Sigma_{ff}) - \alpha_t \gamma_t M_f''  (I + \Sigma_{ff})
%     \\ 
%     & \succeq \left(1 - 2 \alpha_t \nu_{ff}\right) (X_t - \alpha_t \Sigma_{ff})
%     - \alpha_t \gamma_t M_f'' (I + \Sigma_{ff})  .
% \end{align*}
% And then by induction, 
% \begin{align*}
%     X_{t+1} - \alpha_t \Sigma_{ff} \succeq \prod_{j=1}^t (1 - 2 \alpha_j \nu_{ff}) (X_1 - \alpha_1 \Sigma_{ff}) - \sum_{j=1}^t \gamma_j \left(\alpha_j \prod_{k=j+1}^t (1 - 2 \alpha_k \nu_{ff})\right) M_f'' (I + \Sigma_{ff}) .
% \end{align*}
% The first part's coefficient decays exponentially fast, and we have by $I + \Sigma_{ff} \succ 0$ and Proposition \ref{prop:induction_size} that
% \begin{align*}
%     \sum_{j=1}^t \gamma_j \left(\alpha_j \prod_{k=j+1}^t (1 - 2 \alpha_k \nu_{ff})\right) (I + \Sigma_{ff}) 
%     \preceq 
%     \frac{2 \alpha_1}{b} \gamma_t  (I + \Sigma_{ff}) .
% \end{align*}
% from which we conclude 
% \begin{equation}
%     X_{t+1} - \alpha_t \Sigma_{ff} \succeq - \frac{2 M_f'' \alpha_1}{b} \gamma_t (I + \Sigma_{ff}) - o(\gamma_t).
% \end{equation}

\subsection{Joint-Time-Scale Mean Square Convergence}\label{sec:joint_mse}
% {\color{red}Check if the proof holds for martingales; also, I think the result has $\gamma_n$. not $\gamma_{n+1}$.}
The analysis of the slow iterate's covariance $\mathbb{E} \hat{y}_t \hat{y}_t^T$ involves an error term that depends on the deviation of cross covariance $C_t = \mathbb{E} [\tilde{x}_t \hat{y}_t^T]$ from $\gamma_t \Sigma_{fs}$. 
% Recall the expressions for $B_t^{sf}$ and $B_t^{ss}$ from Eq. \eqref{eq:slow_derivation}:
% \begin{align*}
%     B_t^{sf} = A_{sf}, \quad 
%     B_t^{ss} = \Delta - A_{sf} L_t .
% \end{align*}
% Substituting for the stochastic updates, we have the recursions
Recall the recursions in Eq. \eqref{eq:fast_slow_updates} derived in Section \ref{sec:recursion_mse}, restated for reference below:
\begin{align*}
    \tilde{x}_{t+1} &= (I - \alpha_t A_{ff}) \tilde{x}_t - \gamma_t \delta_t A_{sf} \tilde{x}_t + \alpha_t W_t + \gamma_t \delta_t V_t \\ 
    \hat{y}_{t+1} &= (I - \gamma_t \Delta) \hat{y}_t + \gamma_t A_{sf} L_t \hat{y}_t - \gamma_t A_{sf} \tilde{x}_t + \gamma_t V_t .
\end{align*}
This section is dedicated to proving the following intermediate result in Theorem \ref{thm:mse}:
\begin{lemma}\label{lem:joint_mse}
    Let $\Sigma_{fs}$ 
    be the unique solution to
    \begin{equation}\label{eq:joint_covariance}
        A_{ff} \Sigma_{fs} + \Sigma_{ff} A_{sf}^T = \Gamma_{fs} .
    \end{equation}
    Under Assumptions \ref{assumption:first}--\ref{assumption:last}, 
    there is a problem-dependent constant $M_{fs} > 0$ such that
    \begin{equation}
        \lVert C_{n+1} - \gamma_{n+1} \Sigma_{fs} \rVert 
        \leq 
    \prod_{t=1}^n \left(1 - \alpha_t \frac{\mu_{ff}}{4}\right) \lVert C_1 - \gamma_1 \Sigma_{fs} \rVert         
        + M_{fs} \frac{\gamma_{n}^2}{\alpha_n}
    %     \left(
    %         \frac{\gamma_n}{n \alpha_n} 
    %         + \frac{\gamma_n^2}{\alpha_n} 
    %         + \alpha_n \gamma_n
    %     \right) .   
    \end{equation}
    %     In particular, 
    % \begin{equation}
    %     \lVert C_{t+1} \rVert \leq \gamma_{t+1} \lVert \Sigma_{fs} \rVert + \mathcal{O}\left(\frac{\gamma_t^2}{\alpha_t}\right) .
    % \end{equation}
\end{lemma}

\textbf{Proof:}
Using the expressions in \eqref{eq:fast_slow_updates}, we obtain that $C_{t+1} = \mathbb{E} \tilde{x}_{t+1} \hat{y}_{t+1}^T$ is given by
\begin{align*}
    C_{t+1} &= (I - {\color{violet}\alpha_t A_{ff}}) C_t 
    - \gamma_t C_t \Delta^T
    + \alpha_t \gamma_t A_{ff} C_t \Delta^T
    + \gamma_t \left(C_t L_t^T A_{sf}^T - {\color{violet}X_t A_{sf}^T} \right)     
    \\ 
    &
    - \alpha_t \gamma_t A_{ff}     \left(C_t L_t^T A_{sf}^T - X_t A_{sf}^T \right) 
    - \gamma_t \delta_t A_{sf} C_t
    + \gamma_t^2 \delta_t A_{sf} \left(C_t \Delta^T + C_t L_t^T A_{sf}^T - X_t A_{sf}^T\right) 
    \\ &
    - \alpha_t {\color{violet}\gamma_t \Gamma_{fs}} + \gamma_t^2 \delta_t \Gamma_{ss} .
\end{align*}
Grouping terms by their rates (i.e., order-dependency on $t$ as determined by the step sizes),
% and writing in decreasing order of magnitude,
\begin{align*}
    C_{t+1} 
    &= C_t - \gamma_t \left( \frac{\alpha_t}{\gamma_t} A_{ff} C_t + X_t A_{sf}^T - \alpha_t \Gamma_{fs}
    \right)
    - \gamma_t \delta_t \left( A_{sf} C_t + C_t \Delta^T - \gamma_t \Gamma_{ss}\right) 
    \\ &
    + \alpha_t \gamma_t A_{ff} X_t A_{sf}^T
    % \\ &
    +
    \gamma_t \left(\alpha_t A_{ff} C_t \Delta^T - \gamma_t \delta_t A_{sf} X_t A_{sf}^T\right) 
    \\ &
    - \gamma_t \left(\alpha_t A_{ff} C_t L_t^T A_{sf}^T - \gamma_t \delta_t A_{sf} C_t \Delta^T\right) 
    % \\ &
    + \gamma_t C_t L_t^T A_{sf}^T 
    + \gamma_t^2 \delta_t A_{sf} C_t L_t^T A_{sf}^T 
    \\ & 
    = C_t - \gamma_t \left( \frac{\alpha_t}{\gamma_t} A_{ff} C_t + X_t A_{sf}^T - \alpha_t \Gamma_{fs}
    \right)
    + \gamma_t R_t^{fs} (C_t) ,
\end{align*}
where $R_t^{fs} (C_t)$ subsumes all remaining terms.
Substituting for $\Gamma_{fs}$ using Eq. \eqref{eq:covariances},
\begin{align*}
    C_{t+1} &= \left(I - \alpha_t A_{ff} \right) C_t - \gamma_t \left(X_t A_{sf}^T - \alpha_t \Gamma_{fs}\right)
    + \gamma_t R_t^{fs} (C_t) 
    \\ &
    % = (I - \alpha_t A_{ff}) C_t - \gamma_t\left( (X_t - \alpha_t \Sigma_{ff}) A_{sf}^T + \alpha_t \Sigma_{ff} A_{sf}^T - \alpha_t \Gamma_{fs}\right)
    % + \gamma_t R_t^{fs} (C_t) 
    % \\ & 
    = (I - \alpha_t A_{ff}) C_t - \gamma_t \left(
        (X_t - \alpha_t \Sigma_{ff}) A_{sf}^T - \alpha_t A_{ff} \Sigma_{fs}
    \right) + \gamma_t R_t^{fs} (C_t) 
    \\ &
    = C_t - \alpha_t A_{ff}( C_t - \gamma_t \Sigma_{fs}) 
    - \gamma_t \left(X_t - \alpha_t \Sigma_{ff}\right) A_{sf}^T + \gamma_t R_t^{fs} (C_t) 
    . \numberthis \label{eq:joint_recursion}
\end{align*}
% where $G_t (C_t)$ subsumes all the remainder terms. 
Subtracting $\gamma_{t+1} \Sigma_{fs}$ on both sides, 
\begin{align*}
    &C_{t+1} - \gamma_{t+1} \Sigma_{fs}
    % &\preceq 
    \\ &=
    (I - \alpha_t A_{ff}) (C_t - \gamma_t \Sigma_{fs}) + (\gamma_t - \gamma_{t+1}) \Sigma_{fs}
    % \\ & 
    - \gamma_t (X_t - \alpha_t \Sigma_{ff} )A_{sf}^T + \gamma_t R_t^{fs} (C_t) .
\end{align*}
Using Lemma \ref{lem:fast_mse} to get $\lVert X_t - \alpha_t \Sigma_{ff}\rVert = \mathcal{O}(\gamma_t)$, and defining a constant $M_{fs}'$ such that
\begin{align*}
    \lVert R_t^{fs} (C_t) \rVert \leq M_{fs}' 
    \left(\lVert C_t - \gamma_t \Sigma_{fs} \rVert + \alpha_t \lVert X_t \rVert + \gamma_t \lVert \Sigma_{fs} \rVert \right)
    % \left(\lVert C_t \rVert + \alpha_t \lVert X_t \rVert\right) ,
    % g_t (C_t) \leq \mathrm{Tr} C_t A_{sf} + \alpha_t \mathrm{Tr} X_t .
\end{align*}
which exists by sub-multiplicativity of operator norms, we then have for $\alpha_t \mu_{ff}/2 - \gamma_t M_{fs}' \geq \mu_{ff}/4$
\begin{align*}
    \lVert C_{t+1} - \gamma_{t+1} \Sigma_{fs}\rVert 
    &\leq 
    \left(1 - \alpha_t \frac{\mu_{ff}}{2} + \gamma_t M_{fs}'\right) \lVert C_t - \gamma_t \Sigma_{fs} \rVert
    + \frac{\gamma_t}{t} \lVert \Sigma_{fs} \rVert 
    \\ & + \gamma_t \lVert X_t - \alpha_t \Sigma_{ff}\rVert \lVert A_{sf}\rVert + \alpha_t \gamma_t M_{fs} \lVert X_t \rVert.
    \\ &    
    \leq \left(1 - \alpha_t \frac{\mu_{ff}}{4}\right) \lVert C_t - \gamma_t \Sigma_{fs}\rVert 
    + \frac{\gamma_t}{t} \lVert \Sigma_{fs} \rVert 
    + \gamma_t^2 M_f'' \lVert A_{sf}\rVert + \alpha_t^2 \gamma_t M_f'' \lVert \Sigma_{ff}\rVert 
\end{align*}
for some constant $M_f''$.
Since $b < 1$ and $\alpha_t^2 \leq \gamma_t$, $\gamma_t^2$ dominates and the above is a recursion of the form
\begin{equation}
    \lVert C_{t+1} - \gamma_{t+1} \Sigma_{fs} \rVert 
    \leq \left(1 - \alpha_t \frac{\mu_{ff}}{4}\right) \lVert C_t - \gamma_t \Sigma_{fs} \rVert + M_{fs}' \gamma_t^2
\end{equation}
for some constant $M_{fs}' > 0$. 
By induction and Eq. \eqref{eq:induction_size_equation}, we have that for a constant $M_{fs} > 0$,
\begin{align*}
    \lVert C_{n+1} - \gamma_{n+1} \Sigma_{fs} \rVert 
    & \leq 
    \prod_{t=1}^n \left(1 - \alpha_t \frac{\mu_{ff}}{4}\right) \lVert C_1 - \gamma_1 \Sigma_{fs} \rVert 
        + M_{fs}' \sum_{t=1}^n \gamma_t^2 \prod_{j=t+1}^n \left(1 - \alpha_t \frac{\mu_{ff}}{4}\right) 
    \\
    &\leq 
    \prod_{t=1}^n \left(1 - \alpha_t \frac{\mu_{ff}}{4}\right) \lVert C_1 - \gamma_1 \Sigma_{fs} \rVert 
    + M_{fs} \frac{\gamma_{n}^2}{\alpha_n} .
\end{align*}


% Using Eq. \eqref{eq:initial_steps} for $\alpha_t$ and $\gamma_t/\alpha_t$,
% \begin{align*}
%     \lVert C_{t+1} \rVert 
%     &\leq \left(1 - \alpha_t \frac{\mu_{ff}}{2} + \gamma_t M_g\right) \lVert C_t \rVert 
%     + \alpha_t \gamma_t \lVert A_{ff} \Sigma_{fs} \rVert 
%     + \gamma_t \lVert (X_t - \alpha_t \Sigma_{ff}) A_{sf}^T \rVert 
%     + \alpha_t \gamma_t M_g \lVert X_t \rVert 
%     \\ & 
%     \leq 
%     \left(1 - \alpha_t \frac{\mu_{ff}}{4}\right) \lVert C_t \rVert 
%     % + \gamma_t \lVert X_t A_{sf}^T - \alpha_t \Gamma_{fs}\rVert 
%     + \alpha_t \gamma_t \lVert A_{ff} \Sigma_{fs} \rVert 
%     + \gamma_t \lVert (X_t - \alpha_t \Sigma_{ff}) A_{sf}^T \rVert     
%     + \alpha_t \gamma_t M_g \lVert X_t \rVert 
% \end{align*}
% From Lemma \ref{lem:fast_mse} {\color{red}Make sure Lemma contains the new general bound}, we have that $X_t - \alpha_t \Sigma_{ff} = \mathcal{O}(\gamma_t) I$. 
% Define $M_f' > 0$ so that $\lVert X_t - \alpha_t \Sigma_{ff} \rVert \leq M_f' \gamma_t$.
% \begin{align*}
%     \lVert C_{t+1} \rVert 
%     \leq 
%     \left(1 - \alpha_t \frac{\mu_{ff}}{4}\right) \lVert C_t \rVert 
%     + \alpha_t \gamma_t \lVert A_{ff} \Sigma_{fs} \rVert 
%     + \gamma_t^2 M_f' \lVert A_{sf}^T \rVert     
%     + \alpha_t \gamma_t M_g (\alpha_t \Sigma_{ff} + \gamma_t M_f' I)  .
% \end{align*}
% By induction and again applying Prop. \ref{prop:induction_size}, we then have
% \begin{align*}
%     \lVert C_{t+1} \rVert \leq \exp \left(- \frac{\mu_{ff}}{4 (1-a)} t^{1-a}\right) \lVert C_1 \rVert 
%     + \frac{2 \alpha_1}{b} \gamma_t \lVert A_{ff} \Sigma_{fs} \rVert 
%     + \mathcal{O}\left(\frac{\gamma_t^2}{\alpha_t} + \alpha_t \gamma_t + \gamma_t^2 \right) .
% \end{align*}

% {\color{red}I need to go back and show that $\mathrm{Tr} A_{sf} (C_t - \gamma_t \Sigma_{fs}) = o(\gamma_t)$.
% A operator norm on this could suffice, but still I need this difference's bound. 
% }


\textbf{Lower Bound: {\color{red}Make this compatible with my current technique.}} Holds for the same choice of step size. {\color{red}Refine below:}
We start from Eq. \eqref{eq:slow_identity}: Using that $h_t (Y_t) \leq M_h \mathrm{Tr} Y_t$ and that there exists a $\Sigma_t^{ss}$ that satisfies
\begin{align*}
    \mathrm{Tr}\left( \Delta Y_t + Y_t \Delta^T - (C_t A_{sf}^T + A_{sf} C^T + \gamma \Gamma)\right)
    \leq \nu_\Delta \mathrm{Tr} - \gamma_t \mu_\Delta \mathrm{Tr} \Sigma^t_{ss} 
\end{align*}
for every $t$.
Therefore,
\begin{align*}
    \mathrm{Tr}Y_{t+1} &= \mathrm{Tr} Y_t - \gamma_t \mathrm{Tr}\left(\Delta Y_t 
    + Y_t \Delta^T - (A_{sf} C_t + C_t^T A_{sf}^T) - \gamma_t \Gamma_{ss}  \right)
    + \frac{\gamma_t^2}{\alpha_t} h_t (Y_t) ,   
    \\ &
    \geq \left(1 - \gamma_t \nu_\Delta + \frac{\gamma_t^2}{\alpha_t} M_h \right) \mathrm{Tr} Y_t
    + \gamma_t \mu_\Delta \mathrm{Tr} \Sigma^{ss}_t 
    -  \gamma_t^2 \lvert M_h \left(\mathrm{Tr} A_{sf} C_t + \mathrm{Tr} X_t\right) \rvert .
\end{align*}
Again we assume WLOG that $\gamma_t \nu_\Delta - \frac{\gamma_t^2}{\alpha_t} M_h \geq \gamma_t \nu_\Delta / 2$ and use that $\mathrm{Tr} A_{sf} C_t + \mathrm{Tr} X_t = \mathcal{O}(\gamma_t + \alpha_t)$ to obtain
\begin{align*}
    \mathrm{Tr} Y_{t+1} \geq \left(1 - \gamma_t \frac{\nu_\Delta}{2}\right) \mathrm{Tr} Y_t + \gamma_t \mu_\Delta \mathrm{Tr} \Sigma_{ss}
    - \gamma_t \mu_\Delta \mathrm{Tr}(\Sigma_t^{ss} -  \Sigma_{ss}) - \gamma_t^2 \lvert M_h (\gamma_t C_\infty + \alpha_t X_\infty) \rvert ,
\end{align*}
where $C_\infty, X_\infty$ are uniform bounds on $\gamma_t \mathrm{Tr} C_t, \alpha_t \mathrm{Tr} X_t$.
{\color{red}If the above is correct, then we should get a lower bound that matches from this. }



\section{Non-Asymptotic CLT for TSA-PR}\label{sec:clt}
We establish finite-time bounds on the Wasserstein-1 distance defined in Eq. \eqref{eq:wasserstein_definition} between the deviation of the Polyak-Ruppert average $(\bar{x}_n, \bar{y}_n)$ around the solution (zeros) and the limiting Gaussian vector.
Here we restate Theorem \ref{thm:clt} for reference.
\begin{theorem}
    \CLT
\end{theorem}
The asymptotic covariances $\bar{\Sigma}_{ff}$ and $\bar{\Sigma}_{ss}$ satisfy
\begin{equation}
    \begin{split}
        G \bar{\Sigma}_{ff} G^T
        &= \Gamma_{ff} + A_{fs} A_{ss}^{-1} \Gamma_{ss} (A_{fs} A_{ss}^{-1})^T 
        - \left(\Gamma_{fs} (A_{fs} A_{ss}^{-1})^T + A_{fs} A_{ss}^{-1} \Gamma_{sf}\right)  \\
        \Delta \bar{\Sigma}_{ss} \Delta^T
        &= \Gamma_{ss} + A_{sf} A_{ff}^{-1} \Gamma_{ff} (A_{sf} A_{ff}^{-1})^T
        - \left(\Gamma_{sf} (A_{sf} A_{ff}^{-1})^T + A_{sf} A_{ff}^{-1} \Gamma_{fs}\right)         
    \end{split} . \label{eq:ttsapr_covariance}
\end{equation}


\subsection{Proof of Theorem \ref{thm:clt}}
\textbf{Step 1:} We write the recursions \eqref{eq:ttsa} in the form
% We follow \citep{mokkadem2006convergence} and write the fast and slow timescale recursions as
\begin{align*}
    G x_t = (W_t - A_{fs} A_{ss}^{-1} V_t) + \alpha_t^{-1} (x_t - x_{t+1}) - \gamma_t^{-1} A_{fs} A_{ss}^{-1} (y_t - y_{t+1}) \numberthis \label{eq:fast_last} \\ 
    \Delta y_t = (V_t - A_{sf} A_{ff}^{-1} W_t) + \gamma_t^{-1} (y_t - y_{t+1}) - A_{sf} A_{ff}^{-1} \alpha_t^{-1} (x_{t+1} - x_t) ,
\end{align*}
where $G, \Delta$ are the Schur complements
\begin{align*}
    G = A_{ff} - A_{fs} A_{ss}^{-1} A_{sf} , \quad
    \Delta = A_{ss} - A_{sf} A_{ff}^{-1} A_{fs} .
\end{align*}
\begin{proof}    
    Let us start with the fast-time-scale variable update
    \begin{align*}
        x_{t+1} &= x_t - \alpha_t (A_{ff} x_t + A_{fs} y_t - W_t) 
        \\
        \Rightarrow x_t &= (\alpha_t A_{ff})^{-1} (x_{t+1} - x_t) - A_{ff}^{-1} (A_{fs} y_t - W_t) .
    \end{align*}
    Substituting into the slow-time-scale variable's recursion,
    \begin{align*}
        y_{t+1} &= y_t - \gamma_t \left(
            A_{sf} x_t + A_{ss} y_t - V_t 
        \right)
        \\ 
        &= y_t - \gamma_t A_{sf} \left( A_{ff}^{-1} \alpha_t^{-1} (x_{t+1} - x_t) \right) + \gamma_t A_{sf} A_{ff}^{-1} A_{fs} y_t - \gamma_t A_{sf} A_{ff}^{-1} W_t
        - \gamma_t A_{ss} y_t + \gamma_t V_t .
    \end{align*}
    Using $\Delta = A_{ss} - A_{sf} A_{ff}^{-1} A_{fs}$, we have 
    \begin{align*}
         y_{t+1} &= (I - \gamma_t \Delta) y_t - \frac{\gamma_t}{\alpha_t} A_{sf} A_{ff}^{-1} (x_{t+1} - x_t) + \gamma_t (V_t - A_{sf} A_{ff}^{-1} W_t)  \numberthis \label{eq:prelim}\\
        \Leftrightarrow 
         \gamma_t \Delta y_t &=  y_t - y_{t+1} - \frac{\gamma_t}{\alpha_t} A_{sf} A_{ff}^{-1} (x_{t+1} - x_t) + \gamma_t (V_t - A_{sf} A_{ff}^{-1} W_t).
    \end{align*}
    Dividing both sides by the step size $\gamma_t$, we have
    \begin{equation}
        \Delta y_t = \gamma_t^{-1} (y_t - y_{t+1}) - \alpha_t^{-1} A_{sf} A_{ff}^{-1} (x_{t+1} - x_t) +(V_t - A_{sf} A_{ff}^{-1} W_t) .
    \end{equation}

    We repeat the same steps for the fast iterate:
    Using that 
    \begin{align*}
        A_{ss} y_t = \gamma_t^{-1} (y_t - y_{t+1}) - (A_{sf} x_t - V_t) ,
    \end{align*}
    we have by substitution
    \begin{align*}
        x_{t+1} &= x_t - \alpha_t (A_{ff} x_t - W_t) - \alpha_t A_{fs} y_t 
        \\
        &= x_t - \alpha_t A_{ff} x_t + \alpha_t W_t - \alpha_t A_{fs} A_{ss}^{-1}\left(\gamma_t^{-1} (y_t - y_{t+1}) - (A_{sf} x_t - V_t) \right) 
        \\ & = 
        x_t - \alpha_t \left(A_{ff} - A_{fs} A_{ss}^{-1} A_{sf} \right) x_t + \alpha_t (W_t - A_{fs} A_{ss}^{-1} V_t) - \frac{\alpha_t}{\gamma_t} A_{fs} A_{ss}^{-1} (y_t - y_{t+1}) .
    \end{align*}
    Denoting $G = A_{ff} - A_{fs} A_{ss}^{-1} A_{sf}$, we have 
    \begin{equation}
        G x_t = \alpha_t^{-1} (x_t - x_{t+1}) + (W_t - A_{fs} A_{ss}^{-1} V_t) - \gamma_t^{-1} A_{fs} A_{ss}^{-1} (y_t - y_{t+1}) .
    \end{equation}
\end{proof}
The Polyak-Ruppert averages $\bar{x}_n, \bar{y}_t$ are obtained by taking the average on both sides:
\begin{align*}
    G \bar{x}_n &= \frac{1}{n} \sum_{t=1}^n (W_t - A_{fs} A_{ss}^{-1} V_t) + \frac{1}{n} \sum_{t=1}^n \alpha_t^{-1} (x_{t} - x_{t+1}) - \frac{1}{n} \sum_{t=1}^n \gamma_t^{-1} A_{fs} A_{ss}^{-1} (y_t - y_{t+1}) \\ 
    \Delta \bar{y}_n &= \frac{1}{n} \sum_{t=1}^n (V_t - A_{sf} A_{ff}^{-1} W_t) 
    + \frac{1}{n}\sum_{t=1}^n A_{sf} A_{ff}^{-1} \alpha_t^{-1} (x_{t} - x_{t+1})
    + \frac{1}{n}\sum_{t=1}^n \gamma_t^{-1} (y_{t} - y_{t+1})  .    
\end{align*}


\textbf{Step 2:} Quantitative bounds on the central limit theorem. 

Asymptotic convergence of $\sqrt{n} (\bar{x}_n, \bar{y}_n)$ to a normal distribution can then be proved by showing that (1) the first noise terms above satisfy Lindeberg's condition and that (2) the remaining terms converge to zero with probability 1.
Slutsky's theorem then implies that the sum $N_t + E_t \to N + c$ when the noise $N_t$ converges in distribution to a random variable $N$ and the error terms $E_t$ converge to a constant $c$ with probability 1. 



For non-asymptotic analysis, we use the mean-square results for the last iterates to obtain finite-time bounds on the telescoped weighted differences above, where they converge in probability to 0.
Next, we use a bound on the Wasserstein-1 distance between the first terms and a normal random variable. 
These results can be combined using Lindeberg's decomposition to obtain finite-time bounds on the distance between $\bar{x}_n$ and $\bar{y}_n$ to a Gaussian random variable. 


Lemma \ref{lem:slutsky} resembles Slutsky's theorem for a sum of two convergent random sequences, but can be used to obtain finite-time bounds on the Wasserstein-1 distance when the random sequences being summed converge with respect to the Wasserstein-1 distance and mean absolute error. 
% \begin{lemma}[Lemma \ref{lem:slutsky}]
% \begin{lemma}[Lemma \ref{lem:slutsky}]
%     Consider two random sequences $\{X_t\}, \{Y_t\}$ such that $d_1 (X_t, X) \leq r_t$ for some random variable $X$ and $\mathbb{E}\lVert Y_t \rVert \leq r'_t$ for some constant $Y$.
%     Then, $d_1 (X_t + Y_t, X) \leq r_t + r'_t$.
% \end{lemma}
\begin{lemma*}
    (Restated Lemma \ref{lem:slutsky})
    Consider two random sequences $\{X_t\}, \{Y_t\}$ such that $d_1 (X_t, X) \leq r_t$ for some random variable $X$ and $\mathbb{E}\lVert Y_t \rVert \leq r'_t$.
    Then, $d_1 (X_t + Y_t, X) \leq r_t + r'_t$.
\end{lemma*}
\begin{proof}
    By the definition \eqref{eq:wasserstein_definition}, we have that
    \begin{align*}
        d_1 (X_t + Y_t, X) &= \sup_{h \in \mathrm{Lip}_1} \mathbb{E}[h(X_t + Y_t) - h(X)] 
        \\ 
        &= \mathbb{E}\left[h(X_t + Y_t) - h(X_t)\right] + \mathbb{E}h(X_t) - \mathbb{E} h(X) 
        \\ & \leq 
        \mathbb{E}\lVert Y_t \rVert + d_1 (X_t, X) 
        ,
    \end{align*}
    where the last inequality uses that $h$ in 1-Lipschitz. 
    This statement demonstrates that the Wasserstein-1 distance between a sum and its limit can be decomposed into mean absolute errors and the Wasserstein-1 distance of a summand. 
\end{proof}
Next, we use a non-asymptotic central limit theorem (Theorem 1, \citet{srikant2024CLT}) for martingale differences, simplified for our application.
\begin{lemma*}
    % {\color{red}Revise the statement in the main draft;}
    (Restated Lemma \ref{lem:CLT})
    Let $\{N_t\}$ be a martingale difference sequence satisfying Assumption \ref{assumption:noise}. 
    Denoting $Z \sim \mathcal{N}(0, I)$ to be the standard Gaussian vector, we have that
    \begin{equation}
        d_1 \left(n^{-1/2} \sum_{t=1}^n N_t, \Gamma^{1/2} Z\right) \leq 
        \mathcal{O}\left(\frac{d}{1-\beta}\right) \frac{\lVert \Gamma^{1/2} \rVert }{n^{\beta / 2}} \left(\lVert \Gamma^{-1/2} \rVert^{2 + \beta} + \lVert \Gamma^{-1/2} \rVert^\beta \right).
    \end{equation}
\end{lemma*}
% \begin{proof}
%     Assumption \ref{assumption:noise} implies that all $\leq 2 + \beta$ moments of $N_t$ are finite. 
%     Therefore, the assumptions in Theorem 1 of \citep{srikant2024CLT} are satisfied.
%     % This is a special case of Theorem 1 in \citep{srikant2024CLT}.     
% \end{proof}
It remains to show that the expected norm of the remaining terms decay to zero.
Observe that 
\begin{align*}
    \sum_{t=1}^n \alpha_t^{-1} (x_t - x_{t+1}) = \frac{x_1}{\alpha_1} - \frac{x_{n+1}}{\alpha_{n}} + \sum_{t=1}^{n-1} (\alpha_{t+1}^{-1} - \alpha_{t}^{-1}) x_t .
\end{align*}
The step sizes in Assumption \ref{assumption:steps} satisfy
\begin{align*}
    \alpha_{t+1}^{-1} - \alpha_{t}^{-1} \leq (t \alpha_t)^{-1} ,
\end{align*}
and we have from triangle inequality that
\begin{equation}\label{eq:fast_telescope}
    \mathbb{E} \lVert \sum_{t=1}^n \alpha_t^{-1} (x_t - x_{t+1}) \rVert \leq \alpha_1^{-1} \mathbb{E}\lVert x_1 \rVert  + \alpha_n^{-1} \mathbb{E}\lVert x_{n+1} \rVert + (\alpha_1)^{-1} \sum_{t=1}^n t^{a - 1} \mathbb{E} \lVert  x_t \rVert .
\end{equation}
Repeating for $\{y_t\}$, we have
\begin{equation}\label{eq:slow_telescope}
    \mathbb{E} \lVert \sum_{t=1}^n \gamma_t^{-1} (y_t - y_{t+1}) \rVert \leq \gamma_1^{-1} \mathbb{E} \lVert y_1 \rVert + \gamma_n^{-1} \mathbb{E} \lVert y_{n+1} \rVert + (\gamma_1)^{-1} \sum_{t=1}^n t^{b - 1} \mathbb{E} \lVert  y_t \rVert .
\end{equation}
We have from Theorem \ref{thm:mse} and Jensen's inequality that
\begin{align*}
    \mathbb{E}\lVert x_{t} \rVert = \mathcal{O}(\sqrt{\alpha_t} + \sqrt{\gamma_t}), 
    \quad 
    \mathbb{E} \lVert y_{t} \rVert = \mathcal{O}(\sqrt{\gamma_t}) .
\end{align*}
Substituting, we obtain
\begin{align*}
    \mathbb{E} \lVert \sum_{t=1}^n \alpha_t^{-1} (x_t - x_{t+1}) \rVert + \mathbb{E}\lVert \sum_{t=1}^n \gamma_t^{-1} (y_t - y_{t+1}) \rVert 
    = \mathcal{O}\left(n^{a/2} + n^{a-b/2} + n^{b/2}\right) .
\end{align*}
Combining with Lemma \ref{lem:CLT} and \ref{lem:slutsky} yields Theorem \ref{thm:clt}.
Similar to how we included the transient term $\sqrt{\gamma_t}$ for $\mathbb{E}\lVert x_t\rVert$ to obtain the $n^{a-b/2}$ in the above equation, a refined inspection of all rates in the proof of Theorem \ref{thm:mse} illustrate how to control the transient terms in the Wasserstein-1 distance for the Polyak-Ruppert averaged errors in Theorem \ref{thm:clt}. 





% \section{Auxiliary Lemmas}
% \subsection{Two Timescale Generalized Single Timescale}
Under Assumption \ref{assumption:structure}, there exists a matrix $A$ that is not Lyapunov stable. 
\begin{align*}
    A_{ff} = \begin{pmatrix}
        2 & 1 \\ 1 & 2
    \end{pmatrix} ,
    A_{fs} = \begin{pmatrix}
        1 & 0
    \end{pmatrix}^T , 
    A_{sf} = A_{fs}^T, 
    A_{ss} = 1 .
\end{align*}
Here we see that the eigenvalues of $A_{ff}$ are $1$ and $3$, so Assumption \ref{assumption:structure} is satisfied. 
But $A + A^T$ has an eigenvalue $4 - \sqrt{6}$, and therefore is not positive definite. 


To see that Assumption \ref{assumption:structure} is more general, it can also be shown that $A + A^T \succ 0$ implies Assumption \ref{assumption:structure}.
{\color{red}This is not true; $A + A^T \succ 0$ implies $A_{ff} + A_{ff}^T \succ 0$ but not $\Delta + \Delta^T \succ 0$.}



% \begin{proposition}
%     Assumption \ref{assumption:structure} {\color{red}implies that} $A_{ss}$ is invertible and that its Schur complement $G$ also satisfies $G + G^T \succ 0$. 
% \end{proposition}
% Let us start with the Schur decomposition of $A$:
% \begin{equation}
%     A = \begin{pmatrix}
%         I & 0 \\ A_{sf} A_{ff}^{-1} & I 
%     \end{pmatrix}
%     \begin{pmatrix}
%         A_{ff} & 0 \\ 0 & \Delta 
%     \end{pmatrix}
%     \begin{pmatrix}
%         I &  A_{ff}^{-1} A_{fs} \\ 0 & I
%     \end{pmatrix} .
% \end{equation}
% To 


% {\color{blue}
% Reason for checking: Is $A + A^T \succ 0$? If so, what is the reason for using TTSA? 
% If in TTSA we can set $a \approx b$, and we know that STSA achieves optimal rate, what is the motivation for using TTSA?
% TTSA would only be beneficial if $A + A^T$ is not necessarily positive definite, meaning that we can apply TTSA to a broader class of problems/update rules. 
% }

% (1) First check that $A_{ss}$ is invertible; if so, we can write the other Schur decomposition and compare the two. 
% {\color{blue}Fix this and the parts that use this formula (in the main draft). 
The changes won't be too big, but I need to make things precise. 
}
\begin{proposition}\label{prop:induction_size}
    Let $\alpha_t = \alpha_1 t^{-a}$ with $a \in (0, 1)$.
    For any $0 < \alpha_1 \mu < 1$ and $n \geq 1$, 
    \begin{equation}
        \sum_{t=1}^n \frac{\alpha_t}{t} \prod_{j=t+1}^n \left(1 - \mu \alpha_j \right) 
            \leq {\color{red}K}\frac{\alpha_n}{n}
        % \frac{1}{1-a} \frac{\alpha_n}{n} \exp \left(-\mu \alpha_n\right) .
        % K_1 \alpha_n,
    \end{equation}
\end{proposition}

\begin{proof}
    This is shown by induction.
    The base case clearly holds, where $\prod_{j=n+1}^n (1 - \mu \alpha_j)$ is defined to be one. 
    Next,
    \begin{align*}
        \sum_{t=1}^n \frac{\alpha_t}{t} \prod_{j=t+1}^n \left(1 - \mu \alpha_j\right) 
        \leq 
        \frac{\alpha_n}{n} + {\color{red}K} \frac{\alpha_{n-1}}{n-1} .
    \end{align*}

    Using that $\alpha_t/\alpha_{t+1} = (1 + 1/t)^a \leq 2$ is a decreasing sequence, we have
    \begin{align*}
        \sum_{t=1}^n \frac{\alpha_t}{t} \prod_{j=t+1}^n \left(1 - \mu \alpha_j \right) 
        &= \alpha_{n+1} \sum_{t=1}^n \frac{1}{t}\frac{\alpha_{n}}{\alpha_{n+1}} \frac{\alpha_t}{\alpha_n} \prod_{j=t+1}^n \left(1 - \mu \alpha_j \right) 
        \\
        & \leq 2 \alpha_{n+1} \sum_{t=1}^n \frac{1}{t} \prod_{j=t+1}^n \frac{\alpha_j}{\alpha_{j+1}} \left(1 - \mu \alpha_j\right) .
    \end{align*}
    % Using Lemma 12 in \citep{kaledin2020finite}, we have that
    % Using $1 - x \leq \exp (-x)$, we have that 
    % \begin{align*}
    %     \prod_{j=t+1}^n (1 - \mu \alpha_j) 
    %     \leq 
    %     \exp \left(-\frac{\mu \alpha_1}{1-a} \left(
    %         n^{1-a} - (n-1)^{1-a}
    %         \right)
    %     \right)
    %     \leq 
    %     \exp \left(-\mu \alpha_{n-1} \right) ,
    % \end{align*}
    % where the last inequality uses $n^{1-a} - (n-1)^{1-a} \geq (1-a) (n-1)^{-a}$. 
    % Therefore,
    % \begin{align*}
    %     \sum_{t=1}^n \frac{\alpha_t}{t} \prod_{j=t+1}^n (1 - \mu \alpha_j) 
    %     \leq \exp \left(-\mu \alpha_{n-1}\right) \sum_{t=1}^n \frac{\alpha_t}{t} 
    %     \leq \frac{1}{1-a}\exp \left(-\mu \alpha_{n-1}\right) (n+1)^{1-a}.
    % \end{align*}
    
    % For $a < 1$, we have by $1 - x \leq \exp (-x)$ for all $x \in [0, 1]$,
    % \begin{align*}
    %     \prod_{j=t+1}^n (1 - \mu \alpha_j) \leq \exp \left(-\mu \alpha_1 (\sum_{j=t}^n j^{-a} - t^{-a})\right) 
    %     & \leq 
    %     \exp \left(-\frac{\mu \alpha_1}{1-a} 
    %         \left(n^{1-a} - t^{1-a} - 1 \right)
    %     + \mu \alpha_1 t^{-a}
    %     \right)
    %     \\ &
    %     = \exp \left(-\frac{\mu \alpha_1}{1-a}  (n^{1-a} - 2) \right) \exp \left( \frac{\mu \alpha_1}{1-a} t^{1-a}\right) .
    % \end{align*}
    % Next,
    % \begin{align*}
    %     \sum_{t=1}^n t^{-a-1} \exp \left(K_0 t^{1 - a} \right)
    %     % \leq K_1 \int_1^n t^{-a-1} \exp \left(K_0 t^{1-a}\right) dt
    %     \leq \left( \sum_{t=1}^n t^{-4}\right)^{1/2} \left(\sum_{t=1}^n t^{2(1 - a)} \exp \left(2 K_0 t^{1- a}\right)\right)^{1/2}
    % \end{align*}
    % {\color{red}First is $t^{-3/2}$, second is roughly $t^{1-a} \exp (K_0 t^{1-a})$ due to $x \exp x$ having an integral similar to a exp without the $x$.
    % Therefore, if the exponentials cancel out then we will have maybe something like $t^{1/2 - a}$ (if the 2 in the $t$ exponent remains, which gives after subtracting by $3/2$ the half); therefore, this may go to zero when $a > 1/2$, and may be where I need the requirement.  
    % But this wouldn't give $o(\alpha_n)$; need to make sure that whatever bound we get, we have a $o(\alpha_n)$, i.e. $t^{-a}$ times a decaying term. 
    % }


    % {\color{red}Case $a = 1$; may not be necessary.}
    % \begin{align*}
    %     \prod_{j=t+1}^n (1 - \mu \alpha_j ) 
    %         = 
    %     \prod_{j=t+1}^n \left(1 - \frac{\mu \alpha_1}{t}\right) 
    %         &\leq
    %     \exp \left(-\mu \alpha_1 (\log n - \log t)\right) \exp\left(\mu \alpha_1 + \alpha_t\right)          
    %     \\
    %     &        \leq
    %     \exp \left((\mu + 1) \alpha_1\right) \left(\frac{t}{n}\right)^{\mu \alpha_1} .
    %     % &
    %     % \leq \exp \left(-K_0 (\log n - \log t)\right)
    %     % = \left(\frac{t}{n}\right)^{K_0} .
    %     % \leq \left(\frac{2}{n}\right)^{\mu \alpha_1} .
    % \end{align*}
    % Therefore, there exists some constant $K_1 > 0$ such that
    % \begin{align*}
    %     \sum_{t=1}^n \frac{\alpha_t}{t} \prod_{j=t+1}^n \left(1 - \mu \alpha_1 \right) 
    %         \leq 
    %     \frac{\alpha_1 e^{(\mu + 1) \alpha_1}}{n^{\mu \alpha_1}} \sum_{t=1}^n t^{-a - 1 + \mu \alpha_1}
    %     \leq K_1 \frac{\alpha_1 e^{(\mu + 1)\alpha_1}}{\mu \alpha_1 - a} n^{-a} .
    % \end{align*}
    % {\color{red}Finish up!
    % Note that I need to strengthen the bound by, instead of using $a=1$ as the extreme case for the above bound, derive (and re-state for $a < 1$ instead of $a \leq 1$) a bound that decays at rate $o(\alpha_n)$. 
    % }


    % {\color{blue}A stronger bound for $a < 1$;}
    % \begin{align*}
    %     \prod_{j=t+1}^n 
    % \end{align*}
\end{proof}



% {\color{red}Lower bound:} Setting $\phi_j = j^{-a} \prod_{k=j+1}^t (1 - c k^{-a})$, we use $\log (1 - x) \geq - (x + x^2/2)$ for $x \in (0, 1)$ to get {\color{red}Check}
% \begin{align*}
%     \log \phi_j = -a \log j + \sum_{k=j+1}^t \log (1 - c k^{-a}) 
%     &\geq - a \log j - \sum_{k=j+1}^t (ck^{-a} + \frac{c^2}{2} k^{-2a}) 
%     \\ & 
%     \geq 
%     -a \log j - \frac{2 t^{1-a}}{1-a} + \frac{2 t^{1-2a}}{1 - 2a}
% \end{align*}


% \begin{proposition}\label{prop:fast_decay_steps}
%     If $\mu I \preceq A + A^T \preceq \nu I$ and $\alpha \leq 2 \mu/\nu^2$,
%     \begin{align*}
%         \lVert I - \alpha A \rVert \leq 1 - \frac{\mu}{2} \alpha .
%     \end{align*}
% \end{proposition}
% \begin{proof}
%     \begin{align*}
%         \lVert I - \alpha A \rVert^2 = \sup_{x \neq 0} \left\{1 - \alpha \frac{x^T (A^T + A) x}{\lVert x \rVert^2} + \alpha^2 \frac{x^T A^T A x}{\lVert x \rVert^2}\right\} 
%         \leq 
%         1 - \alpha \mu + \alpha^2 \frac{\nu^2}{4} ,
%     \end{align*}
%     where the last step uses that $\nu = \lVert A + A^T \rVert = 2 \lVert A \rVert$ to get $\lVert A^T A \rVert \leq \lVert A \rVert^2 \leq \nu^2/4$.
% \end{proof}

% Using $1 - x \leq e^{-x}$ for all $x \in [0, 1]$, $\alpha_k = t^{-a} $, and $\sum_{k=j+1}^t \alpha_k \leq \frac{t^{1-a}}{1-a} - \frac{j^{1-a}}{1-a}$, we have 
% \begin{align*}
%     \prod_{k=j+1}^t (1 - \alpha_k \frac{\mu_{ff}}{2}) 
%         &\leq \exp \left(-\frac{\mu_{ff}}{2(1-a)} \left(t^{-a+1} - (j+1)^{-a+1}\right) \right) .
% \end{align*}
% % Moreover, we use that $t^{-a+1} - (t-1)^{-a + 1} \leq \alpha_t$ {\color{red}Check! This is kind of like binomial expansion; or maybe this is related to the step-size condition?} to obtain that
% % \begin{align*}
% %     \prod_{k=j+1}^t (1 - \alpha_k \frac{\mu_{ff}}{2}) \leq \exp\left(-\mu j^{-a}\right) \leq \frac{1}{j \alpha_j}
% % \end{align*}
% {\color{red}Show below or faster rates; I think this is the correct rate based on expected results:
% Note that equality holds when $\alpha_t \approx t^{-1}$. Maybe show that this quantity is monotonic in the exponent.
% }
% \begin{align*}
%     \phi_{jt} = \alpha_j \prod_{k=j+1}^t \left(1 - \alpha_k \frac{\mu_{ff}}{2}\right) \leq \frac{1}{j} .
% \end{align*}

% 

\subsection{Trace inequality for $g_t$}
{\color{red}Exercise verifying trace inequality.}
\begin{align*}
    g_t (C_t) &= 
    -\mathrm{Tr}\delta_t \left( A_{sf} C_t + \Delta^T \right) A_{sf}
    + 
    \alpha_t \mathrm{Tr} A_{ff} C_t \Delta^T  A_{sf}
    - \alpha_t \mathrm{Tr} A_{ff} C_t L_t^T A_{sf}^T A_{sf}
    \\ & 
    + \gamma_t \mathrm{Tr} \delta_t A_{sf} C_t \Delta^T A_{sf}
    + \mathrm{Tr} C_t L_t^T A_{sf}^T A_{sf}
    + \gamma_t \mathrm{Tr} \delta_t A_{sf} C_t L_t^T A_{sf}^T A_{sf}
    \\ &
    + \alpha_t \mathrm{Tr} \left(A_{ff} X_t A_{sf}^T 
    + \frac{\gamma_t}{\alpha_t} \left(\delta_t \Gamma_{ss} - \delta_t A_{sf} X_t A_{sf}^T \right) A_{sf}\right) .
\end{align*}


We can use the inequality that for arbitrary matrices $A, X$ of appropriate size, 
\begin{align*}
    \lvert \mathrm{Tr} A^T X \rvert \leq \lVert A \rVert \mathrm{Tr} \lvert X \rvert ,
\end{align*}
where $\lvert X \rvert = (X^T X)^{1/2}$ and $\lVert \cdot \rVert$ is the spectral norm.
Then we have that 
\begin{align*}
    \lvert g_t (C_t) \rvert \leq 
    \lVert \delta_t A_{sf} \rVert \mathrm{Tr} \lvert A_t C_t \rvert 
    + 
    \left(
        \alpha_t \lVert \Delta^T A_{sf} \rVert 
        - \alpha_t \lVert L_t^T A_{sf}^T \rVert 
    \right)
    \mathrm{Tr} \lvert A_{sf} A_{ff} C_t \rvert
    + \gamma_t \lVert \delta_t + \cdots .
\end{align*}
For the terms where there are more matrices inside, I think we can pull them out. 


Q. 
Can we instead write the recursion in terms of $\mathrm{Tr} \lvert C_t A_{sf} \rvert$?
Then, prove that 
\begin{align*}
    \mathrm{Tr} C_t A_{sf} \leq \mathrm{Tr} \lvert C_t A_{sf} \rvert .
\end{align*}
This is true because
\begin{align*}
    \mathrm{Tr} C_t A_{sf} \leq \lvert \mathrm{Tr} C_t A_{sf} \rvert \leq \mathrm{Tr} \lvert C_t A_{sf} \rvert
\end{align*}
(think of the singular values).
{\color{red}Actually, I think it will be easiest if we just prove the inequality for $\lvert g_t (C_t)$, since then the result is immediate.
Problem is, then we can't subsume the terms in $g_t (C_t)$ into the recursion; since we don't a-priori know that $C_t$ is uniformly bounded, we can't really say that $g_t (C_t)$ is bounded. 
}

% We want an inequality of the form
% \begin{align*}
%     \mathrm{Tr} A X \leq a \lvert \mathrm{Tr} X \rvert .
% \end{align*}
% Then go back and consider the two cases separately, when $\mathrm{Tr} C_t A_{sf}$ is positive vs. negative.
% One simple inequality is
% \begin{align*}
%     \lvert \mathrm{Tr} A X \rvert = \lvert \sum_{ij} A_{ij} X_{ij} \rvert \leq \max_{ij} \lvert X_{ij} \rvert \lVert A \rVert_{11}
% \end{align*}

% We can use the Frobenius inner product to express the trace of a product of two arbitrary rectaungular matrices $A, B$, which is subject to Cauchy-Schwartz inequality
% \begin{align*}
%     \mathrm{Tr} A^T B = \langle A, B \rangle_F \leq \lVert A \rVert_F \lVert B \rVert_F .
% \end{align*}
% % Using that $\lVert A \rVert_F = \sqrt{\sum_{i} \sigma_i^2 (A)} \leq \sum_{i} \sigma_i (A) = \sum_i \lvert \lambda_i (A)$, we then have, using the notation that for a matrix $A = U \Lambda V$, where $\Lambda$ is the singular value matrix of $A$, $\lvert A \rvert = U \lvert \Lambda \rvert V$ with $\lvert \cdot \rvert$ applied element-wise to a diagonal matrix:
% Therefore, we can express the trace of the product as a weighted multiple of the norm of singular values of $B$:
% \begin{align*}
%     \mathrm{Tr} A^T B \leq \lVert A \rVert_F \lVert \sigma_B \rVert_2 .
% \end{align*}


% For example if we want to obtain a bound on the first term in $g_t (C_t)$,
% \begin{align*}
%     \mathrm{Tr} (\delta_t A_{sf}) (C_t A_{sf}) \leq  \lVert \delta_t A_{sf} \rVert_F \lVert C_t A_{sf} \rVert_F
% \end{align*}

% \section{Below sections are old.}


% Delegated details on bounds, following Polyak Juditsky, Srikant, Thinh, and Kaledin
% \section{Detailed Method}\label{sec:details}

\subsection{Nested lattice codebook}

In this section, we describe the construction for a Vector Quantization (VQ) codebook of size $q^d$ for quantizing an $d$-dimensional vector, where $q$ is an integer parameter. To quantize a vector, we find the closest codebook element by Euclidean norm. We describe efficient encoding and decoding algorithms to a quantized representation in $\Z_q^d$.

Let $\Lambda$ be a lattice in $\RR^d$ with generator matrix $G$. We define the coordinates of a point $x \in \Lambda$ to be an integer vector $v$ such that $x = Gv$. Each point $P \in \Lambda$ has a corresponding Voronoi region $\m{V}_\Lambda(P)$, for which $P$ is the closest point in $\Lambda$ with respect to $L^2$ metric. To define the codebook, we consider the scaled lattice $q\Lambda$. Then:

\begin{definition}
    $x \in \Lambda$ belongs to codebook $C$ iff $x \in \m{V}_{q\Lambda}(0)$. Let $v$ be the coordinates of $x$. Then, the quantized representation of $x$ is $\mathcal{Q}(x) := v \mmod q$. Note that $\mathcal{Q}$ is a bijection between $C$ and $\Z_q^d$
\end{definition}

Using this representation, we can describe the encoding and decoding functions, assuming the point $x$ we are quantizing is in $\m{V}_{q\Lambda}(0)$. We will also need an oracle $Q_{\Lambda}(x)$, which maps $x$ to the closest point in $\Lambda$ to $x$.

\begin{algorithm}[h]
   \caption{Encode}
   \label{alg:encode}
\begin{algorithmic}
   \State {\bfseries Input:} $x \in V_{q\Lambda}(0)$, $Q_{\Lambda}$
   \State $p \leftarrow Q_{\Lambda}(x)$
   \State $v \leftarrow G^{-1}p$ \Comment{coordinates of $p$}
   \State {\bfseries return} {$v \mmod q$} \Comment{quantized representation of $p$}
\end{algorithmic}
\end{algorithm}




\begin{algorithm}[h]
\caption{Decode}
\label{decode-algo}
\begin{algorithmic}
   \State {\bfseries Input:} $c \in \Z_q^d$, $Q_{\Lambda}$
   \State $p \leftarrow Gc$ \Comment{equivalent to answer modulo $q\Lambda$}
   \State {\bfseries return} $p - q\,Q_{\Lambda}\!\bigl(\tfrac{p}{q}\bigr)$
\end{algorithmic}
\end{algorithm}

In practice, we will be using the Gosset ($E_8$) lattice as $\Lambda$ with $d = 8$. This lattice is a union of $D_8$ and $D_8 + \frac{1}{2}$, where $D_8$ contains elements of $\Z^8$ with even sum of coordinates. There is a simple algorithm for finding the closest point in the Gosset lattice, first described in \cite{1056484}. We provide the pseudocode for this algorithm together with the estimation of its runtime in Appendix \ref{sec:oracle}.

\subsection{Matrix quantization}

\label{matrix-quant}

When quantizing a matrix, we normalize its rows, and quantize each block of $d$ entries using the codebook. The algorithm \ref{alg:nestquant} describes the quantization procedure for each row of the matrix.

\begin{algorithm}[h]
\caption{NestQuant}
\label{alg:nestquant}
\begin{algorithmic}
   \State {\bfseries Input:} $A$ --- a vector of size $n = db$, $q$, array of $\beta$
   \State $QA$ --- $n$ integers \Comment{quantized representation}
   \State $B$ --- $b$ integers \Comment{scaling coefficient indices}
   \State \label{norm_nestquant} $s \leftarrow \lVert A_i\rVert_2$ \Comment{normalization coefficient}
   \State $A \leftarrow \frac{A\sqrt{n}}{s}$
   \For{$j = 0$ {\bfseries to} $b-1$}
        \State $err = \infty$
        \For{$p = 1$ {\bfseries to} $k$}
            \State $v \leftarrow A[dj+1..dj+d]$
            \State $enc \leftarrow \text{Encode}\left(\frac{v}{\beta_p}\right)$
            \State $recon \leftarrow \text{Decode}(enc) \cdot \beta_p$
            \If{$err > |recon - v|_2^2$}
                \State $err \leftarrow |recon - v|_2^2$
                \State $QA[dj+1..dj+d] \leftarrow enc$
                \State $B_{j} \leftarrow p$
            \EndIf
        \EndFor
   \EndFor
   \State {\bfseries Output:} $QA$, $B$, $s$
\end{algorithmic}
\end{algorithm}

We can take dot products of quantized vectors without complete dequantization using algorithm \ref{alg:dotproduct}. We use it in the generation stage on linear layers and for querying the KV cache.

\begin{algorithm}[h]
\caption{Dot product}
\label{alg:dotproduct}
\begin{algorithmic}
   \State {\bfseries Input:} $QA_1$, $B_1$, $s_1$ and $QA_2$, $B_2$, $s_2$ --- representations of two vectors of size $n = db$ from Algorithm \ref{alg:nestquant}, array $\beta$
   \State $ans \leftarrow 0$
   \For{$j = 0$ {\bfseries to} $b-1$}
        \State $p_1 \leftarrow \text{Decode}(QA_1[dj+1..dj+d])$
        \State $p_2 \leftarrow \text{Decode}(QA_2[dj+1..dj+d])$
        \State $ans \leftarrow ans + (p_1 \cdot p_2)\beta_{B_1[j]}\beta_{B_2[j]}$
   \EndFor
   \State {\bfseries return} $ans$
\end{algorithmic}
\end{algorithm}

\subsection{LLM quantization}

\label{subsec:llm-quant}

\ifisicml
\begin{figure}
    \centering
    \includegraphics[width=\linewidth]{figures/kv.pdf}
    \caption{The quantization scheme of multi-head attention. $H$ is Hadamard rotation described in \ref{subsec:llm-quant}. $\mathcal{Q}$ is the quantization function described in \ref{matrix-quant}}
    \label{fig:scheme}
\end{figure}

\else
\begin{figure}[h]
    \centering
    \includegraphics[width=0.5\linewidth]{figures/kv.pdf}
    \caption{The quantization scheme of multi-head attention. $H$ is Hadamard rotation described in \ref{subsec:llm-quant}. $\mathcal{Q}$ is the quantization function described in \ref{matrix-quant}}
    \label{fig:scheme}
\end{figure}

\fi

Recall that we apply a rotation matrix $H$ to every weight-activation pair of a linear layer without changing the output of the network. Let $n$ be the number of input features to the layer.

\begin{itemize}
    \item If $n = 2^k$, we set $H$ to be Hadamard matrix obtained by Sylvester's construction
    \item Otherwise, we decompose $n = 2^km$, such that $m$ is small and there exists a Hadamard matrix $H_1$ of size $m$. We construct Hadamard matrix $H_2$ of size $2^k$ using Sylvester's construction, and set $U = H_1 \otimes H_2$.
\end{itemize}

Note that it's possible to multiply an $r \times n$ matrix by $H$ in $O(rn \log n)$ in the first case and $O(rn(\log n + m))$ in the second case, which is negligible to other computational costs and can be done online.

In NestQuant, we quantize all weights, activations, keys, and values using Algorithm \ref{alg:nestquant}. We merge the Hadamard rotation with the weights and quantize them. We also apply the Hadamard rotation and quantization to the activations before linear layers. We also apply rotation to keys and queries, because it will not change the attention scores, and we quantize keys and values before putting them in the KV cache. Figure \ref{fig:scheme} illustrates the procedure for multi-head attention layers.

When quantizing a weight, we modify the NestQuant algorithm by introducing corrections to unquantized weights when a certain vector piece is quantized. We refer the reader to section 4.1 of \cite{tseng2024} for a more detailed description.

\subsection{Optimal scaling coefficients}

One of the important parts of the algorithm is finding the optimal set of $\beta_i$. Given the distribution of 8-vectors that are quantized via a codebook, it is possible to find an optimal set of given size exactly using a dynamic programming approach, which is described in Appendix \ref{dp-section}.

\subsection{Algorithm summary}
\label{algo-summary}

Here we describe the main steps of NestQuant.

\begin{enumerate}
    \item Collect the statistics for LDLQ. For each linear layer with in-dimension $d$, we compute a $d \times d$ matrix $H$.
    \item We choose an initial set of scaling coefficients $\hat{\beta}$, and for each weight we simulate LDLQ quantization with these coefficients, getting a set of 8-dimensional vectors to quantize.
    \item We run a dynamic programming algorithm described in Appendix \ref{dp-section} on the 8-vectors to find the optimal $\beta$-values for each weight matrix.
    \item We also run the dynamic programming algorithm for activations, keys, and values for each layer. To get the distribution of 8-vectors, we run the model on a small set of examples.
    \item We quantize the weights using LDLQ and precomputed $\beta$.
    \item During inference, we quantize any activation before it's passed to the linear layer, and any KV cache entry before it is saved.
\end{enumerate}
Note the complete lack of fine-tuning needed to make our method work.


\end{document}