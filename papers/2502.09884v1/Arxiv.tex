\documentclass[12pt]{arxiv} % Anonymized submission
%\documentclass[final,12pt]{colt2025} % Include author names
% \documentclass[12pt]{article}

% TODO: Remove the page numbers [start]-[end] on top of first page. 
% \usepackage{fancyhdr}
% \pagestyle{empty}
% \fancypagestyle{firstpage}{%
%   \fancyhf{}  % Clear headers and footers
%   \renewcommand{\headrulewidth}{0pt}  % Remove top rule
% }
% \thispagestyle{firstpage}  % Apply to first page

% From the jmlr.cls file. jmlr.cls is likely being retrieved from overleaf. 
\makeatletter
\def\ps@jmlrtps{%
  \let\@mkboth\@gobbletwo
  \def\@oddhead{}  % Remove the first page header
  \let\@evenhead\@oddhead
  \def\@oddfoot{}  % Remove footer if necessary
  \let\@evenfoot\@oddfoot
}
\makeatother
% TODO: Remove the page numbers [start]-[end] on top of first page. 


\usepackage[utf8]{inputenc} % allow utf-8 input
\usepackage[T1]{fontenc}    % use 8-bit T1 fonts
\usepackage{hyperref}       % hyperlinks
\usepackage{booktabs}       % professional-quality tables
\usepackage{amsfonts}       % blackboard math symbols
\usepackage{nicefrac}       % compact symbols for 1/2, etc.
\usepackage{microtype}      % microtypography
\usepackage{bbold}
\usepackage{capt-of}

% \usepackage[center]{caption}
% \usepackage{caption}


\usepackage{cite,epstopdf,color,soul,mathabx}
\usepackage{makecell}
% \usepackage[ruled,vlined]{algorithm2e}
\allowdisplaybreaks
\usepackage{color}
\usepackage{enumerate}
\usepackage[shortlabels]{enumitem}
\usepackage{multicol}
\usepackage{empheq}
% \usepackage{caption} 
% \captionsetup[table]{skip=10pt}
\usepackage{algorithm,algorithmic}

  
\let\underbrace\LaTeXunderbrace
\let\overbrace\LaTeXoverbrace

% \documentclass{l4dc2024}

% \title{Non-asymptotic Central Limit Theorem for Two Timescale Stochastic Approximation}


% The following packages will be automatically loaded:
% amsmath, amssymb, natbib, graphicx, url, algorithm2e

% \title[Short Title]{Full Title of Article}
% \title[Quantitative CLT for Two Timescale Stochastic Approximation]{Non-asymptotic Central Limit Theorem for Two Timescale Stochastic Approximation}
\title[Quantitative CLT for Two-Time-Scale Stochastic Approximation]{
Nonasymptotic CLT and Error Bounds for Two-Time-Scale Stochastic Approximation
% Non-asymptotic CLT and Error Bounds for Averaged Iterates in Two-Time-Scale Linear Stochastic Approximation
% Averaged Iterates Achieve Optimal Rates in Two Timescale Stochastic Approximation
% Achieving Optimal Rates in Two Timescale Stochastic Approximation with Polyak Ruppert Averaging and Confidence Intervals
}
% \usepackage{times}
% Use \Name{Author Name} to specify the name.
% If the surname contains spaces, enclose the surname
% in braces, e.g. \Name{John {Smith Jones}} similarly
% if the name has a "von" part, e.g \Name{Jane {de Winter}}.
% If the first letter in the forenames is a diacritic
% enclose the diacritic in braces, e.g. \Name{{\'E}louise Smith}

% Two authors with the same address
% \coltauthor{\Name{Author Name1} \Email{abc@sample.com}\and
%  \Name{Author Name2} \Email{xyz@sample.com}\\
%  \addr Address}

% Three or more authors with the same address:
% \coltauthor{\Name{Author Name1} \Email{an1@sample.com}\\
%  \Name{Author Name2} \Email{an2@sample.com}\\
%  \Name{Author Name3} \Email{an3@sample.com}\\
%  \addr Address}

% Authors with different addresses:
% \coltauthor{%
%  \Name{Author Name1} \Email{abc@sample.com}\\
%  \addr Address 1
%  \AND
%  \Name{Author Name2} \Email{xyz@sample.com}\\
%  \addr Address 2%
% }

\coltauthor{%
 \Name{Seo Taek Kong} \Email{skong10@illinois.edu}\\
 \addr University of Illinois, Urbana-Champaign
 \AND
 \Name{Sihan Zeng} \Email{szeng2017@gmail.com}\\
 \addr JPMorgan AI Research
 \AND
 \Name{Thinh T. Doan} \Email{thinhdoan@utexas.edu}\\
 \addr University of Texas at Austin
 \AND 
 \Name{R. Srikant} \Email{rsrikant@illinois.edu}\\
 \addr University of Illinois, Urbana-Champaign
}

\usepackage{times}

\usepackage{mathtools}
\usepackage{nccmath}
\usepackage{esvect}
\usepackage{cleveref} % Custom numbering of theorems.
% \usepackage{amsthm} % Conflict.
\usepackage{mathrsfs}

% Theorems
\newtheorem{innercustomgeneric}{\customgenericname}
\providecommand{\customgenericname}{}
\newcommand{\newcustomtheorem}[2]{%
  \newenvironment{#1}[1]
  {%
   \ifdefined\crefalias\crefalias{innercustomgeneric}{#2}\fi
   \renewcommand\customgenericname{#2}%
   \renewcommand\theinnercustomgeneric{##1}%
   \innercustomgeneric
  }
  {\endinnercustomgeneric}%
  \ifdefined\crefname\crefname{#2}{#2}{#2s}\fi
}

\newtheorem{assumption}{Assumption}
\newtheorem{proposition*}{Proposition} % No counting\newcustomtheorem{customthm}{Theorem}
\newcustomtheorem{customlemma}{Lemma}
\newcustomtheorem{customprop}{Proposition}


\newenvironment{mpmatrix}{\begin{medsize}\begin{pmatrix}}{\end{pmatrix}\end{medsize}}%

\newcommand\numberthis{\addtocounter{equation}{1}\tag{\theequation}}

\DeclarePairedDelimiter\ceil{\lceil}{\rceil}
\DeclarePairedDelimiter\floor{\lfloor}{\rfloor}
\DeclarePairedDelimiter\abs{\lvert}{\rvert}%
\DeclarePairedDelimiter\norm{\lVert}{\rVert}%
\DeclarePairedDelimiterX{\inp}[2]{\langle}{\rangle}{#1, #2}

\newcommand{\expect}[1]{\mathbb{E}\left[#1\right]}
\newcommand{\history}{\mathcal{H}_t}
\newcommand{\historyprev}{\mathcal{H}_{t-1}}
% \newcommand{\historynext}{\mathcal{H}_{t+1}}

% \usepackage{subcaption}
% \usepackage[compatibility=false]{subcaption}
\usepackage{graphicx}
\usepackage{float}
\newcommand{\CLT}{%
Let $\bar{z}_n \coloneqq (\bar{x}_n, \bar{y}_n)$ be the Ruppert-Polyak average of $(x_t, y_t)$ generated by \eqref{eq:ttsa}. 
Define the asymptotic covariance of $\sqrt{n}(\bar{x}_n - x^*)$ and $\sqrt{n}(\bar{y}_n - y^*)$ as
\begin{equation}
    \bar{\Sigma}_{ff} = \lim_{n \to \infty } n \mathbb{E} (\bar{x}_n - x^*) (\bar{x}_n - x^*) ^T , \quad
    \bar{\Sigma}_{ss} = \lim_{n \to \infty } n \mathbb{E} (\bar{y}_n - y^*) (\bar{y}_n - y^*)^T .
\end{equation}
Under Assumptions \ref{assumption:first}--\ref{assumption:last}, the above limits exist, and the rate of convergence in the Wasserstein-1 distance is given by:
    \begin{equation}
        \begin{split}
            d_1 \left(\sqrt{n} G (\bar{x}_n - x^*), (G \bar{\Sigma}_{ff} G^T)^{1/2} Z_1 \right) 
            &= 
            \mathcal{O}\left(\frac{1}{\sqrt{n}} \left(n^{a/2} + n^{a-b/2} + n^{b/2}\right) \right), 
            \\ 
            d_1 \left(\sqrt{n} \Delta (\bar{y}_n - y^*), (\Delta \bar{\Sigma}_{ss} \Delta^T)^{1/2} Z_2 \right) &= 
            \mathcal{O}\left(\frac{1}{\sqrt{n}} \left(n^{a/2} + n^{a-b/2} + n^{b/2}\right) \right), 
        \end{split} %\label{eq:WassersteinBound}
    \end{equation}
    where $Z_1$ and $Z_2$ are standard Gaussian vectors of appropriate dimensions.
}

\newcommand{\MSE}{%
    Assume \ref{assumption:first}--\ref{assumption:last}.
    Let $\Sigma$ be the asymptotic covariance of $(x_t - x^*, y_t - y^*)$, evaluated as
    \begin{align*}
        A_{ff} \Sigma_{ff} + \Sigma_{ff} A_{ff}^T &= \Gamma_{ff} , \\
        A_{ff} \Sigma_{fs} + \Sigma_{ff} A_{sf}^T &= \Gamma_{fs} , \\ 
        \Delta \Sigma_{ss} + \Sigma_{ss} \Delta^T + A_{sf} \Sigma_{fs} + \Sigma_{sf} A_{sf}^T &= \Gamma_{ss} 
        .
        \numberthis \label{eq:covariances}
    \end{align*}
    For some problem-dependent constants $M_f, M_s > 0$ and every $n \geq 1$, it holds that
    \begin{equation}\label{eq:thm2_bounds}
        \begin{split}
            \lVert \mathbb{E} \hat{x}_{n+1} \hat{x}_{n+1}^T - \alpha_{n+1} \Sigma_{ff}\rVert &\leq 
            \prod_{t=1}^n \left(1 - \alpha_t \frac{\mu_{ff}}{4}\right) \lVert \mathbb{E} \hat{x}_1 \hat{x}_1^T - \alpha_1 \Sigma_{ff}\rVert 
            + M_f \gamma_{n}
            % \mathcal{O}\left(\gamma_n + \frac{1}{n}\right)
            % \gamma_n M_f (1 + \lVert \Sigma_{ff} \rVert) 
            % + \frac{M_f}{n} \lVert \Sigma_{ff} \rVert 
            % + o(n^{-1}) 
            ,
            \\ 
            \lVert \mathbb{E} \hat{y}_{n+1} \hat{y}_{n+1}^T - \gamma_{n+1} \Sigma_{ss} \rVert 
        & \leq 
        \prod_{t=1}^n \left(1 - \gamma_t \frac{\mu_\Delta}{4}\right) \lVert \mathbb{E}\hat{y}_1 \hat{y}_1^T - \gamma_1 \Sigma_{ss} \rVert 
        + \frac{M_s}{n} 
        % \frac{\gamma_{n}^2}{\alpha_n}.
        \end{split}
    \end{equation}
    }


\newtheorem*{lemma*}{Lemma}

% Use \Name{Author Name} to specify the name.
% If the surname contains spaces, enclose the surname
% in braces, e.g. \Name{John {Smith Jones}} similarly
% if the name has a "von" part, e.g \Name{Jane {de Winter}}.
% If the first letter in the forenames is a diacritic
% enclose the diacritic in braces, e.g. \Name{{\'E}louise Smith}

% Two authors with the same address
% \coltauthor{\Name{Author Name1} \Email{abc@sample.com}\and
%  \Name{Author Name2} \Email{xyz@sample.com}\\
%  \addr Address}

% Three or more authors with the same address:
% \coltauthor{\Name{Author Name1} \Email{an1@sample.com}\\
%  \Name{Author Name2} \Email{an2@sample.com}\\
%  \Name{Author Name3} \Email{an3@sample.com}\\
%  \addr Address}

% % Authors with different addresses:
% \author{%
%  \Name{Author Name1} \Email{abc@sample.com}\\
%  \addr Address 1
%  \AND
%  \Name{Author Name2} \Email{xyz@sample.com}\\
%  \addr Address 2%
% }
\newcommand{\tdoan}[1]{{\color{red}\bf [thinh: #1]}}



\begin{document}

\maketitle

\begin{abstract}
    We consider linear two-time-scale  stochastic approximation algorithms driven by martingale noise. 
    Recent applications in machine learning motivate the need to understand finite-time error rates, but conventional stochastic approximation analysis focus on either asymptotic convergence in distribution or finite-time bounds that are far from optimal.
    Prior work on asymptotic central limit theorems (CLTs) suggest that two-time-scale algorithms may be able to achieve $1/\sqrt{n}$ error in expectation, with a constant given by the expected norm of the limiting Gaussian vector.
    However, the best known finite-time rates are much slower.
    We derive the first non-asymptotic central limit theorem with respect to the Wasserstein-1 distance for two-time-scale stochastic approximation with Polyak-Ruppert averaging.
    As a corollary, we show that expected error achieved by Polyak-Ruppert averaging decays at rate $1/\sqrt{n}$, which significantly improves on the rates of convergence in prior works.
\end{abstract}

\begin{keywords}
    Two-time-scale stochastic approximation, Polyak-Ruppert averaging, Central Limit Theorem
\end{keywords}

% \section{Multi-Stage Robotic Manipulation Planning Tasks}\label{sec:tasks}
%%

\begin{figure*}[t!]
    \centering
    \includegraphics[width=0.99\textwidth]{figs/filmstrip.pdf}
    \caption{\textbf{Filmstrip of our method solving a complicated assembly task.} Frames are indexed by timestep. The goal image is in the top-left corner (with a green border). Each frame is the observation after executing the action (in black) above it. The other action in gray is the original action proposed by the VLM if it is revised after reflection. We highlight the reflection process at timestep 15, where the VLM first proposes an action to pick up the purple brick, but after reflection, it chooses to pick up the yellow brick instead as the generated future state (red-bordered image) shows little progress towards the goal.}
    \label{fig:filmstrip}
\end{figure*}
Inspired by~\citet{luo2024fmb}, we procedurally generated a suite of multi-stage long-horizon manipulation tasks that require understanding of physical interactions and reasoning about the effects of long-term action sequences. The task is initialized with a board and a set of small pieces randomly placed on a table. The goal is to fully assemble the board by inserting the pieces into the board one by one. Examples of the initial and goal configurations are shown in Fig.~\ref{fig:tasks}. Detailed task generation process is included in App.~\ref{sec:app_task_gen}. Notably, most tasks include inter-locking pieces so that they can be inserted into the board only in a specific order. This requires strategically choosing the object to be manipulated at each step and inferring possible interaction between this object and the other objects already in the board. 
As an example, Fig.~\ref{fig:tasks}(b) shows the dependencies between the pieces in one of the tasks. 
The interlocking feature further necessitates the agent’s ability to replan, enabling it to recover from failures caused by previous mistakes or bad initialization. 


\begin{figure}[h!]
    \centering
    \includegraphics[width=0.49\textwidth]{figs/tasks_single_column.pdf}
    \vspace{-0.1in}
    \caption{\textbf{Task examples.} (a) Generated multi-stage manipulation tasks with interlocking pieces. Top: initial configurations. Bottom: goal configurations. See App.~\ref{sec:app_more_task_samples} for more examples. (b) The graph shows the dependencies between the objects in the blue assembly board on the left. Each node represents an object, and each directed edge indicates the predecessor object should be assembled before the successor object.}
    \label{fig:tasks}
\end{figure}

% \begin{figure}[h!]
%     \centering
%     \includegraphics[width=0.95\linewidth]{figs/dependencies.pdf}
%     \caption{\textbf{A dependency graph of interlocking objects.} The right graph shows the dependencies between the objects in the assembly task on the left. Each node represents an object, and each directed edge indicates the predecessor object should be assembled before the successor object.}
%     \label{fig:dependencies}
% \end{figure}

We focus on the high-level planning of this long-horizon manipulation task. We define a set of actions in the form of ``{\tt [act] [obj]}", where $\text{\tt [act]}\in \{\text{\tt pick up}, \text{\tt insert}, \text{\tt reorient}, \text{\tt put down}\}$ is an action primitive, and {\tt [obj]} denotes the object to be manipulated. Specifically, ``{\tt pick up}" grasps a piece that is not in hand and picks it up. It can then be inserted into the board using the ``{\tt insert}" action, or put back on the table using ``{\tt put down}". By invoking ``{\tt reorient}", the object in hand can be reoriented with the black fixture if necessary, so that it is in a suitable pose for insertion. Each action primitive is implemented as a rule-based script controller; however, integrating other low-level controllers, such as learning-based policies like behavior cloning, is also possible. We also designed an expert policy for the mentioned motor primitives, see App.~\ref{sec:app_expert} for implementation details.



% \section{Introduction}
% Re-position our contributions.
% \begin{enumerate}
%     \item We prove convergence rate in distribuion of two timescale stochastic approximation. 
%     \item As a consequence, we obtain the optimal rate of convergence in expected norm of linear TTSA. 
%     \item Just combining some techniques (e.g. MSE \citep{konda2004convergence} and Jensen) gives sub-optimal rates. Therefore, we improve the analysis techniques to arrive at the optimal rate. 
%     \item Non-trivial combination of techniques? Can I outline them well, or is it hard to describe? 
% \end{enumerate}


\section{Preliminaries}\label{sec:preliminaries}



%We denote by $(\Ac(x_\Ac),\Bc(x_\Bc))(z)$ a random execution of $\pi$ with private inputs $(x_\Ac,y_\Ac)$, and common input $z$.

%\Jnote{Move to DP}
% At the end of such an execution, the protocol outputs a public transcript denoted by the random variable $\trans_\pi(x_\Ac,x_\Ac,z)$ we denotes the common as $\out(\trans_\pi(x_\Ac,x_\Ac,z)$, and each party $\Pc \in \set{\Ac,\Bc}$ obtains his view denoted $\view^\Pc_\pi(x_\Ac,x_\Bc,z)$, which may also contain a ``local output'' \Jnote{Local} $\out^\Pc(x_\Ac,x_\Bc,z)$ (if the protocol specifies such an output). \Jnote{Common output, and parties output}


\subsection{Distributions and Random Variables}\label{sec:prelim:dist}
The support of a distribution $P$ over a finite set $\cS$ is defined by $\Supp(P) \eqdef \set{x\in \cS: P(x)>0}$. For a distribution or a random variable $D$, let $d\from D$ denote that $d$ was sampled according to $D$. Similarly,  for a set $\cS$, let $x \from \cS$ denote that $x$ is drawn uniformly from $\cS$, and denote by $\cU_{\cS}$ the uniform distribution over $\cS$. For a finite set $\cX$ and a distribution $C_X$ over $\cX$, we use the capital letter $X$ to denote the random variable that takes values in $\cX$ and is sampled according to $C_X$. The {\sf statistical distance} (\aka {\sf~variation distance}) of two distributions $P$ and $Q$ over a discrete domain $\cX$ is defined by $\sdist{P}{Q} \eqdef \max_{\cS\subseteq \cX} \size{P(\cS)-Q(\cS)} = \frac{1}{2} \sum_{x \in \cS}\size{P(x)-Q(x)}$. 
For a vector $x = (x_1,\ldots,x_n)$ and index $i\in [n]$, we let $x_{-i} = (x_1,\ldots,x_{i-1},x_{i+1},\ldots,x_n)$ and $x^{(i)} = (x_1,\ldots,x_{i-1}, -x_i, x_{i+1},\ldots,x_n)$, for a set $\cS \subseteq [n]$ we let $x_{\cS} = (x_i)_{i \in \cS}$ and $x_{-\cS} = (x_i)_{i \in [n]\setminus \cS}$, and for a vector $r \in \zo^n$ we let $x_r = (x_i)_{\set{i \colon r_i = 1}}$ and $x_{-r} = (x_i)_{\set{i \colon r_i = 0}}$.

%For $n \in \N$ we let $U_n$ be the uniform distribution over $\oo^n$, and let $S_n$ be the distribution induces by the sum of $n$ i.i.d.\ random variables, each is distributed according to $U_1$. Let $\cN(0,1)$ be the standard normal distribution.
%For a distribution $\cD$ and a function $f$, we define by $f(\cD)$ the distribution that is induced by the output of $f(x)$ for $x \from \cD$. 





% \begin{theorem}[\cite{McGregorMPRTV10}]\label{thm:sv-extracotr}
% 	\Enote{Remove if not needed}
% 	There is a constant $c$ to make the following holds. Let $X$ be an $\alpha$-SV source on $\{0,1\}^n$, let $Y$ be a source on $\{0,1\}^n$ with min-entropy at least $\beta n$ (independent from $X$), and let $Z=\ip{X,Y}\mbox{mod m}$ for some $m\in\mathbb{N}$. Then for every $\delta\in[0,1]$, the random variable $(Y,Z)$ is $\delta$-close to $(Y,U)$ where $U$ is uniform on $\mathbb{Z}_m$ and independent of $Y$, provided that
% 	$$
% 	n\geq c\cdot\frac{m^2}{\alpha\beta}\cdot\log(\frac{m}{\beta})\cdot\log(\frac{m}{\delta}).
% 	$$
% \end{theorem}



\Enote{I removed the definition of DP since it already appears in the intro}
\remove{
\subsection{Differential Privacy}\label{sec:prelim:DP}
We use the following standard definition of (information theoretic) differential privacy, due to \citet{DMNS06}. For notational convenience, we focus on databases over $\oo$.
\begin{definition}[Differentially private mechanisms]\label{def:mech}
	A randomized function $f\colon\oo^n\mapsto \zs$ is an {\sf $n$-size, $(\eps,\delta)$-differentially private mechanism} (denoted $(\eps,\delta)$-\DP) if for every neighboring $w,w'\in \oo^n$ and every function $g\colon \zs\mapsto \zo$, it holds that 
	$$
	\pr{g(f(w))=1}\leq e^{\eps}\cdot \pr{g(f(w'))=1} +\delta.
	$$ 	
	If $\delta=0$, we omit it from the notation.
\end{definition}
}


\subsubsection{Computational Differential Privacy}
There are several ways for defining computational differential privacy (see \cref{sec:related-works}). We use the most relaxed version due to \cite{BNO08}. For notational convenience, we focus on databases over $\oo$.
\begin{definition}[Computational differentially private mechanisms]\label{def:ComMech}
	A randomized function ensemble $f=\set{f_\pk\colon\oo^{n(\pk)}\mapsto \zs}$ is an {\sf $n$-size, $(\eps,\delta)$-computationally differentially private} (denoted $(\eps,\delta)$-$\CDP$) if for every poly-size circuit family $\set{\Ac_\pk}_{\pk\in \N}$, the following holds for every large enough $\pk$ and every neighboring $w,w'\in\oo^{n(\pk)}$:
	$$
	\pr{\Ac_\pk(f_\pk(w))=1}\leq e^{\eps(\pk)}\cdot \pr{\Ac_\pk(f_\pk(w'))=1} +\delta(\pk).
	$$ 
	If $\delta(\pk) = \negl(\pk)$, we omit it from the notation. 
\end{definition}



\subsubsection{Two-Party Differential Privacy}\label{sec:DP}
In this section we formally define distributed differential privacy mechanism (\ie protocols). %For the ease of notation, we consider protocol with no common input.

\begin{definition}\label{def:DP}%\Nnote{fix security parameter}
	A two-party protocol $\Pi=(\Ac,\Bc)$ is {\sf $(\eps,\delta)$-differentially private}, denoted $(\eps,\delta)$-$\DP$, if the following holds for every algorithm $\Dc$: let $\V^\Pc(x,y)(\pk)$ be the view of party $\Pc$ in a random execution of $\Pi(x,y)(1^\pk)$. Then for every $\pk,n \in \N$, $x\in \oo^n$ and neighboring $y,y'\in\oo^n$:
	\begin{align*}
	\pr{\Dc(V^\Ac(x,y)(\pk))=1}\le e^{\eps(\pk)}\cdot \pr{\Dc(V^\Ac (x,y')(\pk))=1}+\delta(\pk),
	\end{align*} 
	and for every $y\in \oo^n$ and neighboring $x,x'\in\oo^{n}$:
	\begin{align*}
	\pr{\Dc(V^\Bc(x,y)(\pk))=1}\le e^{\eps(\pk)}\cdot \pr{\Dc(V^\Bc (x',y)(\pk))=1}+\delta(\pk).
	\end{align*} 	
	Protocol $\Pi$ is {\sf $(\eps,\delta)$-computational differentially private}, denoted $(\eps,\delta)$-$\CDP$, if the above inequalities only hold for a non-uniform \ppt $\Dc$ and large enough $\pk$. We omit $\delta = \negl(\pk)$ from the notation. \footnote{Note that define we give for two-party differentially private protocols is a semi-honest definition, in which we ask for the security to hold when the parties interact in an honest execution of the protocol. Since we are proving a lower bound, starting from this weaker guarantee (as opposed to security against malicious players), yields a stronger result.}
\end{definition}
%We omit $\delta$ from the notation if $\delta$ is a negligible function of $n$.

%\Enote{simulation-based}
\begin{remark}[The definition for computational differential privacy we use]\label{rem:comDPChannel} 
	An alternative, stronger definition of computational differential privacy, known as simulation-based computational differential privacy, requires that the distribution of each party’s view be computationally indistinguishable from a distribution that ensures privacy in an information-theoretic sense. \cref{def:DP} is a weaker notion in comparison. Consequently, establishing a lower bound for a protocol that satisfies this weaker guarantee (as we do in this work) yields a stronger result.%Actually, our lower bound only requires the privacy to hold against \emph{uniform} external observer.
	%\Nnote{Maybe add: When only interesting in \Dp against external observer, the two definitions can be achieve using key-agreement and (single-party) \Dp mechanism. }
\end{remark}




\subsection{Useful Claims}
\remove{
In this section, we state generic lemmas and propositions that we will use later in our proofs.

The following lemma which we prove in \cref{sec:missing-proofs:distance-I}, measures the distance between two uniform stings conditioned one a random index $i$ either being fixed to $0$ or to $1$.

\def\distanceILemma{
    Let $R \la \zo^n$. For any (randomized) function $f:\{0,1\}^n\rightarrow \{0,1\}$ and $\alpha > 0$, it holds that
    \begin{align}\label{eq:f-alpha}
        \ppr{i \la [n]}{\size{\:\ex{f(R) \mid R_i = 0}-\ex{f(R) \mid R_i = 1}\:}\geq \alpha} \leq \frac{2}{n \alpha^2},
    \end{align}
    where the expectations are taken over $R$ and the randomness of $f$.
}

\begin{lemma}\label{lem:distance-I}
    \distanceILemma
\end{lemma}
}

The following two propositions state that given the output of a differentially private function, it is not possible to predict well even a random index (even if all other indexes are leaked). The first proposition handles the information-theoretic case and the second handles the computation case. Both propositions are proven in \cref{sec:missing-proofs:hard-to-guess}. 

\def\propHardToGuessInf{
    Let $f\colon \oo^n \rightarrow \cY$ be an $(\eps,\delta)$-\DP function, let $g \colon [n] \times \oo^{n-1} \times \cY \rightarrow \set{-1,1,\bot}$ be a (randomized) function, and let $X = (X_1,\ldots,X_n) \la \oo^n$. Then the following holds for every $i \in [n]$ where $X_i^* = g(i,X_{-i},f(X_1,\ldots,X_n))$:
    \begin{align*}
        \pr{X_i^* = X_i} \leq e^{\eps}\cdot \pr{X_i^* = -X_i} + \delta.
    \end{align*}
}

\begin{proposition}\label{prop:hard-to-guess-inf}
    \propHardToGuessInf
\end{proposition}


\def\propHardToGuessComp{
    Let $f = \set{f_{\pk} \colon \oo^{n(\pk)} \rightarrow \zo^{m(\pk)}}_{\pk \in \bbN}$ be an $(\eps,\delta)$-\CDP function ensemble, and let $\set{g_{\pk}}_{\pk \in \bbN}$ be a poly-size circuit family. Then, for large enough $\pk$ and $X = (X_1,\ldots,X_{n(\pk)}) \la \oo^{n(\pk)}$, the following holds for every $i \in [n(\pk)]$ where $X_i^* = g_{\pk}(i,X_{-i},f_{\pk}(X_1,\ldots,X_n))$:
    \begin{align*}
        \pr{X_i^* = X_i} \leq e^{\eps(\pk)}\cdot \pr{X_i^* = -X_i} + \delta(\pk).
    \end{align*}
}

\begin{proposition}\label{prop:hard-to-guess-comp}
    \propHardToGuessComp
\end{proposition}





\remove{
\Enote{Chao's old statement:}
\begin{lemma}\label{lem:distance-I-old}
        Let $R \la \zo^n$. 
	For any function $f:\{0,1\}^n\rightarrow \{0,1\}$ and $\alpha<0.01$, it holds that
	$$
	\Pr_{i\la[n]}\left[\: \size{\:\mathbb{E}[f(R) \mid R_i = 0]-\mathbb{E}[f(R) \mid R_i = 1]\:}\geq \alpha\right]\leq \frac{2+2\log(\frac{1}{\alpha})}{n\alpha^2}.
	$$
\end{lemma}
\begin{proof}
	Define $S_1=\{r \in \zo^n \colon f(r)=1\}$. Then for any $i\in[n]$, we have
	$$
	\begin{array}{rl}
		\size{\mathbb{E}[f(R) \mid R_i = 0]-\mathbb{E}[f(R) \mid R_i = 1]}
		&=\size{\Pr[R\in S_1|R_i=0]-\Pr[R\in S_1|R_i=1]}\\
		&=\size{\frac{\Pr[R_i=0|R\in S_1]\cdot\Pr[R\in S_1]}{\Pr[R_i=0]}-\frac{\Pr[R_i=1|R\in S_1]\cdot\Pr[R\in S_1]}{\Pr[R_i=1]}}\\
		&=\frac{2\size{S_1}}{2^n}\size{\Pr[R_i=0|R\in S_1]-\Pr[R_i=1|R\in S_1]}
	\end{array}
	$$
	When $|S_1|\leq \alpha\cdot 2^{n-1}$, we have $\size{\mathbb{E}[f(R) \mid R_i = 0]-\mathbb{E}[f(R) \mid R_i = 1]}\leq\frac{2\size{S_1}}{2^n}\leq \alpha$ for any $i\in[n]$. Hence, in the following, we assume $|S_1|> \alpha\cdot 2^{n-1}$.

	%Define $I_{bad}=\{i|\size{\Pr[R_i=0|R\in S_1]-\Pr[R_i=1|R\in S_1]}>2\alpha\}$ and $k=\size{I_{bad}}$, then for any $i\notin I_{bad}$, we have 
    %$$
    %\begin{array}{rl}
    %    2\alpha&\geq \size{\Pr[R_i=0|R\in S_1]-\Pr[R_i=1|R\in S_1]}\\
    %    &=\size{\frac{\Pr[R\in S_1|R_i=0]\cdot\Pr[R_i=0]}{\Pr[R\in S_1]}-\frac{\Pr[R\in S_1|R_i=1]\cdot\Pr[R_i=1]}{\Pr[R\in S_1]}}\\
    %    &=\size{\Pr[R\in S_1|R_i=0]-\Pr[R\in S_1|R_i=1]}\cdot\frac{1}{2\Pr[R\in S_1]}\\
    %    &\geq \size{\mathbb{E}[f(R) \mid R_i = 0]-\mathbb{E}[f(R) \mid R_i = 1]}\cdot \frac{1}{2},
    %\end{array}
    %$$ 
    %where the last inequality is because $\Pr[R\in S_1]\leq 1$. So that $\size{\mathbb{E}}[f(R) \mid R_i = 0]-\mathbb{E}[f(R) \mid R_i = 1]\leq %4\alpha$.
    Define $I_{bad}=\{i \colon \size{\Pr[R_i=0|R\in S_1]-\Pr[R_i=1|R\in S_1]} \geq 2\alpha\}$ and $k=\size{I_{bad}}$, and denote $I_{bad}=\{i_1,\dots,i_k\}$. Define $(X_{i_1}, \ldots X_{i_k}) = (R_{i_1},\dots,R_{i_k})\mid_{R \in S_1}$. 
    Consider the min-entropy
	$$
	\begin{array}{rl}
		H_{min}(X_{i_1},\dots,X_{i_k})&\leq H(X_{i_1},\dots,X_{i_k})\\
		&\leq \sum_{j=1}^k H(X_{i_j})\\
		&\leq k\cdot \left(-(\frac{1}{2}+2\alpha)\cdot\log(\frac{1}{2}+2\alpha)-(\frac{1}{2}-2\alpha)\cdot\log(\frac{1}{2}-2\alpha)\right)\\
            &=k\cdot \left(-(\frac{1}{2}+2\alpha)\cdot(\log(1+4\alpha)-1)-(\frac{1}{2}-2\alpha)\cdot(\log(1-4\alpha)-1)\right)\\
            &=k\cdot \left(1-(\frac{1}{2}+2\alpha)\cdot\log(1+4\alpha)-(\frac{1}{2}-2\alpha)\cdot\log(1-4\alpha)\right),
		
	\end{array}
	$$
	where $H_{min}(Y)$ is the minimum entropy of $Y$ and $H(Y)$ is the Shannon entropy of $Y$.\Enote{add to preliminaries.}
        The third inequality holds since by the definition of $I_{bad}$, for every $j \in [k]$ it holds that $\size{\pr{X_{i_j} = 1}-\pr{X_{i_j} = 0}} > 2\alpha$, and therefore $H(X_{i_j}) \leq H(1/2 + 2\alpha)$\Enote{define}.
	
	Therefore, there exists $b_1,\dots,b_k\in\{0,1\}$, such that 
	
	\begin{align}\label{eq:min-entropy-result}
		\Pr\left[(R_{i_1},\ldots,R_{i_k}) = (b_1,\ldots,b_k) \mid R\in S_1\right]
		&= \pr{(X_{i_1},\ldots,X_{i_k}) = (b_1,\ldots,b_k)}\\
		&= 2^{-H_{min}(X_{i_1},\dots,X_{i_k})}\nonumber\\
		&\geq 2^{k\cdot \left(-1+(\frac{1}{2}+2\alpha)\cdot\log(1+4\alpha)+(\frac{1}{2}-2\alpha)\cdot\log(1-4\alpha)\right)}.\nonumber
	\end{align}
	
	Let $S_{bad}=\{r \in \zo^n  \colon \set{(r_{i_1},\ldots,r_{i_k}) = (b_1,\ldots,b_k)} \land \set{r\in S_1}\}$.
	It holds that
	\begin{align*}
		|S_{bad}|
		&= \size{S_1} \cdot \Pr\left[(R_{i_1},\ldots,R_{i_k}) = (b_1,\ldots,b_k) \mid R\in S_1\right]\\
		&\geq \alpha\cdot 2^{n-1}\cdot2^{k\cdot \left(-1+(\frac{1}{2}+2\alpha)\cdot\log(1+4\alpha)+(\frac{1}{2}-2\alpha)\cdot\log(1-4\alpha)\right)},
	\end{align*} 
	where the inequality holds by \cref{eq:min-entropy-result} and since $\size{S_1} \geq \alpha\cdot 2^{n-1}$.
	Notice that any string in $S_{bad}$ depends on at most $n-k$ bits. It implies that $|S_{bad}|\leq 2^{n-k}$. Therefore, we have
	$$
	\begin{array}{rl}
		&2^{n-k}\geq \alpha\cdot 2^{n-1}\cdot2^{k\cdot \left(-1+(\frac{1}{2}+2\alpha)\cdot\log(1+4\alpha)+(\frac{1}{2}-2\alpha)\cdot\log(1-4\alpha)\right)} \\
		\Rightarrow& n-k \geq \log \alpha+n-1+k\cdot \left(-1+(\frac{1}{2}+2\alpha)\cdot\log(1+4\alpha)+(\frac{1}{2}-2\alpha)\cdot\log(1-4\alpha)\right)\\
		\Rightarrow& 1-\log \alpha \geq k\cdot((\frac{1}{2}+2\alpha)\cdot\log(1+4\alpha)+(\frac{1}{2}-2\alpha)\cdot\log(1-4\alpha))\\
		\Rightarrow& 1-\log \alpha \geq k\cdot(4\alpha\cdot\log(1+4\alpha)+(\frac{1}{2}-2\alpha)\cdot\log(1-16\alpha^2))\\
        \Rightarrow& 1-\log\alpha \geq k\cdot(15.9\alpha^2-8\alpha^2+32\alpha^3)=k\cdot(7.9\alpha^2+32\alpha^3)>0.5k\alpha^2\\
		\Rightarrow& k\leq \frac{2-2\log \alpha}{\alpha^2} = \frac{2+2\log (1/\alpha)}{\alpha^2},
	\end{array}
	$$
	Where the third transition holds since 
	\begin{align*}
		\lefteqn{(\frac{1}{2}+2\alpha)\cdot\log(1+4\alpha)+(\frac{1}{2}-2\alpha)\cdot\log(1-4\alpha)}\\
		&= 4\alpha\cdot\log(1+4\alpha) + (\frac{1}{2}-2\alpha)\paren{\log(1+4\alpha)+\log(1-4\alpha)}\\
		&= 4\alpha\cdot\log(1+4\alpha)+(\frac{1}{2}-2\alpha)\cdot\log(1-16\alpha^2),
	\end{align*}
	and the forth transition holds since $4\alpha\cdot\log(1+4\alpha)+(\frac{1}{2}-2\alpha)\cdot\log(1-16\alpha^2) > 15.9\alpha^2-8\alpha^2+32\alpha^3$ for $\alpha < 0.01$.
	Thus, we conclude that 
	$$
	\Pr_{i\la[n]}\left[\size{\mathbb{E}[f(R) \mid R_i=0]-\mathbb{E}[f(R) \mid R_i = 1]}\geq \alpha\right]\leq \frac{k}{n}\leq \frac{2+2\log (1/\alpha)}{n\alpha^2}.
	$$
\end{proof}
}


\subsection{Channels and Two-Party Protocols}\label{sec:protocol}

\paragraph{Channels.}A channel is simply a distribution of a pair of tuples defined as follows. 
\begin{definition}[Channels]\label{def:channel} A {\sf channel} $C_{(X,U)(Y,V)}$ of size $\isize$ over alphabet $\Sigma$ is a probability distribution over $(\Sigma^\isize \times\zo^\ast) \times(\Sigma^\isize \times\zo^\ast)$. The ensemble $C_{(X,U)(Y,V)}= \set{C_{(X_\pk,U_\pk)(Y_\pk,V_\pk)}}_{\pk\in \N}$ is an $\isize$-size channel ensemble, if for every $\pk\in \N$, $C_{(X_\pk,U_\pk)(Y_\pk,V_\pk)}$ is an $\isize(\pk)$-size channel. %We denote a channel of size one by a \emph{single-bit} channel. 
We refer to $X$ and $Y$ as the {\sf local outputs}, and to $U$ and $V$ as the {\sf views}.	
\end{definition}

We view a  channel as the experiment in which there are two parties $\Ac$ and $\Bc$.  Party $\Ac$ receives ``output'' $X$ and ``view'' $U$, and party $\Bc$ receives ``output'' $Y$ and ``view'' $V$. Unless stated otherwise, the channels we consider are over the alphabet $\Sigma = \oo$. We naturally identify channels with the distribution that characterizes their output.








\subsubsection{Two-Party Protocols}

A two-party protocol $\Pi=(\Ac,\Bc)$ is \ppt if the running time of both parties is polynomial in their input length. We let $\Pi(x,y)(z)$ or $(\Ac(x),\Bc(y))(z)$ denote a random execution of $\Pi$ on a common input $z$, and private inputs $x,y$.%We assume \wlg that a protocol has a common output (part of its transcript).\Jnote{This is not really the case we consider in this paper..}

\begin{definition}[Oracle-aided protocols]\label{def:ChannelAidedProtocol}
	In a two-party protocol $\Pi$ with oracle access to a {\sf protocol} $\Psi$, denoted $\Pi^\Psi$, the parties make use of the \textit{next-message function} of $\Psi$.\footnote{The function that on a partial view of one of the parties, returns its next message.} In a two-party protocol $\Pi$ with oracle access to a {\sf channel} $C_{Z W}$, denoted $\Pi^C$, the parties can jointly invoke $C$ for several times. In each call, an independent pair $(z,w)$ is sampled according to $C_{Z W}$, one party gets $z$, the other gets $w$.
\end{definition}


\begin{definition}[The channel of a protocol]\label{def:ChannlOfProtocol}
	For a no-input two-party protocol $\Pi= (\Ac,\Bc)$, we associate the channel $C_\Pi$, defined by $\C_\Pi= C_{(X, U),(Y, V)}$, where $X$ and $Y$ are the local outputs of $\Ac$ and $\Bc$ (respectively) and
	$U$ and $V$ are the local views of $\Ac$ and $\Bc$ (respectively).
    
	For a two-party protocol $\Pi$ that gets a security parameter $1^\pk$ as its (only, common) input, we associate the channel ensemble $ \set{C_{\Pi(1^\pk)}}_{\pk\in \N}$. 
\end{definition}

\begin{definition}[$(\alpha,\gamma)$-Accurate channel]\label{def:accurate-func}
	A channel $C = C_{(X, U),(Y, V)}$ is {\sf $(\alpha,\gamma)$-accurate for the function $f$}, if $\ppr{C}{\size{\out(V)-f(X,Y)}\leq \alpha}\ge \gamma$, where $\out(V)$ is the designated output.
    A channel ensemble $C_{(X, U),(Y, V)}= \set{C_{(X_\pk, U_\pk),(Y_\pk, V_\pk)}}_{\pk\in \N}$ is  $(\alpha,\gamma)$-accurate for  $f$ if $C_{(X_\pk, U_\pk),(Y_\pk, V_\pk)}$ is $(\alpha(\pk),\gamma(\pk))$-accurate for $f$, for every $\pk \in \N$.
\end{definition}

\subsubsection{Differentially Private Channels}\label{sec:DPChannel}
Differentially private channels are naturally defined as follows:
\begin{definition}[Differentially private channels]\label{def:DPChannel}
	An $n$-size channel $C = C_{(X, U),(Y, V)}$ with $X, Y$ over $\oo^n$ 
	is {\sf$(\eps,\delta)$-differentially private} (denoted $(\eps,\delta)$-$\DP$) if for every $x \in \Supp(X)$ there exists an $n$-size $(\eps,\delta)$-$\DP$ mechanisms $\Mc_x$ such that $(X,Y,U) \equiv (X,Y,\Mc_X(Y))$, and for every $y \in \Supp(Y)$ there exists an $n$-size $(\eps,\delta)$-$\DP$ mechanisms $\Mc_y'$ such that $(X,Y,V) \equiv (X,Y,\Mc_Y'(X))$. In addition, we say that the channel is \emph{uniform} if $X$ and $Y$ are independent random variables uniformly distributed in $\oo^n$. 
\end{definition}

\begin{definition}[Computational differentially private channels]\label{def:CDPChannel}
	An $n$-size channel ensemble $C = \set{C_{(X_\pk, U_\pk),(Y_\pk, V_\pk)}}_{\pk\in\N}$ with $X_\pk, Y_\pk$ over $\oo^n$ 
	is {\sf$(\eps,\delta)$-computationally differentially private} (denoted $(\eps,\delta)$-$\CDP$) if for every ensemble $\set{x_\pk \in \Supp(X_\pk)}_{\pk\in\N}$ there exists an $n$-size $(\eps,\delta)$-\CDP mechanisms ensemble $\set{\Mc_{x_\pk}}_{\pk\in\N}$ such that $(X_\pk,Y_\pk,U_\pk) \equiv (X_\pk,Y_\pk,\Mc_{X_\pk}(Y_\pk))$, for every $\pk\in\N$, and for every ensemble $\set{y_\pk \in \Supp(Y_\pk)}_{\pk\in\N}$ there exists an $n$-size $(\eps,\delta)$-$\CDP$ mechanisms ensemble $\set{\Mc'_{y_\pk}}_{\pk\in\N}$ such that $(X_\pk,Y_\pk,V_\pk) \equiv (X_\pk,Y_\pk,\Mc_{Y_\pk}'(X_\pk))$ for every $\pk\in \N$. In addition, we say that the channel is \emph{uniform} if $X_\pk$ and $Y_\pk$ are independent random variables uniformly distributed in $\{\pm 1\}^n$ for all $\pk\in\N$.
\end{definition}




% \begin{lemma}~\label{lem:dp-sv-source}
% 	Let $P$ be an $\varepsilon$-DP randomized protocol. Let $X$ and $Y$ be independent random variables uniformly distributed in $\{\pm 1\}^n$ and let random variable $\Pi(X,Y)$ denote the transcript of running $P(X,y)$. Then for every $\pi\in Supp(\Pi)$, the random variables corresponding to the inputs conditioned on transcript $\pi$, $X_\pi$ and $Y_\pi$, are independent $e^{-\varepsilon}$-strong SV source.
% \end{lemma}





\subsubsection{Weak Erasure Channel (\WEC)}

\begin{definition}[\WEC]\label{def:WEC}
	A channel $((O_A,V_A), (O_B,V_B))$ with $O_A \in \set{0,1}$ and $O_B \in \set{0,1,\bot}$ is a {\sf weak erasure channel}, denoted $(\alpha,p,q)$-$\WEC$, if:
	\begin{itemize}
		%\item $O_A\in \set{-1,1}$ and $O_B\in \set{-1,1,\bot}$.
		\item Random erasure: $\pr{O_B = \perp} = 1/2$.
		
		\item Agreement: $\pr{O_A\ne O_B\mid O_B\ne \bot}\le \alpha$.
		
		\item Secrecy:
		
		\begin{enumerate}
			\item For every algorithm $\Dc$ it holds that\label{WEC:item:A}
			\begin{align*}
				%\size{\pr{\Ac(O_A,V_A) = 1 \mid O_B \neq \perp} - \pr{\Ac(O_A,V_A) = 1 \mid O_B = \perp}} \le p
				\size{\pr{\Dc(V_A) = 1 \mid O_B \neq \perp} - \pr{\Dc(V_A) = 1 \mid O_B = \perp}} \le p
			\end{align*}
			(Alice doesn't know if $O_B = \perp$.)
			
			\item For every algorithm $\Dc$ it holds that\label{WEC:item:B}
			\begin{align*}
				\pr{\Dc(V_B) = O_A \mid O_B=\bot} \leq \frac{1+q}{2}.
			\end{align*}
			(i.e., if $O_B=\bot$, Bob don't know what is the value of $O_A$).
			
			%\item $SD((O_A U|O_B=\bot),(O_A U|O_B\ne \bot))\le p$ (The sender don't know if $O_B=\bot$).
			
			%\item $SD(V O_A|O_B=\bot,V(-O_A)|O_B=\bot)\le q$ (If $O_B=\bot$, Bob don't know what the value of $O_A$).
		\end{enumerate}
	\end{itemize}
   We say that a channel ensemble $C=\set{C_\pk}_{\pk\in N}$ is a {\sf computational weak erasure channel}, denoted $(\alpha,p,q)$-\CompWEC, if for every \ppt algorithm $\Dc$ and every sufficiently large $\pk\in\N$, $C_\pk$ satisfies the properties stated in the items above, where the secrecy property holds with respect to a \ppt algorithm $\Dc$. A protocol $\Lambda$ is said to be $(\alpha,p,q)$-$\CompWEC$, if the ensemble induces by the protocol (that is, $C=\set{C_{\Lambda(\pk)}}_{\pk\in\N}$) is $(\alpha,p,q)$-$\CompWEC$.  
\end{definition}



\subsubsection{Approximate Weak Erasure Channel (\AWEC)}\label{sec:AWEC}

\begin{definition}[\AWEC]\label{def:AWEC}
	A channel $C = ((O_A,V_A), (O_B,V_B))$ over $([-n,n] \times \zo^*) \times (([-n,n] \cup \bot)  \times \zo^*)$ is an {\sf approximate weak erasure channel}, denoted $(\ell,\alpha,p,q)$-\AWEC if:
	\begin{itemize}
		
		\item Random erasure: $\pr{O_B = \perp} = 1/2$.
		
		\item Accuracy: $\pr{\size{O_A - O_B} > \ell \mid O_B \ne \bot}\le \alpha$.
		
		\item Secrecy:
		
		\begin{enumerate}
			\item For every algorithm $\Dc$ it holds that\label{AWEC:item:A}
			\begin{align*}
				%\size{\pr{\Ac(O_A,V_A) = 1 \mid O_B \neq \perp} - \pr{\Ac(O_A,V_A) = 1 \mid O_B = \perp}} \le p
				\size{\pr{\Dc(V_A) = 1 \mid O_B \neq \perp} - \pr{\Dc(V_A) = 1 \mid O_B = \perp}} \le p
			\end{align*}
			(Alice doesn't know if $O_B=\bot$).
			
			\item For every algorithm $\Dc$ it holds that\label{AWEC:item:B}
			\begin{align*}
				\pr{\size{\Dc(V_B) - O_A} \leq 1000 \ell \mid O_B=\bot} \leq q.
			\end{align*}
			(i.e., if $O_B=\bot$, Bob can't estimate the value of $O_A$ with error $\leq 1000 \ell$).
		\end{enumerate}
	\end{itemize}
     We say that a channel ensemble $C=\set{C_\pk}_{\pk\in N}$ is a {\sf computational approximate weak erasure channel}, denoted $(\ell,\alpha,p,q)$-\CompAWEC, if for every \ppt algorithm $\Dc$ and every sufficiently large $\pk\in\N$, $C_\pk$ satisfies the properties stated in the items above. A protocol $\Gamma$ is said to be $(\ell,\alpha,p,q)$-$\CompAWEC$, if the ensemble induced by the protocol (that is, $C=\set{C_{\Gamma(\pk)}}_{\pk\in\N}$) is $(\ell,\alpha,p,q)$-$\CompAWEC$.  
\end{definition}

We will make use of the following lemma, which shows that for some choices of the parameters, \AWEC implies \WEC. The lemma is proven in \cref{sec:AWEC-to-WEC}.

\begin{lemma}\label{lemma:AWEC-to-WEC}
	For every $\ell> 0$, there exists a \ppt protocol $\Lambda = (\Pc_1,\Pc_2)$ such that given an oracle access to an $(\ell,\alpha,p,q)$-\AWEC $C$, the channel $\tilde{C}$ induced by $\Lambda^C$ is $(\alpha'=\alpha+0.001,\: p' = p ,\:  q' = 1/2 + 2(q+0.01))$-\WEC.
	Furthermore, the proof is constructive in a black-box manner:
	\begin{enumerate}
		\item There exists an oracle-aided \ppt algorithm $\Ec_1$ such that for every channel $C = ((\OA,\VA), (\OB,\VB))$ and algorithm $\Dc$ violating the \WEC secrecy property~\ref{WEC:item:A} of $\tilde{C}$, algorithm $\Ec_1^{\Dc}$ violates the \AWEC secrecy property~\ref{AWEC:item:A} of $C$.
		
		\item There exists an oracle-aided \ppt algorithm $\Ec_2$ such that for every channel $C = ((\OA,\VA), (\OB,\VB))$ and algorithm $\Dc$ violating the \WEC secrecy property~\ref{WEC:item:B} of $\tilde{C}$, algorithm $\Ec_2^{\Dc}$ violates the \AWEC secrecy property~\ref{AWEC:item:B} of $C$.
	\end{enumerate}
\end{lemma}

Since \cref{lemma:AWEC-to-WEC} is constructive, the following is an immediate corollary.
\begin{corollary}\label{cor:CompAWEC to CompWEC}
There exists an oracle aided \ppt protocol $\Lambda$, such that given a protocol $\Gamma$ that induces $(\ell,\alpha,p,q)$-\CompAWEC, it holds that $\Lambda^\Gamma$ is $(\alpha'=\alpha+0.001,\: p' = p ,\:  q' = 1/2 + 2(q+0.01))$-\CompWEC.  
\end{corollary}
\begin{proof}[Proof of \ref{cor:CompAWEC to CompWEC}]
Let $\Lambda$ be the \ppt algorithm guaranteed  by Lemma \ref{lemma:AWEC-to-WEC}. Given an $(\ell,\alpha,p,q)$-\CompAWEC protocol $\Gamma$, we define $\Lambda(\pk)={\Lambda^{\Gamma(\pk)}(\pk)}$. Assume towards a contradiction that $\Lambda$ is not a $(\alpha',p',q')$-\CompWEC. It follows that there exists a \ppt $\Dc$ that for infinity many $\pk\in\N$ contradicts one of the \WEC secrecy properties of channel ensemble $\set{C_{\Lambda(\pk)}}_{\pk\in\N}$. Fix $\pk\in\N$ for which this holds. By Lemma \ref{lemma:AWEC-to-WEC}, there exists a \ppt $\Ec^\Dc$ that for every such $\pk$  contradicts one of the secrecy properties of the channel $C_{\Gamma(\pk)}$. This implies that for infinity many $\pk\in\N$, $\Ec^\Dc$  contradict the secrecy of the channel ensemble $\set{C_{\Gamma(\pk)}}_{\pk\in\N}$, which is a contradiction since this would means that $\Gamma$ is not a $(\ell,\alpha,p,q)$-\CompAWEC.       
\end{proof}



\subsection{Oblivious Transfer (\OT)}

\paragraph{Secure Computation.}
We use the standard notion of securely computing a functionality, \cf  \cite{Goldreich04}.
\begin{definition}[Secure computation]\label{def:SFE}
	A two-party protocol {\sf securely computes a functionality $f$}, if it does so according to the real/ideal paradigm.   We add the term perfectly/statistically/computationally/non-uniform computationally, if the simulator's output is  perfect/statistical/computationally indistinguishable/  non-uniformly indistinguishable from  the real distribution.  The protocol have the above notions of security {\sf against semi-honest  adversaries}, if its security only  guaranteed to holds against an adversary that follows the prescribed protocol.   Finally, for the case of perfectly secure computation, we naturally apply the above notion also to the non-asymptotic case: the protocol with no security parameter perfectly  compute a functionality $f$.
	
	A two-party protocol {\sf securely computes a functionality ensemble $f$ with oracle to a channel $C$}, if it does so according to the above definition when the parties have access to a trusted party computing $C$. All the above adjectives naturally extend to this setting.
\end{definition}

\paragraph{Oblivious Transfer.}
The (one-out-of-two) oblivious transfer functionality is defined as follows.
\begin{definition}[oblivious transfer functionality $f_{\OT}$]\label{def:OTfunc}
	The oblivious transfer functionality over $\zo \times (\zs)^2$ is defined by  $f_{\OT} (i,(\sigma_0,\sigma_1)) = (\perp,\sigma_i)$.
\end{definition}
A protocol is $\ast$ secure OT,   for \\$\ast\in \set{\text{semi-honest statistically/computationally/computationally non-uniform}}$, if it  compute the $f_{\OT}$  functionality with $\ast$ security.





% \begin{definition}[Computational oblivious transfer, semi-honest model]
% A protocol $\Pi=(\Ac,\Bc)$ is a semi-honest 1-out-of-2 computational oblivious transfer (comp-OT) protocol if the following holds. Given a common input $1^{\pk}$, the parties $\Ac$ and $\Bc$ run the protocol $\Pi(1^\pk)$ (in an honest manner) and    
% $\Ac$ outputs $X=(m_1,m_2)\in \zo\times\zo$ and has a view $U$ and $\Bc$ outputs $Y=(i,\hat{m})\in\zo\times\zo$ and has a view $V$, and the following properties are satisfied:
% \begin{enumerate}
%     \item \textbf{Correctness:} 
%     $\pr{\hat{m}\neq m_i}<\negl(\pk).$ 
    
%     \item \textbf{A's Privacy:} For every \ppt $\Dc$ and every sufficiently large $\pk$:
%     $\pr{\Dc(V)=m_{i-1}}<(1+\negl(\pk))/2$
    
%     \item \textbf{B's Privacy:} For every \ppt $\Dc$ and every sufficiently large $\pk$:
%     $\pr{\Dc(U)=i}<(1+\negl(\pk))/2$  
% \end{enumerate}
% \end{definition}

We make use of the following useful results by Wullschleger on oblivious transfer amplification from weak channels.
\begin{theorem}[\cite{Wullschleger09}, from \WEC to statistically secure \OT]\label{thm:WEC TO OT IT}
    There exists an oracle aided protocol $\Pi$ such that the following holds: Given a $(\alpha,p,q)$-\WEC $C$, if $44(\alpha+p)\le 1-q$ then $\Pi^{C}(1^\pk)$ is a semi-honest statistically secure \OT.
\end{theorem}

The following computational version of \cref{thm:WEC TO OT IT} is implicit in \cite{Wullschleger09} and is based on the computational proof explicitly stated in \cite{Wul07} (see Section 6 in \cite{Wullschleger09} for discussion).   

\begin{theorem}[\cite{Wullschleger09,   Wul07}, from \CompWEC to computinally secure \OT]\label{thm:WEC TO OT Comp}
    There exists an oracle aided protocol $\Pi$ such that the following holds: Given a $(\alpha,p,q)$-\CompWEC protocol $\Lambda$, if $44(\alpha+p)\le 1-q$ then $\Pi^{\Lambda}$ is a semi-honest computational secure \OT.
\end{theorem}



% \begin{definition}[Computational 1-out-of-2 Oblivious Transfer, semi-honest model]
% A protocol $\Pi=(\Ac,\Bc)$ is a semi-honest 1-out-of-2 $(\eps,\alpha,\beta)$-oblivious transfer (OT) protocol if the following holds. 

% The parties $\Ac$ and $\Bc$ run the protocol (in an honest manner) and    
% $\Ac$ outputs $X=(m_1,m_2)\in \zo\times\zo$ and has a view $U$ and $\Bc$ outputs $Y=(i,\hat{m})\in\zo\times\zo$ and has a view $V$, and following properties are satisfied:
% \begin{enumerate}
%     \item \textbf{Correctness:} 
%     $\pr{\hat{m}\neq m_i}<\eps.$ 
    
%     \item \textbf{A's Privacy:} For every adversary $\Dc$:
%     $\pr{\Dc(V)=m_{i-1}}<(1+\alpha)/2$
    
%     \item \textbf{B's Privacy:} For every adversary $\Dc$: $\pr{\Dc(U)=i}<(1+\beta)/2$  
% \end{enumerate}
% \end{definition}
%\section{Related Work}

\subsection{View-Dependent Control}
View-dependent representations have a long history in computer graphics.
In his pioneering work, Rademacher proposed interpolating between \textit{key viewpoints} and associated \textit{key deformations} to manipulate 3D models~\cite{rademacher1999view}.
Other researchers have extended the idea to create 3D animation systems~\cite{10.1111:j.1467-8659.2004.00772.x}, streamline the modeling process~\cite{DBLP:journals/corr/abs-2103-15472}, and integrate physical simulation\cite{koyama2013view}.
Of particular note, Rivers et al.~\cite{rivers25Dcartoonmodels} introduced \textit{2.5D Cartoon Models}, a combination of planar meshes transformed, based upon view angle, so as to appears three dimensional.
Our work draws upon these works but is, to our knowledge, the first work to attempt to use view-dependent techniques to retarget 3D motion onto 2D characters.   

\subsection{Animation from 2D Images}

% output is still 2D
Many researchers have proposed different methods for creating animations from 2D images. Hornung et al.~\cite{Hornung2007anim2Dpicmotion} presented a method to deform a character from a photograph given user-provided joint annotations.
\textit{Toonsynth}~\cite{Dvoroznak18-SIG} and \textit{Neural Puppet}~\cite{poursaeed2020neural} both present methods to create new images of hand-drawn characters from examples.
% output is 3D model
Other researchers have proposed methods of obtaining 3D geometry from 2D sketches~\cite{igarashi2006teddy, Dvoroznak20-SA} and images~\cite{ArtiSketch,weng2019photo}.
% focus on sketches specifically
A number of works have specifically focused on childlike drawings.
Lingens et al.~\cite{lingens2020towards} proposed an evolutionary algorithm for animating children's drawings. 
\textit{MagicToon}~\cite{feng2017magictoon} creates a 3D model from childlike drawings for AR applications.
Similar to our work, Smith et al.~\cite{SmithHodgins} proposed a method for animating childlike drawings using 3D skeletal motion. 
However, the resulting animations are only suitable for use in 2D applications and the type of motions it supports are limited.

Unlike these previous works, our solution can be used in 3D contexts but does not create a 3D model. We instead relying upon a view-dependent formulation of the animated character.
\begin{figure*}[t]
\begin{center}
\includegraphics[width=.85\linewidth]{fig_overview_v3.pdf}
\end{center}
\caption{
FastAtlas Overview: In each frame, we compute charts spanning fully or partially visible triangles (a), determine texture space bounding boxes for the visible portions of the view-space projections of each chart, and tightly pack these boxes into atlases (b, here $2K \times 2K$). We simultaneously bijectively parameterize and shade the charts into their atlas boxes, obtaining high quality texture space shading (c), and use this shading to render the shaded frames (d).}
\label{fig:overview}
\label{fig:alg_overview}
\end{figure*}

\section{Overview}
\label{sec:overview}
Our work has two core contributions: a real-time, GPU-based algorithm for tight packing of general parameterized charts into compact atlases; and a real-time TSS method that
utilizes this packing.  

\paragraph*{FastAtlas Packing.}
FastAtlas runs entirely on the GPU as a series of compute shaders. It takes the bounding boxes of parameterized charts as input, and packs them into an atlas (Fig~\ref{fig:overview}b, Sec.~\ref{sec:pack}). As such, the only input it requires are the dimensions of the bounding boxes.
Its outputs are deterministic; identical input charts are packed into identical atlases. This is critical for TSS and similar applications, as it ensures that consecutive frames taken from the same camera view have the same shading. Even minute shading differences across such frames can cause sampling jitter, leading to undesirable flicker \cite{baker2012rock}. 
While prior methods such as \cite{mueller2018shading,hladky2019tessellated,hladky2021snakebinning,Neff2022MSA} cap the dimensions of the charts that can be packed as-is for a given atlas size, and scale down all charts that exceed these dimensions, we scale all charts by the same factor, and do so only when strictly necessary to achieve packing success (Figs~\ref{fig:atlas},~\ref{fig:sas_issues}). 

\paragraph*{TSS using FastAtlas.}
Our end-to-end TSS atlas generation method combines the packing method above with a novel approach for computing seamless per-frame charts. 
We define our charts as the connected components of the visible surfaces in each frame (Fig.~\ref{fig:overview}a), and efficiently compute them using a parallel union-find algorithm (Sec.~\ref{sec:visible}). Since the boundaries of these charts coincide with the contours of the rendered surface, they are {\em invisible} to the viewer. This approach 
eliminates the artifacts caused by shading discontinuities along visible seams (Fig.~\ref{fig:seams}). 

\begin{parWithWrapFigure}
\begin{wrapfigure}{l}{.27\columnwidth}%
\includegraphics[width=\linewidth]{fig_inset_view_plane.pdf}%
\end{wrapfigure}
We bijectively parametrize the {\em visible portions} of our charts by projecting them to view space (inset). This maps a constant number of texels to each pixel in the final rendered output, evenly distributing residual undersampling error across all image pixels. While conceptually straightforward, efficiently parameterizing charts containing partially visible triangles using viewspace projection is non-trivial, as the visible portions may no longer be triangular (e.g. green triangle in the inset); applying naive projection to triangles with vertices behind the camera may produce ill-posed results. Clipping triangles before projection is both computationally expensive and significantly complicates downstream operations. We avoid explicit clipping by observing that all that is required for atlas packing is the dimensions of, potentially conservative, bounding boxes of these projected visible portions. We compute such bounding boxes without explicit chart clipping by adapting a conservative screen coverage estimator \shortcite{Blinn:CalculatingScreenCoverage} (Sec.~\ref{sec:box}). We then pack the computed boxes using FastAtlas. 
\end{parWithWrapFigure}

Finally, we shade the visible portion of each chart into its corresponding atlas bounding box (Fig~\ref{fig:overview}c). 
The resulting texture is then used during rasterization as a standard texture map (Fig. ~\ref{fig:overview}d). 
Our framework is compatible with all existing approaches for texture space shading, including forward shading, raytraced illumination, or deferred shading in texture space \cite{baker:2016}. In the examples shown, we use the standard forward shading based rendering pipeline included in the G3D Innovation Engine \cite{G3D17}, a commercial grade renderer.

% This work identifies signal collapse as a critical bottleneck in one-shot neural network pruning. Performance loss in pruned networks is due to \textbf{signal collapse} in addition to the removal of critical parameters. We propose \textbf{REFLOW} (\textbf{Re}storing \textbf{F}low of \textbf{Low}-variance signals), a simple yet effective method that mitigates signal collapse without computationally expensive weight updates. By focusing on signal preservation, REFLOW highlights the importance of mitigating signal collapse in sparse networks and enables magnitude pruning to match or surpass state-of-the-art one-shot pruning methods such as CHITA, CBS, and WF.

REFLOW consistently achieves state-of-the-art accuracy across diverse architectures, restoring ResNeXt-101 from under 4.1\% to 78.9\% top-1 accuracy at 80\% sparsity on ImageNet. Its lightweight design makes it a practical solution for both research and deployment, delivering high-quality sparse models without the overhead of traditional approaches. These findings challenge the traditional emphasis on weight selection strategies and underscore the critical role of signal propagation for achieving high-quality sparse networks in the context of one-shot pruning.



\section{Experiments: Planning outperforms Heuristics}
\label{sec:experiment}

We begin our empirical demonstrations by showcasing the effectiveness of our planning framework on both synthetic and real datasets. We focus on the simplest planning algorithm, 1-step lookaheads (Algorithm~\ref{alg:complete}), and show that even basic planning can hold great promise. 
We illustrate our framework using two uncertainty quantification modules---GPs and 
\ensembles/ \ensembleplus. 

Throughout this section, we focus on evaluating the mean squared error of 
a regression model $\model$,  and develop adaptive policies that minimize uncertainty on $g(f)$ defined in~\eqref{eqn:l2-g-f}.
When GPs provide a valid model of uncertainty, 
our experiments show that our planning framework significantly outperforms other baselines. 
We further demonstrate that our conceptual framework extends to deep learning-based uncertainty quantification methods such as  \ensembleplus while highlighting computational challenges that need to be resolved in order to scale our ideas. 
For simplicity, we assume a naive predictor, i.e., $\psi(\cdot) \equiv 0$. However, we emphasize that this problem is just as complex as if we were using a sophisticated model $\psi(.)$. The performance gap between the algorithms 
primarily depends
on the level  of uncertainty in our prior beliefs.

To evaluate the performance of our algorithm, we benchmark it against several baselines. 
%Active learning baselines use an acquisition function $\ac$ to select points that have the highest   function value: $X\opt_t \in \argmax_{X \in \xpoolj{t}} \ac({X})$ at every step $t$. These methods may also need an UQ module, which we simply use the same UQ module as in our algorithm, and it  outputs $V(X)$ that measures the the uncertainty of each point $X \in \xpoolj{t}$.
Our first set of baselines are from active learning~\citep{AggarwalKoGuHaPh14}:
\\ % \noindent\textbf{Active Learning Heuristics:} 
\textbf{(1)} 
\textsf{Uncertainty Sampling (Static):}  In this approach, we query the samples for which the model is least certain about. Specifically, we estimate the variance of the latent output $f(X)$ for each $X \in \xpool$ using the UQ module and select the top-$K$ points with the highest uncertainty. \\
\textbf{(2)} \textsf{Uncertainty Sampling (Sequential):} This is a greedy heuristic that sequentially selects the points with the highest uncertainty within a batch, while updating the posterior beliefs using pseudo labels from the current posterior state. Unlike \textsf{Uncertainty Sampling (Static)}, this method takes into account the information gained from each point within batch, and hence tries to diversify the selected points within a batch. 

 
We also compare our approach to the  \textbf{(3)} \textsf{Random Sampling}, which selects each batch uniformly at random from the pool. Additionally, we compare solving the planning problem using  \textsf{REINFORCE}-based policy gradients with   $\mathsf{Smoothed\text{-}Autodiff}$ policy gradients.\footnote{Our code repository is available at
  \url{https://github.com/namkoong-lab/adaptive-labeling}.}
%Detailed experimental setups are provided in Section \ref{sec:details-experiments}.

%We repeat all experiments with 10 random seeds.




\begin{figure}[t]
\centering
\begin{minipage}[b]{0.49\textwidth}
\centering
\includegraphics[width=\textwidth, height=5cm]{figures/original_scale/Var_of_l_2_loss.pdf}
\caption{(Synthetic data) Variance of mean squared loss evaluated through the posterior belief $\mu_t$ at each horizon $t$. This is the objective that policy gradient methods like \textsf{REINFORCE} and $\ouralgo$ optimizes. 1-step lookaheads are surprisingly effective even in long horizons.}
\label{fig:var-l2-sim}
\end{minipage}
\hfill
\begin{minipage}[b]{0.49\textwidth}
\centering \includegraphics[width=\textwidth, height=5cm]{figures/original_scale/Error_of_estimated_model_l_2_loss.pdf}
\caption{(Synthetic data) Error between MSE calculated based on collected data $\mc{D}^{0:T}$ vs. population oracle MSE over $\mc{D}_{\rm eval} \sim P_X$. Reducing uncertainty over posteriors directly leads to better OOD evaluations. 1-step lookaheads significantly outperform active learning heuristics in small horizons.}
\label{fig:mean-l2-sim}
\end{minipage}
%\caption{Simulated data for GPs}
%\label{fig:both_plots}
\end{figure}

\subsection{Planning with Gaussian processes}
\label{sec:experiment-plan-GP}
We now briefly describe the data generation process for the GP experiments,  deferring a more detailed discussion of the dataset generation to Section~\ref{sec:details-experiments}. 
We use both the synthetic data and the real data to test our methodology.
For the \emph{simulated data},  we construct a setting where the general population is distributed across \emph{51 non-overlapping clusters} while the initial labeled data $\dtrain$ just comes from one cluster. In contrast, both $\dpool \defeq (\xpool,\ypool),\deval \defeq (\xeval,\yeval)$ are generated   from all the clusters. 
We begin with a low-dimensional scenario, generating a one-dimensional regression setting using a GP. %Gaussian Process (GP).
Although the data-generating process is not known to the algorithms,  we assume that the GP hyperparameters are known to all the algorithms
to ensure fair comparisons. This can be viewed as a setting where our prior is well-specified, allowing us to isolate the effects
of different policy optimization approaches
 without any concerns about the misspecified priors. We select $10$ batches, each of size $K=5$ across $T = 10$ time horizons.

To examine the robustness of our method against the distributional assumptions made  in the simulated case, we then move to a real dataset where the correct prior is not known. We simulate selection bias from the eICU dataset~\citep{PollardJoRaCeMaBa18}, which contains real-world patient data with in-hospital mortality outcomes. 
We conduct a $k$-means clustering to generate 51 clusters and then select data from those clusters. We view this to be a credible replication of practice, as severe distribution shifts are common due to selection bias in clinical labels.  To convert the binary mortality labels into a regression setting, we train a  random forest classifier and fit a GP on predicted scores, which serves as the UQ module for all the algorithms. As before, the task is to select 10 batches, each consisting of 5 samples, across 10 time horizons.

 In Figures~\ref{fig:var-l2-sim} and~\ref{fig:mean-l2-sim}, we present results for the simulated data. 
Figure~\ref{fig:var-l2-sim} shows the variance of $\ell_2$ loss, and Figure~\ref{fig:mean-l2-sim} presents the error in the estimated $\ell_2$ loss using $\mu_t$ (relative to true $\ell_2$ loss, that is unknown to the algorithm). 
As we can see from these plots, our method one-step lookahead  gives substantial improvements  over active learning baselines and random sampling. In addition,
compared to the one-step lookahead planning approach using \textsf{REINFORCE}-based policy gradients, 
we observe that $\mathsf{Smoothed\text{-}Autodiff}$-based policy gradients provide significantly more robust performance over all horizons.

In Figures~\ref{fig:var-l2-real}~and~\ref{fig:mean-l2-real}, we observe similar findings on the eICU data. We see that planning policies (\textsf{REINFORCE} and $\mathsf{Smoothed\text{-}Autodiff}$) consistently outperform other heuristics by a large margin.  Active learning baselines perform poorly in these small-horizon batched problems and can sometimes be even worse than the random search baselines.  Overall, our results show the importance of careful planning in adaptive labeling for reliable model evaluation. 

We offer some intuition as to why one-step lookahead planning may outperform other heuristic algorithms. 
 First,  \textsf{Uncertainty sampling (Static)} while myopically selects the
 top-$K$ inputs with the highest uncertainty, it fails to consider 
the overlap in information content among the ``best” instances; see \citep{AggarwalKoGuHaPh14} for more details. 
In other words,  it might acquire points from the same region with high uncertainty while failing to induce diversity among the batch.
Although \textsf{Uncertainty Sampling (Sequential)} somewhat addresses the issue of information overlap, a significant drawback of 
this algorithm
is the disconnect between the objective we aim to optimize and the algorithm. For example, it might sample from a region with high uncertainty but very low density. 

\begin{figure}[t]
\centering
\begin{minipage}[b]{0.48\textwidth}
\centering
\includegraphics[width=\textwidth, height=5cm]{figures/original_scale/Var_of_l_2_loss_real.pdf}
\caption{(Real-world eICU data) Variance of mean squared loss evaluated through the posterior belief $\mu_t$ at each horizon $t$. Even 1-step lookaheads are extremely effective planners, and auto-differentiation-based pathwise policy gradients provide a reliable optimization algorithm based on low-variance gradient estimates.}
\label{fig:var-l2-real}
\end{minipage}
\hfill
\begin{minipage}[b]{0.48\textwidth}
\centering \includegraphics[width=\textwidth, height=5cm]{figures/original_scale/Error_of_estimated_model_l_2_loss_real.pdf}
\caption{(Real-world eICU data) Error between MSE calculated based on collected data $\mc{D}^{0:T}$ vs. population oracle MSE over $\mc{D}_{\rm eval} \sim P_X$. Reducing uncertainty over posteriors directly leads to better OOD evaluations. Our method significantly outperforms active learning-based heuristics, and random sampling.}
\label{fig:mean-l2-real}
\end{minipage}
%\caption{Real data for GPs}
\end{figure}
 
%\vspace{-1.5cm}
% \begin{wrapfigure}{r}{.32\columnwidth}
%   \vspace{-.5cm} 
%   \centering
% \includegraphics[scale=.29]{figures/Var of l2l_2 loss.pdf}
%   \vspace{-0.2cm}
%   \caption{Results of GP}
% \label{fig:var-l2-gp}
%   \vspace{-0.1cm}
% \end{wrapfigure}


% Attempts have been made  in the past to address these  drawbacks heuristically  (see \citep{AggarwalKoGuHaPh14}). We give a unified computational framework while approaching the problem in a more principled manner and solving it more optimally.




\subsection{Planning with  neural network-based uncertainty quantification methods ($\ensembleplus$)}


We now provide a proof-of-concept that shows the generalizability of our conceptual framework  to the deep learning-based UQ modules, specifically focusing on $\ensembleplus$ due to their previously observed superior performance~\citep{OsbandWenAsDwIbLuRo23}. Recall that implementing our framework with deep learning-based UQ modules  requires us to retrain the model across multiple possible random actions $\bm{a}(\theta)$ sampled from the current policy $\pi_\theta$.
This requires significant computational resources, in sharp contrast to the GPs where the posteriors are in closed form and can be readily updated and differentiated. 

Due to the computational constraints, we test $\ensembleplus$ on a toy setting to demonstrate the generalizability of our framework. We consider a setting where the general population consists of four clusters, while the initial labeled data only comes from one cluster. Again we generate data using GPs.  The task is to select a batch of 2 points in one horizon. We detail the $\ensembleplus$ architecture in Section \ref{sec:details-experiments}, and we assume prior uncertainty to be large (depends on the scaling of the prior generating functions). 
The results are summarized in the Table~\ref{tab:UQ_ensemble}.

% \begin{table}[H]
% \vspace{-10pt}
% \caption{Performance under \ensembleplus as UQ module}
%     \centering
%     \begin{tabular}{|m{3cm}|m{2.5cm}|m{2cm}|} 
%     \hline
%       Algorithm   & Variance of $\loss_2$ loss estimate & Error of $\loss_2$ loss estimate  \\ \hline Random Sampling 
%          & $1710.9 \pm 1352.1$ & $8.67\pm6.62$ 
%       \\ \hline \ouralgo & $1.30 \pm 0.68$ & $0.91\pm0.25$ \\ \hline
%     \end{tabular}
%     \label{tab:UQ_ensemble}
%     %\vspace{-10pt}
% \end{table}




\begin{table}[h]
\vspace{-10pt}
\caption{Performance under \ensembleplus as the UQ module}
\centering
\begin{tabular}{|l|l|l|}
\hline
Algorithm   & Variance of $\loss_2$ loss estimate & Error of $\loss_2$ loss estimate  \\
\hline
\textsf{Random sampling} & 7129.8 $\pm$ 1027.0 & 136.2 $\pm$ 8.28 \\ \hline
\textsf{Uncertainty sampling (Static)} & 10852 $\pm$ 0.0 & 162.156 $\pm$ 0.0 \\ \hline
\textsf{Uncertainty sampling (Sequential)} & 8585.5 $\pm$ 898.9 & 144 $\pm$ 6.93 \\ \hline
\textsf{REINFORCE} & 1697.1 $\pm$ 0.0 & 45.27 $\pm$ 0.0 \\ \hline
\ouralgo & 1697.1 $\pm$ 0.0 & 45.27 $\pm$ 0.0 \\ \hline
\end{tabular}
%\caption{Comparison of different algorithms based on variance   and   error in $\ell_2$ loss estimation with Ensemble $+$ as the UQ module. Our results demonstrate that {\ouralgo} and REINFORCE outperformthe other active learning based heuristics, confirming the benefits of our MDP formulation for the adaptive labeling problem, as also demonstrated in Section 4.\\
%\footnotesize{Experimental details: We use Gaussian Processes as our data generating process, GP parameters are the same as in Section D.3.  The task is to select a batch of 2 points along one horizon.The marginal distribution $p_X$ has 4 \textit{non-overlapping} clusters. Initial data comes from one cluster, while pool and evaluation points comes from all the clusters. We have $20$ initial labeled data points, $10$ pool points, and $252$ evaluation points.  Training procedures are similar to the one in Section D.3.} }
\label{tab:UQ_ensemble}
\end{table}



% We faced  issues in scaling up these experiments which will be our focus in the future. 





% \begin{itemize}
%     \item Posteriors should be consistent. Two dimensions: even with less training,  
%     \item the inference should be  fast enough
% \end{itemize}


% Potential research directions for uncertainty quantification

% In this section we consider a simple setting We consider a simpler setting and 


% For synthetic dataset generation, we use ...... For real datasets, we use ...... We compare our methodolgy to several baselines ()    This Section is structured as follows:
% \begin{itemize}
%     \item \textbf{GPs, square loss objective} (Section \ref{}): 
%     %the broad aim of the experiments  in this section is to isolate the performance of our methodology without any concerns for the inefficiencies induced due to a mis-specified prior or imperfect posterior inference. To accomplish this we generate synthetic datasets using GPs (detailed later). We use the well specified prior (GPs - with same hyperparameter setting) as our UQ module.   
%      As GPs provide differentaible posterior inference - any errors induced due to imperfect posterior updates are also isolated. We note that under this setting
%      \item In Section\ref{} we demonstrate why our methodology performs better than other baselines - by devising various synthetic experiments ()
%     \item  \textbf{UQ Benchmarking }(Section \ref{}): Before diving into the experiments using $\ensembleplus$ and ENNs,  we showcase our benchmarking experiments in Section \ref{}. We use real datasets We observe that ENNs perform better
%      \item \textbf{Ensemble $+$}, objective: recall, accuracy
%     \item \textbf{ENN}, objective: recall, accuracy
% \end{itemize}




% In Section {}, we test 
% \subsection{Experimental details}

% \begin{itemize}
%     \item UQ methodologies - GPs, ENNs
%     \item Objectives - Recall,  ATE
%     \item Datasets - ATE-synthetic datasets, Recall-synthetic, real datasets
%     \item Baselines - 
%     \begin{itemize}
%         \item Random sampling
%         \item Active learning - Uncertainty based sampling - In regression setting almost all of the 
%         \item Myopic greedy - Greedy Batch based sampling
%         \item Policy Gradient
%     \end{itemize}
    
% \end{itemize}

% \subsection{Experiments}
%     \begin{itemize}
%     \item GPs with square loss
%     \item Benchmarking ENN
%         \item ENNs with ATE
%         \item ENNs with Recall
%     \end{itemize}

% \subsection{Benefits over other algorithms - intuition and experiments}

%Active learning - Myopic greedy / Don't rely on the objective rather some entropy version.


%%% Local Variables:
%%% mode: latex
%%% TeX-master: "main"
%%% End:

\section{Proof Sketches for Theorems \ref{thm:mse} and \ref{thm:clt}}
The analysis of two-time-scale algorithms is difficult because of the coupling between the fast- and slow-time-scale iterates.
Methods for establishing finite-time MSE bounds on TSA are often based on a decoupling technique introduced by \citet{konda2004convergence}. 
The complexity involved in the MSE analysis is further accentuated when analyzing the averaged iterates, because it involves the inner product between the averages which can be intractable. 
We observe that the Wasserstein-1 distance between errors achieved by TSA-PR and the limiting Gaussian variables can be made tractable by decomposing the TSA-PR errors as standardized martingale sums and weighted averages of the TSA following \citep{mokkadem2006convergence}. 
While \citet{mokkadem2006convergence} prove asymptotic convergence by establishing that the TSA converges almost surely to zero and that the martingales converge in distribution, proving the rates of convergence requires a different approach. 
By considering the Wasserstein-1 distance as the metric, we exploit the useful property in Lemma \ref{lem:slutsky} to bound the Wasserstein-1 distance with that of standardized martingale sums and terms that decay to zero in expected norm by Theorem \ref{thm:mse}. 
The martingale component converges in distribution to a non-zero limit, hence determines the limiting distribution.
It is shown that the expected error achieved by TSA dominates the bound on the Wasserstein-1 distance.
% Convergence of TTSA determines the rate of decay of the metric. 


\textbf{Proof of Theorem \ref{thm:mse}:}
Our mean square analysis starts by using the decoupling technique from \citep{konda2004convergence}, which is the cornerstone of two-time-scale analysis \citep{kaledin2020finite,haque2023tightfinitetimebounds}.
A derivation of the details is provided in Section \ref{sec:recursion}.
% The common technique of writing the second moments as a contraction leads to sub-optimal constants, whereas we show that the difference between the second moments and their asymptotic covariances is contractive. 
Recall the definition of the residues
\begin{equation}
    \hat{x}_t \coloneqq x_t - x_\infty (y_t), \quad \hat{y}_t \coloneqq y_t - y^* ,
\end{equation}
which measure the deviation from the steady-state solution $x_\infty (y)$ of the fast-time-scale system satisfying $F(x_\infty (y), y) = 0$ for every $y$.
Using a coordinate transformation 
\begin{equation}
    \tilde{x}_t = \hat{x}_t + L_t \hat{y}_t ,
\end{equation}
the recursions for $(\tilde{x}_t, \hat{y}_t)$ can then be written as updates with respect to a new system $B$ obtained by a change of basis on the system $A$:
\begin{equation}\label{eq:recursion_expression}
    \begin{split}
        \tilde{x}_{t+1} &= \tilde{x}_t - \alpha_t (B_t^{ff} \tilde{x}_t + B_t^{fs} \hat{y}_t) + \alpha_t W_t + \gamma_t (L_{t+1} + A_{ff}^{-1} A_{fs}) V_t ,
        \\
        \hat{y}_{t+1} &= \hat{y}_t - \gamma_t (B_t^{sf} \tilde{x}_t + B_t^{ss} \hat{y}_t) + \gamma_t V_t ,        
    \end{split}
\end{equation}
where the specific expression for the transformed system $B$ is deferred to Sections \ref{sec:fast_mse}--\ref{sec:slow_mse}.
When $\{L_t\}$ is defined recursively with initial condition $L_0 = 0$ as
\begin{equation}\label{eq:Lt}
    L_{t+1} \coloneqq \left(L_t - \alpha_t A_{ff} L_t + \gamma_t A_{ff}^{-1} A_{fs}(\Delta - A_{sf} L_t)\right) (I - \gamma_t (\Delta - A_{sf} L_t))^{-1}  ,
\end{equation}
the sequence $\{B_t^{fs}\}$ in Eq. \eqref{eq:recursion_expression} is zero for every $t$, effectively removing the explicit dependency of the fast-time-scale $\tilde{x}_t$ from the slow-time-scale $\hat{y}_t$.
A derivation for Eq. \eqref{eq:Lt} is provided in Section \ref{sec:recursion}
The recursion is shown to be well defined in Proposition \ref{prop:Lt_welldefined}, and the mean square error of the transformed iterate $\tilde{x}_t$ is then analyzed in Section \ref{sec:fast_mse}. 
Specifically, we show that for some constants $M_f', M_s' > 0$, 
\begin{equation}
    \begin{split}
        \lVert \mathbb{E} [\tilde{x}_{t+1} \tilde{x}_{t+1}^T ] - \alpha_{t+1} \Sigma_{ff} \rVert &\leq \left(1 - \alpha_t \frac{\mu_{ff}}{2} + \gamma_t M_f'\right) \lVert \mathbb{E} \tilde{x}_t \tilde{x}_t^T - \alpha_t \Sigma_{ff} \rVert + \mathcal{O}\left(\alpha_t \gamma_t \right)
            , 
        \\ 
        \lVert \mathbb{E} [\hat{y}_{t+1} \hat{y}_{t+1}^T ] - \gamma_{t+1} \Sigma_{ss} \rVert &\leq \left(1 - \gamma_t \frac{\mu_\Delta}{2} + \frac{\gamma_t^2}{\alpha_t} M_s'\right) \lVert \mathbb{E}\hat{y}_t \hat{y}_t^T - \gamma_t \Sigma_{ss} \rVert + \mathcal{O}\left(\gamma_t^2\right) .
    \end{split}
\end{equation}
Once this is shown, a standard induction step can be applied with the choice of step sizes in Assumption \ref{assumption:steps} to obtain Theorem \ref{thm:mse}.


The second moment of $x_t - x^*$ is then recovered using the identities $x_t - x^* = \hat{x}_t + (x_\infty (y_t) - x^*) = \hat{x}_t + H \hat{y}_t$ to get
\begin{align*}
    \mathbb{E}\left(x_t - x^*\right) \left(x_t - x^* \right)^T 
    = \mathbb{E} \tilde{x}_t \tilde{x}_t^T
    + (H - L_t ) \mathbb{E} \hat{y}_t \hat{y}_t^T (H - L_t)^T
    + \mathbb{E} \tilde{x}_t \hat{y}_t^T (H - L_t)^T .    
\end{align*}
It is shown in Sections \ref{sec:joint_mse} and \ref{sec:slow_mse} that $\lVert \mathbb{E} \tilde{x}_t \hat{y}_t^T \rVert = \mathcal{O}(\gamma_t)$ and $\lVert \mathbb{E} \hat{y}_t \hat{y}_t^T \rVert = \mathcal{O}(\gamma_t)$.
Proposition \ref{prop:Lt_welldefined} establishes that $\lVert L_t \rVert$ is uniformly bounded, from which we recover the mean square error of $x_{t+1} - x^*$ as
\begin{equation}\label{eq:fast_notransformation_mse}
    \mathrm{Tr} \mathbb{E} (x_{t+1} - x^*) (x_{t+1} - x^*) 
    \leq
    \alpha_t \mathrm{Tr} \Sigma_{ff} 
    + \mathcal{O}\left(\gamma_t \right) .
    % + \gamma_t \left(M_f (I + \Sigma_{ff}) + H \Sigma_{ss} H^T + \Sigma_{fs} H^T\right)
    % + o(\gamma_t) .
    % + \gamma_t H \left(\Sigma_{ss} -  H^T 
    % + \mathcal{O}\left(\gamma_t \mathrm{Tr} (H \Sigma_{ss} H^T + H \Sigma_{sf})
    % + \gamma_t \mathrm{Tr} (H \Sigma_{fs} H^T) \right).
\end{equation}
This completes the mean square error bounds. 


\textbf{Proof of Theorem \ref{thm:clt}:}
Without loss of generality, let $x^* = 0$ and $y^* = 0$.
In Section \ref{sec:clt}, following \citet{mokkadem2006convergence}, we then express the PR averages as a telescoped sum
\begin{equation}
    \begin{split}
        G \bar{x}_n  &= \frac{1}{n} \sum_{t=1}^n (W_t - A_{fs} A_{ss}^{-1} V_t) + \frac{1}{n} \sum_{t=1}^n \alpha_t^{-1} (x_{t} - x_{t+1}) - \frac{1}{n} \sum_{t=1}^n \gamma_t^{-1} A_{fs} A_{ss}^{-1} (y_t - y_{t+1}) \\ 
        \Delta \bar{y}_n &= \frac{1}{n} \sum_{t=1}^n (V_t - A_{sf} A_{ff}^{-1} W_t) 
        + \frac{1}{n}\sum_{t=1}^n A_{sf} A_{ff}^{-1} \alpha_t^{-1} (x_{t} - x_{t+1})
        + \frac{1}{n}\sum_{t=1}^n \gamma_t^{-1} (y_{t} - y_{t+1})  .    
    \end{split}
    \label{eq:polyak_ruppert_expression}
\end{equation}
After scaling Eq. \eqref{eq:polyak_ruppert_expression} by $\sqrt{n}$, we see that the first terms are standardized sums of martingales which converge in distribution. 
To obtain their rates of convergence, we use the quantitative bound from \citep{srikant2024CLT} on the Wasserstein-1 distance between a standardized sum of martingales and its limiting Gaussian variable. 
\begin{lemma}\label{lem:CLT}
    Let $\{N_t\}$ be a martingale difference sequence in $\mathbb{R}^d$ satisfying $\mathbb{E}N_t N_t^T = \Gamma$ and $\mathbb{E}\lVert N_t \rVert^{2 + \beta}<\infty$ for every $t \geq 1$ and some $\beta \in (0, 1)$. 
    For a standard Gaussian vector $Z$, it holds that
    \begin{equation}
        d_1 \left(n^{-1/2} \sum_{t=1}^n N_t , \Gamma^{1/2} Z \right) \leq \mathcal{O}\left(\frac{d}{1 - \beta}\right) \frac{\lVert \Gamma^{1/2} \rVert \lVert \Gamma^{-1/2}\rVert^{2 + \beta}}{n^{\beta/2}} .
    \end{equation}
\end{lemma}
Next, we relate the telescoped terms in Eq. \eqref{eq:polyak_ruppert_expression} to their respective errors in expectation as
\begin{equation}
    \begin{split}
        \mathbb{E}\lVert \sum_{t=1}^n \alpha_t^{-1} (x_t - x_{t+1}) \rVert 
            &\leq 
        \alpha_1^{-1} \mathbb{E} \lVert x_1 \rVert + \alpha_n^{-1} \mathbb{E}\lVert x_n \rVert + \alpha_1^{-1} \sum_{t=1}^n t^{a-1} \mathbb{E}\lVert x_t \rVert  
        \\
        \mathbb{E}\lVert \sum_{t=1}^n \gamma_t^{-1} (y_t - y_{t+1}) \rVert 
            &\leq 
        \gamma_1^{-1} \mathbb{E} \lVert y_1 \rVert + \gamma_n^{-1} \mathbb{E}\lVert y_n \rVert + \gamma_1^{-1} \sum_{t=1}^n t^{b-1} \mathbb{E}\lVert y_t \rVert .
    \end{split}
    \label{eq:pr_expression}
\end{equation}
The expected errors of $x_t$ and $y_t$ for every $t \leq n$ are recovered from Theorem \ref{thm:mse} and Jensen's inequality, which gives $\mathbb{E}\lVert x_t \rVert = \mathcal{O}(t^{-a/2})$ and $\mathbb{E}\lVert y_t \rVert = \mathcal{O}(t^{-b/2})$.
Combining, we conclude that the sum of telescoped sums in Eq. \eqref{eq:pr_expression} grows with $n^{a/2}, n^{b/2}$, and $n^{a-b/2}$, where the $a-b/2$ is obtained from the $\gamma_{n+1}$ dependence of $\mathbb{E}\lVert \hat{x}_n \rVert^2$ in Eq. \eqref{eq:fast_notransformation_mse}.
% on the convergence of $\mathrm{Tr} (\mathbb{E} \hat{x}_{n+1} \hat{x}_{n+1}^T - \alpha_{n+1} \Sigma_{ff})$ deduced from Theorem \ref{thm:mse}. 



To combine these results, we then use the following property of the Wasserstein-1 distance.
\begin{lemma}\label{lem:slutsky}
    % I.e. this is Lindeberg's decomposition.
    Consider two random sequences $\{X_t\}, \{Y_t\}$ such that $d_1 (X_t, X) \leq r_t$ for some random variable $X$ and $\mathbb{E}\lVert Y_t \rVert \leq r'_t$.
    Then, $d_1 (X_t + Y_t, X) \leq r_t + r'_t$.
\end{lemma}
% To finish, we use the quantitative bound for convergence of vector martingale sums to a normal distribution in Theorem 1 of \citep{srikant2024CLT}.
Theorem \ref{thm:clt} is then obtained by applying Lemma \ref{lem:slutsky} to bound the distance between $G \bar{x}_n$ and its Gaussian limit with the Wasserstein-1 distance between the martingale terms in Eq. \eqref{eq:polyak_ruppert_expression} and the expected norms in Eq. \eqref{eq:pr_expression}. 

% Applying Lemma \ref{lem:CLT} to the first term in Eq. \eqref{eq:polyak_ruppert_expression} 
% Together with Eq. \eqref{eq:polyak_ruppert_expression}, we obtain Theorem \ref{thm:clt}.
% where the bound is dominated by the mean absolute errors of the last iterates. 


To obtain Corollary \ref{cor:mae}, we then use $h(x) = \lVert x \rVert$ as the 1-Lipschitz test function for the Wasserstein-1 distance in Eq. \eqref{eq:wasserstein_definition} to obtain
\begin{equation}
    \begin{split}
        \mathbb{E}\lVert G \bar{x}_n\rVert - n^{-1/2} \mathbb{E} \lVert (G \bar{\Sigma}_{ff} G^T)^{1/2} Z_1 \rVert 
        &\leq d_1 \left(G \bar{x}_n, n^{-1/2}(G \bar{\Sigma}_{ff} G^T)^{1/2} Z_1 \right) = \mathcal{O}(n^{b/2 - 1}) ,
        \\
        \mathbb{E}\lVert \Delta \bar{y}_n \rVert - n^{-1/2} \mathbb{E} \lVert (\Delta \bar{\Sigma}_{ss} \Delta^T)^{1/2} Z_2 \rVert 
        &\leq d_1 \left(\Delta \bar{y}_n, n^{-1/2} (\Delta \bar{\Sigma}_{ss} \Delta^T)^{1/2} Z_2 \right) = \mathcal{O}(n^{b/2 - 1}) .
    \end{split}
\end{equation}
Because $b < 1$, we have that $\mathbb{E} \lVert G \bar{x}_n \rVert$ and $\mathbb{E} \lVert \Delta \bar{y}_n \rVert$ are both $\mathcal{O}(n^{-1/2})$.
% , where the inequality in the above equation uses that the function  is 1-Lipschitz and therefore is a valid test function in the definition of the Wasserstein-1 distance (Eq. \eqref{eq:wasserstein_definition}).  
The lower bound is obtained by using the test function $-\lVert x \rVert$. 
\section*{Conclusion}
This paper aims to enhance our understanding of the computational complexity of computing various Shapley value variants. We found that for various ML models --- including decision trees, regression tree ensembles, weighted automata, and linear regression --- both local and global interventional and baseline SHAP can be computed in polynomial time under HMM modeled distributions. This extends popular algorithms, such as TreeSHAP, beyond their empirical distributional scope. We also establish strict complexity gaps between the various SHAP variants (baseline, interventional, and conditional) and prove the intractability of computing SHAP for tree ensembles and neural networks in simplified scenarios. Overall, we present SHAP as a versatile framework whose complexity depends on four key factors: \begin{inparaenum}[(i)] \item model type, \item SHAP variant, \item distribution modeling approach, \item and local vs. global explanations\end{inparaenum}. We believe this perspective provides deeper insight into the computational complexity of SHAP, paving the way for future work.




%We believe that our framework provides a more intricate understanding of SHAP computation complexity across different models, distributions, and variants, paving the way for further research.

Our work opens promising directions for future research. First, expanding our computational analysis to other SHAP-related metrics, such as asymmetric SHAP~\citep{frye20} and SAGE~\citep{covert2020understanding}, would be valuable. Additionally, we aim to explore more expressive distribution classes and relaxed assumptions beyond those in Section \ref{sec:tractable} while maintaining tractable SHAP computation. Finally, when exact computation is intractable (Section \ref{sec:intractable}), investigating the approximability of SHAP metrics through approximation and parameterized complexity theory~\citep{downey2012parameterized} is an important direction.

%Our work opens several promising avenues for future research on the computational properties of explainable AI methods, with a particular focus on SHAP. First, it would be interesting to broaden the computational analysis conducted in this work to include other popular SHAP-related metrics in the literature, such as asymmetric SHAP \cite{frye20} and SAGE \cite{covert2020understanding}. Also, in the future, we aim to explore more expressive distribution classes and relaxed distributional assumptions—extending beyond those examined in Section \ref{sec:tractable} —that still yield tractable SHAP computation. Finally, when exact computation proves intractable (Section \ref{sec:intractable}), it is worthwhile to theoretically investigate the question of the approximability of computing the SHAP metrics across various configurations, through the lens of approximation and parametrized complexity theory \cite{arora2009computational}.

%This paper aims to deepen our understanding of the computational complexity involved in obtaining different Shapley value variants. We found that for a variety of ML models, including decision trees, tree ensembles for regression, weighted automata, and linear regression models — computing both local and global interventional and baseline SHAP can be done in polynomial time when distributions are modeled by HMMs. This extends the distributional scope of popular algorithms like TreeSHAP, which is limited to empirical distributions. Additionally, we demonstrate a strict complexity gap between SHAP variants, showing that interventional and baseline SHAP can be strictly easier to compute than conditional SHAP. Despite these positive results, we uncovered intractability for various SHAP variants in neural networks and tree ensembles. Finally, we provided generalized complexity relations across SHAP variants. We believe that our framework offers a deeper understanding of the complexity involved in computing SHAP across various variants, models, distributions, as well as in both local and global computations, laying the groundwork for future research.


\section*{Disclaimer}
This paper was prepared for informational purposes [“in part” if the work is collaborative with external partners]  by the Artificial Intelligence Research group of JPMorgan Chase \& Co. and its affiliates ("JP Morgan'') and is not a product of the Research Department of JP Morgan. JP Morgan makes no representation and warranty whatsoever and disclaims all liability, for the completeness, accuracy or reliability of the information contained herein. This document is not intended as investment research or investment advice, or a recommendation, offer or solicitation for the purchase or sale of any security, financial instrument, financial product or service, or to be used in any way for evaluating the merits of participating in any transaction, and shall not constitute a solicitation under any jurisdiction or to any person, if such solicitation under such jurisdiction or to such person would be unlawful.

\bibliography{bibliography}

\newpage
\appendix 
% MSE
% \section{Mean Square Analysis of TTSA Last Iterates}
\section{Auxiliary Results}
In this section, we first present the derivation for \eqref{eq:recursion_expression}.
This expression will be used to derive the finite-time bounds on the second moments of $\tilde{x}_t$ and $\hat{y}_t$. 
Some useful properties of recurring quantities will be derived, before presenting the proof of Theorems \ref{thm:mse} and \ref{thm:clt}.


\subsection{Recursion Updates}\label{sec:recursion_mse}
%The recursive expression in Eq. \eqref{eq:recursion_expression} is standard. Here, we provide an alternative derivation. 
The goal of this sub-section is to derive the expressions for $B_t$ and $L_t$ in Eqs. \eqref{eq:recursion_expression} and \eqref{eq:Lt}, which is similar to the one in \citep{konda2004convergence}. 
For convenience, recall the updates in \eqref{eq:ttsa}
\begin{equation*}
    \begin{split}
        x_{t+1} &= x_t - \alpha_t \left(A_{ff} x_t + A_{fs} y_t - W_t \right),
        \\
        y_{t+1} &= y_t - \gamma_t \left(A_{sf} x_t + A_{ss} y_t - V_t \right),
    \end{split}
\end{equation*}
and the residues $\hat{x}_t \coloneqq x_t - x_\infty (y_t)$ and $\hat{y}_t \coloneqq y_t - y^*$, where $x_\infty (y) = -A_{ff}^{-1} A_{fs} y$. 
The next steps will describe how to choose a sequence of matrices $\{L_t\}$ such that the recursion for $(\tilde{x}_{t+1}, \hat{y}_{t+1})$ with $\tilde{x}_t \coloneqq \hat{x}_t + L_t \hat{y}_t$ can be simplified, and state the expressions for the resulting system. 



Let $F, S$ be the operators corresponding to the fast- and slow-time-scale systems, i.e., $F(x, y) = A_{ff} x + A_{fs} y$ and $S(x, y) = A_{sf} x + A_{ss}y$. 
Using the definition of $x_{\infty}(y)$ we have
\begin{align*}
    F(x_{\infty}(y),y) = 0\;\text{ for all } y \quad &\text{ and }\quad S(x_{\infty}(y^*),y^*) = 0, 
\end{align*}
and we express $F$ and $S$ as a function of $\hat{x}_t$ and $\hat{y}_t$
\begin{align*}
    F(x_{t},y_{t}) &= F(x_t, y_t) - F(x_\infty (y_t), y_t) = A_{ff}\hat{x}_{t} ,\\
    S(x_t, y_t) &= S(x_t, y_t) - S(x_\infty (y_t), y_t) + S(x_\infty (y_t), y_t) - S(x_\infty (y^*), y^*)
    \\ & 
    =A_{sf} \hat{x}_t + A_{sf} x_\infty (\hat{y}_t) + A_{ss} \hat{y}_t 
    \\ & = A_{sf} \hat{x}_t + \Delta \hat{y}_t .
\end{align*}
% Thus, we have the following relations
% \begin{align*}
% F(x_{t},y_{t}) = A_{ff}\hat{x}_{t}\quad &\text{ and }\quad S(x_{t},y_{t})  = A_{sf} \hat{x}_t + \Delta \hat{y}_t.\\
% F(x_{\infty}(y),y) = 0\;\text{ for all } y \quad &\text{ and }\quad S(x_{\infty}(y^*),y^*) = 0. 
% \end{align*}
First, we obtain a recursion for the fast variable $\hat{x}_t$
% First, we will derive the recursion for $\hat{z}_t = (\hat{x}_t, \hat{y}_t)$ using $x_\infty (y) = -A_{ff}^{-1} A_{fs} y$ and the relations above.
% , by linearity and the definitions of $x_\infty$ and $y^*$,
% \begin{align*}
%     F(x_t, y_t) &= F(x_t, y_t) - F(x_\infty (y_t), y_t) = A_{ff} \hat{x}_t, 
% \\ 
%     S(x_t, y_t) &= S(x_t, y_t) - S(x_\infty (y_t), y_t) + S(x_\infty (y_t), y_t) - S(x_\infty (y^*), y^*)
%     \\ & = A_{sf} \hat{x}_t + A_{sf} x_\infty (\hat{y}_t) + A_{ss} \hat{y}_t
%     \\  & = A_{sf} \hat{x}_t + \Delta \hat{y}_t .
% \end{align*}
% Elementary algebra gives
% Consider the fast iterate 
\begin{align*}
    \hat{x}_{t+1} 
    &= x_{t+1} - x_\infty (y_{t+1})
    \\ & 
    = \hat{x}_t - x_\infty (y_{t+1} - y_t) - \alpha_t F(x_t, y_t) + \alpha_t W_t
    \\ & 
    = \hat{x}_t - \gamma_t A_{ff}^{-1} A_{fs} S(x_t, y_t) - \alpha_t A_{ff} \hat{x}_t + \alpha_t W_t - \gamma_t A_{ff}^{-1} A_{fs} V_t
    \\ & = 
    (I - \alpha_t A_{ff} - \gamma_t A_{ff}^{-1} A_{fs} A_{sf}) \hat{x}_t - \gamma_t A_{ff}^{-1} A_{fs} \Delta \hat{y}_t + \alpha_t W_t - \gamma_t A_{ff}^{-1} A_{fs} V_t .
    \numberthis \label{eq:fast_untransformed}
\end{align*}
Similarly, the slow-time-scale iterate satisfies the recursion
\begin{align*}
    \hat{y}_{t+1} &= y_{t+1} - y^*
    = \hat{y}_t - \gamma_t S(x_t, y_t) + \gamma_t V_t = 
    (I - \gamma_t \Delta) \hat{y}_t - \gamma_t A_{sf} \hat{x}_t + \gamma_t V_t .
    \numberthis \label{eq:slow_transformed}
\end{align*}
Equations \eqref{eq:fast_untransformed} and \eqref{eq:slow_transformed} can be combined to write the update rule for $\hat{z}_t = (\hat{x}_t, \hat{y}_t)$ in terms of a new system $\hat{A}_t$ and $Q_t$, where
\begin{align*}
    &\hat{z}_{t+1} = \hat{z}_t - \hat{A}_t \hat{z}_t + Q_t N_t,
    \\
    & \hat{A}_t = \begin{pmatrix}
        \alpha_t A_{ff} + \gamma_t A_{ff}^{-1} A_{fs} A_{sf} & \gamma_t A_{ff}^{-1} A_{fs} \Delta 
        \\ 
        \gamma_t A_{sf} & \gamma_t \Delta 
    \end{pmatrix} ,
    \quad 
    Q_t = \begin{pmatrix}
        \alpha_t I & - \gamma_t A_{ff}^{-1} A_{fs} \\ 
        0 & \gamma_t I
    \end{pmatrix} .
\end{align*}
Second, we design a sequence of matrices $\{L_t\}$ such that the recursion for $\tilde{x}_{t} = \hat{x}_{t} + L_{t}\hat{y}_{t}$ does not depend on $\hat{y}_t$ explicitly. 
The choice of $L_t$ determines the coordinate transformation
\begin{align*}
    \tilde{z}_t = P_t \hat{z}_t, \quad P_t = \begin{pmatrix}
        I & L_t \\ 0 & I
    \end{pmatrix} .
\end{align*}
For any $L_t$, it holds that
\begin{equation}\label{eq:ztilde_derivation}
    \tilde{z}_{t+1} = P_{t+1} (I - \hat{A}_t) P_t^{-1} \tilde{z}_t + P_{t+1} Q_t N_t .
\end{equation}
The matrix $L_t$ is to be chosen such that $P_{t+1} P_t^{-1} - P_{t+1} \hat{A}_t P_t^{-1}$ is lower triangular so that $\tilde{x}_{t+1}$ is decoupled from $\hat{y}_t$.
% , thus removing the dependency on $\hat{y}_t$ of $\tilde{x}_{t+1}$. 
Using $\hat{A}_t^{ff}$ and so on to denote the blocks in $\hat{A}_t$, we then have that the upper-right component of $P_{t+1} \hat{A}_t P_t^{-1}$ is given by 
\begin{align*}
    L_{t+1} - L_t
    + (\hat{A}_t^{ff} + L_{t+1} \hat{A}_t^{sf}) L_t - (\hat{A}_t^{fs} + L_{t+1} \hat{A}_t^{ss}) = 0 
    .
\end{align*}
The above calculation is obtained by matrix multiplication and inverting $P_t$, which is equated to zero to achieve the decoupling. 
Substituting the expressions for $\hat{A}_t$ in the identity gives
\begin{align*}
    L_{t+1} (I + \hat{A}_t^{sf} L_t - \hat{A}_t^{ss}) = \hat{A}_t^{fs} + (I - \hat{A}_t^{ff}) L_t,
\end{align*}
which yields the recursion for $L_t$ in \eqref{eq:Lt}. Given $L_{t}$, we finally derive the system matrix $B_t$ in Eq. \eqref{eq:recursion_expression}.
The expression for $B_t^{ff}$ can be easily obtained from Eq. \eqref{eq:ztilde_derivation}, where $B_t^{fs} = 0 $ by design, as
\begin{equation*}
    B_t^{ff} = A_{ff} + \frac{\gamma_t}{\alpha_t} (L_{t+1} + A_{ff}^{-1} A_{fs}) A_{sf} .
\end{equation*}
In addition, for any choice of $L_t$, $\hat{y}_{t+1}$ is related to $(\tilde{x}_t, \hat{y}_t)$ as follows:
\begin{equation*}
    \hat{y}_{t+1} = (I - \gamma_t (\Delta - A_{sf} L_t)) \hat{y}_t - \gamma_t A_{sf} \tilde{x}_t + \gamma_t V_t,
\end{equation*}
which yields the derivation for $B_t^{sf} = A_{sf}$ and $B_t^{ss} = \Delta - A_{sf} L_t$, where
\begin{equation}\label{eq:slow_derivation}
    \hat{y}_{t+1} = \hat{y}_t - \gamma_t (B_t^{sf} \tilde{x}_t + B_t^{ss} \hat{y}_t - V_t ) .
\end{equation}
In summary, the choice of $L_t$ in Eq. \eqref{eq:Lt} results in the updates
\begin{equation}\label{eq:fast_slow_updates}
    \begin{split}
    \tilde{x}_{t+1} &= (I - \alpha_t A_{ff}) \tilde{x}_t - \gamma_t \delta_t A_{sf} \tilde{x}_t + \alpha_t W_t + \gamma_t \delta_t V_t, 
    \\ 
    \hat{y}_{t+1} &= (I - \gamma_t \Delta) \hat{y}_t + \gamma_t A_{sf} L_t \hat{y}_t - \gamma_t A_{sf} \tilde{x}_t + \gamma_t V_t.     
    \end{split}
\end{equation}
% which can be rearranged to obtain 
% \begin{align*}
%     L_{t+1}  = (\hat{A}_{ff} - \hat{A}_{fs}) (\hat{A}_{sf} L_t + \hat{A}_{ss})^{-1}
% \end{align*}



\subsection{Usage of Assumptions \ref{assumption:first}--\ref{assumption:last}}\label{sec:recursion}
Here we state a few elementary properties that will be used to prove Theorems \ref{thm:mse} and \ref{thm:clt}.
% useful in the analysis of mean square analysis.
We will express the difference between finite-time second moments and the asymptotic covariance as a contraction, and so we require that the contraction is well defined for symmetric matrices, rather than for definite matrices.  
The contraction that appears often is as follows. Given any matrix $A$, we denote $\|A\|$ is the matrix norm of $A$. Let $A$ satisfy $\mu I \preceq A + A^T \preceq \nu I$. Given any symmetric matrix $R$, we have
\begin{align*}
    R - \alpha (A R + R A^T)
    &= (I - \alpha A) R (I - \alpha A)^T - \alpha^2 A R A^T,
\end{align*}
which implies 
\begin{align*}
    % R - \alpha (A R + R A^T)
    % &= (I - \alpha A) R (I - \alpha A)^T - \alpha^2 A R A^T
    % \\
    % \Rightarrow 
    \lVert R - \alpha (A R + R A^T) \rVert &\leq \lVert I - \alpha A \rVert^2 \lVert R \rVert + \alpha^2 \lVert A R A^T \rVert.
\end{align*}
In addition, we have $\lVert A \rVert \leq \nu / 2$, which gives
\begin{align*}
    \lVert I - \alpha A \rVert^2 = \sup_{x \neq 0} \left(1 - \alpha \frac{x^T (A + A^T) x}{\lVert x \rVert^2} + \alpha^2 \frac{\lVert A x \rVert^2}{\lVert x \rVert^2}\right) 
    \leq 1 - \alpha \mu + \alpha^2 \lVert A \rVert^2.
\end{align*}
Thus, for $\alpha \leq \mu/\nu^2$ we obtain
\begin{equation}\label{eq:contraction}
    \lVert R - \alpha (A R + R A^T) \rVert 
    \leq 
    \left(1 - \alpha \mu + \alpha^2 \frac{\nu^2}{2}\right) \lVert R \rVert 
    \leq \left(1 - \alpha \frac{\mu}{2}\right)\lVert R \rVert     .
\end{equation}
% For a symmetric matrix $S$ and $\mu I \preceq A + A^T \preceq \nu I$, we have that 
% \begin{align*}
%     S - \alpha (AS + S A^T) = (I/2 - \alpha A) S + S (I/2 - \alpha A)^T ,
% \end{align*}
% and using Weyl's inequality to deduce that $\lambda_{\max} (A) \leq \nu/2$ and $\lVert I/2 - \alpha A\rVert \leq 1 - \alpha $

% When $\nu I \succeq A + A^T \succeq \mu I$ and $X \succeq 0$, we have by Lyapunov stability that 
% \begin{align*}
%     \nu X \succeq A X + X A^T \succeq \mu X .
% \end{align*}
% This can be seen by rearranging $(A + \mu/2 I) X + X (A + \mu/2 I)^T \succ 0$.
% Using this, we have that for a symmetric matrix $S$,
% \begin{align*}
%     \lVert S - (A S + S A^T)\rVert 
% \end{align*}
% {\color{red}Finish this part up and see if I'm using it correctly. }


Next, the choice of step size in Assumption \ref{assumption:steps} implies that
\begin{align*}
    \alpha_t - \alpha_{t+1} \leq \frac{\alpha_{t+1}}{t} ,
\end{align*}
obtained from the elementary inequality $(\frac{t+1}{t})^a \leq (1+\frac{1}{t})$ for every $a \in (0, 1]$. 
Moreover, this implies $\alpha_{t+1}^{-1} - \alpha_t^{-1} \leq (\alpha_{t} t)^{-1}$ and is also true for the sequence $\{\gamma_t\}$, which is tight for the range of values $a, b$ we consider. 

Finally, we derive a recursive bound for the absolute error of the second moments of fast and slow iterates from their asymptotic covariances.
After an induction step, the dominant rates are then obtained using Lemma 14 \citep{kaledin2020finite}, which states that for $\alpha_1 \mu_{ff}/4 \leq 1$ and constants
\begin{align*}
    \xi = 1 + \max \left\{\alpha_1 \frac{\mu_{ff}}{4}, \gamma_1 \frac{\mu_\Delta}{16} \right\},
    \quad 
    \xi' = \frac{8}{\mu_{ff}}\max \left\{\xi, \frac{\mu_{ff}}{4\mu_{\Delta}} \xi, 2 \xi^3
        \right\},
\end{align*}
it holds that
\begin{equation}
    \begin{split}
        \sum_{t=1}^n \gamma_t \prod_{j=t+1}^n \left(1 - \frac{\mu_{ff}}{4} \alpha_j\right) 
        \leq \xi' \frac{\gamma_{n}}{\alpha_n}
        \\
        \sum_{t=1}^n \alpha_t \gamma_t \prod_{j=t+1}^n \left(1 - \frac{\mu_{ff}}{4} \alpha_j\right) 
        \leq \xi' \gamma_{n}
        ,
        \\
        \sum_{t=1}^n \frac{\alpha_t}{t} \prod_{j=t+1}^n \left(1 - \frac{\mu_{ff}}{4} \alpha_j \right) 
        \leq 
        \xi' \frac{1}{n} .
    \end{split} 
    \label{eq:induction_size_equation}
\end{equation}
Variants of the above inequalities can also be deduced to be true, for example when $\gamma_t$ in the first inequality is replaced by $\gamma_t^2$. 


\subsection{Size of Linear Transformation}
\begin{proposition}\label{prop:Lt_welldefined}
    The sequence $\{L_t\}$ is well-defined and converges to zero at rate $\mathcal{O}(\gamma_t/\alpha_t)$.
\end{proposition}
The recursive definition for $L_{t+1}$ in Eq. \eqref{eq:Lt} is equivalently expressed as 
\begin{align*}
    L_{t+1} (I - \gamma_t \Delta + \gamma_t A_{sf} L_t) 
    &= \left(I - \alpha_t A_{ff} - \gamma_t A_{ff}^{-1} A_{fs} A_{sf} \right) L_t + \gamma_t A_{ff}^{-1} A_{fs} \Delta 
    \\ & 
    = \left(I - \alpha_t A_{ff}\right) L_t + \gamma_t A_{ff}^{-1} A_{fs}(\Delta - A_{sf} L_t) .
\end{align*}
Denoting $B_t^{ss} = \Delta - A_{sf} L_t$ as before, 
\begin{equation}
    L_{t+1} (I - \gamma_t B_t^{ss}) = (I - \alpha_t A_{ff}) L_t + \gamma_t A_{ff}^{-1} A_{fs} B_t^{ss} .
\end{equation}
It was shown in Lemma 18 of \citep{kaledin2020finite} that $I - \gamma_t B_t^{ss}$ is non-singular and that $\lVert L_t \rVert$ is uniformly bounded.
Therefore, $ \max_{t \leq n} \prod_{j=t+1}^n \lVert (I - \gamma_t B_t^{ss})^{-1} \rVert$ is uniformly bounded by some constant $K_{ss}$ and we have
\begin{align*}
    \lVert L_{t+1} \rVert 
    &\leq \lVert I - \alpha_t A_{ff} \rVert \lVert L_t (I - \gamma_t B_t^{ss})^{-1} \rVert + \gamma_t \lVert A_{ff}^{-1} A_{fs} B_t^{ss} \rVert 
    \\ & \leq 
    \sqrt{1 - \alpha_t \frac{\mu_{ff}}{2}} \lVert (I - \gamma_t B_n^{ss})^{-1}\rVert \lVert L_t \rVert 
    + \gamma_t \lVert A_{ff}^{-1} A_{fs} B_n^{ss} \rVert 
    \\ & \leq \left(1 - \frac{\mu_{ff}}{4} \alpha_t\right) \lVert (I - \gamma_t B_t^{ss})^{-1} \rVert \lVert L_t \rVert + \gamma_t \lVert A_{ff}^{-1} A_{fs} B_t^{ss} \rVert 
\end{align*}
when $\alpha_t \leq \mu/\nu^2$, where we used that $\sqrt{1 - x} \leq 1 - x/2$ when $x \in [0, 1]$. 
By induction and Eq. \eqref{eq:induction_size_equation}, there exists a constant $K_L$ such that
\begin{align*}
    \frac{1}{K_{ss}}\lVert L_{n+1} \rVert 
    & \leq \prod_{t=1}^n \left(1 - \frac{\mu_{ff}}{4} \alpha_t \right) \lVert L_1 \rVert + \sum_{t=1}^n \gamma_t  \prod_{j=t+1}^n \left(1 - \frac{\mu_{ff}}{4} \alpha_j\right)  \lVert A_{ff}^{-1} A_{fs} B_t^{ss} \rVert 
    \\ & \leq 
    \prod_{t=1}^n \left(1 - \frac{\mu_{ff}}{4} \alpha_t \right) \lVert L_1 \rVert + K_L \frac{\gamma_n}{\alpha_n}
    .
\end{align*}



% \subsection{Remove after completing above; Proof of Proposition \ref{prop:Lt_welldefined}}




% {\color{red}
% Remove after finishing above:
% I need to merge this to blend in with this paper. 
% After parsing the requirements on my own, merge them into Assumption \ref{assumption:steps}.
% }
% \begin{proposition}\label{prop:Lt_welldefined}
%     {\color{red}Needs fixing/proving:}
%     When step sizes are chosen so that
%     \begin{equation}\label{eq:stepsize}
%         \gamma_1 \leq \min \frac{1}{2} \left\{ \frac{1}{\lVert Q_\Delta \rVert^2 \lVert \Delta\rVert^2_{Q_\Delta}}, \frac{\lVert Q_\Delta\rVert^2}{\lVert \Delta \rVert_{Q_\Delta} \lVert Q_\Delta\rVert^2 + 1} \right\}
%         , \quad 
%         \alpha_1 \leq \frac{1}{2 \lVert Q_{ff}\rVert^2 \lVert A_{ff}\rVert_{Q_ff}^2} ,
%     \end{equation}    
%     there exists a sequence of matrices $\{L_t\}$ with $L_0 = 0$ such that $I - \gamma_t (\Delta - A_{sf}L_t)$ is non-singular. 
%     Consequently, there exists a sequence satisfying the recursion \eqref{eq:Lt} for all $t >0$.
% \end{proposition}
% % \begin{customprop}{\ref{prop:Lt_welldefined}}
% %     When step sizes are chosen so that
% %     \begin{equation}\label{eq:stepsize1}
% %         % \gamma_0 \leq \min \frac{1}{2} \left\{ \frac{1}{\lVert Q_\Delta \rVert^2 \lVert \Delta\rVert^2_{Q_\Delta}}, \frac{\lVert Q_\Delta\rVert^2}{\lVert \Delta \rVert_{Q_\Delta} \lVert Q_\Delta\rVert^2 + 1} \right\}
% %         % , \quad 
% %         % \alpha_0 \leq \frac{1}{2 \lVert Q_{ff}\rVert^2 \lVert A_{ff}\rVert_{Q_ff}^2} ,
% %         \alpha_t \leq \frac{1}{2} \lVert Q_{ff} \rVert^{-2} \lVert A_{ff} \rVert^{-2}_{Q_{ff}} , \quad
% %         \gamma_t \leq \frac{1}{2} \left(\lVert \Delta \rVert_{Q_\Delta} + L_\infty \lVert A_{sf}\rVert_{Q_{ff}, Q_\Delta}\right)^{-1} ,
% %     \end{equation}    
% %     there exists a sequence of matrices $\{L_t\}$ with $L_0 = 0$ such that $I - \gamma_t (\Delta - A_{sf}L_t)$ is non-singular. 
% %     Consequently, there exists a sequence satisfying the recursion \eqref{eq:Lt} for all $t >0$.
% %     {\color{red}     There might be slight edits we need to make on the step sizes. 
% %     Must guarantee that all conditions used are met. }
% % \end{customprop}
% This will be proved with induction: (1) Show that $L_1$ is well-defined with bounded norm.
% (2) If $\lVert L_t \rVert_{Q_\Delta, Q_{ff}} \leq L_\infty$.
% {\color{red}To do: Step (1).
% }


% The second step is shown below.
% First observe that the recursion for $L_t$ can be written with a matrix identity on $L = L_t, L' = L_{t+1}$:
% \begin{align*}
%     L' (I-\gamma_t \Delta + \gamma_t A_{sf} L) 
%     &= \left(I - \alpha_t A_{ff} - \gamma_t A_{ff}^{-1} A_{fs} A_{sf}\right) L + \gamma_t A_{ff}^{-1} A_{fs} \Delta  
%     \\
%     &= (I - \alpha_t A_{ff}) L + \gamma_t A_{ff}^{-1} A_{fs} (\Delta - A_{sf} L)
%     .
%     \end{align*}
% Denoting $B^{ss}_t = \Delta - A_{sf} L_t$, we then have the expression
% \begin{equation}\label{eq:Ltidentity}
%     L' (I - \gamma_t B_t^{ss}) = (I - \alpha_t A_{ff})L + \gamma_t A_{ff}^{-1} A_{fs} B_t^{ss}.
% \end{equation}
% We will now see that there exists a unique solution to \eqref{eq:Ltidentity} if $\alpha_t, \gamma_t$ are sufficiently small.
% An upper bound on the norm of $L'$ can then be derived using that (Lemma 17, \citet{kaledin2020finite}) when $\alpha_t \leq \lVert Q_{ff} \rVert^2$, then
% \begin{equation}
%     \lVert I - \alpha_t A_{ff} \rVert_{Q_{ff}} \leq 1 - \alpha_t a_{ff} , \quad
%     a_{ff} = \frac{1}{2} \lVert Q_{ff} \rVert^{-2} .
% \end{equation}


% \begin{lemma}[Lemma 18, \citet{kaledin2020finite}]\label{lem:Ltsize}
%     Suppose $\lVert L \rVert_{Q_\Delta, Q_{22}} \leq L_\infty$ and consider the step sizes
%     \begin{align*}
%         \alpha_t \leq \frac{1}{2} \lVert Q_{ff} \rVert^{-2} \lVert A_{ff} \rVert^{-2}_{Q_{ff}} , \quad
%         \gamma_t \leq \frac{1}{2} \left(\lVert \Delta \rVert_{Q_\Delta} + L_\infty \lVert A_{sf}\rVert_{Q_{ff}, Q_\Delta}\right)^{-1} .
%     \end{align*}
%     Then there exists a unique solution $L'$ to \eqref{eq:Ltidentity} that satisfies
%     \begin{align*}
%         \lVert L' \rVert_{Q_\Delta, Q_{ff}} 
%             &\leq 
%         (1 - \alpha_t a_{ff}) \lVert L \rVert_{Q_{\Delta}, Q_{ff}} + \gamma_t C_D (L_\infty) , 
%         \\
%         C_D (L_\infty) 
%             &= 
%         2
%         \left(
%             \lVert A_{ff}^{-1} A_{fs} \rVert_{Q_\Delta, Q_{ff}} + L_\infty
%         \right) 
%         \left(
%             \lVert \Delta \rVert_{Q_\Delta} + L_\infty \lVert A_{sf} \rVert_{Q_{ff}, Q_\Delta} 
%         \right)
%     \end{align*}
%     Moreover, this implies that if $\gamma_t/\alpha_t \leq \epsilon a_{ff}/C_D(L_\infty)$, then $\lVert L'\rVert_{Q_\Delta, Q_{ff}} \leq L_\infty$.
% \end{lemma}
% Substituting back $L_t$ and $L_{t+1}$, we then have by induction that
% \begin{align*}
%     \lVert L_t\rVert_{Q_\Delta, Q_{ff}}
%         \leq 
%     C_D (L_\infty)
%     \sum_{j=0}^t \gamma_j \prod_{k=j+1}^t (1 - \alpha_k a_{ff} ) .
% \end{align*}
% The last term including summation is a recurring quantity which is extensively analyzed in Lemma 14 \citep{kaledin2020finite}, in this case (iv) asserts that for $\alpha_0 \leq a_{ff}^{-1} $,
% \begin{align*}
%     & \sum_{j=0}^t \gamma_j \prod_{k=j+1}^t (1 - \alpha_k a_{ff} ) \leq
%     \frac{2\gamma_t}{a_{ff} \alpha_t}  
%         \max \left\{
%              \xi \max \left\{1, \frac{a_{ff}}{4 a_\Delta}\right\},
%             2 \xi^3
%     \right\} , \\
%     & \xi = 1 + \max\left\{ \frac{a_{ff}}{8}\alpha_0, \frac{a_{\Delta}}{16} \gamma_0\right\} , \quad
%     a_{\Delta} = \frac{1}{2 \lVert Q_\Delta \rVert^2} .
% \end{align*}
% {\color{red}Lastly, we need to show that this upper bound is again bounded above by $L_\infty$, which I don't think is shown in \citep{kaledin2020finite}.}




% \textbf{Relating operator norm with weighted norm:}
% From the elementary inequalities
% \begin{align*}
%     \lambda_{\min}(Q) \lVert x \rVert^2 \leq x^T Q x \leq \lambda_{\max}(Q) \lVert x \rVert^2 ,
% \end{align*}
% we obtain that substituting $L x$ for $x$ above, the following is satisfied for any $Q \succ 0$:
% \begin{equation}
%     \frac{\lambda_{\min}(Q)}{\lambda_{\max}(Q)} \lVert L \rVert^2
%     \leq 
%     \lVert L \rVert_Q^2 = \sup_{x \neq 0} \frac{x^T L Q L x}{x^T Q x} 
%     \leq 
%     \frac{\lambda_{\max}(Q) }{\lambda_{\min}(Q)} \lVert L \rVert^2 .
% \end{equation}
% This is obtained by optimizing the numerator and denominator separately.
% From the lower bound, we have that for any positive definite $Q$
% \begin{equation}
%     \lVert L \rVert^2 \leq \kappa (Q) \lVert L \rVert_Q^2 . 
% \end{equation}
% More generally,
% \begin{equation}
%     \lVert L_t \rVert^2 \leq \frac{\lambda_{\max}(Q_{\Delta})}{\lambda_{\min}(Q_{ff})} \lVert L_t \rVert_{Q_\Delta, Q_{ff}}^2
% \end{equation}
% To relate this back to the properties of $\Delta$ and $A_{ff}$, we need to evaluate the minimum and maximum eigenvalues of a matrix $Q$ that solves the Lyapunov equation.
% \begin{proposition}[Theorem 10, \citet{lancaster1970explicit}]
%     Let $Q \succ 0$ be the unique solution to the Lypaunov equation
%     \begin{equation}
%         A Q + Q A^T = I 
%     \end{equation}
%     for a stable matrix $-A$.
%     Then,
%     \begin{align}
%         \frac{1}{\lvert \lambda_{\min}(A + A^T) \rvert} &\leq \lambda_{\min}(Q) \leq \frac{1}{2 \lvert \lambda_{\min}(A) \rvert} 
%         \\ 
%         \frac{1}{2 \lvert \lambda_{\max}(A)\rvert } &\leq \lambda_{\max}(Q) \leq \frac{1}{\lvert \lambda_{\max}(A + A^T)\rvert} . 
%     \end{align}
%     % {\color{red}But $A + A^T$ may not be Hurwitz? I may need to refine this. 
%     % Check details of this reference.
%     % }
% \end{proposition}
% {\color{blue}To do:
%     Using this, obtain a bound on the unweighed norm $\lVert L_t\rVert$ for all $t$ as a function of $\Delta$ and $A_{ff}$, in place of $Q_\Delta, Q_{ff}$.
% }
% \begin{lemma}\label{lem:unweighted_norm}
%     The unweighted norm $\lVert L_t \rVert$ is uniformly bounded for all time. {\color{red}Specify problem parameters.}
% \end{lemma}



% \subsection{Bound on $\Phi$: Lemma 12 \& 14 in \citep{kaledin2020finite}}
% Step 1: (E.g. Lemma 17 in \citep{kaledin2020finite}).
% For every $\alpha_t \leq a_{ff}$, {\color{red}Correct step size requirement.}
% \begin{equation}
%     \lVert I - \alpha_t A_{ff} \rVert_{Q_{ff}} \leq 1 - \alpha_t a_{ff} .
%     % \sum_{j=0}^{T-1} \alpha_j \sum_{}
% \end{equation}

% Step 2: Lemma 12 in \citep{kaledin2020finite}.

% Step 3: Relate back to unweighted norm. 

% Note that we also need to show that
% \begin{equation}
%     \frac{1}{T}\sum_{j=0}^{T-1} \lVert \Phi_{jT} \rVert^2 = \mathcal{O}(T^{-(1-\delta)})
% \end{equation}
% for some $\delta$ as in Appendix B \citep{srikant2024CLT}.
% Universal bound is not sufficient, since it would give a much worse rate of a constant above. 

\textbf{Lower Bound:} {\color{red}Revise to be compatible with above.}
It is also possible to derive a lower bound for $\mathrm{Tr} X_t$.
In this context, a lower bound demonstrates that the analysis is tight for the class of algorithms considered.
% , rather than an information-theoretic bound.
\begin{lemma}\label{lem:fast_lb}
    \begin{equation}
        \mathrm{Tr} X_{t+1} = \Omega\left(\mu_{ff} \sum_{j=1}^t \alpha_j^2 \prod_{k=j+1}^t \left(1 - \alpha_k \frac{\mu_{ff}}{2}\right) \mathrm{Tr} \Sigma_{ff}\right) .
    \end{equation}    
    {\color{red}In particular, this can be simplified to?}
\end{lemma}

Starting from Eq. \eqref{eq:fast_recursion}, we have 
\begin{align*}
    X_{t+1} \succeq (1 - \alpha_t \nu_{ff}) X_t + \alpha_t^2 \mu_{ff} \Sigma_{ff} + \gamma_t F_t (X_t) .
\end{align*}
Using that $\mathrm{Tr} A B \leq \mathrm{Tr} \Lambda_A \Lambda_B$ for symmetric matrices $A, B$ with eigenvalue matrices $\Lambda_A, \Lambda_B$ \citep{Theobald_1975},
\begin{align*}
    f_t (X_t) 
    &= 
    -\mathrm{Tr}\left( (\delta_t A_{sf} + (\delta_t A_{sf})^T) X_t \right) + \mathrm{Tr} \left(\gamma_t \delta_t \Gamma_{ss} \delta_t^T - (\Gamma_{fs} \delta_t^T + (\delta_t \Gamma_{fs})^T)\right) 
    \\
    &\geq - M_{f} \mathrm{Tr} X_t
    + \lvert \mathrm{Tr}\left(\Gamma_{fs} \delta_t^T + \delta_t \Gamma_{sf} )\right) 
    \rvert
    =: -M_f \mathrm{Tr}X_t + v_t
    ,
\end{align*}
where we use that $\delta_t$ is uniformly bounded to get the constant $M_{f}$ defined in the upper bound proof, and that $v_t$ is uniformly bounded by some $v$.
Substituting the lower bound for $\mathrm{Tr} F_t (X_t)$, we then have
\begin{equation}
    \mathrm{Tr} X_{t+1} \geq (1 - \alpha_t \nu_{ff} - \gamma_t M_f) \mathrm{Tr} X_t + \alpha_t^2 \mu_{ff} \mathrm{Tr} \Sigma_{ff} + \gamma_t v .
\end{equation}
Using that $\alpha_t < 1/\nu_{ff}$ {\color{red}step!},
% Using that $\alpha_t \nu_{ff} + \gamma_t M_f \leq \alpha_t 2 \nu_{ff}$ whenever $\gamma_t/\alpha_t \leq \nu_{ff}/M_f$, {\color{red}This is for lower bound; not mentioned in step size assumption}
\begin{align*}
    \mathrm{Tr}X_{t+1} 
    &\geq \prod_{k=1}^t (1- \alpha_k \nu_{ff}) \mathrm{Tr} X_1 
    - \sum_{j=1}^t \gamma_j \prod_{k=j+1}^t \left(1 - \alpha_k \nu_{ff}\right) \mathrm{Tr} X_j \\
    &+ \sum_{j=1}^t \alpha_j^2 \prod_{k=j+1}^t (1 - \alpha_k \nu_{ff}) \mathrm{Tr} \Gamma_{ff} 
    + \sum_{j=1}^t \gamma_j \prod_{k=j+1}^t (1 - \alpha_k \nu_{ff}) v.    
\end{align*}
Using our previous result that $\mathrm{Tr} X_j = \mathcal{O}(\alpha_j)$ and that $\gamma_j < \alpha_j$, we see that the second term is smaller than the third term.
To bring in the dependency on $\Sigma_{ff}$, we use that $\mathrm{Tr} \Gamma_{ff} \geq \mu_{ff} \mathrm{Tr} \Sigma_{ff}$ to obtain
\begin{equation}
    \mathrm{Tr} X_{t+1} = \Omega\left(
    % \sum_{j=1}^t \gamma_j \prod_{k=j+1}^t (1 - \alpha_k \nu_{ff}) +
    \mu_{ff} \sum_{j=1}^t \alpha_j^2 \prod_{k=j+1}^t \left(1 - \alpha_k \nu_{ff} \right) \mathrm{Tr} \Sigma_{ff}\right) .
\end{equation}
{\color{red}Simplify: The inner part should give a $1/\mu_{ff}$, which is used in the above text description.}
% \section{Main Results}
\section{Two Time-Scale Convergence}
This is left as the next step. 
Once we have a term-by-term convergence result in distribution, we will need to refine the proof so that it works for the two together. 
The idea is to re-derive the decomposition for two time-scale iterates jointly, where the individual iterate convergence will be given by the previous section.
Then, we can substitute these expression to that required to analyze the cross terms.
{\color{red}Below is based on discussions before deciding to do each time-scale separately.}



Using this transformation, we have the state transition terms given by
\begin{align*}
    B_t = \begin{pmatrix}
        \alpha_t B_t^{ff} & \alpha_t B_t^{fs} \\ \gamma_t A_{sf} & \gamma_t B_t^{ss}
    \end{pmatrix}  
\end{align*}
with the systems $B$ defined as
\begin{align*}
    B_t^{ff} = \frac{\gamma_t}{\alpha_t}(L_{t+1} + A_{ff}^{-1} A_{fs}) A_{sf} + A_{ff} \\ 
    B_t^{fs} = \alpha_t^{-1} \left(L_t - L_{t+1}\right) + \frac{\gamma_t}{\alpha_t}\left(L_{t+1} + A_{ff}^{-1} A_{fs}\right) B_t^{ss} - A_{ff} L_t \\ 
    B_t^{ss} = \Delta - A^{sf} L_t .
\end{align*}
Here we see that if $L_t \to 0$ in the limit $t \to \infty$, we have $-B$ is asymptotically H\"{u}rwitz by inspecting the block-diagonal of the asymptotically lower triangular matrix.
The noise term is
\begin{equation}
    N_t = \begin{pmatrix}
        \alpha_t \xi_t + \gamma_t (A_{ff}^{-1} A_{fs} + L_{t+1}) \psi_t \\ \gamma_t \psi_t 
    \end{pmatrix} 
\end{equation}



For finite-time analysis, it will be useful to define scaled versions of the above quantities.
Specifically, let's use bar to normalize matrices so that they don't go to 0 asymptotically:
\begin{equation}
    B_t = S_t \bar{B}_t, S_t = \begin{pmatrix}
        \alpha_t I & 0 \\ 0 & \gamma_t I 
    \end{pmatrix}, 
    \bar{B}_t = \begin{pmatrix}
        B_t^{ff} & B_t^{fs} \\ A_{sf} & B_t^{ss}
    \end{pmatrix} 
    \to \bar{B}_\infty = \begin{pmatrix}
        A_{ff} & 0 \\ A_{sf} & \Delta
    \end{pmatrix} .
\end{equation}
We can also write the noise term as
\begin{align*}
    N_t = \left(S_t + \gamma_t M_k
    \right) \begin{pmatrix}
        \xi_t \\ \psi_t
    \end{pmatrix},
    M_k = \begin{pmatrix}
        0 &  (A_{ff}^{-1} A_{fs} + L_{k+1}) \\ 0 & 0 
    \end{pmatrix} .
\end{align*}
From this, we can write 
\begin{equation}\label{eq:master}
    \hat{z}_t = \prod_{k=1}^t (I - B_k) \hat{z}_0 + \sum_j \prod_k \left(I - S_k \bar{B}_k\right) \left(S_j + M_j\right) \begin{pmatrix}
        \xi_j \\ \psi_j 
    \end{pmatrix} .
\end{equation}


Note that $\hat{\bar{z}}_T = T^{-1} \sum_t \hat{z}_t$ is an adequate approximation of the Polyak-Ruppet average of $(x_t, y_t)$, where the error of approximation decays with the average of $\{L_t\}$ (see backup CLT.tex).
So for the following sections, we will focus on analyzing $\hat{z}_t$ before moving on to the Polyak-Ruppert average.


{\color{blue}
To do: Derive formal approximation error bounds and verify that they decay fast enough.
Most of the remaining analysis can be completed by generalizing scalar-versions of inequalities in \citep{konda2004convergence,kaledin2020finite} to matrices. 
The reason we need such generalizations is that convergence in distribution requires keeping track of the exact expressions as much as possible, whereas mean-squared convergence results can at some point replace operators with scalar versions of their operator norms. 
}

\subsection{Noise Decomposition}
We will write the noise in the form
\begin{equation}
    \sum_j \prod_k \left(I - S_k \bar{B}_k\right) \left(S_j + M_j\right) \begin{pmatrix}
        \xi_j \\ \psi_j 
    \end{pmatrix} = 
    \bar{B}_\infty^{-1} \sum_j \begin{pmatrix}
        \xi_t \\ \psi_t
    \end{pmatrix}
    + \text{Transient Terms}
\end{equation}
where the Transient Terms are error bounds that are shown to decay to 0.


First note that $S_k$ in \eqref{eq:master} is diagonal hence commutes with any matrix, which we use to get
\begin{equation}
    \sum_j \prod_k (I - S_k \bar{B}_k) S_k \begin{pmatrix}
        \xi_j \\ \psi_j 
    \end{pmatrix}
    = \sum_j S_j \prod_k (I - S_k \bar{B}_k) \begin{pmatrix}
        \xi_j \\ \psi_j 
    \end{pmatrix} .
\end{equation}
We would like to approximate the noise terms with 
\begin{align*}
    \sum_j S_j \prod_k (I - S_k \bar{B}_\infty) \begin{pmatrix}
        \xi_j \\ \psi_j 
    \end{pmatrix} , 
\end{align*}
which resembles the $\sum_j \gamma_j (1 - \gamma_j A)$ shown to converge to $A^{-1}$ for a H\"{u}rwitz matrix $-A$ in \citep{konda2004convergence}.
Specifically, we have that $-\bar{B}_\infty$ defined above is H\"{u}rwitz, and we need to generalize this for a matrix $S_j$ in place of scalar $\gamma_j$.
{\color{red}TODO: Prove the rate of the above term.
If we succeed, we should be able to prove convergence of the Polyak-Ruppert average.
}


Assuming the above gives a nice analysis, we would like to obtain an error bound on 
\begin{equation}
    \prod_k (I - S_k \bar{B}_k) - \prod_k (I - S_k \bar{B}_\infty) .
\end{equation}
Note that the LHS terms can be factorized as
\begin{align*}
    I - S_k \bar{B}_k = (I - S_k \bar{B}_\infty) (I - (I - S_k \bar{B}_\infty)^{-1}(S_k \bar{B}_k - S_k \bar{B}_\infty)).
\end{align*}
{\color{red}To Prove: A formal proof for the below approximation. 
Currently, this is based on the fact that if the last term on RHS is a scalar, or if we can somehow isolate its operator norm, then we get an error rate of}
\begin{equation}
    \lVert \prod_k (I - S_k \bar{B}_\infty) - \prod_k (I - S_k \bar{B}_k)\rVert \approx \lVert \prod_k \left(I - (I - S_k \bar{B}_\infty)^{-1}(S_k \bar{B}_k - S_k \bar{B}_\infty)\right) - I \rVert,
\end{equation}
where $S_k \to 0$ and so $I - S_k \bar{B}_\infty \to I$, and the transient term $\bar{B}_t - \bar{B}_\infty$ is the transient component
\begin{align*}
    S_t \bar{B}_t - S_t \bar{B}_\infty = S_t \begin{pmatrix}
        \frac{\gamma_t}{\alpha_t}(L_{t+1} + A_{ff}^{-1} A_{fs})A_{sf} & B_t^{fs} \\ 0 & -A_{sf} L_t 
    \end{pmatrix} .
\end{align*}
{\color{red}At this point, we need that}
\begin{align*}
    \lVert \prod_k (I - (I - S_k \bar{B}_\infty)^{-1}(S_k \bar{B}_k - S_k \bar{B}_\infty)) \rVert
        \approx 
    \prod_k (1 - o(k))
\end{align*}
which holds {\color{red}wrt a weighted norm} when the minus term is sufficiently small (see \citet{kaledin2020finite} Eq. 8 or Lemma 17). 
For this error to decay to 0, we need $o(k) \leq 1/k$.
{\color{red}To do: Combine to recover the error rate of the approximation above from this.}



Assuming we can prove the above, we now have that with $\Xi_j = [\xi_j, \psi_j]$ that
\begin{align*}
    \sum_j \prod_k (I - S_k \bar{B}_k) (S_j + M_j) \Xi_j 
    = \sum_j S_j \prod_k (I - S_k \bar{B}_k) \Xi_j + \sum_j \prod_k (I - S_k \bar{B}_k) M_j \Xi_j 
    \\
    \approx \sum_j S_j \prod_k (I - S_k \bar{B}_\infty) \Xi_j + \sum_j \prod_k (I - S_k \bar{B}_k) M_j \Xi_j 
    \\
    = \bar{B}_\infty^{-1} \sum_j \Xi_j + \sum_j \left(S_j \prod_k (I - S_k \bar{B}_\infty) - \bar{B}_{\infty}^{-1}\right) \Xi_j 
    + \sum_j \prod_k (I - S_K \bar{B}_k) M_j \Xi_j .
\end{align*}
{\color{red}
My intention now is to show that the (1) we can apply CLT to the first one, (2) Second term we are working on an error bound in the above section, and (3) $M_j$ is upper triangular with eigenvalues of 0 and $\lVert \prod_k (I - S_k \bar{B}_k) Z \rVert \leq \prod_k (1 - s_k)$ as used in \citep{konda2004convergence,kaledin2020finite}.
}

The reason we can suspect
\begin{align*}
    \sum_j S_j \prod_{k=j+1}^t (I - S_k \bar{B}_\infty) \to \bar{B}_{\infty}^{-1}
\end{align*}
is that (1) if $S_k$ is understood to be a scalar step-size, then \citep{konda2004convergence} has this type of result (their $\alpha$ and $w$); (2) Again viewing $S_j$ to be a scalar step-size, we have (after switching sums in Polyak Ruppert avg) $s_t \sum_j \prod_k (1 - s_k) \approx s_t \sum_j \exp(-\sum_{k=j+1}^t s_k) = s_t (\sum_k s_k r_\infty)^{-1} \approx r_\infty^{-1}$ (refine this argument in matrix form). 
The first reason from \citep{konda2004convergence} is sufficinet to see this case. 
Note that in this case, we are saying that the approximation is such that for every finite time $t$, we have a constant bound on the distance to its limit.

\section{Slow Iterate}

\subsection{Why Same Steps as in Fast Isn't Good Enough for Slow}
Using induction,
\begin{align*}
    &\hat{y}_{t} 
    = \prod_{k=1}^t (I - \gamma_k \Delta) \hat{y}_0 
    + \sum_{j=1}^t \gamma_j \prod_{j=k+1}^t (I - \gamma_k \Delta) A_{sf} L_j \hat{y}_j 
    \\ 
    &+ \sum_{j=1}^t \gamma_j \prod_{j=k+1}^t (I - \gamma_k \Delta) A_{sf} \tilde{x}_j 
    + \sum_{j=1}^t \gamma_j \prod_{j=k+1}^t (I - \gamma_k \Delta)V_j 
    \\
    & \hat{\mu}_{yn} = n^{-1} \sum_{t=1}^n \prod_{k=1}^t (I - \gamma_k \Delta) \hat{y}_0 + n^{-1} \sum_{t=1}^n \gamma_t \sum_{j=t+1}^n \prod_{k=j+1}^n (I - \gamma_k \Delta) A_{sf} L_t \hat{y}_t 
\\ & + n^{-1} \sum_{t=1}^n \gamma_t \sum_{j=t+1}^n \prod_{k=t+1}^n (I - \gamma_k \Delta) A_{sf} \tilde{x}_t + n^{-1} \sum_{t=1}^n \gamma_t \sum_{j=t+1}^n \prod_{k=t+1}^n (I - \gamma_k \Delta) V_t
    ,
\end{align*}
where we changed the order of summation as earlier.
Overloading $\Phi_{tn}$ to be defined with the step-size schedule $\{\gamma_t\}$ and $\Delta$, we then repeat the same analysis for 
\begin{align*}
    \sqrt{n} \hat{\mu}_{yn} = \frac{1}{\sqrt{n}} \sum_{t=1}^n \prod_{k=1}^t (I - \gamma_k \Delta) \hat{y}_0 + \frac{1}{\sqrt{n}} \sum_{t=1}^n \Phi_{tn} A_{sf} L_t \hat{y}_t + \frac{1}{\sqrt{n}} \sum_{t=1}^n \Delta^{-1} A_{sf} L_t \hat{y}_t 
    \\
    + \frac{1}{\sqrt{n}} \sum_{t=1}^n \Phi_{tn} A_{sf} \tilde{x}_t + \frac{1}{\sqrt{n}} \sum_{t=1}^n \Delta^{-1} A_{sf} \tilde{x}_t + 
    \frac{1}{\sqrt{n}} \sum_{t=1}^n \Phi_{tn} V_t + \frac{1}{\sqrt{n}} \sum_{t=1}^n \Delta^{-1} V_t .
\end{align*}
{\color{red}To proceed with the analysis, we need to understand the behavior of $\Phi$ with respect to $\gamma$, i.e. when $\delta = 1$. Is it still uniformly bounded?}
{\color{red}Assuming $\Phi$ is uniformly bounded for $\gamma$, }
the terms are 
\begin{enumerate}
    \item Second term summand $\lVert L_t \hat{y}_t \rVert \sim t^{-1 + \delta} t^{-1/2}$ which results in a sum followed by normalization to get $(\delta - 1/2)^{-1} n^{-1 + \delta}$.
    Third term is the same as this.
    \item Fourth and fifth terms (especially fifth) don't decay fast enough: using $\tilde{x}_t \sim t^{-\delta/2}$, we have that after normalization that the sum if $n^{(1 - \delta)/2}$ which grows with $n$. 
    We're not using that $Cov(\tilde{x}_t, \hat{y}_t) \asymp t^{-1}$, which is perhaps what we need to use. 
    {\color{blue}For $\gamma_t = t^{-1}$, we have $\lVert \Phi \rVert \sim \log n$ simply by integrating. 
    So we need to write out the terms using a different decomposition such that the fourth and fifth terms do not appear. 
    } 
\end{enumerate}
We can invoke Theorem 1 in \citep{srikant2024CLT} to get $n^{-\beta / 2}$ for the last term.



Overloading $\Phi_{tn}$ to be defined with the step-size schedule $\{\gamma_t\}$ and $\Delta$ as
\begin{equation}
    \Phi_{tn} = \gamma_t \sum_{j=t+1}^n \prod_{k=t+1}^j (I - \gamma_k \Delta) - \Delta^{-1},
\end{equation}
we then repeat the same analysis for 
\begin{align*}
    \sqrt{n} \hat{\mu}_{yn} = \frac{1}{\sqrt{n}} \sum_{t=1}^n \prod_{k=1}^t (I - \gamma_k \Delta) \hat{y}_0 + \frac{1}{\sqrt{n}} \sum_{t=1}^n \Phi_{tn} A_{sf} L_t \hat{y}_t + \frac{1}{\sqrt{n}} \sum_{t=1}^n \Delta^{-1} A_{sf} L_t \hat{y}_t 
    \\
    + \frac{1}{\sqrt{n}} \sum_{t=1}^n \Phi_{tn} A_{sf} \tilde{x}_t + \frac{1}{\sqrt{n}} \sum_{t=1}^n \Delta^{-1} A_{sf} \tilde{x}_t + 
    \frac{1}{\sqrt{n}} \sum_{t=1}^n \Phi_{tn} V_t + \frac{1}{\sqrt{n}} \sum_{t=1}^n \Delta^{-1} V_t .
\end{align*}
Note that $\Phi_{tn}$ is not analyzed in \citep{srikant2024CLT,polyakJuditsky} because $\gamma_t \propto t^{-1}$.
Let us roughly calculate the rate of $\Phi_{tn}$ using $\gamma_t \propto (c t)^{-1}$:
\begin{align*}
    \lVert \gamma_t \sum_{j=t+1}^n \prod_{k=t+1}^j (I - \gamma_k \Delta) \rVert \leq \gamma_t \sum_{j=t+1}^n \prod_{k=t+1}^j (1 - c \gamma_k) 
    \propto t^{-1} \sum_{j=t+1}^n \frac{t+1}{j}
    % \\ 
    \approx \log \frac{n}{t+1}
    % \\
    % \approx \frac{\log }
    % \leq \frac{1}{t} \sum_{j=t+1}^n \exp\left(-c \log \frac{j}{t+1}\right) \\ 
    % = \frac{1}{t} \sum_{j=t+1}^n \left(\frac{j}{t+1}\right)^c
    % \leq t^{-1 - c} \sum_{j=t+1}^n j^c 
    % \approx t^{-1 - c} \left(n^{1+c} - t^{1+c}\right) ,
\end{align*}
where the first inequality uses that $\lVert I - \gamma_t \Delta \rVert \leq 1 - c \gamma_t$ by relating the weighted norm with the unweighted norm.
The second inequality holds with $\gamma_t = t^{-1}$ and $c \in (0, 1)$.
Note that this approximation shows that $\lVert \Phi_{tn}\rVert$ does not decay. 
As a result, the recursion in the current form is not contractive.


Even if $\Phi_{tn}$ is shown to exhibit contractive properties, the fourth and fifth terms are problematic: Using $\tilde{x}_t \sim t^{-\delta/2}$ we have that $n^{-1/2} \sum_t t^{-\delta / 2} \sim n^{1/2 - \delta/2}$ which grows with $n$.
Therefore, we try to improve our analysis for the fourth and fifth terms in the next part.


% \subsubsection{Attempt 3: Re-scaling}
% {\color{blue}This works, but not to the correct distribution (covariance is off). 
% Rate matches that of fast iterate. 
% Correction: Problem persists, since we only get the ratio in front of $\Phi_{tn}$, not in front of $\Delta^{-1}$. 
% }
% % Our analysis relies on the fast iterates converging for every $y$.
% % Therefore, we can try to write $\tilde{x}_t = x_\infty(\hat{y}_t) + (\tilde{x}_t - x_\infty (\hat{y}_t))$:
% % \begin{align*}
% %     \hat{y}_{t+1} = (I - \gamma_t B_t^{ss}) \hat{y}_t - \gamma_t A_{sf} (x_t - x_\infty (y_t) + L_t \hat{y}_t) + \gamma_t V_t .
% % \end{align*}
% % {\color{red}Not sure how to proceed; the $x_t$ term still poses a problem of being too slow. 
% % I was hoping that we can bring out the $\hat{y}_t$ term from inside $x_\infty$, since $\hat{y}_t$ decays fast. 
% % }
% \begin{align*}
%     \hat{y}_{t+1} = (I - \alpha_t B_t^{ss}) \hat{y}_t + (\alpha_t - \gamma_t) B_t^{ss}\hat{y}_t + \gamma_t \tilde{x}_t + \gamma_t V_t
%     \\
%     = (I - \alpha_t \Delta) \hat{y}_t + \alpha_t (\Delta - B_t^{ss}) \hat{y}_t + (\alpha_t - \gamma_t)B_t^{ss} \hat{y}_t + \gamma_t \tilde{x}_t + \gamma_t V_t .
% \end{align*}
% Defining
% \begin{align*}
%     \Phi_{tn} = \alpha_t \sum_{j=t+1}^n \prod_{k=t+1}^j (I - \alpha_t \Delta) - \Delta^{-1} ,
% \end{align*}
% we can write out the terms as {\color{red}Check calculations! I think there should be a $\gamma_t$ coefficient. }
% \begin{align*}
%     \bar{y}_n = \bar{b}_n + \frac{1}{n}\sum_{t=1}^n (\frac{\gamma_t}{\alpha_t} \Phi_{tn} + \Delta^{-1}) (\Delta - B_t^{ss}) \hat{y}_t - \frac{1}{n} \sum_{t=1}^n (\frac{\gamma_t}{\alpha_t} \Phi_{tn} + \Delta^{-1}) B_t^{ss} \hat{y}_t \\
%     + 
%     \frac{1}{n}\sum_{t=1}^n (\frac{\gamma_t}{\alpha_t} \Phi_{tn} + \Delta^{-1}) \tilde{x}_t + Noise .
% \end{align*}
% Therefore using $\sum_{t=1}^n \lVert \Phi_{tn} \rVert^2 = O(n^\delta)$ as before, the $n^{-1/2} \sum_t \frac{\gamma_t}{\alpha_t}\tilde{x}_t$ term goes away at rate established for the fast iterate, which is $n^{\delta/2}$.
% {\color{red}Problem: Applying telescoping we obtain convergence to normal distribution with covariance $\Delta^{-1} \Gamma_{ss} \Delta^{-T}$, which may not be the correct one. 
% We can derive the correct covariance through recursion following \citep{konda2004convergence}.
% But then what is the flaw in the above argument?
% There is a chance that everything is correct, and that for the Polyak-Ruppert average we indeed get the simple $\Delta^{-1} \Gamma_{ss} \Delta^{-T}$ term instead of the more complicated covariance established in \citep{konda2004convergence}.
% }



% \subsubsection{{\color{red}New!} Handling individual terms: Linear Innovations / LMMSE}
% {\color{blue}Summary of idea and reason for failure:}
% The $\tilde{x}_t$ term decays too slow, and we need a different decomposition so that individual terms exhibit fast convergence, in the sense that when added and divided by $\sqrt{n}$ the norm decays to 0 (without the $\gamma_t$ term).
% Here, $\tilde{x}_t \sim t^{-\delta/2}$ becomes $n^{-1/2} \sum_t t^{-\delta / 2} \sim n^{1/2 - \delta / 2}$ which grows with $n$.


% In hopes that we can exploit the covariance between $\tilde{x}_t$ and $\hat{y}_t$, because this term decays fast, we can try to approximate $\tilde{x}_t$:
% \begin{align*}
%     \tilde{x}_t = Cov(\tilde{x}_t, \hat{y}_t) Cov(\hat{y}_t)^{-1} \hat{y}_t + (\tilde{x}_t - Cov(\tilde{x}_t, \hat{y}_t) Cov(\hat{y}_t)^{-1} \hat{y}_t ),
% \end{align*}
% which resembles the LMMSE.
% While the former term has norm of order $\hat{y}_t \sim \gamma_t$ (which still isn't fast enough, but is faster), the latter term understood as the error term in LMMSE has variance
% \begin{align*}
%     Var(e) = Var(\tilde{x}_t) - Cov(\tilde{x}_t, \hat{y}_t) Cov(\hat{y}_t)^{-1} Cov(\hat{y}_t, \tilde{x}_t) ,
% \end{align*}
% which is of order $Var \tilde{x}_t \sim \alpha_t$. 
% Therefore, this doesn't eliminate the slowness of $\tilde{x}_t$.




% {\color{red}
% The $\Sigma^{ff}_t$ term has a second order $\gamma_t$ and is thus neglected for covariance. 
% Ie the covariance depends only between $\hat{y}_t$ (because of the $I$) and inter-covariances. 
% So it is unclear how we can get rid of the $x_t$ term in the analysis, which results in rates that is too slow. 

% \textbf{Mean-square analysis / covariance methods will all be successful because they can get rid of the $\gamma_t^2$ term. 
% But for Stein's method analysis, we need to write $\hat{y}_t$ or $\bar{y}_n$ as a linear combination whose individual terms has expected norms that converge to zero.}
% }

% The issue with the above decomposition is that $\tilde{x}_t$ appears to decay slowly, whereas there may be a different decomposition that uses some correlation between $\tilde{x}_t$ and $\hat{y}_t$.
% Specifically, we aren't using the covariance matrix $Cov(\tilde{x}_t, \hat{y}_t)$ whose magnitude decays fast at rate $\gamma_t$.


% To this end, we introduce $E[\tilde{x}_t | \hat{y}_t]$ and obtain the following recursion for the slow iterates:
% \begin{align*}
%     \hat{y}_{t+1} &= (I - \gamma_t B_t^{ss}) \hat{y}_t 
%     - \gamma_t A_{sf} \left(\tilde{x}_t - E[\tilde{x}_t | \hat{y}_t] \right) + \gamma_t E[\tilde{x}_t | \hat{y}_t] + \gamma_t V_t .
% \end{align*}
% Let us use the LMMSE $E[\tilde{x}_t | \hat{y}_t] = H_t \hat{y}_t$, where we now have
% \begin{align*}
%     \hat{y}_{t+1} &= (I - \gamma_t (B_t^{ss} + A_{sf} H_t))\hat{y}_t - \gamma_t A_{sf} (\tilde{x}_t - H_t \hat{y}_t) + \gamma_t V_t , \\ 
%     H_t &= Cov(\tilde{x}_t, \hat{y}_t) Cov^{-1} (\hat{y}_t) \to \Sigma_{fs} \Sigma_{ss}^{-1} .
% \end{align*}
% The role of introducing a LMMSE $H_t \hat{y}_t$ is to re-write the recursion above using some matrix $Q_t$, where 
% \begin{align*}
%     \hat{y}_{t+1} = (I - \gamma_t Q_t) \hat{y}_t + \gamma_t (Q_t - B_t^{ss} - A_{sf} H_t) \hat{y}_t - \gamma_t A_{sf} (\tilde{x}_t - H_t \hat{y}_t) + \gamma_t V_t .
% \end{align*}
% Since the above is satisfied for any $Q_t$, we can check for which values of $Q_t$ the covariance of $\hat{y}_t$ matches that in \citep{konda2004convergence}:
% \begin{align*}
%     \Sigma_{t+1}^{ss} = &\Sigma_t^{ss} - \gamma_t (Q_t \Sigma_t^{ss} + \Sigma_t^{ss} Q_t^T) 
%     \\
%     &+ \gamma_t^2 (Q_t - \Delta - A_{sf} H_t) \Sigma_t^{ss}(Q_t - \Delta - A_{sf} H_t)^T
%     - \gamma_t^2 A_{sf} Cov(\tilde{x}_t - H_t \hat{y}_t) A_{sf}^T + \gamma_t^2 \Gamma_{ss}
%     \\
%     &+ \gamma_t \left(\Sigma_t^{ss} (Q_t - B_t^{ss} - A_{sf} H_t)^T +  (Q_t - B_t^{ss} - A_{sf} H_t) \Sigma_t^{ss}\right)
%     \\
%     &+ \gamma_t \left( Cov(\hat{y}_t, A_{sf}(\tilde{x}_t - H_t \hat{y}_t)) + Cov(A_{sf} (\tilde{x}_t - H_t \hat{y}_t), \hat{y}_t)\right)
%     \\
%     &+ \gamma_t^2 Q_t \Sigma_t^{ss} Q_t^T
%     - \gamma_t^2 Cov( A_{sf} L_t \hat{y}_t) 
%     + \gamma_t^2 \left(Q_t \Sigma_t^{ss} (Q_t - B_t^{ss} - A_{sf} H_t) + (Q_t - B_t^{ss} - A_{sf} H_t) \Sigma_t^{ss} Q_t^T\right)
%     \\
%     &+ \gamma_t^2 \left(Cov(Q_t \hat{y}_t, A_{sf} (\tilde{x}_t - H_t \hat{y}_t)) + Cov(A_{sf} (\tilde{x}_t - H_t \hat{y}_t, Q_t \hat{y}_t)\right)  .
% \end{align*}
% Similar to how \citet{konda2004convergence} neglect the higher-order $\gamma_t$ terms, we can also use that $\Sigma_t^{ss} \approx \gamma_t \Sigma_{ss}$ and hide the higher order terms to get the limit $\Sigma_{t+1}^{ss} = \Sigma_t^{ss}$ so that the lower-order terms (that depends on $\gamma_t \Sigma_t^{ss}$ and $\gamma_t^2 \Gamma_{ss}$, discrading those with $\gamma_t^2 \Sigma_t^{ss} \sim \gamma_t^3$)
% \begin{align*}
%     0 = -(Q \Sigma_{ss} + \Sigma_{ss} Q_t^T) + \Sigma_{ss} (Q - \Delta - A_{sf} H)^T + (Q - \Delta - A_{sf} H) \Sigma_{ss}
%     \\
%     + 
%     \Sigma_{sf} A_{sf}^T + A_{sf} \Sigma_{fs} - (\Sigma_{ss} H^T A_{sf}^T + A_{sf} H \Sigma_{ss}) 
%     + \Gamma_{ss}
%     \\
%     =
%     - (\Delta + A_{sf} H) \Sigma_{ss} - \Sigma_{ss} (\Delta + A_{sf} H)^T + \Gamma_{ss}
%     \\
%     = - (\Delta \Sigma_{ss} + \Sigma_{ss} \Delta^T) - (A_{sf} \Sigma_{fs} + \Sigma_{fs} A_{sf}^T)  + \Gamma_{ss}
%     \\
%     \Leftrightarrow \Delta \Sigma_{ss} + \Sigma_{ss} \Delta^T +  (A_{sf} \Sigma_{fs} + \Sigma_{fs} A_{sf}^T)  = \Gamma_{ss} ,
% \end{align*}
% where we used that $H_t \to \Sigma_{fs} \Sigma_{ss}^{-1}$.
% {\color{red}
% This gives the correct $\Sigma_{ss}$ as in Eq. 2.6 in \citep{konda2004convergence} except above is missing a $-\bar{\gamma} \Sigma_{ss}$ term. 
% I think this term can be recovered if instead of above, we use the more precise limit $\Sigma_{t+1}^{ss} = \frac{\gamma_{t+1}}{\gamma_t} \Sigma_t^{ss} \Rightarrow \Sigma_{t+1}^{ss} - \Sigma_t^{ss} = -1 + \frac{\gamma_{t+1}}{\gamma_t}$. 
% }
% Here we see that any $Q_t$ can be chosen as long as $\gamma_t^3 Q_t \ll \gamma_t^2$ so that the higher order terms are indeed negligible. 

% Now we will need to choose $Q_t$ so that it is Hurwitz, from which we can prove decay of the induction
% \begin{align*}
%     \bar{y}_n = \bar{b}_n + \frac{1}{n} \sum_{t=1}^n \Phi_{tn} (Q_t - B_t^{ss} - A_{sf} H_t) \hat{y}_t + \frac{1}{n} \sum_{t=1}^n \Phi_{tn} A_{sf} (\tilde{x}_t - H_t \hat{y}_t) + \frac{1}{n} \sum_{t=1}^n \Phi_{tn} V_t \\ 
%     + \frac{1}{n} \sum_{t=1}^n Q^{-1} (Q_t - B_t^{ss} - A_{sf} H_t) \hat{y}_t + \frac{1}{n} \sum_{t=1}^n Q^{-1} A_{sf} (\tilde{x}_t - H_t \hat{y}_t) + \frac{1}{n}\sum_{t=1}^n Q^{-1} V_t ,
% \end{align*}
% where we overload notation for 
% \begin{align*}
%     \Phi_{tn} = \gamma_t \sum_{j=t+1}^n \prod_{k=t+1}^j (I - \gamma_k Q_k) .
% \end{align*}
% {\color{red}What if we choose $Q^{-1} = \Sigma_{ss}^{1/2} \Gamma_{ss}^{-1/2}$ and prove that for this choice we have $\Phi_{tn} (\tilde{x}_t - H_t \hat{y}_t)$ decays sufficiently fast?
% The reason for this choice is so that we can apply CLT to the last term. 
% But the problem is $Q_t$ is then not necessarily H\"{u}rwitz. 
% }
% Ultimately, we will get a different covariance matrix for $\bar{y}_n$ because $\Sigma_{ss}$ is the covariance for the last iterate.


% % If $\Delta + A_{sf} H_t$ is H\"{u}rwitz, then we have that the norm of $I - ()$ can be bounded using weighted norm with weight satisfying Lyapunov equation. 
% % {\color{red}Let us define $\Phi_{tn}$} and write
% % \begin{equation}
% %     \sqrt{n} \hat{\mu}_{yn} = \frac{1}{\sqrt{n}} \sum_{t=1}^n \prod_{k=1}^t (I - \gamma_k (\Delta + A_{sf} H_t )) \hat{y}_0 + \frac{1}{\sqrt{n}} \sum_{t=1}^n \Phi_{tn} A_{sf} (\tilde{x}_t - H_t \hat{y}_t) + \frac{1}{\sqrt{n}} \sum_{t=1}^n \Phi_{tn} V_t ,
% % \end{equation}
% % where $\Phi_{tn}$ {\color{red}A modified version} has to satisfy the last iterate's covariance \citep{konda2004convergence}
% % \begin{equation}
% %     Cov(\Phi_{tn} V_t) \approx \Sigma_{ss}: \Delta \Sigma_{ss} + \Sigma_{ss} \Delta^T - \bar{\gamma} \Sigma_{ss} = - (A_{sf} \Sigma_{fs} + \Sigma_{sf} A_{sf}^T) + \Gamma_{ss} , 
% % \end{equation}
% % where $\bar{\gamma} = \lim_{t \to \infty} \gamma_{t+1}^{-1} - \gamma_t^{-1}$.
% % {\color{red}Note that this is a rough estimate, since the covariance of Polyak-Ruppert average on the slow iterate is not established in literature.}
% % Using the LMMSE property (\citet{HajekRandomProcess} Eq. 3.7)
% % \begin{equation}
% %     Cov(\tilde{x}_t - E[\tilde{x}_t | \hat{y}_t]) = Cov(\tilde{x}_t) - Cov(\tilde{x}_t, \hat{y}_t) Cov^{-1} (\hat{y}_t) Cov(\hat{y}_t, \tilde{x}_t) ,
% % \end{equation}
% % \textbf{\color{blue}Most important part: can we show that the $\tilde{x}_t$ covariance terms cancel out to obtain a rate of $Cov (\hat{y}_t) \sim \gamma_t$?}
% % Note that this implies 
% % \begin{align*}
% %     \alpha_t \Sigma_{ff} - \gamma_t H_t \Sigma_{fs} = O(\gamma_t) \Leftrightarrow 
% %     H_t = \frac{\alpha_t}{\gamma_t} (\Sigma_{ff} \Sigma_{fs}^{-1} + O(\gamma_t)) .
% % \end{align*}
% % {\color{blue}Since in this case we have that $H_t \gg \Delta$, perhaps we need to use induction by defining $\Phi$ with respect to $H_t$ excluding $\Delta$?}
% % If this is all true, then we can proceed with the former method of obtaining a bound on the expected norm of individual terms.


% Note that by definition of LMMSE, we have from Eq. 3.7 in \citep{hajekRandomProcess} that
% \begin{align*}
%     Cov(\tilde{x}_t - H_t \hat{y}_t ) = Cov(\tilde{x}_t) - Cov(\tilde{x}_t, \hat{y}_t) Cov(\hat{y}_t)^{-1} Cov(\hat{y}_t, \tilde{x}_t) .
% \end{align*}
% In the limit, we already know the RHS terms to be
% \begin{align*}
%     Cov(\tilde{x}_t - H_t \hat{y}_t ) \to \alpha_t \Sigma_{ff} - \gamma_t \Sigma_{fs} \Sigma_{ss}^{-1} \Sigma_{sf} .
% \end{align*}
% {\color{blue}The following may be useful:
% In the univariate case ($d = 1$), we have that
% }
% \begin{align*}
%     Cov(\tilde{x}_t - E[\tilde{x}_t | \hat{y}_t]) &= (1 - \rho_t^2) Var(\tilde{x}_t) \sim (1 - \rho_t^2) \alpha_t \\
%     E \lVert E[\tilde{x}_t | \hat{y}_t] \rVert &\leq \sqrt{Var E[\tilde{x}_t | \hat{y}_t]} \leq \sqrt{Var \tilde{x}_t} \sim \sqrt{\alpha_t} ,
% \end{align*}
% where the second line uses Jensen, law of total variance, and the MSE of $\tilde{x}_t$. 
% If we obtain a similar expression for the matrix case with $\rho_t^2 \geq 1 - t^{-1}$, then the covariance of LMMSE error term decays fast enough as 
% \begin{align*}
%     n^{-1/2} \sum_{t=1}^n \sqrt{(1 - \rho_t^2) \alpha_t} = n^{-1/2} \sum_{t=1}^n t^{-1/2 - \delta / 2} \sim n^{-\delta / 2}
% \end{align*}
% which decays for every $\delta > 0$ (this is the corresponding CLT rate if this term dominates).
% {\color{red}Notice that if this is true, then we have that while fast term benefits from small $\delta$, the slow term here benefits from large $\delta$.}






\section{Non-Asymptotic CLT for TSA-PR}\label{sec:clt}
We establish finite-time bounds on the Wasserstein-1 distance defined in Eq. \eqref{eq:wasserstein_definition} between the deviation of the Polyak-Ruppert average $(\bar{x}_n, \bar{y}_n)$ around the solution (zeros) and the limiting Gaussian vector.
Here we restate Theorem \ref{thm:clt} for reference.
\begin{theorem}
    \CLT
\end{theorem}
The asymptotic covariances $\bar{\Sigma}_{ff}$ and $\bar{\Sigma}_{ss}$ satisfy
\begin{equation}
    \begin{split}
        G \bar{\Sigma}_{ff} G^T
        &= \Gamma_{ff} + A_{fs} A_{ss}^{-1} \Gamma_{ss} (A_{fs} A_{ss}^{-1})^T 
        - \left(\Gamma_{fs} (A_{fs} A_{ss}^{-1})^T + A_{fs} A_{ss}^{-1} \Gamma_{sf}\right)  \\
        \Delta \bar{\Sigma}_{ss} \Delta^T
        &= \Gamma_{ss} + A_{sf} A_{ff}^{-1} \Gamma_{ff} (A_{sf} A_{ff}^{-1})^T
        - \left(\Gamma_{sf} (A_{sf} A_{ff}^{-1})^T + A_{sf} A_{ff}^{-1} \Gamma_{fs}\right)         
    \end{split} . \label{eq:ttsapr_covariance}
\end{equation}


\subsection{Proof of Theorem \ref{thm:clt}}
\textbf{Step 1:} We write the recursions \eqref{eq:ttsa} in the form
% We follow \citep{mokkadem2006convergence} and write the fast and slow timescale recursions as
\begin{align*}
    G x_t = (W_t - A_{fs} A_{ss}^{-1} V_t) + \alpha_t^{-1} (x_t - x_{t+1}) - \gamma_t^{-1} A_{fs} A_{ss}^{-1} (y_t - y_{t+1}) \numberthis \label{eq:fast_last} \\ 
    \Delta y_t = (V_t - A_{sf} A_{ff}^{-1} W_t) + \gamma_t^{-1} (y_t - y_{t+1}) - A_{sf} A_{ff}^{-1} \alpha_t^{-1} (x_{t+1} - x_t) ,
\end{align*}
where $G, \Delta$ are the Schur complements
\begin{align*}
    G = A_{ff} - A_{fs} A_{ss}^{-1} A_{sf} , \quad
    \Delta = A_{ss} - A_{sf} A_{ff}^{-1} A_{fs} .
\end{align*}
\begin{proof}    
    Let us start with the fast-time-scale variable update
    \begin{align*}
        x_{t+1} &= x_t - \alpha_t (A_{ff} x_t + A_{fs} y_t - W_t) 
        \\
        \Rightarrow x_t &= (\alpha_t A_{ff})^{-1} (x_{t+1} - x_t) - A_{ff}^{-1} (A_{fs} y_t - W_t) .
    \end{align*}
    Substituting into the slow-time-scale variable's recursion,
    \begin{align*}
        y_{t+1} &= y_t - \gamma_t \left(
            A_{sf} x_t + A_{ss} y_t - V_t 
        \right)
        \\ 
        &= y_t - \gamma_t A_{sf} \left( A_{ff}^{-1} \alpha_t^{-1} (x_{t+1} - x_t) \right) + \gamma_t A_{sf} A_{ff}^{-1} A_{fs} y_t - \gamma_t A_{sf} A_{ff}^{-1} W_t
        - \gamma_t A_{ss} y_t + \gamma_t V_t .
    \end{align*}
    Using $\Delta = A_{ss} - A_{sf} A_{ff}^{-1} A_{fs}$, we have 
    \begin{align*}
         y_{t+1} &= (I - \gamma_t \Delta) y_t - \frac{\gamma_t}{\alpha_t} A_{sf} A_{ff}^{-1} (x_{t+1} - x_t) + \gamma_t (V_t - A_{sf} A_{ff}^{-1} W_t)  \numberthis \label{eq:prelim}\\
        \Leftrightarrow 
         \gamma_t \Delta y_t &=  y_t - y_{t+1} - \frac{\gamma_t}{\alpha_t} A_{sf} A_{ff}^{-1} (x_{t+1} - x_t) + \gamma_t (V_t - A_{sf} A_{ff}^{-1} W_t).
    \end{align*}
    Dividing both sides by the step size $\gamma_t$, we have
    \begin{equation}
        \Delta y_t = \gamma_t^{-1} (y_t - y_{t+1}) - \alpha_t^{-1} A_{sf} A_{ff}^{-1} (x_{t+1} - x_t) +(V_t - A_{sf} A_{ff}^{-1} W_t) .
    \end{equation}

    We repeat the same steps for the fast iterate:
    Using that 
    \begin{align*}
        A_{ss} y_t = \gamma_t^{-1} (y_t - y_{t+1}) - (A_{sf} x_t - V_t) ,
    \end{align*}
    we have by substitution
    \begin{align*}
        x_{t+1} &= x_t - \alpha_t (A_{ff} x_t - W_t) - \alpha_t A_{fs} y_t 
        \\
        &= x_t - \alpha_t A_{ff} x_t + \alpha_t W_t - \alpha_t A_{fs} A_{ss}^{-1}\left(\gamma_t^{-1} (y_t - y_{t+1}) - (A_{sf} x_t - V_t) \right) 
        \\ & = 
        x_t - \alpha_t \left(A_{ff} - A_{fs} A_{ss}^{-1} A_{sf} \right) x_t + \alpha_t (W_t - A_{fs} A_{ss}^{-1} V_t) - \frac{\alpha_t}{\gamma_t} A_{fs} A_{ss}^{-1} (y_t - y_{t+1}) .
    \end{align*}
    Denoting $G = A_{ff} - A_{fs} A_{ss}^{-1} A_{sf}$, we have 
    \begin{equation}
        G x_t = \alpha_t^{-1} (x_t - x_{t+1}) + (W_t - A_{fs} A_{ss}^{-1} V_t) - \gamma_t^{-1} A_{fs} A_{ss}^{-1} (y_t - y_{t+1}) .
    \end{equation}
\end{proof}
The Polyak-Ruppert averages $\bar{x}_n, \bar{y}_t$ are obtained by taking the average on both sides:
\begin{align*}
    G \bar{x}_n &= \frac{1}{n} \sum_{t=1}^n (W_t - A_{fs} A_{ss}^{-1} V_t) + \frac{1}{n} \sum_{t=1}^n \alpha_t^{-1} (x_{t} - x_{t+1}) - \frac{1}{n} \sum_{t=1}^n \gamma_t^{-1} A_{fs} A_{ss}^{-1} (y_t - y_{t+1}) \\ 
    \Delta \bar{y}_n &= \frac{1}{n} \sum_{t=1}^n (V_t - A_{sf} A_{ff}^{-1} W_t) 
    + \frac{1}{n}\sum_{t=1}^n A_{sf} A_{ff}^{-1} \alpha_t^{-1} (x_{t} - x_{t+1})
    + \frac{1}{n}\sum_{t=1}^n \gamma_t^{-1} (y_{t} - y_{t+1})  .    
\end{align*}


\textbf{Step 2:} Quantitative bounds on the central limit theorem. 

Asymptotic convergence of $\sqrt{n} (\bar{x}_n, \bar{y}_n)$ to a normal distribution can then be proved by showing that (1) the first noise terms above satisfy Lindeberg's condition and that (2) the remaining terms converge to zero with probability 1.
Slutsky's theorem then implies that the sum $N_t + E_t \to N + c$ when the noise $N_t$ converges in distribution to a random variable $N$ and the error terms $E_t$ converge to a constant $c$ with probability 1. 



For non-asymptotic analysis, we use the mean-square results for the last iterates to obtain finite-time bounds on the telescoped weighted differences above, where they converge in probability to 0.
Next, we use a bound on the Wasserstein-1 distance between the first terms and a normal random variable. 
These results can be combined using Lindeberg's decomposition to obtain finite-time bounds on the distance between $\bar{x}_n$ and $\bar{y}_n$ to a Gaussian random variable. 


Lemma \ref{lem:slutsky} resembles Slutsky's theorem for a sum of two convergent random sequences, but can be used to obtain finite-time bounds on the Wasserstein-1 distance when the random sequences being summed converge with respect to the Wasserstein-1 distance and mean absolute error. 
% \begin{lemma}[Lemma \ref{lem:slutsky}]
% \begin{lemma}[Lemma \ref{lem:slutsky}]
%     Consider two random sequences $\{X_t\}, \{Y_t\}$ such that $d_1 (X_t, X) \leq r_t$ for some random variable $X$ and $\mathbb{E}\lVert Y_t \rVert \leq r'_t$ for some constant $Y$.
%     Then, $d_1 (X_t + Y_t, X) \leq r_t + r'_t$.
% \end{lemma}
\begin{lemma*}
    (Restated Lemma \ref{lem:slutsky})
    Consider two random sequences $\{X_t\}, \{Y_t\}$ such that $d_1 (X_t, X) \leq r_t$ for some random variable $X$ and $\mathbb{E}\lVert Y_t \rVert \leq r'_t$.
    Then, $d_1 (X_t + Y_t, X) \leq r_t + r'_t$.
\end{lemma*}
\begin{proof}
    By the definition \eqref{eq:wasserstein_definition}, we have that
    \begin{align*}
        d_1 (X_t + Y_t, X) &= \sup_{h \in \mathrm{Lip}_1} \mathbb{E}[h(X_t + Y_t) - h(X)] 
        \\ 
        &= \mathbb{E}\left[h(X_t + Y_t) - h(X_t)\right] + \mathbb{E}h(X_t) - \mathbb{E} h(X) 
        \\ & \leq 
        \mathbb{E}\lVert Y_t \rVert + d_1 (X_t, X) 
        ,
    \end{align*}
    where the last inequality uses that $h$ in 1-Lipschitz. 
    This statement demonstrates that the Wasserstein-1 distance between a sum and its limit can be decomposed into mean absolute errors and the Wasserstein-1 distance of a summand. 
\end{proof}
Next, we use a non-asymptotic central limit theorem (Theorem 1, \citet{srikant2024CLT}) for martingale differences, simplified for our application.
\begin{lemma*}
    % {\color{red}Revise the statement in the main draft;}
    (Restated Lemma \ref{lem:CLT})
    Let $\{N_t\}$ be a martingale difference sequence satisfying Assumption \ref{assumption:noise}. 
    Denoting $Z \sim \mathcal{N}(0, I)$ to be the standard Gaussian vector, we have that
    \begin{equation}
        d_1 \left(n^{-1/2} \sum_{t=1}^n N_t, \Gamma^{1/2} Z\right) \leq 
        \mathcal{O}\left(\frac{d}{1-\beta}\right) \frac{\lVert \Gamma^{1/2} \rVert }{n^{\beta / 2}} \left(\lVert \Gamma^{-1/2} \rVert^{2 + \beta} + \lVert \Gamma^{-1/2} \rVert^\beta \right).
    \end{equation}
\end{lemma*}
% \begin{proof}
%     Assumption \ref{assumption:noise} implies that all $\leq 2 + \beta$ moments of $N_t$ are finite. 
%     Therefore, the assumptions in Theorem 1 of \citep{srikant2024CLT} are satisfied.
%     % This is a special case of Theorem 1 in \citep{srikant2024CLT}.     
% \end{proof}
It remains to show that the expected norm of the remaining terms decay to zero.
Observe that 
\begin{align*}
    \sum_{t=1}^n \alpha_t^{-1} (x_t - x_{t+1}) = \frac{x_1}{\alpha_1} - \frac{x_{n+1}}{\alpha_{n}} + \sum_{t=1}^{n-1} (\alpha_{t+1}^{-1} - \alpha_{t}^{-1}) x_t .
\end{align*}
The step sizes in Assumption \ref{assumption:steps} satisfy
\begin{align*}
    \alpha_{t+1}^{-1} - \alpha_{t}^{-1} \leq (t \alpha_t)^{-1} ,
\end{align*}
and we have from triangle inequality that
\begin{equation}\label{eq:fast_telescope}
    \mathbb{E} \lVert \sum_{t=1}^n \alpha_t^{-1} (x_t - x_{t+1}) \rVert \leq \alpha_1^{-1} \mathbb{E}\lVert x_1 \rVert  + \alpha_n^{-1} \mathbb{E}\lVert x_{n+1} \rVert + (\alpha_1)^{-1} \sum_{t=1}^n t^{a - 1} \mathbb{E} \lVert  x_t \rVert .
\end{equation}
Repeating for $\{y_t\}$, we have
\begin{equation}\label{eq:slow_telescope}
    \mathbb{E} \lVert \sum_{t=1}^n \gamma_t^{-1} (y_t - y_{t+1}) \rVert \leq \gamma_1^{-1} \mathbb{E} \lVert y_1 \rVert + \gamma_n^{-1} \mathbb{E} \lVert y_{n+1} \rVert + (\gamma_1)^{-1} \sum_{t=1}^n t^{b - 1} \mathbb{E} \lVert  y_t \rVert .
\end{equation}
We have from Theorem \ref{thm:mse} and Jensen's inequality that
\begin{align*}
    \mathbb{E}\lVert x_{t} \rVert = \mathcal{O}(\sqrt{\alpha_t} + \sqrt{\gamma_t}), 
    \quad 
    \mathbb{E} \lVert y_{t} \rVert = \mathcal{O}(\sqrt{\gamma_t}) .
\end{align*}
Substituting, we obtain
\begin{align*}
    \mathbb{E} \lVert \sum_{t=1}^n \alpha_t^{-1} (x_t - x_{t+1}) \rVert + \mathbb{E}\lVert \sum_{t=1}^n \gamma_t^{-1} (y_t - y_{t+1}) \rVert 
    = \mathcal{O}\left(n^{a/2} + n^{a-b/2} + n^{b/2}\right) .
\end{align*}
Combining with Lemma \ref{lem:CLT} and \ref{lem:slutsky} yields Theorem \ref{thm:clt}.
Similar to how we included the transient term $\sqrt{\gamma_t}$ for $\mathbb{E}\lVert x_t\rVert$ to obtain the $n^{a-b/2}$ in the above equation, a refined inspection of all rates in the proof of Theorem \ref{thm:mse} illustrate how to control the transient terms in the Wasserstein-1 distance for the Polyak-Ruppert averaged errors in Theorem \ref{thm:clt}. 





% \section{Auxiliary Lemmas}
% \subsection{Two Timescale Generalized Single Timescale}
Under Assumption \ref{assumption:structure}, there exists a matrix $A$ that is not Lyapunov stable. 
\begin{align*}
    A_{ff} = \begin{pmatrix}
        2 & 1 \\ 1 & 2
    \end{pmatrix} ,
    A_{fs} = \begin{pmatrix}
        1 & 0
    \end{pmatrix}^T , 
    A_{sf} = A_{fs}^T, 
    A_{ss} = 1 .
\end{align*}
Here we see that the eigenvalues of $A_{ff}$ are $1$ and $3$, so Assumption \ref{assumption:structure} is satisfied. 
But $A + A^T$ has an eigenvalue $4 - \sqrt{6}$, and therefore is not positive definite. 


To see that Assumption \ref{assumption:structure} is more general, it can also be shown that $A + A^T \succ 0$ implies Assumption \ref{assumption:structure}.
{\color{red}This is not true; $A + A^T \succ 0$ implies $A_{ff} + A_{ff}^T \succ 0$ but not $\Delta + \Delta^T \succ 0$.}



% \begin{proposition}
%     Assumption \ref{assumption:structure} {\color{red}implies that} $A_{ss}$ is invertible and that its Schur complement $G$ also satisfies $G + G^T \succ 0$. 
% \end{proposition}
% Let us start with the Schur decomposition of $A$:
% \begin{equation}
%     A = \begin{pmatrix}
%         I & 0 \\ A_{sf} A_{ff}^{-1} & I 
%     \end{pmatrix}
%     \begin{pmatrix}
%         A_{ff} & 0 \\ 0 & \Delta 
%     \end{pmatrix}
%     \begin{pmatrix}
%         I &  A_{ff}^{-1} A_{fs} \\ 0 & I
%     \end{pmatrix} .
% \end{equation}
% To 


% {\color{blue}
% Reason for checking: Is $A + A^T \succ 0$? If so, what is the reason for using TTSA? 
% If in TTSA we can set $a \approx b$, and we know that STSA achieves optimal rate, what is the motivation for using TTSA?
% TTSA would only be beneficial if $A + A^T$ is not necessarily positive definite, meaning that we can apply TTSA to a broader class of problems/update rules. 
% }

% (1) First check that $A_{ss}$ is invertible; if so, we can write the other Schur decomposition and compare the two. 
% {\color{blue}Fix this and the parts that use this formula (in the main draft). 
The changes won't be too big, but I need to make things precise. 
}
\begin{proposition}\label{prop:induction_size}
    Let $\alpha_t = \alpha_1 t^{-a}$ with $a \in (0, 1)$.
    For any $0 < \alpha_1 \mu < 1$ and $n \geq 1$, 
    \begin{equation}
        \sum_{t=1}^n \frac{\alpha_t}{t} \prod_{j=t+1}^n \left(1 - \mu \alpha_j \right) 
            \leq {\color{red}K}\frac{\alpha_n}{n}
        % \frac{1}{1-a} \frac{\alpha_n}{n} \exp \left(-\mu \alpha_n\right) .
        % K_1 \alpha_n,
    \end{equation}
\end{proposition}

\begin{proof}
    This is shown by induction.
    The base case clearly holds, where $\prod_{j=n+1}^n (1 - \mu \alpha_j)$ is defined to be one. 
    Next,
    \begin{align*}
        \sum_{t=1}^n \frac{\alpha_t}{t} \prod_{j=t+1}^n \left(1 - \mu \alpha_j\right) 
        \leq 
        \frac{\alpha_n}{n} + {\color{red}K} \frac{\alpha_{n-1}}{n-1} .
    \end{align*}

    Using that $\alpha_t/\alpha_{t+1} = (1 + 1/t)^a \leq 2$ is a decreasing sequence, we have
    \begin{align*}
        \sum_{t=1}^n \frac{\alpha_t}{t} \prod_{j=t+1}^n \left(1 - \mu \alpha_j \right) 
        &= \alpha_{n+1} \sum_{t=1}^n \frac{1}{t}\frac{\alpha_{n}}{\alpha_{n+1}} \frac{\alpha_t}{\alpha_n} \prod_{j=t+1}^n \left(1 - \mu \alpha_j \right) 
        \\
        & \leq 2 \alpha_{n+1} \sum_{t=1}^n \frac{1}{t} \prod_{j=t+1}^n \frac{\alpha_j}{\alpha_{j+1}} \left(1 - \mu \alpha_j\right) .
    \end{align*}
    % Using Lemma 12 in \citep{kaledin2020finite}, we have that
    % Using $1 - x \leq \exp (-x)$, we have that 
    % \begin{align*}
    %     \prod_{j=t+1}^n (1 - \mu \alpha_j) 
    %     \leq 
    %     \exp \left(-\frac{\mu \alpha_1}{1-a} \left(
    %         n^{1-a} - (n-1)^{1-a}
    %         \right)
    %     \right)
    %     \leq 
    %     \exp \left(-\mu \alpha_{n-1} \right) ,
    % \end{align*}
    % where the last inequality uses $n^{1-a} - (n-1)^{1-a} \geq (1-a) (n-1)^{-a}$. 
    % Therefore,
    % \begin{align*}
    %     \sum_{t=1}^n \frac{\alpha_t}{t} \prod_{j=t+1}^n (1 - \mu \alpha_j) 
    %     \leq \exp \left(-\mu \alpha_{n-1}\right) \sum_{t=1}^n \frac{\alpha_t}{t} 
    %     \leq \frac{1}{1-a}\exp \left(-\mu \alpha_{n-1}\right) (n+1)^{1-a}.
    % \end{align*}
    
    % For $a < 1$, we have by $1 - x \leq \exp (-x)$ for all $x \in [0, 1]$,
    % \begin{align*}
    %     \prod_{j=t+1}^n (1 - \mu \alpha_j) \leq \exp \left(-\mu \alpha_1 (\sum_{j=t}^n j^{-a} - t^{-a})\right) 
    %     & \leq 
    %     \exp \left(-\frac{\mu \alpha_1}{1-a} 
    %         \left(n^{1-a} - t^{1-a} - 1 \right)
    %     + \mu \alpha_1 t^{-a}
    %     \right)
    %     \\ &
    %     = \exp \left(-\frac{\mu \alpha_1}{1-a}  (n^{1-a} - 2) \right) \exp \left( \frac{\mu \alpha_1}{1-a} t^{1-a}\right) .
    % \end{align*}
    % Next,
    % \begin{align*}
    %     \sum_{t=1}^n t^{-a-1} \exp \left(K_0 t^{1 - a} \right)
    %     % \leq K_1 \int_1^n t^{-a-1} \exp \left(K_0 t^{1-a}\right) dt
    %     \leq \left( \sum_{t=1}^n t^{-4}\right)^{1/2} \left(\sum_{t=1}^n t^{2(1 - a)} \exp \left(2 K_0 t^{1- a}\right)\right)^{1/2}
    % \end{align*}
    % {\color{red}First is $t^{-3/2}$, second is roughly $t^{1-a} \exp (K_0 t^{1-a})$ due to $x \exp x$ having an integral similar to a exp without the $x$.
    % Therefore, if the exponentials cancel out then we will have maybe something like $t^{1/2 - a}$ (if the 2 in the $t$ exponent remains, which gives after subtracting by $3/2$ the half); therefore, this may go to zero when $a > 1/2$, and may be where I need the requirement.  
    % But this wouldn't give $o(\alpha_n)$; need to make sure that whatever bound we get, we have a $o(\alpha_n)$, i.e. $t^{-a}$ times a decaying term. 
    % }


    % {\color{red}Case $a = 1$; may not be necessary.}
    % \begin{align*}
    %     \prod_{j=t+1}^n (1 - \mu \alpha_j ) 
    %         = 
    %     \prod_{j=t+1}^n \left(1 - \frac{\mu \alpha_1}{t}\right) 
    %         &\leq
    %     \exp \left(-\mu \alpha_1 (\log n - \log t)\right) \exp\left(\mu \alpha_1 + \alpha_t\right)          
    %     \\
    %     &        \leq
    %     \exp \left((\mu + 1) \alpha_1\right) \left(\frac{t}{n}\right)^{\mu \alpha_1} .
    %     % &
    %     % \leq \exp \left(-K_0 (\log n - \log t)\right)
    %     % = \left(\frac{t}{n}\right)^{K_0} .
    %     % \leq \left(\frac{2}{n}\right)^{\mu \alpha_1} .
    % \end{align*}
    % Therefore, there exists some constant $K_1 > 0$ such that
    % \begin{align*}
    %     \sum_{t=1}^n \frac{\alpha_t}{t} \prod_{j=t+1}^n \left(1 - \mu \alpha_1 \right) 
    %         \leq 
    %     \frac{\alpha_1 e^{(\mu + 1) \alpha_1}}{n^{\mu \alpha_1}} \sum_{t=1}^n t^{-a - 1 + \mu \alpha_1}
    %     \leq K_1 \frac{\alpha_1 e^{(\mu + 1)\alpha_1}}{\mu \alpha_1 - a} n^{-a} .
    % \end{align*}
    % {\color{red}Finish up!
    % Note that I need to strengthen the bound by, instead of using $a=1$ as the extreme case for the above bound, derive (and re-state for $a < 1$ instead of $a \leq 1$) a bound that decays at rate $o(\alpha_n)$. 
    % }


    % {\color{blue}A stronger bound for $a < 1$;}
    % \begin{align*}
    %     \prod_{j=t+1}^n 
    % \end{align*}
\end{proof}



% {\color{red}Lower bound:} Setting $\phi_j = j^{-a} \prod_{k=j+1}^t (1 - c k^{-a})$, we use $\log (1 - x) \geq - (x + x^2/2)$ for $x \in (0, 1)$ to get {\color{red}Check}
% \begin{align*}
%     \log \phi_j = -a \log j + \sum_{k=j+1}^t \log (1 - c k^{-a}) 
%     &\geq - a \log j - \sum_{k=j+1}^t (ck^{-a} + \frac{c^2}{2} k^{-2a}) 
%     \\ & 
%     \geq 
%     -a \log j - \frac{2 t^{1-a}}{1-a} + \frac{2 t^{1-2a}}{1 - 2a}
% \end{align*}


% \begin{proposition}\label{prop:fast_decay_steps}
%     If $\mu I \preceq A + A^T \preceq \nu I$ and $\alpha \leq 2 \mu/\nu^2$,
%     \begin{align*}
%         \lVert I - \alpha A \rVert \leq 1 - \frac{\mu}{2} \alpha .
%     \end{align*}
% \end{proposition}
% \begin{proof}
%     \begin{align*}
%         \lVert I - \alpha A \rVert^2 = \sup_{x \neq 0} \left\{1 - \alpha \frac{x^T (A^T + A) x}{\lVert x \rVert^2} + \alpha^2 \frac{x^T A^T A x}{\lVert x \rVert^2}\right\} 
%         \leq 
%         1 - \alpha \mu + \alpha^2 \frac{\nu^2}{4} ,
%     \end{align*}
%     where the last step uses that $\nu = \lVert A + A^T \rVert = 2 \lVert A \rVert$ to get $\lVert A^T A \rVert \leq \lVert A \rVert^2 \leq \nu^2/4$.
% \end{proof}

% Using $1 - x \leq e^{-x}$ for all $x \in [0, 1]$, $\alpha_k = t^{-a} $, and $\sum_{k=j+1}^t \alpha_k \leq \frac{t^{1-a}}{1-a} - \frac{j^{1-a}}{1-a}$, we have 
% \begin{align*}
%     \prod_{k=j+1}^t (1 - \alpha_k \frac{\mu_{ff}}{2}) 
%         &\leq \exp \left(-\frac{\mu_{ff}}{2(1-a)} \left(t^{-a+1} - (j+1)^{-a+1}\right) \right) .
% \end{align*}
% % Moreover, we use that $t^{-a+1} - (t-1)^{-a + 1} \leq \alpha_t$ {\color{red}Check! This is kind of like binomial expansion; or maybe this is related to the step-size condition?} to obtain that
% % \begin{align*}
% %     \prod_{k=j+1}^t (1 - \alpha_k \frac{\mu_{ff}}{2}) \leq \exp\left(-\mu j^{-a}\right) \leq \frac{1}{j \alpha_j}
% % \end{align*}
% {\color{red}Show below or faster rates; I think this is the correct rate based on expected results:
% Note that equality holds when $\alpha_t \approx t^{-1}$. Maybe show that this quantity is monotonic in the exponent.
% }
% \begin{align*}
%     \phi_{jt} = \alpha_j \prod_{k=j+1}^t \left(1 - \alpha_k \frac{\mu_{ff}}{2}\right) \leq \frac{1}{j} .
% \end{align*}

% 

\subsection{Trace inequality for $g_t$}
{\color{red}Exercise verifying trace inequality.}
\begin{align*}
    g_t (C_t) &= 
    -\mathrm{Tr}\delta_t \left( A_{sf} C_t + \Delta^T \right) A_{sf}
    + 
    \alpha_t \mathrm{Tr} A_{ff} C_t \Delta^T  A_{sf}
    - \alpha_t \mathrm{Tr} A_{ff} C_t L_t^T A_{sf}^T A_{sf}
    \\ & 
    + \gamma_t \mathrm{Tr} \delta_t A_{sf} C_t \Delta^T A_{sf}
    + \mathrm{Tr} C_t L_t^T A_{sf}^T A_{sf}
    + \gamma_t \mathrm{Tr} \delta_t A_{sf} C_t L_t^T A_{sf}^T A_{sf}
    \\ &
    + \alpha_t \mathrm{Tr} \left(A_{ff} X_t A_{sf}^T 
    + \frac{\gamma_t}{\alpha_t} \left(\delta_t \Gamma_{ss} - \delta_t A_{sf} X_t A_{sf}^T \right) A_{sf}\right) .
\end{align*}


We can use the inequality that for arbitrary matrices $A, X$ of appropriate size, 
\begin{align*}
    \lvert \mathrm{Tr} A^T X \rvert \leq \lVert A \rVert \mathrm{Tr} \lvert X \rvert ,
\end{align*}
where $\lvert X \rvert = (X^T X)^{1/2}$ and $\lVert \cdot \rVert$ is the spectral norm.
Then we have that 
\begin{align*}
    \lvert g_t (C_t) \rvert \leq 
    \lVert \delta_t A_{sf} \rVert \mathrm{Tr} \lvert A_t C_t \rvert 
    + 
    \left(
        \alpha_t \lVert \Delta^T A_{sf} \rVert 
        - \alpha_t \lVert L_t^T A_{sf}^T \rVert 
    \right)
    \mathrm{Tr} \lvert A_{sf} A_{ff} C_t \rvert
    + \gamma_t \lVert \delta_t + \cdots .
\end{align*}
For the terms where there are more matrices inside, I think we can pull them out. 


Q. 
Can we instead write the recursion in terms of $\mathrm{Tr} \lvert C_t A_{sf} \rvert$?
Then, prove that 
\begin{align*}
    \mathrm{Tr} C_t A_{sf} \leq \mathrm{Tr} \lvert C_t A_{sf} \rvert .
\end{align*}
This is true because
\begin{align*}
    \mathrm{Tr} C_t A_{sf} \leq \lvert \mathrm{Tr} C_t A_{sf} \rvert \leq \mathrm{Tr} \lvert C_t A_{sf} \rvert
\end{align*}
(think of the singular values).
{\color{red}Actually, I think it will be easiest if we just prove the inequality for $\lvert g_t (C_t)$, since then the result is immediate.
Problem is, then we can't subsume the terms in $g_t (C_t)$ into the recursion; since we don't a-priori know that $C_t$ is uniformly bounded, we can't really say that $g_t (C_t)$ is bounded. 
}

% We want an inequality of the form
% \begin{align*}
%     \mathrm{Tr} A X \leq a \lvert \mathrm{Tr} X \rvert .
% \end{align*}
% Then go back and consider the two cases separately, when $\mathrm{Tr} C_t A_{sf}$ is positive vs. negative.
% One simple inequality is
% \begin{align*}
%     \lvert \mathrm{Tr} A X \rvert = \lvert \sum_{ij} A_{ij} X_{ij} \rvert \leq \max_{ij} \lvert X_{ij} \rvert \lVert A \rVert_{11}
% \end{align*}

% We can use the Frobenius inner product to express the trace of a product of two arbitrary rectaungular matrices $A, B$, which is subject to Cauchy-Schwartz inequality
% \begin{align*}
%     \mathrm{Tr} A^T B = \langle A, B \rangle_F \leq \lVert A \rVert_F \lVert B \rVert_F .
% \end{align*}
% % Using that $\lVert A \rVert_F = \sqrt{\sum_{i} \sigma_i^2 (A)} \leq \sum_{i} \sigma_i (A) = \sum_i \lvert \lambda_i (A)$, we then have, using the notation that for a matrix $A = U \Lambda V$, where $\Lambda$ is the singular value matrix of $A$, $\lvert A \rvert = U \lvert \Lambda \rvert V$ with $\lvert \cdot \rvert$ applied element-wise to a diagonal matrix:
% Therefore, we can express the trace of the product as a weighted multiple of the norm of singular values of $B$:
% \begin{align*}
%     \mathrm{Tr} A^T B \leq \lVert A \rVert_F \lVert \sigma_B \rVert_2 .
% \end{align*}


% For example if we want to obtain a bound on the first term in $g_t (C_t)$,
% \begin{align*}
%     \mathrm{Tr} (\delta_t A_{sf}) (C_t A_{sf}) \leq  \lVert \delta_t A_{sf} \rVert_F \lVert C_t A_{sf} \rVert_F
% \end{align*}

% \section{Below sections are old.}


% Delegated details on bounds, following Polyak Juditsky, Srikant, Thinh, and Kaledin
% \section{Further Experimental Details}
\label{apdx:details}
In this section, we append further experimental details and provide formal definitions of the baselines evaluated in the manuscript.

\subsection{Implementation Details}
For supervised fine-tuning, we utilize Low-rank Adaptation~\cite{hulora}.
In Table~\ref{tab:hyperparam}, we disclose detailed LoRA configurations and other training hyperparameters used for supervised fine-tuning.

\begin{figure*}[!ht]
    \centering
    \includegraphics[width=\linewidth]{rebuttal-figures-src/hyperparams.pdf}
    \vspace{-1.5em}
    \caption{Concept Sliders Comparison \& Hyperparameter analysis: (Left) Impact of PCA directions: SliderSpace with 10 directions matches the FID of 64 Concept Sliders. More directions, upto 40, leads to improved FID. (Right) Effect of LoRA rank: Given a fixed training budget rank-one sliders are efficient than higher rank versions and outperforms Concept Sliders}
    \vspace{-0.3em}
    \label{fig:reb-hyperparam}
\end{figure*}



\subsection{Baseline Definitions}
\label{app:baseline}
Here, we provide formal definitions for each baseline compared in Table~\ref{tab:req1}.


\noindent\textbf{Definition 1.} \textit{(\textbf{Zlib Score}) is the negated ratio of the log perplexity and the zlib compression size:}
\begin{equation}
    -\frac{1}{n}\sum_{i=1}^n \frac{ -\frac{1}{|\mathcal{T}_i|}\sum_{x_j \in \mathcal{T}_i} \log P_\theta(x_j | x_{<j})}{\text{Zlib}(\mathbf{x}_i).\text{size}},
\end{equation}
\textit{where $\mathcal{T}_i$ is the set of tokens from sample $i$.}~\cite{carlini2021extracting}

\noindent\textbf{Definition 2.} \textit{(\textbf{Perplexity Score}) is the negated average perplexity across samples:}
\begin{equation}
    -\frac{1}{n}\sum_{i=1}^n \text{exp}\bigg(-\frac{1}{|\mathcal{T}_i|}\sum_{x_j \in \mathcal{T}_i} \log P_\theta(x_j | x_{<j})\bigg),
\end{equation}
\textit{where $\mathcal{T}_i$ is the set of tokens from sample $i$.}~\cite{li2023estimating}


\noindent\textbf{Definition 3.} \textit{(\textbf{Min-K\% Score}) is the negated mean probability from bottom-$k\%$ tokens averaged across samples:}
\begin{equation}
    -\frac{1}{n \cdot |\mathcal{K}_i|}\sum_{i=1}^n \sum_{x_j \in \mathcal{K}_i} \log P_\theta(x_j | x_{<j}),
\end{equation}
\textit{where $\mathcal{K}_i$ is the set of bottom-$k\%$ tokens from sample $i$.}~\cite{shidetecting}

\noindent\textbf{Definition 4.} \textit{(\textbf{Min-K\%++ Score}) is the negated mean normalized probability from bottom-$k\%$ tokens averaged across samples:}
\begin{equation}
    -\frac{1}{n \cdot |\mathcal{K}_i|}\sum_{i=1}^n \sum_{x_j \in \mathcal{K}_i} \frac{\log P_\theta(x_j | x_{<j}) - \mu_{x_{<j}}}{\sigma_{x_{<j}}},
\end{equation}
\textit{where $\mathcal{K}_i$ is the set of bottom-$k\%$ tokens from sample $i$, $\mu_{x_{<j}} = \mathbb{E}_{z\sim p(\cdot | x_{<j})} [\log p(z | x_{<j})]$ is the expected log probability over the vocabulary of the model, and $\sigma_{x_{<j}} = \sqrt{\mathbb{E}_{z\sim p(\cdot | x_{<j})} [(\log p(z | x_{<j}) - \mu_{x_{<j}})^2]}$ is the standard deviation.}~\cite{zhang2024min}

Following the general guideline from \citet{shidetecting}, we take the bottom 20\% tokens for the Min-K\% Score and Min-K\%++ Score.

\noindent\textbf{Definition 5.} \textit{(\textbf{Fine-tuned Score Deviation}) is the difference of scores before and after supervised fine-tuning, averaged across samples:}
\begin{equation}
    \frac{1}{n}\sum_{i=1}^n S(\mathbf{x}_i; \theta) - S(\mathbf{x}_i; \theta'),
\end{equation}
\textit{where $\mathbf{x}_i$ is the $i$-th sample in the dataset, $S(\cdot;\cdot)$ is an existing scoring function~(e.g., Min-K\% or Perplexity Score), and $\theta, \theta'$ are models before and after fine-tuning, respectively.}~\cite{zhang2024fine}


\noindent\textbf{Definition 6.} \textit{(\textbf{Sharded Rank Comparison Test}) is the difference between the log likelihood of the canonical dataset sample ordering from the mean over shuffled sample orderings, averaged across dataset shards:}
\begin{equation}
    \frac{1}{r} \sum_{k=1}^r \bigg[ \log P([x_i^{(k)}]_{i=1}^n) - \frac{1}{|\frak{S}|} \sum_{\sigma \in \frak{S}} \log P([x_{\sigma(i)}^{(k)}]_{i=1}^n) \bigg],
\end{equation}
\textit{where $r$ is the number of shards, $\frak{S}$ is the set of sample permutations, and $[x_i^{(k)}]_{i=1}^n$ is the sequence of samples $x_1, x_2, \ldots, x_n$ in $k$-th shard of the dataset.}~\cite{orenproving}


\end{document}