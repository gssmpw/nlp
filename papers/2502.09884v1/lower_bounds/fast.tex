
\textbf{Lower Bound:} {\color{red}Revise to be compatible with above.}
It is also possible to derive a lower bound for $\mathrm{Tr} X_t$.
In this context, a lower bound demonstrates that the analysis is tight for the class of algorithms considered.
% , rather than an information-theoretic bound.
\begin{lemma}\label{lem:fast_lb}
    \begin{equation}
        \mathrm{Tr} X_{t+1} = \Omega\left(\mu_{ff} \sum_{j=1}^t \alpha_j^2 \prod_{k=j+1}^t \left(1 - \alpha_k \frac{\mu_{ff}}{2}\right) \mathrm{Tr} \Sigma_{ff}\right) .
    \end{equation}    
    {\color{red}In particular, this can be simplified to?}
\end{lemma}

Starting from Eq. \eqref{eq:fast_recursion}, we have 
\begin{align*}
    X_{t+1} \succeq (1 - \alpha_t \nu_{ff}) X_t + \alpha_t^2 \mu_{ff} \Sigma_{ff} + \gamma_t F_t (X_t) .
\end{align*}
Using that $\mathrm{Tr} A B \leq \mathrm{Tr} \Lambda_A \Lambda_B$ for symmetric matrices $A, B$ with eigenvalue matrices $\Lambda_A, \Lambda_B$ \citep{Theobald_1975},
\begin{align*}
    f_t (X_t) 
    &= 
    -\mathrm{Tr}\left( (\delta_t A_{sf} + (\delta_t A_{sf})^T) X_t \right) + \mathrm{Tr} \left(\gamma_t \delta_t \Gamma_{ss} \delta_t^T - (\Gamma_{fs} \delta_t^T + (\delta_t \Gamma_{fs})^T)\right) 
    \\
    &\geq - M_{f} \mathrm{Tr} X_t
    + \lvert \mathrm{Tr}\left(\Gamma_{fs} \delta_t^T + \delta_t \Gamma_{sf} )\right) 
    \rvert
    =: -M_f \mathrm{Tr}X_t + v_t
    ,
\end{align*}
where we use that $\delta_t$ is uniformly bounded to get the constant $M_{f}$ defined in the upper bound proof, and that $v_t$ is uniformly bounded by some $v$.
Substituting the lower bound for $\mathrm{Tr} F_t (X_t)$, we then have
\begin{equation}
    \mathrm{Tr} X_{t+1} \geq (1 - \alpha_t \nu_{ff} - \gamma_t M_f) \mathrm{Tr} X_t + \alpha_t^2 \mu_{ff} \mathrm{Tr} \Sigma_{ff} + \gamma_t v .
\end{equation}
Using that $\alpha_t < 1/\nu_{ff}$ {\color{red}step!},
% Using that $\alpha_t \nu_{ff} + \gamma_t M_f \leq \alpha_t 2 \nu_{ff}$ whenever $\gamma_t/\alpha_t \leq \nu_{ff}/M_f$, {\color{red}This is for lower bound; not mentioned in step size assumption}
\begin{align*}
    \mathrm{Tr}X_{t+1} 
    &\geq \prod_{k=1}^t (1- \alpha_k \nu_{ff}) \mathrm{Tr} X_1 
    - \sum_{j=1}^t \gamma_j \prod_{k=j+1}^t \left(1 - \alpha_k \nu_{ff}\right) \mathrm{Tr} X_j \\
    &+ \sum_{j=1}^t \alpha_j^2 \prod_{k=j+1}^t (1 - \alpha_k \nu_{ff}) \mathrm{Tr} \Gamma_{ff} 
    + \sum_{j=1}^t \gamma_j \prod_{k=j+1}^t (1 - \alpha_k \nu_{ff}) v.    
\end{align*}
Using our previous result that $\mathrm{Tr} X_j = \mathcal{O}(\alpha_j)$ and that $\gamma_j < \alpha_j$, we see that the second term is smaller than the third term.
To bring in the dependency on $\Sigma_{ff}$, we use that $\mathrm{Tr} \Gamma_{ff} \geq \mu_{ff} \mathrm{Tr} \Sigma_{ff}$ to obtain
\begin{equation}
    \mathrm{Tr} X_{t+1} = \Omega\left(
    % \sum_{j=1}^t \gamma_j \prod_{k=j+1}^t (1 - \alpha_k \nu_{ff}) +
    \mu_{ff} \sum_{j=1}^t \alpha_j^2 \prod_{k=j+1}^t \left(1 - \alpha_k \nu_{ff} \right) \mathrm{Tr} \Sigma_{ff}\right) .
\end{equation}
{\color{red}Simplify: The inner part should give a $1/\mu_{ff}$, which is used in the above text description.}