% \section{Problem Statement and Overview}
\section{Introduction}
% \tdoan{Outline}
% \begin{itemize}
%     \item A paragraph for SA, bringing the idea of P-R averaging.
%     \item A paragraph for two-time-scale SA (TSA) and its applications in optimization and RL
%     \item A few sentences discussing the existing results of TSA
%     \item State the main contributions
% \end{itemize}
Stochastic approximation (SA), introduced by \citep{RobbinsMonro}, is an iterative sample-based approach to find the root (or fixed point)  $x^{\star}$  of some unknown operator $F$. 
% In particular, SA iteratively updates $x$, an estimate of $x^{\star}$, by moving along the direction of the samples $F(x;\xi)$ of $F(x)$ scaled by some step size, where $\xi$ is a random variable representing sampling noise. 
In particular, SA iteratively updates $x$, an estimate of $x^{\star}$, by moving along the direction of the sample $F(x;\xi)$ scaled by some step size, where $\xi$ is a random variable representing sampling noise. 
The simplicity of its implementation has enabled broad applications in many areas including stochastic optimization, machine learning, and reinforcement learning \citep{SBbook2018,LanBook2020}. 
The convergence properties of SA are well-studied, where the error rate achieved by SA with Polyak-Ruppert (PR) averaging is known to decay faster than that of SA without PR averaging \citep{polyakJuditsky}. 




In this paper, we study the finite-time error rates of the linear two-time-scale SA (TSA) algorithm, a generalized variant of SA, for solving a system of two coupled linear equations, i.e., we seek to find a pair $(x^{*},y^{*})$ that solves 
\begin{align}\label{eq:problem}
\left\{\begin{array}{ll}
\mathbb{E}_{W}[A_{ff}x^{*}+A_{fs}y^{*}-W] = 0,\\
\mathbb{E}_{V}[A_{sf}x^{*}+A_{ss}y^{*}-V] = 0,
\end{array}\right.    
\end{align}
where $W, V$ are random vectors with unknown distributions and $A_{ff}, A_{fs}, A_{sf}, A_{ss}$ are the system parameters. 
In this setting, TSA iteratively updates $(x,y)$ as follows:
%
%Linear stochastic approximation is a framework for solving a linear system given access to noisy observations.  Many algorithms in machine learning can be formulated as stochastic approximation methods, where two iterates $x_t$ and $y_t$ are updated as
\begin{equation}
    \begin{split}
        x_{t+1} &= x_t - \alpha_t \left(A_{ff} x_t + A_{fs} y_t - W_t \right) \\ 
        y_{t+1} &= y_t - \gamma_t \left(A_{sf} x_t + A_{ss} y_t - V_t \right), 
    \end{split}
    \label{eq:ttsa_general}
\end{equation}
where $\{(W_t, V_t)\}$ is a martingale difference sequence representing the sampling noise. Here, $\alpha_{t}\gg\gamma_{t}$ are two different step sizes. The variable $x$ is often referred to as the fast-time-scale variable (using larger step sizes) while $y$ is called the slow-time-scale variable (using smaller step sizes).%, explaining for the name of two-time-scale SA. These two-step sizes have to be carefully designed to guarantee the convergence of TSA.       


% The step sizes $\{\alpha_t\}$ and $\{\gamma_t\}$ are to be designed so that the solution $(x^*, y^*)$ to the system $A$ with components $A_{ff}, A_{fs}, A_{sf}, A_{ss}$ is obtained, where $\{(W_t, V_t)\}$ is a martingale difference sequence. 


TSA has received a great amount of interest due to its application in many areas where the classic SA is not applicable. Prominent examples include gradient temporal-difference learning \citep{sutton2008convergent,sutton2009fast,xu2019two}, actor-critic methods in reinforcement learning \citep{konda1999actor,bhatnagar2012stochastic,harsh19,wu2020finite,zeng2024accelerated}, and gradient descent-ascent methods in min-max optimization for zero-sum games \citep{zeng2022regularized}. These methods require one iterate being updated using a smaller step size than the other to guarantee convergence.  




Unlike the literature on SA where convergence properties are well understood, theoretical results for TSA are not complete.
% While the benefit of using PR averaging has been established for SA \citep{polyakJuditsky}, a similar result has been studied for TSA in \citep{mokkadem2006convergence}, where only an asymptotic convergence in distribution to a non-zero limit was established.  
While the benefit of using PR averaging has been established for SA \citep{polyakJuditsky}, its benefit has only been demonstrated with respect to asymptotic convergence in distribution for TSA \citep{mokkadem2006convergence}.
In this paper, we focus on understanding the finite-time error rates at which the PR average of the iterates generated by TSA converge to the desired solution. 
% We use TSA to refer to the variables $(x_t, y_t)$ generated as in Eq. \ref{eq:ttsa_general}, to distinguish from TSA with PR averaging (TSA-PR) that produces $\bar{x}_n = n^{-1} \sum_{t=1}^n x_t$ and $\bar{y}_n = n^{-1} \sum_{t=1}^n y_t$.
We use TSA to refer to the sequence $\{(x_t, y_t)\}$ in Eq. \eqref{eq:ttsa_general} without Polyak-Ruppert averaging, to distinguish from TSA-PR which is used to refer to the averages $(\bar{x}_n, \bar{y}_n)$. 
\textbf{Our contributions} are summarized below. 
\vspace{-0.3cm}
\begin{enumerate}
    \item 
    We establish in Theorem \ref{thm:clt} a finite-time bound on the Wasserstein-1 distance between the scaled deviations $\sqrt{n}(\bar{x}_n - x^*)$ and $\sqrt{n}(\bar{y}_n - y^*)$ generated by TSA-PR and their Gaussian limits. 
    This is the first non-asymptotic central limit theorem (CLT) in the context of two-time-scale algorithms, complementing the asymptotic CLT established in \citep{mokkadem2006convergence}. 
    % {\color{red}Similar to the classical SA setting, we do not require the knowledge of the problem parameters to schedule the step sizes.}
    \vspace{-0.3cm}
 
    \item Our bound on the Wasserstein-1 distance implies that the expected error achieved by TSA-PR decays at rate $n^{-1/2}$ (Corollary \ref{cor:mae}), where error is defined to be the norms $\lVert \bar{x}_n - x^* \rVert$ and $\lVert \bar{y}_n - y^*\rVert$. 
    This rate significantly improves the rate achieved by TSA without PR averaging, as explained in Section \ref{sec:PR_TTSA} and Remark \ref{rem:comparison_with_mse}.\vspace{-0.3cm}

    \item In proving these results, we also establish in Theorem \ref{thm:mse} lower and upper bounds on the expected square error (MSE) achieved by TSA (i.e., without PR averaging). 
    When compared to similar bounds in prior works, our result provides a more convenient representation to establish our main results above while using a much simpler proof technique.      
    % This result may be of independent interest since there are a lot of prior work on TSA, but to the best of our knowledge, prior results specialized to our model provide weaker results.
    \vspace{-0.3cm}
    
    % The result captures both the magnitude of their second moments, as well as the rate at which the second moments converge to their asymptotic covariances. 
\end{enumerate}






\section{Related Work}

\subsection{View-Dependent Control}
View-dependent representations have a long history in computer graphics.
In his pioneering work, Rademacher proposed interpolating between \textit{key viewpoints} and associated \textit{key deformations} to manipulate 3D models~\cite{rademacher1999view}.
Other researchers have extended the idea to create 3D animation systems~\cite{10.1111:j.1467-8659.2004.00772.x}, streamline the modeling process~\cite{DBLP:journals/corr/abs-2103-15472}, and integrate physical simulation\cite{koyama2013view}.
Of particular note, Rivers et al.~\cite{rivers25Dcartoonmodels} introduced \textit{2.5D Cartoon Models}, a combination of planar meshes transformed, based upon view angle, so as to appears three dimensional.
Our work draws upon these works but is, to our knowledge, the first work to attempt to use view-dependent techniques to retarget 3D motion onto 2D characters.   

\subsection{Animation from 2D Images}

% output is still 2D
Many researchers have proposed different methods for creating animations from 2D images. Hornung et al.~\cite{Hornung2007anim2Dpicmotion} presented a method to deform a character from a photograph given user-provided joint annotations.
\textit{Toonsynth}~\cite{Dvoroznak18-SIG} and \textit{Neural Puppet}~\cite{poursaeed2020neural} both present methods to create new images of hand-drawn characters from examples.
% output is 3D model
Other researchers have proposed methods of obtaining 3D geometry from 2D sketches~\cite{igarashi2006teddy, Dvoroznak20-SA} and images~\cite{ArtiSketch,weng2019photo}.
% focus on sketches specifically
A number of works have specifically focused on childlike drawings.
Lingens et al.~\cite{lingens2020towards} proposed an evolutionary algorithm for animating children's drawings. 
\textit{MagicToon}~\cite{feng2017magictoon} creates a 3D model from childlike drawings for AR applications.
Similar to our work, Smith et al.~\cite{SmithHodgins} proposed a method for animating childlike drawings using 3D skeletal motion. 
However, the resulting animations are only suitable for use in 2D applications and the type of motions it supports are limited.

Unlike these previous works, our solution can be used in 3D contexts but does not create a 3D model. We instead relying upon a view-dependent formulation of the animated character.

\section{Model and Preliminaries}\label{sec:preliminaries}
In this paper, we consider two time-scale stochastic approximation algorithms of the form
\begin{equation}
    \begin{split}
        x_{t+1} &= x_t - \alpha_t \left(A_{ff} x_t + A_{fs} y_t - W_t \right),
        \\
        y_{t+1} &= y_t - \gamma_t \left(A_{sf} x_t + A_{ss} y_t - V_t \right).
    \end{split}\label{eq:ttsa}
\end{equation}
While we do not consider the so-called ODE approach here to analyze the system, the assumptions on the system matrices in Eq. \eqref{eq:ttsa} are easy to explain by relating the above to a singularly perturbed differential equation (ODE); see \citep{borkar2008stochastic}:
\begin{equation}
    \dot{x}_t = - (A_{ff} x_t + A_{fs} y_t) , 
    \quad
    \dot{y}_t = - \frac{\gamma}{\alpha} \left(A_{sf} x_t + A_{ss} y_t \right).
\end{equation}
In the limit $\gamma/\alpha \to 0$, $x_t$ evolves much faster than $y_t$.
When the system governing $\dot{x}_t$ is stable, the slow-time-scale $y_t$ is analyzed assuming a stationary solution $x_\infty (y) = -A_{ff}^{-1} A_{fs} y$:
\begin{equation}
    \dot{y}_t = -\left(A_{sf} x_\infty (y_t) + A_{ss} y_t\right) 
    = -(A_{ss} - A_{sf} A_{ff}^{-1} A_{fs}) y_t
    .
\end{equation}
To ensure that both $x_t$ and $y_t$ converge to their respective limits, it is therefore assumed that $-A_{ff}$ and $-\Delta = -(A_{ss} - A_{sf} A_{ff}^{-1} A_{fs})$ are both H\"{u}rwitz stable, i.e., the eigenvalues lie in the left-half of the complex plane. 
The rates at which the discretized system in Eq. \eqref{eq:ttsa} approach their limits are studied under the same setting, which we now state formally. 
% Motivated by this ODE analysis, the same assumption is used for the convergence analysis of the discretized system in Eq. \eqref{eq:ttsa}.
% Next, we state a few standards assumptions used in the TTSA literature \cite{}.
% ***Clearly make the connection to TTSA*** Consider the root $x_\infty (y)$ that solves the fast system $F(x_\infty (y), y) = 0$ for every $y$.
% A closed form solution is given by $x_\infty (y) = - A_{ff}^{-1} A_{fs} y \eqqcolon H y$.
% Our first assumption is that the fast and slow systems are both asymptotically stable, i.e., $x_n \to x_\infty (y)$ whenever $y$ is fixed and $y_n \to y^*$ when $x_n$ is taken to be $x_\infty (y_n)$.
% When the solution $x_\infty (y)$ to the fast system $F(x, y) = 0$ can be solved for any $y$, $y^*$ is a global attractor for the slow system $S$ iff
% \begin{align*}
%     \hat{y} \coloneqq y - y^* = S(x_\infty (y), y) - S(x_{\infty}(y^*), y^*) =
%     (A_{ss} -A_{sf} A_{ff}^{-1} A_{fs}) \hat{y} 
% \end{align*}
% is stable for every $y$.
\begin{assumption}[System Parameters]\label{assumption:structure}\label{assumption:first}
    The matrix $A_{ff}$ and its Schur complement $\Delta = A_{ss} - A_{sf} A_{ff}^{-1} A_{fs}$ are real and satisfy 
    \begin{equation}
        A_{ff} + A_{ff}^T \succ 0, \quad \Delta + \Delta^T \succ 0 .
    \end{equation}
\end{assumption}
% In other words, $-A_{ff}, -\Delta$ are both H\"{u}rwitz stable. 
When a matrix $-A$ is H\"{u}rwitz stable, we use $\mu_A, \nu_A$ to denote the smallest and largest eigenvalues of $A + A^T$, respectively.







\begin{assumption}[Noise]\label{assumption:noise}
    Let $\{N_t\} \coloneqq \{(W_t, V_t)\}_{t=1}^\infty$ be a martingale difference sequence drawn independently of the iterates $x_t$ and $y_t$.
    We assume that for every $t \geq 1$, $\mathbb{E}\lVert N_t \rVert^{2 + \beta} < \infty$ for some $\beta \in (1/2, 1)$ and that
    \begin{align*}
        \mathbb{E} [W_t W_t^T | \historyprev] = \Gamma_{ff}, \mathbb{E} [V_t V_t^T | \historyprev] = \Gamma_{ss}, \mathbb{E} [W_t V_t^T | \historyprev] = \Gamma_{fs} ,
    \end{align*} 
    where  $\history = \{x_1, y_1, W_1, V_1, \cdots, W_t, V_t\}$ and $\Gamma$ is covariance matrix of $(W_t, V_t)$ conditioned on the history.
\end{assumption}
Our last assumption is a guidance on how to choose the step sizes.
\begin{assumption}[Step Size]\label{assumption:steps}\label{assumption:last}
    The step sizes are chosen to be $\alpha_t = \alpha_1 t^{-a}, \gamma_t = \gamma_1 t^{-b}$
    for $1/2 < a < b < 1$ and any $\alpha_1, \gamma_1 > 0$.
\end{assumption}
\begin{remark}
    Both $x_n$ and $y_n$ converge even when $b = 1$. 
    This parameter is crucial when deducing the $n^{-1}$ rate for the second moment of the slow-time-scale iterate. 
    As we will show that the Polyak-Ruppert averaging scheme achieves this fast rate for both fast- and slow-time-scale variables without having to set $b = 1$, we assume $b < 1$ which introduces minor technical conditions on the requirement for $\gamma_1$.
\end{remark}
Here we allow arbitrary $\alpha_1, \gamma_1 > 0$, but clarify a few technical conditions. 
Define $(\mu_{ff}, \nu_{ff})$ and $(\mu_\Delta, \nu_\Delta)$ such that $\mu_{ff} I \preceq A_{ff} + A_{ff}^T \preceq \nu_{ff} I$ and $\mu_\Delta I \preceq \Delta + \Delta^T \preceq \nu_\Delta I$.  
Because $\alpha_t, \gamma_t \to 0$ and $\gamma_t/\alpha_t \to 0$, there exists a time $t_0$ and problem-dependent constants $K_1, K_2, K_3$ such that the following holds for all $t \geq t_0$: 
\begin{align*}
    % \alpha_t &\leq \frac{2}{\mu_{ff}} , 
    \alpha_t &< K_1 \frac{\mu_{ff}}{\nu_{ff}^2} , 
    \quad 
    \gamma_t < K_2 \frac{\mu_{\Delta}}{\nu_\Delta^2} ,
    % \\ 
    \quad
    \frac{\gamma_t}{\alpha_t} = K_3 \min\left\{
        \frac{\mu_{ff}}{\nu_{ff}^2}, \frac{\mu_{\Delta}}{\nu_{\Delta}^2}
    \right\} .
    % \leq \min\left\{
    %     \frac{\mu_{ff}}{2 M_f},
    %     % \frac{M_g}{4 \mu_{ff}},
    %     \frac{\nu_{ff}}{4 M_g} ,
    %     \frac{2\mu_\Delta}{M_h}
    % \right\},  
    \numberthis \label{eq:initial_steps}
\end{align*}
% These conditions must be satisfied to ensure that the error rates do not increase.
% In many applications, it is hard to verify the above conditions because the system parameters are unknown. 
Since this condition will be satisfied at some finite-time $t_0$, the analysis in this paper holds for all $t \geq t_0.$ We set $t_0 = 1$ for simplicity of exposition.
% {\color{red}Include above: $\alpha_t, \gamma_t$ must be defined so that $\{L_t\}$ is well-defined, e.g. see \citep{kaledin2020finite}.
%     Note that their condition that $L_t \leq L_\infty$ (before Eq. 17) is fine but not strong enough; we can work from the recursion and prove when it is finite. 
% }




