\section{Preliminaries: Relation between diffusions and PDEs}
In this section we introduce the connections between diffusions generated by Stochastic Differential Equations (SDEs) and Partial Differential Equations (PDEs). We first recall the well known relation between 
diffusions and Fokker-Planck equations, and then we briefly review the connection between FBSDEs and non-linear parabolic PDEs. 

Let $\{W_t\}_{t\in[0,T]}$ be a standard Brownian motion. Consider a stochastic process $\{X_t\}_{t\in[0,T]}$ generated by the SDE:
\begin{equation*}
    dX_t = \mu(t,X_t)dt+\sigma(t)dW_t,\quad X_0\sim\nu,
\end{equation*}
where $\mu:[0,T]\times\R^d\rightarrow\R^d$ is the drift coefficient and $\sigma:[0,T]\rightarrow\R$ is the diffusion coefficient. The diffusion coefficient $\sigma$ can in general take a matrix form and may vary as a function of space. However, for the sake of simplicity in our presentation, we limit our discussion to the scenario where $\sigma$ is a scalar-valued function solely dependent on $t$. Let $\rho:[0,T]\times\R^d\rightarrow\R_+$ denote the probability density of $\{X_t\}_{t\in[0,T]}$, i.e., $X_t\sim\rho(t,\cdot)$. Then, $\rho$ satisfies a Fokker-Planck equation given by
\begin{equation*}
    \partial_t\rho-\frac{\sigma^2}{2}\Delta\rho+\nabla\cdot(\mu\rho)=0,\quad \rho(0,x) = \nu(x).
\end{equation*}
This constitutes a forward Kolmogorov PDE, which is well-defined as an initial value problem. Next, consider the following backward Kolmogorov PDE also known as \textit{quasi-linear parabolic partial differential equation}:
\begin{equation}\label{eqn:semiParaPDE}
    \partial_t u+ \frac{\sigma^2}{2}\Delta u + \mu^T\nabla u + f(t,x,u,\sigma^T\nabla u) = 0,%\quad u(T,x) = \varphi(x),
\end{equation}
with terminal condition $u(T,x) = \varphi(x)$, where $u:[0,T]\times \R^d\rightarrow\R$ is the solution of the PDE, $\mu:[0,T]\times\R^d\times\R\times\R^d\rightarrow\R^d$ and $\sigma:[0,T]\rightarrow\R$ are coefficient functions, $f:[0,T]\times\R^d\times\R\times\R^d\rightarrow\R$ is a non-linearity function, and $\varphi:\R^d\rightarrow\R$ a given terminal condition. The solution to PDE (\ref{eqn:semiParaPDE}) is related to diffusion processes via the so-called forward-backward stochastic differential equations (FBSDE). Consider the following set of stochastic differential equations:
\begin{align}\label{eqn:pardouxSDE}
\begin{split}
    dX_t &= \mu (t,X_t,Y_t,Z_t)dt + \sigma (t)dW_t,\\
    dY_t &= -f(t,X_t,Y_t,Z_t)dt + Z_t^TdW_t, \,\, Y_T = \varphi(X_T)
\end{split}
\end{align}
where $(X_t,Y_t,Z_t)$ are stochastic processes adapted to the natural filtration of $W_t$. Following pioneering work of Bismuth \cite{bismut_conjugate_1973}, it was shown by Pardoux and Peng \cite{pardoux_adapted_1990,pardoux_backward_1998,pardoux_forward-backward_1999} that under certain regularity conditions on $\mu,\sigma,$ and $f$, quite remarkably, there exists a unique solution $(X_t,Y_t,Z_t)$ to the above set of SDEs. It can be shown that the solution must satisfy $Y_t = u(t,X_t)$ and $Z_t = \sigma(t)\nabla u(t,X_t)$, where $u$ is the solution to PDE (\ref{eqn:semiParaPDE}). Thus we can solve the PDE (\ref{eqn:semiParaPDE}) by solving FBSDE (\ref{eqn:pardouxSDE}).

