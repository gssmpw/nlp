This section details theoretical underpinnings and intuition for our design in \autoref{sec:design}.
Recent literature has studied carbon-aware scheduling problems with a theoretical lens~\cite{Lechowicz:23,Lechowicz:24,Lechowicz:24CFL,Bostandoost:24}, spanning relatively simple suspend-resume~\cite{Lechowicz:23} to settings considering scaling and uncertainty in job lengths~\cite{Bostandoost:24}.
In these online carbon-aware scheduling problems, the key challenge is the inherent uncertainty in future carbon intensity values due to the proliferation of intermittent renewable energy sources. 

A common approach to manage this uncertainty is \textit{threshold-based design}~\cite{Lechowicz:23, Bostandoost:24:HotCarbon, Wiesner:2021:WaitAwhile}, %
that uses a predetermined and parameterized threshold function to inform decisions. 
Among studies that use this design paradigm, a common theoretical performance metric is \textit{competitive ratio}, %
which is the worst-case ratio ($\geq 1$) between the cost of an online algorithm vs. that of an optimal solution.  Algorithms designed using this metric are known to be pessimistic in practice~\cite{Lykouris:18, Purohit:18}.  Moreover, existing theoretical studies on carbon-aware scheduling focus on simple settings where, e.g., the job is bound by a deadline, the objective is only to reduce carbon, and precedence constraints are not considered.
However, threshold-based algorithms have been demonstrated to work well in practice: they are often close to optimal provided their inputs are reasonably accurate~\cite{Daneshvaramoli:24}.



Carbon-aware DAG scheduling exhibits an inherent trade-off between carbon savings and job completion time (JCT).  
Although worst-case metrics %
(i.e., bounds with respect to an intractable offline solution) have limited utility in this setting, it is still useful to quantify a trade-off between carbon and JCT -- to this end, we introduce two metrics that we use in the following sections.
We start by introducing some notation:  let $\OPT_K(\mathcal{J})$ denote the optimal makespan for job $\mathcal{J}$, and let $\ALG_K(\mathcal{J})$ denote the makespan for the schedule generated by some scheduler $\ALG$ (all using a maximum of $K$ machines).

\vspace{-0.5em}
\begin{dfn}[Carbon Stretch Factor (CSF)] \label{dfn:csf}
Given a scheduling policy (e.g., FIFO), let $\texttt{AG}$ denote the regular (i.e., carbon-agnostic) scheduling policy, and let $\texttt{CA}$ denote a carbon-aware variant of the same scheduling policy.
If $a$ is an upper bound such that $\texttt{AG}_K(\mathcal{J}) \leq a \cdot \OPT_K(\mathcal{J}) : \forall \mathcal{J}$, and $b$ is an upper bound such that $\texttt{CA}_K(\mathcal{J}) \leq b \cdot \OPT_K(\mathcal{J}) : \forall \mathcal{J}$, where $b \geq a$, then the \textbf{carbon stretch factor} is defined as $\nicefrac{b}{a}$, which indicates (multiplicatively) how much worse the makespan of $\texttt{CA}$ is compared to $\texttt{AG}$. Note that $\nicefrac{b}{a} \geq 1$. 
\end{dfn}
\vspace{-0.5em}



\noindent To quantify carbon savings, we define $C_\ALG(t)$ as the instantaneous carbon emissions at time $t$ due to decisions by scheduler $\ALG$.  It is a function of the number of executors active in $\ALG$'s schedule at time $t$ (denoted by $E_\ALG(t)$) and the current carbon intensity: $C_\ALG(t) \coloneqq c(t) E_\ALG (t)$.

\begin{dfn}[Carbon Savings] \label{dfn:carbonsavings}
Let $\texttt{AG}$ and $\texttt{CA}$ denote a carbon-agnostic and carbon-aware scheduler as outlined in \sref{Def}{dfn:csf}. %
For a job $\mathcal{J}$, if $\texttt{AG}$ runs from time step $0$ until $T$ (its completion time), and $\texttt{CA}$ operates from time $0$ to $T'$, %
then $\texttt{CA}$'s \textbf{carbon savings} are given by $\int_0^T C_{\texttt{AG}}(t) - \int_0^{T'} C_{\texttt{CA}}(t)$.
\end{dfn}

\noindent Using CSF and carbon savings, we describe the desired behavior of a carbon-aware scheduler for data processing.
A basic intuition in threshold-based designs is ``hedging'' between completing tasks now and waiting for lower-carbon periods that may arrive. 
To do this, thresholds rely on the \textit{range} of carbon intensities that are expected to appear in the near future (i.e., $L$ and $U$).  In the context of CSF, this translates into two conditions that a scheduler should satisfy:

\textbf{i) } If the fluctuation of carbon intensity is \textit{low} (e.g., $L$ and $U$ are close), the CSF should be close to $1$, i.e., JCT should be close to that of the carbon-agnostic algorithm.  

\textbf{ii) } If the fluctuation is \textit{high} (e.g., $L$ and $U$ are not close), the CSF should be \textit{finite}, i.e., the scheduler does not wait indefinitely to complete the job.  In threshold-based designs, this is often met by imposing a \textit{deadline} on the job~\cite{Goiri:2012:GreenHadoop, Lechowicz:23}.

In the context of the DAG scheduling for data processing workloads, additional unique challenges exist. For instance, in the single job settings considered by prior work, specifying a deadline for each job is straightforward~\cite{Lechowicz:23, Bostandoost:24}.  However, on a cluster scale that considers multiple jobs of unknown length and different arrival times, setting a proper deadline quickly becomes complicated. Instead, our schedulers (see \autoref{sec:design}) guarantee a minimum amount of job progress whenever there are outstanding tasks in the queue.  

Due to precedence constraints, carbon-aware scheduling actions that do not consider the structure of the DAG may inadvertently block bottleneck tasks from processing, having a large negative impact on JCT.  This gives a third condition:

\textbf{iii) } When fluctuation is \textit{high} (i.e., $L$ and $U$ are not close) and the system is in a high-carbon period,
a scheduler should carefully consider the \textit{structure} of a job's DAG, prioritizing \textit{bottleneck} tasks to use the limited cluster resources.

\noindent Conditions \textbf{i - iii)} summarize the desired high-level behavior of a carbon-aware scheduler for data processing workloads.  In the following section, we present \PCAPS that takes into account the above conditions in its design.




















