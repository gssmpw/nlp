

As large-scale data processing workloads continue to grow, their carbon footprint raises concerns. Prior research on carbon-aware schedulers has focused on shifting computation to align with availability of low-carbon energy, but these approaches assume that each task can be executed independently. 
In contrast, data processing jobs have precedence constraints (i.e., outputs of one task are inputs for another) that complicate decisions, since delaying an upstream ``bottleneck'' task to a low-carbon period will also block downstream tasks, impacting the entire job's completion time. 
In this paper, we show that carbon-aware scheduling for data processing benefits from knowledge of both time-varying carbon and precedence constraints. 
Our main contribution is $\texttt{PCAPS}$, a carbon-aware scheduler that interfaces with modern ML scheduling policies to explicitly consider the precedence-driven importance of  each task in addition to carbon.  
To illustrate the gains due to fine-grained task information, we also study $\texttt{CAP}$, a wrapper for any carbon-agnostic scheduler that adapts the key provisioning ideas of $\texttt{PCAPS}$.
Our schedulers enable a configurable priority between carbon reduction and job completion time, and we give analytical results characterizing the trade-off between the two.
Furthermore, our Spark prototype on a 100-node Kubernetes cluster shows that a moderate configuration of $\texttt{PCAPS}$ reduces carbon footprint by up to 32.9\% without significantly impacting the cluster's total efficiency.







