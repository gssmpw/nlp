In this section, we give full proofs for analytical results and detailed discussion for the \DANISH and \CAP scheduler designs introduced in \autoref{sec:design}.

\subsection{Deferred Proofs and Discussion from \autoref{sec:danish-design} (Precedence- and carbon-aware provisioning and scheduling)} \label{apx:danish-proofs}

In this section, we discuss and prove the analytical results for the \DANISH scheduler, introduced in \autoref{sec:danish-design}.  Throughout this section, we let $\texttt{PB}$ denote a \textit{carbon-agnostic, probabilistic} baseline scheduler, as outlined in \sref{Def.}{dfn:pb}.

For the sake of analysis, in the following results we leverage the classic makespan bound of Graham's list scheduling algorithm~\cite{Graham:66}, which is known to produce a schedule that is a $(2 - \nicefrac{1}{K})$-approximation for the optimal makespan on $K$ identical machines.  
Note that any carbon-agnostic probabilistic scheduler is an instance of list scheduling where the list (of tasks) is random, and the next task is assigned to a machine as soon as it becomes idle.


Recall the definition of \DANISH's carbon-awareness filter, parameterized by the $\Psi_\gamma$ function (see \autoref{alg:danish}). $\Psi_\gamma$ exhibits an exponential dependence on $r$, the relative importance of a task.  This is inspired by literature on one-way trading and related online problems~\cite{ElYaniv:01, Zhou:08}, where such an exponential trade-off is derived by balancing the marginal reward of the current price against the risk that better prices may arrive in the future.  
For DAG scheduling, we interpret relative importance in an analogous way: tasks with high importance have a substantial negative impact if they are not scheduled, so they are almost always scheduled at the current ``price'' (i.e., carbon intensity).  On the other hand, tasks with low relative importance (e.g., short tasks) may not significantly affect the DAG's completion time if they are deferred to start later, so they can ``wait'' for better carbon intensities.

To analyze the carbon stretch factor of \DANISH, we define $\mathcal{D}(\gamma, \mbf{c}) \in [0,1]$, a quantity that describes the fraction of tasks (in terms of total runtime) that are deferred by \DANISH when scheduling with a given value of $\gamma$ and given carbon trace $\mbf{c}$.  It is $\leq 1$ for any value of $\gamma$, and $\mathcal{D}(0, \mbf{c}) = 0$ for any $\mbf{c}$.  As $\gamma$ grows and \DANISH becomes ``more carbon-aware'',  $\mathcal{D}(\gamma, \mbf{c})$ grows in expectation.%

\subsubsection{\textbf{Proof of \autoref{thm:danishMakespan}}}\label{apx:danishMakespan}

We are ready to prove \autoref{thm:danishMakespan}, which states that for a given carbon intensity trace $\mbf{c}$, \DANISH's carbon stretch factor is $1 + \frac{\mathcal{D}(\gamma, \mbf{c}) K}{2 - \frac{1}{K}}$.


\begin{proof}
Let $\texttt{PB}_K(\mathcal{J})$ denote the schedule produced by the carbon-agnostic probabilistic baseline scheduler (e.g., Decima) using $K$ machines, and let $\DANISH_K(\mathcal{J})$ denote the schedule produced by $\DANISH$ for the same job $\mathcal{J}$ with $n$ tasks and a maximum of $K$ machines. We denote the carbon intensity trace by $\mbf{c} \coloneqq \{c(t) : t \ge 0\}$.

We henceforth use $\texttt{PB}_\text{K}^\texttt{D}$ to denote the instance of $\texttt{PB}$ that $\DANISH$ interfaces with.

Consider the perspective of a single node $v$: suppose that $\texttt{PB}$ samples node $v$ to be scheduled at time $t \geq 0$, where node $v$ has probability $p_{v,t} > 0$ and $v \in \mathcal{A}_t$.  By definition of $\texttt{PB}$, as soon as a task (node) is sampled, it runs on the available machine in $\texttt{PB}$'s schedule.

Now we consider the same node sampled by $\texttt{PB}_\text{K}^\texttt{D}$.  We denote $D_{v}$ as a random variable that denotes the number of times that node $v$ is \textit{not} scheduled (i.e., deferred) when it is sampled by $\texttt{PB}_\text{K}^\texttt{D}$. It is defined as:
\[
D_{v} = \sum_{t \in \mathcal{T}_v} \mathbb{I} \{ c(t) > \Psi_\gamma(r_{v,t}) \},
\]
where $\mathcal{T}_v$ denotes the times at which node $v$ is sampled by $\texttt{PB}_\text{K}^\texttt{D}$.
In the worst-case, observe that whenever a task $v$ is sampled but not scheduled, a deferral increases the total makespan by at most $X_v$, where $X_v$ is the runtime of task $v$.  Consider the edge case where all other tasks in $\mathcal{A}_t$ (i.e., all other tasks that are ready to run at the same time as task $v$) are scheduled on other machines at time $t$, and all further tasks (i.e., tasks that have not yet been completed but also were not in $\mathcal{A}_t$) are successors of $v$ (i.e., they cannot run until $v$ is completed).

In this case, assuming that the other tasks in $\mathcal{A}_t$ run to completion, there will be some time step $t' > t$ such that $\mathcal{A}_{t'} = \{ v \}$ -- i.e., the only task available to run is task $v$.  As soon as this is the case, $v$ will run -- this is because the \textit{relative importance} of any task in a set of size $1$ is always $1$.  Thus, in the worst-case, the schedule length increases by exactly $X_v$ -- this is the case if all of the other tasks in $\mathcal{A}_t$ finish at the same time $t'$ (i.e., no overlap with $v$).

This gives the following makespan bound in terms of $D_{v}$:
\[
\E [ \DANISH_K(\mathcal{J}) ] \le \E [ \texttt{PB}_K(\mathcal{J}) ] + \E \left[ \sum_{v \in \mathcal{J}} D_v X_v \right]
\]
By linearity of expectation, we have:
\[
\E [ \DANISH_K(\mathcal{J}) ] \le \E [ \texttt{PB}_K(\mathcal{J}) ] + \sum_{v \in \mathcal{J}} \E \left[ D_v \right] X_v
\]

Consider the \textit{total} number of deferrals $D = \sum_{v \in \mathcal{J}} D_v$, and note that $D$ must be $\leq n-1$ -- since at least one machine is active at all times, the maximum number of deferrals is that which causes the schedule to drop to a single machine across the entire time interval.  This immediately implies that $\E[D] \leq n-1$.
Define a sorted list $\{ X'_i : i \in n\}$ such that $X'_0 = \max_{v \in \mathcal{A}} X_v, \dots, X'_n = \min_{v \in \mathcal{A}} X_v$, and note that we can upper bound the above as follows:
\begin{align*}
\E [ \texttt{PB}_K(\mathcal{J}) ] + \sum_{v \in \mathcal{J}} \E \left[ D_v \right] X_v &\le \E [ \texttt{PB}_K(\mathcal{J}) ] + \sum_{i=0}^{\E[D]} X'_i,
\end{align*}
Note that this is a worst-case assumption -- in reality, the tasks with low relative importance (that are likely to be deferred) are unlikely to be the longest tasks for any reasonable scheduler \texttt{PB}.
Note that $\sum_{i=0}^n X'_i = \sum_{v \in \mathcal{J}} = \OPT_1(\mathcal{J})$, i.e., the optimal makespan using a single machine.  To simplify the above bound, we can define a function $\mathcal{D}(\gamma, \mbf{c})$ as follows:
\[
\mathcal{D}(\gamma, \mbf{c}) = \frac{\sum_{i=0}^{\E[D]} X'_i}{\OPT_1(\mathcal{J})}.
\]
Note that $\mathcal{D}(\gamma, \mbf{c}) \le 1$ for any $\gamma$ and any $\mbf{c}$, and $\mathcal{D}(0, \mbf{c}) = 0$ for any $\mbf{c}$.  In practice, $\E [D]$ can be estimated from e.g., historical carbon traces and the characteristic behavior of $\texttt{PB}$ (i.e., in terms of relative importance).
We have the following bound:
\begin{align*}
\E [ \DANISH_K(\mathcal{J}) ] &\le \E [ \texttt{PB}_K(\mathcal{J}) ] +  \mathcal{D}(\gamma, \mbf{c}) \OPT_1(\mathcal{J}).
\end{align*}
This gives insight into the tuning of hyperparameter $\gamma$ -- for a low-carbon period $\mbf{c}'$ where a practitioner desires full throughput, one should tune $\gamma$ such that $\mathcal{D}(\gamma, \mbf{c'}) \approx 0$.

Since \texttt{PB} follows the list scheduling paradigm of scheduling tasks in an order that respects precedence constraints, it inherits the following worst-case theoretical bound on  its makespan:
\begin{align*}
\E [ \texttt{PB}_K(\mathcal{J}) ] \le \left(2 - \frac{1}{K} \right) \OPT_K(\mathcal{J}).
\end{align*}

From the proof of \autoref{thm:ksMakespan}, we also have the following bound for $\OPT_1(\mathcal{J})$:
\begin{align*}
\OPT_1(\mathcal{J}) \le K\cdot \OPT_K(\mathcal{J}).
\end{align*}

Combining these results, we have the following:
\begin{align*}
\E [ \DANISH_K(\mathcal{J}) ] &\le \left(2 - \frac{1}{K} + \mathcal{D}(\gamma, \mbf{c}) K \right) \OPT_K(\mathcal{J}).
\end{align*}

Combined with the bound for $\texttt{PB}$, this shows that $\DANISH$ has a carbon stretch factor of $1 + \frac{\mathcal{D}(\gamma, \mbf{c}) K}{2 - \frac{1}{K}}$.

\end{proof}

We now turn to the question of carbon savings, and the result stated in \autoref{thm:danishCarbonSavings}. 
First, for ease of analysis, we define a \textit{discretized time model} that is motivated by the empirical fact that carbon intensities are reported in discrete time intervals~\cite{electricity-map, watttime}.
Assuming that new carbon intensity values arrive at integers in continuous time, we define a discretized carbon signal for any discrete time step $t$ as $c_{t} \coloneqq c(t') : t' \in [t, t+1)$, where note that $c(t')$ does not change on the interval $t \in [t, t+1)$.

Slightly abusing notation, we let $C_{\DANISH}(t)$ denote the carbon emissions of $\DANISH$'s schedule during discrete time step $t$, and let $C_{\texttt{PB}}(t)$ denote the carbon emissions of $\texttt{PB}$ at discrete time step $t$, respectively.  
The schedules generated by $\DANISH$ and $\texttt{PB}$ each use a certain number of machines during any discrete time interval -- to capture this, we let $E^\DANISH_t : t' \in [t, t+1)$ denote the number of active machines during discrete time step $t$ in \DANISH's schedule, and $E^\texttt{PB}_t \leq K$ denotes the same for \texttt{PB}'s schedule.  In what follows, we use $W$ as shorthand to denote the \textit{excess work} that \DANISH must ``make up'' with respect to \texttt{PB}'s schedule (i.e., due to deferral actions). Formally, $W = \sum_{i=0}^T \max\{ E^\texttt{PB}_t - E^\DANISH_t , \ 0 \}$.

\subsubsection{\textbf{Proof of \autoref{thm:danishCarbonSavings}}}\label{apx:danishCarbonSavings}

In what follows, we prove \autoref{thm:danishCarbonSavings}, which states that for a given carbon intensity trace $\mbf{c}$, \DANISH yields carbon savings of $W \left( \overline{s}_{-}^{(0,T)} - \overline{s}_{+}^{(0,T)} - \overline{c}^{(T, T')} \right)$ compared to a carbon-agnostic baseline \texttt{PB}. %

\begin{proof}
We let $C_s(t)$ denote the \textit{carbon savings} of \DANISH at discrete time step $t$.  Formally, we have:
\[
C_s(t) = \begin{cases}
    C_{\texttt{PB}}(t) - C_{\DANISH}(t) &  1 \leq t \leq T,\\
    - C_{\DANISH}(t) & T < t \leq T'.
\end{cases}
\]
By definition, we have the following by substituting for the carbon emissions of $\texttt{PB}$ and \DANISH:
\[
C_s(t) = \begin{cases}
    E^\texttt{PB}_t c_t  - E^\DANISH_t c_t &  1 \leq t \leq T,\\
    - E^\DANISH_t c_t& T < t \leq T'.
\end{cases}
\]
Summing over all time steps, we have that the carbon savings is given by:
\[
\sum_{i=0}^{T'} C_s(i) = \sum_{i=0}^{T} (E^\texttt{PB}_i-E^\DANISH_i) c_i - \sum_{i=T+1}^{T'} E^\DANISH_i c_i
\]

Note that because of \DANISH's job-specific decisions, it is not necessarily the case that $E^\texttt{PB}_t \geq E^\DANISH_t$ for any $t \leq T$ -- namely, if \texttt{PB}'s schedule is constrained by a bottleneck task during a low-carbon time step, \DANISH's schedule may use that low-carbon time step to schedule deferred tasks from previous time steps.

Thus, to begin simplifying this expression, we consider two cases for the sum from $0$ to $T$ as follows:
\begin{align*}
\sum_{i=0}^{T'} C_s(i) &= \sum_{i=0}^{T} (E^\texttt{PB}_i-E^\DANISH_i) c_i \mathbb{I}(E^\texttt{PB}_i \geq E^\DANISH_i),\\
& - \sum_{i=0}^{T} (E^\DANISH_i - E^\texttt{PB}_i) c_i \mathbb{I}(E^\texttt{PB}_i < E^\DANISH_i) - \sum_{i=T+1}^{T'} E^\DANISH_i c_i.
\end{align*}

We define three terms that capture the \textit{weighted average} carbon intensity per unit of work that is deferred, opportunistically completed, or completed later as follows.  
Note that $\sum_{i=0}^{T} (E^\texttt{PB}_i-E^\DANISH_i) = \sum_{i=T+1}^{T'} E^\DANISH_i$.

We let $\overline{s}_{-}^{(0,T)}$ denote the weighted average carbon intensity of the machine time (work) that is \textit{deferred} in \DANISH's schedule (i.e., carbon saved due to \DANISH's carbon-aware filtering):
\[
\overline{s}^{(0,T)} = \frac{\sum_{i=0}^{T} (E^\texttt{PB}_i-E^\DANISH_i) c_i}{W} \mathbb{I}(E^\texttt{PB}_i \geq E^\DANISH_i)
\]

Furthermore, we let $\overline{s}_{+}^{(0,T)}$ denote the weighted average carbon intensity of the machine time (work) that is \textit{opportunistically completed} in \DANISH's schedule (i.e., when \DANISH does more work than \texttt{PB}, likely during a low-carbon period):
\[
\overline{s}_{+}^{(0,T)} = \frac{\sum_{i=0}^{T} (E^\texttt{PB}_i-E^\DANISH_i) c_i}{W} \mathbb{I}(E^\texttt{PB}_i < E^\DANISH_i)
\]

Finally, we let $\overline{c}^{(T, T')}$ denote the weighted average carbon intensity of the work completed by \DANISH after time $T$:
\[
\overline{c}^{(T,T')} = \frac{\sum_{i=T+1}^{T'} E^\DANISH_i c_i}{W}
\]
Under the above definitions, we can pose the total carbon savings as:
\[
\sum_{i=0}^{T'} C_s(i) = W \left( \overline{s}_{-}^{(0,T)} -\overline{s}_{+}^{(0,T)} - \overline{c}^{(T,T')} \right)
\]


\end{proof}

\noindent We note that an adversary could construct instances such that \PCAPS  uses more carbon than a carbon-agnostic baseline -- for instance, consider the case where the carbon intensity at each time step is strictly increasing over time.  In such a scenario, the ``carbon-optimal'' solution simply finishes the job as soon as it can, and such a scenario implies that $\overline{c}^{(T,T')} + \overline{s}_{+}^{(0,T)} > \overline{s}_{-}^{(0,T)}$, meaning that \PCAPS's carbon savings are negative.  However, we note such scenarios for the carbon intensity on the grid are unrealistic.  In reality, grid carbon intensity exhibits \textit{diurnal (i.e., daily) patterns} that mediate this effect -- see \autoref{fig:CI-traces} for an example. 


The above result contextualizes the total carbon savings achieved by \DANISH for a single job, but we also consider the average carbon savings at each (discrete) carbon intensity interval as follows.

Let $\rho_{\DANISH}(c)$ denote the average machine utilization for \DANISH's schedule conditioned on the fact that the current carbon intensity is $c = c_t$.  Denoting the set of discrete time steps with carbon intensity $c$ by $\mathcal{T}_c$, we have the following: $\rho_{\DANISH}(c) = \lim_{T\to \infty} \frac{\sum_{i\in\mathcal{T}_c} \nicefrac{E^\DANISH_i}{K}}{\vert \mathcal{T}_c \vert}$.  
Note that $\rho_{\DANISH}(c)$ can be estimated based on e.g., the observed relative importances of tasks produced by \texttt{PB} -- this \textit{distribution} of relative importances maps (via $\Psi_\gamma$) into a distribution of carbon intensity values -- the fraction of these values that lie below $c$ is proportional to $\rho_{\DANISH}(c)$, since the fraction of values above $c$ correspond to tasks that would be deferred by $\DANISH$.

\begin{cor} \label{cor:danishCarbonSavingsContinuous}
In a setting where there are always jobs with outstanding tasks in the data processing queue, the average carbon savings of \DANISH at any given discrete time step $t$ is given by $\left(\rho_{\texttt{PB}} K - \rho_{\DANISH}(c_t) K \right) c_t$.
\end{cor}
\begin{proof}
In this setting, the expression of the \textit{average} carbon savings at any given discrete time step simplifies as follows:

Let $\rho_{\texttt{PB}}$ denote the average machine utilization of \texttt{PB}'s schedule, i.e., $\rho_{\texttt{PB}} = \lim_{T\to \infty} \frac{\sum_{i=0}^T \nicefrac{E^\texttt{PB}_i}{K}}{T}$.

Then by \autoref{thm:danishCarbonSavings}, the average carbon savings $\overline{C}_s$ at  any time step $t$ is given by the following:
\begin{align*}
\overline{C}_s(t) &= \left(\rho_{\texttt{PB}} K - \rho_{\DANISH}(c_t) K \right) c_t.
\end{align*}
\end{proof}

\subsection{Deferred Proofs and Discussion from \autoref{sec:cap-design} (Carbon-aware provisioning (\CAP))} \label{apx:cap-proofs}

In this section, we discuss and prove the analytical results for \CAP, introduced in \autoref{sec:cap-design}.  Throughout this section, we let $\texttt{AG}$ denote a \textit{carbon-agnostic} baseline scheduler.

For the sake of analysis, in the following results we leverage the classic makespan bound of Graham's list scheduling algorithm~\cite{Graham:66}, which is known to produce a schedule that is a $(2 - \nicefrac{1}{K})$-approximation for the optimal makespan on $K$ identical machines.  
Note that FIFO is an instance of list scheduling where the list (of tasks) is a FIFO queue, and the next task is assigned to a machine as soon as it becomes idle.

Suppose that for a job $\mathcal{J}$, \CAP's schedule completes it at time $T' \geq 0$.
Note that if $c(t) = L$ for all time steps, the schedule of \CAP is identical to that of \texttt{AG} because \CAP always sets $r(t) = K$.  Otherwise, we let $\OPT_{K}(\mathcal{J})$ denote the optimal makespan on $K$ machines, and let $\mathcal{M}(B, \mbf{c})$ denote the minimum resource cap specified by \CAP at any point in its schedule (note this depends on the carbon signal $\mbf{c}$).  Formally, we let: $\mathcal{M}(B, \mbf{c}) = \arg \max_{i\in[K]} \Phi_i : \Phi_i \le c(t) \ \forall t \in [0, T']$.

\subsubsection{\textbf{Proof of \autoref{thm:ksMakespan}}}\label{apx:ksMakespan}

We are now ready to prove \autoref{thm:ksMakespan}, which states that \CAP's
carbon stretch factor is $\left(\frac{K}{\mathcal{M}(B, \mbf{c})}\right)^2 \frac{2\mathcal{M}(B, \mbf{c})-1}{2K-1}$.
We prove the statement by first showing that \CAP's makespan is at most $\left( \frac{2K}{\mathcal{M}(B, \mbf{c})} - \frac{K}{\mathcal{M}(B, \mbf{c})^{2}} \right) \OPT_{K}(\mathcal{J})$.

\smallskip

\begin{proof}
Let $\CAP_K(\mathcal{J} \mid \mathcal{M}(B, \mbf{c}) )$ denote the makespan of \CAP given $K$ machines conditioned on the value of $\mathcal{M}(B, \mbf{c})$, and let $\texttt{AG}_K(\mathcal{J})$ denote the makespan of \texttt{AG} (i.e., Graham's list scheduling with $K$ machines).

First, note that the following holds by~\cite{Graham:66}:
\[
\texttt{AG}_K(\mathcal{J}) \le \left(2 - \frac{1}{K}\right) \OPT_K(\mathcal{J}).
\]

Second, note that the makespan of $\CAP_K(\mathcal{J} \mid \mathcal{M}(B, \mbf{c}) )$ is upper-bounded by that of $\texttt{AG}_{\mathcal{M}(B, \mbf{c})}(\mathcal{J}) $.  By definition of $\mathcal{M}(B, \mbf{c})$, $\CAP$ always has $\mathcal{M}(B, \mbf{c})$ machines available, which is the same as $\texttt{AG}_{\mathcal{M}(B, \mbf{c})}(\mathcal{J})$.  If any other machines become available and process jobs during the schedule of $\CAP_K(\mathcal{J} \mid \mathcal{M}(B, \mbf{c}) )$, these \textit{strictly help} the makespan with respect to $\texttt{AG}_{\mathcal{M}(B, \mbf{c})}(\mathcal{J})$.  Thus, we have:
\[
\CAP_K(\mathcal{J} \mid \mathcal{M}(B, \mbf{c}) ) \le \texttt{AG}_{\mathcal{M}(B, \mbf{c})}(\mathcal{J})
\]

Furthermore, we have the following relationship between the optimal makespans (for the same job) when given different amounts of machines.  Letting $\mathcal{M}(B, \mbf{c}) \le K$, we have that:
\[
\OPT_{\mathcal{M}(B, \mbf{c})}(\mathcal{J}) \le \frac{K}{\mathcal{M}(B, \mbf{c})} \OPT_K(\mathcal{J}).
\]

To observe this, consider the limiting case as $\mathcal{M}(B, \mbf{c}) \to 1$.     When $\mathcal{M}(B, \mbf{c}) = 1$, the optimal makespan contains no ``gaps'', in the sense that the single machine is always being utilized.  If the job is perfectly parallelizable and subdividable, we have that $\OPT_{\mathcal{M}(B, \mbf{c})}(\mathcal{J}) = \frac{K}{\mathcal{M}(B, \mbf{c})} \OPT_K(\mathcal{J})$ by a geometric proof (i.e., $\OPT_K(\mathcal{J})$ has a makespan that is $\nicefrac{1}{K}$ as long as $\OPT_1(\mathcal{J})$).  For any other job, as the number of machines increases, the utilization of machines worsens.  

\begin{figure}[h]
    \centering
    \includegraphics[width=0.5\linewidth]{figures/cap-makespan-explainer.png} \vspace{-1em}
    \caption{ An example to contextualize how the optimal makespan differs when a job is given different amounts of machines.  In the case of a single machine (i.e., $\OPT_1(\mathcal{J})$), the precedence constraints defined by the DAG on the right-hand side of the figure are non-binding -- there is always one or more tasks that are ready to execute.  As the number of machines increases, situations arise where some machines must be idle (e.g., in $\OPT_3(\mathcal{J})$), indicated by ``blank slots'' in the optimal schedule.  The relation between makespans is then formalized by considering a hypothetical schedule (in brackets) that ignores precedence constraints and subdivides individual tasks across machines -- while this is an infeasible schedule, it forms a \textit{lower bound} on the makespan of any feasible one. }
    \label{fig:cap-makespan-explainer}
\end{figure}

We give a visual example of such a job $\mathcal{J}$ with $N=10$ tasks in \autoref{fig:cap-makespan-explainer} -- observe that respecting the precedence constraints in the case with $K=3$ machines necessarily forces a makespan that is greater than the hypothetical best makespan if jobs are perfectly parallelizable and subdividable. 
Thus, we have that scaling $\OPT_K(\mathcal{J})$ by $\frac{K}{\mathcal{M}(B, \mbf{c})}$ is always an upper bound on the optimal makespan $\OPT_{\mathcal{M}(B, \mbf{c})}(\mathcal{J})$.



Combining the above bounds, we obtain the following:
\begin{align*}
\CAP_K(\mathcal{J} \mid \mathcal{M}(B, \mbf{c}) ) &\le \texttt{AG}_{\mathcal{M}(B, \mbf{c})}(\mathcal{J}),\\
&\le \left(2 - \frac{1}{\mathcal{M}(B, \mbf{c})}\right) \OPT_{\mathcal{M}(B, \mbf{c})}(\mathcal{J}),\\
&\le \left( \frac{2K}{\mathcal{M}(B, \mbf{c})} - \frac{K}{\mathcal{M}(B, \mbf{c})^{2}} \right) \OPT_K(\mathcal{J}).
\end{align*}

This gives that the carbon stretch factor (\sref{Definition}{dfn:csf}) of \CAP is given by $\left(\frac{K}{\mathcal{M}(B, \mbf{c})}\right)^2 \frac{2\mathcal{M}(B, \mbf{c})-1}{2K-1}$. 
\end{proof}

\noindent We now turn to the question of carbon savings, and the result stated in \autoref{thm:ksCarbonSavings}.  
For ease of analysis, we again consider a \textit{discretized time model} as defined in \sref{Appendix}{apx:danishCarbonSavings}.

Slightly abusing notation, we let $C_{\CAP}(t)$ denote the carbon emissions of $\CAP$'s schedule during discrete time step $t$, and let $C_{\texttt{AG}}(t)$ denote the carbon emissions of $\texttt{AG}$ at discrete time step $t$, respectively.  
Schedules generated by $\CAP$ and $\texttt{AG}$ each use a certain number of machines during any discrete time interval -- to capture this, we let $E^\CAP_t \leq r(t') : t' \in [t, t+1)$ denote the number of active machines during discrete time step $t$ in \CAP's schedule.  Note that  $r(t')$ is constant on the interval $t \in [t, t+1)$, and that $E^\CAP_t$ \textit{need not be} an integer -- i.e., if a machine is active for only half of the discrete time interval, that machine contributes fractionally to $E^\CAP_t$.  We let $E^\texttt{AG}_t \leq K$ denote the same quantity for \texttt{AG}'s schedule.

On a per-job basis, let $T$ denote the makespan of $\texttt{AG}$ (i.e., $T = \texttt{AG}_K(\mathcal{J})$), where note that $T \le T'$.
In what follows, we use $W$ as shorthand to denote the \textit{excess work} that \CAP must complete after time $T$ (i.e., after $\texttt{AG}$ has completed).  Formally, $W = \sum_{i=0}^T E^\texttt{AG}_t - E^\CAP_t$.  We also define quantities $\overline{s}^{(0,T)}$ and $\overline{c}^{(T, T')}$, which are weighted averages based on a combination of the carbon intensity and schedules of \texttt{AG} and \CAP, respectively. 
These can be interpreted as the realization of carbon intensities that \CAP ``waited for'' -- in other words, it deferred some work in between time $0$ and time $T$ (saving $\overline{s}^{(0,T)} $ amount of carbon), so it must make up the difference after time $T$.  

\subsubsection{\textbf{Proof of \autoref{thm:ksCarbonSavings}}}\label{apx:ksCarbonSavings}

We are ready prove \autoref{thm:ksMakespan}, which states that \CAP yields carbon savings compared to a carbon-agnostic baseline of $W \left( \overline{s}^{(0,T)} - \overline{c}^{(T, T')} \right)$.

\begin{proof}
Slightly abusing notation, we let $C_s(t)$ denote the \textit{carbon savings} of \CAP at discrete time step $t$.  Formally, we have:
\[
C_s(t) = \begin{cases}
    C_{\texttt{AG}}(t) - C_{\CAP}(t) &  1 \leq t \leq T,\\
    - C_{\CAP}(t) & T < t \leq T'.
\end{cases}
\]
By definition, we have the following by substituting for the carbon emissions of $\texttt{ECA}$ and \CAP:
\[
C_s(t) = \begin{cases}
    E^\texttt{AG}_t c_t  - E^\CAP_t c_t &  1 \leq t \leq T,\\
    - E^\CAP_t c_t& T < t \leq T'.
\end{cases}
\]
Summing over all time steps, we have that the carbon savings is given by:
\[
\sum_{i=0}^{T'} C_s(i) = \sum_{i=0}^{T} (E^\texttt{AG}_i-E^\CAP_i) c_i - \sum_{i=T+1}^{T'} E^\CAP_i c_i
\]
To simplify this expression, we define two terms that capture the \textit{weighted average} carbon intensity per unit of work completed/deferred.  First, note that $\sum_{i=0}^{T} (E^\texttt{AG}_i-E^\CAP_i) = \sum_{i=T+1}^{T'} E^\CAP_i = W$.

Formally, we let $\overline{s}^{(0,T)}$ denote the weighted average carbon intensity of the deferred work $W$ that is completed by \texttt{AG} before time $T$ but must be completed after time $T$ by \CAP:
\[
\overline{s}^{(0,T)} = \nicefrac{\sum_{i=0}^{T} (E^\texttt{AG}_i-E^\CAP_i) c_i}{W}
\]

Similarly, we let $\overline{c}^{(T, T')}$ denote the weighted average carbon intensity of the deferred work $W$ that is completed by \CAP after time $T$:
\[
\overline{c}^{(T,T')} = \nicefrac{\sum_{i=T+1}^{T'} E^\CAP_i c_i}{W}
\]
Under the above definitions, we can pose the total carbon savings as:
\[
\sum_{i=0}^{T'} C_s(i) = W \left( \overline{s}^{(0,T)} - \overline{c}^{(T,T')} \right)
\]
\end{proof}

The above result contextualizes the total carbon savings achieved by \CAP for a single job, but we may also consider the average carbon savings at each (discrete) carbon intensity interval as follows.
We let $\rho_{\texttt{AG}} \in [0,1)$ denote the average cluster utilization of $\texttt{AG}$, and we let $\rho_{\texttt{CAP}} \in [0,1)$ denote the average cluster  utilization of \CAP.  In general, since \CAP allows jobs to use less than or equal the amount of resource that \texttt{AG} allows, we expect $\rho_{\texttt{AG}} \leq \rho_{\texttt{CAP}}$ as for the same number of jobs submitted (e.g., during a given period), jobs will take up a greater proportion of the resources that \CAP allows. 


\begin{cor} \label{cor:ksCarbonSavingsContinuous}
In a setting where there are always jobs with outstanding tasks in the data processing queue, the average carbon savings of \CAP at any given discrete time step $t$ is given by $(\rho_{\texttt{AG}} K - \rho_{\CAP} r_t ) \Phi_{r_t + B}$.
\end{cor}
\begin{proof}
In a setting where there are always jobs with outstanding tasks in the data processing queue, the expression of the \textit{average} carbon savings at any given discrete time step simplifies as follows:

Let $\rho_{\texttt{AG}}$ denote the average machine utilization of \texttt{AG}'s schedule, i.e., $\rho_{\texttt{AG}} = \lim_{T\to \infty} \frac{\sum_{i=0}^T \nicefrac{E^\texttt{AG}_i}{K}}{T}$, and let $\rho_{\CAP}$ denote the same for \CAP's schedule, i.e., $\rho_{\CAP} = \lim_{T\to \infty} \frac{\sum_{i=0}^T \nicefrac{E^\CAP_i}{j_i}}{T}$

Then the average carbon savings $\overline{C}_s$ at  any time step $t$ is given by the following, where $r_t \coloneqq r(t') : t' \in [t, t+1)$:
\begin{align*}
\overline{C}_s(t) &= \left(\rho_{\texttt{AG}} K - \rho_{\CAP} r_t \right) c_t, \\
& \ge \left(\rho_{\texttt{AG}} K - \rho_{\CAP} r_t \right) \Phi_{r_t + B}.
\end{align*}
\end{proof}
