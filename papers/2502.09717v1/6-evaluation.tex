\noindent We evaluate our carbon-aware schedulers in a prototype cluster and a realistic Spark simulator, using workloads from TPC-H benchmarks~\cite{TPCH:18} and Alibaba production DAG traces~\cite{Alibaba:18}.  We answer the following questions:
\begin{enumerate}[leftmargin=*]
    \item How do \DANISH and \CAP navigate the trade-off between carbon emissions and job completion time?
    \item How do \DANISH and \CAP adapt to changes in carbon intensity characteristics and workload characteristics?
\end{enumerate}
\vspace{-0.5em}


\subsection{Experimental setup} 
\label{sec:carbon-traces}


\begin{table}[t]
\caption{Summary of carbon intensity trace characteristics, including the duration, granularity, minimum, maximum, mean, and coefficient of variation (\textit{higher value implies more variation}) for carbon intensities.} \label{tab:characteristics} \vspace{-1em}
\begin{tabular}{|l|lllll|}
\hline
  & \multicolumn{5}{c|}{\begin{tabular}[c]{@{}c@{}}Avg. Carbon Intensity \vspace{-0.1em}\\ {\footnotesize \textit{(in gCO$_2$eq./kWh)} \cite{electricity-map}} \end{tabular}} \\ \cline{2-6} 
\multirow{-2}{*}{\begin{tabular}[c]{@{}c@{}}Grid \vspace{-0.1em}\\ Code \end{tabular}}           & \multicolumn{1}{l|}{ {\small Duration} }   & \multicolumn{1}{l|}{{\small Min.}}  & \multicolumn{1}{l|}{{\small Max.}} & \multicolumn{1}{l|}{{\small Mean}} & {\small Coeff. Var.}  \\ \hline
\worldflag[length=0.75em, width=0.6em]{US} {\small \texttt{\textbf{PJM}}}                               & \multicolumn{1}{l|}{}                                                                                      & \multicolumn{1}{l|}{293} & \multicolumn{1}{l|}{567} & \multicolumn{1}{l|}{425} & \multicolumn{1}{l|}{0.110}   \\ \cline{1-1} \cline{3-6} 
\worldflag[length=0.75em, width=0.6em]{US} {\small \texttt{\textbf{CAISO}}}      & \multicolumn{1}{l|}{}                                                                                      & \multicolumn{1}{l|}{83}  & \multicolumn{1}{l|}{451} & \multicolumn{1}{l|}{274} & \multicolumn{1}{l|}{0.309}    \\ \cline{1-1} \cline{3-6} 
\worldflag[length=0.75em, width=0.6em]{CA} {\small \texttt{\textbf{ON}}}   & \multicolumn{1}{l|}{}                                                                                      & \multicolumn{1}{l|}{12}  & \multicolumn{1}{l|}{179}  & \multicolumn{1}{l|}{50} & \multicolumn{1}{l|}{0.654}   \\ \cline{1-1} \cline{3-6}
\worldflag[length=0.75em, width=0.6em]{DE} {\small \texttt{\textbf{DE}}}   & \multicolumn{1}{l|}{}                                                                                      & \multicolumn{1}{l|}{130} & \multicolumn{1}{l|}{765} & \multicolumn{1}{l|}{440} &\multicolumn{1}{l|}{0.280}  \\ \cline{1-1} \cline{3-6} 
\worldflag[length=0.75em, width=0.6em]{AU} {\small \texttt{\textbf{NSW}}} & \multicolumn{1}{l|}{}                                                                                      & \multicolumn{1}{l|}{267} & \multicolumn{1}{l|}{817} & \multicolumn{1}{l|}{647} & \multicolumn{1}{l|}{0.143}    \\ \cline{1-1} \cline{3-6} 
\worldflag[length=0.75em, width=0.6em]{ZA} {\small \texttt{\textbf{ZA}}}     & \multicolumn{1}{l|}{\multirow{-6}{*}{\small \begin{tabular}[c]{@{}l@{}}01/01/2020-\\ 12/31/2022 \\ Hourly \\ granularity\\ 26,304\\ data points\end{tabular}}}                                                                                      & \multicolumn{1}{l|}{586} & \multicolumn{1}{l|}{785}  & \multicolumn{1}{l|}{713} & \multicolumn{1}{l|}{0.046}                                                                                               \\ \hline
\end{tabular}
\end{table}
\begin{figure}[t]
    \centering 
    \vspace{-1em}
    \includegraphics[width=\linewidth]{figures/CI-traces.png} \vspace{-2.5em}
    \caption{ Time-varying carbon intensity for six grids (detailed in \autoref{tab:characteristics}) over 48 hours in January 2021.}
    \label{fig:CI-traces} \vspace{-1em}
\end{figure}


\noindent
\textbf{Carbon intensity traces.}
We use historical carbon traces from six regions~\cite{electricity-map} -- each trace provides hourly carbon intensity data in grams of CO$_2$ equivalent per kilowatt-hour (gCO$_2$eq./kWh).  The chosen power grids represent different energy generation mixes and thus different characteristics in terms of average carbon intensity and variability; we evaluate how these impact the behavior of \DANISH and \CAP.
In \autoref{tab:characteristics} and \autoref{fig:CI-traces}, we give snapshots of each region, showing how grid characteristics impact time-varying carbon intensity.  Larger \textit{coefficients of variation} (the ratio of the standard deviation to the mean) correspond to greater renewable penetration -- for instance, a large fraction of \verb|CAISO|'s capacity is solar PV, while the capacity in \verb|ZA| is predominantly coal.



To better observe the behavior of our carbon-aware schedulers, we follow prior work~\cite{Goiri:2012:GreenHadoop} and scale time in our experiments such that $1$ minute of \textit{real time} corresponds to $1$ hour of \textit{experiment time} -- since carbon intensity is reported hourly, this approximates a scenario where each job works with large amounts of data and runs for several hours, as is becoming common in e.g., data curation for LLMs~\cite{Liu:24, Chen:24, Villalobos:24, Bother:24, Brown:20}.

\vspace{0.05cm}
\noindent
\textbf{Workload traces.}
For workloads, we use TPC-H benchmarks~\cite{TPCH:18} and real DAG traces from a production Alibaba cluster~\cite{Alibaba:18}.  
We construct workloads such that the inter-arrival times follow a Poisson distribution while specific jobs are randomly picked from the respective traces. 
In the main body, we consider an average inter-arrival time of 30 minutes (30 real-time seconds), with additional experiments measuring the impact of this parameter in \sref{Appendix}{apx:exp}.

The TPC-H queries we experiment with operate on synthetic data with scales of 2 GB, 10 GB, and 50 GB -- these correspond to average real durations of 180 seconds, 386 seconds, and 1,261 seconds when given a single executor.
In our prototype experiments, we also construct workloads based on DAG information from the Alibaba trace~\cite{Alibaba:18}.  These DAGs exhibit a realistic power law distribution (many DAGs of short duration, few DAGs of long duration), they have 66 nodes on average, and an average total duration (on one executor) of 7,989 seconds. %
We scale all durations by $\nicefrac{1}{60}$ to match our experiment scale -- this yields jobs that take 2.2 real-time minutes to complete on average.  %

In the simulator, each experiment is run over a full carbon trace (spanning three years of data).  In the prototype, each experiment is run for several trials, where each starts at a uniformly randomly chosen time in the carbon trace, and new carbon intensities are reported once per (real-time) minute.  In both implementations, the upper and lower bounds of $U$ and $L$ correspond to the maximum and minimum forecasted carbon intensities over a lookahead window of 48 hours.

\begin{figure*}[t]
    \centering
    \includegraphics[width=0.95\linewidth]{figures/sim/multiplot.png}
    \vspace{-0.2cm}
    {\raggedleft
    \hspace{5.5em} \textbf{\textit{(a)}} Decima \hspace{6em} \textbf{\textit{(b)}} \PCAPS \hspace{6em} \textbf{\textit{(c)}} \CAP-FIFO \hspace{6em} \textbf{\textit{(d)}} Carbon intensity \hfill}
    \caption{ Visualizing executor usage over time for three schedulers, \textit{(a)} Decima, \textit{(b)} \PCAPS, and \textit{(c)} \CAP-FIFO in a small simulator cluster with 5 executors and 20 TPC-H jobs, over a 15 hour period in the \textnormal{\texttt{DE}} grid. \textit{(d)} In the executor plots, each job is a unique shade of blue, while ``idle'' executors are indicated by a white background. }
    \vspace{-0.4cm}
    \label{fig:multiplot}
\end{figure*}

\vspace{0.05cm}
\noindent\textbf{Baselines.}
We compare against the following baselines:


$\blacktriangleright$ \textbf{Default Spark/Kubernetes behavior (default)}:
The default behavior of Spark on Kubernetes -- Spark uses first in, first out (FIFO) to choose stages within a job, while the Kubernetes scheduler mediates between pods of each job during execution~\cite{SparkKubernetes}. 
In the simulator's Spark standalone mode, this baseline implements only the FIFO scheduling.  



$\blacktriangleright$ \textbf{Decima}:
An RL scheduler for Spark that is optimized for job completion time~\cite{Hongzi:2019:Decima}.  We use the simulator's training environment to train Decima for 20,000 epochs.


$\blacktriangleright$ \textbf{Weighted Fair}: A heuristic that assigns executors proportionally to each job's workload, with tuned weights to improve performance on the simulated workloads~\cite{Hongzi:2019:Decima}. 

$\blacktriangleright$ \textbf{GreenHadoop:} A MapReduce framework proposed to leverage green energy by matching workloads with the availability of solar~\cite{Goiri:2012:GreenHadoop}.  This framework predates Spark, so we adapt its key ideas for DAG scheduling in the simulator -- see \autoref{apx:extra-details} for implementation details.



\vspace{0.05cm}
\noindent 
\textbf{Metrics.} We use three metrics to evaluate the fidelity of our approach in reducing carbon footprint without significantly impacting the completion time. 

\vspace{0.05cm}
$\blacktriangleright$ \textbf{Carbon Footprint}: 
We report the carbon footprint for various scheduling policies as a percentage decrease of the carbon-agnostic default baseline unless stated otherwise. 
The values are in the range of [-100\%, $\infty$), with negative values indicating a carbon reduction and positive values indicating an increase relative to the baseline. 
 
$\blacktriangleright$ \textbf{Job Completion Time (JCT)}:
We report the average job completion time across all the jobs in each experimental run. 
We report JCT as a fraction of the average JCT for the carbon-agnostic default baseline unless stated otherwise. 
The values can be in the (0, $\infty$) range, with below 1 indicating a reduction in JCT and above 1 indicating an increase in JCT.

$\blacktriangleright$ \textbf{End-to-end Completion Time (ECT)}:
We report the total time to complete all the jobs in a given experiment as the end-to-end completion time (ECT) as a fraction of the ECT for carbon-agnostic default. 
Its values lie in the same range as JCT. 
However, while JCT focuses on individual jobs, ECT represents the system's throughput and efficiency. 
Also, as \DANISH and \CAP focus on minimizing the total carbon footprint for a set of jobs and not the instantaneous rate of carbon consumption, ECT is a better metric for performance with respect to time in our case.

\subsection{Carbon-aware schedulers in action} \label{sec:in-action}

Before moving to our main results, we demonstrate the carbon-aware behavior of our schedulers in \autoref{fig:multiplot}, which visualizes the schedules generated by Decima, \PCAPS and \CAP-FIFO during a short period in the \verb|DE| grid on the simulator.
\PCAPS makes fine-grained scheduling decisions, idling specific executors during the high-carbon period ($t = (5,8)$) while keeping bottleneck tasks running, achieving the lowest carbon footprint of the three schedules shown.  This is in contrast to \CAP-FIFO, which applies a resource quota uniformly across the cluster without consideration of bottlenecks -- note the gaps in \CAP-FIFO's schedule that are straight vertical lines across multiple executors.  Just before $t = 5$, a large gap in the schedule indicates that \CAP-FIFO cannot run tasks because it did not prioritize bottleneck tasks early on.


\begin{table}[h]
\caption{Summary of prototype results averaged over all six carbon traces.  Each metric is normalized with respect to the Spark / Kubernetes default.  \DANISH and \CAP are configured to be moderately carbon aware. %
} 
\vspace{-1em} 
\label{tab:results-prototype}
{\small
\begin{tabular}{|l|l|l|l|l|}
\hline
{\footnotesize \textit{\textbf{\begin{tabular}[c]{@{}l@{}}Metric normalized \\ w.r.t. Default\end{tabular}}}} & {\footnotesize \textbf{Default} } & {\footnotesize \textbf{Decima}~\cite{Hongzi:2019:Decima} } & {\footnotesize \textbf{\CAP} } & {\footnotesize \textbf{\DANISH} } \\ \hline
{\footnotesize CO$_2$ Reduction (\%) } & 0\% & 1.2\% & 24.7\% & \textbf{32.9\%} \\ \hline
{\footnotesize Avg. ECT } & 1.0 & 0.857 & 1.126 & \textbf{1.013 }\\ \hline
{\footnotesize Avg. JCT } & 1.0 & 0.852 & 1.996 & \textbf{1.381} \\ \hline
\end{tabular}
} \vspace{-1em}
\end{table}

\subsection{Prototype experiments} \label{sec:eval-proto}


Our prototype is deployed on an OpenStack cluster running Kubernetes v1.31 and Spark v3.5.3 (both modified per \autoref{sec:impl}) in Chameleon Cloud~\cite{Keahey:20:Chameleon}. 
Our testbed consists of 51 \verb|m1.xlarge| virtual machines, each with 8 VCPUs and 16GB of RAM.  
One VM is designated as the control plane node, while the remaining 50 are workers, each hosting two executor pods.
Our Spark configuration allocates 4 VCPUs and 7GB of RAM to each of the 100 executors\footnote{ Spark's default \textit{memory overhead factor} is 10\%. The difference between 7GB ($\times 2$) and the 16GB RAM of each worker is to accommodate this~\cite{SparkConfiguration}.}.  
To avoid a known issue with Spark's dynamic allocation feature that can cause it to hang on Kubernetes~\cite{SparkKubernetes}, we configure an upper limit of 25 executors that can be allocated to any single job.  
We implement a carbon intensity API that replays historical traces to test our carbon-aware schedulers in the prototype.  
In our prototype, we implement default and Decima as the baselines. 
Unless stated otherwise, the results are averaged over the batch sizes of 25, 50, and 100 jobs. 
Furthermore, the results for each experimental configuration are averaged over 10 trials. 


\begin{figure}[t]
    \centering 
    \includegraphics[width=0.8\linewidth]{figures/proto/gamma-tradeoff-proto.png} \vspace{-0.4cm}
    \caption{Relative carbon footprint and end-to-end completion times (w.r.t. the Spark/Kubernetes default) for \DANISH in the prototype, with five different degrees of carbon-awareness ($\gamma$). The shaded region denotes the standard deviation across 10 random trials. }
    \label{fig:gamma-tradeoff-proto}
    \includegraphics[width=0.8\linewidth]{figures/proto/B-tradeoff-proto.png} 
    \vspace{-0.4cm}
    \caption{ Relative carbon footprint and end-to-end completion times (w.r.t. the Spark/Kubernetes default) for \CAP in the prototype, with five different degrees of carbon-awareness ($B$).  The shaded region denotes the standard deviation across 10 random trials. }
    \label{fig:B-tradeoff-proto} \vspace{-0.5cm}
\end{figure}





\smallskip
\noindent \textbf{Results. \ } \autoref{tab:results-prototype} presents the results for our prototype experiments.
\DANISH and \CAP configured to be \textit{moderately carbon-aware} (\CAP is configured with $B = 20$ and \DANISH is configured with $\gamma = 0.5$) achieve average carbon reductions of 32.8\% and 24.6\% compared to the default baseline, respectively.  Compared to Decima, \DANISH reduces carbon by 32.1\%.  

In terms of JCT, \DANISH performs significantly worse than the default and Decima baselines; JCT increases by 38.1\% and 62\% with respect to the default and Decima, respectively. 
This is expected since Decima targets minimizing average JCT. 
However, the key objective of \DANISH is to reduce the total carbon footprint without increasing ECT. 
\DANISH increases average ECT by only 12.4\% compared to Decima and 1.3\% compared to the default.  \CAP also performs well and increases average ECT by 12.6\% on top of the default. 





\begin{figure}[t]
    \centering 
    \includegraphics[width=0.8\linewidth]{figures/proto/scatter-proto.png} 
    \vspace{-0.4cm}
    \caption{ Per-job metrics for \DANISH and \CAP in the prototype -- each point represents a single trial's average JCT and per-job carbon footprint.  Trials are normalized so that the Spark/Kubernetes default is represented by $(1, 1)$ -- this splits the plot into quadrants, and each is annotated with the percentage of trials it holds (\DANISH is the top percentage, and \CAP is the bottom).  Contour lines outline a Gaussian KDE for each point cluster. %
    }
    \label{fig:scatter-proto} \vspace{-0.6cm}
\end{figure} 


\noindent \textbf{Trade-offs between carbon and job completion time. \ }
We next test several parameter settings for \DANISH and \CAP to configure their carbon awareness in the \verb|DE| grid region with batches of 50 TPC-H or Alibaba jobs.  
\autoref{fig:gamma-tradeoff-proto} plots the carbon-time trade-off for five settings of \DANISH relative to the Spark/Kubernetes default.
Increasing the carbon awareness of \DANISH improves carbon savings at the expense of longer ECT, with a most pronounced effect for values of $\gamma$ approaching $1$.  
Conversely, \autoref{fig:B-tradeoff-proto} plots the same carbon-time trade-off for five settings of \CAP; \CAP sacrifices more in ECT (relative to \DANISH) for the same amount of carbon savings.



On a per-job level, we observe similar trends in \autoref{fig:scatter-proto}, which plots average JCT and average \textit{per-job} carbon footprint across trials of prototype experiments where \DANISH and \CAP are configured to be moderately carbon-aware.  Contour lines represent Gaussian kernel density estimators of the \textit{outcome distribution} -- note that the location of each ``hot spot'' represents the scheduler average.  Splitting the plot into quadrants, we observe that \DANISH improves on the baseline scheduler's per-job carbon footprint in 95.8\% of trials, corresponding to the lower two quadrants.  \DANISH improves on \textit{both} carbon and completion time in 25.7\% of cases, while \CAP does so in only 2.1\% of cases. 








\noindent \textbf{Effects of carbon intensity trace characteristics. \ }
Next, we analyze the effect of grid characteristics on the carbon-time trade-offs of our carbon-aware schedulers using subsets of each carbon trace (via 30 trials with 25, 50, and 100 jobs).

\autoref{fig:traces-proto} plots the carbon reduction and average ECT of \DANISH, \CAP, and Decima.  
Decima is carbon-agnostic and shows a minimal reduction in carbon that stays relatively constant across all regions.  \DANISH and \CAP incorporate grid behavior into their decisions, and we observe a positive relationship between the variability of a carbon trace and the resulting carbon reduction.
For example, in \verb|ZA| where carbon is relatively constant, high-carbon periods do not prompt \DANISH to defer tasks since the \textit{future potential reductions} are insignificant.  
In contrast, high-carbon periods in the \verb|CAISO| grid correspond to nighttime scenarios on the grid, where the prospect of daytime solar bolsters future potential reductions.
These interactions are further illustrated through ECT -- grid regions with more intermittent and variable energy mixes drive increases in ECT in exchange for more carbon reduction, since \DANISH and \CAP \textit{wait} for potential reductions.


\begin{figure}[t]
    \centering 
    \includegraphics[width=0.9\linewidth]{figures/proto/traces-proto.png}  
    \vspace{-0.4cm}
    \caption{ Carbon reduction \textit{(left)} and ECT \textit{(right)} for \DANISH, \CAP, and Decima in six grid regions. Shaded regions denote standard deviation across 30 trials. }
    \label{fig:traces-proto} 
    \vspace{-0.5cm}
\end{figure} 


\vspace{-0.5em}

\subsection{Simulator experiments}
\label{sec:eval-sim}

In the simulator, we evaluate \DANISH and \CAP using TPC-H workloads, comparing them against \textbf{Weighted Fair} and \textbf{GreenHadoop} baselines, in addition to the Decima and default baselines from prototype experiments. 
We renamed the default baseline as FIFO for simulation-based experiments for accurate representation.



\noindent\textbf{Simulator fidelity.} To establish the fidelity of our simulator, we illustrate the granular effect of differences for a batch of 50 TPC-H jobs in \sref{Appendix}{apx:fifo-vs-k8s}.  
A notable difference between the prototype and the simulator is the relative performance of the main baseline (FIFO in the simulator, Spark/Kubernetes default in the prototype).  
In short, the simulator's FIFO scheduler \textit{over-assigns} executors to individual jobs, blocking others from entering service (thus increasing JCT) -- this also increases its relative carbon footprint compared to the default behavior of our prototype.






\begin{table*}[t]
\caption{Summary of results for simulator experiments averaged over all 6 tested carbon traces.  Each metric is normalized with respect to the default Spark FIFO behavior.  \DANISH and \CAP are configured to be moderately carbon-aware, and end-to-end completion time measures the total flow time for batches of jobs arriving continuously. } \vspace{-1em} \label{tab:results-simulator}
{\footnotesize
\begin{tabular}{|l|l|l|l|l|l|l|l|l|}
\hline
\multirow{2}{*}{\footnotesize \textbf{Metric} \textit{(normalized with respect to FIFO)}} & \multirow{2}{*}{\textbf{FIFO}} & \multirow{2}{*}{\textbf{W. Fair}} & \multirow{2}{*}{\textbf{Decima}~\cite{Hongzi:2019:Decima}} & \multirow{2}{*}{\textbf{GreenHadoop}~\cite{Goiri:2012:GreenHadoop}} & \multicolumn{3}{c|}{\textbf{\CAP}} & \multirow{2}{*}{\textbf{\DANISH}} \\ \cline{6-8} 
& & & & & FIFO & W. Fair & Decima & \\ \hline
Carbon Reduction (\%)                                                                & 0\% & 12.1\% & 21.5\% & 8.2\%  & 22.7\% & 34.2\% & 31.1\% & \textbf{39.7\%}                             \\ \hline
Avg. End-to-End Completion Time                                                 & 1.0 & 0.972 & 0.970 & 1.077 & 1.108 & 1.011 & 1.061 & \textbf{1.045 }                            \\ \hline
Avg. Job Completion Time                                                        & 1.0 & 0.652 & 0.654 & 1.918 & 2.274 & 1.217 & 1.479 & \textbf{1.436}                             \\ \hline
\end{tabular}
} \vspace{-0.1cm}
\end{table*}


\smallskip
\noindent \textbf{Results. \ } 
\autoref{tab:results-simulator} presents our top-line results showing \DANISH and \CAP achieve significant reductions in carbon emissions compared to the baselines. 
Configured to be moderately carbon-aware, \DANISH achieves an average reduction of 23.1\% compared to Decima, and a reduction of 39.7\% compared to FIFO.  \CAP achieves an average carbon reduction of 22.7\% when implemented on top of FIFO, 25.1\% on top of Weighted Fair, and 14.5\% on top of Decima. 
For 25, 50, and 100 jobs, \DANISH increases average end-to-end completion time (ECT) by 7.7\% with respect to Decima, which is only a 4.5\% degradation compared to FIFO.  For \CAP, average ECT increases by 10.8\% when implemented on top of (and compared to) FIFO, 4.0\% on top of Weighted Fair, and 9.3\% on top of Decima.   

At a per-job level, \DANISH increases the average job completion time (JCT) by 119.57\% compared to Decima.  Similarly, \CAP increases average JCT by 127.4\%, 86.6\%, and 126.8\% when implemented with FIFO, Weighted Fair, and Decima respectively.  Larger increases in JCT relative to ECT 
happen because \DANISH allows more queue build-up during high-carbon periods, but ``makes up for lost time'' in low-carbon periods.  

\begin{figure}[t]
    \centering 
    \includegraphics[width=0.8\linewidth]{figures/sim/gamma-tradeoff-sim.png} \vspace{-1em}
    \caption{ Relative carbon footprint and end-to-end completion times (with respect to FIFO) for \DANISH in simulator experiments, given different values of $\gamma$ that correspond to degrees of carbon-awareness. Shaded region denotes standard deviation across carbon trace. }
    \vspace{-0.5cm}
    \label{fig:gamma-tradeoff-sim}
\end{figure}
\begin{figure}[h]
    \centering 
    \includegraphics[width=0.8\linewidth]{figures/sim/B-tradeoff-sim.png} \vspace{-1em}
    \caption{ Relative carbon footprint and end-to-end completion times (with respect to FIFO) for \CAP-FIFO in simulator experiments, given different values of $B$ that correspond to degrees of carbon-awareness.  Shaded region denotes standard deviation across carbon trace. }
    \label{fig:B-tradeoff-sim} \vspace{-0.5cm}
\end{figure}




\noindent \textbf{Trade-offs between carbon and job completion time. \ }
In the results summary, we observe positive carbon reduction in exchange for degradation in JCT and ECT.  Since \DANISH and \CAP can be configured to be more or less carbon-aware, we explore this \textit{trade-off} in the \verb|DE| grid with batches of 50 jobs.  We vary hyperparameters $\gamma$ (\DANISH) and $B$ (\CAP) to measure the impact of configuration on both carbon and JCT.



In \autoref{fig:gamma-tradeoff-sim}, we illustrate this trade-off for \DANISH compared against FIFO.  As the carbon-awareness of \DANISH increases (indicated by the value of $\gamma$), the carbon savings of \DANISH improve at the expense of longer ECT.  This effect is most pronounced for large values of $\gamma$ approaching $1$, because $\DANISH$ defers many tasks to lower carbon periods.
Conversely, \autoref{fig:B-tradeoff-sim} illustrates the trade-off for \CAP-FIFO, showing a similar trend of improving carbon at the expense of longer ECT.  Compared to \autoref{fig:gamma-tradeoff-sim}, \CAP-FIFO sacrifices more in terms of ECT for the same or lower amounts of carbon savings, and the increase in completion time begins earlier (at lower degrees of carbon-awareness).


\noindent \textbf{Advantages of relative importance. \ }
Between \DANISH and \CAP-Decima, the carbon-agnostic scheduler is identical -- thus, performance differences can be attributed to the key ideas behind \DANISH, namely relative importance (see \autoref{sec:danish-design}).  In what follows, we examine this in detail using the \verb|DE| grid region with batches of 50 jobs.  We configure \DANISH and \CAP-Decima with ten parameter settings for varying carbon-awareness.  \autoref{fig:capdecima-vs-danish} plots the result of this experiment, where each dot denotes the outcome of one trial.  We fit a cubic polynomial to the outcomes of both methods to illustrate the key trend: \DANISH exhibits a strictly better trade-off between carbon footprint and ECT.  For trials where either method achieves carbon savings between 35\% and 45\%, \DANISH increases ECT by an average of 7.9\%, while \CAP-Decima increases it by an average of 42.7\%.  Conversely, for those trials where either method increases ECT by between 0\% and 10\%, \DANISH achieves average carbon savings of 35.6\%, while \CAP-Decima achieves an average savings of only 20.1\%.

\begin{figure}[t]
    \centering 
    \includegraphics[width=0.8\linewidth]{figures/sim/capdecima-vs-danish-sim.png} \vspace{-1em}
    \caption{ Relative carbon footprint %
    vs. end-to-end completion time %
    for \DANISH and \CAP-Decima in simulator experiments, given varying parameters $\gamma \in [0.1, 1.0]$ and $B \in \{ 5, 10, \dots , 85\}$ that correspond to degrees of carbon-awareness.  Each dot represents an individual trial, and lines represent a cubic polynomial of best fit. } %
    \label{fig:capdecima-vs-danish} \vspace{-0.6cm}
\end{figure}


\noindent \textbf{Effects of carbon intensity trace characteristics. \ } The top-line results in \autoref{tab:results-simulator} average over all six grid regions.  
We configure both schedulers to be moderately carbon-aware and characterize each carbon trace based on its \textit{coefficient of variation}. In \autoref{fig:traces-sim}, we plot the carbon reduction and ECT for each of \CAP-FIFO, \DANISH, and Decima.  
Decima's carbon reduction relative to the 
default is higher relative to that observed in the prototype -- this is due to differences between Spark's standalone FIFO scheduler and the Spark/Kubernetes behavior of our prototype (see \sref{Appendix}{apx:fifo-vs-k8s}).  Similarly, \DANISH's ``baseline'' carbon reductions increase alongside Decima's, and \CAP gains relative ground compared to \CAP-FIFO in the simulator.
We observe similar trends overall -- grid regions with more intermittent and variable energy mixes due to renewables drive increases in both carbon reduction and ECT, as observed in \autoref{fig:traces-proto}.

\begin{figure}[t]
    \centering 
    \vspace{-0.5em}
    \includegraphics[width=0.9\linewidth]{figures/sim/traces-sim.png} \vspace{-0.4cm}
    \caption{ Carbon reduction \textit{(left)} and increase in end-to-end completion time \textit{(right)} for \CAP, \DANISH, and Decima (relative to FIFO) in six grid regions.  Shaded areas denote the standard deviation across a carbon trace. }
    \label{fig:traces-sim} 
    \vspace{-0.4cm}
\end{figure}






\subsection{Takeaways}
Through evaluation in a realistic simulator and a prototype cluster, we show that a moderately carbon-aware \DANISH reduces carbon emissions by up to 39.7\% in exchange for modest increases ($< 10\%$) in ECT for batches of 25, 50, and 100 data processing jobs.  
\CAP configured to be moderately carbon-aware ($B=20$) is also effective, reducing carbon by up to 25.1\% with respect to the scheduler it is implemented on, in exchange for slightly larger increases in ECT.  While \CAP does not take dependencies into account and is suboptimal in terms of the carbon-time trade-off (see \autoref{fig:capdecima-vs-danish}), it is easier to implement and more general than \DANISH.
Intuitively, the performance of all carbon-aware techniques exhibits a dependence on the time-varying behavior of the power grid.
Greater carbon savings can be achieved in regions with more renewables (thus more variability) in carbon intensity.
