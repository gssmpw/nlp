

\noindent In this section, we present \PCAPS, our Precedence- and Carbon-Aware Provisioning and Scheduling system, and then, as a flexible and easy-to-implement alternative, we present \CAP (Carbon-Aware Provisioning).

\vspace{-0.5em}
\subsection{\PCAPS} \label{sec:danish-design}

From the discussion in \autoref{sec:theory}, we seek an interpretable and configurable scheduler that satisfies the conditions $\textbf{i - iii)}$.
To this end, we introduce \PCAPS (Precedence- and Carbon-Aware Provisioning and Scheduling), which interfaces with a probabilistic DAG scheduler such as Decima~\cite{Hongzi:2019:Decima}.





\PCAPS's key idea is a metric of \textit{relative importance} (\sref{Def.}{dfn:rel-imp}) that is implicitly embedded in a probability distribution over tasks.  We define a configurable carbon and importance-aware threshold function that uses the \textit{relative importance} metric to make per-task fine-grained scheduling decisions -- as illustrated in \autoref{fig:danish-diagram}. 
Next, we detail how the theoretical literature on threshold-based algorithms inspires \PCAPS, describe its operation, and discuss analytical results that characterize the trade-off between carbon savings and JCT. 

\noindent \textbf{\PCAPS design. }
We first formalize a class of \textit{probabilistic schedulers} that \PCAPS interfaces with, giving a concrete example of an ML scheduler in this class.

\vspace{-0.5em}
\begin{dfn}[Probabilistic Scheduler] \label{dfn:pb}
At each \textbf{scheduling event},\footnote{Scheduling events include job arrivals, task completions, and machines becoming available.} a probabilistic scheduler generates a distribution $\{ p_{v,t} : v \in \mathcal{A}_t\}$, where $\mathcal{A}_t$ denotes the set of tasks that are ready to be executed at time $t$.
\end{dfn}
\vspace{-0.5em}

\noindent One example of a probabilistic scheduler is Decima~\cite{Hongzi:2019:Decima}, an RL-based scheduler for data processing workloads.  
Decima learns actions in the form of scores for each task -- a masked softmax is applied to these scores to obtain a probability distribution over $\mathcal{A}_t$, and the next scheduled task is sampled from this distribution.
Recall the motivation behind \PCAPS: 
in addition to ramping down during high-carbon periods and ramping up during low-carbon periods, bottleneck tasks (i.e., tasks with a large score) should be scheduled even if the carbon intensity is high to reduce JCT.
To this end, we define a notion of \textit{relative importance} that compares the probability mass assigned to a single task $v$ against other tasks in $\mathcal{A}_t$.

\begin{figure}[t]
    \centering 
    \includegraphics[width=0.95\linewidth]{figures/pcapsfig.png} \vspace{-1em}
    \caption{\PCAPS interfaces with a probabilistic (\texttt{PB}) scheduling policy. Given a probability distribution over nodes \ding{202}, \PCAPS computes a \textit{relative importance} score \ding{203} that is used to determine which nodes should run based on the current carbon intensity \ding{204} -- e.g., bottleneck nodes impeding job completion run regardless of carbon \ding{205}, while less important nodes can be deferred for lower carbon periods \ding{206}.}
    \label{fig:danish-diagram} 
\end{figure}


 \vspace{-0.5em}
\begin{dfn}[Relative Importance] \label{dfn:rel-imp}
Given a time $t \geq 0$ and node $v \in \mathcal{A}_t$, the relative importance $r_{v,t}$ is defined:
{\small
\begin{equation*}
    r_{v,t} \coloneqq \frac{p_{v,t}}{\max_{u \in \mathcal{A}_t} p_{u,t}} \in [0,1].
\end{equation*}}
\end{dfn}
 \vspace{-0.5em}

\noindent If a task's relative importance is closer to $1$, the task is relatively \textit{more important}, and a value closer to $0$ implies the opposite.
Note that if $\vert \mathcal{A}_t \vert =1$ (i.e., only one task can be scheduled), the importance of that task is $1$. Leveraging inspiration from threshold-based design, we define scheduling decisions using a threshold function $\Psi_\gamma$ that considers the current carbon intensity and the relative importance of a task.
$\gamma \in [0,1]$ is a carbon-awareness parameter that controls the ``strictness'' of the function: $\gamma = 0$ recovers carbon-agnostic actions, while $\gamma = 1$ is maximally carbon-aware for tasks with low relative importance.  We define $\Psi_\gamma$ as:
{\small
\begin{equation*}
    \Psi_\gamma(r) \coloneqq \left( \gamma L + (1-\gamma) U \right) + \left[ U - \left( \gamma L + (1-\gamma) U \right)\right] \frac{\exp (\gamma r) -1 }{\exp (\gamma) -1}, \label{eq:Psi}
\end{equation*}
}
\noindent The function $\Psi_\gamma(.)$ exhibits an exponential dependence on $r$, the relative importance of a task.  This draws on continuous versions of online search~\cite{ElYaniv:01, Zhou:08}, where an exponential trade-off is derived by balancing the marginal reward of the current price against the risk that better prices exist in the future. 
We interpret relative importance analogously: high-importance tasks are scheduled regardless of carbon intensity to avoid the negative impact of not scheduling them, while low-importance tasks can be deferred to wait for lower-carbon periods with less impact on JCT. 
This threshold function is used in a \textit{carbon-awareness filter} of sampled tasks before they are scheduled-- we formalize this in \autoref{alg:danish}:











\begin{algorithm}[t]
	\caption{\PCAPS (Precedence- and Carbon-Aware Provisioning and Scheduling) }
	\label{alg:danish}
    {\small
	\begin{algorithmic}[1]
		\State \textbf{input:} hyperparameter $\gamma$, threshold function $\Psi_\gamma(\cdot)$, probabilistic (carbon-agnostic) scheduler \texttt{PB}
        \State \textbf{define:} a \textit{scheduling event} occurs whenever \texttt{PB} is invoked or the carbon intensity $c(t)$ changes
        \While{cluster active at time $t \ge 0$} 
        \If{scheduling event at time $t$} 
        \State Sample $v \in \mathcal{A}_t$ and probabilities $p_{v,t} : v \in \mathcal{A}_t$ from \texttt{PB}
        \State Compute relative importance $r_{v,t} = \frac{p_{v,t}}{\max_{u \in \mathcal{A}_t} p_{u,t}}$
        \If{$\Psi_\gamma(r_{v,t}) \geq c(t)$ \textbf{or} no machines currently busy}
        \State Send task $v$ to an available machine at time $t$
        \Else
        \State Idle until next scheduling event
        \EndIf
        \EndIf
        \EndWhile
	\end{algorithmic}
    }
\end{algorithm}	

\noindent
\PCAPS's carbon-awareness filter accomplishes all three of the motivation points defined in \autoref{sec:motiv}.
It schedules (or defers) tasks based on the current carbon intensity $c(t)$, with the effect of reducing execution during high-carbon periods.  Furthermore, the likelihood of a task being scheduled irrespective of the current carbon intensity is proportional to its importance in the DAG (i.e., in terms of precedence constraints).
Note that $\Psi_\gamma(1) = U$ -- tasks with high relative importance are always scheduled.  
While deferring a task has a negative impact on an individual job's JCT, \PCAPS is optimized for the case where multiple DAGs share a cluster.  Prioritizing outstanding bottleneck tasks across all jobs helps to manage the system's \textit{end-to-end completion time} (ECT) for e.g., a set of jobs.
In \autoref{fig:rel-imp}, we illustrate the intuitions behind \PCAPS's carbon-awareness filter using two sample job DAGs. 

\noindent\textbf{Analytical results. }
We analyze the \textit{carbon stretch factor} (\sref{Def.}{dfn:csf}) and \textit{carbon savings} (\sref{Def.}{dfn:carbonsavings}) for \PCAPS, with full proofs in \sref{Appendix}{apx:danish-proofs}.  $\texttt{PB}$ denotes a carbon-agnostic probabilistic scheduler throughout.
For an arbitrary job $\mathcal{J}$, we let $\mathcal{D}(\gamma, \mbf{c}) \in [0,1]$ denote a function that depends on the \textit{expected amount of deferrals}
(i.e., the tasks that $\PCAPS$ prevents from being scheduled, which depends on the carbon signal $\mbf{c}$).  See \sref{Appendix}{apx:danishMakespan} for a formal definition.
\begin{thm}\label{thm:danishMakespan}
    For time-varying carbon intensities given by $\mbf{c}$, the carbon stretch factor of \PCAPS is $1 + \frac{\mathcal{D}(\gamma, \mbf{c}) K}{2 - \frac{1}{K}}$.
\end{thm}
\noindent At a high level, $\mathcal{D}(\gamma, \mbf{c})$ describes the fraction of tasks (in terms of total runtime) that are deferred by \PCAPS with a given $\gamma$ and carbon trace $\mbf{c}$.  It is $\leq 1$ for any $\gamma$, and $\mathcal{D}(0, \mbf{c}) = 0$ for any $\mbf{c}$.  As $\gamma$ grows and \PCAPS becomes ``more carbon-aware'',  $\mathcal{D}(\gamma, \mbf{c})$ grows and CSF increases accordingly.


\begin{figure}[t]
    \centering 
    \includegraphics[width=\linewidth]{figures/rel-imp.png} \vspace{-2em}
    \caption{Illustrating \PCAPS's carbon-awareness filter.  Jobs A and B are  DAGs found in TPC-H queries and Alibaba traces, respectively~\cite{TPCH:18, Alibaba:18}.  Highlighted nodes explain two scheduling outcomes.  In job A, the sampled node has low relative importance, so it is deferred.  In contrast, job B's sampled node is a \textit{bottleneck} task with high relative importance: even when the current carbon intensity is high, such tasks are scheduled to avoid increasing job completion time.}
    \label{fig:rel-imp} \vspace{-1em}
\end{figure}







We also analyze the \textit{carbon savings} of \PCAPS.  
Suppose \texttt{PB}'s schedule finishes at time $T$, and \PCAPS's finishes at time $T'$ (where $T \leq T'$).   We let $W$ denote the \textit{excess work} that \PCAPS must ``make up'' with respect to \texttt{PB}'s schedule (i.e., due to deferrals) -- note that this implicitly depends on the carbon stretch factor.
We let $\overline{s}_{-}^{(0,T)}$, $\overline{s}_{+}^{(0,T)}$, and $\overline{c}^{(T, T')}$ denote weighted average carbon intensity values based on the schedules of \texttt{PB} and \PCAPS.  In short, $\overline{s}_{-}^{(0,T)}$ captures the carbon emissions that \PCAPS avoids due to deferrals between time $0$ and time $T$,  
$\overline{s}_{+}^{(0,T)}$ captures the extra carbon (if any) incurred by \PCAPS due to higher utilization relative to \texttt{PB} between time $0$ and time $T$, %
while $\overline{c}^{(T, T')}$ captures the emissions that \PCAPS incurs after time $T$.  
See \sref{Appendix}{apx:danishMakespan} for formal definitions of these quantities.
\begin{thm}\label{thm:danishCarbonSavings}
   For time-varying carbon intensities given by $\mbf{c}$, \PCAPS yields carbon savings of $W ( \overline{s}_{-}^{(0,T)} - \overline{s}_{+}^{(0,T)} - \overline{c}^{(T, T')} )$.
\end{thm}
\noindent Taken together, \autoref{thm:danishMakespan} and \ref{thm:danishCarbonSavings}  characterize 
the carbon-time trade-off for \PCAPS, implying that a larger CSF unlocks greater potential carbon savings.





\vspace{-0.5em}
\subsection{Carbon-aware provisioning (\CAP)}  \label{sec:cap-design}

While \PCAPS captures all three intuition points in \autoref{sec:theory} by interfacing with a probabilistic scheduler, many existing data processing schedulers use simple policies such as FIFO~\cite{SparkScheduling}.  This naturally prompts the question of how \PCAPS can be \textit{simplified} to retain many of the same qualities, while interoperating with \textit{any} scheduler.
In particular, \PCAPS implicitly performs \textit{resource provisioning}, changing the amount of resources available to the cluster based on carbon -- this is a key technique used by prior work in carbon-aware scheduling~\cite{radovanovic2022carbon, Hanafy:23:CarbonScaler}.  In this section, we introduce \CAP (Carbon-Aware Provisioning), a simplified policy that applies a time-varying resource quota to the cluster and coexists with any underlying scheduler.  In what follows, we motivate the design and discuss analytical results on its carbon-JCT trade-off.


\begin{figure}[t]
    \centering 
    \includegraphics[width=\linewidth]{figures/CAPfig.png}\vspace{-1em}
    \caption{The \CAP (Carbon-Aware Provisioning) module interacts directly with a \textit{cluster manager} %
    to specify the \textit{amount of resources} (e.g., no. of machines) that can be used at any given time, based on a \textit{carbon intensity signal}.  \CAP can be implemented without changes to an existing scheduling policy and/or the cluster manager.}
    \label{fig:cap-diagram}\vspace{-0.5cm}
\end{figure}


\noindent\textbf{\CAP design. } Given a cluster with $K$ machines, the possible resource quotas are given by $\{0, 1, \dots, K\}$. 
This set calls to mind the $k$-search problem~\cite{Lorenz:08}, where an online player must choose when to purchase $k$ items over a deadline-constrained sequence of time-varying prices. Variants of $k$-search have been applied to carbon-aware scheduling with deadlines~\cite{Lechowicz:23, Hanafy:23}.
\CAP uses the $k$-search threshold set, which captures the trade-off between executing now and waiting for better carbon intensities. 
Instead of using a deadline, \CAP frames the problem of determining a resource quota as \textit{repeated rounds} of $(K-B)$-search, where a minimum quota $B \in \{1, \dots, K\}$ always allows the cluster to use $\le B$ machines, ensuring continuous progress on jobs.  The thresholds are given by:
{\small
\begin{equation*}
\Phi_B = U; \ 
\Phi_{i+B} = U - \left(U - \frac{U}{\alpha} \right) \left( 1 + \frac{1}{\left(K-B\right)\alpha} \right)^{i-1} \kern-1em :  i \in \{1, \dots, K-B\},
\end{equation*}}
\noindent where $\alpha$ is the solution to $\left( 1 + \frac{1}{\left(K-B\right)\alpha} \right)^{\left(K-B\right)} \kern-1em = \frac{U-L}{U(1 - \frac{1}{\alpha})}$.  Each of these thresholds corresponds to a carbon intensity, and a quota is set based on how many values are \textit{above} the current carbon intensity.  Formally, the resource quota at time $t$ is $r(t) \gets \arg \max_{i\in \mathcal{R}} \Phi_i : \Phi_i \le c(t)$.  For ease of implementation, this quota is enforced without preemption; when machines become available, new task assignments are only allowed if $r(t)$ is greater than the number of busy machines.

\noindent\textbf{Analytical results. }
We analyze the \textit{carbon stretch factor} (\sref{Def.}{dfn:csf}) and \textit{carbon savings} (\sref{Def.}{dfn:carbonsavings}) for \CAP, with full proofs in \sref{Appendix}{apx:cap-proofs}.  $\texttt{AG}$ denotes a carbon-agnostic baseline scheduler throughout. Suppose \CAP's schedule completes at time $T'$.
We let $\mathcal{M}(B, \mbf{c})$ denote the minimum resource cap specified by \CAP at any point in its schedule (note this depends on the carbon signal $\mbf{c}$).  Formally, $\mathcal{M}(B, \mbf{c}) \coloneqq \arg \max_{i\in[K]} \Phi_i : \Phi_i \le c(t) \ \forall t \in [0, T']$.

\begin{thm}\label{thm:ksMakespan}
    For time-varying carbon intensities given by $\mbf{c}$, the carbon stretch factor of \CAP is $\left(\frac{K}{\mathcal{M}(B, \mbf{c})}\right)^2 \frac{2\mathcal{M}(B, \mbf{c})-1}{2K-1}$.
\end{thm}

\noindent We also analyze the \textit{carbon savings} of \CAP.  
If \texttt{AG}'s schedule finishes at time $T$ (where $T \leq T'$), we use $W$ as shorthand to denote the \textit{excess work} that \CAP must complete after time $T$ (i.e., after $\texttt{AG}$ has completed).  
As in \autoref{thm:danishCarbonSavings}, we let $\overline{s}^{(0,T)}$ and $\overline{c}^{(T, T')}$ denote weighted average carbon intensity values based on the schedules of \texttt{AG} and \CAP, respectively -- in short, $\overline{s}^{(0,T)}$ captures the carbon emissions that \CAP avoids by deferring $W$ amount of work relative to \texttt{AG}, while $\overline{c}^{(T, T')}$ captures the emissions that \CAP incurs after time $T$.  
See \sref{Appendix}{apx:ksCarbonSavings} for formal definitions of all three quantities.

\begin{thm}\label{thm:ksCarbonSavings}
   For time-varying carbon intensities given by $\mbf{c}$, \CAP yields carbon savings of $W ( \overline{s}^{(0,T)} - \overline{c}^{(T, T')} )$.
\end{thm}

\noindent \autoref{thm:ksMakespan} and \ref{thm:ksCarbonSavings} imply that a larger CSF unlocks greater carbon savings for \CAP.  We explore the relative performance of \PCAPS vs. \CAP in our experiments, in \autoref{sec:eval}.




















