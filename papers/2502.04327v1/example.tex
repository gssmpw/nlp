





















\documentclass[11pt, letterpaper, shortlabels]{archer}


\usepackage{microtype}
\usepackage{graphicx}
\usepackage{xcolor}
\usepackage{pgf}
\usepackage{booktabs}
\usepackage{subcaption}
\usepackage{booktabs}
\usepackage{xcolor}         %
\usepackage{color}         %
\usepackage{algorithm}
\usepackage{algorithmicx}
\usepackage{algpseudocode}
\usepackage{multirow}
\usepackage{subcaption}
\usepackage{resizegather}
\usepackage{hyperref}
\usepackage{color-edits}
\addauthor{yifei}{blue}
\title{\ourmethodnospace~(\ouracronymnospace): \\Digi-Q}




\usepackage{algorithm}
\usepackage{amsmath}
\usepackage{amssymb}
\usepackage{mathtools}
\usepackage{amsthm}
\newcommand\numberthis{\addtocounter{equation}{1}\tag{\theequation}}
\ifx\assumption\undefined
\newtheorem{assumption}{Assumption}
\fi

\usepackage[capitalize,noabbrev]{cleveref}


\usepackage[textsize=tiny]{todonotes}
\usepackage{wrapfig}
\captionsetup[figure]{font=small,skip=0pt}
\setlength{\belowcaptionskip}{0pt}








\newcommand{\yifei}[1]{{\textcolor{blue}{[Yifei: #1]}}}
\newcommand{\qianlan}[1]{{\textcolor{green}{[Qianlan: #1]}}}
\usepackage{multirow}
\newcommand{\ourmethod}{Digi-Q}
\newcommand{\ourmethodnospace}{Proposer-Agent-Evaluator}
\newcommand{\ouracronym}{PAE }
\newcommand{\ouracronymnospace}{PAE}
\newcommand{\argmax}{\arg \max}

\usepackage[all]{hypcap}

\usepackage[authoryear, round]{natbib}

\usepackage{hyperref}[citecolor=magenta,linkcolor=magenta]

\hypersetup{
    colorlinks = true,
    citecolor = {magenta},
}

\usepackage{microtype}
\usepackage{graphicx}
\usepackage{booktabs} %
\usepackage{float}

\usepackage{amsmath}
\usepackage{amssymb}
\usepackage{mathtools}
\usepackage{amsthm}
\usepackage{mathrsfs}
\usepackage{nicefrac}
\usepackage{dsfont}
\usepackage{enumitem}
\usepackage{float}
\usepackage{enumitem}
\usepackage{comment}
\usepackage{etoolbox}
\usepackage{ifthen}
\usepackage{mathrsfs}
\usepackage{upquote}
\usepackage{graphicx}
\usepackage{caption}
\usepackage{subcaption}
\usepackage{algorithm}
\usepackage{algpseudocode}
\usepackage{arydshln}
\usepackage{longtable}
\usepackage{hyperref}
\usepackage{url}
\usepackage{graphicx}
\usepackage{booktabs}
\usepackage{adjustbox}
\usepackage{amsmath}
\usepackage{dsfont}
\usepackage{multirow}
\usepackage{mdframed}
\usepackage{xcolor}
\usepackage{blindtext}
\usepackage{setspace}
\usepackage{xcolor,colortbl}
\definecolor{Gray}{gray}{0.90}
\definecolor{LightCyan}{rgb}{0.88,1,1}
\usepackage{multirow}
\usepackage{wrapfig}

\setlength\parindent{0pt}


\usepackage{xspace}
\usepackage[capitalize,noabbrev]{cleveref}
\bibliographystyle{plainnat}
\usepackage{subcaption}
\usepackage{wrapfig}
\usepackage{lipsum}
\usepackage{listings}

\usepackage{amsmath}
\usepackage{amssymb}
\usepackage{mathtools}
\usepackage{amsthm}
\usepackage{bbm}

\usepackage{algpseudocode}
\usepackage{setspace}

\usepackage{color}
\definecolor{deepblue}{rgb}{0,0,0.5}
\definecolor{deepred}{rgb}{0.6,0,0}
\definecolor{deepgreen}{rgb}{0,0.5,0}


\newcommand\pythonstyle{\lstset{
basicstyle=\ttfamily\footnotesize,
language=Python,
morekeywords={self, clip, exp, mse_loss, uniform_sample, concatenate, logsumexp},              %
keywordstyle=\color{deepblue},
emph={MyClass,__init__},          %
emphstyle=\color{deepred},    %
stringstyle=\color{deepgreen},
frame=single,                         %
showstringspaces=false
}}

\lstnewenvironment{python}[1][]
{
\pythonstyle
\lstset{#1}
}
{}

\DeclareRobustCommand{\StartCrate}{%
  \begingroup\normalfont
  \raisebox{-0.3ex}{\smash{\includegraphics[height=2.0\fontcharht\font`\B]{figures/icon.png}}}%
  \endgroup
}

\newcommand\pythonexternal[2][]{{
\pythonstyle
\lstinputlisting[#1]{#2}}}

\newcommand\pythoninline[1]{{\pythonstyle\lstinline!#1!}}

\newcommand{\Dcal}{\mathcal{D}}
\newcommand{\EE}{\mathbb{E}}


\makeatletter
\def\mathcolor#1#{\@mathcolor{#1}}
\def\@mathcolor#1#2#3{%
  \protect\leavevmode
  \begingroup
    \color#1{#2}#3%
  \endgroup
}
\makeatother



\Crefformat{equation}{#2Eq.\;(#1)#3}

\Crefformat{figure}{#2Figure #1#3}
\Crefformat{assumption}{#2Assumption #1#3}
\Crefname{assumption}{Assumption}{Assumptions}

\usepackage{crossreftools}
\pdfstringdefDisableCommands{%
    \let\Cref\crtCref
    \let\cref\crtcref
}
\newcommand{\creftitle}[1]{\crtcref{#1}}

\usepackage{dsfont}
\usepackage{nicefrac}

\author[1,2\textbf{*}]{Hao Bai}
\author[1\textbf{*}]{Yifei Zhou}
\author[3]{Erran Li}
\author[1]{Sergey Levine}
\author[4]{Aviral Kumar}

\affil[*]{Equal contributions}
\affil[1]{UC Berkeley}
\affil[2]{UIUC}
\affil[3]{Amazon}
\affil[4]{CMU}

\correspondingauthor{haob2@illinois.edu, yifei\_zhou@berkeley.edu, aviralku@andrew.cmu.edu}



\title{\StartCrate{} Digi-Q: Learning VLM Q-Value Functions for Training Device-Control Agents}



\begin{abstract}
\textbf{Abstract:} While most paradigms for building foundation model agents rely on prompting or fine-tuning on demonstrations, it is not sufficient in dynamic environments (e.g., mobile device control). On-policy reinforcement learning (RL) should address these limitations, but collecting actual rollouts in an environment is often undesirable when using truly open-ended agentic tasks such as mobile device, where simulation is a bottleneck. In such scenarios, an offline method for policy improvement that utilizes a trained value-function for training the policy is much more practical. In this paper, we develop a scalable value-based offline RL approach called \ourmethod{} to train VLM agents for device control entirely from static data. The key idea in \ourmethod{} is to train a value function using offline temporal-difference (TD) learning. We show that this can be done by fine-tuning a Q-function on top of frozen, intermediate-layer features of a VLM rather than fine-tuning the whole VLM itself, which saves us compute and enhances training stability. To make the VLM features amenable for representing the value function, we need to employ an initial phase of fine-tuning to amplify coverage over actionable information critical for Q-functions. Once trained, we use this value function alongside a best-of-N policy extraction operator that imitates the best action out of multiple candidate actions from the current policy as ranked by the value function, enabling policy improvement without ever needing to use the simulator. \ourmethod{} outperforms several prior methods on  user-scale device control tasks in Android-in-the-Wild, attaining 9.9\% improvement 
over prior best-performing method.

\end{abstract}

\begin{document}

\maketitle

 \section{Introduction}\label{sec:intro}

In computational finance, Monte Carlo simulations are used extensively to estimate the expected value of financial payoffs based on the solution of stochastic differential equations (SDEs) which model the evolution of stock prices, interest rates, exchange rates and other quantities \cite{glasserman04}.  Monte Carlo methods are very general and flexible, but for high accuracy it requires generating a large number of costly SDE path approximations, which has motivated research into a number of variance reduction or, equivalently, cost reduction techniques. One such method is
Multilevel Monte Carlo (MLMC), which was proposed in \cite{GILES2008} and was adapted for various applications that are summarised in \cite{Giles_overview17} and successfully combined with other methods such as quasi-Monte Carlo methods. The main idea of MLMC is to approximate the payoff using different time stepping resolutions when numerically solving the underlying SDE and to generate an optimal number of samples on each level, such that the overall computational cost is minimised subject to the desired bound on the variance. %, such that the total computational cost is minimised. 
The computational savings come from the fact that most samples are computed on the coarser levels and hence are less expensive while only a few samples from the finest levels are required \cite{GILES2008}.


Among the directions in which the computational cost 
of MLMC methods could further be reduced, an important avenue is the use of lower precision calculations, especially for the first Monte Carlo levels where the targeted accuracy is relatively low. 
 An overview of the research on mixed precision for the standard Monte Carlo (MC) framework is provided in \cite{ChowMixedPrecisionStandardMC} but only a few references study the potential of low precision computation in the MLMC framework \cite{Rounding_error_oliver}. To the best of our knowledge, the only MLMC framework with customised precision in the literature is \cite{brugger2014mixed}, but they use a uniform precision for all operations on each Monte Carlo level instead of optimising 
 the precision of each intermediary variable to reduce as much as possible the cost of path generation.
 
An important motivation for an MLMC framework with variable precision would be performing the low precision computations on reconfigurable hardware devices such as Field Programmable Gate Arrays (FPGAs). FPGAs contain customizable logic blocks and connectors that make it easy to adapt the digital circuit architecture for a specific application, leading to a highly parallel and optimised implementation. Therefore they are successfully exploited in applications that require high speed and have high computational workload, such as signal processing \cite{woods2008fpga}, and real time applications like high frequency trading \cite{HFT1,HFT2}. That is why a number of previous works in hardware architecture design implemented the MLMC algorithm to price financial options using FPGAs as accelerators, which resulted in improved speed and power efficiency compared to full CPU architectures \cite{Schryver2013AMM}. The paper \cite{lindsey2016domain} also proposed 
a Domain Specific Language to automate the configuration of FPGAs for this specific application. However, only \cite{brugger2014mixed} proposed a heuristic to reduce the precision in calculations.

In addition, all aforementioned works considered that the random number generation (RNG) is performed in single or double precision. Yet in most cases an important portion of the workload in the overall MLMC simulation comes from the RNG and in \cite{brugger2014mixed} this limited the total computational savings.
To reduce the cost of MLMC simulations in particular those based on the Geometric Brownian Motion (GBM), \cite{approximateICDF_Oliver, NestedOliver} have proposed to use approximate random numbers that are generated by applying an approximation of the inverse CDF to uniform random numbers. In \cite{NestedOliver}, the authors proposed a way to integrate these lower precision random variables into a \textit{nested} MLMC framework and completed a numerical analysis to bound the resulting error at each MC level by a product of the time step and the error in the random number approximation. The same authors show in \cite{approximateICDF_Oliver} that using approximate random variables reduces the cost of path generation by a factor 7.


In this paper we propose a nested MLMC framework that combines the use of approximate random normal variables and lower precision calculations to reduce the computational cost of MLMC even further than \cite{brugger2014mixed,NestedOliver}. We illustrate the efficiency of our framework in Matlab, after making several assumptions on the cost of operations and size of the errors that we carefully justify. We focus on the case of GBM and use the approximate RNG methods presented in \cite{approximateICDF_Oliver} as well as a new slightly modified method that combines CDF inversion and the central limit theorem. To choose the precision of the variables in the low precision path generation, we introduce a novel method to optimise the bit-widths. This optimisation is performed before the main path generation loop is executed and is based on a linear model of the payoff error  
due to rounding when computing in low precision. The error model relies on algorithmic differentiation in a similar manner to \cite{unifying-bwoptim,bitwidth-AD,ADAPT}. The bit-width optimisation procedure can be performed off-line, so this stage can be excluded from the on-line time complexity of our framework. The user specified desired accuracy is then enforced by calculating on-line the number of samples that need to be generated.

In terms of hardware design, we suggest implementing the low precision path generation on FPGAs and the full-precision ones on a CPU or GPU. 
The FPGA offers enough flexibility to define a separate bit-width for every variable in the low precision path generation, and can be reconfigured periodically to update the bit-widths when the market parameters have changed considerably. 


The paper is organized as follows : \Cref{sec:MLMC} introduces MLMC and nested MLMC to make clear the estimator that is implemented in our framework. Then in \Cref{sec:RNG} we detail the methods that could be used to obtain approximate random normally distributed numbers very cheaply for the low precision path generation. In \Cref{sec:error_model} and \Cref{sec:costModel} we propose an error model and a cost model (resp.) that we then use to formulate the optimisation problem that is solved to obtain the optimal bit-widths of fixed point variables in \Cref{sec:optimisation}. Finally we summarise our results and future directions in \Cref{sec:conclusion}.



 \section{Related Work}

\subsection{First-order logic for natural entailment}

Since the start of the RTE challenge \citep{rte}, multiple works have attempted using FOL representations to solve natural language entailment. These methods first obtain the syntactic/semantic parse tree and apply a rule-based transformation to get the FOL representation \citep{bos-markert-2005-recognising, bos-nli}. However, it was repeatedly shown that these FOL representations are not empirically effective in solving natural language entailment. For instance, \citet{bos-nli} reported that FOL representations translated from the discourse representation structure (DRS) yield only 1.9\% recall in detecting the entailment in the single-premise RTE benchmark \citep{rte}.

Independently from these works, multi-premise logical entailment benchmarks \citep{tafjord-etal-2021-proofwriter, logicnli, folio} were developed to evaluate the reasoning ability of generative models. These benchmarks adopt the classic 3-way entailment label classification format (\textit{entailment, contradiction, neutral}) of single-premise RTE tasks, in which both the NL sentences and their gold FOL representations point to the same entailment label. 

Recent works have applied LLMs to obtain FOL representations for these multi-premise logical entailment tasks \citep{logiclm, linc, divide-and-translate}, fueled by the code generation ability of LLMs. While they achieve significant performance in synthetic, controlled logical reasoning benchmarks, whether they can generalize to natural entailment has remained unanswered. Furthermore, \citet{linc} observed that LLMs are highly susceptible to \textit{arbitrariness}, as they fail to produce coherent predicate names or numbers of arguments even when generating FOL representations of premises and hypotheses in a single inference.

\subsection{Executable semantic representations}

Apart from FOL, a stream of research focuses on the \textit{executability} of semantic representations. From this perspective, semantic representations are \textit{program codes} that can be executed to solve downstream tasks, such as query intent analysis \citep{spider, dligach-etal-2022-exploring} and question answering \citep{semparse-qa}. The performance of the semantic parser is directly assessed by the accuracy of execution results for the downstream tasks, rather than the similarity between the prediction and the reference parse.

To improve the execution accuracy that is often non-differentiable, reinforcement learning (RL) and its variants have been applied to train neural semantic parsers \citep{cheng-etal-2019-learning, cheng-lapata-2018-weakly}. Using only the input sentence and the desired execution result, these methods learn to maximize the probability of the representations that lead to the correct execution result. However, these approaches are not directly applicable to EPF, as EPF requires taking account of \textit{interactions between premises and hypotheses} during execution (\textit{i.e.} theorem proving) while these methods assume that sentences are isolated.


 

\section{Methodology}
\paragraph{Preliminaries.}
We primarily focus on the homologous model merging, in which $\boldsymbol{\theta}_i$ all come from the same base model $\boldsymbol{\theta}_{\rm{base}}$. Given $K$ tasks $\{T_1,T_2,\cdots,T_K\}$ and $K$ corresponding fine-tuned models with parameters $\{\boldsymbol{\theta}_1,\boldsymbol{\theta}_2,\cdots,\boldsymbol{\theta}_K\}$, model merging aims to combine $K$ fine-tuned models into one single model simultaneously performing on $\{T_1,T_2,\cdots,T_K\}$ without post-training~\cite{method_p1_1,method_p1_2}.
Task vector~\cite{ilharco2023editing,yang2024adamerging} is a key element in merging method which could enhances the base model‘s ability or enable the model to handle other tasks. Specifically, for task $T_i$, the task vector $\boldsymbol\tau_i\in \mathbb{R}^D$ is defined as the vector obtained by subtracting the SFT weights $\boldsymbol{\theta}_i$ from the base model weight
$\boldsymbol{\theta}_{\rm{base}}$, \emph{i.e.}, $\boldsymbol\tau_i=\boldsymbol{\theta}_i-\boldsymbol{\theta}_{\rm{base}}$. The merged model could be denoted as $\boldsymbol{\theta}_m=\boldsymbol{\theta}_{\rm{base}}+\sum_i \lambda_i\boldsymbol{\tau}_i$, which $\lambda_i$ is the scaling factor measuring the importance of task vector. For clarification, we also denote the neuron set in $\boldsymbol{\theta}_i$ as $\mathcal{N}_i$, the neuron set in $\boldsymbol{\tau}_i$ as $\mathcal{T}_i$.



\begin{algorithm}[!ht]
    \caption{LED-Merging}
    \label{alg1}
    \begin{algorithmic}[1]
        \REQUIRE  base model $\boldsymbol{\theta}_{\rm{base}}$, SFT models $\{\boldsymbol{\theta}_{i}\mid i\in [K]\}$, mask ratios \{$r_{i} \mid i\in [K]\}$, scaling factors $\{\lambda_i\mid i\in[K]\}$, location datasets $\{\mathcal{X}_{i}\mid i\in[K]\}$
        \ENSURE merged parameter $\boldsymbol{\theta}_{m}$
        \STATE $\mathcal{M}\leftarrow\phi$
        \STATE $\boldsymbol{\theta}_{m}\leftarrow \boldsymbol{\theta}_{\rm{base}}$
        \FOR{$i\in [K]$}
        \STATE $I(\boldsymbol{\theta}_i)=\mathbb{E}_{x\sim \mathcal{X}_i}|\boldsymbol{\theta}_{i}\odot \nabla_{\boldsymbol{\theta}_i}\mathcal{L}(x)|$
        \STATE $I(\boldsymbol{\theta}_{\rm{base}})=\mathbb{E}_{x\sim \mathcal{X}_i}|\boldsymbol{\theta}_{\rm{base}}\odot \nabla_{\boldsymbol{\theta}_{\rm{base}}}\mathcal{L}(x)|$
        
        \STATE calculate $\mathcal{T}^{r_i}_{i}$ following Equation \ref{vote}
        \STATE  $\mathcal{M}\leftarrow \mathcal{M}\cup\{\mathcal{T}^{r_i}_i\}$
       
        
   
        
        
        \ENDFOR  
        \FOR{$i\in [K]$}
        
        \STATE calculate $\text{Disjoint}(\mathcal{T}_i^{r_i})$ use Equation~\ref{disjoint_safety}
        \STATE $\boldsymbol{m}_i \leftarrow \boldsymbol{0}$
        \FOR{$d\in \mathcal{T}_i^{r_i}$}
        \STATE $\boldsymbol{m}_{i,d}=1$
        \ENDFOR
        \STATE $\boldsymbol{\theta}_{m}\leftarrow \boldsymbol{\theta}_{m}+\lambda_i \boldsymbol{\tau}_i\odot \boldsymbol{m}_{i}$
        \ENDFOR
    \end{algorithmic}
\end{algorithm}
    %\vspace{-5pt}
\begin{figure*}[h!]
    \centering
    \includegraphics[width=\linewidth]{figs/pipeline_v2.pdf}
    \vspace{-40mm}
    \caption{Overview of our two-stage training pipeline {\ours}.}
    \label{fig:pipeline}
\end{figure*}


\paragraph{LED-Merging: Location, Election, and Disjoint Merging}
To address the neuron misidentification and interference issues in existing model merging methods, we propose LED-Merging (Location, Election, and Disjoint Merging). Specifically, previous studies \cite{modelstock, ilharco2023editing, tiesmerging} fail to accurately identify safety-related neurons in task vectors with a single magnitude score, namely \textit{neuron misidentification}. Meanwhile, there exists an interference between safety-related and utility-related task vector neurons during the merging process, namely \textit{neuron interference}. To address neuron misidentification, we first locate important neurons both in the base and fine-tuned models and then elect neurons from the task vector considering these two scores together. Subsequently, to mitigate the interference, we introduce a disjoint step, isolating these important neurons so that they influence different base neurons. The whole process is illustrated in Figure~\ref{fig:method}. 




In the location and election step, we consider the importance score from base and fine-tuned models simultaneously to locate task-specific neurons. In this way, it is more accurate than relying on the magnitude score alone because task-specific neurons with high importance score in the fine-tuned model may not necessarily score high in the base model, and vice versa.

{\textbf{Location}}.  We first calculate importance scores for each neuron in a base/fine-tuned model. Given a location dataset $\mathcal{X}_i=\{(x,y)_k\}$, where $x$ is the question and $y$ is the answer, we calculate the importance scores for the weight $\boldsymbol{\theta}_i\in\mathbb{R}^D$ in any  layer as follows~\cite{snip,spareseGPT,sun2024a}:
\begin{equation}
    I(\boldsymbol{\theta}_i)=\mathbb{E}_{x\sim \mathcal{X}_i}[\boldsymbol{\theta}_i\odot \nabla _{\boldsymbol{\theta}_i}\mathcal{L}(x)],
    \label{location}
\end{equation}
which $\mathcal{L}(x)=-\log p(y\mid x)$ is the conditional negative log-likelihood loss. We choose the SNIP score~\cite{snip} because it balances computational efficiency and performance~\cite{cq}. Please refer to Sec.~\ref{sec:ablation} for the comparison between different location methods. After computing importance scores, we choose top-$r_i$ neurons as the important neuron subset $\mathcal{N}_{i}^{r_i}$ from $I(\boldsymbol{\theta}_i)$.
 
 % After computing locating scores, we select the neurons scoring both high in base and fine-tuned models as important neurons in task vectors. Then in the disjoint step,  with preventing  polysemantic neurons  from receiving gradient updates towards different directions,
 % we use set difference to isolate the safety   and utility-related neurons  and construct corresponding masks for merging process,

{\textbf{Election}}. A natural question is how to select important neurons in the task vector $\boldsymbol{\tau}_i$ based on $I(\boldsymbol{\theta}_{\rm{base}})$ and $I(\boldsymbol{\theta}_{i})$. The important neurons in the base model may be different from neurons in the fine-tuned model. Therefore, we introduce the following election strategy to select neurons with high scores in both base and fine-tuned models:
\begin{equation}
    \mathcal{T}_i^{r_i}=\mathcal{N}_i^{r_i}\cap \mathcal{N}_{\rm{base}}^{r_i}.
    \label{vote}
\end{equation}
\emph{Remark}. We compare different choosing methods, including scoring low or high in base or fine-tuned model in Section~\ref{sec:ablation} and find that Equation \ref{vote} achieves the best performance.





{\textbf{Disjoint}}. As important neurons from different task vectors may conflict with each other at the same position, we use the set difference to disjoint the neurons from others to prevent interference:
\begin{equation}
    \text{Disjoint}(\mathcal{T}^{r_i}_{i})=\mathcal{T}^{r_i}_{i}-\mathop{\cup}\limits_{{J}\subsetneqq [K],|J|\geq 2}\mathop{\cap}\limits_{j\in {J}}\mathcal{T}^{r_j}_{j}.
    \label{disjoint_safety}
\end{equation}

Next, we construct a mask $\boldsymbol{m}_i\in\mathbb{R}^D$ to implement disjoint in the merging process. Specifically, this mask $\boldsymbol{m}_i$ is used to select neurons from $\mathcal{T}_i$. The mask ratio is $r_i$, where $r\in(0,1]$. The mask $\boldsymbol{m}_i$ can be derived from:
\begin{equation}
    \boldsymbol{m}_{i,d}=\begin{aligned} &\left\{ \begin{array}{ll} 1, & \text{if } d\in \text{Disjoint}(\mathcal{T}_{i}^{r_i}), \\ 0, & \text{otherwise}. \end{array} \right. \end{aligned}
    \label{mask_safety}
\end{equation}


% \subsection{Merging Models with Masks}
{\textbf{Merging}}. The final
merged task vector $\boldsymbol{\tau}_m$ is as follows:
\begin{equation}
    \boldsymbol{\tau}_m= \sum_i \lambda_i\boldsymbol{\tau}_{i}\odot\boldsymbol{m}_i.
    \label{merged_task_vector}
\end{equation}
We summarize the workflow in Algorithm \ref{alg1}.



 \section{Experiments}
\label{sec:experiments}

\begin{figure*}[t]
\vspace{-6mm}
    \centering
    \includegraphics[width=0.8\linewidth]{figs/compare.pdf}
    \vspace{-4mm}
    \caption{\textbf{Qualitative comparison} with the baseline for generating a sequence of novel view images.  
    The results demonstrate that our method synthesizes more consistent multi-view images compared to our baseline model (Zero123). In addition, compared to SyncDreamer, our method visually maintains better similarity to the conditioned image and appears more natural.}
    \label{fig:sota_compare}
\vspace{-5mm}
\end{figure*}

\subsection{Experimental Setups}
\textbf{Dataset.}
Following previous work~\cite{zero123, SyncDreamer}, we evaluate our work on the Google Scanned Object (GSO)~\cite{GSO} dataset to verify the zero-shot novel view image synthesis capability. 
We also provide results for additional datasets in the Supplementary Material.
Specifically, we randomly select 30 objects from the GSO dataset with various object categories. 
Unlike recent approaches~\cite{mvdream, SyncDreamer} that aim to enhance the consistency of novel view synthesis models by generating multiple fixed-view images, our method can generate images from any camera pose and any number of views. Therefore, we conduct experiments under different camera pose settings to validate our approach:
specifically, 
1) \textit{16-views with free camera pose}: for each object, we circularly render 16 views with the elevation angles ranging in $[-10\degree, 40\degree]$ and the azimuth angles are evenly distributed in $[0\degree, 360\degree]$. 
2) \textit{16-views with fixed camera pose}: We maintain a constant elevation angle of $30\degree$ and uniformly sample azimuth angles (same as SyncDreamer~\cite{SyncDreamer}).
3) \textit{32-views with free camera pose}: Similar to the first setting, but we sample 32 views.
It's important to note that our method does not require additional training or fine-tuning on any datasets.

\noindent\textbf{Metrics.}
To validate the effectiveness of our method, we mainly evaluate it based on three criteria:
1) \textit{Quality Score}. We evaluate the image quality of synthesized multi-view images by measuring their similarity with ground truth images. Following prior research~\cite{zero123, sparsefusion}, we report the similarity between the synthesized images and the ground truth images with standard metrics: PSNR, SSIM~\cite{ssim}, and LPIPS~\cite{lpips}.
2) \textit{Multi-view Consistency Score}. As the primary goal of our work is to improve the consistency of generated images, we also employ the 3D consistency score~\cite{3dim} to verify the consistency among the synthesized images. Specifically, we train an Instant-NGP~\cite{instant_ngp} with the input image and part of the synthesized novel view images of our model and evaluate the similarity between the remaining synthesized images and the rendered images of Instant-NGP. For the synthesized multi-view images of each object, we allocate $3/4$ for training and reserve the remaining $1/4$ for validation.
Intuitively, if the consistency of synthesized images is improved, the NeRF-like model will train a better object representation, and the re-rendered images will agree more with the validation images.
3) \textit{Input Consistency Score}. To assess the faithfulness of synthesized images in preserving the identity of the input condition image, we introduce the input consistency score. This score calculates the similarity of each synthesized image with the input condition image, utilizing the LPIPS metric.

In addition, we use synthesized multi-view images to train a neural 3D reconstruction model (NeuS~\cite{neus}) and report commonly used Chamfer Distances (CD) and Volume IoUs between the trained 3D model and the ground truth.

\noindent\textbf{Baselines.}
Given that our main goal is to improve the consistency of the trained baseline model without further fine-tuning, we mainly compare our approach with the used baseline model Zero123~\cite{zero123}. Additionally, we compare our method to the SOTA approaches such as PGD~\cite{tseng2023consistent} and SyncDreamer~\cite{SyncDreamer} using the same Zero123 base model.

\noindent\textbf{Implementation Details.}
We use the official checkpoint provided by Zero123~\cite{zero123}, which is trained on objaverse~\cite{objaverse} for 165,000 steps. We inject our epipolar attention layer after step $T=4$ and layer $L=10$ by default. We find that feature fusion weight $\alpha=0.5$, and the number of context views $M=2$ work better.

\begin{table}[t]
\centering
\caption{Comparison of multi-view consistency, image quality, and input consistency of synthesized multi-view images at the 16-view setting with free camera pose.}
\label{tab:view16_free_compare}
\vspace{-2mm}
\scalebox{0.6}{
\begin{tabular}{c ccc ccc c}
\toprule
              & \multicolumn{3}{c}{Multi-view Consistency} & \multicolumn{3}{c}{Quality Score} & \multicolumn{1}{c}{Input Consis.} \\
              \cmidrule(lr){2-4} \cmidrule(lr){5-7} \cmidrule(lr){8-8}
              & PSNR$\uparrow$  & SSIM$\uparrow$ & LPIPS$\downarrow$ 
              & PSNR$\uparrow$  & SSIM$\uparrow$ & LPIPS$\downarrow$ 
              & LPIPS$\downarrow$ 
              \\ \midrule

Zero123
& 15.225        & 0.645       & 0.408
& 14.255        & 0.747       &	0.208
& 0.303         
\\
SyncDreamer
& 14.830        & 0.626       & 0.434
& 12.650        & 0.713       &	0.254
& 0.317         
\\
Ours 
& \best{18.300}	& \best{0.734}	& \best{0.355}
& \best{14.947}	& \best{0.763}	& \best{0.191}
& \best{0.282}
\\

\bottomrule
\end{tabular}
}
\end{table}

\begin{table}[t]
\vspace{-1mm}
\centering
\caption{Comparison of multi-view consistency, image quality, and input consistency at the 16-view setting with fixed camera pose as SyncDreamer~\cite{SyncDreamer}.}
\label{tab:view16_fxied_compare}
\vspace{-3mm}
\scalebox{0.6}{
\begin{tabular}{c ccc ccc c}
\toprule
              & \multicolumn{3}{c}{Multi-view Consistency} & \multicolumn{3}{c}{Quality Score} & \multicolumn{1}{c}{Input Consis.} \\
              \cmidrule(lr){2-4} \cmidrule(lr){5-7} \cmidrule(lr){8-8}
              & PSNR$\uparrow$  & SSIM$\uparrow$ & LPIPS$\downarrow$ 
              & PSNR$\uparrow$  & SSIM$\uparrow$ & LPIPS$\downarrow$ 
              & LPIPS$\downarrow$ 
              \\ \midrule

Zero123
& 16.556        & 0.682       & 0.378
& 14.592        & 0.750       &	0.207
& 0.305         
\\
SyncDreamer
& \best{22.424}        & \best{0.812}       & \best{0.268}
& 15.269        & 0.749       &	0.196
& 0.300         
\\
Ours 
& 21.151	& 0.780	& 0.302
& \best{15.293}	& \best{0.764}	& \best{0.184}
& \best{0.287}
\\

\bottomrule
\end{tabular}
}
\vspace{-4mm}
\end{table}


\subsection{Comparison With Baseline Models}
The quantitative comparison on three settings are shown in Tab.~\ref{tab:view16_free_compare}, Tab.~\ref{tab:view16_fxied_compare}, and Tab.~\ref{tab:view32_free_compare}. The qualitative comparison is shown in Fig.~\ref{fig:sota_compare}.

\begin{table}[t]
\centering
\caption{Comparison of multi-view consistency and image quality scores of synthesized multi-view images at the 32-view setting with free camera pose.}
\vspace{-3mm}
\label{tab:view32_free_compare}
\scalebox{0.7}{
\begin{tabular}{c ccc ccc}
\toprule
              & \multicolumn{3}{c}{Multi-view Consistency} & \multicolumn{3}{c}{Quality Score} \\
              \cmidrule(lr){2-4} \cmidrule(lr){5-7}
              & PSNR$\uparrow$  & SSIM$\uparrow$ & LPIPS$\downarrow$ 
              & PSNR$\uparrow$  & SSIM$\uparrow$ & LPIPS$\downarrow$ 
              \\ \midrule

Zero123
& 16.515        & 0.694       & 0.378
& 15.142        & 0.733       &	0.211
\\
PGD~\cite{tseng2023consistent}
& 18.481        & 0.720       & 0.343
& 15.281        & 0.739       &	0.205
\\
Ours 
& \best{20.655}	& \best{0.792}	& \best{0.305}
& \best{15.268}	& \best{0.742}	& \best{0.203}
\\

\bottomrule
\end{tabular}
}
\vspace{-3mm}
\end{table}

\begin{table*}
  [t]
  \centering
  \resizebox{\textwidth}{!}{%
  \begin{tabular}{cccccccccccc}
    \toprule \multicolumn{2}{c}{Components}                                                             & \multicolumn{5}{c}{Re-executability Rate (\%)} & \multicolumn{5}{c}{Readability (\#)} \\
    \cmidrule(lr){1-2} \cmidrule(lr){3-7} \cmidrule(lr){8-12}        \hspace{8pt}\labelemoji\hspace{8pt}                                                                & \hspace{8pt}\toolemoji\hspace{8pt}                                      & O0                                 & O1             & O2             & O3             & AVG            & O0             & O1             & O2             & O3             & AVG            \\
    \hline
    \rowcolor[rgb]{0.93,0.93,0.93}\multicolumn{12}{c}{\textbf{Initialize with LLM4Decompile-End-6.7B~\citep{llm4decompile}}}   \\
    \xmark                                                                                              & \xmark                                    & 69.51                              & 46.95          & 50.61          & 46.34          & 53.35          & 3.98 & 3.41 & 3.44 & 3.38 & 3.55 \\
    \cmark                                                                                              & \xmark                                    & 75.61                              & 50.61          & 50.00          & 50.00          & 56.55          & 4.01 & 3.44 & 3.39 & \textbf{3.49} & 3.58 \\
    \xmark                                                                                              & \cmark                                    & 83.54                     & \textbf{56.10}          & 51.22          & 50.61 & 60.37 & 4.05 & 3.51 & 3.51 & 3.42 & 3.62 \\
    \cmark                                                                                              & \cmark                                    & \textbf{85.37}                            & \textbf{56.10}                     & \textbf{51.83} & \textbf{52.43}          & \textbf{61.43} & \textbf{4.13} & \textbf{3.60} & \textbf{3.54} & \textbf{3.49} & \textbf{3.69} \\

    \rowcolor[rgb]{0.93,0.93,0.93}\multicolumn{12}{c}{\textbf{Initialize with Deepseek-Coder-6.7B-base~\citep{deepseekcoder}}} \\
    \xmark                                                                                              & \xmark                                    & 59.15                              & 35.98          & 39.02          & 37.80          & 42.99          & 3.71 & 3.05 & 3.16 & 3.05 & 3.24 \\
    \cmark                                                                                              & \xmark                                    & 66.46                              & 41.46          & 38.41          & 36.59          & 45.73          & 3.76 & 3.17 & \textbf{3.21} & 3.08 & 3.31 \\
    \xmark                                                                                              & \cmark                                    & 70.73                              & 39.63          & 39.02          & 40.24          & 47.41          & 3.90 & 3.17 & 3.08 & 3.11 & 3.31 \\
    \cmark                                                                                              & \cmark                                    & \textbf{79.88}                     & \textbf{45.73} & \textbf{43.90} & \textbf{42.68} & \textbf{53.05} & \textbf{3.96} & \textbf{3.21} & 3.18 & \textbf{3.19} & \textbf{3.38} \\
    \bottomrule
  \end{tabular}%
  }
  \caption{The ablation study of different methods across four optimization levels
  (O0, O1, O2, O3), as well as their average scores (AVG). The results in bold represent the optimal performance. The ~\labelemoji~ and ~\toolemoji~ means Relabedling and Function Call. \textbf{Bold} denotes the best performance.}
  \label{tab:ablation}
\end{table*}



\begin{figure*}[ht]
    \centering
    \begin{minipage}{0.65\textwidth}
        \centering
        \includegraphics[width=0.95\linewidth]{figs/ablation.pdf}
        \vspace{-2mm}
        \captionof{figure}{Qualitative Comparison for different design choices. Our method, employing multi-view epipolar attention, demonstrates the best consistency.}
        \label{fig:ablation}
    \end{minipage}\hfill
    \begin{minipage}{0.33\textwidth}
        \centering
        \includegraphics[width=0.8\linewidth]{figs/neus_ver.pdf}
        \vspace{-3mm}
        \caption{Our method shows better direct 3D reconstruction~\cite{neus}.}
        \label{fig:neus}
    \end{minipage}
    \vspace{-5mm}
\end{figure*}

\noindent\textbf{Multi-view Consistency.}
Tab.~\ref{tab:view16_fxied_compare} presents the 3D consistency scores compared to our baseline model (Zero123) and SyncDreamer. The results indicate a significant improvement across all three metrics achieved by our method when compared with Zero123.
While our method exhibits a marginally lower numerical consistency score compared to SyncDreamer, it enables the synthesis of images with arbitrary camera poses.	
This capability is illustrated in Tab.~\ref{tab:view16_free_compare}, where our method consistently enhances consistency with changes in camera pose settings, whereas SyncDreamer fails to do so and exhibits inferior results compared to Zero123.
Furthermore, our method facilitates the synthesis of multi-view images with any number of camera views. This versatility is demonstrated in Tab.~\ref{tab:view32_free_compare}, where our method continues to achieve significant improvements in consistency scores, while SyncDreamer is unable to operate under such conditions.	

Meanwhile, Fig.~\ref{fig:sota_compare} provides a qualitative comparison with the baseline. While both our method and SyncDreamer enhance consistency, our method visually preserves better similarity to the input image, including color and texture details. The input consistency score further corroborates this.

\noindent\textbf{Image Quality.}
While our primary goal centers around enhancing the consistency of synthesized multi-view images, we also evaluate the image quality by comparing the similarity with the ground truth images. The results shown in Tab.~\ref{tab:view16_free_compare}, Tab.~\ref{tab:view16_fxied_compare}, and Tab.~\ref{tab:view32_free_compare} indicate that our method also enhances the image quality under different settings besides improving the consistency.
Moreover, our method shows better image quality compared with SyncDreamer even in the 16-view setting with fixed camera pose.

\noindent\textbf{Input Consistency.}
Input consistency terms whether the results align with the input image.
Fig.~\ref{fig:sota_compare} illustrates that both our method and SyncDreamer enhance multi-view consistency. However, the color and texture details of SyncDreamer's results diverge from the input image and appear visually unnatural.
This discrepancy is evident in the input consistency score presented in Tab.~\ref{tab:view16_fxied_compare}, indicating lower similarity with the condition image in the SyncDreamer results.	

\subsection{Ablation Study}
The overall quantitative results are shown in Tab.~\ref{tab:ablation}, and the qualitative comparisons are shown in Fig.~\ref{fig:ablation}.

\noindent \textbf{Full Attention \vs Epipolar Attention.}
The results presented in Tab.\ref{tab:ablation} and Fig.\ref{fig:ablation} demonstrate that our epipolar attention mechanism can synthesize more consistent multi-view images compared with full attention. Furthermore, our epipolar attention achieves a greater performance improvement compared to full attention when using multiple reference images. This could be attributed to the fact that our epipolar attention more effectively localizes target information, as depicted in Fig.~\ref{fig:full_attn_compare}, thereby reducing noise from the reference images. In the multi-view setting, where multiple reference images are utilized, this noise reduction becomes particularly crucial.
Moreover, it is noteworthy that the epipolar attention mechanism consumes less GPU memory compared to our baseline, as discussed in Sec.~\ref{sec:attn_analysis}.

\noindent \textbf{Attending Single-View \vs Multi-View.}
Applying the epipolar attention significantly improves the consistency between the input and target views. However, the consistency between different views in the unobserved regions of the input view is not well preserved.
After implementing our epipolar attention in the multi-view setting, the consistency across the generated multi-view images is further improved. The last row in Tab.~\ref{tab:ablation} shows that after applying our multi-view epipolar attention, the consistency score is further improved compared with the single-view setting. Besides, the qualitative result in Fig.~\ref{fig:ablation} also shows better consistency among different target views.



\begin{table}[t]
\centering
\vspace{-1mm}
\caption{Comparison of 3D reconstruction results. Our method significantly improves the reconstruction quality.}
\vspace{-3mm}
\label{tab:neus}
\scalebox{0.7}{
\begin{tabular}{c cc}
\toprule
              &  Chamfer Dist.$\downarrow$  & Volume IoU$\uparrow$
\\ \midrule

            Zero123         & 0.017         & 0.819    \\
            SyncDreamer     & \best{0.013}         & \best{0.847}    \\
            Ours            & 0.014	& 0.842 \\

\bottomrule
\end{tabular}
}
\vspace{-5mm}
\end{table}


\vspace{-2mm}
\subsection{Downstream Application}
\vspace{-2mm}
To demonstrate the effectiveness of our method, we also applied it to the downstream 3D reconstruction task. Specifically, we trained the NeuS model~\cite{neus} directly using images synthesized by our method, Zero123, and SyncDreamer, respectively.
The quantitative results in Tab.~\ref{tab:neus} show that the consistent multi-view images synthesized by our method can significantly improve the 3D reconstruction quality.
Additionally, our method exhibits similar performance to SyncDreamer which requires time-consuming re-training.
The qualitative results in Fig.~\ref{fig:neus} show that it is challenging to train the NeuS model directly due to the lack of consistency in the images generated by Zero123. In contrast, our method generates more consistent multi-view images and, therefore, better reconstructs the geometry and texture details.
We show improvements on other downstream applications such as image-to-3D in the Supplementary Material.


 \section{Conclusion}

%In this paper, w
We propose a new PEFT method called DiffoRA, which enables efficient and adaptive LLM fine-tuning based on LoRA. 
Instead of adjusting every interior rank, 
%of the decomposition matrices 
%of all modules, 
we argue that adopting LoRA module-wisely is sufficient. 
To achieve this, we construct a DAM to select the modules that are most suitable and essential to fine-tune. We theoretically analyze how the DAM impacts the convergence rate and generalization capability.
%of the pre-trained model. 
Furthermore, we adopt continuous relaxation and discretization to establish DAM.
%for each task. 
To alleviate the issue of discretization discrepancy, we utilize the weight-sharing strategy for optimization. 
%We fully implement our method and t
The experimental results demonstrate that our DiffoRA works consistently better than the baselines across all benchmarks. 
 \section*{Acknowledgements}
This is acknowledgment.






\bibliography{neurips2024}

\newpage

\appendix
\onecolumn
\part*{Appendices}

\section{Details on the Algorithm} \label{app:algorithm}

For completeness, we include a detailed pseudo-code of \ourmethod{} in~\Cref{alg:archer_detail}. After initializing the parameters, we perform the representation fine-tuning procedure on top of VLM to obtain actionable features for later TD-learning. Then the VLM parameters will be kept frozen and we train the Q- and V- functions using TD-learning on top of frozen VLM representations. After both value functions are trained, we perform gradient updates on the actor with Best-of-N policy extraction.

\definecolor{darkgreen}{rgb}{0, 0.5, 0}
\begin{algorithm}[!htp]
\caption{\ourmethod{}: Practical Framework}
\label{alg:archer_detail}
\begin{algorithmic}[1]
\State Initialize parameters $\phi, \psi_\mathrm{MLP}, \bar{\psi}_\mathrm{MLP}, \theta_\mathrm{MLP}, \bar{\theta}_\mathrm{MLP}$.
% \State Import pretrained parameters $\theta_{\mathrm{VLM}}$.
\State Initialize replay buffer $\mathcal{D}$ (from an offline dataset).
\For{each VLM iteration}
\State $\theta_\mathrm{VLM} \leftarrow \nabla J_\mathcal P(\theta_\mathrm{VLM})$
\Comment{Equation~\ref{equation:vlm}}
\EndFor
% \State Get representation $f_{\theta_{\mathrm{VLM}}}(s_t,a_t)$ for all $(s_t, a_t)$ in buffer $\mathcal D$.
% \For{each local iteration}
% \State \textcolor{darkgreen}{\#\# Data Collection.}\Comment[only online mode] 
% \For{each environment step}
% \State Execute $a_t \sim \pi_\phi(\cdot|s_t)$ , obtain the next state $s_{t+1}$, add to buffer $\Dcal$.
% \EndFor
\For{each critic step}
\State \textcolor{darkgreen}{\#\# Update high-level Q and V functions by target function bootstrapping.}
\State $\theta_\mathrm{MLP} \leftarrow \theta_\mathrm{MLP} - \nabla J_{\theta_\mathrm{MLP}}(Q)$ \Comment{Equation~\ref{equation: JQ_MLP}}
\State $\psi_\mathrm{MLP} \leftarrow \psi_\mathrm{MLP} - \nabla J_{\psi_\mathrm{MLP}}(V)$ \Comment{Equation~\ref{equation: JV_MLP}}
\State \textcolor{darkgreen}{\#\# Update target Q and V functions.}
\State $\bar{\theta}_\mathrm{MLP} \leftarrow (1 - \tau)\bar{\theta}_\mathrm{MLP} + \tau\theta_\mathrm{MLP}$
\State $\bar{\psi}_\mathrm{MLP} \leftarrow (1 - \tau)\bar{\psi}_\mathrm{MLP} + \tau\psi_\mathrm{MLP}$
\EndFor
\State \textcolor{darkgreen}{\#\# Update low-level actor with high-level critic.}
\For{each actor step}
\State $\phi \leftarrow \phi - \nabla J_\phi(\pi)$ \Comment{Equation~\ref{equation:best-of-n}}
% \EndFor
\EndFor
\end{algorithmic}
\end{algorithm}

\section{Experimental Details}

\vspace{-0.2cm}
\subsection{Compute Efficiency Comparison}\label{app:compute_efficiency}
\vspace{-0.2cm}

A common concern with deploying TD-learning methods to train large-scale foundation models is their compute inefficiency~\citep{abdulhai2023lmrlgymbenchmarksmultiturn,chebotar2023qtransformerscalableofflinereinforcement}. Therefore, we attempted to understand the compute-performance tradeoffs associated with \ourmethod{}  by comparing it against end-to-end TD-learning on VLMs without using any representation fine-tuning or frozen pre-trained representations.  We plot the performance-compute tradeoff curve for \ourmethod{} on the web-shopping subset of the AitW dataset in~\Cref{fig:flops}. We found it a bit hard to fine-tune an entire VLM with TD-learning, which required iteration on hyperparameters such as learning rate and soft update rates for target networks. Due to the compute-intensive nature, we use a 3B VLM (PaLiGemma~\citep{beyer2024paligemmaversatile3bvlm}) for these runs instead of our 7B VLM~\citep{liu2024llavanext}, and evaluate the performance of the critic as measured by the correlation between advantage predictions and ground-truth notion of human judgement on a held-out set of trajectories.  
In particular, we find that end-to-end TD-learning exhibits a much worse performance-compute frontier, to the extent that beyond a point more training FLOPS hurts performance. We conjecture that this behavior is likely a result of well-known pathologies of training large models with TD learning~\citep{kumar2022offline}, though we leave it for future work to fully understand these pathologies in our context. In contrast, while \ourmethod{} invests an initial amount of computation for representation fine-tuning, its accuracy quickly rises up and results in much better frontiers, with no instability. The calculation of the FLOPS is shown below.

% \begin{figure}[!htp]
%      \centering
%      \begin{subfigure}[b]{0.45\textwidth}
%          \centering
%     \includegraphics[width=\linewidth]{figures/ablation_flops.pdf}
%      \end{subfigure}
%         \caption{\textbf{Offline critic evaluation accuracy as a function of compute} measured in terms of training FLOPS, compared across \ourmethod{}, end-to-end TD-learning on a VLM, and MC return. Observe that the critic accuracy is much better for our approach over end-to-end TD-learning as the amount of compute increases.}
%         \label{fig:ablation-n-actions-2curves}
%         \vspace{-0.2cm}
% \end{figure}

\begin{wrapfigure}{r}{0.45\textwidth}
  \centering
  \includegraphics[width=\linewidth]{figures/ablation_flops.pdf}
  \caption{\textbf{Offline critic evaluation accuracy as a function of compute} measured in terms of training FLOPS, compared across \ourmethod{}, end-to-end TD-learning on a VLM, and MC return. Observe that the critic accuracy is much better for our approach over end-to-end TD-learning as the amount of compute increases.}
  \label{fig:flops}
  \vspace{-0.2cm}
\end{wrapfigure}

\textbf{FLOPS Calculation.} The 3B VLM takes $45.6\times10^{12}$ FLOPS for \textit{each sample} for forward plus backward process. As the end-to-end TD learning contains one VLM as part of the Q function and one VLM as the target Q function (which only do forward pass), one sample takes $68.4\times10^{12}$ FLOPS (according to \citet{hoffmann2022training}, the FLOPS incurred by the forward prrcess is approximately half of the backward process). Thus, as the longest run takes 15k samples, the last point of the end-to-end run in~\Cref{fig:flops} takes around $1\times10^{18}$ FLOPS. Also, the first logged point takes 128 samples, so the starting point should have $8.3\times10^{15}$ FLOPS.

On the other hand, in \ourmethod{}, we first finetune the 3B VLM, which incurs only one forward and backward process. Thus, finetuning the 3B VLM on $2000$ samples takes $91.2\times 10^{15}$ FLOPS. After that, we infer the representations of these samples with the 3B VLM, which includes one forward pass. This sums up to $136.8\times10^{15}$ FLOPs, which explains the starting point of the \ourmethod{} curve. Then we only train the value head using the VLM representations.\footnote{In this experiment, we fix the BERT model when running Digi-Q.} The size of the value head is 0.07B, incurring $1.1\times10^{12}$ FLOPS for each sample. The longest run of \ourmethod{} takes 0.46M samples, thus incurring $506.9\times 10^{15}$ FLOPS ($10\times 10^{17}$).

Thus, the end-to-end TD learning should range from $0.0083\times10^{15}$ to $1\times10^{18}$ FLOPS, while \ourmethod{} should range from $0.137\times10^{18}$ FLOPS to $0.644\times10^{18}$ FLOPS, which is shown in~\Cref{fig:flops}.




\textbf{Critic Accuracy.} We manually label 483 states with binary advantages, and normalize the advantages produced by the agents to have a mean of zero before thresholding and calculating its accuracy with human annotations.

\subsection{Critic Model Architecture} \label{app: arch}

We show the details of the critic model architecture in~\Cref{fig:arch}. In our environment setting, the states are composed of task, observation (screenshot at step $t$), previous observation (screenshot at step $t-1$), and previous action (action at timestep $t-1$). The task and previous action are text strings, while observations are images. We encode the text strings with BERT and images with BLIP-2 model. Then we concatenate all these feature vectors and pass them through a MLP that tries to predict the V value. The target of the V value is calculated by Equation~\ref{equation: JV_MLP}.

The state-action features are modeled by the current action as well, which is a string passed into not only the BERT encoder but also a part of the prompt passed into the VLM. The prompt is described in~\Cref{app:vlm-prompts}. In the end, the Q features include the BERT embeddings, the BLIP-2 embeddings, and the VLM intermediate-layer representations. We concatenate all of these feature vectors and pass into the another MLP that predicts the Q value. The target of Q value is calculated by Equation \ref{equation: JQ_MLP}.

\begin{figure}[!t]
     \centering
    \begin{subfigure}[b]{1.0\textwidth}
         \centering
    \includegraphics[width=0.99\textwidth]{figures/arch.pdf}
     \end{subfigure}
     ~\vspace{-0.2cm}
        \caption{\textbf{Q-function architecture.} The modules marked \textcolor{orange}{orange} are trained, otherwise the module is kept fixed.}
        \label{fig:arch}
\end{figure}

\subsection{Training Dataset Construction} \label{app:offline-dataset-construction}

We use the pre-trained AutoUI checkpoint to collect offline trajectories. Specifically, to collect each trajectory, starting from the home screen, the agent generates an action, and then the environment takes the action and transitions to the next state. It iterates until a maximum number of steps have been reached or the autonomous evaluator has decided to be a success. We collect 1296 trajectories this way for both AitW Webshop and AitW General subsets. The horizon $H$ of the Webshop subset is set to 20, and the horizon of the General subset is set to 10, which aligns with~\citep{bai2024digirltraininginthewilddevicecontrol}. Each trajectory is composed of state-action-reward-next-state pairs $(s, a, r, s')$, which is also referred to as ``transitions".

The $N$ actions in the offline dataset used for Best-of-N loss are sampled post-hoc from the pre-trained AutoUI checkpoint. When training the actor offline, as we use the Best-of-N loss, we want to sample more than one action. From an engineering aspect, collecting actions each time we sample from the offline dataset $\Dcal$ during training is not efficient. Thus, in practice, we pre-collect $K=64$ actions for each state, and store them in the offline dataset. As $N\in\{1,2,4,8,16\}$ is much smaller than $64$, this strategy serves as a good approximation and results in good performance. It suffices to give enough variety compared to sampling the actions when training the actor model. Note that in this case, the original action will always appear in the offline dataset.

\subsection{Additional Method Details} \label{app:additional-exp-details}

\textbf{Task set formulations.} The two task sets (Webshop and General) in the AitW dataset have different horizons $H$ (maximum number of steps allowed) in a trajectory to improve computational efficiency. Specifically, $H=20$ for AitW Webshop and $H=10$ for AitW General. Following tradition~\citep{bai2024digirltraininginthewilddevicecontrol}, we keep $A>1/H$ (e.g. 0.05 for AitW Webshop) as a threshold for the actor model to learn the state-action pair.

\textbf{Ablation on representation fine-tuning and TD learning as opposed to MC.} In the ablation study on representation fine-tuning, for all configurations, we train the actor model with Best-of-N loss where $N=16$ to keep computation efficient. This is also the case for the ablation on the TD learning as opposed to MC ablations.

\textbf{Ablation on actor loss.} For the ablation study on the actor loss, we keep the same trained Q function, while we ablate only on the loss used to train the actor model. We use $30$ actor epochs for the Best-of-N loss and AWR loss, and $120$ epochs for the REINFORCE loss as the magnitide of the raw advantage is very small. We use $N=16$ for the Best-of-N loss, while REINFORCE and AWR both uses the original action in the offlin dataset.

\textbf{Value function}. In practice, we find the V function significantly easier to train, and it suffices to only use the representations of the state from the vision encoder of the VLM (CLIP) to train the V functions. This simplification significantly saves time and space required, and aligns with previous work~\citep{bai2024digirltraininginthewilddevicecontrol}.

% We call this strategy \textit{approximate random action sampling}.

% \textit{Determinisitc action sampling}, on the other hand, always sample the first $n$ actions from the pre-collected action set. In this case, if a state is sampled multiple times in one iteration, the action set will always be the same, and thus the critic will always produce the same action for the actor to learn. We observe that the performance of determinisitc action sampling is significantly worse than approximate random action sampling, as shown in~\Cref{fig:ablation-n-actions-2curves} (\textit{Left}).

% Note that the $n=1$ case for both approximate random and determinisitc action sampling uses the original action (that causes the transition) instead of sampling it from the action set. Also, for all cases where $n>1$, we always include the original action into the sampled actions.



\section{More Qualitative Examples}

\subsection{Environment Errors} \label{app:env-errors}

We observe that several tasks has problems working in the environment introduced in~\citet{bai2024digirltraininginthewilddevicecontrol}. We observe that (1) the \url{newegg.com} domain has a high probability of blocking the agent from accessing it, and (2) the \url{costco.com} domain prevents the agent from typing the \texttt{<ENTER>} key. Examples are shown in~\Cref{fig:qual-env-error}. These problems were not observed in ~\citet{bai2024digirltraininginthewilddevicecontrol}. This is the main reason why some scores on the AitW Webshop subset in this paper falls a little behind~\citet{bai2024digirltraininginthewilddevicecontrol}.

\begin{figure}[!htp]
     \centering
    \begin{subfigure}[b]{0.85\textwidth}
         \centering
    \includegraphics[width=\textwidth]{figures/qual-env-error.pdf}
     \end{subfigure}
        \caption{\textbf{Environment errors.} These errors are systematic and can not be removed by the agent.}
        \label{fig:qual-env-error}
\end{figure}


\subsection{Example Trajectory Comparing REINFORCE and Best-of-N Loss} \label{app:pg-example}

We show a typical trajectory produced by the agent trained with REINFORCE in~\Cref{fig:pg-example}. We observe that the agent frequently diverges from the target and is too ``stubborn" to recover from errors. 

In this task towards searching for an item on costco.com, the agent has successfully arrived at costco.com, but (1) it takes some bad actions and (2) cannot recover. Specifically, after the agent clicks the warehouse button, it keeps clicking on the same button for 10 times until it clicks on somewhere else. This situation rarely appear in any trajectories collect by the agent trained with the Best-of-N loss.

\begin{figure}[!t]
     \centering
    \begin{subfigure}[b]{0.85\textwidth}
         \centering
    \includegraphics[width=\textwidth]{figures/pg-bad-example.pdf}
     \end{subfigure}
        \caption{\footnotesize{\textbf{Example trajectory of the agent trained with REINFORCE and Best-of-N loss.} Results show that the agent trained with REINFORCE tends to get stuck at a specific state because it's ``stubborn", while agent trained with Best-of-N loss effectively solves the task.}}
        \label{fig:pg-example}
        \vspace{2mm}
\end{figure}

\subsection{Benefits of dynamic programming} \label{app:dyn-prog}

\begin{figure}[t]
     \centering
    \begin{subfigure}[b]{0.85\textwidth}
         \centering
    \includegraphics[width=\textwidth]{figures/stitching-example.pdf}
     \end{subfigure}
        \caption{\footnotesize{\textbf{Trajectory examples showing benefits of Q-functions.} Our method can combine the best of a successful but lengthy (\textit{A}) trajectory and a failed but short trajectory (\textit{B}), to produce successful and short trajectories (\textit{C}).}}
        \label{fig:stitching-example}
\end{figure}

An appealing property of value-based RL is \emph{dynamic programming}: the ability to identify optimal behaviors from overall suboptimal rollouts. We present a qualitative example in~\Cref{fig:stitching-example} that illustrates this ability of \ourmethod{} in learning optimal behaviors from sub-optimal data. 
%%AK.10.1: readers will wonder -- how cherry picked is this?
In this example, trajectory (A) and (B) are from the offline dataset where trajectory (A) successfully completes the task but has many redundant actions while trajectory (B) does not have redundant actions but fails to complete the task. It turns out that \ourmethod{} is able to learn a policy that performs dynamic programming with trajectory (A) and (B) to produce a trajectory (C) that completes the task in the most efficient way. Neither trajectory (A) nor (B) is the optimal trajectory for solving the task but this example shows the ability of \ourmethod{} to learn an optimal policy from sub-optimal data, which is theoretically impossible through imitation alone.


\section{VLM Prompts} \label{app:vlm-prompts}

The prompt we use for fine-tuning and inferring the VLM is shown in~\Cref{fig:vlm-prompt}. The prompt template is designed to be action-type-specific, in order to facilitate the VLM to better differentiate different types of actions, which promotes fine-grained representations within the same action type. The input to the VLM is constructed by the image and the text prompt. Note that the VLM only sees the current image (overlayed with a cursor if the action is to click), and the next image is only used to calculate whether the target should be ``yes" or ``no". The target is a single token to promote computational efficiency. In practice, we find that a long target sequence introduces challenges for the VLM to fine-tune the representations.

\begin{figure}[!htp]
     \centering
    \begin{subfigure}[b]{0.85\textwidth}
         \centering
    \includegraphics[width=\textwidth]{figures/vlm-prompt.pdf}
     \end{subfigure}
     \vspace{2mm}
        \caption{\textbf{Prompt template we use to fine-tune and infer the VLM.} The input prompt consists of an input image and text input. The text input include a template prompt concatenated with an action-specific prompt. The action-specific prompt includes specific information about the input image. The output (target) prompt is just a word ``yes'' or ``no''.}
        \label{fig:vlm-prompt}
\end{figure}


\section{Experimental Details}

In this section, we provide more experimental details.

\subsection{\mnist{} with Text Rendering Program}

\cref{tab:hyper_mnist_cate,tab:hyper_mnist_nume} show the list of the parameters and their associated feasible sets and variation degrees in the \mnist{} with Text Rendering Program experiments. The total number of \pe{} iterations is 4.

\begin{table}[!h]
    \centering
    \caption{The configurations of the categorical parameters in \mnist{} with Text Rendering Program experiments.}
    \label{tab:hyper_mnist_cate}
    \begin{tabular}{c|c|c}
    \toprule
       Categorical Parameter ($\cate$)  & Feasible Set ($\cateset$)  & Variation Degrees ($\catevariationdegree$) across \pe{} Iterations\\\midrule
       Font  & 1 - 3589 & 0.8, 0.4, 0.2, 0.0\\
       Text & `0' - `9' & 0, 0, 0, 0\\
         \bottomrule
    \end{tabular}
\end{table}

\begin{table}[!h]
    \centering
    \caption{The configurations of the numerical parameters in \mnist{} with Text Rendering Program experiments.}
    \label{tab:hyper_mnist_nume}
    \begin{tabular}{c|c|c}
    \toprule
       Numerical Parameter ($\nume$)  & Feasible Set ($\numeset$)  & Variation Degrees ($\numevariationdegree$) across \pe{} Iterations\\\midrule
       Font size  & [10, 30] & 5, 4, 3, 2\\
       Font rotation & [-30, 30] & 9, 7, 5, 3\\
       Stroke width & [0, 2] & 1, 1, 0, 0\\
         \bottomrule
    \end{tabular}
\end{table}



\subsection{\celeba{} with Generated Images from Computer Graphics-based Render}

The variation degrees $\nnvariationdegree$ across \pe{} iterations are [1000, 500, 200, 100, 50, 20]. The total number of \pe{} iterations is 6.

\subsection{\celeba{} with Rule-based Avatar Generator}

The full list of the categorical parameters are 
    \begin{packeditemize}
    \item Style
    \item Background color
    \item Top
    \item Hat color
    \item Eyebrows
    \item Eyes
    \item Nose
    \item Mouth
    \item Facial hair
    \item Skin color
    \item Hair color
    \item Facial hair color
    \item Accessory
    \item Clothing
    \item Clothing color
    \item Shirt graphic
\end{packeditemize}
These are taken from the input parameters to the library \cite{pythonavatar}. There is no numerical parameter. 

For the experiments with only the simulator,
the variation degrees $\catevariationdegree$ across \pe{} iterations are [0.8, 0.6, 0.4, 0.2, 0.1, 0.08, 0.06].
The total number of \pe{} iterations is 7.

For the experiments with both foundation models and the simulator, we use a total of 5 \pe{} iterations so as to be consistent with the setting in \citet{dpimagebench}.
For the \randomsampleapiname{} and the first \pe{} iteration, we use the simulator ($\catevariationdegree=0.8$). For the next 4 \pe{} iterations, we use the same foundation model as in \citet{lin2023differentially} with variation degrees [96, 94, 92, 90].




\end{document}
