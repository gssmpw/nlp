\documentclass{proc}


\usepackage{microtype}
\usepackage{graphicx}
\usepackage{subcaption}
\usepackage{multirow}
\usepackage{booktabs} %

\usepackage{xcolor}
\usepackage{xspace}

\usepackage[round]{natbib}
\renewcommand{\bibname}{References}
\renewcommand{\bibsection}{\subsubsection*{\bibname}}


\bibliographystyle{apalike}

\usepackage{hyperref}


\newcommand{\theHalgorithm}{\arabic{algorithm}}
\newcommand{\suppmat}{appendix\xspace}


%%%%% NEW MATH DEFINITIONS %%%%%

\usepackage{amsmath,amsfonts,bm}
\usepackage{derivative}
% Mark sections of captions for referring to divisions of figures
\newcommand{\figleft}{{\em (Left)}}
\newcommand{\figcenter}{{\em (Center)}}
\newcommand{\figright}{{\em (Right)}}
\newcommand{\figtop}{{\em (Top)}}
\newcommand{\figbottom}{{\em (Bottom)}}
\newcommand{\captiona}{{\em (a)}}
\newcommand{\captionb}{{\em (b)}}
\newcommand{\captionc}{{\em (c)}}
\newcommand{\captiond}{{\em (d)}}

% Highlight a newly defined term
\newcommand{\newterm}[1]{{\bf #1}}

% Derivative d 
\newcommand{\deriv}{{\mathrm{d}}}

% Figure reference, lower-case.
\def\figref#1{figure~\ref{#1}}
% Figure reference, capital. For start of sentence
\def\Figref#1{Figure~\ref{#1}}
\def\twofigref#1#2{figures \ref{#1} and \ref{#2}}
\def\quadfigref#1#2#3#4{figures \ref{#1}, \ref{#2}, \ref{#3} and \ref{#4}}
% Section reference, lower-case.
\def\secref#1{section~\ref{#1}}
% Section reference, capital.
\def\Secref#1{Section~\ref{#1}}
% Reference to two sections.
\def\twosecrefs#1#2{sections \ref{#1} and \ref{#2}}
% Reference to three sections.
\def\secrefs#1#2#3{sections \ref{#1}, \ref{#2} and \ref{#3}}
% Reference to an equation, lower-case.
\def\eqref#1{equation~\ref{#1}}
% Reference to an equation, upper case
\def\Eqref#1{Equation~\ref{#1}}
% A raw reference to an equation---avoid using if possible
\def\plaineqref#1{\ref{#1}}
% Reference to a chapter, lower-case.
\def\chapref#1{chapter~\ref{#1}}
% Reference to an equation, upper case.
\def\Chapref#1{Chapter~\ref{#1}}
% Reference to a range of chapters
\def\rangechapref#1#2{chapters\ref{#1}--\ref{#2}}
% Reference to an algorithm, lower-case.
\def\algref#1{algorithm~\ref{#1}}
% Reference to an algorithm, upper case.
\def\Algref#1{Algorithm~\ref{#1}}
\def\twoalgref#1#2{algorithms \ref{#1} and \ref{#2}}
\def\Twoalgref#1#2{Algorithms \ref{#1} and \ref{#2}}
% Reference to a part, lower case
\def\partref#1{part~\ref{#1}}
% Reference to a part, upper case
\def\Partref#1{Part~\ref{#1}}
\def\twopartref#1#2{parts \ref{#1} and \ref{#2}}

\def\ceil#1{\lceil #1 \rceil}
\def\floor#1{\lfloor #1 \rfloor}
\def\1{\bm{1}}
\newcommand{\train}{\mathcal{D}}
\newcommand{\valid}{\mathcal{D_{\mathrm{valid}}}}
\newcommand{\test}{\mathcal{D_{\mathrm{test}}}}

\def\eps{{\epsilon}}


% Random variables
\def\reta{{\textnormal{$\eta$}}}
\def\ra{{\textnormal{a}}}
\def\rb{{\textnormal{b}}}
\def\rc{{\textnormal{c}}}
\def\rd{{\textnormal{d}}}
\def\re{{\textnormal{e}}}
\def\rf{{\textnormal{f}}}
\def\rg{{\textnormal{g}}}
\def\rh{{\textnormal{h}}}
\def\ri{{\textnormal{i}}}
\def\rj{{\textnormal{j}}}
\def\rk{{\textnormal{k}}}
\def\rl{{\textnormal{l}}}
% rm is already a command, just don't name any random variables m
\def\rn{{\textnormal{n}}}
\def\ro{{\textnormal{o}}}
\def\rp{{\textnormal{p}}}
\def\rq{{\textnormal{q}}}
\def\rr{{\textnormal{r}}}
\def\rs{{\textnormal{s}}}
\def\rt{{\textnormal{t}}}
\def\ru{{\textnormal{u}}}
\def\rv{{\textnormal{v}}}
\def\rw{{\textnormal{w}}}
\def\rx{{\textnormal{x}}}
\def\ry{{\textnormal{y}}}
\def\rz{{\textnormal{z}}}

% Random vectors
\def\rvepsilon{{\mathbf{\epsilon}}}
\def\rvphi{{\mathbf{\phi}}}
\def\rvtheta{{\mathbf{\theta}}}
\def\rva{{\mathbf{a}}}
\def\rvb{{\mathbf{b}}}
\def\rvc{{\mathbf{c}}}
\def\rvd{{\mathbf{d}}}
\def\rve{{\mathbf{e}}}
\def\rvf{{\mathbf{f}}}
\def\rvg{{\mathbf{g}}}
\def\rvh{{\mathbf{h}}}
\def\rvu{{\mathbf{i}}}
\def\rvj{{\mathbf{j}}}
\def\rvk{{\mathbf{k}}}
\def\rvl{{\mathbf{l}}}
\def\rvm{{\mathbf{m}}}
\def\rvn{{\mathbf{n}}}
\def\rvo{{\mathbf{o}}}
\def\rvp{{\mathbf{p}}}
\def\rvq{{\mathbf{q}}}
\def\rvr{{\mathbf{r}}}
\def\rvs{{\mathbf{s}}}
\def\rvt{{\mathbf{t}}}
\def\rvu{{\mathbf{u}}}
\def\rvv{{\mathbf{v}}}
\def\rvw{{\mathbf{w}}}
\def\rvx{{\mathbf{x}}}
\def\rvy{{\mathbf{y}}}
\def\rvz{{\mathbf{z}}}

% Elements of random vectors
\def\erva{{\textnormal{a}}}
\def\ervb{{\textnormal{b}}}
\def\ervc{{\textnormal{c}}}
\def\ervd{{\textnormal{d}}}
\def\erve{{\textnormal{e}}}
\def\ervf{{\textnormal{f}}}
\def\ervg{{\textnormal{g}}}
\def\ervh{{\textnormal{h}}}
\def\ervi{{\textnormal{i}}}
\def\ervj{{\textnormal{j}}}
\def\ervk{{\textnormal{k}}}
\def\ervl{{\textnormal{l}}}
\def\ervm{{\textnormal{m}}}
\def\ervn{{\textnormal{n}}}
\def\ervo{{\textnormal{o}}}
\def\ervp{{\textnormal{p}}}
\def\ervq{{\textnormal{q}}}
\def\ervr{{\textnormal{r}}}
\def\ervs{{\textnormal{s}}}
\def\ervt{{\textnormal{t}}}
\def\ervu{{\textnormal{u}}}
\def\ervv{{\textnormal{v}}}
\def\ervw{{\textnormal{w}}}
\def\ervx{{\textnormal{x}}}
\def\ervy{{\textnormal{y}}}
\def\ervz{{\textnormal{z}}}

% Random matrices
\def\rmA{{\mathbf{A}}}
\def\rmB{{\mathbf{B}}}
\def\rmC{{\mathbf{C}}}
\def\rmD{{\mathbf{D}}}
\def\rmE{{\mathbf{E}}}
\def\rmF{{\mathbf{F}}}
\def\rmG{{\mathbf{G}}}
\def\rmH{{\mathbf{H}}}
\def\rmI{{\mathbf{I}}}
\def\rmJ{{\mathbf{J}}}
\def\rmK{{\mathbf{K}}}
\def\rmL{{\mathbf{L}}}
\def\rmM{{\mathbf{M}}}
\def\rmN{{\mathbf{N}}}
\def\rmO{{\mathbf{O}}}
\def\rmP{{\mathbf{P}}}
\def\rmQ{{\mathbf{Q}}}
\def\rmR{{\mathbf{R}}}
\def\rmS{{\mathbf{S}}}
\def\rmT{{\mathbf{T}}}
\def\rmU{{\mathbf{U}}}
\def\rmV{{\mathbf{V}}}
\def\rmW{{\mathbf{W}}}
\def\rmX{{\mathbf{X}}}
\def\rmY{{\mathbf{Y}}}
\def\rmZ{{\mathbf{Z}}}

% Elements of random matrices
\def\ermA{{\textnormal{A}}}
\def\ermB{{\textnormal{B}}}
\def\ermC{{\textnormal{C}}}
\def\ermD{{\textnormal{D}}}
\def\ermE{{\textnormal{E}}}
\def\ermF{{\textnormal{F}}}
\def\ermG{{\textnormal{G}}}
\def\ermH{{\textnormal{H}}}
\def\ermI{{\textnormal{I}}}
\def\ermJ{{\textnormal{J}}}
\def\ermK{{\textnormal{K}}}
\def\ermL{{\textnormal{L}}}
\def\ermM{{\textnormal{M}}}
\def\ermN{{\textnormal{N}}}
\def\ermO{{\textnormal{O}}}
\def\ermP{{\textnormal{P}}}
\def\ermQ{{\textnormal{Q}}}
\def\ermR{{\textnormal{R}}}
\def\ermS{{\textnormal{S}}}
\def\ermT{{\textnormal{T}}}
\def\ermU{{\textnormal{U}}}
\def\ermV{{\textnormal{V}}}
\def\ermW{{\textnormal{W}}}
\def\ermX{{\textnormal{X}}}
\def\ermY{{\textnormal{Y}}}
\def\ermZ{{\textnormal{Z}}}

% Vectors
\def\vzero{{\bm{0}}}
\def\vone{{\bm{1}}}
\def\vmu{{\bm{\mu}}}
\def\vtheta{{\bm{\theta}}}
\def\vphi{{\bm{\phi}}}
\def\va{{\bm{a}}}
\def\vb{{\bm{b}}}
\def\vc{{\bm{c}}}
\def\vd{{\bm{d}}}
\def\ve{{\bm{e}}}
\def\vf{{\bm{f}}}
\def\vg{{\bm{g}}}
\def\vh{{\bm{h}}}
\def\vi{{\bm{i}}}
\def\vj{{\bm{j}}}
\def\vk{{\bm{k}}}
\def\vl{{\bm{l}}}
\def\vm{{\bm{m}}}
\def\vn{{\bm{n}}}
\def\vo{{\bm{o}}}
\def\vp{{\bm{p}}}
\def\vq{{\bm{q}}}
\def\vr{{\bm{r}}}
\def\vs{{\bm{s}}}
\def\vt{{\bm{t}}}
\def\vu{{\bm{u}}}
\def\vv{{\bm{v}}}
\def\vw{{\bm{w}}}
\def\vx{{\bm{x}}}
\def\vy{{\bm{y}}}
\def\vz{{\bm{z}}}

% Elements of vectors
\def\evalpha{{\alpha}}
\def\evbeta{{\beta}}
\def\evepsilon{{\epsilon}}
\def\evlambda{{\lambda}}
\def\evomega{{\omega}}
\def\evmu{{\mu}}
\def\evpsi{{\psi}}
\def\evsigma{{\sigma}}
\def\evtheta{{\theta}}
\def\eva{{a}}
\def\evb{{b}}
\def\evc{{c}}
\def\evd{{d}}
\def\eve{{e}}
\def\evf{{f}}
\def\evg{{g}}
\def\evh{{h}}
\def\evi{{i}}
\def\evj{{j}}
\def\evk{{k}}
\def\evl{{l}}
\def\evm{{m}}
\def\evn{{n}}
\def\evo{{o}}
\def\evp{{p}}
\def\evq{{q}}
\def\evr{{r}}
\def\evs{{s}}
\def\evt{{t}}
\def\evu{{u}}
\def\evv{{v}}
\def\evw{{w}}
\def\evx{{x}}
\def\evy{{y}}
\def\evz{{z}}

% Matrix
\def\mA{{\bm{A}}}
\def\mB{{\bm{B}}}
\def\mC{{\bm{C}}}
\def\mD{{\bm{D}}}
\def\mE{{\bm{E}}}
\def\mF{{\bm{F}}}
\def\mG{{\bm{G}}}
\def\mH{{\bm{H}}}
\def\mI{{\bm{I}}}
\def\mJ{{\bm{J}}}
\def\mK{{\bm{K}}}
\def\mL{{\bm{L}}}
\def\mM{{\bm{M}}}
\def\mN{{\bm{N}}}
\def\mO{{\bm{O}}}
\def\mP{{\bm{P}}}
\def\mQ{{\bm{Q}}}
\def\mR{{\bm{R}}}
\def\mS{{\bm{S}}}
\def\mT{{\bm{T}}}
\def\mU{{\bm{U}}}
\def\mV{{\bm{V}}}
\def\mW{{\bm{W}}}
\def\mX{{\bm{X}}}
\def\mY{{\bm{Y}}}
\def\mZ{{\bm{Z}}}
\def\mBeta{{\bm{\beta}}}
\def\mPhi{{\bm{\Phi}}}
\def\mLambda{{\bm{\Lambda}}}
\def\mSigma{{\bm{\Sigma}}}

% Tensor
\DeclareMathAlphabet{\mathsfit}{\encodingdefault}{\sfdefault}{m}{sl}
\SetMathAlphabet{\mathsfit}{bold}{\encodingdefault}{\sfdefault}{bx}{n}
\newcommand{\tens}[1]{\bm{\mathsfit{#1}}}
\def\tA{{\tens{A}}}
\def\tB{{\tens{B}}}
\def\tC{{\tens{C}}}
\def\tD{{\tens{D}}}
\def\tE{{\tens{E}}}
\def\tF{{\tens{F}}}
\def\tG{{\tens{G}}}
\def\tH{{\tens{H}}}
\def\tI{{\tens{I}}}
\def\tJ{{\tens{J}}}
\def\tK{{\tens{K}}}
\def\tL{{\tens{L}}}
\def\tM{{\tens{M}}}
\def\tN{{\tens{N}}}
\def\tO{{\tens{O}}}
\def\tP{{\tens{P}}}
\def\tQ{{\tens{Q}}}
\def\tR{{\tens{R}}}
\def\tS{{\tens{S}}}
\def\tT{{\tens{T}}}
\def\tU{{\tens{U}}}
\def\tV{{\tens{V}}}
\def\tW{{\tens{W}}}
\def\tX{{\tens{X}}}
\def\tY{{\tens{Y}}}
\def\tZ{{\tens{Z}}}


% Graph
\def\gA{{\mathcal{A}}}
\def\gB{{\mathcal{B}}}
\def\gC{{\mathcal{C}}}
\def\gD{{\mathcal{D}}}
\def\gE{{\mathcal{E}}}
\def\gF{{\mathcal{F}}}
\def\gG{{\mathcal{G}}}
\def\gH{{\mathcal{H}}}
\def\gI{{\mathcal{I}}}
\def\gJ{{\mathcal{J}}}
\def\gK{{\mathcal{K}}}
\def\gL{{\mathcal{L}}}
\def\gM{{\mathcal{M}}}
\def\gN{{\mathcal{N}}}
\def\gO{{\mathcal{O}}}
\def\gP{{\mathcal{P}}}
\def\gQ{{\mathcal{Q}}}
\def\gR{{\mathcal{R}}}
\def\gS{{\mathcal{S}}}
\def\gT{{\mathcal{T}}}
\def\gU{{\mathcal{U}}}
\def\gV{{\mathcal{V}}}
\def\gW{{\mathcal{W}}}
\def\gX{{\mathcal{X}}}
\def\gY{{\mathcal{Y}}}
\def\gZ{{\mathcal{Z}}}

% Sets
\def\sA{{\mathbb{A}}}
\def\sB{{\mathbb{B}}}
\def\sC{{\mathbb{C}}}
\def\sD{{\mathbb{D}}}
% Don't use a set called E, because this would be the same as our symbol
% for expectation.
\def\sF{{\mathbb{F}}}
\def\sG{{\mathbb{G}}}
\def\sH{{\mathbb{H}}}
\def\sI{{\mathbb{I}}}
\def\sJ{{\mathbb{J}}}
\def\sK{{\mathbb{K}}}
\def\sL{{\mathbb{L}}}
\def\sM{{\mathbb{M}}}
\def\sN{{\mathbb{N}}}
\def\sO{{\mathbb{O}}}
\def\sP{{\mathbb{P}}}
\def\sQ{{\mathbb{Q}}}
\def\sR{{\mathbb{R}}}
\def\sS{{\mathbb{S}}}
\def\sT{{\mathbb{T}}}
\def\sU{{\mathbb{U}}}
\def\sV{{\mathbb{V}}}
\def\sW{{\mathbb{W}}}
\def\sX{{\mathbb{X}}}
\def\sY{{\mathbb{Y}}}
\def\sZ{{\mathbb{Z}}}

% Entries of a matrix
\def\emLambda{{\Lambda}}
\def\emA{{A}}
\def\emB{{B}}
\def\emC{{C}}
\def\emD{{D}}
\def\emE{{E}}
\def\emF{{F}}
\def\emG{{G}}
\def\emH{{H}}
\def\emI{{I}}
\def\emJ{{J}}
\def\emK{{K}}
\def\emL{{L}}
\def\emM{{M}}
\def\emN{{N}}
\def\emO{{O}}
\def\emP{{P}}
\def\emQ{{Q}}
\def\emR{{R}}
\def\emS{{S}}
\def\emT{{T}}
\def\emU{{U}}
\def\emV{{V}}
\def\emW{{W}}
\def\emX{{X}}
\def\emY{{Y}}
\def\emZ{{Z}}
\def\emSigma{{\Sigma}}

% entries of a tensor
% Same font as tensor, without \bm wrapper
\newcommand{\etens}[1]{\mathsfit{#1}}
\def\etLambda{{\etens{\Lambda}}}
\def\etA{{\etens{A}}}
\def\etB{{\etens{B}}}
\def\etC{{\etens{C}}}
\def\etD{{\etens{D}}}
\def\etE{{\etens{E}}}
\def\etF{{\etens{F}}}
\def\etG{{\etens{G}}}
\def\etH{{\etens{H}}}
\def\etI{{\etens{I}}}
\def\etJ{{\etens{J}}}
\def\etK{{\etens{K}}}
\def\etL{{\etens{L}}}
\def\etM{{\etens{M}}}
\def\etN{{\etens{N}}}
\def\etO{{\etens{O}}}
\def\etP{{\etens{P}}}
\def\etQ{{\etens{Q}}}
\def\etR{{\etens{R}}}
\def\etS{{\etens{S}}}
\def\etT{{\etens{T}}}
\def\etU{{\etens{U}}}
\def\etV{{\etens{V}}}
\def\etW{{\etens{W}}}
\def\etX{{\etens{X}}}
\def\etY{{\etens{Y}}}
\def\etZ{{\etens{Z}}}

% The true underlying data generating distribution
\newcommand{\pdata}{p_{\rm{data}}}
\newcommand{\ptarget}{p_{\rm{target}}}
\newcommand{\pprior}{p_{\rm{prior}}}
\newcommand{\pbase}{p_{\rm{base}}}
\newcommand{\pref}{p_{\rm{ref}}}

% The empirical distribution defined by the training set
\newcommand{\ptrain}{\hat{p}_{\rm{data}}}
\newcommand{\Ptrain}{\hat{P}_{\rm{data}}}
% The model distribution
\newcommand{\pmodel}{p_{\rm{model}}}
\newcommand{\Pmodel}{P_{\rm{model}}}
\newcommand{\ptildemodel}{\tilde{p}_{\rm{model}}}
% Stochastic autoencoder distributions
\newcommand{\pencode}{p_{\rm{encoder}}}
\newcommand{\pdecode}{p_{\rm{decoder}}}
\newcommand{\precons}{p_{\rm{reconstruct}}}

\newcommand{\laplace}{\mathrm{Laplace}} % Laplace distribution

\newcommand{\E}{\mathbb{E}}
\newcommand{\Ls}{\mathcal{L}}
\newcommand{\R}{\mathbb{R}}
\newcommand{\emp}{\tilde{p}}
\newcommand{\lr}{\alpha}
\newcommand{\reg}{\lambda}
\newcommand{\rect}{\mathrm{rectifier}}
\newcommand{\softmax}{\mathrm{softmax}}
\newcommand{\sigmoid}{\sigma}
\newcommand{\softplus}{\zeta}
\newcommand{\KL}{D_{\mathrm{KL}}}
\newcommand{\Var}{\mathrm{Var}}
\newcommand{\standarderror}{\mathrm{SE}}
\newcommand{\Cov}{\mathrm{Cov}}
% Wolfram Mathworld says $L^2$ is for function spaces and $\ell^2$ is for vectors
% But then they seem to use $L^2$ for vectors throughout the site, and so does
% wikipedia.
\newcommand{\normlzero}{L^0}
\newcommand{\normlone}{L^1}
\newcommand{\normltwo}{L^2}
\newcommand{\normlp}{L^p}
\newcommand{\normmax}{L^\infty}

\newcommand{\parents}{Pa} % See usage in notation.tex. Chosen to match Daphne's book.

\DeclareMathOperator*{\argmax}{arg\,max}
\DeclareMathOperator*{\argmin}{arg\,min}

\DeclareMathOperator{\sign}{sign}
\DeclareMathOperator{\Tr}{Tr}
\let\ab\allowbreak


\usepackage[capitalize,noabbrev]{cleveref}
\usepackage{algorithm}
\usepackage{algorithmic}

\renewcommand\algorithmiccomment[1]{\{\textit{#1}\}}


\usepackage[textsize=tiny]{todonotes}

\title{Solving Linear-Gaussian Bayesian Inverse Problems with\\Decoupled Diffusion Sequential Monte Carlo}

\author{Filip Ekström Kelvinius, Zheng Zhao, Fredrik Lindsten
\\
Linköping University
\\
\texttt{\{filip.ekstrom, zheng.zhao, fredrik.lindsten\}@liu.se}
}

\begin{document}
\maketitle

\begin{abstract}
A recent line of research has exploited pre-trained generative diffusion models as priors for solving Bayesian inverse problems. We contribute to this research direction by designing a sequential Monte Carlo method for linear-Gaussian inverse problems which builds on ``decoupled diffusion", where the generative process is designed such that larger updates to the sample are possible. The method is asymptotically exact and we demonstrate the effectiveness of our Decoupled Diffusion Sequential Monte Carlo (DDSMC) algorithm on both synthetic data and image reconstruction tasks. Further, we demonstrate how the approach can be extended to discrete data.
\end{abstract}

\section{Introduction}
Generative diffusion models \cite{sohl-dickstein_deep_2015, ho_denoising_2020-2, song_score-based_2021} have sparked an enormous interest from the research community, and shown impressive results on a wide variety of modeling task, ranging from image synthesis \citep{dhariwal_diffusion_2021-1, rombach_high-resolution_2022, saharia_photorealistic_2022} and audio generation \cite{chen_wavegrad_2020,kong_diffwave_2021} to molecule and protein generation \cite{hoogeboom_equivariant_2022, xu_geodiff_2022, corso_diffdock_2023}. A diffusion model consists of a neural network which implicitly, through a generative procedure, defines an approximation of the data distribution, $p_\theta(\x)$. While methods~\citep[e.g.][]{ho_classifier-free_2022} exist where the model is explicitly trained to define a conditional distribution $p_\theta(\x|\y)$, this conditional training is not always possible. Alternatively, a domain-specific likelihood $p(\y|\x)$ might be known, and in this case, using $p_\theta(\x)$ as a prior in a Bayesian inverse problem setting, i.e., sampling from the posterior distribution $p_\theta(\x|\y) \propto p(\y|\x)p_\theta(\x)$, becomes an appealing alternative. This approach is flexible since many different likelihoods can be used with the same diffusion model prior, without requiring retraining or access to paired training data. Previous methods for posterior sampling with diffusion priors, while providing impressive results on tasks like image reconstruction \citep{kawar_denoising_2022, chung_diffusion_2023, song2023pseudoinverseguided}, often rely on approximations and fail or perform poorly on simple tasks \citep[and our \Cref{sec:gmm_exp}]{cardoso_monte_2023-2}, making it uncertain to what extent they can solve Bayesian inference problems in general.

Sequential Monte Carlo (SMC) 
is a well-established method for Bayesian inference,
and its use of sequences of distributions makes it a natural choice to combine with diffusion priors. It also offers asymptotical guarantees, and the combination of SMC and diffusion models has recently seen a spark of interest \cite{trippe_diffusion_2023,dou_diffusion_2023-1,wu_practical_2023, cardoso_monte_2023-2}. Moreover, the design of an efficient SMC algorithm involves a high degree of flexibility while guaranteeing asymptotic exactness, which makes it an interesting framework for continued exploration. 

As previous (non-SMC) works on posterior sampling has introduced different approximations, and SMC offers a flexible framework with asymptotic guarantees, we aim to further investigate the use of SMC for Bayesian inverse problems with diffusion priors. In particular, we target the case of linear-Gaussian likelihood models. We develop a method which we call Decoupled Diffusion SMC (DDSMC) utilizing and extending a previously introduced technique for posterior sampling based on decoupled diffusion \citep{zhang_improving_2024}. With this approach, it is possible to make larger updates to the sample during the generative procedure, and it also opens up new ways of taking the conditioning on $\y$ into account in the design of the SMC proposal distribution. We show how this method can effectively perform posterior sampling for both synthetic data and image reconstruction tasks. We also further generalize DDSMC for discrete data.

\section{Background}
As mentioned in the introduction, we are interested in sampling from the posterior distribution
\(
    p_\theta(\x|\y) \propto p(\y|\x)p_\theta(\x),
\)
where the prior $p_\theta(\x)$ is implicitly defined by a pre-trained diffusion model and the likelihood is linear-Gaussian, i.e., 
\(
    p(\y|\x) = \mathcal{N}(\y|A\x, \sigma_y^2I).
\)
We will use the notation $p_\theta(\cdot)$ for any distribution that is defined via the diffusion model, the notation $p(\cdot)$ for the likelihood (which is assumed to be known), and $q(\cdot)$ for any distribution related to the fixed \emph{forward} diffusion process (see next section).

\subsection{Diffusion Models}
\label{sec:diffmodels}
Diffusion models are based on transforming data, $\x_0$, to Gaussian noise by a Markovian \emph{forward} process of the form\footnote{In principle, $\xt$ could also be multiplied by some time-dependent parameter, but we skip this here for simplicity.}
\begin{align}
    q(\xtplusone|\xt) = \mathcal{N}(\xtplusone|\xt, \beta_{t+1}I). \label{eq:difftrans}
\end{align}
The generative process then consists in trying to reverse this process, and is parametrized as a backward Markov process
\begin{align}
    p_\theta(\x_{0:T}) = p_\theta(\x_T)\prod_{t=0}^{T-1} p_\theta(\xt|\xtplusone). \label{eq:diffbackward_seq}
\end{align}
The reverse process $p_\theta$ is fitted to approximate the reversal of the forward process, i.e., $p_\theta(\xt|\xtplusone) \approx q(\xt|\xtplusone)$, where the latter can be expressed as 
\begin{align}
q(\xt|\xtplusone) = \int q(\xt|\xtplusone, \x_0)q(\x_0|\xtplusone)d\x_0. \label{eq:diffbackwardint}
\end{align}
While $q(\xt|\xtplusone, \x_0)$ is available in closed form, $q(\x_0|\xtplusone)$ is not, thereby rendering \Cref{eq:diffbackwardint} intractable. In practice, this is typically handled by replacing the conditional $q(\x_0|\xtplusone)$ with a point estimate $\delta_{\recon(\xtplusone)}(\x_0)$, where $\recon(\xtplusone)$ is a ``reconstruction" of $\x_0$, computed by a neural network. This is typically done with Tweedie's formula, where $\recon(\xtplusone) = \xt + \sigma_{t+1}^2 s_\theta(\xtplusone, t+1) \approx \mathbb{E}[\x_0|\xtplusone]$ and $s_\theta(\xtplusone, t+1)$ is a neural network which approximates the score $\nabla_{\xtplusone}\log q(\xtplusone)$ (and using the true score would result in the reconstruction being equal to the expected value). With this approximation, the backward kernel becomes $p_\theta(\xt|\xtplusone) = q(\xt|\xtplusone, \x_0=\recon(\xtplusone))$. 

\paragraph{Probability Flow ODE}
\citet{song_score-based_2021} described how diffusion models can be generalized as time-continuous stochastic differential equations (SDEs), and how sampling form a diffusion model can be viewed as solving the corresponding reverse-time SDE. They further derive a ``probability flow ordinary differential equation (PF-ODE)'', which allows sampling from the same distribution $p_\theta(\xt)$ as the SDE by solving a deterministic ODE (initialized randomly).

It can be noted that the probability flow formulation conceptually can be used to sample from \Cref{eq:diffbackwardint}. By solving the PF-ODE from state $\xtplusone \sim q(\xtplusone)$ to time $t=0$ we obtain a sample from the conditional $q(\x_0|\xtplusone)$, which can be plugged in to \Cref{eq:diffbackwardint} to generate a sample from $q(\xt|\xtplusone)$. For unconditional sampling this is a convoluted way of performing the generation, since the $\x_0$ obtained after solving the PF-ODE is already a sample from the approximate data distribution. However, as we discuss in \Cref{sec:ddsmc}, this approach does provide an interesting possibility for conditional sampling.

\subsection{Sequential Monte Carlo}
\label{sec:smc_background}
Sequential Monte Carlo \citep[SMC; see, e.g.,][for an overview]{NaessethLS:2019a} is a class of methods that enable sampling from a sequence of distributions $\{\pi_t(\xtt)\}_{t=0}^T$, which are known and can be evaluated up to a normalizing constant, i.e., $\pi_{t}(\xtt) = \gammat/Z_t$, where $\gammat$ can be evaluated pointwise. In an SMC algorithm, a set of $N$ samples, or \emph{particles}, are generated in parallel, and they are weighted such that a set of (weighted) particles $\{(\xtt^i, w_t^i)\}_{i=1}^N$ are approximate draws from the target distribution $\pi_t(\xtt)$. For each $t$, an SMC algorithm consists of three steps. The \emph{resampling} step samples a new set of particles $\{(\xtt^i)\}_{i=1}^N$ from $\text{Multinomial}(\{(\xtt^i)\}_{i=1}^N; \{w_t^i\}_{i=1}^N)$\footnote{The resampling procedure is a design choice. In this paper we use Multinomial with probabilities $\{w_t^i\}_{i=1}^N$ for simplicity.}. The second step is the \emph{proposal} step, where new samples $\{\xtonet^i\}_{i=1}^N$ are proposed as $\xtone^i\sim r_{t-1}(\xtone^i|\xtt^i), \ \xtonet^i = (\xtone^i, \xtt^i)$, and finally a \emph{weighting} step, $w^i_{t-1} \propto \gamma_{t-1}(\xtonet^i) / (r_{t-1}(\xtone^i|\xtt^i) \gamma_{t}(\xtt^i))$. To construct an SMC algorithm, it is hence necessary to determine two components: the target distributions $\{\pi_t(\xtt)\}_{t=0}^T$ (or rather, the unnormalized distributions $\{\gammat\}_{t=0}^T$), and the proposals $\{r_{t-1}(\xtone|\xtt)\}_{t=1}^{T}$. 

\paragraph{SMC as a General Sampler}
Even though SMC relies on a sequence of (unnormalized) distributions, it can still be used as a general sampler to sample from some ``static'' distribution $\phi(\x)$ by introducing auxiliary variables $\x_{0:T}$ and \emph{constructing} a sequence of distributions over $\{ \x_{t:T} \}_{t=0}^T$. As long as the marginal distribution of $\x_0$ w.r.t.\ the \emph{final} distribution $\pi_0(\xzerot)$ is equal to $\phi(\x_0)$, the SMC algorithm will provide a consistent approximation of $\phi(\x)$ (i.e.,\ increasingly accurate as the number of particles $N$ increases). The \emph{intermediate target} distributions $\{\pi_t(\xtt)\}_{t=1}^T$ are then merely a means for approximating the final target
$\pi_0(\xzerot)$.

\section{Decoupled Diffusion SMC}
\label{sec:ddsmc}
\subsection{Target Distributions}
\label{sec:ddsmc_prior}
The sequential nature of the generative diffusion model naturally suggests that SMC can be used for the Bayesian inverse problem, by constructing a sequence of target distributions based on \Cref{eq:diffbackward_seq}. This approach has been explored in several recent works \citep[see also \Cref{sec:related_work}]{wu_practical_2023, cardoso_monte_2023-2}. However, \citet{zhang_improving_2024} recently proposed an alternative simulation protocol for tackling inverse problems with diffusion priors, based on a ``decoupling" argument: they propose to simulate (approximately) from $p_\theta(\x_0|\xtplusone, \y)$ and then push this sample forward to diffusion time $t$ by sampling from the forward kernel $q(\xt|\x_0)$ instead of $q(\xt|\x_0, \xtplusone)$. The motivation is to reduce the autocorrelation in the generative process to enable making transitions with larger updates and thus correct larger, global errors. The resulting method is referred to as Decoupled Annealing Posterior Sampling (DAPS).

To leverage this idea, we can realize that the SMC framework is in fact very general and, as discussed in \Cref{sec:smc_background}, the sequence of target distributions can be seen as a design choice, as long as the final target $\pi_0(\xzerot)$ admits the distribution of interest as a marginal. Thus, it is possible to use the DAPS sampling protocol as a basis for SMC. This corresponds to redefining the prior over trajectories, from \Cref{eq:diffbackward_seq} to $p_\theta^0(\xzerot) = p_\theta(\x_T)\prod_{t=0}^{T-1}p_\theta^0(\xt|\xtplusone)$ where
\begin{align}
    p_\theta^0(\xt|\xtplusone) = q(\xt|\x_0=\recon(\xtplusone)).
\end{align}
Conceptually, this corresponds to reconstructing $\x_0$ conditionally on the current state $\xtplusone$, followed by adding noise to the reconstructed sample using the forward model. As discussed by \citet{zhang_improving_2024}, and also proven by us in \Cref{app:proofs}, the two transitions can still lead to the same marginal distributions for all time points $t$, i.e., $\int p_\theta^0(\xtt)d\xtplusoneT = \int p_\theta(\xtt)d\xtplusoneT$,  under the assumption that $\x_0 = \recon(\xtplusone)$ is a sample from $p_\theta(\x_0)$. This can be approximately obtained by, e.g., solving the reverse-time PF-ODE, and using this solution as reconstruction model $\recon(\xtplusone)$.

\paragraph{Generalizing the DAPS Prior}
By rewriting the conditional forward kernel $q(\xt|\xtplusone, \x_0)$ using Bayes theorem,
\begin{align}
    q(\xt|\xtplusone, \x_0) = \frac{q(\xtplusone|\xt)q(\xt|\x_0)}{q(\xtplusone|\x_0)},
\end{align}
we can view the standard diffusion backward kernel \Cref{eq:diffbackwardint} as applying the DAPS kernel, but conditioning the sample on the previous state $\xtplusone$, which acts as an ``observation'' with likelihood $q(\xt|\xtplusone)$. We can thus generalize the kernel in \Cref{eq:diffbackwardint} by annealing this likelihood with the inverse temperature $\eta$, 
\begin{align}
    p_\theta^\eta(\xt|\xtplusone) 
    \propto \int 
    q(\xtplusone|\xt)^\eta q(\xt|\x_0)
    \delta_{\recon(\xtplusone)}(\x_0)
    d\x_0
\end{align}
which allows us to smoothly transition between the DAPS ($\eta=0$) and standard ($\eta=1$) backward kernels.

\paragraph{Likelihood}
Having defined the prior as the generalized DAPS prior $p_\theta^\eta(\xtt)$, we also need to incorporate the conditioning on the observation $\y$. 
A natural starting point would be to choose as targets
\begin{align}
    \gammat = p(\y|\xt)p_\theta^\eta(\xtt) \label{eq:smc_ideal_target},
\end{align}
as this for $t=0$ leads to a distribution with marginal $p_\theta^\eta(\x|\y)$. However, the likelihood $p(\y|\xt) = \int p(\y|\x_0) p_\theta(\x_0|\xt)d\x_0$ in \Cref{eq:smc_ideal_target} is not tractable for $t>0$, and needs to be approximated. As the reconstruction $\recon(\xtplusone)$ played a central role in the prior, we can utilize this also for our likelihood. \citet{song2023pseudoinverseguided} proposed to use a Gaussian approximation $\tilde p_\theta(\x_0|\xt) \eqdef \mathcal{N} \left(\x_0|\recon(\xt), \rho_t^2I\right)$, resulting in
\begin{align}
p(\y|\xt) 
&\approx \tilde p(\y|\xt)
=\mathcal{N}\bigl(\y|A\recon(\xt), \sigma_y^2 I + \rho_t^2 AA^T\bigr), \label{eq:tilde_likelihood}
\end{align}
as the likelihood is linear and Gaussian. From hereon we will use the notation $\tilde p_\theta$ for any distribution derived from the Gaussian approximation $\tilde p_\theta(\x_0|\xt)$. As for $t=0$ the likelihood is known, we can rely on the consistency of SMC to obtain asymptotically exact samples, even if using approximate likelihoods in the intermediate targets.

\paragraph{Putting it Together}
To summarize, we define a sequence of intermediate target distributions for SMC according to
\begin{align}
    \gammat 
    &= \tilde p(\y|\xt) p_\theta(\x_T)\prod_{s=t}^{T-1} p^\eta_\theta(\xs|\xsplusone) \nonumber \\
    &= \frac{\tilde p(\y|\xt)}{\tilde p(\y|\xtplusone)} p_\theta^\eta(\xt|\xtplusone)\gammatplusone,
\end{align}
where, for $t>0$,
\begin{align}
    \tilde p(\y|\xt) 
    &= \mathcal{N}\left(\y|A\recon(\xt), \sigma_y^2 I + \rho_t^2 AA^T\right) \label{eq:ddsmc_likelihood}
\end{align}
and, for $\eta=0$,\footnote{See \Cref{app:general_DAPS_prior} for an expression for general $\eta$.}
\begin{align}
    p_\theta^{\eta=0}(\xt|\xtplusone) 
    &= \mathcal{N}\left(\xt|\recon(\xtplusone),
    \sigma^2_t I\right).
    \label{eq:ddsmc_target_transition}
\end{align}

\begin{algorithm}[tb]
   \caption{Decoupled Diffusion SMC (DDSMC). All operations for $i=1, \dots, N$}
   \label{algo:ddsmc}
   \hspace*{\algorithmicindent} \textbf{Input:}
   Score model $s_\theta$, 
   measurement $\y$
   \\
   \hspace*{\algorithmicindent} \textbf{Output:} Sample $\x_0$
\begin{algorithmic}
\STATE Sample $\x_T^i \sim p(\x_T)$
\FOR{$t$ in $T \dots, 1$}
\STATE Predict $\hat \x_{0, t}^i = \recon(\xt^i)$
\STATE Compute $\tilde p(\y|\xt^i)$ \COMMENT{\Cref{eq:ddsmc_likelihood}}
\IF{$t=T$}
\STATE Set $\tilde w^i = \tilde p(\y |\x_T^i)$ 
\ELSE
\STATE Set $\tilde w^i = \frac{\tilde p(\y |\xt^i)p_\theta^\eta(\xt^i|\xtplusone^i)}{\tilde p(\y |\xtplusone^i) r_t(\xt^i|\xtplusone^i, \y)}$
\ENDIF
\STATE Compute  $w^i = \tilde w^i /\sum_{j=1}^N \tilde w^j$
\STATE Resample $\{\x_t^i, \hat \x_{0, t}^i\}_{i=1}^N$
\STATE Sample $\x_{t-1}^i \sim r_{t-1}(\x_{t-1}|\xt^i, \y)$ \COMMENT{\Cref{eq:ddsmc_proposal}, or \Cref{eq:general_eta_proposal} for general $\eta$.}
\ENDFOR
\STATE Compute $\tilde w^i = \frac{p(\y | \x_0^i)}{p(\y | \x_1^i)}$ and $w^i = \tilde w^i /\sum_{j=1}^N \tilde w^j$
\STATE Sample $\x_0\sim \text{Multinomial}(\{\x_0^j\}_{j=1}^N, \{w^j\}_{j=1}^N)$
\RETURN $\x_0$ \COMMENT{Or the full set of weighted samples depending on application}
\end{algorithmic}
\end{algorithm}

\subsection{Proposal}
\label{sec:ddsmc_proposal}
As the efficiency of the SMC algorithm in practice very much depends on the proposal, we will, as previous works on SMC (e.g., TDS \citep{wu_practical_2023} and MCGDiff \citep{cardoso_monte_2023-2}), use a proposal which incorporates information about the measurement $\y$. Again we are inspired by DAPS where the Gaussian approximation $\tilde p(\x_0|\xt)$ plays a central role. In their method, they use this approximation as a prior over $\x_0$, which together with the likelihood $p(\y|\x_0)$ form an (approximate) posterior
\begin{align}
    \tilde p_\theta(\x_0|\xtplusone, \y) \propto p(\y|\x_0)\tilde p_\theta(\x_0|\xtplusone).\label{eq:x0_posterior_general}
\end{align}
The DAPS method is designed for general likelihoods and makes use of Langevin dynamics to sample from this approximate posterior. However, in the linear-Gaussian case we can, similarly to \Cref{eq:tilde_likelihood}, obtain a closed form expression as
\begin{align}
    \tilde p_\theta(\x_0|\xtplusone, \y) = \mathcal{N}(\x_0|\tilde \mu_\theta^t(\xtplusone, \y), \mM^{-1}_{t+1}), \label{eq:x0_posterior_closed_form}
\end{align}
with $\tilde \mu_\theta^t(\xtplusone, \y) = \mM^{-1}_{t+1}\rvb_{t+1}$ and
\begin{subequations}
\label{eq:x0_posterior_mean_precisionMat}
\begin{align}
    \mM_{t+1} &= \frac{1}{\sigma_y^2}\mA^T\mA + \frac{1}{\rho_{t+1}^{2}}I, \\ 
    \rvb_{t+1} &= \frac{1}{\sigma_y^2}\mA^T\y + \frac{1}{\rho_{t+1}^2}\recon(\xtplusone).
\end{align}
\end{subequations}
Although this is an approximation of the true posterior, it offers an interesting venue for an SMC proposal by propagating a sample from the posterior forward in time. Assuming $\eta=0$ for notational brevity (see \cref{app:general_DAPS_prior} for the general expression) we can write this as 
\begin{align}
    r_t(\xt|\xtplusone, \y) = \int \tilde q(\xt|\x_0)\tilde p_\theta(\x_0|\xtplusone, \y)d\x_0. \label{eq:ddsmc_proposal_general}
\end{align}
This proposal is similar to one step of the generative procedure used in DAPS, where in their case the transition would correspond to using the diffusion forward kernel, i.e., $\tilde q(\xt|\x_0)=q(\xt|\x_0)$. We make a slight adjustment based on the following intuition. Considering the setting of non-informative measurements, we expect our posterior $p_\theta(\x_0|\y)$ to coincide with the prior. To achieve this, we could choose $\tilde q(\xt|\x_0)$ such that $r_t(\xt|\xtplusone) = p_\theta^0(\xt|\xtplusone)$. Using $q(\xt|\x_0)$ together with the prior $\tilde p_\theta(\x_0|\xtplusone)$ will, however, not match the covariance in $p^0_\theta(\xt|\xtplusone)$, since the Gaussian approximation $\tilde p_\theta(\x_0|\xtplusone)$ will inflate the variance by a term $\rho_{t+1}^2$.
To counter this effect, we instead opt for a covariance $\lambda_t^2I$ (i.e., $\tilde q(\xt|\x_0) = \mathcal{N}\left(\xt|\x_0, \lambda_t^2I\right)$) such that in the unconditional case, $r_t(\xt|\xtplusone) = p_\theta^0(\xt|\xtplusone)$. As everything is Gaussian, the marginalization in \Cref{eq:ddsmc_proposal_general} can be computed exactly, and we obtain our proposal for $t>0$ as
\begin{align}
    r_t(\xt|\xtplusone, \y) 
    = 
    \mathcal{N}\left( \xt|
    \tilde \mu_\theta^t(\xtplusone, \y), \lambda_{t}^2I + \mM_{t+1}^{-1}
    \right)  %
    \label{eq:ddsmc_proposal}
\end{align}
where $\lambda_t^2 = \sigma^2_t - \rho_{t+1}^2$ and, for $t=0$, $r_0(\x_0|\x_1, \y) = \delta_{\tilde \mu_\theta^0(\x_1, \y)}(\x_0)$. We can directly see that in the non-informative case (i.e., $\sigma_y \rightarrow \infty$), we will recover the prior as $\mM_{t+1}^{-1} = \rho_{t+1}^2I$ and $\rvb_{t+1} = \frac{1}{\rho_{t+1}^2}f_\theta(\xtplusone)$. 

If using the generalized DAPS prior from the previous section, we can also replace $\tilde q(\xt|\x_0)$ with $\tilde q_\eta(\xt|\xtplusone, \x_0)$ and make the corresponding matching of covariance.


\paragraph{Detailed Expressions for Diagonal $A$}
If first assuming that $A$ is diagonal in the sense that it is of shape $d_y \times d_x$ ($d_y \leq d_x$) with non-zero elements $(a_1, \dots, a_{d_y})$ only along the main diagonal, we can write out explicit expressions for the proposal. In this case, the covariance matrix $\mM_{t+1}^{-1}$ is diagonal with the $i$:th diagonal element ($i=1, \dots, d_x$) becoming
\begin{align}
    (\mM_{t+1}^{-1})_{ii} = 
    \begin{cases}
    \frac{\rho_{t+1}^2\sigma_y^2}{a_i^2\rho_{t+1}^2 + \sigma_y^2} & i\leq d_y\\
    \rho_{t+1}^2 & i>d_y
    \end{cases}.
\end{align}
This means that the covariance matrix in the proposal is diagonal with the elements $\sigma_t^2 - \rho_{t+1}^2 + \frac{\rho_{t+1}^2\sigma_y^2}{a_i^2\rho_{t+1}^2 + \sigma_y^2}$ for $i\leq d_y$, and $\sigma_t^2$ for $i> d_y$. Further, the mean of the $i$:th variable in the proposal becomes
\begin{multline}
    \tilde \mu_\theta^t(\xtplusone, \y)_i =  \\
    \begin{cases}
    \frac{a_i\rho_{t+1}^2}{a_i^2\rho_{t+1}^2 + \sigma_y^2} y_i + \frac{\sigma_y^2}{a_i^2\rho^2_{t+1} + \sigma_y^2}\recon(\xtplusone)_i & i\leq d_y \\
    \recon(\xtplusone)_i & i>d_y.
    \end{cases}
\end{multline}
We can see the effect of the decoupling clearly by considering the noise-less case, $\sigma_y=0$. In that case, we are always using the known observed value $(\x_0)_i=y_i/a_i$ as the mean, not taking the reconstruction, nor the previous value $(\xtplusone)_i$, into account. 

As the proposal is a multivariate Gaussian with diagonal covariance matrix, we can efficiently sample from it using independent samples from a standard Gaussian. We provide expressions for general choices of $\eta$ in \Cref{app:general_DAPS_prior}.

\paragraph{Non-diagonal $A$}
For a general non-diagonal $A$, the proposal at first glance looks daunting as it requires inverting $M_{t+1}$ which is a (potentially very large) non-diagonal matrix, and sampling from a multivariate Gaussian with non-diagonal covariance matrix (which requires a computationally expensive Cholesky decomposition). However, similar to previous work \citep{kawar_snips_2021, kawar_denoising_2022, cardoso_monte_2023-2}, by writing $A$ in terms of its singular value decomposition $A=USV^T$ where $U\in\R^{d_y\times d_y}$ and $V\in \R^{d_x\times d_x}$ are orthonormal matrices, and $S$ a $d_y \times d_x$-dimensional matrix with non-zero elements only on its main diagonal, we can multiply both sides of the measurement equation by $U^T$ to obtain
\begin{align}
    U^T \y = SV^T \x + \sigma_y U^T\epsilon.
\end{align}
By then defining $\y' = U^T\y$, $\x' = V^T \x$, $A'=S$, and $\epsilon' = U^T \eps \sim\mathcal{N}(0, UIU^T) = \mathcal{N}(0, I)$, we obtain the new measurement equation
\begin{align}
    \y' = A'\x' + \sigma_y \epsilon', \ \eps' \sim \mathcal{N}(0, I).
\end{align}
We can now run our DDSMC algorithm in this new basis and use the expressions from the diagonal case by replacing all variables with their corresponding primed versions, enabling efficient sampling also for non-diagonal $A$.

An algorithm outlining an implementation of DDSMC can be found in \Cref{algo:ddsmc}.

\begin{table*}[tb!]
\caption{Results on the Gaussian mixture model experiment when using DDSMC and reconstructing $\x_0$ using either Tweedie's formula ($\recon(\xtplusone) = \mathbb{E}[\x_0|\xt]$) or solving the PF-ODE. Metric is the sliced Wasserstein distance between true posterior samples and samples from DDSMC, averaged over 20 seeds with $95 \%$ CLT confidence intervals. }
\label{tab:gmm-table-ddsmc}
\vskip 0.15in
\begin{center}
\begin{small}
\begin{sc}
\begin{tabular}{cccccccc}
\toprule
                     &       & \multicolumn{3}{c}{Tweedie}                                  & \multicolumn{3}{c}{PF-ODE}                 \\
                     \cmidrule(lr){3-5}                                                 \cmidrule(lr){6-8}
$d_x$                & $d_y$ & $\eta=0.0$         & $\eta=0.5$         & $\eta=1.0$         & $\eta=0.0$         & $\eta=0.5$         & $\eta=1.0$          \\
\midrule
\multirow{3}{*}{8}   & 1     & $1.90\pm0.48$      & $1.78\pm0.44$      & $1.57\pm0.43$      & $1.15\pm0.24$      & $0.88\pm0.33$      & $1.01\pm0.39$       \\
                     & 2     & $0.69\pm0.27$      & $0.64\pm0.29$      & $0.50\pm0.27$      & $0.56\pm0.23$      & $0.34\pm0.20$      & $0.39\pm0.22$       \\
                     & 4     & $0.28\pm0.04$      & $0.22\pm0.03$      & $0.13\pm0.05$      & $0.21\pm0.10$      & $0.09\pm0.08$      & $0.17\pm0.06$       \\ \midrule
\multirow{3}{*}{80}  & 1     & $1.26\pm0.34$      & $1.07\pm0.31$      & $0.96\pm0.28$      & $0.69\pm0.17$      & $0.50\pm0.16$      & $0.77\pm0.12$       \\
                     & 2     & $1.14\pm0.40$      & $0.92\pm0.36$      & $0.81\pm0.35$      & $0.69\pm0.36$      & $0.37\pm0.20$      & $0.69\pm0.15$       \\
                     & 4     & $0.66\pm0.29$      & $0.50\pm0.27$      & $0.44\pm0.22$      & $0.41\pm0.23$      & $0.21\pm0.13$      & $0.54\pm0.04$       \\ \midrule
\multirow{3}{*}{800} & 1     & $1.73\pm0.54$      & $1.64\pm0.61$      & $1.86\pm0.63$      & $1.37\pm0.43$      & $1.68\pm0.52$      & $2.09\pm0.51$       \\
                     & 2     & $1.06\pm0.56$      & $0.90\pm0.65$      & $1.34\pm0.68$      & $0.81\pm0.40$      & $1.17\pm0.65$      & $1.76\pm0.67$       \\
                     & 4     & $0.35\pm0.08$      & $0.16\pm0.10$      & $0.56\pm0.19$      & $0.23\pm0.06$      & $0.47\pm0.22$      & $1.24\pm0.47$       \\
\bottomrule
\end{tabular}
\end{sc}
\end{small}
\end{center}
\vskip -0.1in
\end{table*}



\begin{figure*}[tb!]
        \centering
        \includegraphics[width=\textwidth]{figures/ddsmc_gmm/xdim=800_ydim1_single_row=True_left=True.pdf}\vspace{-2ex}%
\caption{Samples from DDSMC from the GMM experiments ($d_x=800$ and $d_y=1$) using either Tweedie's formula (three left-most figures) or the solution of the PF-ODE (three right-most figures) as a reconstruction, and using different values of $\eta$ in the generalized DAPS prior. Blue samples are from the posterior, while red samples are from DDSMC. More examples can be found in \Cref{app:gmm}.}
\label{fig:ddsmc-gmm} 
\end{figure*}



\subsection{D3SMC -- A Discrete Version}
We also extend our DDSMC algorithm to a discrete setting which we call Discrete Decoupled Diffusion SMC (D3SMC). Using $\x$ to denote a one-hot encoding of a single variable, we target the case where the measurement can be described by a transition matrix $Q_y$ as
\begin{align}
    p(\y|\x_0) = \text{Categorical}(\rvp = \x_0 Q_y), \label{eq:discrete_meas}
\end{align}
and when the data consists of $D$ variables, the measurement factorizes over these variables (i.e., for each of the $D$ variables, we have a corresponding measurement). This setting includes both inpainting (a variable is observed with some probability, otherwise it is in an ``unknown'' state) and denoising (a variable randomly transitions into a different state). To tackle this problem, we explore the D3PM-uniform model \citep{austin_structured_2021} and derive a discrete analogy to DDSMC which uses the particular structure of D3PM. We elaborate on the D3PM model in general and our D3SMC algorithm in particular in \Cref{app:d3smc}.

\section{Related Work}
\label{sec:related_work}
\paragraph{SMC and Diffusion Models} 
The closest related SMC methods for Bayesian inverse problems with diffusion priors are the Twisted Diffusion Sampler (TDS) \citep{wu_practical_2023} and Monte Carlo Guided Diffusion (MCGDiff) \citep{cardoso_monte_2023-2}. TDS is a general approach for solving inverse problems, while MCGDiff specifically focuses on the linear-Gaussian setting. These methods differ in the choices of intermediate targets and proposals.
TDS makes use of the reconstruction network to approximate the likelihood at time $t$ as $p(\y \mid \x_0 = f_\theta(\x_{t}))$ and then add the score of this approximate likelihood as a drift in the transition kernel. This requires differentiating through the reconstruction model w.r.t. $\x_{t}$, which can incur a significant computational overhead. MCGDiff instead uses the forward diffusion model to push the observation $\y$ "forward in time". Specifically, they introduce a potential function at time $t$ which can be seen as a likelihood corresponding to the observation model $\hat \y_t = A\x_t$, where $\hat \y_t$ is a noised version (according to the forward model at time $t$) of the original observation $\y$. 
DDSMC differs from both of these methods, and is conceptually based on reconstructing $\x_0$ from the current state $\x_t$, performing an explicit conditioning on $\y$ at time $t=0$, and then pushing the \emph{posterior distribution} forward to time $t-1$ using the forward model.
We further elaborate on the differences between TDS, MCGDiff, and DDSMC in \Cref{app:smc_comparison}~(see also the discussion by \citet{Zhao2024rsta}). 

SMCDiff \citep{trippe_diffusion_2023} and FPS \citep{dou_diffusion_2023-1} are two other SMC algorithms that target posterior sampling with diffusion priors, but these rely on the assumption that the learned backward process is an exact reversal of the forward process, and are therefore not consistent in general. SMC has also been used as a type of \emph{discriminator} guidance \cite{kim_refining_2023-1} of diffusion models in both the continuous \cite{liu_correcting_2024} and discrete \cite{ekstrom_kelvinius_discriminator_2024} setting.

\paragraph{Posterior Sampling with Diffusion Priors}
The closest non-SMC work to ours is of course DAPS \cite{zhang_improving_2024}, which we build upon, but also generalize in several ways. DDSMC is also related to $\Pi$GDM \citep{song2023pseudoinverseguided}, which introduces the Gaussian approximation used in \Cref{eq:tilde_likelihood,eq:x0_posterior_general}. %
Other proposed methods include DDRM \cite{kawar_denoising_2022} which defines a task-specific \emph{conditional} diffusion process which depends on a reconstruction $\recon(\xt)$, but where the optimal solution can be approximated with a model trained on the regular unconditional task. Recently, \citet{janati_divide-and-conquer_2024} proposed a method referred to as DCPS, which also builds on the notion of intermediate targets, but not within an SMC framework. Instead, to sample from these intermediate targets, they make use of Langevin sampling and a variational approximation that is optimized with stochastic gradient descent within each step of the generative model. 
All of these methods include various approximations, and contrary to DDSMC, none of them provides a consistent approximation of a given posterior. 


\section{Experiments}
\subsection{Gaussian Mixture Model}
\label{sec:gmm_exp}
We first experiment on synthetic data, and use the Gaussian mixture model problem from \citet{cardoso_monte_2023-2}. Here, the prior on $\x$ is a Gaussian mixture, and both the posterior and score are therefore known on closed form\footnote{This includes using the DDPM \citep{ho_denoising_2020-2} or "Variance-preserving" formulation of a diffusion model. We have included a derivation of DDSMC for this setting in \Cref{app:ddsmc_ddpm}} (we give more details in \Cref{app:gmm}), enabling us verify the efficiency of DDSMC while ablating all other errors. As a metric, we use the sliced Wasserstein distance \cite{flamary_pot_2021} between 10k samples from the true posterior and each sampling algorithm. 


\begin{table*}[tb!]
\caption{Comparison of DDSMC with other methods in the Gaussian mixture setting. Numbers for DDSMC are the best numbers from \Cref{tab:gmm-table-ddsmc}. *All methods have been run for 20 different seeds, but TDS shows instability and crashes, meaning that their numbers are computed over fewer runs.}
\label{tab:gmm-table-other}
\vskip 0.15in
\begin{center}
\begin{small}
\begin{sc}
\begin{tabular}{cccccccc}
\toprule
$d_x$                & $d_y$ & DDSMC              & MCGDiff       &  TDS*            & DCPS           & DDRM         & DAPS  \\
\midrule
\multirow{3}{*}{8}   & 1     & $0.88\pm0.33$      & $1.79\pm0.54$ & $9.72\pm9.89$  & $2.51\pm1.02$  & $3.83\pm1.05$ & $5.63\pm0.90$ \\
                     & 2     & $0.34\pm0.20$      & $0.80\pm0.41$ & $5.23\pm2.57$   & $1.35\pm0.70$  & $2.25\pm0.92$ & $5.93\pm1.16$ \\
                     & 4     & $0.09\pm0.08$      & $0.26\pm0.21$ & $3.02\pm2.58$   & $0.44\pm0.25$  & $0.55\pm0.29$ & $4.85\pm1.34$ \\ \midrule
\multirow{3}{*}{80}  & 1     & $0.50\pm0.16$      & $1.06\pm0.36$ & $7.37\pm7.62$   & $1.19\pm0.41$  & $5.19\pm1.07$ & $6.85\pm1.16$ \\
                     & 2     & $0.37\pm0.20$      & $1.04\pm0.49$ & $2.63\pm1.40$   & $1.10\pm0.55$  & $5.62\pm1.09$ & $8.49\pm0.92$ \\
                     & 4     & $0.21\pm0.13$      & $0.80\pm0.42$ & $1.47\pm1.42$   & $0.56\pm0.24$  & $4.95\pm1.25$ & $9.04\pm0.74$ \\ \midrule
\multirow{3}{*}{800} & 1     & $1.37\pm0.43$      & $1.64\pm0.48$ & $2.45\pm0.96$   & $2.45\pm0.58$  & $7.15\pm1.11$ & $7.03\pm1.20$ \\
                     & 2     & $0.81\pm0.40$      & $1.29\pm0.66$ & $2.84\pm1.32$   & $2.80\pm0.91$  & $8.21\pm0.88$ & $8.31\pm1.01$ \\
                     & 4     & $0.16\pm0.10$      & $1.21\pm0.94$ & $3.58\pm3.18$   & $1.84\pm0.76$  & $8.66\pm0.87$ & $9.21\pm0.83$ \\
\bottomrule
\end{tabular}
\end{sc}
\end{small}
\end{center}
\vskip -0.1in
\end{table*}

\begin{figure*}[tb!]
        \centering
        \includegraphics[width=\textwidth]{figures/ddsmc_gmm/comparison_xdim=800_ydim=1.pdf}\vspace{-2ex}%
\caption{Qualitative comparison between DDSMC (using PF-ODE as reconstruction and $\eta=0$) and other methods on the GMM experiments, $d_x=800$ and $d_y=1$. More qualitative comparisons can be found in \Cref{app:gmm}}
\label{fig:gmm-comparison} 
\end{figure*}




We start with investigating the influence of $\eta$ and the reconstruction function $\recon$ in \Cref{tab:gmm-table-ddsmc}. 
We run DDSMC with $N=256$ particles, and use $T=20$ steps in the generative process. 
As a reconstruction, we compare Tweedie's formula and the DDIM ODE solver \cite{song_denoising_2021}, see \Cref{eq:ddim_ode} in the \suppmat, where we solve the ODE for the ``remaining steps" $t, t-1, \dots, 0$. In the table, we see that using Tweedie's reconstruction requires a larger value of $\eta$. This can be explained by the fact that $\eta=0$ (DAPS prior) requires exact samples from $p_\theta(\x_0|\xt)$ for the DAPS prior to agree with the original prior (see \Cref{app:proofs}), and this distribution is more closely approximated using the ODE reconstruction. For higher dimensions (i.e., $d_x=800$) and using $\eta >0$, Tweedie's formula works slightly better than the ODE solver. Qualitatively, we find that using a smaller $\eta$ and/or using Tweedie's reconstruction tends to concentrate the samples around the different modes, see \Cref{fig:ddsmc-gmm} for an example of $d_x=800$ and $d_y=1$ (additional examples in \Cref{app:gmm}). This plot also shows how, in high dimensions, using a smaller $\eta$ is preferable, which could be attributed to lower $\eta$ enabling larger updates necessary in higher dimensions. 

Next, we compare DDSMC with other methods, both SMC-based (MCGDiff and TDS) and non-SMC-based (DDRM, DCPS, DAPS). For all SMC methods, we use 256 particles, and accelerated sampling according to DDIM \cite{song_denoising_2021} with 20 steps, which we also used for DAPS to enabling evaluating the benefits of our generalization of the DAPS method without confounding the results with the effect of common hyperparameters. For DDRM and DCPS, however, we used \thsnd{1} steps. More details are available in \Cref{app:gmm}. We see quantitatively in \Cref{tab:gmm-table-other} that DDSMC outperforms all other methods, even when using Tweedie's reconstruction. A qualitative inspection of generated samples in \Cref{fig:gmm-comparison} shows how DDSMC is the most resistant method to sampling from spurious modes, while DAPS, which we have based our method on, struggles to sample from the posterior. These are general trends we see when repeating with different seeds, see \Cref{app:gmm}.

\subsection{Image Restoration}
\begin{figure*}[h]
    \centering
    \begin{subfigure}[t]{0.083\linewidth}
        \centering
        \includegraphics[width=\linewidth]{figures/ffhq_main/main_gt_grid.png}
    \end{subfigure}%
    \hfill
    \begin{subfigure}[t]{0.083\linewidth}
        \centering
        \includegraphics[width=\linewidth]{figures/ffhq_main/outpainting_half_meas_grid.png}
    \end{subfigure}%
    \hfill
    \begin{subfigure}[t]{0.25\linewidth}
        \centering
        \includegraphics[width=\linewidth]{figures/ffhq_main/ddsmc_Tweedie_outpainting_half_grid.png}
    \end{subfigure}%
    \hfill
    \begin{subfigure}[t]{0.25\linewidth}
        \centering
        \includegraphics[width=\linewidth]{figures/ffhq_main/ddsmc_ODE_outpainting_half_grid.png}
    \end{subfigure}%
    \hfill
    \begin{subfigure}[t]{0.083\linewidth}
        \centering
        \includegraphics[width=\linewidth]{figures/ffhq_main/dcps_outpainting_half_grid.png}
    \end{subfigure}%
    \begin{subfigure}[t]{0.083\linewidth}
        \centering
        \includegraphics[width=\linewidth]{figures/ffhq_main/ddrm_outpainting_half_grid.png}
    \end{subfigure}%
    \begin{subfigure}[t]{0.083\linewidth}
        \centering
        \includegraphics[width=\linewidth]{figures/ffhq_main/daps_outpainting_half_grid.png}
    \end{subfigure}%
    \begin{subfigure}[t]{0.083\linewidth}
        \centering
        \includegraphics[width=\linewidth]{figures/ffhq_main/mcgdiff_outpainting_half_grid.png}
    \end{subfigure}%
\caption{Examples of generated images for the outpainting task. The DDSMC samples are ordered with $\eta=0$ to the left, $\eta=0.5$ in the middle, and $\eta=1$ to the right. }
\label{fig:ffhq_main}
\end{figure*}

\begin{table}[tb!]
\caption{LPIPS results on FFHQ experiments. The noise level is $\sigma_y=0.05$, and the tasks are inpainting a box in the \textbf{middle}, outpainting right \textbf{half} of the image, and \textbf{s}uper-\textbf{r}esolution $4\times$. \textbf{Lower is better.}}
\label{tab:image-table-lpips}
\vskip 0.15in
\begin{center}
\begin{small}
\begin{sc}
\begin{tabular}{clccc}
\toprule
&            & Middle & Half & SR4 \\
            \midrule\midrule
& DDRM        & 0.04   & 0.26 & 0.21 \\
& DCPS        & 0.03   & 0.21 & 0.10 \\
& DAPS        & 0.05   & 0.25 & 0.15 \\
& MCGDiff     & 0.12   & 0.33 & 0.15 \\
\midrule
\multirow{3}{*}{\rotatebox[origin]{0}{Tweedie}}
& DDSMC-$0.0$ & 0.07  & 0.27 & 0.27  \\ 
& DDSMC-$0.5$ & 0.07  & 0.28 & 0.20  \\
& DDSMC-$1.0$ & 0.05  & 0.24 & 0.14  \\
\midrule
\multirow{3}{*}{\rotatebox[origin]{0}{ODE}}
& DDSMC-$0.0$ & 0.05  & 0.24 & 0.21  \\ 
& DDSMC-$0.5$ & 0.05  & 0.24 & 0.15  \\
& DDSMC-$1.0$ & 0.08  & 0.40 & 0.36   \\
\bottomrule

\end{tabular}
\end{sc}
\end{small}
\end{center}
\vskip -0.1in
\end{table}

We now turn our attention to the problem of image restoration, and use our method for inpainting, outpainting, and super-resolution on the FFHQ \citep{karras_style-based_2019} dataset, using a pretrained model by \citet{chung_diffusion_2023}. We implement DDSMC in the codebase of DCPS\footnote{\url{https://github.com/Badr-MOUFAD/dcps/}}, and follow their protocol of using 100 images from the validation set and compute LPIPS \citep{zhang_unreasonable_2018} as a quantitative metric. The results can be found in \Cref{tab:image-table-lpips}, in addition to numbers for DDRM, DCPS, DAPS, and MCGDiff. We used 5 particles for our method, and when using the PF-ODE as reconstruction, we used 5 ODE steps. Further implementation details are available in \Cref{app:image}. It should be noted that LPIPS measures perceptual similarity between the sampled image and the ground truth, which is not the same as a measurement of how well the method samples from the true posterior. Specifically, it does not say anything about diversity of samples or how well the model captures the posterior uncertainty. Nevertheless, the numbers indicate that we perform on par with previous methods. Notably, MCGDiff, which is also an SMC method and have the same asymptotic guarantees in terms of posterior sampling, performs similar or worse to DDSMC. 

Inspecting the generated images, we can see a visible effect when altering the reconstruction function and $\eta$, see \Cref{fig:ffhq_main} and more examples in \Cref{app:image}. It seems like increasing $\eta$ and/or using the PF-ODE as a reconstruction function adds more details to the image. With the GMM experiments in mind, where we saw that lower $\eta$ and using Tweedie's formula as reconstruction function made samples more concentrated around the modes, it can be argued that there is a similar effect here: using Tweedie's formula and lower $\eta$ lead to sampled images closer to a ``mode'' of images, meaning details are averaged out. On the other hand, changing to ODE-reconstruction and increasing $\eta$ further away from the mode, which corresponds to more details in the images. For the case of ODE and $\eta=1.0$, we see clear artifacts which explains the poor quantitative results.

\subsection{Discrete Data}
\begin{figure}[tb!]
        \centering
        \includegraphics[width=\columnwidth]{figures/bmnist/bmnist_10_5particles.png}
\caption{\label{fig:bmnist} Qualitative results on binary MNIST using the discrete counterpart of DDSMC, D3SMC. Top is the ground truth, middle is measurement, and bottom is sampled image. More results can be found in \Cref{app:bmnist}}
\end{figure}

As a proof of concept of our discrete algorithm D3SMC (see detailed description of the algorithm in \Cref{app:d3smc}), we try denoising on the binary MNIST dataset (i.e., each pixel is either 0 or 1), cropped to $24 \times 24$ pixels to remove padding. As a backbone neural network, we used a U-Net \citep{ronneberger_u-net_2015} trained with cross-entropy loss. We use a measurement model $\y = \x Q_y$ with $Q_y = (1-\beta_y) I + \beta_y \mathbbm{1}\mathbbm{1}^T/d$ and $\beta_y=0.6$, i.e., for the binary case, this means that the observed pixel has the same value as the original image with probability $0.7$, and changed with probability $0.3$. We present qualitative results for running D3SMC with $N=5$ particles in \Cref{fig:bmnist}, and more qualitative results can be found in \Cref{app:bmnist}. While the digits are often recovered, there are cases when they are not (e.g., a 4 becoming a 9, and a 9 becoming a 7). However, looking at multiple draws in the \suppmat, we can see how the model in such cases can sample different digits, suggesting a multi-modal nature of the posterior. 

\section{Discussion \& Conclusion}
In this paper, we have designed an SMC algorithm, DDSMC, for posterior sampling with diffusion priors which we also extended to discrete data. The method is based on decoupled diffusion, which we generalize by introducing a hyperparameter that allows bridging between full decoupling and standard diffusion (no decoupling). We demonstrate the superior performance of DDSMC compared to the state-of-the-art on a synthetic problem, which enables a quantitative evaluation of how well the different methods approximate the true posterior. We additionally test DDSMC on image reconstruction where it performs on par with previous methods. Specifically, it performs slightly better than the alternative SMC-based method MCGDiff, while slightly underperforming compared to DCPS. However, while LPIPS indeed is a useful quantitative metric for evaluating image reconstruction, it does not inherently capture how well a method approximates the true posterior, which is the aim of our method. We found that the methods performing particularly well on image reconstruction struggle on the synthetic task, highlighting a gap between perceived image quality and their ability to approximate the true posterior. As DDSMC still performs on par with previous work on image reconstruction while at the same time showing excellent performance on the GMM task, we view DDSMC as a promising method for bridging between solving challenging high-dimensional inverse problems and maintaining exact posterior sampling capabilities.
 

\section*{Acknowledgments}
This research was supported by 
Swedish Research Council (VR) grant no. 2020-04122, 2024-05011,
the Wallenberg AI, Autonomous Systems and Software Program (WASP) funded by the Knut and Alice Wallenberg Foundation,
and
the Excellence Center at Linköping--Lund in Information Technology (ELLIIT).
Computations were enabled by the Berzelius resource provided by the Knut and Alice Wallenberg Foundation at the National Supercomputer Centre.


\bibliography{ddsmc_references}


\newpage
\appendix
\onecolumn
\subsection{Lloyd-Max Algorithm}
\label{subsec:Lloyd-Max}
For a given quantization bitwidth $B$ and an operand $\bm{X}$, the Lloyd-Max algorithm finds $2^B$ quantization levels $\{\hat{x}_i\}_{i=1}^{2^B}$ such that quantizing $\bm{X}$ by rounding each scalar in $\bm{X}$ to the nearest quantization level minimizes the quantization MSE. 

The algorithm starts with an initial guess of quantization levels and then iteratively computes quantization thresholds $\{\tau_i\}_{i=1}^{2^B-1}$ and updates quantization levels $\{\hat{x}_i\}_{i=1}^{2^B}$. Specifically, at iteration $n$, thresholds are set to the midpoints of the previous iteration's levels:
\begin{align*}
    \tau_i^{(n)}=\frac{\hat{x}_i^{(n-1)}+\hat{x}_{i+1}^{(n-1)}}2 \text{ for } i=1\ldots 2^B-1
\end{align*}
Subsequently, the quantization levels are re-computed as conditional means of the data regions defined by the new thresholds:
\begin{align*}
    \hat{x}_i^{(n)}=\mathbb{E}\left[ \bm{X} \big| \bm{X}\in [\tau_{i-1}^{(n)},\tau_i^{(n)}] \right] \text{ for } i=1\ldots 2^B
\end{align*}
where to satisfy boundary conditions we have $\tau_0=-\infty$ and $\tau_{2^B}=\infty$. The algorithm iterates the above steps until convergence.

Figure \ref{fig:lm_quant} compares the quantization levels of a $7$-bit floating point (E3M3) quantizer (left) to a $7$-bit Lloyd-Max quantizer (right) when quantizing a layer of weights from the GPT3-126M model at a per-tensor granularity. As shown, the Lloyd-Max quantizer achieves substantially lower quantization MSE. Further, Table \ref{tab:FP7_vs_LM7} shows the superior perplexity achieved by Lloyd-Max quantizers for bitwidths of $7$, $6$ and $5$. The difference between the quantizers is clear at 5 bits, where per-tensor FP quantization incurs a drastic and unacceptable increase in perplexity, while Lloyd-Max quantization incurs a much smaller increase. Nevertheless, we note that even the optimal Lloyd-Max quantizer incurs a notable ($\sim 1.5$) increase in perplexity due to the coarse granularity of quantization. 

\begin{figure}[h]
  \centering
  \includegraphics[width=0.7\linewidth]{sections/figures/LM7_FP7.pdf}
  \caption{\small Quantization levels and the corresponding quantization MSE of Floating Point (left) vs Lloyd-Max (right) Quantizers for a layer of weights in the GPT3-126M model.}
  \label{fig:lm_quant}
\end{figure}

\begin{table}[h]\scriptsize
\begin{center}
\caption{\label{tab:FP7_vs_LM7} \small Comparing perplexity (lower is better) achieved by floating point quantizers and Lloyd-Max quantizers on a GPT3-126M model for the Wikitext-103 dataset.}
\begin{tabular}{c|cc|c}
\hline
 \multirow{2}{*}{\textbf{Bitwidth}} & \multicolumn{2}{|c|}{\textbf{Floating-Point Quantizer}} & \textbf{Lloyd-Max Quantizer} \\
 & Best Format & Wikitext-103 Perplexity & Wikitext-103 Perplexity \\
\hline
7 & E3M3 & 18.32 & 18.27 \\
6 & E3M2 & 19.07 & 18.51 \\
5 & E4M0 & 43.89 & 19.71 \\
\hline
\end{tabular}
\end{center}
\end{table}

\subsection{Proof of Local Optimality of LO-BCQ}
\label{subsec:lobcq_opt_proof}
For a given block $\bm{b}_j$, the quantization MSE during LO-BCQ can be empirically evaluated as $\frac{1}{L_b}\lVert \bm{b}_j- \bm{\hat{b}}_j\rVert^2_2$ where $\bm{\hat{b}}_j$ is computed from equation (\ref{eq:clustered_quantization_definition}) as $C_{f(\bm{b}_j)}(\bm{b}_j)$. Further, for a given block cluster $\mathcal{B}_i$, we compute the quantization MSE as $\frac{1}{|\mathcal{B}_{i}|}\sum_{\bm{b} \in \mathcal{B}_{i}} \frac{1}{L_b}\lVert \bm{b}- C_i^{(n)}(\bm{b})\rVert^2_2$. Therefore, at the end of iteration $n$, we evaluate the overall quantization MSE $J^{(n)}$ for a given operand $\bm{X}$ composed of $N_c$ block clusters as:
\begin{align*}
    \label{eq:mse_iter_n}
    J^{(n)} = \frac{1}{N_c} \sum_{i=1}^{N_c} \frac{1}{|\mathcal{B}_{i}^{(n)}|}\sum_{\bm{v} \in \mathcal{B}_{i}^{(n)}} \frac{1}{L_b}\lVert \bm{b}- B_i^{(n)}(\bm{b})\rVert^2_2
\end{align*}

At the end of iteration $n$, the codebooks are updated from $\mathcal{C}^{(n-1)}$ to $\mathcal{C}^{(n)}$. However, the mapping of a given vector $\bm{b}_j$ to quantizers $\mathcal{C}^{(n)}$ remains as  $f^{(n)}(\bm{b}_j)$. At the next iteration, during the vector clustering step, $f^{(n+1)}(\bm{b}_j)$ finds new mapping of $\bm{b}_j$ to updated codebooks $\mathcal{C}^{(n)}$ such that the quantization MSE over the candidate codebooks is minimized. Therefore, we obtain the following result for $\bm{b}_j$:
\begin{align*}
\frac{1}{L_b}\lVert \bm{b}_j - C_{f^{(n+1)}(\bm{b}_j)}^{(n)}(\bm{b}_j)\rVert^2_2 \le \frac{1}{L_b}\lVert \bm{b}_j - C_{f^{(n)}(\bm{b}_j)}^{(n)}(\bm{b}_j)\rVert^2_2
\end{align*}

That is, quantizing $\bm{b}_j$ at the end of the block clustering step of iteration $n+1$ results in lower quantization MSE compared to quantizing at the end of iteration $n$. Since this is true for all $\bm{b} \in \bm{X}$, we assert the following:
\begin{equation}
\begin{split}
\label{eq:mse_ineq_1}
    \tilde{J}^{(n+1)} &= \frac{1}{N_c} \sum_{i=1}^{N_c} \frac{1}{|\mathcal{B}_{i}^{(n+1)}|}\sum_{\bm{b} \in \mathcal{B}_{i}^{(n+1)}} \frac{1}{L_b}\lVert \bm{b} - C_i^{(n)}(b)\rVert^2_2 \le J^{(n)}
\end{split}
\end{equation}
where $\tilde{J}^{(n+1)}$ is the the quantization MSE after the vector clustering step at iteration $n+1$.

Next, during the codebook update step (\ref{eq:quantizers_update}) at iteration $n+1$, the per-cluster codebooks $\mathcal{C}^{(n)}$ are updated to $\mathcal{C}^{(n+1)}$ by invoking the Lloyd-Max algorithm \citep{Lloyd}. We know that for any given value distribution, the Lloyd-Max algorithm minimizes the quantization MSE. Therefore, for a given vector cluster $\mathcal{B}_i$ we obtain the following result:

\begin{equation}
    \frac{1}{|\mathcal{B}_{i}^{(n+1)}|}\sum_{\bm{b} \in \mathcal{B}_{i}^{(n+1)}} \frac{1}{L_b}\lVert \bm{b}- C_i^{(n+1)}(\bm{b})\rVert^2_2 \le \frac{1}{|\mathcal{B}_{i}^{(n+1)}|}\sum_{\bm{b} \in \mathcal{B}_{i}^{(n+1)}} \frac{1}{L_b}\lVert \bm{b}- C_i^{(n)}(\bm{b})\rVert^2_2
\end{equation}

The above equation states that quantizing the given block cluster $\mathcal{B}_i$ after updating the associated codebook from $C_i^{(n)}$ to $C_i^{(n+1)}$ results in lower quantization MSE. Since this is true for all the block clusters, we derive the following result: 
\begin{equation}
\begin{split}
\label{eq:mse_ineq_2}
     J^{(n+1)} &= \frac{1}{N_c} \sum_{i=1}^{N_c} \frac{1}{|\mathcal{B}_{i}^{(n+1)}|}\sum_{\bm{b} \in \mathcal{B}_{i}^{(n+1)}} \frac{1}{L_b}\lVert \bm{b}- C_i^{(n+1)}(\bm{b})\rVert^2_2  \le \tilde{J}^{(n+1)}   
\end{split}
\end{equation}

Following (\ref{eq:mse_ineq_1}) and (\ref{eq:mse_ineq_2}), we find that the quantization MSE is non-increasing for each iteration, that is, $J^{(1)} \ge J^{(2)} \ge J^{(3)} \ge \ldots \ge J^{(M)}$ where $M$ is the maximum number of iterations. 
%Therefore, we can say that if the algorithm converges, then it must be that it has converged to a local minimum. 
\hfill $\blacksquare$


\begin{figure}
    \begin{center}
    \includegraphics[width=0.5\textwidth]{sections//figures/mse_vs_iter.pdf}
    \end{center}
    \caption{\small NMSE vs iterations during LO-BCQ compared to other block quantization proposals}
    \label{fig:nmse_vs_iter}
\end{figure}

Figure \ref{fig:nmse_vs_iter} shows the empirical convergence of LO-BCQ across several block lengths and number of codebooks. Also, the MSE achieved by LO-BCQ is compared to baselines such as MXFP and VSQ. As shown, LO-BCQ converges to a lower MSE than the baselines. Further, we achieve better convergence for larger number of codebooks ($N_c$) and for a smaller block length ($L_b$), both of which increase the bitwidth of BCQ (see Eq \ref{eq:bitwidth_bcq}).


\subsection{Additional Accuracy Results}
%Table \ref{tab:lobcq_config} lists the various LOBCQ configurations and their corresponding bitwidths.
\begin{table}
\setlength{\tabcolsep}{4.75pt}
\begin{center}
\caption{\label{tab:lobcq_config} Various LO-BCQ configurations and their bitwidths.}
\begin{tabular}{|c||c|c|c|c||c|c||c|} 
\hline
 & \multicolumn{4}{|c||}{$L_b=8$} & \multicolumn{2}{|c||}{$L_b=4$} & $L_b=2$ \\
 \hline
 \backslashbox{$L_A$\kern-1em}{\kern-1em$N_c$} & 2 & 4 & 8 & 16 & 2 & 4 & 2 \\
 \hline
 64 & 4.25 & 4.375 & 4.5 & 4.625 & 4.375 & 4.625 & 4.625\\
 \hline
 32 & 4.375 & 4.5 & 4.625& 4.75 & 4.5 & 4.75 & 4.75 \\
 \hline
 16 & 4.625 & 4.75& 4.875 & 5 & 4.75 & 5 & 5 \\
 \hline
\end{tabular}
\end{center}
\end{table}

%\subsection{Perplexity achieved by various LO-BCQ configurations on Wikitext-103 dataset}

\begin{table} \centering
\begin{tabular}{|c||c|c|c|c||c|c||c|} 
\hline
 $L_b \rightarrow$& \multicolumn{4}{c||}{8} & \multicolumn{2}{c||}{4} & 2\\
 \hline
 \backslashbox{$L_A$\kern-1em}{\kern-1em$N_c$} & 2 & 4 & 8 & 16 & 2 & 4 & 2  \\
 %$N_c \rightarrow$ & 2 & 4 & 8 & 16 & 2 & 4 & 2 \\
 \hline
 \hline
 \multicolumn{8}{c}{GPT3-1.3B (FP32 PPL = 9.98)} \\ 
 \hline
 \hline
 64 & 10.40 & 10.23 & 10.17 & 10.15 &  10.28 & 10.18 & 10.19 \\
 \hline
 32 & 10.25 & 10.20 & 10.15 & 10.12 &  10.23 & 10.17 & 10.17 \\
 \hline
 16 & 10.22 & 10.16 & 10.10 & 10.09 &  10.21 & 10.14 & 10.16 \\
 \hline
  \hline
 \multicolumn{8}{c}{GPT3-8B (FP32 PPL = 7.38)} \\ 
 \hline
 \hline
 64 & 7.61 & 7.52 & 7.48 &  7.47 &  7.55 &  7.49 & 7.50 \\
 \hline
 32 & 7.52 & 7.50 & 7.46 &  7.45 &  7.52 &  7.48 & 7.48  \\
 \hline
 16 & 7.51 & 7.48 & 7.44 &  7.44 &  7.51 &  7.49 & 7.47  \\
 \hline
\end{tabular}
\caption{\label{tab:ppl_gpt3_abalation} Wikitext-103 perplexity across GPT3-1.3B and 8B models.}
\end{table}

\begin{table} \centering
\begin{tabular}{|c||c|c|c|c||} 
\hline
 $L_b \rightarrow$& \multicolumn{4}{c||}{8}\\
 \hline
 \backslashbox{$L_A$\kern-1em}{\kern-1em$N_c$} & 2 & 4 & 8 & 16 \\
 %$N_c \rightarrow$ & 2 & 4 & 8 & 16 & 2 & 4 & 2 \\
 \hline
 \hline
 \multicolumn{5}{|c|}{Llama2-7B (FP32 PPL = 5.06)} \\ 
 \hline
 \hline
 64 & 5.31 & 5.26 & 5.19 & 5.18  \\
 \hline
 32 & 5.23 & 5.25 & 5.18 & 5.15  \\
 \hline
 16 & 5.23 & 5.19 & 5.16 & 5.14  \\
 \hline
 \multicolumn{5}{|c|}{Nemotron4-15B (FP32 PPL = 5.87)} \\ 
 \hline
 \hline
 64  & 6.3 & 6.20 & 6.13 & 6.08  \\
 \hline
 32  & 6.24 & 6.12 & 6.07 & 6.03  \\
 \hline
 16  & 6.12 & 6.14 & 6.04 & 6.02  \\
 \hline
 \multicolumn{5}{|c|}{Nemotron4-340B (FP32 PPL = 3.48)} \\ 
 \hline
 \hline
 64 & 3.67 & 3.62 & 3.60 & 3.59 \\
 \hline
 32 & 3.63 & 3.61 & 3.59 & 3.56 \\
 \hline
 16 & 3.61 & 3.58 & 3.57 & 3.55 \\
 \hline
\end{tabular}
\caption{\label{tab:ppl_llama7B_nemo15B} Wikitext-103 perplexity compared to FP32 baseline in Llama2-7B and Nemotron4-15B, 340B models}
\end{table}

%\subsection{Perplexity achieved by various LO-BCQ configurations on MMLU dataset}


\begin{table} \centering
\begin{tabular}{|c||c|c|c|c||c|c|c|c|} 
\hline
 $L_b \rightarrow$& \multicolumn{4}{c||}{8} & \multicolumn{4}{c||}{8}\\
 \hline
 \backslashbox{$L_A$\kern-1em}{\kern-1em$N_c$} & 2 & 4 & 8 & 16 & 2 & 4 & 8 & 16  \\
 %$N_c \rightarrow$ & 2 & 4 & 8 & 16 & 2 & 4 & 2 \\
 \hline
 \hline
 \multicolumn{5}{|c|}{Llama2-7B (FP32 Accuracy = 45.8\%)} & \multicolumn{4}{|c|}{Llama2-70B (FP32 Accuracy = 69.12\%)} \\ 
 \hline
 \hline
 64 & 43.9 & 43.4 & 43.9 & 44.9 & 68.07 & 68.27 & 68.17 & 68.75 \\
 \hline
 32 & 44.5 & 43.8 & 44.9 & 44.5 & 68.37 & 68.51 & 68.35 & 68.27  \\
 \hline
 16 & 43.9 & 42.7 & 44.9 & 45 & 68.12 & 68.77 & 68.31 & 68.59  \\
 \hline
 \hline
 \multicolumn{5}{|c|}{GPT3-22B (FP32 Accuracy = 38.75\%)} & \multicolumn{4}{|c|}{Nemotron4-15B (FP32 Accuracy = 64.3\%)} \\ 
 \hline
 \hline
 64 & 36.71 & 38.85 & 38.13 & 38.92 & 63.17 & 62.36 & 63.72 & 64.09 \\
 \hline
 32 & 37.95 & 38.69 & 39.45 & 38.34 & 64.05 & 62.30 & 63.8 & 64.33  \\
 \hline
 16 & 38.88 & 38.80 & 38.31 & 38.92 & 63.22 & 63.51 & 63.93 & 64.43  \\
 \hline
\end{tabular}
\caption{\label{tab:mmlu_abalation} Accuracy on MMLU dataset across GPT3-22B, Llama2-7B, 70B and Nemotron4-15B models.}
\end{table}


%\subsection{Perplexity achieved by various LO-BCQ configurations on LM evaluation harness}

\begin{table} \centering
\begin{tabular}{|c||c|c|c|c||c|c|c|c|} 
\hline
 $L_b \rightarrow$& \multicolumn{4}{c||}{8} & \multicolumn{4}{c||}{8}\\
 \hline
 \backslashbox{$L_A$\kern-1em}{\kern-1em$N_c$} & 2 & 4 & 8 & 16 & 2 & 4 & 8 & 16  \\
 %$N_c \rightarrow$ & 2 & 4 & 8 & 16 & 2 & 4 & 2 \\
 \hline
 \hline
 \multicolumn{5}{|c|}{Race (FP32 Accuracy = 37.51\%)} & \multicolumn{4}{|c|}{Boolq (FP32 Accuracy = 64.62\%)} \\ 
 \hline
 \hline
 64 & 36.94 & 37.13 & 36.27 & 37.13 & 63.73 & 62.26 & 63.49 & 63.36 \\
 \hline
 32 & 37.03 & 36.36 & 36.08 & 37.03 & 62.54 & 63.51 & 63.49 & 63.55  \\
 \hline
 16 & 37.03 & 37.03 & 36.46 & 37.03 & 61.1 & 63.79 & 63.58 & 63.33  \\
 \hline
 \hline
 \multicolumn{5}{|c|}{Winogrande (FP32 Accuracy = 58.01\%)} & \multicolumn{4}{|c|}{Piqa (FP32 Accuracy = 74.21\%)} \\ 
 \hline
 \hline
 64 & 58.17 & 57.22 & 57.85 & 58.33 & 73.01 & 73.07 & 73.07 & 72.80 \\
 \hline
 32 & 59.12 & 58.09 & 57.85 & 58.41 & 73.01 & 73.94 & 72.74 & 73.18  \\
 \hline
 16 & 57.93 & 58.88 & 57.93 & 58.56 & 73.94 & 72.80 & 73.01 & 73.94  \\
 \hline
\end{tabular}
\caption{\label{tab:mmlu_abalation} Accuracy on LM evaluation harness tasks on GPT3-1.3B model.}
\end{table}

\begin{table} \centering
\begin{tabular}{|c||c|c|c|c||c|c|c|c|} 
\hline
 $L_b \rightarrow$& \multicolumn{4}{c||}{8} & \multicolumn{4}{c||}{8}\\
 \hline
 \backslashbox{$L_A$\kern-1em}{\kern-1em$N_c$} & 2 & 4 & 8 & 16 & 2 & 4 & 8 & 16  \\
 %$N_c \rightarrow$ & 2 & 4 & 8 & 16 & 2 & 4 & 2 \\
 \hline
 \hline
 \multicolumn{5}{|c|}{Race (FP32 Accuracy = 41.34\%)} & \multicolumn{4}{|c|}{Boolq (FP32 Accuracy = 68.32\%)} \\ 
 \hline
 \hline
 64 & 40.48 & 40.10 & 39.43 & 39.90 & 69.20 & 68.41 & 69.45 & 68.56 \\
 \hline
 32 & 39.52 & 39.52 & 40.77 & 39.62 & 68.32 & 67.43 & 68.17 & 69.30  \\
 \hline
 16 & 39.81 & 39.71 & 39.90 & 40.38 & 68.10 & 66.33 & 69.51 & 69.42  \\
 \hline
 \hline
 \multicolumn{5}{|c|}{Winogrande (FP32 Accuracy = 67.88\%)} & \multicolumn{4}{|c|}{Piqa (FP32 Accuracy = 78.78\%)} \\ 
 \hline
 \hline
 64 & 66.85 & 66.61 & 67.72 & 67.88 & 77.31 & 77.42 & 77.75 & 77.64 \\
 \hline
 32 & 67.25 & 67.72 & 67.72 & 67.00 & 77.31 & 77.04 & 77.80 & 77.37  \\
 \hline
 16 & 68.11 & 68.90 & 67.88 & 67.48 & 77.37 & 78.13 & 78.13 & 77.69  \\
 \hline
\end{tabular}
\caption{\label{tab:mmlu_abalation} Accuracy on LM evaluation harness tasks on GPT3-8B model.}
\end{table}

\begin{table} \centering
\begin{tabular}{|c||c|c|c|c||c|c|c|c|} 
\hline
 $L_b \rightarrow$& \multicolumn{4}{c||}{8} & \multicolumn{4}{c||}{8}\\
 \hline
 \backslashbox{$L_A$\kern-1em}{\kern-1em$N_c$} & 2 & 4 & 8 & 16 & 2 & 4 & 8 & 16  \\
 %$N_c \rightarrow$ & 2 & 4 & 8 & 16 & 2 & 4 & 2 \\
 \hline
 \hline
 \multicolumn{5}{|c|}{Race (FP32 Accuracy = 40.67\%)} & \multicolumn{4}{|c|}{Boolq (FP32 Accuracy = 76.54\%)} \\ 
 \hline
 \hline
 64 & 40.48 & 40.10 & 39.43 & 39.90 & 75.41 & 75.11 & 77.09 & 75.66 \\
 \hline
 32 & 39.52 & 39.52 & 40.77 & 39.62 & 76.02 & 76.02 & 75.96 & 75.35  \\
 \hline
 16 & 39.81 & 39.71 & 39.90 & 40.38 & 75.05 & 73.82 & 75.72 & 76.09  \\
 \hline
 \hline
 \multicolumn{5}{|c|}{Winogrande (FP32 Accuracy = 70.64\%)} & \multicolumn{4}{|c|}{Piqa (FP32 Accuracy = 79.16\%)} \\ 
 \hline
 \hline
 64 & 69.14 & 70.17 & 70.17 & 70.56 & 78.24 & 79.00 & 78.62 & 78.73 \\
 \hline
 32 & 70.96 & 69.69 & 71.27 & 69.30 & 78.56 & 79.49 & 79.16 & 78.89  \\
 \hline
 16 & 71.03 & 69.53 & 69.69 & 70.40 & 78.13 & 79.16 & 79.00 & 79.00  \\
 \hline
\end{tabular}
\caption{\label{tab:mmlu_abalation} Accuracy on LM evaluation harness tasks on GPT3-22B model.}
\end{table}

\begin{table} \centering
\begin{tabular}{|c||c|c|c|c||c|c|c|c|} 
\hline
 $L_b \rightarrow$& \multicolumn{4}{c||}{8} & \multicolumn{4}{c||}{8}\\
 \hline
 \backslashbox{$L_A$\kern-1em}{\kern-1em$N_c$} & 2 & 4 & 8 & 16 & 2 & 4 & 8 & 16  \\
 %$N_c \rightarrow$ & 2 & 4 & 8 & 16 & 2 & 4 & 2 \\
 \hline
 \hline
 \multicolumn{5}{|c|}{Race (FP32 Accuracy = 44.4\%)} & \multicolumn{4}{|c|}{Boolq (FP32 Accuracy = 79.29\%)} \\ 
 \hline
 \hline
 64 & 42.49 & 42.51 & 42.58 & 43.45 & 77.58 & 77.37 & 77.43 & 78.1 \\
 \hline
 32 & 43.35 & 42.49 & 43.64 & 43.73 & 77.86 & 75.32 & 77.28 & 77.86  \\
 \hline
 16 & 44.21 & 44.21 & 43.64 & 42.97 & 78.65 & 77 & 76.94 & 77.98  \\
 \hline
 \hline
 \multicolumn{5}{|c|}{Winogrande (FP32 Accuracy = 69.38\%)} & \multicolumn{4}{|c|}{Piqa (FP32 Accuracy = 78.07\%)} \\ 
 \hline
 \hline
 64 & 68.9 & 68.43 & 69.77 & 68.19 & 77.09 & 76.82 & 77.09 & 77.86 \\
 \hline
 32 & 69.38 & 68.51 & 68.82 & 68.90 & 78.07 & 76.71 & 78.07 & 77.86  \\
 \hline
 16 & 69.53 & 67.09 & 69.38 & 68.90 & 77.37 & 77.8 & 77.91 & 77.69  \\
 \hline
\end{tabular}
\caption{\label{tab:mmlu_abalation} Accuracy on LM evaluation harness tasks on Llama2-7B model.}
\end{table}

\begin{table} \centering
\begin{tabular}{|c||c|c|c|c||c|c|c|c|} 
\hline
 $L_b \rightarrow$& \multicolumn{4}{c||}{8} & \multicolumn{4}{c||}{8}\\
 \hline
 \backslashbox{$L_A$\kern-1em}{\kern-1em$N_c$} & 2 & 4 & 8 & 16 & 2 & 4 & 8 & 16  \\
 %$N_c \rightarrow$ & 2 & 4 & 8 & 16 & 2 & 4 & 2 \\
 \hline
 \hline
 \multicolumn{5}{|c|}{Race (FP32 Accuracy = 48.8\%)} & \multicolumn{4}{|c|}{Boolq (FP32 Accuracy = 85.23\%)} \\ 
 \hline
 \hline
 64 & 49.00 & 49.00 & 49.28 & 48.71 & 82.82 & 84.28 & 84.03 & 84.25 \\
 \hline
 32 & 49.57 & 48.52 & 48.33 & 49.28 & 83.85 & 84.46 & 84.31 & 84.93  \\
 \hline
 16 & 49.85 & 49.09 & 49.28 & 48.99 & 85.11 & 84.46 & 84.61 & 83.94  \\
 \hline
 \hline
 \multicolumn{5}{|c|}{Winogrande (FP32 Accuracy = 79.95\%)} & \multicolumn{4}{|c|}{Piqa (FP32 Accuracy = 81.56\%)} \\ 
 \hline
 \hline
 64 & 78.77 & 78.45 & 78.37 & 79.16 & 81.45 & 80.69 & 81.45 & 81.5 \\
 \hline
 32 & 78.45 & 79.01 & 78.69 & 80.66 & 81.56 & 80.58 & 81.18 & 81.34  \\
 \hline
 16 & 79.95 & 79.56 & 79.79 & 79.72 & 81.28 & 81.66 & 81.28 & 80.96  \\
 \hline
\end{tabular}
\caption{\label{tab:mmlu_abalation} Accuracy on LM evaluation harness tasks on Llama2-70B model.}
\end{table}

%\section{MSE Studies}
%\textcolor{red}{TODO}


\subsection{Number Formats and Quantization Method}
\label{subsec:numFormats_quantMethod}
\subsubsection{Integer Format}
An $n$-bit signed integer (INT) is typically represented with a 2s-complement format \citep{yao2022zeroquant,xiao2023smoothquant,dai2021vsq}, where the most significant bit denotes the sign.

\subsubsection{Floating Point Format}
An $n$-bit signed floating point (FP) number $x$ comprises of a 1-bit sign ($x_{\mathrm{sign}}$), $B_m$-bit mantissa ($x_{\mathrm{mant}}$) and $B_e$-bit exponent ($x_{\mathrm{exp}}$) such that $B_m+B_e=n-1$. The associated constant exponent bias ($E_{\mathrm{bias}}$) is computed as $(2^{{B_e}-1}-1)$. We denote this format as $E_{B_e}M_{B_m}$.  

\subsubsection{Quantization Scheme}
\label{subsec:quant_method}
A quantization scheme dictates how a given unquantized tensor is converted to its quantized representation. We consider FP formats for the purpose of illustration. Given an unquantized tensor $\bm{X}$ and an FP format $E_{B_e}M_{B_m}$, we first, we compute the quantization scale factor $s_X$ that maps the maximum absolute value of $\bm{X}$ to the maximum quantization level of the $E_{B_e}M_{B_m}$ format as follows:
\begin{align}
\label{eq:sf}
    s_X = \frac{\mathrm{max}(|\bm{X}|)}{\mathrm{max}(E_{B_e}M_{B_m})}
\end{align}
In the above equation, $|\cdot|$ denotes the absolute value function.

Next, we scale $\bm{X}$ by $s_X$ and quantize it to $\hat{\bm{X}}$ by rounding it to the nearest quantization level of $E_{B_e}M_{B_m}$ as:

\begin{align}
\label{eq:tensor_quant}
    \hat{\bm{X}} = \text{round-to-nearest}\left(\frac{\bm{X}}{s_X}, E_{B_e}M_{B_m}\right)
\end{align}

We perform dynamic max-scaled quantization \citep{wu2020integer}, where the scale factor $s$ for activations is dynamically computed during runtime.

\subsection{Vector Scaled Quantization}
\begin{wrapfigure}{r}{0.35\linewidth}
  \centering
  \includegraphics[width=\linewidth]{sections/figures/vsquant.jpg}
  \caption{\small Vectorwise decomposition for per-vector scaled quantization (VSQ \citep{dai2021vsq}).}
  \label{fig:vsquant}
\end{wrapfigure}
During VSQ \citep{dai2021vsq}, the operand tensors are decomposed into 1D vectors in a hardware friendly manner as shown in Figure \ref{fig:vsquant}. Since the decomposed tensors are used as operands in matrix multiplications during inference, it is beneficial to perform this decomposition along the reduction dimension of the multiplication. The vectorwise quantization is performed similar to tensorwise quantization described in Equations \ref{eq:sf} and \ref{eq:tensor_quant}, where a scale factor $s_v$ is required for each vector $\bm{v}$ that maps the maximum absolute value of that vector to the maximum quantization level. While smaller vector lengths can lead to larger accuracy gains, the associated memory and computational overheads due to the per-vector scale factors increases. To alleviate these overheads, VSQ \citep{dai2021vsq} proposed a second level quantization of the per-vector scale factors to unsigned integers, while MX \citep{rouhani2023shared} quantizes them to integer powers of 2 (denoted as $2^{INT}$).

\subsubsection{MX Format}
The MX format proposed in \citep{rouhani2023microscaling} introduces the concept of sub-block shifting. For every two scalar elements of $b$-bits each, there is a shared exponent bit. The value of this exponent bit is determined through an empirical analysis that targets minimizing quantization MSE. We note that the FP format $E_{1}M_{b}$ is strictly better than MX from an accuracy perspective since it allocates a dedicated exponent bit to each scalar as opposed to sharing it across two scalars. Therefore, we conservatively bound the accuracy of a $b+2$-bit signed MX format with that of a $E_{1}M_{b}$ format in our comparisons. For instance, we use E1M2 format as a proxy for MX4.

\begin{figure}
    \centering
    \includegraphics[width=1\linewidth]{sections//figures/BlockFormats.pdf}
    \caption{\small Comparing LO-BCQ to MX format.}
    \label{fig:block_formats}
\end{figure}

Figure \ref{fig:block_formats} compares our $4$-bit LO-BCQ block format to MX \citep{rouhani2023microscaling}. As shown, both LO-BCQ and MX decompose a given operand tensor into block arrays and each block array into blocks. Similar to MX, we find that per-block quantization ($L_b < L_A$) leads to better accuracy due to increased flexibility. While MX achieves this through per-block $1$-bit micro-scales, we associate a dedicated codebook to each block through a per-block codebook selector. Further, MX quantizes the per-block array scale-factor to E8M0 format without per-tensor scaling. In contrast during LO-BCQ, we find that per-tensor scaling combined with quantization of per-block array scale-factor to E4M3 format results in superior inference accuracy across models. 

\end{document}
