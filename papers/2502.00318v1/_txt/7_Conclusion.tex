\vspace{-3mm}
\section{Conclusion}
In this paper, we reveal that the mismatch between discrete training of PINNs and the continuous nature of PDEs, as well as simplicity bias are the key of failure modes. 
    In combating with such failure modes, we propose PINNMamba, an SSM-based sub-sequence learning framework. 
    PINNMamba successfully eliminates the failure modes, and meanwhile becomes the new state-of-the-art PINN architecture.
    
\section*{Impact Statement}

The development of physics-informed neural networks represents a transformative approach to solving differential equations by integrating physical laws directly into the learning process. 
    This work explores novel advancements in PINN architecture, to improve accuracy and eliminate the potential failure modes. 
    By refining PINN architectures, this study contributes to the broader adoption of physics-informed machine learning in fields such as computational fluid dynamics, material science, and engineering simulations. 
        The proposed enhancements lead to more robust and scalable models, facilitating real-world applications where conventional PINNs struggle with over-smoothing. There is no known negative impact from this study at this time.