%%%%%%%% ICML 2025 EXAMPLE LATEX SUBMISSION FILE %%%%%%%%%%%%%%%%%
\pdfoutput=1
\documentclass{article}

% Recommended, but optional, packages for figures and better typesetting:
\usepackage{microtype}
\usepackage{graphicx}
%usepackage{graphicx}
\usepackage{grffile}
\usepackage{subfigure}
\usepackage{booktabs} % for professional tables

% hyperref makes hyperlinks in the resulting PDF.
% If your build breaks (sometimes temporarily if a hyperlink spans a page)
% please comment out the following usepackage line and replace
\usepackage[accepted]{_sty/icml2025} %with \usepackage[nohyperref]{icml2025} above.
%\usepackage{hyperref}


% Attempt to make hyperref and algorithmic work together better:
\newcommand{\theHalgorithm}{\arabic{algorithm}}
\newcommand{\QX}[1]{{\textcolor{teal}{#1}}}
\newcommand{\dcheng}[1]{{\textcolor{orange}{#1}}}
\newcommand{\CX}[1]{{\textcolor{purple}{#1}}}
\newcommand{\jx}[1]{{\textcolor{blue}{#1}}}

% Use the following line for the initial blind version submitted for review:
% \usepackage{icml2025}

% If accepted, instead use the following line for the camera-ready submission:
%\usepackage[accepted]{_sty/icml2025}

% For theorems and such
\usepackage{amsmath}
\usepackage{amssymb}
\usepackage{mathtools}
\usepackage{amsthm}
\usepackage{comment}
\usepackage{multirow}
\usepackage{float}

% if you use cleveref..
\usepackage[capitalize,noabbrev]{cleveref}

%%%%%%%%%%%%%%%%%%%%%%%%%%%%%%%%
% THEOREMS
%%%%%%%%%%%%%%%%%%%%%%%%%%%%%%%%
\theoremstyle{plain}
\newtheorem{theorem}{Theorem}[section]
\newtheorem{proposition}[theorem]{Proposition}
\newtheorem{lemma}[theorem]{Lemma}
\newtheorem{corollary}[theorem]{Corollary}
\theoremstyle{definition}
\newtheorem{definition}[theorem]{Definition}
\newtheorem{assumption}[theorem]{Assumption}
\theoremstyle{remark}
\newtheorem{remark}[theorem]{Remark}

% Todonotes is useful during development; simply uncomment the next line
%    and comment out the line below the next line to turn off comments
%\usepackage[disable,textsize=tiny]{todonotes}
\usepackage[textsize=tiny]{todonotes}

\usepackage{colortbl}
\usepackage{wrapfig} % For annotations
\usepackage{xcolor}   
\newcommand{\yuting}[1]{\textcolor{purple}{Yuting: #1}}
\definecolor{mygray}{gray}{.9}


% The \icmltitle you define below is probably too long as a header.
% Therefore, a short form for the running title is supplied here:
\icmltitlerunning{Sub-Sequential Physics-Informed Learning with State Space Model}

\begin{document}

\twocolumn[
\icmltitle{Sub-Sequential Physics-Informed Learning with State Space Model}

% It is OKAY to include author information, even for blind
% submissions: the style file will automatically remove it for you
% unless you've provided the [accepted] option to the icml2025
% package.

% List of affiliations: The first argument should be a (short)
% identifier you will use later to specify author affiliations
% Academic affiliations should list Department, University, City, Region, Country
% Industry affiliations should list Company, City, Region, Country

% You can specify symbols, otherwise they are numbered in order.
% Ideally, you should not use this facility. Affiliations will be numbered
% in order of appearance and this is the preferred way.
\icmlsetsymbol{equal}{*}

\begin{icmlauthorlist}
\icmlauthor{Chenhui Xu}{yyy}
\icmlauthor{Dancheng Liu}{yyy}
\icmlauthor{Yuting Hu}{yyy}
\icmlauthor{Jiajie Li}{yyy}
\icmlauthor{Ruiyang Qin}{yyy,zzz}
\icmlauthor{Qingxiao Zheng}{yyy}
\icmlauthor{Jinjun Xiong}{yyy}
%\icmlauthor{}{sch}
%\icmlauthor{Firstname8 Lastname8}{sch}
%\icmlauthor{Firstname8 Lastname8}{yyy,comp}
%\icmlauthor{}{sch}
%\icmlauthor{}{sch}
\end{icmlauthorlist}

\icmlaffiliation{yyy}{University at Buffalo, SUNY}
%\icmlaffiliation{comp}{Company Name, Location, Country}
\icmlaffiliation{zzz}{University of Notre Dame}

\icmlcorrespondingauthor{Chenhui Xu}{cxu26@buffalo.edu}
\icmlcorrespondingauthor{Jinjun Xiong}{jinjun@buffalo.edu}

% You may provide any keywords that you
% find helpful for describing your paper; these are used to populate
% the "keywords" metadata in the PDF but will not be shown in the document
\icmlkeywords{Machine Learning, ICML}

\vskip 0.3in
]

% this must go after the closing bracket ] following \twocolumn[ ...

% This command actually creates the footnote in the first column
% listing the affiliations and the copyright notice.
% The command takes one argument, which is text to display at the start of the footnote.
% The \icmlEqualContribution command is standard text for equal contribution.
% Remove it (just {}) if you do not need this facility.

\printAffiliationsAndNotice{}  % leave blank if no need to mention equal contribution
%\printAffiliationsAndNotice{\icmlEqualContribution} % otherwise use the standard text.
\begin{comment}

\begin{abstract}
This document provides a basic paper template and submission guidelines.
Abstracts must be a single paragraph, ideally between 4--6 sentences long.
Gross violations will trigger corrections at the camera-ready phase.
\end{abstract}

\section{Electronic Submission}
\label{submission}

Submission to ICML 2025 will be entirely electronic, via a web site
(not email). Information about the submission process and \LaTeX\ templates
are available on the conference web site at:
\begin{center}
\textbf{\texttt{http://icml.cc/}}
\end{center}

The guidelines below will be enforced for initial submissions and
camera-ready copies. Here is a brief summary:
\begin{itemize}
\item Submissions must be in PDF\@. 
\item If your paper has appendices, submit the appendix together with the main body and the references \textbf{as a single file}. Reviewers will not look for appendices as a separate PDF file. So if you submit such an extra file, reviewers will very likely miss it.
\item Page limit: The main body of the paper has to be fitted to 8 pages, excluding references and appendices; the space for the latter two is not limited in pages, but the total file size may not exceed 10MB. For the final version of the paper, authors can add one extra page to the main body.
\item \textbf{Do not include author information or acknowledgements} in your
    initial submission.
\item Your paper should be in \textbf{10 point Times font}.
\item Make sure your PDF file only uses Type-1 fonts.
\item Place figure captions \emph{under} the figure (and omit titles from inside
    the graphic file itself). Place table captions \emph{over} the table.
\item References must include page numbers whenever possible and be as complete
    as possible. Place multiple citations in chronological order.
\item Do not alter the style template; in particular, do not compress the paper
    format by reducing the vertical spaces.
\item Keep your abstract brief and self-contained, one paragraph and roughly
    4--6 sentences. Gross violations will require correction at the
    camera-ready phase. The title should have content words capitalized.
\end{itemize}

\subsection{Submitting Papers}

\textbf{Anonymous Submission:} ICML uses double-blind review: no identifying
author information may appear on the title page or in the paper
itself. \cref{author info} gives further details.

\medskip

Authors must provide their manuscripts in \textbf{PDF} format.
Furthermore, please make sure that files contain only embedded Type-1 fonts
(e.g.,~using the program \texttt{pdffonts} in linux or using
File/DocumentProperties/Fonts in Acrobat). Other fonts (like Type-3)
might come from graphics files imported into the document.

Authors using \textbf{Word} must convert their document to PDF\@. Most
of the latest versions of Word have the facility to do this
automatically. Submissions will not be accepted in Word format or any
format other than PDF\@. Really. We're not joking. Don't send Word.

Those who use \textbf{\LaTeX} should avoid including Type-3 fonts.
Those using \texttt{latex} and \texttt{dvips} may need the following
two commands:

{\footnotesize
\begin{verbatim}
dvips -Ppdf -tletter -G0 -o paper.ps paper.dvi
ps2pdf paper.ps
\end{verbatim}}
It is a zero following the ``-G'', which tells dvips to use
the config.pdf file. Newer \TeX\ distributions don't always need this
option.

Using \texttt{pdflatex} rather than \texttt{latex}, often gives better
results. This program avoids the Type-3 font problem, and supports more
advanced features in the \texttt{microtype} package.

\textbf{Graphics files} should be a reasonable size, and included from
an appropriate format. Use vector formats (.eps/.pdf) for plots,
lossless bitmap formats (.png) for raster graphics with sharp lines, and
jpeg for photo-like images.

The style file uses the \texttt{hyperref} package to make clickable
links in documents. If this causes problems for you, add
\texttt{nohyperref} as one of the options to the \texttt{icml2025}
usepackage statement.


\subsection{Submitting Final Camera-Ready Copy}

The final versions of papers accepted for publication should follow the
same format and naming convention as initial submissions, except that
author information (names and affiliations) should be given. See
\cref{final author} for formatting instructions.

The footnote, ``Preliminary work. Under review by the International
Conference on Machine Learning (ICML). Do not distribute.'' must be
modified to ``\textit{Proceedings of the
$\mathit{42}^{nd}$ International Conference on Machine Learning},
Vancouver, Canada, PMLR 267, 2025.
Copyright 2025 by the author(s).''

For those using the \textbf{\LaTeX} style file, this change (and others) is
handled automatically by simply changing
$\mathtt{\backslash usepackage\{icml2025\}}$ to
$$\mathtt{\backslash usepackage[accepted]\{icml2025\}}$$
Authors using \textbf{Word} must edit the
footnote on the first page of the document themselves.

Camera-ready copies should have the title of the paper as running head
on each page except the first one. The running title consists of a
single line centered above a horizontal rule which is $1$~point thick.
The running head should be centered, bold and in $9$~point type. The
rule should be $10$~points above the main text. For those using the
\textbf{\LaTeX} style file, the original title is automatically set as running
head using the \texttt{fancyhdr} package which is included in the ICML
2025 style file package. In case that the original title exceeds the
size restrictions, a shorter form can be supplied by using

\verb|\icmltitlerunning{...}|

just before $\mathtt{\backslash begin\{document\}}$.
Authors using \textbf{Word} must edit the header of the document themselves.

\section{Format of the Paper}

All submissions must follow the specified format.

\subsection{Dimensions}




The text of the paper should be formatted in two columns, with an
overall width of 6.75~inches, height of 9.0~inches, and 0.25~inches
between the columns. The left margin should be 0.75~inches and the top
margin 1.0~inch (2.54~cm). The right and bottom margins will depend on
whether you print on US letter or A4 paper, but all final versions
must be produced for US letter size.
Do not write anything on the margins.

The paper body should be set in 10~point type with a vertical spacing
of 11~points. Please use Times typeface throughout the text.

\subsection{Title}

The paper title should be set in 14~point bold type and centered
between two horizontal rules that are 1~point thick, with 1.0~inch
between the top rule and the top edge of the page. Capitalize the
first letter of content words and put the rest of the title in lower
case.

\subsection{Author Information for Submission}
\label{author info}

ICML uses double-blind review, so author information must not appear. If
you are using \LaTeX\/ and the \texttt{icml2025.sty} file, use
\verb+\icmlauthor{...}+ to specify authors and \verb+\icmlaffiliation{...}+ to specify affiliations. (Read the TeX code used to produce this document for an example usage.) The author information
will not be printed unless \texttt{accepted} is passed as an argument to the
style file.
Submissions that include the author information will not
be reviewed.

\subsubsection{Self-Citations}

If you are citing published papers for which you are an author, refer
to yourself in the third person. In particular, do not use phrases
that reveal your identity (e.g., ``in previous work \cite{langley00}, we
have shown \ldots'').

Do not anonymize citations in the reference section. The only exception are manuscripts that are
not yet published (e.g., under submission). If you choose to refer to
such unpublished manuscripts \cite{anonymous}, anonymized copies have
to be submitted
as Supplementary Material via OpenReview\@. However, keep in mind that an ICML
paper should be self contained and should contain sufficient detail
for the reviewers to evaluate the work. In particular, reviewers are
not required to look at the Supplementary Material when writing their
review (they are not required to look at more than the first $8$ pages of the submitted document).

\subsubsection{Camera-Ready Author Information}
\label{final author}

If a paper is accepted, a final camera-ready copy must be prepared.
%
For camera-ready papers, author information should start 0.3~inches below the
bottom rule surrounding the title. The authors' names should appear in 10~point
bold type, in a row, separated by white space, and centered. Author names should
not be broken across lines. Unbolded superscripted numbers, starting 1, should
be used to refer to affiliations.

Affiliations should be numbered in the order of appearance. A single footnote
block of text should be used to list all the affiliations. (Academic
affiliations should list Department, University, City, State/Region, Country.
Similarly for industrial affiliations.)

Each distinct affiliations should be listed once. If an author has multiple
affiliations, multiple superscripts should be placed after the name, separated
by thin spaces. If the authors would like to highlight equal contribution by
multiple first authors, those authors should have an asterisk placed after their
name in superscript, and the term ``\textsuperscript{*}Equal contribution"
should be placed in the footnote block ahead of the list of affiliations. A
list of corresponding authors and their emails (in the format Full Name
\textless{}email@domain.com\textgreater{}) can follow the list of affiliations.
Ideally only one or two names should be listed.

A sample file with author names is included in the ICML2025 style file
package. Turn on the \texttt{[accepted]} option to the stylefile to
see the names rendered. All of the guidelines above are implemented
by the \LaTeX\ style file.

\subsection{Abstract}

The paper abstract should begin in the left column, 0.4~inches below the final
address. The heading `Abstract' should be centered, bold, and in 11~point type.
The abstract body should use 10~point type, with a vertical spacing of
11~points, and should be indented 0.25~inches more than normal on left-hand and
right-hand margins. Insert 0.4~inches of blank space after the body. Keep your
abstract brief and self-contained, limiting it to one paragraph and roughly 4--6
sentences. Gross violations will require correction at the camera-ready phase.

\subsection{Partitioning the Text}

You should organize your paper into sections and paragraphs to help
readers place a structure on the material and understand its
contributions.

\subsubsection{Sections and Subsections}

Section headings should be numbered, flush left, and set in 11~pt bold
type with the content words capitalized. Leave 0.25~inches of space
before the heading and 0.15~inches after the heading.

Similarly, subsection headings should be numbered, flush left, and set
in 10~pt bold type with the content words capitalized. Leave
0.2~inches of space before the heading and 0.13~inches afterward.

Finally, subsubsection headings should be numbered, flush left, and
set in 10~pt small caps with the content words capitalized. Leave
0.18~inches of space before the heading and 0.1~inches after the
heading.

Please use no more than three levels of headings.

\subsubsection{Paragraphs and Footnotes}

Within each section or subsection, you should further partition the
paper into paragraphs. Do not indent the first line of a given
paragraph, but insert a blank line between succeeding ones.

You can use footnotes\footnote{Footnotes
should be complete sentences.} to provide readers with additional
information about a topic without interrupting the flow of the paper.
Indicate footnotes with a number in the text where the point is most
relevant. Place the footnote in 9~point type at the bottom of the
column in which it appears. Precede the first footnote in a column
with a horizontal rule of 0.8~inches.\footnote{Multiple footnotes can
appear in each column, in the same order as they appear in the text,
but spread them across columns and pages if possible.}

\begin{figure}[ht]
\vskip 0.2in
\begin{center}
\centerline{\includegraphics[width=\columnwidth]{icml_numpapers}}
\caption{Historical locations and number of accepted papers for International
Machine Learning Conferences (ICML 1993 -- ICML 2008) and International
Workshops on Machine Learning (ML 1988 -- ML 1992). At the time this figure was
produced, the number of accepted papers for ICML 2008 was unknown and instead
estimated.}
\label{icml-historical}
\end{center}
\vskip -0.2in
\end{figure}

\subsection{Figures}

You may want to include figures in the paper to illustrate
your approach and results. Such artwork should be centered,
legible, and separated from the text. Lines should be dark and at
least 0.5~points thick for purposes of reproduction, and text should
not appear on a gray background.

Label all distinct components of each figure. If the figure takes the
form of a graph, then give a name for each axis and include a legend
that briefly describes each curve. Do not include a title inside the
figure; instead, the caption should serve this function.

Number figures sequentially, placing the figure number and caption
\emph{after} the graphics, with at least 0.1~inches of space before
the caption and 0.1~inches after it, as in
\cref{icml-historical}. The figure caption should be set in
9~point type and centered unless it runs two or more lines, in which
case it should be flush left. You may float figures to the top or
bottom of a column, and you may set wide figures across both columns
(use the environment \texttt{figure*} in \LaTeX). Always place
two-column figures at the top or bottom of the page.

\subsection{Algorithms}

If you are using \LaTeX, please use the ``algorithm'' and ``algorithmic''
environments to format pseudocode. These require
the corresponding stylefiles, algorithm.sty and
algorithmic.sty, which are supplied with this package.
\cref{alg:example} shows an example.

\begin{algorithm}[tb]
   \caption{Bubble Sort}
   \label{alg:example}
\begin{algorithmic}
   \STATE {\bfseries Input:} data $x_i$, size $m$
   \REPEAT
   \STATE Initialize $noChange = true$.
   \FOR{$i=1$ {\bfseries to} $m-1$}
   \IF{$x_i > x_{i+1}$}
   \STATE Swap $x_i$ and $x_{i+1}$
   \STATE $noChange = false$
   \ENDIF
   \ENDFOR
   \UNTIL{$noChange$ is $true$}
\end{algorithmic}
\end{algorithm}

\subsection{Tables}

You may also want to include tables that summarize material. Like
figures, these should be centered, legible, and numbered consecutively.
However, place the title \emph{above} the table with at least
0.1~inches of space before the title and the same after it, as in
\cref{sample-table}. The table title should be set in 9~point
type and centered unless it runs two or more lines, in which case it
should be flush left.

% Note use of \abovespace and \belowspace to get reasonable spacing
% above and below tabular lines.

\begin{table}[t]
\caption{Classification accuracies for naive Bayes and flexible
Bayes on various data sets.}
\label{sample-table}
\vskip 0.15in
\begin{center}
\begin{small}
\begin{sc}
\begin{tabular}{lcccr}
\toprule
Data set & Naive & Flexible & Better? \\
\midrule
Breast    & 95.9$\pm$ 0.2& 96.7$\pm$ 0.2& $\surd$ \\
Cleveland & 83.3$\pm$ 0.6& 80.0$\pm$ 0.6& $\times$\\
Glass2    & 61.9$\pm$ 1.4& 83.8$\pm$ 0.7& $\surd$ \\
Credit    & 74.8$\pm$ 0.5& 78.3$\pm$ 0.6&         \\
Horse     & 73.3$\pm$ 0.9& 69.7$\pm$ 1.0& $\times$\\
Meta      & 67.1$\pm$ 0.6& 76.5$\pm$ 0.5& $\surd$ \\
Pima      & 75.1$\pm$ 0.6& 73.9$\pm$ 0.5&         \\
Vehicle   & 44.9$\pm$ 0.6& 61.5$\pm$ 0.4& $\surd$ \\
\bottomrule
\end{tabular}
\end{sc}
\end{small}
\end{center}
\vskip -0.1in
\end{table}

Tables contain textual material, whereas figures contain graphical material.
Specify the contents of each row and column in the table's topmost
row. Again, you may float tables to a column's top or bottom, and set
wide tables across both columns. Place two-column tables at the
top or bottom of the page.

\subsection{Theorems and such}
The preferred way is to number definitions, propositions, lemmas, etc. consecutively, within sections, as shown below.
\begin{definition}
\label{def:inj}
A function $f:X \to Y$ is injective if for any $x,y\in X$ different, $f(x)\ne f(y)$.
\end{definition}
Using \cref{def:inj} we immediate get the following result:
\begin{proposition}
If $f$ is injective mapping a set $X$ to another set $Y$, 
the cardinality of $Y$ is at least as large as that of $X$
\end{proposition}
\begin{proof} 
Left as an exercise to the reader. 
\end{proof}
\cref{lem:usefullemma} stated next will prove to be useful.
\begin{lemma}
\label{lem:usefullemma}
For any $f:X \to Y$ and $g:Y\to Z$ injective functions, $f \circ g$ is injective.
\end{lemma}
\begin{theorem}
\label{thm:bigtheorem}
If $f:X\to Y$ is bijective, the cardinality of $X$ and $Y$ are the same.
\end{theorem}
An easy corollary of \cref{thm:bigtheorem} is the following:
\begin{corollary}
If $f:X\to Y$ is bijective, 
the cardinality of $X$ is at least as large as that of $Y$.
\end{corollary}
\begin{assumption}
The set $X$ is finite.
\label{ass:xfinite}
\end{assumption}
\begin{remark}
According to some, it is only the finite case (cf. \cref{ass:xfinite}) that is interesting.
\end{remark}
%restatable

\subsection{Citations and References}

Please use APA reference format regardless of your formatter
or word processor. If you rely on the \LaTeX\/ bibliographic
facility, use \texttt{natbib.sty} and \texttt{icml2025.bst}
included in the style-file package to obtain this format.

Citations within the text should include the authors' last names and
year. If the authors' names are included in the sentence, place only
the year in parentheses, for example when referencing Arthur Samuel's
pioneering work \yrcite{Samuel59}. Otherwise place the entire
reference in parentheses with the authors and year separated by a
comma \cite{Samuel59}. List multiple references separated by
semicolons \cite{kearns89,Samuel59,mitchell80}. Use the `et~al.'
construct only for citations with three or more authors or after
listing all authors to a publication in an earlier reference \cite{MachineLearningI}.

Authors should cite their own work in the third person
in the initial version of their paper submitted for blind review.
Please refer to \cref{author info} for detailed instructions on how to
cite your own papers.

Use an unnumbered first-level section heading for the references, and use a
hanging indent style, with the first line of the reference flush against the
left margin and subsequent lines indented by 10 points. The references at the
end of this document give examples for journal articles \cite{Samuel59},
conference publications \cite{langley00}, book chapters \cite{Newell81}, books
\cite{DudaHart2nd}, edited volumes \cite{MachineLearningI}, technical reports
\cite{mitchell80}, and dissertations \cite{kearns89}.

Alphabetize references by the surnames of the first authors, with
single author entries preceding multiple author entries. Order
references for the same authors by year of publication, with the
earliest first. Make sure that each reference includes all relevant
information (e.g., page numbers).

Please put some effort into making references complete, presentable, and
consistent, e.g. use the actual current name of authors.
If using bibtex, please protect capital letters of names and
abbreviations in titles, for example, use \{B\}ayesian or \{L\}ipschitz
in your .bib file.

\section*{Accessibility}
Authors are kindly asked to make their submissions as accessible as possible for everyone including people with disabilities and sensory or neurological differences.
Tips of how to achieve this and what to pay attention to will be provided on the conference website \url{http://icml.cc/}.

\section*{Software and Data}

If a paper is accepted, we strongly encourage the publication of software and data with the
camera-ready version of the paper whenever appropriate. This can be
done by including a URL in the camera-ready copy. However, \textbf{do not}
include URLs that reveal your institution or identity in your
submission for review. Instead, provide an anonymous URL or upload
the material as ``Supplementary Material'' into the OpenReview reviewing
system. Note that reviewers are not required to look at this material
when writing their review.

% Acknowledgements should only appear in the accepted version.
\section*{Acknowledgements}

\textbf{Do not} include acknowledgements in the initial version of
the paper submitted for blind review.

If a paper is accepted, the final camera-ready version can (and
usually should) include acknowledgements.  Such acknowledgements
should be placed at the end of the section, in an unnumbered section
that does not count towards the paper page limit. Typically, this will 
include thanks to reviewers who gave useful comments, to colleagues 
who contributed to the ideas, and to funding agencies and corporate 
sponsors that provided financial support.

\section*{Impact Statement}

Authors are \textbf{required} to include a statement of the potential 
broader impact of their work, including its ethical aspects and future 
societal consequences. This statement should be in an unnumbered 
section at the end of the paper (co-located with Acknowledgements -- 
the two may appear in either order, but both must be before References), 
and does not count toward the paper page limit. In many cases, where 
the ethical impacts and expected societal implications are those that 
are well established when advancing the field of Machine Learning, 
substantial discussion is not required, and a simple statement such 
as the following will suffice:

``This paper presents work whose goal is to advance the field of 
Machine Learning. There are many potential societal consequences 
of our work, none which we feel must be specifically highlighted here.''

The above statement can be used verbatim in such cases, but we 
encourage authors to think about whether there is content which does 
warrant further discussion, as this statement will be apparent if the 
paper is later flagged for ethics review.
\end{comment}

% In the unusual situation where you want a paper to appear in the
% references without citing it in the main text, use \nocite
%\nocite{langley00}

\begin{abstract}

Hierarchical clustering is a powerful tool for exploratory data analysis, organizing data into a tree of clusterings from which a partition can be chosen. This paper generalizes these ideas by proving that, for any reasonable hierarchy, one can optimally solve any center-based clustering objective over it (such as $k$-means). Moreover, these solutions can be found exceedingly quickly and are \emph{themselves} necessarily hierarchical. 
%Thus, given a cluster tree, we show that one can quickly generate a myriad of \emph{new} hierarchies from it. 
Thus, given a cluster tree, we show that one can quickly access a plethora of new, equally meaningful hierarchies.
Just as in standard hierarchical clustering, one can then choose any desired partition from these new hierarchies. We conclude by verifying the utility of our proposed techniques across datasets, hierarchies, and partitioning schemes.


\end{abstract}

\section{Introduction}

In the past few years, Physics-Informed Neural Networks~(PINNs)~\cite{raissi2019physics} have emerged as a novel approach for numerically solving partial differential equations~(PDEs). 
    PINN takes a neural network $u_{\theta}(x,t)$, whose parameters $\theta$ are trained with physics PDE residual loss, as the numerical solution $u(x,t)$ of the PDE, where $x$ and $t$ are spatial and temporal coordinates. 
        The core idea behind PINNs is to take advantage of the universal approximation property of neural networks~\cite{hornik1989multilayer} and automatic differentiation implemented by mainstream deep learning frameworks, such as PyTorch~\cite{paszke2019pytorch} and Tensorflow~\cite{abadi2016tensorflow}, 
            so that PINNs can achieve potentially more precise and efficient PDE solution approximation compared with traditional numerical approaches like finite element methods~\cite{reddy1993introduction}.


%Current mainstream PINNs use multilayer perceptrons (MLPs) as their backbone network. 
 %   However, despite the universal approximation of the MLPs, an arbitrarily accurate approximation to the numerical solution of the PDE is not always guaranteed to be learned correctly. 
%This has resulted in the observation of multiple failure modes in PINNs, especially when approximating high-frequency or complex patterns~\cite{krishnapriyan2021characterizing}.
%As shown in Fig.~\ref{fig1}, %these failure modes of PINNs exhibit a gradual distortion over time.This is due to that MLPs lack \textit{Inductive Bias} that can effectively model time dependencies of a system, resulting in the failure of propagating physical information that is defined by the initial condition. 
%these failure modes exhibit a gradual temporal distortion of the solution. This degradation stems from MLPs' inherent lack of inductive bias for modeling time dependencies of a system, which impairs their ability to effectively propagate physical information defined by the initial condition.%constraints.


The mainstream PINNs predominantly employ multilayer perceptrons (MLPs) as their backbone architecture. However, despite the universal approximation capability of MLPs, they do not always guarantee the accurate learning of numerical solutions to PDEs in practice. This phenomenon is observed as the failure modes in PINNs, in which case PINN provides a completely wrong approximation~\cite{krishnapriyan2021characterizing}. As illustrated in Fig. \ref{fig1}, the failure modes often manifest as a temporal gradual distortion. This distortion arises because MLPs lack the necessary inductive bias to effectively capture the temporal dependencies of a system, ultimately hindering the accurate propagation of physical patterns informed by the initial conditions.

%Current mainstream PINNs employ multilayer perceptrons (MLPs) as their backbone. However, despite the universal approximation capabilities of MLPs, they do not always guarantee an arbitrarily accurate approximation of the numerical solution to PDEs. Consequently, multiple failure modes have been observed in PINNs, particularly when approximating high-frequency or complex patterns~\cite{krishnapriyan2021characterizing}. As illustrated in Fig.~\ref{fig1}, these failure modes manifest as a gradual distortion over time. This issue arises because MLPs lack an \textit{inductive bias} that effectively models temporal dependencies, leading to failures in propagating physical information dictated by the initial conditions.

\begin{figure}[t!]
    \centering
    \includegraphics[width=\linewidth]{_fig/fig1.pdf}
    \vspace{-6mm}
    \caption{PINN gradually distorts on convection equation.}
    \label{fig1}
    \vspace{-5mm}
  %  \vspace{-1mm}
\end{figure}


To introduce such an inductive bias, several sequence-to-sequence approaches have been proposed~\cite{krishnapriyan2021characterizing,zhao2024pinnsformer,yang2022learning,gao2022earthformer}. Specifically, \citet{krishnapriyan2021characterizing} propose training a new network at each time step and recursively using its output as the initial condition for the next step. This method incurs significant computational and memory overhead while exhibiting poor generalization. Furthermore, Transformer-based approaches~\cite{zhao2024pinnsformer,yang2022learning,gao2022earthformer} are proposed to address the time-dependency issue. Yet, these methods are based on discrete sequences, making their model produce incorrect pattern mutations in some cases due to their ignorance of the basic principle that PINNs approximate continuous dynamical systems. Thus, there is still an open question:

%To introduce such an \textit{Inductive Bias}, several sequence-to-sequence approaches are proposed~\cite{krishnapriyan2021characterizing,zhao2024pinnsformer,yang2022learning,gao2022earthformer}. Specifically, \citet{krishnapriyan2021characterizing} propose training a new network at every time step, then take the output as the initial condition for next step, but with tremendous computation and memory overheads and poor generalization. Meanwhile, Transformer-based approaches~\cite{zhao2024pinnsformer,yang2022learning,gao2022earthformer} are proposed to address the time-dependency issue. Yet, these methods are based on discrete sequences, which leads to that Transformers, in some cases, have incorrect mutations of the pattern. This is due to the fact that they ignore an important assumption that PINNs are approximating continuous dynamical systems.
\begin{comment}
To address this problem, sequence-to-sequence learning~\cite{sutskever2014sequence} is proposed as the inductive bias for modeling time-dependent equations' solution~\cite{krishnapriyan2021characterizing}.
    But \citet{sutskever2014sequence}'s method requires training a new network at every time step, which significantly increases the computation and memory overhead.
To make matters worse, we can't simply use recurrent neural networks for sequence-to-sequence learning to solve such failures. Since PINNs often involve higher order differentiation, the problem of gradient vanishing inherent in recurrent models~\cite{hochreiter1998vanishing} will be exacerbated, indicating that recurrent long-sequence modeling is not a good inductive bias either. PINNsFormer~\cite{zhao2024pinnsformer} attempts to use the Transformer model, but its construction of a pseudo-sequence within a small neighborhood of sampled time points does not really reflect the propagation of sequence information between time collection points. 
\end{comment}
%Chenhui: 这个问题和上面这段话的衔接并不够紧密,上一段话在强调seq2seq,为什么突然跳到inductive bias. 但是inductive bias是引出后面分析的keyword,怎么解决这个问题。

\vspace{-2mm}
\begin{center}
%\textit{What kinds of Inductive Bias should be introduced into Physics-Informed Neural Networks? And how?}
\textit{How can we effectively introduce sequentiality to PINNs?}
\end{center}
\vspace{-2mm}

To answer this question, we need to understand the essential difficulties of training PINNs.
First, PINNs assume temporal continuity, whereas, during their actual training, the spatio-temporal collection points used to construct the PDE residual loss are sampled discretely.
       %In the absence of a well-defined continuous-discrete articulation, the real trajectory is not necessarily recovered correctly in the training process, since the interpolation/extrapolation might be over-smoothed. 
       We define this nature of PINN as   \textit{Continuous-Discrete Mismatch}. 
       In the absence of a well-defined continuous-discrete articulation, the real trajectory of physical system is not necessarily recovered correctly in the training process, since such \textit{Continuous-Discrete Mismatch} would block the propagation of the initial condition.
%So for the successful training of PINNs, we need ensure the computability of the differentiation for time coordinates, i.e., the scale of the gradient is aligned without vanishing or explosion. 
   % Meanwhile, it is necessary to ensure that the model has the ability to propagate information directly over discrete time points, i.e., the ability to sequential modeling.


To respect the inherent \textit{Continuous-Discrete Mismatch}, we reveal that the State Space Models (SSM)~\cite{kalman1960new} can be a good continuous-discrete articulation. 
    SSMs parametrically model a discrete temporal sequence as a continuous dynamic system. An SSM's discrete form approximates the instantaneous states and rates of change of its continuous form via integrating the system's dynamics over each time interval, which more accurately responds to the trajectory of a continuous system. Meanwhile, the SSM unifies the scale of derivatives of different orders, making it easier to be optimized.    
So far, SSMs have shown their insane capacity in language~\cite{gu2023mamba} and vision~\cite{liu2024vmamba} tasks, but its potential for solving PDEs remains unexplored. We propose to construct PINNs with SSMs to unleash their excellent properties of continuous-discrete articulation.

%Therefore, the discrete modeling process for SSMs more accurately responds to the trajectory of a continuous system.
%Moreover, parametric modeling of the instantaneous rate of change allows the differentiation over time to be unified to the same scale, avoiding the hard-to-optimize problem endogenous to PINNs that is caused by the inconsistency of differentiation scales.
%SSMs parametrically model the state evolution of a continuous system. 

%SSMs effectively utilizes discrete time collection points to construct behavior of continuous dynamic systems.


    %In SSMs, there is a smooth articulation between its discrete-time form and continuous-time essence. Moreover, the differentiation of SSM for time is realized by multiplication between state transfer matrices, which effectively aligns the scales of the differentiation of each order. Therefore, it can avoid the PINNs' inherent difficulty of optimization due to the inconsistency of gradient scales.

Next, we remark that the simplicity bias~\cite{shah2020pitfalls} of neural networks is another crucial contributing factor to PINN training difficulty. The simplicity bias will lead the model to choose the pattern with the simplest hypothesis. This results in an over-smoothed solution when approximating PDEs. Because, for data-free PINNs, there might be a very simple function in the feasible domain that can make the residual loss zero. For example, for convection equation, $\bar u(x,t)=0$ leads to zero empirical loss on every collection point except when $t=0$. While the correct pattern is hard to fight against over-smoothed patterns during training.

A major way to eliminate simplicity bias is to construct agreements over diversity predictions~\cite{teney2022evading}. Following this principle, we propose a novel sub-sequence alignment approach, which allows the diverse predictions of time-varying SSMs to form such agreements. %The sub-sequence modeling successfully models the temporal pattern propagation while avoiding the difficulty of optimizing due to the large inertia of modeling long sequences.
Sub-sequence modeling adopts a medium sequence granularity, forming the time dependency that a small sequence fails to capture while avoiding the optimization problem along with the long sequence. 
Meanwhile, the alignment of the sub-sequence predictions ensures the global pattern propagation and the formation of an agreement that eliminates simplicity bias.

%Next, we remark that a suitable granularity of sequence is also crucial for PINNs. 
   % Sequences with too little granularity, for example, the pseudo-sequence within a small neighborhood of collection points implemented by PINNsFormer~\cite{zhao2024pinnsformer}, do not really reflect the propagation of information. 
   % A large sequence, on the other hand, converges to an over-smoothed solution due to the imperfectly reliable supervision of PINNs. This is because, for data-free PINNs, every function in the feasible domain can make the loss of a collection point to zero. For example, for a 1d-convection equation, $\bar u(x,t)=0$ leads to 0-loss on every collection point except when $t=0$. These points contribute uniformly over long sequences with the total loss function, leading to trapping in a local optimum of over-smoothing.
 %First, we remark that simply modeling the sequences over entire time intervals is unreliable, since PINNs are not data-driven, that their supervision is not a direct mapping from a data point sequence to a label sequence but residuals constructed from differential equations. In this physical law-driven learning paradigm, governing in late stages of a long sequence is ineffective, because there is a wide solution space in which functions can all make residual loss low but such functions lack information from previous time. 
%For exerting the influence of the initial conditions, short sequence modeling might be a useful inductive bias. Base on this, we propose a learning scheme with overlapping short sub-sequences. First, it constructs a sub-sequence at each time collection point to predict the output at the current collection point and the next few collection points. Then, at the next collection point, the output of model is softly aligned with previous predictions, as a way to propagate the information defined by the initial conditions over time.
%To address such issues, we propose a moderately granular approach, the sub-sequential modeling for correct propagation. The sub-sequential modeling first constructs a short sub-sequence of the collection points such that the initial condition can be fully propagated over the sub-sequence. Then it recursively propagates over the entire time-domain sequence via a sub-sequence state contrastive alignment.



In this paper, we introduce a novel learning framework to solve physics PDE's numerically, named PINNMamba, which performs time sub-sequences modeling with the Selective SSMs (Mamba)~\cite{gu2023mamba}. PINNMamba successfully captures the temporal dependence within the PDE when training the continuous dynamical systems with discretized collection points.
To the best of our knowledge, PINNMamba is the first data-free SSM-based model that effectively solve physics PDE.
%Our approach eliminates distortions due to discrete-time sampling in PDE's neural modeling of continuous systems. 
Experiments show that PINNMamba outperforms other PINN approaches
such as PINNsFormer~\cite{zhao2024pinnsformer} and KAN~\cite{liu2024kan} on multiple hard problems, achieving a new state-of-the-art.

\textbf{Contributions.} We make the following contributions:
\vspace{-4mm}
\begin{itemize}
    \item We reveal that the mismatch between the discrete nature of the training collection points and the continuous nature of the function approximated by the PINNs is an important factor that prevents the propagation of the initial condition pattern over time in PINNs.
%    \vspace{-6mm}
    \vspace{-2mm}
    \item We also note that the simplicity bias of neural networks is a key contributing factor to the over-smoothing pattern that causes gradual distortion in PINNs.
    \vspace{-2mm}
    \item We propose PINNMamba, which eliminates the discrete-continuity mismatch with SSM and combats simplicity bias with sub-sequential modeling, resulting in state-of-the-art on several PINN benchmarks.
\end{itemize}
\vspace{-2mm}
%\section{Introduction}

In the past few years, Physics-Informed Neural Networks~(PINNs)~\cite{raissi2019physics} have emerged as a novel approach for numerically solving partial differential equations~(PDEs). PINN takes a neural network $u_{\theta}(x,t)$, whose parameters $\theta$ are trained with physics PDE residual loss, as the numerical solution $u(x,t)$ of the PDE, where $x$ and $t$ are spatial and temporal coordinates. The core idea behind PINNs is to take advantage of the universal approximation property of neural networks~\cite{hornik1989multilayer} and automatic differentiation implemented by mainstream deep learning frameworks, such as PyTorch~\cite{paszke2019pytorch} and Tensorflow~\cite{abadi2016tensorflow}, so that PINNs can achieve potentially more precise and efficient PDE solution approximation compared with traditional numerical approaches like finite element methods~\cite{reddy1993introduction}.

Current mainstream PINNs use multilayer perceptrons (MLPs) as their backbone model architecture. However, despite the universal approximation of the MLPs, an arbitrarily accurate approximation to the numerical solution of the PDE is not always guaranteed. This has resulted in the observation of multiple failure modes in PINNs, especially when approximating high-frequency or complex patterns~\cite{krishnapriyan2021characterizing}. As shown in Fig.~\ref{fig1}, these failure modes exhibit a gradual temporal distortion of the solution. This degradation stems from MLPs' inherent lack of inductive bias for modeling time dependencies of a system, which impairs their ability to effectively propagate physical information defined by the initial condition.


\begin{figure}[t!]
    \centering
    \includegraphics{_fig/Fig1.pdf}
    \vspace{-2mm}
    \caption{PINN gradually distorts on convection equation.}
    \label{fig1}
    \vspace{-5mm}
  %  \vspace{-1mm}
\end{figure}

To incorporate \textit{Inductive Bias} into the model design, recent work proposes several sequence-to-sequence approaches~\cite{krishnapriyan2021characterizing,zhao2024pinnsformer,yang2022learning,gao2022earthformer}. \citet{krishnapriyan2021characterizing} introduce a novel method that trains a new network at every time step and then uses the output as the initial condition for the next step. However, this approach suffers from substantial computational and memory overhead while exhibiting poor generalization. Furthermore, Transformer-based approaches~\cite{zhao2024pinnsformer,yang2022learning,gao2022earthformer} are proposed to address the time-dependency issue. Yet, these methods are based on discrete sequences, making their model produce incorrect pattern mutations in some cases due to their ignorance of a basic principle that PINNs approximate continuous dynamical systems. Therefore, there remains a critical question unanswered by current PINNs research:
\vspace{-2mm}
\begin{center}
%\textit{What kinds of Inductive Bias should be introduced into Physics-Informed Neural Networks? And how?}
\textit{How can we effectively introduce sequentiality to PINNs?}
\end{center}
\vspace{-2mm}


First of all, we remark that PINNs have a nature of \textit{Continuous-Discrete Duality}. Paradoxically, although PINNs assume temporal continuity (or even high-order differentiability), the spatio-temporal points used to construct the PDE residual loss are discretely sampled during training. Consequently, without a well-defined continuous-discrete articulation, the real trajectory might not be recovered correctly in the training process due to potential over-smoothing of interpolation/extrapolation.
%So for the successful training of PINNs, we need ensure the computability of the differentiation for time coordinates, i.e., the scale of the gradient is aligned without vanishing or explosion. 
   % Meanwhile, it is necessary to ensure that the model has the ability to propagate information directly over discrete time points, i.e., the ability to sequential modeling.

To respect the inherent \textit{Continuous-Discrete Duality} of PINNs, we propose to use the State Space Models (SSM)~\cite{gu2023mamba,gu2022efficiently} as network backbones. SSMs parametrically model a discrete temporal sequence as a continuous dynamic system. Instead of simple interpolation, its discrete form approximates the instantaneous states and changing rates described by its continuous form by integrating the system's dynamics over each time interval. Therefore, the discrete modeling process for SSMs can respond to the trajectory of a continuous system more precisely. Moreover, the parametric modeling of the instantaneous rate of change allows the differentiation over time to be unified to the same scale, mitigating the hard-to-optimize problem of PINNs caused by the differentiation scale inconsistency.
%SSMs parametrically model the state evolution of a continuous system. 

%SSMs effectively utilizes discrete time collection points to construct behavior of continuous dynamic systems.


    %In SSMs, there is a smooth articulation between its discrete-time form and continuous-time essence. Moreover, the differentiation of SSM for time is realized by multiplication between state transfer matrices, which effectively aligns the scales of the differentiation of each order. Therefore, it can avoid the PINNs' inherent difficulty of optimization due to the inconsistency of gradient scales.

Secondly, we remark that the sequence granularity is crucial for PINNs. Sequences with fine-grained granularity, i.e., the pseudo-sequence within a small neighborhood of collection points adopted in PINNsFormer~\cite{zhao2024pinnsformer}, cannot authentically reflect the propagation of information. On the contrary, using a large sequence can lead to an over-smoothed solution because of the imperfectly reliable supervision of PINNs. In other words, for data-free PINNs, every function in the feasible domain can make the loss at a collection point to zero. For example, in 1d-convection equation, $\bar u(x,t)=0$ leads to 0-loss at every collection point except at $t=0$. As these points contribute uniformly to the total loss over long sequences, the model can become trapped in an over-smoothed local optimum.
 %First, we remark that simply modeling the sequences over entire time intervals is unreliable, since PINNs are not data-driven, that their supervision is not a direct mapping from a data point sequence to a label sequence but residuals constructed from differential equations. In this physical law-driven learning paradigm, governing in late stages of a long sequence is ineffective, because there is a wide solution space in which functions can all make residual loss low but such functions lack information from previous time. 
%For exerting the influence of the initial conditions, short sequence modeling might be a useful inductive bias. Base on this, we propose a learning scheme with overlapping short sub-sequences. First, it constructs a sub-sequence at each time collection point to predict the output at the current collection point and the next few collection points. Then, at the next collection point, the output of model is softly aligned with previous predictions, as a way to propagate the information defined by the initial conditions over time.
To address this challenge, we propose a sub-sequential modeling approach with moderate granularity to enable correct propagation. First, it constructs a short sub-sequence from the collection points to fully propagate the initial condition. Then, it recursively advances through the entire time-domain sequence via a sub-sequence state contrastive alignment.

In this paper, we propose PINNMamba, a novel learning framework for numerically solving physics PDEs. Our approach leverages Selective SSMs (Mamba)~\cite{gu2023mamba} to model temporal sub-sequences, enabling effective capture of temporal dependencies in continuous dynamical systems while training on discretized collection points. PINNMamba successfully addresses the distortions typically induced by discrete-time sampling in neural PDE modeling. Through extensive experimentation, we demonstrate that PINNMamba achieves state-of-the-art performance, consistently outperforming existing approaches such as PINNsFormer~\cite{zhao2024pinnsformer} and QRes ~\cite{bu2021quadratic} over multiple challenging problems.

\textbf{Contributions.} We make the following contributions:
\vspace{-2mm}
\begin{itemize}
    \item We reveal that the mismatch between the discrete nature of the training collection points and the continuous nature of the function approximated by the PINN is an important factor in causing distortion in the propagation of the initial condition over time in the PINN.
    \vspace{-6mm}
    \item We also note that, for PINNs, the lack of direct supervision of their data-label pairs and the multitude of functions conforming to a single differential equation create an intrinsic difficulty in modeling long sequences.
    \vspace{-2mm}
    \item We propose PINNMamba, which eliminates the discrete-continuity mismatch and sub-sequentially models the systems, resulting in state-of-the-art performance on several PINN benchmarks.
\end{itemize}

\section{Preliminary}
Due to page limit, we discuss related works in Appendix~\ref{apx:rw}.


\textbf{Physics-Informed Neural Networks.}
The PDE systems that are defined on spatio-temporal set $\Omega \times [0,T] \subseteq  \mathbb R^{d+1}$ and described by equation constraints, boundary conditions, and initial conditions can be formulated as:
\vspace{-2mm}
\begin{equation}
    \mathcal F(u(x,t)) = 0,\forall (x,t)\in\Omega\times[0,T];
\end{equation}
\begin{equation}
    \mathcal I(u(x,t)) = 0,\forall (x,t)\in\Omega\times\{0\};
\end{equation}
\begin{equation}
    \mathcal B(u(x,t)) = 0,\forall (x,t)\in\partial\Omega\times [0,T],
\end{equation}
where $u:\mathbb R^{d+1}\rightarrow \mathbb R^m$ is the solution of the PDE, $x\in\Omega$ is the spatial coordinate, $\partial\Omega$ is the boundary of $\Omega$, $t \in [0,T]$ is the temporal coordinate and $T$ is the time horizon. The
$\mathcal F,\mathcal I, \mathcal B$ denote the operators defined by PDE equations, initial conditions, and boundary conditions respectively.

A physics-driven PINN first builds a finite collection point set $\chi \subset \Omega\times[0,T]$, and its spatio (temporal) boundary $\partial\chi \subset \partial\Omega\times[0,T]$ ($\chi_0 \subset \Omega\times\{0\}$), then employs a neural network $u_\theta(x,t)$ which is parameterized by $\theta$ to approximate $u(x,t)$ by optimizing the residual loss as defined in Eq.~\ref{equ:loss}:
\vspace{-2mm}
\begin{equation}
    \mathcal L_{\mathcal F}(u_\theta)= \frac{1}{|\chi|}\sum_{(x_i,t_i)\in \chi}\|\mathcal F(u_\theta(x_i,t_i)\|^2;
    \label{equ:lossequ}
\end{equation}
\begin{equation}
    \mathcal L_{\mathcal I}(u_\theta)= \frac{1}{|\chi_0|}\sum_{(x_i,t_i)\in \chi_0}\|\mathcal I(u_\theta(x_i,t_i)\|^2;
    \label{equ:lossinit}
\end{equation}
\begin{equation}
    \mathcal L_{\mathcal B}(u_\theta)= \frac{1}{|\partial\chi|}\sum_{(x_i,t_i)\in \partial\chi}\|\mathcal B(u_\theta(x_i,t_i)\|^2;
    \label{equ:lossbound}
\end{equation}
\begin{equation}
    \mathcal L(u_\theta)=\lambda_{\mathcal F}\mathcal L_{\mathcal F}(u_\theta)+\lambda_{\mathcal I}\mathcal L_{\mathcal I}(u_\theta)+\lambda_{\mathcal B}\mathcal L_{\mathcal B}(u_\theta),
    \label{equ:loss}
\end{equation}
%where $\chi$,$\chi_0$, and $\partial \chi$ are sets of spatio-temporal collection points corresponding to $\Omega\times[0,T]$, $\Omega\times\{0\}$, and $\partial\Omega\times[0,T]$ respectively.
where $\lambda_\mathcal F$,$\lambda_\mathcal I$,$\lambda_\mathcal B$ are the weights for loss that are adjustable by auto-balancing or hyperparameters. $\|\cdot\|$ denotes $l^2$-norm.


\textbf{State Space Models.} An SSM describes and analyzes a continuous dynamic system. It is typically described by:
\begin{align}    \label{equ:hiddenssm}
   \mathbf {\dot h(t)} &= A\mathbf h(t) + B\mathbf x(t),\\
     \mathbf u(t) &= C\mathbf h(t) + D\mathbf x(t),
     \label{equ:outputssm}
\end{align}

where $\mathbf h(t)$ is hidden state of time $t$, $\mathbf {\dot  h}(t)$ is the derivative of $\mathbf h(t)$. $\mathbf x(t)$ is the input state of time $t$, $\mathbf u(t)$ is the output state, and $A,B,C,D$ are state transition matrices. 
    
    In real-world applications, we can only sample in discrete time for building a deep SSM model. 
        We usually omit the term $D\mathbf x(t)$ in deep SSM models because it can be easily implemented by residual connection \cite{he2016deep}. So we create a discrete time counterpart:
\begin{align}
    \mathbf h_k &= \bar A \mathbf h_{k-1}+\bar B \mathbf x_k,
\\
    \mathbf u_k &= C \mathbf h_k,
\end{align}
with discretization rules such as zero-order hold (ZOH):
\begin{align}\label{equ:disc1}
    \bar A &= \exp{(\mathrm{\Delta}A)},\\
    \bar B &= (\mathrm{\Delta}A)^{-1}( \exp{(\mathrm{\Delta}A)}-I)\cdot \mathrm{\Delta}B,
    \label{equ:disc2}
\end{align}
where $\bar A$ and $\bar B$ is discrete time state transfer and input matrix, and $\mathrm{\Delta}$ is a step size parameter. By parameterizing $A$ using HiPPO  matrix~\cite{gu2020hippo}, and parameterizing $(\Delta,B,C)$ with input-dependency, a time-varying Selective SSM can be constructed~\cite{gu2023mamba}. Such a Selective SSM can capture the long-time continual dependencies in dynamic systems. We will argue that this makes SSM a good continuous-discrete articulation for modeling PINN.
%从collection point loss构建的角度,说明初始条件的影响力与位置的关系。
\begin{figure}[t!]
    \centering
    \includegraphics[width=\linewidth]{_fig/fig2}
    \vspace{-8mm}
    \caption{Failure mode of PINN on convection equation, the over-smooth solution brings the losses down to 0 almost everywhere.}
    \label{fig2}
    \vspace{-3mm}
  %  \vspace{-1mm}
\end{figure}

\section{Why PINNs present Failure Modes?}
\label{sec:fail}
%\section{Continuous-Discrete Mismatch of PINNs Can Result in Over-Smooth Failure Modes}

%\CX{The logic of this section is completely out of order and needs to be adjusted.}

A counterintuitive fact of PINNs is that the failure modes are not devoid of optimizing their residual loss to a very low degree.
As shown in Fig.~\ref{fig2}, for the convection equation, the converged PINN almost completely crashes in the domain, but its loss maintains a near-zero degree at almost every collection point. This is the result of the combined effects of the simplicity bias~\cite{shah2020pitfalls,pezeshki2021gradient} of neural networks and the \textit{Continuous-Discrete Mismatch} of PINNs, as shown in Fig.~\ref{fig5}. 
    The simplicity bias is the phenomenon that the model tends to choose the one with the simplest hypothesis among all feasible solutions, which we demonstrate in Fig.~\ref{fig5}~(b).
        \textit{Continuous-Discrete Mismatch} refers to the inconsistency between the continuity of the PDE and the discretization of PINN's training process.
As shown in Eq.~\ref{equ:lossequ} - \ref{equ:lossbound}, to construct the empirical loss for the PINN training process, we need to determine a discrete and finite set of collection points on $\Omega\times[0,T]$. 
This is usually done with a grid or uniform sampling. But a PDE system is usually continuous and its solutions should be regular enough to satisfy the differential operator $\mathcal F$, $\mathcal B$, and $\mathcal I$.

\begin{figure}[t!]
    \centering
    \includegraphics[width=\linewidth]{_fig/fig5}    \vspace{-3mm}
    \caption{The correct Pattern determined by the initial conditions faces two resistances in propagation: (a) the difficulty of propagating information directly through the gradient among discrete collection points, and (b) the need to fight against over-smoothed solutions with near-zero loss caused by simplicity bias.}
    \label{fig5}
    \vspace{-3mm}
  %  \vspace{-1mm}
\end{figure}
%We call this inconsistency between theory and practice \textit{Continuous-Discrete Mismatch}. 

%This \textit{Mismatch} is the essence of the difficulty of optimizing PINNs over complex patterns, since discrete sampling can make over-smoothed solutions that in some cases bring the PINN's empirical loss down to 0, but do not at all match the actual solution.  
\textbf{Continuous-Discrete Mismatch.} \textit{Continuous-Discrete Mismatch} will cause correct local patterns hard to propagate over the global domain.
Because the loss on discrete collection points does not necessarily respond to the correct pattern on the continuous domain, instead, only responds to its small neighborhood. %Without loss of generality, we consider the case where the output of $u$ is one-dimensional:
To show such \textit{Continuous-Discrete Mismatch}, we first present the following theorem:

\begin{theorem}\label{thm:continuous-discrete}
    Let $\chi^* = \{(x^*_1,t^*_1),\dots,(x^*_N,t^*_N)\}\subset \Omega\times[0,T]$. Then for differential operator $\mathcal M$ there exist infinitely many functions
$u_\theta : \Omega \to \mathbb{R}^m$ parametrized by $\theta$ , s.t.
$$ \mathcal{M}(u_\theta(x^*_i,t^*_i)) = 0 \quad \text{for } i=1,\dots,N,$$ $$ 
   \mathcal{M}(u_\theta(x,t)) \neq 0
   \quad \text{for a.e. } x \in \Omega\times[0,T] \setminus \chi^*.$$
\end{theorem}

\begin{figure*}[t!]
    \centering
    \includegraphics[width=\textwidth]{_fig/main}
    \vspace{-6mm}
    \caption{PINNMamba Overview. PINNMamba takes the sub-sequence as input which is a composite of several consecutive collection points on the time axis. For each sub-sequence, the prediction of the first collection point is taken as the output of PINNMamba, while the others are used to align the prediction of different sub-sequences, that can propagate information among time coordinates.}
    \label{fig:main}
    \vspace{-4mm}
  %  \vspace{-1mm}
\end{figure*}

 By Theorem~\ref{thm:continuous-discrete}, enforcing the PDE only at a finite set of points does not guarantee a globally correct solution. This can be performed by simply constructing a Bump function in a small neighborhood of points in $\chi^*$ so that it satisfies $\mathcal{M}(u_\theta(x^*,t^*)) = 0$ for $(x^*,t^*) \in \chi^*$. This means that the information of the equation determined by the initial conditional differential operator $\mathcal I$ may act only on a small neighborhood of collection points with $t = 0$. The other collection points in the $\Omega\times(0,T]$, on the other hand, might fall into a local optimum that can make $\mathcal L_{\mathcal F}(u_\theta)$ defined by Eq.~\ref{equ:lossequ} to near 0. 
 Because the function $u_{\theta}$ determined by $\mathcal F$ and $\mathcal I$ together on the collection points at $t = 0$ may not be generalized outside its small neighborhood. The detailed proof
of Theorem~\ref{thm:continuous-discrete} can be found in Appendix~\ref{apx:proof3_1}.
 
\textbf{Simplicity Bias.} Meanwhile, the simplicity bias of neural networks will make the PINNs always tend to choose the simplest solution in optimizing $\mathcal L_{\mathcal F}(u_\theta)$. This implies that PINN will easily fall into an over-smoothed solution. For example, as shown in Fig.~\ref{fig2}, the PINN's prediction is 0 in most regions. The loss of this over-smoothed feasible solution is almost identical to that of the true solution, and the existence of an insurmountable ridge between the two loss basins results in a PINN that is extremely susceptible to falling into local optimums. As in Fig~\ref{fig5}, the over-smoothed pattern yields an advantage against the correct pattern.

Under the effect of difficulty in passing locally correct patterns to the global due to \textit{Continuous-Discrete Mismatch} and over-smoothing due to simplicity bias, PINNs present failure modes. Therefore, to address such failure modes, the key points in designing the PINN models lie in: (1) a mechanism for information propagation in continuous time and (2) a mechanism to eliminate the simplicity bias of models.


\subsection{Background}
In this paper, we mainly focus on the multi-constraint instruction $I_c$. It can be formulated as a seed instruction incorporated with ${n}$ constraints:
\begin{equation}
\label{eq1}
    I_c = I_s \oplus C_1 \oplus ... \oplus C_n,
\end{equation}
where the seed instructions $I_s$ describe a task, e.g., write a story, while these constraints $\sum_{i=1}^n C_i$ limit the output from different aspects, e.g., format, length, content, etc. $\oplus$ stands for the concatenation operation. 
% \footnote{We provide more examples of multi-constraint instruction in Appx.~\ref{appx:con_samp}.} 
% seed + 1 2 3 4


\subsection{Probing Task} \label{method}
% 我们尝试从约束的难度角度出发 量化指令中的不同排布情况 to achieve this, two problems need to be soluted: how to quantify the 
% 我们参考约束的分类建模
\subsubsection{Task Formulation}
To investigate the impact of constraint order, we introduce a probing task. In this task, the LLM is given multi-constraint instructions with constraints arranged in various orders. The LLM's task is to generate a response that follows all constraints. We evaluate the LLM in two practical scenarios: single-round and multi-round inference. The LLM's responses are then evaluated to determine its performance across various constraints. The overall procedure is illustrated in Fig.~\ref{fig:method}. In the following sections, we will provide a detailed explanation.





\subsubsection{Multi-constraint Instruction Synthesis}\label{sec:ins_cons}
To ensure the generalizability of probing data, we construct the initial multi-constraint instructions which include a variety of tasks and diverse constraint combinations. The multi-constraint instruction synthesis can be further divided into two parts: seed sampling and constraint sampling. 

For the seed sampling, we sample data from three source datasets: (1) Natural Instructions V2~\cite{wang2022supernaturalinstructions}. It is an instruction collection covering more than 1600 NLP tasks. We filter those tasks that are too easy and could potentially conflict with complex constraints, e.g., object classification and sentiment tagging. Then, we randomly sample 52 instructions from the remaining tasks. (2) Self-Instruct~\cite{wang2023self}. We only sample 83 instances from their initial 175 seed instructions which are formulated by humans. (3) Open Assistant~\cite{kopf2024openassistant}. Following the strategy of Suri~\cite{li2023self}, we filter out the first turn of the conversation with the highest quality (marked as rank 0 in the dataset) and sample 65 instances from them. Overall, we obtain 200 seed instructions, where the number of instructions is denoted as $n_{seed}$.

As for the constraint sampling, we first categorize the constraints into 8 groups with 25 fine-grained types~\cite{zhou2023instructionfollowing}. For each type of constraint, we employ 8 different expressions to describe it\footnote{More details are shown in Appx.~\ref{appx:cons_tax}}. Then, we sample $n$ constraints from the constraint taxonomy and use the predefined rules to avoid possible conflicts. To ensure diversity, we repeat the sampling process to obtain $n_{cc}$ distinct constraint combinations, deriving $n_{seed}\times n_{cc}$ multi-constraint instructions.






\subsubsection{Constraint Reordering} \label{reorder}
To quantitatively construct instructions with different constraint orders, here are two questions that need to be answered: (1) \textit{How do we distinguish the disparity of different constraints}? (2) After we order the constraints based on their disparity, \textit{how do we quantitatively describe the disparity of constraint orders}?

An appropriate solution for the first question is to categorize the constraints based on their difficulty~\cite{chen2024sifo}. In this paper, we also sort the constraints based on their difficulty. However, different from existing works which designate the difficulty of the constraints based on handcraft rules, we measure the difficulty of a constraint via the overall accuracy of following it in our probing datasets. The formulation is as follows:
\begin{equation}
\label{eq2}
    % \small
    \text{Dff}_{C_x}= \text{Softmax}(1-\text{Acc}_{C_{x}}), 
\end{equation}
\begin{equation}
    \label{eq3}
    \text{Acc}_{C_x} = \frac{1}{N_{x}}\sum_{i=1}^{N_{x}}c_x^i.
\end{equation}
The $C_x$ refers to a specific type of constraint, the $N_{x}$ stands for the total number of instructions corresponding to the constraint $C_{x}$, and the $c_x^i$ is a binary value to reflect whether the constraint $C_{x}$ is followed in the $i^{th}$ instruction. 

To quantitatively describe the disparity of constraint order, we propose a novel metric called the Constraint Difficulty Distribution Index (CDDI) which quantifies a specific constraint order based on its difficulty distribution. Given the difficulty of different types of constraints, we can readily attain the difficulty distribution of the constraints incorporated in the multi-constraint instructions. Specifically, for a multi-constraint instruction, we rank the incorporated constraints based on their difficulty, from the hardest to the easiest. We set this “hard-to-easy” constraint order as an anchor since it depicts an extreme situation, i.e., we designate the $\text{CDDI}=1$ when the constraints fall in this order. Consequently, akin to the Kendall tau distance~\cite{cicirello2020kendall}, we measure the difficulty distribution of a specific constraint order $o$ by comparing it with the ``hard-to-easy'' constraint order $o_{max}$. The formula is shown as:
\begin{equation}
    \label{eq4}
    \text{CDDI}_{o} = \frac{N_{con}-N_{dis}}{N_{pair}} = \frac{2(N_{con}-N_{dis})}{n(n-1)}.
\end{equation}
where $N_{con}$ and $N_{dis}$ represent the number of concordant and discordant distribution pairs of constraints between $o$ and $o_{max}$, respectively. The $N_{pair}$ is the total number of compared constraint pairs. Overall, we select $n_{dd}$ different difficulty distributions, finally comprising $n_{seed}\times n_{cc}\times n_{dd}$ instances.







\subsubsection{Sequential-Sensitive Inference}
Given the multi-constraint instructions with different constraint orders, we evaluate the model's performance in two common scenarios: single-round inference and multi-round inference. In single-round inference, the LLM is directly given the multi-constraint instructions with different constraint distributions. We argue that different constraint distributions could impose different levels of difficulty on the LLM to handle. The multi-round inference introduces a more typical setting: the user will first provide the LLM with the core intention (i.e., the seed instruction in this work), and then iteratively put forward the constraints in order to obtain a final response.

To evaluate the model performance, apart from the constraint following accuracy mentioned in Eq.(\ref{eq3}), we also verify its constraint-level accuracy $Acc_{cons}$ and instruction-level accuracy $Acc_{inst}$. Corresponding formulas are shown below:
\begin{equation}
    \label{eq5}
    \small
    \text{Acc}_{\text{cons}} = \frac{1}{mn}\sum_{i=1}^{m}\sum_{j=1}^{n}c_i^j,     \text{Acc}_{\text{inst}} = \frac{1}{m}\sum_{i=1}^{m}\prod_{j=1}^{n}c_i^j.
\end{equation}
% \begin{equation}
%     \label{eq5_1}
%     \text{Acc}_{\text{inst}} = \frac{1}{m}\sum_{i=1}^{m}\prod_{j=1}^{n}c_i^j.
% \end{equation}
where $m$ and $n$ refer to the number of instructions and constraints in the instruction, respectively. Similar to Eq.(\ref{eq3}), the $c_i^j$ is a binary value which equals 1 when the constraint is followed in the $i^{th}$ instruction. All the evaluation is conducted by leveraging the script introduced in ~\cite{zhou2023instructionfollowing}. We only evaluate the final responses produced by the LLMs.


\begin{figure}[t] 
    \centering
        \includegraphics[width=0.5\textwidth]{statistic.pdf}
    % \captionsetup{font={small}} 
    \caption{The statistic of different types of constraints in the probing data. The 7cons and 9cons stand for the setting when $n$=7 and $n$=9, respectively.}
    \label{fig:statistic}
\end{figure}
\section{Experiments}
\label{sec: experiments}

\subsection{Experimental Setup}
\label{sec: experimental_setup}
\begin{figure}[t]
\centering \includegraphics[width=\linewidth]{figure_2.png} \caption{The handheld platform configuration, including the radar, IMU, and onboard computer. The experiments are conducted in a room equipped with a motion capture system to obtain accurate ground truth.}
\label{fig2}
\end{figure}

We conduct experiments using three datasets, comprising a total of 15 sequences. One is our self-collected dataset, captured with a handheld platform as shown in Fig.~\ref{fig2}, while the other two are public radar datasets: ICINS2021~\cite{9470842}, and ColoRadar~\cite{kramer2022coloradar}. The sensors on our platform include a 4D FMCW radar, specifically the Texas Instruments AWR1843BOOST, and an Xsens MTI-670-DK IMU. No additional hardware triggers are used between the sensors, and the sensor data is recorded using an Intel NUC i7 onboard computer. The experiments are conducted in an indoor area equipped with a motion capture system to obtain precise ground truth. The extrinsic calibration between the IMU and the radar is performed manually. To highlight the significance of temporal calibration in RIO, we design the dataset with two levels of difficulty. Sequences 1 to 3 feature standard motion patterns, while Sequences 4 to 7 introduce more rotational motion to induce larger errors due to the time offset, providing a clearer demonstration of its impact.

\begin{figure*}[t]
\centering
\includegraphics[width=\linewidth]{figure_3.png}
\caption{Comparison of estimated trajectories with the ground truth. The \textcolor{black}{black} trajectory is the ground truth, the \textcolor{blue}{blue} one is the EKF-RIO, which does not account for temporal calibration, and the \textcolor{red}{red} one is the proposed RIO with online temporal calibration. Results are presented for Sequence 4, ICINS 1, and ColoRadar 1, representing one sequence from each of the three datasets.}
\label{trajectory}
\end{figure*}

In~\cite{9470842}, the ICINS2021 dataset is collected using a Texas Instruments IWR6843AOP radar sensor, an Analog Devices ADIS16448 IMU sensor, and a camera. A microcontroller board is used for active hardware triggering to accurately capture the timing of the radar measurements. Data is collected using both handheld and drone platforms. The handheld sequences, ``carried\_1'' and ``carried\_2'', are referred to as ``ICINS 1'' and ``ICINS 2'', while the drone sequences, ``flight\_1'' and ``flight\_2'', are referred to as ``ICINS 3'' and ``ICINS 4'', respectively. The ground truth is provided through visual-inertial SLAM, which performs multiple loop closures, offering a pseudo-ground truth. In~\cite{kramer2022coloradar}, the ColoRadar dataset is collected using a Texas Instruments AWR1843BOOST radar sensor, a Microstrain 3DM-GX5-25 IMU sensor, and a LiDAR mounted on a handheld platform. No specific synchronization setup is used between the sensors. The sequences, ``arpg\_lab\_run0'' and ``arpg\_lab\_run1'', are referred to as ``ColoRadar 1'' and ``ColoRadar 2'', while the sequences ``ec\_hallways\_run0'' and ``ec\_hallways\_run1'' are referred to as ``ColoRadar 3'' and ``ColoRadar 4'', respectively. The ground truth is generated via LiDAR-inertial SLAM, which includes loop closures, offering a pseudo-ground truth.
\subsection{Evaluation}
\label{sec: evaluation}

\begin{table}[t]
\centering
\caption{Quantitative Results of Fixed Offset and Online Estimation}
\label{fixed_offset}
\resizebox{\linewidth}{!}{
\begin{tblr}{
  cells = {c},
  cell{1}{1} = {r=2}{},
  cell{1}{2} = {r=2}{},
  cell{1}{3} = {r=2}{},
  cell{1}{4} = {c=2}{},
  cell{1}{6} = {c=2}{},
  cell{3}{1} = {r=6}{},
  cell{3}{2} = {r=5}{},
  cell{3}{5} = {fg=red},
  cell{4}{4} = {fg=red},
  cell{5}{4} = {fg=blue},
  cell{5}{5} = {fg=blue},
  cell{5}{6} = {fg=blue},
  cell{5}{7} = {fg=red},
  cell{6}{6} = {fg=red},
  cell{6}{7} = {fg=blue},
  cell{9}{1} = {r=6}{},
  cell{9}{2} = {r=5}{},
  cell{11}{4} = {fg=red},
  cell{11}{5} = {fg=blue},
  cell{11}{6} = {fg=red},
  cell{11}{7} = {fg=red},
  cell{12}{4} = {fg=blue},
  cell{12}{5} = {fg=red},
  cell{12}{6} = {fg=blue},
  cell{12}{7} = {fg=blue},
  hline{1,3,9,15} = {-}{},
  hline{2} = {4-7}{},
}
\textbf{Sequence} & \textbf{Method} &  \textbf{Time Offset (s)}            & \textbf{APE RMSE} &                & \textbf{RPE RMSE} &                   \\
                  &                 &                                      & Trans. (m)        & Rot. (\degree) & Trans. (m)        & Rot. (\degree)    \\
                  \hline
Sequence 1        & Fixed Offset    & 0.0             & 0.985             & 1.872          & 0.264             & 1.230          \\
                  &                 & -0.05           & 0.647             & 7.561          & 0.166             & 1.549          \\
                  &                 & -0.10           & 0.661             & 2.438          & 0.138             & 0.948          \\
                  &                 & -0.15           & 0.826             & 5.151          & \textbf{0.131}    & 1.196          \\
                  &                 & -0.20           & 0.974             & 2.698          & 0.156             & 1.274          \\
                  & Online Est.     & \textbf{-0.114} & \textbf{0.646}    & \textbf{0.935} & 0.132    & \textbf{0.774} \\
Sequence 4        & Fixed Offset    & 0.0             & 1.737             & 25.885         & 0.118             & 4.074          \\
                  &                 & -0.05           & 1.028             & 15.460         & 0.091             & 2.313          \\
                  &                 & -0.10           & 0.635             & 4.655          & 0.061             & 0.994          \\
                  &                 & -0.15           & 0.649             & 4.275          & 0.068             & 1.083          \\
                  &                 & -0.20           & 0.716             & 12.461         & 0.092             & 2.526          \\
                  & Online Est.     & \textbf{-0.115} & \textbf{0.610}    & \textbf{3.099} & \textbf{0.057}    & \textbf{0.944} 
\end{tblr}
}
\vspace{0.3em}
{\raggedright
\noindent\par {\footnotesize \textsuperscript{*}The initial time offset of `Online Est.' is set to 0.0 and the converged values are shown above.}
\noindent\par {\footnotesize \textsuperscript{**}For each sequence, the lowest error values among the fixed offsets are highlighted in \textcolor{red}{red}, and the second-lowest in \textcolor{blue}{blue}.}
\par}

\end{table}
For the performance comparison, the open-source EKF-RIO \cite{9235254}, which uses the same measurement model but does not account for temporal calibration, is employed. All parameters are kept identical to ensure a fair comparison. In the proposed method, the time offset \( t_d \) is initialized to 0.0 seconds for all sequences, reflecting a typical scenario where the initial time offset is unknown. The experimental results are evaluated using the open-source tool EVO \cite{grupp2017evo}. Figure~\ref{trajectory} illustrates the estimated trajectories compared to the ground truth for visual comparison, with one representative result from each dataset. Due to the stochastic nature of the RANSAC algorithm used in radar ego-velocity estimation, the averaged results from 100 trials across all datasets are presented. We compare the root mean square error (RMSE) of both absolute pose error (APE) and relative pose error (RPE), with the RPE calculated at 10-meter intervals.

APE evaluates the overall trajectory by calculating the difference between the ground truth and the estimated poses for all frames, making it particularly useful for assessing the global accuracy of the estimated trajectory. However, APE can be sensitive to significant rotational errors that occur early or in specific sections, potentially overshadowing smaller errors later in the trajectory. In contrast, RPE focuses on local accuracy by aligning poses at regular intervals and calculating the error, allowing discrepancies over shorter segments to be highlighted. When the temporal calibration between sensors is not accounted for, errors can accumulate over time, making RPE evaluation essential. Both metrics offer valuable insights, providing a comprehensive evaluation of the trajectory.

\subsubsection{Self-Collected Dataset}
The purpose of the self-collected dataset is to identify the actual time offset between the IMU and the radar and evaluate its impact on the accuracy of RIO. Since the handheld platform does not utilize a hardware trigger to synchronize the sensors, the exact time offset is unknown and must be estimated. To address this uncertainty, we evaluate the performance of fixed time offsets over a range of values to determine the interval that provides the best accuracy and estimate the likely time offset range.

As shown in Table \ref{fixed_offset}, error values are analyzed with fixed offsets set at 0.05-second intervals for both Sequence 1 and Sequence 4, which feature different motion patterns. The results show that the time offset falls within the -0.10 to -0.15 second range, where the highest accuracy in terms of APE and RPE is observed for both sequences. The proposed method, which utilizes online temporal calibration, estimates the time offset as -0.114 seconds for Sequence 1 and -0.115 seconds for Sequence 4, closely matching the range found through fixed offset testing. In both cases, the proposed method achieves improved performance in terms of both APE and RPE, demonstrates its effectiveness in accurately estimating the time offset.

\begin{table}[t]
\centering
\caption{Quantitative Results of Comparison study on Self-collected dataset}
\label{table_self}
\resizebox{\linewidth}{!}{
\begin{tblr}{
  cells = {c},
  cell{1}{1} = {r=2}{},
  cell{1}{2} = {r=2}{},
  cell{1}{3} = {c=2}{},
  cell{1}{5} = {c=2}{},
  cell{3}{1} = {r=2}{},
  cell{5}{1} = {r=2}{},
  cell{7}{1} = {r=2}{},
  cell{9}{1} = {r=2}{},
  cell{11}{1} = {r=2}{},
  cell{13}{1} = {r=2}{},
  cell{15}{1} = {r=2}{},
  cell{17}{1} = {r=2}{},
  hline{1,3,5,7,9,11,13,15,17,19} = {-}{},
  hline{2} = {3-6}{},
}
{\textbf{Sequence }\\\textbf{(Trajectory Length)}} & {\textbf{Method } \textbf{($\hat{t}_d$)}} & \textbf{APE RMSE } &                & \textbf{RPE RMSE } &                \\
                                                   &                                         & Trans. (m)         & Rot. (\degree)        & Trans. (m)         & Rot. (\degree)        \\
                                                   \hline
{Sequence 1\\(177 m)}                              & {EKF-RIO (N/A)}                        & 0.985              & 1.872           & 0.264              & 1.230          \\
                                                   & {Ours (-0.114 s)}                      & \textbf{0.646}     & \textbf{0.935}  & \textbf{0.132}     & \textbf{0.774} \\
{Sequence 2\\(197 m)}                              & {EKF-RIO}                              & 2.269              & 2.161           & 0.136              & 1.414          \\
                                                   & {Ours (-0.114 s)}                      & \textbf{0.587}     & \textbf{1.650}  & \textbf{0.064}     & \textbf{0.774} \\
{Sequence 3\\(144 m)}                              & {EKF-RIO}                              & 1.368              & 2.331           & 0.167              & 1.347          \\
                                                   & {Ours (-0.113 s)}                      & \textbf{0.414}     & \textbf{1.140}  & \textbf{0.088}     & \textbf{0.613} \\
{Sequence 4\\(197 m)}                              & {EKF-RIO}                              & 1.737              & 25.885          & 0.118              & 4.074          \\
                                                   & {Ours (-0.115 s)}                      & \textbf{0.610}     & \textbf{3.099}  & \textbf{0.057}     & \textbf{0.944} \\
{Sequence 5\\(190 m)}                              & {EKF-RIO}                              & 2.375              & 7.702           & 0.122              & 1.600          \\
                                                   & {Ours (-0.115 s)}                      & \textbf{1.150}     & \textbf{1.304}  & \textbf{0.069}     & \textbf{0.814} \\
{Sequence 6\\(179 m)}                              & {EKF-RIO}                              & 1.267              & 17.907          & 0.117              & 2.828          \\
                                                   & {Ours (-0.111 s)}                      & \textbf{0.661}     & \textbf{2.551}  & \textbf{0.051}     & \textbf{0.809} \\
{Sequence 7\\(223 m)}                              & {EKF-RIO}                              & 2.757              & 10.092          & 0.116              & 1.863          \\
                                                   & {Ours (-0.112 s)}                      & \textbf{1.596}     & \textbf{6.039}  & \textbf{0.057}     & \textbf{1.365} \\
{Average}                                          & {EKF-RIO}                              & 1.822              & 9.707            & 0.148             & 2.051          \\
                                                   & {Ours (-0.113 s)}                      & \textbf{0.809}     & \textbf{2.388}   & \textbf{0.074}    & \textbf{0.870}   
\end{tblr}
}
\end{table}

Since the radar delay is generally larger than IMU delay, the time offset \( t_d \), representing the difference between these delays, typically takes a negative value. To evaluate the robustness of the estimation, different initial values of \( t_d \) ranging from 0.0 to -0.3 seconds are tested. Figure \ref{sq5} illustrates the estimated time offset for each initial setting, along with the 3-sigma boundaries. As \( t_d \) is estimated from radar ego-velocity, it cannot be determined while the platform is stationary. Once the platform starts moving, the filter begins estimating \( t_d \) and quickly converges to a stable value. The filter converges to a stable time offset of -0.114 ± 0.001 seconds in Sequence 1 and -0.115 ± 0.001 seconds in Sequence 4.

Table \ref{table_self} presents the performance comparison between the proposed method with online temporal calibration and EKF-RIO across seven sequences. The proposed method outperforms EKF-RIO, significantly reducing both APE and RPE across all sequences. Specifically, it reduces APE translation error by an average of 56\%, APE rotation error by 75\%, RPE translation error by 50\%, and RPE rotation error by 58\% compared with EKF-RIO. Despite using the same measurement model, the performance improvement is achieved solely by applying propagation and updates based on a common time stream through the proposed online temporal calibration.

On average, the time offset \( t_d \) is estimated to be -0.113 ± 0.002 seconds, confirming consistent temporal calibration throughout the experiments. Compared with LiDAR-inertial and visual-inertial systems, radar-inertial systems exhibit a significantly larger time offset, as shown in Table~\ref{time_offset_comparison}. Given the radar sensor rate (10 Hz), such a large time offset is significant enough to cause a misalignment spanning more than one data frame. These findings highlight the necessity of temporal calibration in RIO, which is crucial for accurate sensor fusion and reliable pose estimation in real-world applications.

\begin{figure}[t]
\centering
\includegraphics[width=\linewidth]{figure_4.png}
\caption{Time offset estimation with 3-sigma boundaries for different initial values in Sequence 1 and 4.}
\label{sq5}
\end{figure}

\begin{table}[t]
\centering
\caption{Comparison of Time Offset in Multi-Sensor Fusion Systems}
\label{time_offset_comparison}
\begin{tabular}{|c|c|c|} 
\hline
\textbf{Systems} & \textbf{Sensor} & \textbf{Time Offset} \\ 
\hline
LiDAR-Inertial~\cite{10113826} & Velodyne VLP-32 & 0.006 s\\ 
\hline
Visual-Inertial~\cite{li2014online} & PointGrey Bumblebee2 & 0.047 s\\ 
\hline
Radar-Inertial & TI AWR1843BOOST & \textbf{0.113 s} \\
\hline
\end{tabular}
\end{table}

\subsubsection{Open Datasets}
Table \ref{opendataset} presents the results from the two open datasets. The ICINS dataset incorporates a hardware trigger for the radar, which we use to validate the accuracy of the time offset estimation for the proposed method. In this setup, a microcontroller sends radar trigger signals, prompting the radar to begin scanning. The radar data is timestamped based on the actual trigger signal, providing a pseudo-ground truth for time offset estimation. Theoretically, if the sensors are time-synchronized through triggers, the time offset \( t_d \) is expected to be close to 0.0 seconds. The proposed method estimates the time offset to be an average of 0.016 ± 0.003 seconds. Despite this slight discrepancy, the proposed method demonstrates comparable or improved performance on average in both APE and RPE compared with EKF-RIO. Although the ICINS dataset includes hardware-triggered signals for the radar, there is no such trigger signal for the IMU in the dataset, which may introduce a delay in IMU measurements. As defined in Eq.~\eqref{time_offset}, we attribute the estimated positive time offset to this IMU delay, explaining the difference from the expected value.

The ColoRadar dataset, widely used for performance comparison in the RIO field, is utilized to assess if the proposed method generalizes well across different datasets. As shown in Table \ref{opendataset}, the proposed method also demonstrates performance improvements over EKF-RIO in terms of both APE and RPE on average. However, the extent of improvement is smaller compared with the self-collected dataset, which can be explained by differences in trajectory characteristics. The radar ego-velocity model utilizes not only the accelerometer but also the gyroscope measurements. As illustrated in Fig.~\ref{trajectory}, the ColoRadar dataset involves movement over a larger area with less rotation, leading to a smaller impact of the time offset on performance. Nonetheless, the proposed method achieves 33\% reduction in RPE translation error, demonstrating its effectiveness even in this less challenging trajectory. On average, the time offset \( t_d \) is estimated to be -0.111 ± 0.003 seconds, similar to the time offset found in the self-collected dataset. This consistency is likely due to the use of the same radar sensor model in both datasets, further validating the reliability of the proposed method across different environments.

\begin{table}[t]
\centering
\caption{Quantitative Results of Comparison study on Open datasets}
\label{opendataset}
\resizebox{\linewidth}{!}{
\begin{tblr}{
  cells = {c},
  cell{1}{1} = {r=2}{},
  cell{1}{2} = {r=2}{},
  cell{1}{3} = {c=2}{},
  cell{1}{5} = {c=2}{},
  cell{3}{1} = {r=2}{},
  cell{5}{1} = {r=2}{},
  cell{7}{1} = {r=2}{},
  cell{9}{1} = {r=2}{},
  cell{11}{1} = {r=2}{},
  cell{13}{1} = {r=2}{},
  cell{15}{1} = {r=2}{},
  cell{17}{1} = {r=2}{},
  cell{19}{1} = {r=2}{},
  cell{21}{1} = {r=2}{},
  hline{1,3,5,7,9,11,13,15,17,19,21,23} = {-}{},
  hline{2-3} = {3-6}{},
}
{\textbf{Sequence }\\\textbf{(Trajectory Length)}}       & \textbf{Method ($\hat{t}_d$)} & \textbf{APE RMSE}        &                                           & \textbf{RPE RMSE}       &                         \\
                        &                               & Trans. (m)               & Rot. (\degree)                                   & Trans. (m)              & Rot. (\degree)                 \\
                        \hline
{ICINS 1\\(295 m)}      & EKF-RIO (N/A)                 & 1.959                    & 10.694                                    & \textbf{0.093}          & \textbf{0.896}          \\
                        & Ours (0.016 s)                & \textbf{1.922}           & \textbf{10.135}                           & 0.098                   & 0.918          \\
{ICINS 2\\(468 m)}      & EKF-RIO                       & 3.830                    & 23.151                                    & \textbf{0.114}          & 1.289                   \\
                        & Ours (0.013 s)                & \textbf{3.198}           & \textbf{19.235}                           & 0.121                   & \textbf{1.076}          \\
{ICINS 3\\(150 m)}      & EKF-RIO                       & \textbf{1.502}           & \textbf{9.905}                            & 0.130                   & \textbf{1.512}           \\
                        & Ours (0.015 s)                & 1.530                    & 10.189                                    & \textbf{0.126}          & 1.553          \\
{ICINS 4\\(50 m)}       & EKF-RIO                       & \textbf{0.213}           & \textbf{2.091}                            & \textbf{0.076}          & \textbf{0.923}           \\
                        & Ours (0.019 s)                & 0.216                    & 2.098                                     & 0.081                   & \textbf{0.923}          \\
Average                 & EKF-RIO                       & 1.876                    & 11.460                                    & \textbf{0.103}          & 1.155                   \\
                        & Ours (0.016 s)                & \textbf{1.716}           & \textbf{10.414}                           & 0.106                   & \textbf{1.117}          \\
                        \hline
{ColoRadar 1\\(178 m) } & EKF-RIO (N/A)                 & 6.556                    & \textbf{\textbf{1.354}}                   & 0.182                   & \textbf{1.071} \\
                        & Ours (-0.110 s)               & \textbf{\textbf{6.173}}  & 1.382                                     & \textbf{\textbf{0.155}} & 1.188                   \\
{ColoRadar 2\\(197 m) } & EKF-RIO                       & \textbf{\textbf{4.747}}  & 1.238                                     & 0.372                   & 1.375                   \\
                        & Ours (-0.114 s)               & 4.826                    & \textbf{\textbf{0.960}}                   & \textbf{\textbf{0.292}} & \textbf{\textbf{1.180}} \\
{ColoRadar 3\\(197 m) } & EKF-RIO                       & \textbf{\textbf{8.307}}  & 1.969                                     & 0.259                   & 1.015                   \\
                        & Ours (-0.108 s)               & 8.550                    & \textbf{\textbf{1.852}}                   & \textbf{\textbf{0.221}} & \textbf{\textbf{0.879}} \\
{ColoRadar 4\\(144 m) } & EKF-RIO                       & 12.111                   & 2.815                                     & 0.488                   & 1.263                   \\
                        & Ours (-0.112 s)               & \textbf{11.946}          & \textbf{2.756}                            & \textbf{0.200}          & \textbf{1.116} \\
Average                 & EKF-RIO                       & 7.930                    & 1.844                                     & 0.325                   & 1.181                   \\
                        & Ours(-0.111 s)                & \textbf{7.874}           & \textbf{1.737}                            & \textbf{0.217}          & \textbf{1.091}          
\end{tblr}
}
\end{table}

%\input{_txt/6_Ablation}

%\section{Related Works}
\label{Related Work}

\noindent\textbf{Non-Blind Video Inpainting.}
% \subsection{Video Inpainting}
% With the rapid development of deep learning, video inpainting has made great progress. 
The video inpainting methods can be roughly divided into three lines: 3D convolution-based~\cite{chang2019free,Kim_2019_CVPR,9558783}, flow-based~\cite{Gao-ECCV-FGVC,Kang2022ErrorCF,Ke2021OcclusionAwareVO,li2022towards,xu2019deep,Zhang_2022_CVPR,zou2020progressive}, and attention-based methods~\cite{cai2022devit,lee2019cpnet,Li2020ShortTermAL,liu2021fuseformer,Ren_2022_CVPR,9010390,srinivasan2021spatial,yan2020sttn,zhang2022flow}. 

%\noindent\textbf{3D convolution-based methods.}
The methods~\cite{chang2019free,Kim_2019_CVPR,9558783} based on 3D convolution usually reconstruct the corrupted contents by directly aggregating complementary information in a local temporal window through 3D temporal convolution. 
%For example,
%Wang et al.~\cite{wang2018video} proposed the first deep learning-based video inpainting network using a 3D encoder-decoder network.
%Further,
%Kim et al.~\cite{Kim_2019_CVPR} aggregated the temporal information of the neighbor frames into missing regions of the target frame by a recurrent 3D-2D feed-forward network.
Nevertheless, they often yield temporally inconsistent completed results due to the limited temporal receptive fields.
%\noindent\textbf{Flow-based methods.} 
The methods~\cite{Gao-ECCV-FGVC,Kang2022ErrorCF,Ke2021OcclusionAwareVO,li2022towards,xu2019deep,Zhang_2022_CVPR,zou2020progressive} based on optical flow treat the video inpainting as a pixel propagation problem. 
Generally, they first introduce a deep flow completion network to complete the optical flow, and then utilize the completed flow to guide the valid pixels into the corrupted regions. 
%For instance, 
%Xu et al.~\cite{xu2019deep} used the flow field completed by a coarse-to-fine deep flow completion network to capture the correspondence between the valid regions and the corrupted regions, and guide relevant pixels into the corrupted regions. 
%Based on this, 
%Gao et al.~\cite{Gao-ECCV-FGVC} further improved the performance of video inpainting by explicitly completing the flow edges. 
%Zou et al.~\cite{zou2020progressive} corrected the spatial misalignment in the temporal feature propagation stage by the completed optical flow. 
However, these methods fail to capture the visible contents of long-distance frames, thus reducing the inpainting performance in the scene of large objects and slowly moving objects. 


%\noindent\textbf{Attention-based methods.} 
Due to its outstanding long-range modeling capacity, attention-based methods, especially transformer-based methods, have shed light on the video inpainting community. 
These methods~\cite{cai2022devit,lee2019cpnet,Li2020ShortTermAL,liu2021fuseformer,Ren_2022_CVPR,9010390,srinivasan2021spatial,yan2020sttn,zhang2022flow} first find the most relevant pixels in the video frame with the corrupted regions by the attention module, and then aggregate them to complete the video frame. 
%For example,
%Zeng et al.~\cite{yan2020sttn} filled the missing regions of multi-frames simultaneously by learning a spatial-temporal transformer network. 
%Further, Liu et al.~\cite{liu2021fuseformer} improved edge details of missing contents by novel soft split and soft composition operations.
Although the existing video inpainting methods have shown promising results, 
%they usually assume that the corrupted regions of the video are known, 
they usually need to elaborately annotate the corrupted regions of each frame in the video,
limiting its application scope.
Unlike these approaches, we propose a blind video inpainting network in this paper, which can automatically identify and complete the corrupted regions in the video.


\begin{figure*}[tb]
\centering%height=3.0cm,width=15.5cm
\includegraphics[scale=0.66]{Fig/Fig_KT.pdf}
\vspace{-0.15cm}
\caption{\textbf{The overview of the proposed blind video inpainting framework}. Our framework are composed of a mask prediction network (MPNet) and a video completion network (VCNet). The former aims to predict the masks of corrupted regions by detecting semantic-discontinuous regions of the frame and utilizing temporal consistency prior of the video, while the latter perceive valid context information from uncorrupted regions using predicted mask to generate corrupted contents.
}
\label{Fig_KT}
\vspace{-0.5cm}
\end{figure*}

\noindent\textbf{Blind Image Inpainting.}
% \subsection{Blind Image Inpainting}
% \subsection{Image Inpainting}
In contrast to video inpainting, image inpainting solely requires consideration of the spatial consistency of the inpainted results. In the few years, the success of deep learning has brought new opportunities to many vision tasks, which promoted the development of a large number of deep learning-based image inpainting methods~\cite{shamsolmoali2023transinpaint,dong2022incremental,liu2022reduce,li2022misf,cao2022learning}. 
As a sub-task of image inpainting, blind image inpainting~\cite{wang2020vcnet,zhao2022transcnn,li2024semid,li2023decontamination,10147235} has been preliminarily explored. 
For example,
Nian et al.~\cite{cai2017blind} proposed a novel blind inpainting method based on a fully convolutional neural network. 
Liu et al.~\cite{BII} designed a deep CNN to directly restore a clear image from a corrupted input. However, these blind inpainting work assumes that the corrupted regions are filled with constant values or Gaussian noise, which may be problematic when corrupted regions contain unknown content. To improve the applicability, Wang et al.~\cite{wang2020vcnet} relaxed this assumption and proposed a two-stage visual consistency network. 
%Jenny et al.~\cite{schmalfuss2022blind} improved inpainting quality by integrating theoretically founded concepts from transform domain methods and sparse approximations into a CNN-based approach. 
Compared with blind image inpainting, blind video inpainting presents an additional challenge in preserving temporal consistency. Naively applying blind image inpainting algorithms on individual video frame to fill corrupted regions will lose inter-frame motion continuity, resulting in flicker artifacts. Inspired by the success of deep learning in blind image inpainting task, we propose the first deep blind video inpainting model in this paper, which provides a strong benchmark for subsequent research.

This work presented \ac{deepvl}, a Dynamics and Inertial-based method to predict velocity and uncertainty which is fused into an EKF along with a barometer to perform long-term underwater robot odometry in lack of extroceptive constraints. Evaluated on data from the Trondheim Fjord and a laboratory pool, the method achieves an average of \SI{4}{\percent} RMSE RPE compared to a reference trajectory from \ac{reaqrovio} with $30$ features and $4$ Cameras. The network contains only $28$K parameters and runs on both GPU and CPU in \SI{<5}{\milli\second}. While its fusion into state estimation can benefit all sensor modalities, we specifically evaluate it for the task of fusion with vision subject to critically low numbers of features. Lastly, we also demonstrated position control based on odometry from \ac{deepvl}.

%\clearpage
\bibliography{_bib/example_paper}
\bibliographystyle{_bib/icml2025}


%%%%%%%%%%%%%%%%%%%%%%%%%%%%%%%%%%%%%%%%%%%%%%%%%%%%%%%%%%%%%%%%%%%%%%%%%%%%%%%
%%%%%%%%%%%%%%%%%%%%%%%%%%%%%%%%%%%%%%%%%%%%%%%%%%%%%%%%%%%%%%%%%%%%%%%%%%%%%%%
% APPENDIX
%%%%%%%%%%%%%%%%%%%%%%%%%%%%%%%%%%%%%%%%%%%%%%%%%%%%%%%%%%%%%%%%%%%%%%%%%%%%%%%
%%%%%%%%%%%%%%%%%%%%%%%%%%%%%%%%%%%%%%%%%%%%%%%%%%%%%%%%%%%%%%%%%%%%%%%%%%%%%%%
%\newpage
\appendix
\onecolumn

\section{Related Works}
\label{apx:rw}

\textbf{Physics-Informed Neural Networks.}
Physics-Informed Neural Networks~\cite{raissi2019physics} are a class of deep learning models designed to solve problems governed by physical laws described in PDEs. 
    They integrate physics-based constraints directly into the training process in the loss function, allowing them to numerically solve many key physical equations, such as Navier-Stokes equations\cite{jin2021nsfnets}, Euler equations~\cite{mao2020physics}, heat equatuons~\cite{cai2021physics}. Several advanced learning schemes such as gPINN~\cite{kharazmi2019variational}, vPINN\cite{yu2022gradient}, and RoPINN\cite{wu2024ropinn}, model architectures such as QRes~\cite{bu2021quadratic}, FLS~\cite{wong2022learning}, PINNsFormer~\cite{zhao2024pinnsformer}, KAN~\cite{liu2024kan,liu2024kanw} are proposed in terms of convergence, optimization, and generalization.

\textbf{Failure Modes in PINNs.}
Despite these efforts, PINN still has some inherently intractable failure modes. 
\citet{krishnapriyan2021characterizing} identify several types of equations that are vulnerable to difficulties in solving by PINNs.  
    These equations are usually manifested by the presence of a parameter in them that makes their pattern behave as a high frequency or a complex state~\cite{pmlr-v235-cho24b}, failing to propagate the initial condition. 
        In such cases, an empirical loss constructed using a collection point can easily fall into an over-smooth solution (e.g. $\bar u(x,t)=0$ can make the loss of all collection points except whose $t=0$ descend to 0 for 1d-wave equations). Several methods regarding optimization~\cite{wu2024ropinn,wang20222}, sampling~\cite{gao2023failure,wu2023comprehensive}, model architecture~\cite{zhao2024pinnsformer,pmlr-v235-cho24b,pmlr-v235-nguyen24c}, transfer learning~\cite{xu2023transfer,pmlr-v235-cho24b} are proposed to mitigate such failure modes. 
            However, the above approaches do not focus on the fact that a PDE system should be modeled as a continuous dynamic, leading to difficulties in generalization over a wide range of problems.




\textbf{State Space Models.} The state space model~\cite{kalman1960new} is a mathematical representation of a physical system in terms of state variables. 
    Modern SSMs~\cite{gu2022efficiently,smith2023simplified,gu2023mamba} combine the representational power of neural networks with their own superior long-range dependency capturing and parallel computing capabilities and thus are widely used in many fields, such as language modeling~\cite{fu2023hungry,poli2023hyena,gu2023mamba,pmlr-v235-dao24a}, computer vision~\cite{pmlr-v235-zhu24f,liu2024vmamba}, and genomics~\cite{gu2023mamba,nguyen2024sequence}. Specifically, Structured SSMs~(S4)~\cite{gu2022efficiently} decomposing the structured state matrices as the sum of a low-rank
and normal terms to improve the efficiency of state-space-based deep models. Further, Selective SSMs (Mamba)~\cite{gu2023mamba} eliminates the Linear Time Invariance~\cite{sain1969invertibility} of SSMs by introducing a gating mechanism, allowing the model to selectively propagate or forget information and greatly enhancing the model performance. In physics, SSMs are used in conjunction with Neural Operator to form a data-driven solution to PDEs~\cite{zheng2024aliasfree,hu2024state}. 
However, these methods are data-driven which lack generalization ability in some scenarios where real data is not available. Unlike these methods, our approach, PINNMamba is fully physics-driven, relying only on residuals constructed using PDEs without any training data.

%



\section{Proof of Theorem \ref{thm:continuous-discrete}}
\label{apx:proof3_1}

We start with a function $v$ such that $\mathcal{M}(v)$ is non-zero almost everywhere. Such a function exists because $\mathcal{M}$ is a non-zero differential operator. For example, if $\mathcal{M}$ is the Laplacian, a non-harmonic function can be chosen.

\begin{lemma}[Existence of Base Function]
    Let $\mathcal{M}$ be a non-degenerate differential operator on $\Omega \times [0,T]$, where $\Omega \subset \mathbb{R}^n$ is a domain. There exists a function $v \in C^\infty(\Omega \times [0,T])$ such that:  
$$
\mathcal{M}(v) \neq 0 \quad \text{for almost every } (x,t) \in \Omega \times [0,T].
$$
\end{lemma}


\begin{proof}
    Since $\mathcal{M}$ is non-degenerate (i.e., not identically zero), there exists at least one function $w \in C^\infty(\Omega \times [0,T])$ and a point $(x_0, t_0) \in \Omega \times [0,T]$ such that:  
   $$
   \mathcal{M}(w)(x_0, t_0) \neq 0.
   $$  
   By continuity of $\mathcal{M}(w)$ (assuming smooth coefficients for $\mathcal{M}$), there is an open neighborhood $U \subset \Omega \times [0,T]$ around $(x_0, t_0)$ where $\mathcal{M}(w) \neq 0$.
   
   Construct a smooth bump function $\phi \in C^\infty(\Omega \times [0,T])$ with:  
   $\phi \equiv 1$ on a smaller neighborhood $V \subset U$,  
   and $\phi \equiv 0$ outside $U$.  
      Define $v_0 = \phi \cdot w$. Then $\mathcal{M}(v_0) = \mathcal{M}(\phi w)$ is non-zero on $V$ and smooth everywhere.  
   Let $\{(x_k, t_k)\}_{k=1}^\infty$ be a countable dense subset of $\Omega \times [0,T]$. For each $k$, repeat the above construction to obtain a function $v_k \in C^\infty(\Omega \times [0,T])$ such that: $\mathcal{M}(v_k) \neq 0$ in a neighborhood $U_k$ of $(x_k, t_k)$,  
   $\text{supp}(v_k) \subset U_k$,  
   and the supports $\{U_k\}$ are pairwise disjoint. 

   Define the function:  
   $$
   v = \sum_{k=1}^\infty \epsilon_k v_k,
   $$  
   where $\epsilon_k > 0$ are chosen such that the series converges in $C^\infty(\Omega \times [0,T])$ (e.g., $\epsilon_k = 2^{-k}/\max\{\|v_k\|_{C^k}, 1\}$).

   The set $\bigcup_{k=1}^\infty U_k$ is open and dense in $\Omega \times [0,T]$. Since $\mathcal{M}(v) \neq 0$ on this dense open set, the zero set $Z = \{(x,t) : \mathcal{M}(v)(x,t) = 0\}$ is contained in the complement of $\bigcup_{k=1}^\infty U_k$, which is nowhere dense and hence has Lebesgue measure zero. Therefore:  
   $$
   \mathcal{M}(v) \neq 0 \quad \text{for almost every } (x,t) \in \Omega \times [0,T].
   $$
\end{proof}

\begin{lemma}[Local Correction Functions]\label{lem:B2}
    Let $\mathcal{M}$ be a non-degenerate differential operator on $\Omega \times [0,T]$, and let $\chi^* = \{(x^*_1,t^*_1),\dots,(x^*_N,t^*_N)\} \subset \Omega \times [0,T]$. There exist smooth functions $\{w_i\}_{i=1}^N \subset C^\infty(\Omega \times [0,T])$ and radii $\epsilon_1, \dots, \epsilon_N > 0$ such that for each $i$:  
    
1. Compact Support: $\text{supp}(w_i) \subset B_{\epsilon_i}(x^*_i,t^*_i)$,  

2. Non-Vanishing Action: $\mathcal{M}(w_i)(x^*_i,t^*_i) \neq 0$, 

3. Disjoint Supports: $B_{\epsilon_i}(x^*_i,t^*_i) \cap B_{\epsilon_j}(x^*_j,t^*_j) = \emptyset$ for $i \neq j$. 
\end{lemma} 


\begin{proof}
    Let $d_{\text{min}} = \min_{i \neq j} \text{dist}\left((x^*_i,t^*_i), (x^*_j,t^*_j)\right)$ be the minimal distance between distinct points in $\chi^*$. For all $i$, choose radii $\epsilon_i > 0$ such that:  
$$
\epsilon_i < \frac{d_{\text{min}}}{2}.
$$  
This ensures the balls $B_{\epsilon_i}(x^*_i,t^*_i)$ are pairwise disjoint.  

For each $(x^*_i,t^*_i)$, since $\mathcal{M}$ is non-degenerate, there exists a smooth function $f_i \in C^\infty(\Omega \times [0,T])$ such that:  
$$
\mathcal{M}(f_i)(x^*_i,t^*_i) \neq 0.
$$  This is because, when $\mathcal{M}$ is non-degenerate, its action cannot vanish on all smooth functions at $(x^*_i,t^*_i)$. For instance, if $\mathcal{M}$ contains a derivative $\partial_{x_k}$, take $f_i = x_k$ near $(x^*_i,t^*_i)$.

Then for each $i$, construct a smooth bump function $\phi_i \in C^\infty(\Omega \times [0,T])$ satisfying:  

1. $\phi_i \equiv 1$ on $B_{\epsilon_i/2}(x^*_i,t^*_i)$, 

2. $\phi_i \equiv 0$ outside $B_{\epsilon_i}(x^*_i,t^*_i)$, 

3. $0 \leq \phi_i \leq 1$ everywhere.  

Therefore, define the localized function:  
$$
w_i = \phi_i \cdot f_i.
$$  
By construction:  

1. $\text{supp}(w_i) \subset B_{\epsilon_i}(x^*_i,t^*_i)$,

2. $w_i = f_i$ on $B_{\epsilon_i/2}(x^*_i,t^*_i)$, so  
$$
\mathcal{M}(w_i)(x^*_i,t^*_i) = \mathcal{M}(f_i)(x^*_i,t^*_i) \neq 0.
$$  

Since $\epsilon_i < \frac{d_{\text{min}}}{2}$, the distance between any two balls $B_{\epsilon_i}(x^*_i,t^*_i)$ and $B_{\epsilon_j}(x^*_j,t^*_j)$ is at least $d_{\text{min}} - 2\epsilon_i > 0$. Thus, the supports of $w_i$ and $w_j$ are disjoint for $i \neq j$. 

Therefore, the functions $\{w_i\}_{i=1}^N$ satisfy all required conditions.  


\end{proof}



We now state the one-dimensional case of Theorem~\ref{thm:continuous-discrete} here:

\begin{lemma}[One-Dimensional Case of Theorem~\ref{thm:continuous-discrete}]
\label{lemma:1d}
    Let $\chi^* = \{(x^*_1,t^*_1),\dots,(x^*_N,t^*_N)\}\subset \Omega\times[0,T]$. Then for differential operator $\mathcal M$ there exist infinitely many functions
$u_\theta : \Omega \to \mathbb{R}$ parametrized by $\theta$ , s.t.
$$ \mathcal{M}(u_\theta(x^*_i,t^*_i)) = 0 \quad \text{for } i=1,\dots,N,$$ $$ 
   \mathcal{M}(u_\theta(x,t)) \neq 0
   \quad \text{for a.e. } x \in \Omega\times[0,T] \setminus \chi^*.$$
\end{lemma}

\begin{proof}
    Define the corrected function:
$$
u_\theta = v + \sum_{i=1}^N \alpha_i w_i,
$$
where $w_i$ is the local correction function defined in Lemma~\ref{lem:B2}, $\alpha_i \in \mathbb{R}$ are scalars chosen such that:
$$
\mathcal{M}(u_\theta)(x_i^*, t_i^*) = \mathcal{M}(v)(x_i^*, t_i^*) + \alpha_i \mathcal{M}(w_i)(x_i^*, t_i^*) = 0.
$$

Since $\mathcal{M}(w_i)(x_i^*, t_i^*) \neq 0$, we can solve for $\alpha_i$:
$$
\alpha_i = -\frac{\mathcal{M}(v)(x_i^*, t_i^*)}{\mathcal{M}(w_i)(x_i^*, t_i^*)}.
$$

Outside the union of supports $\bigcup_{i=1}^N B_{\epsilon_i}(x_i^*, t_i^*)$, we have:
$$
\mathcal{M}(u_\theta) = \mathcal{M}(v) + \sum_{i=1}^N \alpha_i \mathcal{M}(w_i) = \mathcal{M}(v),
$$
since $w_i \equiv 0$ outside $B_{\epsilon_i}(x_i^*, t_i^*)$. By construction, $\mathcal{M}(v) \neq 0$ almost everywhere. 

The parameters $\theta = (\epsilon_1, \dots, \epsilon_N, \alpha_1, \dots, \alpha_N)$ can be varied infinitely by varying $w_i$: The bump functions $w_i$ can be scaled, translated, or reshaped (e.g., Gaussian vs. polynomial) while retaining the properties of Local Correction in Lemma~\ref{lem:B2} and varying $\epsilon_i$: For each $i$, choose $\epsilon_i$ from a continuum $(0, \delta_i)$, where $\delta_i$ ensures disjointness.

Thus, the family $\{u_\theta\}$ is uncountably infinite.

The set $\chi^*$ by definition has Lebesgue measure zero in $\Omega \times [0,T]$. The corrections $\sum_{i=1}^N \alpha_i w_i$ are confined to the measure-zero set $\bigcup_{i=1}^N B_{\epsilon_i}(x_i^*, t_i^*)$. Therefore:
$$
\mathcal{M}(u_\theta) \neq 0 \quad \text{for a.e. } (x,t) \in \Omega \times [0,T] \setminus \chi^*.
$$
\end{proof}


We now generalize Lemma~\ref{lemma:1d} to $m$-dimension, to get Theorem~\ref{thm:continuous-discrete}.

\begin{theorem}[Theorem~\ref{thm:continuous-discrete}]
    Let $\chi^* = \{(x^*_1,t^*_1),\dots,(x^*_N,t^*_N)\}\subset \Omega\times[0,T]$. Then for differential operator $\mathcal M$ there exist infinitely many functions
$u_\theta : \Omega \to \mathbb{R}^m$ parametrized by $\theta$ , s.t.
$$ \mathcal{M}(u_\theta(x^*_i,t^*_i)) = 0 \quad \text{for } i=1,\dots,N,$$ $$ 
   \mathcal{M}(u_\theta(x,t)) \neq 0
   \quad \text{for a.e. } x \in \Omega\times[0,T] \setminus \chi^*.$$
\end{theorem}

\begin{proof}
    It is trivial to generalize the Lemma~\ref{lemma:1d} to the case $u_\theta : \Omega \to \mathbb{R}^m$, by constructing:
    $$
   u_\theta = v + \sum_{i=1}^N \sum_{j=1}^m \alpha_{i,j} w_{i,j},
   $$
   where $ \alpha = (\alpha_{i,j}) \in \mathbb{R}^{N \cdot m} $. Adjust $ \alpha_{i,j} $ such that:
   $$
   \mathcal{M}(u_\theta)(x_i^*, t_i^*) = \mathcal{M}(v)(x_i^*, t_i^*) + \sum_{j=1}^m \alpha_{i,j} \mathcal{M}(w_{i,j})(x_i^*, t_i^*) = 0.
   $$
   This gives a linear system for $ \alpha $, which is solvable because the $ w_{i,j} $ are linearly independent.
\end{proof}


\section{Linear Time-Varying System}
\label{apx:LTI}

To adjust the given Linear Time-Invariant system to a Linear Time-Varying system, we replace the constant matrices $ \bar{A} $, $ \bar{B} $, and $ C $ with their time-varying counterparts $ \bar{A}(k) $, $ \bar{B}(k) $, and $ C(k) $. The state transition matrix $ \bar{A}^{k-i} $ in the LTI system becomes the product of time-varying matrices from time $ i $ to $ k-1 $. The resulting time-varying output equation is:

\begin{equation}
    \mathbf{u}_k = C(k) \Phi(k, 0) \mathbf{h}_0 + C(k) \sum_{i=0}^k \Phi(k, i) \bar{B}(i) \mathbf{x}_i,
\end{equation}



where $ \Phi(k, i) $ is the state transition matrix from time $ i $ to $ k $, defined as:
\begin{equation}
      \Phi(k, i) = \begin{cases} 
    \bar{A}(k-1) \bar{A}(k-2) \cdots \bar{A}(i) & \text{if } k > i, \\
    I & \text{if } k = i.
  \end{cases}
\end{equation}


  
and the term $ \Phi(k, 0) \mathbf{h}_0 $ represents the free response due to the initial condition $ \mathbf{h}_0 $.

The summation $ \sum_{i=0}^k \Phi(k, i) \bar{B}(i) \mathbf{x}_i $ includes contributions from all inputs $ \mathbf{x}_i $ up to time $ k $, with $ \Phi(k, i) \bar{B}(i) $ capturing the time-varying dynamics.

To adjust the Eq.~\ref{equ:timeloss} to a Time-Varying system The state transition term $ \bar{A}^{k-i} $ becomes the time-ordered product $ \Phi(k, i) $, and the output $ \mathbf{u}_k $ now explicitly depends on time-varying dynamics. The adjusted equation becomes:

\begin{equation}
    \sum_{i=1}^M \mathcal{L}_{\mathcal{F}}(u(x_i, k\Delta t)) = \frac{1}{M} \left\| \mathcal{F}\left( \mathbf{1}_M \cdot \mathbf{u}_k \right) \right\|^2= \frac{1}{M} \left\| \mathcal{F}\left( \mathbf{1}_M \cdot \mathbf{u}_k = C(k) \Phi(k, 0) \mathbf{h}_0 + C(k) \sum_{i=0}^k \Phi(k, i) \bar{B}(i) \mathbf{x}_i\right) \right\|^2.
\end{equation}




This modification ensures consistency with the Time-Varying system’s time-dependent parameters while preserving the structure of the original loss function.

\section{PDEs Setups}
\label{apx:setup}

\subsection{1-D Convection}

The 1-D convection equation, also known as the 1-D advection equation, is a partial differential equation that models the transport of a scalar quantity $ u(x,t) $ (such as temperature, concentration, or momentum) due to fluid motion at a constant velocity $ c $. It is a fundamental equation in fluid dynamics and transport phenomena. The equation is given by:
\begin{gather}
    \frac{\partial u}{\partial t} + \beta \frac{\partial u}{\partial x} = 0,\; \forall x \in[0,2\pi], t\in [0,1],\nonumber\\
    u(x,0) = \sin x,\;\forall x \in[0,2\pi],\\
    u(0,t)=u(2\pi,t),\;\forall  t\in [0,1],\nonumber
\end{gather}
where $\beta$ is the constant convection (advection) speed. As $\beta$ increases, the equation will be harder for PINN to approximate. It is a well-known equation with failure mode for PINN. We set $\beta=50$ following common practice~\cite{zhao2024pinnsformer,wu2024ropinn}.

The 1-D convection equation's analytical solution is given by:
\begin{equation}
    u_\text{ana}(x,t) = \sin(x-\beta t).
\end{equation}


\subsection{1-D Reaction}

The 1-D reaction equation is a partial differential equation that models how a chemical species reacts over time and (optionally) varies along a single spatial dimension. The equation is given by:
\begin{gather}
    \frac{\partial u}{\partial t} -\rho u(1-u) = 0,\; \forall x \in[0,2\pi], t\in [0,1],\nonumber\\
    u(x,0) = \exp(-\frac{(x-\pi)^2}{2(\pi/4)^2}),\;\forall x \in[0,2\pi],\\
    u(0,t)=u(2\pi,t),\; \forall  t\in [0,1],\nonumber
\end{gather}
where $\rho$ is the growth rate coefficient. As $\rho$ increases, the equation will be harder for PINN to approximate. It is a well-known equation with failure mode for PINN. We set $\rho=5$ following common practice~\cite{zhao2024pinnsformer,wu2024ropinn}.

The 1-D reaction equation's analytical solution is given by:
\begin{equation}
    u_\text{ana}=\frac{\exp(-\frac{(x-\pi)^2}{2(\pi/4)^2})\exp(\rho t)}{\exp(-\frac{(x-\pi)^2}{2(\pi/4)^2})(\exp(\rho t)-1)+1}.
\end{equation}

\subsection{1-D Wave}

The 1-D wave equation is a partial differential equation that describes how a wave propagates through a medium, such as a vibrating string.  We consider such an equation given by:
\begin{gather}
    \frac{\partial^2 u}{\partial t^2} - 4\frac{\partial^2 u}{\partial x^2} = 0,\; \forall x \in[0,1], t\in [0,1],\nonumber\\
    u(x,0) = \sin(\pi x)+\frac{1}{2}\sin(\beta \pi x), \;\forall x\in[0,1],\\
    \frac{\partial u(x,0)}{\partial t} = 0, \;\forall x\in[0,1],\nonumber\\
    u(0,t)=u(1,t) = 0, \; \forall  t\in [0,1],\nonumber
\end{gather}
where $\beta$ is a wave frequency coefficient. We set $\beta$ as 3 following common practice~\cite{zhao2024pinnsformer,wu2024ropinn}. The wave equation contains second-order derivative terms in the equation and first-order derivative terms in the initial condition, which is considered to be hard to optimize~\cite{wu2024ropinn}. This example illustrates that PINNMamba can better capture the time continuum because its differentiation for time is directly defined by the matrix, whose differential scale is uniform for multiple orders.

The 1-D wave equation's analytical solution is given by:
\begin{equation}
    u_\text{ana}(x,t)=\sin(\pi x)\cos(2\pi t)+\sin(\beta \pi x)\cos(2\beta \pi t).
\end{equation}

\subsection{2-D Navier-Stokes}

The 2-D Navier-Stokes equation describes the motion of fluid in two spatial dimensions $x$ and $y$. It is fundamental in fluid dynamics and is used to model incompressible fluid flows. We consider such an equation given by:
\begin{gather}
    \frac{\partial u}{\partial t} + \lambda_1 (u\frac{\partial u}{\partial x} + v \frac{\partial u}{\partial y}) = - \frac{\partial p}{\partial x} + \lambda_2 (\frac{\partial^2 u}{\partial x^2} + \frac{\partial^2 u}{\partial v^2}), \nonumber \\
    \frac{\partial v}{\partial t} + \lambda_1 (u\frac{\partial v}{\partial x} + v \frac{\partial v}{\partial y}) = - \frac{\partial p}{\partial y} + \lambda_2 (\frac{\partial^2 u}{\partial x^2} + \frac{\partial^2 u}{\partial v^2}),
\end{gather}
where $u(x,y,t)$, $v(x,y,t)$, and $p(x,y,t)$ are the x-coordinate velocity field, y-coordinate velocity field, and pressure field, respectively. We set $\lambda_1 = 1$ and $\lambda_2 = 0.01$ following common practice~\cite{zhao2024pinnsformer,raissi2019physics}. 

The 2-dimensional Navier-Stokes equation doesn't have an analytical solution that can be described by existing mathematical symbols, we take~\citet{raissi2019physics}'s finite-element numerical simulation as ground truth. 

\subsection{PINNNacle}
PINNacle~\cite{hao2023pinnacle} contains 16 hard PDE problems, which can be classified as Burges, Poisson, Heat, Navier-Stokes, Wave, Chaotic, and other High-dimensional problems. We only test PINNmamba on 6 problems, because solving the remaining problems with a sequence-based PINN model will cause an out-of-memory issue, even on the most advanced NVIDIA H100 GPU. Please refer to the original paper of PINNacle~\cite{hao2023pinnacle} for the details of the benchmark.

\section{Training Details}
\label{apx:hyperparam}

\textbf{Hyperparameters.} We provide the training hyperparameters of the main experiments in Table~\ref{tab:hyperpara}.


\begin{table}[H]
\vspace{-3mm}
  \caption{Hyperparameters for main results.}
  
  \centering
    \small
  \begin{tabular}{l|c|c}

    \toprule 
    Model & Hyperparameter Type & Value\\
    \midrule
   \multirow{ 2}{*}{PINN} & network depth & 4\\
   & network width & 512 \\
    \midrule
   \multirow{ 2}{*}{QRes}& network depth & 4 \\
    & network width & 256 \\
    \midrule
    \multirow{ 3}{*}{KAN}   & network width  & [2,5,5,1] \\
     & grid size & 5\\
       & grid\_epsilon  & 1.0 \\
       \midrule
        \multirow{ 7}{*}{PINNsFormer}   & \# of encoder  & 1 \\
     & \# of decoder & 1\\
       & embedding size  & 32 \\
        & attention head  & 2 \\
                & MLP hidden width  & 512 \\
            & sequence length $k$  & 5 \\
            & sequence interval $\Delta t$  & 1e-4 \\
            \midrule
            \multirow{ 7}{*}{PINNMamba}   & \# of encoder  & 1 \\
       & embedding size  & 32 \\
        & $\Delta,B,C$ width & 8 \\
                & MLP hidden width  & 512 \\
            & sequence length $k$  & 7 \\
            & sequence interval $\Delta t$  & 1e-2 \\
                
       

    \bottomrule
  \end{tabular}
  \normalsize
  \label{tab:hyperpara}

\end{table} 

\textbf{Computation Overhead.} We report the training time and memory consumption of baseline models and PINNMamba on the convection equation in Table~\ref{tab:training}. 

We compare the performance of agents trained on data from the InSTA pipeline to agents trained on human demonstrations from WebLINX \citep{WebLINX} and Mind2Web \citep{Mind2Web}, two recent and popular benchmarks for web navigation. Recent works that mix synthetic data with real data control the real data sampling probability in the batch $p_{\text{real}}$ independently from data size \citep{DAFusion}. We employ $p_{\text{real}} = 0.5$ in few-shot experiments and $p_{\text{real}} = 0.8$ otherwise. Shown in Figure~\ref{fig:data-statistics}, our data have a wide spread in performance, so we apply several filtering rules to select high-quality training data. First, we require the evaluator to return \texttt{conf} = 1 that the task was successfully completed, and that the agent was on the right track (this selects data where the actions are reliable, and directly caused the task to be solved). Second, we filter data where the trajectory contains at least three actions. Third, we remove data where the agent encountered any type of server error, was presented with a captcha, or was blocked at any point. These steps produce $7,463$ high-quality demonstrations in which agents successfully completed tasks on diverse websites. We sample 500 demonstrations uniformly at random from this pool to create a diverse test set, and employ the remaining $6,963$ demonstrations to train agents on a mix of real and synthetic data.

\subsection{Improving Data-Efficiency}
\label{sec:few-shot}

\begin{wrapfigure}{r}{0.48\textwidth}
    \centering
    \vspace{-0.8cm}
    \includegraphics[width=\linewidth]{assets/few_shot_results_weblinx_mind2web.pdf}
    \vspace{-0.3cm}
    \caption{\small \textbf{Data from InSTA improves efficiency.} Language model agents trained on mixtures of our data and human demonstrations scale faster than agents trained on human data. In a setting with 32 human actions, adding our data improves \textit{Step Accuracy} by +89.5\% relative to human data for Mind2Web, and +122.1\% relative to human data for WebLINX.}
    \vspace{-0.2cm}
    \label{fig:few-shot-results}
\end{wrapfigure}

In a data-limited setting derived from WebLINX \citep{WebLINX} and Mind2Web \citep{Mind2Web}, agents trained on our data \textit{scale faster with increasing data size} than human data alone. Without requiring laborious human annotations, the data produced by our pipeline leads to improvements on Mind2Web that range from +89.5\% in \textit{Step Accuracy} (the rate at which the correct element is selected and the correct action is performed on that element) with 32 human actions, to +77.5\% with 64 human actions, +13.8\% with 128 human actions, and +12.1\% with 256 human actions. For WebLINX, our data improves by +122.1\% with 32 human actions, and +24.6\% with 64 human actions, and +6.2\% for 128 human actions. Adding our data is comparable in performance gained to doubling the amount of human data from 32 to 64 actions. Performance on the original test sets for Mind2Web and WebLINX appears to saturate as the amount of human data increases, but these benchmark only test agent capabilities for a limited set of 150 popular sites.

\subsection{Improving Generalization} 
\label{sec:generalization}

\begin{wrapfigure}{r}{0.48\textwidth}
    \centering
    \vspace{-1.0cm}
    \includegraphics[width=\linewidth]{assets/diverse_results_weblinx_mind2web.pdf}
    \vspace{-0.3cm}
    \caption{\small \textbf{Our data improves generalization.} We train agents with all human data from the WebLINX and Mind2Web training sets, and resulting agents struggle to generalize to more diverse test data. Adding our data improves generalization by +149.0\% for WebLINX, and +156.3\% for Mind2Web.}
    \vspace{-0.3cm}
    \label{fig:generalization-results}
\end{wrapfigure}

To understand how agents trained on data from our pipeline generalize to diverse real-world sites, we construct a more diverse test set than Mind2Web and WebLINX using 500 held-out demonstrations produced by our pipeline. Shown in Figure~\ref{fig:generalization-results}, we train agents using all human data in the training sets for WebLINX and Mind2Web, and compare the performance with agents trained on 80\% human data, and 20\% data from our pipeline. Agents trained with our data achieve comparable performance to agents trained purely on human data on the official test sets for the WebLINX and Mind2Web benchmarks, suggesting that when enough human data are available, synthetic data may not be necessary. However, when evaluated in a more diverse test set that includes 500 sites not considered by existing benchmarks, agents trained purely on existing human data struggle to generalize. Training with our data improves generalization to these sites by +149.0\% for WebLINX agents, and +156.3\% for Mind2Web agents, with the largest gains in generalization \textit{Step Accuracy} appearing for harder tasks.

\section{Sensitivity Analysis}

PINNMamba can be further improved by hyper-parameters tuning, we test the sub-sequence length, interval and activation selection in this section.

\label{apx:sense}

\textbf{Sub-sequence Length.} We test the effect of different sub-sequence lengths on model performance. As shown in Table~\ref{tab:length}, we test the length of 3, 5, 7, 9, 21.  Length $k =7$ achieves the best performance on reaction and wave equations, while  $k =5$ achieves the best performance on convection equation.

\begin{table}[h]
\centering

\resizebox{\columnwidth}{!}{
\begin{tabular}{c|c|c|c}
\hline
\multicolumn{1}{l}{Sentence Length Range} & \multicolumn{1}{l}{{\#} of Sentences} & \multicolumn{1}{l}{UAS} & \multicolumn{1}{l}{LAS} \\ \hline\hline
1-10           & 270& 97.24& 96.30\\
11-20          & 764& 97.54& 96.35\\
21-30          & 778& 96.87& 95.83\\
31-40          & 433& 96.73& 95.69\\
41-50          & 135& 97.19& 96.12\\
51-60          & 28 & 94.89& 93.77\\
61-70          & 8  & 94.65& 94.26\\ \hline
all            & 2416               & 96.95& 95.88\\ \hline
\end{tabular}}
\caption{A Table of statistics and performance according to sentence length (based on word count)}
\label{tab:length}
\end{table}

\textbf{Sub-Sequence Interval.}  We test the effect of different sub-sequence intervals on model performance. As shown in Table~\ref{tab:interval}, we test the intervals of $2e-3$, $5e-3$, $1e-2$, $1e-1$. The interval $\Delta t =1e-2$ achieves the best performance on convection and wave equations, while $\Delta t = 5e-3$ achieves the best performance on reaction. Note that, when $\Delta t = 1e-1$, we cannot build the sub-sequence contrastive alignment.
\begin{table}[H]
\vspace{-3mm}
  \caption{Results with different Sub-Sequence Interval of PINNmamba, $k$ is set to 7.}
  
  \centering
    \small
  \begin{tabular}{c|cc|cc|cc}

    \toprule 
      &\multicolumn{2}{c}{Convection }&\multicolumn{2}{c}{Reaction}&\multicolumn{2}{c}{Wave}\\
    \cmidrule(lr){2-3}\cmidrule(lr){4-5}\cmidrule(lr){6-7}
   Interval & rMAE & rRMSE & rMAE & rRMSE & rMAE & rRMSE\\
   \midrule
   2e-3 &0.0249& 0.0257 & 0.0739 & 0.1389 &0.1693 &0.1903  \\
 5e-3 & 0.0243& 0.0287& \textbf{0.0083} & \textbf{0.0185} & 0.2492 & 0.2690 \\
 1e-2 &\textbf{0.0188} & \textbf{0.0201} &0.0094 &0.0217 & \textbf{0.0197} &  \textbf{0.0199}\\
 1e-1 & 1.2169 &1.3480 &0.4324& 0.5034   &0.0666 &  0.0703\\

    

   

    \bottomrule
  \end{tabular}
  \normalsize
  \label{tab:interval}

\end{table}


\textbf{Activation Function.} We test the activation function's effect on the performance of PINNMamba. We report the results of ReLU~\cite{nair2010rectified}, Tanh~\cite{fan2000extended}, and Wavelet~\cite{zhao2024pinnsformer} in Table~\ref{tab:activation}.
\begin{table}[H]
\vspace{-3mm}
  \caption{Results with different activation function in PINNmamba.}
  
  \centering
    \small
  \begin{tabular}{c|cc|cc|cc}

    \toprule 
      &\multicolumn{2}{c}{Convection }&\multicolumn{2}{c}{Reaction}&\multicolumn{2}{c}{Wave}\\
    \cmidrule(lr){2-3}\cmidrule(lr){4-5}\cmidrule(lr){6-7}
   Activation & rMAE & rRMSE & rMAE & rRMSE & rMAE & rRMSE\\
   \midrule
   ReLU & 0.4695& 0.4722 & 0.0865 & 0.1583 &0.4139 &0.4203  \\
 Tanh & 0.4531& 0.4601& 0.0299 & 0.0568  & 0.3515 & 0.3539  \\
Wavelet &0.0188 & 0.0201 &0.0094 &0.0217 & 0.0197 &  0.0199\\


    

   

    \bottomrule
  \end{tabular}
  \normalsize
  \label{tab:activation}

\end{table}

\section{Complex Problem Results}
\label{apx:comp}

\subsection{2D Navier-Stokes Equations}

Although PINN can already handle Navier-Stokes equations well, we still tested the performance of PINN Mamba on Navier-Stokes equations to check the generalization performance of our method on high-dimensional problems. As shown in Fig.~\ref{fig:nss}, our method achieves good results on Navier-Stokes pressure prediction. Since there is no initial condition information for the N-S equation for pressure, we took the data from the only collection point for pattern alignment.

\begin{figure*}[t]
    \centering
    \includegraphics[width=\textwidth]{_fig/nss}
    \vspace{-3mm}
    \caption{The ground truth solution, prediction (top), and absolute error (bottom) on Navier-Stokes equations.}
    \label{fig:nss}
    %\vspace{-5mm}
  %  \vspace{-1mm}
\end{figure*}

\subsection{PINNacle Benchmark}

Like PINNsFormer, PINNMamba is a sequence model. The sequence model suffers from Out-of-Memory problems when dealing with some of the problems in the PINNacle Benchmark~\cite{hao2023pinnacle}, even when running on the advanced Nvidia H100 GPU. We report here the results of the sub-problems for which results can be obtained in Table~\ref{tab:pinnacle}. PINNMamba can solve the Out-of-Memory problem by distributed training over multiple cards, which we leave as a follow-up work.

\begin{table}[H]
\vspace{-3mm}
  \caption{Results on PINNacle. Baseline results are from RoPINN paper~\cite{wu2024ropinn}. OOM means Out-of-Memroy.}
  
  \centering
    \small
  \begin{tabular}{c|cc|cc|cc}

    \toprule 
      &\multicolumn{2}{c}{PINN }&\multicolumn{2}{c}{PINNsFormer}&\multicolumn{2}{c}{PINNMamba}\\
    \cmidrule(lr){2-3}\cmidrule(lr){4-5}\cmidrule(lr){6-7}
   Equation & rMAE & rRMSE & rMAE & rRMSE & rMAE & rRMSE\\
   \midrule
   Burgers 1d-C &1.1e-2& 3.3e-2 & 9.3e-3 & 1.4e-2 & 3.7e-3 & 1.1e-3 \\
 Burgers 2d-C & 4.5e-1& 5.2e-1&  OOM & OOM & OOM & OOM \\
 Poisson 2d-C & 7.5e-1 & 6.8e-1 & 7.2e-1 & 6.6e-1 & 6.2 e-1 & 5.7e-1 \\
Poisson 2d-CG & 5.4e-1 & 6.6e-1 & 5.4e-1& 6.3e-1 & 1.2e-1 & 1.4e-1 \\
Poisson 3d-CG & 4.2e-1 & 5.0e-1 & OOM& OOM & OOM & OOM \\
Poisson 2d-MS & 7.8e-1 & 6.4e-1 & 1.3e+0& 1.1e+0 & 7.2e-1& 6.0e-1 \\
Heat 2d-VC & 1.2e+0 & 9.8e-1 & OOM& OOM &OOM &OOM  \\
Heat 2d-MS & 4.7e-2 & 6.9e-2 &OOM & OOM &OOM &OOM  \\
Heat 2d-CG & 2.7e-2 & 2.3e-2 & OOM& OOM &OOM &OOM  \\
NS 2d-C & 6.1e-2 & 5.1e-2 & OOM& OOM & OOM& OOM \\
NS 2d-CG & 1.8e-1 & 1.1e-1 & 1.0e-1& 7.0e-2 & 1.1e-2& 7.8e-3  \\
Wave 1d-C & 5.5e-1 & 5.5e-1 & 5.0e-1 & 5.1e-1 & 1.0e-1 & 1.0e-1 \\ 
Wave 2d-CG & 2.3e+0 & 1.6e+0 &OOM & OOM  &OOM &OOM  \\
Chaotic GS & 2.1e-2 & 9.4e-2 & OOM& OOM & OOM &OOM  \\
High-dim PNd & 1.2e-3 & 1.1e-3 &OOM &OOM  &OOM &OOM  \\
High-dim HNd & 1.2e-2 & 5.3e-3 &OOM &OOM  &OOM &OOM  \\
   

    \bottomrule
  \end{tabular}
  \normalsize
  \label{tab:pinnacle}

\end{table}



%%%%%%%%%%%%%%%%%%%%%%%%%%%%%%%%%%%%%%%%%%%%%%%%%%%%%%%%%%%%%%%%%%%%%%%%%%%%%%%
%%%%%%%%%%%%%%%%%%%%%%%%%%%%%%%%%%%%%%%%%%%%%%%%%%%%%%%%%%%%%%%%%%%%%%%%%%%%%%%


\end{document}


% This document was modified from the file originally made available by
% Pat Langley and Andrea Danyluk for ICML-2K. This version was created
% by Iain Murray in 2018, and modified by Alexandre Bouchard in
% 2019 and 2021 and by Csaba Szepesvari, Gang Niu and Sivan Sabato in 2022.
% Modified again in 2023 and 2024 by Sivan Sabato and Jonathan Scarlett.
% Previous contributors include Dan Roy, Lise Getoor and Tobias
% Scheffer, which was slightly modified from the 2010 version by
% Thorsten Joachims & Johannes Fuernkranz, slightly modified from the
% 2009 version by Kiri Wagstaff and Sam Roweis's 2008 version, which is
% slightly modified from Prasad Tadepalli's 2007 version which is a
% lightly changed version of the previous year's version by Andrew
% Moore, which was in turn edited from those of Kristian Kersting and
% Codrina Lauth. Alex Smola contributed to the algorithmic style files.
