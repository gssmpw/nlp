\section{Related Works}
\subsection{LoRA and its variants}
As one of the parameter-efficient fine-tuning methods, LoRA~\citep{DBLP:conf/iclr/HuSWALWWC22} has been widely adopted. However, it still has room for improvement in terms of accuracy.
Current enhancements follow two main pathways: optimizing initialization and refining the fine-tuning process. For initialization, methods like PiSSA~\citep{meng2024pissa} and MiLoRA~\citep{wang2024miloraharnessingminorsingular} use Singular Value Decomposition on base model weights, with PiSSA focusing on principal singular values and MiLoRA on minor ones for initializing LoRA before fine-tuning.
% To address this issue, current works primarily follow two pathways. The first pathway focuses on optimizing LoRA’s initialization, such as PiSSA~\citep{meng2024pissa} and MiLoRA~\citep{wang2024miloraharnessingminorsingular}. They apply Singular Value Decomposition to the weights of the base model. PiSSA utilizes the principle singular values, while MiLoRA uses the minor singular values for initializing LoRA before fine-tuning. 
For fine-tuning, DoRA~\citep{pmlr-v235-liu24bn} splits LoRA’s fine-tuning into magnitude and direction components. ReLoRA~\citep{lialin2024relora} continuously merges the fine-tuned LoRA modules into the base model.  AdaLoRA~\citep{zhang2023adaptive} and IncreLoRA~\citep{DBLP:journals/corr/abs-2308-12043} optimize rank allocation across modules.
% , the former starts with a large rank and gradually prunes it, while the latter begins with a small rank and progressively increases it. 
Unlike these approaches, BeamLoRA revisits the foundational aspects of LoRA and recognizes the varying importance of ranks within a module. It compresses less important ranks to free up space for expanding the important ones, thereby allowing them to be better optimized.
\subsection{Model Pruning and Expansion}
Model pruning is typically used to remove redundant parameters in models, thereby improving efficiency~\citep{kurtic-etal-2022-optimal,ma2023llmpruner}. Unlike previous works, our primary goal for pruning is to free up space for expanding important parameters.
% Model pruning is typically used to remove redundant parameters in models, thereby improving efficiency~\citep{kurtic-etal-2022-optimal,ma2023llmpruner}. Unlike previous works, our primary goal for pruning is to free up space for expanding important parameters. 
% Model expansion can be categorized into two types: depth expansion and width expansion. Our approach is more aligned with the latter.
Model Width expansion is first introduced by Net2Net~\citep{DBLP:journals/corr/ChenGS15} and applied to CNNs. bert2BERT~\citep{chen-etal-2022-bert2bert} extends this method to the pre-training of language models, and the recent work Scaling Smart~\citep{DBLP:journals/corr/abs-2409-12903} applies width expansion to large scale base models. Unlike these approaches, we focus on parameter-efficient fine-tuning and propose compressing unimportant parameters within a limited space to expand important ones for better performance. Additionally, due to the shorter nature of the fine-tuning process than pre-training, we propose to use historical states to break symmetry in expansion, thereby ensuring fast convergence.