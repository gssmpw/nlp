
\documentclass{article} % For LaTeX2e
\usepackage{iclr2025_conference,times}
\usepackage{natbib}

% Optional math commands from https://github.com/goodfeli/dlbook_notation.
%%%%% NEW MATH DEFINITIONS %%%%%

\usepackage{amsmath,amsfonts,bm}
\usepackage{derivative}
% Mark sections of captions for referring to divisions of figures
\newcommand{\figleft}{{\em (Left)}}
\newcommand{\figcenter}{{\em (Center)}}
\newcommand{\figright}{{\em (Right)}}
\newcommand{\figtop}{{\em (Top)}}
\newcommand{\figbottom}{{\em (Bottom)}}
\newcommand{\captiona}{{\em (a)}}
\newcommand{\captionb}{{\em (b)}}
\newcommand{\captionc}{{\em (c)}}
\newcommand{\captiond}{{\em (d)}}

% Highlight a newly defined term
\newcommand{\newterm}[1]{{\bf #1}}

% Derivative d 
\newcommand{\deriv}{{\mathrm{d}}}

% Figure reference, lower-case.
\def\figref#1{figure~\ref{#1}}
% Figure reference, capital. For start of sentence
\def\Figref#1{Figure~\ref{#1}}
\def\twofigref#1#2{figures \ref{#1} and \ref{#2}}
\def\quadfigref#1#2#3#4{figures \ref{#1}, \ref{#2}, \ref{#3} and \ref{#4}}
% Section reference, lower-case.
\def\secref#1{section~\ref{#1}}
% Section reference, capital.
\def\Secref#1{Section~\ref{#1}}
% Reference to two sections.
\def\twosecrefs#1#2{sections \ref{#1} and \ref{#2}}
% Reference to three sections.
\def\secrefs#1#2#3{sections \ref{#1}, \ref{#2} and \ref{#3}}
% Reference to an equation, lower-case.
\def\eqref#1{equation~\ref{#1}}
% Reference to an equation, upper case
\def\Eqref#1{Equation~\ref{#1}}
% A raw reference to an equation---avoid using if possible
\def\plaineqref#1{\ref{#1}}
% Reference to a chapter, lower-case.
\def\chapref#1{chapter~\ref{#1}}
% Reference to an equation, upper case.
\def\Chapref#1{Chapter~\ref{#1}}
% Reference to a range of chapters
\def\rangechapref#1#2{chapters\ref{#1}--\ref{#2}}
% Reference to an algorithm, lower-case.
\def\algref#1{algorithm~\ref{#1}}
% Reference to an algorithm, upper case.
\def\Algref#1{Algorithm~\ref{#1}}
\def\twoalgref#1#2{algorithms \ref{#1} and \ref{#2}}
\def\Twoalgref#1#2{Algorithms \ref{#1} and \ref{#2}}
% Reference to a part, lower case
\def\partref#1{part~\ref{#1}}
% Reference to a part, upper case
\def\Partref#1{Part~\ref{#1}}
\def\twopartref#1#2{parts \ref{#1} and \ref{#2}}

\def\ceil#1{\lceil #1 \rceil}
\def\floor#1{\lfloor #1 \rfloor}
\def\1{\bm{1}}
\newcommand{\train}{\mathcal{D}}
\newcommand{\valid}{\mathcal{D_{\mathrm{valid}}}}
\newcommand{\test}{\mathcal{D_{\mathrm{test}}}}

\def\eps{{\epsilon}}


% Random variables
\def\reta{{\textnormal{$\eta$}}}
\def\ra{{\textnormal{a}}}
\def\rb{{\textnormal{b}}}
\def\rc{{\textnormal{c}}}
\def\rd{{\textnormal{d}}}
\def\re{{\textnormal{e}}}
\def\rf{{\textnormal{f}}}
\def\rg{{\textnormal{g}}}
\def\rh{{\textnormal{h}}}
\def\ri{{\textnormal{i}}}
\def\rj{{\textnormal{j}}}
\def\rk{{\textnormal{k}}}
\def\rl{{\textnormal{l}}}
% rm is already a command, just don't name any random variables m
\def\rn{{\textnormal{n}}}
\def\ro{{\textnormal{o}}}
\def\rp{{\textnormal{p}}}
\def\rq{{\textnormal{q}}}
\def\rr{{\textnormal{r}}}
\def\rs{{\textnormal{s}}}
\def\rt{{\textnormal{t}}}
\def\ru{{\textnormal{u}}}
\def\rv{{\textnormal{v}}}
\def\rw{{\textnormal{w}}}
\def\rx{{\textnormal{x}}}
\def\ry{{\textnormal{y}}}
\def\rz{{\textnormal{z}}}

% Random vectors
\def\rvepsilon{{\mathbf{\epsilon}}}
\def\rvphi{{\mathbf{\phi}}}
\def\rvtheta{{\mathbf{\theta}}}
\def\rva{{\mathbf{a}}}
\def\rvb{{\mathbf{b}}}
\def\rvc{{\mathbf{c}}}
\def\rvd{{\mathbf{d}}}
\def\rve{{\mathbf{e}}}
\def\rvf{{\mathbf{f}}}
\def\rvg{{\mathbf{g}}}
\def\rvh{{\mathbf{h}}}
\def\rvu{{\mathbf{i}}}
\def\rvj{{\mathbf{j}}}
\def\rvk{{\mathbf{k}}}
\def\rvl{{\mathbf{l}}}
\def\rvm{{\mathbf{m}}}
\def\rvn{{\mathbf{n}}}
\def\rvo{{\mathbf{o}}}
\def\rvp{{\mathbf{p}}}
\def\rvq{{\mathbf{q}}}
\def\rvr{{\mathbf{r}}}
\def\rvs{{\mathbf{s}}}
\def\rvt{{\mathbf{t}}}
\def\rvu{{\mathbf{u}}}
\def\rvv{{\mathbf{v}}}
\def\rvw{{\mathbf{w}}}
\def\rvx{{\mathbf{x}}}
\def\rvy{{\mathbf{y}}}
\def\rvz{{\mathbf{z}}}

% Elements of random vectors
\def\erva{{\textnormal{a}}}
\def\ervb{{\textnormal{b}}}
\def\ervc{{\textnormal{c}}}
\def\ervd{{\textnormal{d}}}
\def\erve{{\textnormal{e}}}
\def\ervf{{\textnormal{f}}}
\def\ervg{{\textnormal{g}}}
\def\ervh{{\textnormal{h}}}
\def\ervi{{\textnormal{i}}}
\def\ervj{{\textnormal{j}}}
\def\ervk{{\textnormal{k}}}
\def\ervl{{\textnormal{l}}}
\def\ervm{{\textnormal{m}}}
\def\ervn{{\textnormal{n}}}
\def\ervo{{\textnormal{o}}}
\def\ervp{{\textnormal{p}}}
\def\ervq{{\textnormal{q}}}
\def\ervr{{\textnormal{r}}}
\def\ervs{{\textnormal{s}}}
\def\ervt{{\textnormal{t}}}
\def\ervu{{\textnormal{u}}}
\def\ervv{{\textnormal{v}}}
\def\ervw{{\textnormal{w}}}
\def\ervx{{\textnormal{x}}}
\def\ervy{{\textnormal{y}}}
\def\ervz{{\textnormal{z}}}

% Random matrices
\def\rmA{{\mathbf{A}}}
\def\rmB{{\mathbf{B}}}
\def\rmC{{\mathbf{C}}}
\def\rmD{{\mathbf{D}}}
\def\rmE{{\mathbf{E}}}
\def\rmF{{\mathbf{F}}}
\def\rmG{{\mathbf{G}}}
\def\rmH{{\mathbf{H}}}
\def\rmI{{\mathbf{I}}}
\def\rmJ{{\mathbf{J}}}
\def\rmK{{\mathbf{K}}}
\def\rmL{{\mathbf{L}}}
\def\rmM{{\mathbf{M}}}
\def\rmN{{\mathbf{N}}}
\def\rmO{{\mathbf{O}}}
\def\rmP{{\mathbf{P}}}
\def\rmQ{{\mathbf{Q}}}
\def\rmR{{\mathbf{R}}}
\def\rmS{{\mathbf{S}}}
\def\rmT{{\mathbf{T}}}
\def\rmU{{\mathbf{U}}}
\def\rmV{{\mathbf{V}}}
\def\rmW{{\mathbf{W}}}
\def\rmX{{\mathbf{X}}}
\def\rmY{{\mathbf{Y}}}
\def\rmZ{{\mathbf{Z}}}

% Elements of random matrices
\def\ermA{{\textnormal{A}}}
\def\ermB{{\textnormal{B}}}
\def\ermC{{\textnormal{C}}}
\def\ermD{{\textnormal{D}}}
\def\ermE{{\textnormal{E}}}
\def\ermF{{\textnormal{F}}}
\def\ermG{{\textnormal{G}}}
\def\ermH{{\textnormal{H}}}
\def\ermI{{\textnormal{I}}}
\def\ermJ{{\textnormal{J}}}
\def\ermK{{\textnormal{K}}}
\def\ermL{{\textnormal{L}}}
\def\ermM{{\textnormal{M}}}
\def\ermN{{\textnormal{N}}}
\def\ermO{{\textnormal{O}}}
\def\ermP{{\textnormal{P}}}
\def\ermQ{{\textnormal{Q}}}
\def\ermR{{\textnormal{R}}}
\def\ermS{{\textnormal{S}}}
\def\ermT{{\textnormal{T}}}
\def\ermU{{\textnormal{U}}}
\def\ermV{{\textnormal{V}}}
\def\ermW{{\textnormal{W}}}
\def\ermX{{\textnormal{X}}}
\def\ermY{{\textnormal{Y}}}
\def\ermZ{{\textnormal{Z}}}

% Vectors
\def\vzero{{\bm{0}}}
\def\vone{{\bm{1}}}
\def\vmu{{\bm{\mu}}}
\def\vtheta{{\bm{\theta}}}
\def\vphi{{\bm{\phi}}}
\def\va{{\bm{a}}}
\def\vb{{\bm{b}}}
\def\vc{{\bm{c}}}
\def\vd{{\bm{d}}}
\def\ve{{\bm{e}}}
\def\vf{{\bm{f}}}
\def\vg{{\bm{g}}}
\def\vh{{\bm{h}}}
\def\vi{{\bm{i}}}
\def\vj{{\bm{j}}}
\def\vk{{\bm{k}}}
\def\vl{{\bm{l}}}
\def\vm{{\bm{m}}}
\def\vn{{\bm{n}}}
\def\vo{{\bm{o}}}
\def\vp{{\bm{p}}}
\def\vq{{\bm{q}}}
\def\vr{{\bm{r}}}
\def\vs{{\bm{s}}}
\def\vt{{\bm{t}}}
\def\vu{{\bm{u}}}
\def\vv{{\bm{v}}}
\def\vw{{\bm{w}}}
\def\vx{{\bm{x}}}
\def\vy{{\bm{y}}}
\def\vz{{\bm{z}}}

% Elements of vectors
\def\evalpha{{\alpha}}
\def\evbeta{{\beta}}
\def\evepsilon{{\epsilon}}
\def\evlambda{{\lambda}}
\def\evomega{{\omega}}
\def\evmu{{\mu}}
\def\evpsi{{\psi}}
\def\evsigma{{\sigma}}
\def\evtheta{{\theta}}
\def\eva{{a}}
\def\evb{{b}}
\def\evc{{c}}
\def\evd{{d}}
\def\eve{{e}}
\def\evf{{f}}
\def\evg{{g}}
\def\evh{{h}}
\def\evi{{i}}
\def\evj{{j}}
\def\evk{{k}}
\def\evl{{l}}
\def\evm{{m}}
\def\evn{{n}}
\def\evo{{o}}
\def\evp{{p}}
\def\evq{{q}}
\def\evr{{r}}
\def\evs{{s}}
\def\evt{{t}}
\def\evu{{u}}
\def\evv{{v}}
\def\evw{{w}}
\def\evx{{x}}
\def\evy{{y}}
\def\evz{{z}}

% Matrix
\def\mA{{\bm{A}}}
\def\mB{{\bm{B}}}
\def\mC{{\bm{C}}}
\def\mD{{\bm{D}}}
\def\mE{{\bm{E}}}
\def\mF{{\bm{F}}}
\def\mG{{\bm{G}}}
\def\mH{{\bm{H}}}
\def\mI{{\bm{I}}}
\def\mJ{{\bm{J}}}
\def\mK{{\bm{K}}}
\def\mL{{\bm{L}}}
\def\mM{{\bm{M}}}
\def\mN{{\bm{N}}}
\def\mO{{\bm{O}}}
\def\mP{{\bm{P}}}
\def\mQ{{\bm{Q}}}
\def\mR{{\bm{R}}}
\def\mS{{\bm{S}}}
\def\mT{{\bm{T}}}
\def\mU{{\bm{U}}}
\def\mV{{\bm{V}}}
\def\mW{{\bm{W}}}
\def\mX{{\bm{X}}}
\def\mY{{\bm{Y}}}
\def\mZ{{\bm{Z}}}
\def\mBeta{{\bm{\beta}}}
\def\mPhi{{\bm{\Phi}}}
\def\mLambda{{\bm{\Lambda}}}
\def\mSigma{{\bm{\Sigma}}}

% Tensor
\DeclareMathAlphabet{\mathsfit}{\encodingdefault}{\sfdefault}{m}{sl}
\SetMathAlphabet{\mathsfit}{bold}{\encodingdefault}{\sfdefault}{bx}{n}
\newcommand{\tens}[1]{\bm{\mathsfit{#1}}}
\def\tA{{\tens{A}}}
\def\tB{{\tens{B}}}
\def\tC{{\tens{C}}}
\def\tD{{\tens{D}}}
\def\tE{{\tens{E}}}
\def\tF{{\tens{F}}}
\def\tG{{\tens{G}}}
\def\tH{{\tens{H}}}
\def\tI{{\tens{I}}}
\def\tJ{{\tens{J}}}
\def\tK{{\tens{K}}}
\def\tL{{\tens{L}}}
\def\tM{{\tens{M}}}
\def\tN{{\tens{N}}}
\def\tO{{\tens{O}}}
\def\tP{{\tens{P}}}
\def\tQ{{\tens{Q}}}
\def\tR{{\tens{R}}}
\def\tS{{\tens{S}}}
\def\tT{{\tens{T}}}
\def\tU{{\tens{U}}}
\def\tV{{\tens{V}}}
\def\tW{{\tens{W}}}
\def\tX{{\tens{X}}}
\def\tY{{\tens{Y}}}
\def\tZ{{\tens{Z}}}


% Graph
\def\gA{{\mathcal{A}}}
\def\gB{{\mathcal{B}}}
\def\gC{{\mathcal{C}}}
\def\gD{{\mathcal{D}}}
\def\gE{{\mathcal{E}}}
\def\gF{{\mathcal{F}}}
\def\gG{{\mathcal{G}}}
\def\gH{{\mathcal{H}}}
\def\gI{{\mathcal{I}}}
\def\gJ{{\mathcal{J}}}
\def\gK{{\mathcal{K}}}
\def\gL{{\mathcal{L}}}
\def\gM{{\mathcal{M}}}
\def\gN{{\mathcal{N}}}
\def\gO{{\mathcal{O}}}
\def\gP{{\mathcal{P}}}
\def\gQ{{\mathcal{Q}}}
\def\gR{{\mathcal{R}}}
\def\gS{{\mathcal{S}}}
\def\gT{{\mathcal{T}}}
\def\gU{{\mathcal{U}}}
\def\gV{{\mathcal{V}}}
\def\gW{{\mathcal{W}}}
\def\gX{{\mathcal{X}}}
\def\gY{{\mathcal{Y}}}
\def\gZ{{\mathcal{Z}}}

% Sets
\def\sA{{\mathbb{A}}}
\def\sB{{\mathbb{B}}}
\def\sC{{\mathbb{C}}}
\def\sD{{\mathbb{D}}}
% Don't use a set called E, because this would be the same as our symbol
% for expectation.
\def\sF{{\mathbb{F}}}
\def\sG{{\mathbb{G}}}
\def\sH{{\mathbb{H}}}
\def\sI{{\mathbb{I}}}
\def\sJ{{\mathbb{J}}}
\def\sK{{\mathbb{K}}}
\def\sL{{\mathbb{L}}}
\def\sM{{\mathbb{M}}}
\def\sN{{\mathbb{N}}}
\def\sO{{\mathbb{O}}}
\def\sP{{\mathbb{P}}}
\def\sQ{{\mathbb{Q}}}
\def\sR{{\mathbb{R}}}
\def\sS{{\mathbb{S}}}
\def\sT{{\mathbb{T}}}
\def\sU{{\mathbb{U}}}
\def\sV{{\mathbb{V}}}
\def\sW{{\mathbb{W}}}
\def\sX{{\mathbb{X}}}
\def\sY{{\mathbb{Y}}}
\def\sZ{{\mathbb{Z}}}

% Entries of a matrix
\def\emLambda{{\Lambda}}
\def\emA{{A}}
\def\emB{{B}}
\def\emC{{C}}
\def\emD{{D}}
\def\emE{{E}}
\def\emF{{F}}
\def\emG{{G}}
\def\emH{{H}}
\def\emI{{I}}
\def\emJ{{J}}
\def\emK{{K}}
\def\emL{{L}}
\def\emM{{M}}
\def\emN{{N}}
\def\emO{{O}}
\def\emP{{P}}
\def\emQ{{Q}}
\def\emR{{R}}
\def\emS{{S}}
\def\emT{{T}}
\def\emU{{U}}
\def\emV{{V}}
\def\emW{{W}}
\def\emX{{X}}
\def\emY{{Y}}
\def\emZ{{Z}}
\def\emSigma{{\Sigma}}

% entries of a tensor
% Same font as tensor, without \bm wrapper
\newcommand{\etens}[1]{\mathsfit{#1}}
\def\etLambda{{\etens{\Lambda}}}
\def\etA{{\etens{A}}}
\def\etB{{\etens{B}}}
\def\etC{{\etens{C}}}
\def\etD{{\etens{D}}}
\def\etE{{\etens{E}}}
\def\etF{{\etens{F}}}
\def\etG{{\etens{G}}}
\def\etH{{\etens{H}}}
\def\etI{{\etens{I}}}
\def\etJ{{\etens{J}}}
\def\etK{{\etens{K}}}
\def\etL{{\etens{L}}}
\def\etM{{\etens{M}}}
\def\etN{{\etens{N}}}
\def\etO{{\etens{O}}}
\def\etP{{\etens{P}}}
\def\etQ{{\etens{Q}}}
\def\etR{{\etens{R}}}
\def\etS{{\etens{S}}}
\def\etT{{\etens{T}}}
\def\etU{{\etens{U}}}
\def\etV{{\etens{V}}}
\def\etW{{\etens{W}}}
\def\etX{{\etens{X}}}
\def\etY{{\etens{Y}}}
\def\etZ{{\etens{Z}}}

% The true underlying data generating distribution
\newcommand{\pdata}{p_{\rm{data}}}
\newcommand{\ptarget}{p_{\rm{target}}}
\newcommand{\pprior}{p_{\rm{prior}}}
\newcommand{\pbase}{p_{\rm{base}}}
\newcommand{\pref}{p_{\rm{ref}}}

% The empirical distribution defined by the training set
\newcommand{\ptrain}{\hat{p}_{\rm{data}}}
\newcommand{\Ptrain}{\hat{P}_{\rm{data}}}
% The model distribution
\newcommand{\pmodel}{p_{\rm{model}}}
\newcommand{\Pmodel}{P_{\rm{model}}}
\newcommand{\ptildemodel}{\tilde{p}_{\rm{model}}}
% Stochastic autoencoder distributions
\newcommand{\pencode}{p_{\rm{encoder}}}
\newcommand{\pdecode}{p_{\rm{decoder}}}
\newcommand{\precons}{p_{\rm{reconstruct}}}

\newcommand{\laplace}{\mathrm{Laplace}} % Laplace distribution

\newcommand{\E}{\mathbb{E}}
\newcommand{\Ls}{\mathcal{L}}
\newcommand{\R}{\mathbb{R}}
\newcommand{\emp}{\tilde{p}}
\newcommand{\lr}{\alpha}
\newcommand{\reg}{\lambda}
\newcommand{\rect}{\mathrm{rectifier}}
\newcommand{\softmax}{\mathrm{softmax}}
\newcommand{\sigmoid}{\sigma}
\newcommand{\softplus}{\zeta}
\newcommand{\KL}{D_{\mathrm{KL}}}
\newcommand{\Var}{\mathrm{Var}}
\newcommand{\standarderror}{\mathrm{SE}}
\newcommand{\Cov}{\mathrm{Cov}}
% Wolfram Mathworld says $L^2$ is for function spaces and $\ell^2$ is for vectors
% But then they seem to use $L^2$ for vectors throughout the site, and so does
% wikipedia.
\newcommand{\normlzero}{L^0}
\newcommand{\normlone}{L^1}
\newcommand{\normltwo}{L^2}
\newcommand{\normlp}{L^p}
\newcommand{\normmax}{L^\infty}

\newcommand{\parents}{Pa} % See usage in notation.tex. Chosen to match Daphne's book.

\DeclareMathOperator*{\argmax}{arg\,max}
\DeclareMathOperator*{\argmin}{arg\,min}

\DeclareMathOperator{\sign}{sign}
\DeclareMathOperator{\Tr}{Tr}
\let\ab\allowbreak


\usepackage{hyperref}
\usepackage{url}
\usepackage{graphicx}
\usepackage{booktabs}
\usepackage{siunitx} % for number formatting, optional
\usepackage{booktabs}
\usepackage{threeparttable}
\usepackage[utf8]{inputenc}
\usepackage{amsmath, amssymb}
\usepackage{graphicx}
\usepackage{enumitem}
\usepackage{hyperref}

\title{ChameleonLLM: Batch-Aware Dynamic Low-Rank Adaptation via Inference-Time Clusters}

% Authors must not appear in the submitted version. They should be hidden
% as long as the \iclrfinalcopy macro remains commented out below.
% Non-anonymous submissions will be rejected without review.

\author{Kamer Ali Yuksel \& Hassan Sawaf \\
        aiXplain Inc., San Jose, CA, USA \\
  \texttt{\{kamer, hassan\}@aixplain.com} \\}

% The \author macro works with any number of authors. There are two commands
% used to separate the names and addresses of multiple authors: \And and \AND.
%
% Using \And between authors leaves it to \LaTeX{} to determine where to break
% the lines. Using \AND forces a linebreak at that point. So, if \LaTeX{}
% puts 3 of 4 authors names on the first line, and the last on the second
% line, try using \AND instead of \And before the third author name.

\newcommand{\fix}{\marginpar{FIX}}
\newcommand{\new}{\marginpar{NEW}}

\iclrfinalcopy % Uncomment for camera-ready version, but NOT for submission.
\begin{document}

\maketitle

\begin{abstract}
Recent advances in large language models (LLMs) have shown remarkable performance across diverse tasks. However, these models are typically deployed with fixed weights, which limits their ability to adapt dynamically to the variability inherent in real-world data during inference. This paper introduces ChameleonLLM, a novel framework that enables inference-time adaptation of LLMs by leveraging batch-aware clustering and on-the-fly generation of low-rank updates. Unlike traditional fine-tuning approaches such as Low-Rank Adaptation (LoRA) or methods that rely on a fixed set of pre-learned uniforms (changeable masks), our method dynamically generates adaptive modifications to the decoder weights based on the aggregated statistics of clustered batches. By intelligently grouping similar inputs and computing context-aware low-rank updates via a hyper-network, ChameleonLLM achieves significant performance gains, outperforming conventional LoRA methods while eliminating the overhead of maintaining multiple expert models. Our experiments highlight the potential of our approach to serve as a versatile and highly adaptive solution for language model inference. ChameleonLLM is \href{https://anonymous.4open.science/r/ChamaleonLLM/}{open-sourced} to ensure the reproducibility of our experiments.
\end{abstract}

\section{Introduction}

Large language models have revolutionized natural language processing by demonstrating unprecedented text generation, summarization, translation, and beyond capabilities. Despite their impressive performance, a persistent limitation is their static nature during inference: once deployed, the weights of these models remain fixed regardless of the variability in input data. This static configuration can lead to suboptimal performance when encountering novel or contextually distinct inputs. Traditionally, fine-tuning methods such as Low-Rank Adaptation (LoRA) have been employed during training to inject task-specific updates into the model parameters in a computationally efficient manner \citep{hu2021lora}. LoRA achieves this by freezing most of the pre-trained model’s parameters and introducing trainable low-rank matrices that serve as corrections. However, even with these efficient updates, the low-rank modifications remain static during inference, meaning that the model cannot adjust to nuances in the input data it receives on-the-fly. 

ChameleonLLM is designed to address this limitation by enabling the model to adapt its decoder weights during inference based on the structure and statistics of the input batch. Inputs are grouped into clusters based on their semantic and syntactic similarities. By leveraging precomputed token embeddings, the inference engine identifies coherent groups within each batch, ensuring that similar examples are processed together. Rather than relying on pre-learned and fixed uniforms (changeable masks) or static LoRA updates, ChameleonLLM employs a hyper-network \citep{ha2016hypernetworks} to generate low-rank adaptation parameters in real-time. This dynamic generation is based on the aggregated statistics of the clustered batch, allowing the model to tailor its adaptations to the prevailing context. This approach not only removes the necessity for multiple expert models or a vast wardrobe of pre-stored masks but also leverages the collective context of the batch, leading to enhanced performance across a range of tasks. The primary contributions of this work are three-fold: (1) We introduce a method to dynamically generate low-rank updates based on batch statistics, resulting in a self-adaptive model during inference. (2)  We employ a clustering-based on normalized token embeddings that groups similar inputs, ensuring context-aware adaptation. (3) We provide experimental results demonstrating the superiority of ChameleonLLM over traditional LoRA fine-tuning.

\section{Related Work}

LoRA has emerged as a popular method for fine-tuning large pre-trained models with minimal computational resources. The fundamental idea behind LoRA is to freeze the pre-trained weights and introduce a pair of trainable, low-rank matrices that approximate the necessary updates for a given task. By operating in a low-dimensional subspace, LoRA significantly reduces the number of trainable parameters while enabling effective adaptation. Despite its success, standard LoRA is inherently static during inference. Once the low-rank updates are learned, they are fixed and applied uniformly, regardless of the heterogeneity of the incoming data. This static nature can limit the model's responsiveness to dynamic input distribution or context changes. Several methods have been proposed that explore dynamic and parameter-efficient adaptation from various perspectives. For instance, \citet{chen2024multiplicative} introduced a multiplicative sparse factorization technique that leverages the low-rank structure in fine-tuning, showing that a single adjustment may not be sufficient for robust performance. Similarly, \citet{kim2024single} demonstrated that a single linear layer can yield task-adapted low-rank matrices, highlighting the potential for dynamic low-rank adaptations. \citet{xiao2023task} with task-agnostic low-rank adapters for unseen English dialects, has also demonstrated the power of adaptive methods by addressing robustness across diverse linguistic scenarios.

Building on these advancements, recent work has sought to further enhance adaptability through more modular and context-sensitive designs. \cite{ostapenko2024modular} proposed a modular approach for LLMs by building and reusing a library of LoRAs. Unlike standard LoRA—which applies a single static low-rank update during inference—this modular method dynamically selects or combines specialized LoRA modules based on the task or context, thereby enabling a more flexible and responsive adaptation to diverse and evolving input distributions. In contrast, an alternative strategy employs pre-learned changeable masks \citep{sun2025text}. This approach involves training a set of masks—each tailored to specific tasks or data distributions—that are blended or selected during inference via a softmax-based selector informed by a task embedding or similar indicator. While these methods introduce dynamic adjustment, they come with some practical limitations, such as:
\begin{itemize}
    \item \textbf{Storage Overhead:} A large set of masks must be maintained, and these masks need to be loaded and blended during inference, increasing both memory and computational demands.
    \item \textbf{Adaptation Flexibility:} Even though blending introduces some variability, the masks are fine-tuned individually and not co-optimized for joint adaptation. This can lead to suboptimal performance when the masks are applied in a blended fashion.
    \item \textbf{Task Limitation:} Many of these approaches rely on task embeddings, which restrict adaptation to known task categories rather than capturing the broader context of the input batch.
\end{itemize}

Another line of research has focused on hypernetwork-based task conditioning and multi-task learning. As an early attempt, \citet{karimi2021parameter} proposed a shared hypernetwork framework for parameter‑efficient multi‑task fine‑tuning. Later, \citet{he2022hyperprompt} introduced Hyperprompt, a method that conditions transformers using hypernetwork‑generated prompts, while \citet{ivison2022hyperdecoders} presented hyperdecoders capable of producing instance‑specific decoders for multi‑task NLP. More recently, \citet{zheng2024dlo} proposed a strategy that dynamically adjusts model depth on a per‑sample basis by generating context‑dependent low‑rank updates via a hypernetwork, and \citet{schug2024attention} reinterpreted the attention mechanism as a hypernetwork, embedding dynamic parameter generation in the model’s core architecture. Lastly, \citet{ortiz2024hyperloader} integrated hypernetwork‑based LoRA and adapter layers into multi‑task transformers for sequence labeling, while \citet{lv2024hyperlora} leveraged constrained low‑rank adapter generation to achieve efficient cross‑task generalization.


Recent research has also explored using task description embeddings derived from powerful models, such as GPT-4, to generate low-rank updates on a per-sample basis \citep{charakorn2024instant}. This method generates the low-rank parameters for each input based on a task representation. While effective in certain scenarios, this per-sample adaptation may not fully exploit the shared context available when processing batches of similar inputs. ChameleonLLM differentiates itself by using batch-level context for adaptation. By clustering inputs and computing aggregated statistics, our method generates low-rank updates that reflect the collective characteristics of the batch. Aggregating across a batch helps mitigate the noise or outlier effects in single-sample adaptation. A uniform adaptation across a cluster of similar inputs can lead to more coherent and consistent outputs. By not being limited to task-specific embeddings, our method can adapt to a broader range of scenarios, including open-domain and instruction-based tasks. Methods that use task embeddings typically generate per-sample low-rank updates based on a fixed task identifier. In contrast:
\begin{itemize}
    \item \textbf{Batch vs. Sample-Level Adaptation:} Our method leverages batch-level statistics, leading to a more coherent adaptation that benefits from the collective context of multiple samples.
    \item \textbf{Flexibility in Unstructured Environments:} Task embeddings require clear task boundaries, whereas our context-based approach naturally adapts to real-world data.
    \item \textbf{Out-of-Sample Robustness:} Aggregating over a batch reduces the over-fitting sensitivity of training, an important advantage when dealing with heterogeneous or noisy datasets.
\end{itemize}

To sum up, in contrast to per-sample or depth-focused adaptation methods, ChameleonLLM exploits batch-level context to generate adaptive low-rank updates. Rather than adapting each sample independently, we cluster similar inputs and compute aggregated token embedding statistics across the batch; these statistics then drive our hyper-network to generate low-rank updates that reflect the collective context. ChameleonLLM achieves a coherent and context-aware adaptation that naturally overcomes the limitations of fixed, per-sample updates and the storage overhead associated with large sets of pre-learned masks, such as the impact of noisy or outlier samples, and the storage overhead associated with maintaining pre-learned masks. To conclude, while traditional methods like static LoRA and changeable masks offer efficient adaptation strategies, the recent surge in dynamic approaches demonstrates a growing consensus: incorporating context—whether at the sample or task level—can significantly enhance the adaptability and robustness of LLMs. Our work builds on previous research by uniquely leveraging batch-level clustering to generate context-aware low-rank updates, thereby improving the hypernetwork-based adaptation using aggregated batch statistics.

\section{Methodology}

In this work, we introduce \textbf{ChameleonLLM}, a dynamic adaptation framework leveraging batch-aware clustering and hyper-network–driven low-rank adaptations to efficiently tailor the model to the input context. ChameleonLLM can be built upon any pre-trained causal language model (e.g., GPT‑2) by augmenting the conventional transformer layers and language modeling head with low-rank adaptation modules. The innovation lies in the dynamic generation of low-rank parameters based on batch-aware statistics. This process involves two major components: (1) batch-aware clustering and (2) adaptive low-rank update generation. The central hypothesis is that inputs arriving in a batch often exhibit significant similarity---whether semantic, syntactic, or stylistic. By grouping similar inputs, we can extract shared context and exploit common features in the batch. This grouping not only aids in noise reduction by averaging but also ensures that the low-rank updates are tailored to the collective characteristics of the batch rather than isolated samples. Using the normalized token embeddings, we perform k‑means clustering. The number of clusters is chosen based on the batch size to ensure each cluster is large enough to provide robust statistics while capturing meaningful variations between groups. The clustering algorithm minimizes the Euclidean distance between points and cluster centroids, and through iterative refinement, it assigns each input to a cluster that best represents its features. Once the clusters are defined, the batch is reconstructed such that each mini-batch contains inputs from the same cluster. This reorganization is critical because it ensures that subsequent low-rank adaptations are computed on a homogenous set of inputs. 

The core idea of dynamic adaptation is to generate low-rank modifications specific to the current batch's context. This work proposes a more flexible alternative to LoRA that uses a hyper-network. Instead of applying a fixed set of low-rank updates (as in standard LoRA), ChameleonLLM employs a hyper-network to produce these parameters on-the-fly. The hyper-network takes as input the mean of the token embeddings from the entire batch, processes this aggregated statistic through several fully connected layers with non-linear activations, and outputs the parameters for the low-rank update. This on-the-fly generation allows the model to adjust to the specifics of the batch dynamically, capturing nuances that a static update might miss. The overall architecture maintains the integrity of the pre-trained model while introducing minimal additional parameters. The transformer layers are wrapped with LoRA modules that operate similarly to conventional implementations, ensuring that most of the model’s capacity remains unchanged. For the LM head, the hyper-network variant ensures that adaptation is context-dependent. During inference, the model first processes the batch to compute the necessary embeddings to cluster the inputs and then generates a custom low-rank update for the LM head based on the cluster's aggregated statistics. The final output is produced by applying this adapted LM head to the last hidden states of the transformer.

\subsection{Implementation}

To leverage the inherent similarity between samples, we first cluster the normalized token embeddings using the k-means algorithm.  This ensures that mini-batches are contextually coherent and provide reliable aggregate statistics for conditioning the hyper-network. The number of clusters is determined dynamically (typically by dividing the dataset size by the batch size) to ensure each cluster is sufficiently large to provide reliable aggregate statistics while preserving the similarity among examples. Once clustering is complete, we construct the training and validation data loaders that generate mini-batches composed of homogeneous examples from the same cluster. This \emph{clustered batching} strategy is critical because it ensures that the subsequent dynamic adaptation via the hyper-network is conditioned on inputs with similar features, enhancing the adaptability and robustness of the generated low-rank updates. By integrating detailed data preprocessing, batch-aware clustering, and dynamic low-rank adaptation, our methodology efficiently leverages the strengths of a pre-trained transformer backbone while adapting to the specific context of each batch. 


In our architecture, the transformer layers are augmented with the standard LoRA modules, while the LM head is replaced with a hyper-network LoRA module. Designed specifically for the language modeling (LM) head, this module dynamically predicts update matrices based on the aggregated token embeddings from the current batch. The hyper-network processes the mean token embedding through several fully connected layers with non-linear activations (e.g., ReLU) and dropout, ultimately outputting the low-rank matrices (denoted as A and/or B). Initialization is performed using Xavier and small-scale normal initialization. A dedicated procedure handles wrapping by freezing the original model parameters and applying the appropriate LoRA modifications. This ensures that only the LoRA and hyper-network components are trainable, thereby preserving the representational capacity of the pre-trained model. The model is trained using a standard next-token prediction objective with cross-entropy loss. During each training iteration, the following steps are executed:
\begin{enumerate}
    \item \textbf{Transformer Processing}: The input batch of clustered examples, is passed through the frozen transformer layers augmented with LoRA, yielding hidden states for each token.
    
    \item \textbf{Dynamic LM Head Adaptation}: The last hidden state is passed to the LM head. Simultaneously, the LM head receives the aggregated token embeddings (computed as the mean over the batch) as input, allowing the hyper-network to generate dynamic low-rank updates.
    
    \item \textbf{Prediction and Loss Computation}: The adapted LM head produces logits for next-token prediction, which are compared against the target tokens to compute the cross-entropy loss.
    
    \item \textbf{Optimization}: Gradients are computed using the AdamW optimizer, and only the LoRA and hyper-network parameters are updated.
\end{enumerate}

\section{Experiments}

For our experiments, we have used WikiText-2 \citep{merity2016pointer} and Alpaca \citep{alpaca} datasets, providing meaningful benchmarks due to their diverse and natural language texts. The datasets are split into training and validation sets. Each text sample is tokenized using a pre-trained GPT-2 tokenizer by truncating to a maximum length and padding to ensure uniform sequence lengths. LM input token embeddings are computed for every example, and normalized to facilitate robust clustering. For the Alpaca dataset, we calculate the token embeddings from the instruction part of the input prompts. A crucial part of ChameleonLLM is creating data loaders that reflect the clustered nature of the inputs. Using the precomputed token embeddings, we apply k‑means clustering with the number of clusters based on the desired batch size and overall dataset size. After clustering, indices are grouped so each mini-batch contains examples from a single cluster. This ensures that each batch is contextually coherent. We implement a custom data loader that leverages these clusters as batch samplers. The data loader returns batches fed into the model during training and evaluation. The evaluation is performed at the end of each epoch on the validation set, where batches are processed through the clustering pipeline, and the average loss is computed. This systematic evaluation ensures that our comparisons are fair and reflect real-world performance. All codes for ChameleonLLM are \href{https://anonymous.4open.science/r/ChamaleonLLM/}{open-sourced} to ensure the reproducibility of experimental results.

\begin{table}[htbp]
  \centering
  \caption{Comparison of Low-Rank Adaptation Regimes on WikiText-2 Dataset}
  \label{tab:gpt2_comparison_wikitext}
  \begin{tabular}{lccccc}
    \toprule
    Adaptation Regime & Parameters & Training Loss & Validation Loss & Val. Perplexity \\
    \midrule
    Unadapted GPT-2        & 124,439,808       & 13.8144 & 13.7876   & 972,500 \\
    Traditional LoRA        & 204,100  & 0.5504  & 0.5023   & 1.6525 \\
    \textbf{ChameleonLLM}       & 6,786,596 & \textbf{0.2359}  & \textbf{0.3753}    & \textbf{1.4554} \\
    \bottomrule
  \end{tabular}
\end{table}

\begin{table}[htbp]
  \centering
  \begin{threeparttable}
    \caption{Comparison of Low-Rank Adaptation Regimes on Alpaca Dataset}
    \label{tab:gpt2_comparison_alpaca}
    \begin{tabular}{lccccc}
      \toprule
      Adaptation Regime & Parameters & Training Loss & Validation Loss & Val. Perplexity \\
      \midrule
      Unadapted GPT-2         & 124,439,808 & 13.3881 & 13.3881 & 652,166 \\
      Traditional LoRA        & 204,100     & 0.1254  & 0.1527  & 1.1650 \\
      \textbf{ChameleonLLM\tnote{*}} & 6,786,596 & \textbf{0.0810} & \textbf{0.1082} & \textbf{1.1143} \\
      \bottomrule
    \end{tabular}
    \begin{tablenotes}
      \footnotesize
      \item[*] Token embeddings for adaptation are calculated from the instruction part of the prompt only.
    \end{tablenotes}
  \end{threeparttable}
\end{table}


\begin{itemize}
    \item \textbf{Traditional LoRA Fine-Tuning:} Here, all transformer layers and the LM head use the static LoRA adaptation. Standard cross-entropy loss (with appropriate masking for padding tokens) is computed, and the low-rank parameters are optimized using AdamW optimizer.
    \item \textbf{ChameleonLLM Fine-Tuning:} In this regime, while the transformer layers continue to use static LoRA modules, the LM head is adapted by a hyper-network LoRA module that generates low-rank update parameters based on the mean token embeddings of the batch. The loss is computed on the LM head outputs, and optimized similarly.
\end{itemize}

Experimental results demonstrate that ChameleonLLM significantly improves over the traditional LoRA approach, consistently achieving a lower average validation loss on both datasets, and suggest that dynamic adaptation based on batch statistics leads to better convergence and generalization. As shown in Table~\ref{tab:gpt2_comparison_wikitext}, ChameleonLLM achieves a validation loss reduction of approximately \textbf{25\%} compared to traditional LoRA, along with a perplexity improvement of roughly 12\%. Table~\ref{tab:gpt2_comparison_alpaca} further indicates that it reduces the validation loss by almost \textbf{30\%} on instruction fine-tuning. These improvements underscore the effectiveness of the batch-level context adaptation strategy and can largely be attributed to the exploitation of batch-level context. By averaging token embeddings over each batch, the hyper-network input becomes less sensitive to individual outliers, leading to robust low-rank parameter generation. Clustering groups together semantically or stylistically similar inputs allows the hyper-network to extract stronger, more relevant signals for adaptation. 

Although ChameleonLLM has more trainable parameters than LoRA, this increase is justified by many practical benefits, as it dynamically generates low-rank updates on-the-fly rather than training and storing separate adaptation matrices for each sample or task. Avoiding the storage of numerous pre-learned matrices significantly reduces memory and processing overheads during inference. The additional parameters enable context-aware, batch-level adaptation, leading to better performance. Despite it also introduces some extra computational steps during inference, the overhead is modest compared to the overall inference time. The trade-off between increased computation and improved performance is highly favorable. Our approach is not tied to pre-defined task categories and adapts to the nuances in the data at inference time, offering a versatile solution for open-domain tasks.
    
\section{Conclusion}

In this paper, we presented ChameleonLLM---a novel framework that enables inference-time adaptation of large language models through batch-aware clustering and dynamic low-rank parameter generation. By clustering similar inputs and using a hyper-network to generate low-rank updates based on the aggregated batch statistics, our method provides a flexible and efficient alternative to static fine-tuning approaches such as traditional LoRA or uniform mask blending. Our extensive experiments on WikiText-2 and Alpaca demonstrate that ChameleonLLM achieves lower validation loss and perplexity than baseline methods with dynamic batch-aware contextual adaptation. The proposed framework reduces the need for storing multiple expert masks and adapts to diverse input distributions in real-time, making it highly suitable for modern, dynamic inference environments.

\bibliography{iclr2025_conference}
\bibliographystyle{iclr2025_conference}


\end{document}


\section{Submission of conference papers to ICLR 2025}

ICLR requires electronic submissions, processed by
\url{https://openreview.net/}. See ICLR's website for more instructions.

If your paper is ultimately accepted, the statement {\tt
  {\textbackslash}iclrfinalcopy} should be inserted to adjust the
format to the camera ready requirements.

The format for the submissions is a variant of the NeurIPS format.
Please read carefully the instructions below, and follow them
faithfully.

\subsection{Style}

Papers to be submitted to ICLR 2025 must be prepared according to the
instructions presented here.

%% Please note that we have introduced automatic line number generation
%% into the style file for \LaTeXe. This is to help reviewers
%% refer to specific lines of the paper when they make their comments. Please do
%% NOT refer to these line numbers in your paper as they will be removed from the
%% style file for the final version of accepted papers.

Authors are required to use the ICLR \LaTeX{} style files obtainable at the
ICLR website. Please make sure you use the current files and
not previous versions. Tweaking the style files may be grounds for rejection.

\subsection{Retrieval of style files}

The style files for ICLR and other conference information are available online at:
\begin{center}
   \url{http://www.iclr.cc/}
\end{center}
The file \verb+iclr2025_conference.pdf+ contains these
instructions and illustrates the
various formatting requirements your ICLR paper must satisfy.
Submissions must be made using \LaTeX{} and the style files
\verb+iclr2025_conference.sty+ and \verb+iclr2025_conference.bst+ (to be used with \LaTeX{}2e). The file
\verb+iclr2025_conference.tex+ may be used as a ``shell'' for writing your paper. All you
have to do is replace the author, title, abstract, and text of the paper with
your own.

The formatting instructions contained in these style files are summarized in
sections \ref{gen_inst}, \ref{headings}, and \ref{others} below.

\section{General formatting instructions}
\label{gen_inst}

The text must be confined within a rectangle 5.5~inches (33~picas) wide and
9~inches (54~picas) long. The left margin is 1.5~inch (9~picas).
Use 10~point type with a vertical spacing of 11~points. Times New Roman is the
preferred typeface throughout. Paragraphs are separated by 1/2~line space,
with no indentation.

Paper title is 17~point, in small caps and left-aligned.
All pages should start at 1~inch (6~picas) from the top of the page.

Authors' names are
set in boldface, and each name is placed above its corresponding
address. The lead author's name is to be listed first, and
the co-authors' names are set to follow. Authors sharing the
same address can be on the same line.

Please pay special attention to the instructions in section \ref{others}
regarding figures, tables, acknowledgments, and references.


There will be a strict upper limit of 10 pages for the main text of the initial submission, with unlimited additional pages for citations. 

\section{Headings: first level}
\label{headings}

First level headings are in small caps,
flush left and in point size 12. One line space before the first level
heading and 1/2~line space after the first level heading.

\subsection{Headings: second level}

Second level headings are in small caps,
flush left and in point size 10. One line space before the second level
heading and 1/2~line space after the second level heading.

\subsubsection{Headings: third level}

Third level headings are in small caps,
flush left and in point size 10. One line space before the third level
heading and 1/2~line space after the third level heading.

\section{Citations, figures, tables, references}
\label{others}

These instructions apply to everyone, regardless of the formatter being used.

\subsection{Citations within the text}

Citations within the text should be based on the \texttt{natbib} package
and include the authors' last names and year (with the ``et~al.'' construct
for more than two authors). When the authors or the publication are
included in the sentence, the citation should not be in parenthesis using \verb|\citet{}| (as
in ``See \citet{Hinton06} for more information.''). Otherwise, the citation
should be in parenthesis using \verb|\citep{}| (as in ``Deep learning shows promise to make progress
towards AI~\citep{Bengio+chapter2007}.'').

The corresponding references are to be listed in alphabetical order of
authors, in the \textsc{References} section. As to the format of the
references themselves, any style is acceptable as long as it is used
consistently.

\subsection{Footnotes}

Indicate footnotes with a number\footnote{Sample of the first footnote} in the
text. Place the footnotes at the bottom of the page on which they appear.
Precede the footnote with a horizontal rule of 2~inches
(12~picas).\footnote{Sample of the second footnote}

\subsection{Figures}

All artwork must be neat, clean, and legible. Lines should be dark
enough for purposes of reproduction; art work should not be
hand-drawn. The figure number and caption always appear after the
figure. Place one line space before the figure caption, and one line
space after the figure. The figure caption is lower case (except for
first word and proper nouns); figures are numbered consecutively.

Make sure the figure caption does not get separated from the figure.
Leave sufficient space to avoid splitting the figure and figure caption.

You may use color figures.
However, it is best for the
figure captions and the paper body to make sense if the paper is printed
either in black/white or in color.
\begin{figure}[h]
\begin{center}
%\framebox[4.0in]{$\;$}
\fbox{\rule[-.5cm]{0cm}{4cm} \rule[-.5cm]{4cm}{0cm}}
\end{center}
\caption{Sample figure caption.}
\end{figure}

\subsection{Tables}

All tables must be centered, neat, clean and legible. Do not use hand-drawn
tables. The table number and title always appear before the table. See
Table~\ref{sample-table}.

Place one line space before the table title, one line space after the table
title, and one line space after the table. The table title must be lower case
(except for first word and proper nouns); tables are numbered consecutively.

\begin{table}[t]
\caption{Sample table title}
\label{sample-table}
\begin{center}
\begin{tabular}{ll}
\multicolumn{1}{c}{\bf PART}  &\multicolumn{1}{c}{\bf DESCRIPTION}
\\ \hline \\
Dendrite         &Input terminal \\
Axon             &Output terminal \\
Soma             &Cell body (contains cell nucleus) \\
\end{tabular}
\end{center}
\end{table}

\section{Default Notation}

In an attempt to encourage standardized notation, we have included the
notation file from the textbook, \textit{Deep Learning}
\cite{goodfellow2016deep} available at
\url{https://github.com/goodfeli/dlbook_notation/}.  Use of this style
is not required and can be disabled by commenting out
\texttt{math\_commands.tex}.


\centerline{\bf Numbers and Arrays}
\bgroup
\def\arraystretch{1.5}
\begin{tabular}{p{1in}p{3.25in}}
$\displaystyle a$ & A scalar (integer or real)\\
$\displaystyle \va$ & A vector\\
$\displaystyle \mA$ & A matrix\\
$\displaystyle \tA$ & A tensor\\
$\displaystyle \mI_n$ & Identity matrix with $n$ rows and $n$ columns\\
$\displaystyle \mI$ & Identity matrix with dimensionality implied by context\\
$\displaystyle \ve^{(i)}$ & Standard basis vector $[0,\dots,0,1,0,\dots,0]$ with a 1 at position $i$\\
$\displaystyle \text{diag}(\va)$ & A square, diagonal matrix with diagonal entries given by $\va$\\
$\displaystyle \ra$ & A scalar random variable\\
$\displaystyle \rva$ & A vector-valued random variable\\
$\displaystyle \rmA$ & A matrix-valued random variable\\
\end{tabular}
\egroup
\vspace{0.25cm}

\centerline{\bf Sets and Graphs}
\bgroup
\def\arraystretch{1.5}

\begin{tabular}{p{1.25in}p{3.25in}}
$\displaystyle \sA$ & A set\\
$\displaystyle \R$ & The set of real numbers \\
$\displaystyle \{0, 1\}$ & The set containing 0 and 1 \\
$\displaystyle \{0, 1, \dots, n \}$ & The set of all integers between $0$ and $n$\\
$\displaystyle [a, b]$ & The real interval including $a$ and $b$\\
$\displaystyle (a, b]$ & The real interval excluding $a$ but including $b$\\
$\displaystyle \sA \backslash \sB$ & Set subtraction, i.e., the set containing the elements of $\sA$ that are not in $\sB$\\
$\displaystyle \gG$ & A graph\\
$\displaystyle \parents_\gG(\ervx_i)$ & The parents of $\ervx_i$ in $\gG$
\end{tabular}
\vspace{0.25cm}


\centerline{\bf Indexing}
\bgroup
\def\arraystretch{1.5}

\begin{tabular}{p{1.25in}p{3.25in}}
$\displaystyle \eva_i$ & Element $i$ of vector $\va$, with indexing starting at 1 \\
$\displaystyle \eva_{-i}$ & All elements of vector $\va$ except for element $i$ \\
$\displaystyle \emA_{i,j}$ & Element $i, j$ of matrix $\mA$ \\
$\displaystyle \mA_{i, :}$ & Row $i$ of matrix $\mA$ \\
$\displaystyle \mA_{:, i}$ & Column $i$ of matrix $\mA$ \\
$\displaystyle \etA_{i, j, k}$ & Element $(i, j, k)$ of a 3-D tensor $\tA$\\
$\displaystyle \tA_{:, :, i}$ & 2-D slice of a 3-D tensor\\
$\displaystyle \erva_i$ & Element $i$ of the random vector $\rva$ \\
\end{tabular}
\egroup
\vspace{0.25cm}


\centerline{\bf Calculus}
\bgroup
\def\arraystretch{1.5}
\begin{tabular}{p{1.25in}p{3.25in}}
% NOTE: the [2ex] on the next line adds extra height to that row of the table.
% Without that command, the fraction on the first line is too tall and collides
% with the fraction on the second line.
$\displaystyle\frac{d y} {d x}$ & Derivative of $y$ with respect to $x$\\ [2ex]
$\displaystyle \frac{\partial y} {\partial x} $ & Partial derivative of $y$ with respect to $x$ \\
$\displaystyle \nabla_\vx y $ & Gradient of $y$ with respect to $\vx$ \\
$\displaystyle \nabla_\mX y $ & Matrix derivatives of $y$ with respect to $\mX$ \\
$\displaystyle \nabla_\tX y $ & Tensor containing derivatives of $y$ with respect to $\tX$ \\
$\displaystyle \frac{\partial f}{\partial \vx} $ & Jacobian matrix $\mJ \in \R^{m\times n}$ of $f: \R^n \rightarrow \R^m$\\
$\displaystyle \nabla_\vx^2 f(\vx)\text{ or }\mH( f)(\vx)$ & The Hessian matrix of $f$ at input point $\vx$\\
$\displaystyle \int f(\vx) d\vx $ & Definite integral over the entire domain of $\vx$ \\
$\displaystyle \int_\sS f(\vx) d\vx$ & Definite integral with respect to $\vx$ over the set $\sS$ \\
\end{tabular}
\egroup
\vspace{0.25cm}

\centerline{\bf Probability and Information Theory}
\bgroup
\def\arraystretch{1.5}
\begin{tabular}{p{1.25in}p{3.25in}}
$\displaystyle P(\ra)$ & A probability distribution over a discrete variable\\
$\displaystyle p(\ra)$ & A probability distribution over a continuous variable, or over
a variable whose type has not been specified\\
$\displaystyle \ra \sim P$ & Random variable $\ra$ has distribution $P$\\% so thing on left of \sim should always be a random variable, with name beginning with \r
$\displaystyle  \E_{\rx\sim P} [ f(x) ]\text{ or } \E f(x)$ & Expectation of $f(x)$ with respect to $P(\rx)$ \\
$\displaystyle \Var(f(x)) $ &  Variance of $f(x)$ under $P(\rx)$ \\
$\displaystyle \Cov(f(x),g(x)) $ & Covariance of $f(x)$ and $g(x)$ under $P(\rx)$\\
$\displaystyle H(\rx) $ & Shannon entropy of the random variable $\rx$\\
$\displaystyle \KL ( P \Vert Q ) $ & Kullback-Leibler divergence of P and Q \\
$\displaystyle \mathcal{N} ( \vx ; \vmu , \mSigma)$ & Gaussian distribution %
over $\vx$ with mean $\vmu$ and covariance $\mSigma$ \\
\end{tabular}
\egroup
\vspace{0.25cm}

\centerline{\bf Functions}
\bgroup
\def\arraystretch{1.5}
\begin{tabular}{p{1.25in}p{3.25in}}
$\displaystyle f: \sA \rightarrow \sB$ & The function $f$ with domain $\sA$ and range $\sB$\\
$\displaystyle f \circ g $ & Composition of the functions $f$ and $g$ \\
  $\displaystyle f(\vx ; \vtheta) $ & A function of $\vx$ parametrized by $\vtheta$.
  (Sometimes we write $f(\vx)$ and omit the argument $\vtheta$ to lighten notation) \\
$\displaystyle \log x$ & Natural logarithm of $x$ \\
$\displaystyle \sigma(x)$ & Logistic sigmoid, $\displaystyle \frac{1} {1 + \exp(-x)}$ \\
$\displaystyle \zeta(x)$ & Softplus, $\log(1 + \exp(x))$ \\
$\displaystyle || \vx ||_p $ & $\normlp$ norm of $\vx$ \\
$\displaystyle || \vx || $ & $\normltwo$ norm of $\vx$ \\
$\displaystyle x^+$ & Positive part of $x$, i.e., $\max(0,x)$\\
$\displaystyle \1_\mathrm{condition}$ & is 1 if the condition is true, 0 otherwise\\
\end{tabular}
\egroup
\vspace{0.25cm}



\section{Final instructions}
Do not change any aspects of the formatting parameters in the style files.
In particular, do not modify the width or length of the rectangle the text
should fit into, and do not change font sizes (except perhaps in the
\textsc{References} section; see below). Please note that pages should be
numbered.

\section{Preparing PostScript or PDF files}

Please prepare PostScript or PDF files with paper size ``US Letter'', and
not, for example, ``A4''. The -t
letter option on dvips will produce US Letter files.

Consider directly generating PDF files using \verb+pdflatex+
(especially if you are a MiKTeX user).
PDF figures must be substituted for EPS figures, however.

Otherwise, please generate your PostScript and PDF files with the following commands:
\begin{verbatim}
dvips mypaper.dvi -t letter -Ppdf -G0 -o mypaper.ps
ps2pdf mypaper.ps mypaper.pdf
\end{verbatim}

\subsection{Margins in LaTeX}

Most of the margin problems come from figures positioned by hand using
\verb+\special+ or other commands. We suggest using the command
\verb+\includegraphics+
from the graphicx package. Always specify the figure width as a multiple of
the line width as in the example below using .eps graphics
\begin{verbatim}
   \usepackage[dvips]{graphicx} ...
   \includegraphics[width=0.8\linewidth]{myfile.eps}
\end{verbatim}
or % Apr 2009 addition
\begin{verbatim}
   \usepackage[pdftex]{graphicx} ...
   \includegraphics[width=0.8\linewidth]{myfile.pdf}
\end{verbatim}
for .pdf graphics.
See section~4.4 in the graphics bundle documentation (\url{http://www.ctan.org/tex-archive/macros/latex/required/graphics/grfguide.ps})

A number of width problems arise when LaTeX cannot properly hyphenate a
line. Please give LaTeX hyphenation hints using the \verb+\-+ command.

\subsubsection*{Author Contributions}
If you'd like to, you may include  a section for author contributions as is done
in many journals. This is optional and at the discretion of the authors.

\subsubsection*{Acknowledgments}
Use unnumbered third level headings for the acknowledgments. All
acknowledgments, including those to funding agencies, go at the end of the paper.



\appendix
\section{Appendix}
You may include other additional sections here.


\end{document}
