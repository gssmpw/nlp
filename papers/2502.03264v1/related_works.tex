\section{Related works}
\label{related}

Due to page limit, we focus primarily on MTS foundation models trained from scratch, 
additional literature review and comparison can be found in Section \ref{sec:addRelatedWork}.  

\begin{table*}[]
\centering
\caption{Comparison between GTM and SOTA time series foundation models trained from scratch. The models are characterized by their approach to representation learning, ability to handle downstream tasks, and adaptability to multi-task scenarios.}
\label{table:related}
\scriptsize  % 调整字体大小 small/scriptsize/footnotesize
\setlength{\tabcolsep}{3pt} % 调整列间距
\renewcommand{\arraystretch}{0.8} % 调整行间距
\begin{tabular}{c|ccc|cccc|c}
\toprule
     &\multicolumn{3}{c|}{Time Series Features}  & \multicolumn{4}{c|}{Downstream Tasks}   & Adaptivity  \\ \midrule
     & Temporal Domain & Freq. Domain  & Time Gran.  & Forecasting  & Anomaly Detection & Imputation & CLF. & W/o inference adaption      \\ \midrule
     \begin{tabular}[c]{@{}c@{}}PatchTST, Lag-Llama, GPD\\ GPHT, TimesFM, MOIRAI,\\ UTSD, TTMs, TIME-MOE\end{tabular} & $\checkmark$ & $\times$ & $\times$ & $\checkmark$ & $\times$ & $\times$ & $\times$ & $\times$ \\ \midrule
TimeSiam, LPTM & $\checkmark$ & $\times$ & $\times$ & $\checkmark$ & $\times$ & $\times$ & $\checkmark$ & $\times$ \\ \midrule
TIMER, UP2ME & $\checkmark$ & $\times$ & $\times$ & $\checkmark$ & $\checkmark$ & $\checkmark$ & $\times$ & $\times$ \\ \midrule
UniTS & $\checkmark$ & $\times$ & $\times$ & $\checkmark$ & $\checkmark$ & $\checkmark$ & $\checkmark$ & $\times$ \\ \midrule
GTM(ours) & $\checkmark$ & $\checkmark$ & $\checkmark$ & $\checkmark$ & $\checkmark$ & $\checkmark$ & $\times$ & $\checkmark$
 \\ 
            \bottomrule
\end{tabular}
\vspace{-0.1in}
\end{table*}   

 

{\bf Early attempts.}  
TimesNet\cite{Wu23} achieves good performance across various generative downstream tasks.
The idea of adding a new dimension of multi-periodicity to 
temporal modeling is a novel approach, proving effective for enabling multi-task adaption.  
PatchTST\cite{Nie23} unlocks the potential of 
Transformer for MTS forecasting.   
Two pioneering components, 
i.e., Channel Independence and Patching, 
were introduced to Transformer, 
opening new possibilities for 
time series foundation models.    

{\bf MTS foundation model for forecasting.} 
This line of works focus only on the forecasting task, 
aiming to enable adaptivity to diverse data domains.  
Lag-Llama\cite{Rasul23} is one effort in this research line. Built on decoder-only architecture that incorporates lags as covariates and constructs features from timestamps, Lag-Llama has been shown to outperform previous deep learning approaches through fine-tuning on relatively small subsets of unseen datasets.  
GPHT\cite{Liuzd24} extends PatchTST by incorporating a hierarchical decoder-only backbone and employs an auto-regressive forecasting approach. One key advantage of GPHT is its ability to forecast across arbitrary horizon settings with a single model.  
TimesFM \cite{Das24} is based on stacked decoder-only 
transformer backbone with patching.  
With 200M parameters and pretraining on $O(100B)$ data points, 
it yields accurate zero-shot forecasts  
across different domains, 
forecasting horizons and temporal granularities.  
GPD\cite{Yang24} and UTSD \cite{Ma24} 
aim to address the across-domain issue of 
MTS forecasting. 
They utilize diffusion models to model 
the mixture distribution of the cross-domain data. 
MOIRAI \cite{Woo24} is built on a masked encoder-only Transformer backbone, but specially focus on 
tackling the cross-frequency learning challenge and 
accommodating an arbitrary number
of variates for MTS. 
The idea of flattening the MTS into a single sequence is novel, 
which enables it to learn multivariate interactions while considering exogenous covariates.   
TTMs \cite{Ekambaram24} reduces the computational cost of existing models while capturing cross-channel and exogenous correlations that are often missed by traditional approaches. 
%It uses finetuning to capture cross-channel and exogenous correlations. 
TIME-MOE\cite{Shi24} also reduces the 
computational cost by using a decoder-only forecasting model with a sparse mixture-of-experts (MOE) design. During training, it optimizes forecasting heads at multiple resolutions with varying prediction lengths, and dynamically schedules these heads for flexible forecasting during inference.  

{\bf Multi-task MTS foundation model.} 
This line of works aim to enable adaptivity to a wide 
range of down stream tasks.
UP2ME\cite{Zhang24} is built on a Transformer backbone and 
uses Masked AutoEncoder for pre-training.  
It introduces two instance generation techniques: variable window lengths and channel decoupling to remove cross-channel dependencies. During fine-tuning, it employs a Graph Transformer, freezing the backbone parameters while adding learnable Temporal-Channel (TC) layers. 
Timer\cite{Liu24} is built on a decode-only backbone and uses autoregressive approach with causal attention 
for generative pre-training. 
It defines a unified single-series sequence(S3) data format to curate 1 billion time points datasets for pre-training.  
Its pre-training approach fits well with forecasting and prediction-based anomaly detection tasks, but can’t provide sufficient context information in imputation task.  
TimeSiam \cite{Dong24} and LPTM\cite{Kamarthi23} are 
tailored to time series forecasting and classification tasks. 
TimeSiam uses the 
Siamese networks (Bromley et al., 1993) as its backbone and employs contrastive learning for pretraining. It aims to address the challenge that randomly masking time series or calculating series-wise similarity can distort or neglect the inherent temporal correlations that are critical in time series data. While contrastive learning enhances its performance in certain tasks, it results in limited adaptability for generative tasks.
LPTM\cite{Kamarthi23} aims to address the cross-domain challenge of extracting semantically meaningful tokenized inputs from heterogeneous time series across different domains. It combines a Transformer and GRU as its backbone and employs an adaptive segmentation method that automatically identifies the optimal segmentation strategy during pretraining. However, like TimeSiam, it has limited adaptability for generative tasks.   
UniTS \cite{gao24} is designed to handle both generative and classification tasks simultaneously. It uses task tokenization to integrate these tasks into a unified framework. The model employs a modified transformer block with two separate towers: one tailored for classification tasks and the other for generative tasks. This design enables effective transfer from a heterogeneous, multi-domain pretraining dataset to a variety of downstream datasets with varied task specifications and data domains.

 
{\bf Summary of difference.}   
Table\ref{table:related} summarizes the key differences between our work and the above models.
First, previous foundation models rely only on temporal information from discrete scalar values, while ours utilize both temporal and frequency domain information. 
Second, previous models require token, pre-training strategy or model level customization for down stream tasks, 
while ours does not due to new pre-training strategy design.   
Finally, our work introduces two architectural innovations: the Fourier Knowledge Attention mechanism, which learns time granularity-aware representations from both domains, and an autoregressive blank infilling pre-training framework, enabling a generative task-agnostic pre-training strategy. 





 
