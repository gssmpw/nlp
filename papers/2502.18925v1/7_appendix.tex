
\section{Metric}
\label{sec:metric}

\textbf{Mean Squared Error (MSE)} Mean Squared Error (MSE) is a common statistical metric used to assess the difference between predicted and actual values. The formula is:
\begin{equation}
    MSE = \frac{1}{n} \sum_{i=1}^{n} (y_i - \hat{y}_i)^2
\end{equation}
where $ n $ is the number of samples, $ y_i $ is the actual value, and $ \hat{y}_i $ is the predicted value.

\textbf{Relative L2 Error} Relative L2 error measures the relative difference between predicted and actual values, commonly used in time series prediction. The formula is:
\begin{equation}
    \text{Relative L2 Error} = \frac{\| Y_{\text{pred}} - Y_{\text{true}} \|_2}{\| Y_{\text{true}} \|_2}
\end{equation}
where $ Y_{\text{pred}} $ is the predicted value and $ Y_{\text{true}} $ is the actual value.

\textbf{Structural Similarity Index Measure (SSIM)} The Structural Similarity Index (SSIM) measures the similarity between two images in terms of luminance, contrast, and structure. The formula is:
\begin{equation}
    SSIM(x, y) = \frac{(2\mu_x \mu_y + C_1)(2\sigma_{xy} + C_2)}{(\mu_x^2 + \mu_y^2 + C_1)(\sigma_x^2 + \sigma_y^2 + C_2)}
\end{equation}
where $ \mu_x $ and $ \mu_y $ are the mean values, $ \sigma_x $ and $ \sigma_y $ are the standard deviations, $ \sigma_{xy} $ is the covariance.

\section{Related Work}
\label{sec:rw}

Our work lies at the intersection of three lines of inquiry: research on technologies supporting health services (Section \ref{sec:rw:tech-services}), mental health data collection and storage (Section \ref{sec:rw:data}), and value-based mental healthcare (Section \ref{sec:rw:vbc}).

\subsection{Designing Technologies for Health Services}
\label{sec:rw:tech-services}

In this work, we studied technologies that support value-based care and the delivery of \textit{health services}, which encompass the people, organizations, and technology involved in healthcare delivery \cite{issues_working_1994, sanford_schwartz_chapter_2017}.
These people and organizations include \textit{healthcare providers}, the clinicians or hospital systems that provide treatments or preventive care (the ``services''); as well as \textit{healthcare payers}, the government agencies or private health insurance companies that pay for health services.
We review specific technologies supporting mental health services in Section \ref{sec:rw:data}.
To design technologies for health services, we need to confront more than the hardware or software capabilities of a specific technology, or the effectiveness of interventions that use technologies to improve health outcomes.
We also need to confront sociotechnical factors that affect the implementation and effectiveness of these technologies in real-world care. 
Norman and Stappers categorize sociotechnical factors that affect technology implementation as political, economic, cultural, organizational, and structural \cite{norman_designx_2015}.
Blandford states that, for health services specifically, HCI scholars should \textit{``consider stages (of identifying technical possibilities or early adopters and planning for adoption and diffusion) that are rarely discussed in HCI, but that are necessary to deliver real impact from HCI innovations in healthcare''} \cite{blandford_hci_2019}.
Thus, we were motivated to improve the design of technologies supporting health services by understanding factors that affect their implementation and adoption in care.

Recently, HCI scholars have considered adopting ideas from health services research to improve both the design and effectiveness of health technologies.
Scholars have considered how HCI research can integrate aspects of \textit{implementation science} -- the health services field examining the real-world adoption of evidence-based interventions \cite{lyon_bridging_2023}. 
Interviews with HCI and implementation science researchers uncovered that HCI tends to de-prioritize factors that influence long-term adoption of technologies in their initial design, including the financial incentives that affect adoption, and an understanding of how technologies support providers after implementation \cite{dopp_aligning_2020}.
Moreover, HCI scholars have stated that if technologies are to impact real-world care, HCI researchers should focus on how technology is consumed in care, including developing an understanding of the technical and market incentives to use new tools \cite{colusso_translational_2019}.
Inspired by this work, we considered these aspects of adoption in the initial design of technologies that support value-based mental healthcare.
Specifically, we considered how technologies can support healthcare providers -- practicing clinicians -- including how these technologies can be integrated into clinicians' workflows to support care, and the financial incentives that influence HIT adoption as a part of value-based care.

\subsection{Health Information Technologies for Collecting and Storing Mental Health Data}
\label{sec:rw:data}

HCI, health informatics, and mental health researchers have collaborated to build health information technologies (HITs) for collecting and storing mental health data.
In this work, we focus on three categories of mental health data: clinical data, active data, and passive data.
\textit{Clinical data} can be retrieved from \textit{electronic health records} (EHRs), which record information collected during clinical visits including patient demographics, diagnoses, health and family history, treatments provided, and unstructured clinical notes \cite{birkhead_uses_2015}.
\rev{That said, to protect patient privacy, not all mental health data may be contained within the EHR, and exporting EHR data for VBC may require patient consent \cite{shenoy_safeguarding_2017, leventhal_designing_2015}.}
Clinical data can also be retrieved from \textit{administrative claims databases}, which log diagnostic, treatment, and medication information used to bill healthcare payers \cite{karve_prospective_2009, davis_can_2016}.
Clinics or hospitals may also collect measures of patient satisfaction to understand patients' perceptions of their care \cite{carr-hill_measurement_1992}.

\textit{Active data} require active patient or clinician engagement to be collected, and can be collected with technologies that support digital surveys (eg, smartphones, iPads, computers, \rev{patient portals}) and pen-and-paper questionnaires.
This data include validated self-reported \textit{measures of mental health symptoms}, which quantify symptom presence and/or severity for specific mental health disorders, such as the PHQ-9 for major depressive disorder \cite{kroenke_phq-9_2001}, or the GAD-7 for generalized anxiety disorder \cite{spitzer_brief_2006}.
Active data can also include clinician-rated scales, collected during clinical interviews \cite{andersen_brief_1986}.
Outside of symptoms, self-reported and clinician-rated measures can also quantify \textit{functioning}, as mental health symptoms can impair functioning including cognition, mobility, self-care, and sociality \cite{ustun_measuring_2010}. 
Self-reported measures can also quantify how well patients and their mental health clinicians collaborate towards shared goals, complete tasks, and bond, called \textit{working alliance} \cite{hatcher_development_2006}.
The discussed scales typically quantify persistent symptoms or functional impairment.
Researchers have used everyday devices, such as smartphones, to collect more in-the-moment symptoms via questionnaires called ecological momentary assessments (EMAs) \cite{wang_crosscheck_2016, hsieh_using_2008}.
EMAs can also collect \textit{engagement data}, measuring, for example, medication adherence, or participation in behavioral interventions, such as mindfulness exercises \cite{militello_digital_2022, klasnja_how_2011}.
Active data can be stored in clinical records, like an EHR, but significant investments have not been made to build structured EHR fields for storing active data \cite{pincus_quality_2016}.

In addition to active data, sensors embedded in devices (eg, smartphones, wearables) and online platforms have created opportunities to collect \textit{passive data} -- data collected with little-to-no effort -- on behavior and physiology \cite{nghiem_understanding_2023}.
Passive data can be used to estimate signals related to functioning, including social behaviors, mobility, and sleep \cite{mohr_personal_2017, saeb_relationship_2016, saeb_scalable_2017}, and more recently, researchers have investigated if passive data can measure engagement in therapeutic exercises \cite{evans_using_2024}.
Prior work has also studied whether passive data can estimate symptom severity \cite{adler_measuring_2024, das_swain_semantic_2022, meyerhoff_evaluation_2021, currey_digital_2022}.
The use of passive data in treatment is limited: \rev{while passive data can be collected within EHRs \cite{apple_healthcare_2024, metrohealth_track_2024, pennic_novant_2015}, established clinical guidelines for passive data use in care do not exist, and use is often limited to patients who are motivated to share passive data with their healthcare provider \cite{nghiem_understanding_2023}}.

It is challenging to identify what mental health data are most relevant to HITs in certain contexts, given their variety.
Li et al. proposed a 5-stage model to work through these challenges, specifically in the context of \textit{personal informatics systems}, where users collect data for self-reflection and gaining self-knowledge.
These five stages are preparation, collection, integration, reflection, and action \cite{li_stage-based_2010}.
In this work, we study how HITs can support mental health outcomes data as a part of value-based mental healthcare, inspired by three out of these five stages, specifically \textit{preparation}, understanding what data to collect; \textit{collection}, gathering data; and \textit{action}, how data is used.
We focus on these three stages because they capture existing challenges to design HITs that support VBC, which we review in Section \ref{sec:rw:vbc}.

\subsection{Value-based Mental Healthcare}
\label{sec:rw:vbc}
The World Economic Forum defines \textit{value-based care} (VBC) as a \textit{``patient-centric way to design and manage health systems''} and \textit{``align industry stakeholders around the shared objective of improving health outcomes delivered to patients at a given cost''} \cite{world_economic_forum_value_2017}.
VBC intends to change how healthcare is paid for, away from \textit{fee-for-service} payment models -- where payers reimburse providers for the number of services they provide -- towards paying for services if they deliver ``value'' to the healthcare system \cite{brown_key_2017}.
In practice, VBC is implemented by paying providers a set rate for managing patients' health, sharing savings if specific cost or utilization targets are met, and/or by offering financial incentives for payers and providers based upon \textit{quality measures}, which quantify the ``value'' of care \cite{world_economic_forum_moment_2023, health_care_payment_learning__action_network_alternative_2017}.
These changes shift some of the financial risk of healthcare from payers to providers.
In fee-for-service models, providers continue to be paid as they provide more services.
In VBC, providers may lose money if services cost more than set rates, specific cost/utilization targets are not met, or if care quality suffers \cite{novikov_historical_2018, health_care_payment_learning__action_network_alternative_2017}.

Standardized quality measures guide payers and providers to deliver services that improve health outcomes and reduce cost.
% Quality measures can be derived from administrative claims, EHRs, and patient self-report; are validated for their reliability and validity; importance for improving quality; feasibility to collect; and are certified by country-specific organizations like the NCQA in the United States, or the National Institute for Health and Care Excellence (NICE) in the UK \cite{center_for_medicare__medicaid_services_your_2021, national_institute_for_health_and_care_excellence_nice_2019}
The Donabedian model categorizes quality measures into three areas: (1) \textit{structure} -- the material, human, and organizational resources used in care (eg, the ratio of patients to providers); (2) \textit{process} -- the services provided in care (eg, the percentage of patients receiving immunizations); and (3) \textit{outcomes} -- measuring the effectiveness of care (eg, surgical mortality rates) \cite{donabedian_quality_1988, endeshaw_healthcare_2020,agency_for_healthcare_research_and_quality_types_2015}.
% Each category of measures has strengths and weaknesses.
While structure and process measures are more actionable -- hospital systems can hire more staff, or modify care practices -- their relationship to outcomes can be ambiguous \cite{quentin_measuring_2019}. 
In contrast, outcome measures most clearly represent the goals of care, but can be biased by factors outside of providers' direct control, including co-occurring health conditions that complicate treatment success \cite{lilienfeld_why_2013, quentin_measuring_2019}.
To reduce bias, statisticians apply a \textit{risk-adjustment} to outcome measures, using regression to model expected care outcomes observed in real-world data, based upon variables known to moderate treatment effects \cite{lane-fall_outcomes_2013}.
The quality of provided health services for a specific patient can then be determined based upon whether a patient's health outcomes exceed or underperform expectations.

Mental healthcare has faced specific challenges implementing VBC.
Some of these challenges can be attributed to ambiguity on how to design health information technologies (HITs) that store outcomes data tying provided services to value \cite{world_economic_forum_value_2017}.
\textit{Preparation challenges} revolve around identifying standardized outcome metrics to store in HITs.
Current quality monitoring programs incentivize using symptom scales as standardized care outcomes \cite{morden_health_2022}.
Patients often experience a unique constellation of symptoms that cut across multiple disorders (eg, major depressive disorder and generalized anxiety disorder) \cite{boschloo_network_2015, cramer_comorbidity_2010, barkham_routine_2023}, making it difficult to identify a limited set of symptom scales to track outcomes across patients.
Given these challenges, researchers have proposed using other data types as an alternative to symptom scales within VBC \cite{hobbs_knutson_driving_2021, oslin_provider_2019}. 
For example, scholars and healthcare providers have argued that functional and engagement outcomes may be a promising alternative to symptom scales. 
Engagement is the proximal outcome of many mental health treatments, improved functioning is often more important to patients than symptom reduction, and functional outcomes measure treatment progress across patients living with different mental health symptoms or disorders \cite{stewart_can_2017, tauscher_what_2021, pincus_quality_2016}.

In terms of \textit{data collection}, it is estimated that less than 20\% of mental health clinicians practice measurement-based care (MBC) -- the process of collecting, planning, and adjusting treatment based on outcomes data -- specifically symptom scales \cite{zimmerman_why_2008, fortney_tipping_2017}, despite evidence that MBC improves outcomes \cite{barkham_routine_2023}. 
MBC is usually implemented by having patients routinely self-report symptoms during clinical encounters using validated symptom scales, like the PHQ-9 for depression, or the GAD-7 for anxiety \cite{wray_enhancing_2018}.
Mental health clinicians choose to not practice MBC for many reasons. 
Electronic health records (EHRs) often do not have standardized fields to support symptom data collection, clinicians perceive that symptom scale administration disrupts the therapeutic relationship, and clinicians are often not paid to administer symptom scales \cite{lewis_implementing_2019, desimone_impact_2023, oslin_provider_2019}.
These barriers call for work centering mental health providers in designing HITs that effectively engage providers in outcomes data collection.

\textit{Action} challenges stem from both perceptions of how outcomes data could be used in care, and challenges towards attributing accountability for care.
For example, clinicians are often not trained to use outcomes data in care, and worry that they will be held accountable and penalized if outcomes data reveal that their patients are not improving \cite{lewis_implementing_2019, desimone_impact_2023}.
There are also concerns that outcomes data could be gamed: biased reporting that artificially inflates performance metrics \cite{kilbourne_measuring_2018}.
In addition, it is difficult in mental healthcare to attribute accountability to specific actors (eg, specific providers) in care systems.
Mental healthcare is often ``siloed'' from physical healthcare, though both physical and mental health outcomes are strongly intertwined (eg, individuals living with schizophrenia suffer from chronic physical health conditions) \cite{pincus_quality_2016}.
Thus, existing value-based mental healthcare programs may hold both physical and mental health clinicians \textit{jointly accountable} by sharing cost savings across different types of providers \cite{hobbs_knutson_driving_2021}.

Taken together, this prior work demonstrates challenges designing HITs that support value-based mental healthcare.
Integral to the design of these HITs are mental health clinicians, who are asked to participate in outcomes data collection, which clinicians have found challenging, and will be held financially accountable to the outcomes data HITs store.
Given these challenges, this work centers mental health clinicians' perspectives on how to design HITs that support value-based mental healthcare.
By centering clinicians' perspectives, we looked to gain a deeper understanding of their workflows and incentives to adopt HITs, and integrate this knowledge into the design and development of HITs supporting value-based care. 
The following section details the methodology used in this study.


\section{Detailed Mathematical Proof}
\label{sec:proof}
\textbf{Proof of Theorem 1}

Now we have N augmented data and we need to select the best from them. We consider both the quality and the diversity of these data and get the sampling strategy from an optimization problem.

We model the sampling strategy as a multinomial distribution supported on all the augmented data $S = \{\mathbf{X}_j\}_{j=1}^N$, which means that the sampling strategy $\pi=(\pi_1,...,\pi_N)^\top$ is the corresponding probabilities of selecting $\mathbf{X}_1,...,\mathbf{X}_N$, then we can model the expectation of the similarity as:
$$\begin{aligned}
 & \mathbb{E}_{Y_x,Y_{x^{\prime}}\in\mathcal{C}}\{g(x,x^{\prime})\mid S\} \\
 & =\quad\int g(\mathbf{x},\mathbf{x}^{\prime})\boldsymbol{\pi}(\mathbf{x})\mathrm{Pr}_{S}(Y_{x}\in\mathcal{C}\mid\boldsymbol{x}=\mathbf{x})\boldsymbol{\pi}(\mathbf{x}^{\prime})\mathrm{Pr}_{S}(Y_{x}\in\mathcal{C}\mid\boldsymbol{x}=\mathbf{x}^{\prime})d\mathbf{x}d\mathbf{x}^{\prime} \\
 & =\quad\sum_{i,j=1}^Ng(\mathbf{X}_i,\mathbf{X}_j)\pi_i\pi_j\mathrm{Pr}_{S}(Y_x\in\mathcal{C}\mid\boldsymbol{x}=\mathbf{X}_i)\mathrm{Pr}_{S}(Y_x\in\mathcal{C}\mid\boldsymbol{x}=\mathbf{X}_j),
\end{aligned}$$
where the set $\mathcal{C}$ denotes the criterion of selection we are using, the function $g$ can be chosen as any similarity metric function and $x$ means a random variable.

The core to solving the above optimization problem is to use predictive inference to approximate the conditional probability of $\{Y_x\in\mathcal{C}\}$ given $x = \mathbf{X}$
Let $\mu ( \mathbf{x} ) : = \mathbb{E} ( Y\mid \mathbf{X} = \mathbf{x} )$ be the oracle associated with $( \mathbf{X} , Y) .$ Denote $\theta_j=\mathbb{I}\{Y_j\in\mathcal{C}\}$. As the augmented data
$\mathbf{X}_1,...,\mathbf{X}_N$ are independently identically distributed, $\theta_1,...,\theta_N$ can be regarded as independent Bernoulli($q)$ variables with $q=\Pr(Y_j\in\mathcal{C}).$ The probability distribution of the predicted result $W_j$ for $j=1,...,N$ is
$$\Pr(W_j\mid\theta_j)=(1-\theta_j)f_0+\theta_jf_1,\quad$$
where $f_0$ and $f_1$ are the conditional distributions of $W_j$ on $Y_j \in \mathcal{C}$ or not.

Denote $T(w) = \frac{(1-q)f_0(W_j)}{f(W_j)}$, we can rewrite the expectation of the similarity as
$$\mathbb{E}_{Y_x,Y_{x^{\prime}}\in\mathcal{C}}\{g(x,x^{\prime})|S\}=\sum_{i,j=1}^Ng(\mathbf{X}_i,\mathbf{X}_j)\pi_i\pi_j(1-T_i)(1-T_j)=\boldsymbol{\pi}^\top A_\mathbb{T}\boldsymbol{\pi},$$

Next, we use the expectation to control the quality of the data.
$$\mathbb{E}\{\mathbb{I}(Y_x\not\in\mathcal{C})\mid S\}=\sum_{i=1}^N\Pr(Y_i\not\in\mathcal{C}\mid\mathbf{X}_i)\pi_i=\sum_{i=1}^N\pi_iT_i\leq\alpha,$$

In all, the optimization problem can be modeled as 
\begin{align}
    & \arg\min_{\boldsymbol{\pi}}\quad h(\boldsymbol{\pi},\mathbb{T}):=\boldsymbol{\pi}^\top A_\mathbb{T}\boldsymbol{\pi}, \\
    & \text{subject to} \quad
        \begin{cases}
            \sum_{i = 1}^N\pi_iT_i\leq\alpha, \\
            \sum_{i = 1}^N\pi_i = 1, \\
            0\leq\pi_i\leq m^{-1}, \quad 1\leq i\leq N.
        \end{cases}
\end{align}

where $m$ is used to control the maximum selection.

The best selection of K is determined by the strategy $\pi$ which serves as the solution to the above optimization problem.

\section{Additional Experiments}
\label{sec:more_experiments}
\subsection{Long-term forecasting experiment expansion}

In the long-term forecasting experiments, we compare the performance of different backbone models on the SWE benchmark, evaluating the relative L2 error for three variables (U, V, and H). Our setup inputs 5 frames and predicts 50 frames. For the SimVP-v2 model, using \method{} reduces the relative L2 error for SWE (u) from 0.0187 to 0.0154, SWE (v) from 0.0387 to 0.0342, and SWE (h) from 0.0443 to 0.0397. We visualize SWE (h) in 3D as shown in Figure~\ref{fig:case} [\textcolor{red}{I}]. For the ConvLSTM model, applying \method{} reduces the relative L2 error for SWE (u) from 0.0487 to 0.0321, SWE (v) from 0.0673 to 0.0351, and SWE (h) from 0.0762 to 0.0432. For the FNO model, using \method{} reduces the relative L2 error for SWE (u) from 0.0571 to 0.0502, SWE (v) from 0.0832 to 0.0653, and SWE (h) from 0.0981 to 0.0911. Overall, \method{} significantly improves the long-term forecasting accuracy of different backbone models.

\begin{figure*}[h]
    \centering
    \includegraphics[width=\textwidth]{image/casestudy.pdf}
    \caption{
    \textcolor{red}{I.} 3D visualization of the SWE(h), showing Ground-truth, SimVP-V2+BeamVQ predictions, and Error at T=1, 10, 20, 30, 40, 50. The first row shows Ground-truth, the second SimVP-V2+BeamVQ predictions, and the third Error. \textcolor{red}{II.} A case study. Building fire simulation with ventilation settings added to Wu's Prometheus~\cite{wu2024prometheus}. (a) Layout and HRR growth. (b) Comparison of physical metrics for different methods. (c) Ground-truth, ResNet+BeamVQ, and ResNet predictions.
    }
    \label{fig:case} 
\end{figure*}


\subsection{Experiment Statistical Significance}
\label{sec:significance}
To measure the statistical significance of our main experiment results, we choose three backbones to train on two datasets to run 5 times. 
Table~\ref{tab:significance} records the average and standard deviation of the test MSE loss.
The results prove that our method is statistically significant to outperform the baselines
because our confidence interval is always upper than the confidence interval of the baselines. 
Due to limited computation resources, we do not cover all ten backbones and five datasets, 
but we believe these results have shown that our method has consistent advantages.


\begin{table}[h]
\label{tab:significance}
\centering
\begin{scriptsize}
    \begin{sc}
    \caption{ The average and standard deviation of MSE in 5 runs}
    \label{tab:significance}
    \centering
        \renewcommand{\multirowsetup}{\centering}
        \setlength{\tabcolsep}{10pt}
        \begin{tabular}{l|cc|cc}
            \toprule
            
            \multirow{4}{*}{Model} & \multicolumn{4}{c}{Benchmarks}  \\
            \cmidrule(lr){2-5}
            & \multicolumn{2}{c}{NSE} &   \multicolumn{2}{c}{SEVIR}   \\
            \cmidrule(lr){2-5}
           & Ori & + BeamVQ & Ori & + BeamVQ  \\
            \midrule
            ConvLSTM &0.4092$\pm$0.0002 &\textbf{0.1277$\pm$0.0001}  & 0.1762 0.0007  & \textbf{0.1279$\pm$0.0009}  \\
            FNO &  0.2227$\pm$0.0003 &\textbf{0.1007 $\pm$0.0002}& 0.0787$\pm$0.0012 & \textbf{ 0.0437$\pm$0.0013} \\
            CNO & 0.2192 $\pm$0.0008 &\textbf{ 0.1492$\pm$0.0011}& 0.0057$\pm$0.0005 & \textbf{ 0.0053$\pm$0.0006} \\
            \bottomrule
        \end{tabular}
    \end{sc}

\end{scriptsize}
\end{table}
