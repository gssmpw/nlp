In practice,  physical spatiotemporal forecasting can suffer from data scarcity, because collecting large-scale data is non-trivial, especially for extreme events. 
Hence, we propose \method{}, a novel probabilistic framework to realize iterative self-training with new self-ensemble strategies, 
achieving better physical consistency and generalization on extreme events. 
Following any base forecasting model, 
we can encode its deterministic outputs into a latent space and retrieve multiple codebook entries to generate probabilistic outputs. 
Then \method{} extends the beam search from discrete spaces to the continuous state spaces in this field.
We can further employ domain-specific metrics (e.g., Critical Success Index for extreme events) to filter out the top-k candidates and develop the new self-ensemble strategy by combining the high-quality candidates. 
The self-ensemble can not only improve the inference quality and robustness but also iteratively augment the training datasets during continuous self-training. 
Consequently, \method{} realizes the exploration of rare but critical phenomena beyond the original dataset. 
Comprehensive experiments on different benchmarks and backbones show that \method{} consistently reduces forecasting MSE (up to 39\%), enhancing extreme events detection and proving its effectiveness in handling data scarcity. Our codes are available at~\url{https://github.com/easylearningscores/BeamVQ}.



% 在气象预报、流体模拟以及基于偏微分方程(PDE)的多物理系统模型中,数据稀缺下的时空预测仍然是一个关键挑战。本文提出了\method{},一个统一的框架,旨在同时解决标注数据有限以及在确保物理一致性的前提下捕捉极端事件的难题。首先,我们训练了一个确定性的基础模型,从小规模数据中学习主要动力学。随后,通过Top-K 向量量化变分自编码器(VQ-VAE)对基础模型的输出进行增强,该模块将确定性预测编码到潜在空间,并检索多个码本条目以生成多样化且物理上合理的重构结果。一个新颖的联合优化过程利用领域特定的指标(例如关键成功指数)引导基础模型向更准确且对极端事件敏感的预测方向优化。在推理阶段,我们采用束搜索策略,维持多个候选轨迹并通过指标感知评分进行迭代剪枝,从而在探索罕见但关键现象与利用最可能的系统轨迹之间实现平衡。在多个气象和流体流动基准数据集上的大量实验表明,\method{}显著提升了预测精度,增强了对极端状态的检测能力,并保持了物理合理性,证明了其在数据稀缺场景下进行时空预测的优越性。
