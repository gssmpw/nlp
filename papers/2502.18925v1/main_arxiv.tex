\documentclass{article}
\usepackage{microtype}
\usepackage{graphicx}
\usepackage{subfigure}
\usepackage{booktabs} 
\usepackage{multirow,bm}
\usepackage{tikz}
\usepackage{threeparttable}
\usepackage{tikz-cd,mathtools}
\usepackage{mathtools}
\usetikzlibrary{arrows, matrix} 
\usepackage{xcolor}
\definecolor{darkblue}{rgb}{0.0, 0.0, 0.55}
\definecolor{darkred}{rgb}{0.55, 0.0, 0.0}
\usepackage{hyperref}
\usepackage{comment}
\usepackage[accepted]{icml2024}  % 如果你是在icml2024模板下编译
\usepackage{amsmath}
\usepackage{amssymb}
\usepackage{mathtools}
\usepackage{amsthm}
\usepackage{multirow}
\usepackage[capitalize,noabbrev]{cleveref}
\usepackage{enumitem}
\usepackage{microtype}
\usepackage{graphicx}
\usepackage{subfigure}
\usepackage{booktabs} 
\usepackage{multirow,bm}
\usepackage{tikz}
\usepackage{threeparttable}
\usepackage{tikz-cd,mathtools}
\usepackage{mathtools}
\usetikzlibrary{arrows, matrix} 
\usepackage{xcolor}
\definecolor{darkblue}{rgb}{0.0, 0.0, 0.55}
\definecolor{darkred}{rgb}{0.55, 0.0, 0.0}
\usepackage{hyperref}
% 必要的宏包
\usepackage{wrapfig}
\usepackage[utf8]{inputenc} % allow utf-8 input
\usepackage[T1]{fontenc}    % use 8-bit T1 fonts
\usepackage{hyperref}       % hyperlinks
\usepackage{url}            % simple URL typesetting
\usepackage{booktabs}       % professional-quality tables
\usepackage{amsfonts}       % blackboard math symbols
\usepackage{nicefrac}       % compact symbols for 1/2, etc.
\usepackage{microtype}      % microtypography
\usepackage{xcolor}         % colors
\usepackage{tikz}
\newcommand\dashedrightarrow{\mathrel{\tikz[baseline=-.5ex]\draw[dashed,->](0,0)--(1em,0);}}
\usepackage{booktabs}
\usepackage{subfigure}
\usepackage{enumitem}
\usepackage{rotating}
\let\Bbbk\relax % to solve the incompatible between amssymb & acmart(containing newtxmath)
\usepackage[utf8]{inputenc}

\let\algorithm\undefined
\let\endalgorithm\undefined
\let\algorithmic\undefined
\let\endalgorithmic\undefined
\usepackage[ruled,vlined]{algorithm2e}

\usepackage{amsmath,amssymb,amsfonts}
\newtheorem{theorem}{Theorem}
\newtheorem{definition}{Definition}
\newtheorem{lemma}{Lemma}
%\newtheorem{proof}{Proof}[section]
\usepackage{multirow}
\usepackage{booktabs,bm}

\usepackage{colortbl}
\usepackage{multirow}
\usepackage{booktabs}
\usepackage{adjustbox}
% \definecolor{grey}{rgb}{0.89,0.71,0.57}
% \definecolor{pink}{rgb}{1,0.94,1}
% \definecolor{purple}{rgb}{0.84,0.78,1}
% \definecolor{white}{rgb}{1,1,1}
\usepackage{amsmath}
\usepackage{mathtools}
\usepackage[utf8]{inputenc}
\usepackage{hyperref}
\usepackage{comment}
%\usepackage{algorithm}
%\usepackage{algorithmicx}
%\usepackage{algpseudocode}
\usepackage{wrapfig}
\usepackage{amsmath}
\usepackage{graphicx}
\usepackage{subcaption}
\usepackage{multirow}
\usepackage{booktabs}
% \hypersetup{
%     colorlinks=true,
%     linkcolor=red,
%     citecolor=cyan,
%     filecolor=magenta,      
%     urlcolor=magenta,
%     }
%\usepackage{algcompatible}  % 兼容性更好的算法包
%\usepackage{algpseudocode}  % 现代伪代码样式
\newcommand{\showcomments}{yes}
\newcommand\hao[1]{
    \ifthenelse{\equal{\showcomments}{yes}}{{\color{red} Hao: #1}}{\ignorespaces}
}
\newcommand\xingjian[1]{
    \ifthenelse{\equal{\showcomments}{yes}}{{\color{blue} Xingjian: #1}}{\ignorespaces}
}
\newcommand\weiyan[1]{
    \ifthenelse{\equal{\showcomments}{yes}}{{\color{green} Weiyan: #1}}{\ignorespaces}
}

\usepackage{amsmath}         % 数学公式支持
\usepackage{enumitem}        % 列表定制
\setlist[itemize]{leftmargin=*, align=left} % 全局列表左对齐设置

\usepackage{enumitem}    % 用于自定义列表
\usepackage{amsmath}     % 增强数学排版
\usepackage{natbib}      % 处理引用
\usepackage{graphicx}    % 插入图片
\usepackage{comment}     % 使用 comment 环境
\usepackage{wrapfig}     % 环绕图像

%%%%%%%%%%%%%%%%%%%%%%%%%%%%%%%%
% THEOREMS
%%%%%%%%%%%%%%%%%%%%%%%%%%%%%%%%
% \theoremstyle{plain}
% \newtheorem{theorem}{Theorem}[section]
% \newtheorem{proposition}[theorem]{Proposition}
% \newtheorem{lemma}[theorem]{Lemma}
% \newtheorem{corollary}[theorem]{Corollary}
% \theoremstyle{definition}
% \newtheorem{definition}[theorem]{Definition}
% \newtheorem{assumption}[theorem]{Assumption}
% \theoremstyle{remark}
% \newtheorem{remark}[theorem]{Remark}

% Todonotes is useful during development; simply uncomment the next line
%    and comment out the line below the next line to turn off comments
%\usepackage[disable,textsize=tiny]{todonotes}
% \usepackage[textsize=tiny]{todonotes}

\def\method{BeamVQ}% The \icmltitle you define below is probably too long as a header.

% Therefore, a short form for the running title is supplied here:
\icmltitlerunning{\method: Beam Search with Vector Quantization to Mitigate Data Scarcity in Physical Spatiotemporal Forecasting}

\begin{document}

\twocolumn[
\icmltitle{\method: Beam Search with Vector Quantization to Mitigate Data Scarcity in Physical Spatiotemporal Forecasting}

% It is OKAY to include author information, even for blind
% submissions: the style file will automatically remove it for you
% unless you've provided the [accepted] option to the icml2024
% package.

% List of affiliations: The first argument should be a (short)
% identifier you will use later to specify author affiliations
% Academic affiliations should list Department, University, City, Region, Country
% Industry affiliations should list Company, City, Region, Country

% You can specify symbols, otherwise they are numbered in order.
% Ideally, you should not use this facility. Affiliations will be numbered
% in order of appearance and this is the preferred way.
% \icmlsetsymbol{equal}{*}


\begin{icmlauthorlist}
\icmlauthor{Weiyan Wang}{tx}
\icmlauthor{Xingjian Shi}{Boson}
\icmlauthor{Ruiqi Shu}{thu1}
\icmlauthor{Yuan Gao}{thu1}
\icmlauthor{Rui Ray Chen}{thu2}
\icmlauthor{Kun Wang}{ntu}
\icmlauthor{Fan Xu}{ustc}
\icmlauthor{Jinbao Xue}{tx}
\icmlauthor{Shuaipeng Li}{tx}
\icmlauthor{Yangyu Tao}{tx}
\icmlauthor{Di Wang}{tx}
\icmlauthor{Hao Wu}{tx,thu1,ustc}
\icmlauthor{Xiaomeng Huang}{thu1}
\end{icmlauthorlist}
\vskip -0.03in
\icmlaffiliation{tx}{TEG, Tencent}
\icmlaffiliation{Boson}{Boson AI}
\icmlaffiliation{thu1}{Department of Earth System Science, Ministry of Education Key Laboratory for Earth System Modeling, Institute for Global Change Studies, Tsinghua University}
\icmlaffiliation{ustc}{Department and Computer and Science, University of Science and Technology of China}
\icmlaffiliation{thu2}{Institute for Interdisciplinary Information Sciences, Tsinghua University}
\icmlaffiliation{ntu}{School of Computer Science and Engineering, Nanyang Technological University}

\icmlcorrespondingauthor{Xiaomeng Huang}{hxm@tsinghua.edu.cn}

\icmlkeywords{Machine Learning, ICML}

\vskip 0.3in
]



\printAffiliationsAndNotice{}  

\begin{abstract}
In practice,  physical spatiotemporal forecasting can suffer from data scarcity, because collecting large-scale data is non-trivial, especially for extreme events. 
Hence, we propose \method{}, a novel probabilistic framework to realize iterative self-training with new self-ensemble strategies, 
achieving better physical consistency and generalization on extreme events. 
Following any base forecasting model, 
we can encode its deterministic outputs into a latent space and retrieve multiple codebook entries to generate probabilistic outputs. 
Then \method{} extends the beam search from discrete spaces to the continuous state spaces in this field.
We can further employ domain-specific metrics (e.g., Critical Success Index for extreme events) to filter out the top-k candidates and develop the new self-ensemble strategy by combining the high-quality candidates. 
The self-ensemble can not only improve the inference quality and robustness but also iteratively augment the training datasets during continuous self-training. 
Consequently, \method{} realizes the exploration of rare but critical phenomena beyond the original dataset. 
Comprehensive experiments on different benchmarks and backbones show that \method{} consistently reduces forecasting MSE (up to 39\%), enhancing extreme events detection and proving its effectiveness in handling data scarcity. Our codes are available at~\url{https://github.com/easylearningscores/BeamVQ}.



% 在气象预报、流体模拟以及基于偏微分方程(PDE)的多物理系统模型中,数据稀缺下的时空预测仍然是一个关键挑战。本文提出了\method{},一个统一的框架,旨在同时解决标注数据有限以及在确保物理一致性的前提下捕捉极端事件的难题。首先,我们训练了一个确定性的基础模型,从小规模数据中学习主要动力学。随后,通过Top-K 向量量化变分自编码器(VQ-VAE)对基础模型的输出进行增强,该模块将确定性预测编码到潜在空间,并检索多个码本条目以生成多样化且物理上合理的重构结果。一个新颖的联合优化过程利用领域特定的指标(例如关键成功指数)引导基础模型向更准确且对极端事件敏感的预测方向优化。在推理阶段,我们采用束搜索策略,维持多个候选轨迹并通过指标感知评分进行迭代剪枝,从而在探索罕见但关键现象与利用最可能的系统轨迹之间实现平衡。在多个气象和流体流动基准数据集上的大量实验表明,\method{}显著提升了预测精度,增强了对极端状态的检测能力,并保持了物理合理性,证明了其在数据稀缺场景下进行时空预测的优越性。

\end{abstract}

The ubiquitous question "How did I do this before ChatGPT?" has become a cultural touch point, highlighting how Large Language Models (LLMs) have gradually permeated people's everyday lives. While initially introduced as general-purpose chatbots, LLMs have been adopted in unexpectedly diverse ways \cite{chkirbene2024applications}. These systems now play multiple roles in decision-making processes and tasks, ranging from information providers to triggers for human self-reflection \cite{kim2022bridging, kmmer2024effects}. This widespread integration has raised the question about how users develop dependencies on and relationships with these AI systems \cite{he2025conversational}.

Previous Human-Computer Interaction (HCI) research has extensively examined domain-specific LLM applications \cite{jin2024teach, liu2024selenite}. These studies have yielded insights into specialized use cases and led to targeted interaction design improvements. However, the broader impact of LLMs on everyday decision-making and tasks remains under-explored. As users increasingly integrate these tools into their daily routines, understanding the tangible impacts of habitual use becomes crucial \cite{kmmer2024effects}. 

Recent studies have attempted to measure the impact of LLM use through quantitative metrics such as task performance and decision accuracy \cite{kim2025fostering}. However, these measurement-based approaches cannot fully capture how people delegate everyday decisions to LLMs or the resulting meta-cognitive effects. Furthermore, everyday decisions encompass a broad spectrum of choices, from routine task management to social interaction planning \cite{eigner2024determinants, dhami2012cct}, making them challenging to examine through purely quantitative and task-specific approaches.

To address this gap, we conducted a qualitative study examining heavy LLM users who regularly rely on these systems for everyday decisions and tasks. Through interviews and analysis, we explored how these users integrate LLMs into their decision-making processes, what types of decisions they choose to delegate, and how this delegation affects their cognitive patterns and decision-making confidence. Our study addressed three research questions: 

\begin{itemize}
\item RQ1: How do heavy LLM users integrate LLMs into their everyday decision-making process?
\item RQ2: What underlying needs do heavy LLM users seek to fulfill through LLM assistance?
\item RQ3: How do heavy LLM users conceptualize and evaluate their relationship with LLMs?
\end{itemize}

Through these research questions, we examine three key aspects of heavy LLM use. RQ1 explores emergent use cases and notable patterns in how users incorporate LLMs into their decision-making processes. RQ2 investigates the fundamental motivations and needs that drive sustained LLM usage. RQ3 examines how users develop their mental models of LLMs and reflect on their extensive interaction with these systems.

% \section{Preliminaries}
% \textbf{Problem Definition.} We investigate spatiotemporal prediction tasks spanning meteorological forecasting~\cite{bi2023accurate}, computational fluid dynamics~\cite{wu2024prometheus}, and PDE-based systems~\cite{wu2024neural}. The observational data is structured as a 4D tensor $\mathbf{X} \in \mathbb{R}^{T \times C \times H \times W}$, where $T$ denotes temporal steps, $C$ represents physical variables (temperature, pressure, velocity fields), and $(H,W)$ specify spatial resolution. Our objective is to learn a parametric mapping $f_\Theta: \mathbf{X}_t \mapsto \hat{\mathbf{Y}}_{t+1}$ that predicts subsequent system states from historical sequences $\mathbf{X}_t = \{\mathbf{X}_1, ..., \mathbf{X}_t\}$. The parameters $\Theta$ are optimized through maximum likelihood estimation:
% \begin{equation}
% \Theta^* = \arg\max_{\Theta} \sum_{i=1}^T \log P(\mathbf{Y}_{t+1}^i | \mathbf{X}_t^i; \Theta)
% \end{equation}
% where $P(\mathbf{Y}_{t+1}^i | \mathbf{X}_t^i; \Theta)$ defines the predictive distribution. The optimized model enables multi-step forecasting via iterative rollout $\hat{\mathbf{Y}}_{t+k} = f_\Theta(\{\mathbf{X}_t, \hat{\mathbf{Y}}_{t+1}, ..., \hat{\mathbf{Y}}_{t+k-1}\})$, crucial for applications requiring temporal extrapolation in climate modeling~\cite{bi2023accurate}, combustion dynamics~\cite{anonymous2024openck}, and fluid simulations~\cite{wupure}.

% \begin{figure*}[t]
% \centering
% \includegraphics[width=1\textwidth]{image/ICML_beamvq_main.png}
% \caption{\textbf{Architecture Overview of~\method{}.}  
% (a) \textbf{Stage $1$: Base Model Training}: A deterministic predictor (FNO/ViT/ConvLSTM) learns single-step mappings $\mathbf{X}_t \xrightarrow{f_{\Theta_f}} \hat{\mathbf{Y}}_{t+1}$ via MSE minimization.  
% (b) \textbf{Stage $2$: Top-K VQ-VAE}: Latent code $\mathbf{z}$ from encoder $e_{\Phi_h}$ is quantized to $K$ nearest codebook vectors $\{\mathbf{q}^{(k)}\}$, decoded to diverse predictions $\{\tilde{\mathbf{Y}}_{t+1}^{(k)}\}$.  
% (c) \textbf{Joint Optimization}: The optimal reconstruction $\tilde{\mathbf{Y}}_{t+1}^*$ (selected by metric $M$) guides base model refinement, while top-$K'$ ensemble $\bar{\mathbf{Y}}_{t+1}$ enables self-training.} 
% \label{fig:Idea_main} 
% \end{figure*}
\section{Method}
\begin{figure*}
  \centering
  \includegraphics[width=1\linewidth]{figures/icml_main.png}
     \vspace{-16pt}
  \caption{Overview of Our~\method{}. \textbf{(a)} The overall architecture includes input variables, an encoder, a message passing module, a decoder, and visualization of forecast variables; \textbf{(b)} The global forecasts module uses rollout technology to generate future forecasts; \textbf{(c)} The neural nested grid method specializes in regional high-resolution weather forecasts tasks; and \textbf{(d)} The ensemble forecasting module generates long-term forecast results.}
  \label{fig:ICML_yuan}
\vspace{-10pt}
\end{figure*}

\textbf{Problem Definition}\label{sec:problem}
In this study, we model weather forecasting as an autoregressive problem~\cite{lam2023learning}. At each time step $t$, we use the meteorological state comprising surface variables $\mathbf{X}_t$ and pressure level variables $\mathbf{P}_t$ to forecast the state at the next time step. We concatenate the surface and pressure level variables along the channel dimension to form the combined input: $\mathbf{Z}_t = [\mathbf{X}_t, \mathbf{P}_t] \in \mathbb{R}^{N \times d}$, where $N = H \times W$ represents the number of grid locations (nodes), and $d = d_x + d_p$ is the total number of variables. Here, $d_x$ and $d_p$ are the numbers of surface level and pressure level variables, respectively. In our setup, the initial input contains 69 variables: 4 surface level variables and 65 pressure level variables. Our model aims to forecast the combined variables at the next time step $\hat{\mathbf{Z}}_{t+1}$ using the current input $\mathbf{Z}_t$, capturing the spatiotemporal evolution of the atmosphere: $\hat{\mathbf{Z}}_{t+1} = \text{Model}(\mathbf{Z}_t; \Theta),$ where $\Theta$ denotes the model parameters. The training objective is to minimize the relative mean squared error (MSE) between the forecasts and the true values across all time steps: $    \min_{\Theta} \frac{1}{T} \sum_{t=0}^{T-1} \frac{ \left\| \hat{\mathbf{Z}}_{t+1} - \mathbf{Z}_{t+1} \right\|_2^2}{\left\| \mathbf{Z}_{t+1} \right\|_2^2}$. During inference, we adopt a rollout strategy to forecast longer sequences. Starting from the initial state $\mathbf{Z}_0$, the model recursively uses its previous forecasts as the next input: $ \hat{\mathbf{Z}}_{t+1} = \text{Model}(\hat{\mathbf{Z}}_t; \Theta), \quad t = 0, 1, 2, \dots, T-1.$ This strategy allows the model to generate extended weather forecasts using its own forecasts.

\subsection{Earth-specific Region Refined Graph Encoder}
In the encoder of~\method{}, inspired by~\cite{fortunato2022multiscale} ~\cite{lam2023learning}, we introduce an Earth-specific Region Refined Multi-scale Graph to improve the interaction of node features in complex dynamical systems. Inspired by the idea of multigrid methods~\cite{he2019mgnet}, we construct a multi-level Graph Neural Network architecture that includes grids of multiple granularities. Each grid has the same number of nodes but different grid densities, thereby capturing spatial features at different scales. Specifically, we define the multi-scale graph structure as:
\begin{equation}
\mathcal{G}=\left(\mathcal{V}^G, \mathcal{V}, \mathcal{E}^{(1)}, \mathcal{E}^{(2)}, \ldots, \mathcal{E}^{(L)}, \mathcal{E}^{(R)}, \mathcal{E}^{\mathrm{G} 2 \mathrm{M}}, \mathcal{E}^{\mathrm{M} 2 \mathrm{G}}\right),
\end{equation}
where ${\mathcal{V}}^{G}$ represents the set of lat-lon grid nodes, with a total of $N = H \times W$ nodes; $\mathcal{V}$ represents the mesh nodes. $\mathcal{E}^{(l)}$ denotes the edge set at the $l$-th scale, corresponding to grids of different granularities, where $l = 1, 2, \dots, L$, and $\mathcal{E}^{(R)}$ represents regional refined edges. $\mathcal{E}^{\mathrm{G} 2 \mathrm{M}}$ and $\mathcal{E}^{\mathrm{M} 2 \mathrm{G}}$ are the unidirectional edges that connect lat-lon grid nodes and mesh nodes. All scales share the same set of nodes $\mathcal{V}$. More details can be found in Appendix \ref{Appendix:model_details}.

In the encoder, we first map the input meteorological state $\mathbf{Z}_t \in \mathbb{R}^{N \times d}$ to the initial node feature representation:
\begin{equation}
\mathbf{h}_i^{(0)} = \phi (\mathbf{Z}_{t, i}), \quad i = 1, 2, \ldots, N,
\end{equation}
where $\phi(\cdot)$ is the feature mapping function, and $\mathbf{Z}_{t,i}$ is the input feature at node $i$. Next, we iteratively update the node features on the multi-scale graph structure. At iteration $k$, the feature update formula for node $i$ is:
\begin{equation}
\mathbf{h}_i^{(k)} = \sigma \left( \sum_{l=1}^L \sum_{j \in \mathcal{N}_i^{(l)}} \mathbf{W}^{(l)} \mathbf{h}_j^{(k-1)} + \mathbf{b}^{(l)} \right),
\end{equation}
where $\mathcal{N}_i^{(l)}$ is the set of nodes adjacent to node $i$ at the $l$-th scale, $\mathbf{W}^{(l)}$ is the weight matrix at the $l$-th scale, $\mathbf{b}$ is the bias term, and $\sigma(\cdot)$ is the activation function. To enhance the forecasting accuracy in specific regions, we introduce a region-refined grid on the finest global grid. For nodes within the target region, we add denser edge connections to capture local high-frequency features. In this way, the update of node features not only considers global multi-scale information but also incorporates region-specific fine-grained information.

\subsection{Multi-stream Messaging}
To address the issue of information transmission between nodes in complex dynamic systems, we propose a module called \textit{Multi-stream Messaging} (MSM). This module consists of an adaptive messaging mechanism, including a dynamic multi-head gated edge update module and a multi-head node attention mechanism. And OneForecast includes 16 MSMs for messaging.

\textbf{Dynamic Multi-head Gated Edge Update Module.} Unlike traditional message passing methods based on MLPs, we introduce dynamic gating and multi-head mechanisms to control the information flow more precisely. For each edge, we concatenate its own features with those of the source node and the target node:
\begin{equation}
\mathbf{c}_i = \operatorname{Concat}\left( \mathbf{e}_i, \mathbf{h}_{s(i)}, \mathbf{h}_{d(i)} \right) \in \mathbb{R}^{D_e + 2 D_h},
\end{equation}
where $\mathbf{e}_i$ is the feature of edge $i$, $\mathbf{h}_{s(i)}$ and $\mathbf{h}_{d(i)}$ are the features of the source and target nodes of edge $i$, $D_e$ is the edge feature dimension, and $D_h$ is the node feature dimension. Next, we generate gating vectors through a two-layer MLP to regulate the information flow. Specifically, the gating vector is divided into three parts: edge feature update gate $g_e$, source node feature gate $g_s$, and destination node feature gate $g_d$. First, we perform the first layer linear transformation and activation:
\begin{equation}
\mathbf{z}_i = \operatorname{SiLU}\left( \mathbf{W}_1 \mathbf{c}_i + \mathbf{b}_1 \right),
\end{equation}
where $\mathbf{W}_1 \in \mathbb{R}^{h \times (D_e + 2 D_h)}$ is the weight matrix of the first layer, and $\mathbf{b}_1 \in \mathbb{R}^h$ is the bias term. Then, we perform the second layer linear transformation and Sigmoid activation:
\begin{equation}
\mathbf{g}_i = \sigma\left( \mathbf{W}_2 \mathbf{z}_i + \mathbf{b}_2 \right) \in \mathbb{R}^{3 H D},
\end{equation}
where, $\mathbf{W}_2 \in \mathbb{R}^{3 H D\times h}$ is the weight matrix of the linear transformation of the second layer, and $D$ is the feature dimension for each gate. For each head $h$, the gating values $\mathbf{g}_i^{(h,e)}$, $\mathbf{g}_i^{(h,s)}$, and $\mathbf{g}_i^{(h,d)}$ are vectors of dimension $D$, corresponding to the edge feature gate, source node feature gate, and destination node feature gate, respectively.

Subsequently, we use Edge Sum MLP (ESMLP) to perform linear transformation and nonlinear activation on the edge features to generate the updated edge features:
\begin{equation}
\mathbf{e}_i' = \operatorname{ESMLP}_e\left( \mathbf{e}_i, \mathbf{h}_{s(i)}, \mathbf{h}_{d(i)} \right) \in \mathbb{R}^{D_e'},
\end{equation}
where $D_e'$ is the dimension of the updated edge features. Finally, we combine the gating vectors and the updated edge features to generate the final updated edge features through weighted averaging and residual connections:
\begin{equation}
\begin{aligned}
\mathbf{e}_i^{\text{new}} = &\frac{1}{3} \sum_{h=1}^H \bigg(
\mathbf{g}_i^{(h,e)} \odot \mathbf{e}_i^{\prime}
+ \mathbf{g}_i^{(h,s)} \odot \mathbf{h}_{s(i)} \\
&\quad + \mathbf{g}_i^{(h,d)} \odot \mathbf{h}_{d(i)}
\bigg) + \mathbf{e}_i,
\end{aligned}
\end{equation}
where $\odot$ denotes element-wise multiplication.

\textbf{Multi-head Node Attention Mechanism.} Compared to traditional message passing mechanisms, multi-head attention mechanisms can more precisely capture complex dependencies between nodes and dynamically adjust the way information is aggregated through attention weights. For each edge $e_i = (j \rightarrow k)$, we use an MLP to calculate the attention score:
\begin{equation}
\mathbf{a}_i = \operatorname{MLP}_a\left( \mathbf{e}_i^{\text{new}} \right) \in \mathbb{R}^H,
\end{equation}
then, we normalize the attention scores:
\begin{equation}
\alpha_i^{(h)} = \frac{\exp\left( \mathbf{a}_i^{(h)} \right)}{ \sum_{e_j \in \mathcal{E}(k)} \exp\left( \mathbf{a}_j^{(h)} \right) }, \quad \forall h = 1, 2, \dots, H,
\end{equation}
where $\mathcal{E}(k)$ denotes the set of all incoming edges to node $k$, and $\alpha_i^{(h)}$ is the attention weight of the $h$-th attention head for edge $e_i$. Next, we perform weighted aggregation of the edge features. For each node $k$, based on the attention weights, we compute the weighted sum of the features of all edges incoming to node $k$, generating the aggregated feature for each head:
\begin{equation}
\mathbf{m}_k^{(h)} = \sum_{e_i \in \mathcal{E}(k)} \alpha_i^{(h)} \cdot \mathbf{e}_i^{\text{new}} \in \mathbb{R}^{D_e'},
\end{equation}
then, we flatten and concatenate the aggregated features from all heads:
\begin{equation}
\mathbf{M}_k = \operatorname{Flatten}\left[ \mathbf{m}_k^{(1)}, \mathbf{m}_k^{(2)}, \dots, \mathbf{m}_k^{(H)} \right] \in \mathbb{R}^{D_e' \cdot H}.
\end{equation}
Finally, we concatenate the aggregated edge features with the original node features and, through an MLP, perform a nonlinear transformation to generate the updated node features:
\begin{equation}
\mathbf{h}_k^{\text{new}} = \operatorname{MLP}_n\left( \operatorname{Concat}\left( \mathbf{M}_k, \mathbf{h}_k \right) \right) + \mathbf{h}_k.
\end{equation}
In summary, in each iteration, we use the multi-stream messaging module to update the node features. Specifically, the node feature update formula is:
\begin{equation}
\mathbf{h}_i^{(k)} = \sigma \left( \sum_{l=1}^{L} \operatorname{MSM}\left( \mathbf{h}_i^{(k-1)}, \mathcal{E}^{(l)} \right) + \mathbf{b} \right),
\end{equation}
where $\operatorname{MSM}$ represents the aforementioned multi-stream messaging operation, $\mathcal{E}^{(l)}$ is the set of edges at the $l$-th scale, and $\sigma(\cdot)$ is the activation function. In the region-refined graph structure, for nodes within the target region, we additionally consider the set of edges within the region $\mathcal{E}^{\text{region}}$ to capture finer local information.

\textbf{Theoretical Analysis.} From a theoretical perspective, we explain why our method helps capture high-frequency information. This enhances long-term prediction ability and improves the ability to detect extreme events.
\begin{theorem}\label{thm:bias}
\textbf{High-pass Filtering Property of Multi-stream Messaging.} Considering the improved multi-stream message passing mechanism, suppose the graph signal $\bm{f} \in \mathbb{R}^N$ has a spectrum $\hat{\bm{f}} = \bm{U}^\top \bm{f}$ under the graph Fourier basis $\bm{U} = [\bm{u}_1, ..., \bm{u}_N]$, where $\bm{L} = \bm{U} \bm{\Lambda} \bm{U}^\top$ is the normalized graph Laplacian matrix and $\bm{\Lambda} = \operatorname{diag}(\lambda_1, ..., \lambda_N)$ is its eigenvalue diagonal matrix ($0 \leq \lambda_1 \leq ... \leq \lambda_N \leq 2$). Define the frequency response function of the message passing operator as $\rho: \lambda \mapsto \mathbb{R}$. If the dynamic gating weights satisfy:

\begin{equation}
    g^{(h,e)}_i, g^{(h,s)}_i, g^{(h,d)}_i \propto |\lambda_i - 1| + \epsilon \quad (\epsilon > 0)
\end{equation}

then there exist constants $\alpha > 0$ and $\kappa > 0$ such that the frequency response of the operator satisfies:

\begin{equation}
    \rho(\lambda_i) \geq \alpha |\lambda_i - 1| \quad \text{and} \quad \rho(\lambda_i) \geq \kappa \lambda_i
\end{equation}

that is, the operator is a strictly high-pass filter.
\end{theorem}
The proof of Theorem~\ref{thm:bias} can be found in Appendix~\ref{appendix_theorems}.
\subsection{Decoding and Optimization}
The decoder's goal is to decode the latent information back to meteorological variables on the latitude-longitude grid. We obtain the updated feature representation $\mathbf{h}_i$ for each node. For each node $i$, the decoder applies the mapping function:
\begin{equation}
\hat{\mathbf{Z}}_{t+1, i} = \psi(\mathbf{h}_i),
\end{equation}
where $\psi(\cdot)$ is an MLP that converts the latent node features into the predicted variables $\hat{\mathbf{Z}}_{t+1, i}$ for the next time step.

We use relative $L_2$ loss function for model training. The loss function is defined as:
\begin{equation}
\mathcal{L} = \frac{1}{K H W} \sum_{k=1}^K  \sum_{i=1}^H \sum_{j=1}^W \frac{\left( \hat{x}_{i, j, k}^{t + l \delta t} - x_{i, j, k}^{t + l \delta t} \right)^2}{\left( x_{i, j, k}^{t + l \delta t} \right)^2},
\end{equation}
where $\hat{x}_{i, j, k}^{t + l \delta t}$ and $x_{i, j, k}^{t + l \delta t}$ are the predicted and true values for variable (channel) $k$ at spatial location $(i, j)$ and time $t + l \delta t$; $K$ is the number of variables (channels); $H$ and $W$ are the height and width of the spatial dimensions, respectively; $\delta t$ is the time interval of single-step prediction (we use $\delta t = 6$ hours).

\subsection{Downstream Tasks}

We consider three principal downstream tasks:

\textbf{Global Weather Forecasting.} As detailed in Section~\ref{sec:problem} and illustrated in Figure~\ref{fig:ICML_yuan}(b), we employ a rollout approach during inference, using the trained model for multi-step extrapolation. Specifically, starting from the initial state~$\mathbf{Z}_0$, the model recursively uses its previous predictions as inputs for subsequent time steps, generating a sequence of future global weather forecasts.

\textbf{Regional High-Resolution Forecasting.} To enhance the accuracy of high-resolution forecasts in specific regions, we propose a neural nested grid method, illustrated in Figure~\ref{fig:ICML_yuan}(c). This method combines global low-resolution future forecasts with regional high-resolution data to produce detailed forecasts for the target region. We first input the global low-resolution data at time~$t$ into the pre-trained global model to obtain the global forecasts $\hat{\mathbf{Z}}_{t+1}^{\text{global}}$ at time~$t+1$. We extract $\hat{\mathbf{Z}}_{t+1}^{\text{global1}}$ from $\hat{\mathbf{Z}}_{t+1}^{\text{global}}$, which shares the same spatial range as the region, and $\hat{\mathbf{Z}}_{t+1}^{\text{global2}}$, which includes the boundary of the region (the boundary are defined as two grid points around the region). Both $\hat{\mathbf{Z}}_{t+1}^{\text{global1}}$ and $\hat{\mathbf{Z}}_{t+1}^{\text{global2}}$ are then interpolated to match the resolution of the high-resolution regional data, which are concatenate as $\hat{\mathbf{Z}}_{t+1}^{\text{global}}$ to acted as global force. We then combine the regional high-resolution data at time~$t$ with the $\hat{\mathbf{Z}}_{t+1}^{\text{global}}$ to form the input to the regional model. The global forecasts provide the necessary boundary conditions for the regional forecasts. The regional model then produces the high-resolution forecasts for the regional state at time~$t+1$:
\begin{equation}\small
       \hat{\mathbf{Z}}_{t+1}^{\text{region}} = \text{Model}_{\text{region}}\left( \operatorname{Concat}\left( \hat{\mathbf{Z}}_{t+1}^{\text{global}},\, \mathbf{Z}_t^{\text{region}} \right);\, \Theta_{\text{region}} \right).
\end{equation}

\textbf{Long-Term and Esemble Weather Forecasting.} The initial condition of the atmospheric state is uncertain, so reasonable quantification of this uncertainty is conducive to improve to forecast performance. In this work, we set N=50. To account for the uncertainty in the atmospheric initial state for long-term ensemble forecasting, we generate~$N$ perturbed initial conditions~$\mathbf{Z}_0^{(n)}$ by adding Perlin noise~$\varepsilon^{(n)}$ to~$\mathbf{Z}_0$~\cite{chen2023fuxi}. Each perturbed initial condition is input into the model, and through recursive rollout over~$T$ time steps, we obtain individual forecasts~$\hat{\mathbf{Z}}_{t+1}^{(n)}$. Finally, at each time step~$t+1$, we compute the ensemble mean prediction~$ \hat{\mathbf{Z}}_{t+1}^{\text{ensemble}} = \frac{1}{N} \sum_{n=1}^N \hat{\mathbf{Z}}_{t+1}^{(n)}$ by averaging the forecasts from all~$N$ ensemble members. 

% \subsection{Theoretical Analysis}
% To illustrate the ability of graph neural networks with different grid resolutions to express data information, we denote the continuous form of the input data as \( f(x) \). For simplicity, assume the data is one-dimensional, with \( x \in [-\pi, \pi] \). For the input data, we use the trigonometric functions \( \{1, \cos t, \sin t, \cos 2t, \sin 2t, \cdots \} \) as an orthogonal basis, and expand the input data in this basis to obtain signals of different frequencies:
%  \[
%  x(t) = \frac{a_0}{2} + \sum_{n=1}^{\infty} \left( a_n \cos nt + b_n \sin nt \right)
%  \]

%  Assume the grid on the data is \( \{x_1, x_2, \cdots, x_N\} \in [-\pi, \pi] \). For any sampling point \( x_i \), we have:
%  \[
%  \cos(k_1 x_i) \approx \cos(k_2 x_i)
%  \]
%  where
%  \[
%  k_1 \approx k_2 - \frac{2\pi n}{x_i}, \quad n \in \mathbb{Z}
%  \]
%  This indicates that at the point \( x_i \), signals with frequencies \( k_1 \) and \( k_2 \) will experience aliasing.

%  Furthermore, for
%  \[
%  k_1' \approx k_2' - \frac{2\pi n}{\prod_{i=1}^{N} x_i}, \quad n \in \mathbb{Z}
%  \]
%  the signals with frequencies \( k_1' \) and \( k_2' \) will experience aliasing at all sampling points, meaning that the discrete data will cause the model to misinterpret the high-frequency signal of \( k_2' \) as the low-frequency signal of \( k_1' \).

%  To precisely describe this phenomenon, we introduce the famous \textbf{Nyquist-Shannon sampling theorem}:
%  \begin{theorem}
%  When the sampling rate of a continuous-time signal is less than twice the highest frequency component in the signal's Fourier series representation, aliasing will occur.
%  \end{theorem}
%  This theorem indicates that to capture higher-frequency signals through discrete sampling points, the sampling frequency must be increased. In the process of converting data into grids, this means that to allow a neural network to learn higher-frequency information, a finer grid is required. That is, graph neural networks learn high-frequency information more accurately on finer grids than on coarser grids, reducing distortion.
\section{Experiments}
\label{section5}

In this section, we conduct extensive experiments to show that \ourmethod~can significantly speed up the sampling of existing MR Diffusion. To rigorously validate the effectiveness of our method, we follow the settings and checkpoints from \cite{luo2024daclip} and only modify the sampling part. Our experiment is divided into three parts. Section \ref{mainresult} compares the sampling results for different NFE cases. Section \ref{effects} studies the effects of different parameter settings on our algorithm, including network parameterizations and solver types. In Section \ref{analysis}, we visualize the sampling trajectories to show the speedup achieved by \ourmethod~and analyze why noise prediction gets obviously worse when NFE is less than 20.


\subsection{Main results}\label{mainresult}

Following \cite{luo2024daclip}, we conduct experiments with ten different types of image degradation: blurry, hazy, JPEG-compression, low-light, noisy, raindrop, rainy, shadowed, snowy, and inpainting (see Appendix \ref{appd1} for details). We adopt LPIPS \citep{zhang2018lpips} and FID \citep{heusel2017fid} as main metrics for perceptual evaluation, and also report PSNR and SSIM \citep{wang2004ssim} for reference. We compare \ourmethod~with other sampling methods, including posterior sampling \citep{luo2024posterior} and Euler-Maruyama discretization \citep{kloeden1992sde}. We take two tasks as examples and the metrics are shown in Figure \ref{fig:main}. Unless explicitly mentioned, we always use \ourmethod~based on SDE solver, with data prediction and uniform $\lambda$. The complete experimental results can be found in Appendix \ref{appd3}. The results demonstrate that \ourmethod~converges in a few (5 or 10) steps and produces samples with stable quality. Our algorithm significantly reduces the time cost without compromising sampling performance, which is of great practical value for MR Diffusion.


\begin{figure}[!ht]
    \centering
    \begin{minipage}[b]{0.45\textwidth}
        \centering
        \includegraphics[width=1\textwidth, trim=0 20 0 0]{figs/main_result/7_lowlight_fid.pdf}
        \subcaption{FID on \textit{low-light} dataset}
        \label{fig:main(a)}
    \end{minipage}
    \begin{minipage}[b]{0.45\textwidth}
        \centering
        \includegraphics[width=1\textwidth, trim=0 20 0 0]{figs/main_result/7_lowlight_lpips.pdf}
        \subcaption{LPIPS on \textit{low-light} dataset}
        \label{fig:main(b)}
    \end{minipage}
    \begin{minipage}[b]{0.45\textwidth}
        \centering
        \includegraphics[width=1\textwidth, trim=0 20 0 0]{figs/main_result/10_motion_fid.pdf}
        \subcaption{FID on \textit{motion-blurry} dataset}
        \label{fig:main(c)}
    \end{minipage}
    \begin{minipage}[b]{0.45\textwidth}
        \centering
        \includegraphics[width=1\textwidth, trim=0 20 0 0]{figs/main_result/10_motion_lpips.pdf}
        \subcaption{LPIPS on \textit{motion-blurry} dataset}
        \label{fig:main(d)}
    \end{minipage}
    \caption{\textbf{Perceptual evaluations on \textit{low-light} and \textit{motion-blurry} datasets.}}
    \label{fig:main}
\end{figure}

\subsection{Effects of parameter choice}\label{effects}

In Table \ref{tab:ablat_param}, we compare the results of two network parameterizations. The data prediction shows stable performance across different NFEs. The noise prediction performs similarly to data prediction with large NFEs, but its performance deteriorates significantly with smaller NFEs. The detailed analysis can be found in Section \ref{section5.3}. In Table \ref{tab:ablat_solver}, we compare \ourmethod-ODE-d-2 and \ourmethod-SDE-d-2 on the \textit{inpainting} task, which are derived from PF-ODE and reverse-time SDE respectively. SDE-based solver works better with a large NFE, whereas ODE-based solver is more effective with a small NFE. In general, neither solver type is inherently better.


% In Table \ref{tab:hazy}, we study the impact of two step size schedules on the results. On the whole, uniform $\lambda$ performs slightly better than uniform $t$. Our algorithm follows the method of \cite{lu2022dpmsolverplus} to estimate the integral part of the solution, while the analytical part does not affect the error.  Consequently, our algorithm has the same global truncation error, that is $\mathcal{O}\left(h_{max}^{k}\right)$. Note that the initial and final values of $\lambda$ depend on noise schedule and are fixed. Therefore, uniform $\lambda$ scheduling leads to the smallest $h_{max}$ and works better.

\begin{table}[ht]
    \centering
    \begin{minipage}{0.5\textwidth}
    \small
    \renewcommand{\arraystretch}{1}
    \centering
    \caption{Ablation study of network parameterizations on the Rain100H dataset.}
    % \vspace{8pt}
    \resizebox{1\textwidth}{!}{
        \begin{tabular}{cccccc}
			\toprule[1.5pt]
            % \multicolumn{6}{c}{Rainy} \\
            % \cmidrule(lr){1-6}
             NFE & Parameterization      & LPIPS\textdownarrow & FID\textdownarrow &  PSNR\textuparrow & SSIM\textuparrow  \\
            \midrule[1pt]
            \multirow{2}{*}{50}
             & Noise Prediction & \textbf{0.0606}     & \textbf{27.28}   & \textbf{28.89}     & \textbf{0.8615}    \\
             & Data Prediction & 0.0620     & 27.65   & 28.85     & 0.8602    \\
            \cmidrule(lr){1-6}
            \multirow{2}{*}{20}
              & Noise Prediction & 0.1429     & 47.31   & 27.68     & 0.7954    \\
              & Data Prediction & \textbf{0.0635}     & \textbf{27.79}   & \textbf{28.60}     & \textbf{0.8559}    \\
            \cmidrule(lr){1-6}
            \multirow{2}{*}{10}
              & Noise Prediction & 1.376     & 402.3   & 6.623     & 0.0114    \\
              & Data Prediction & \textbf{0.0678}     & \textbf{29.54}   & \textbf{28.09}     & \textbf{0.8483}    \\
            \cmidrule(lr){1-6}
            \multirow{2}{*}{5}
              & Noise Prediction & 1.416     & 447.0   & 5.755     & 0.0051    \\
              & Data Prediction & \textbf{0.0637}     & \textbf{26.92}   & \textbf{28.82}     & \textbf{0.8685}    \\       
            \bottomrule[1.5pt]
        \end{tabular}}
        \label{tab:ablat_param}
    \end{minipage}
    \hspace{0.01\textwidth}
    \begin{minipage}{0.46\textwidth}
    \small
    \renewcommand{\arraystretch}{1}
    \centering
    \caption{Ablation study of solver types on the CelebA-HQ dataset.}
    % \vspace{8pt}
        \resizebox{1\textwidth}{!}{
        \begin{tabular}{cccccc}
			\toprule[1.5pt]
            % \multicolumn{6}{c}{Raindrop} \\     
            % \cmidrule(lr){1-6}
             NFE & Solver Type     & LPIPS\textdownarrow & FID\textdownarrow &  PSNR\textuparrow & SSIM\textuparrow  \\
            \midrule[1pt]
            \multirow{2}{*}{50}
             & ODE & 0.0499     & 22.91   & 28.49     & 0.8921    \\
             & SDE & \textbf{0.0402}     & \textbf{19.09}   & \textbf{29.15}     & \textbf{0.9046}    \\
            \cmidrule(lr){1-6}
            \multirow{2}{*}{20}
              & ODE & 0.0475    & 21.35   & 28.51     & 0.8940    \\
              & SDE & \textbf{0.0408}     & \textbf{19.13}   & \textbf{28.98}    & \textbf{0.9032}    \\
            \cmidrule(lr){1-6}
            \multirow{2}{*}{10}
              & ODE & \textbf{0.0417}    & 19.44   & \textbf{28.94}     & \textbf{0.9048}    \\
              & SDE & 0.0437     & \textbf{19.29}   & 28.48     & 0.8996    \\
            \cmidrule(lr){1-6}
            \multirow{2}{*}{5}
              & ODE & \textbf{0.0526}     & 27.44   & \textbf{31.02}     & \textbf{0.9335}    \\
              & SDE & 0.0529    & \textbf{24.02}   & 28.35     & 0.8930    \\
            \bottomrule[1.5pt]
        \end{tabular}}
        \label{tab:ablat_solver}
    \end{minipage}
\end{table}


% \renewcommand{\arraystretch}{1}
%     \centering
%     \caption{Ablation study of step size schedule on the RESIDE-6k dataset.}
%     % \vspace{8pt}
%         \resizebox{1\textwidth}{!}{
%         \begin{tabular}{cccccc}
% 			\toprule[1.5pt]
%             % \multicolumn{6}{c}{Raindrop} \\     
%             % \cmidrule(lr){1-6}
%              NFE & Schedule      & LPIPS\textdownarrow & FID\textdownarrow &  PSNR\textuparrow & SSIM\textuparrow  \\
%             \midrule[1pt]
%             \multirow{2}{*}{50}
%              & uniform $t$ & 0.0271     & 5.539   & 30.00     & 0.9351    \\
%              & uniform $\lambda$ & \textbf{0.0233}     & \textbf{4.993}   & \textbf{30.19}     & \textbf{0.9427}    \\
%             \cmidrule(lr){1-6}
%             \multirow{2}{*}{20}
%               & uniform $t$ & 0.0313     & 6.000   & 29.73     & 0.9270    \\
%               & uniform $\lambda$ & \textbf{0.0240}     & \textbf{5.077}   & \textbf{30.06}    & \textbf{0.9409}    \\
%             \cmidrule(lr){1-6}
%             \multirow{2}{*}{10}
%               & uniform $t$ & 0.0309     & 6.094   & 29.42     & 0.9274    \\
%               & uniform $\lambda$ & \textbf{0.0246}     & \textbf{5.228}   & \textbf{29.65}     & \textbf{0.9372}    \\
%             \cmidrule(lr){1-6}
%             \multirow{2}{*}{5}
%               & uniform $t$ & 0.0256     & 5.477   & \textbf{29.91}     & 0.9342    \\
%               & uniform $\lambda$ & \textbf{0.0228}     & \textbf{5.174}   & 29.65     & \textbf{0.9416}    \\
%             \bottomrule[1.5pt]
%         \end{tabular}}
%         \label{tab:ablat_schedule}



\subsection{Analysis}\label{analysis}
\label{section5.3}

\begin{figure}[ht!]
    \centering
    \begin{minipage}[t]{0.6\linewidth}
        \centering
        \includegraphics[width=\linewidth, trim=0 20 10 0]{figs/trajectory_a.pdf} %trim左下右上
        \subcaption{Sampling results.}
        \label{fig:traj(a)}
    \end{minipage}
    \begin{minipage}[t]{0.35\linewidth}
        \centering
        \includegraphics[width=\linewidth, trim=0 0 0 0]{figs/trajectory_b.pdf} %trim左下右上
        \subcaption{Trajectory.}
        \label{fig:traj(b)}
    \end{minipage}
    \caption{\textbf{Sampling trajectories.} In (a), we compare our method (with order 1 and order 2) and previous sampling methods (i.e., posterior sampling and Euler discretization) on a motion blurry image. The numbers in parentheses indicate the NFE. In (b), we illustrate trajectories of each sampling method. Previous methods need to take many unnecessary paths to converge. With few NFEs, they fail to reach the ground truth (i.e., the location of $\boldsymbol{x}_0$). Our methods follow a more direct trajectory.}
    \label{fig:traj}
\end{figure}

\textbf{Sampling trajectory.}~ Inspired by the design idea of NCSN \citep{song2019ncsn}, we provide a new perspective of diffusion sampling process. \cite{song2019ncsn} consider each data point (e.g., an image) as a point in high-dimensional space. During the diffusion process, noise is added to each point $\boldsymbol{x}_0$, causing it to spread throughout the space, while the score function (a neural network) \textit{remembers} the direction towards $\boldsymbol{x}_0$. In the sampling process, we start from a random point by sampling a Gaussian distribution and follow the guidance of the reverse-time SDE (or PF-ODE) and the score function to locate $\boldsymbol{x}_0$. By connecting each intermediate state $\boldsymbol{x}_t$, we obtain a sampling trajectory. However, this trajectory exists in a high-dimensional space, making it difficult to visualize. Therefore, we use Principal Component Analysis (PCA) to reduce $\boldsymbol{x}_t$ to two dimensions, obtaining the projection of the sampling trajectory in 2D space. As shown in Figure \ref{fig:traj}, we present an example. Previous sampling methods \citep{luo2024posterior} often require a long path to find $\boldsymbol{x}_0$, and reducing NFE can lead to cumulative errors, making it impossible to locate $\boldsymbol{x}_0$. In contrast, our algorithm produces more direct trajectories, allowing us to find $\boldsymbol{x}_0$ with fewer NFEs.

\begin{figure*}[ht]
    \centering
    \begin{minipage}[t]{0.45\linewidth}
        \centering
        \includegraphics[width=\linewidth, trim=0 0 0 0]{figs/convergence_a.pdf} %trim左下右上
        \subcaption{Sampling results.}
        \label{fig:convergence(a)}
    \end{minipage}
    \begin{minipage}[t]{0.43\linewidth}
        \centering
        \includegraphics[width=\linewidth, trim=0 20 0 0]{figs/convergence_b.pdf} %trim左下右上
        \subcaption{Ratio of convergence.}
        \label{fig:convergence(b)}
    \end{minipage}
    \caption{\textbf{Convergence of noise prediction and data prediction.} In (a), we choose a low-light image for example. The numbers in parentheses indicate the NFE. In (b), we illustrate the ratio of components of neural network output that satisfy the Taylor expansion convergence requirement.}
    \label{fig:converge}
\end{figure*}

\textbf{Numerical stability of parameterizations.}~ From Table 1, we observe poor sampling results for noise prediction in the case of few NFEs. The reason may be that the neural network parameterized by noise prediction is numerically unstable. Recall that we used Taylor expansion in Eq.(\ref{14}), and the condition for the equality to hold is $|\lambda-\lambda_s|<\boldsymbol{R}(s)$. And the radius of convergence $\boldsymbol{R}(t)$ can be calculated by
\begin{equation}
\frac{1}{\boldsymbol{R}(t)}=\lim_{n\rightarrow\infty}\left|\frac{\boldsymbol{c}_{n+1}(t)}{\boldsymbol{c}_n(t)}\right|,
\end{equation}
where $\boldsymbol{c}_n(t)$ is the coefficient of the $n$-th term in Taylor expansion. We are unable to compute this limit and can only compute the $n=0$ case as an approximation. The output of the neural network can be viewed as a vector, with each component corresponding to a radius of convergence. At each time step, we count the ratio of components that satisfy $\boldsymbol{R}_i(s)>|\lambda-\lambda_s|$ as a criterion for judging the convergence, where $i$ denotes the $i$-th component. As shown in Figure \ref{fig:converge}, the neural network parameterized by data prediction meets the convergence criteria at almost every step. However, the neural network parameterized by noise prediction always has components that cannot converge, which will lead to large errors and failed sampling. Therefore, data prediction has better numerical stability and is a more recommended choice.


\paragraph{Summary}
Our findings provide significant insights into the influence of correctness, explanations, and refinement on evaluation accuracy and user trust in AI-based planners. 
In particular, the findings are three-fold: 
(1) The \textbf{correctness} of the generated plans is the most significant factor that impacts the evaluation accuracy and user trust in the planners. As the PDDL solver is more capable of generating correct plans, it achieves the highest evaluation accuracy and trust. 
(2) The \textbf{explanation} component of the LLM planner improves evaluation accuracy, as LLM+Expl achieves higher accuracy than LLM alone. Despite this improvement, LLM+Expl minimally impacts user trust. However, alternative explanation methods may influence user trust differently from the manually generated explanations used in our approach.
% On the other hand, explanations may help refine the trust of the planner to a more appropriate level by indicating planner shortcomings.
(3) The \textbf{refinement} procedure in the LLM planner does not lead to a significant improvement in evaluation accuracy; however, it exhibits a positive influence on user trust that may indicate an overtrust in some situations.
% This finding is aligned with prior works showing that iterative refinements based on user feedback would increase user trust~\cite{kunkel2019let, sebo2019don}.
Finally, the propensity-to-trust analysis identifies correctness as the primary determinant of user trust, whereas explanations provided limited improvement in scenarios where the planner's accuracy is diminished.

% In conclusion, our results indicate that the planner's correctness is the dominant factor for both evaluation accuracy and user trust. Therefore, selecting high-quality training data and optimizing the training procedure of AI-based planners to improve planning correctness is the top priority. Once the AI planner achieves a similar correctness level to traditional graph-search planners, strengthening its capability to explain and refine plans will further improve user trust compared to traditional planners.

\paragraph{Future Research} Future steps in this research include expanding user studies with larger sample sizes to improve generalizability and including additional planning problems per session for a more comprehensive evaluation. Next, we will explore alternative methods for generating plan explanations beyond manual creation to identify approaches that more effectively enhance user trust. 
Additionally, we will examine user trust by employing multiple LLM-based planners with varying levels of planning accuracy to better understand the interplay between planning correctness and user trust. 
Furthermore, we aim to enable real-time user-planner interaction, allowing users to provide feedback and refine plans collaboratively, thereby fostering a more dynamic and user-centric planning process.



% \clearpage
\bibliography{main_arxiv}
\bibliographystyle{icml2024}

\newpage
\appendix
\onecolumn

\clearpage
\section{Secure Token Pruning Protocols}
\label{app:a}
We detail the encrypted token pruning protocols $\Pi_{prune}$ in Figure \ref{fig:protocol-prune} and $\Pi_{mask}$ in Figure \ref{fig:protocol-mask} in this section.

%Optionally include supplemental material (complete proofs, additional experiments and plots) in appendix.
%All such materials \textbf{SHOULD be included in the main submission.}
\begin{figure}[h]
%vspace{-0.2in}
\begin{protocolbox}
\noindent
\textbf{Parties:} Server $P_0$, Client $P_1$.

\textbf{Input:} $P_0$ and $P_1$ holds $\{ \left \langle Att \right \rangle_{0}^{h}, \left \langle Att \right \rangle_{1}^{h}\}_{h=0}^{H-1} \in \mathbb{Z}_{2^{\ell}}^{n\times n}$ and $\left \langle x \right \rangle_{0}, \left \langle x \right \rangle_{1} \in \mathbb{Z}_{2^{\ell}}^{n\times D}$ respectively, where H is the number of heads, n is the number of input tokens and D is the embedding dimension of tokens. Additionally, $P_1$ holds a threshold $\theta \in \mathbb{Z}_{2^{\ell}}$.

\textbf{Output:} $P_0$ and $P_1$ get $\left \langle y \right \rangle_{0}, \left \langle y \right \rangle_{1} \in \mathbb{Z}_{2^{\ell}}^{n'\times D}$, respectively, where $y=\mathsf{Prune}(x)$ and $n'$ is the number of remaining tokens.

\noindent\rule{13.2cm}{0.1pt} % This creates the horizontal line
\textbf{Protocol:}
\begin{enumerate}[label=\arabic*:, leftmargin=*]
    \item For $h \in [H]$, $P_0$ and $P_1$ compute locally with input $\left \langle Att \right \rangle^{h}$, and learn the importance score in each head $\left \langle s \right \rangle^{h} \in \mathbb{Z}_{2^{\ell}}^{n} $, where $\left \langle s \right \rangle^{h}[j] = \frac{1}{n} \sum_{i=0}^{n-1} \left \langle Att \right \rangle^{h}[i,j]$.
    \item $P_0$ and $P_1$ compute locally with input $\{ \left \langle s \right \rangle^{i} \in \mathbb{Z}_{2^{\ell}}^{n}  \}_{i=0}^{H-1}$, and learn the final importance score $\left \langle S \right \rangle \in \mathbb{Z}_{2^{\ell}}^{n}$ for each token, where  $\left \langle S \right \rangle[i] = \frac{1}{H} \sum_{h=0}^{H-1} \left \langle s \right \rangle^{h}[i]$.
    \item  For $i \in [n]$, $P_0$ and $P_1$ invoke $\Pi_{CMP}$ with inputs  $\left \langle S \right \rangle$ and $ \theta $, and learn  $\left \langle M \right \rangle \in \mathbb{Z}_{2^{\ell}}^{n}$, such that$\left \langle M \right \rangle[i] = \Pi_{CMP}(\left \langle S \right \rangle[i] - \theta) $, where: \\
    $M[i] = \begin{cases}
        1  &\text{if}\ S[i] > \theta, \\
        0  &\text{otherwise}.
            \end{cases} $
    % \item If the pruning location is insensitive, $P_0$ and $P_1$ learn real mask $M$ instead of shares $\left \langle M \right \rangle$. $P_0$ and $P_1$ compute $\left \langle y \right \rangle$ with input $\left \langle x \right \rangle$ and $M$, where  $\left \langle x \right \rangle[i]$ is pruned if $M[i]$ is $0$.
    \item $P_0$ and $P_1$ invoke $\Pi_{mask}$ with inputs  $\left \langle x \right \rangle$ and pruning mask $\left \langle M \right \rangle$, and set outputs as $\left \langle y \right \rangle$.
\end{enumerate}
\end{protocolbox}
\setlength{\abovecaptionskip}{-1pt} % Reduces space above the caption
\caption{Secure Token Pruning Protocol $\Pi_{prune}$.}
\label{fig:protocol-prune}
\end{figure}




\begin{figure}[h]
\begin{protocolbox}
\noindent
\textbf{Parties:} Server $P_0$, Client $P_1$.

\textbf{Input:} $P_0$ and $P_1$ hold $\left \langle x \right \rangle_{0}, \left \langle x \right \rangle_{1} \in \mathbb{Z}_{2^{\ell}}^{n\times D}$ and  $\left \langle M \right \rangle_{0}, \left \langle M \right \rangle_{1} \in \mathbb{Z}_{2^{\ell}}^{n}$, respectively, where n is the number of input tokens and D is the embedding dimension of tokens.

\textbf{Output:} $P_0$ and $P_1$ get $\left \langle y \right \rangle_{0}, \left \langle y \right \rangle_{1} \in \mathbb{Z}_{2^{\ell}}^{n'\times D}$, respectively, where $y=\mathsf{Prune}(x)$ and $n'$ is the number of remaining tokens.

\noindent\rule{13.2cm}{0.1pt} % This creates the horizontal line
\textbf{Protocol:}
\begin{enumerate}[label=\arabic*:, leftmargin=*]
    \item For $i \in [n]$, $P_0$ and $P_1$ set $\left \langle M \right \rangle$ to the MSB of $\left \langle x \right \rangle$ and learn the masked tokens $\left \langle \Bar{x} \right \rangle \in Z_{2^{\ell}}^{n\times D}$, where
    $\left \langle \Bar{x}[i] \right \rangle = \left \langle x[i] \right \rangle + (\left \langle M[i] \right \rangle << f)$ and $f$ is the fixed-point precision.
    \item $P_0$ and $P_1$ compute the sum of $\{\Pi_{B2A}(\left \langle M \right \rangle[i]) \}_{i=0}^{n-1}$, and learn the number of remaining tokens $n'$ and the number of tokens to be pruned $m$, where $m = n-n'$.
    \item For $k\in[m]$, for $i\in[n-k-1]$, $P_0$ and $P_1$ invoke $\Pi_{msb}$ to learn the highest bit of $\left \langle \Bar{x}[i] \right \rangle$, where $b=\mathsf{MSB}(\Bar{x}[i])$. With the highest bit of $\Bar{x}[i]$, $P_0$ and $P_1$ perform a oblivious swap between $\Bar{x}[i]$ and $\Bar{x}[i+1]$:
    $\begin{cases}
        \Tilde{x}[i] = b\cdot \Bar{x}[i] + (1-b)\cdot \Bar{x}[i+1] \\
        \Tilde{x}[i+1] = b\cdot \Bar{x}[i+1] + (1-b)\cdot \Bar{x}[i]
    \end{cases} $ \\
    $P_0$ and $P_1$ learn the swapped token sequence $\left \langle \Tilde{x} \right \rangle$.
    \item $P_0$ and $P_1$ truncate $\left \langle \Tilde{x} \right \rangle$ locally by keeping the first $n'$ tokens, clear current MSB (all remaining token has $1$ on the MSB), and set outputs as $\left \langle y \right \rangle$.
\end{enumerate}
\end{protocolbox}
\setlength{\abovecaptionskip}{-1pt} % Reduces space above the caption
\caption{Secure Mask Protocol $\Pi_{mask}$.}
\label{fig:protocol-mask}
%\vspace{-0.2in}
\end{figure}

% \begin{wrapfigure}{r}{0.35\textwidth}  % 'r' for right, and the width of the figure area
%   \centering
%   \includegraphics[width=0.35\textwidth]{figures/msb.pdf}
%   \caption{Runtime of $\Pi_{prune}$ and $\Pi_{mask}$ in different layers. We compare different secure pruning strategies based on the BERT Base model.}
%   \label{fig:msb}
%   \vspace{-0.1in}
% \end{wrapfigure}

% \begin{figure}[h]  % 'r' for right, and the width of the figure area
%   \centering
%   \includegraphics[width=0.4\textwidth]{figures/msb.pdf}
%   \caption{Runtime of $\Pi_{prune}$ and $\Pi_{mask}$ in different layers. We compare different secure pruning strategies based on the BERT Base model.}
%   \label{fig:msb}
%   % \vspace{-0.1in}
% \end{figure}

\textbf{Complexity of $\Pi_{mask}$.} The complexity of the proposed $\Pi_{mask}$ mainly depends on the number of oblivious swaps. To prune $m$ tokens out of $n$ input tokens, $O(mn)$ swaps are needed. Since token pruning is performed progressively, only a small number of tokens are pruned at each layer, which makes $\Pi_{mask}$ efficient during runtime. Specifically, for a BERT base model with 128 input tokens, the pruning protocol only takes $\sim0.9$s on average in each layer. An alternative approach is to invoke an oblivious sort algorithm~\citep{bogdanov2014swap2,pang2023bolt} on $\left \langle \Bar{x} \right \rangle$. However, this approach is less efficient because it blindly sort the whole token sequence without considering $m$. That is, even if only $1$ token needs to be pruned, $O(nlog^{2}n)\sim O(n^2)$ oblivious swaps are needed, where as the proposed $\Pi_{mask}$ only need $O(n)$ swaps. More generally, for an $\ell$-layer Transformer with a total of $m$ tokens pruned, the overall time complexity using the sort strategy would be $O(\ell n^2)$ while using the swap strategy remains an overall complexity of $O(mn).$ Specifically, using the sort strategy to prune tokens in one BERT Base model layer can take up to $3.8\sim4.5$ s depending on the sorting algorithm used. In contrast, using the swap strategy only needs $0.5$ s. Moreover, alternative to our MSB strategy, one can also swap the encrypted mask along with the encrypted token sequence. However, we find that this doubles the number of swaps needed, and thus is less efficient the our MSB strategy, as is shown in Figure \ref{fig:msb}.

\section{Existing Protocols}
\label{app:protocol}
\noindent\textbf{Existing Protocols Used in Our Private Inference.}  In our private inference framework, we reuse several existing cryptographic protocols for basic computations. $\Pi_{MatMul}$ \citep{pang2023bolt} processes two ASS matrices and outputs their product in SS form. For non-linear computations, protocols $\Pi_{SoftMax}, \Pi_{GELU}$, and $\Pi_{LayerNorm}$\citep{lu2023bumblebee, pang2023bolt} take a secret shared tensor and return the result of non-linear functions in ASS. Basic protocols from~\citep{rathee2020cryptflow2, rathee2021sirnn} are also utilized. $\Pi_{CMP}$\citep{EzPC}, for example, inputs ASS values and outputs a secret shared comparison result, while $\Pi_{B2A}$\citep{EzPC} converts secret shared Boolean values into their corresponding arithmetic values.

\section{Polynomial Reduction for Non-linear Functions}
\label{app:b}
The $\mathsf{SoftMax}$ and $\mathsf{GELU}$ functions can be approximated with polynomials. High-degree polynomials~\citep{lu2023bumblebee, pang2023bolt} can achieve the same accuracy as the LUT-based methods~\cite{hao2022iron-iron}. While these polynomial approximations are more efficient than look-up tables, they can still incur considerable overheads. Reducing the high-degree polynomials to the low-degree ones for the less important tokens can imporve efficiency without compromising accuracy. The $\mathsf{SoftMax}$ function is applied to each row of an attention map. If a token is to be reduced, the corresponding row will be computed using the low-degree polynomial approximations. Otherwise, the corresponding row will be computed accurately via a high-degree one. That is if $M_{\beta}'[i] = 1$, $P_0$ and $P_1$ uses high-degree polynomials to compute the $\mathsf{SoftMax}$ function on token $x[i]$:
\begin{equation}
\mathsf{SoftMax}_{i}(x) = \frac{e^{x_i}}{\sum_{j\in [d]}e^{x_j}}
\end{equation}
where $x$ is a input vector of length $d$ and the exponential function is computed via a polynomial approximation. For the $\mathsf{SoftMax}$ protocol, we adopt a similar strategy as~\citep{kim2021ibert, hao2022iron-iron}, where we evaluate on the normalized inputs $\mathsf{SoftMax}(x-max_{i\in [d]}x_i)$. Different from~\citep{hao2022iron-iron}, we did not used the binary tree to find max value in the given vector. Instead, we traverse through the vector to find the max value. This is because each attention map is computed independently and the binary tree cannot be re-used. If $M_{\beta}[i] = 0$, $P_0$ and $P_1$ will approximate the $\mathsf{SoftMax}$ function with low-degree polynomial approximations. We detail how $\mathsf{SoftMax}$ can be approximated as follows:
\begin{equation}
\label{eq:app softmax}
\mathsf{ApproxSoftMax}_{i}(x) = \frac{\mathsf{ApproxExp}(x_i)}{\sum_{j\in [d]}\mathsf{ApproxExp}(x_j)}
\end{equation}
\begin{equation}
\mathsf{ApproxExp}(x)=\begin{cases}
    0  &\text{if}\ x \leq T \\
    (1+ \frac{x}{2^n})^{2^n} &\text{if}\ x \in [T,0]\\
\end{cases}
\end{equation}
where the $2^n$-degree Taylor series is used to approximate the exponential function and $T$ is the clipping boundary. The value $n$ and $T$ determines the accuracy of above approximation. With $n=6$ and $T=-13$, the approximation can achieve an average error within $2^{-10}$~\citep{lu2023bumblebee}. For low-degree polynomial approximation, $n=3$ is used in the Taylor series.

Similarly, $P_0$ or $P_1$ can decide whether or not to approximate the $\mathsf{GELU}$ function for each token. If $M_{\beta}[i] = 1$, $P_0$ and $P_1$ use high-degree polynomials~\citep{lu2023bumblebee} to compute the $\mathsf{GELU}$ function on token $x[i]$ with high-degree polynomial:
% \begin{equation}
% \mathsf{GELU}(x) = 0.5x(1+\mathsf{Tanh}(\sqrt{2/\pi}(x+0.044715x^3)))
% \end{equation}
% where the $\mathsf{Tanh}$ and square root function are computed via a OT-based lookup-table.

\begin{equation}
\label{eq:app gelu}
\mathsf{ApproxGELU}(x)=\begin{cases}
    0  &\text{if}\ x \leq -5 \\
    P^3(x), &\text{if}\ -5 < x \leq -1.97 \\
    P^6(x), &\text{if}\ -1.97 < x \leq 3  \\
    x, &\text{if}\ x >3 \\
\end{cases}
\end{equation}
where $P^3(x)$ and $P^6(x)$ are degree-3 and degree-6 polynomials respectively. The detailed coefficient for the polynomial is: 
\begin{equation*}
    P^3(x) = -0.50540312 -  0.42226581x - 0.11807613x^2 - 0.01103413x^3
\end{equation*}
, and
\begin{equation*}
    P^6(x) = 0.00852632 + 0.5x + 0.36032927x^2 - 0.03768820x^4 + 0.00180675x^6
\end{equation*}

For BOLT baseline, we use another high-degree polynomial to compute the $\mathsf{GELU}$ function.

\begin{equation}
\label{eq:app gelu}
\mathsf{ApproxGELU}(x)=\begin{cases}
    0  &\text{if}\ x < -2.7 \\
    P^4(x), &\text{if}\   |x| \leq 2.7 \\
    x, &\text{if}\ x >2.7 \\
\end{cases}
\end{equation}
We use the same coefficients for $P^4(x)$ as BOLT~\citep{pang2023bolt}.

\begin{figure}[h]
 % \vspace{-0.1in}
    \centering
    \includegraphics[width=1\linewidth]{figures/bumble.pdf}
    % \captionsetup{skip=2pt}
    % \vspace{-0.1in}
    \caption{Comparison with prior works on the BERT model. The input has 128 tokens.}
    \label{fig:bumble}
\end{figure}

If $M_{\beta}'[i] = 0$, $P_0$ and $P_1$ will use low-degree 
polynomial approximation to compute the $\mathsf{GELU}$ function instead. Encrypted polynomial reduction leverages low-degree polynomials to compute non-linear functions for less important tokens. For the $\mathsf{GELU}$ function, the following degree-$2$ polynomial~\cite{kim2021ibert} is used:
\begin{equation*}
\mathsf{ApproxGELU}(x)=\begin{cases}
    0  &\text{if}\ x <  -1.7626 \\
    0.5x+0.28367x^2, &\text{if}\ x \leq |1.7626| \\
    x, &\text{if}\ x > 1.7626\\
\end{cases}
\end{equation*}


\section{Comparison with More Related Works.}
\label{app:c}
\textbf{Other 2PC frameworks.} The primary focus of CipherPrune is to accelerate the private Transformer inference in the 2PC setting. As shown in Figure \ref{fig:bumble}, CipherPrune can be easily extended to other 2PC private inference frameworks like BumbleBee~\citep{lu2023bumblebee}. We compare CipherPrune with BumbleBee and IRON on BERT models. We test the performance in the same LAN setting as BumbleBee with 1 Gbps bandwidth and 0.5 ms of ping time. CipherPrune achieves more than $\sim 60 \times$ speed up over BOLT and $4.3\times$ speed up over BumbleBee.

\begin{figure}[t]
 % \vspace{-0.1in}
    \centering
    \includegraphics[width=1\linewidth]{figures/pumab.pdf}
    % \captionsetup{skip=2pt}
    % \vspace{-0.1in}
    \caption{Comparison with MPCFormer and PUMA on the BERT models. The input has 128 tokens.}
    \label{fig:pumab}
\end{figure}

\begin{figure}[h]
 % \vspace{-0.1in}
    \centering
    \includegraphics[width=1\linewidth]{figures/pumag.pdf}
    % \captionsetup{skip=2pt}
    % \vspace{-0.1in}
    \caption{Comparison with MPCFormer and PUMA on the GPT2 models. The input has 128 tokens. The polynomial reduction is not used.}
    \label{fig:pumag}
\end{figure}

\textbf{Extension to 3PC frameworks.} Additionally, we highlight that CipherPrune can be also extended to the 3PC frameworks like MPCFormer~\citep{li2022mpcformer} and PUMA~\citep{dong2023puma}. This is because CipherPrune is built upon basic primitives like comparison and Boolean-to-Arithmetic conversion. We compare CipherPrune with MPCFormer and PUMA on both the BERT and GPT2 models. CipherPrune has a $6.6\sim9.4\times$ speed up over MPCFormer and $2.8\sim4.6\times$ speed up over PUMA on the BERT-Large and GPT2-Large models.


\section{Communication Reduction in SoftMax and GELU.}
\label{app:e}

\begin{figure}[h]
    \centering
    \includegraphics[width=0.9\linewidth]{figures/layerwise.pdf}
    \caption{Toy example of two successive Transformer layers. In layer$_i$, the SoftMax and Prune protocol have $n$ input tokens. The number of input tokens is reduced to $n'$ for the Linear layers, LayerNorm and GELU in layer$_i$ and SoftMax in layer$_{i+1}$.}
    \label{fig:layer}
\end{figure}

\begin{table*}[h]
\captionsetup{skip=2pt}
\centering
\scriptsize
\caption{Communication cost (in MB) of the SoftMax and GELU protocol in each Transformer layer.}
\begin{tblr}{
    colspec = {c |c c c c c c c c c c c c},
    row{1} = {font=\bfseries},
    row{2-Z} = {rowsep=1pt},
    % row{4} = {bg=LightBlue},
    colsep = 2.5pt,
    }
\hline
\textbf{Layer Index} & \textbf{0}  & \textbf{1}  & \textbf{2} & \textbf{3} & \textbf{4} & \textbf{5} & \textbf{6} & \textbf{7} & \textbf{8} & \textbf{9} & \textbf{10} & \textbf{11} \\
\hline
Softmax & 642.19 & 642.19 & 642.19 & 642.19 & 642.19 & 642.19 & 642.19 & 642.19 & 642.19 & 642.19 & 642.19 & 642.19 \\
Pruned Softmax & 642.19 & 129.58 & 127.89 & 119.73 & 97.04 & 71.52 & 43.92 & 21.50 & 10.67 & 6.16 & 4.65 & 4.03 \\
\hline
GELU & 698.84 & 698.84 & 698.84 & 698.84 & 698.84 & 698.84 & 698.84 & 698.84 & 698.84 & 698.84 & 698.84 & 698.84\\
Pruned GELU  & 325.10 & 317.18 & 313.43 & 275.94 & 236.95 & 191.96 & 135.02 & 88.34 & 46.68 & 16.50 & 5.58 & 5.58\\
\hline
\end{tblr}
\label{tab:layer}
\end{table*}

{
In Figure \ref{fig:layer}, we illustrate why CipherPrune can reduce the communication overhead of both  SoftMax and GELU. Suppose there are $n$ tokens in $layer_i$. Then, the SoftMax protocol in the attention module has a complexity of $O(n^2)$. CipherPrune's token pruning protocol is invoked to select $n'$ tokens out of all $n$ tokens, where $m=n-n'$ is the number of tokens that are removed. The overhead of the GELU function in $layer_i$, i.e., the current layer, has only $O(n')$ complexity (which should be $O(n)$ without token pruning). The complexity of the SoftMax function in $layer_{i+1}$, i.e., the following layer, is reduced to $O(n'^2)$ (which should be $O(n^2)$ without token pruning). The SoftMax protocol has quadratic complexity with respect to the token number and the GELU protocol has linear complexity. Therefore, CipherPrune can reduce the overhead of both the GELU protocol and the SoftMax protocols by reducing the number of tokens. In Table \ref{tab:layer}, we provide detailed layer-wise communication cost of the GELU and the SoftMax protocol. Compared to the unpruned baseline, CipherPrune can effectively reduce the overhead of the GELU and the SoftMax protocols layer by layer.
}

\section{Analysis on Layer-wise redundancy.}
\label{app:f}

\begin{figure}[h]
    \centering
    \includegraphics[width=0.9\linewidth]{figures/layertime0.pdf}
    \caption{The number of pruned tokens and pruning protocol runtime in different layers in the BERT Base model. The results are averaged across 128 QNLI samples.}
    \label{fig:layertime}
\end{figure}

{
In Figure \ref{fig:layertime}, we present the number of pruned tokens and the runtime of the pruning protocol for each layer in the BERT Base model. The number of pruned tokens per layer was averaged across 128 QNLI samples, while the pruning protocol runtime was measured over 10 independent runs. The mean token count for the QNLI samples is 48.5. During inference with BERT Base, input sequences with fewer tokens are padded to 128 tokens using padding tokens. Consistent with prior token pruning methods in plaintext~\citep{goyal2020power}, a significant number of padding tokens are removed at layer 0.  At layer 0, the number of pruned tokens is primarily influenced by the number of padding tokens rather than token-level redundancy.
%In Figure \ref{fig:layertime}, we demonstrate the number of pruned tokens and the pruning protocol runtime in each layer in the BERT Base model. We averaged the number of pruned tokens in each layer across 128 QNLI samples and then tested the pruning protocol runtime in 10 independent runs. The mean token number of the QNLI samples is 48.5. During inference with BERT Base, input sequences with small token number are padded to 128 tokens with padding tokens. Similar to prior token pruning methods in the plaintext~\citep{goyal2020power}, a large number of padding tokens can be removed at layer 0. We remark that token-level redundancy builds progressively throughout inference~\citep{goyal2020power, kim2022LTP}. The number of pruned tokens in layer 0 mostly depends on the number of padding tokens instead of token-level redundancy.
}

{
%As shown in Figure \ref{fig:layertime}, more tokens are removed in the intermediate layers, e.g., layer $4$ to layer $7$. This suggests there is more redundant information in these intermediate layers. 
In CipherPrune, tokens are removed progressively, and once removed, they are excluded from computations in subsequent layers. Consequently, token pruning in earlier layers affects computations in later layers, whereas token pruning in later layers does not impact earlier layers. As a result, even if layers 4 and 7 remove the same number of tokens, layer 7 processes fewer tokens overall, as illustrated in Figure \ref{fig:layertime}. Specifically, 8 tokens are removed in both layer $4$ and layer $7$, but the runtime of the pruning protocol in layer $4$ is $\sim2.4\times$ longer than that in  layer $7$.
}

\section{Related Works}
\label{app:g}

{
In response to the success of Transformers and the need to safeguard data privacy, various private Transformer Inferences~\citep{chen2022thex,zheng2023primer,hao2022iron-iron,li2022mpcformer, lu2023bumblebee, luo2024secformer, pang2023bolt}  are proposed. To efficiently run private Transformer inferences, multiple cryptographic primitives are used in a popular hybrid HE/MPC method IRON~\citep{hao2022iron-iron}, i.e., in a Transformer, HE and SS are used for linear layers, and SS and OT are adopted for nonlinear layers. IRON and BumbleBee~\citep{lu2023bumblebee} focus on optimizing linear general matrix multiplications; SecFormer~\cite{luo2024secformer} improves the non-linear operations like the exponential function with polynomial approximation; BOLT~\citep{pang2023bolt} introduces the baby-step giant-step (BSGS) algorithm to reduce the number of HE rotations, proposes a word elimination (W.E.) technique, and uses polynomial approximation for non-linear operations, ultimately achieving state-of-the-art (SOTA) performance.
}

{Other than above hybrid HE/MPC methods, there are also works exploring privacy-preserving Transformer inference using only HE~\citep{zimerman2023converting, zhang2024nonin}. The first HE-based private Transformer inference work~\citep{zimerman2023converting} replaces \mysoftmax function with a scaled-ReLU function. Since the scaled-ReLU function can be approximated with low-degree polynomials more easily, it can be computed more efficiently using only HE operations. A range-loss term is needed during training to reduce the polynomial degree while maintaining high accuracy. A training-free HE-based private Transformer inference was proposed~\citep{zhang2024nonin}, where non-linear operations are approximated by high-degree polynomials. The HE-based methods need frequent bootstrapping, especially when using high-degree polynomials, thus often incurring higher overhead than the hybrid HE/MPC methods in practice.
}

\end{document}
