
\section{The VI Approximation as $N\rightarrow\infty$}\label{sec:theoretical_tool}

Our first set of theoretical results presented in this chapter makes the connection between the NE and a solution of \eqref{eq:mfg_vi_statement} explicit.
We will show that solutions of \eqref{eq:mfg_vi_statement} form good approximations of the true NE in the $N$-player game if $N$ is large.

\begin{remark}[Existence and Uniqueness of MF-NE]
\label{remark:vi_existence}
Let $\vecF:\Delta_\setA \rightarrow [0,1]^K$ be a continuous function.
Then $\vecF$ has at least one MF-NE $\vecpi^*$, and the set of MF-NE is compact.
Furthermore, if $\vecF$ is also $\lambda$-strongly monotone for some $\lambda > 0$, then the MF-NE is unique.
This can be seen as follows.
The MF-NE corresponds to solutions of the VI: $\forall \vecpi \in \Delta_\setA, \vecF(\vecpi^*)^\top (\vecpi^* - \vecpi) \geq 0$.
The domain set $\Delta_\setA$ is compact and convex, and the assumption that $\vecF$ is continuous yields the existence of a solution using Corollary~2.2.5 of \cite{facchinei2003finite}.
For uniqueness in the case of strong monotonicity, see Theorem~2.3.3 of \cite{facchinei2003finite}.
\end{remark}








The following theorem shows that the solution of \eqref{eq:mfg_vi_statement}, when deployed by all players, is a $\mathcal{O}\left(\sfrac{1}{\sqrt{N}}\right)$ solution of the $N$-player game.
Therefore, the MF-NE solution will be an arbitrarily good approximation of the true NE when $N\rightarrow\infty$, and the bias introduced by studying the $N$-player game can be explicitly quantified.

\begin{theorem}\label{theorem:mfg_ne}
    Let $\vecF$ be $L$-Lipschitz, $\delta\geq 0$ arbitrary, and let $\vecpi^*$ be a $\delta$-MF-NE.
Then, the strategy profile $(\vecpi^*, \ldots, \vecpi^*) \in \Delta_\setA^N$ is a $\mathcal{O}\left(\delta + \frac{L}{\sqrt{N}}\right)$-NE of the $N$-player SMFG. 
\end{theorem}

\begin{proof}
Firstly, define the independent random variables $a^j \sim \vecpi^*$ for all $j\in\setN$ for a $\delta$-MF-NE $\vecpi^* \in \Delta_\setA$.
Define the random variable $\widehat{\vecmu} := \sfrac{1}{N} \sum_{j=1}^N \vece_{a^j}$, which is the empirical distribution of players over actions in a single round of an SMFG.
The proof will proceed by formally proving that if $N$ is large enough, then $\Exop[\vecF(\widehat{\vecmu})] \approx \vecF(\vecpi^*)$ and $a^j$ is approximately independent from $\vecF(\widehat{\vecmu})$.

It is straightforward that $\Exop \left[ \widehat{\vecmu}  \right] = \vecpi^*$.
Furthermore, by independence of the random vectors $\vece_{a^j}$, we have
\begin{align*}
    \Exop \left[ \left\| \widehat{\vecmu} - \vecpi^* \right\|_2  \right] \leq
    &\sqrt{\Exop \Big[ \Big\| \frac{1}{N} \sum_{j=1}^N \vece_{a^j} - \vecpi^* \Big\|_2^2  \Big]} 
    \leq  \sqrt{\frac{1}{N^2} \sum_{j=1}^N \Exop \left[  \left\| \vece_{a^j} - \vecpi^* \right\|_2^2 \right]} \leq \frac{2}{\sqrt{N}}.
\end{align*}
Hence, as $\vecF$ is $L$-Lipschitz, we have that
\begin{align}\label{eq:theorem1:ineq2}
\|\Exop[\vecF(\widehat{\vecmu})|a_j\sim \vecpi^*] - \vecF(\vecpi^*)\|_2 \leq \Exop[\|\vecF(\widehat{\vecmu}) - \vecF(\vecpi^*) \|_2] \leq \frac{2L}{\sqrt{N}}.
\end{align}

Now let $i\in \setN$ be arbitrary, and let $\vecpi' \in \Delta_\setA$ be any distribution over actions that satisfies $V^i(\vecpi', \vecpi^{*, -i}) = \max_{\vecpi} V^i(\vecpi, \vecpi^{*, -i})$.
We also define the quantities
\begin{align*}
    \overline{\vecF}_1 = \Exop\left[\vecF(\widehat{\vecmu}) \middle| a^j \sim \vecpi^*, \forall j\in\setN \right], \qquad
    \overline{\vecF}_2 = \Exop\left[\vecF(\widehat{\vecmu}) \middle| a^j \sim \vecpi^* \text{ for } \forall i \neq j, \quad a^i \sim \vecpi' \right].
\end{align*}
We will bound $V^i(\vecpi', \vecpi^{*, -i}) - V^i(\vecpi^*, \vecpi^{*, -i})$.
Combining Lemma~\ref{lemma:technical_bound_1N} and the inequality \eqref{eq:theorem1:ineq2}, we observe
\begin{align*}
    |V^i(\vecpi', \vecpi^{*, -i}) - \vecpi'^\top\vecF(\vecpi^*)| \leq & |V^i(\vecpi', \vecpi^{*, -i}) -  \vecpi'^\top \overline{\vecF}_2 |
    + \Big|\vecpi'^\top \overline{\vecF}_2 - \vecpi'^\top\vecF\Big(\frac{N-1}{N}\vecpi^* + \frac{1}{N} \vecpi'\Big)\Big| \\
    &+ \Big|\vecpi'^\top\vecF\Big(\frac{N-1}{N}\vecpi^* + \frac{1}{N} \vecpi'\Big) - \vecpi'^\top\vecF(\vecpi^*)\Big| \\
    \leq & \frac{L\sqrt{2}}{N} + \frac{2L}{\sqrt{N}} + \frac{2L}{N},
\end{align*}
since $\vecF$ is $L$-Lipschitz.
Likewise, using Lemma~\ref{lemma:technical_bound_1N} once again, we have
\begin{align*}
    |V^i(\vecpi^*, \vecpi^{*, -i}) - \vecpi^{*,\top}\vecF(\vecpi^*)| &\leq |V^i(\vecpi^*, \vecpi^{*, -i}) - \vecpi^{*,\top}\overline{\vecF}_1| + |\vecpi^{*,\top} \overline{\vecF}_1 - \vecpi^{*,\top}\vecF(\vecpi^*)|  \\
    &\leq \frac{L\sqrt{2}}{N} + \frac{2L}{\sqrt{N}}.
\end{align*}
Finally, using the definition of a $\delta$-MF-NE, it holds that
\begin{align*}
V^i(\vecpi', \vecpi^{*, -i}) - V^i(\vecpi^*, \vecpi^{*, -i}) \leq &\vecF(\vecpi^*)^\top (\vecpi' - \vecpi^*) +|V^i(\vecpi^*, \vecpi^{*, -i}) -  \vecpi^{*,\top}\vecF(\vecpi^*)| \\
    & + |V^i(\vecpi', \vecpi^{*, -i}) - \vecpi'^\top\vecF(\vecpi^*)| \\
\leq &\delta + \frac{L(2\sqrt{2} + 4)}{N} + \frac{4L}{\sqrt{N}}.
\end{align*}
\end{proof}

Recall our goal in the context of the $N$-player SMFG is to find policies $\{\vecpi^j\}_{j=1}^N$ with low exploitability $\setE_{\text{exp}}^i$ for all $i$. Theorem~\ref{theorem:mfg_ne} considers that agents adopt the same policy $\vecpi^*$ from solving the VI corresponding to operator $\vecF$ to obtain a low-exploitability approximation. 
We will generalize this result to explicitly bound $\setE_{\text{exp}}^i$ when agent policies can also deviate and when agents can employ regularization.
In our algorithms, regularizing the MF-VI problem will play a crucial role in the IL setting, as it will prevent the policies of agents from diverging when there is no centralized controller synchronizing the policies of agents.
For this reason, our algorithms in the later sections will introduce extraneous regularization to \eqref{eq:mfg_vi_statement} and instead solve the following $\tau$-Tikhonov regularized VI problem:
\begin{align}\label{eq:mfg_rvi_statement}
    \text{Find } \vecpi^* \in \Delta_\setA \text{ s.t. } (\vecF - \tau \matI)(\vecpi^*)^\top (\vecpi^* - \vecpi) \geq 0, \forall \vecpi\in \Delta_\setA. \tag{MF-RVI}
\end{align}

The following theorem quantifies the additional exploitability incurred in the $N$-player game due to (1) extraneous regularization, which is useful for algorithm design, and (2) deviations in agent policies from the MF-NE, potentially due to stochasticity in learning.
Theorem~\ref{theorem:mfgrvi_and_explotability} will be a more useful result later in a learning setting since the learned policies $(\vecpi^1, \ldots, \vecpi^N)$ will only approximate the solution of \eqref{eq:mfg_rvi_statement}.

\begin{theorem}
\label{theorem:mfgrvi_and_explotability}
Let $\vecF$ be monotone, $L$-Lipschitz.
Let $\vecpi_{\tau}^* \in \Delta_\setA$ be the (unique) MF-NE of the regularized map $\vecF - \tau \matI$.
Let $\vecpi^1, \ldots, \vecpi^N \in \Delta_\setA$ be such that $\|\vecpi^i - \vecpi_\tau^*\|_2 \leq \delta$ for all $i$, then it holds that $\setE^i_{\text{exp}}(\{\vecpi^j\}_{j=1}^N) = \mathcal{O}(\tau + \delta + \sfrac{1}{\sqrt{N}})$ for all $i\in\setN$.
\end{theorem}
\begin{proof}
By the Lipschitz continuity of exploitability (Lemma~\ref{lemma:phi_lipschitz}), we have
\begin{align}
    \setE^i_{\text{exp}}(\{ \vecpi^j \}_{j=1}^N) \leq &\setE^i_{\text{exp}}(\{ \vecpi_\tau^* \}_{j=1}^N) + \sqrt{K} \| \vecpi^i - \vecpi_\tau^* \|_2 + \sum_{j\neq i} \frac{4L\sqrt{2K}}{N} \| \vecpi^j - \vecpi_\tau^*\|_2 \notag \\
        \leq & \setE^i_{\text{exp}}(\{ \vecpi_\tau^* \}_{j=1}^N) + \delta \sqrt{K} + 4L\sqrt{2K} \delta. \label{eq:theorem:rviexpbound}
\end{align}
Since $\vecpi_\tau^*$ is the unique MF-NE of the operator $\vecF - \tau \matI$, it holds by definition that
\begin{align*}
    (\vecF - \tau \matI)(\vecpi_{\tau})^\top \vecpi_{\tau} &\geq (\vecF - \tau \matI)(\vecpi_{\tau})^\top \vecpi.
\end{align*}
Organizing both sides, we have
\begin{align*}
    \vecF(\vecpi_{\tau})^\top\vecpi_{\tau} &\geq \vecF(\vecpi_{\tau})^\top \vecpi + \tau \vecpi_{\tau}^\top (\vecpi_{\tau} - \vecpi) \geq \vecF(\vecpi_{\tau})^\top \vecpi - 2\tau,
\end{align*}
as $|\vecpi_{\tau}^\top (\vecpi_{\tau} - \vecpi)| \leq \|\vecpi_{\tau}\|_2 \|\vecpi_{\tau} - \vecpi\|_2 \leq 2$.
Then, $\vecpi^*_{\tau}$ is a $2\tau$-MF-NE for the operator $\vecF$, and by Theorem~\ref{theorem:mfg_ne}, $\setE^i_{\text{exp}}(\{ \vecpi_\tau^* \}_{j=1}^N) \leq \mathcal{O}(\tau + \sfrac{1}{\sqrt{N}})$.
Placing this in~(\ref{eq:theorem:rviexpbound}) proves the theorem.
\end{proof}






To summarize, this section presented key approximation results linking the solutions of \eqref{eq:mfg_vi_statement} and \eqref{eq:mfg_rvi_statement} to the $N$-player exploitability in the SMFG.
The next sections will be devoted to designing sample-efficient IL algorithms.
