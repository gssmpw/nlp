\pdfoutput=1

\documentclass[twoside,11pt]{article}
\usepackage{jmlr2e}



\usepackage{xcolor}
\usepackage[utf8]{inputenc} %
\usepackage[T1]{fontenc}    %
\usepackage{hyperref}       %
\usepackage{url}            %
\usepackage{booktabs}       %
\usepackage{amsfonts}       %

\usepackage{booktabs} %
\usepackage{longtable}

\usepackage{hyperref}

\newcommand{\theHalgorithm}{\arabic{algorithm}}



\usepackage{algorithm}%
\usepackage{algorithmicx}%
\usepackage{algpseudocode}%

\usepackage{amsmath,amssymb}%
\usepackage{graphicx}
\usepackage{mathrsfs}%

\usepackage{mathrsfs}%
\usepackage[title]{appendix}%
\usepackage{xcolor}%
\usepackage{textcomp}%
\usepackage{manyfoot}%

\usepackage{enumitem}

 \usepackage[english]{babel}
\usepackage{mathtools}
\usepackage{thmtools}
\usepackage{thm-restate}
\usepackage{algorithm}[l]
\usepackage{algpseudocode}
\usepackage{multirow}
\usepackage{chngpage}
\usepackage{xfrac}

\usepackage{mathrsfs}
\usepackage{xspace}
\usepackage{bm}
\usepackage{upgreek}
\usepackage{bbm}


\usepackage{subfig}

\section{Problem Studied}\label{sec:def}
We first present Fixed-Radius Near Neighbor (FRNN) queries and then formalize Aggregation Queries over Nearest Neighbors (AQNNs) that build on them. We then state our problem.

\subsection{Nearest Neighbor Queries}\label{subsec:FRNN}
We build on generalized Fixed-Radius Near Neighbor (FRNN) queries \cite{FRNNSurvey}. Given a dataset \( D \), a query object \( q \), a radius \( r \), and a distance function \( dist \), a generalized FRNN query retrieves all nearest neighbors of \( q \) within radius \( r \). More formally:
\[
NN_D(q, r) = \{x \in D \mid dist(x, q) \leq r\},
\]
where \(x\) is any data point in \(D\) and \(dist(x, q)\) denotes the distance between them. We use \(|NN_D(q,r)|\) to denote the neighborhood size of \(q\). As shown in Fig. \ref{fig:framework}, given a radius \(r\) and a target patient \(q\), patients in the dotted circle are nearest neighbors, and the neighborhood size is 6.

\subsection{Aggregation Queries over Nearest Neighbors}\label{subsec:AQNN} 
Given an FRNN query object \(q\) in dataset \(D\), a radius \(r\), and an attribute \(\texttt{attr}\), an Aggregation Query over Nearest Neighbors (AQNN) is defined as:
\[ \text{agg}(NN_D(q,r)[\texttt{attr}]) \]
where agg is an aggregation function, such as $\mathtt{AVG}$, $\mathtt{SUM}$, and $\mathtt{PCT}$, and \(NN_D(q,r)[\texttt{attr}]\) denotes the bag of values of attribute \texttt{attr} of all FRNN results of \(q\) within radius \(r\). 
% \end{definition}

An AQNN expresses aggregation operations to capture key insights about the neighborhood of a query object. For example, \(\mathtt{AVG}\) can be used to reflect the average heart rate or systolic blood pressure of patients in the neighborhood, providing a measure of typical health conditions. \(\mathtt{SUM}\) is useful for assessing cumulative effects, such as the total cost of treatments in the neighborhood that instructs public policy in terms of health. Similarly, $\mathtt{PCT}$ can be used to find the proportion of patients in the neighborhood of a patient of interest, relative to the population in the dataset.
%\laks{Why is finding the total \#meds to NNs or the total treatment cost of everyone in the NN interesting?}

% \texttt{MIN} and \texttt{MAX} are not included in the aggregation functions because they only capture extreme values, which may not represent the typical characteristics of the nearest neighbors and are more sensitive to outliers. 
% \laks{AVG is also sensitive to outliers, but we still allow it. isn't the real reason we don't consider MIN/MAX because they are amenable to estimation via sampling?} We choose \texttt{PCT} instead of \texttt{COUNT} in order to provide a normalized measure that remains comparable across different neighborhood sizes. It allows for more consistent interpretation of relative popularity \cite{moore1989introduction}.


Fig. \ref{fig:framework} illustrates an example of an AQNN: ``\textit{Find the average systolic blood pressure of patients similar to an insomnia patient \(q\)}''. The aggregation function is \(\mathtt{AVG}\) and the target attribute of interest is systolic blood pressure. Exact query evaluation requires consulting physicians (or predicting embeddings by an expensive machine learning model) for all 500 patients in \(D\) and calculate \(q\)'s nearest neighbors wrt \(r\) \cite{DBLP:journals/isci/RodriguesGSBA21}. We refer to such highly accurate but computationally expensive models as \textit{oracle models}, denoted as \(O\), including deep learning models trained on domain-specific data or human expert annotations \cite{DBLP:conf/sigmod/LuCKC18}. Using oracle models is very expensive \cite{sze2017efficient, DujianPQA, DBLP:journals/pvldb/KangGBHZ20}. To address that, we seek an approximate solution by \textit{proxy models}, denoted as \(P\), that are at least one order of magnitude cheaper than oracle models. In the example, if consulting physicians for one patient incurs one cost unit, calling a cheap machine learning model instead incurs at most \(0.1\) cost unit. Once the similar patients are identified, their systolic blood pressure values are averaged and returned as  output. The use of a proxy model may reduce the accuracy of the neighborhood prediction and hence, we should judiciously call oracle and proxy models to minimize the error of aggregate results.

Note that the values of the target attribute \texttt{attr} are \textit{not} predicted but are instead known quantities.

\subsection{Problem Statement}
Given an AQNN, our goal is to return an approximate aggregate result by leveraging both oracle and proxy models while reducing error and cost.




\looseness=-1





\ShortHeadings{VI Approach to Independent Learning in Static MFGs}{VI Approach to Independent Learning in Static MFGs}
\title{
A Variational Inequality Approach to Independent Learning in Static Mean-Field Games
}


\author{\name Batuhan Yardim \email alibatuhan.yardim@inf.ethz.ch \\
     \addr Department of Computer Science\\
          ETH Z\"urich\\
          Z\"urich, Switzerland\\
     \AND
     \name Semih Cayci   \email cayci@mathc.rwth-aachen.de \\
     \addr Department of Mathematics\\
          RWTH Aachen University \\
          Aachen, Germany\\
     \AND
     \name Niao He   \email niao.he@inf.ethz.ch \\
     \addr Department of Computer Science\\
          ETH Z\"urich\\
          Z\"urich, Switzerland
}



\begin{document}

\maketitle

\begin{abstract}%
Competitive games involving thousands or even millions of players are prevalent in real-world contexts, such as transportation, communications, and computer networks. However, learning in these large-scale multi-agent environments presents a grand challenge, often referred to as the "curse of many agents". In this paper, we formalize and analyze the Static Mean-Field Game (SMFG) under both full and bandit feedback, offering a generic framework for modeling large population interactions while enabling independent learning. 


We first establish close connections between SMFG and variational inequality (VI),  showing that SMFG can be framed as a VI problem in the infinite agent limit. Building on the VI perspective, we propose independent learning and exploration algorithms that efficiently converge to approximate Nash equilibria, when dealing with a finite number of agents.   Theoretically, we provide explicit finite sample complexity guarantees for independent learning across various feedback models in repeated play scenarios, assuming (strongly-)monotone payoffs. Numerically, we validate our results through both simulations and real-world applications in city traffic and network access management. 

\end{abstract}

\begin{keywords}
variational inequality, independent learning, mean-field games, multi-agent systems, bandit feedback
\end{keywords}


\section{Introduction}
\label{sec:intro}

\begin{figure*}[tb]
    \centering
    \includegraphics[width=0.848\linewidth]{figs/circuitnn.pdf} 
    \caption{Illustration of differentiable CircuitNN. CircuitNN is designed based on differentiable NAND gates. After DAS is guided by PI and PO pairs of the truth table, CircuitNN can get the precise circuit architecture logic equivalent to the truth table.}
    \label{fig:circuitnn}
\end{figure*}

% 1. Describe the importance of logic synthesis
% 2. Existing Problems
% (a) Neural Architecture Search: Unstable, Predefined Setting, etc.
% (b) Circuit Generation: Probabilistic Model, Logic Equivalence

With the rapid advancement of technology, the scale of integrated circuits (ICs) has expanded exponentially. 
This expansion has introduced significant challenges in chip manufacturing, particularly concerning power and area metrics.
A primary objective in IC design is achieving the same circuit function with fewer transistors, thereby reducing power usage and area occupancy.

Logic synthesis~\cite{hachtel2005logicsynth}, a critical step in electronic design automation (EDA), transforms behavioral-level circuit designs into optimized gate-level circuits, ultimately yielding the final IC layout. 
The primary goal of logic synthesis is to identify the physical implementation with the fewest gates for a given circuit function. 
This task constitutes a challenging NP-hard combinatorial optimization problem. 
Current logic synthesis tools~\cite{brayton2010abc, wolf2013yosys} rely on human-designed heuristics, often leading to sub-optimal outcomes.

Differentiable architecture search (DAS) techniques~\cite{liu2018darts, chu2020darts} offer novel perspectives on addressing challenges in this problem.
Circuit functions can be represented through truth tables, which map binary inputs to their corresponding outputs. 
Truth tables provide a precise representation of input-output relationships, ensuring the design of functionally equivalent circuits.
Inspired by this, researchers~\cite{deepmind2024ai4sys, wang2024tnet} have begun exploring the application of DAS to synthesize circuits directly from truth tables.
Specifically, \citet{deepmind2024ai4sys} proposed CircuitNN, a framework that learns differentiable connection structures with logic gates, enabling the automatic generation of logic circuits from truth tables.
This approach significantly reduces the complexity of traditional circuit generation. 
Building on this, \citet{wang2024tnet} introduced T-Net, a triangle-shaped variant of CircuitNN, incorporating regularization techniques to enhance the efficiency of DAS.

Despite these advancements, several challenges remain. 
The computational complexity of DAS grows quadratically with the number of gates, posing scalability issues.
Although triangle-shaped architecture~\cite{wang2024tnet} partially mitigates this problem, redundancy persists. 
%Additionally, DAS is susceptible to converging to local optima, limiting the ability to search architectures that satisfy the given truth tables~\cite{liu2018darts}. 
%Furthermore, hyperparameters (network depth and layer width) require extensive searches, introducing complexity and prolonging the synthesis process. 
Additionally, DAS is susceptible to converging to local optima~\cite{liu2018darts} and hyperparameters (network depth and layer width) require extensive searches. 
The challenges arise from the vast search space in DAS. 
% Even with predefined settings for CircuitNN, finding a configuration that meets the truth table requires extensive trial and error during the DAS process. 
Intuitively, limiting the search space through predefined parameters (network depth, gates per layer, and connection probabilities) can significantly reduce the complexity.

Recent advances~\cite{openai2023gpt4, abramson2024alphafold3, esser2024sd3, li2024mar} in conditional generative models have demonstrated remarkable performance across language, vision, and graph generation tasks. 
Motivated by these developments, we propose a novel approach to circuit generation that generates preliminary circuit structures to guide DAS in generating refined circuits matching specified truth tables. 
Firstly, we introduce CircuitVQ, a tokenizer with a discrete codebook for circuit tokenization. 
Built upon our Circuit AutoEncoder framework~\cite{hou2022graphmae,li2023maskgae,wu2025mgvga}, CircuitVQ is trained through a circuit reconstruction task. 
Specifically, the CircuitVQ encoder encodes input circuits into discrete tokens using a learnable codebook, while the decoder reconstructs the circuit adjacency matrix based on these tokens.
Subsequently, the CircuitVQ encoder serves as a circuit tokenizer for CircuitAR pretraining, which employs a masked autoregressive modeling paradigm~\cite{chang2022maskgit, li2023mage}. 
In this process, the discrete codes function as supervision signals. 
After training, CircuitAR can generate discrete tokens progressively, which can be decoded into initial circuit structures by the decoder of the CircuitVQ. 
These prior insights can guide DAS in producing refined circuits that match the target truth tables precisely.

Our key contributions can be summarized as follows:
\begin{itemize}
\item We introduce CircuitVQ, a circuit tokenizer that facilitates graph autoregressive modeling for circuit generation, based on our Circuit AutoEncoder framework;
\item Develop CircuitAR, a model trained using masked autoregressive modeling, which generates initial circuit structures conditioned on given truth tables;
\item Propose a refinement framework that integrates differentiable architecture search to produce functionally equivalent circuits guided by target truth tables;
\item Comprehensive experiments demonstrating the scalability and capability emergence of our CircuitAR and the superior performance of the proposed circuit generation approach.
\end{itemize}

% Motivation
% (a) Diffusion (Vision, Graph), Autoregressive (Language, Vision)
% (b) Circuit Generation for Predefined Setting
% (c) Neural Architecture Search for Strict Logic Equivalence

% Contribution
% (a) Circuit Tokenizer (new transformer arch, training strategy)
% (b) CircuitAR (train and gen strategies, post-ar strategy)
% (c) Extensive Evaluation including BitD (Bit Distance) for Scalability

\section{Related Work}
% \subsection{Vision Language Model}
% 시각장애인에서 상황을 설명할 DB가 없으니 만들었다. 그리고 이를 VLM에 튜닝했다.
\subsection{Technical approaches for assisting the visually-impaired}


\subsection{Datasets for visual instruction tuning}

\section{Adaptive labeling as a Markov decision process} 
\label{sec:formulation}

We illustrate our formulation for model evaluation, and extend it to the ATE estimation setting at the end of the section. 
Our goal is to evaluate the performance of a prediction model $\model: \statdomain \to \mathcal \labeldomain$ over the input distribution $P_X$ that we expect to see during deployment.  Given inputs $X  \in \mc{X}$,   labels/outcomes are generated
 from some unknown function $f\opt$: $
      Y = f\opt(X) + \varepsilon$, where $\varepsilon$ is the noise.
      %~~~\mbox{where}~~\varepsilon \sim N(0, \sigma^2)
  % \begin{equation*}
  %     Y = f\opt(X) + \varepsilon~~~\mbox{where}~~\varepsilon \sim N(0, \sigma^2).
  % \end{equation*}
When ground truth outcomes are costly to obtain, previously collected labeled data $\mc{D}^0 := \{(X_i,Y_i)\}_{i \in \mc{I}}$ 
typically suffers selection bias and covers only a subset of the support of input distribution $P_X$ over which we aim to evaluate the model performance. 

Assuming we have a   pool of data $\xpool$, we design
 adaptive sampling algorithms that iteratively select
inputs in $\xpool$ to be labeled.
Since labeling inputs takes time in practice, we model
real-world instances by considering \emph{batched} settings. Our goal is to sequentially label batches of data to accurately estimate model performance over $P_X$ and therefore we assume we have access to a set of inputs $\xeval \sim P_X$. %We assume the modeler pays a fixed and equal cost for each label/outcome. 
%Our framework is general as we do not assume \xpool∼PX\xpool \sim P_X.
We use the squared loss to illustrate our framework,
where our goal is to evaluate $\E_{X \sim P_X}[ (Y - \model(X))^2]$. Under the ``likelihood" function $p(y | f, x) = p_{\varepsilon}(y - f(x))$,  let $g(f)$ be the performance of the AI model $\model(\cdot)$ under the  data generating function $f$, which we refer to as our estimand of interest.
When we consider the mean squared loss,  $g(f)$ is given by 
\begin{align}
    g(f) \defeq \E_{X \sim P_X}\left[ \E_{Y \sim p(\cdot|f,X) } \Big[ (Y - \model(X))^2 \Big] \mid f \right]. \label{eqn:l2-g-f}
\end{align}
Our framework is general and can be extended to other settings. For example, a clinically useful  metric is \texttt{Recall}, defined as the fraction of individuals that the model $\model(\cdot)$ correctly labels as positive  among all the individuals who actually have the positive label 
\begin{align*}
    g(f) \defeq  \E_{X \sim P_X}\left[ \E_{Y \sim p(\cdot|f,X) } \Big[\indic{\model(X)>0}|Y=1\Big] \mid f\right].
\end{align*}
 
  Since the true function $f\opt$ is unknown, we  model it from a Bayesian perspective by formulating a posterior given the available supervised data. We refer to uncertainty over the data generating function $f$ as \emph{epistemic} uncertainty---since we can resolve it with more data---and that over
 the measurement noise $\varepsilon$ as \emph{aleatoric} uncertainty. 
Assuming independence given features $X$, we model the  likelihood of the data via the product 
$p({Y}_{1:m}|f, {X}_{1:m}) = \prod_{i=1}^m p(Y_i|f,X_i)$.
 Our prior belief  $\mu$ over functions $f$   reflects our uncertainty about how
labels are generated given features. 
To adaptively label inputs from $\mc{X}_{\rm pool}$, we assume access to an uncertainty quantification (UQ) method that provides posterior beliefs $\mu(f \mid \mc{D})$ given
any supervised data $\mc{D}:= \{(X_i,Y_i)\}_{i \in \mc{I}}$. As we detail  in Section~\ref{sec:uq}, our framework can leverage both classical 
Bayesian models like Gaussian processes and recent advancements in deep learning-based UQ  methods.

As new batches are labeled, we update our posterior beliefs about $f$ over time, which we view as ``state transitions'' of a dynamical system.
Recalling the Markov decision process depicted in Figure~\ref{fig:overview}, we sequentially label a batch of inputs from $\mc{X}_{\rm pool}$ (actions), which lead to state transitions (posterior updates).
Specifically, our initial state is given by $\mu_0(\cdot) = \mu(\cdot \mid \mc{D}^0)$, where $\mc{D}^0$ represents the initial labeled dataset.
At each period $t$, we label a batch of $K_t$ inputs $\mc{X}^{t+1} \subset \mc{X}_{\rm pool}$ resulting in labeled data $ \mc{D}^{t+1} = (\mc{X}^{t+1}, \datay^{t+1})$. After acquiring the labels at each step $t$, we update the posterior state to $\mu_{t+1}(\cdot) = \mu_t(\cdot \mid \mc{D}^{t+1})$. Modeling practical instances, we consider a small horizon problem with limited adaptivity $T$. Formulating an MDP over posterior states has long conceptual roots, dating back to the Gittin's index for multi-armed bandits~\citep{Gittins79}.

 We denote by $\pi_t$ the adaptive labeling policy at period $t$. We account for randomized policies $\datax^{t+1} \sim \pi_t(\mu_t)$ with a flexible batch size $|\datax^{t+1}| = \batchsize_t$.   
We assume $\pi_t$ is $\mc{F}_t-$measurable for all $t < T$, where $\mc{F}_t$ is the filtration generated by the observations up to the end of step $t$.
 Observe that $\mu_{t+1}$ contains randomness in the policy $\pi_t$ as well as randomness in $\datay^{t+1} \mid (\datax^{t+1},\mu_t)$. Letting $\pi = \set{\pi_0,....,\pi_{T-1}}$,  we minimize the uncertainty over $g(f)$
 at the end of data collection
\begin{align}
H(\pi) \defeq \E_{\mc{D}^{1:T} \sim \pi} \left[G(\mu_{T}) \right]  \defeq  \E_{\mc{D}^{1:T} \sim \pi} \left[G(\mu(\cdot \mid \mc{D}^{0:T})) \right]
%\defeq \E_{\mc{D}^{1:T} \sim \pi} \left[ \V_{f \sim \mu_{T}}  g(f)  \right]
     = \E_{\mc{D}^{1:T} \sim \pi} \left[ \V_{f \sim \mu(\cdot \mid \mc{D}^{0:T})}  g(f)  \right],
     \label{eqn:general-obj}
\end{align}   
where $G(\mu_T) = \V_{f \sim \mu_T}  g(f)$.
In the above objective~\eqref{eqn:general-obj}, we assume
that the modeler pays a fixed and equal cost for each outcome. 
Our framework can also seamlessly accommodate variable labeling cost. Specifically, we can define a cost function $c(\cdot)$ 
 applied on the selected subsets 
 and update the objective~\eqref{eqn:general-obj} accordingly to include the term  $\lambda c(\mc{D}^{1:T})$,
 where $\lambda$
 is the penalty factor that controls the trade-off between minimizing variance and cost of acquiring samples.


Our framework can be easily extended to causal estimation problems.  Consider a feature vector ${X}$ and suppose we have two treatment arms $Z \in \{0,1\}$. Our objective is to evaluate the average treatment effect over the population distribution $P_X$.  Given feature vector $X$, and treatment $Z$,  outcomes are generated from an unknown function $f\opt$: 
$Y = f\opt(X,Z) + \varepsilon.$
%~~~\mbox{where}~~\varepsilon \sim N(0, \sigma^2)
  % \begin{equation*}
  %     Y = f\opt(X) + \varepsilon~~~\mbox{where}~~\varepsilon \sim N(0, \sigma^2).
  % \end{equation*}
The available data is denoted by $\mc{D}^0 := \{X_i,Y_i,Z_i\}_{i \in \mc{I}}$ and given a pool of candidates 
$\xpool$, we want an
 adaptive sampling algorithms that iteratively select
candidates in $\xpool$ to be assigned a random treatment so that we can estimate average treatment effect efficiently. Under the ``likelihood" function $p(y | f, x, z) = p_{\varepsilon}(y - f(x,z))$,  let $g(f)$ represent  the average treatment effect, which is our estimand of interest. Formally, this is expressed as:
\begin{align}
g(f) \defeq \E_{X \sim P_X} \left[\E_{Y_1 \sim p(\cdot|f,X,Z=1) , Y_0 \sim p(\cdot|f,X,Z=0)} \left[Y_1 -  Y_0 \right]\mid f \right]. \label{eqn:ate-g-f}
\end{align}




 %Again the true function $f\opt$ is unknown and we model it from a Bayesian perspective by formulating a posterior  given available data.
 Our prior belief  $\mu$ over functions $f$, now 
 reflects our uncertainty about how
outcomes are generated given features and treatments. 
  We sequentially observe outcome of a batch of inputs from $\mc{X}_{\rm pool}$ (actions), and treatments assigned to this batch. We assume that selected batch of inputs $\mc{X}^t$ is randomly assigned treatments $\mc{Z}^t$ with each $Z\sim p_Z$. We summarize our formulation in Figure~\ref{fig:MDP_framework_flowchart}.
%Specifically, our initial state is given by $\mu_0(\cdot) = \mu(\cdot \mid \mc{D}^0)$ and at each period $t$, we get outcomes for a batch of $K$ candidates $\mc{X}^{t+1} \subset \mc{X}_{\rm pool}$, with randomly assigned treatments $\mc{Z}^{t+1}$ (with each $Z\sim p_Z$) and get the data $ \mc{D}^{t+1} = (\mc{X}^{t+1} \times \datay^{t+1} \times \mc{Z}^{t+1} )$. After acquiring the data at each step $t$ we update posterior state to $\mu_{t+1}(\cdot) = \mu_t(\cdot \mid \mc{D}^{t+1})$. Modeling practical instances, we consider a small horizon problem with limited adaptivity $T$.  We denote by $\pi_t$ the adaptive labeling policy at period $t$. We account for randomized policies $\datax^{t+1} \sim \pi_t(\mu_t)$ with a flexible batch size $|\datax^{t+1}| = \batchsize_t$.   
Again, we assume $\mu_t$ is $\mc{F}_t-$measurable for all $t < T$, where $\mc{F}_t$ is the filtration generated by the observations up to the end of step $t$.
 Observe that $\mu_{t+1}$ contains randomness in the policy $\pi_t$, randomness in treatment assignment $\mc{Z}^{t+1}$ and randomness in $\datay^{t+1} \mid (\datax^{t+1}, \mc{Z}^{t+1},\mu_t)$. Letting $\pi = \set{\pi_0,....,\pi_{T-1}}$,  we minimize the uncertainty over $g(f)$
 at the end of data collection:
\begin{align}
\E_{\mc{D}^{1:T} \sim \pi} \left[G(\mu_{T}) \right] \defeq
\E_{\mc{D}^{1:T} \sim \pi} \left[ \V_{f \sim \mu_{T}}  g(f)  \right]
= \E_{\mc{D}^{1:T} \sim \pi} \left[ \V_{f \sim \mu(\cdot \mid \mc{D}^{0:T})}  g(f)  \right].
\label{eqn:general-ate-obj}
\end{align}  


 

\begin{figure}[ht]
\centering
\begin{tikzpicture}
[
roundnode/.style={circle, draw=black!60, very thick, minimum size=10mm},
squarednode/.style={rectangle, draw=black!60, very thick, minimum size=10mm, align =center,text width = 26mm},
]
%Nodes
\node[roundnode]      (maintopic)                              {$\mu$};
\node[roundnode]        (circle1)       [right=20mm of maintopic] {$\mu_0$};
\node[roundnode]      (circle2)       [right=5mm of circle1] {$\mu_t$};
\node[roundnode]        (circle3)       [right=20mm of circle2] {$\mu_{t+1}$};
\node[roundnode]        (circle4)       [right=5mm of circle3] {$\mu_T$};
\node[squarednode]        (circle5)       [right=5mm of circle4] {Reward/Cost $\E \left[ \V_{\mu_T} (g(f))\right]$};


%Lines
\draw[thick, ->, >=stealth] (maintopic.east) -- node[anchor=south] {$(\mathcal{X}^0,\mathcal{Y}^0,\mathcal{Z}^0)$} (circle1.west);
\draw[thick, ->, dashed] (circle1.east) --  (circle2.west);
\draw[thick, ->] (circle2.east)  -- node[above, align =center, text width = 18mm] { Query $(\mathcal{X}^t,\mathcal{Y}^t,\mathcal{Z}^t)$} (circle3.west);
\draw[thick, ->,dashed] (circle3.east) --  (circle4.west);
\draw[thick, ->] (circle4.east) --  (circle5.west);


\end{tikzpicture}
\caption{MDP framework for adaptive labeling to efficiently estimate the average treatment effect (ATE).}
\label{fig:MDP_framework_flowchart}
\end{figure}
 









\begin{comment}
\subsection{Broader applicability of the framework to other problem settings} \label{sec:broad-framework-accuracy}
 

Although  we describe our setting in a healthcare setting with the objective  to estimate the recall of a trained AI model $\model(\cdot)$, the framework caters to many other problem settings. The extension to the evaluation of model based on accuracy (in regression setting) is straightforward, we simply replace the definition of recall $g(f)$ in~\eqref{eqn:l2-g-f} with
\begin{align*}
    g(f) = \E_{\substack{ y \sim p(y|f,x) \\  \forall x \in \mathcal X}} \big( \E_{{\textbf x} \sim p_x} [y-\model(x)]^2 \big).
\end{align*}


\textcolor{red}{To discuss if we need to have it here}
We can also extend this setting to the efficient estimation of the ATE as well. We describe these in detail below:

\begin{itemize}
    
    \item Estimating accuracy:  \[g(f) = \E_{\substack{ y \sim p(y|f,x) \\  \forall x \in \mathcal X}} \big( \E_{{\textbf x} \sim p_x} [y-\model(x)]^2 \big)\]
%    \item Estimating ATE with known control arm: 
%\[g(f) = \E_{\substack{ y \sim p(y|f,x) \\  \forall x \in \mathcal X}} \big( \E_{{\textbf x} \sim p_x} [y-\model(x)] \big)\]
\item Estimating ATE  (with minor modifications - broad structure remains similar) : 


Consider feature vector ${\mathbf x} \in \mathcal X $  distributed as ${\mathbf x}  \sim p_{\mathbf x}$, treatment $z \in {\mathcal Z} = \{0,1\}$, and a class of random functions $f: {\mathcal X} \times {\mathcal Z} \to {\mathcal Y}$, which determines the likelihood $p(y_i|f,{\mathbf x_i},z_i)$. Note that $f$ is random and reflects our uncertainty about how
labels are generated given features and the treatment. Additionally, the joint likelihood is determined as follows,  

\[p(Y|f,X,Z) = \prod_{i} p(y_i|f,{\mathbf x_i}, z_i) \]

Assuming the prior over functions $f$ to be $\mu$, therefore we have 
\[p(Y|X,Z) = \int \prod_{i} p(y_i|f,{\mathbf x_i},z_i) d\mu(f) \]


Also, assuming that under the  true data generating function $f$ (if known precisely - which we don't), the estimand of interest is

\[ \E_{{\textbf x} \sim p_x}  \left( \E_{\substack{ y \sim p(y|x,f,z=1) }} y - \E_{\substack{ y \sim p(y|x,f,z=0) }} y \right) \]


Throughout the paper we assume the above data generating process.  Now, suppose we have some labeled  data $(\datax^0,\datay^0,Z^0) =({\mathbf x}_{1:m}^0,y_{1:m}^0, z_{1:m}^0)$. 
    We run a experiment, in which we want to query the labels (in batches), so as to minimize the uncertainty of the estimand of interest. Suppose, the horizon of the experiment is $T$. Now, given prior $\mu$ and labeled data $\datax^0,\datay^0,Z^0$, in the beginning of our experiment the posterior state is $\mu_0$.

 At each step $j$ ($j \geq 1$), we query labels for a batch (with size $k$) of unlabeled data $(\datax^j,Z^j) \subset \mathcal X \times \mathcal Z$  and get labels $\datay^j$. After acquiring the labels at each step $j$ we update posterior state to $\mu_{j+1}$, informed by $\mu_j$ and $(\datax^j,\datay^j,Z^j)$. 
 
 Let the policy at step $j$ be $\pi_j$ (potentially random), which gives $\datax^{j+1},Z^{j+1} \sim \pi_j(\mu_j)$.  Observe that $\mu_{j+1}$ is random because of the randomness of the policy $\pi_j$ and $\datay^{j+1}|\{\datax^{j+1},Z^{j+1},\mu_j\}$ (\textcolor{red}{can this be written in a better way?}). Let, $\pi = \{\pi_0,....,\pi_{T-1}\}$. Therefore, our objective is to

 
\[ \min_{\pi} \E \left[ {\mathbf {Var}}_{f \sim \mu_T} \left( \E_{{\textbf x} \sim p_x}  \left( \E_{\substack{ y \sim p(y|x,f,z=1) }} y - \E_{\substack{ y \sim p(y|x,f,z=0) }} y \right) \right) \right]\]

where, $\mu_T$ depends on $\{(\datax^i,\datay^i,Z^i)\}_{i=0}^T$ and outer expectation is over both $\pi$ and  $\datay^{j+1}|\{\datax^{j+1}, Z^{j+1},\mu_j\}$ for all $j \in [0,T-1]$.


%Constraining the action space is straightforward - by first choosing set of x's using k-subset and then assigning treatment with learnable probability parameters $w_1,...,w_n$.

\end{itemize}

 \[ g(f) = \E_{\substack{ y \sim p(y|f,x) \\  \forall x \in \mathcal X}}\E_{{\textbf x} \sim p_x} g(y,{\textbf x}) \approx \E_{\substack{ y \sim p(y|f,x) \\  \forall x \in  \datax^u}} \left( \frac{1}{n}\sum_{i=1}^n \tilde{g}(y,{\textbf x}_i^u) \right)\]




Notation borrowed from a combination of the following papers 
~\citep{LeeYuNaFoLe23, KatoOgKoIn24, FongHoWa24}

%  
\end{comment}


%%% Local Variables:
%%% mode: latex
%%% TeX-master: "main"
%%% End:



\section{The VI Approximation as $N\rightarrow\infty$}\label{sec:theoretical_tool}

Our first set of theoretical results presented in this chapter makes the connection between the NE and a solution of \eqref{eq:mfg_vi_statement} explicit.
We will show that solutions of \eqref{eq:mfg_vi_statement} form good approximations of the true NE in the $N$-player game if $N$ is large.

\begin{remark}[Existence and Uniqueness of MF-NE]
\label{remark:vi_existence}
Let $\vecF:\Delta_\setA \rightarrow [0,1]^K$ be a continuous function.
Then $\vecF$ has at least one MF-NE $\vecpi^*$, and the set of MF-NE is compact.
Furthermore, if $\vecF$ is also $\lambda$-strongly monotone for some $\lambda > 0$, then the MF-NE is unique.
This can be seen as follows.
The MF-NE corresponds to solutions of the VI: $\forall \vecpi \in \Delta_\setA, \vecF(\vecpi^*)^\top (\vecpi^* - \vecpi) \geq 0$.
The domain set $\Delta_\setA$ is compact and convex, and the assumption that $\vecF$ is continuous yields the existence of a solution using Corollary~2.2.5 of \cite{facchinei2003finite}.
For uniqueness in the case of strong monotonicity, see Theorem~2.3.3 of \cite{facchinei2003finite}.
\end{remark}








The following theorem shows that the solution of \eqref{eq:mfg_vi_statement}, when deployed by all players, is a $\mathcal{O}\left(\sfrac{1}{\sqrt{N}}\right)$ solution of the $N$-player game.
Therefore, the MF-NE solution will be an arbitrarily good approximation of the true NE when $N\rightarrow\infty$, and the bias introduced by studying the $N$-player game can be explicitly quantified.

\begin{theorem}\label{theorem:mfg_ne}
    Let $\vecF$ be $L$-Lipschitz, $\delta\geq 0$ arbitrary, and let $\vecpi^*$ be a $\delta$-MF-NE.
Then, the strategy profile $(\vecpi^*, \ldots, \vecpi^*) \in \Delta_\setA^N$ is a $\mathcal{O}\left(\delta + \frac{L}{\sqrt{N}}\right)$-NE of the $N$-player SMFG. 
\end{theorem}

\begin{proof}
Firstly, define the independent random variables $a^j \sim \vecpi^*$ for all $j\in\setN$ for a $\delta$-MF-NE $\vecpi^* \in \Delta_\setA$.
Define the random variable $\widehat{\vecmu} := \sfrac{1}{N} \sum_{j=1}^N \vece_{a^j}$, which is the empirical distribution of players over actions in a single round of an SMFG.
The proof will proceed by formally proving that if $N$ is large enough, then $\Exop[\vecF(\widehat{\vecmu})] \approx \vecF(\vecpi^*)$ and $a^j$ is approximately independent from $\vecF(\widehat{\vecmu})$.

It is straightforward that $\Exop \left[ \widehat{\vecmu}  \right] = \vecpi^*$.
Furthermore, by independence of the random vectors $\vece_{a^j}$, we have
\begin{align*}
    \Exop \left[ \left\| \widehat{\vecmu} - \vecpi^* \right\|_2  \right] \leq
    &\sqrt{\Exop \Big[ \Big\| \frac{1}{N} \sum_{j=1}^N \vece_{a^j} - \vecpi^* \Big\|_2^2  \Big]} 
    \leq  \sqrt{\frac{1}{N^2} \sum_{j=1}^N \Exop \left[  \left\| \vece_{a^j} - \vecpi^* \right\|_2^2 \right]} \leq \frac{2}{\sqrt{N}}.
\end{align*}
Hence, as $\vecF$ is $L$-Lipschitz, we have that
\begin{align}\label{eq:theorem1:ineq2}
\|\Exop[\vecF(\widehat{\vecmu})|a_j\sim \vecpi^*] - \vecF(\vecpi^*)\|_2 \leq \Exop[\|\vecF(\widehat{\vecmu}) - \vecF(\vecpi^*) \|_2] \leq \frac{2L}{\sqrt{N}}.
\end{align}

Now let $i\in \setN$ be arbitrary, and let $\vecpi' \in \Delta_\setA$ be any distribution over actions that satisfies $V^i(\vecpi', \vecpi^{*, -i}) = \max_{\vecpi} V^i(\vecpi, \vecpi^{*, -i})$.
We also define the quantities
\begin{align*}
    \overline{\vecF}_1 = \Exop\left[\vecF(\widehat{\vecmu}) \middle| a^j \sim \vecpi^*, \forall j\in\setN \right], \qquad
    \overline{\vecF}_2 = \Exop\left[\vecF(\widehat{\vecmu}) \middle| a^j \sim \vecpi^* \text{ for } \forall i \neq j, \quad a^i \sim \vecpi' \right].
\end{align*}
We will bound $V^i(\vecpi', \vecpi^{*, -i}) - V^i(\vecpi^*, \vecpi^{*, -i})$.
Combining Lemma~\ref{lemma:technical_bound_1N} and the inequality \eqref{eq:theorem1:ineq2}, we observe
\begin{align*}
    |V^i(\vecpi', \vecpi^{*, -i}) - \vecpi'^\top\vecF(\vecpi^*)| \leq & |V^i(\vecpi', \vecpi^{*, -i}) -  \vecpi'^\top \overline{\vecF}_2 |
    + \Big|\vecpi'^\top \overline{\vecF}_2 - \vecpi'^\top\vecF\Big(\frac{N-1}{N}\vecpi^* + \frac{1}{N} \vecpi'\Big)\Big| \\
    &+ \Big|\vecpi'^\top\vecF\Big(\frac{N-1}{N}\vecpi^* + \frac{1}{N} \vecpi'\Big) - \vecpi'^\top\vecF(\vecpi^*)\Big| \\
    \leq & \frac{L\sqrt{2}}{N} + \frac{2L}{\sqrt{N}} + \frac{2L}{N},
\end{align*}
since $\vecF$ is $L$-Lipschitz.
Likewise, using Lemma~\ref{lemma:technical_bound_1N} once again, we have
\begin{align*}
    |V^i(\vecpi^*, \vecpi^{*, -i}) - \vecpi^{*,\top}\vecF(\vecpi^*)| &\leq |V^i(\vecpi^*, \vecpi^{*, -i}) - \vecpi^{*,\top}\overline{\vecF}_1| + |\vecpi^{*,\top} \overline{\vecF}_1 - \vecpi^{*,\top}\vecF(\vecpi^*)|  \\
    &\leq \frac{L\sqrt{2}}{N} + \frac{2L}{\sqrt{N}}.
\end{align*}
Finally, using the definition of a $\delta$-MF-NE, it holds that
\begin{align*}
V^i(\vecpi', \vecpi^{*, -i}) - V^i(\vecpi^*, \vecpi^{*, -i}) \leq &\vecF(\vecpi^*)^\top (\vecpi' - \vecpi^*) +|V^i(\vecpi^*, \vecpi^{*, -i}) -  \vecpi^{*,\top}\vecF(\vecpi^*)| \\
    & + |V^i(\vecpi', \vecpi^{*, -i}) - \vecpi'^\top\vecF(\vecpi^*)| \\
\leq &\delta + \frac{L(2\sqrt{2} + 4)}{N} + \frac{4L}{\sqrt{N}}.
\end{align*}
\end{proof}

Recall our goal in the context of the $N$-player SMFG is to find policies $\{\vecpi^j\}_{j=1}^N$ with low exploitability $\setE_{\text{exp}}^i$ for all $i$. Theorem~\ref{theorem:mfg_ne} considers that agents adopt the same policy $\vecpi^*$ from solving the VI corresponding to operator $\vecF$ to obtain a low-exploitability approximation. 
We will generalize this result to explicitly bound $\setE_{\text{exp}}^i$ when agent policies can also deviate and when agents can employ regularization.
In our algorithms, regularizing the MF-VI problem will play a crucial role in the IL setting, as it will prevent the policies of agents from diverging when there is no centralized controller synchronizing the policies of agents.
For this reason, our algorithms in the later sections will introduce extraneous regularization to \eqref{eq:mfg_vi_statement} and instead solve the following $\tau$-Tikhonov regularized VI problem:
\begin{align}\label{eq:mfg_rvi_statement}
    \text{Find } \vecpi^* \in \Delta_\setA \text{ s.t. } (\vecF - \tau \matI)(\vecpi^*)^\top (\vecpi^* - \vecpi) \geq 0, \forall \vecpi\in \Delta_\setA. \tag{MF-RVI}
\end{align}

The following theorem quantifies the additional exploitability incurred in the $N$-player game due to (1) extraneous regularization, which is useful for algorithm design, and (2) deviations in agent policies from the MF-NE, potentially due to stochasticity in learning.
Theorem~\ref{theorem:mfgrvi_and_explotability} will be a more useful result later in a learning setting since the learned policies $(\vecpi^1, \ldots, \vecpi^N)$ will only approximate the solution of \eqref{eq:mfg_rvi_statement}.

\begin{theorem}
\label{theorem:mfgrvi_and_explotability}
Let $\vecF$ be monotone, $L$-Lipschitz.
Let $\vecpi_{\tau}^* \in \Delta_\setA$ be the (unique) MF-NE of the regularized map $\vecF - \tau \matI$.
Let $\vecpi^1, \ldots, \vecpi^N \in \Delta_\setA$ be such that $\|\vecpi^i - \vecpi_\tau^*\|_2 \leq \delta$ for all $i$, then it holds that $\setE^i_{\text{exp}}(\{\vecpi^j\}_{j=1}^N) = \mathcal{O}(\tau + \delta + \sfrac{1}{\sqrt{N}})$ for all $i\in\setN$.
\end{theorem}
\begin{proof}
By the Lipschitz continuity of exploitability (Lemma~\ref{lemma:phi_lipschitz}), we have
\begin{align}
    \setE^i_{\text{exp}}(\{ \vecpi^j \}_{j=1}^N) \leq &\setE^i_{\text{exp}}(\{ \vecpi_\tau^* \}_{j=1}^N) + \sqrt{K} \| \vecpi^i - \vecpi_\tau^* \|_2 + \sum_{j\neq i} \frac{4L\sqrt{2K}}{N} \| \vecpi^j - \vecpi_\tau^*\|_2 \notag \\
        \leq & \setE^i_{\text{exp}}(\{ \vecpi_\tau^* \}_{j=1}^N) + \delta \sqrt{K} + 4L\sqrt{2K} \delta. \label{eq:theorem:rviexpbound}
\end{align}
Since $\vecpi_\tau^*$ is the unique MF-NE of the operator $\vecF - \tau \matI$, it holds by definition that
\begin{align*}
    (\vecF - \tau \matI)(\vecpi_{\tau})^\top \vecpi_{\tau} &\geq (\vecF - \tau \matI)(\vecpi_{\tau})^\top \vecpi.
\end{align*}
Organizing both sides, we have
\begin{align*}
    \vecF(\vecpi_{\tau})^\top\vecpi_{\tau} &\geq \vecF(\vecpi_{\tau})^\top \vecpi + \tau \vecpi_{\tau}^\top (\vecpi_{\tau} - \vecpi) \geq \vecF(\vecpi_{\tau})^\top \vecpi - 2\tau,
\end{align*}
as $|\vecpi_{\tau}^\top (\vecpi_{\tau} - \vecpi)| \leq \|\vecpi_{\tau}\|_2 \|\vecpi_{\tau} - \vecpi\|_2 \leq 2$.
Then, $\vecpi^*_{\tau}$ is a $2\tau$-MF-NE for the operator $\vecF$, and by Theorem~\ref{theorem:mfg_ne}, $\setE^i_{\text{exp}}(\{ \vecpi_\tau^* \}_{j=1}^N) \leq \mathcal{O}(\tau + \sfrac{1}{\sqrt{N}})$.
Placing this in~(\ref{eq:theorem:rviexpbound}) proves the theorem.
\end{proof}






To summarize, this section presented key approximation results linking the solutions of \eqref{eq:mfg_vi_statement} and \eqref{eq:mfg_rvi_statement} to the $N$-player exploitability in the SMFG.
The next sections will be devoted to designing sample-efficient IL algorithms.

\section{Convergence in the Full Feedback Case}\label{sec:expert_feedback_results}

We first present an IL algorithm for the full feedback setting, as a first step towards analyzing the more interesting bandit feedback setting.
In this setting, while there is no centralized controller, independent noisy reports of all action payoffs are available to each agent after each round.

\textbf{Is it possible to simply solve MF-RVI in our IL setting?}
Before we present our results, we note the following:
Past works in MFG have already proved approximation results of $N$-agent games by MFG albeit in different settings \citep{saldi2019approximate, yardim2024mean}, but these results do not consider when \emph{learning itself} is carried out with $N$ agents. 
If $N$ agents can not communicate, it is theoretically challenging to approximate the MF-RVI and to tackle bandit feedback.
Most importantly, the IL algorithms formalized in Section~\ref{section:alg_formalization} can not query an operator oracle or maintain a common iterate throughout repeated plays.
Therefore, the approximation properties of \eqref{eq:mfg_vi_statement} do not immediately imply the MF-NE can be learned using VI algorithms.
In this section and the next, we prove the more challenging result of convergence with IL, first under full feedback and later under partial (bandit) feedback.

Our analysis builds up on Tikhonov regularized projected ascent (TRPA).
The TRPA operator is defined as
\begin{align}
    \Gamma^{\eta, \tau}(\vecpi) := \Pi_{\Delta_\setA} ( \vecpi + \eta (\vecF - \tau \matI)(\vecpi) ) = \Pi_{\Delta_\setA} ( (1-\eta\tau) \vecpi + \eta \vecF(\vecpi) ), \tag{TRPA}
\end{align}
for a learning rate $\eta > 0$ and regularization $\tau > 0$.
Intuitively, $\Gamma^{\eta, \tau}$ uses $\vecF$ evaluated at $\vecpi$ to modify action probabilities in the direction of the greatest payoff, incorporating an $\ell_2$ regularizer of $\tau$.
Furthermore, the unique MF-NE $\vecpi^*$ of \eqref{eq:mfg_rvi_statement} is also a fixed point of $\Gamma^{\eta, \tau}$.
The analysis of TRPA is standard and known to converge for monotone $\vecF$ \citep{facchinei2003finite, nemirovski2004prox}, when (stochastic) oracle access to $\vecF$ is assumed.
Naturally, the main complication in applying the method above will be the fact that in the IL setting, agents can not evaluate the operator $\vecF$ arbitrarily, but rather can only observe (a noisy) estimate of $\vecF$ as a function of the empirical population distribution and not of their policy $\vecpi$.
In the full feedback setting, we analyze the following dynamics:
\begin{align}
     \vecpi_0^i := \operatorname{Unif} (\setA) = \frac{1}{K}\vecone_K, \hspace{1em} \vecpi^i_{t+1} =\Pi_{\Delta_\setA} ( (1 - \tau \eta_t) \vecpi_t^i + \eta_t \vecr_t^i ), \tag{TRPA-Full}
\end{align}
for a time varying learning rate $\eta_t$, for each agent $i \in \setN$.
The extraneous $\ell_2$-regularization incorporated in each agent running TRPA-Full is critical for the analysis and convergence in IL, as it allows explicit synchronization of policies of agents without communication.
We state the TRPA-Full algorithm in Algorithm~\ref{alg:full} for reference.

We state the following standard result regarding the TRPA operator without proof, as it will be used later.

\begin{lemma}[cf. Theorem 12.1.2 of \cite{facchinei2003finite}]\label{lemma:contraction_pg}
Assume $\vecF$ is $\lambda\geq 0$-monotone and $L$-Lipschitz.
Then $\Gamma^{\eta,\tau}$ is Lipschitz with constant $\sqrt{1 - 2 (\lambda + \tau) \eta + \eta^2 (L+\tau)^2}$ with respect to the $\ell_2$-norm.
\end{lemma}

\begin{algorithm}
    \caption{TRPA-Full: IL with full feedback algorithm for each agent $i \in \setN$.}\label{alg:full}
    \begin{algorithmic}
    \Require Number of actions $K$, regularization $\tau > 0$, learning rate $\{\eta_t\}_{t=0}^T$, rounds $T > 0$.
    \State $\vecpi^i_0 \leftarrow \frac{1}{K} \vecone$
    \For{$t = 0, \ldots, T-1$}
    \State \text{Play action with current policy $a^i_{t}\sim \vecpi^i_t$}.
    \State \text{Observe payoff $\vecr^i_{t}$}
    \State $\vecpi^i_{t+1} = \Pi_{\Delta_\setA} ( (1 - \tau \eta_t) \vecpi^i_t + \eta_t \vecr^i_t )$
    \EndFor
    \State Return $\vecpi^i_T$
    \end{algorithmic}
\end{algorithm}

For abuse of notation, let $\vecpi^* \in \Delta_\setA$ be the unique solution of \eqref{eq:mfg_rvi_statement} for the regularization $\tau > 0$.
Also define the sigma algebra $\mathcal{F}_{t} := \mathcal{F}(\{ \vecpi_{t'}^i \}_{t'=0, \ldots, t}^{i=1, \ldots, N})$.
We maintain the definitions of the core random variables of the SMFG dynamics introduced in Section~\ref{sec:game_initial_formulation},
\begin{align*}
    \widehat{\vecmu}_t := \frac{1}{N} \sum_{i=1}^N \vece_{a_t^i}, \quad \vecr^i_t := \vecF(\widehat{\vecmu}_t) + \vecn_t^i.
\end{align*}
Under TRPA-Full dynamics, we also define the following random variables that assist our analysis.
\begin{align*}
    \bar{\vecmu}_t &:= \frac{1}{N} \sum_{i=1}^N \vecpi_t^i, \quad e_t^i := \|\vecpi^i_t - \bar{\vecmu}_t \|_2^2, \quad
    u_t^i := \Exop\left[\| \vecpi_t^i - \vecpi^* \|_2^2\right].
\end{align*}
We call $\bar{\vecmu}_t$ the mean policy, $e_t^i$ the mean policy deviation, and $u_t^i$ the expected $\ell_2$-deviation from the regularized MF-NE.
Our goal is to bound the sequence or error terms $u_t^i$; however, the process is complicated by the fact that in general the policy deviations of agents $e_t^i$ are nonzero.
Our strategy is as follows: 
(1) derive a  recursion for $u_t^i$ incorporating the terms $e_t^i$, 
(2) bound the terms $e_t^i$, showing the deviation of the policies of the agents goes to zero in expectation, and
(3) solve the recursion to obtain the convergence rate.

The following lemma captures the first step and provides a recurrence for the evolution of $u_t^i$ under TRPA-Full.

\begin{lemma}[Error recurrence under full feedback]\label{lemma:full_error_recurrence}
    Under TRPA-Full with learning rates $\eta_t$, it holds for $L$-Lipschitz and $\lambda$-strongly monotone $\vecF$ that
    \begin{align*}
    \Exop\left[\| \vecpi_{t+1}^i - \vecpi^* \|_2^2\right] \leq &3\eta_t^2 K(1 + \sigma^2) + 2\eta_t^2(L+\tau)^2 + \frac{4\eta_t L^2 \lambda^{-1}}{ N } \\
        & + 2\eta_t L^2 \lambda^{-1} \Exop\left[e_t^i\right] + \left(1 - 2 \eta_t(\sfrac{\lambda}{2} + \tau)\right) \Exop\left[\| \vecpi_t^i - \vecpi^* \|_2^2\right],
\end{align*}
and for $L$-Lipschitz and monotone $\vecF$ that
\begin{align*}
    \Exop\left[\| \vecpi_{t+1}^i - \vecpi^* \|_2^2\right] \leq &3\eta_t^2 K(1 + \sigma^2) + 2\eta_t^2(L+\tau)^2 + \frac{4\tau^{-1} \eta_t L^2 \delta^{-1}}{ N} \\
        & + \tau^{-1}\eta_t L^2\delta^{-1} \Exop\left[e_t^i\right] + \left(1 - 2\tau \eta_t (1-\delta)\right) \Exop\left[\| \vecpi_t^i - \vecpi^* \|_2^2\right],
\end{align*}
for arbitrary $\delta \in (0,1)$.
\end{lemma}
\begin{proof}
We analyze for any $i\in[N]$ the error term $\| \vecpi_t^i - \vecpi^*\|_2^2$.
Denote $\alpha_t := (1 - \tau \eta_t)$.
For the regularized solution $\vecpi^*$, we have the fixed point result
\begin{align*}
    \Pi_{\Delta_{\setA}} ((1 - \tau \eta_t) \vecpi^* + \eta_t \vecF(\vecpi^*)) = \Pi_{\Delta_{\setA}} (\vecpi^* + \eta_t (\vecF - \tau \matI)(\vecpi^*)) = \vecpi^*.
\end{align*}
The proof strategy is to decompose the $\ell_2$ distance of player policies to $\vecpi^*$ into 3 components using this property.
We can bound the quantity $\| \vecpi_{t+1}^i - \vecpi^*\|_2^2$ by using the non-expansiveness of $\Pi_{\Delta_{\setA}}$:
\begin{align}
    \| \vecpi_{t+1}^i - \vecpi^*\|_2^2 = &\| \Pi_{\Delta_{\setA}}(\alpha_t \vecpi^i_t + \eta_t \vecr_t^i) - \Pi_{\Delta_{\setA}} (\alpha_t \vecpi^* + \eta_t \vecF(\vecpi^*)) \|_2^2 \notag \\
        \leq &\| \alpha_t \vecpi_t^i + \eta_t \vecF(\vecpi_t^i) - \alpha_t \vecpi^* - \eta_t \vecF(\vecpi^*) + \eta_t (\vecr_t^i - \vecF(\vecpi_t^i) )\|_2^2 \notag \\
        = & \eta_t^2\| \vecr_t^i - \vecF(\vecpi_t^i) \|_2^2 + 2\eta_t (\alpha_t (\vecpi_t^i - \vecpi^*) + \eta_t (\vecF(\vecpi_t^i) - \vecF(\vecpi^*)) )^\top (\vecr_t^i - \vecF(\vecpi_t^i)) \notag \\
         & + \|\alpha_t (\vecpi_t^i - \vecpi^*) + \eta_t (\vecF(\vecpi_t^i) - \vecF(\vecpi^*))\|_2^2 \notag \\
         \leq & \underbrace{\eta_t^2\| \vecr_t^i - \vecF(\vecpi_t^i) \|_2^2 + 2\eta_t^2 (\vecF(\vecpi_t^i) - \vecF(\vecpi^*))^\top (\vecr_t^i - \vecF(\vecpi_t^i))}_{(a)} \notag \\
            &+ \underbrace{2\eta_t\alpha_t (\vecpi_t^i - \vecpi^*)^\top (\vecr_t^i - \vecF(\vecpi_t^i))}_{(b)} + \underbrace{\|\alpha_t (\vecpi_t^i - \vecpi^*) + \eta_t (\vecF(\vecpi_t^i) - \vecF(\vecpi^*))\|_2^2 }_{(c)}. \label{ineq:decomp_abc_full_recur}
\end{align}

We analyze the three marked terms separately.
For term $(a)$, using the independence assumption of the noise vectors and Young's inequality, in expectation we obtain
\begin{align*}
    \Exop[(a)] \leq &\eta_t^2 \Exop[\| \vecr_t^i - \vecF(\vecpi_t^i) \|_2^2] + \eta_t^2 \Exop[\|\vecF(\vecpi_t^i) - \vecF(\vecpi^*)\|_2^2] + \eta_t^2 \Exop[\| \vecr_t^i - \vecF(\vecpi_t^i) \|_2^2] \\
    \leq &2\eta_t^2 \Exop[\| \vecr_t^i - \vecF(\vecpi_t^i) \|_2^2] + \eta_t^2 \Exop[\|\vecF(\vecpi_t^i) - \vecF(\vecpi^*)\|_2^2] \\
    \leq & 2\eta_t^2 \Exop[\| \vecr_t^i - \vecF(\widehat{\vecmu}_t)\|_2^2 + \| \vecF(\widehat{\vecmu}_t) - \vecF(\vecpi^i_t)\|_2^2 ] + \eta_t^2 K \\
    \leq & 2\eta_t^2 \sigma^2 K + 3\eta_t^2 K \leq 3\eta_t^2 K(\sigma^2 + 1)
\end{align*}
For the term $(c)$, we obtain
\begin{align*}
    (c) = &\|\alpha_t (\vecpi_t^i - \vecpi^*) + \eta_t (\vecF(\vecpi_t^i) - \vecF(\vecpi^*))\|_2^2 \\
        = & \|(\vecpi_t^i - \vecpi^*) + \eta_t (\vecF(\vecpi_t^i) - \tau \vecpi_t^i - \vecF(\vecpi^*) + \tau \vecpi^* )\|_2^2 \\
        \leq & \left(1 - 2 (\lambda + \tau) \eta_t + (L + \tau)^2 \eta_t^2\right) \| \vecpi_t^i - \vecpi^* \|_2^2 \\
        \leq & \left(1 - 2 (\lambda + \tau) \eta_t \right) \| \vecpi_t^i - \vecpi^* \|_2^2 + 2(L + \tau)^2 \eta_t^2
\end{align*}
where the last inequality holds from the Lipschitz continuity result of Lemma~\ref{lemma:contraction_pg}.

For the term $(b)$, first taking the strongly monotone problem $\lambda > 0$ , we have that
\begin{align*}
(b) = & 2\eta_t\alpha_t (\vecpi_t^i - \vecpi^*)^\top (\vecr_t^i - \vecF(\vecpi_t^i)) \\
 = & 2\eta_t\alpha_t (\vecpi_t^i - \vecpi^*)^\top (\vecr_t^i - \vecF(\widehat{\vecmu}_t)) + 2\eta_t\alpha_t (\vecpi_t^i - \vecpi^*)^\top (\vecF(\widehat{\vecmu}_t) - \vecF(\bar{\vecmu}_t)) \\
    & + 2\eta_t\alpha_t (\vecpi_t^i - \vecpi^*)^\top (\vecF(\bar{\vecmu}_t) - \vecF(\vecpi_t^i)) \\
\leq &2\eta_t\alpha_t \left( \frac{\lambda}{4} \|\vecpi_t^i - \vecpi^* \|_2^2 + \frac{1}{\lambda} \|\vecF(\widehat{\vecmu}_t) - \vecF(\bar{\vecmu}_t)\|_2^2\right) + 2\eta_t\alpha_t \left(\frac{\lambda}{4} \|\vecpi_t^i - \vecpi^* \|_2^2 + \frac{1}{\lambda} \|\vecF(\bar{\vecmu}_t) - \vecF(\vecpi_t^i)\|_2^2 \right) \\
    &+2\eta_t\alpha_t (\vecpi_t^i - \vecpi^*)^\top (\vecr_t^i - \vecF(\widehat{\vecmu}_t)) \\
\leq & \eta_t \lambda \|\vecpi_t^i - \vecpi^* \|_2^2 + 2\eta_t\lambda^{-1}\|\vecF(\widehat{\vecmu}_t) - \vecF(\bar{\vecmu}_t)\|_2^2 + 2\eta_t\lambda^{-1} \|\vecF(\bar{\vecmu}_t) - \vecF(\vecpi_t^i)\|_2^2 \\
    &+2\eta_t\alpha_t (\vecpi_t^i - \vecpi^*)^\top (\vecr_t^i - \vecF(\widehat{\vecmu}_t)),
\end{align*}
which follows from applications of Young's inequality.
For the last three terms we observe:
\begin{align*}
    \Exop\left[2\eta_t\alpha_t (\vecpi_t^i - \vecpi^*)^\top (\vecr_t^i - \vecF(\widehat{\vecmu}_t)) | \mathcal{F}_t\right] = &0, \\
    \Exop[\|\vecF(\widehat{\vecmu}_t) - \vecF(\bar{\vecmu}_t)\|_2^2 | \mathcal{F}_{t}] \leq & L^2 \Exop\left[ \|\widehat{\vecmu}_t - \bar{\vecmu}_t\|_2^2 | \mathcal{F}_{t}\right] \\
    \leq & L^2\Exop\left[\frac{1}{N^2}\|\sum_{i}\vecpi_t^i - \sum_{i} \vece_{a_t^i}\|^2_2 | \mathcal{F}_{t}\right] \\
    = & \frac{L^2}{N^2}\sum_{i}\Exop[\|\vecpi_t^i - \vece_{a_t^i}\|^2_2 | \mathcal{F}_{t}] \leq \frac{2L^2}{N}, \\
    \|\vecF(\bar{\vecmu}_t) - \vecF(\vecpi_t^i)\|_2^2 \leq & L^2 \|\bar{\vecmu}_t - \vecpi_t^i\|_2^2 = L^2 e_t^i.
\end{align*}
The second inequality above follows from the fact that $\widehat{\vecmu}_t$ is the sum of $N$ independent random variables and has expectation $\bar{\vecmu}_t$.
Hence, putting in the bounds for $(a), (b), (c)$ and taking expectations, we obtain the inequality
\begin{align*}
    \Exop\left[\| \vecpi_{t+1}^i - \vecpi^*\|_2^2 \right] \leq & 3 \eta_t^2 K(1 + \sigma^2) + \frac{4\eta_t L^2}{\lambda N} + \frac{2\eta_t L^2}{\lambda} \Exop\left[e_t^i\right] \\
        &+\left(1 - 2 (\sfrac{\lambda}{2} + \tau) \eta_t \right) \Exop\left[\| \vecpi_t^i - \vecpi^* \|_2^2\right] + 2(L + \tau)^2 \eta_t^2.
\end{align*}

Turning back to the monotone case, if $\lambda=0$, vary the upper bound on $(b)$ as follows.
Take any arbitrary $\delta \in (0,1)$.
Then, once again applying Young's inequality, we obtain
\begin{align*}
(b) = & 2\eta_t\alpha_t (\vecpi_t^i - \vecpi^*)^\top (\vecr_t^i - \vecF(\vecpi_t^i)) \\
 = & 2\eta_t\alpha_t (\vecpi_t^i - \vecpi^*)^\top (\vecr_t^i - \vecF(\widehat{\vecmu}_t)) + 2\eta_t\alpha_t (\vecpi_t^i - \vecpi^*)^\top (\vecF(\widehat{\vecmu}_t) - \vecF(\bar{\vecmu}_t)) \\
    & + 2\eta_t\alpha_t (\vecpi_t^i - \vecpi^*)^\top (\vecF(\bar{\vecmu}_t) - \vecF(\vecpi_t^i)) \\
\leq &2\eta_t\alpha_t \left( \frac{\tau\delta}{2} \|\vecpi_t^i - \vecpi^* \|_2^2 + \frac{1}{2\tau\delta} \|\vecF(\widehat{\vecmu}_t) - \vecF(\bar{\vecmu}_t)\|_2^2\right) + 2\eta_t\alpha_t \left(\frac{\tau\delta}{2} \|\vecpi_t^i - \vecpi^* \|_2^2 + \frac{1}{2\tau\delta} \|\vecF(\bar{\vecmu}_t) - \vecF(\vecpi_t^i)\|_2^2 \right)  \\
    &+2\eta_t\alpha_t (\vecpi_t^i - \vecpi^*)^\top (\vecr_t^i - \vecF(\widehat{\vecmu}_t)) \\
\leq & 2 \eta_t \tau\delta \|\vecpi_t^i - \vecpi^* \|_2^2 + \frac{\eta_t}{\tau\delta}\|\vecF(\widehat{\vecmu}_t) - \vecF(\bar{\vecmu}_t)\|_2^2 + \frac{\eta_t}{\tau\delta} \|\vecF(\bar{\vecmu}_t) - \vecF(\vecpi_t^i)\|_2^2 \\
    &+2\eta_t\alpha_t (\vecpi_t^i - \vecpi^*)^\top (\vecr_t^i - \vecF(\widehat{\vecmu}_t)).
\end{align*}
Applying the same bounds for the terms $(a), (c)$ as before yields  the lemma.
\end{proof}

This above lemma has two key features: a dependence on expected mean policy deviation $\Exop\left[e_t^i\right]$, and a term that scales as $\mathcal{O}(\sfrac{1}{N})$.
While the $\mathcal{O}(\sfrac{1}{N})$ term can be anticipated (and asymptotically ignored when $N$ is large) due to the finite-agent mean-field bias (as shown previously in Section~\ref{sec:theoretical_tool}), the term $\Exop\left[e_t^i\right]$ must be controlled separately in the independent learning setting, where policies cannot be synchronized through explicit communication between agents.
The term $\Exop\left[e_t^i\right]$ reflects the core difference of the SMFG model from typical VI stochastic oracles.
Unlike typical VI oracle models, in SMFG the operator $\vecF$ cannot be evaluated at the current iterate $\vecpi^i_t$ of a player $i$  but only approximately at the mean $\bar{\vecmu}_t$.
This is due to decentralized learning: players can only evaluate the current payoffs at the ``mean-iterate'' given by $\vecF(\widehat{\vecmu}_t) \approx \vecF(\bar{\vecmu}_t)$ (up to some stochastic noise) that is almost surely different than their iterates $\{\vecpi^i_t\}_i$ apart from the case with degenerate/zero noise.
Furthermore, Lemma~\ref{lemma:full_error_recurrence} suggests that the algorithmic scheme must guarantee that $\Exop\left[e_t^i\right]$ decays with the rate at least $\mathcal{O}(\sfrac{1}{t})$ to obtain a non-vacuous bound on exploitability.
Taking inspiration from algorithmic stability literature \citep{ahn2022reproducibility, zhang2024optimal}, we utilize a regularization scheme to ensure the iterates of players do not diverge.
The following lemma shows that by introducing explicit regularization $\tau>0$, the expected mean policy deviation can be controlled throughout training.



\begin{lemma}[Policy variations bound]\label{lemma:policy_variations_bound_trpa_full}
    Under TRPA-Full with learning rates $\eta_t :=\frac{\tau^{-1}}{t+2}$, we have $\Exop\left[e_t^i\right] \leq \frac{14 \tau^{-2} K\sigma^2 + 14}{t+2}$.
\end{lemma}
\begin{proof}
Note that for any $i,j\in\setN$ such that $i\neq j$, using the non-expansiveness of the projection operator it holds that
\begin{align*}
    \| \vecpi^i_{t+1} - \vecpi^j_{t+1} \|_2^2 = &  \| \Pi_{\Delta_\setA}((1 - \tau \eta_t) \vecpi^i_t + \eta_t \vecr_t^i) - \Pi_{\Delta_\setA}((1 - \tau \eta_t) \vecpi^j_t + \eta_t \vecr_t^j) \|_2^2 \\
    \leq & \| (1 - \tau \eta_t) \vecpi^i_t + \eta_t \vecr_t^i - (1 - \tau \eta_t) \vecpi^j_t - \eta_t \vecr_t^j \|_2^2 \\
    \leq & \| (1 - \tau \eta_t) (\vecpi^i_t - \vecpi^j_t) + \eta_t (\vecr_t^i - \vecr_t^j) \|_2^2 \\
    = & (1 - \tau \eta_t)^2 \| \vecpi^i_t - \vecpi^j_t \|_2^2 + \eta_t^2 \|  \vecr_t^i - \vecr_t^j \|_2^2 + 2 (1 - \tau \eta_t) \eta_t (\vecpi^i_t - \vecpi^j_t) ^ \top ( \vecr_t^i - \vecr_t^j )
\end{align*}
Taking the conditional expectation on both sides, we obtain
\begin{align*}
    \Exop \left[ \| \vecpi^i_{t+1} - \vecpi^j_{t+1} \|_2^2 | \mathcal{F}_t \right] \leq & (1 - \tau \eta_t)^2 \| \vecpi^i_t - \vecpi^j_t \|_2^2 + \Exop \left[ \eta_t^2 \|  \vecr_t^i - \vecr_t^j \|_2^2 | \mathcal{F}_t \right] \\
        &+ 2 (1 - \tau \eta_t) \eta_t (\vecpi^i_t - \vecpi^j_t) ^ \top \Exop\left[\vecr_t^i - \vecr_t^j | \mathcal{F}_t \right] \\
    = & (1 - \tau \eta_t)^2 \|  \vecpi^i_t - \vecpi^j_t \|_2^2 + \eta_t^2 \Exop \left[\|\vecn^i_t - \vecn^j_t\|_2^2 | \mathcal{F}_t \right] \\
    = & (1 - \tau \eta_t)^2 \|  \vecpi^i_t - \vecpi^j_t \|_2^2 + 2\eta_t^2 K \sigma^2
\end{align*}
almost surely, since we have $\vecr_t^i := \vecF(\widehat{\vecmu}_t) + \vecn^i_t$.
Then, taking the expectation on both sides, 
\begin{align*}
    \Exop \left[ \| \vecpi^i_{t+1} - \vecpi^j_{t+1} \|_2^2 \right] \leq &(1 - \tau \eta_t)^2 \Exop\left[\|  \vecpi^i_t - \vecpi^j_t \|_2^2\right] + 2\eta_t^2 K\sigma^2 \\
    \leq & \left(1 - \frac{1}{t+2}\right)^2 \Exop\left[\|  \vecpi^i_t - \vecpi^j_t \|_2^2\right] + \left(\frac{\tau^{-1}}{t+2}\right)^2 2K\sigma^2 \\
    \leq & \left(1 - \frac{2}{t+2}\right) \Exop\left[\|  \vecpi^i_t - \vecpi^j_t \|_2^2\right] + \frac{1}{(t+2)^2} \Exop\left[\|  \vecpi^i_t - \vecpi^j_t \|_2^2\right] + \frac{2\tau^{-2}K\sigma^2}{(t+2)^2} \\
    \leq & \left(1 - \frac{2}{t+2}\right) \Exop\left[\|  \vecpi^i_t - \vecpi^j_t \|_2^2\right] + \frac{2\tau^{-2}K\sigma^2 + 2}{(t+2)^2}
\end{align*}
To bound the recurrence, we can use the recurrence lemma (Lemma~\ref{lemma:general_recurrence}, noting $\gamma=2, a = 2, u_0 = 0, c_0 = 0, c_1 = 2\tau^{-2}K\sigma^2 + 2$ in its statement):
\begin{align*}
    \Exop \left[ \| \vecpi^i_{t+1} - \vecpi^j_{t+1} \|_2^2 \right] \leq & 5\frac{2\tau^{-2}K\sigma^2 + 2}{(t+2)^2} + 3\frac{2\tau^{-2}K\sigma^2 + 2}{t+2} + \frac{2\tau^{-2}K\sigma^2 + 2}{(t+2)^2} \leq \frac{14 \tau^{-2} K\sigma^2 + 14}{t+2}.
\end{align*}
Then, the expected values of $e_t^i$ can be bounded using:
\begin{align*}
     e_t^i = &\|\vecpi^i_t - \bar{\vecmu}_t \|_2^2 
     =  \left\|\vecpi^i_t - \frac{1}{N} \sum_{j=1}^N \vecpi^j_t  \right\|_2^2 
     \leq \frac{1}{N} \sum_{j=1}^N \| \vecpi^i_t - \vecpi^j_t \|_2^2
\end{align*}
by an application of Jensen's inequality.
Then we have $\Exop\left[e_t^i\right] \leq \frac{14 \tau^{-2} K\sigma^2 + 14}{t+2}$.
\end{proof}

With an explicit bound in expectation on the mean policy deviation $e_t^i$, we can now proceed to the main recurrence for the expected error terms $u_t^i$ in order to prove our main convergence result.
We state our main convergence result for TRPA-Full dynamics in Theorem~\ref{theorem:expert_short} by solving these two recurrences for the monotone and strongly monotone cases.

\begin{theorem}[Convergence, full feedback]\label{theorem:expert_short}
Assume $\vecF$ is Lipschitz, monotone.
Assume $N$ agents run the TRPA-Full update rule for $T$ time steps with learning rates $\eta_t := \frac{\tau^{-1}}{t+2}$ and arbitrary regularization $\tau>0$.
Then it holds for any $i\in[N]$ that $\Exop\left[ \setE^i_{\text{exp}}( \{\vecpi^j_{T}\}_{j=1}^N ) \right] \leq \mathcal{O} (\frac{\tau^{-2}}{\sqrt{T}}+ \frac{\tau^{-1}}{\sqrt{N}} + \tau)$.
Furthermore, if $\vecF$ is $\lambda$-strongly monotone, then $\Exop\left[ \setE^i_{\text{exp}}( \{\vecpi^j_{T}\}_{j=1}^N ) \right] \leq \mathcal{O} (\frac{\tau^{-\sfrac{3}{2}} \lambda^{-\sfrac{1}{2}}}{\sqrt{T}} + \frac{\tau^{-\sfrac{1}{2}} \lambda^{-\sfrac{1}{2}}}{\sqrt{N}} + \tau)$.
\end{theorem}
\begin{proof}
Note that the exploitability in the main statement of the theorem can be related to $u_t^i$ as follows using Lemma~\ref{lemma:phi_lipschitz}:
\begin{align*}
    \Exop[\setE^i_{\text{exp}}(\{\vecpi^j_{t}\}_{j=1}^N)] \leq &\setE^i_{\text{exp}}(\{\vecpi^*\}_{j=1}^N) + \sqrt{K} \Exop[\| \vecpi_t^i - \vecpi^* \|_2] + \frac{4L\sqrt{2K}}{N} \sum_{j\neq i} \Exop[\| \vecpi_t^j - \vecpi^* \|_2] \\
    \leq & \setE^i_{\text{exp}}(\{\vecpi^*\}_{j=1}^N) + \sqrt{K} \sqrt{u_t^i} +  \frac{4L\sqrt{2K}}{N} \sum_{j\neq i} \sqrt{u_t^j} \\
    \leq & \setE^i_{\text{exp}}(\{\vecpi^*\}_{j=1}^N) + \frac{\max\{ \sqrt{K}, 4L\sqrt{2K} \}}{N} \sum_{j} \sqrt{u_t^j}
\end{align*}
Hence the bounds on $u_t^j$ will yield the result of the theorem by linearity of expectation, along with an invocation of Theorem~\ref{theorem:mfgrvi_and_explotability}.

Finally, we solve the recurrences for $\lambda = 0$ and $\lambda > 0$ using Lemma~\ref{lemma:full_error_recurrence}.
For the case $\lambda > 0$, if $\eta_t=\frac{\tau^{-1}}{t+2}$, Lemma~\ref{lemma:full_error_recurrence} provides the bound 
\begin{align*}
    u_{t+1}^i \leq &\frac{ 3\tau^{-2} K(1 + \sigma^2) + 2\tau^{-2}(L+\tau)^2}{(t+2)^2} + \frac{4\tau^{-1} L^2 \lambda^{-1}}{ N (t+2)} + \frac{2\tau^{-1} L^2 \lambda^{-1}}{t+2} \Exop\left[e_t^i\right] \\
        &+ \left(1 - \frac{2 \tau^{-1}(\sfrac{\lambda}{2} + \tau)}{t+2}\right) u_{t}^i. 
\end{align*}
By placing $\Exop\left[ e^i_t\right] \leq \frac{14 \tau^{-2} K\sigma^2 + 14}{t+2}$ due to Lemma~\ref{lemma:policy_variations_bound_trpa_full}, we obtain
\begin{align*}
    u_{t+1}^i \leq &\frac{ 3\tau^{-2}K(1 + \sigma^2) + 2\tau^{-2}(L+\tau)^2 +  28\tau^{-3} K \sigma^2 \lambda^{-1} L^2 + 28 \tau^{-1} L^2 \lambda^{-1}}{(t+2)^2} \\
        &+ \frac{4\tau^{-1} L^2 \lambda^{-1}}{ N (t+2)} + \left(1 - \frac{2}{t+2}\right) u_{t}^i.
\end{align*}
Invoking a generic recurrence lemma (Lemma~\ref{lemma:general_recurrence} in Appendix~\ref{app:basic_inequalities}) leads to the main statement of the theorem.

For the monotone case $\lambda = 0$, we have the recursion:
\begin{align*}
    u_{t+1}^i \leq &\frac{ 3\tau^{-2}K(1 + \sigma^2) + 2\tau^{-2}(L+\tau)^2}{(t+2)^2} + \frac{4\tau^{-2} L^2 \delta^{-1}}{ N (t+2)} + \frac{ \tau^{-2} L^2\delta^{-1}}{ t+2 } \Exop\left[e_t^i\right] \\
        & + \left(1 - \frac{2 (1-\delta)}{t+2}\right) u_{t}^i.
\end{align*}
and once again placing the upper bound on expected policy deviation due to Lemma~\ref{lemma:policy_variations_bound_trpa_full},
\begin{align*}
    u_{t+1}^i \leq &\frac{ 3\tau^{-2}K(1 + \sigma^2) + 2\tau^{-2}(L+\tau)^2 + 28 K \tau^{-4} L^2\delta^{-1}\sigma^2 +28 \tau^{-2} L^2\delta^{-1}}{(t+2)^2} \\
        &+ \frac{2\tau^{-2} L^2 \delta^{-1}}{ N (t+2)} + \left(1 - \frac{2 (1-\delta)}{t+2}\right) u_{t}^i.
\end{align*}
Another invocation of Lemma~\ref{lemma:general_recurrence} concludes the proof, choosing $\delta=\sfrac{1}{4}$.
\end{proof}

This convergence result is stated in terms of exploitability of the unregularized game, leading to an additional $\mathcal{O}(\tau)$ term.
However, in many cases, the Nash equilibrium of the regularized game itself is of interest, in which case the upper bounds should read
$\mathcal{O} (\frac{\tau^{-2}}{\sqrt{T}}+ \frac{\tau^{-1}}{\sqrt{N}})$
and
$\mathcal{O} (\frac{\tau^{-\sfrac{3}{2}} \lambda^{-\sfrac{1}{2}}}{\sqrt{T}} + \frac{\tau^{-\sfrac{1}{2}} \lambda^{-\sfrac{1}{2}}}{\sqrt{N}})$
for the monotone and strongly monotone cases respectively.

In the choice of learning rate $\eta_t$ above, no intrinsic problem parameter is assumed to be known.
Furthermore, due to (1) the regularization $\tau$ and (2) a finite population, a non-vanishing exploitability of $\mathcal{O}(\tau + \sfrac{\tau^{-1}}{\sqrt{N}})$ will be induced in terms of the NE in the monotone case.
While Theorem~\ref{theorem:mfgrvi_and_explotability} readily suggested a bias of order $\mathcal{O}(\sfrac{1}{\sqrt{N}})$ is fundamental, when learning is conducted with finitely many agents Theorem~\ref{theorem:expert_short} shows this is amplified to $\mathcal{O}(\sfrac{\tau^{-1}}{\sqrt{N}})$.
Since for finite population SMFG, there will always be a non-vanishing exploitability in terms of NE due to the mean-field approximation, in practice $\tau$ could be chosen to incorporate an acceptable bias level.
Alternatively, if the exact value of the number of players $N$ is known by each agent, one could choose $\tau$ optimally, to obtain the following corollary.

\begin{corollary}[Optimal $\tau$, full feedback]\label{corollary:expert}
Assume the conditions of Theorem~\ref{theorem:expert_short}.
For monotone $\vecF$, choosing regularization parameter $\tau = \sfrac{1}{\sqrt[4]{N}}$ yields
$\Exop\left[\setE^i_{\text{exp}}(\{\vecpi^j_T\}_{j=1}^N) \right] \leq \mathcal{O}(\frac{\sqrt{N}}{\sqrt{T}} + \frac{1}{\sqrt[4]{N}})$ for any $i$.
For $\lambda$-strongly monotone $\vecF$, choosing $\tau = \sfrac{1}{\sqrt[3]{N}}$ yields $\Exop\left[\setE^i_{\text{exp}}(\{\vecpi^j_T\}_{j=1}^N) \right] \leq \mathcal{O}(\frac{ \lambda^{-\sfrac{1}{2}} \sqrt{N}}{\sqrt{T}} + \frac{\lambda^{-\sfrac{1}{2}}}{\sqrt[3]{N}})$.
\end{corollary}

Even though TRPA-Full solves the regularized (hence strongly monotone) problem, compared to the $\mathcal{O}(\sfrac{1}{T})$ rate in classical strongly monotone VI \citep{kotsalis2022simple} or strongly convex optimization \citep{rakhlin2011making},
our worse $\mathcal{O}(\sfrac{1}{\sqrt{T}})$ time dependence is due to independent learning.
Intuitively, additional time is required to ensure the policies of independent learners are sufficiently close when ``collectively'' evaluating $\vecF$.
The additional dependence of the time-vanishing term on $\sqrt{N}$ is also a result of this fact.
Furthermore, when learning itself is performed by $N$ agents, we note that the bias as a function of $N$ decreases with $\mathcal{O}(\sfrac{1}{\sqrt[4]{N}})$ (or $\mathcal{O}(\sfrac{1}{\sqrt[3]{N}})$ for strongly monotone problems), and not with $\mathcal{O}(\sfrac{1}{\sqrt{N}})$ as Theorem~\ref{theorem:mfg_ne} might suggest.
We leave the question of whether this gap can be improved and whether knowledge of $N$ is required in Corollary~\ref{corollary:expert}, as future work.



\section{Convergence in the Bandit Feedback Case}\label{sec:bandit_feedback_results}

We now move on to the more challenging and realistic bandit feedback case, where agents can only observe the payoffs of the actions they have chosen.
Once again, we analyze the IL setting (or in bandits terminology, the ``no communications'' setting) where agents can not interact or coordinate with each other.
One of the main challenges of bandit feedback with IL in our setting is that it is difficult for each agent to identify itself (i.e., assign itself a unique number between $1,\ldots,N$) so that exploration of action payoffs can be performed in turns.
For instance, in MMAB algorithms, this is typically achieved using variants of the so-called musical chairs algorithm \citep{lugosi2022multiplayer}, which is not available in our formulation.
Instead, we adopt a \emph{probabilistic} exploration scheme where each agent probabilistically decides it is its turn to explore payoffs while the rest of the agents induce the required empirical population distribution on which $\vecF$ should be evaluated.

Our algorithm, which we call TRPA-Bandit, is straightforward and relies on exploration occurring over epochs, where policies are updated once in between epochs using the estimate of action payoffs constructed during the exploration phase.
We use the subscript $h$ to index epochs, which consist of $T_h$ repeated plays indexed by $(h,t)$ for $t=1,\ldots,T_h$.
While we formally presented TRPA-Bandit (Algorithm~\ref{alg:bandit}), the procedure informally is as follows for each agent, fixing an exploration parameter $\varepsilon \in (0,1)$ and an agent $i\in\setN$:
\begin{enumerate}
    \item At each epoch $h$, for $T_h > 0$ time steps, repeat the following:
    \begin{enumerate}
        \item With probability $\varepsilon$, sample uniformly an action $a^i_{h,t}$, observe the payoff $r^i_{h,t}$, and keep the importance sampling estimate $\widehat{\vecr}^i_h \leftarrow K r_{h,t}^i \vece_{a^i_{h,t}}$.
        \item Otherwise (with probability $1-\varepsilon$), sample action according to current policy $\vecpi^i_h$.
    \end{enumerate}
    \item Update the policy using TRPA, $\vecpi^i_{h+1} = \Pi_{\Delta_\setA} ( (1 - \tau \eta_h) \vecpi^i_h + \eta_h \widehat{\vecr}^i_h )$.
    If the agent did not explore this epoch, use $\widehat{\vecr}^i_h = \veczero$.
\end{enumerate}
Intuitively, the probabilistic sampling scheme allows some agents to build a low-variance estimate of $\vecF$, while others simply sample actions with their current policy in order to induce the empirical population distribution at which $\vecF$ should be evaluated.

\begin{algorithm}
    \caption{TRPA-Bandit: IL with bandit feedback algorithm for each agent $i\in\setN$.}\label{alg:bandit}
    \begin{algorithmic}
    \Require Number of actions $K$, regularization $\tau > 0$, exploration probability $\varepsilon > 0$, number of epochs $H$, epoch lengths $\{T_h\}_h$, learning rates $\{\eta_h\}_h$
    \State $\vecpi^i_0 \leftarrow \frac{1}{K} \vecone$
    \For{$h = 0, \ldots, H-1$}
    \State $\widehat{\vecr}^i_h \leftarrow \veczero$ %
    \For{$t = 1, \ldots, T_h$} \Comment{Exploration for $T_h$ rounds before policy update,}
    \State Sample Bernoulli r.v. $X_{h,t}^i \sim \operatorname{Ber}(\varepsilon)$.
    \If{$X_{h,t}^i=1$}
        \State \text{Play action $a^i_{h,t} \sim \operatorname{Unif}(\setA)$ uniformly at random}.
        \Comment{Explore with prob. $\varepsilon$,}
        \State \text{Observe payoff $r^i_{h,t}$}, set  $\widehat{\vecr}^i_h \leftarrow K r^i_{h,t}\vece_{a^i_{h,t}}$.
    \ElsIf{$X_{h,t}^i=0$}
        \State \text{Play action with current policy $a^i_{h,t}\sim \vecpi^i_h$}.
        \Comment{Else, play the current policy.}
    \EndIf
    \EndFor
    \State $\vecpi^i_{h+1} = \Pi_{\Delta_\setA} ( (1 - \tau \eta_h) \vecpi^i_h + \eta_h \widehat{\vecr}^i_h )$
    \Comment{After each epoch, update policy.}
    \EndFor
    \State Return $\vecpi^i_H$
    \end{algorithmic}
    \end{algorithm}



Similar to the full feedback setting, we introduce useful notation used throughout this chapter.
We define the sigma algebra $\mathcal{F}_{h} := \mathcal{F}(\{ \vecpi_{h'}^i \}_{h'=0, \ldots, h}^{i=1, \ldots, N})$.
Adapting the notation from Section~\ref{sec:game_initial_formulation} to the case with multiple epochs, we use
\begin{align*}
\widehat{\vecmu}_{h, t} := \frac{1}{N} \sum_{i=1}^N \vece_{a_{h, t}^i}, \qquad
    \vecr^i_{h, t} := \vecF(\widehat{\vecmu}_{h, t}) + \vecn_{h, t}^i,
\end{align*}
where the updated time indices $h, t$ simply refer to the $t$-th round of play in epoch $h$, and $a_{h, t}^i$ is the action played by player $i$ at epoch $h$, round $t$.
Under the dynamics of Algorithm~\ref{alg:bandit}, we define the following random variables to assist our proofs:
\begin{align*}
    \bar{\vecmu}_h &:= \frac{1}{N} \sum_{i=1}^N \vecpi_h^i, 
    \qquad
    e_t^i := \|\vecpi^i_h - \bar{\vecmu}_h \|_2^2, 
    \qquad
    u_h^i := \Exop\left[\| \vecpi_h^i - \vecpi^* \|_2^2\right],
\end{align*}
which correspond to the mean policy at epoch $h$, the policy deviation of agent $i$ from the mean at epoch $h$ and the $\ell_2$ distance from the MF-NE.
Note that since policies are updated only in between epochs, the above quantities are indexed by epochs $h$ rather than rounds $h, t$.

Our analysis follows the ideas in the case of expert feedback, the main difference being randomization due to the exploration probabilities and the errors being analyzed per epoch rather than per round.
Similar to the full feedback setting, we will proceed in several steps expressed as intermediate lemmas:
(1) we bound the added bias and variance due to the importance sampling strategy,
(2) we obtain a non-linear recursion for the expectation of the terms $e_t^i$ and possible sampling bias, 
(3) we bound the expected differences of each agent's action probabilities $e_t^i$, showing the deviation of the policies of the agents goes to zero in expectation, 
(4) we solve the recursion to obtain the convergence rate.

The next result, Lemma~\ref{lemma:exploration_bias_trpa_bandit}, provides an answer to the first step.
We show that despite the probabilistic exploration step, the estimates $\widehat{\vecr}_h^i$ in Algorithm~\ref{alg:bandit} have low bias and variance.

\begin{lemma}[Exploration bias]\label{lemma:exploration_bias_trpa_bandit}
Under the dynamics of TRPA-Bandit, it holds almost surely for each epoch $h \geq 0$ that
\begin{align*}
    \| \Exop[\widehat{\vecr}_h^i | 
 \mathcal{F}_h ] - \vecF(\varepsilon \frac{1}{K}\vecone + (1-\varepsilon) \bar{\vecmu}_h) \|_2 \leq K^{\sfrac{3}{2}} \sqrt{1 + \sigma^2} \exp\left\{ -\varepsilon T_{h}\right\} + \frac{2L}{N} + \frac{2L}{\sqrt{N}}.
\end{align*}
\end{lemma}

The full proof has been postponed to Appendix~\ref{sec:proof_lemma_bandit_exploration_bias}.
In summary, the proof strategy is to decompose and analyze the bias due to the possibility of no exploration round happening (the term $K^{\sfrac{3}{2}} \sqrt{1 + \sigma^2} \exp\left\{ -\varepsilon T_{h}\right\}$),  the impact of the exploring agent on payoffs (the term $\frac{2L}{N}$), and bias due to the finitely many agents, similar to Theorem~\ref{theorem:mfg_ne} (the term $\frac{2L}{\sqrt{N}}$).
The additional bias due to probabilistic exploration originates from the possibility that no exploration occurs in a given epoch: the probability of this event can be bounded by $\exp\left\{ -\varepsilon T_{h}\right\}$.

Lemma~\ref{lemma:exploration_bias_trpa_bandit} shows that even if the players do not have full feedback, they can obtain low-bias, low-variance estimates of $\vecF(\varepsilon \frac{1}{K}\vecone + (1-\varepsilon) \bar{\vecmu}_h) \approx \vecF(\bar{\vecmu}_h)$ when $\varepsilon$ is small.
It guarantees that even if the agents do not explore each epoch, in expectation the probabilistic exploration scheme yields a low bias if the epoch lengths $T_h$ are logarithmically large: hence, full feedback can be simulated by paying a logarithmic cost.
Therefore, in our epoched exploration scheme, the bias in ``querying'' the payoff operator $\vecF$ due to an exploring population can be controlled by tuning $\varepsilon$ and $T_h$.

We next state the error recursion in Lemma~\ref{lemma:bandit_main_recurrence}, which uses the result of Lemma~\ref{lemma:exploration_bias_trpa_bandit} to construct the main recurrence for the bandit feedback case.

\begin{lemma}[Main recurrence for TRPA-Bandit]\label{lemma:bandit_main_recurrence}
Under TRPA-Bandit dynamics, it holds for any $i \in \{1, \ldots, N \}$ and each epoch $h\geq 0$ that
\begin{align*}
    \Exop\left[\| \vecpi_{h+1}^i - \vecpi^* \|_2^2\right] \leq & 4 \eta_h^2 K^3(1 + \sigma^2) + 8\eta_h^2(L+\tau)^2 + 8 K^{\sfrac{3}{2}}\eta_h \sqrt{1+\sigma^2}  \exp\{-\varepsilon T_h\} \\
        &+128\eta_h\lambda^{-1} L^2 N^{-1} + 16\eta_h\lambda^{-1}L^2\varepsilon^2 + 2\eta_h\lambda^{-1} L^2 \Exop\left[e_h^i\right] \\
        &+\left(1 - 2 \eta_h(\sfrac{\lambda}{2} + \tau)\right) \Exop\left[\| \vecpi_h^i - \vecpi^* \|_2^2\right],
\end{align*}
for strongly monotone $\lambda >0$ payoffs, and
\begin{align*}
    \Exop\left[\| \vecpi_{h+1}^i - \vecpi^* \|_2^2\right] \leq &  4\eta_h^2 K^3(\sigma^2 + 1) + 8 \eta_h^2 (L+\tau)^2 + 8 K^{\sfrac{3}{2}} \eta_h \sqrt{1+\sigma^2} \exp\{-\varepsilon T_h\} \\
    &+64\tau^{-1} \eta_h \delta^{-1}L^2 N^{-1}+8\tau^{-1} \eta_h \delta^{-1}L^2 \varepsilon^{2} + \tau^{-1} \eta_h\delta^{-1}L^2 \Exop\left[e_h^i\right] \\  
        &+ \left(1 - 2 \tau \eta_h (1-\delta)\right) \Exop\left[\| \vecpi_h^i - \vecpi^* \|_2^2\right],
\end{align*}
for monotone payoffs for arbitrary $\delta \in (0,1)$.
\end{lemma}
Once again, the full proof has been postponed to Appendix~\ref{sec:proof_lemma_bandit_recurrence}.
The proof of Lemma~\ref{lemma:bandit_main_recurrence} follows a similar path as in the recurrence in the full feedback case (Lemma~\ref{lemma:full_error_recurrence}), with the exception that $\vecr_{h,t}^i$ has been replaced by the importance sampling estimator $\widehat{\vecr}^i_h$.
In the decomposition due to Inequality~\eqref{ineq:decomp_abc_full_recur}, the analysis of term (a) remains the same, whereas the terms (b), (c) must be further analyzed using Lemma~\ref{lemma:exploration_bias_trpa_bandit} to account for deviations between $\vecr_{h,t}^i$ and $\widehat{\vecr}^i_h$, as well as the $\varepsilon$ fraction of the population now exploring each round.

The recurrence in Lemma~\ref{lemma:bandit_main_recurrence} is similar in form to the full feedback case (Lemma~\ref{lemma:full_error_recurrence}), apart from the term $8 K^{\sfrac{3}{2}}\eta_h \sqrt{1+\sigma^2}  \exp\{-\varepsilon T_h\}$ due to the exploration scheme.
However, keeping the exploration epoch lengths $T_h$ logarithmically large can make this term small.
Furthermore, once again the recursion produces a dependence on expected mean policy deviations, $\Exop[e_h^i]$.
Hence, the expected policy deviation $\Exop[e_h^i]$ at epoch $h$ must be bounded once again at a rate $\mathcal{O}(\sfrac{1}{h})$ in order to obtain a non-vacuous upper bound on exploitability.
As in the full feedback case, we employ regularization to ensure $\Exop[e_h^i]$ is small.
The next lemma presents our upper bound.

\begin{lemma}[Policy deviation under TRPA-Bandit]\label{lemma:bandit_pol_deviation}
Under TRPA-Bandit dynamics, with learning rates $\eta_h:=\frac{\tau^{-1}}{h+2}$, arbitrary exploration rate $\varepsilon > 0$ and epoch lengths $T_h := \lceil \varepsilon^{-1} \log(h+2) \rceil$
it holds for any $i,j \in \{1, \ldots, N \}, i\neq j$ and each epoch $h\geq 0$ that
\begin{align*}
    \Exop[e_h^i] \leq \frac{24\tau^{-2} K^3 (\sigma^2 + 2) + 48\tau^{-2} + 24}{h+1} + \frac{16\tau^{-2} L ^ 2}{N^2}.
\end{align*}
\end{lemma}
The proof of Lemma~\ref{lemma:bandit_pol_deviation} follows similar ideas to Lemma~\ref{lemma:policy_variations_bound_trpa_full}, while accounting for (a) the increased variance due to importance sampling, and (b) potential further deviation between agent policies due to the $\mathcal{O}(\sfrac{1}{N})$ impact of exploration on $\widehat{\vecmu}_{h,t}$.
In particular, an additional source of policy deviation occurs when an agent does not explore in a given epoch, in which case the payoff estimator is uninformative ($\widehat{\vecr}_h^i = \veczero$) causing additional policy deviation.
The full proof has been postponed to Appendix~\ref{sec:proof_lemma_bandit_pol_dev}.

In the case of TRPA-Bandit, due to the additional variance of probabilistic exploration, the policy deviations between agents might be larger: compare the upper bounds of Lemma~\ref{lemma:bandit_pol_deviation} and Lemma~\ref{lemma:policy_variations_bound_trpa_full}.
In particular, the upper bound of Lemma~\ref{lemma:bandit_pol_deviation} contains a non-vanishing term unlike Lemma~\ref{lemma:policy_variations_bound_trpa_full}.
Nevertheless, they can still be controlled of order $\mathcal{O}(\sfrac{1}{(h+1)} + \sfrac{1}{N^2})$, where the additional $\mathcal{O}(\sfrac{1}{N^2})$ term compared to TRPA-Full vanishes very quickly when $N$ is large.

With these intermediate lemmas established, we state and prove the main convergence result for TRPA-Bandit  Theorem~\ref{theorem:bandit_short}, the main result of this work.
We provide asymptotic rates for brevity, although the proof of the theorem provides explicit bounds.

\begin{theorem}[Convergence, bandit feedback]\label{theorem:bandit_short}
Assume $\vecF$ is Lipschitz, monotone.
Assume $N$ agents run TRPA-Bandit (Algorithm~\ref{alg:bandit}) for $T$ time steps with regularization $\tau>0$ and exploration parameter $\varepsilon > 0$, and agents return policies $\{\vecpi^i\}_i$ after executing Algorithm~\ref{alg:bandit}.
Then, for any agent $i \in \setN$ that $\Exop\left[ \setE^i_{\text{exp}}( \{\vecpi^j\}_{j=1}^N ) \right] \leq \widetilde{\mathcal{O}} (\frac{\tau^{-2}\varepsilon^{-\sfrac{1}{2}}}{\sqrt{T}} + \tau^{-1}\varepsilon + \tau + \frac{\tau^{-1}}{\sqrt{N}} + \frac{\tau^{-\sfrac{3}{2}} }{N} )$.
If $\vecF$ is $\lambda$-strongly monotone, then $\Exop\left[ \setE^i_{\text{exp}}( \{\vecpi^j\}_{j=1}^N ) \right] \leq \widetilde{\mathcal{O}} (\frac{\tau^{-\sfrac{3}{2}} \lambda^{-\sfrac{1}{2}} \varepsilon^{-\sfrac{1}{2}}}{\sqrt{T}} + \tau^{-\sfrac{1}{2}} \lambda^{-\sfrac{1}{2}} \varepsilon + \tau + \frac{\tau^{-\sfrac{1}{2}} \lambda^{-\sfrac{1}{2}}}{\sqrt{N}} + \frac{\tau^{-1} \lambda^{-\sfrac{1}{2}} }{N} )$.
\end{theorem}
 The proof follows from a straightforward combination of Lemmas~\ref{lemma:exploration_bias_trpa_bandit}, \ref{lemma:bandit_main_recurrence}, and \ref{lemma:bandit_pol_deviation} as in the full feedback case.
The exact bounds and proof are in the appendix, Section~\ref{sec:bandit_theorem_full}.


Once again, the values of $\tau$ and exploration probability $\varepsilon$ can be chosen to incorporate tolerable exploitability.
In the case where the number of participants $N$ in the game is known, the following corollary indicates the asymptotically optimal choices for the hyperparameters.

\begin{corollary}[Optimal $\varepsilon, \tau$, bandit feedback]\label{corollary:bandit}
Assume the conditions of Theorem~\ref{theorem:bandit_short} for $N$ agents running TRPA-Bandit.
For monotone $\vecF$, choosing $\tau = \sfrac{1}{\sqrt[4]{N}}$ and $\varepsilon = \sfrac{1}{\sqrt{N}}$ yields
$\Exop\left[\setE^i_{\text{exp}}(\{\vecpi^j\}_{j=1}^N) \right] \leq \widetilde{\mathcal{O}}(\frac{N^{\sfrac{3}{4}}}{\sqrt{T}} + \frac{1}{\sqrt[4]{N}})$ for any $i$.
For strongly monotone $\vecF$, choosing $\tau = \sfrac{1}{\sqrt[3]{N}}$ and $\varepsilon = \sfrac{1}{\sqrt{N}}$ yields $\Exop\left[\setE^i_{\text{exp}}(\{\vecpi^j\}_{j=1}^N) \right] \leq \widetilde{\mathcal{O}}(\frac{N^{\sfrac{3}{4}} \lambda^{-\sfrac{1}{2}}}{\sqrt{T}} + \frac{\lambda^{-\sfrac{1}{2}}}{\sqrt[3]{N}})$.
\end{corollary}

The dependence of $N$ of the sample complexity in the bandit case is worse compared to the full feedback setting as expected: intuitively the agents must take turns to estimate the payoffs of each action in bandit feedback.
Furthermore, while our problem framework is different and a direct comparison is difficult in terms of bounds, we point out that classical MMAB results such as \citep{lugosi2022multiplayer} have a linear dependence on $N$, while in our case the dependence on $N$ scales with $N^{\sfrac{3}{4}}$.
We emphasize that the time-dependence is sublinear in terms of $N$, up to the non-vanishing finite population bias.
As in the full feedback case, the non-vanishing finite population bias in the bandit feedback case scales with $\mathcal{O}(\sfrac{1}{\sqrt[4]{N}})$ or $\mathcal{O}(\sfrac{1}{\sqrt[3]{N}})$, rather than $\mathcal{O}(\sfrac{1}{\sqrt{N}})$ which would match Theorem~\ref{theorem:mfg_ne}.
Note that the dependence of the bias on $N$ varies in various mean-field game results \citep{saldi2019approximate}, but asymptotically is known to converge to zero as $N\rightarrow\infty$, as our explicit finite-agent bound also demonstrates.

Finally, we note that as expected the algorithm for bandit feedback has a worse dependency on the number of actions.
This is as expected due to the fact that (i) the importance sampling estimator increases variance on payoff estimators by a factor of $K$, and (2) in other words, a factor of $\mathcal{O}(K)$ is introduced in order to explore all actions.


\section{Experiments: Planning outperforms Heuristics}
\label{sec:experiment}

We begin our empirical demonstrations by showcasing the effectiveness of our planning framework on both synthetic and real datasets. We focus on the simplest planning algorithm, 1-step lookaheads (Algorithm~\ref{alg:complete}), and show that even basic planning can hold great promise. 
We illustrate our framework using two uncertainty quantification modules---GPs and 
\ensembles/ \ensembleplus. 

Throughout this section, we focus on evaluating the mean squared error of 
a regression model $\model$,  and develop adaptive policies that minimize uncertainty on $g(f)$ defined in~\eqref{eqn:l2-g-f}.
When GPs provide a valid model of uncertainty, 
our experiments show that our planning framework significantly outperforms other baselines. 
We further demonstrate that our conceptual framework extends to deep learning-based uncertainty quantification methods such as  \ensembleplus while highlighting computational challenges that need to be resolved in order to scale our ideas. 
For simplicity, we assume a naive predictor, i.e., $\psi(\cdot) \equiv 0$. However, we emphasize that this problem is just as complex as if we were using a sophisticated model $\psi(.)$. The performance gap between the algorithms 
primarily depends
on the level  of uncertainty in our prior beliefs.

To evaluate the performance of our algorithm, we benchmark it against several baselines. 
%Active learning baselines use an acquisition function $\ac$ to select points that have the highest   function value: $X\opt_t \in \argmax_{X \in \xpoolj{t}} \ac({X})$ at every step $t$. These methods may also need an UQ module, which we simply use the same UQ module as in our algorithm, and it  outputs $V(X)$ that measures the the uncertainty of each point $X \in \xpoolj{t}$.
Our first set of baselines are from active learning~\citep{AggarwalKoGuHaPh14}:
\\ % \noindent\textbf{Active Learning Heuristics:} 
\textbf{(1)} 
\textsf{Uncertainty Sampling (Static):}  In this approach, we query the samples for which the model is least certain about. Specifically, we estimate the variance of the latent output $f(X)$ for each $X \in \xpool$ using the UQ module and select the top-$K$ points with the highest uncertainty. \\
\textbf{(2)} \textsf{Uncertainty Sampling (Sequential):} This is a greedy heuristic that sequentially selects the points with the highest uncertainty within a batch, while updating the posterior beliefs using pseudo labels from the current posterior state. Unlike \textsf{Uncertainty Sampling (Static)}, this method takes into account the information gained from each point within batch, and hence tries to diversify the selected points within a batch. 

 
We also compare our approach to the  \textbf{(3)} \textsf{Random Sampling}, which selects each batch uniformly at random from the pool. Additionally, we compare solving the planning problem using  \textsf{REINFORCE}-based policy gradients with   $\mathsf{Smoothed\text{-}Autodiff}$ policy gradients.\footnote{Our code repository is available at
  \url{https://github.com/namkoong-lab/adaptive-labeling}.}
%Detailed experimental setups are provided in Section \ref{sec:details-experiments}.

%We repeat all experiments with 10 random seeds.




\begin{figure}[t]
\centering
\begin{minipage}[b]{0.49\textwidth}
\centering
\includegraphics[width=\textwidth, height=5cm]{figures/original_scale/Var_of_l_2_loss.pdf}
\caption{(Synthetic data) Variance of mean squared loss evaluated through the posterior belief $\mu_t$ at each horizon $t$. This is the objective that policy gradient methods like \textsf{REINFORCE} and $\ouralgo$ optimizes. 1-step lookaheads are surprisingly effective even in long horizons.}
\label{fig:var-l2-sim}
\end{minipage}
\hfill
\begin{minipage}[b]{0.49\textwidth}
\centering \includegraphics[width=\textwidth, height=5cm]{figures/original_scale/Error_of_estimated_model_l_2_loss.pdf}
\caption{(Synthetic data) Error between MSE calculated based on collected data $\mc{D}^{0:T}$ vs. population oracle MSE over $\mc{D}_{\rm eval} \sim P_X$. Reducing uncertainty over posteriors directly leads to better OOD evaluations. 1-step lookaheads significantly outperform active learning heuristics in small horizons.}
\label{fig:mean-l2-sim}
\end{minipage}
%\caption{Simulated data for GPs}
%\label{fig:both_plots}
\end{figure}

\subsection{Planning with Gaussian processes}
\label{sec:experiment-plan-GP}
We now briefly describe the data generation process for the GP experiments,  deferring a more detailed discussion of the dataset generation to Section~\ref{sec:details-experiments}. 
We use both the synthetic data and the real data to test our methodology.
For the \emph{simulated data},  we construct a setting where the general population is distributed across \emph{51 non-overlapping clusters} while the initial labeled data $\dtrain$ just comes from one cluster. In contrast, both $\dpool \defeq (\xpool,\ypool),\deval \defeq (\xeval,\yeval)$ are generated   from all the clusters. 
We begin with a low-dimensional scenario, generating a one-dimensional regression setting using a GP. %Gaussian Process (GP).
Although the data-generating process is not known to the algorithms,  we assume that the GP hyperparameters are known to all the algorithms
to ensure fair comparisons. This can be viewed as a setting where our prior is well-specified, allowing us to isolate the effects
of different policy optimization approaches
 without any concerns about the misspecified priors. We select $10$ batches, each of size $K=5$ across $T = 10$ time horizons.

To examine the robustness of our method against the distributional assumptions made  in the simulated case, we then move to a real dataset where the correct prior is not known. We simulate selection bias from the eICU dataset~\citep{PollardJoRaCeMaBa18}, which contains real-world patient data with in-hospital mortality outcomes. 
We conduct a $k$-means clustering to generate 51 clusters and then select data from those clusters. We view this to be a credible replication of practice, as severe distribution shifts are common due to selection bias in clinical labels.  To convert the binary mortality labels into a regression setting, we train a  random forest classifier and fit a GP on predicted scores, which serves as the UQ module for all the algorithms. As before, the task is to select 10 batches, each consisting of 5 samples, across 10 time horizons.

 In Figures~\ref{fig:var-l2-sim} and~\ref{fig:mean-l2-sim}, we present results for the simulated data. 
Figure~\ref{fig:var-l2-sim} shows the variance of $\ell_2$ loss, and Figure~\ref{fig:mean-l2-sim} presents the error in the estimated $\ell_2$ loss using $\mu_t$ (relative to true $\ell_2$ loss, that is unknown to the algorithm). 
As we can see from these plots, our method one-step lookahead  gives substantial improvements  over active learning baselines and random sampling. In addition,
compared to the one-step lookahead planning approach using \textsf{REINFORCE}-based policy gradients, 
we observe that $\mathsf{Smoothed\text{-}Autodiff}$-based policy gradients provide significantly more robust performance over all horizons.

In Figures~\ref{fig:var-l2-real}~and~\ref{fig:mean-l2-real}, we observe similar findings on the eICU data. We see that planning policies (\textsf{REINFORCE} and $\mathsf{Smoothed\text{-}Autodiff}$) consistently outperform other heuristics by a large margin.  Active learning baselines perform poorly in these small-horizon batched problems and can sometimes be even worse than the random search baselines.  Overall, our results show the importance of careful planning in adaptive labeling for reliable model evaluation. 

We offer some intuition as to why one-step lookahead planning may outperform other heuristic algorithms. 
 First,  \textsf{Uncertainty sampling (Static)} while myopically selects the
 top-$K$ inputs with the highest uncertainty, it fails to consider 
the overlap in information content among the ``best” instances; see \citep{AggarwalKoGuHaPh14} for more details. 
In other words,  it might acquire points from the same region with high uncertainty while failing to induce diversity among the batch.
Although \textsf{Uncertainty Sampling (Sequential)} somewhat addresses the issue of information overlap, a significant drawback of 
this algorithm
is the disconnect between the objective we aim to optimize and the algorithm. For example, it might sample from a region with high uncertainty but very low density. 

\begin{figure}[t]
\centering
\begin{minipage}[b]{0.48\textwidth}
\centering
\includegraphics[width=\textwidth, height=5cm]{figures/original_scale/Var_of_l_2_loss_real.pdf}
\caption{(Real-world eICU data) Variance of mean squared loss evaluated through the posterior belief $\mu_t$ at each horizon $t$. Even 1-step lookaheads are extremely effective planners, and auto-differentiation-based pathwise policy gradients provide a reliable optimization algorithm based on low-variance gradient estimates.}
\label{fig:var-l2-real}
\end{minipage}
\hfill
\begin{minipage}[b]{0.48\textwidth}
\centering \includegraphics[width=\textwidth, height=5cm]{figures/original_scale/Error_of_estimated_model_l_2_loss_real.pdf}
\caption{(Real-world eICU data) Error between MSE calculated based on collected data $\mc{D}^{0:T}$ vs. population oracle MSE over $\mc{D}_{\rm eval} \sim P_X$. Reducing uncertainty over posteriors directly leads to better OOD evaluations. Our method significantly outperforms active learning-based heuristics, and random sampling.}
\label{fig:mean-l2-real}
\end{minipage}
%\caption{Real data for GPs}
\end{figure}
 
%\vspace{-1.5cm}
% \begin{wrapfigure}{r}{.32\columnwidth}
%   \vspace{-.5cm} 
%   \centering
% \includegraphics[scale=.29]{figures/Var of l2l_2 loss.pdf}
%   \vspace{-0.2cm}
%   \caption{Results of GP}
% \label{fig:var-l2-gp}
%   \vspace{-0.1cm}
% \end{wrapfigure}


% Attempts have been made  in the past to address these  drawbacks heuristically  (see \citep{AggarwalKoGuHaPh14}). We give a unified computational framework while approaching the problem in a more principled manner and solving it more optimally.




\subsection{Planning with  neural network-based uncertainty quantification methods ($\ensembleplus$)}


We now provide a proof-of-concept that shows the generalizability of our conceptual framework  to the deep learning-based UQ modules, specifically focusing on $\ensembleplus$ due to their previously observed superior performance~\citep{OsbandWenAsDwIbLuRo23}. Recall that implementing our framework with deep learning-based UQ modules  requires us to retrain the model across multiple possible random actions $\bm{a}(\theta)$ sampled from the current policy $\pi_\theta$.
This requires significant computational resources, in sharp contrast to the GPs where the posteriors are in closed form and can be readily updated and differentiated. 

Due to the computational constraints, we test $\ensembleplus$ on a toy setting to demonstrate the generalizability of our framework. We consider a setting where the general population consists of four clusters, while the initial labeled data only comes from one cluster. Again we generate data using GPs.  The task is to select a batch of 2 points in one horizon. We detail the $\ensembleplus$ architecture in Section \ref{sec:details-experiments}, and we assume prior uncertainty to be large (depends on the scaling of the prior generating functions). 
The results are summarized in the Table~\ref{tab:UQ_ensemble}.

% \begin{table}[H]
% \vspace{-10pt}
% \caption{Performance under \ensembleplus as UQ module}
%     \centering
%     \begin{tabular}{|m{3cm}|m{2.5cm}|m{2cm}|} 
%     \hline
%       Algorithm   & Variance of $\loss_2$ loss estimate & Error of $\loss_2$ loss estimate  \\ \hline Random Sampling 
%          & $1710.9 \pm 1352.1$ & $8.67\pm6.62$ 
%       \\ \hline \ouralgo & $1.30 \pm 0.68$ & $0.91\pm0.25$ \\ \hline
%     \end{tabular}
%     \label{tab:UQ_ensemble}
%     %\vspace{-10pt}
% \end{table}




\begin{table}[h]
\vspace{-10pt}
\caption{Performance under \ensembleplus as the UQ module}
\centering
\begin{tabular}{|l|l|l|}
\hline
Algorithm   & Variance of $\loss_2$ loss estimate & Error of $\loss_2$ loss estimate  \\
\hline
\textsf{Random sampling} & 7129.8 $\pm$ 1027.0 & 136.2 $\pm$ 8.28 \\ \hline
\textsf{Uncertainty sampling (Static)} & 10852 $\pm$ 0.0 & 162.156 $\pm$ 0.0 \\ \hline
\textsf{Uncertainty sampling (Sequential)} & 8585.5 $\pm$ 898.9 & 144 $\pm$ 6.93 \\ \hline
\textsf{REINFORCE} & 1697.1 $\pm$ 0.0 & 45.27 $\pm$ 0.0 \\ \hline
\ouralgo & 1697.1 $\pm$ 0.0 & 45.27 $\pm$ 0.0 \\ \hline
\end{tabular}
%\caption{Comparison of different algorithms based on variance   and   error in $\ell_2$ loss estimation with Ensemble $+$ as the UQ module. Our results demonstrate that {\ouralgo} and REINFORCE outperformthe other active learning based heuristics, confirming the benefits of our MDP formulation for the adaptive labeling problem, as also demonstrated in Section 4.\\
%\footnotesize{Experimental details: We use Gaussian Processes as our data generating process, GP parameters are the same as in Section D.3.  The task is to select a batch of 2 points along one horizon.The marginal distribution $p_X$ has 4 \textit{non-overlapping} clusters. Initial data comes from one cluster, while pool and evaluation points comes from all the clusters. We have $20$ initial labeled data points, $10$ pool points, and $252$ evaluation points.  Training procedures are similar to the one in Section D.3.} }
\label{tab:UQ_ensemble}
\end{table}



% We faced  issues in scaling up these experiments which will be our focus in the future. 





% \begin{itemize}
%     \item Posteriors should be consistent. Two dimensions: even with less training,  
%     \item the inference should be  fast enough
% \end{itemize}


% Potential research directions for uncertainty quantification

% In this section we consider a simple setting We consider a simpler setting and 


% For synthetic dataset generation, we use ...... For real datasets, we use ...... We compare our methodolgy to several baselines ()    This Section is structured as follows:
% \begin{itemize}
%     \item \textbf{GPs, square loss objective} (Section \ref{}): 
%     %the broad aim of the experiments  in this section is to isolate the performance of our methodology without any concerns for the inefficiencies induced due to a mis-specified prior or imperfect posterior inference. To accomplish this we generate synthetic datasets using GPs (detailed later). We use the well specified prior (GPs - with same hyperparameter setting) as our UQ module.   
%      As GPs provide differentaible posterior inference - any errors induced due to imperfect posterior updates are also isolated. We note that under this setting
%      \item In Section\ref{} we demonstrate why our methodology performs better than other baselines - by devising various synthetic experiments ()
%     \item  \textbf{UQ Benchmarking }(Section \ref{}): Before diving into the experiments using $\ensembleplus$ and ENNs,  we showcase our benchmarking experiments in Section \ref{}. We use real datasets We observe that ENNs perform better
%      \item \textbf{Ensemble $+$}, objective: recall, accuracy
%     \item \textbf{ENN}, objective: recall, accuracy
% \end{itemize}




% In Section {}, we test 
% \subsection{Experimental details}

% \begin{itemize}
%     \item UQ methodologies - GPs, ENNs
%     \item Objectives - Recall,  ATE
%     \item Datasets - ATE-synthetic datasets, Recall-synthetic, real datasets
%     \item Baselines - 
%     \begin{itemize}
%         \item Random sampling
%         \item Active learning - Uncertainty based sampling - In regression setting almost all of the 
%         \item Myopic greedy - Greedy Batch based sampling
%         \item Policy Gradient
%     \end{itemize}
    
% \end{itemize}

% \subsection{Experiments}
%     \begin{itemize}
%     \item GPs with square loss
%     \item Benchmarking ENN
%         \item ENNs with ATE
%         \item ENNs with Recall
%     \end{itemize}

% \subsection{Benefits over other algorithms - intuition and experiments}

%Active learning - Myopic greedy / Don't rely on the objective rather some entropy version.


%%% Local Variables:
%%% mode: latex
%%% TeX-master: "main"
%%% End:


This work identifies signal collapse as a critical bottleneck in one-shot neural network pruning. Performance loss in pruned networks is due to \textbf{signal collapse} in addition to the removal of critical parameters. We propose \textbf{REFLOW} (\textbf{Re}storing \textbf{F}low of \textbf{Low}-variance signals), a simple yet effective method that mitigates signal collapse without computationally expensive weight updates. By focusing on signal preservation, REFLOW highlights the importance of mitigating signal collapse in sparse networks and enables magnitude pruning to match or surpass state-of-the-art one-shot pruning methods such as CHITA, CBS, and WF.

REFLOW consistently achieves state-of-the-art accuracy across diverse architectures, restoring ResNeXt-101 from under 4.1\% to 78.9\% top-1 accuracy at 80\% sparsity on ImageNet. Its lightweight design makes it a practical solution for both research and deployment, delivering high-quality sparse models without the overhead of traditional approaches. These findings challenge the traditional emphasis on weight selection strategies and underscore the critical role of signal propagation for achieving high-quality sparse networks in the context of one-shot pruning.




\section*{Acknowledgements}

We thank the editor and reviewers for their detailed comments and insightful feedback. This project was supported by Swiss National Science Foundation (SNSF) under the framework of NCCR Automation and SNSF Starting Grant.
S. Cayci's research was funded under the Excellence Strategy of the Federal Government and the L{\"a}nder.

\bibliography{sn-bibliography}

\appendix

\section{Table for Notation}
\begin{longtable}{ p{.23\textwidth}  p{.77\textwidth} } 
\underline{General notation:} & \\
$\Delta_\setX$ & probability simplex on discrete set $\setX$ \\ 
$[M]$ & $:= \{1, \ldots, M\}$, for any $M\in\mathbb{N}_{>0}$ \\
$\Delta_{\setX, N}$ & $:= \{ \vecu = \{u_i\}_i \in\Delta_\setX | N u_i \in \mathbb{N}_{\geq 0} \}$, for discrete set $\setX$, $N\in\mathbb{N}_{>0}$ \\
$\mathbb{M}^{D_1,D_2}$ & $D_1 \times D_2$ matrices \\
$\mathbb{S}_{++}^D$ & positive definite $D \times D$ matrices \\
$\Pi_K$ & projection onto convex, compact set $K\subset \mathbb{R}^H$\\
$\vece_a$ & $\in \mathbb{R}^\setA$, standard unit vector with coordinate $a$ set to 1 \\
 &  \\ 
\underline{For SMFGs:} & \\
$\vecF$ & payoff function, $\vecF: \Delta_\setA \rightarrow [0,1]^K$  \\
$\lambda$ & strong monotonicity modulus of $\vecF$ \\  
$L$ & Lipschitz modulus of $\vecF$ \\
$\tau$ & Tikhonov regularization parameter \\ 
$\vecpi^*$ & unique solution of $\tau$ regularized \eqref{eq:mfg_rvi_statement} \\
$K$ & number of actions \\ 
$N$ & number of players \\  
$\setN$ & $:= \{1, \ldots, N \}$, the set of players \\ 
$\setA$ & the set of actions, $|\setA| = K$. \\ 
$\sigma^2$ & upper bound of the standard deviation of received payoff \\
$\widehat{\vecmu}$ & $:= \frac{1}{N} \sum_{i=1}^N \vece_{a^i}$, when actions $\{a^i\}_{i=1}^N$ clear in context \\
$V^i(\vecpi^1, \ldots, \vecpi^N)$ & expected payoff of player $i$ under strategy profile $(\vecpi^1, \ldots, \vecpi^N)$ \\
$\setE^i_{\text{exp}}(\{\vecpi^j\}_{j=1}^N)$ & $:= \max_{\vecpi'\in \Delta_\setA} V^i(\vecpi', \vecpi^{-i}) - V^i(\vecpi^1, \ldots, \vecpi^N)$, exploitability \\
$\widehat{\vecmu}(\{a^i\}_{i=1}^N)$ & $:= \frac{1}{N} \sum_{i=1}^N \vece_{a^i}$, the action distribution induced by particular $\{a^i\}_{i=1}^N \in \setA^N$ \\
 &  \\ 
\underline{For full feedback:} & \\
$t$ & round of play \\
$a^i_t$ & $\in\setA$ action take by player $i$ at round $t$ \\
$\widehat{\vecmu}_t $ & $:= \frac{1}{N} \sum_{i=1}^N \vece_{a_t^i}$, empirical distribution over actions $\setA$ on round $t$ \\ 
$\vecr^i_t$ & $:= \vecF(\widehat{\vecmu}_t) + \vecn_t^i$, payoff vector observed by player $i$ \\ 
$\eta_t$ & learning rate \\
$ \bar{\vecmu}_t $ & $ := \frac{1}{N} \sum_{i=1}^N \vecpi_t^i,$  mean policy at round $t$\\ 
$e_t^i$ & $:= \|\vecpi^i_t - \bar{\vecmu}_t \|_2^2 $, deviation of policy of player $i$ \\
$u_t^i $ & $:= \Exop\left[\| \vecpi_t^i - \vecpi^* \|_2^2\right]$ \\
 &  \\ 
\underline{For bandit feedback:} & \\
$h$ & epoch of play \\
$t$ & round of play in epoch \\
$a^i_{h,t}$ & $\in\setA$ action take by player $i$ at round $t$ of epoch $h$ \\
$T_h$ & number of rounds in epoch $h$ \\
$\varepsilon$ & exploration probability \\
$\widehat{\vecmu}_{h,t} $ & $:= \frac{1}{N} \sum_{i=1}^N \vece_{a_{h, t}^i}$, empirical distribution over actions $\setA$ on round $t$ of epoch $h$ \\ 
$\vecr^i_{h,t}$ & $:= \vecF(\widehat{\vecmu}_{h,t}) + \vecn_{h,t}^i$, (unobserved) payoff vector by player $i$ \\ 
$r^i_{h,t}$ & $:= \vecr^i_{h,t} (a^i_{h,t})$, payoff observed by player $i$ at round $t$ of epoch $h$ \\
$X_{h,t}^i$ & $\sim \operatorname{Ber}(\varepsilon)$ indicator variable, $1$ if $i$ explores at round $t$ of epoch $h$ \\
$\widehat{\vecr}^i_h $ & $:= K r^i_{h,t}\vece_{a^i_{h,t}}$ if $X_{h,t}^i=1$, importance sampling estimate of player $i$ \\
$\eta_h$ & learning rate \\
$ \bar{\vecmu}_h $ & $ := \frac{1}{N} \sum_{i=1}^N \vecpi_h^i,$  mean policy at epoch $h$\\ 
$e_h^i$ & $:= \|\vecpi^i_h - \bar{\vecmu}_h \|_2^2 $, deviation of policy of player $i$ \\
$u_h^i $ & $:= \Exop\left[\| \vecpi_h^i - \vecpi^* \|_2^2\right]$ \\
\end{longtable}


\section{A Detailed Comparison to the Setting in \cite{gummadi2013mean}}\label{sec:detailed_comparison}

Since specific keywords seem to correspond to the works on mean-field approximations with bandits, we provide a greater discussion of our setting and the results by \citet{gummadi2013mean}.
In general, our settings and models are very different, hence almost none of the results between our work and Gummadi et al. are transferable to the other.
Our problem formulation, analysis, and results are fundamentally different from their setting due to the following points.

\textbf{Stationary equilibrium vs Nash equilibrium.}
The most critical difference between the two works is the solution concepts.
Our setting is competitive, as a natural extension, the solution concept is that of a Nash equilibrium where each agent has no incentive to change their policy.
On the other hand, the setting of Gummadi et al. need not be competitive or collaborative and this distinction is not significant for their framework, their goal is to characterize convergence of the population to a stationary distribution.
Their main results show that if a particular policy map $\sigma: \mathbb{Z}_{\geq 0}^{2 n} \rightarrow \Delta_\setA$ is prescribed on agents, the population distribution will converge to a steady state.
The equilibrium concept of \cite{gummadi2013mean} is not \emph{Nash}, rather stationarity.

\textbf{Optimality considerations.}
As a consequence of analyzing stationarity, the results by \citep{gummadi2013mean} do not analyze or aim to characterize optimality.
In their analysis, a fixed map $\sigma: \mathbb{Z}_{\geq 0}^{2 n} \rightarrow \Delta_\setA$ is assumed to be the policy/strategy of a continuum of (i.e., infinitely many) agents, which maps observed loss/win counts (from Bernoulli distributed arm rewards) to arm probabilities.
The stationary distribution in general obtained from $\sigma$ in \cite{gummadi2013mean} does not have optimality properties, for instance, a fixed agent will can have arbitrary large exploitability.
The main goal of \cite{gummadi2013mean} is to prove the convergence of the population distribution to a steady state behaviour.

\textbf{Algorithms.}
As a consequence of the previous points, Gummadi et al. abstract away any algorithmic considerations to the fixed map $\sigma$ and the particular algorithms employed by agents do not directly have significance in terms of their theoretical conclusions.
Since we analyze optimality in our setting, we require a specific algorithm to be employed (TRPA and Algorithm~\ref{alg:bandit}).

\textbf{Independent learning.}
In our setting, the notion of learning and independent learning become significant since we are aiming to obtain an approximate NE.
Hence, our theoretical results bound the expected exploitability (Theorems~\ref{theorem:expert_short}, \ref{theorem:bandit_short}) in terms of number of samples.
In the work of \cite{gummadi2013mean}, the main aim is convergence to a steady state rather than learning.

\textbf{Population regeneration.}
Finally, to be able to obtain a contractive mapping yielding a population stationary distribution/steady state, \cite{gummadi2013mean} assume that the population regenerates at a constant rate $\beta$, implying agents are constantly being replaced by oblivious agents that have not observed the game dynamics.
This smooths the dynamics by introducing a forgetting mechanism to game participants.
Our results on the other hand are closer to the traditional bandits/independent learning setting.
For instance, this would correspond to non-vanishing exploitability scaling with $\mathcal{O}(\beta)$ in our system as agents constantly ``forget'' what they learned.

\textbf{Other model differences.}
In our setting, we assume general noisy rewards while in \cite{gummadi2013mean}, the rewards are Bernoulli random variables with success probability dependent on the population.

\section{Formalizing Learning Algorithms}\label{section:alg_formalization}

In this section, we formalize the concept of an independent learning algorithm in the full feedback and bandit feedback setting.
In general, we formalize the notion of an algorithm as a map $A_t^i: \mathcal{H}_{t}^i \rightarrow \Delta_\setA$ that maps the set of past observations of agent $i$ at time $t$ to action selection probabilities.
The definition of the set $\mathcal{H}_{t}^i$ varies between the feedback models.

\begin{definition}[Learning algorithm with full feedback]
\label{definition:alg_expert}
An independent learning algorithm with full information $\mathbf{A} = \{ A_t^i\}_{i,t}$ is a sequence of mappings for each player with
\begin{align*}
    &A_t^i : \Delta_\setA^{t-1} \times \setA^{t-1} \times [0,1]^{(t-1) \times K} \rightarrow \Delta_\setA, \text{ for all $t > 0$}, \\
    &A_0^i \in \Delta_\setA,
\end{align*}
that maps past $t-1$ observations from previous rounds to a mixed strategy on actions $\setA$ at time $t > 0$ for each agent $i$.
\end{definition}

Naturally, we are interested in algorithms that yield a good approximation of the NE in expectation.
More explicitly, we will be interested in designing algorithms that converge to policy profiles with low expected explotability.

\begin{definition}[Rational learning algorithm with full feedback]
Let $\mathbf{A}$ be an algorithm with full feedback as defined in Definition~\ref{definition:alg_expert}.
We call $\mathbf{A}$ $\delta$-rational if it holds that for all $i$, the induced mixed strategies $\vecpi_t^i$ under $\vecpi_0^i = A_0^i, \vecpi_t^i = A_t^i(\vecpi^i_0, \ldots, \vecpi_{t-1}^i, a_0^i, \ldots, a_{t-1}^i, \vecr_{0}^i, \ldots, \vecr_{t-1}^i)$ satisfy
\begin{align*}
    \lim_{t\rightarrow \infty} \Exop[ \setE^i_{\text{exp}}(\{\vecpi_t^j \}_{j=1}^N)] \leq \delta, \text{ for all } i \in \setN.
\end{align*}
\end{definition}

Note that while not specified in the definition above, we will also be interested in the rate of convergence of the exploitability term for a rational algorithm.
Since we are solving the SMFG at the finite-agent regime, we will be interested in $\delta$-rational algorithms that have $\delta \rightarrow 0$ as $N\rightarrow\infty$, that is, the non-vanishing bias should scale inversely with the number of agents.

Finally, we also formalize the concepts of a learning algorithm and $\delta$-rationality in the bandit setting.

\begin{definition}[Algorithm with bandit feedback]
\label{definition:alg_bandit}
An algorithm with bandit feedback $\mathbf{A} = \{ A_t^i\}_{i,t}$ is a sequence of mappings for each player with
\begin{align*}
    &A_t^i : \Delta_\setA^{t-1} \times \setA^{t-1} \times [0,1]^{(t-1) } \rightarrow \Delta_\setA, \text{ for all $t > 0$}, \\
    &A_0^i \in \Delta_\setA,
\end{align*}
that maps past $t-1$ observations from previous rounds at all times $t > 0$ (only including the payoffs of the \emph{played} actions) to a probability distribution on actions $\setA$.
\end{definition}

\begin{definition}[Rational algorithm with bandit feedback]
Let $\mathbf{A}$ be an algorithm with bandit feedback as defined in Definition~\ref{definition:alg_bandit}.
We call $\mathbf{A}$ $\delta$-rational if it holds that for all $i$, the induced (random) mixed strategies $\vecpi_t^i$ under $\vecpi_0^i = A_0^i, \vecpi_t^i = A_t^i(\vecpi^i_0, \ldots, \vecpi_{t-1}^i, a_0^i, \ldots, a_{t-1}^i, r_{0}^i, \ldots, r_{t-1}^i)$ satisfy
\begin{align*}
    \lim_{t\rightarrow \infty} \Exop[ \setE^i_{\text{exp}}(\{\vecpi_t^j \}_{j=1}^N)] \leq \delta, \text{ for all } i \in \setN.
\end{align*}
\end{definition}


\section{Basic Inequalities}\label{app:basic_inequalities}

In our proofs, we will need to repeatedly bound certain recurrences and sums.
In this section, we present useful inequalities to this end.

\begin{lemma}[Harmonic partial sum bound]\label{lemma:harmonic}
For any integers $s,\bar{s}$, constant $c\in\mathbb{R}$ such that $1 \leq \bar{s} < s$, $p \neq -1$, and $a \geq 0$, it holds that
\begin{align*}
   \log (s + a) - \log (\bar{s} + a ) + \frac{1}{s + a} &\leq \sum_{n = \bar{s}}^{s} \frac{1}{n + a} \leq \frac{1}{\bar{s} + a} + \log ( s + a) - \log (\bar{s} + a), \\
   \frac{(s + a)^{p+1}}{p+1} - \frac{(\bar{s} + a)^{p+1}}{(p+1)} + (\bar{s} + a)^p &\leq \sum_{n = \bar{s}}^{s} (n + a)^p \leq \frac{(s + a)^{p+1}}{p+1} - \frac{(\bar{s} + a)^{p+1}}{p+1} + (s + a)^p, \text{ if } p \geq 0 \\
   \frac{(s + a)^{p+1}}{p+1} - \frac{(\bar{s} + a)^{p+1}}{p+1} + (s + a)^p &\leq \sum_{n = \bar{s}}^{s} (n + a)^p \leq \frac{(s + a)^{p+1}}{p+1} - \frac{(\bar{s} + a)^{p+1}}{p+1} + (\bar{s} + a)^p, \text{ if } p \leq 0
\end{align*}
\end{lemma}
\begin{proof}
Let $f_1:\mathbb{R}_{\geq 0} \rightarrow \mathbb{R}_{\geq 0}$ be a non-decreasing positive function and $f_2:\mathbb{R}_{\geq 0} \rightarrow \mathbb{R}_{\geq 0}$ be a non-increasing positive function.
Then it holds that
\begin{align*}
    \int_{x = \bar{s}}^s f_1(x) dx + f_1(\bar{s}) \leq \sum_{n=\bar{s}}^s f_1(n) \leq \int_{x = \bar{s}}^s f_1(x) dx  + f_1(s), \\
    \int_{x = \bar{s}}^s f_2(x) dx + f_2(s) \leq \sum_{n=\bar{s}}^s f_2(n) \leq \int_{x = \bar{s}}^s f_2(x) dx + f_2(\bar{s}).
\end{align*}
The result follows from a simple integral bound with $\int \frac{1}{x} dx = \log x$ and $\int x^p dx = \frac{x^{p+1}}{p+1}$.
\end{proof}

We state a certain recurrence inequality that appears several times in our analysis as a lemma, in order to shorten some proofs.

\begin{lemma}[General error recurrence]\label{lemma:general_recurrence}
Let $c_0 \geq 0, c_1 \geq 0, \gamma > 1, a \geq 0$ be arbitrary constants such that $a \geq \gamma$.
Furthermore, let $\{u_t\}_{t=0}^\infty$ be a sequence of non-negative numbers such that for all $t \geq 0$, it holds that
\begin{align*}
    u_{t+1} \leq \frac{c_0}{t+a} + \frac{c_1}{(t+a)^2} + \left( 1 - \frac{\gamma}{t+a}\right) u_t.
\end{align*}
Then, for all values of $t\geq 0$, it holds that:
\begin{align*}
u_{t+1} \leq &\frac{u_0 a ^ \gamma + c_1 (a^{\gamma-2} + 1) (1+a^{-1})^\gamma + c_0 (1+a^{-1})^\gamma a^{\gamma-1}}{\left(t+a\right)^\gamma}  \\
    &\quad + \frac{c_0 + c_1 (1+a^{-1})^\gamma (\gamma - 1)^{-1}}{t+a} + \frac{c_1}{(t+a)^2} + \gamma^{-1}(1+a^{-1})^\gamma c_0 
\end{align*}
    
\end{lemma}
\begin{proof}
We note that inductively, we have
\begin{align*}
u_{t+1} \leq &\frac{c_0}{t+a} + \frac{c_1}{(t+a)^2} + u_0 \prod_{s=0}^{t} \left( 1 - \frac{\gamma}{s+a} \right) 
     + \sum_{s=0}^{t-1} \left( \frac{c_0}{s+a} + \frac{c_1}{(s+a)^2} \right) \prod_{s'=s+1}^{t} \left(1 - \frac{\gamma}{s'+a} \right).
\end{align*}
Using the inequality $1 + x \leq e^{x}$, we obtain
\begin{align*}
u_{t+1} \leq &\frac{c_0}{t+a} + \frac{c_1}{(t+a)^2} + u_0 \prod_{s=0}^{t} \exp\left\{- \frac{\gamma}{s+a}\right\} + \sum_{s=0}^{t-1} \left( \frac{c_0}{s+a} + \frac{c_1}{(s+a)^2} \right) \prod_{s'=s+1}^{t} \exp\left\{ - \frac{\gamma}{s'+a}\right\} \\
\leq &\frac{c_0}{t+a} + \frac{c_1}{(t+a)^2} + u_0 \exp\left\{- \sum_{s=0}^{t}\frac{\gamma}{s+a}\right\} + \sum_{s=0}^{t-1} \left( \frac{c_0}{s+a} + \frac{c_1}{(s+a)^2} \right)  \exp\left\{ -\sum_{s'=s+1}^{t} \frac{\gamma}{s'+a}\right\}. 
\end{align*}
Here, using Lemma~\ref{lemma:harmonic}, since $a - 1 > 0$, it holds that
\begin{align*}
  \sum_{s=0}^{t}\frac{\gamma}{s+a} &\geq \sum_{s = 1}^{t+1} \frac{\gamma}{s+(a - 1)} \geq \gamma \log(t+a) - \gamma \log a = \log \left(\frac{(t+1)^\gamma}{a^\gamma}\right) \\
  \sum_{s'=s+1}^{t} \frac{\gamma}{s'+a} &\geq \gamma \log(t + a) - \gamma \log(s + a + 1) \geq \log \left\{ \frac{(t+a)^\gamma}{(s+a+1)^\gamma} \right\},
\end{align*}
therefore $u_{t+1}$ can be further upper bounded by
\begin{align*}
u_{t+1} \leq &\frac{c_0}{t+a} + \frac{c_1}{(t+a)^2} + u_0 \frac{a^\gamma}{ \left(t+a\right)^\gamma} + \sum_{s=0}^{t-1} \left( \frac{c_0}{s+a}  + \frac{c_1}{(s+a)^2} \right) \left(\frac{s + a + 1}{t+a}\right)^\gamma \\
\leq &\frac{c_0}{t+a} + \frac{c_1}{(t+a)^2} + \frac{u_0 a ^ \gamma}{\left(t+a\right)^\gamma} + (t+a)^{-\gamma} \sum_{s=0}^{t-1} \left( \frac{c_0}{s+a} + \frac{c_1}{(s+a)^2} \right) \left(s+a+1\right)^\gamma \\
\leq &\frac{c_0}{t+a} + \frac{c_1}{(t+a)^2} + \frac{u_0 a ^ \gamma}{\left(t+a\right)^\gamma} + (t+a)^{-\gamma} (1 + a^{-1})^\gamma \sum_{s=0}^{t-1} \left( c_0 \left(s+a\right)^{\gamma-1} + c_1\left(s+a\right)^{\gamma-2} \right) .
\end{align*}
The last term can be bound with the corresponding integral (see Lemma~\ref{lemma:harmonic}), yielding (since $\gamma-1 > 0$):
\begin{align*}
    \sum_{s=0}^{t-1}(s+a)^{\gamma - 1} &\leq \frac{(t+a)^\gamma}{\gamma} + a ^ {\gamma - 1}.
\end{align*}
For the term $\sum_{s=0}^{t-1}(s+1)^{\gamma - 2}$, we analyzing the two cases $1 < \gamma \leq 2$ and $\gamma > 2$ using Lemma~\ref{lemma:harmonic} we obtain
\begin{align*}
    \sum_{s=0}^{t-1}(s+a)^{\gamma - 2} &\leq \frac{(t+a)^{\gamma-1}}{\gamma - 1} + a^{\gamma - 2} + 1.
\end{align*}
The two inequalities combined yield the stated bound,
\begin{align*}
    u_{t+1} \leq & \frac{c_0}{t+a} + \frac{c_1}{(t+a)^2} + \frac{u_0 a ^ \gamma}{\left(t+a\right)^\gamma} + (t+a)^{-\gamma} (1 + a^{-1})^\gamma \sum_{s=0}^{t-1} \left( c_0(s+a)^{\gamma-1} + c_1(s+a)^{\gamma - 2} \right) \\
    \leq & \frac{c_0}{t+a} + \frac{c_1}{(t+a)^2} + \frac{u_0 a ^ \gamma}{\left(t+a\right)^\gamma} + \gamma^{-1}(1+a^{-1})^\gamma c_0 + \frac{c_0 (1+a^{-1})^\gamma a^{\gamma-1}}{(t+a)^\gamma} \\ 
    &\quad +\frac{c_1 (1+a^{-1})^\gamma}{(t+a)(\gamma - 1)} + \frac{c_1 (a^{\gamma-2} + 1) (1+a^{-1})^\gamma}{(t+a)^\gamma} \\
    \leq & \frac{u_0 a ^ \gamma + c_1 (a^{\gamma-2} + 1) (1+a^{-1})^\gamma + c_0 (1+a^{-1})^\gamma a^{\gamma-1}}{\left(t+a\right)^\gamma}  \\
    &\quad + \frac{c_0 + c_1 (1+a^{-1})^\gamma (\gamma - 1)^{-1}}{t+a} + \frac{c_1}{(t+a)^2} + \gamma^{-1}(1+a^{-1})^\gamma c_0 .
\end{align*}
\end{proof}


\section{Proofs of Technical Results}


\subsection{Proof of Lemma~\ref{lemma:technical_bound_1N}}\label{app:technical_lemma}
\begin{proof}
We introduce an auxiliary random variable $\bar{a}^i$ which is independent from other players' actions $\{ a^j\}_{j=1}^N$ and has distribution $\vecpi^i$, that is, we introduce the random variable $\bar{a}^i$ as an identically distributed independent copy of $a^i$.
Then, it holds by simple computation that
\begin{align*}
V^i(\vecpi^1, \ldots, \vecpi^N) = & \Exop \Big[ \vecF\Big( \frac{1}{N} \sum_{j=1}^N \vece_{a^j}, a^i\Big) \Big]\\
= &\Exop \Big[\vecF\Big(\frac{1}{N}\Big(\sum_{j=1, j\neq i}^N \vece_{a^j} + \vece_{\bar{a}^i}\Big), a^i \Big)\Big] \\ 
    &+ \Exop \Big[ \vecF\Big(\frac{1}{N}\sum_{j=1}^N \vece_{a^j}, a^i\Big) -\vecF\Big( \frac{1}{N}\Big(\sum_{j=1, j\neq i}^N \vece_{a^j} + \vece_{\bar{a}^i}\Big), a^i\Big)\Big].
\end{align*}
For the first term above, we observe that
\begin{align*}
   \Exop \Big[\vecF\Big( \frac{1}{N}\Big(\sum_{j=1, j\neq i}^N \vece_{a^j} + \vece_{\bar{a}^i}\Big), a^i\Big)\Big] = &\Exop \Big[ \Exop \Big[\vecF\Big( \frac{1}{N}\Big(\sum_{j=1, j\neq i}^N \vece_{a^j} + \vece_{\bar{a}^i}\Big), a^i\Big) \Big| a^i\Big]\Big] \\
   = &\Exop \Big[ \Exop \Big[\vece_{a^i}^\top \vecF\Big(\frac{1}{N}\Big(\sum_{j=1, j\neq i}^N \vece_{a^j} + \vece_{\bar{a}^i}\Big)\Big) \Big| a^i\Big]\Big] \\
   = &\Exop [ \vece_{a^i}^\top] \Exop \Big[ \vecF\Big(\frac{1}{N}\Big(\sum_{j=1, j\neq i}^N \vece_{a^j} + \vece_{\bar{a}^i}\Big)\Big)\Big] \\
   = & \vecpi^{i, \top} \Exop [\vecF(\widehat{\vecmu})],
\end{align*}
since $\{a^j \}_{j=1}^N$ and $(\bar{a}^i,\veca^{-i})$ are identically distributed by the independence of both $\bar{a}^i$ and $a^i$ from $\veca^{-i}$, and since
$(\bar{a}^i,\veca^{-i})$ is independent of $a^i$.

The second term above can be bounded using
\begin{align*}
    \Big|\vecF\Big(&\frac{1}{N}\sum_{j=1}^N \vece_{a^j}, a^i\Big) -\vecF\Big( \frac{1}{N}\Big(\sum_{j=1, j\neq i}^N \vece_{a^j} + \vece_{\bar{a}^i}\Big), a^i\Big)\Big| \\
    = & \Big|\vece_{a^i}^\top\vecF\Big(\frac{1}{N}\sum_{j=1}^N \vece_{a^j}\Big) -\vece_{a^i}^\top \vecF\Big( \frac{1}{N}\Big(\sum_{j=1, j\neq i}^N \vece_{a^j} + \vece_{\bar{a}^i}\Big)\Big)\Big| \\
    \leq & \left\| \vece_{a^i} \right\|_2 \Big\| \vecF\Big(\frac{1}{N}\sum_{j=1}^N \vece_{a^j}\Big) -\vecF\Big( \frac{1}{N}\Big(\sum_{j=1, j\neq i}^N \vece_{a^j} + \vece_{\bar{a}^i}\Big)\Big) \Big\|_2 \\
    \leq & \frac{L}{N} \left\| \vece_{a^i} - \vece_{\bar{a}^i}\right\|_2 \leq \frac{L\sqrt{2}}{N}.
\end{align*}
The last line follows from the fact that $\vecF$ is $L$-Lipschitz.
\end{proof}

\subsection{Proof of Lemma~\ref{lemma:phi_lipschitz}}\label{sec:extended_proof_lema_lipschitz_phi}

We first prove the fact that $V^i$ is Lipschitz.
In the proof, we denote the empirical action distribution induced by actions $\{a^j\}_{j=1}^N\in\setA^N$ by $\widehat{\vecmu}(\{a^j\}_{j=1}^N) \in \Delta_\setA$ to simplify notation.
We analyze two cases of Lipschitz moduli stated in the lemma separately. In the first case, we compute the Lipschitz moduli $L_{i,i}$ for any $i\in\setN$ as follows:
\begin{align*}
    |V^i&(\vecpi, \vecpi^{-i}) - V^i (\vecpi', \vecpi^{-i})| \\
    \leq & \Big| \Exop \left[ \vecF\left(\widehat{\vecmu}(\{a^k\}_{k=1}^N), a^i\right) \middle|
a^j \sim \vecpi^j, \forall j \neq i, a^i \sim \vecpi
\right] \\
    &\quad - \Exop \left[ \vecF\left(\widehat{\vecmu}(\{a^k\}_{k=1}^N),a^i\right) \middle|
a^j \sim \vecpi^j, \forall j \neq i, a^i \sim \vecpi' \right] \Big| \\
\leq & \big| \sum_{\substack{a^j \in \setA \\ j\neq i}}  \sum_{a^i \in \setA}\vecpi(a^i) \vecF(\widehat{\vecmu}(\{a^k\}_{k=1}^N), a^i)\prod_{j\neq i} \vecpi^j(a^j) - \sum_{\substack{a^j \in \setA \\ j\neq i}} \sum_{a^i \in \setA} \vecpi'(a^i) \vecF(\widehat{\vecmu}(\{a^k\}_{k=1}^N), a^i) \prod_{j\neq i} \vecpi^j(a^j) 
 \Big| \\
\leq & \sum_{\substack{a^j \in \setA \\ j\neq i}}  \Big|\sum_{a^i \in \setA}\left[ \vecpi(a^i) - \vecpi'(a^i)\right] \vecF(\widehat{\vecmu}(\{a^k\}_{k=1}^N), a^i) \Big| \prod_{j\neq i} \vecpi^j(a^j) \\
\leq & \sum_{\substack{a^j \in \setA \\ j\neq i}}  \left\| \vecpi - \vecpi'\right\|_2 \sqrt{\sum_{a^i \in \setA} \vecF(\widehat{\vecmu}(\{a^k\}_{k=1}^N), a^i)^2} \prod_{j\neq i} \vecpi^j(a^j) 
\leq \left\| \vecpi - \vecpi'\right\|_2 \sqrt{K}.
\end{align*}
where we use Jensen's inequality in the penultimate step and the Cauchy-Schwartz inequality in the final step.

In the second case, for any $k \neq i$, it holds that
\begin{align*}
|V^i&(\vecpi, \vecpi^{-k}) - V^i (\vecpi', \vecpi^{-k})| \\
    \leq & \Exop \left[ \vecF\left(\widehat{\vecmu}(\{a^l\}_{l=1}^N), a^i\right) \middle|
a^j \sim \vecpi^j, \forall j \neq k, a^k \sim \vecpi
\right] - \Exop \left[ \vecF\left(\widehat{\vecmu}(\{a^l\}_{l=1}^N),a^i\right) \middle|
a^j \sim \vecpi^j, \forall j \neq k, a^k \sim \vecpi' \right] \\
    \leq & \Big| \sum_{\substack{a^j \in \setA \\ j\neq k}} \prod_{j\neq k} \vecpi^j(a^j) \sum_{a^k \in \setA}\vecpi(a^k) \vecF(\widehat{\vecmu}(\{a^l\}_{l=1}^N), a^i) - \sum_{\substack{a^j \in \setA \\ j\neq k}} \prod_{j\neq k} \vecpi^j(a^j) \sum_{a^k \in \setA} \vecpi'(a^k) \vecF(\widehat{\vecmu}(\{a^l\}_{l=1}^N), a^i) \Big| \\
\leq & \sum_{\substack{a^j \in \setA \\ j\neq k}} \prod_{j\neq k} \vecpi^j(a^j) \Big|\sum_{a^k \in \setA}\left[ \vecpi(a^k) - \vecpi'(a^k)\right] \vecF(\widehat{\vecmu}(\{a^l\}_{l=1}^N), a^i) \Big|
\end{align*}
In this case, note that for any $a, a' \in \setA, \veca\in \setA^K$, we have $|\vecF(\widehat{\vecmu}(a, \veca^{-k}), a^i) - \vecF(\widehat{\vecmu}(a', \veca^{-k}), a^i)| \leq \|\vecF(\widehat{\vecmu}(a, \veca^{-k})) - \vecF(\widehat{\vecmu}(a', \veca^{-k})) \|_2 \leq \sfrac{L\sqrt{2}}{N}$.
That is, the set $\{\vecF(\widehat{\vecmu}(a, \veca^{-k}), a^i) \, : a\in\setA \} \subset \mathbb{R}$ has diameter $\sfrac{2L}{\sqrt{N}}$, and
there exists a constant $v_{k} \in \mathbb{R}$ such that $|\vecF(\widehat{\vecmu}(a, \veca^{-k}), a^i) - v_{k}| \leq \sfrac{2L\sqrt{2}}{N}$ for all $a$.
Then,
\begin{align*}
|V^i&(\vecpi, \vecpi^{-k}) - V^i (\vecpi', \vecpi^{-k})| \\
    \leq & \sum_{\substack{a^j \in \setA \\ j\neq k}} \prod_{j\neq k} \vecpi^j(a^j) \Big|\sum_{a^k \in \setA}\left[ \vecpi(a^k) - \vecpi'(a^k)\right] \left[ \vecF(\widehat{\vecmu}(\{a^l\}_{l=1}^N), a^i) - v_{k}\right] \Big| \\
    \leq & \sum_{\substack{a^j \in \setA \\ j\neq k}} \prod_{j\neq k} \vecpi^j(a^j) \|\vecpi - \vecpi'\|_2 \sqrt{\sum_{a^k\in \setA} \left[ \vecF(\widehat{\vecmu}(\{a^l\}_{l=1}^N), a^i) - v_{k}\right]^2 } \\
    \leq & \sum_{\substack{a^j \in \setA \\ j\neq k}} \prod_{j\neq k} \vecpi^j(a^j) \|\vecpi - \vecpi'\|_2 \frac{2L\sqrt{2K}}{N} \leq \|\vecpi - \vecpi'\|_2 \frac{2L \sqrt{2K}}{N}.
\end{align*}

We establish the Lipschitz continuity of $\setE^i_{\text{exp}}$ with the above inequalities.
\begin{align*}
    |\setE^i_{\text{exp}} &(\vecpi, \vecpi^{-i}) - \setE^i_{\text{exp}}(\vecpi', \vecpi^{-i}) | \\
    \leq & \left|\max_{\overline{\vecpi}\in \Delta_\setA} V^i(\overline{\vecpi}, \vecpi^{-i}) - V^i(\vecpi, \vecpi^{-i}) - \max_{\overline{\vecpi}\in \Delta_\setA} V^i(\overline{\vecpi}, \vecpi^{-i}) + V^i(\vecpi', \vecpi^{-i})\right| \\
    \leq & \left| V^i(\vecpi', \vecpi^{-i}) - V^i(\vecpi, \vecpi^{-i})\right| 
    \leq \sqrt{K} \|\vecpi - \vecpi'\|_2.
\end{align*}
Similarly, for $k\neq i$,
\begin{align*}
|\setE^i_{\text{exp}} &(\vecpi, \vecpi^{-k}) - \setE^i_{\text{exp}}(\vecpi', \vecpi^{-k}) | \\
    \leq & \left|\max_{\overline{\vecpi}\in \Delta_\setA} [ V^i( \vecpi, \overline{\vecpi},\vecpi^{-k,i}) - V^i(\vecpi, \vecpi^{-k}) ] - \max_{\overline{\vecpi}\in \Delta_\setA} [V^i( \vecpi', \overline{\vecpi},\vecpi^{-k,i}) - V^i(\vecpi', \vecpi^{-k}) ]\right| \\
    \leq & \max_{\overline{\vecpi}\in \Delta_\setA} \left| V^i( \vecpi, \overline{\vecpi},\vecpi^{-k,i}) - V^i( \vecpi',\overline{\vecpi}, \vecpi^{-k,i}) \right| + \left| V^i(\vecpi, \vecpi^{-k}) - V^i(\vecpi', \vecpi^{-k})\right| \\
    \leq & \frac{4L\sqrt{2K}}{N} \| \vecpi - \vecpi'\|_2.
\end{align*}

\subsection{Extended Definitions for Bandit Feedback}\label{sec:bandit_extended_defs}

When analyzing the TRPA-Bandit dynamics, several random variables and events will be reused to assist analysis.
For brevity, we define them here.
We define the following random variables:
\begin{align*}
    \mathbbm{1}_{h,t}^i := &\mathbbm{1}\{\text{player $i$ explores at round $t$ of epoch $h$}\} = X_{h,t}^i \\
    E_{h,t}^i := &\{\mathbbm{1}_{h,t}^i = 1 \} \\
    \mathbbm{1}_{h}^i := &\mathbbm{1}\{\text{player $i$ explores at least once during epoch $h$}\} = \max_{t = 1, \ldots, T_h} \mathbbm{1}_{h,t}^i  \\
    E_{h}^i := &\{\mathbbm{1}_{h}^i = 1 \} = \bigcup_{t=1}^{T_h} E_{h,t}^i\\
    a_{h}^i := &\text{Last explored action in epoch $h$ by agent $i$,} \\
        &\text{and $a_0$ if no exploration occurred}. \\
    s_h^i := &\text{Timestep when exploration last occurred in epoch $h$ by agent $i$, }\\
        &\text{and $0$ if no exploration occurred.} \in \{1, \ldots, T_h \}
\end{align*}

\subsection{Proof of Lemma~\ref{lemma:exploration_bias_trpa_bandit}}\label{sec:proof_lemma_bandit_exploration_bias}

We will reuse the definitions of Section~\ref{sec:bandit_extended_defs}.
By the definition of the events and the probabilistic exploration scheme, we have $\widehat{\vecr}_h^i = K \left( \vecF(\widehat{\vecmu}_{s_h^{i},h}, a_h^{i}) + \vecn_{s_h^{i},h}^{i}(a_h^{i}) \right) \mathbbm{1}_h^i$.
Firstly, by the law of total expectations and the fact that $E_h^i$ are independent of $\mathcal{F}_h$,
\begin{align*}
    \Exop\left[\widehat{\vecr}_h^i \middle| \mathcal{F}_h\right] = &\Exop\left[\widehat{\vecr}_h^i \middle|E_{h}^i, \mathcal{F}_h\right] \Prob(E_{h}^i) + \Exop\left[\widehat{\vecr}_h^i\middle|\overline{E_{h}^{i}}, \mathcal{F}_h\right] \Prob(\overline{E_{h}^{i}}) \\
    = & \Exop\left[\widehat{\vecr}_h^i \middle|E_{h}^i, \mathcal{F}_h\right] - \underbrace{\Exop\left[\widehat{\vecr}_h^i\middle|E_{h}^{i}, \mathcal{F}_h\right] \Prob(\overline{E_{h}^{i}}) + \Exop\left[\widehat{\vecr}_h^i\middle|\overline{E_{h}^{i}}, \mathcal{F}_h\right] \Prob(\overline{E_{h}^{i}})}_{:=\vecb_h^i} \\
    = & \Exop\left[\widehat{\vecr}_h^i \middle|E_{h}^i, \mathcal{F}_h\right] + \vecb_h^i
\end{align*}
for $\vecb_h^i$ quantifying a bias induced due to the probability of no exploration.
We have that
\begin{align*}
    \| \vecb_h^i \|_2 \leq K\sqrt{K} \sqrt{1 + \sigma^2} \exp\left\{ -\varepsilon T_{h}\right\}
\end{align*}
since $\Exop\left[\widehat{\vecr}_h^i\middle|\overline{E_{h}^{i}}, \mathcal{F}_h\right] = 0$ and exploration probabilities are determined by independent random Bernoulli variables hence $\Prob(\overline{E_{h}^{i}}) = \left( 1 - \varepsilon \right)^{T_h} \leq \exp\left\{ -\varepsilon T_{h}\right\}.$
To further characterize the bias, we introduce a coupling argument.
Define independent random variables $\bar{a}_h^j \sim \varepsilon \vecpi_\text{unif} + (1-\varepsilon) \vecpi_h^j$ for all $j\in\setN$ and $\bar{a}^i_{h, exp} \sim \vecpi_\text{unif}$.
By the definition, it holds that (where $\widehat{\vecmu}(\{ \bar{a}_h ^ j \}_j) \in \Delta_\setA$ denotes the empirical distribution induced by actions $\{ \bar{a}_h ^ j \}_j$):
\begin{align*}
    \|\Exop \left[\widehat{\vecr}_h^i \middle|E_{h}^i, \mathcal{F}_h\right] - \Exop\left[ \vecF(\widehat{\vecmu}(\{ \bar{a}_h ^ j \}_j)) \middle| \mathcal{F}_h \right]\|_2 
    \leq & \|\Exop \left[\vecF(\widehat{\vecmu}(\bar{a}_{h,exp}^i, \bar{a}_h ^ {-i})) \middle| \mathcal{F}_h \right] - \Exop\left[ \vecF(\widehat{\vecmu}(\{ \bar{a}_h ^ j \}_j))  \middle| \mathcal{F}_h\right]\|_2 \\
    \leq & \Exop \left[\|\vecF(\widehat{\vecmu}(\bar{a}_{h,exp}^i, \bar{a}_h ^ {-i})) - \vecF(\widehat{\vecmu}(\{ \bar{a}_h ^ j \}_j)) \|_2 \middle| \mathcal{F}_h\right] \leq \frac{2 L }{N}.
\end{align*}
With the additional bound $\Exop \left[\|\vecF(\varepsilon \vecpi_\text{unif} + (1-\varepsilon) \bar{\vecmu}_h) - \vecF(\widehat{\vecmu}(\{ \bar{a}_h ^ j \}_j)) \|_2 \middle| \mathcal{F}_h \right] \leq \frac{2 L }{\sqrt{N}} $, we obtain the lemma.

\subsection{Proof of Lemma~\ref{lemma:bandit_main_recurrence}}\label{sec:proof_lemma_bandit_recurrence}

We will reuse the definitions of Section~\ref{sec:bandit_extended_defs}.
We formulate a recurrence for the main error term of interest, $\| \vecpi_{t+1}^i - \vecpi^*\|_2^2$,
Repeating the steps presented in Lemma~\ref{lemma:full_error_recurrence}, (noting $\alpha_h := 1 - \tau\eta_h$), our proof strategy is to analyze the three terms in the following decomposition:
\begin{align*}
    \| \vecpi_{h+1}^i - \vecpi^*\|_2^2  
    \leq & \underbrace{\eta_h^2\| \widehat{\vecr}_h^i - \vecF(\vecpi_h^i) \|_2^2 + 2\eta_h^2 (\vecF(\vecpi_h^i) - \vecF(\vecpi^*)))^\top (\widehat{\vecr}_h^i - \vecF(\vecpi_h^i))}_{(a)} \\
     &+ \underbrace{2\eta_h\alpha_h (\vecpi_h^i - \vecpi^*)^\top (\widehat{\vecr}_h^i - \vecF(\vecpi_h^i))}_{(b)} + \underbrace{\|\alpha_h (\vecpi_h^i - \vecpi^*) + \eta_h (\vecF(\vecpi_h^i) - \vecF(\vecpi^*))\|_2^2 }_{(c)}
\end{align*}
Once again, we will need to upper bound the three terms above.
For the term $(a)$ we have $\Exop\left[ (a)\right] \leq 4\eta_t^2 K^3(\sigma^2 + 1)$, noting that $\|\widehat{\vecr}_h^i\|_2^2 \leq K^3$ almost surely.
Likewise, it still holds for the term $(c)$ by Lemma~\ref{lemma:contraction_pg} that
\begin{align*}
    (c) 
        \leq & \left(1 - 2 (\lambda + \tau) \eta_h + (L + \tau)^2 \eta_h^2\right) \| \vecpi_h^i - \vecpi^* \|_2^2.
\end{align*}
However, unlike the bound in Lemma~\ref{lemma:full_error_recurrence}, the exploration parameter $\varepsilon$ will cause additional bias in the term $(b)$.
Define the random vector $\widetilde{\vecr}_h^i = \Exop[ \widehat{\vecr}_h^i | E_h^i, \mathcal{F}_h]$.
\begin{align*}
(b) = & 2\eta_h\alpha_h (\vecpi_h^i - \vecpi^*)^\top (\widehat{\vecr}_h^i - \vecF(\vecpi_h^i)) \\
 = & 2\eta_h\alpha_h (\vecpi_h^i - \vecpi^*)^\top (\widehat{\vecr}_h^i - \widetilde{\vecr}_h^i) + 2\eta_h\alpha_h (\vecpi_h^i - \vecpi^*)^\top (\widetilde{\vecr}_h^i - \vecF(\bar{\vecmu}_h)) \\
    & + 2\eta_h\alpha_h (\vecpi_h^i - \vecpi^*)^\top (\vecF(\bar{\vecmu}_h) - \vecF(\vecpi_h^i)) \\
\leq &2\eta_h\alpha_h \left( \frac{\lambda}{4} \|\vecpi_h^i - \vecpi^* \|_2^2 + \frac{1}{\lambda} \|\widetilde{\vecr}_h^i - \vecF(\bar{\vecmu}_h)\|_2^2\right) \\
    & + 2\eta_h\alpha_h \left(\frac{\lambda}{4} \|\vecpi_h^i - \vecpi^* \|_2^2 + \frac{1}{\lambda} \|\vecF(\bar{\vecmu}_h) - \vecF(\vecpi_h^i)\|_2^2 \right) \\
    &+ 2\eta_h\alpha_h (\vecpi_h^i - \vecpi^*)^\top (\widehat{\vecr}_h^i - \widetilde{\vecr}_h^i) \\
\leq & 2 \eta_h \frac{\lambda}{2} \|\vecpi_h^i - \vecpi^* \|_2^2 + \frac{2\eta_h}{\lambda}\|\widetilde{\vecr}_h^i - \vecF(\bar{\vecmu}_h)\|_2^2 + \frac{2\eta_h}{\lambda} \|\vecF(\bar{\vecmu}_h) - \vecF(\vecpi_h^i)\|_2^2 \\
    &+2\eta_h\alpha_h (\vecpi_h^i - \vecpi^*)^\top (\widehat{\vecr}_h^i - \widetilde{\vecr}_h^i),
\end{align*}
and similarly, if $\lambda = 0$, we have
\begin{align*}
(b) = & 2\eta_h\alpha_h (\vecpi_h^i - \vecpi^*)^\top (\widehat{\vecr}_h^i - \vecF(\vecpi_h^i)) \\
 = & 2\eta_h\alpha_h (\vecpi_h^i - \vecpi^*)^\top (\widehat{\vecr}_h^i - \widetilde{\vecr}_h^i) + 2\eta_h\alpha_h (\vecpi_h^i - \vecpi^*)^\top (\widetilde{\vecr}_h^i - \vecF(\bar{\vecmu}_h)) \\
    & + 2\eta_h\alpha_h (\vecpi_h^i - \vecpi^*)^\top (\vecF(\bar{\vecmu}_h) - \vecF(\vecpi_h^i)) \\
\leq &2\eta_h\alpha_h \left( \frac{\tau\delta}{2} \|\vecpi_h^i - \vecpi^* \|_2^2 + \frac{1}{2\tau\delta} \|\widetilde{\vecr}_h^i - \vecF(\bar{\vecmu}_h)\|_2^2\right) \\
    & + 2\eta_h\alpha_h \left(\frac{\tau\delta}{2} \|\vecpi_h^i - \vecpi^* \|_2^2 + \frac{1}{2\tau\delta} \|\vecF(\bar{\vecmu}_h) - \vecF(\vecpi_h^i)\|_2^2 \right) \\
    &+ 2\eta_h\alpha_h (\vecpi_h^i - \vecpi^*)^\top (\widehat{\vecr}_h^i - \widetilde{\vecr}_h^i) \\
\leq & 2 \eta_h \tau\delta \|\vecpi_h^i - \vecpi^* \|_2^2 + \frac{\eta_h}{\tau\delta}\|\widetilde{\vecr}_h^i - \vecF(\bar{\vecmu}_h)\|_2^2 + \frac{\eta_h}{\tau\delta} \|\vecF(\bar{\vecmu}_h) - \vecF(\vecpi_h^i)\|_2^2 \\
    &+2\eta_h\alpha_h (\vecpi_h^i - \vecpi^*)^\top (\widehat{\vecr}_h^i - \widetilde{\vecr}_h^i).
\end{align*}

Denote $\vecpi_{unif} := \frac{1}{K}\vecone_K$.
The remaining error terms we can bound by (using the auxiliary coupling random actions $\bar{a}_h^j$ from Lemma~\ref{lemma:exploration_bias_trpa_bandit}):
\begin{align*}
    |\Exop[2\eta_h\alpha_h (\vecpi_h^i - \vecpi^*)^\top (\widehat{\vecr}_h^i - \widetilde{\vecr}_h^i) | \mathcal{F}_h] | 
        \leq & |\Exop[2\eta_h\alpha_h (\vecpi_h^i - \vecpi^*)^\top (\widehat{\vecr}_h^i - \widetilde{\vecr}_h^i) | E_h^i, \mathcal{F}_h] \mathbb{P}(E_h^i) \\
        & + \Exop[2\eta_h\alpha_h (\vecpi_h^i - \vecpi^*)^\top (\widehat{\vecr}_h^i - \widetilde{\vecr}_h^i) | \overline{E_h^i}, \mathcal{F}_h]\mathbb{P}(\overline{E_h^i})| \\
    \leq & 2\eta_h \Exop\left[\|\vecpi_h^i - \vecpi^*\|_2 \|\widehat{\vecr}_h^i - \widetilde{\vecr}_h^i\|_2 | \overline{E_h^i}, \mathcal{F}_h\right] \mathbb{P}(\overline{E_h^i}) \\
    \leq & 8\eta_h K^{\sfrac{3}{2}} \sqrt{1 + \sigma^2} \mathbb{P}(\overline{E_h^i}) 
\end{align*}
and $\mathbb{P}(\overline{E_h^i}) \leq \exp\{-\varepsilon T_h\}$.
Furthermore,
\begin{align*}
    \Exop[&\|\widetilde{\vecr}_h^i - \vecF(\bar{\vecmu}_h)\|_2^2 | \mathcal{F}_{h}] \\
    \leq & 2\Exop\left[ \|\widetilde{\vecr}_h^i - \vecF(\widehat{\vecmu}(\bar{a}_{h,exp}^i, \bar{a}_h ^ {-i})) \|_2^2\middle|\mathcal{F}_{h}\right] + 2\Exop\left[ \|\vecF(\widehat{\vecmu}(\bar{a}_{h,exp}^i, \bar{a}_h ^ {-i})) - \vecF(\bar{\vecmu}_h)\|_2^2 | \mathcal{F}_{h}\right] \\
    \leq &  \frac{8 L^2}{N}  +  2L^2 \Exop[ \|\widehat{\vecmu}(\bar{a}_{exp}^i, \bar{a} ^ {-i}) - \frac{1}{N}\sum_{j=1}^N \vecpi^j_h\|_2^2 | \mathcal{F}_{h}] \\
    \leq &  \frac{8 L^2}{N}  +  4L^2 \Exop[ \|\widehat{\vecmu}(\bar{a}_{exp}^i, \bar{a} ^ {-i}) - \frac{1}{N}\sum_{j\neq i} (\varepsilon\vecpi_{unif} + (1-\varepsilon)\vecpi^j_h) - \frac{\vecpi_{unif}}{N} \|_2^2 | \mathcal{F}_{h}] \\
        &+ 4L^2 \Exop[ \| \varepsilon\frac{1}{N}\sum_{j\neq i} ( \varepsilon\vecpi^j_h - \vecpi_{unif}) +  \frac{\vecpi^i_h - \vecpi_{unif}}{N} \|_2^2 | \mathcal{F}_{h}] \\
    \leq &  \frac{8 L^2}{N}  +  \frac{8L^2}{N} + 4L^2 (2\varepsilon^2 + \frac{4}{N}) \\
    \leq & \frac{64 L ^2}{N} + 8 L^2 \varepsilon^2,
\end{align*}
and finally, using the trivial Lipschitz continuity property: $\|\vecF(\bar{\vecmu}_h) - \vecF(\vecpi_h^i)\|_2^2 \leq L^2 \|\bar{\vecmu}_h - \vecpi_h^i\|_2^2 = L^2 e_h^i$.

\subsection{Proof of Lemma~\ref{lemma:bandit_pol_deviation}}\label{sec:proof_lemma_bandit_pol_dev}

We will reuse the definitions of Section~\ref{sec:bandit_extended_defs}.
Once again, repeating the derivations from Lemma~\ref{lemma:policy_variations_bound_trpa_full}, we have that
\begin{align*}
    \| \vecpi^i_{h+1} - \vecpi^j_{h+1} \|_2^2 
    = & (1 - \tau \eta_h)^2 \| \vecpi^i_h - \vecpi^j_h \|_2^2 + \eta_h^2 \|  \widehat{\vecr}_h^i - \widehat{\vecr}_h^j \|_2^2 + 2 (1 - \tau \eta_h) \eta_h (\vecpi^i_h - \vecpi^j_h) ^ \top ( \widehat{\vecr}_h^i - \widehat{\vecr}_h^j ).
\end{align*}
Unlike in the expert feedback case, the last term does not vanish in expectation when bounding policy deviation.
\begin{align*}
    \Exop \left[ \| \vecpi^i_{h+1} - \vecpi^j_{h+1} \|_2^2 | \mathcal{F}_h \right] \leq & (1 - \tau \eta_h)^2 \| \vecpi^i_h - \vecpi^j_h \|_2^2 + \Exop \left[ \eta_h^2 \|  \widehat{\vecr}_h^i - \widehat{\vecr}_h^j \|_2^2 | \mathcal{F}_h \right] \\
        &+ 2 (1 - \tau \eta_h) \eta_h (\vecpi^i_h - \vecpi^j_h) ^ \top \Exop\left[\widehat{\vecr}_h^i - \widehat{\vecr}_h^j | \mathcal{F}_h \right] \\
    \leq & (1 - \tau \eta_h)^2 \| \vecpi^i_h - \vecpi^j_h \|_2^2 + \Exop \left[ \eta_h^2 \|  \widehat{\vecr}_h^i - \widehat{\vecr}_h^j \|_2^2 | \mathcal{F}_h \right] \\
        &+ 2 \eta_h (\vecpi^i_h - \vecpi^j_h) ^ \top \left[ \Exop\left[\widehat{\vecr}_h^i \middle|E_{h}^i, \mathcal{F}_h\right] + \vecb_h^i - \Exop\left[\widehat{\vecr}_h^j \middle|E_{h}^j, \mathcal{F}_h\right] - \vecb_h^j \right]  \\
    \leq & (1 - \tau \eta_h)^2 \| \vecpi^i_h - \vecpi^j_h \|_2^2 + \Exop \left[ \eta_h^2 \|  \widehat{\vecr}_h^i - \widehat{\vecr}_h^j \|_2^2 | \mathcal{F}_h \right] \\
        &+ 2 \eta_h (\vecpi^i_h - \vecpi^j_h) ^ \top \left[ \Exop\left[\widehat{\vecr}_h^i \middle|E_{h}^i, \mathcal{F}_h\right] - \Exop\left[\widehat{\vecr}_h^j \middle|E_{h}^j, \mathcal{F}_h\right] \right] + 8 \eta_h \exp\left\{ -\varepsilon T_{h}\right\},
\end{align*}
where the last line follows from the bound on $\vecb_h^i$ in Lemma~\ref{lemma:exploration_bias_trpa_bandit}.
Furthermore, using Young's inequality, we obtain
\begin{align*}
    \Exop \left[ \| \vecpi^i_{h+1} - \vecpi^j_{h+1} \|_2^2 | \mathcal{F}_h \right] \leq & (1 - \tau \eta_h)^2 \| \vecpi^i_h - \vecpi^j_h \|_2^2 + \Exop \left[ \eta_h^2 \|  \widehat{\vecr}_h^i - \widehat{\vecr}_h^j \|_2^2 | \mathcal{F}_h \right] + 8 \eta_h \exp\left\{ -\varepsilon T_{h}\right\} \\
        &+ \frac{\tau\eta_h}{2} \|\vecpi^i_h - \vecpi^j_h\|_2^2 + \frac{ \eta_h\tau^{-1}}{2}\left\| \Exop\left[\widehat{\vecr}_h^i \middle|E_{h}^i, \mathcal{F}_h\right] - \Exop\left[\widehat{\vecr}_h^j \middle|E_{h}^j, \mathcal{F}_h\right] \right\|_2^2 \\
     \leq & (1 - \tau \eta_h)^2 \| \vecpi^i_h - \vecpi^j_h \|_2^2 + \Exop \left[ \eta_h^2 \|  \widehat{\vecr}_h^i - \widehat{\vecr}_h^j \|_2^2 | \mathcal{F}_h \right] + 8 \eta_h \exp\left\{ -\varepsilon T_{h}\right\} \\
        &+ \frac{\tau\eta_h}{2} \|\vecpi^i_h - \vecpi^j_h\|_2^2 + \frac{2\eta_h\tau^{-1} L ^ 2}{N^2}.
\end{align*}
With the choice of $T_h = \varepsilon^{-1} \log (h+2)$ and noting that $\| \vecpi^i_{h} - \vecpi_h^j\|_2 \leq 2$, we obtain the recurrence
\begin{align*}
    \Exop \left[ \| \vecpi^i_{h+1} - \vecpi^j_{h+1} \|^2_2 \right] \leq & \left(1 - \frac{\sfrac{3}{2}}{h+2}\right) \Exop \left[ \| \vecpi^i_h - \vecpi^j_h \|_2^2\right] + \frac{4\tau^{-2} K^3 (\sigma^2 + 1) + 8\tau^{-1} + 4}{(h+2)^2}   \\
        & + \frac{4\tau^{-2} L ^ 2}{N^2 (h+2)}.
\end{align*}
Hence, by invoking the recurrence lemma (Lemma~\ref{lemma:general_recurrence}, with $a=2, c_0 = \frac{4\tau^{-2} L ^ 2}{N^2}, c_1 = 4\tau^{-2} K^3 (\sigma^2 + 1) + 8\tau^{-1} + 4, \gamma = \sfrac{3}{2}, u_0 = 0$), we have
\begin{align*}
    \Exop \left[ \| \vecpi^i_{h+1} - \vecpi^j_{h+1} \|_2^2 \right] \leq & \frac{24\tau^{-2} K^3 (\sigma^2 + 1) + 48\tau^{-2} + 24}{h+2} + \frac{16\tau^{-2} L ^ 2}{N^2}.
\end{align*}
The statement in the lemma follows as in the full feedback case.

\subsection{Proof of Theorem~\ref{theorem:bandit_short}}\label{sec:bandit_theorem_full}

Using Lemma~\ref{lemma:bandit_main_recurrence}, for the strongly monotone case $\lambda > 0$ we have that
\begin{align*}
    u_{h+1}^i \leq & 4 \eta_h^2 K^3(1 + \sigma^2) + 8\eta_h^2(L+\tau)^2 + 8 K^{\sfrac{3}{2}}\eta_h \sqrt{1+\sigma^2}  \exp\{-\varepsilon T_h\} \\
        &+128\eta_h\lambda^{-1} L^2 N^{-1} + 16\eta_h\lambda^{-1}L^2\varepsilon^2 + 2\eta_h\lambda^{-1} L^2 \Exop\left[e_h^i\right] \\
        &+\left(1 - 2 \eta_h(\sfrac{\lambda}{2} + \tau)\right) u_{h}^i,
\end{align*}
and for the monotone case it holds that 
\begin{align*}
    u_{h+1}^i \leq &  4\eta_h^2 K^3(\sigma^2 + 1) + 8 \eta_h^2 (L+\tau)^2 + 8 K^{\sfrac{3}{2}} \eta_h \sqrt{1+\sigma^2} \exp\{-\varepsilon T_h\} \\
    &+64\tau^{-1} \eta_h \delta^{-1}L^2 N^{-1}+8\tau^{-1} \eta_h \delta^{-1}L^2 \varepsilon^{2} + \tau^{-1} \eta_h\delta^{-1}L^2 \Exop\left[e_h^i\right] \\  
        &+ \left(1 - 2 \tau \eta_h (1-\delta)\right) u_{h}^i.
\end{align*}
By Lemma~\ref{lemma:bandit_pol_deviation}, we know that
\begin{align*}
    \Exop[e_h^i] \leq \frac{12\tau^{-2} K^3 (\sigma^2 + 1) + 48\tau^{-2} + 24}{h+2} + \frac{16\tau^{-2} L ^ 2}{N^2}.
\end{align*}
Placing this bound as well as $T_h = \lceil \varepsilon^{-1} \log (h+2) \rceil$ and $\eta_h = \frac{\tau^{-1}}{h+2}$, we obtain the recurrences which are solved by using Lemma~\ref{lemma:general_recurrence}.
In the monotone case, we pick $\delta=\sfrac{1}{4}$ as before.



The bound in the statement of the theorem in the main body of the paper follows from the fact that the lengths of the exploration epochs scale with $T_h = \mathcal{O}(\varepsilon^{-1} \log (h+2)) = \widetilde{\mathcal{O}}(\varepsilon^{-1})$.

\section{Experimental details}
\label{appendix:experiments}

We use the following search space for hyperparameters:
\begin{itemize}
    \item $c_s \in \mathbb{Z}^+$: Maximum number of tokens in each document chunk.
    \item $c_n \in \mathbb{Z}^+$: Number of chunks retrieved from the vector database for each query.
    \item $o \in \mathbb{Z}^+$: Number of tokens which overlap between adjacent chunks in a document.
    \item $t \in [0,1.2]$: Temperature of the LLM when generating responses.
    \item $r \in [0, 1]$: Rerank threshold used to set the minimum similarity between the context chunk and query, as evaluated by the reranker\footnote{We use a fixed rerank model \texttt{Salesforce/Llama-Rank-V1} provided by TogetherAI for all RAG systems.}. Retrieved documents which are below this threshold are ignored and not passed to the LLM as context. If no chunks exceed this threshold, we choose only the highest scoring chunk as context.
    \item $\ell \in \{\text{gpt-4o}, \text{gpt-4o-mini}, \text{llama-3.2-3B}, \text{llama-3.1-8B}\}$: Choice of LLM used to generate the response.
    \item $e \in \{\text{text-embedding-3-large},\text{text-embedding-3-small}
    \}$: Choice of embedding model when embedding the queries and document chunks.
\end{itemize}


\end{document}
