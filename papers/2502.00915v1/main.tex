\pdfoutput=1

\documentclass[twoside,11pt]{article}
\usepackage{jmlr2e}



\usepackage{xcolor}
\usepackage[utf8]{inputenc} %
\usepackage[T1]{fontenc}    %
\usepackage{hyperref}       %
\usepackage{url}            %
\usepackage{booktabs}       %
\usepackage{amsfonts}       %

\usepackage{booktabs} %
\usepackage{longtable}

\usepackage{hyperref}

\newcommand{\theHalgorithm}{\arabic{algorithm}}



\usepackage{algorithm}%
\usepackage{algorithmicx}%
\usepackage{algpseudocode}%

\usepackage{amsmath,amssymb}%
\usepackage{graphicx}
\usepackage{mathrsfs}%

\usepackage{mathrsfs}%
\usepackage[title]{appendix}%
\usepackage{xcolor}%
\usepackage{textcomp}%
\usepackage{manyfoot}%

\usepackage{enumitem}

 \usepackage[english]{babel}
\usepackage{mathtools}
\usepackage{thmtools}
\usepackage{thm-restate}
\usepackage{algorithm}[l]
\usepackage{algpseudocode}
\usepackage{multirow}
\usepackage{chngpage}
\usepackage{xfrac}

\usepackage{mathrsfs}
\usepackage{xspace}
\usepackage{bm}
\usepackage{upgreek}
\usepackage{bbm}


\usepackage{subfig}

\section{Problem Studied}\label{sec:def}
We first present Fixed-Radius Near Neighbor (FRNN) queries and then formalize Aggregation Queries over Nearest Neighbors (AQNNs) that build on them. We then state our problem.

\subsection{Nearest Neighbor Queries}\label{subsec:FRNN}
We build on generalized Fixed-Radius Near Neighbor (FRNN) queries \cite{FRNNSurvey}. Given a dataset \( D \), a query object \( q \), a radius \( r \), and a distance function \( dist \), a generalized FRNN query retrieves all nearest neighbors of \( q \) within radius \( r \). More formally:
\[
NN_D(q, r) = \{x \in D \mid dist(x, q) \leq r\},
\]
where \(x\) is any data point in \(D\) and \(dist(x, q)\) denotes the distance between them. We use \(|NN_D(q,r)|\) to denote the neighborhood size of \(q\). As shown in Fig. \ref{fig:framework}, given a radius \(r\) and a target patient \(q\), patients in the dotted circle are nearest neighbors, and the neighborhood size is 6.

\subsection{Aggregation Queries over Nearest Neighbors}\label{subsec:AQNN} 
Given an FRNN query object \(q\) in dataset \(D\), a radius \(r\), and an attribute \(\texttt{attr}\), an Aggregation Query over Nearest Neighbors (AQNN) is defined as:
\[ \text{agg}(NN_D(q,r)[\texttt{attr}]) \]
where agg is an aggregation function, such as $\mathtt{AVG}$, $\mathtt{SUM}$, and $\mathtt{PCT}$, and \(NN_D(q,r)[\texttt{attr}]\) denotes the bag of values of attribute \texttt{attr} of all FRNN results of \(q\) within radius \(r\). 
% \end{definition}

An AQNN expresses aggregation operations to capture key insights about the neighborhood of a query object. For example, \(\mathtt{AVG}\) can be used to reflect the average heart rate or systolic blood pressure of patients in the neighborhood, providing a measure of typical health conditions. \(\mathtt{SUM}\) is useful for assessing cumulative effects, such as the total cost of treatments in the neighborhood that instructs public policy in terms of health. Similarly, $\mathtt{PCT}$ can be used to find the proportion of patients in the neighborhood of a patient of interest, relative to the population in the dataset.
%\laks{Why is finding the total \#meds to NNs or the total treatment cost of everyone in the NN interesting?}

% \texttt{MIN} and \texttt{MAX} are not included in the aggregation functions because they only capture extreme values, which may not represent the typical characteristics of the nearest neighbors and are more sensitive to outliers. 
% \laks{AVG is also sensitive to outliers, but we still allow it. isn't the real reason we don't consider MIN/MAX because they are amenable to estimation via sampling?} We choose \texttt{PCT} instead of \texttt{COUNT} in order to provide a normalized measure that remains comparable across different neighborhood sizes. It allows for more consistent interpretation of relative popularity \cite{moore1989introduction}.


Fig. \ref{fig:framework} illustrates an example of an AQNN: ``\textit{Find the average systolic blood pressure of patients similar to an insomnia patient \(q\)}''. The aggregation function is \(\mathtt{AVG}\) and the target attribute of interest is systolic blood pressure. Exact query evaluation requires consulting physicians (or predicting embeddings by an expensive machine learning model) for all 500 patients in \(D\) and calculate \(q\)'s nearest neighbors wrt \(r\) \cite{DBLP:journals/isci/RodriguesGSBA21}. We refer to such highly accurate but computationally expensive models as \textit{oracle models}, denoted as \(O\), including deep learning models trained on domain-specific data or human expert annotations \cite{DBLP:conf/sigmod/LuCKC18}. Using oracle models is very expensive \cite{sze2017efficient, DujianPQA, DBLP:journals/pvldb/KangGBHZ20}. To address that, we seek an approximate solution by \textit{proxy models}, denoted as \(P\), that are at least one order of magnitude cheaper than oracle models. In the example, if consulting physicians for one patient incurs one cost unit, calling a cheap machine learning model instead incurs at most \(0.1\) cost unit. Once the similar patients are identified, their systolic blood pressure values are averaged and returned as  output. The use of a proxy model may reduce the accuracy of the neighborhood prediction and hence, we should judiciously call oracle and proxy models to minimize the error of aggregate results.

Note that the values of the target attribute \texttt{attr} are \textit{not} predicted but are instead known quantities.

\subsection{Problem Statement}
Given an AQNN, our goal is to return an approximate aggregate result by leveraging both oracle and proxy models while reducing error and cost.




\looseness=-1





\ShortHeadings{VI Approach to Independent Learning in Static MFGs}{VI Approach to Independent Learning in Static MFGs}
\title{
A Variational Inequality Approach to Independent Learning in Static Mean-Field Games
}


\author{\name Batuhan Yardim \email alibatuhan.yardim@inf.ethz.ch \\
     \addr Department of Computer Science\\
          ETH Z\"urich\\
          Z\"urich, Switzerland\\
     \AND
     \name Semih Cayci   \email cayci@mathc.rwth-aachen.de \\
     \addr Department of Mathematics\\
          RWTH Aachen University \\
          Aachen, Germany\\
     \AND
     \name Niao He   \email niao.he@inf.ethz.ch \\
     \addr Department of Computer Science\\
          ETH Z\"urich\\
          Z\"urich, Switzerland
}



\begin{document}

\maketitle

\begin{abstract}%
Competitive games involving thousands or even millions of players are prevalent in real-world contexts, such as transportation, communications, and computer networks. However, learning in these large-scale multi-agent environments presents a grand challenge, often referred to as the "curse of many agents". In this paper, we formalize and analyze the Static Mean-Field Game (SMFG) under both full and bandit feedback, offering a generic framework for modeling large population interactions while enabling independent learning. 


We first establish close connections between SMFG and variational inequality (VI),  showing that SMFG can be framed as a VI problem in the infinite agent limit. Building on the VI perspective, we propose independent learning and exploration algorithms that efficiently converge to approximate Nash equilibria, when dealing with a finite number of agents.   Theoretically, we provide explicit finite sample complexity guarantees for independent learning across various feedback models in repeated play scenarios, assuming (strongly-)monotone payoffs. Numerically, we validate our results through both simulations and real-world applications in city traffic and network access management. 

\end{abstract}

\begin{keywords}
variational inequality, independent learning, mean-field games, multi-agent systems, bandit feedback
\end{keywords}


\section{Introduction}


\begin{figure}[t]
\centering
\includegraphics[width=0.6\columnwidth]{figures/evaluation_desiderata_V5.pdf}
\vspace{-0.5cm}
\caption{\systemName is a platform for conducting realistic evaluations of code LLMs, collecting human preferences of coding models with real users, real tasks, and in realistic environments, aimed at addressing the limitations of existing evaluations.
}
\label{fig:motivation}
\end{figure}

\begin{figure*}[t]
\centering
\includegraphics[width=\textwidth]{figures/system_design_v2.png}
\caption{We introduce \systemName, a VSCode extension to collect human preferences of code directly in a developer's IDE. \systemName enables developers to use code completions from various models. The system comprises a) the interface in the user's IDE which presents paired completions to users (left), b) a sampling strategy that picks model pairs to reduce latency (right, top), and c) a prompting scheme that allows diverse LLMs to perform code completions with high fidelity.
Users can select between the top completion (green box) using \texttt{tab} or the bottom completion (blue box) using \texttt{shift+tab}.}
\label{fig:overview}
\end{figure*}

As model capabilities improve, large language models (LLMs) are increasingly integrated into user environments and workflows.
For example, software developers code with AI in integrated developer environments (IDEs)~\citep{peng2023impact}, doctors rely on notes generated through ambient listening~\citep{oberst2024science}, and lawyers consider case evidence identified by electronic discovery systems~\citep{yang2024beyond}.
Increasing deployment of models in productivity tools demands evaluation that more closely reflects real-world circumstances~\citep{hutchinson2022evaluation, saxon2024benchmarks, kapoor2024ai}.
While newer benchmarks and live platforms incorporate human feedback to capture real-world usage, they almost exclusively focus on evaluating LLMs in chat conversations~\citep{zheng2023judging,dubois2023alpacafarm,chiang2024chatbot, kirk2024the}.
Model evaluation must move beyond chat-based interactions and into specialized user environments.



 

In this work, we focus on evaluating LLM-based coding assistants. 
Despite the popularity of these tools---millions of developers use Github Copilot~\citep{Copilot}---existing
evaluations of the coding capabilities of new models exhibit multiple limitations (Figure~\ref{fig:motivation}, bottom).
Traditional ML benchmarks evaluate LLM capabilities by measuring how well a model can complete static, interview-style coding tasks~\citep{chen2021evaluating,austin2021program,jain2024livecodebench, white2024livebench} and lack \emph{real users}. 
User studies recruit real users to evaluate the effectiveness of LLMs as coding assistants, but are often limited to simple programming tasks as opposed to \emph{real tasks}~\citep{vaithilingam2022expectation,ross2023programmer, mozannar2024realhumaneval}.
Recent efforts to collect human feedback such as Chatbot Arena~\citep{chiang2024chatbot} are still removed from a \emph{realistic environment}, resulting in users and data that deviate from typical software development processes.
We introduce \systemName to address these limitations (Figure~\ref{fig:motivation}, top), and we describe our three main contributions below.


\textbf{We deploy \systemName in-the-wild to collect human preferences on code.} 
\systemName is a Visual Studio Code extension, collecting preferences directly in a developer's IDE within their actual workflow (Figure~\ref{fig:overview}).
\systemName provides developers with code completions, akin to the type of support provided by Github Copilot~\citep{Copilot}. 
Over the past 3 months, \systemName has served over~\completions suggestions from 10 state-of-the-art LLMs, 
gathering \sampleCount~votes from \userCount~users.
To collect user preferences,
\systemName presents a novel interface that shows users paired code completions from two different LLMs, which are determined based on a sampling strategy that aims to 
mitigate latency while preserving coverage across model comparisons.
Additionally, we devise a prompting scheme that allows a diverse set of models to perform code completions with high fidelity.
See Section~\ref{sec:system} and Section~\ref{sec:deployment} for details about system design and deployment respectively.



\textbf{We construct a leaderboard of user preferences and find notable differences from existing static benchmarks and human preference leaderboards.}
In general, we observe that smaller models seem to overperform in static benchmarks compared to our leaderboard, while performance among larger models is mixed (Section~\ref{sec:leaderboard_calculation}).
We attribute these differences to the fact that \systemName is exposed to users and tasks that differ drastically from code evaluations in the past. 
Our data spans 103 programming languages and 24 natural languages as well as a variety of real-world applications and code structures, while static benchmarks tend to focus on a specific programming and natural language and task (e.g. coding competition problems).
Additionally, while all of \systemName interactions contain code contexts and the majority involve infilling tasks, a much smaller fraction of Chatbot Arena's coding tasks contain code context, with infilling tasks appearing even more rarely. 
We analyze our data in depth in Section~\ref{subsec:comparison}.



\textbf{We derive new insights into user preferences of code by analyzing \systemName's diverse and distinct data distribution.}
We compare user preferences across different stratifications of input data (e.g., common versus rare languages) and observe which affect observed preferences most (Section~\ref{sec:analysis}).
For example, while user preferences stay relatively consistent across various programming languages, they differ drastically between different task categories (e.g. frontend/backend versus algorithm design).
We also observe variations in user preference due to different features related to code structure 
(e.g., context length and completion patterns).
We open-source \systemName and release a curated subset of code contexts.
Altogether, our results highlight the necessity of model evaluation in realistic and domain-specific settings.





\section{RELATED WORK}
\label{sec:relatedwork}
In this section, we describe the previous works related to our proposal, which are divided into two parts. In Section~\ref{sec:relatedwork_exoplanet}, we present a review of approaches based on machine learning techniques for the detection of planetary transit signals. Section~\ref{sec:relatedwork_attention} provides an account of the approaches based on attention mechanisms applied in Astronomy.\par

\subsection{Exoplanet detection}
\label{sec:relatedwork_exoplanet}
Machine learning methods have achieved great performance for the automatic selection of exoplanet transit signals. One of the earliest applications of machine learning is a model named Autovetter \citep{MCcauliff}, which is a random forest (RF) model based on characteristics derived from Kepler pipeline statistics to classify exoplanet and false positive signals. Then, other studies emerged that also used supervised learning. \cite{mislis2016sidra} also used a RF, but unlike the work by \citet{MCcauliff}, they used simulated light curves and a box least square \citep[BLS;][]{kovacs2002box}-based periodogram to search for transiting exoplanets. \citet{thompson2015machine} proposed a k-nearest neighbors model for Kepler data to determine if a given signal has similarity to known transits. Unsupervised learning techniques were also applied, such as self-organizing maps (SOM), proposed \citet{armstrong2016transit}; which implements an architecture to segment similar light curves. In the same way, \citet{armstrong2018automatic} developed a combination of supervised and unsupervised learning, including RF and SOM models. In general, these approaches require a previous phase of feature engineering for each light curve. \par

%DL is a modern data-driven technology that automatically extracts characteristics, and that has been successful in classification problems from a variety of application domains. The architecture relies on several layers of NNs of simple interconnected units and uses layers to build increasingly complex and useful features by means of linear and non-linear transformation. This family of models is capable of generating increasingly high-level representations \citep{lecun2015deep}.

The application of DL for exoplanetary signal detection has evolved rapidly in recent years and has become very popular in planetary science.  \citet{pearson2018} and \citet{zucker2018shallow} developed CNN-based algorithms that learn from synthetic data to search for exoplanets. Perhaps one of the most successful applications of the DL models in transit detection was that of \citet{Shallue_2018}; who, in collaboration with Google, proposed a CNN named AstroNet that recognizes exoplanet signals in real data from Kepler. AstroNet uses the training set of labelled TCEs from the Autovetter planet candidate catalog of Q1–Q17 data release 24 (DR24) of the Kepler mission \citep{catanzarite2015autovetter}. AstroNet analyses the data in two views: a ``global view'', and ``local view'' \citep{Shallue_2018}. \par


% The global view shows the characteristics of the light curve over an orbital period, and a local view shows the moment at occurring the transit in detail

%different = space-based

Based on AstroNet, researchers have modified the original AstroNet model to rank candidates from different surveys, specifically for Kepler and TESS missions. \citet{ansdell2018scientific} developed a CNN trained on Kepler data, and included for the first time the information on the centroids, showing that the model improves performance considerably. Then, \citet{osborn2020rapid} and \citet{yu2019identifying} also included the centroids information, but in addition, \citet{osborn2020rapid} included information of the stellar and transit parameters. Finally, \citet{rao2021nigraha} proposed a pipeline that includes a new ``half-phase'' view of the transit signal. This half-phase view represents a transit view with a different time and phase. The purpose of this view is to recover any possible secondary eclipse (the object hiding behind the disk of the primary star).


%last pipeline applies a procedure after the prediction of the model to obtain new candidates, this process is carried out through a series of steps that include the evaluation with Discovery and Validation of Exoplanets (DAVE) \citet{kostov2019discovery} that was adapted for the TESS telescope.\par
%



\subsection{Attention mechanisms in astronomy}
\label{sec:relatedwork_attention}
Despite the remarkable success of attention mechanisms in sequential data, few papers have exploited their advantages in astronomy. In particular, there are no models based on attention mechanisms for detecting planets. Below we present a summary of the main applications of this modeling approach to astronomy, based on two points of view; performance and interpretability of the model.\par
%Attention mechanisms have not yet been explored in all sub-areas of astronomy. However, recent works show a successful application of the mechanism.
%performance

The application of attention mechanisms has shown improvements in the performance of some regression and classification tasks compared to previous approaches. One of the first implementations of the attention mechanism was to find gravitational lenses proposed by \citet{thuruthipilly2021finding}. They designed 21 self-attention-based encoder models, where each model was trained separately with 18,000 simulated images, demonstrating that the model based on the Transformer has a better performance and uses fewer trainable parameters compared to CNN. A novel application was proposed by \citet{lin2021galaxy} for the morphological classification of galaxies, who used an architecture derived from the Transformer, named Vision Transformer (VIT) \citep{dosovitskiy2020image}. \citet{lin2021galaxy} demonstrated competitive results compared to CNNs. Another application with successful results was proposed by \citet{zerveas2021transformer}; which first proposed a transformer-based framework for learning unsupervised representations of multivariate time series. Their methodology takes advantage of unlabeled data to train an encoder and extract dense vector representations of time series. Subsequently, they evaluate the model for regression and classification tasks, demonstrating better performance than other state-of-the-art supervised methods, even with data sets with limited samples.

%interpretation
Regarding the interpretability of the model, a recent contribution that analyses the attention maps was presented by \citet{bowles20212}, which explored the use of group-equivariant self-attention for radio astronomy classification. Compared to other approaches, this model analysed the attention maps of the predictions and showed that the mechanism extracts the brightest spots and jets of the radio source more clearly. This indicates that attention maps for prediction interpretation could help experts see patterns that the human eye often misses. \par

In the field of variable stars, \citet{allam2021paying} employed the mechanism for classifying multivariate time series in variable stars. And additionally, \citet{allam2021paying} showed that the activation weights are accommodated according to the variation in brightness of the star, achieving a more interpretable model. And finally, related to the TESS telescope, \citet{morvan2022don} proposed a model that removes the noise from the light curves through the distribution of attention weights. \citet{morvan2022don} showed that the use of the attention mechanism is excellent for removing noise and outliers in time series datasets compared with other approaches. In addition, the use of attention maps allowed them to show the representations learned from the model. \par

Recent attention mechanism approaches in astronomy demonstrate comparable results with earlier approaches, such as CNNs. At the same time, they offer interpretability of their results, which allows a post-prediction analysis. \par


\section{Auxiliary-Variable Adaptive Control Barrier Functions}
\label{sec:AVBCBF}

In this section, we introduce Auxiliary-Variable Adaptive Control Barrier Functions (AVCBFs) for safety-critical control.
We start with a simple example to motivate the need for AVCBFs.

\subsection{Motivation Example: Simplified Adaptive Cruise Control}
\label{subsec:SACC-problem}

Consider a Simplified Adaptive Cruise Control (SACC) problem with the dynamics of ego vehicle expressed as 
\begin{small}
\begin{equation}
\label{eq:SACC-dynamics}
\underbrace{\begin{bmatrix}
\dot{z}(t) \\
\dot{v}(t) 
\end{bmatrix}}_{\dot{\boldsymbol{x}}(t)}  
=\underbrace{\begin{bmatrix}
 v_{p}-v(t) \\
 0
\end{bmatrix}}_{f(\boldsymbol{x}(t))} 
+ \underbrace{\begin{bmatrix}
  0 \\
  1 
\end{bmatrix}}_{g(\boldsymbol{x}(t))}u(t),
\end{equation}
\end{small}
where $v_{p}>0, v(t)>0$ denote the velocity of the lead vehicle (constant velocity) and ego vehicle, respectively, $z(t)$ denotes the distance between the lead and ego vehicle and $u(t)$ denotes the acceleration (control) of ego vehicle, subject to the control constraints
\begin{equation}
\label{eq:simple-control-constraint}
u_{min}\le u(t) \le u_{max}, \forall t \ge0,
\end{equation}
where $u_{min}<0$ and $u_{max}>0$ are the minimum and maximum control input, respectively.

 For safety, we require that $z(t)$ always be greater than or equal to the safety distance denoted by $l_{p}>0,$ i.e., $z(t)\ge l_{p}, \forall t \ge 0.$ Based on Def. \ref{def:HOCBF}, let $\psi_{0}(\boldsymbol{x})\coloneqq b(\boldsymbol{x})=z(t)-l_{p}.$ From \eqref{eq:sequence-f1} and \eqref{eq:sequence-set1}, since the relative degree of $b(\boldsymbol{x})$ is 2, we have
\begin{equation}
\label{eq:SACC-HOCBF-sequence}
\begin{split}
&\psi_{1}(\boldsymbol{x})\coloneqq v_{p}-v(t)+k_{1}\psi_{0}(\boldsymbol{x})\ge 0
,\\
&\psi_{2}(\boldsymbol{x})\coloneqq -u(t)+k_{1}(v_{p}-v(t))+k_{2}\psi_{1}(\boldsymbol{x})\ge 0,
\end{split}
\end{equation}
where $\alpha_{1}(\psi_{0}(\boldsymbol{x}))\coloneqq k_{1}\psi_{0}(\boldsymbol{x}), \alpha_{2}(\psi_{1}(\boldsymbol{x}))\coloneqq k_{2}\psi_{1}(\boldsymbol{x}), k_{1}>0, k_{2}>0.$ The constant class $\kappa$ coefficients $k_{1},k_{2}$ are always chosen small to equip ego vehicle with a conservative control strategy to keep it safe, i.e., smaller $k_{1},k_{2}$ make ego vehicle brake earlier (see \cite{xiao2021high}). Suppose we wish to minimize the energy cost as $\int_{0}^{T} u^{2}(t)dt.$ We can then formulate the cost in the QP with constraint $\psi_{2}(\boldsymbol{x})\ge0$ and control input constraint \eqref{eq:simple-control-constraint} to get the optimal controller for the SACC problem. However, the feasible set of input can easily become empty if $u(t)\le k_{1}(v_{p}-v(t))+k_{2}\psi_{1}(\boldsymbol{x})<u_{min}$,  which causes infeasibility of the optimization. In \cite{xiao2021adaptive}, the authors introduced penalty variables in front of class $\kappa$ functions to enhance the feasibility. This approach defines $\psi_{0}(\boldsymbol{x})\coloneqq b(\boldsymbol{x})=z(t)-l_{p}$ as PACBF and other constraints can be further defined as
\begin{equation}
\label{eq:SACC-PACBF-sequence}
\begin{split}
\psi_{1}(\boldsymbol{x},p_{1}(t))&\coloneqq v_{p}-v(t)+p_{1}(t)k_{1}\psi_{0}(\boldsymbol{x})\ge 0,\\
\psi_{2}(\boldsymbol{x},p_{1}(t),&\boldsymbol{\nu})\coloneqq \nu_{1}(t)k_{1}\psi_{0}(\boldsymbol{x})+p_{1}(t)k_{1}(v_{p}\\
-v(t))&-u(t)+\nu_{2}(t)k_{2}\psi_{1}(\boldsymbol{x},p_{1}(t))\ge 0,
\end{split}
\end{equation}
where $\nu_{1}(t)=\dot{p}_{1}(t),\nu_{2}(t)=p_{2}(t), p_{1}(t)\ge0,p_{2}(t)\ge0,\boldsymbol{\nu}=(\nu_{1}(t),\nu_{2}(t)).$ $p_{1}(t),p_{2}(t)$ are time-varying penalty variables, which alleviate the conservativeness of the control strategy and $\nu_{1}(t),\nu_{2}(t)$ are auxiliary inputs, which relax the constraints for $u(t)$ in $\psi_{2}(\boldsymbol{x},p_{1}(t),\boldsymbol{\nu})\ge0$ and \eqref{eq:simple-control-constraint}. However, in practice, we need to define several additional constraints to make PACBF valid as shown in Eqs. (24)-(27) in \cite{xiao2021adaptive}. First, we need to define HOCBFs ($b_{1}(p_{1}(t))=p_{1}(t),b_{2}(p_{2}(t))=p_{2}(t))$ based on Def. \ref{def:HOCBF} to ensure $p_{1}(t)\ge0,p_{2}(t)\ge0.$ Next we need to define HOCBF ($b_{3}(p_{1}(t))=p_{1,max}-p_{1}(t)$) to confine the value of $p_{1}(t)$ in the range $[0,p_{1,max}].$ We also need to define CLF ($V(p_{1}(t))=(p_{1}(t)-p_{1}^{\ast})^{2}$) based on Def. \ref{def:control-l-f} to keep $p_{1}(t)$ close to a small value $p_{1}^{\ast}.$ $b_{3}(p_{1}(t)), V(p_{1}(t))$ are necessary since $\psi_{0}(\boldsymbol{x},p_{1}(t))\coloneqq p_{1}(t)k_{1}\psi_{0}(\boldsymbol{x})$ in first constraint in \eqref{eq:SACC-PACBF-sequence} is not a class $\kappa$ function with respect to $\psi_{0}(\boldsymbol{x}),$ i.e., $p_{1}(t)k_{1}\psi_{0}(\boldsymbol{x})$ is not guaranteed to strictly increase since $\psi_{0}(\boldsymbol{x},p_{1}(t))$ is in fact a class $\kappa$ function with respect to $p_{1}(t)\psi_{0}(\boldsymbol{x})$, which is against Thm. \ref{thm:safety-guarantee}, therefore $\psi_{1}(\boldsymbol{x},p_{1}(t))\ge 0$ in \eqref{eq:SACC-PACBF-sequence} may not guarantee $\psi_{0}(\boldsymbol{x})\ge 0.$ This illustrates why we have to limit the growth of $p_{1}(t)$ by defining $b_{3}(p_{1}(t)),V(p_{1}(t)).$ However, the way to choose appropriate values for $p_{1,max},p_{1}^{\ast}$ is not straightforward. We can imagine as the relative degree of $b(\boldsymbol{x})$ gets higher, the number of additional constraints we should define also gets larger, which results in complicated parameter-tuning process. To address this issue, we introduce $a_{1}(t),a_{2}(t)$ in the form
\begin{small}
\begin{equation}
\label{eq:SACC-AVBCBF-sequence}
\begin{split}
\psi_{1}(\boldsymbol{x},\boldsymbol{a},\dot{a}_{1}(t))\coloneqq a_{2}(t)(\dot{\psi}_{0}(\boldsymbol{x},a_{1}(t))
+k_{1}\psi_{0}(\boldsymbol{x},a_{1}(t)))\ge 0,\\
\psi_{2}(\boldsymbol{x},\boldsymbol{a},\dot{a}_{1}(t),\boldsymbol{\nu})\coloneqq \nu_{2}(t)\frac{\psi_{1}(\boldsymbol{x},\boldsymbol{a},\dot{a}_{1}(t))}{a_{2}(t)} +a_{2}(t)(\nu_{1}(t)(z(t)\\
-l_{p})+2\dot{a}_{1}(t)(v_{p}-v(t))-a_{1}(t)u(t)+k_{1}\dot{\psi}_{0}(\boldsymbol{x},a_{1}(t)))\\
+k_{2}\psi_{1}(\boldsymbol{x},\boldsymbol{a},\dot{a}_{1}(t))\ge 0, 
\end{split}
\end{equation}
\end{small}
where $\psi_{0}(\boldsymbol{x},a_{1}(t))\coloneqq a_{1}(t)b (\boldsymbol{x})=a_{1}(t)(z(t)-l_{p}),\boldsymbol{\nu}=[\nu_{1}(t),\nu_{2}(t)]^{T}=[\ddot{a}_{1}(t),\dot{a}_{2}(t)]^{T},\boldsymbol{a}=[a_{1}(t),a_{2}(t)]^{T},$ $a_{1}(t),a_{2}(t)$ are time-varying auxiliary variables. Since $\psi_{0}(\boldsymbol{x},a_{1}(t))\ge0,\psi_{1}(\boldsymbol{x},\boldsymbol{a},\dot{a}_{1}(t))\ge 0$ will not be against $b(\boldsymbol{x})\ge 0,\dot{\psi}_{0}(\boldsymbol{x},a_{1}(t))
+k_{1}\psi_{0}(\boldsymbol{x},a_{1}(t))\ge 0$ iff $a_{1}(t)>0,a_{2}(t)>0,$ we need to define HOCBFs for auxiliary variables to make $a_{1}(t)>0,a_{2}(t)>0,$ which will be illustrated in Sec. \ref{sec:AVCBFs}.  $\nu_{1}(t),\nu_{2}(t)$ are auxiliary inputs which are used to alleviate the restriction of constraints for $u(t)$ in $\psi_{2}(\boldsymbol{x},\boldsymbol{a},\dot{a}_{1}(t),\boldsymbol{\nu})\ge0$ and \eqref{eq:simple-control-constraint}. Different from the first constraint in \eqref{eq:SACC-PACBF-sequence}, $k_{1}\psi_{0}(\boldsymbol{x},a_{1}(t))$ is still a class $\kappa$ function with respect to $\psi_{0}(\boldsymbol{x},a_{1}(t)),$ therefore we do not need to define additional HOCBFs and CLFs like $b_{3}(p_{1}(t)),V(p_{1}(t))$ to limit the growth of $a_{1}(t).$
We can rewrite $\psi_{1} (\boldsymbol{x},\boldsymbol{a},\dot{a}_{1}(t))$ in \eqref{eq:SACC-AVBCBF-sequence} as
\begin{equation}
\label{eq:SACC-AVBCBF-sequence-rewrite}
\begin{split}
\psi_{1}(\boldsymbol{x},\boldsymbol{a},\dot{a}_{1}(t))\coloneqq a_{2}(t)a_{1}(t)(v_{p}-v(t)\\
+k_{1}(1+\frac{\dot{a}_{1}(t)}{k_{1}a_{1}(t)})b(\boldsymbol{x}))\ge 0.
\end{split}
\end{equation}
Compared to the first constraint in \eqref{eq:SACC-HOCBF-sequence}, $\frac{\dot{a}_{1}(t)}{a_{1}(t)}$ is a time-varying auxiliary term to alleviate the conservativeness of control that small $k_{1}$ originally has, which shows the adaptivity of auxiliary terms to constant class $\kappa$ coefficients. 

% There is another type of adaptive CBFs called Relaxation-Adaptive Control Barrier Functions (RACBFs) in \cite{xiao2021adaptive}. The RACBF $b(\boldsymbol{x})$ is in the form:
% \begin{equation}
% \label{eq:RACBF}
% \psi_{0}(\boldsymbol{x},r(t))\coloneqq b(\boldsymbol{x})-r(t),
% \end{equation}
% where $r(t)\ge0$ is a relaxation that plays the similar role as Backup policy introduced in \cite{chen2021backup} {\color{red} How a relaxation is related to the backup policy?}. However, it is difficult for us to find the appropriate backup policy for controller of complicated dynamic system. Two main drawbacks affect the performance of RACBFs. {\color{red}wording} In the first place, $r(t)$ contracts the coverage of feasible space of states defined by $b(\boldsymbol{x})\ge0$, i.e., the distance $z(t)$ allowable for two vehicles is even smaller {\color{red}This should be larger} by $z(t)-l_{p}-r(t)\ge0$ because of the existence of non-negative $r(t)$. Secondly, the feasibility of solving QP with RACBF constraints is limited by the existence of upper bound of auxiliary input $\nu_{r}(t)$ related to $r(t)$ defined in Eq. (29) in \cite{xiao2021adaptive} {\color{red}What is $\nu_r$? you should make it self-contained.}. We can define the highest order {\color{red}what is this?} of $r(t)$ to be 2, then from \eqref{eq:SACC-HOCBF-sequence} normally we have
% \begin{equation}
% \label{eq:highest-order-RACBF}
% \begin{split}
% \psi_{2}(\boldsymbol{x},r(t),\dot{r}(t),\nu_{r}(t))\coloneqq -u(t)-\nu_{r}(t)\\
% +k_{1}(v_{p}-v(t)-\dot{r}(t))+k_{2}(v_{p}-v(t)-\dot{r}(t)\\
% +k_{1}(z(t)-l_{p}-r(t))\ge0, \nu_{r}(t)=\ddot{r}(t),
% \end{split}
% \end{equation}
% which sets the upper bound {\color{red}This is not clear} for $\nu_{r}(t)$ and there will easily be empty feasible set for $\nu_{r}(t)$ if the lower bound of $\nu_{r}(t)$ defined by constraint (31) in \cite{xiao2021adaptive} is too large. Compared to RACBFs, AVCBFs will neither contract the feasible space of states, nor set the upper bound for $\boldsymbol{\nu}$ (at least no upper bound for $\nu_{1}(t))$ as shown in the proof of Thm. \ref{thm:feasibility-guarantee} in Sec. \ref{subsec: optimal-control}, which shows the great benefits of AVCBFs in terms of safety and feasibility. 

% \subsection{HOCBFs for Auxiliary Coefficients}
\subsection{Adaptive HOCBFs for Safety:\ AVCBFs}
\label{sec:AVCBFs}

Motivated by the SACC example in Sec. \ref{subsec:SACC-problem}, given a function $b:\mathbb{R}^{n}\to\mathbb{R}$ with relative degree $m$ for system \eqref{eq:affine-control-system} and a time-varying vector $\boldsymbol{a}(t)\coloneqq [a_{1}(t),\dots,a_{m}(t)]^{T}$ with positive components called auxiliary variables, the key idea in converting a regular HOCBF into an adaptive
one without defining excessive constraints is to place one auxiliary variable in front of each function in \eqref{eq:sequence-f1} similar to \eqref{eq:SACC-AVBCBF-sequence}. 
As described in Sec. \ref{subsec:SACC-problem}, we only need to define HOCBFs for auxiliary variables to ensure each $a_{i}(t)>0, i \in \{1,...,m\}.$ To realize this, we need to define auxiliary systems that contain auxiliary states $\boldsymbol{\pi}_{i}(t)$ and inputs $\nu_{i}(t)$, through which systems we can extend each HOCBF to desired relative degree, just like $b(\boldsymbol{x})$ has relative degree $m$
with respect to the dynamics \eqref{eq:affine-control-system}. Consider $m$ auxiliary systems in the form 
\begin{equation}
\label{eq:virtual-system}
\dot{\boldsymbol{\pi}}_{i}=F_{i}(\boldsymbol{\pi}_{i})+G_{i}(\boldsymbol{\pi}_{i})\nu_{i}, i \in \{1,...,m\},
\end{equation}
where $\boldsymbol{\pi}_{i}(t)\coloneqq [\pi_{i,1}(t),\dots,\pi_{i,m+1-i}(t)]^{T}\in \mathbb{R}^{m+1-i}$ denotes an auxiliary state with $\pi_{i,j}(t)\in \mathbb{R}, j \in \{1,...,m+1-i\}.$ $\nu_{i}\in \mathbb{R}$ denotes an auxiliary input for \eqref{eq:virtual-system}, $F_{i}:\mathbb{R}^{m+1-i}\to\mathbb{R}^{m+1-i}$ and $G_{i}:\mathbb{R}^{m+1-i}\to\mathbb{R}^{m+1-i}$ are locally Lipschitz. For simplicity, we just build up the connection between an auxiliary variable and the system as $a_{i}(t)=\pi_{i,1}(t), \dot{\pi}_{i,1}(t)=\pi_{i,2}(t),\dots,\dot{\pi}_{i,m-i}(t)=\pi_{i,m+1-i}(t)$ and make $\dot{\pi}_{i,m+1-i}(t)=\nu_{i},$ then we can define many specific HOCBFs $h_{i}$ to enable $a_{i}(t)$ to be positive. 

Given a function $h_{i}:\mathbb{R}^{m+1-i}\to\mathbb{R},$ we can define a sequence of functions $\varphi_{i,j}:\mathbb{R}^{m+1-i}\to\mathbb{R}, i \in\{1,...,m\}, j \in\{1,...,m+1-i\}:$
\begin{equation}
\label{eq:virtual-HOCBFs}
\varphi_{i,j}(\boldsymbol{\pi}_{i})\coloneqq\dot{\varphi}_{i,j-1}(\boldsymbol{\pi}_{i})+\alpha_{i,j}(\varphi_{i,j-1}(\boldsymbol{\pi}_{i})),
\end{equation}
where $\varphi_{i,0}(\boldsymbol{\pi}_{i})\coloneqq h_{i}(\boldsymbol{\pi}_{i}),$ $\alpha_{i,j}(\cdot)$ are $(m+1-i-j)^{th}$ order differentiable class $\kappa$ functions. Sets $\mathcal{B}_{i,j}$ are defined as
\begin{equation}
\label{eq:virtual-sets}
\mathcal B_{i,j}\coloneqq \{\boldsymbol{\pi}_{i}\in\mathbb{R}^{m+1-i}:\varphi_{i,j}(\boldsymbol{\pi}_{i})>0\}, \ j\in \{0,...,m-i\}. 
\end{equation}
Let $\varphi_{i,j}(\boldsymbol{\pi}_{i}),\ j\in \{1,...,m+1-i\}$ and $\mathcal B_{i,j},\ j\in \{0,...,m-i\}$ be defined by \eqref{eq:virtual-HOCBFs} and \eqref{eq:virtual-sets} respectively. By Def. \ref{def:HOCBF}, a function $h_{i}:\mathbb{R}^{m+1-i}\to\mathbb{R}$ is a HOCBF with relative degree $m+1-i$ for system \eqref{eq:virtual-system} if there exist class $\kappa$ functions $\alpha_{i,j},\ j\in \{1,...,m+1-i\}$ as in \eqref{eq:virtual-HOCBFs} such that
\begin{small}
\begin{equation}
\label{eq:highest-SHOCBF}
\begin{split}
\sup_{\nu_{i}\in \mathbb{R}}[L_{F_{i}}^{m+1-i}h_{i}(\boldsymbol{\pi}_{i})+L_{G_{i}}L_{F_{i}}^{m-i}h_{i}(\boldsymbol{\pi}_{i})\nu_{i}+O_{i}(h_{i}(\boldsymbol{\pi}_{i}))\\
+ \alpha_{i,m+1-i}(\varphi_{i,m-i}(\boldsymbol{\pi}_{i}))] \ge \epsilon,
\end{split}
\end{equation}
\end{small}
$\forall\boldsymbol{\pi}_{i}\in \mathcal B_{i,0}\cap,...,\cap \mathcal B_{i,m-i}$. $O_{i}(\cdot)=\sum_{j=1}^{m-i}L_{F_{i}}^{j}(\alpha_{i,m-i}\circ\varphi_{i,m-1-i})(\boldsymbol{\pi}_{i}) $ where $\circ$ denotes the composition of functions. $\epsilon$ is a positive constant which can be infinitely small. 

\begin{remark}
\label{rem:safety-guarantee-2}
If $h_{i}(\boldsymbol{\pi}_{i})$ is a HOCBF illustrated above and $\boldsymbol{\pi}_{i}(0) \in \mathcal {B}_{i,0}\cap \dots \cap \mathcal {B}_{i,m-i},$ then satisfying constraint in \eqref{eq:highest-SHOCBF} is equivalent to making $\varphi_{i,m+1-i}(\boldsymbol{\pi}_{i}(t))\ge \epsilon>0, \forall t\ge 0.$ Based on
\eqref{eq:virtual-HOCBFs}, since $\boldsymbol{\pi}_{i}(0) \in \mathcal {B}_{i,m-i}$ (i.e., $\varphi_{i,m-i}(\boldsymbol{\pi}_{i}(0))>0),$ then we have $\varphi_{i,m-i}(\boldsymbol{\pi}_{i}(t))>0$ (If there exists a $t_{1}\in (0,t_{2}]$, which makes $\varphi_{i,m-i}(\boldsymbol{\pi}_{i}(t_{1}))=0,$ then we have $\dot{\varphi}_{i,m-i}((\boldsymbol{\pi}_{i}(t_{1}))>0\Leftrightarrow \varphi_{i,m-i}(\boldsymbol{\pi}_{i}(t_{1}^{-}))\varphi_{i,m-i}(\boldsymbol{\pi}_{i}(t_{1}^{+}))<0,$ which is against the definition of $\alpha_{i,m+1-i}(\cdot),$ therefore $\forall t_{1}>0, \varphi_{i,m-i}(\boldsymbol{\pi}_{i}(t_{1}))>0,$ note that $t_{1}^{-},t_{1}^{+}$ denote the left and right limit). Based on \eqref{eq:virtual-HOCBFs}, since $\boldsymbol{\pi}_{i}(0) \in \mathcal {B}_{i,m-1-i},$ then similarly we have $\varphi_{i,m-1-i}(\boldsymbol{\pi}_{i}(t))>0,\forall t\ge 0.$ Repeatedly, we have $\varphi_{i,0}(\boldsymbol{\pi}_{i}(t))>0,\forall t\ge 0,$ therefore the sets $\mathcal {B}_{i,0},\dots,\mathcal {B}_{i,m-i}$ are forward invariant.
\end{remark}

For simplicity, we can make $h_{i}(\boldsymbol{\pi}_{i})=\pi_{i,1}(t)=a_{i}(t).$ Based on Rem. \ref{rem:safety-guarantee-2}, each $a_{i}(t)$ will be positive.

The remaining question is how to define an adaptive HOCBF to guarantee $b(\boldsymbol{x})\ge0$ with the assistance of auxiliary variables. Let $\boldsymbol{\Pi}(t)\coloneqq [\boldsymbol{\pi}_{1}(t),\dots,\boldsymbol{\pi}_{m}(t)]^{T}$ and $\boldsymbol{\nu}\coloneqq [\nu_{1},\dots,\nu_{m}]^{T}$ denote the auxiliary states and control inputs of system \eqref{eq:virtual-system}. We can define a sequence of functions 
\begin{small}
\begin{equation}
\label{eq:AVBCBF-sequence}
\begin{split}
&\psi_{0}(\boldsymbol{x},\boldsymbol{\Pi}(t))\coloneqq a_{1}(t)b(\boldsymbol{x}),\\
&\psi_{i}(\boldsymbol{x},\boldsymbol{\Pi}(t))\coloneqq a_{i+1}(t)(\dot{\psi}_{i-1}(\boldsymbol{x},\boldsymbol{\Pi}(t))+\alpha_{i}(\psi_{i-1}(\boldsymbol{x},\boldsymbol{\Pi}(t)))),
\end{split}
\end{equation}
\end{small}
where $i \in \{1,...,m-1\}, \psi_{m}(\boldsymbol{x},\boldsymbol{\Pi}(t))\coloneqq \dot{\psi}_{m-1}(\boldsymbol{x},\boldsymbol{\Pi}(t))+\alpha_{m}(\psi_{m-1}(\boldsymbol{x},\boldsymbol{\Pi}(t))).$ We further define a sequence of sets $\mathcal{C}_{i}$ associated with \eqref{eq:AVBCBF-sequence} in the form 
\begin{equation}
\label{eq:AVBCBF-set}
\begin{split}
\mathcal C_{i}\coloneqq \{(\boldsymbol{x},\boldsymbol{\Pi}(t)) \in \mathbb{R}^{n} \times \mathbb{R}^{m}:\psi_{i}(\boldsymbol{x},\boldsymbol{\Pi}(t))\ge 0\}, 
\end{split}
\end{equation}
where $i \in \{0,...,m-1\}.$
Since $a_{i}(t)$ is a HOCBF with relative degree $m+1-i$ for \eqref{eq:virtual-system}, based on \eqref{eq:highest-SHOCBF}, we define a constraint set $\mathcal{U}_{\boldsymbol{a}}$ for $\boldsymbol{\nu}$ as 
\begin{small}
\begin{equation}
\label{eq:constraint-up}
\begin{split}
\mathcal{U}_{\boldsymbol{a}}(\boldsymbol{\Pi})\coloneqq \{\boldsymbol{\nu}\in\mathbb{R}^{m}:   L_{F_{i}}^{m+1-i}a_{i}+[L_{G_{i}}L_{F_{i}}^{m-i}a_{i}]\nu_{i}\\
+O_{i}(a_{i})+ \alpha_{i,m+1-i}(\varphi_{i,m-i}(a_{i})) \ge \epsilon, i\in \{1,\dots,m\}\},
\end{split}
\end{equation}
\end{small}
where $\varphi_{i,m-i}(\cdot)$ is defined similar to \eqref{eq:virtual-HOCBFs} and $a_{i}(t)$ is ensured positive. $\epsilon$ is a positive constant which can be infinitely small. 

\begin{definition}[AVCBF]
\label{def:AVBCBF}
Let $\psi_{i}(\boldsymbol{x},\boldsymbol{\Pi}(t)),\ i\in \{1,...,m\}$ be defined by \eqref{eq:AVBCBF-sequence} and $\mathcal C_{i},\ i\in \{0,...,m-1\}$ be defined by \eqref{eq:AVBCBF-set}. A function $b(\boldsymbol{x}):\mathbb{R}^{n}\to\mathbb{R}$ is an Auxiliary-Variable Adaptive Control Barrier Function (AVCBF) with relative degree $m$ for system \eqref{eq:affine-control-system} if every $a_{i}(t),i\in \{1,...,m\}$ is a
HOCBF with relative degree $m+1-i$ for the auxiliary system
\eqref{eq:virtual-system}, and there exist $(m-j)^{th}$ order differentiable class $\kappa$ functions $\alpha_{j},j\in \{1,...,m-1\}$
and a class $\kappa$ functions $\alpha_{m}$ s.t.
\begin{small}
\begin{equation}
\label{eq:highest-AVBCBF}
\begin{split}
\sup_{\boldsymbol{u}\in \mathcal{U},\boldsymbol{\nu}\in \mathcal{U}_{\boldsymbol{a}}}[\sum_{j=2}^{m-1}[(\prod_{k=j+1}^{m}a_{k})\frac{\psi_{j-1}}{a_{j}}\nu_{j}] + \frac{\psi_{m-1}}{a_{m}}\nu_{m} \\ +(\prod_{i=2}^{m}a_{i})b(\boldsymbol{x})\nu_{1} +(\prod_{i=1}^{m}a_{i})(L_{f}^{m}b(\boldsymbol{x})+L_{g}L_{f}^{m-1}b(\boldsymbol{x})\boldsymbol{u})\\+R(b(\boldsymbol{x}),\boldsymbol{\Pi})
+ \alpha_{m}(\psi_{m-1})] \ge 0,
\end{split}
\end{equation}
\end{small}
$\forall (\boldsymbol{x},\boldsymbol{\Pi})\in \mathcal C_{0}\cap,...,\cap \mathcal C_{m-1}$ and each $a_{i}>0, i\in\{1,\dots,m\}.$ In \eqref{eq:highest-AVBCBF}, $R(b(\boldsymbol{x}),\boldsymbol{\Pi})$ denotes the remaining Lie derivative terms of $b(\boldsymbol{x})$ (or $\boldsymbol{\Pi}$) along $f$ (or $F_{i},i\in\{1,\dots,m\}$) with degree less than $m$ (or $m+1-i$), which is similar to the form of $O(\cdot )$ in \eqref{eq:highest-HOCBF}.
\end{definition}

\begin{theorem}
\label{thm:safety-guarantee-3}
Given an AVCBF $b(\boldsymbol{x})$ from Def. \ref{def:AVBCBF} with corresponding sets $\mathcal{C}_{0}, \dots,\mathcal {C}_{m-1}$ defined by \eqref{eq:AVBCBF-set}, if $(\boldsymbol{x}(0),\boldsymbol{\Pi}(0)) \in \mathcal {C}_{0}\cap \dots \cap \mathcal {C}_{m-1},$ then if there exists solution of Lipschitz controller $(\boldsymbol{u},\boldsymbol{\nu})$ that satisfies the constraint in \eqref{eq:highest-AVBCBF} and also ensures $(\boldsymbol{x},\boldsymbol{\Pi})\in \mathcal {C}_{m-1}$ for all $t\ge 0,$ then $\mathcal {C}_{0}\cap \dots \cap \mathcal {C}_{m-1}$ will be rendered forward invariant for system \eqref{eq:affine-control-system}, $i.e., (\boldsymbol{x},\boldsymbol{\Pi}) \in \mathcal {C}_{0}\cap \dots \cap \mathcal {C}_{m-1}, \forall t\ge 0.$ Moreover, $b(\boldsymbol{x})\ge 0$ is ensured for all $t\ge 0.$
\end{theorem}

\begin{proof}
If $b(\boldsymbol{x})$ is an AVCBF that is $m^{th}$ order differentiable, then satisfying constraint in \eqref{eq:highest-AVBCBF} while ensuring $(\boldsymbol{x},\boldsymbol{\Pi})\in \mathcal {C}_{m-1}$ for all $t\ge 0$ is equivalent to make $\psi_{m-1}(\boldsymbol{x},\boldsymbol{\Pi})\ge 0, \forall t\ge 0.$ Since $a_{m}(t)>0$, we have $\frac{\psi_{m-1}(\boldsymbol{x},\boldsymbol{\Pi})}{a_{m}(t)}\ge 0.$ Based on
\eqref{eq:AVBCBF-sequence}, since $(\boldsymbol{x}(0),\boldsymbol{\Pi}(0)) \in \mathcal {C}_{m-2}$ (i.e., $\frac{\psi_{m-2}(\boldsymbol{x}(0),\boldsymbol{\Pi}(0))}{a_{m-1}(0)}\ge 0),a_{m-1}(t)>0,$ then we have $\psi_{m-2}(\boldsymbol{x},\boldsymbol{\Pi})\ge 0$ (The proof of this is similar to the proof in Rem. \ref{rem:safety-guarantee-2}), and also $\frac{\psi_{m-2}(\boldsymbol{x},\boldsymbol{\Pi})}{a_{m-1}(t)}\ge 0.$ Based on \eqref{eq:AVBCBF-sequence}, since $(\boldsymbol{x}(0),\boldsymbol{\Pi}(0)) \in \mathcal {C}_{m-3},a_{m-2}(t)>0$ then similarly we have $\psi_{m-3}(\boldsymbol{x},\boldsymbol{\Pi})\ge 0$ and $\frac{\psi_{m-3}(\boldsymbol{x},\boldsymbol{\Pi})}{a_{m-2}(t)}\ge 0,\forall t\ge 0.$ Repeatedly, we have $\psi_{0}(\boldsymbol{x},\boldsymbol{\Pi})\ge 0$ and $\frac{\psi_{0}(\boldsymbol{x},\boldsymbol{\Pi})}{a_{1}(t)}\ge 0,\forall t\ge 0.$ Therefore the sets $\mathcal {C}_{0},\dots,\mathcal {C}_{m-1}$ are forward invariant and $b(\boldsymbol{x})=\frac{\psi_{0}(\boldsymbol{x},\boldsymbol{\Pi})}{a_{1}(t)}\ge 0$ is ensured for all $t\ge 0$.
\end{proof}
Based on Thm. \ref{thm:safety-guarantee-3}, the safety regarding $b(\boldsymbol{x})=\frac{\psi_{0}(\boldsymbol{x},\boldsymbol{\Pi})}{a_{1}(t)}\ge 0$ is guaranteed.

\begin{remark}[Limitation of Approaches with Auxiliary Inputs]
\label{rem: PACBF-AVBCBF} 
Ensuring the satisfaction of the $i^{th}$ order AVCBF constraint as shown in \eqref{eq:AVBCBF-set} when $i\in\{1,\dots,m-1\},$ i.e., $\psi_{i}(\boldsymbol{x},\boldsymbol{\Pi})\ge 0$ will guarantee $\psi_{i-1}(\boldsymbol{x},\boldsymbol{\Pi})\ge 0$ based on the proof of Thm. \ref{thm:safety-guarantee-3}, which theoretically outperforms PACBF. However, both approaches can not ensure satisfying $\psi_{m}(\boldsymbol{x},\boldsymbol{\Pi})\ge 0$ will guarantee $\psi_{m-1}(\boldsymbol{x},\boldsymbol{\Pi})\ge 0$ since the growth of $\boldsymbol{\nu}_{i}$ is unbounded. Therefore in Thm. \ref{thm:safety-guarantee-3}, $(\boldsymbol{x},\boldsymbol{\Pi})\in \mathcal {C}_{m-1}$ for all $t\ge 0$ also needs to be satisfied to guarantee the forward invariance of the intersection of sets. 
\end{remark}

\subsection{Optimal Control with AVCBFs}
\label{subsec: optimal-control}
Consider an optimal control problem as
\begin{small}
\begin{equation}
\label{eq:cost-function-1}
\begin{split}
 \min_{\boldsymbol{u}} \int_{0}^{T} 
 D(\left \| \boldsymbol{u} \right \| )dt,
\end{split}
\end{equation}
\end{small}
where $\left \| \cdot \right \|$ denotes the 2-norm of a vector, $D(\cdot)$ is a strictly increasing function of its argument and $T>0$ denotes the ending time. Since we need to introduce auxiliary inputs $v_{i}$ to enhance the feasibility of optimization, we should reformulate the cost in \eqref{eq:cost-function-1} as
\begin{small}
\begin{equation}
\label{eq:cost-function-2}
\begin{split}
 \min_{\boldsymbol{u},\boldsymbol{\nu}} \int_{0}^{T} 
 [D(\left \| \boldsymbol{u} \right \| )+\sum_{i=1}^{m}W_{i}(\nu_{i}-a_{i,w})^{2}]dt.
\end{split}
\end{equation}
\end{small}
In \eqref{eq:cost-function-2}, $W_{i}$ is a positive scalar and $a_{i,w}\in \mathbb{R}$ is the scalar to which we hope each auxiliary input $\nu_{i}$ converges. Both are chosen to tune the performance of the controller. We can formulate the CLFs, HOCBFs and AVCBFs introduced in Def. \ref{def:control-l-f}, Sec. \ref{sec:AVCBFs} and Def. \ref{def:AVBCBF} as constraints of the QP with cost function \eqref{eq:cost-function-2} to realize safety-critical control. Next we will show AVCBFs can be used to enhance the feasibility of solving QP compared with classical HOCBFs in Def. \ref{def:HOCBF}.

In auxiliary system \eqref{eq:virtual-system}, if we define $a_{i}(t)=\pi_{i,1}(t)=1, \dot{\pi}_{i,1}(t)=\dot{\pi}_{i,2}(t)=\cdots=\dot{\pi}_{i,m+1-i}(t)=0,$ then the way we construct functions and sets in \eqref{eq:virtual-HOCBFs} and \eqref{eq:virtual-sets} are exactly the same as \eqref{eq:sequence-f1} and \eqref{eq:sequence-set1}, which means classical HOCBF is in fact one specific case of AVCBF. Assume that the highest order HOCBF constraint \eqref{eq:highest-HOCBF} conflicts with control input constraints \eqref{eq:control-constraint} at $t=t_{b},$ i.e., we can not find a feasible controller $u(t_{b})$ to satisfy \eqref{eq:highest-HOCBF} and \eqref{eq:control-constraint}. Instead, starting from a time slot $t=t_{a}$ which is just before $t=t_{b}$ ($t_{b}-t_{a}=\varepsilon$ where $\varepsilon$ is an infinitely small positive value), we exchange the control framework of classical HOCBF into AVCBF instantly. Suppose we can find appropriate hyperparameters to ensure two constraints in \eqref{eq:constraint-up} and \eqref{eq:highest-AVBCBF}
% \begin{small}
% \begin{equation}
% \label{eq:constraint-fea-12}
% \begin{split}
%  \nu_{i}
%   > \frac{-L_{F_{i}}^{m+1-i}a_{i}-O_{i}(a_{i})-\alpha_{i,m+1-i}(\varphi_{i,m-i}(a_{i}))}{L_{G_{i}}L_{F_{i}}^{m-i}a_{i}},\\
%   \sum_{j=2}^{m-1}[(\prod_{k=j+1}^{m}a_{k})\frac{\psi_{j-1}}{a_{j}}\nu_{j}] + \frac{\psi_{m-1}}{a_{m}}\nu_{m} +(\prod_{i=2}^{m}a_{i})b(\boldsymbol{x})\nu_{1} \\ \ge -(\prod_{i=1}^{m}a_{i})(L_{f}^{m}b(\boldsymbol{x})+L_{g}L_{f}^{m-1}b(\boldsymbol{x})\boldsymbol{u})-R(b(\boldsymbol{x}),\boldsymbol{\Pi}) \\
% - \alpha_{m}(\psi_{m-1}),  i\in \{1,\dots,m\}
% \end{split}
% \end{equation}
% \end{small}
are satisfied given $\boldsymbol{u}$ constrained by \eqref{eq:control-constraint} at $t_{b},$ then there exists solution $\boldsymbol{u}(t_{b})$ for the optimal control problem and the feasibility of solving QP is enhanced. Relying on AVCBF, We can discretize the whole time period $[0,T]$ into several small time intervals like $[t_{a},t_{b}]$ to maximize the feasibility of solving QP under safety constraints, which calls for the development of automatic parameter-tuning techniques in future.
% \begin{theorem}
% \label{thm:feasibility-guarantee}
% Given an AVCBF $b(\boldsymbol{x})$ from Def. \ref{def:AVBCBF} with corresponding sets $\mathcal{C}_{0}, \dots,\mathcal {C}_{m-1}$ defined by \eqref{eq:AVBCBF-set}, if $(\boldsymbol{x}(0),\boldsymbol{\Pi}(0)) \in \mathcal {C}_{0}\cap \dots \cap \mathcal {C}_{m-1}$ and $L_{G_{i}}L_{F_{i}}^{m-i}a_{i}>0,a_{i}(t)>0, i\in\{1,\dots,m\}$ in \eqref{eq:constraint-up}, then if there exists solution of Lipschitz controller $(\boldsymbol{u},\boldsymbol{\nu})$ that satisfies the constraint in \eqref{eq:highest-AVBCBF} and also ensures $\psi_{0}>0,\dots,\psi_{s}>0,s\in \{0,\dots,m-1\}$ in \eqref{eq:AVBCBF-set}, then the QP with cost function \eqref{eq:cost-function-2} and constraints \eqref{eq:control-constraint},\eqref{eq:AVBCBF-set}-\eqref{eq:highest-AVBCBF} is guranteed to be feasible.
% \end{theorem}

% \begin{proof}
% Rewrite the constraint \eqref{eq:constraint-up} as 
% \begin{equation}
% \label{eq:constraint-fea-1}
% \begin{split}
%  \nu_{i}
%   > \frac{-L_{F_{i}}^{m+1-i}a_{i}-O_{i}(a_{i})-\alpha_{i,m+1-i}(\varphi_{i,m-i}(a_{i}))}{L_{G_{i}}L_{F_{i}}^{m-i}a_{i}},
% \end{split}
% \end{equation}
% where $i\in \{1,\dots,m\}.$ Rewrite the constraint \eqref{eq:highest-AVBCBF} as
% \begin{equation}
% \label{eq:constraint-fea-2}
% \begin{split}
% \sum_{j=2}^{m-1}[(\prod_{k=j+1}^{m}a_{k})\frac{\psi_{j-1}}{a_{j}}\nu_{j}] + \frac{\psi_{m-1}}{a_{m}}\nu_{m} +(\prod_{i=2}^{m}a_{i})b(\boldsymbol{x})\nu_{1} \\ \ge -(\prod_{i=1}^{m}a_{i})(L_{f}^{m}b(\boldsymbol{x})+L_{g}L_{f}^{m-1}b(\boldsymbol{x})\boldsymbol{u})-R(b(\boldsymbol{x}),\boldsymbol{\Pi}) \\
% - \alpha_{m}(\psi_{m-1}),  i\in \{1,\dots,m\}.
% \end{split}
% \end{equation}
% Since $L_{G_{i}}L_{F_{i}}^{m-i}a_{i}>0$ in \eqref{eq:constraint-fea-1}, $\psi_{0}>0,\dots,\psi_{s}>0,s\in \{0,\dots,m-1\}$ in \eqref{eq:constraint-fea-2} and $a_{1}>0,\dots,a_{m}>0,$ we have $(\prod_{i=2}^{m}a_{i})b(\boldsymbol{x})>0,(\prod_{k=j+1}^{m}a_{k})\frac{\psi_{j-1}}{a_{j}}\nu_{j}>0,j\in \{2,\dots,s\}$ are always positive, 
% then there always exist large enough $\nu_{1},\dots,\nu_{s}$ satisfying constraints above {\color{red}you are assuming a very specific (13).} (the upper bounds of $\nu_{1},\dots,\nu_{s}$ are unlimited), hence the feasibility of QP with cost function \eqref{eq:cost-function-2} and constraints \eqref{eq:control-constraint},\eqref{eq:AVBCBF-set}-\eqref{eq:highest-AVBCBF} is guaranteed.  {\color{red}Control limitations (2) are the most critical factor in the feasibility. You completely ignore this. The proof is very sloppy.}
% \end{proof}

Besides safety and feasibility, another benefit of using AVCBFs is that the conservativeness of the control strategy can also be ameliorated. For example, from \eqref{eq:AVBCBF-sequence}, we can rewrite $\psi_{i}(\boldsymbol{x},\boldsymbol{\Pi})\ge 0$ as
\begin{equation}
\label{eq:AVCBF-rewrite}
\begin{split}
\dot{\phi}_{i-1}(\boldsymbol{x},\boldsymbol{\Pi})+k_{i}(1+\frac{\dot{a}_{i}(t)}{k_{i}a_{i}(t)}) \phi_{i-1}(\boldsymbol{x},\boldsymbol{\Pi})\ge0,
\end{split}
\end{equation}
where $\phi_{i-1}(\boldsymbol{x},\boldsymbol{\Pi})=\frac{\psi_{i-1}(\boldsymbol{x},\boldsymbol{\Pi})}{a_{i}(t)},\alpha_{i}(\psi_{i-1}(\boldsymbol{x},\boldsymbol{\Pi}))=k_{i}a_{i}(t)\phi_{i-1}(\boldsymbol{x},\boldsymbol{\Pi}), k_{i}>0, i\in \{1,\dots,m\}.$ Similar to PACBFs, we require $1+\frac{\dot{a}_{i}(t)}{k_{i}a_{i}(t)}\ge0,$ which gives us $\dot{a}_{i}(t)+k_{i}a_{i}(t)\ge0.$
The term $\frac{\dot{a}_{i}(t)}{a_{i}(t)}$ can be adjusted adaptable  to ameliorate the conservativeness of control strategy that $k_{i}\phi_{i-1}(\boldsymbol{x},\boldsymbol{\Pi})$ may have, i.e., the ego vehicle can brake earlier or later given time-varying control constraint $\boldsymbol{u}_{min}(t)\le \boldsymbol{u} \le\boldsymbol{u}_{max}(t),$ which confirms the adaptivity of AVCBFs to control constraint and conservativeness of control strategy. 

\begin{remark}[Parameter-Tuning for AVCBFs]
\label{rem: parameter-tuning}
Based on the analysis of \eqref{eq:AVCBF-rewrite}, we require $\dot{a}_{i}(t)+k_{i}a_{i}(t)\ge0.$ If we define first order HOCBF constraint for $a_{i}(t)>0$ as $\dot{a}_{i}(t)+l_{i}a_{i}(t)\ge0,$ we should choose hyperparameter $l_{i}\le k_{i}$ to guarantee $\dot{a}_{i}(t)+k_{i}a_{i}(t)\ge\dot{a}_{i}(t)+l_{i}a_{i}(t)\ge 0.$ For simplicity, we can use $l_{i}=k_{i}.$ In cost function \eqref{eq:cost-function-2}, we can tune hyperparameters $W_{i}$ and $a_{i,w}$ to adjust the corresponding ratio $\frac{\dot{a}_{i}(t)}{a_{i}(t)}$ to change the performance of the optimal controller.
\end{remark}

\begin{remark}
\label{rem: sufficient-con}
Note that the satisfaction of the constraint in \eqref{eq:highest-AVBCBF} is a sufficient condition for the satisfaction of the original constraint $\psi_{0}(\boldsymbol{x},\boldsymbol{\Pi})>0,$ it is not necessary to introduce auxiliary variables as many as from $a_{1}(t)$ to $a_{m}(t),$ which allows us to choose an appropriate
number of auxiliary variables for the AVCBF constraints to reduce the complexity. In other words, the number of auxiliary variables can be less than or equal to the relative degree $m$.
\end{remark}


\section{The VI Approximation as $N\rightarrow\infty$}\label{sec:theoretical_tool}

Our first set of theoretical results presented in this chapter makes the connection between the NE and a solution of \eqref{eq:mfg_vi_statement} explicit.
We will show that solutions of \eqref{eq:mfg_vi_statement} form good approximations of the true NE in the $N$-player game if $N$ is large.

\begin{remark}[Existence and Uniqueness of MF-NE]
\label{remark:vi_existence}
Let $\vecF:\Delta_\setA \rightarrow [0,1]^K$ be a continuous function.
Then $\vecF$ has at least one MF-NE $\vecpi^*$, and the set of MF-NE is compact.
Furthermore, if $\vecF$ is also $\lambda$-strongly monotone for some $\lambda > 0$, then the MF-NE is unique.
This can be seen as follows.
The MF-NE corresponds to solutions of the VI: $\forall \vecpi \in \Delta_\setA, \vecF(\vecpi^*)^\top (\vecpi^* - \vecpi) \geq 0$.
The domain set $\Delta_\setA$ is compact and convex, and the assumption that $\vecF$ is continuous yields the existence of a solution using Corollary~2.2.5 of \cite{facchinei2003finite}.
For uniqueness in the case of strong monotonicity, see Theorem~2.3.3 of \cite{facchinei2003finite}.
\end{remark}








The following theorem shows that the solution of \eqref{eq:mfg_vi_statement}, when deployed by all players, is a $\mathcal{O}\left(\sfrac{1}{\sqrt{N}}\right)$ solution of the $N$-player game.
Therefore, the MF-NE solution will be an arbitrarily good approximation of the true NE when $N\rightarrow\infty$, and the bias introduced by studying the $N$-player game can be explicitly quantified.

\begin{theorem}\label{theorem:mfg_ne}
    Let $\vecF$ be $L$-Lipschitz, $\delta\geq 0$ arbitrary, and let $\vecpi^*$ be a $\delta$-MF-NE.
Then, the strategy profile $(\vecpi^*, \ldots, \vecpi^*) \in \Delta_\setA^N$ is a $\mathcal{O}\left(\delta + \frac{L}{\sqrt{N}}\right)$-NE of the $N$-player SMFG. 
\end{theorem}

\begin{proof}
Firstly, define the independent random variables $a^j \sim \vecpi^*$ for all $j\in\setN$ for a $\delta$-MF-NE $\vecpi^* \in \Delta_\setA$.
Define the random variable $\widehat{\vecmu} := \sfrac{1}{N} \sum_{j=1}^N \vece_{a^j}$, which is the empirical distribution of players over actions in a single round of an SMFG.
The proof will proceed by formally proving that if $N$ is large enough, then $\Exop[\vecF(\widehat{\vecmu})] \approx \vecF(\vecpi^*)$ and $a^j$ is approximately independent from $\vecF(\widehat{\vecmu})$.

It is straightforward that $\Exop \left[ \widehat{\vecmu}  \right] = \vecpi^*$.
Furthermore, by independence of the random vectors $\vece_{a^j}$, we have
\begin{align*}
    \Exop \left[ \left\| \widehat{\vecmu} - \vecpi^* \right\|_2  \right] \leq
    &\sqrt{\Exop \Big[ \Big\| \frac{1}{N} \sum_{j=1}^N \vece_{a^j} - \vecpi^* \Big\|_2^2  \Big]} 
    \leq  \sqrt{\frac{1}{N^2} \sum_{j=1}^N \Exop \left[  \left\| \vece_{a^j} - \vecpi^* \right\|_2^2 \right]} \leq \frac{2}{\sqrt{N}}.
\end{align*}
Hence, as $\vecF$ is $L$-Lipschitz, we have that
\begin{align}\label{eq:theorem1:ineq2}
\|\Exop[\vecF(\widehat{\vecmu})|a_j\sim \vecpi^*] - \vecF(\vecpi^*)\|_2 \leq \Exop[\|\vecF(\widehat{\vecmu}) - \vecF(\vecpi^*) \|_2] \leq \frac{2L}{\sqrt{N}}.
\end{align}

Now let $i\in \setN$ be arbitrary, and let $\vecpi' \in \Delta_\setA$ be any distribution over actions that satisfies $V^i(\vecpi', \vecpi^{*, -i}) = \max_{\vecpi} V^i(\vecpi, \vecpi^{*, -i})$.
We also define the quantities
\begin{align*}
    \overline{\vecF}_1 = \Exop\left[\vecF(\widehat{\vecmu}) \middle| a^j \sim \vecpi^*, \forall j\in\setN \right], \qquad
    \overline{\vecF}_2 = \Exop\left[\vecF(\widehat{\vecmu}) \middle| a^j \sim \vecpi^* \text{ for } \forall i \neq j, \quad a^i \sim \vecpi' \right].
\end{align*}
We will bound $V^i(\vecpi', \vecpi^{*, -i}) - V^i(\vecpi^*, \vecpi^{*, -i})$.
Combining Lemma~\ref{lemma:technical_bound_1N} and the inequality \eqref{eq:theorem1:ineq2}, we observe
\begin{align*}
    |V^i(\vecpi', \vecpi^{*, -i}) - \vecpi'^\top\vecF(\vecpi^*)| \leq & |V^i(\vecpi', \vecpi^{*, -i}) -  \vecpi'^\top \overline{\vecF}_2 |
    + \Big|\vecpi'^\top \overline{\vecF}_2 - \vecpi'^\top\vecF\Big(\frac{N-1}{N}\vecpi^* + \frac{1}{N} \vecpi'\Big)\Big| \\
    &+ \Big|\vecpi'^\top\vecF\Big(\frac{N-1}{N}\vecpi^* + \frac{1}{N} \vecpi'\Big) - \vecpi'^\top\vecF(\vecpi^*)\Big| \\
    \leq & \frac{L\sqrt{2}}{N} + \frac{2L}{\sqrt{N}} + \frac{2L}{N},
\end{align*}
since $\vecF$ is $L$-Lipschitz.
Likewise, using Lemma~\ref{lemma:technical_bound_1N} once again, we have
\begin{align*}
    |V^i(\vecpi^*, \vecpi^{*, -i}) - \vecpi^{*,\top}\vecF(\vecpi^*)| &\leq |V^i(\vecpi^*, \vecpi^{*, -i}) - \vecpi^{*,\top}\overline{\vecF}_1| + |\vecpi^{*,\top} \overline{\vecF}_1 - \vecpi^{*,\top}\vecF(\vecpi^*)|  \\
    &\leq \frac{L\sqrt{2}}{N} + \frac{2L}{\sqrt{N}}.
\end{align*}
Finally, using the definition of a $\delta$-MF-NE, it holds that
\begin{align*}
V^i(\vecpi', \vecpi^{*, -i}) - V^i(\vecpi^*, \vecpi^{*, -i}) \leq &\vecF(\vecpi^*)^\top (\vecpi' - \vecpi^*) +|V^i(\vecpi^*, \vecpi^{*, -i}) -  \vecpi^{*,\top}\vecF(\vecpi^*)| \\
    & + |V^i(\vecpi', \vecpi^{*, -i}) - \vecpi'^\top\vecF(\vecpi^*)| \\
\leq &\delta + \frac{L(2\sqrt{2} + 4)}{N} + \frac{4L}{\sqrt{N}}.
\end{align*}
\end{proof}

Recall our goal in the context of the $N$-player SMFG is to find policies $\{\vecpi^j\}_{j=1}^N$ with low exploitability $\setE_{\text{exp}}^i$ for all $i$. Theorem~\ref{theorem:mfg_ne} considers that agents adopt the same policy $\vecpi^*$ from solving the VI corresponding to operator $\vecF$ to obtain a low-exploitability approximation. 
We will generalize this result to explicitly bound $\setE_{\text{exp}}^i$ when agent policies can also deviate and when agents can employ regularization.
In our algorithms, regularizing the MF-VI problem will play a crucial role in the IL setting, as it will prevent the policies of agents from diverging when there is no centralized controller synchronizing the policies of agents.
For this reason, our algorithms in the later sections will introduce extraneous regularization to \eqref{eq:mfg_vi_statement} and instead solve the following $\tau$-Tikhonov regularized VI problem:
\begin{align}\label{eq:mfg_rvi_statement}
    \text{Find } \vecpi^* \in \Delta_\setA \text{ s.t. } (\vecF - \tau \matI)(\vecpi^*)^\top (\vecpi^* - \vecpi) \geq 0, \forall \vecpi\in \Delta_\setA. \tag{MF-RVI}
\end{align}

The following theorem quantifies the additional exploitability incurred in the $N$-player game due to (1) extraneous regularization, which is useful for algorithm design, and (2) deviations in agent policies from the MF-NE, potentially due to stochasticity in learning.
Theorem~\ref{theorem:mfgrvi_and_explotability} will be a more useful result later in a learning setting since the learned policies $(\vecpi^1, \ldots, \vecpi^N)$ will only approximate the solution of \eqref{eq:mfg_rvi_statement}.

\begin{theorem}
\label{theorem:mfgrvi_and_explotability}
Let $\vecF$ be monotone, $L$-Lipschitz.
Let $\vecpi_{\tau}^* \in \Delta_\setA$ be the (unique) MF-NE of the regularized map $\vecF - \tau \matI$.
Let $\vecpi^1, \ldots, \vecpi^N \in \Delta_\setA$ be such that $\|\vecpi^i - \vecpi_\tau^*\|_2 \leq \delta$ for all $i$, then it holds that $\setE^i_{\text{exp}}(\{\vecpi^j\}_{j=1}^N) = \mathcal{O}(\tau + \delta + \sfrac{1}{\sqrt{N}})$ for all $i\in\setN$.
\end{theorem}
\begin{proof}
By the Lipschitz continuity of exploitability (Lemma~\ref{lemma:phi_lipschitz}), we have
\begin{align}
    \setE^i_{\text{exp}}(\{ \vecpi^j \}_{j=1}^N) \leq &\setE^i_{\text{exp}}(\{ \vecpi_\tau^* \}_{j=1}^N) + \sqrt{K} \| \vecpi^i - \vecpi_\tau^* \|_2 + \sum_{j\neq i} \frac{4L\sqrt{2K}}{N} \| \vecpi^j - \vecpi_\tau^*\|_2 \notag \\
        \leq & \setE^i_{\text{exp}}(\{ \vecpi_\tau^* \}_{j=1}^N) + \delta \sqrt{K} + 4L\sqrt{2K} \delta. \label{eq:theorem:rviexpbound}
\end{align}
Since $\vecpi_\tau^*$ is the unique MF-NE of the operator $\vecF - \tau \matI$, it holds by definition that
\begin{align*}
    (\vecF - \tau \matI)(\vecpi_{\tau})^\top \vecpi_{\tau} &\geq (\vecF - \tau \matI)(\vecpi_{\tau})^\top \vecpi.
\end{align*}
Organizing both sides, we have
\begin{align*}
    \vecF(\vecpi_{\tau})^\top\vecpi_{\tau} &\geq \vecF(\vecpi_{\tau})^\top \vecpi + \tau \vecpi_{\tau}^\top (\vecpi_{\tau} - \vecpi) \geq \vecF(\vecpi_{\tau})^\top \vecpi - 2\tau,
\end{align*}
as $|\vecpi_{\tau}^\top (\vecpi_{\tau} - \vecpi)| \leq \|\vecpi_{\tau}\|_2 \|\vecpi_{\tau} - \vecpi\|_2 \leq 2$.
Then, $\vecpi^*_{\tau}$ is a $2\tau$-MF-NE for the operator $\vecF$, and by Theorem~\ref{theorem:mfg_ne}, $\setE^i_{\text{exp}}(\{ \vecpi_\tau^* \}_{j=1}^N) \leq \mathcal{O}(\tau + \sfrac{1}{\sqrt{N}})$.
Placing this in~(\ref{eq:theorem:rviexpbound}) proves the theorem.
\end{proof}






To summarize, this section presented key approximation results linking the solutions of \eqref{eq:mfg_vi_statement} and \eqref{eq:mfg_rvi_statement} to the $N$-player exploitability in the SMFG.
The next sections will be devoted to designing sample-efficient IL algorithms.

\section{Convergence in the Full Feedback Case}\label{sec:expert_feedback_results}

We first present an IL algorithm for the full feedback setting, as a first step towards analyzing the more interesting bandit feedback setting.
In this setting, while there is no centralized controller, independent noisy reports of all action payoffs are available to each agent after each round.

\textbf{Is it possible to simply solve MF-RVI in our IL setting?}
Before we present our results, we note the following:
Past works in MFG have already proved approximation results of $N$-agent games by MFG albeit in different settings \citep{saldi2019approximate, yardim2024mean}, but these results do not consider when \emph{learning itself} is carried out with $N$ agents. 
If $N$ agents can not communicate, it is theoretically challenging to approximate the MF-RVI and to tackle bandit feedback.
Most importantly, the IL algorithms formalized in Section~\ref{section:alg_formalization} can not query an operator oracle or maintain a common iterate throughout repeated plays.
Therefore, the approximation properties of \eqref{eq:mfg_vi_statement} do not immediately imply the MF-NE can be learned using VI algorithms.
In this section and the next, we prove the more challenging result of convergence with IL, first under full feedback and later under partial (bandit) feedback.

Our analysis builds up on Tikhonov regularized projected ascent (TRPA).
The TRPA operator is defined as
\begin{align}
    \Gamma^{\eta, \tau}(\vecpi) := \Pi_{\Delta_\setA} ( \vecpi + \eta (\vecF - \tau \matI)(\vecpi) ) = \Pi_{\Delta_\setA} ( (1-\eta\tau) \vecpi + \eta \vecF(\vecpi) ), \tag{TRPA}
\end{align}
for a learning rate $\eta > 0$ and regularization $\tau > 0$.
Intuitively, $\Gamma^{\eta, \tau}$ uses $\vecF$ evaluated at $\vecpi$ to modify action probabilities in the direction of the greatest payoff, incorporating an $\ell_2$ regularizer of $\tau$.
Furthermore, the unique MF-NE $\vecpi^*$ of \eqref{eq:mfg_rvi_statement} is also a fixed point of $\Gamma^{\eta, \tau}$.
The analysis of TRPA is standard and known to converge for monotone $\vecF$ \citep{facchinei2003finite, nemirovski2004prox}, when (stochastic) oracle access to $\vecF$ is assumed.
Naturally, the main complication in applying the method above will be the fact that in the IL setting, agents can not evaluate the operator $\vecF$ arbitrarily, but rather can only observe (a noisy) estimate of $\vecF$ as a function of the empirical population distribution and not of their policy $\vecpi$.
In the full feedback setting, we analyze the following dynamics:
\begin{align}
     \vecpi_0^i := \operatorname{Unif} (\setA) = \frac{1}{K}\vecone_K, \hspace{1em} \vecpi^i_{t+1} =\Pi_{\Delta_\setA} ( (1 - \tau \eta_t) \vecpi_t^i + \eta_t \vecr_t^i ), \tag{TRPA-Full}
\end{align}
for a time varying learning rate $\eta_t$, for each agent $i \in \setN$.
The extraneous $\ell_2$-regularization incorporated in each agent running TRPA-Full is critical for the analysis and convergence in IL, as it allows explicit synchronization of policies of agents without communication.
We state the TRPA-Full algorithm in Algorithm~\ref{alg:full} for reference.

We state the following standard result regarding the TRPA operator without proof, as it will be used later.

\begin{lemma}[cf. Theorem 12.1.2 of \cite{facchinei2003finite}]\label{lemma:contraction_pg}
Assume $\vecF$ is $\lambda\geq 0$-monotone and $L$-Lipschitz.
Then $\Gamma^{\eta,\tau}$ is Lipschitz with constant $\sqrt{1 - 2 (\lambda + \tau) \eta + \eta^2 (L+\tau)^2}$ with respect to the $\ell_2$-norm.
\end{lemma}

\begin{algorithm}
    \caption{TRPA-Full: IL with full feedback algorithm for each agent $i \in \setN$.}\label{alg:full}
    \begin{algorithmic}
    \Require Number of actions $K$, regularization $\tau > 0$, learning rate $\{\eta_t\}_{t=0}^T$, rounds $T > 0$.
    \State $\vecpi^i_0 \leftarrow \frac{1}{K} \vecone$
    \For{$t = 0, \ldots, T-1$}
    \State \text{Play action with current policy $a^i_{t}\sim \vecpi^i_t$}.
    \State \text{Observe payoff $\vecr^i_{t}$}
    \State $\vecpi^i_{t+1} = \Pi_{\Delta_\setA} ( (1 - \tau \eta_t) \vecpi^i_t + \eta_t \vecr^i_t )$
    \EndFor
    \State Return $\vecpi^i_T$
    \end{algorithmic}
\end{algorithm}

For abuse of notation, let $\vecpi^* \in \Delta_\setA$ be the unique solution of \eqref{eq:mfg_rvi_statement} for the regularization $\tau > 0$.
Also define the sigma algebra $\mathcal{F}_{t} := \mathcal{F}(\{ \vecpi_{t'}^i \}_{t'=0, \ldots, t}^{i=1, \ldots, N})$.
We maintain the definitions of the core random variables of the SMFG dynamics introduced in Section~\ref{sec:game_initial_formulation},
\begin{align*}
    \widehat{\vecmu}_t := \frac{1}{N} \sum_{i=1}^N \vece_{a_t^i}, \quad \vecr^i_t := \vecF(\widehat{\vecmu}_t) + \vecn_t^i.
\end{align*}
Under TRPA-Full dynamics, we also define the following random variables that assist our analysis.
\begin{align*}
    \bar{\vecmu}_t &:= \frac{1}{N} \sum_{i=1}^N \vecpi_t^i, \quad e_t^i := \|\vecpi^i_t - \bar{\vecmu}_t \|_2^2, \quad
    u_t^i := \Exop\left[\| \vecpi_t^i - \vecpi^* \|_2^2\right].
\end{align*}
We call $\bar{\vecmu}_t$ the mean policy, $e_t^i$ the mean policy deviation, and $u_t^i$ the expected $\ell_2$-deviation from the regularized MF-NE.
Our goal is to bound the sequence or error terms $u_t^i$; however, the process is complicated by the fact that in general the policy deviations of agents $e_t^i$ are nonzero.
Our strategy is as follows: 
(1) derive a  recursion for $u_t^i$ incorporating the terms $e_t^i$, 
(2) bound the terms $e_t^i$, showing the deviation of the policies of the agents goes to zero in expectation, and
(3) solve the recursion to obtain the convergence rate.

The following lemma captures the first step and provides a recurrence for the evolution of $u_t^i$ under TRPA-Full.

\begin{lemma}[Error recurrence under full feedback]\label{lemma:full_error_recurrence}
    Under TRPA-Full with learning rates $\eta_t$, it holds for $L$-Lipschitz and $\lambda$-strongly monotone $\vecF$ that
    \begin{align*}
    \Exop\left[\| \vecpi_{t+1}^i - \vecpi^* \|_2^2\right] \leq &3\eta_t^2 K(1 + \sigma^2) + 2\eta_t^2(L+\tau)^2 + \frac{4\eta_t L^2 \lambda^{-1}}{ N } \\
        & + 2\eta_t L^2 \lambda^{-1} \Exop\left[e_t^i\right] + \left(1 - 2 \eta_t(\sfrac{\lambda}{2} + \tau)\right) \Exop\left[\| \vecpi_t^i - \vecpi^* \|_2^2\right],
\end{align*}
and for $L$-Lipschitz and monotone $\vecF$ that
\begin{align*}
    \Exop\left[\| \vecpi_{t+1}^i - \vecpi^* \|_2^2\right] \leq &3\eta_t^2 K(1 + \sigma^2) + 2\eta_t^2(L+\tau)^2 + \frac{4\tau^{-1} \eta_t L^2 \delta^{-1}}{ N} \\
        & + \tau^{-1}\eta_t L^2\delta^{-1} \Exop\left[e_t^i\right] + \left(1 - 2\tau \eta_t (1-\delta)\right) \Exop\left[\| \vecpi_t^i - \vecpi^* \|_2^2\right],
\end{align*}
for arbitrary $\delta \in (0,1)$.
\end{lemma}
\begin{proof}
We analyze for any $i\in[N]$ the error term $\| \vecpi_t^i - \vecpi^*\|_2^2$.
Denote $\alpha_t := (1 - \tau \eta_t)$.
For the regularized solution $\vecpi^*$, we have the fixed point result
\begin{align*}
    \Pi_{\Delta_{\setA}} ((1 - \tau \eta_t) \vecpi^* + \eta_t \vecF(\vecpi^*)) = \Pi_{\Delta_{\setA}} (\vecpi^* + \eta_t (\vecF - \tau \matI)(\vecpi^*)) = \vecpi^*.
\end{align*}
The proof strategy is to decompose the $\ell_2$ distance of player policies to $\vecpi^*$ into 3 components using this property.
We can bound the quantity $\| \vecpi_{t+1}^i - \vecpi^*\|_2^2$ by using the non-expansiveness of $\Pi_{\Delta_{\setA}}$:
\begin{align}
    \| \vecpi_{t+1}^i - \vecpi^*\|_2^2 = &\| \Pi_{\Delta_{\setA}}(\alpha_t \vecpi^i_t + \eta_t \vecr_t^i) - \Pi_{\Delta_{\setA}} (\alpha_t \vecpi^* + \eta_t \vecF(\vecpi^*)) \|_2^2 \notag \\
        \leq &\| \alpha_t \vecpi_t^i + \eta_t \vecF(\vecpi_t^i) - \alpha_t \vecpi^* - \eta_t \vecF(\vecpi^*) + \eta_t (\vecr_t^i - \vecF(\vecpi_t^i) )\|_2^2 \notag \\
        = & \eta_t^2\| \vecr_t^i - \vecF(\vecpi_t^i) \|_2^2 + 2\eta_t (\alpha_t (\vecpi_t^i - \vecpi^*) + \eta_t (\vecF(\vecpi_t^i) - \vecF(\vecpi^*)) )^\top (\vecr_t^i - \vecF(\vecpi_t^i)) \notag \\
         & + \|\alpha_t (\vecpi_t^i - \vecpi^*) + \eta_t (\vecF(\vecpi_t^i) - \vecF(\vecpi^*))\|_2^2 \notag \\
         \leq & \underbrace{\eta_t^2\| \vecr_t^i - \vecF(\vecpi_t^i) \|_2^2 + 2\eta_t^2 (\vecF(\vecpi_t^i) - \vecF(\vecpi^*))^\top (\vecr_t^i - \vecF(\vecpi_t^i))}_{(a)} \notag \\
            &+ \underbrace{2\eta_t\alpha_t (\vecpi_t^i - \vecpi^*)^\top (\vecr_t^i - \vecF(\vecpi_t^i))}_{(b)} + \underbrace{\|\alpha_t (\vecpi_t^i - \vecpi^*) + \eta_t (\vecF(\vecpi_t^i) - \vecF(\vecpi^*))\|_2^2 }_{(c)}. \label{ineq:decomp_abc_full_recur}
\end{align}

We analyze the three marked terms separately.
For term $(a)$, using the independence assumption of the noise vectors and Young's inequality, in expectation we obtain
\begin{align*}
    \Exop[(a)] \leq &\eta_t^2 \Exop[\| \vecr_t^i - \vecF(\vecpi_t^i) \|_2^2] + \eta_t^2 \Exop[\|\vecF(\vecpi_t^i) - \vecF(\vecpi^*)\|_2^2] + \eta_t^2 \Exop[\| \vecr_t^i - \vecF(\vecpi_t^i) \|_2^2] \\
    \leq &2\eta_t^2 \Exop[\| \vecr_t^i - \vecF(\vecpi_t^i) \|_2^2] + \eta_t^2 \Exop[\|\vecF(\vecpi_t^i) - \vecF(\vecpi^*)\|_2^2] \\
    \leq & 2\eta_t^2 \Exop[\| \vecr_t^i - \vecF(\widehat{\vecmu}_t)\|_2^2 + \| \vecF(\widehat{\vecmu}_t) - \vecF(\vecpi^i_t)\|_2^2 ] + \eta_t^2 K \\
    \leq & 2\eta_t^2 \sigma^2 K + 3\eta_t^2 K \leq 3\eta_t^2 K(\sigma^2 + 1)
\end{align*}
For the term $(c)$, we obtain
\begin{align*}
    (c) = &\|\alpha_t (\vecpi_t^i - \vecpi^*) + \eta_t (\vecF(\vecpi_t^i) - \vecF(\vecpi^*))\|_2^2 \\
        = & \|(\vecpi_t^i - \vecpi^*) + \eta_t (\vecF(\vecpi_t^i) - \tau \vecpi_t^i - \vecF(\vecpi^*) + \tau \vecpi^* )\|_2^2 \\
        \leq & \left(1 - 2 (\lambda + \tau) \eta_t + (L + \tau)^2 \eta_t^2\right) \| \vecpi_t^i - \vecpi^* \|_2^2 \\
        \leq & \left(1 - 2 (\lambda + \tau) \eta_t \right) \| \vecpi_t^i - \vecpi^* \|_2^2 + 2(L + \tau)^2 \eta_t^2
\end{align*}
where the last inequality holds from the Lipschitz continuity result of Lemma~\ref{lemma:contraction_pg}.

For the term $(b)$, first taking the strongly monotone problem $\lambda > 0$ , we have that
\begin{align*}
(b) = & 2\eta_t\alpha_t (\vecpi_t^i - \vecpi^*)^\top (\vecr_t^i - \vecF(\vecpi_t^i)) \\
 = & 2\eta_t\alpha_t (\vecpi_t^i - \vecpi^*)^\top (\vecr_t^i - \vecF(\widehat{\vecmu}_t)) + 2\eta_t\alpha_t (\vecpi_t^i - \vecpi^*)^\top (\vecF(\widehat{\vecmu}_t) - \vecF(\bar{\vecmu}_t)) \\
    & + 2\eta_t\alpha_t (\vecpi_t^i - \vecpi^*)^\top (\vecF(\bar{\vecmu}_t) - \vecF(\vecpi_t^i)) \\
\leq &2\eta_t\alpha_t \left( \frac{\lambda}{4} \|\vecpi_t^i - \vecpi^* \|_2^2 + \frac{1}{\lambda} \|\vecF(\widehat{\vecmu}_t) - \vecF(\bar{\vecmu}_t)\|_2^2\right) + 2\eta_t\alpha_t \left(\frac{\lambda}{4} \|\vecpi_t^i - \vecpi^* \|_2^2 + \frac{1}{\lambda} \|\vecF(\bar{\vecmu}_t) - \vecF(\vecpi_t^i)\|_2^2 \right) \\
    &+2\eta_t\alpha_t (\vecpi_t^i - \vecpi^*)^\top (\vecr_t^i - \vecF(\widehat{\vecmu}_t)) \\
\leq & \eta_t \lambda \|\vecpi_t^i - \vecpi^* \|_2^2 + 2\eta_t\lambda^{-1}\|\vecF(\widehat{\vecmu}_t) - \vecF(\bar{\vecmu}_t)\|_2^2 + 2\eta_t\lambda^{-1} \|\vecF(\bar{\vecmu}_t) - \vecF(\vecpi_t^i)\|_2^2 \\
    &+2\eta_t\alpha_t (\vecpi_t^i - \vecpi^*)^\top (\vecr_t^i - \vecF(\widehat{\vecmu}_t)),
\end{align*}
which follows from applications of Young's inequality.
For the last three terms we observe:
\begin{align*}
    \Exop\left[2\eta_t\alpha_t (\vecpi_t^i - \vecpi^*)^\top (\vecr_t^i - \vecF(\widehat{\vecmu}_t)) | \mathcal{F}_t\right] = &0, \\
    \Exop[\|\vecF(\widehat{\vecmu}_t) - \vecF(\bar{\vecmu}_t)\|_2^2 | \mathcal{F}_{t}] \leq & L^2 \Exop\left[ \|\widehat{\vecmu}_t - \bar{\vecmu}_t\|_2^2 | \mathcal{F}_{t}\right] \\
    \leq & L^2\Exop\left[\frac{1}{N^2}\|\sum_{i}\vecpi_t^i - \sum_{i} \vece_{a_t^i}\|^2_2 | \mathcal{F}_{t}\right] \\
    = & \frac{L^2}{N^2}\sum_{i}\Exop[\|\vecpi_t^i - \vece_{a_t^i}\|^2_2 | \mathcal{F}_{t}] \leq \frac{2L^2}{N}, \\
    \|\vecF(\bar{\vecmu}_t) - \vecF(\vecpi_t^i)\|_2^2 \leq & L^2 \|\bar{\vecmu}_t - \vecpi_t^i\|_2^2 = L^2 e_t^i.
\end{align*}
The second inequality above follows from the fact that $\widehat{\vecmu}_t$ is the sum of $N$ independent random variables and has expectation $\bar{\vecmu}_t$.
Hence, putting in the bounds for $(a), (b), (c)$ and taking expectations, we obtain the inequality
\begin{align*}
    \Exop\left[\| \vecpi_{t+1}^i - \vecpi^*\|_2^2 \right] \leq & 3 \eta_t^2 K(1 + \sigma^2) + \frac{4\eta_t L^2}{\lambda N} + \frac{2\eta_t L^2}{\lambda} \Exop\left[e_t^i\right] \\
        &+\left(1 - 2 (\sfrac{\lambda}{2} + \tau) \eta_t \right) \Exop\left[\| \vecpi_t^i - \vecpi^* \|_2^2\right] + 2(L + \tau)^2 \eta_t^2.
\end{align*}

Turning back to the monotone case, if $\lambda=0$, vary the upper bound on $(b)$ as follows.
Take any arbitrary $\delta \in (0,1)$.
Then, once again applying Young's inequality, we obtain
\begin{align*}
(b) = & 2\eta_t\alpha_t (\vecpi_t^i - \vecpi^*)^\top (\vecr_t^i - \vecF(\vecpi_t^i)) \\
 = & 2\eta_t\alpha_t (\vecpi_t^i - \vecpi^*)^\top (\vecr_t^i - \vecF(\widehat{\vecmu}_t)) + 2\eta_t\alpha_t (\vecpi_t^i - \vecpi^*)^\top (\vecF(\widehat{\vecmu}_t) - \vecF(\bar{\vecmu}_t)) \\
    & + 2\eta_t\alpha_t (\vecpi_t^i - \vecpi^*)^\top (\vecF(\bar{\vecmu}_t) - \vecF(\vecpi_t^i)) \\
\leq &2\eta_t\alpha_t \left( \frac{\tau\delta}{2} \|\vecpi_t^i - \vecpi^* \|_2^2 + \frac{1}{2\tau\delta} \|\vecF(\widehat{\vecmu}_t) - \vecF(\bar{\vecmu}_t)\|_2^2\right) + 2\eta_t\alpha_t \left(\frac{\tau\delta}{2} \|\vecpi_t^i - \vecpi^* \|_2^2 + \frac{1}{2\tau\delta} \|\vecF(\bar{\vecmu}_t) - \vecF(\vecpi_t^i)\|_2^2 \right)  \\
    &+2\eta_t\alpha_t (\vecpi_t^i - \vecpi^*)^\top (\vecr_t^i - \vecF(\widehat{\vecmu}_t)) \\
\leq & 2 \eta_t \tau\delta \|\vecpi_t^i - \vecpi^* \|_2^2 + \frac{\eta_t}{\tau\delta}\|\vecF(\widehat{\vecmu}_t) - \vecF(\bar{\vecmu}_t)\|_2^2 + \frac{\eta_t}{\tau\delta} \|\vecF(\bar{\vecmu}_t) - \vecF(\vecpi_t^i)\|_2^2 \\
    &+2\eta_t\alpha_t (\vecpi_t^i - \vecpi^*)^\top (\vecr_t^i - \vecF(\widehat{\vecmu}_t)).
\end{align*}
Applying the same bounds for the terms $(a), (c)$ as before yields  the lemma.
\end{proof}

This above lemma has two key features: a dependence on expected mean policy deviation $\Exop\left[e_t^i\right]$, and a term that scales as $\mathcal{O}(\sfrac{1}{N})$.
While the $\mathcal{O}(\sfrac{1}{N})$ term can be anticipated (and asymptotically ignored when $N$ is large) due to the finite-agent mean-field bias (as shown previously in Section~\ref{sec:theoretical_tool}), the term $\Exop\left[e_t^i\right]$ must be controlled separately in the independent learning setting, where policies cannot be synchronized through explicit communication between agents.
The term $\Exop\left[e_t^i\right]$ reflects the core difference of the SMFG model from typical VI stochastic oracles.
Unlike typical VI oracle models, in SMFG the operator $\vecF$ cannot be evaluated at the current iterate $\vecpi^i_t$ of a player $i$  but only approximately at the mean $\bar{\vecmu}_t$.
This is due to decentralized learning: players can only evaluate the current payoffs at the ``mean-iterate'' given by $\vecF(\widehat{\vecmu}_t) \approx \vecF(\bar{\vecmu}_t)$ (up to some stochastic noise) that is almost surely different than their iterates $\{\vecpi^i_t\}_i$ apart from the case with degenerate/zero noise.
Furthermore, Lemma~\ref{lemma:full_error_recurrence} suggests that the algorithmic scheme must guarantee that $\Exop\left[e_t^i\right]$ decays with the rate at least $\mathcal{O}(\sfrac{1}{t})$ to obtain a non-vacuous bound on exploitability.
Taking inspiration from algorithmic stability literature \citep{ahn2022reproducibility, zhang2024optimal}, we utilize a regularization scheme to ensure the iterates of players do not diverge.
The following lemma shows that by introducing explicit regularization $\tau>0$, the expected mean policy deviation can be controlled throughout training.



\begin{lemma}[Policy variations bound]\label{lemma:policy_variations_bound_trpa_full}
    Under TRPA-Full with learning rates $\eta_t :=\frac{\tau^{-1}}{t+2}$, we have $\Exop\left[e_t^i\right] \leq \frac{14 \tau^{-2} K\sigma^2 + 14}{t+2}$.
\end{lemma}
\begin{proof}
Note that for any $i,j\in\setN$ such that $i\neq j$, using the non-expansiveness of the projection operator it holds that
\begin{align*}
    \| \vecpi^i_{t+1} - \vecpi^j_{t+1} \|_2^2 = &  \| \Pi_{\Delta_\setA}((1 - \tau \eta_t) \vecpi^i_t + \eta_t \vecr_t^i) - \Pi_{\Delta_\setA}((1 - \tau \eta_t) \vecpi^j_t + \eta_t \vecr_t^j) \|_2^2 \\
    \leq & \| (1 - \tau \eta_t) \vecpi^i_t + \eta_t \vecr_t^i - (1 - \tau \eta_t) \vecpi^j_t - \eta_t \vecr_t^j \|_2^2 \\
    \leq & \| (1 - \tau \eta_t) (\vecpi^i_t - \vecpi^j_t) + \eta_t (\vecr_t^i - \vecr_t^j) \|_2^2 \\
    = & (1 - \tau \eta_t)^2 \| \vecpi^i_t - \vecpi^j_t \|_2^2 + \eta_t^2 \|  \vecr_t^i - \vecr_t^j \|_2^2 + 2 (1 - \tau \eta_t) \eta_t (\vecpi^i_t - \vecpi^j_t) ^ \top ( \vecr_t^i - \vecr_t^j )
\end{align*}
Taking the conditional expectation on both sides, we obtain
\begin{align*}
    \Exop \left[ \| \vecpi^i_{t+1} - \vecpi^j_{t+1} \|_2^2 | \mathcal{F}_t \right] \leq & (1 - \tau \eta_t)^2 \| \vecpi^i_t - \vecpi^j_t \|_2^2 + \Exop \left[ \eta_t^2 \|  \vecr_t^i - \vecr_t^j \|_2^2 | \mathcal{F}_t \right] \\
        &+ 2 (1 - \tau \eta_t) \eta_t (\vecpi^i_t - \vecpi^j_t) ^ \top \Exop\left[\vecr_t^i - \vecr_t^j | \mathcal{F}_t \right] \\
    = & (1 - \tau \eta_t)^2 \|  \vecpi^i_t - \vecpi^j_t \|_2^2 + \eta_t^2 \Exop \left[\|\vecn^i_t - \vecn^j_t\|_2^2 | \mathcal{F}_t \right] \\
    = & (1 - \tau \eta_t)^2 \|  \vecpi^i_t - \vecpi^j_t \|_2^2 + 2\eta_t^2 K \sigma^2
\end{align*}
almost surely, since we have $\vecr_t^i := \vecF(\widehat{\vecmu}_t) + \vecn^i_t$.
Then, taking the expectation on both sides, 
\begin{align*}
    \Exop \left[ \| \vecpi^i_{t+1} - \vecpi^j_{t+1} \|_2^2 \right] \leq &(1 - \tau \eta_t)^2 \Exop\left[\|  \vecpi^i_t - \vecpi^j_t \|_2^2\right] + 2\eta_t^2 K\sigma^2 \\
    \leq & \left(1 - \frac{1}{t+2}\right)^2 \Exop\left[\|  \vecpi^i_t - \vecpi^j_t \|_2^2\right] + \left(\frac{\tau^{-1}}{t+2}\right)^2 2K\sigma^2 \\
    \leq & \left(1 - \frac{2}{t+2}\right) \Exop\left[\|  \vecpi^i_t - \vecpi^j_t \|_2^2\right] + \frac{1}{(t+2)^2} \Exop\left[\|  \vecpi^i_t - \vecpi^j_t \|_2^2\right] + \frac{2\tau^{-2}K\sigma^2}{(t+2)^2} \\
    \leq & \left(1 - \frac{2}{t+2}\right) \Exop\left[\|  \vecpi^i_t - \vecpi^j_t \|_2^2\right] + \frac{2\tau^{-2}K\sigma^2 + 2}{(t+2)^2}
\end{align*}
To bound the recurrence, we can use the recurrence lemma (Lemma~\ref{lemma:general_recurrence}, noting $\gamma=2, a = 2, u_0 = 0, c_0 = 0, c_1 = 2\tau^{-2}K\sigma^2 + 2$ in its statement):
\begin{align*}
    \Exop \left[ \| \vecpi^i_{t+1} - \vecpi^j_{t+1} \|_2^2 \right] \leq & 5\frac{2\tau^{-2}K\sigma^2 + 2}{(t+2)^2} + 3\frac{2\tau^{-2}K\sigma^2 + 2}{t+2} + \frac{2\tau^{-2}K\sigma^2 + 2}{(t+2)^2} \leq \frac{14 \tau^{-2} K\sigma^2 + 14}{t+2}.
\end{align*}
Then, the expected values of $e_t^i$ can be bounded using:
\begin{align*}
     e_t^i = &\|\vecpi^i_t - \bar{\vecmu}_t \|_2^2 
     =  \left\|\vecpi^i_t - \frac{1}{N} \sum_{j=1}^N \vecpi^j_t  \right\|_2^2 
     \leq \frac{1}{N} \sum_{j=1}^N \| \vecpi^i_t - \vecpi^j_t \|_2^2
\end{align*}
by an application of Jensen's inequality.
Then we have $\Exop\left[e_t^i\right] \leq \frac{14 \tau^{-2} K\sigma^2 + 14}{t+2}$.
\end{proof}

With an explicit bound in expectation on the mean policy deviation $e_t^i$, we can now proceed to the main recurrence for the expected error terms $u_t^i$ in order to prove our main convergence result.
We state our main convergence result for TRPA-Full dynamics in Theorem~\ref{theorem:expert_short} by solving these two recurrences for the monotone and strongly monotone cases.

\begin{theorem}[Convergence, full feedback]\label{theorem:expert_short}
Assume $\vecF$ is Lipschitz, monotone.
Assume $N$ agents run the TRPA-Full update rule for $T$ time steps with learning rates $\eta_t := \frac{\tau^{-1}}{t+2}$ and arbitrary regularization $\tau>0$.
Then it holds for any $i\in[N]$ that $\Exop\left[ \setE^i_{\text{exp}}( \{\vecpi^j_{T}\}_{j=1}^N ) \right] \leq \mathcal{O} (\frac{\tau^{-2}}{\sqrt{T}}+ \frac{\tau^{-1}}{\sqrt{N}} + \tau)$.
Furthermore, if $\vecF$ is $\lambda$-strongly monotone, then $\Exop\left[ \setE^i_{\text{exp}}( \{\vecpi^j_{T}\}_{j=1}^N ) \right] \leq \mathcal{O} (\frac{\tau^{-\sfrac{3}{2}} \lambda^{-\sfrac{1}{2}}}{\sqrt{T}} + \frac{\tau^{-\sfrac{1}{2}} \lambda^{-\sfrac{1}{2}}}{\sqrt{N}} + \tau)$.
\end{theorem}
\begin{proof}
Note that the exploitability in the main statement of the theorem can be related to $u_t^i$ as follows using Lemma~\ref{lemma:phi_lipschitz}:
\begin{align*}
    \Exop[\setE^i_{\text{exp}}(\{\vecpi^j_{t}\}_{j=1}^N)] \leq &\setE^i_{\text{exp}}(\{\vecpi^*\}_{j=1}^N) + \sqrt{K} \Exop[\| \vecpi_t^i - \vecpi^* \|_2] + \frac{4L\sqrt{2K}}{N} \sum_{j\neq i} \Exop[\| \vecpi_t^j - \vecpi^* \|_2] \\
    \leq & \setE^i_{\text{exp}}(\{\vecpi^*\}_{j=1}^N) + \sqrt{K} \sqrt{u_t^i} +  \frac{4L\sqrt{2K}}{N} \sum_{j\neq i} \sqrt{u_t^j} \\
    \leq & \setE^i_{\text{exp}}(\{\vecpi^*\}_{j=1}^N) + \frac{\max\{ \sqrt{K}, 4L\sqrt{2K} \}}{N} \sum_{j} \sqrt{u_t^j}
\end{align*}
Hence the bounds on $u_t^j$ will yield the result of the theorem by linearity of expectation, along with an invocation of Theorem~\ref{theorem:mfgrvi_and_explotability}.

Finally, we solve the recurrences for $\lambda = 0$ and $\lambda > 0$ using Lemma~\ref{lemma:full_error_recurrence}.
For the case $\lambda > 0$, if $\eta_t=\frac{\tau^{-1}}{t+2}$, Lemma~\ref{lemma:full_error_recurrence} provides the bound 
\begin{align*}
    u_{t+1}^i \leq &\frac{ 3\tau^{-2} K(1 + \sigma^2) + 2\tau^{-2}(L+\tau)^2}{(t+2)^2} + \frac{4\tau^{-1} L^2 \lambda^{-1}}{ N (t+2)} + \frac{2\tau^{-1} L^2 \lambda^{-1}}{t+2} \Exop\left[e_t^i\right] \\
        &+ \left(1 - \frac{2 \tau^{-1}(\sfrac{\lambda}{2} + \tau)}{t+2}\right) u_{t}^i. 
\end{align*}
By placing $\Exop\left[ e^i_t\right] \leq \frac{14 \tau^{-2} K\sigma^2 + 14}{t+2}$ due to Lemma~\ref{lemma:policy_variations_bound_trpa_full}, we obtain
\begin{align*}
    u_{t+1}^i \leq &\frac{ 3\tau^{-2}K(1 + \sigma^2) + 2\tau^{-2}(L+\tau)^2 +  28\tau^{-3} K \sigma^2 \lambda^{-1} L^2 + 28 \tau^{-1} L^2 \lambda^{-1}}{(t+2)^2} \\
        &+ \frac{4\tau^{-1} L^2 \lambda^{-1}}{ N (t+2)} + \left(1 - \frac{2}{t+2}\right) u_{t}^i.
\end{align*}
Invoking a generic recurrence lemma (Lemma~\ref{lemma:general_recurrence} in Appendix~\ref{app:basic_inequalities}) leads to the main statement of the theorem.

For the monotone case $\lambda = 0$, we have the recursion:
\begin{align*}
    u_{t+1}^i \leq &\frac{ 3\tau^{-2}K(1 + \sigma^2) + 2\tau^{-2}(L+\tau)^2}{(t+2)^2} + \frac{4\tau^{-2} L^2 \delta^{-1}}{ N (t+2)} + \frac{ \tau^{-2} L^2\delta^{-1}}{ t+2 } \Exop\left[e_t^i\right] \\
        & + \left(1 - \frac{2 (1-\delta)}{t+2}\right) u_{t}^i.
\end{align*}
and once again placing the upper bound on expected policy deviation due to Lemma~\ref{lemma:policy_variations_bound_trpa_full},
\begin{align*}
    u_{t+1}^i \leq &\frac{ 3\tau^{-2}K(1 + \sigma^2) + 2\tau^{-2}(L+\tau)^2 + 28 K \tau^{-4} L^2\delta^{-1}\sigma^2 +28 \tau^{-2} L^2\delta^{-1}}{(t+2)^2} \\
        &+ \frac{2\tau^{-2} L^2 \delta^{-1}}{ N (t+2)} + \left(1 - \frac{2 (1-\delta)}{t+2}\right) u_{t}^i.
\end{align*}
Another invocation of Lemma~\ref{lemma:general_recurrence} concludes the proof, choosing $\delta=\sfrac{1}{4}$.
\end{proof}

This convergence result is stated in terms of exploitability of the unregularized game, leading to an additional $\mathcal{O}(\tau)$ term.
However, in many cases, the Nash equilibrium of the regularized game itself is of interest, in which case the upper bounds should read
$\mathcal{O} (\frac{\tau^{-2}}{\sqrt{T}}+ \frac{\tau^{-1}}{\sqrt{N}})$
and
$\mathcal{O} (\frac{\tau^{-\sfrac{3}{2}} \lambda^{-\sfrac{1}{2}}}{\sqrt{T}} + \frac{\tau^{-\sfrac{1}{2}} \lambda^{-\sfrac{1}{2}}}{\sqrt{N}})$
for the monotone and strongly monotone cases respectively.

In the choice of learning rate $\eta_t$ above, no intrinsic problem parameter is assumed to be known.
Furthermore, due to (1) the regularization $\tau$ and (2) a finite population, a non-vanishing exploitability of $\mathcal{O}(\tau + \sfrac{\tau^{-1}}{\sqrt{N}})$ will be induced in terms of the NE in the monotone case.
While Theorem~\ref{theorem:mfgrvi_and_explotability} readily suggested a bias of order $\mathcal{O}(\sfrac{1}{\sqrt{N}})$ is fundamental, when learning is conducted with finitely many agents Theorem~\ref{theorem:expert_short} shows this is amplified to $\mathcal{O}(\sfrac{\tau^{-1}}{\sqrt{N}})$.
Since for finite population SMFG, there will always be a non-vanishing exploitability in terms of NE due to the mean-field approximation, in practice $\tau$ could be chosen to incorporate an acceptable bias level.
Alternatively, if the exact value of the number of players $N$ is known by each agent, one could choose $\tau$ optimally, to obtain the following corollary.

\begin{corollary}[Optimal $\tau$, full feedback]\label{corollary:expert}
Assume the conditions of Theorem~\ref{theorem:expert_short}.
For monotone $\vecF$, choosing regularization parameter $\tau = \sfrac{1}{\sqrt[4]{N}}$ yields
$\Exop\left[\setE^i_{\text{exp}}(\{\vecpi^j_T\}_{j=1}^N) \right] \leq \mathcal{O}(\frac{\sqrt{N}}{\sqrt{T}} + \frac{1}{\sqrt[4]{N}})$ for any $i$.
For $\lambda$-strongly monotone $\vecF$, choosing $\tau = \sfrac{1}{\sqrt[3]{N}}$ yields $\Exop\left[\setE^i_{\text{exp}}(\{\vecpi^j_T\}_{j=1}^N) \right] \leq \mathcal{O}(\frac{ \lambda^{-\sfrac{1}{2}} \sqrt{N}}{\sqrt{T}} + \frac{\lambda^{-\sfrac{1}{2}}}{\sqrt[3]{N}})$.
\end{corollary}

Even though TRPA-Full solves the regularized (hence strongly monotone) problem, compared to the $\mathcal{O}(\sfrac{1}{T})$ rate in classical strongly monotone VI \citep{kotsalis2022simple} or strongly convex optimization \citep{rakhlin2011making},
our worse $\mathcal{O}(\sfrac{1}{\sqrt{T}})$ time dependence is due to independent learning.
Intuitively, additional time is required to ensure the policies of independent learners are sufficiently close when ``collectively'' evaluating $\vecF$.
The additional dependence of the time-vanishing term on $\sqrt{N}$ is also a result of this fact.
Furthermore, when learning itself is performed by $N$ agents, we note that the bias as a function of $N$ decreases with $\mathcal{O}(\sfrac{1}{\sqrt[4]{N}})$ (or $\mathcal{O}(\sfrac{1}{\sqrt[3]{N}})$ for strongly monotone problems), and not with $\mathcal{O}(\sfrac{1}{\sqrt{N}})$ as Theorem~\ref{theorem:mfg_ne} might suggest.
We leave the question of whether this gap can be improved and whether knowledge of $N$ is required in Corollary~\ref{corollary:expert}, as future work.



\section{Convergence in the Bandit Feedback Case}\label{sec:bandit_feedback_results}

We now move on to the more challenging and realistic bandit feedback case, where agents can only observe the payoffs of the actions they have chosen.
Once again, we analyze the IL setting (or in bandits terminology, the ``no communications'' setting) where agents can not interact or coordinate with each other.
One of the main challenges of bandit feedback with IL in our setting is that it is difficult for each agent to identify itself (i.e., assign itself a unique number between $1,\ldots,N$) so that exploration of action payoffs can be performed in turns.
For instance, in MMAB algorithms, this is typically achieved using variants of the so-called musical chairs algorithm \citep{lugosi2022multiplayer}, which is not available in our formulation.
Instead, we adopt a \emph{probabilistic} exploration scheme where each agent probabilistically decides it is its turn to explore payoffs while the rest of the agents induce the required empirical population distribution on which $\vecF$ should be evaluated.

Our algorithm, which we call TRPA-Bandit, is straightforward and relies on exploration occurring over epochs, where policies are updated once in between epochs using the estimate of action payoffs constructed during the exploration phase.
We use the subscript $h$ to index epochs, which consist of $T_h$ repeated plays indexed by $(h,t)$ for $t=1,\ldots,T_h$.
While we formally presented TRPA-Bandit (Algorithm~\ref{alg:bandit}), the procedure informally is as follows for each agent, fixing an exploration parameter $\varepsilon \in (0,1)$ and an agent $i\in\setN$:
\begin{enumerate}
    \item At each epoch $h$, for $T_h > 0$ time steps, repeat the following:
    \begin{enumerate}
        \item With probability $\varepsilon$, sample uniformly an action $a^i_{h,t}$, observe the payoff $r^i_{h,t}$, and keep the importance sampling estimate $\widehat{\vecr}^i_h \leftarrow K r_{h,t}^i \vece_{a^i_{h,t}}$.
        \item Otherwise (with probability $1-\varepsilon$), sample action according to current policy $\vecpi^i_h$.
    \end{enumerate}
    \item Update the policy using TRPA, $\vecpi^i_{h+1} = \Pi_{\Delta_\setA} ( (1 - \tau \eta_h) \vecpi^i_h + \eta_h \widehat{\vecr}^i_h )$.
    If the agent did not explore this epoch, use $\widehat{\vecr}^i_h = \veczero$.
\end{enumerate}
Intuitively, the probabilistic sampling scheme allows some agents to build a low-variance estimate of $\vecF$, while others simply sample actions with their current policy in order to induce the empirical population distribution at which $\vecF$ should be evaluated.

\begin{algorithm}
    \caption{TRPA-Bandit: IL with bandit feedback algorithm for each agent $i\in\setN$.}\label{alg:bandit}
    \begin{algorithmic}
    \Require Number of actions $K$, regularization $\tau > 0$, exploration probability $\varepsilon > 0$, number of epochs $H$, epoch lengths $\{T_h\}_h$, learning rates $\{\eta_h\}_h$
    \State $\vecpi^i_0 \leftarrow \frac{1}{K} \vecone$
    \For{$h = 0, \ldots, H-1$}
    \State $\widehat{\vecr}^i_h \leftarrow \veczero$ %
    \For{$t = 1, \ldots, T_h$} \Comment{Exploration for $T_h$ rounds before policy update,}
    \State Sample Bernoulli r.v. $X_{h,t}^i \sim \operatorname{Ber}(\varepsilon)$.
    \If{$X_{h,t}^i=1$}
        \State \text{Play action $a^i_{h,t} \sim \operatorname{Unif}(\setA)$ uniformly at random}.
        \Comment{Explore with prob. $\varepsilon$,}
        \State \text{Observe payoff $r^i_{h,t}$}, set  $\widehat{\vecr}^i_h \leftarrow K r^i_{h,t}\vece_{a^i_{h,t}}$.
    \ElsIf{$X_{h,t}^i=0$}
        \State \text{Play action with current policy $a^i_{h,t}\sim \vecpi^i_h$}.
        \Comment{Else, play the current policy.}
    \EndIf
    \EndFor
    \State $\vecpi^i_{h+1} = \Pi_{\Delta_\setA} ( (1 - \tau \eta_h) \vecpi^i_h + \eta_h \widehat{\vecr}^i_h )$
    \Comment{After each epoch, update policy.}
    \EndFor
    \State Return $\vecpi^i_H$
    \end{algorithmic}
    \end{algorithm}



Similar to the full feedback setting, we introduce useful notation used throughout this chapter.
We define the sigma algebra $\mathcal{F}_{h} := \mathcal{F}(\{ \vecpi_{h'}^i \}_{h'=0, \ldots, h}^{i=1, \ldots, N})$.
Adapting the notation from Section~\ref{sec:game_initial_formulation} to the case with multiple epochs, we use
\begin{align*}
\widehat{\vecmu}_{h, t} := \frac{1}{N} \sum_{i=1}^N \vece_{a_{h, t}^i}, \qquad
    \vecr^i_{h, t} := \vecF(\widehat{\vecmu}_{h, t}) + \vecn_{h, t}^i,
\end{align*}
where the updated time indices $h, t$ simply refer to the $t$-th round of play in epoch $h$, and $a_{h, t}^i$ is the action played by player $i$ at epoch $h$, round $t$.
Under the dynamics of Algorithm~\ref{alg:bandit}, we define the following random variables to assist our proofs:
\begin{align*}
    \bar{\vecmu}_h &:= \frac{1}{N} \sum_{i=1}^N \vecpi_h^i, 
    \qquad
    e_t^i := \|\vecpi^i_h - \bar{\vecmu}_h \|_2^2, 
    \qquad
    u_h^i := \Exop\left[\| \vecpi_h^i - \vecpi^* \|_2^2\right],
\end{align*}
which correspond to the mean policy at epoch $h$, the policy deviation of agent $i$ from the mean at epoch $h$ and the $\ell_2$ distance from the MF-NE.
Note that since policies are updated only in between epochs, the above quantities are indexed by epochs $h$ rather than rounds $h, t$.

Our analysis follows the ideas in the case of expert feedback, the main difference being randomization due to the exploration probabilities and the errors being analyzed per epoch rather than per round.
Similar to the full feedback setting, we will proceed in several steps expressed as intermediate lemmas:
(1) we bound the added bias and variance due to the importance sampling strategy,
(2) we obtain a non-linear recursion for the expectation of the terms $e_t^i$ and possible sampling bias, 
(3) we bound the expected differences of each agent's action probabilities $e_t^i$, showing the deviation of the policies of the agents goes to zero in expectation, 
(4) we solve the recursion to obtain the convergence rate.

The next result, Lemma~\ref{lemma:exploration_bias_trpa_bandit}, provides an answer to the first step.
We show that despite the probabilistic exploration step, the estimates $\widehat{\vecr}_h^i$ in Algorithm~\ref{alg:bandit} have low bias and variance.

\begin{lemma}[Exploration bias]\label{lemma:exploration_bias_trpa_bandit}
Under the dynamics of TRPA-Bandit, it holds almost surely for each epoch $h \geq 0$ that
\begin{align*}
    \| \Exop[\widehat{\vecr}_h^i | 
 \mathcal{F}_h ] - \vecF(\varepsilon \frac{1}{K}\vecone + (1-\varepsilon) \bar{\vecmu}_h) \|_2 \leq K^{\sfrac{3}{2}} \sqrt{1 + \sigma^2} \exp\left\{ -\varepsilon T_{h}\right\} + \frac{2L}{N} + \frac{2L}{\sqrt{N}}.
\end{align*}
\end{lemma}

The full proof has been postponed to Appendix~\ref{sec:proof_lemma_bandit_exploration_bias}.
In summary, the proof strategy is to decompose and analyze the bias due to the possibility of no exploration round happening (the term $K^{\sfrac{3}{2}} \sqrt{1 + \sigma^2} \exp\left\{ -\varepsilon T_{h}\right\}$),  the impact of the exploring agent on payoffs (the term $\frac{2L}{N}$), and bias due to the finitely many agents, similar to Theorem~\ref{theorem:mfg_ne} (the term $\frac{2L}{\sqrt{N}}$).
The additional bias due to probabilistic exploration originates from the possibility that no exploration occurs in a given epoch: the probability of this event can be bounded by $\exp\left\{ -\varepsilon T_{h}\right\}$.

Lemma~\ref{lemma:exploration_bias_trpa_bandit} shows that even if the players do not have full feedback, they can obtain low-bias, low-variance estimates of $\vecF(\varepsilon \frac{1}{K}\vecone + (1-\varepsilon) \bar{\vecmu}_h) \approx \vecF(\bar{\vecmu}_h)$ when $\varepsilon$ is small.
It guarantees that even if the agents do not explore each epoch, in expectation the probabilistic exploration scheme yields a low bias if the epoch lengths $T_h$ are logarithmically large: hence, full feedback can be simulated by paying a logarithmic cost.
Therefore, in our epoched exploration scheme, the bias in ``querying'' the payoff operator $\vecF$ due to an exploring population can be controlled by tuning $\varepsilon$ and $T_h$.

We next state the error recursion in Lemma~\ref{lemma:bandit_main_recurrence}, which uses the result of Lemma~\ref{lemma:exploration_bias_trpa_bandit} to construct the main recurrence for the bandit feedback case.

\begin{lemma}[Main recurrence for TRPA-Bandit]\label{lemma:bandit_main_recurrence}
Under TRPA-Bandit dynamics, it holds for any $i \in \{1, \ldots, N \}$ and each epoch $h\geq 0$ that
\begin{align*}
    \Exop\left[\| \vecpi_{h+1}^i - \vecpi^* \|_2^2\right] \leq & 4 \eta_h^2 K^3(1 + \sigma^2) + 8\eta_h^2(L+\tau)^2 + 8 K^{\sfrac{3}{2}}\eta_h \sqrt{1+\sigma^2}  \exp\{-\varepsilon T_h\} \\
        &+128\eta_h\lambda^{-1} L^2 N^{-1} + 16\eta_h\lambda^{-1}L^2\varepsilon^2 + 2\eta_h\lambda^{-1} L^2 \Exop\left[e_h^i\right] \\
        &+\left(1 - 2 \eta_h(\sfrac{\lambda}{2} + \tau)\right) \Exop\left[\| \vecpi_h^i - \vecpi^* \|_2^2\right],
\end{align*}
for strongly monotone $\lambda >0$ payoffs, and
\begin{align*}
    \Exop\left[\| \vecpi_{h+1}^i - \vecpi^* \|_2^2\right] \leq &  4\eta_h^2 K^3(\sigma^2 + 1) + 8 \eta_h^2 (L+\tau)^2 + 8 K^{\sfrac{3}{2}} \eta_h \sqrt{1+\sigma^2} \exp\{-\varepsilon T_h\} \\
    &+64\tau^{-1} \eta_h \delta^{-1}L^2 N^{-1}+8\tau^{-1} \eta_h \delta^{-1}L^2 \varepsilon^{2} + \tau^{-1} \eta_h\delta^{-1}L^2 \Exop\left[e_h^i\right] \\  
        &+ \left(1 - 2 \tau \eta_h (1-\delta)\right) \Exop\left[\| \vecpi_h^i - \vecpi^* \|_2^2\right],
\end{align*}
for monotone payoffs for arbitrary $\delta \in (0,1)$.
\end{lemma}
Once again, the full proof has been postponed to Appendix~\ref{sec:proof_lemma_bandit_recurrence}.
The proof of Lemma~\ref{lemma:bandit_main_recurrence} follows a similar path as in the recurrence in the full feedback case (Lemma~\ref{lemma:full_error_recurrence}), with the exception that $\vecr_{h,t}^i$ has been replaced by the importance sampling estimator $\widehat{\vecr}^i_h$.
In the decomposition due to Inequality~\eqref{ineq:decomp_abc_full_recur}, the analysis of term (a) remains the same, whereas the terms (b), (c) must be further analyzed using Lemma~\ref{lemma:exploration_bias_trpa_bandit} to account for deviations between $\vecr_{h,t}^i$ and $\widehat{\vecr}^i_h$, as well as the $\varepsilon$ fraction of the population now exploring each round.

The recurrence in Lemma~\ref{lemma:bandit_main_recurrence} is similar in form to the full feedback case (Lemma~\ref{lemma:full_error_recurrence}), apart from the term $8 K^{\sfrac{3}{2}}\eta_h \sqrt{1+\sigma^2}  \exp\{-\varepsilon T_h\}$ due to the exploration scheme.
However, keeping the exploration epoch lengths $T_h$ logarithmically large can make this term small.
Furthermore, once again the recursion produces a dependence on expected mean policy deviations, $\Exop[e_h^i]$.
Hence, the expected policy deviation $\Exop[e_h^i]$ at epoch $h$ must be bounded once again at a rate $\mathcal{O}(\sfrac{1}{h})$ in order to obtain a non-vacuous upper bound on exploitability.
As in the full feedback case, we employ regularization to ensure $\Exop[e_h^i]$ is small.
The next lemma presents our upper bound.

\begin{lemma}[Policy deviation under TRPA-Bandit]\label{lemma:bandit_pol_deviation}
Under TRPA-Bandit dynamics, with learning rates $\eta_h:=\frac{\tau^{-1}}{h+2}$, arbitrary exploration rate $\varepsilon > 0$ and epoch lengths $T_h := \lceil \varepsilon^{-1} \log(h+2) \rceil$
it holds for any $i,j \in \{1, \ldots, N \}, i\neq j$ and each epoch $h\geq 0$ that
\begin{align*}
    \Exop[e_h^i] \leq \frac{24\tau^{-2} K^3 (\sigma^2 + 2) + 48\tau^{-2} + 24}{h+1} + \frac{16\tau^{-2} L ^ 2}{N^2}.
\end{align*}
\end{lemma}
The proof of Lemma~\ref{lemma:bandit_pol_deviation} follows similar ideas to Lemma~\ref{lemma:policy_variations_bound_trpa_full}, while accounting for (a) the increased variance due to importance sampling, and (b) potential further deviation between agent policies due to the $\mathcal{O}(\sfrac{1}{N})$ impact of exploration on $\widehat{\vecmu}_{h,t}$.
In particular, an additional source of policy deviation occurs when an agent does not explore in a given epoch, in which case the payoff estimator is uninformative ($\widehat{\vecr}_h^i = \veczero$) causing additional policy deviation.
The full proof has been postponed to Appendix~\ref{sec:proof_lemma_bandit_pol_dev}.

In the case of TRPA-Bandit, due to the additional variance of probabilistic exploration, the policy deviations between agents might be larger: compare the upper bounds of Lemma~\ref{lemma:bandit_pol_deviation} and Lemma~\ref{lemma:policy_variations_bound_trpa_full}.
In particular, the upper bound of Lemma~\ref{lemma:bandit_pol_deviation} contains a non-vanishing term unlike Lemma~\ref{lemma:policy_variations_bound_trpa_full}.
Nevertheless, they can still be controlled of order $\mathcal{O}(\sfrac{1}{(h+1)} + \sfrac{1}{N^2})$, where the additional $\mathcal{O}(\sfrac{1}{N^2})$ term compared to TRPA-Full vanishes very quickly when $N$ is large.

With these intermediate lemmas established, we state and prove the main convergence result for TRPA-Bandit  Theorem~\ref{theorem:bandit_short}, the main result of this work.
We provide asymptotic rates for brevity, although the proof of the theorem provides explicit bounds.

\begin{theorem}[Convergence, bandit feedback]\label{theorem:bandit_short}
Assume $\vecF$ is Lipschitz, monotone.
Assume $N$ agents run TRPA-Bandit (Algorithm~\ref{alg:bandit}) for $T$ time steps with regularization $\tau>0$ and exploration parameter $\varepsilon > 0$, and agents return policies $\{\vecpi^i\}_i$ after executing Algorithm~\ref{alg:bandit}.
Then, for any agent $i \in \setN$ that $\Exop\left[ \setE^i_{\text{exp}}( \{\vecpi^j\}_{j=1}^N ) \right] \leq \widetilde{\mathcal{O}} (\frac{\tau^{-2}\varepsilon^{-\sfrac{1}{2}}}{\sqrt{T}} + \tau^{-1}\varepsilon + \tau + \frac{\tau^{-1}}{\sqrt{N}} + \frac{\tau^{-\sfrac{3}{2}} }{N} )$.
If $\vecF$ is $\lambda$-strongly monotone, then $\Exop\left[ \setE^i_{\text{exp}}( \{\vecpi^j\}_{j=1}^N ) \right] \leq \widetilde{\mathcal{O}} (\frac{\tau^{-\sfrac{3}{2}} \lambda^{-\sfrac{1}{2}} \varepsilon^{-\sfrac{1}{2}}}{\sqrt{T}} + \tau^{-\sfrac{1}{2}} \lambda^{-\sfrac{1}{2}} \varepsilon + \tau + \frac{\tau^{-\sfrac{1}{2}} \lambda^{-\sfrac{1}{2}}}{\sqrt{N}} + \frac{\tau^{-1} \lambda^{-\sfrac{1}{2}} }{N} )$.
\end{theorem}
 The proof follows from a straightforward combination of Lemmas~\ref{lemma:exploration_bias_trpa_bandit}, \ref{lemma:bandit_main_recurrence}, and \ref{lemma:bandit_pol_deviation} as in the full feedback case.
The exact bounds and proof are in the appendix, Section~\ref{sec:bandit_theorem_full}.


Once again, the values of $\tau$ and exploration probability $\varepsilon$ can be chosen to incorporate tolerable exploitability.
In the case where the number of participants $N$ in the game is known, the following corollary indicates the asymptotically optimal choices for the hyperparameters.

\begin{corollary}[Optimal $\varepsilon, \tau$, bandit feedback]\label{corollary:bandit}
Assume the conditions of Theorem~\ref{theorem:bandit_short} for $N$ agents running TRPA-Bandit.
For monotone $\vecF$, choosing $\tau = \sfrac{1}{\sqrt[4]{N}}$ and $\varepsilon = \sfrac{1}{\sqrt{N}}$ yields
$\Exop\left[\setE^i_{\text{exp}}(\{\vecpi^j\}_{j=1}^N) \right] \leq \widetilde{\mathcal{O}}(\frac{N^{\sfrac{3}{4}}}{\sqrt{T}} + \frac{1}{\sqrt[4]{N}})$ for any $i$.
For strongly monotone $\vecF$, choosing $\tau = \sfrac{1}{\sqrt[3]{N}}$ and $\varepsilon = \sfrac{1}{\sqrt{N}}$ yields $\Exop\left[\setE^i_{\text{exp}}(\{\vecpi^j\}_{j=1}^N) \right] \leq \widetilde{\mathcal{O}}(\frac{N^{\sfrac{3}{4}} \lambda^{-\sfrac{1}{2}}}{\sqrt{T}} + \frac{\lambda^{-\sfrac{1}{2}}}{\sqrt[3]{N}})$.
\end{corollary}

The dependence of $N$ of the sample complexity in the bandit case is worse compared to the full feedback setting as expected: intuitively the agents must take turns to estimate the payoffs of each action in bandit feedback.
Furthermore, while our problem framework is different and a direct comparison is difficult in terms of bounds, we point out that classical MMAB results such as \citep{lugosi2022multiplayer} have a linear dependence on $N$, while in our case the dependence on $N$ scales with $N^{\sfrac{3}{4}}$.
We emphasize that the time-dependence is sublinear in terms of $N$, up to the non-vanishing finite population bias.
As in the full feedback case, the non-vanishing finite population bias in the bandit feedback case scales with $\mathcal{O}(\sfrac{1}{\sqrt[4]{N}})$ or $\mathcal{O}(\sfrac{1}{\sqrt[3]{N}})$, rather than $\mathcal{O}(\sfrac{1}{\sqrt{N}})$ which would match Theorem~\ref{theorem:mfg_ne}.
Note that the dependence of the bias on $N$ varies in various mean-field game results \citep{saldi2019approximate}, but asymptotically is known to converge to zero as $N\rightarrow\infty$, as our explicit finite-agent bound also demonstrates.

Finally, we note that as expected the algorithm for bandit feedback has a worse dependency on the number of actions.
This is as expected due to the fact that (i) the importance sampling estimator increases variance on payoff estimators by a factor of $K$, and (2) in other words, a factor of $\mathcal{O}(K)$ is introduced in order to explore all actions.


\section{Experiments}
\label{sec:experiments}
The experiments are designed to address two key research questions.
First, \textbf{RQ1} evaluates whether the average $L_2$-norm of the counterfactual perturbation vectors ($\overline{||\perturb||}$) decreases as the model overfits the data, thereby providing further empirical validation for our hypothesis.
Second, \textbf{RQ2} evaluates the ability of the proposed counterfactual regularized loss, as defined in (\ref{eq:regularized_loss2}), to mitigate overfitting when compared to existing regularization techniques.

% The experiments are designed to address three key research questions. First, \textbf{RQ1} investigates whether the mean perturbation vector norm decreases as the model overfits the data, aiming to further validate our intuition. Second, \textbf{RQ2} explores whether the mean perturbation vector norm can be effectively leveraged as a regularization term during training, offering insights into its potential role in mitigating overfitting. Finally, \textbf{RQ3} examines whether our counterfactual regularizer enables the model to achieve superior performance compared to existing regularization methods, thus highlighting its practical advantage.

\subsection{Experimental Setup}
\textbf{\textit{Datasets, Models, and Tasks.}}
The experiments are conducted on three datasets: \textit{Water Potability}~\cite{kadiwal2020waterpotability}, \textit{Phomene}~\cite{phomene}, and \textit{CIFAR-10}~\cite{krizhevsky2009learning}. For \textit{Water Potability} and \textit{Phomene}, we randomly select $80\%$ of the samples for the training set, and the remaining $20\%$ for the test set, \textit{CIFAR-10} comes already split. Furthermore, we consider the following models: Logistic Regression, Multi-Layer Perceptron (MLP) with 100 and 30 neurons on each hidden layer, and PreactResNet-18~\cite{he2016cvecvv} as a Convolutional Neural Network (CNN) architecture.
We focus on binary classification tasks and leave the extension to multiclass scenarios for future work. However, for datasets that are inherently multiclass, we transform the problem into a binary classification task by selecting two classes, aligning with our assumption.

\smallskip
\noindent\textbf{\textit{Evaluation Measures.}} To characterize the degree of overfitting, we use the test loss, as it serves as a reliable indicator of the model's generalization capability to unseen data. Additionally, we evaluate the predictive performance of each model using the test accuracy.

\smallskip
\noindent\textbf{\textit{Baselines.}} We compare CF-Reg with the following regularization techniques: L1 (``Lasso''), L2 (``Ridge''), and Dropout.

\smallskip
\noindent\textbf{\textit{Configurations.}}
For each model, we adopt specific configurations as follows.
\begin{itemize}
\item \textit{Logistic Regression:} To induce overfitting in the model, we artificially increase the dimensionality of the data beyond the number of training samples by applying a polynomial feature expansion. This approach ensures that the model has enough capacity to overfit the training data, allowing us to analyze the impact of our counterfactual regularizer. The degree of the polynomial is chosen as the smallest degree that makes the number of features greater than the number of data.
\item \textit{Neural Networks (MLP and CNN):} To take advantage of the closed-form solution for computing the optimal perturbation vector as defined in (\ref{eq:opt-delta}), we use a local linear approximation of the neural network models. Hence, given an instance $\inst_i$, we consider the (optimal) counterfactual not with respect to $\model$ but with respect to:
\begin{equation}
\label{eq:taylor}
    \model^{lin}(\inst) = \model(\inst_i) + \nabla_{\inst}\model(\inst_i)(\inst - \inst_i),
\end{equation}
where $\model^{lin}$ represents the first-order Taylor approximation of $\model$ at $\inst_i$.
Note that this step is unnecessary for Logistic Regression, as it is inherently a linear model.
\end{itemize}

\smallskip
\noindent \textbf{\textit{Implementation Details.}} We run all experiments on a machine equipped with an AMD Ryzen 9 7900 12-Core Processor and an NVIDIA GeForce RTX 4090 GPU. Our implementation is based on the PyTorch Lightning framework. We use stochastic gradient descent as the optimizer with a learning rate of $\eta = 0.001$ and no weight decay. We use a batch size of $128$. The training and test steps are conducted for $6000$ epochs on the \textit{Water Potability} and \textit{Phoneme} datasets, while for the \textit{CIFAR-10} dataset, they are performed for $200$ epochs.
Finally, the contribution $w_i^{\varepsilon}$ of each training point $\inst_i$ is uniformly set as $w_i^{\varepsilon} = 1~\forall i\in \{1,\ldots,m\}$.

The source code implementation for our experiments is available at the following GitHub repository: \url{https://anonymous.4open.science/r/COCE-80B4/README.md} 

\subsection{RQ1: Counterfactual Perturbation vs. Overfitting}
To address \textbf{RQ1}, we analyze the relationship between the test loss and the average $L_2$-norm of the counterfactual perturbation vectors ($\overline{||\perturb||}$) over training epochs.

In particular, Figure~\ref{fig:delta_loss_epochs} depicts the evolution of $\overline{||\perturb||}$ alongside the test loss for an MLP trained \textit{without} regularization on the \textit{Water Potability} dataset. 
\begin{figure}[ht]
    \centering
    \includegraphics[width=0.85\linewidth]{img/delta_loss_epochs.png}
    \caption{The average counterfactual perturbation vector $\overline{||\perturb||}$ (left $y$-axis) and the cross-entropy test loss (right $y$-axis) over training epochs ($x$-axis) for an MLP trained on the \textit{Water Potability} dataset \textit{without} regularization.}
    \label{fig:delta_loss_epochs}
\end{figure}

The plot shows a clear trend as the model starts to overfit the data (evidenced by an increase in test loss). 
Notably, $\overline{||\perturb||}$ begins to decrease, which aligns with the hypothesis that the average distance to the optimal counterfactual example gets smaller as the model's decision boundary becomes increasingly adherent to the training data.

It is worth noting that this trend is heavily influenced by the choice of the counterfactual generator model. In particular, the relationship between $\overline{||\perturb||}$ and the degree of overfitting may become even more pronounced when leveraging more accurate counterfactual generators. However, these models often come at the cost of higher computational complexity, and their exploration is left to future work.

Nonetheless, we expect that $\overline{||\perturb||}$ will eventually stabilize at a plateau, as the average $L_2$-norm of the optimal counterfactual perturbations cannot vanish to zero.

% Additionally, the choice of employing the score-based counterfactual explanation framework to generate counterfactuals was driven to promote computational efficiency.

% Future enhancements to the framework may involve adopting models capable of generating more precise counterfactuals. While such approaches may yield to performance improvements, they are likely to come at the cost of increased computational complexity.


\subsection{RQ2: Counterfactual Regularization Performance}
To answer \textbf{RQ2}, we evaluate the effectiveness of the proposed counterfactual regularization (CF-Reg) by comparing its performance against existing baselines: unregularized training loss (No-Reg), L1 regularization (L1-Reg), L2 regularization (L2-Reg), and Dropout.
Specifically, for each model and dataset combination, Table~\ref{tab:regularization_comparison} presents the mean value and standard deviation of test accuracy achieved by each method across 5 random initialization. 

The table illustrates that our regularization technique consistently delivers better results than existing methods across all evaluated scenarios, except for one case -- i.e., Logistic Regression on the \textit{Phomene} dataset. 
However, this setting exhibits an unusual pattern, as the highest model accuracy is achieved without any regularization. Even in this case, CF-Reg still surpasses other regularization baselines.

From the results above, we derive the following key insights. First, CF-Reg proves to be effective across various model types, ranging from simple linear models (Logistic Regression) to deep architectures like MLPs and CNNs, and across diverse datasets, including both tabular and image data. 
Second, CF-Reg's strong performance on the \textit{Water} dataset with Logistic Regression suggests that its benefits may be more pronounced when applied to simpler models. However, the unexpected outcome on the \textit{Phoneme} dataset calls for further investigation into this phenomenon.


\begin{table*}[h!]
    \centering
    \caption{Mean value and standard deviation of test accuracy across 5 random initializations for different model, dataset, and regularization method. The best results are highlighted in \textbf{bold}.}
    \label{tab:regularization_comparison}
    \begin{tabular}{|c|c|c|c|c|c|c|}
        \hline
        \textbf{Model} & \textbf{Dataset} & \textbf{No-Reg} & \textbf{L1-Reg} & \textbf{L2-Reg} & \textbf{Dropout} & \textbf{CF-Reg (ours)} \\ \hline
        Logistic Regression   & \textit{Water}   & $0.6595 \pm 0.0038$   & $0.6729 \pm 0.0056$   & $0.6756 \pm 0.0046$  & N/A    & $\mathbf{0.6918 \pm 0.0036}$                     \\ \hline
        MLP   & \textit{Water}   & $0.6756 \pm 0.0042$   & $0.6790 \pm 0.0058$   & $0.6790 \pm 0.0023$  & $0.6750 \pm 0.0036$    & $\mathbf{0.6802 \pm 0.0046}$                    \\ \hline
%        MLP   & \textit{Adult}   & $0.8404 \pm 0.0010$   & $\mathbf{0.8495 \pm 0.0007}$   & $0.8489 \pm 0.0014$  & $\mathbf{0.8495 \pm 0.0016}$     & $0.8449 \pm 0.0019$                    \\ \hline
        Logistic Regression   & \textit{Phomene}   & $\mathbf{0.8148 \pm 0.0020}$   & $0.8041 \pm 0.0028$   & $0.7835 \pm 0.0176$  & N/A    & $0.8098 \pm 0.0055$                     \\ \hline
        MLP   & \textit{Phomene}   & $0.8677 \pm 0.0033$   & $0.8374 \pm 0.0080$   & $0.8673 \pm 0.0045$  & $0.8672 \pm 0.0042$     & $\mathbf{0.8718 \pm 0.0040}$                    \\ \hline
        CNN   & \textit{CIFAR-10} & $0.6670 \pm 0.0233$   & $0.6229 \pm 0.0850$   & $0.7348 \pm 0.0365$   & N/A    & $\mathbf{0.7427 \pm 0.0571}$                     \\ \hline
    \end{tabular}
\end{table*}

\begin{table*}[htb!]
    \centering
    \caption{Hyperparameter configurations utilized for the generation of Table \ref{tab:regularization_comparison}. For our regularization the hyperparameters are reported as $\mathbf{\alpha/\beta}$.}
    \label{tab:performance_parameters}
    \begin{tabular}{|c|c|c|c|c|c|c|}
        \hline
        \textbf{Model} & \textbf{Dataset} & \textbf{No-Reg} & \textbf{L1-Reg} & \textbf{L2-Reg} & \textbf{Dropout} & \textbf{CF-Reg (ours)} \\ \hline
        Logistic Regression   & \textit{Water}   & N/A   & $0.0093$   & $0.6927$  & N/A    & $0.3791/1.0355$                     \\ \hline
        MLP   & \textit{Water}   & N/A   & $0.0007$   & $0.0022$  & $0.0002$    & $0.2567/1.9775$                    \\ \hline
        Logistic Regression   &
        \textit{Phomene}   & N/A   & $0.0097$   & $0.7979$  & N/A    & $0.0571/1.8516$                     \\ \hline
        MLP   & \textit{Phomene}   & N/A   & $0.0007$   & $4.24\cdot10^{-5}$  & $0.0015$    & $0.0516/2.2700$                    \\ \hline
       % MLP   & \textit{Adult}   & N/A   & $0.0018$   & $0.0018$  & $0.0601$     & $0.0764/2.2068$                    \\ \hline
        CNN   & \textit{CIFAR-10} & N/A   & $0.0050$   & $0.0864$ & N/A    & $0.3018/
        2.1502$                     \\ \hline
    \end{tabular}
\end{table*}

\begin{table*}[htb!]
    \centering
    \caption{Mean value and standard deviation of training time across 5 different runs. The reported time (in seconds) corresponds to the generation of each entry in Table \ref{tab:regularization_comparison}. Times are }
    \label{tab:times}
    \begin{tabular}{|c|c|c|c|c|c|c|}
        \hline
        \textbf{Model} & \textbf{Dataset} & \textbf{No-Reg} & \textbf{L1-Reg} & \textbf{L2-Reg} & \textbf{Dropout} & \textbf{CF-Reg (ours)} \\ \hline
        Logistic Regression   & \textit{Water}   & $222.98 \pm 1.07$   & $239.94 \pm 2.59$   & $241.60 \pm 1.88$  & N/A    & $251.50 \pm 1.93$                     \\ \hline
        MLP   & \textit{Water}   & $225.71 \pm 3.85$   & $250.13 \pm 4.44$   & $255.78 \pm 2.38$  & $237.83 \pm 3.45$    & $266.48 \pm 3.46$                    \\ \hline
        Logistic Regression   & \textit{Phomene}   & $266.39 \pm 0.82$ & $367.52 \pm 6.85$   & $361.69 \pm 4.04$  & N/A   & $310.48 \pm 0.76$                    \\ \hline
        MLP   &
        \textit{Phomene} & $335.62 \pm 1.77$   & $390.86 \pm 2.11$   & $393.96 \pm 1.95$ & $363.51 \pm 5.07$    & $403.14 \pm 1.92$                     \\ \hline
       % MLP   & \textit{Adult}   & N/A   & $0.0018$   & $0.0018$  & $0.0601$     & $0.0764/2.2068$                    \\ \hline
        CNN   & \textit{CIFAR-10} & $370.09 \pm 0.18$   & $395.71 \pm 0.55$   & $401.38 \pm 0.16$ & N/A    & $1287.8 \pm 0.26$                     \\ \hline
    \end{tabular}
\end{table*}

\subsection{Feasibility of our Method}
A crucial requirement for any regularization technique is that it should impose minimal impact on the overall training process.
In this respect, CF-Reg introduces an overhead that depends on the time required to find the optimal counterfactual example for each training instance. 
As such, the more sophisticated the counterfactual generator model probed during training the higher would be the time required. However, a more advanced counterfactual generator might provide a more effective regularization. We discuss this trade-off in more details in Section~\ref{sec:discussion}.

Table~\ref{tab:times} presents the average training time ($\pm$ standard deviation) for each model and dataset combination listed in Table~\ref{tab:regularization_comparison}.
We can observe that the higher accuracy achieved by CF-Reg using the score-based counterfactual generator comes with only minimal overhead. However, when applied to deep neural networks with many hidden layers, such as \textit{PreactResNet-18}, the forward derivative computation required for the linearization of the network introduces a more noticeable computational cost, explaining the longer training times in the table.

\subsection{Hyperparameter Sensitivity Analysis}
The proposed counterfactual regularization technique relies on two key hyperparameters: $\alpha$ and $\beta$. The former is intrinsic to the loss formulation defined in (\ref{eq:cf-train}), while the latter is closely tied to the choice of the score-based counterfactual explanation method used.

Figure~\ref{fig:test_alpha_beta} illustrates how the test accuracy of an MLP trained on the \textit{Water Potability} dataset changes for different combinations of $\alpha$ and $\beta$.

\begin{figure}[ht]
    \centering
    \includegraphics[width=0.85\linewidth]{img/test_acc_alpha_beta.png}
    \caption{The test accuracy of an MLP trained on the \textit{Water Potability} dataset, evaluated while varying the weight of our counterfactual regularizer ($\alpha$) for different values of $\beta$.}
    \label{fig:test_alpha_beta}
\end{figure}

We observe that, for a fixed $\beta$, increasing the weight of our counterfactual regularizer ($\alpha$) can slightly improve test accuracy until a sudden drop is noticed for $\alpha > 0.1$.
This behavior was expected, as the impact of our penalty, like any regularization term, can be disruptive if not properly controlled.

Moreover, this finding further demonstrates that our regularization method, CF-Reg, is inherently data-driven. Therefore, it requires specific fine-tuning based on the combination of the model and dataset at hand.

\section{Discussion of Assumptions}\label{sec:discussion}
In this paper, we have made several assumptions for the sake of clarity and simplicity. In this section, we discuss the rationale behind these assumptions, the extent to which these assumptions hold in practice, and the consequences for our protocol when these assumptions hold.

\subsection{Assumptions on the Demand}

There are two simplifying assumptions we make about the demand. First, we assume the demand at any time is relatively small compared to the channel capacities. Second, we take the demand to be constant over time. We elaborate upon both these points below.

\paragraph{Small demands} The assumption that demands are small relative to channel capacities is made precise in \eqref{eq:large_capacity_assumption}. This assumption simplifies two major aspects of our protocol. First, it largely removes congestion from consideration. In \eqref{eq:primal_problem}, there is no constraint ensuring that total flow in both directions stays below capacity--this is always met. Consequently, there is no Lagrange multiplier for congestion and no congestion pricing; only imbalance penalties apply. In contrast, protocols in \cite{sivaraman2020high, varma2021throughput, wang2024fence} include congestion fees due to explicit congestion constraints. Second, the bound \eqref{eq:large_capacity_assumption} ensures that as long as channels remain balanced, the network can always meet demand, no matter how the demand is routed. Since channels can rebalance when necessary, they never drop transactions. This allows prices and flows to adjust as per the equations in \eqref{eq:algorithm}, which makes it easier to prove the protocol's convergence guarantees. This also preserves the key property that a channel's price remains proportional to net money flow through it.

In practice, payment channel networks are used most often for micro-payments, for which on-chain transactions are prohibitively expensive; large transactions typically take place directly on the blockchain. For example, according to \cite{river2023lightning}, the average channel capacity is roughly $0.1$ BTC ($5,000$ BTC distributed over $50,000$ channels), while the average transaction amount is less than $0.0004$ BTC ($44.7k$ satoshis). Thus, the small demand assumption is not too unrealistic. Additionally, the occasional large transaction can be treated as a sequence of smaller transactions by breaking it into packets and executing each packet serially (as done by \cite{sivaraman2020high}).
Lastly, a good path discovery process that favors large capacity channels over small capacity ones can help ensure that the bound in \eqref{eq:large_capacity_assumption} holds.

\paragraph{Constant demands} 
In this work, we assume that any transacting pair of nodes have a steady transaction demand between them (see Section \ref{sec:transaction_requests}). Making this assumption is necessary to obtain the kind of guarantees that we have presented in this paper. Unless the demand is steady, it is unreasonable to expect that the flows converge to a steady value. Weaker assumptions on the demand lead to weaker guarantees. For example, with the more general setting of stochastic, but i.i.d. demand between any two nodes, \cite{varma2021throughput} shows that the channel queue lengths are bounded in expectation. If the demand can be arbitrary, then it is very hard to get any meaningful performance guarantees; \cite{wang2024fence} shows that even for a single bidirectional channel, the competitive ratio is infinite. Indeed, because a PCN is a decentralized system and decisions must be made based on local information alone, it is difficult for the network to find the optimal detailed balance flow at every time step with a time-varying demand.  With a steady demand, the network can discover the optimal flows in a reasonably short time, as our work shows.

We view the constant demand assumption as an approximation for a more general demand process that could be piece-wise constant, stochastic, or both (see simulations in Figure \ref{fig:five_nodes_variable_demand}).
We believe it should be possible to merge ideas from our work and \cite{varma2021throughput} to provide guarantees in a setting with random demands with arbitrary means. We leave this for future work. In addition, our work suggests that a reasonable method of handling stochastic demands is to queue the transaction requests \textit{at the source node} itself. This queuing action should be viewed in conjunction with flow-control. Indeed, a temporarily high unidirectional demand would raise prices for the sender, incentivizing the sender to stop sending the transactions. If the sender queues the transactions, they can send them later when prices drop. This form of queuing does not require any overhaul of the basic PCN infrastructure and is therefore simpler to implement than per-channel queues as suggested by \cite{sivaraman2020high} and \cite{varma2021throughput}.

\subsection{The Incentive of Channels}
The actions of the channels as prescribed by the DEBT control protocol can be summarized as follows. Channels adjust their prices in proportion to the net flow through them. They rebalance themselves whenever necessary and execute any transaction request that has been made of them. We discuss both these aspects below.

\paragraph{On Prices}
In this work, the exclusive role of channel prices is to ensure that the flows through each channel remains balanced. In practice, it would be important to include other components in a channel's price/fee as well: a congestion price  and an incentive price. The congestion price, as suggested by \cite{varma2021throughput}, would depend on the total flow of transactions through the channel, and would incentivize nodes to balance the load over different paths. The incentive price, which is commonly used in practice \cite{river2023lightning}, is necessary to provide channels with an incentive to serve as an intermediary for different channels. In practice, we expect both these components to be smaller than the imbalance price. Consequently, we expect the behavior of our protocol to be similar to our theoretical results even with these additional prices.

A key aspect of our protocol is that channel fees are allowed to be negative. Although the original Lightning network whitepaper \cite{poon2016bitcoin} suggests that negative channel prices may be a good solution to promote rebalancing, the idea of negative prices in not very popular in the literature. To our knowledge, the only prior work with this feature is \cite{varma2021throughput}. Indeed, in papers such as \cite{van2021merchant} and \cite{wang2024fence}, the price function is explicitly modified such that the channel price is never negative. The results of our paper show the benefits of negative prices. For one, in steady state, equal flows in both directions ensure that a channel doesn't loose any money (the other price components mentioned above ensure that the channel will only gain money). More importantly, negative prices are important to ensure that the protocol selectively stifles acyclic flows while allowing circulations to flow. Indeed, in the example of Section \ref{sec:flow_control_example}, the flows between nodes $A$ and $C$ are left on only because the large positive price over one channel is canceled by the corresponding negative price over the other channel, leading to a net zero price.

Lastly, observe that in the DEBT control protocol, the price charged by a channel does not depend on its capacity. This is a natural consequence of the price being the Lagrange multiplier for the net-zero flow constraint, which also does not depend on the channel capacity. In contrast, in many other works, the imbalance price is normalized by the channel capacity \cite{ren2018optimal, lin2020funds, wang2024fence}; this is shown to work well in practice. The rationale for such a price structure is explained well in \cite{wang2024fence}, where this fee is derived with the aim of always maintaining some balance (liquidity) at each end of every channel. This is a reasonable aim if a channel is to never rebalance itself; the experiments of the aforementioned papers are conducted in such a regime. In this work, however, we allow the channels to rebalance themselves a few times in order to settle on a detailed balance flow. This is because our focus is on the long-term steady state performance of the protocol. This difference in perspective also shows up in how the price depends on the channel imbalance. \cite{lin2020funds} and \cite{wang2024fence} advocate for strictly convex prices whereas this work and \cite{varma2021throughput} propose linear prices.

\paragraph{On Rebalancing} 
Recall that the DEBT control protocol ensures that the flows in the network converge to a detailed balance flow, which can be sustained perpetually without any rebalancing. However, during the transient phase (before convergence), channels may have to perform on-chain rebalancing a few times. Since rebalancing is an expensive operation, it is worthwhile discussing methods by which channels can reduce the extent of rebalancing. One option for the channels to reduce the extent of rebalancing is to increase their capacity; however, this comes at the cost of locking in more capital. Each channel can decide for itself the optimum amount of capital to lock in. Another option, which we discuss in Section \ref{sec:five_node}, is for channels to increase the rate $\gamma$ at which they adjust prices. 

Ultimately, whether or not it is beneficial for a channel to rebalance depends on the time-horizon under consideration. Our protocol is based on the assumption that the demand remains steady for a long period of time. If this is indeed the case, it would be worthwhile for a channel to rebalance itself as it can make up this cost through the incentive fees gained from the flow of transactions through it in steady state. If a channel chooses not to rebalance itself, however, there is a risk of being trapped in a deadlock, which is suboptimal for not only the nodes but also the channel.

\section{Conclusion}
This work presents DEBT control: a protocol for payment channel networks that uses source routing and flow control based on channel prices. The protocol is derived by posing a network utility maximization problem and analyzing its dual minimization. It is shown that under steady demands, the protocol guides the network to an optimal, sustainable point. Simulations show its robustness to demand variations. The work demonstrates that simple protocols with strong theoretical guarantees are possible for PCNs and we hope it inspires further theoretical research in this direction.

\section*{Acknowledgements}

We thank the editor and reviewers for their detailed comments and insightful feedback. This project was supported by Swiss National Science Foundation (SNSF) under the framework of NCCR Automation and SNSF Starting Grant.
S. Cayci's research was funded under the Excellence Strategy of the Federal Government and the L{\"a}nder.

\bibliography{sn-bibliography}

\appendix

\section{Table for Notation}
\begin{longtable}{ p{.23\textwidth}  p{.77\textwidth} } 
\underline{General notation:} & \\
$\Delta_\setX$ & probability simplex on discrete set $\setX$ \\ 
$[M]$ & $:= \{1, \ldots, M\}$, for any $M\in\mathbb{N}_{>0}$ \\
$\Delta_{\setX, N}$ & $:= \{ \vecu = \{u_i\}_i \in\Delta_\setX | N u_i \in \mathbb{N}_{\geq 0} \}$, for discrete set $\setX$, $N\in\mathbb{N}_{>0}$ \\
$\mathbb{M}^{D_1,D_2}$ & $D_1 \times D_2$ matrices \\
$\mathbb{S}_{++}^D$ & positive definite $D \times D$ matrices \\
$\Pi_K$ & projection onto convex, compact set $K\subset \mathbb{R}^H$\\
$\vece_a$ & $\in \mathbb{R}^\setA$, standard unit vector with coordinate $a$ set to 1 \\
 &  \\ 
\underline{For SMFGs:} & \\
$\vecF$ & payoff function, $\vecF: \Delta_\setA \rightarrow [0,1]^K$  \\
$\lambda$ & strong monotonicity modulus of $\vecF$ \\  
$L$ & Lipschitz modulus of $\vecF$ \\
$\tau$ & Tikhonov regularization parameter \\ 
$\vecpi^*$ & unique solution of $\tau$ regularized \eqref{eq:mfg_rvi_statement} \\
$K$ & number of actions \\ 
$N$ & number of players \\  
$\setN$ & $:= \{1, \ldots, N \}$, the set of players \\ 
$\setA$ & the set of actions, $|\setA| = K$. \\ 
$\sigma^2$ & upper bound of the standard deviation of received payoff \\
$\widehat{\vecmu}$ & $:= \frac{1}{N} \sum_{i=1}^N \vece_{a^i}$, when actions $\{a^i\}_{i=1}^N$ clear in context \\
$V^i(\vecpi^1, \ldots, \vecpi^N)$ & expected payoff of player $i$ under strategy profile $(\vecpi^1, \ldots, \vecpi^N)$ \\
$\setE^i_{\text{exp}}(\{\vecpi^j\}_{j=1}^N)$ & $:= \max_{\vecpi'\in \Delta_\setA} V^i(\vecpi', \vecpi^{-i}) - V^i(\vecpi^1, \ldots, \vecpi^N)$, exploitability \\
$\widehat{\vecmu}(\{a^i\}_{i=1}^N)$ & $:= \frac{1}{N} \sum_{i=1}^N \vece_{a^i}$, the action distribution induced by particular $\{a^i\}_{i=1}^N \in \setA^N$ \\
 &  \\ 
\underline{For full feedback:} & \\
$t$ & round of play \\
$a^i_t$ & $\in\setA$ action take by player $i$ at round $t$ \\
$\widehat{\vecmu}_t $ & $:= \frac{1}{N} \sum_{i=1}^N \vece_{a_t^i}$, empirical distribution over actions $\setA$ on round $t$ \\ 
$\vecr^i_t$ & $:= \vecF(\widehat{\vecmu}_t) + \vecn_t^i$, payoff vector observed by player $i$ \\ 
$\eta_t$ & learning rate \\
$ \bar{\vecmu}_t $ & $ := \frac{1}{N} \sum_{i=1}^N \vecpi_t^i,$  mean policy at round $t$\\ 
$e_t^i$ & $:= \|\vecpi^i_t - \bar{\vecmu}_t \|_2^2 $, deviation of policy of player $i$ \\
$u_t^i $ & $:= \Exop\left[\| \vecpi_t^i - \vecpi^* \|_2^2\right]$ \\
 &  \\ 
\underline{For bandit feedback:} & \\
$h$ & epoch of play \\
$t$ & round of play in epoch \\
$a^i_{h,t}$ & $\in\setA$ action take by player $i$ at round $t$ of epoch $h$ \\
$T_h$ & number of rounds in epoch $h$ \\
$\varepsilon$ & exploration probability \\
$\widehat{\vecmu}_{h,t} $ & $:= \frac{1}{N} \sum_{i=1}^N \vece_{a_{h, t}^i}$, empirical distribution over actions $\setA$ on round $t$ of epoch $h$ \\ 
$\vecr^i_{h,t}$ & $:= \vecF(\widehat{\vecmu}_{h,t}) + \vecn_{h,t}^i$, (unobserved) payoff vector by player $i$ \\ 
$r^i_{h,t}$ & $:= \vecr^i_{h,t} (a^i_{h,t})$, payoff observed by player $i$ at round $t$ of epoch $h$ \\
$X_{h,t}^i$ & $\sim \operatorname{Ber}(\varepsilon)$ indicator variable, $1$ if $i$ explores at round $t$ of epoch $h$ \\
$\widehat{\vecr}^i_h $ & $:= K r^i_{h,t}\vece_{a^i_{h,t}}$ if $X_{h,t}^i=1$, importance sampling estimate of player $i$ \\
$\eta_h$ & learning rate \\
$ \bar{\vecmu}_h $ & $ := \frac{1}{N} \sum_{i=1}^N \vecpi_h^i,$  mean policy at epoch $h$\\ 
$e_h^i$ & $:= \|\vecpi^i_h - \bar{\vecmu}_h \|_2^2 $, deviation of policy of player $i$ \\
$u_h^i $ & $:= \Exop\left[\| \vecpi_h^i - \vecpi^* \|_2^2\right]$ \\
\end{longtable}


\section{A Detailed Comparison to the Setting in \cite{gummadi2013mean}}\label{sec:detailed_comparison}

Since specific keywords seem to correspond to the works on mean-field approximations with bandits, we provide a greater discussion of our setting and the results by \citet{gummadi2013mean}.
In general, our settings and models are very different, hence almost none of the results between our work and Gummadi et al. are transferable to the other.
Our problem formulation, analysis, and results are fundamentally different from their setting due to the following points.

\textbf{Stationary equilibrium vs Nash equilibrium.}
The most critical difference between the two works is the solution concepts.
Our setting is competitive, as a natural extension, the solution concept is that of a Nash equilibrium where each agent has no incentive to change their policy.
On the other hand, the setting of Gummadi et al. need not be competitive or collaborative and this distinction is not significant for their framework, their goal is to characterize convergence of the population to a stationary distribution.
Their main results show that if a particular policy map $\sigma: \mathbb{Z}_{\geq 0}^{2 n} \rightarrow \Delta_\setA$ is prescribed on agents, the population distribution will converge to a steady state.
The equilibrium concept of \cite{gummadi2013mean} is not \emph{Nash}, rather stationarity.

\textbf{Optimality considerations.}
As a consequence of analyzing stationarity, the results by \citep{gummadi2013mean} do not analyze or aim to characterize optimality.
In their analysis, a fixed map $\sigma: \mathbb{Z}_{\geq 0}^{2 n} \rightarrow \Delta_\setA$ is assumed to be the policy/strategy of a continuum of (i.e., infinitely many) agents, which maps observed loss/win counts (from Bernoulli distributed arm rewards) to arm probabilities.
The stationary distribution in general obtained from $\sigma$ in \cite{gummadi2013mean} does not have optimality properties, for instance, a fixed agent will can have arbitrary large exploitability.
The main goal of \cite{gummadi2013mean} is to prove the convergence of the population distribution to a steady state behaviour.

\textbf{Algorithms.}
As a consequence of the previous points, Gummadi et al. abstract away any algorithmic considerations to the fixed map $\sigma$ and the particular algorithms employed by agents do not directly have significance in terms of their theoretical conclusions.
Since we analyze optimality in our setting, we require a specific algorithm to be employed (TRPA and Algorithm~\ref{alg:bandit}).

\textbf{Independent learning.}
In our setting, the notion of learning and independent learning become significant since we are aiming to obtain an approximate NE.
Hence, our theoretical results bound the expected exploitability (Theorems~\ref{theorem:expert_short}, \ref{theorem:bandit_short}) in terms of number of samples.
In the work of \cite{gummadi2013mean}, the main aim is convergence to a steady state rather than learning.

\textbf{Population regeneration.}
Finally, to be able to obtain a contractive mapping yielding a population stationary distribution/steady state, \cite{gummadi2013mean} assume that the population regenerates at a constant rate $\beta$, implying agents are constantly being replaced by oblivious agents that have not observed the game dynamics.
This smooths the dynamics by introducing a forgetting mechanism to game participants.
Our results on the other hand are closer to the traditional bandits/independent learning setting.
For instance, this would correspond to non-vanishing exploitability scaling with $\mathcal{O}(\beta)$ in our system as agents constantly ``forget'' what they learned.

\textbf{Other model differences.}
In our setting, we assume general noisy rewards while in \cite{gummadi2013mean}, the rewards are Bernoulli random variables with success probability dependent on the population.

\section{Formalizing Learning Algorithms}\label{section:alg_formalization}

In this section, we formalize the concept of an independent learning algorithm in the full feedback and bandit feedback setting.
In general, we formalize the notion of an algorithm as a map $A_t^i: \mathcal{H}_{t}^i \rightarrow \Delta_\setA$ that maps the set of past observations of agent $i$ at time $t$ to action selection probabilities.
The definition of the set $\mathcal{H}_{t}^i$ varies between the feedback models.

\begin{definition}[Learning algorithm with full feedback]
\label{definition:alg_expert}
An independent learning algorithm with full information $\mathbf{A} = \{ A_t^i\}_{i,t}$ is a sequence of mappings for each player with
\begin{align*}
    &A_t^i : \Delta_\setA^{t-1} \times \setA^{t-1} \times [0,1]^{(t-1) \times K} \rightarrow \Delta_\setA, \text{ for all $t > 0$}, \\
    &A_0^i \in \Delta_\setA,
\end{align*}
that maps past $t-1$ observations from previous rounds to a mixed strategy on actions $\setA$ at time $t > 0$ for each agent $i$.
\end{definition}

Naturally, we are interested in algorithms that yield a good approximation of the NE in expectation.
More explicitly, we will be interested in designing algorithms that converge to policy profiles with low expected explotability.

\begin{definition}[Rational learning algorithm with full feedback]
Let $\mathbf{A}$ be an algorithm with full feedback as defined in Definition~\ref{definition:alg_expert}.
We call $\mathbf{A}$ $\delta$-rational if it holds that for all $i$, the induced mixed strategies $\vecpi_t^i$ under $\vecpi_0^i = A_0^i, \vecpi_t^i = A_t^i(\vecpi^i_0, \ldots, \vecpi_{t-1}^i, a_0^i, \ldots, a_{t-1}^i, \vecr_{0}^i, \ldots, \vecr_{t-1}^i)$ satisfy
\begin{align*}
    \lim_{t\rightarrow \infty} \Exop[ \setE^i_{\text{exp}}(\{\vecpi_t^j \}_{j=1}^N)] \leq \delta, \text{ for all } i \in \setN.
\end{align*}
\end{definition}

Note that while not specified in the definition above, we will also be interested in the rate of convergence of the exploitability term for a rational algorithm.
Since we are solving the SMFG at the finite-agent regime, we will be interested in $\delta$-rational algorithms that have $\delta \rightarrow 0$ as $N\rightarrow\infty$, that is, the non-vanishing bias should scale inversely with the number of agents.

Finally, we also formalize the concepts of a learning algorithm and $\delta$-rationality in the bandit setting.

\begin{definition}[Algorithm with bandit feedback]
\label{definition:alg_bandit}
An algorithm with bandit feedback $\mathbf{A} = \{ A_t^i\}_{i,t}$ is a sequence of mappings for each player with
\begin{align*}
    &A_t^i : \Delta_\setA^{t-1} \times \setA^{t-1} \times [0,1]^{(t-1) } \rightarrow \Delta_\setA, \text{ for all $t > 0$}, \\
    &A_0^i \in \Delta_\setA,
\end{align*}
that maps past $t-1$ observations from previous rounds at all times $t > 0$ (only including the payoffs of the \emph{played} actions) to a probability distribution on actions $\setA$.
\end{definition}

\begin{definition}[Rational algorithm with bandit feedback]
Let $\mathbf{A}$ be an algorithm with bandit feedback as defined in Definition~\ref{definition:alg_bandit}.
We call $\mathbf{A}$ $\delta$-rational if it holds that for all $i$, the induced (random) mixed strategies $\vecpi_t^i$ under $\vecpi_0^i = A_0^i, \vecpi_t^i = A_t^i(\vecpi^i_0, \ldots, \vecpi_{t-1}^i, a_0^i, \ldots, a_{t-1}^i, r_{0}^i, \ldots, r_{t-1}^i)$ satisfy
\begin{align*}
    \lim_{t\rightarrow \infty} \Exop[ \setE^i_{\text{exp}}(\{\vecpi_t^j \}_{j=1}^N)] \leq \delta, \text{ for all } i \in \setN.
\end{align*}
\end{definition}


\section{Basic Inequalities}\label{app:basic_inequalities}

In our proofs, we will need to repeatedly bound certain recurrences and sums.
In this section, we present useful inequalities to this end.

\begin{lemma}[Harmonic partial sum bound]\label{lemma:harmonic}
For any integers $s,\bar{s}$, constant $c\in\mathbb{R}$ such that $1 \leq \bar{s} < s$, $p \neq -1$, and $a \geq 0$, it holds that
\begin{align*}
   \log (s + a) - \log (\bar{s} + a ) + \frac{1}{s + a} &\leq \sum_{n = \bar{s}}^{s} \frac{1}{n + a} \leq \frac{1}{\bar{s} + a} + \log ( s + a) - \log (\bar{s} + a), \\
   \frac{(s + a)^{p+1}}{p+1} - \frac{(\bar{s} + a)^{p+1}}{(p+1)} + (\bar{s} + a)^p &\leq \sum_{n = \bar{s}}^{s} (n + a)^p \leq \frac{(s + a)^{p+1}}{p+1} - \frac{(\bar{s} + a)^{p+1}}{p+1} + (s + a)^p, \text{ if } p \geq 0 \\
   \frac{(s + a)^{p+1}}{p+1} - \frac{(\bar{s} + a)^{p+1}}{p+1} + (s + a)^p &\leq \sum_{n = \bar{s}}^{s} (n + a)^p \leq \frac{(s + a)^{p+1}}{p+1} - \frac{(\bar{s} + a)^{p+1}}{p+1} + (\bar{s} + a)^p, \text{ if } p \leq 0
\end{align*}
\end{lemma}
\begin{proof}
Let $f_1:\mathbb{R}_{\geq 0} \rightarrow \mathbb{R}_{\geq 0}$ be a non-decreasing positive function and $f_2:\mathbb{R}_{\geq 0} \rightarrow \mathbb{R}_{\geq 0}$ be a non-increasing positive function.
Then it holds that
\begin{align*}
    \int_{x = \bar{s}}^s f_1(x) dx + f_1(\bar{s}) \leq \sum_{n=\bar{s}}^s f_1(n) \leq \int_{x = \bar{s}}^s f_1(x) dx  + f_1(s), \\
    \int_{x = \bar{s}}^s f_2(x) dx + f_2(s) \leq \sum_{n=\bar{s}}^s f_2(n) \leq \int_{x = \bar{s}}^s f_2(x) dx + f_2(\bar{s}).
\end{align*}
The result follows from a simple integral bound with $\int \frac{1}{x} dx = \log x$ and $\int x^p dx = \frac{x^{p+1}}{p+1}$.
\end{proof}

We state a certain recurrence inequality that appears several times in our analysis as a lemma, in order to shorten some proofs.

\begin{lemma}[General error recurrence]\label{lemma:general_recurrence}
Let $c_0 \geq 0, c_1 \geq 0, \gamma > 1, a \geq 0$ be arbitrary constants such that $a \geq \gamma$.
Furthermore, let $\{u_t\}_{t=0}^\infty$ be a sequence of non-negative numbers such that for all $t \geq 0$, it holds that
\begin{align*}
    u_{t+1} \leq \frac{c_0}{t+a} + \frac{c_1}{(t+a)^2} + \left( 1 - \frac{\gamma}{t+a}\right) u_t.
\end{align*}
Then, for all values of $t\geq 0$, it holds that:
\begin{align*}
u_{t+1} \leq &\frac{u_0 a ^ \gamma + c_1 (a^{\gamma-2} + 1) (1+a^{-1})^\gamma + c_0 (1+a^{-1})^\gamma a^{\gamma-1}}{\left(t+a\right)^\gamma}  \\
    &\quad + \frac{c_0 + c_1 (1+a^{-1})^\gamma (\gamma - 1)^{-1}}{t+a} + \frac{c_1}{(t+a)^2} + \gamma^{-1}(1+a^{-1})^\gamma c_0 
\end{align*}
    
\end{lemma}
\begin{proof}
We note that inductively, we have
\begin{align*}
u_{t+1} \leq &\frac{c_0}{t+a} + \frac{c_1}{(t+a)^2} + u_0 \prod_{s=0}^{t} \left( 1 - \frac{\gamma}{s+a} \right) 
     + \sum_{s=0}^{t-1} \left( \frac{c_0}{s+a} + \frac{c_1}{(s+a)^2} \right) \prod_{s'=s+1}^{t} \left(1 - \frac{\gamma}{s'+a} \right).
\end{align*}
Using the inequality $1 + x \leq e^{x}$, we obtain
\begin{align*}
u_{t+1} \leq &\frac{c_0}{t+a} + \frac{c_1}{(t+a)^2} + u_0 \prod_{s=0}^{t} \exp\left\{- \frac{\gamma}{s+a}\right\} + \sum_{s=0}^{t-1} \left( \frac{c_0}{s+a} + \frac{c_1}{(s+a)^2} \right) \prod_{s'=s+1}^{t} \exp\left\{ - \frac{\gamma}{s'+a}\right\} \\
\leq &\frac{c_0}{t+a} + \frac{c_1}{(t+a)^2} + u_0 \exp\left\{- \sum_{s=0}^{t}\frac{\gamma}{s+a}\right\} + \sum_{s=0}^{t-1} \left( \frac{c_0}{s+a} + \frac{c_1}{(s+a)^2} \right)  \exp\left\{ -\sum_{s'=s+1}^{t} \frac{\gamma}{s'+a}\right\}. 
\end{align*}
Here, using Lemma~\ref{lemma:harmonic}, since $a - 1 > 0$, it holds that
\begin{align*}
  \sum_{s=0}^{t}\frac{\gamma}{s+a} &\geq \sum_{s = 1}^{t+1} \frac{\gamma}{s+(a - 1)} \geq \gamma \log(t+a) - \gamma \log a = \log \left(\frac{(t+1)^\gamma}{a^\gamma}\right) \\
  \sum_{s'=s+1}^{t} \frac{\gamma}{s'+a} &\geq \gamma \log(t + a) - \gamma \log(s + a + 1) \geq \log \left\{ \frac{(t+a)^\gamma}{(s+a+1)^\gamma} \right\},
\end{align*}
therefore $u_{t+1}$ can be further upper bounded by
\begin{align*}
u_{t+1} \leq &\frac{c_0}{t+a} + \frac{c_1}{(t+a)^2} + u_0 \frac{a^\gamma}{ \left(t+a\right)^\gamma} + \sum_{s=0}^{t-1} \left( \frac{c_0}{s+a}  + \frac{c_1}{(s+a)^2} \right) \left(\frac{s + a + 1}{t+a}\right)^\gamma \\
\leq &\frac{c_0}{t+a} + \frac{c_1}{(t+a)^2} + \frac{u_0 a ^ \gamma}{\left(t+a\right)^\gamma} + (t+a)^{-\gamma} \sum_{s=0}^{t-1} \left( \frac{c_0}{s+a} + \frac{c_1}{(s+a)^2} \right) \left(s+a+1\right)^\gamma \\
\leq &\frac{c_0}{t+a} + \frac{c_1}{(t+a)^2} + \frac{u_0 a ^ \gamma}{\left(t+a\right)^\gamma} + (t+a)^{-\gamma} (1 + a^{-1})^\gamma \sum_{s=0}^{t-1} \left( c_0 \left(s+a\right)^{\gamma-1} + c_1\left(s+a\right)^{\gamma-2} \right) .
\end{align*}
The last term can be bound with the corresponding integral (see Lemma~\ref{lemma:harmonic}), yielding (since $\gamma-1 > 0$):
\begin{align*}
    \sum_{s=0}^{t-1}(s+a)^{\gamma - 1} &\leq \frac{(t+a)^\gamma}{\gamma} + a ^ {\gamma - 1}.
\end{align*}
For the term $\sum_{s=0}^{t-1}(s+1)^{\gamma - 2}$, we analyzing the two cases $1 < \gamma \leq 2$ and $\gamma > 2$ using Lemma~\ref{lemma:harmonic} we obtain
\begin{align*}
    \sum_{s=0}^{t-1}(s+a)^{\gamma - 2} &\leq \frac{(t+a)^{\gamma-1}}{\gamma - 1} + a^{\gamma - 2} + 1.
\end{align*}
The two inequalities combined yield the stated bound,
\begin{align*}
    u_{t+1} \leq & \frac{c_0}{t+a} + \frac{c_1}{(t+a)^2} + \frac{u_0 a ^ \gamma}{\left(t+a\right)^\gamma} + (t+a)^{-\gamma} (1 + a^{-1})^\gamma \sum_{s=0}^{t-1} \left( c_0(s+a)^{\gamma-1} + c_1(s+a)^{\gamma - 2} \right) \\
    \leq & \frac{c_0}{t+a} + \frac{c_1}{(t+a)^2} + \frac{u_0 a ^ \gamma}{\left(t+a\right)^\gamma} + \gamma^{-1}(1+a^{-1})^\gamma c_0 + \frac{c_0 (1+a^{-1})^\gamma a^{\gamma-1}}{(t+a)^\gamma} \\ 
    &\quad +\frac{c_1 (1+a^{-1})^\gamma}{(t+a)(\gamma - 1)} + \frac{c_1 (a^{\gamma-2} + 1) (1+a^{-1})^\gamma}{(t+a)^\gamma} \\
    \leq & \frac{u_0 a ^ \gamma + c_1 (a^{\gamma-2} + 1) (1+a^{-1})^\gamma + c_0 (1+a^{-1})^\gamma a^{\gamma-1}}{\left(t+a\right)^\gamma}  \\
    &\quad + \frac{c_0 + c_1 (1+a^{-1})^\gamma (\gamma - 1)^{-1}}{t+a} + \frac{c_1}{(t+a)^2} + \gamma^{-1}(1+a^{-1})^\gamma c_0 .
\end{align*}
\end{proof}


\section{Proofs of Technical Results}


\subsection{Proof of Lemma~\ref{lemma:technical_bound_1N}}\label{app:technical_lemma}
\begin{proof}
We introduce an auxiliary random variable $\bar{a}^i$ which is independent from other players' actions $\{ a^j\}_{j=1}^N$ and has distribution $\vecpi^i$, that is, we introduce the random variable $\bar{a}^i$ as an identically distributed independent copy of $a^i$.
Then, it holds by simple computation that
\begin{align*}
V^i(\vecpi^1, \ldots, \vecpi^N) = & \Exop \Big[ \vecF\Big( \frac{1}{N} \sum_{j=1}^N \vece_{a^j}, a^i\Big) \Big]\\
= &\Exop \Big[\vecF\Big(\frac{1}{N}\Big(\sum_{j=1, j\neq i}^N \vece_{a^j} + \vece_{\bar{a}^i}\Big), a^i \Big)\Big] \\ 
    &+ \Exop \Big[ \vecF\Big(\frac{1}{N}\sum_{j=1}^N \vece_{a^j}, a^i\Big) -\vecF\Big( \frac{1}{N}\Big(\sum_{j=1, j\neq i}^N \vece_{a^j} + \vece_{\bar{a}^i}\Big), a^i\Big)\Big].
\end{align*}
For the first term above, we observe that
\begin{align*}
   \Exop \Big[\vecF\Big( \frac{1}{N}\Big(\sum_{j=1, j\neq i}^N \vece_{a^j} + \vece_{\bar{a}^i}\Big), a^i\Big)\Big] = &\Exop \Big[ \Exop \Big[\vecF\Big( \frac{1}{N}\Big(\sum_{j=1, j\neq i}^N \vece_{a^j} + \vece_{\bar{a}^i}\Big), a^i\Big) \Big| a^i\Big]\Big] \\
   = &\Exop \Big[ \Exop \Big[\vece_{a^i}^\top \vecF\Big(\frac{1}{N}\Big(\sum_{j=1, j\neq i}^N \vece_{a^j} + \vece_{\bar{a}^i}\Big)\Big) \Big| a^i\Big]\Big] \\
   = &\Exop [ \vece_{a^i}^\top] \Exop \Big[ \vecF\Big(\frac{1}{N}\Big(\sum_{j=1, j\neq i}^N \vece_{a^j} + \vece_{\bar{a}^i}\Big)\Big)\Big] \\
   = & \vecpi^{i, \top} \Exop [\vecF(\widehat{\vecmu})],
\end{align*}
since $\{a^j \}_{j=1}^N$ and $(\bar{a}^i,\veca^{-i})$ are identically distributed by the independence of both $\bar{a}^i$ and $a^i$ from $\veca^{-i}$, and since
$(\bar{a}^i,\veca^{-i})$ is independent of $a^i$.

The second term above can be bounded using
\begin{align*}
    \Big|\vecF\Big(&\frac{1}{N}\sum_{j=1}^N \vece_{a^j}, a^i\Big) -\vecF\Big( \frac{1}{N}\Big(\sum_{j=1, j\neq i}^N \vece_{a^j} + \vece_{\bar{a}^i}\Big), a^i\Big)\Big| \\
    = & \Big|\vece_{a^i}^\top\vecF\Big(\frac{1}{N}\sum_{j=1}^N \vece_{a^j}\Big) -\vece_{a^i}^\top \vecF\Big( \frac{1}{N}\Big(\sum_{j=1, j\neq i}^N \vece_{a^j} + \vece_{\bar{a}^i}\Big)\Big)\Big| \\
    \leq & \left\| \vece_{a^i} \right\|_2 \Big\| \vecF\Big(\frac{1}{N}\sum_{j=1}^N \vece_{a^j}\Big) -\vecF\Big( \frac{1}{N}\Big(\sum_{j=1, j\neq i}^N \vece_{a^j} + \vece_{\bar{a}^i}\Big)\Big) \Big\|_2 \\
    \leq & \frac{L}{N} \left\| \vece_{a^i} - \vece_{\bar{a}^i}\right\|_2 \leq \frac{L\sqrt{2}}{N}.
\end{align*}
The last line follows from the fact that $\vecF$ is $L$-Lipschitz.
\end{proof}

\subsection{Proof of Lemma~\ref{lemma:phi_lipschitz}}\label{sec:extended_proof_lema_lipschitz_phi}

We first prove the fact that $V^i$ is Lipschitz.
In the proof, we denote the empirical action distribution induced by actions $\{a^j\}_{j=1}^N\in\setA^N$ by $\widehat{\vecmu}(\{a^j\}_{j=1}^N) \in \Delta_\setA$ to simplify notation.
We analyze two cases of Lipschitz moduli stated in the lemma separately. In the first case, we compute the Lipschitz moduli $L_{i,i}$ for any $i\in\setN$ as follows:
\begin{align*}
    |V^i&(\vecpi, \vecpi^{-i}) - V^i (\vecpi', \vecpi^{-i})| \\
    \leq & \Big| \Exop \left[ \vecF\left(\widehat{\vecmu}(\{a^k\}_{k=1}^N), a^i\right) \middle|
a^j \sim \vecpi^j, \forall j \neq i, a^i \sim \vecpi
\right] \\
    &\quad - \Exop \left[ \vecF\left(\widehat{\vecmu}(\{a^k\}_{k=1}^N),a^i\right) \middle|
a^j \sim \vecpi^j, \forall j \neq i, a^i \sim \vecpi' \right] \Big| \\
\leq & \big| \sum_{\substack{a^j \in \setA \\ j\neq i}}  \sum_{a^i \in \setA}\vecpi(a^i) \vecF(\widehat{\vecmu}(\{a^k\}_{k=1}^N), a^i)\prod_{j\neq i} \vecpi^j(a^j) - \sum_{\substack{a^j \in \setA \\ j\neq i}} \sum_{a^i \in \setA} \vecpi'(a^i) \vecF(\widehat{\vecmu}(\{a^k\}_{k=1}^N), a^i) \prod_{j\neq i} \vecpi^j(a^j) 
 \Big| \\
\leq & \sum_{\substack{a^j \in \setA \\ j\neq i}}  \Big|\sum_{a^i \in \setA}\left[ \vecpi(a^i) - \vecpi'(a^i)\right] \vecF(\widehat{\vecmu}(\{a^k\}_{k=1}^N), a^i) \Big| \prod_{j\neq i} \vecpi^j(a^j) \\
\leq & \sum_{\substack{a^j \in \setA \\ j\neq i}}  \left\| \vecpi - \vecpi'\right\|_2 \sqrt{\sum_{a^i \in \setA} \vecF(\widehat{\vecmu}(\{a^k\}_{k=1}^N), a^i)^2} \prod_{j\neq i} \vecpi^j(a^j) 
\leq \left\| \vecpi - \vecpi'\right\|_2 \sqrt{K}.
\end{align*}
where we use Jensen's inequality in the penultimate step and the Cauchy-Schwartz inequality in the final step.

In the second case, for any $k \neq i$, it holds that
\begin{align*}
|V^i&(\vecpi, \vecpi^{-k}) - V^i (\vecpi', \vecpi^{-k})| \\
    \leq & \Exop \left[ \vecF\left(\widehat{\vecmu}(\{a^l\}_{l=1}^N), a^i\right) \middle|
a^j \sim \vecpi^j, \forall j \neq k, a^k \sim \vecpi
\right] - \Exop \left[ \vecF\left(\widehat{\vecmu}(\{a^l\}_{l=1}^N),a^i\right) \middle|
a^j \sim \vecpi^j, \forall j \neq k, a^k \sim \vecpi' \right] \\
    \leq & \Big| \sum_{\substack{a^j \in \setA \\ j\neq k}} \prod_{j\neq k} \vecpi^j(a^j) \sum_{a^k \in \setA}\vecpi(a^k) \vecF(\widehat{\vecmu}(\{a^l\}_{l=1}^N), a^i) - \sum_{\substack{a^j \in \setA \\ j\neq k}} \prod_{j\neq k} \vecpi^j(a^j) \sum_{a^k \in \setA} \vecpi'(a^k) \vecF(\widehat{\vecmu}(\{a^l\}_{l=1}^N), a^i) \Big| \\
\leq & \sum_{\substack{a^j \in \setA \\ j\neq k}} \prod_{j\neq k} \vecpi^j(a^j) \Big|\sum_{a^k \in \setA}\left[ \vecpi(a^k) - \vecpi'(a^k)\right] \vecF(\widehat{\vecmu}(\{a^l\}_{l=1}^N), a^i) \Big|
\end{align*}
In this case, note that for any $a, a' \in \setA, \veca\in \setA^K$, we have $|\vecF(\widehat{\vecmu}(a, \veca^{-k}), a^i) - \vecF(\widehat{\vecmu}(a', \veca^{-k}), a^i)| \leq \|\vecF(\widehat{\vecmu}(a, \veca^{-k})) - \vecF(\widehat{\vecmu}(a', \veca^{-k})) \|_2 \leq \sfrac{L\sqrt{2}}{N}$.
That is, the set $\{\vecF(\widehat{\vecmu}(a, \veca^{-k}), a^i) \, : a\in\setA \} \subset \mathbb{R}$ has diameter $\sfrac{2L}{\sqrt{N}}$, and
there exists a constant $v_{k} \in \mathbb{R}$ such that $|\vecF(\widehat{\vecmu}(a, \veca^{-k}), a^i) - v_{k}| \leq \sfrac{2L\sqrt{2}}{N}$ for all $a$.
Then,
\begin{align*}
|V^i&(\vecpi, \vecpi^{-k}) - V^i (\vecpi', \vecpi^{-k})| \\
    \leq & \sum_{\substack{a^j \in \setA \\ j\neq k}} \prod_{j\neq k} \vecpi^j(a^j) \Big|\sum_{a^k \in \setA}\left[ \vecpi(a^k) - \vecpi'(a^k)\right] \left[ \vecF(\widehat{\vecmu}(\{a^l\}_{l=1}^N), a^i) - v_{k}\right] \Big| \\
    \leq & \sum_{\substack{a^j \in \setA \\ j\neq k}} \prod_{j\neq k} \vecpi^j(a^j) \|\vecpi - \vecpi'\|_2 \sqrt{\sum_{a^k\in \setA} \left[ \vecF(\widehat{\vecmu}(\{a^l\}_{l=1}^N), a^i) - v_{k}\right]^2 } \\
    \leq & \sum_{\substack{a^j \in \setA \\ j\neq k}} \prod_{j\neq k} \vecpi^j(a^j) \|\vecpi - \vecpi'\|_2 \frac{2L\sqrt{2K}}{N} \leq \|\vecpi - \vecpi'\|_2 \frac{2L \sqrt{2K}}{N}.
\end{align*}

We establish the Lipschitz continuity of $\setE^i_{\text{exp}}$ with the above inequalities.
\begin{align*}
    |\setE^i_{\text{exp}} &(\vecpi, \vecpi^{-i}) - \setE^i_{\text{exp}}(\vecpi', \vecpi^{-i}) | \\
    \leq & \left|\max_{\overline{\vecpi}\in \Delta_\setA} V^i(\overline{\vecpi}, \vecpi^{-i}) - V^i(\vecpi, \vecpi^{-i}) - \max_{\overline{\vecpi}\in \Delta_\setA} V^i(\overline{\vecpi}, \vecpi^{-i}) + V^i(\vecpi', \vecpi^{-i})\right| \\
    \leq & \left| V^i(\vecpi', \vecpi^{-i}) - V^i(\vecpi, \vecpi^{-i})\right| 
    \leq \sqrt{K} \|\vecpi - \vecpi'\|_2.
\end{align*}
Similarly, for $k\neq i$,
\begin{align*}
|\setE^i_{\text{exp}} &(\vecpi, \vecpi^{-k}) - \setE^i_{\text{exp}}(\vecpi', \vecpi^{-k}) | \\
    \leq & \left|\max_{\overline{\vecpi}\in \Delta_\setA} [ V^i( \vecpi, \overline{\vecpi},\vecpi^{-k,i}) - V^i(\vecpi, \vecpi^{-k}) ] - \max_{\overline{\vecpi}\in \Delta_\setA} [V^i( \vecpi', \overline{\vecpi},\vecpi^{-k,i}) - V^i(\vecpi', \vecpi^{-k}) ]\right| \\
    \leq & \max_{\overline{\vecpi}\in \Delta_\setA} \left| V^i( \vecpi, \overline{\vecpi},\vecpi^{-k,i}) - V^i( \vecpi',\overline{\vecpi}, \vecpi^{-k,i}) \right| + \left| V^i(\vecpi, \vecpi^{-k}) - V^i(\vecpi', \vecpi^{-k})\right| \\
    \leq & \frac{4L\sqrt{2K}}{N} \| \vecpi - \vecpi'\|_2.
\end{align*}

\subsection{Extended Definitions for Bandit Feedback}\label{sec:bandit_extended_defs}

When analyzing the TRPA-Bandit dynamics, several random variables and events will be reused to assist analysis.
For brevity, we define them here.
We define the following random variables:
\begin{align*}
    \mathbbm{1}_{h,t}^i := &\mathbbm{1}\{\text{player $i$ explores at round $t$ of epoch $h$}\} = X_{h,t}^i \\
    E_{h,t}^i := &\{\mathbbm{1}_{h,t}^i = 1 \} \\
    \mathbbm{1}_{h}^i := &\mathbbm{1}\{\text{player $i$ explores at least once during epoch $h$}\} = \max_{t = 1, \ldots, T_h} \mathbbm{1}_{h,t}^i  \\
    E_{h}^i := &\{\mathbbm{1}_{h}^i = 1 \} = \bigcup_{t=1}^{T_h} E_{h,t}^i\\
    a_{h}^i := &\text{Last explored action in epoch $h$ by agent $i$,} \\
        &\text{and $a_0$ if no exploration occurred}. \\
    s_h^i := &\text{Timestep when exploration last occurred in epoch $h$ by agent $i$, }\\
        &\text{and $0$ if no exploration occurred.} \in \{1, \ldots, T_h \}
\end{align*}

\subsection{Proof of Lemma~\ref{lemma:exploration_bias_trpa_bandit}}\label{sec:proof_lemma_bandit_exploration_bias}

We will reuse the definitions of Section~\ref{sec:bandit_extended_defs}.
By the definition of the events and the probabilistic exploration scheme, we have $\widehat{\vecr}_h^i = K \left( \vecF(\widehat{\vecmu}_{s_h^{i},h}, a_h^{i}) + \vecn_{s_h^{i},h}^{i}(a_h^{i}) \right) \mathbbm{1}_h^i$.
Firstly, by the law of total expectations and the fact that $E_h^i$ are independent of $\mathcal{F}_h$,
\begin{align*}
    \Exop\left[\widehat{\vecr}_h^i \middle| \mathcal{F}_h\right] = &\Exop\left[\widehat{\vecr}_h^i \middle|E_{h}^i, \mathcal{F}_h\right] \Prob(E_{h}^i) + \Exop\left[\widehat{\vecr}_h^i\middle|\overline{E_{h}^{i}}, \mathcal{F}_h\right] \Prob(\overline{E_{h}^{i}}) \\
    = & \Exop\left[\widehat{\vecr}_h^i \middle|E_{h}^i, \mathcal{F}_h\right] - \underbrace{\Exop\left[\widehat{\vecr}_h^i\middle|E_{h}^{i}, \mathcal{F}_h\right] \Prob(\overline{E_{h}^{i}}) + \Exop\left[\widehat{\vecr}_h^i\middle|\overline{E_{h}^{i}}, \mathcal{F}_h\right] \Prob(\overline{E_{h}^{i}})}_{:=\vecb_h^i} \\
    = & \Exop\left[\widehat{\vecr}_h^i \middle|E_{h}^i, \mathcal{F}_h\right] + \vecb_h^i
\end{align*}
for $\vecb_h^i$ quantifying a bias induced due to the probability of no exploration.
We have that
\begin{align*}
    \| \vecb_h^i \|_2 \leq K\sqrt{K} \sqrt{1 + \sigma^2} \exp\left\{ -\varepsilon T_{h}\right\}
\end{align*}
since $\Exop\left[\widehat{\vecr}_h^i\middle|\overline{E_{h}^{i}}, \mathcal{F}_h\right] = 0$ and exploration probabilities are determined by independent random Bernoulli variables hence $\Prob(\overline{E_{h}^{i}}) = \left( 1 - \varepsilon \right)^{T_h} \leq \exp\left\{ -\varepsilon T_{h}\right\}.$
To further characterize the bias, we introduce a coupling argument.
Define independent random variables $\bar{a}_h^j \sim \varepsilon \vecpi_\text{unif} + (1-\varepsilon) \vecpi_h^j$ for all $j\in\setN$ and $\bar{a}^i_{h, exp} \sim \vecpi_\text{unif}$.
By the definition, it holds that (where $\widehat{\vecmu}(\{ \bar{a}_h ^ j \}_j) \in \Delta_\setA$ denotes the empirical distribution induced by actions $\{ \bar{a}_h ^ j \}_j$):
\begin{align*}
    \|\Exop \left[\widehat{\vecr}_h^i \middle|E_{h}^i, \mathcal{F}_h\right] - \Exop\left[ \vecF(\widehat{\vecmu}(\{ \bar{a}_h ^ j \}_j)) \middle| \mathcal{F}_h \right]\|_2 
    \leq & \|\Exop \left[\vecF(\widehat{\vecmu}(\bar{a}_{h,exp}^i, \bar{a}_h ^ {-i})) \middle| \mathcal{F}_h \right] - \Exop\left[ \vecF(\widehat{\vecmu}(\{ \bar{a}_h ^ j \}_j))  \middle| \mathcal{F}_h\right]\|_2 \\
    \leq & \Exop \left[\|\vecF(\widehat{\vecmu}(\bar{a}_{h,exp}^i, \bar{a}_h ^ {-i})) - \vecF(\widehat{\vecmu}(\{ \bar{a}_h ^ j \}_j)) \|_2 \middle| \mathcal{F}_h\right] \leq \frac{2 L }{N}.
\end{align*}
With the additional bound $\Exop \left[\|\vecF(\varepsilon \vecpi_\text{unif} + (1-\varepsilon) \bar{\vecmu}_h) - \vecF(\widehat{\vecmu}(\{ \bar{a}_h ^ j \}_j)) \|_2 \middle| \mathcal{F}_h \right] \leq \frac{2 L }{\sqrt{N}} $, we obtain the lemma.

\subsection{Proof of Lemma~\ref{lemma:bandit_main_recurrence}}\label{sec:proof_lemma_bandit_recurrence}

We will reuse the definitions of Section~\ref{sec:bandit_extended_defs}.
We formulate a recurrence for the main error term of interest, $\| \vecpi_{t+1}^i - \vecpi^*\|_2^2$,
Repeating the steps presented in Lemma~\ref{lemma:full_error_recurrence}, (noting $\alpha_h := 1 - \tau\eta_h$), our proof strategy is to analyze the three terms in the following decomposition:
\begin{align*}
    \| \vecpi_{h+1}^i - \vecpi^*\|_2^2  
    \leq & \underbrace{\eta_h^2\| \widehat{\vecr}_h^i - \vecF(\vecpi_h^i) \|_2^2 + 2\eta_h^2 (\vecF(\vecpi_h^i) - \vecF(\vecpi^*)))^\top (\widehat{\vecr}_h^i - \vecF(\vecpi_h^i))}_{(a)} \\
     &+ \underbrace{2\eta_h\alpha_h (\vecpi_h^i - \vecpi^*)^\top (\widehat{\vecr}_h^i - \vecF(\vecpi_h^i))}_{(b)} + \underbrace{\|\alpha_h (\vecpi_h^i - \vecpi^*) + \eta_h (\vecF(\vecpi_h^i) - \vecF(\vecpi^*))\|_2^2 }_{(c)}
\end{align*}
Once again, we will need to upper bound the three terms above.
For the term $(a)$ we have $\Exop\left[ (a)\right] \leq 4\eta_t^2 K^3(\sigma^2 + 1)$, noting that $\|\widehat{\vecr}_h^i\|_2^2 \leq K^3$ almost surely.
Likewise, it still holds for the term $(c)$ by Lemma~\ref{lemma:contraction_pg} that
\begin{align*}
    (c) 
        \leq & \left(1 - 2 (\lambda + \tau) \eta_h + (L + \tau)^2 \eta_h^2\right) \| \vecpi_h^i - \vecpi^* \|_2^2.
\end{align*}
However, unlike the bound in Lemma~\ref{lemma:full_error_recurrence}, the exploration parameter $\varepsilon$ will cause additional bias in the term $(b)$.
Define the random vector $\widetilde{\vecr}_h^i = \Exop[ \widehat{\vecr}_h^i | E_h^i, \mathcal{F}_h]$.
\begin{align*}
(b) = & 2\eta_h\alpha_h (\vecpi_h^i - \vecpi^*)^\top (\widehat{\vecr}_h^i - \vecF(\vecpi_h^i)) \\
 = & 2\eta_h\alpha_h (\vecpi_h^i - \vecpi^*)^\top (\widehat{\vecr}_h^i - \widetilde{\vecr}_h^i) + 2\eta_h\alpha_h (\vecpi_h^i - \vecpi^*)^\top (\widetilde{\vecr}_h^i - \vecF(\bar{\vecmu}_h)) \\
    & + 2\eta_h\alpha_h (\vecpi_h^i - \vecpi^*)^\top (\vecF(\bar{\vecmu}_h) - \vecF(\vecpi_h^i)) \\
\leq &2\eta_h\alpha_h \left( \frac{\lambda}{4} \|\vecpi_h^i - \vecpi^* \|_2^2 + \frac{1}{\lambda} \|\widetilde{\vecr}_h^i - \vecF(\bar{\vecmu}_h)\|_2^2\right) \\
    & + 2\eta_h\alpha_h \left(\frac{\lambda}{4} \|\vecpi_h^i - \vecpi^* \|_2^2 + \frac{1}{\lambda} \|\vecF(\bar{\vecmu}_h) - \vecF(\vecpi_h^i)\|_2^2 \right) \\
    &+ 2\eta_h\alpha_h (\vecpi_h^i - \vecpi^*)^\top (\widehat{\vecr}_h^i - \widetilde{\vecr}_h^i) \\
\leq & 2 \eta_h \frac{\lambda}{2} \|\vecpi_h^i - \vecpi^* \|_2^2 + \frac{2\eta_h}{\lambda}\|\widetilde{\vecr}_h^i - \vecF(\bar{\vecmu}_h)\|_2^2 + \frac{2\eta_h}{\lambda} \|\vecF(\bar{\vecmu}_h) - \vecF(\vecpi_h^i)\|_2^2 \\
    &+2\eta_h\alpha_h (\vecpi_h^i - \vecpi^*)^\top (\widehat{\vecr}_h^i - \widetilde{\vecr}_h^i),
\end{align*}
and similarly, if $\lambda = 0$, we have
\begin{align*}
(b) = & 2\eta_h\alpha_h (\vecpi_h^i - \vecpi^*)^\top (\widehat{\vecr}_h^i - \vecF(\vecpi_h^i)) \\
 = & 2\eta_h\alpha_h (\vecpi_h^i - \vecpi^*)^\top (\widehat{\vecr}_h^i - \widetilde{\vecr}_h^i) + 2\eta_h\alpha_h (\vecpi_h^i - \vecpi^*)^\top (\widetilde{\vecr}_h^i - \vecF(\bar{\vecmu}_h)) \\
    & + 2\eta_h\alpha_h (\vecpi_h^i - \vecpi^*)^\top (\vecF(\bar{\vecmu}_h) - \vecF(\vecpi_h^i)) \\
\leq &2\eta_h\alpha_h \left( \frac{\tau\delta}{2} \|\vecpi_h^i - \vecpi^* \|_2^2 + \frac{1}{2\tau\delta} \|\widetilde{\vecr}_h^i - \vecF(\bar{\vecmu}_h)\|_2^2\right) \\
    & + 2\eta_h\alpha_h \left(\frac{\tau\delta}{2} \|\vecpi_h^i - \vecpi^* \|_2^2 + \frac{1}{2\tau\delta} \|\vecF(\bar{\vecmu}_h) - \vecF(\vecpi_h^i)\|_2^2 \right) \\
    &+ 2\eta_h\alpha_h (\vecpi_h^i - \vecpi^*)^\top (\widehat{\vecr}_h^i - \widetilde{\vecr}_h^i) \\
\leq & 2 \eta_h \tau\delta \|\vecpi_h^i - \vecpi^* \|_2^2 + \frac{\eta_h}{\tau\delta}\|\widetilde{\vecr}_h^i - \vecF(\bar{\vecmu}_h)\|_2^2 + \frac{\eta_h}{\tau\delta} \|\vecF(\bar{\vecmu}_h) - \vecF(\vecpi_h^i)\|_2^2 \\
    &+2\eta_h\alpha_h (\vecpi_h^i - \vecpi^*)^\top (\widehat{\vecr}_h^i - \widetilde{\vecr}_h^i).
\end{align*}

Denote $\vecpi_{unif} := \frac{1}{K}\vecone_K$.
The remaining error terms we can bound by (using the auxiliary coupling random actions $\bar{a}_h^j$ from Lemma~\ref{lemma:exploration_bias_trpa_bandit}):
\begin{align*}
    |\Exop[2\eta_h\alpha_h (\vecpi_h^i - \vecpi^*)^\top (\widehat{\vecr}_h^i - \widetilde{\vecr}_h^i) | \mathcal{F}_h] | 
        \leq & |\Exop[2\eta_h\alpha_h (\vecpi_h^i - \vecpi^*)^\top (\widehat{\vecr}_h^i - \widetilde{\vecr}_h^i) | E_h^i, \mathcal{F}_h] \mathbb{P}(E_h^i) \\
        & + \Exop[2\eta_h\alpha_h (\vecpi_h^i - \vecpi^*)^\top (\widehat{\vecr}_h^i - \widetilde{\vecr}_h^i) | \overline{E_h^i}, \mathcal{F}_h]\mathbb{P}(\overline{E_h^i})| \\
    \leq & 2\eta_h \Exop\left[\|\vecpi_h^i - \vecpi^*\|_2 \|\widehat{\vecr}_h^i - \widetilde{\vecr}_h^i\|_2 | \overline{E_h^i}, \mathcal{F}_h\right] \mathbb{P}(\overline{E_h^i}) \\
    \leq & 8\eta_h K^{\sfrac{3}{2}} \sqrt{1 + \sigma^2} \mathbb{P}(\overline{E_h^i}) 
\end{align*}
and $\mathbb{P}(\overline{E_h^i}) \leq \exp\{-\varepsilon T_h\}$.
Furthermore,
\begin{align*}
    \Exop[&\|\widetilde{\vecr}_h^i - \vecF(\bar{\vecmu}_h)\|_2^2 | \mathcal{F}_{h}] \\
    \leq & 2\Exop\left[ \|\widetilde{\vecr}_h^i - \vecF(\widehat{\vecmu}(\bar{a}_{h,exp}^i, \bar{a}_h ^ {-i})) \|_2^2\middle|\mathcal{F}_{h}\right] + 2\Exop\left[ \|\vecF(\widehat{\vecmu}(\bar{a}_{h,exp}^i, \bar{a}_h ^ {-i})) - \vecF(\bar{\vecmu}_h)\|_2^2 | \mathcal{F}_{h}\right] \\
    \leq &  \frac{8 L^2}{N}  +  2L^2 \Exop[ \|\widehat{\vecmu}(\bar{a}_{exp}^i, \bar{a} ^ {-i}) - \frac{1}{N}\sum_{j=1}^N \vecpi^j_h\|_2^2 | \mathcal{F}_{h}] \\
    \leq &  \frac{8 L^2}{N}  +  4L^2 \Exop[ \|\widehat{\vecmu}(\bar{a}_{exp}^i, \bar{a} ^ {-i}) - \frac{1}{N}\sum_{j\neq i} (\varepsilon\vecpi_{unif} + (1-\varepsilon)\vecpi^j_h) - \frac{\vecpi_{unif}}{N} \|_2^2 | \mathcal{F}_{h}] \\
        &+ 4L^2 \Exop[ \| \varepsilon\frac{1}{N}\sum_{j\neq i} ( \varepsilon\vecpi^j_h - \vecpi_{unif}) +  \frac{\vecpi^i_h - \vecpi_{unif}}{N} \|_2^2 | \mathcal{F}_{h}] \\
    \leq &  \frac{8 L^2}{N}  +  \frac{8L^2}{N} + 4L^2 (2\varepsilon^2 + \frac{4}{N}) \\
    \leq & \frac{64 L ^2}{N} + 8 L^2 \varepsilon^2,
\end{align*}
and finally, using the trivial Lipschitz continuity property: $\|\vecF(\bar{\vecmu}_h) - \vecF(\vecpi_h^i)\|_2^2 \leq L^2 \|\bar{\vecmu}_h - \vecpi_h^i\|_2^2 = L^2 e_h^i$.

\subsection{Proof of Lemma~\ref{lemma:bandit_pol_deviation}}\label{sec:proof_lemma_bandit_pol_dev}

We will reuse the definitions of Section~\ref{sec:bandit_extended_defs}.
Once again, repeating the derivations from Lemma~\ref{lemma:policy_variations_bound_trpa_full}, we have that
\begin{align*}
    \| \vecpi^i_{h+1} - \vecpi^j_{h+1} \|_2^2 
    = & (1 - \tau \eta_h)^2 \| \vecpi^i_h - \vecpi^j_h \|_2^2 + \eta_h^2 \|  \widehat{\vecr}_h^i - \widehat{\vecr}_h^j \|_2^2 + 2 (1 - \tau \eta_h) \eta_h (\vecpi^i_h - \vecpi^j_h) ^ \top ( \widehat{\vecr}_h^i - \widehat{\vecr}_h^j ).
\end{align*}
Unlike in the expert feedback case, the last term does not vanish in expectation when bounding policy deviation.
\begin{align*}
    \Exop \left[ \| \vecpi^i_{h+1} - \vecpi^j_{h+1} \|_2^2 | \mathcal{F}_h \right] \leq & (1 - \tau \eta_h)^2 \| \vecpi^i_h - \vecpi^j_h \|_2^2 + \Exop \left[ \eta_h^2 \|  \widehat{\vecr}_h^i - \widehat{\vecr}_h^j \|_2^2 | \mathcal{F}_h \right] \\
        &+ 2 (1 - \tau \eta_h) \eta_h (\vecpi^i_h - \vecpi^j_h) ^ \top \Exop\left[\widehat{\vecr}_h^i - \widehat{\vecr}_h^j | \mathcal{F}_h \right] \\
    \leq & (1 - \tau \eta_h)^2 \| \vecpi^i_h - \vecpi^j_h \|_2^2 + \Exop \left[ \eta_h^2 \|  \widehat{\vecr}_h^i - \widehat{\vecr}_h^j \|_2^2 | \mathcal{F}_h \right] \\
        &+ 2 \eta_h (\vecpi^i_h - \vecpi^j_h) ^ \top \left[ \Exop\left[\widehat{\vecr}_h^i \middle|E_{h}^i, \mathcal{F}_h\right] + \vecb_h^i - \Exop\left[\widehat{\vecr}_h^j \middle|E_{h}^j, \mathcal{F}_h\right] - \vecb_h^j \right]  \\
    \leq & (1 - \tau \eta_h)^2 \| \vecpi^i_h - \vecpi^j_h \|_2^2 + \Exop \left[ \eta_h^2 \|  \widehat{\vecr}_h^i - \widehat{\vecr}_h^j \|_2^2 | \mathcal{F}_h \right] \\
        &+ 2 \eta_h (\vecpi^i_h - \vecpi^j_h) ^ \top \left[ \Exop\left[\widehat{\vecr}_h^i \middle|E_{h}^i, \mathcal{F}_h\right] - \Exop\left[\widehat{\vecr}_h^j \middle|E_{h}^j, \mathcal{F}_h\right] \right] + 8 \eta_h \exp\left\{ -\varepsilon T_{h}\right\},
\end{align*}
where the last line follows from the bound on $\vecb_h^i$ in Lemma~\ref{lemma:exploration_bias_trpa_bandit}.
Furthermore, using Young's inequality, we obtain
\begin{align*}
    \Exop \left[ \| \vecpi^i_{h+1} - \vecpi^j_{h+1} \|_2^2 | \mathcal{F}_h \right] \leq & (1 - \tau \eta_h)^2 \| \vecpi^i_h - \vecpi^j_h \|_2^2 + \Exop \left[ \eta_h^2 \|  \widehat{\vecr}_h^i - \widehat{\vecr}_h^j \|_2^2 | \mathcal{F}_h \right] + 8 \eta_h \exp\left\{ -\varepsilon T_{h}\right\} \\
        &+ \frac{\tau\eta_h}{2} \|\vecpi^i_h - \vecpi^j_h\|_2^2 + \frac{ \eta_h\tau^{-1}}{2}\left\| \Exop\left[\widehat{\vecr}_h^i \middle|E_{h}^i, \mathcal{F}_h\right] - \Exop\left[\widehat{\vecr}_h^j \middle|E_{h}^j, \mathcal{F}_h\right] \right\|_2^2 \\
     \leq & (1 - \tau \eta_h)^2 \| \vecpi^i_h - \vecpi^j_h \|_2^2 + \Exop \left[ \eta_h^2 \|  \widehat{\vecr}_h^i - \widehat{\vecr}_h^j \|_2^2 | \mathcal{F}_h \right] + 8 \eta_h \exp\left\{ -\varepsilon T_{h}\right\} \\
        &+ \frac{\tau\eta_h}{2} \|\vecpi^i_h - \vecpi^j_h\|_2^2 + \frac{2\eta_h\tau^{-1} L ^ 2}{N^2}.
\end{align*}
With the choice of $T_h = \varepsilon^{-1} \log (h+2)$ and noting that $\| \vecpi^i_{h} - \vecpi_h^j\|_2 \leq 2$, we obtain the recurrence
\begin{align*}
    \Exop \left[ \| \vecpi^i_{h+1} - \vecpi^j_{h+1} \|^2_2 \right] \leq & \left(1 - \frac{\sfrac{3}{2}}{h+2}\right) \Exop \left[ \| \vecpi^i_h - \vecpi^j_h \|_2^2\right] + \frac{4\tau^{-2} K^3 (\sigma^2 + 1) + 8\tau^{-1} + 4}{(h+2)^2}   \\
        & + \frac{4\tau^{-2} L ^ 2}{N^2 (h+2)}.
\end{align*}
Hence, by invoking the recurrence lemma (Lemma~\ref{lemma:general_recurrence}, with $a=2, c_0 = \frac{4\tau^{-2} L ^ 2}{N^2}, c_1 = 4\tau^{-2} K^3 (\sigma^2 + 1) + 8\tau^{-1} + 4, \gamma = \sfrac{3}{2}, u_0 = 0$), we have
\begin{align*}
    \Exop \left[ \| \vecpi^i_{h+1} - \vecpi^j_{h+1} \|_2^2 \right] \leq & \frac{24\tau^{-2} K^3 (\sigma^2 + 1) + 48\tau^{-2} + 24}{h+2} + \frac{16\tau^{-2} L ^ 2}{N^2}.
\end{align*}
The statement in the lemma follows as in the full feedback case.

\subsection{Proof of Theorem~\ref{theorem:bandit_short}}\label{sec:bandit_theorem_full}

Using Lemma~\ref{lemma:bandit_main_recurrence}, for the strongly monotone case $\lambda > 0$ we have that
\begin{align*}
    u_{h+1}^i \leq & 4 \eta_h^2 K^3(1 + \sigma^2) + 8\eta_h^2(L+\tau)^2 + 8 K^{\sfrac{3}{2}}\eta_h \sqrt{1+\sigma^2}  \exp\{-\varepsilon T_h\} \\
        &+128\eta_h\lambda^{-1} L^2 N^{-1} + 16\eta_h\lambda^{-1}L^2\varepsilon^2 + 2\eta_h\lambda^{-1} L^2 \Exop\left[e_h^i\right] \\
        &+\left(1 - 2 \eta_h(\sfrac{\lambda}{2} + \tau)\right) u_{h}^i,
\end{align*}
and for the monotone case it holds that 
\begin{align*}
    u_{h+1}^i \leq &  4\eta_h^2 K^3(\sigma^2 + 1) + 8 \eta_h^2 (L+\tau)^2 + 8 K^{\sfrac{3}{2}} \eta_h \sqrt{1+\sigma^2} \exp\{-\varepsilon T_h\} \\
    &+64\tau^{-1} \eta_h \delta^{-1}L^2 N^{-1}+8\tau^{-1} \eta_h \delta^{-1}L^2 \varepsilon^{2} + \tau^{-1} \eta_h\delta^{-1}L^2 \Exop\left[e_h^i\right] \\  
        &+ \left(1 - 2 \tau \eta_h (1-\delta)\right) u_{h}^i.
\end{align*}
By Lemma~\ref{lemma:bandit_pol_deviation}, we know that
\begin{align*}
    \Exop[e_h^i] \leq \frac{12\tau^{-2} K^3 (\sigma^2 + 1) + 48\tau^{-2} + 24}{h+2} + \frac{16\tau^{-2} L ^ 2}{N^2}.
\end{align*}
Placing this bound as well as $T_h = \lceil \varepsilon^{-1} \log (h+2) \rceil$ and $\eta_h = \frac{\tau^{-1}}{h+2}$, we obtain the recurrences which are solved by using Lemma~\ref{lemma:general_recurrence}.
In the monotone case, we pick $\delta=\sfrac{1}{4}$ as before.



The bound in the statement of the theorem in the main body of the paper follows from the fact that the lengths of the exploration epochs scale with $T_h = \mathcal{O}(\varepsilon^{-1} \log (h+2)) = \widetilde{\mathcal{O}}(\varepsilon^{-1})$.

\section{Experimental Details}
\label{appdx:experiment}
Unless specified, we obtain a stream of datasets for all our experiments by simply sampling from the assumed probabilistic model, where the number of observations $n$ is sampled uniformly in the range $[64, 128]$. For efficient mini-batching over datasets with different cardinalities, we sample datasets with maximum cardinality $(128)$ and implement different cardinalities by masking out different numbers of observations for different datasets whenever required. 
% For all our experiments on supervised setups, we sample $\vx_i \sim \mathcal{N}(\mathbf{0}, \mathbf{I})$ for simplicity, but it is possible to explore other proposal distributions (e.g., heavy-tailed distributions) too. 
% In our Bayesian Neural Networks experiments, we considered a single-layered neural network with $\mathrm{Tanh}$ activation function and $32$ hidden dimensions. We considered the likelihood function as either a Gaussian or a categorical distribution using the logits, depending on regression and classification.

To evaluate both our proposed approach and the baselines, we compute an average of the predictive performances across $25$ different posterior samples for each of the $100$ fixed test datasets for all our experiments. 
That means for our proposed approach, we sample $25$ different parameter vectors from the approximate posterior that we obtain. For MCMC, we rely on $25$ MCMC samples, and for optimization, we train $25$ different parameter vectors where the randomness comes from initialization. 
For the optimization baseline, we perform a quick hyperparameter search over the space $\{0.01, 003, 0.001, 0.0003, 0.0001, 0.00003\}$ to pick the best learning rate that works for all of the test datasets and then use it to train for $1000$ iterations using the Adam optimizer~\citep{kingma2014adam}. For the MCMC baseline, we use the open-sourced implementation of Langevin-based MCMC sampling\footnote{\href{https://github.com/alisiahkoohi/Langevin-dynamics}{https://github.com/alisiahkoohi/Langevin-dynamics}} where we leave a chunk of the starting samples as burn-in and then start accepting samples after a regular interval (to not make them correlated). The details about the burn-in time and the regular interval for acceptance are provided in the corresponding experiments' sections below.

For our proposed approach of amortized inference, we do not consider explicit hyperparameter optimization and simply use a learning rate of $1\mathrm{e}\text{-}4$ with the Adam optimizer. For all experiments, we used linear scaling of the KL term in the training objectives as described in~\citep{higgins2017betavae}, which we refer to as warmup. Furthermore, training details for each experiment can be found below. 

\subsection{Fixed-Dim}
\label{appdx:details_fixed_dim}
In this section, we provide the experimental details relevant to reproducing the results of Section~\ref{sec:experiments}. All the models are trained with streaming data from the underlying probabilistic model, such that every iteration of training sees a new set of datasets. Training is done with a batch size of $128$, representing the number of datasets seen during one optimization step. Evaluations are done with $25$ samples and we ensure that the test datasets used for each probabilistic model are the same across all the compared methods, i.e., baselines, forward KL, and reverse KL. We train the amortized inference model and the forward KL baselines for the following different probabilistic models:

\textbf{Mean of Gaussian (GM):} We train the amortization models over $20,000$ iterations for both the $2$-dimensional as well as the $100$-dimensional setup. We use a linear warmup with $5000$ iterations over which the weight of the KL term in our proposed approach scales linearly from $0$ to $1$. We use an identity covariance matrix for the data-generating process, but it can be easily extended to the case of correlated or diagonal covariance-based Gaussian distributions.

\textbf{Gaussian Mixture Model (GMM):} We train the mixture model setup for $200,000$ iterations with $50,000$ iterations of warmup. We mainly experiment with $2$-dimensional and $5$-dimensional mixture models, with $2$ and $5$ mixture components for each setup. While we do use an identity covariance matrix for the data-generating process, again, it can be easily extended to other cases.
% For all our experiments, we compute the average over 25 different samples (either from the approximate posterior, or 25 different optimization runs, etc.) to report the downstream metrics. For the optimization baseline, we perform a quick hyperparameter search for each dataset over the space of $\{\}$

\textbf{Linear Regression (LR):} The amortization models for this setup are trained for $50,000$ iterations with $12,500$ iterations of warmup. The feature dimensions considered for this task are $1$ and $100$ dimensions, and the predictive variance $\sigma^2$ is assumed to be known and set as $0.25$.

\textbf{Nonlinear Regression (NLR):} We train the setup for $100,000$ iterations with $25,000$ iterations consisting of warmup. The feature dimensionalities considered are $1$-dimensional and $25$-dimensional, and training is done with a known predictive variance similar to the LR setup. For the probabilistic model, we consider both a $1$-layered and a $2$-layered multi-layer perceptron (MLP) network with 32 hidden units in each, and either a \textsc{relu} or \textsc{tanh} activation function.

\textbf{Linear Classification (LC):} We experiment with $2$-dimensional and $100$-dimensional setups with training done for $50,000$ iterations, out of which $12,500$ are used for warmup. Further, we train for both binary classification as well as a $5$-class classification setup.

\textbf{Nonlinear Classification (NLC):} We experiment with $2$-dimensional and $25$-dimensional setups with training done for $100,000$ iterations, out of which $2,5000$ are used for warmup. Further, we train for both binary classification as well as a $5$-class classification setup. For the probabilistic model, we consider both a $1$-layered and a $2$-layered multi-layer perceptron (MLP) network with 32 hidden units in each, and either a \textsc{relu} or \textsc{tanh} activation function.

\begin{table*}[t]
    \centering
    % \small
    \footnotesize	    
    \def\arraystretch{1.25}
    \setlength{\tabcolsep}{5pt}
    \begin{tabular}{lcr ccc cccc}
        \cmidrule[\heavyrulewidth]{1-9}
         &  &  & \multicolumn{6}{c}{\textit{$L_2$ Loss} ($\downarrow$)} \\
        \cmidrule(lr){4-9}
        \textbf{Objective} & $q_\varphi$ & \textbf{Model} & \multicolumn{3}{c}{\textbf{Linear Model $|$ MLP-TanH Data}} & \multicolumn{3}{c}{\textbf{MLP-TanH Model $|$ Linear Data}} & $\leftarrow\chi_{real}$ \\
        \cmidrule(lr){4-6}\cmidrule(lr){7-9}
        & & & \textit{LR} & \textit{NLR} & \textit{GP} & \textit{LR} & \textit{NLR} & \textit{GP} & $\leftarrow\chi_{sim}$ \\
        \cmidrule{1-9}
\multirow{4}{*}{Baseline} & - & Random & - & $17.761$\sstd{$0.074$}  & -  & $17.847$\sstd{$0.355$} & -  & -  \\
& - & Optimization & - & $1.213$\sstd{$0.000$} & -  & $0.360$\sstd{$0.001$} & -  & -  \\
& - & Langevin & - & $1.218$\sstd{$0.002$} & -  & $0.288$\sstd{$0.001$} & -  & -  \\
& - & HMC & - & $1.216$\sstd{$0.002$} & -  & $0.275$\sstd{$0.001$} & -  & -  \\
\cmidrule{2-9}
\multirow{3}{*}{Fwd-KL} & \multirow{6}{*}{\rotatebox[origin=c]{90}{Gaussian}} & GRU &$2.415$\sstd{$0.269$} & -  & -  & -  & $15.632$\sstd{$0.283$} & -  \\
& & DeepSets &$1.402$\sstd{$0.017$} & -  & -  & -  & $16.046$\sstd{$0.393$} & -  \\
& & Transformer &$2.216$\sstd{$0.097$} & -  & -  & -  & $15.454$\sstd{$0.246$} & -  \\
\cmidrule{3-9}
\multirow{3}{*}{Rev-KL}& & GRU &$1.766$\sstd{$0.044$} & $1.216$\sstd{$0.001$} & $4.566$\sstd{$0.199$} & $0.375$\sstd{$0.001$} & $0.386$\sstd{$0.002$} & $0.524$\sstd{$0.019$} \\
& & DeepSets &$1.237$\sstd{$0.006$} & $1.216$\sstd{$0.001$} & $3963.694$\sstd{$5602.411$} & $0.365$\sstd{$0.000$} & $0.377$\sstd{$0.003$} & $0.385$\sstd{$0.011$} \\
& & Transformer &$1.892$\sstd{$0.113$} & $1.226$\sstd{$0.001$} & $4.313$\sstd{$0.707$} & $0.367$\sstd{$0.006$} & $0.382$\sstd{$0.003$} & $0.458$\sstd{$0.048$} \\
\cmidrule{2-9}
\multirow{3}{*}{Fwd-KL} & \multirow{6}{*}{\rotatebox[origin=c]{90}{Flow}} & GRU &$2.180$\sstd{$0.024$} & -  & -  & -  & $9.800$\sstd{$0.473$} & -  \\
& & DeepSets &$1.713$\sstd{$0.244$} & -  & -  & -  & $15.253$\sstd{$0.403$} & -  \\
& & Transformer &$1.632$\sstd{$0.070$} & -  & -  & -  & $7.949$\sstd{$0.419$} & -  \\
\cmidrule{3-9}
\multirow{3}{*}{Rev-KL} & & GRU &$1.830$\sstd{$0.081$} & \highlight{$1.214$\sstd{$0.001$}} & $5.690$\sstd{$0.196$} & $0.346$\sstd{$0.004$} & $0.349$\sstd{$0.001$} & $0.520$\sstd{$0.015$} \\
& & DeepSets &$1.282$\sstd{$0.036$} & $1.218$\sstd{$0.001$} & $11.690$\sstd{$10.602$} & \highlight{$0.339$\sstd{$0.003$}} & $0.344$\sstd{$0.002$} & $0.397$\sstd{$0.026$} \\
& & Transformer &$1.471$\sstd{$0.016$} & $1.226$\sstd{$0.004$} & $5.194$\sstd{$0.320$} & $0.346$\sstd{$0.002$} & $0.347$\sstd{$0.001$} & $0.480$\sstd{$0.030$} \\
\cmidrule[\heavyrulewidth]{1-9}
    \end{tabular}
    \caption{\textbf{Model Misspecification}. Results for model misspecification under different training data $\chi_{sim}$, when evaluated under MLP-TanH and Linear Data ($\chi_{real}$), with the underlying model as a linear and MLP-TanH model respectively.}
    \vspace{-4mm}
    \label{tab:misspec_model}
\end{table*}
\begin{table*}[t]
    \centering
    % \small
    \footnotesize	    
    \def\arraystretch{1.25}
    \setlength{\tabcolsep}{5pt}
    \begin{tabular}{lcr ccc cccc}
        \cmidrule[\heavyrulewidth]{1-9}
         &  &  & \multicolumn{6}{c}{\textit{$L_2$ Loss} ($\downarrow$)} \\
        \cmidrule(lr){4-9}
        \textbf{Objective} & $q_\varphi$ & \textbf{Model} & \multicolumn{3}{c}{\textbf{Linear Model $|$ GP Data}} & \multicolumn{3}{c}{\textbf{MLP-TanH Model $|$ GP Data}} & $\leftarrow\chi_{real}$ \\
        \cmidrule(lr){4-6}\cmidrule(lr){7-9}
        & & & \textit{LR} & \textit{NLR} & \textit{GP} & \textit{LR} & \textit{NLR} & \textit{GP}  & $\leftarrow\chi_{sim}$ \\
        \cmidrule{1-9}
\multirow{4}{*}{Baseline} & - & Random & -  & -  & $2.681$\sstd{$0.089$} &  -  & -  & $16.236$\sstd{$0.381$} \\
& - & Optimization & -  & -  & $0.263$\sstd{$0.000$} & -  & -  & $0.007$\sstd{$0.000$} \\
& - & Langevin & -  & -  & $0.266$\sstd{$0.001$} & -  & -  & $0.022$\sstd{$0.001$} \\
& - & HMC & -  & -  & $0.266$\sstd{$0.000$} & -  & -  & $0.090$\sstd{$0.002$} \\
\cmidrule{2-9}
\multirow{3}{*}{Fwd-KL} & \multirow{6}{*}{\rotatebox[origin=c]{90}{Gaussian}} & GRU &$0.268$\sstd{$0.000$} & -  & -  & -  & $14.077$\sstd{$0.368$} & -  \\
& & DeepSets &$0.269$\sstd{$0.001$} & -  & -  & -  & $14.756$\sstd{$0.280$} & -  \\
& & Transformer &$0.270$\sstd{$0.001$} & -  & -  & -  & $14.733$\sstd{$0.513$} & -  \\
\cmidrule{3-9}
\multirow{3}{*}{Rev-KL} & & GRU &$0.268$\sstd{$0.000$} & $0.269$\sstd{$0.000$} & $0.266$\sstd{$0.000$} & $0.334$\sstd{$0.005$} & $0.157$\sstd{$0.003$} & $0.080$\sstd{$0.003$} \\
& & DeepSets &$0.269$\sstd{$0.000$} & $0.269$\sstd{$0.000$} & \highlight{$0.265$\sstd{$0.000$}} & $0.331$\sstd{$0.003$} & $0.146$\sstd{$0.002$} & $0.063$\sstd{$0.000$} \\
& & Transformer &$0.269$\sstd{$0.000$} & $0.269$\sstd{$0.000$} & $0.267$\sstd{$0.000$} & $0.310$\sstd{$0.013$} & $0.155$\sstd{$0.006$} & $0.066$\sstd{$0.004$} \\
\cmidrule{2-9}
\multirow{3}{*}{Fwd-KL} & \multirow{6}{*}{\rotatebox[origin=c]{90}{Flow}} & GRU &$0.268$\sstd{$0.000$} & -  & -  & -  & $9.756$\sstd{$0.192$} & -  \\
& & DeepSets &$0.269$\sstd{$0.001$} & -  & -  & -  & $14.345$\sstd{$0.628$} & -  \\
& & Transformer &$0.269$\sstd{$0.000$} & -  & -  & -  & $8.557$\sstd{$0.561$} & -  \\
\cmidrule{3-9}
\multirow{3}{*}{Rev-KL} & & GRU &$0.268$\sstd{$0.000$} & $0.270$\sstd{$0.001$} & $0.266$\sstd{$0.000$} & $0.289$\sstd{$0.011$} & $0.120$\sstd{$0.004$} & $0.059$\sstd{$0.003$} \\
& & DeepSets &$0.269$\sstd{$0.000$} & $0.269$\sstd{$0.001$} & $0.266$\sstd{$0.000$} & $0.270$\sstd{$0.008$} & $0.115$\sstd{$0.002$} & $0.059$\sstd{$0.002$} \\
& & Transformer &$0.269$\sstd{$0.001$} & $0.270$\sstd{$0.000$} & $0.267$\sstd{$0.000$} & $0.293$\sstd{$0.008$} & $0.120$\sstd{$0.005$} & \highlight{$0.055$\sstd{$0.002$}} \\
\cmidrule[\heavyrulewidth]{1-9}
    \end{tabular}
    \caption{\textbf{Model Misspecification}. Results for model misspecification under different training data $\chi_{sim}$, when evaluated under GP Data ($\chi_{real}$), with the underlying model as a linear and MLP-TanH model respectively.}
    \vspace{-4mm}
    \label{tab:misspec_gp}
\end{table*}
\subsection{Variable-Dim}
\label{appdx:details_max_dim}
In this section, we provide the experimental details relevant to reproducing the results of Section~\ref{sec:experiments}. All the models are trained with streaming data from the underlying probabilistic model, such that every iteration of training sees a new set of datasets. Training is done with a batch size of $128$, representing the number of datasets seen during one optimization step. Further, we ensure that the datasets sampled resemble a uniform distribution over the feature dimensions, ranging from $1$-dimensional to the maximal dimensional setup. Evaluations are done with $25$ samples and we ensure that the test datasets used for each probabilistic model are the same across all the compared methods, i.e., baselines, forward KL, and reverse KL. We train the amortized inference model and the forward KL baselines for the following different probabilistic models:

\textbf{Mean of Gaussian (GM):} We train the amortization models over $50,000$ iterations using a linear warmup with $12,5000$ iterations over which the weight of the KL term in our proposed approach scales linearly from $0$ to $1$. We use an identity covariance matrix for the data-generating process, but it can be easily extended to the case of correlated or diagonal covariance-based Gaussian distributions. In this setup, we consider a maximum of $100$ feature dimensions.

\textbf{Gaussian Mixture Model (GMM):} We train the mixture model setup for $500,000$ iterations with $125,000$ iterations of warmup. We set the maximal feature dimensions as $5$ and experiment with $2$ and $5$ mixture components. While we do use an identity covariance matrix for the data-generating process, again, it can be easily extended to other cases.
% For all our experiments, we compute the average over 25 different samples (either from the approximate posterior, or 25 different optimization runs, etc.) to report the downstream metrics. For the optimization baseline, we perform a quick hyperparameter search for each dataset over the space of $\{\}$

\textbf{Linear Regression (LR):} The amortization models for this setup are trained for $100,000$ iterations with $25,000$ iterations of warmup. The maximal feature dimension considered for this task is $100$-dimensional, and the predictive variance $\sigma^2$ is assumed to be known and set as $0.25$.

\textbf{Nonlinear Regression (NLR):} We train the setup for $250,000$ iterations with $62,500$ iterations consisting of warmup. The maximal feature dimension considered is $100$-dimensional, and training is done with a known predictive variance similar to the LR setup. For the probabilistic model, we consider both a $1$-layered and a $2$-layered multi-layer perceptron (MLP) network with 32 hidden units in each, and either a \textsc{relu} or \textsc{tanh} activation function.

\textbf{Linear Classification (LC):} We experiment with a maximal $100$-dimensional setup with training done for $100,000$ iterations, out of which $25,000$ are used for warmup. Further, we train for both binary classification as well as a $5$-class classification setup.

\textbf{Nonlinear Classification (NLC):} We experiment with a maximal $100$-dimensional setup with training done for $250,000$ iterations, out of which $62,500$ are used for warmup. Further, we train for both binary classification as well as a $5$-class classification setup. For the probabilistic model, we consider both a $1$-layered and a $2$-layered multi-layer perceptron (MLP) network with 32 hidden units in each, and either a \textsc{relu} or \textsc{tanh} activation function.

\begin{figure*}
    \centering
    \captionsetup[subfigure]{font=scriptsize}
    \includegraphics[width=\textwidth]{Draft/Plots/dimension_trends/KL.pdf}
    \vspace{-7mm}
    \caption{\textbf{Trends of Performance over different Dimensions in Variable Dimensionality Setup:} We see that our proposed reverse KL methodology outperforms the forward KL one.}
    \vspace{-5mm}
    \label{fig:dim_kl}
\end{figure*}

\begin{figure*}
    \centering
    \captionsetup[subfigure]{font=scriptsize}
    \includegraphics[width=\textwidth]{Draft/Plots/dimension_trends/Model.pdf}
    \vspace{-7mm}
    \caption{\textbf{Trends of Performance over different Dimensions in Variable Dimensionality Setup:} We see that transformer models generalize better to different dimensional inputs than DeepSets.}
    \vspace{-5mm}
    \label{fig:dim_kl}
\end{figure*}

\begin{figure*}
    \centering
    \captionsetup[subfigure]{font=scriptsize}
    \includegraphics[width=\textwidth]{Draft/Plots/dimension_trends/Variational_Approximation.pdf}
    \vspace{-7mm}
    \caption{\textbf{Trends of Performance over different Dimensions in Variable Dimensionality Setup:} We see that normalizing flows leads to similar performances than Gaussian based variational approximation.}
    \vspace{-5mm}
    \label{fig:dim_kl}
\end{figure*}
\subsection{Model Misspecification}
\label{appdx:details_misspecification}
In this section, we provide the experimental details relevant to reproducing the results of Section~\ref{sec:experiments}.
All models during this experiment are trained with streaming data from the currently used dataset-generating function $\chi$, such that every iteration of training sees a new batch of datasets. Training is done with a batch size of $128$, representing the number of datasets seen during one optimization step. Evaluation for all models is done with $10$ samples from each dataset-generator used in the respective experimental subsection and we ensure that the test datasets are the same across all compared methods, i.e., baselines, forward KL, and reverse KL.

\textbf{Linear Regression Model:} The linear regression amortization models are trained following the training setting for linear regression fixed dimensionality, that is, $50,000$ training iterations with $12,500$ iterations of warmup. The feature dimension considered for this task is $1$-dimension. The model is trained separately on datasets from three different generators $\chi$: linear regression, nonlinear regression, and Gaussian processes, and evaluated after training on test datasets from all of them.
For training with datasets from the linear regression probabilistic model, the predictive variance $\sigma^2$ is assumed to be known and set as $0.25$. 
The same variance is used for generating datasets from the nonlinear regression dataset generator with $1$ layer, $32$ hidden units, and \textsc{tanh} activation function. 
Lastly, datasets from the Gaussian process-based generator are sampled similarly, using the GPytorch library~\cite{gardner2018gpytorch}, where datasets are sampled of varying cardinality, ranging from $64$ to $128$. We use a zero-mean Gaussian Process (GP) with a unit lengthscale radial-basis function (RBF) kernel serving as the covariance matrix. Further, we use a very small noise of $\sigma^2 = 1\mathrm{e}^{-6}$ in the likelihood term of the GP.
Forward KL training in this experiment can only be done when the amortization model and the dataset-generating function are the same: when we train on datasets from the linear regression-based $\chi$. Table \ref{tab:misspec_model} provides a detailed overview of the results.


\textbf{Nonlinear Regression Models:} The nonlinear regression amortization models are trained following the training setting for nonlinear regression fixed dimensionality, that is, $100,000$ training iterations with $25,000$ iterations of warmup. Here, we consider two single-layer perceptions with 32 hidden units with a \textsc{tanh} activation function. The feature dimensionality considered is $1$ dimension.
We consider the same dataset-generating functions as in the misspecification experiment for a linear regression model above. However, the activation function used in the nonlinear regression dataset generator matches the activation function of the currently trained amortization model. In this case, forward KL training is possible in the two instances when trained on datasets from the corresponding nonlinear regression probabilistic model. A more detailed overview of the results can be found in Table \ref{tab:misspec_model} and \ref{tab:misspec_gp}.

\begin{figure}
    \centering
    \includegraphics[width=\textwidth]{Draft/Plots/real_world/linear_regression_Vanilla.pdf}
    \caption{\textbf{Tabular Experiments $|$ Linear Regression with Diagonal Gaussian}: For every regression dataset from the OpenML platform considered, we initialize the parameters of a linear regression-based probabilistic model with the amortized inference models which were trained with a diagonal Gaussian assumption. The parameters are then further trained with maximum-a-posteriori (MAP) estimate with gradient descent. Reverse and Forward KL denote initialization with the correspondingly trained amortized model. Prior refers to a MAP-based optimization baseline initialized from the prior $\gN(0, I)$, whereas Xavier refers to initialization from the Xavier initialization scheme.}
    \label{fig:regression_linear_vanilla}
\end{figure}

\subsection{Tabular Experiments}
\label{appdx:details_tabular}
For the tabular experiments, we train the amortized inference models for (non-)linear regression (NLR/LR) as well as (non-)linear classification (NLC/LC) with $\vx \sim \mathcal{N}(\mathbf{0}, \mathbf{I})$ as opposed to $\vx \sim \gU(-\mathbf{1}, \mathbf{1})$ in the dataset generating process $\chi$, with the rest of the settings the same as \textsc{maximum-dim} experiments. For the nonlinear setups, we only consider the \textsc{relu} case as it has seen predominant success in deep learning. Further, we only consider a 1-hidden layer neural network with 32 hidden dimensions in the probabilistic model. 

After having trained the amortized inference models, both for forward and reverse KL setups, we evaluate them on real-world tabular datasets. We first collect a subset of tabular datasets from the OpenML platform as outlined in Appendix~\ref{appdx:datasets}. Then, for each dataset, we perform a 5-fold cross-validation evaluation where the dataset is chunked into $5$ bins, of which, at any time, $4$ are used for training and one for evaluation. This procedure is repeated five times so that every chunk is used for evaluation once.

For each dataset, we normalize the observations and the targets so that they have zero mean and unit standard deviation. For the classification setups, we only normalize the inputs as the targets are categorical. For both forward KL and reverse KL amortization models, we initialize the probabilistic model from samples from the amortized model and then further finetune it via dataset-specific maximum a posteriori optimization. We repeat this setup over $25$ different samples from the inference model. In contrast, for the optimization baseline, we initialize the probabilistic models' parameters from $\gN(0, I)$, which is the prior that we consider, and then train 25 such models with maximum a posteriori objective using Adam optimizer. 

While we see that the amortization models, particularly the reverse KL model, lead to much better initialization and convergence, it is important to note that the benefits vanish if we initialize using the Xavier-init initialization scheme. However, we believe that this is not a fair comparison as it means that we are considering a different prior now, while the amortized models were trained with $\gN(0, I)$ prior. We defer the readers to the section below for additional discussion and experimental results.



\end{document}
