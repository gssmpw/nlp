\section{INTRODUCTION}
\label{sec:introduction}

Causal inference \citep{pearl2009causality} is fundamental to our scientific understanding and practical decision-making.
In many settings, we do not know the causal relations between the variables, which we can learn with \emph{causal discovery} methods \citep{glymour2019review}.
These methods can be computationally demanding for large numbers of variables.
In many cases, we are only interested in estimating the causal effects between a small subset of variables, which does not require recovering the causal graph over all variables.
We formalize this setting as \emph{targeted causal effect estimation with an unknown graph}, a task that focuses on identifying the causal effects $P(T_i|do(T_j))$  between pairs of target variables $T_i,T_j \in \mathbf{T}$, where $\mathbf{T}$ is a  subset of all variables $\mathbf{V}$ in a \emph{computationally and statistically efficient way}.

We assume that we are in a causally sufficient setting, i.e., there are no unobserved confounders or selection bias.
Under these assumptions, we can use constraint-based causal discovery algorithms \citep{spirtes2000causation} to identify the \ac{MEC} of the causal graph \citep{verma90equivalence}, represented by a mixed graph, called the \ac{CPDAG}.
The \ac{CPDAG} can then be used to identify valid adjustment sets for causal effect estimation \citep{perkovic2015complete}.
However, discovering the \acs{CPDAG} over all variables can scale poorly in terms of \ac{CI} tests for large numbers of nodes \citep{mokhtarian2021recursive}.


\Acl{LCD} methods \citep{wang2014discovering, gupta2023local} address this issue for a pair of targets, a treatment and an outcome, by identifying the parent adjustment set of the treatment, but they cannot learn other types of adjustments that are statistically more efficient.
\citet{maasch2024local} learn to group nodes according to their ancestral relationship to a treatment-outcome pair, which can be used to identify various adjustment sets, but not necessarily optimal ones.
These algorithms focus only on two target variables and assume that we know the causal relations between them.
\citet{watson2022causal} propose an algorithm to discover the causal relations between multiple targets, which they call \emph{foreground variables}, but assume 
that the other variables, the \emph{background variables}, are all non-descendants of the target variables.


In this paper we propose Sequential Non-Ancestor Pruning (SNAP), an approach that efficiently discovers the ancestral relationships between multiple target variables, as well as their adjustments sets, from observational data.
We show that only possible ancestors of the target variables are required to both identify these relationships and provide efficient adjustment sets.
Thus, during the process of causal discovery we identify definite non-ancestors of the targets, remove them from consideration, and continue causal discovery only on the remaining variables.
SNAP is also straightforward to combine with readily available causal discovery algorithms, significantly decreasing their complexity in terms of execution time.
Our contributions are:
\begin{compactitem}
   \item We introduce the \emph{targeted causal effect estimation with an unknown graph} task in which we focus on estimating causal effects for a set of targets in a computationally and statistically efficient way.
    \item To solve this task, we propose the \ac{SNAP} framework.
    SNAP can be used as a sound and complete standalone method,
    or as a preprocessing  step that can be combined with standard causal discovery methods.
    \item We evaluate  \ac{SNAP} on simulated and real-world data, showing their benefits in reducing conditional independence tests and computation time.
\end{compactitem}


