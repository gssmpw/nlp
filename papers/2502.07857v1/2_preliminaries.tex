\section{PRELIMINARIES}
\label{sec:preliminaries}

We focus on graphs $G=(\mathbf{V}, \mathbf{E})$ with nodes $\mathbf{V}$ and edges $\mathbf{E}$.
The edges $\mathbf{E}$ can be undirected edges $X-Y$, directed edges $X \to Y$ or $X \gets Y$ and bidirected edges $X \leftrightarrow Y$ that are directed both towards $X$ and $Y$.
If two nodes are connected by an edge in a graph $G$, we say that they are \emph{adjacent}, and denote the set of nodes adjacent to $X$ as $Adj_G(X)$.
An undirected graph is a graph with only undirected edges, and a directed graph is a graph with only directed edges.
A mixed graph can contain undirected, directed and bidirected edges.

A path between two nodes $X$ and $Y$ is a sequence of distinct adjacent nodes that starts at $X$ and ends at $Y$.
A directed path from $X$ to $Y$ is a path where each edge on the path is directed towards $Y$.
A directed cycle is a directed path from a node to itself.
A \ac{DAG} is a graph with only directed edges and without directed cycles.
If there is a directed path from a node $X$ to another node $Y$ in a graph $G$, then we say that $X$ is an ancestor of $Y$ and $Y$ is a descendant of $X$.
We denote the set of ancestors of $Y$ as $An_G(Y)$ and the set of descendants of $X$ as $De_G(X)$.
We say that a set of nodes $\mathbf{V}' \subseteq \mathbf{V}$ is an \emph{ancestral set} in a DAG $D$ if for all $V \in \mathbf{V}'$ it holds that $An_D(V) \subseteq \mathbf{V}'$.

A causal graph $D$ is a \ac{DAG} that describes the data generating process of a joint observational distribution $p$ over variables $\mathbf{V}$ \citep{pearl2009causality}.
We assume \emph{causal sufficiency}, i.e., the distribution $p$ is over all variables $\mathbf{V}$ and 
there are no latent confounders or selection bias.
Furthermore, we assume that the distribution $p$ is \emph{Markov and faithful} to $D$, i.e., that for all $X,Y \in \mathbf{V}$ and $\mathbf{S} \subseteq \mathbf{V}\setminus \{X,Y\}$ it holds that the \acl{CI} $X \indep Y | \mathbf{S}$ in the distribution $p$ is equivalent to d-separation $X \perp_d Y | \mathbf{S}$ in the true causal graph $D$.

Under these assumptions, constraint-based causal discovery algorithms use conditional independence (CI) tests to identify causal relations between the variables.
However, without further assumptions, \ac{CI} tests alone generally cannot identify all causal relationships, because multiple causal graphs with conflicting causal relations can imply the same set of conditional independence.
This set of causal graphs is called the \emph{Markov Equivalence Class} (\ac{MEC}) of $D$.
\citet{verma90equivalence} show that all graphs in a \ac{MEC} have the same adjacencies and v-structures, i.e., patterns of $X \to Z \gets Y$, such that $X$ and $Y$ are not adjacent. \citet{meek1995causal} shows how to identify some further edge orientations.

We  graphically represent a \ac{MEC} by a mixed graph, called the Completed Partially oriented DAG (\ac{CPDAG}).
In a \ac{CPDAG} $G$, directed edges are oriented the same way in all graphs in the \ac{MEC} of $D$, while undirected edges indicate conflicting orientations between graphs in the \ac{MEC}.
Due to conflicting orientations, some ancestral relationships might hold only in some graphs in the \ac{MEC}, but not all \citep{zhang2006causal, roumpelaki2016marginal}.
If $X$ is an ancestor of $Y$ in at least one graph in the \ac{MEC} represented by the CPDAG $G$, then $X$ is a \emph{possible ancestor} of $Y$.

\begin{definition}
\label{def:possan}
Given a \ac{CPDAG} $G$, the node $X$ is a \emph{possible ancestor} of the node $Y$, denoted as $X\in PossAn_G(Y)$, iff $X \in An_D(Y)$ in at least one graph $D$ in the MEC of $G$, or equivalently, iff there exists a \emph{possibly directed path} from $X$ to $Y$ in $G$, i.e., a path from $X$ to $Y$ that only contains undirected edges or directed edges pointing towards $Y$.
Otherwise, $X$ is a \emph{definite non-ancestor} of $Y$.
The possible ancestors for sets of nodes $\mathbf{V}'$ are the union of the possible ancestors of all nodes
$
    PossAn_G(\mathbf{V}') = \bigcup_{V \in \mathbf{V}'} PossAn(V).
$
\end{definition}

Possible ancestry is transitive in a CPDAG: if $X$ is a possible ancestor of $Y$, and $Y$ is a possible ancestor of $Z$, then $X$ is a possible ancestor of $Z$. 
Similarly, $Y$ is a possible descendant of $X$, i.e., $Y \in PossDe_G(X)$, if there is a possibly directed path from $X$ to $Y$ in $G$.

The induced subgraph of a graph $G = (\mathbf{V}, \mathbf{E})$ over a subset of variables $\mathbf{V}^* \subseteq \mathbf{V}$ is a graph, denoted as $G|_{\mathbf{V}^*}$, with edges $\mathbf{E}' \subseteq \mathbf{E}$ that are between pairs of variables in $\mathbf{V}^*$.
Let $G(\mathbf{V})$ be the \ac{CPDAG} according to the observational distribution over $\mathbf{V}$ and $G(\mathbf{V}^*)$ be the CPDAG according to the marginal observational distribution over $\mathbf{V}^*$.
\citet{guo2023variable} (Lemma D.1) show  that if $\mathbf{V}^*$ is an ancestral set in the underlying causal graph, then $G(\mathbf{V})|_{\mathbf{V}^*} = G(\mathbf{V}^*)$.

Given a \ac{CPDAG} $G$, we can estimate causal effects between its variables \citep{maathuis2009estimating}, either single values, if they are identifiable, or as a set of values.
If a causal effect is identifiable, there are various adjustment sets available to estimate it.
Parents \citep{maathuis2015generalized} and definite ancestors \citep{henckel2024adjustment} of the cause $X$ are valid adjustment sets for any outcome $Y$ that is not a parent of $X$.
If the causal effect is identifiable, then we can always estimate the effect of a cause $X$ on an outcome $Y$ in a CPDAG $G$ with the canonical adjustment \citep{perkovic2018complete}, defined as 
$
PossAn_G(\{X,Y\}) \setminus (Forb_G(X,Y) \cup \{X, Y \}),
$
where $Forb_G(X,Y)$ is the \emph{forbidden set} defined as $PossDe_G(Cn_G(X,Y))$ such that $Cn_G(X,Y)$ is the set of all nodes on  possibly directed paths from $X$ to $Y$ in $G$. While this adjustment always exists if the causal effect is identifiable, it is not necessarily the most statistically efficient adjustment.
\citet{henckel2022graphical} define the asymptotically optimal adjustment set to estimate the causal effect of $X$ on $Y$ as
$$
Pa_G(Cn_G(X,Y)) \setminus (Forb_G(X,Y) \cup \{X\}),
$$
where $Pa_G$ are the definite parents in $G$.
All the listed adjustment sets are subsets of the possible ancestors of the outcome $Y$.
\citet{guo2023variable} show that non-ancestors of the outcome are \emph{uninformative}, in the sense that they have no impact on the statistical efficiency of estimating the causal effect.
Furthermore, \citet{henckel2022graphical} show that the parents of the cause $X$ are usually strongly correlated with $X$, making the asymptotic variance of the parent adjustment set suboptimal.