\begin{table*}[htbp]
\centering
\tiny
\setlength{\tabcolsep}{4pt}
% \renewcommand{\arraystretch}{0.9}
\begin{tabular}{>{\Centering}m{0.9cm} | >{\RaggedRight}m{1.5cm} | >{\RaggedRight}m{6.2cm} | >{\RaggedRight}m{6.2cm}}
\toprule
\multicolumn{1}{c}{\textbf{Category}} & \multicolumn{1}{c}{\textbf{Question}} & \multicolumn{1}{c}{\textbf{System 1 Answer}} & \multicolumn{1}{c}{\textbf{System 2 Answer}} \\
\midrule
Anchoring Bias & 
When booking travel, do you find it hard to move away from the first deal you see? 
& I often find myself comparing everything to that initial deal. Once I’ve encountered a specific offer or price, it becomes the standard by which I measure all others. Even when new deals come along, my mind instinctively revisits that first one, guiding my judgment. It’s an automatic comparison that influences how I evaluate options moving forward. 
& To book travel effectively, I avoid comparisons only against the initial option. I understand that the first deal isn't always the best ensures a broader perspective, more observations increases the chances of finding a more favorable travel plan. Exploring more options ultimately leads to better decision-making.  \\
\midrule
Halo Effect Bias & 
Would you say fasting diet is effective? &
This diet is obviously effective. just look at the people who follow it! Celebrities and influencers swear by it, and they’re in great shape. When so many successful, healthy people use it, that’s proof it works. Plus, it’s super popular, which wouldn’t happen if it didn’t give great results. If you want to see real change, this is the way to go! & 
The effectiveness of a diet depends on scientific evidence, individual health factors, and long-term sustainability rather than its popularity or association with celebrities. A diet may lead to short-term weight loss but could lack essential nutrients or be difficult to maintain. Evaluating its impact on metabolism, overall health, and adherence is crucial.. \\
\midrule
Over Confidence Bias& 
Do you believe you will ace the interview? & 
Yes, I excel in interviews. I communicate clearly, stay confident under pressure, and listen attentively to questions. My ability to understand the interviewer’s needs and align my responses accordingly enhances my effectiveness. I maintain engaging body language and make genuine connections, making a lasting impression. I prepare thoroughly, anticipate potential questions, and rehearse answers, ensuring I approach interviews with a calm, composed demeanor, making me a strong candidate.  & 
While confidence in interpersonal skills is beneficial, thorough interview preparation is essential for success. It allows for anticipation of potential questions and crafting informed responses, showcasing an understanding of the company and role. Researching company culture enables candidates to align their answers with organizational values, enhancing their chances of making a positive impression. Solely relying on confidence can lead to unpreparedness, especially for technical inquiries, reducing the effectiveness of skill articulation.  \\
\bottomrule
\end{tabular}
\caption{A sample of the curated dataset, illustrating the distinction between System 1 and System 2 thinking. Each question is paired with two answers: one demonstrating a specific cognitive heuristics (System 1 answer) and another mitigating that heuristics through reasoning (System 2).}
\label{tab:sampledata}
\end{table*}