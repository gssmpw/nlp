\documentclass{article}
\usepackage{graphicx}
\usepackage{amsmath}
\usepackage{cite}

\title{Super-Resolution via Fourier Transform Implicit Neural Representation}
\date{\today}

\begin{document}

\maketitle

\section{Introduction}
The super-resolution (SR) task is a critical issue widely discussed in image processing and computer vision fields. The main goal is to recover high-resolution (HR) images from low-resolution (LR) ones. Due to the broader popularity and efficient applicability of SR techniques in a wide variety of applications such as medical imaging, satellite data analysis, and video surveillance, addressing the associated issues remains an ongoing area of research. The traditional deep learning-based approaches, such as Super-Resolution Convolutional Neural Networks (SRCNN), have shown promising results; the remaining challenge is to reconstruct the final HR image based on both high-frequency (HF) and low-frequency (LF) components simultaneously. The tendency to learn LF components of the input data compared to the HF ones is called the Spectral Bias problem. 
In this proposal, a novel machine-learning algorithm based on Neural Networks is introduced. Fourier transform filters are used as activation functions to address the spectral bias problem and, hence, improve SR performance.

\section{Problem Statement and Proposed Approach}
One of the main issues in SR tasks is to reconstruct HR images based on HF components found in edges and fast-changing textures, as well as LF ones, which are the overall structure of the image. According to the general characteristics of standard activation functions, such as ReLU or Tanh, they are unable to differentiate between frequency components of the input image, which limits their capabilities as well. In the proposed method, standard activation functions are replaced by Fourier Transformer Filters to improve the limitations of the Neural Networks in SR tasks. By using high-pass and low-pass Fourier transform-based filters, the network can specifically focus on the HF and LF components of the image separately. In the proposed method, both HF and LF output will be combined to construct the final HR output image to ensure that the final HR image consists of both HF and LF components. We are going to test different fusion techniques to find the most effective one, such as concatenating both path features or simply summing the outputs. In the proposed method, high-pass filters help the network learn detailed edges and textures (HF), while low-pass filters focus on the general structure (LF), leading to a more comprehensive learning process.

\section{Benefits and Novelty}
The novelty of the proposed algorithm is the usage of Fourier transform filters as activation functions. By controlling the frequency bands that each neural network is focusing on, we can get to the hyperparameter that helps the algorithm to capture HF and Lf components of the image at the same time and can best describe the HR image. Furthermore, the parallel structure proposed can lead to efficient learning of both high and low-frequency features.

\section{Implementation}
To implement this idea, we design two separate neural networks that use Fourier transformer filters as activation functions rather than conventional activation functions like ReLU. The main challenge is to optimize the frequency bands they are working on in such a way that they complement each other. We need both networks to cover all frequency ranges in the image and to construct the final HR image based on all HF and LF components. Accordingly, different combination techniques will be examined to get to the most efficient one.

\section{conclusion}
The proposed algorithm has the potential to improve Neural Networks' capability in Super-Resolution tasks. While it will specifically focus on the HF and LF components of the image separately, the final fused algorithm is capable of addressing the spectral bias issue associated with Neural Networks.

Generally speaking, this approach can result in sharper, more detailed, high-resolution images, which is a critical issue in fields such as medical imaging and satellite data processing. Enjoying the strengths of frequency-specific learning, we aim to tackle the current SR technique's problems and address them as much as possible.

\bibliographystyle{plain}


\end{document}
