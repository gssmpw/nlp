
% To obtain the four possible sign combinations for $\epsilon_1, \epsilon_2$, we fix the low and high interest rates at $\gamma_l = 0.4$ and $\gamma_h = 0.8$ for these, the truncated gaussian distribution with the following pairs of mean and standard deviation $(\mu, \sigma) = (0.3, 0.1)$, $(0.1, 0.3)$, and $(0.1, 0.2)$ results in the cases $+\,+, -\,-$, and $+\,-$, respectively. These parameters are summarized in Table \ref{tab:distribution_parameters}.

% \juba{please don't use a table for this---we are not conveying so much info that a table is necessary.}\juba{use the following structure:}

% \paragraph{Distribution choices} \juba{fill out here, bullet points with the two distributions we look at. We need to explain what these distributions represent (ok a gaussina is a standard population dist; what does a piecewise linear represent in real life and why is it a good function to look at?)}

% \paragraph{Parameter choices} \juba{explain parameters here. YOu also need to explain that we tried a large number of other parameters and that we pick these parameters because they are representative. Ow it looks like we just tried one pair of arbitrary gammas and our results do not generalize}

% We provide representative figures for the strategies of both banks under Algorithm \ref{alg:Hedge} and \ref{alg:Hedge:stoc} in Section \ref{sec:exp-dyna}, illustrating their convergence to the respective Nash equilibria. Additionally, we plot the distance of the banks' strategies to their Nash equilibrium strategies, averaged over five random initializations, in Section \ref{sec:exp-dist}.  

% \begin{table}[H]
%     \centering
%     \begin{tabular}{c c l}
%         \hline
%         $(\gamma_l, \gamma_h)$ & Credit Distribution & sign($\epsilon_1$), sign($\epsilon_2$) \\
%         \hline
%         $(0.4, 0.8)$ & Truncated Gaussian$(0.3, 0.1)$ & \(+\,+\) \\
%         $(0.4, 0.8)$ & Truncated Gaussian$(0.1, 0.3)$ &  \(-\,-\) \\
%         $(0.4, 0.8)$ & Truncated Gaussian$(0.1, 0.2)$ &  \(+\,-\) \\
%         $(0.6, 0.7)$ & Piecewise uniform (Eq. \eqref{eq:pu-dist}) &  \(-\,+\) \\
%         \hline
%     \end{tabular}
%     \caption{Interest rate values and distribution parameters used to construct different sign cases of $(\epsilon_1, \epsilon_2)$.}
%     \label{tab:distribution_parameters}
% \end{table}


% \juba{this is way too informal and reads like a class project report over a conference paper. This needs to be re-organized. The experimental setup n 5.1. needs to have a clear description of all distributions that are being used. You do not want to argue the technical reason why you pick a piecewise linear distribution---it looks like we are reverse engineering the problem which looks bad. You need to explain instead why this distribution is a natural one to look at}To obtain the  $-\,+$ case, we performed a grid search over $(\gamma_l, \gamma_h)$ for interest rates and $(\mu, \sigma)$ for the truncated normal distribution but were unable to find a game instance that resulted in $\epsilon_1 < 0$ and $\epsilon_2 > 0$. So, we specifically designed the following piecewise uniform distribution that resulted in this sign pair. We set $\gamma_l = 0.6, \gamma_h = 0.7, \tau_l = \frac{1}{2+\gamma_h}, \tau_h = \frac{1}{2+\gamma_l}$ and use a piecewise uniform distribution defined as: 

\paragraph{Credit score distribution} 
We model customer credit scores using two types of distributions:  

\begin{itemize}
    \item Truncated Gaussian (\(\mu, \sigma\)): This is a standard normal distribution conditioned to lie within the normalized credit score range \( y \in [0,1] \), and models customers with credit scores symmetrically distributed around a mean value $\mu \in [0,1]$.
    \item Piecewise Uniform Distribution: This distribution models \(\beta_1\) fraction of customers with credit scores uniformly distributed in \([0, \tau_l]\), a fraction \(\beta_2\) with scores in \((\tau_l, \tau_h)\), and the remainder in \([\tau_h, 1]\).
  \begin{equation}
    \label{eq:pu-dist}
    y \sim  
    \begin{cases}  
        \frac{\beta_1}{\tau_l}, & \quad y \in [0, \tau_l] \\[4pt]  
        \frac{\beta_2}{\tau_h - \tau_l} & \quad y \in (\tau_l, \tau_h) \\[4pt]  
        \frac{1 - (\beta_1 + \beta_2)}{1 - \tau_h}, & \quad y \in [\tau_h, 1]
    \end{cases}
\end{equation}  
\end{itemize}

\paragraph{Interest Rates}  
We ran experiments across a wide range of interest rates, where $0 < \gamma_l < \gamma_h < 1$, and both families of credit score distributions described above. Our findings are consistent across different parameter choices. For the representative figures in Sections \ref{sec:exp-dyna} and \ref{sec:exp-dist}, we use the following distributions and interest rate values, corresponding to the four possible sign pairs for $(\epsilon_1, \epsilon_2)$:
\begin{itemize}
    \item (+, +): $\gamma_l = 0.4$, $\gamma_h = 0.8$, with a truncated Gaussian distribution $(\mu, \sigma) = (0.3, 0.1)$.
    \item (-, -): $\gamma_l = 0.4$, $\gamma_h = 0.8$, with a truncated Gaussian distribution $(\mu, \sigma) = (0.1, 0.3)$.
    \item (+, -): $\gamma_l = 0.4$, $\gamma_h = 0.8$, with a truncated Gaussian distribution $(\mu, \sigma) = (0.1, 0.2)$.
    \item (-, +): $\gamma_l = 0.6$, $\gamma_h = 0.7$, with $\beta_1 = 0.01, \beta_2 = 0.95$ and credit thresholds $\tau_l = \frac{1}{2.7}$ and $\tau_h = \frac{1}{2.6}$ 
\end{itemize}