\subsection{Equilibrium Concepts}

We start by defining the main solution concepts used in this section, in the \emph{one-shot} setting. The main solution concept used in this paper is the concept of \emph{Nash equilibrium}: 

\begin{definition}[Nash equilibrium]
A pair of strategies \((\theta_1^*, \theta_2^*)\) constitutes a \emph{pure Nash equilibrium} (pure NE) iff:
\[
u_1(\theta_1^*, \theta_2^*) \geq \max_{\theta_1\in \Gamma\times \Lambda} u_1(\theta_1, \theta_2^*), \quad \text{and} \quad u_2(\theta_1^*, \theta_2^*) \geq \max_{\theta_2\in \Gamma\times \Lambda} u_2(\theta_1^*, \theta_2).
\]
\noindent A pair of mixed strategies \((p_1^*, p_2^*)\) constitutes a \emph{mixed Nash equilibrium} (mixed NE) if and only if the following conditions hold:  
\[
\E_{\theta_1 \sim p_1^*, \theta_2 \sim p_2^*}[u_1(\theta_1, \theta_2)] \geq \max_{p \in \Delta(\Gamma \times \Lambda)} \E_{\theta_1 \sim p, \theta_2 \sim p_2^*}[u_1(\theta_1, \theta_2)],
\]
and  
\[
\E_{\theta_1 \sim p_1^*, \theta_2 \sim p_2^*}[u_2(\theta_1, \theta_2)] \geq \max_{p \in \Delta(\Gamma \times \Lambda)} \E_{\theta_1 \sim p_1^*, \theta_2 \sim p}[u_2(\theta_1, \theta_2)].
\]
\end{definition}

In Nash equilibria, players pick their actions simultaneously and independently of each other. Beyond Nash Equilibria, we also consider the concept of Correlated Equilibria (CEs) of a game, where coordination among players is allowed: 

\begin{definition}[Correlated Equilibrium (CE)] 
\label{defn:CE}
A joint probability distribution $p$ \text{over} pairs of strategies $(\theta_1,\theta_2) \in \Theta_1 \times \Theta_2$ is a correlated equilibrium if, for each $i \in \{1,2\}$,
\[
\sum_{\theta_{2} \in \Theta_{2}} p(\theta_1,\theta_2)\, u_1(\theta_1,\theta_2)
\geq 
\sum_{\theta_2 \in \Theta_2} p(\theta_1,\theta_2)\, u_1(\theta'_1,\theta_2),
~~\forall \theta_1,\theta'_1 \in \Theta_1,
\]
and 
\[
\sum_{\theta_{1} \in \Theta_{1}} p(\theta_1,\theta_2)\, u_2(\theta_1,\theta_2)
\geq 
\sum_{\theta_1 \in \Theta_1} p(\theta_1,\theta_2)\, u_2(\theta_1,\theta'_2),
~~\forall \theta_2,\theta'_2 \in \Theta_2.
\]
\end{definition}

Informally, correlated equilibria use a correlation or signaling device that recommend an action $(\theta_1,\theta_2)$ to the players according to a \emph{joint} distribution (for example, think of a traffic signal at an intersection). At a correlated equilibrium, no player benefits from deviating from the recommended action in expectation.  
%{\color{red} Vidya: I think some of this is maybe removable, especially since we ultimately do not have real results for CE beyond the one-shot characterization} \gua{I added some discussion under the CE theorem. }


\subsection{One-Shot Equilibrium Characterization for $n=2$ Actions} 

To simplify presentation, we introduce the following function, which is useful in writing down players' utilities: 
%which is essentially the utility under $D_y$ for a particular bank as a function of its interest rate $\gamma$, threshold $\tau$ and threshold range $[\tau_a, \tau_b]$ (recall that the threshold range is determined in part by the bank's own choice of threshold $\tau$ but also the strategy of the competing bank, if it exists).
% {\color{red} Is it always true that $\tau_a = \tau$? Is it ever possible for $\tau_b < \tau_a$, meaning that the bank would never attract customers?}
% Then, if $\tau_a<\tau_b$, the utility of the bank is given by:
\begin{equation}
\label{eqn:utility-onebank}
    f_{D_y}(\gamma,\tau_a,\tau_b)
    \triangleq
    \int_{\tau_a}^{\tau_b}[(2+\gamma)y-1]p(y)dy.
\end{equation}
We often omit the subscript $D_y$ when it is clear from context. Intuitively, this represents the utility for a single bank who sets the interest rate as $\gamma$ and is able to attract all the customers in the range $Y \in [\tau_a,\tau_b]$. %\gua{I added this sentence}

As discussed in Section \ref{sec1}, we focus on the simplest case where there are two choices for each parameter: \(0 \leq \tau_{\ell} < \tau_{h} \leq 1\) and \(0 \leq \gamma_{\ell} < \gamma_{h} \leq 1\). In this scenario, the utility of Bank 1 under each pair of decisions forms a \(4 \times 4\) matrix (as shown in Table \ref{tab:my-table}). We consider the canonical case where:
\[
\tau_{\ell} = \frac{1}{2 + \gamma_{h}}, \quad \text{and} \quad \tau_{h} = \frac{1}{2 + \gamma_{\ell}}.
\]

These choices for the thresholds are natural from Equation~\eqref{eqn:utility-onebank}. Indeed, it is a dominant strategy for a \emph{rational} bank with interest rate $\gamma$ to set a threshold of $\tau^*(\gamma) := \frac{1}{ 2 + \gamma}$. Setting a higher threshold $\tau > \tau^*(\gamma)$ leads to ignoring customers in the range $y \in [\tau^*(\gamma), \tau]$ that provide a utility of $(2 + \gamma) y - 1 > 0$, while setting a lower threshold $\tau < \tau^*(\gamma)$ leads to potentially incorporating some customers in the range $y < \tau^*(\gamma)$ that would yield a utility of $(2 + \gamma) y - 1 < 0$ as a result of low probability of repayment; both cases cannot improve total utility.
%{\color{red} I think the originally written discussion of threshold range $[\tau_a,\tau_b]$ was quite confusing. I have taken a shot at revising it}



\begin{table}[!h]
\centering
\begin{tabular}{@{}ccccc@{}}
\toprule
\textbf{} &
  \textbf{$(\tau_{\ell},\gamma_{\ell})$} &
  \textbf{$(\tau_{\ell},\gamma_{h})$} &
  $(\tau_{h},\gamma_{\ell})$ &
  $(\tau_{h},\gamma_{h})$ \\ \midrule
$(\tau_{\ell},\gamma_{\ell})$ &
  $\frac{1}{2}f(\gamma_{\ell},\tau_{\ell},1)$ &
  0 &
  $\frac{1}{2}f(\gamma_{\ell},\tau_{h},1)$ &
  0 \\ \midrule
$(\tau_{\ell},\gamma_{h})$ &
  $f(\gamma_{\ell},\tau_{\ell},1)$ &
  $\frac{1}{2}f(\gamma_{h},\tau_{\ell},1)$ &
  $f(\gamma_{\ell},\tau_{h},1)$ &
  $\frac{1}{2}f(\gamma_{h},\tau_h,1)$ \\ \midrule
$(\tau_{h},\gamma_{\ell})$ &
  $f(\gamma_{\ell},\tau_{\ell},\tau_h)+\frac{1}{2}f(\gamma_{\ell},\tau_h,1)$ &
  $f(\gamma_{h},\tau_{\ell},\tau_h)$ &
  $\frac{1}{2}f(\gamma_{\ell},\tau_h,1)$ &
  0 \\ \midrule
$(\tau_{h},\gamma_{h})$ &
  $f(\gamma_{\ell},\tau_{\ell},1)$ &
  $f(\gamma_{h},\tau_{\ell},\tau_h)+\frac{1}{2}f(\gamma_{h},\tau_h,1)$ &
  $f(\gamma_{\ell},\tau_{h},1)$ &
  $\frac{1}{2}f(\gamma_{h},\tau_h,1)$ \\ \bottomrule
\end{tabular}
\caption{Utility matrix for bank 1. The rows are the strategies for bank 2, while the columns are the strategies for bank 1.}
\label{tab:my-table}
\end{table}


We now characterize the pure Nash Equilibria of our game. To do so, let us introduce the shorthand notation $\epsilon_1 = \frac{1}{2}f(\gamma_{h},\tau_{\ell},1)-f(\gamma_{\ell},\tau_{h},1)$, and $\epsilon_2 = f(\gamma_{h},\tau_{\ell},\tau_{h})- \frac{1}{2}f(\gamma_{\ell},\tau_h,1)$. $\epsilon_1$ represents the change in utility for a bank when it switches its decision from $(\tau_{\ell},\gamma_{h})$ to $(\tau_{h},\gamma_{\ell})$, while its opponent selects $(\tau_{\ell},\gamma_{h})$. Similarly, $\epsilon_2$ denotes the change in utility for the same bank when it switches its decision from $(\tau_{\ell},\gamma_{h})$ to $(\tau_{h},\gamma_{\ell})$, while its opponent selects $(\tau_{h},\gamma_{\ell})$. Note that, once the values of $\tau$ and $\gamma$ are fixed, these two parameters are functions of the customer distribution $D_y$. 
The following result fully characterizes the pure Nash Equilibria of the Bank Game.
% The pure Nash Equilibria of our problem can be found as follows:


\begin{theorem}
\label{thm:pureNE}
We have that:
\begin{itemize}
    \item  If $\epsilon_1<0$, and $\epsilon_2>0$, then the pure \textbf{asymmetric} Nash equilibria are  given by $((\tau_{\ell},\gamma_h),(\tau_h, \gamma_{\ell}))$, and $((\tau_{h},\gamma_{\ell}),(\tau_{\ell}, \gamma_{h}))$;
    \item If $\epsilon_1<0$ and $\epsilon_2<0$, then the pure \textbf{symmetric} Nash equilibrium is given by  $((\tau_{h},\gamma_{\ell}),(\tau_h, \gamma_{\ell}))$;
    \item If $\epsilon_1>0$ and $\epsilon_2>0$, then the pure \textbf{symmetric} Nash equilibrium is given by $((\tau_{\ell},\gamma_h), (\tau_{\ell},\gamma_{h}))$;
    \item If $\epsilon_1>0$ and $\epsilon_2<0$, then the pure \textbf{symmetric} Nash equilibria are given by $((\tau_{h},\gamma_{\ell}),(\tau_h, \gamma_{\ell}))$ and $((\tau_{\ell},\gamma_h), (\tau_{\ell},\gamma_{h}))$.
\end{itemize}
\end{theorem}

% \begin{proof}
% The full proof can be found in Appendix \ref{proof:Theorem:pure:NE}.
% \end{proof}

The proof of Theorem~\ref{thm:pureNE} is given in Appendix~\ref{proof:Theorem:pure:NE}.
Theorem \ref{thm:pureNE} demonstrates that the structure of pure Nash equilibria of this problem  critically depends on the signs of the customer-distribution-dependent quantities $\epsilon_1$ and $\epsilon_2$. In most cases, there are only symmetric Nash equilibria (i.e., both banks choose the same threshold and interest rate). However, interestingly, there exists an asymmetric Nash equilibrium when $\epsilon_1<0$ and $\epsilon_2>0$. 
%\juba{I am commenting out the continuous distribution case. Please also comment out of Appendix. This is just making the paper a lot weaker and at the same time distracting from the main point + the characterization is incomplete compared to the discrete case}
% Note that this is very different from the continuous case, where there are only \emph{symmetric} pure Nash equilibria. To be more specific, we provide the following conclusions for the continuous case. The proof is given in Appendix \ref{proof:continous}.


% \begin{theorem}
% \label{thm:continous:peq}
% Assume $\emph{Supp}(D_y)=[0,1]$, and let $p(y)$ be the p.d.f of $D_y$. Then, if the following condition on $D_y$ holds:
% $$ \frac{1}{2}\int_{0.5}^1[2y-1]p(y)dy\geq \int^{0.5}_{\frac{1}{3}}[3y-1]p(y)dy,$$
% then the only pure NE of this problem is $\theta_1^*=\theta^*_2=(0.5,0)$. If the condition is violated, then there is no pure Nash equilibrium. Assume $D_y$ is a uniform distribution $\emph{U}([0,1])$. Then $\theta_1^*=\theta^*_2=(0.5,0)$ is a pure NE. 
% \end{theorem}






Next, we consider the \emph{strictly} or \emph{non-pure mixed} Nash Equilibria of our problem, meaning any Nash Equilibrium that puts non-zero probability on more than one pure strategy for at least one of the banks. To simplify notation, we write $(\tau_{\ell},\gamma_{\ell}), (\tau_{\ell},\gamma_h), (\tau_{h},\gamma_{\ell}),$ $(\tau_{h},\gamma_{h})$ as  $\theta^{(1)},\dots, \theta^{(4)}$, 
respectively. 
In this notation, let $p_1,p_{2}\in\Delta_4$ be the candidate mixed strategies for Bank 1 and Bank 2, where $p_{i}=[p_{i,1},\dots,p_{i,4}]$ and $p_{i,j}$ denotes the probability weight that Bank $i$ assigns to decision $\theta^{(j)}$. 
We then have the following result.

\begin{theorem}
\label{thm:mixed-NE}
% Let $p_1,p_{2}\in\Delta_4$ be the mixed strategies for Bank 1 and Bank 2, where $p_{i}=[p_{i,1},\dots,p_{i,4}]$, with $p_{i,j}$ denoting the probability of Bank $i$ assigned to decision $\theta^{(j)}$ in a mixed Nash Equilibrium. 
Let us define
$$ c=\frac{f(\gamma_{\ell},\tau_h,1)-2f(\gamma_h,\tau_{\ell},\tau_h)}{f(\gamma_h,\tau_h,1)-f(\gamma_{\ell},\tau_h,1)-f(\gamma_h,\tau_{\ell},\tau_h)}.$$
\begin{itemize}
    \item If $\epsilon_1>0$ and $\epsilon_2<0$, then we have $c\in(0,1)$, and $p_1=p_2=[0,c,1-c,0]$ is the unique strictly mixed Nash equilibrium;
    \item If $\epsilon_1<0$ and $\epsilon_2>0$, then we have $c\in(0,1)$, and $p_1=p_2=[0,c,1-c,0]$ is the unique strictly mixed Nash equilibrium.  
    \item If $\epsilon_1$ and $\epsilon_2$ have the same sign, no strictly mixed Nash Equilibrium exists.
\end{itemize}
\end{theorem}
The proof of Theorem~\ref{thm:mixed-NE} is provided in Appendix~\ref{Proof:Theorem:mixedNE}.
This result states that when $\epsilon_1$ and $\epsilon_2$ have the same sign, only pure Nash Equilibria  exist. Conversely, when $\epsilon_1$ and $\epsilon_2$ have opposite signs, non-pure mixed Nash equilibria arise. 

Finally, we provide a complete characterization of correlated equilibria of our problem: 
\begin{theorem}
\label{thm:cE}
Let 
\begin{equation*}
P = \begin{pmatrix}
p_{11} & p_{12} & p_{13} & p_{14} \\
p_{21} & p_{22} & p_{23} & p_{24} \\
p_{31} & p_{32} & p_{33} & p_{34} \\
p_{41} & p_{42} & p_{43} & p_{44}
\end{pmatrix}  
\end{equation*}
be the matrix of probability representing the joint strategies of the players; i.e., $p_{ij}$ denotes the probability that Bank $1$ plays $\theta^{(i)}$ and Bank $2$ plays $\theta^{(j)}$. Then:
\begin{itemize}
 \item  If $\epsilon_1<0$ and $\epsilon_2>0$, $P$ is a CE if and only if $\min\{p_{23},p_{32}\}>\max\left\{ \frac{|\epsilon_1|}{\epsilon_2}p_{22}, \frac{\epsilon_2}{|\epsilon_1|}p_{33} \right\}$
 ;
 \item  If $\epsilon_1<0$ and $\epsilon_2<0$, $P$ is a CE if and only if $p_{33}=1$;
  \item  If $\epsilon_1>0$ and $\epsilon_2>0$, $P$ is a CE if and only if $p_{22}=1$;
   \item  If $\epsilon_1>0$ and $\epsilon_2<0$, $P$ is a CE if and only if $\min\{p_{22},p_{33}\}>\max\left\{ \frac{|\epsilon_2|}{\epsilon_1}p_{23}, \frac{\epsilon_1}{|\epsilon_2|}p_{32} \right\}$.
\end{itemize}
\end{theorem}
%
The proof of Theorem~\ref{thm:cE} is provided in Appendix~\ref{proof:CE}. As stated in Theorem~\ref{thm:cE}, when $\epsilon_1$ and $\epsilon_2$ share the same sign, the set of correlated equilibria (CE) coincides with the set of mixed Nash equilibria (NE). However, if $\epsilon_1$ and $\epsilon_2$ have opposite signs, the set of CE becomes strictly larger than the set of mixed NEs. In the next section, we introduce an online learning dynamic that converges specifically to mixed NEs, avoiding other CEs.

% \begin{proof}
% The full proof can be found in Appendix \ref{proof:CE}.
% \end{proof}


