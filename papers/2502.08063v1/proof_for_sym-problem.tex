\noindent \textbf{b) Symmetric initialization:}  
Let $p^{(t)}_{2,2}\epsilon_1+p^{(t)}_{2,3}\epsilon_2=\Delta^{(t)}_2$, and $p^{(t)}_{1,2}\epsilon_1+p^{(t)}_{1,3}\epsilon_2=\Delta^{(t)}_1$. Moreover, let $r^{(t)}_1=\frac{p^{(t)}_{1,2}}{p^{(t)}_{1,3}}$, and $r^{(t)}_2=\frac{p^{(t)}_{2,2}}{p^{(t)}_{2,3}}$. Assume $\max\{\Delta^{(1)}_2,\Delta^{(1)}_1\}\leq 0.$  We have 
\[
  r_{1}^{(t+1)} = r_1^{(t)}\exp\left(\eta{\left(p^{(t)}_{2,2}\epsilon_1 + p^{(t)}_{2,3}\epsilon_2\right)}\right) = r_1^{(t)}\exp(\eta\Delta_2^{(t)}),
\]
and 
\[
  r_{2}^{(t+1)} = r_2^{(t)}\exp\left(\eta{\left(p^{(t)}_{1,2}\epsilon_1 + p^{(t)}_{1,3}\epsilon_2\right)}\right) = r_2^{(t)}\exp(\eta\Delta_1^{(t)}).
\]

\noindent Since $\max\{\Delta^{(1)}_2,\Delta^{(1)}_1\}\leq 0.$, we know that $r_{i}^{(2)}\leq r_{i}^{(1)}$ for $i\in[2]$. We consider two situations: 
\begin{itemize}
    \item Situation 1: $\max\{\Delta^{(t)}_2,\Delta^{(t)}_1\}\leq 0$ for all $t\geq 1$. Then, we know that the sequences $\{r_{i}^{(t)}\}_{t=1}^{\infty}$ is non-increasing, i.e., 
    $$ r_{i}^{(1)}\geq  r_{i}^{(2)}\geq \dots \geq 0>-\infty.$$
    According to the monotone convergence theorem, it implies that $\lim\limits_{t\rightarrow \infty}r_{i}^{(t)}$ exists. Let $\lim\limits_{t\rightarrow \infty}r_{i}^{(t)}=x_i$, for some $x_i\geq 0$. If $x_i=0$, then, combining with the fact that  $p_{i,2}^{(t)}+p_{i,3}^{(t)}=1$, we have 
$$\lim\limits_{t\rightarrow \infty} \Delta_{i}^{(t)}= \epsilon_2>0,$$    
which contradicts with the assumption that $\max\{\Delta^{(t)}_{2},\Delta_{1}^{(t)}\}\leq 0$. Therefore, we draw the conclusion that $x_i>0$ for $i\in[2]$. \\

On the other hand, since both $r_i^{(t)}$ is non-increasing for $i\in[2]$, and $\max\{\Delta^{(t)}_2,\Delta^{(t)}_1\}\leq 0$, we have 
$$\Delta_i^{(1)}\leq \Delta_i^{(2)}\leq \dots\leq 0<+\infty,  $$
which means that the limit of $\Delta_{i}^{(t)}$ also exist. 
Let $\lim\limits_{t\rightarrow \infty} \Delta^{(t)}_i=y_i\leq 0$. If $y_i<0$, it means that $r_j^{(t)}$ (for $j\not=i$) decreases with a constant rate, and that implies that  $x_j=\lim_{t\rightarrow \infty} r^{(t)}_j=0$, which contradicts with the conclusion we obtain above. Therefore, we know that $\lim_{t\rightarrow \infty} \Delta^{(t)}_i=0$. Combining with Theorem \ref{thm:mixed-NE}, it happens at the mixed NEs. 
\item Situation 2: There exists $t^{o}<\infty$, where $\Delta_i^{(t^{(o)})}\leq 0$, and  $\Delta_j^{(t^{(o)})}> 0$. In this case, if $\Delta_i^{(t^{(o)})}< 0$, then we have 
$$r_i^{(t^{(o)}+1)} = r_i^{(t^{(o)})}\exp\left(\eta \Delta_j^{(t^{(o)})}\right)>r_i^{(t^{(o)})}, $$
and 
$$r_j^{(t^{(o)}+1)} = r_j^{(t^{(o)})}\exp\left(\eta \Delta_i^{(t^{(o)})}\right)<r_j^{(t^{(o)})}, $$
since $\epsilon_1<0$ and $\epsilon_2>0$, it means that $\Delta^{t^{(o)+1}}_{i}<\Delta^{t^{(o)}}_{i}<0$, and  $\Delta^{t^{(o)+1}}_{j}>\Delta^{t^{(o)}}_{j}>0$. Therefore, by induction, it can be easily seen that, after $t^{(o)}$, the algorithm converges to the NE $((\tau_{\ell},\gamma_h),(\tau_{h},\gamma_{\ell}))$ exponentially fast. If $\Delta_{i}^{t^{(o)}}=0$, then we have 
$$r_i^{(t^{(o)}+1)} = r_i^{(t^{(o)})}\exp\left(\eta \Delta_j^{(t^{(o)})}\right)>r_i^{(t^{(o)})}, $$
and 
$$r_j^{(t^{(o)}+1)} = r_j^{(t^{(o)})}.$$
We have $\Delta^{t^{(o)+1}}_{i}<\Delta^{t^{(o)}}_{i}<0$, and  $\Delta^{t^{(o)+1}}_{j}=\Delta^{t^{(o)}}_{j}>0$. Then we can apply the same analysis for $t^{(o)+1}$, and the algorithm will still converge to $((\tau_{\ell},\gamma_h),(\tau_{h},\gamma_{\ell}))$.
\end{itemize}
Finally, for the opposite case where $\min\{\Delta^{(1)}_2,\Delta^{(1)}_1\}\geq 0,$ the proof is very similar. Here we only briefly outline the proof to avoid redundancy. Let $q_{i}^{(t)}=\frac{p_{i,3}^{(t)}}{p_{i,2}^{(t)}}$ for $i\in[2]$. We have 
$$ q^{(t+1)}_{i}= q^{(t)}_{i}\exp\left(-\eta\Delta_{j}^{(t)}\right).$$
If $\min\{\Delta^{(t)}_2,\Delta^{(t)}_1\}\geq 0$ for all $t\geq 1$, then $q_{i}^{(t)}$ is non-increasing. Following similar proof as above, we have $\lim_{t\rightarrow \infty}\Delta_{i}^{(t)}=0$, which implies that the algorithm converges to the mixed NEs. On the other hand, if there exists $t^{o'}<\infty$, where $\Delta_i^{(t^{(o)})}\geq 0$ and $\Delta_j^{(t^{(o)})}< 0$, the proof goes back to the asymmetric initialization case after $t^{(o')}$.\\

\noindent \textbf{b) Asymmetric initialization:}   Let $p^{(t)}_{2,2}\epsilon_1+p^{(t)}_{2,3}\epsilon_2=\Delta^{(t)}_2$, and $p^{(t)}_{1,2}\epsilon_1+p^{(t)}_{1,3}\epsilon_2=\Delta^{(t)}_1$. Let $r^{(t)}_1=\frac{p^{(t)}_{1,2}}{p^{(t)}_{1,3}}$, and $r^{(t)}_2=\frac{p^{(t)}_{2,2}}{p^{(t)}_{2,3}}$. Without loss of generality, first assume $\Delta_1^{(1)}<0$, and $\Delta_2^{(1)}>0$.  Since 
\[
  r_{1}^{(t+1)} = r_1^{(t)}\exp\left(\eta{\left(p^{(t)}_{2,2}\epsilon_1 + p^{(t)}_{2,3}\epsilon_2\right)}\right) = r_1^{(t)}\exp(\eta\Delta_2^{(t)}),
\]
and 
\[
  r_{2}^{(t+1)} = r_2^{(t)}\exp\left(\eta{\left(p^{(t)}_{1,2}\epsilon_1 + p^{(t)}_{1,3}\epsilon_2\right)}\right) = r_2^{(t)}\exp(\eta\Delta_1^{(t)}),
\]
we know that $r_1^{(2)}> r_1^{(1)}$, and  $r_2^{(2)}< r_2^{(1)}$, which causes $\Delta_1^{(2)}>\Delta_1^{(1)}$, and $\Delta_2^{(2)}<\Delta_2^{(1)}$. Consider two cases:
\begin{itemize}
    \item Situation 1: $\Delta_1^{(t)}\leq 0$, $\Delta_2^{(t)}\geq 0$ for all $t$. Then, we have that the sequence $\{r_1^{(t)}\}_{t}$ is non-decreasing, and the sequence $\{r_2^{(t)}\}_{t}$ is non-increasing. Combining with the fact that $\epsilon_1>0$, $\epsilon_2<0$, and $p_{i,2}^{(t)}+p_{i,3}^{(t)}=1$, implies that $\Delta_1^{(t)}$ is non-decreasing and $\Delta_2^{(t)}$ is non-increasing. Similar to the proof of Situation 1 in Case III, it leads to the conclusion that $\lim\limits_{t\rightarrow\infty}\Delta_{i}^{(t)}=0$, which only happens at the mixed NE. 
    \item Situation 2: there exists a time $t^{(o)}$, such that $\Delta_1^{(t^{(o)})}>0$
    
\end{itemize}


