For each of the four sign pairs, the Nash equilibria of the one-shot game and the dynamics leading to those equilibria under the Exponential Weights dynamic are summarized in Table \ref{tab:NE_dyna_summary}. We initialize both banks with \( p_1^{(1)} = (0.1, 0.5, 0.3, 0.1) \) and \( p_2^{(1)} = (0.1, 0.3, 0.5, 0.1) \). For the \(+\,-\) and \(-,+\) cases, however, the Nash equilibrium to which the dynamics converge depends on the initial strategies of the banks (which is consistent with Theorem 4).
For these, we also present the dynamics with alternate initializations that violate the conditions in Theorem 4 Parts III and IV; despite this, the dynamics are still observed to converge, but to possibly different pure Nash equilibria.



\begin{table}[H]
    \centering
    \small
    \resizebox{0.95\textwidth}{!}{
    \begin{tabular}{|c|c|c|}
        \hline
        Sign($\epsilon_1,\epsilon_2$) & Nash Equilibria & Dynamics of Algorithm \ref{alg:Hedge} and \ref{alg:Hedge:stoc} \\ \hline
        \(-\,-\) & \(\left( (\tau_h, \gamma_l), (\tau_h, \gamma_l) \right)\) & Converges to unique pure NE (Fig.~\ref{fig:dyna-mm}) \\ \hline
        \(+\,+\) & \(\left( (\tau_l, \gamma_h), (\tau_l, \gamma_h) \right)\) & Converges to unique pure NE (Fig.~\ref{fig:dyna-pp}) \\ \hline
        \(-\,+\) & \(\begin{array}{c} 
            NE_1': \left( (\tau_h, \gamma_l), (\tau_l, \gamma_h) \right), \\ 
            NE_2': \left( (\tau_l, \gamma_h), (\tau_h, \gamma_l) \right), \\ 
            NE_3': ((0,c',1-c',0), (0,c',1-c',0))
        \end{array}\) &  
        \begin{tabular}{c}
            Depends on initialization: \\ 
            Converges to the pure $NE_1'$ or $NE_2'$ (Fig.~\ref{fig:dyna-mp-ass}) \\ 
            or the mixed $NE_3'$ (Fig.~\ref{fig:dyna-mp-sym}) 
        \end{tabular} \\ \hline
        \(+\,-\) & \(\begin{array}{c} 
            NE_1: \left( (\tau_l, \gamma_h), (\tau_l, \gamma_h) \right), \\ 
            NE_2:\left( (\tau_h, \gamma_l), (\tau_h, \gamma_l) \right), \\ 
            NE_3:((0,c,1-c,0), (0,c,1-c,0))
        \end{array}\) &  
        \begin{tabular}{c}
            Depends on initialization: \\ 
            Converges to pure $NE_1$ 
            (Fig.~\ref{fig:dyna-pm-1}) or 
            $NE_2$ (Fig.~\ref{fig:dyna-pm-2}), \\ 
            % or stays at mixed $NE_3$ 
            % if initialized there 
            % (Fig.~\ref{fig:dyna-pm-3})
        \end{tabular} \\ \hline
    \end{tabular}
    }
    \caption{Nash equilibria and convergence of Algorithm \ref{alg:Hedge} and \ref{alg:Hedge:stoc} across different signs for $\epsilon_1$ and $\epsilon_2$.}
    \label{tab:NE_dyna_summary}
\end{table}

In Figures \ref{fig:dyna-pp}--\ref{fig:dyna-mp-sym}, the left panels depict the dynamics when the utility matrix is known (Algorithm \ref{alg:Hedge}), while the right panels correspond to the stochastic setting, where the utility matrix is estimated from samples (Algorithm \ref{alg:Hedge:stoc}). Note that our theoretical results in Section~\ref{sec:stoch-hedge-convergence} require the number of samples, \( n_{\text{samples}} \), used in each round of Algorithm \ref{alg:Hedge:stoc} to be sufficiently large for the dynamics to converge to the Nash equilibrium in a manner similar to the full-information dynamics. However, and perhaps surprisingly, our experiments \emph{consistently} show that setting \( n_{\text{samples}} = 1 \) is practically sufficient to achieve convergence behavior similar to that of the idealized full-information case.

We start with the `-\,-' case in Figure~\ref{fig:dyna-mm} (corresponding to Theorem 4 Part I), and observe convergence to the unique pure symmetric NE $((\tau_h, \gamma_l), (\tau_h, \gamma_l))$.

\begin{figure}[H]
    \centering
    \begin{subfigure}{0.49\linewidth}
        \centering
        \includegraphics[width=\linewidth]{Figures/2gamma/casemm/dynamics/Known-MatrixBank1.pdf}
        % \caption{Bank1: Known utility matrix}
    \end{subfigure}
    \begin{subfigure}{0.49\linewidth}
        \centering
        \includegraphics[width=\linewidth]{Figures/2gamma/casemm/dynamics/Fresh-EstimateBank1.pdf}
        % \caption{Bank1: Estimated utility matrix}
    \end{subfigure}

    \begin{subfigure}{0.49\linewidth}
        \centering
        \includegraphics[width=\linewidth]{Figures/2gamma/casemm/dynamics/Known-MatrixBank2.pdf}
        % \caption{Bank2: Known utility matrix}
    \end{subfigure}
    \begin{subfigure}{0.49\linewidth}
        \centering
        \includegraphics[width=\linewidth]{Figures/2gamma/casemm/dynamics/Fresh-EstimateBank2.pdf}
        % \caption{Bank2: Estimated utility matrix}
    \end{subfigure}
    \caption{\textbf{Case - -} Exponential weights dynamics for Bank1 (top row) and Bank2 (bottom row), converging to the unique pure NE $((\tau_h, \gamma_l),(\tau_h, \gamma_l))$. Left: known utility matrices. Right: estimated utilities using a single sample per round. $y \sim$ truncated Gaussian ($\mu=0.1, \sigma=0.3$), $\gamma_l = 0.4$, $\gamma_h = 0.8$.  Initial weights: Bank1 $(0.1, 0.5, 0.3, 0.1)$, Bank2 $(0.1, 0.3, 0.5, 0.1)$ \label{fig:dyna-mm}}
\end{figure}

We start with the `+\,+' case in Figure~\ref{fig:dyna-pp} (corresponding to Theorem 4 Part II), and see convergence to the unique pure symmetric NE $((\tau_l, \gamma_h), (\tau_l, \gamma_h))$

\begin{figure}[H]
    \centering
    \begin{subfigure}{0.49\linewidth}
        \centering
        \includegraphics[width=\linewidth]{Figures/2gamma/casepp/dynamics/Known-MatrixBank1.pdf}
        % \caption{Bank1: Known utility matrix}
    \end{subfigure}
    \begin{subfigure}{0.49\linewidth}
        \centering
        \includegraphics[width=\linewidth]{Figures/2gamma/casepp/dynamics/Fresh-EstimateBank1.pdf}
        % \caption{Bank1: Estimated utility matrix}
    \end{subfigure}

    \begin{subfigure}{0.49\linewidth}
        \centering
        \includegraphics[width=\linewidth]{Figures/2gamma/casepp/dynamics/Known-MatrixBank2.pdf}
        % \caption{Bank2: Known utility matrix}
    \end{subfigure}
    \begin{subfigure}{0.49\linewidth}
        \centering
        \includegraphics[width=\linewidth]{Figures/2gamma/casepp/dynamics/Fresh-EstimateBank2.pdf}
        % \caption{Bank2: Estimated utility matrix}
    \end{subfigure}
    \caption{\textbf{Case +\,+} Exponential weights dynamics for Bank1 (top row) and Bank2 (bottom row), converging to the unique pure NE $((\tau_l, \gamma_h),(\tau_l, \gamma_h))$. Left: known utility matrices. Right: estimated utilities using a single sample per round. $y \sim$ truncated Gaussian ($\mu=0.3, \sigma=0.1$), $\gamma_l = 0.4$, $\gamma_h = 0.8$. Initial weights: Bank1 $(0.1, 0.5, 0.3, 0.1)$, Bank2 $(0.1, 0.3, 0.5, 0.1)$. \label{fig:dyna-pp}}
\end{figure}


We now present two representative figures for the `\(-\,+\)' case. Figure~\ref{fig:dyna-mp-ass} illustrates convergence to the pure asymmetric NE $((\tau_l, \gamma_h), (\tau_h, \gamma_l))$ (corresponding to Theorem 4 Part III), while Figure~\ref{fig:dyna-mp-sym} shows convergence to the mixed symmetric NE \( ((0,c',1-c',0), (0,c',1-c',0)) \) (not covered by Theorem 4).
\begin{figure}[H]
    \centering
    \begin{subfigure}{0.49\linewidth}
        \centering
        \includegraphics[width=\linewidth]{Figures/2gamma/casemp/dynamics/asymmetric/Known-MatrixBank1.pdf}
        % \caption{Bank1: Known utility matrix}
    \end{subfigure}
    \begin{subfigure}{0.49\linewidth}
        \centering
        \includegraphics[width=\linewidth]{Figures/2gamma/casemp/dynamics/asymmetric/Fresh-EstimateBank1.pdf}
        % \caption{Bank1: Estimated utility matrix}
    \end{subfigure}

    \begin{subfigure}{0.49\linewidth}
        \centering
        \includegraphics[width=\linewidth]{Figures/2gamma/casemp/dynamics/asymmetric/Known-MatrixBank2.pdf}
        % \caption{Bank2: Known utility matrix}
    \end{subfigure}
    \begin{subfigure}{0.49\linewidth}
        \centering
        \includegraphics[width=\linewidth]{Figures/2gamma/casemp/dynamics/asymmetric/Fresh-EstimateBank2.pdf}
        % \caption{Bank2: Estimated utility matrix}
    \end{subfigure}
    \caption{\textbf{Case -\,+ (a)} Exponential weights dynamics for Bank1 (top row) and Bank2 (bottom row) converging to the pure NE $((\tau_l, \gamma_h), (\tau_h, \gamma_l))$, one of the three NE from theory. Left: known utility matrices. Right: estimated utilities using a single sample per round. Parameters: $\epsilon_1<0, \epsilon_2>0$, $y \sim$ piecewise uniform distribution, $\gamma_l = 0.6$, $\gamma_h = 0.7$. Initial weights: Bank1 $(0.1, 0.5, 0.3, 0.1)$, Bank2 $(0.1, 0.3, 0.5, 0.1)$. \label{fig:dyna-mp-ass}}
\end{figure}


\begin{figure}[H]
    \centering
    \begin{subfigure}{0.49\linewidth}
        \centering
        \includegraphics[width=\linewidth]{Figures/2gamma/casemp/dynamics/symmetric/Known-MatrixBank1.pdf}
        % \caption{Bank1: Known utility matrix}
    \end{subfigure}
    \begin{subfigure}{0.49\linewidth}
        \centering
        \includegraphics[width=\linewidth]{Figures/2gamma/casemp/dynamics/symmetric/Fresh-EstimateBank1.pdf}
        % \caption{Bank1: Estimated utility matrix}
    \end{subfigure}

    \begin{subfigure}{0.49\linewidth}
        \centering
        \includegraphics[width=\linewidth]{Figures/2gamma/casemp/dynamics/symmetric/Known-MatrixBank2.pdf}
        % \caption{Bank2: Known utility matrix}
    \end{subfigure}
    \begin{subfigure}{0.49\linewidth}
        \centering
        \includegraphics[width=\linewidth]{Figures/2gamma/casemp/dynamics/symmetric/Fresh-EstimateBank2.pdf}
        % \caption{Bank2: Estimated utility matrix}
    \end{subfigure}
    \caption{\textbf{Case -\,+ (b)} Exponential weights dynamics for Bank1 (top row) and Bank2 (bottom row) converging to the mixed NE $((0,c,1-c,0), (0,c,1-c,0))$, one of the three NE from theory. Left: known utility matrices. Right: estimated utilities using a single sample per round. Parameters: $\epsilon_1<0, \epsilon_2>0$, $y \sim$ piecewise uniform distribution, $\gamma_l = 0.6$, $\gamma_h = 0.7$. Both banks start with weights $(0.25,0.25,0.25,0.25)$.\label{fig:dyna-mp-sym}}
\end{figure}


Finally, we present three figures for the `\(+\,-\)' case. Figure \ref{fig:dyna-pm-1} illustrates convergence to the pure symmetric NE $((\tau_l, \gamma_h), (\tau_l, \gamma_h))$, while Figure \ref{fig:dyna-pm-2} shows convergence to the pure symmetric NE $((\tau_h, \gamma_l), (\tau_h, \gamma_l))$.
Both of these cases are covered by Theorem 4 Part IV.

% In Figure \ref{fig:dyna-pm-3}, we depict the dynamics when initialized at the mixed \( NE \). \kri{For this instance I couldnt find dynamics converging to mixed NE unles initialized there}

\begin{figure}[H]
    \centering
    \begin{subfigure}{0.49\linewidth}
        \centering
        \includegraphics[width=\linewidth]{Figures/2gamma/casepm/dynamics/NE1/Known-MatrixBank1.pdf}
        % \caption{Bank1: Known utility matrix}
    \end{subfigure}
    \begin{subfigure}{0.49\linewidth}
        \centering
        \includegraphics[width=\linewidth]{Figures/2gamma/casepm/dynamics/NE1/Fresh-EstimateBank1.pdf}
        % \caption{Bank1: Estimated utility matrix}
    \end{subfigure}

    \begin{subfigure}{0.49\linewidth}
        \centering
        \includegraphics[width=\linewidth]{Figures/2gamma/casepm/dynamics/NE1/Known-MatrixBank2.pdf}
        % \caption{Bank2: Known utility matrix}
    \end{subfigure}
    \begin{subfigure}{0.49\linewidth}
        \centering
        \includegraphics[width=\linewidth]{Figures/2gamma/casepm/dynamics/NE1/Fresh-EstimateBank2.pdf}
        % \caption{Bank2: Estimated utility matrix}
    \end{subfigure}
    \caption{\textbf{Case +\,- (a)} Exponential weights dynamics for Bank1 (top row) and Bank2 (bottom row), converging to the pure NE $((\tau_l, \gamma_h),(\tau_l, \gamma_h)$. Left: known utility matrices. Right: estimated utilities using a single sample per round. $y \sim$ truncated Gaussian ($\mu=0.1, \sigma=0.2$), $\gamma_l = 0.4$, $\gamma_h = 0.8$.  Initial weights: Bank1 $(0, 0.9, 0.1, 0)$, Bank2 $(0, 0.8, 0.2, 0)$  \label{fig:dyna-pm-1}}
\end{figure}

\begin{figure}[H]
    \centering
    \begin{subfigure}{0.49\linewidth}
        \centering
        \includegraphics[width=\linewidth]{Figures/2gamma/casepm/dynamics/NE2/Known-MatrixBank1.pdf}
        % \caption{Bank1: Known utility matrix}
    \end{subfigure}
    \begin{subfigure}{0.49\linewidth}
        \centering
        \includegraphics[width=\linewidth]{Figures/2gamma/casepm/dynamics/NE2/Fresh-EstimateBank1.pdf}
        % \caption{Bank1: Estimated utility matrix}
    \end{subfigure}

    \begin{subfigure}{0.49\linewidth}
        \centering
        \includegraphics[width=\linewidth]{Figures/2gamma/casepm/dynamics/NE2/Known-MatrixBank2.pdf}
        % \caption{Bank2: Known utility matrix}
    \end{subfigure}
    \begin{subfigure}{0.49\linewidth}
        \centering
        \includegraphics[width=\linewidth]{Figures/2gamma/casepm/dynamics/NE2/Fresh-EstimateBank2.pdf}
        % \caption{Bank2: Estimated utility matrix}
    \end{subfigure}
    \caption{\textbf{Case +\,- (b)}  Exponential weights dynamics for Bank1 (top row) and Bank2 (bottom row), converging to the pure NE $((\tau_h, \gamma_l),(\tau_h, \gamma_l))$. Left: known utility matrices. Right: estimated utilities using a single sample per round. $y \sim$ truncated Gaussian ($\mu=0.1, \sigma=0.2$), $\gamma_l = 0.4$, $\gamma_h = 0.8$. Initial weights: Bank1 $(0.1, 0.5, 0.3, 0.1)$, Bank2 $(0.1, 0.3, 0.5, 0.1)$ \label{fig:dyna-pm-2}}
\end{figure}
