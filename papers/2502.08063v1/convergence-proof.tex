


\section{Proofs for Section~\ref{sec:learntoconvege}: Convergence of Dynamics}



\subsection{Proof of Lemma \ref{lem:1 and 4 p}}
\label{proof:Lemma: converofp_1p_4}
We have for $t>1$,
\begin{equation*}
\begin{split}
    \frac{p^{(t+1)}_{1,1}}{p^{(t+1)}_{1,3}}  = {} & \frac{p^{(t)}_{1,1}\exp\left(\eta \sum_{j=1}^4 p_{2,j}^{(t)}u_1\left(\theta^{(1)},\theta^{(j)}\right) \right)}{p^{(t)}_{1,3}\exp\left(\eta \sum_{j=1}^4 p_{2,j}^{(t)}u_1\left(\theta^{(3)},\theta^{(j)}\right) \right)} = \frac{p_{1,1}^{(t)}}{p_{1,3}^{(t)}}\exp\left(\eta \sum_{j=1}^4p_{2,j}^{(t)}\left(u_1\left(\theta^{(1)},\theta^{(j)}\right)-u_1\left(\theta^{(3)},\theta^{(j)}\right)\right)\right) \\
\leq {} &  \frac{p_{1,1}^{(t)}}{p_{1,3}^{(t)}}\exp\left(-\eta\xi_1\right),
\end{split}
\end{equation*}
where recall that we defined $\xi_1 := \min\{\frac{1}{2}\left|f(\gamma_{\ell},\tau_{\ell},\tau_{h})\right|,\frac{1}{2}|f(\gamma_{h},\tau_{\ell},\tau_{h})|, |\epsilon_1|, |\epsilon_2|\}>0$ as shorthand.
Therefore, we have 
\[
{p^{(t+1)}_{1,1}}\leq p_{1,3}^{(t+1)} \frac{p^{(1)}_{1,1}}{p^{(1)}_{1,3}}\left[\exp\left(-\eta\xi_1\right)\right]^{t}\leq  \frac{p^{(1)}_{1,1}}{p^{(1)}_{1,3}}\left[\exp\left(-\eta\xi_1\right)\right]^{t}\leq c_{ini}\left[\exp\left(-\eta\xi_1\right)\right]^{t},
\]
where the last inequality plugs in the definition of $c_{ini}=\max_{i\in[2]}\left\{\frac{\max_{j\in[4]}p^{(1)}_{i,j}}{\min_{i\in[4]}p^{(1)}_{i,j}}\right\}$.
Similarly, we also have 

\begin{equation*}
\begin{split}
    \frac{p^{(t+1)}_{2,1}}{p^{(t+1)}_{2,3}}  = {} & \frac{p^{(t)}_{2,1}\exp\left(\eta \sum_{j=1}^4 p_{1,j}^{(t)}u_2\left(\theta^{(j)},\theta^{(1)}\right) \right)}{p^{(t)}_{2,3}\exp\left(\eta \sum_{j=1}^4 p_{1,j}^{(t)}u_2\left(\theta^{(j)},\theta^{(3)}\right) \right)} = \frac{p_{2,1}^{(t)}}{p_{2,3}^{(t)}}\exp\left(\eta \sum_{j=1}^4p_{1,j}^{(t)}\left(u_1\left(\theta^{(1)},\theta^{(j)}\right)-u_1\left(\theta^{(3)},\theta^{(j)}\right)\right)\right) \\
\leq {} &  \frac{p_{2,1}^{(t)}}{p_{2,3}^{(t)}}\exp\left(-\eta\xi_1\right),
\end{split}
\end{equation*}
where the second equality is based on the fact the game is symmetric. It implies that 
\[
{p^{(t+1)}_{2,1}}\leq c_{ini}\left[\exp\left(-\eta\xi_1\right)\right]^{t}.
\]
Next, we know that $u_1(\theta^{(2)},\theta^{(1)})=u_1(\theta^{(4)},\theta^{(1)})=0$, but for all $j \neq 1$, we have $u_1\left(\theta^{(2)},\theta^{(j)}\right) - u_1\left(\theta^{(4)},\theta^{(j)}\right) \geq \xi_1$. Therefore, we have for all $t>1$,
\begin{equation}\label{eq:recursive-p14}
   \begin{split}
\frac{p^{(t+1)}_{1,4}}{p^{(t+1)}_{1,2}} =      \frac{p^{(t)}_{1,4}}{p^{(t)}_{1,2}} \exp\left(\eta \sum_{j=2}^4 p^{(t)}_{2,j}\left(u_1\left(\theta^{(4)},\theta^{(j)}\right)-u_1\left(\theta^{(2)},\theta^{(j)}\right)\right) \right)\leq \frac{p^{(t)}_{1,4}}{p^{(t)}_{1,2}} \exp\left(-\eta \xi_1\sum_{j=2}^4 p^{(t)}_{2,j} \right).
   \end{split} 
\end{equation}
Note that it implies that the ratio between $p_{1,4}^{(t)}$ and $p_{1,2}^{(t)}$ is non-increasing. Moreover, we have just shown that
\[
p_{2,1}^{(t)}\leq c_{ini}\exp\left(-\eta\xi_1(t-1)\right)
\]
For $t\geq t'=\max\left\{\left\lceil\frac{\log 2c_{ini}}{\epsilon\eta}+1\right\rceil,2\right\}$, we have $p_{2,1}^{(t)}\leq \frac{1}{2}$, which implies that 
\[
\sum_{j=2}^4p_{2,j}^{(t)}= 1- p_{1,2}^{(t)}\geq\frac{1}{2}.
\] 
Thus, 
\begin{equation*}
    \begin{split}
 \frac{p^{(t+1)}_{1,4}}{p^{(t+1)}_{1,2}}\leq {} &  \frac{p^{(t)}_{1,4}}{p^{(t)}_{1,2}} \exp\left(-\eta \xi_1\sum_{j=2}^4 p^{(t)}_{2,j} \right)\leq \frac{p^{(t'-1)}_{1,4}}{p^{(t'-1)}_{1,2}} \left(\exp\left(-\frac{1}{2}\eta \xi_1 \right)\right)^{t-t'+1}\exp\left(-\eta\xi_1\sum_{j=2}^4p_{2,j}^{(t'-1)}\right)\\
\leq {} &\frac{p^{(1)}_{1,4}}{p^{(1)}_{1,2}} \left(\exp\left(-\frac{1}{2}\eta \xi_1 \right)\right)^{t-t'+1}\leq c_{ini}\exp\left(-\frac{1}{2}\eta \xi_1(t-t'+1) \right).       
    \end{split}
\end{equation*}
where the penultimate inequality uses Equation~\eqref{eq:recursive-p14} recursively for $t = t' - 1, \ldots, 1$. We complete the proof by setting 
$$c_{ini}\exp\left(-\frac{1}{2}\eta \xi_1(t-t'+1) \right)\leq\omega. $$

%This completes the proof of the lemma.

\subsection{Proof of Theorem 4}
\label{proof:Theoerm:conveccc}
In this section, we present the proof of Theorem 4, considering four cases based on the different possible signs of $\epsilon_1$ and $\epsilon_2$.\\
Recall the definitions $\epsilon_1 = \frac{1}{2} f(\gamma_h, \tau_{\ell},1) - f(\gamma_{\ell},\tau_h, 1)$ and $\epsilon_2 = f(\gamma_h, \tau_{\ell}, \tau_h) - \frac{1}{2} f(\gamma_{\ell},\tau_h, 1)$.
Also recall the shorthand definitions $\xi_1=\min\{\frac{1}{2}f(\gamma_{\ell},\tau_{h},1), \frac{1}{2}\left|f(\gamma_{\ell},\tau_{\ell},\tau_{h})\right|,\frac{1}{2}|f(\gamma_{h},\tau_{\ell},\tau_{h})|, |\epsilon_1|, |\epsilon_2|\}>0$ and $c_{ini} := \max_{i\in[2]}\left\{\frac{\max_{j\in[4]}p^{(1)}_{i,j}}{\min_{i\in[4]}p^{(1)}_{i,j}}\right\}$.


\noindent \textbf{Case I: $\epsilon_1<0,$ and $\epsilon_2<0$.} We have 
\begin{equation}
    \begin{split}
    \label{eqn:throrem 4:case1}
     \frac{p^{(t+1)}_{1,2}}{p^{(t+1)}_{1,3}}  = {} &  \frac{p^{(t)}_{1,2}}{p^{(t)}_{1,3}}\exp\left(\eta \sum_{j=1}^4 p^{(t)}_{2,j}\left(u_1\left(\theta^{(2)}, \theta^{(j)}\right)-u_1\left(\theta^{(3)}, \theta^{(j)}\right)\right)\right) \\ 
     \leq  {} &  \frac{p^{(t)}_{1,2}}{p^{(t)}_{1,3}}\exp\left( -\eta \sum_{j=1}^3p^{(t)}_{2,j}\xi_1 + p^{(t)}_{2,4}\left(f(\gamma_{h},\tau_{\ell},\tau_{h})+\frac{1}{2}f(\gamma_{h},\tau_{h},1)-f(\gamma_{\ell},\tau_h,1)\right)  \right)\\
     \leq {} &  \frac{p^{(t)}_{1,2}}{p^{(t)}_{1,3}}\exp\left( -\eta \sum_{j=1}^3p^{(t)}_{2,j}\xi_1 + 3\eta p^{(t)}_{2,4}  \right)
     =  \frac{p^{(t)}_{1,2}}{p^{(t)}_{1,3}}\exp\left( -\eta \sum_{j=1}^4p^{(t)}_{2,j}\xi_1 + \eta(\xi_1+3) p^{(t)}_{2,4}  \right)\\
     = {} &  \frac{p^{(t)}_{1,2}}{p^{(t)}_{1,3}}\exp\left( -\eta\xi_1 + \eta(\xi_1+3) p^{(t)}_{2,4}  \right).\\
    \end{split}
\end{equation}
For the first inequality, it is because:
\begin{equation*}
    \begin{split}
u_1(\theta^{(2)},\theta^{(1)})-u_1(\theta^{(3)},\theta^{(1)})={} & 0-\frac{1}{2}f(\gamma_{\ell},\tau_{h},1)\leq -\xi_1,\\
u_1(\theta^{(2)},\theta^{(2)})-u_1(\theta^{(3)},\theta^{(2)})={} & \frac{1}{2}f(\gamma_{h},\tau_{\ell},1)-f(\gamma_{\ell},\tau_{h},1)=\epsilon_1\leq -\xi_1,\\
u_1(\theta^{(2)},\theta^{(3)})-u_1(\theta^{(3)},\theta^{(3)})={} &f(\gamma_{h},\tau_{\ell},\tau_{h})- \frac{1}{2}f(\gamma_{\ell},\tau_{h},1)=\epsilon_2\leq -\xi_1.
    \end{split}
\end{equation*}
The second inequality of \eqref{eqn:throrem 4:case1} is because the $f$ function is upper bounded by 2 based on Lemma \ref{lem:property of f}, and $f(\gamma_{\ell},\tau_{h},1)$ is positive. Denote as shorthand the constant $t''=t'+1+\left\lceil\frac{2}{\eta\xi_1}\log\frac{2(\xi_1+3)c_{ini}}{\xi_1}\right\rceil$.
Then, from Lemma \ref{lem:1 and 4 p}, we have for $t\geq t''$, 
$$\eta(\xi_1+3)p_{2,4}^{(t)}\leq \frac{\eta\xi_1}{2}.$$
Thus, we have  
\begin{equation*}
    \begin{split}
   \frac{p^{(t+1)}_{1,2}}{p^{(t+1)}_{1,3}} 
   \leq {} & \frac{p^{(t)}_{1,2}}{p^{(t)}_{1,3}} \exp\left(-\frac{\eta\xi_1}{2}\right) \ldots \leq \frac{p^{(t''-1)}_{1,2}}{p^{(t''-1)}_{1,3}} \left[\exp\left(-\frac{\eta\xi_1}{2}\right)\right]^{t-t''+1}\\
   \leq {} & c_{\text{ini}}\exp\left(3\eta t''\right)\exp\left(-\frac{\eta\xi_1}{2}\left(t-t''+1\right)\right),     
    \end{split}
\end{equation*}
where the last inequality is obtained by applying the following inequality recursively for $t=t''-1,\dots,1$:
$$  \frac{p^{(t)}_{1,2}}{p^{(t)}_{1,3}}  \leq  \frac{p^{(t-1)}_{1,2}}{p^{(t-1)}_{1,3}}\exp\left( -\eta\xi_1 + \eta(\xi_1+3) p^{(t-1)}_{2,4}  \right)\leq \frac{p^{(t-1)}_{1,2}}{p^{(t-1)}_{1,3}} \exp(3\eta),$$
which is obtained based on \eqref{eqn:throrem 4:case1}.

Note that $t''$ is a constant. We have for $t\geq t''$,
$$p^{(t+1)}_{1,2}\leq c_{\text{ini}}\exp\left(3\eta t''\right)\exp\left(-\frac{\eta\xi_1}{2}\left(t-t''+1\right)\right), $$ 
and thus 
$$ p^{(t+1)}_{1,3}\geq \max\left\{ 1 -3c_{\text{ini}}\exp\left(3\eta t''\right)\exp\left(-\frac{\eta\xi_1}{2}\left(t-t''+1\right)\right),0\right\} $$
The proof is completed by setting 
$$c_{\text{ini}}\exp\left(3\eta t''\right)\exp\left(-\frac{\eta\xi_1}{2}\left(t-t''+1\right)\right)\leq \omega.$$
It shows that $p_{1,2}^{(t)}$ converges to 0 at an exponential rate after $t''$ iterations, which is a constant factor. Combining with Lemma \ref{lem:1 and 4 p}, we can draw the conclusion that $p_{1,3}^{(t)}$ converges to 1 exponentially fast after a constant number of iterations.
An identical argument reaches the same conclusion for Bank 2.
This implies that the algorithm eventually converges to the symmetric pure NE $((\tau_h, \gamma_{\ell}),(\tau_h, \gamma_{\ell}))$.
\\


\noindent \textbf{Case II: $\epsilon_1>0,$ and $\epsilon_2>0$.} 
The proof for this case proceeds similarly to Case I.
We have:
\begin{equation}
    \begin{split}
    \label{eqn:case2:1113112}
\frac{p^{(t+1)}_{1,3}}{p^{(t+1)}_{1,2}} = {} & \frac{p^{(t)}_{1,3}}{p^{(t)}_{1,2}}\exp\left( \eta \sum_{j=1}^4p^{(t)}_{2,j}\left(u_1\left(\theta^{(3)}, \theta^{(j)}\right)-u_1\left(\theta^{(2)}, \theta^{(j)}\right)\right) \right) \\
\leq {} & \frac{p^{(t)}_{1,3}}{p^{(t)}_{1,2}}\exp\left( -\eta \sum_{j=2}^4p^{(t)}_{2,j}\xi_1   + \eta p^{(t)}_{2,1} \right)
=  \frac{p^{(t)}_{1,3}}{p^{(t)}_{1,2}}\exp\left( -\eta \sum_{j=1}^4p^{(t)}_{2,j}\xi_1   + \eta(1+\xi_1)  p^{(t)}_{2,1}\right).
    \end{split}
\end{equation}
where the first inequality is because 
\begin{equation*}
   \begin{split}
    u_1(\theta^{(3)},\theta^{(2)}) - u_1(\theta^{(2)},\theta^{(2)}) = {} &-\epsilon_1\leq -\xi_1, \\
      u_1(\theta^{(3)},\theta^{(3)}) - u_1(\theta^{(2)},\theta^{(3)}) ={} & -\epsilon_2\leq -\xi_1, \\
      u_1(\theta^{(3)},\theta^{(4)}) - u_1(\theta^{(2)},\theta^{(4)}) ={} & f(\gamma_{\ell},\tau_{h},1) - f(\gamma_{h},\tau_{\ell},\tau_{h})-\frac{1}{2} f(\gamma_{h},\tau_{h},1)\\
      \leq {} & f(\gamma_{\ell},\tau_{h},1) - \frac{1}{2}f(\gamma_{h},\tau_{\ell},\tau_{h})-\frac{1}{2} f(\gamma_{h},\tau_{h},1)\\
      = {} & f(\gamma_{\ell},\tau_{h},1) - \frac{1}{2}f(\gamma_{h},\tau_{\ell},1)=-\epsilon_1\leq -\xi_1
   \end{split} 
\end{equation*}


Denote as shorthand the constant $t'''=\left\lceil\frac{1}{\eta\xi_1}\log\frac{2(1+\xi_1)c_{\text{ini}}}{\xi_1}\right\rceil+1$.
Then, again using Lemma~\ref{lem:1 and 4 p}, we have for all $t \geq t'''$:
$$ \eta(1+\xi_1)p_{2,1}^{(t)}\leq \frac{\eta\xi_1}{2}, $$
In this case, i.e., for $t\geq t'''$, we have 
\begin{align*}
    \frac{p^{(t+1)}_{1,3}}{p^{(t+1)}_{1,2}} \leq \frac{p^{(t)}_{1,3}}{p^{(t)}_{1,2}} \exp\left(-\frac{\eta\xi_1}{2}\right) \ldots &\leq \frac{p^{(t'''-1)}_{1,3}}{p^{(t'''-1)}_{1,2}} \exp\left(-\frac{\eta\xi_1}{2}(t-t'''+1)\right)
    \\& \leq c_{\text{ini}}\exp(t'''\eta)\exp\left(-\frac{\eta\xi_1}{2}(t-t'''+1)\right)\leq \omega. 
\end{align*}
%\vidya{can you replace all appearances of $\left[\exp\left(-\frac{\eta \xi_1}{2}\right)\right]^t$ with $\exp\left(- \frac{\eta \xi_1 t}{2}\right)$? The latter is easier to read/format.}
It shows that $p_{1,3}^{(t+1)}$ converges to 0 exponentially fast after $t'''$, which is a constant factor. Combining with Lemma \ref{lem:1 and 4 p}, we can draw the conclusion that $p_{1,2}^{(t)}$ converges to 1 exponentially fast after a constant number of iterations. 
An identical argument reaches the same conclusion for Bank 2.
This implies that the algorithm eventually converges to the symmetric pure NE $((\tau_{\ell},\gamma_h),(\tau_{\ell},\gamma_h))$.\\

\noindent \textbf{Case III: $\epsilon_1<0,$ and $\epsilon_2>0$.}
Note that according to Lemma \ref{lem:1 and 4 p} and the analysis above, we know that $p^{(t)}_{1,1}$ and $p^{(t)}_{1,4}$ converge to 0 exponentially fast under any conditions, and has little influence on the final results. For now, for the simplicity of the proof, we assume $p^{(t)}_{1,1}=p^{(t)}_{1,4}=0$ for $t\geq1$. This can be understood as starting after an initial phase where $p^{(t)}_{1,1}$ and $p^{(t)}_{1,4}$ decrease near to 0. 
%\vidya{don't think we'll have time to do this for EC but it would be good to formalize this a bit more before the arxiv deadline. in particular we expect the proofs to go through if $p^{(t)}_{1,1}, p^{(t)}_{1,4} \leq \delta$ for some small enough $\delta$?}
\\


%We find that the convergence of the algorithm under this case is related to the algorithm initialization. 

%where two cases, where in the first case  $p^{(1)}_{2,2}\epsilon_1+p^{(1)}_{2,3}\epsilon_2$ and $p^{(1)}_{1,2}\epsilon_1+p^{(1)}_{1,3}\epsilon_2$ have different signs (we call it asymmetric initialization), and in the second case the two have the same sign (we call it symmetric initialization). 
For this case we need to assume an \emph{asymmetric} initialization, where $p^{(1)}_{2,2}\epsilon_1+p^{(1)}_{2,3}\epsilon_2$ and $p^{(1)}_{1,2}\epsilon_1+p^{(1)}_{1,3}\epsilon_2$ have different signs. Without loss of generality, first assume $p^{(1)}_{2,2}\epsilon_1+p^{(1)}_{2,3}\epsilon_2<-\Delta$, and $p^{(1)}_{1,2}\epsilon_1+p^{(1)}_{1,3}\epsilon_2>\Delta$, for some constant $\Delta>0$. 
We have for $t=1$,  
\begin{equation*}
    \begin{split}
\frac{p^{(2)}_{1,2}}{p^{(2)}_{1,3}} = \frac{p^{(1)}_{1,2}}{p^{(1)}_{1,3}}\exp\left(\eta\left(p^{(1)}_{2,2}\epsilon_1 + p^{(1)}_{2,3}\epsilon_2\right)\right) \leq   \frac{p^{(1)}_{1,2}}{p^{(1)}_{1,3}}\exp(-\eta\Delta).,
    \end{split}
\end{equation*}
where the last inequality plugs in our initialization condition.
Note that since $p^{(t)}_{1,2}+p^{(t)}_{1,3}=1$ for all $t$, the deceasing of the ratio means that $p^{(2)}_{1,2}< p^{(1)}_{1,2}$, and $p^{(2)}_{1,3}> p^{(1)}_{1,3}$. On the other hand, 
\begin{equation*}
    \begin{split}
\frac{p^{(2)}_{2,3}}{p^{(2)}_{2,2}} = \frac{p^{(1)}_{2,3}}{p^{(1)}_{2,2}}\exp\left(-\eta\left(p^{(1)}_{1,2}\epsilon_1 + p^{(1)}_{1,3}\epsilon_2\right)\right) \leq   \frac{p^{(1)}_{2,3}}{p^{(1)}_{2,2}}\exp(-\eta \Delta). 
    \end{split}
\end{equation*}
Similar to our reasoning for Bank 1, this implies that $p^{(2)}_{2,3}<p^{(1)}_{2,3}$, and $p^{(2)}_{2,2}>p^{(1)}_{2,2}$. 
% Consequently, noting that $\epsilon_1 < 0$ and $\epsilon_2 > 0$ we have $p_{1,2}^{(2)}\epsilon_1 + p_{1,3}^{(2)} \epsilon_2 > p_{1,2}^{(1)} \epsilon_1 + p_{1,3}^{(1)} \epsilon_2$ and $p_{2,2}^{(2)} \epsilon_1 + p_{2,3}^{(2)} \epsilon_2 < p_{2,2}^{(1)} \epsilon_1 + p_{2,3}^{(1)} \epsilon_2$.
We now show that $p^{(t+1)}_{1,2}< p^{(1)}_{1,2}$,  $p^{(t+1)}_{1,3}> p^{(1)}_{1,3}$, $p^{(t+1)}_{2,3}<p^{(1)}_{2,3}$, and $p^{(t+1)}_{2,2}>p^{(1)}_{2,2}$ through an inductive argument.
Assume at $t\geq 2$, $p^{(t)}_{1,2}< p^{(1)}_{1,2}$,  $p^{(t)}_{1,3}> p^{(1)}_{1,3}$, $p^{(t)}_{2,3}<p^{(1)}_{2,3}$, and $p^{(t)}_{2,2}>p^{(1)}_{2,2}$.
Then, noting that $\epsilon_1 < 0$ and $\epsilon_2 > 0$, we have:
\[
  \frac{p^{(t+1)}_{1,2}}{p^{(t+1)}_{1,3}} = \frac{p^{(t)}_{1,2}}{p^{(t)}_{1,3}}\exp\left(\eta\left(p^{(t)}_{2,2}\epsilon_1 + p^{(t)}_{2,3}\epsilon_2\right)\right) \leq \frac{p^{(t)}_{1,2}}{p^{(t)}_{1,3}}\exp\left(\eta\left(p^{(1)}_{2,2}\epsilon_1 + p^{(1)}_{2,3}\epsilon_2\right)\right)\leq \frac{p^{(t)}_{1,2}}{p^{(t)}_{1,3}}\exp(-\eta \Delta),
\]
and 
\[
  \frac{p^{(t+1)}_{2,3}}{p^{(t+1)}_{2,2}} = \frac{p^{(t)}_{2,3}}{p^{(t)}_{2,2}}\exp\left(-\eta\left(p^{(t)}_{1,2}\epsilon_1 + p^{(t)}_{1,3}\epsilon_2\right)\right) \leq \frac{p^{(t)}_{2,3}}{p^{(t)}_{2,2}}\exp\left(-\eta\left(p^{(1)}_{1,2}\epsilon_1 + p^{(1)}_{1,3}\epsilon_2\right)\right)\leq \frac{p^{(t)}_{2,3}}{p^{(t)}_{2,2}}\exp(-\eta \Delta),
\]
which implies that we have $p^{(t+1)}_{1,2}< p^{(1)}_{1,2}$,  $p^{(t+1)}_{1,3}> p^{(1)}_{1,3}$, $p^{(t+1)}_{2,3}<p^{(1)}_{2,3}$, and $p^{(t+1)}_{2,2}>p^{(1)}_{2,2}$, which finishes the induction. This also implies that 
\[
p^{(t+1)}_{1,2}\leq p^{(t+1)}_{1,3}c_{ini}\exp(-\eta  \Delta t), \quad \text{and} \quad p^{(t+1)}_{2,3}\leq p^{(t+1)}_{2,2}c_{ini}\exp(-\eta  \Delta t).
\]





Then we have 
\[
p^{(t+1)}_{1,2}\geq \frac{1}{1+c_{ini}\exp\left(-\eta\Delta t\right)},\ \ \text{and}\ \ p^{(t+1)}_{2,3}\geq \frac{1}{1+c_{ini}\exp\left(-\eta\Delta t\right)},
\]
which implies that the algorithm converges to the asymmetric NE $((\tau_{\ell},\gamma_{h}),(\tau_{h},\gamma_{\ell}))$. 

Finally, consider the other case, i.e., $p^{(1)}_{2,2}\epsilon_1+p^{(1)}_{2,3}\epsilon_2>\Delta$, and $p^{(1)}_{1,2}\epsilon_1+p^{(1)}_{1,3}\epsilon_2<-\Delta$, for some constant $\Delta>0$. The proof proceeds very similarly as to the previous case, but the algorithm will converge to a different NE. We have for $t=1$: 

\begin{equation*}
    \begin{split}
\frac{p^{(2)}_{2,2}}{p^{(2)}_{2,3}} = \frac{p^{(1)}_{2,2}}{p^{(1)}_{2,3}}\exp\left(\eta\left(p^{(1)}_{1,2}\epsilon_1 + p^{(1)}_{1,3}\epsilon_2\right)\right) \leq   \frac{p^{(1)}_{2,2}}{p^{(1)}_{2,3}}\exp(-\eta\Delta),  
    \end{split}
\end{equation*}
and 
\begin{equation*}
    \begin{split}
\frac{p^{(2)}_{1,3}}{p^{(2)}_{1,2}} = \frac{p^{(1)}_{1,3}}{p^{(1)}_{1,2}}\exp\left(-\eta\left(p^{(1)}_{2,2}\epsilon_1 + p^{(1)}_{2,3}\epsilon_2\right)\right) \leq   \frac{p^{(1)}_{1,3}}{p^{(1)}_{1,2}}\exp(-\eta \Delta). 
    \end{split}
\end{equation*}
Therefore, $p_{2,2}^{(2)}< p_{2,2}^{(1)}$,  $p_{1,3}^{(2)}< p_{1,3}^{(1)}$. Assume for $t\geq 2$, $p_{2,2}^{(t)}< p_{2,2}^{(1)}$,  $p_{1,3}^{(t)}< p_{1,3}^{(1)}$,  $p_{2,3}^{(t)}> p_{2,3}^{(1)}$,  $p_{1,2}^{(t)}> p_{1,2}^{(1)}$. 
Then, noting that $\epsilon_1 < 0$ and $\epsilon_1 > 0$, we have:
\[
  \frac{p^{(t+1)}_{2,2}}{p^{(t+1)}_{2,3}} = \frac{p^{(t)}_{2,2}}{p^{(t)}_{2,3}}\exp\left(\eta\left(p^{(t)}_{1,2}\epsilon_1 + p^{(t)}_{1,3}\epsilon_2\right)\right) \leq \frac{p^{(t)}_{2,2}}{p^{(t)}_{2,3}}\exp\left(\eta\left(p^{(1)}_{1,2}\epsilon_1 + p^{(1)}_{1,3}\epsilon_2\right)\right)\leq \frac{p^{(t)}_{2,2}}{p^{(t)}_{2,3}}\exp(-\eta \Delta),
\]
and 
\[
  \frac{p^{(t+1)}_{1,3}}{p^{(t+1)}_{1,2}} = \frac{p^{(t)}_{1,3}}{p^{(t)}_{1,2}}\exp\left(-\eta\left(p^{(t)}_{2,2}\epsilon_1 + p^{(t)}_{2,3}\epsilon_2\right)\right) \leq \frac{p^{(t)}_{1,3}}{p^{(t)}_{1,2}}\exp\left(-\eta\left(p^{(1)}_{2,2}\epsilon_1 + p^{(1)}_{2,3}\epsilon_2\right)\right)\leq \frac{p^{(t)}_{1,3}}{p^{(t)}_{1,2}}\exp(-\eta \Delta),
\]
which implies that $p_{2,2}^{(t+1)}< p_{2,2}^{(1)}$,  $p_{1,3}^{(t+1)}< p_{1,3}^{(1)}$,  $p_{2,3}^{(t+1)}> p_{2,3}^{(1)}$,  $p_{1,2}^{(t+1)}> p_{1,2}^{(1)}$, and finishes the induction. This shows that that algorithm converges to the asymmetric NE $((\tau_{h},\gamma_{\ell}),(\tau_{\ell},\gamma_h))$.


%\noindent \textbf{a) Symmetric initialization:} Without loss of generality, assume $p^{(1)}_{2,2}\epsilon_1+p^{(1)}_{2,3}\epsilon_2<\Delta$, and $p^{(1)}_{1,2}\epsilon_1+p^{(1)}_{1,3}\epsilon_2<\Delta$, for some constant $\Delta<0$.  We have for $t=1$,  
\noindent \textbf{Case IV: $\epsilon_1>0$, $\epsilon_2<0$.} Similar to Case III, we assume $p^{(t)}_{1,1}=p^{(t)}_{1,4}=0$ for $t\geq1$. The proof here will be in a way symmetric to Case III.   We  consider the symmetric initialization case. Without loss of generality, assume $p^{(1)}_{2,2}\epsilon_1+p^{(1)}_{2,3}\epsilon_2<-\Delta$, and $p^{(1)}_{1,2}\epsilon_1+p^{(1)}_{1,3}\epsilon_2<-\Delta$, for some constant $\Delta>0$.  We have for $t=1$, 
\begin{equation*}
    \begin{split}
\frac{p^{(2)}_{1,2}}{p^{(2)}_{1,3}} = \frac{p^{(1)}_{1,2}}{p^{(1)}_{1,3}}\exp\left(\eta\left(p^{(1)}_{2,2}\epsilon_1 + p^{(1)}_{2,3}\epsilon_2\right)\right)\leq    \frac{p^{(1)}_{1,2}}{p^{(1)}_{1,3}}\exp(-\eta\Delta). 
    \end{split}
\end{equation*}
Note that since $p^{(t)}_{1,2}+p^{(t)}_{1,3}=1$ for all $t$, the deceasing of the ratio means that $p^{(2)}_{1,2}< p^{(1)}_{1,2}$, and $p^{(2)}_{1,3}> p^{(1)}_{1,3}$. Similarly, we have
\begin{equation*}
    \begin{split}
\frac{p^{(2)}_{2,2}}{p^{(2)}_{2,3}} = \frac{p^{(1)}_{2,2}}{p^{(1)}_{2,3}}\exp\left(\eta\left(p^{(1)}_{1,2}\epsilon_1 + p^{(1)}_{1,3}\epsilon_2\right)\right) \leq   \frac{p^{(1)}_{2,2}}{p^{(1)}_{2,3}}\exp(-\eta \Delta). 
    \end{split}
\end{equation*}
Since $\Delta<0$, it also implies that $p^{(2)}_{2,2}<p^{(1)}_{2,2}$, and $p^{(2)}_{2,3}>p^{(1)}_{2,3}$. Assume at $t\geq 2$, $p^{(t)}_{1,2}< p^{(1)}_{1,2}$,  $p^{(t)}_{1,3}> p^{(1)}_{1,3}$, $p^{(t)}_{2,2}<p^{(1)}_{2,2}$, and $p^{(t)}_{2,3}>p^{(1)}_{2,3}$, then we have (note that $\epsilon_1>0, \epsilon_2<0$):
\[
  \frac{p^{(t+1)}_{1,2}}{p^{(t+1)}_{1,3}} = \frac{p^{(t)}_{1,2}}{p^{(t)}_{1,3}}\exp\left(\eta\left(p^{(t)}_{2,2}\epsilon_1 + p^{(t)}_{2,3}\epsilon_2\right)\right) \leq \frac{p^{(t)}_{1,2}}{p^{(t)}_{1,3}}\exp\left(\eta\left(p^{(1)}_{2,2}\epsilon_1 + p^{(1)}_{2,3}\epsilon_2\right)\right)\leq \frac{p^{(t)}_{1,2}}{p^{(t)}_{1,3}}\exp(-\eta \Delta),
\]
and 
\[
  \frac{p^{(t+1)}_{2,2}}{p^{(t+1)}_{2,3}} = \frac{p^{(t)}_{2,2}}{p^{(t)}_{2,3}}\exp\left(\eta\left(p^{(t)}_{1,2}\epsilon_1 + p^{(t)}_{1,3}\epsilon_2\right)\right) \leq \frac{p^{(t)}_{2,3}}{p^{(t)}_{2,2}}\exp\left(-\eta\left(p^{(1)}_{1,2}\epsilon_1 + p^{(1)}_{1,3}\epsilon_2\right)\right)\leq \frac{p^{(t)}_{2,3}}{p^{(t)}_{2,2}}\exp(-\eta \Delta),
\]
which implies that we have $p^{(t+1)}_{1,2}< p^{(1)}_{1,2}$,  $p^{(t+1)}_{1,3}> p^{(1)}_{1,3}$, $p^{(t+1)}_{2,2}<p^{(1)}_{2,2}$, and $p^{(t+1)}_{2,3}>p^{(1)}_{2,3}$, which finishes the induction. This also implies that 
\[
p^{(t+1)}_{1,2}\leq p^{(t+1)}_{1,3}c_{ini}\left[\exp(-\eta \Delta)\right]^t, \quad \text{and} \quad p^{(t+1)}_{2,2}\leq p^{(t+1)}_{2,3}c_{ini}\left[\exp(-\eta \Delta)\right]^t,
\]
i.e., the algorithm converges to the asymmetric NE $((\gamma_{\ell},\tau_{h}),(\gamma_{\ell},\tau_{h}))$. Following a similar argument, it can be shown that, if $p^{(1)}_{2,2}\epsilon_1+p^{(1)}_{2,3}\epsilon_2>\Delta$, and $p^{(1)}_{1,2}\epsilon_1+p^{(1)}_{1,3}\epsilon_2>\Delta$, for some $\Delta>0$. The algorithm converges to $((\gamma_{h},\tau_{\ell}),(\gamma_{h},\tau_{\ell}))$, i.e., the other pure NE. \\

%Then the algorithm either converges to the pure NE in finite time, or mixed NE as $t\rightarrow\infty.$




\subsection{Proof of Theorem 5}
\label{proof:Theorem 5}
Before diving into the details, we first provide some useful lemmas. Firstly, we introduce the following lemma, which shows that the estimated utility values are accurate with a high probability. 
\begin{lemma}
 \label{lem:u-estimation} 
If $n_{samples}\geq\frac{1280\log\frac{64T}{\delta}}{\xi_1^2}$, 
then with probability at least  $1-\delta$, for all round $t\in[T]$, $i\in[2]$, $\theta_1,\theta_2\in\{\theta^{(1)},\dots,\theta^{(4)} \}$, we have 
$$ \left|\widehat{u}_i(\theta_1, \theta_2, \mathcal{C}^{(t)})-{u}_i(\theta_1, \theta_2)\right|\leq \frac{\xi_1}{16}. $$
\end{lemma}
The proof is given in Appendix \ref{Proof of u-estimation}. Next, we have the following lemma, which is the counterpart of Lemma \ref{lem:1 and 4 p} in the stochastic setting. 
\begin{lemma}
\label{lem:1 and 4 p:stochastic}
Let $T$ be the time horizon, $\xi_1=\min\{\frac{1}{2}f(\gamma_{\ell},\tau_{h},1), \frac{1}{2}\left|f(\gamma_{\ell},\tau_{\ell},\tau_{h})\right|,\frac{1}{2}|f(\gamma_{h},\tau_{\ell},\tau_{h})|, |\epsilon_1|, |\epsilon_2|\}>0.$ Let $c_{ini}=\max_{i\in[2]}\left\{\frac{\max_{j\in[4]}p^{(1)}_{i,j}}{\min_{i\in[4]}p^{(1)}_{i,j}}\right\}$, and $t'_s=\max\left\{\left\lceil\frac{2\log 4c_{ini}}{\epsilon\eta}+1\right\rceil,2\right\}$ be constants. Then for Algorithm \ref{alg:Hedge:stoc}, with probability at least $1-\delta$, for any error tolerance $\omega>0$, if $n_{samples}\geq \frac{1280\log\frac{64T}{\delta}}{\xi_1^2}$, and $T\geq t'_s + \frac{4}{\eta\xi_1}\log\frac{c_{ini}}{\omega}$, we have  we have for bank $i=1,2$, $p^{(T)}_{i,1}\leq \omega$ and $p^{(T)}_{i,4}\leq \omega$. 
\end{lemma}

Lemma \ref{lem:1 and 4 p:stochastic} is proved in Appendix \ref{proof:Theoerm:stoc}. Next, we consider the four cases respectively. 

\noindent \textbf{Case I: $\epsilon_1<0,$ and $\epsilon_2<0$.} We have 
\begin{equation}
    \begin{split}
    \label{eqn:throrem 5:case1}
     \frac{p^{(t+1)}_{1,2}}{p^{(t+1)}_{1,3}}  = {} &  \frac{p^{(t)}_{1,2}}{p^{(t)}_{1,3}}\exp\left(\eta \sum_{j=1}^4 p^{(t)}_{2,j}\left(\widehat{u}_1\left(\theta^{(2)}, \theta^{(j)},\C^{(t)}\right)-\widehat{u}_1\left(\theta^{(3)}, \theta^{(j)},\C^{(t)}\right)\right)\right) \\ 
     \leq  {} &  \frac{p^{(t)}_{1,2}}{p^{(t)}_{1,3}}\exp\left(\eta \sum_{j=1}^4 p^{(t)}_{2,j}\left(u_1\left(\theta^{(2)}, \theta^{(j)}\right)-u_1\left(\theta^{(3)}, \theta^{(j)}\right)\right)+\frac{\eta\xi_1}{2}\right) \\
     \leq  {} &  \frac{p^{(t)}_{1,2}}{p^{(t)}_{1,3}}\exp\left( -\eta \sum_{j=1}^3p^{(t)}_{2,j}\xi_1 + p^{(t)}_{2,4}\left(f(\gamma_{h},\tau_{\ell},\tau_{h})+\frac{1}{2}f(\gamma_{h},\tau_{h},1)-f(\gamma_{\ell},\tau_h,1)\right)  +\frac{\eta\xi_1}{2}\right)\\
     \leq {} &  \frac{p^{(t)}_{1,2}}{p^{(t)}_{1,3}}\exp\left( -\eta \sum_{j=1}^3p^{(t)}_{2,j}\xi_1 + 3\eta p^{(t)}_{2,4} +\frac{\eta\xi_1}{2} \right)
     =  \frac{p^{(t)}_{1,2}}{p^{(t)}_{1,3}}\exp\left( -\eta \sum_{j=1}^4p^{(t)}_{2,j}\xi_1 + \eta(\xi_1+3) p^{(t)}_{2,4} +\frac{\eta\xi_1}{2} \right)\\
     = {} &  \frac{p^{(t)}_{1,2}}{p^{(t)}_{1,3}}\exp\left( -\frac{\eta\xi_1}{2} + \eta(\xi_1+3) p^{(t)}_{2,4}  \right).\\
    \end{split}
\end{equation}
Denote as shorthand the constant $t_s''=t_s'+1+\left\lceil\frac{4}{\eta\xi_1}\log\frac{4(\xi_1+3)c_{ini}}{\xi_1}\right\rceil$.
Then, from Lemma \ref{lem:1 and 4 p:stochastic}, we have for $t\geq t_s''$, 
$$\eta(\xi_1+3)p_{2,4}^{(t)}\leq \frac{\eta\xi_1}{4}.$$
Thus, we have  
\begin{equation*}
    \begin{split}
   \frac{p^{(t+1)}_{1,2}}{p^{(t+1)}_{1,3}} 
   \leq {} & \frac{p^{(t)}_{1,2}}{p^{(t)}_{1,3}} \exp\left(-\frac{\eta\xi_1}{4}\right) \ldots \leq \frac{p^{(t''_s-1)}_{1,2}}{p^{(t''_s-1)}_{1,3}} \left[\exp\left(-\frac{\eta\xi_1}{4}\right)\right]^{t-t_s''+1}\\
   \leq {} & c_{\text{ini}}\exp\left(3\eta t_s''\right)\exp\left(-\frac{\eta\xi_1(t-t_s''+1)}{4}\right),     
    \end{split}
\end{equation*}
Note that $t_s''$ is a constant. We have for $t\geq t_s''$,
$$p^{(t+1)}_{1,2}\leq c_{\text{ini}}\exp\left(3\eta t_s''-\frac{\eta\xi_1(t-t_s''+1)}{4}\right), $$ 
and thus 
$$ p^{(t+1)}_{1,3}\geq \max\left\{ 1 -c_{\text{ini}}\exp\left(3\eta t_s''-\frac{\eta\xi_1(t-t_s''+1)}{4}\right),0\right\} $$
It shows that $p_{1,2}^{(t)}$ converges to 0 at an exponential rate after $t_s''$ iterations, which is a constant factor. Combining with Lemma \ref{lem:1 and 4 p:stochastic}, we can draw the conclusion that $p_{1,3}^{(t)}$ converges to 1 exponentially fast after a constant number of iterations.
An identical argument reaches the same conclusion for Bank 2.
This implies that the algorithm eventually converges to the symmetric pure NE $((\tau_h, \gamma_{\ell}),(\tau_h, \gamma_{\ell}))$.
\\


\noindent \textbf{Case II: $\epsilon_1>0,$ and $\epsilon_2>0$.} 
The proof for this case proceeds similarly to Case I.
We have:
\begin{equation}
    \begin{split}
\frac{p^{(t+1)}_{1,3}}{p^{(t+1)}_{1,2}} = {} & \frac{p^{(t)}_{1,3}}{p^{(t)}_{1,2}}\exp\left( \eta \sum_{j=1}^4p^{(t)}_{2,j}\left(\widehat{u}_1\left(\theta^{(3)}, \theta^{(j)},\C^{(t)}\right)-\widehat{u}_1\left(\theta^{(2)}, \theta^{(j)},\C^{(t)}\right)\right) \right) \\
\leq {} & \frac{p^{(t)}_{1,3}}{p^{(t)}_{1,2}}\exp\left( \eta \sum_{j=1}^4p^{(t)}_{2,j}\left(u_1\left(\theta^{(3)}, \theta^{(j)}\right)-u_1\left(\theta^{(2)}, \theta^{(j)}\right)\right) + \frac{\eta\xi_1}{2}\right) \\
\leq {} & \frac{p^{(t)}_{1,3}}{p^{(t)}_{1,2}}\exp\left( -\eta \sum_{j=2}^4p^{(t)}_{2,j}\xi_1   + \eta p^{(t)}_{2,1} + \frac{\eta\xi_1}{2} \right)
\leq  \frac{p^{(t)}_{1,3}}{p^{(t)}_{1,2}}\exp\left( -\eta \sum_{j=1}^4p^{(t)}_{2,j}\xi_1   + \eta(1+\xi_1)  p^{(t)}_{2,1} + \frac{\eta\xi_1}{2}\right)\\
= {} & \frac{p^{(t)}_{1,3}}{p^{(t)}_{1,2}}\exp\left(-\frac{\eta\xi_1}{2} + \eta(1+\xi_1)  p^{(t)}_{2,1}  \right)
    \end{split}
\end{equation}
Where the first inequality is based on Lemma \ref{lem:u-estimation}. Denote as shorthand the constant $t_s'''=\left\lceil\frac{2}{\eta\xi_1}\log\frac{4(1+\xi_1)c_{\text{ini}}}{\xi_1}\right\rceil+1$.
Then, again using Lemma~\ref{lem:1 and 4 p:stochastic}, we have for all $t \geq t_s'''$:
$$ \eta(1+\xi_1)p_{2,1}^{(t)}\leq \frac{\eta\xi_1}{4}, $$
In this case, i.e., for $t\geq t_s'''$, we have 
$$\frac{p^{(t+1)}_{1,3}}{p^{(t+1)}_{1,2}} \leq \frac{p^{(t)}_{1,3}}{p^{(t)}_{1,2}} \exp\left(-\frac{\eta\xi_1}{4}\right) \ldots \leq \frac{p^{(t_s'''-1)}_{1,3}}{p^{(t_s'''-1)}_{1,2}} \left[\exp\left(-\frac{\eta\xi_1}{4}\right)\right]^{t-t_s'''+1}\leq c_{\text{ini}}\exp(t_s'''\eta)\exp\left(-\frac{\eta\xi_1(t-t_s'''+1)}{4}\right). $$
It shows that $p_{1,3}^{(t+1)}$ converges to 0 exponentially fast after $t_s'''$, which is a constant factor. Combining with Lemma \ref{lem:1 and 4 p:stochastic}, we can draw the conclusion that $p_{1,2}^{(t)}$ converges to 1 exponentially fast after a constant number of iterations. 
An identical argument reaches the same conclusion for Bank 2.
This implies that the algorithm eventually converges to the symmetric pure NE $((\tau_{\ell},\gamma_h),(\tau_{\ell},\gamma_h))$.\\

\noindent \textbf{Case III: $\epsilon_1<0,$ and $\epsilon_2>0$.}
We assume $p^{(t)}_{1,1}=p^{(t)}_{1,4}=0$ for $t\geq1$. Let $\epsilon^{(t)}_{1,1} = \widehat{u}_1(\theta^{(2)},\theta^{(2)},\C^{(t)})-\widehat{u}_1(\theta^{(3)},\theta^{(2)},\C^{(t)})$, and $\epsilon^{(t)}_{1,2} = \widehat{u}_1(\theta^{(2)},\theta^{(3)},\C^{(t)})-\widehat{u}_1(\theta^{(3)},\theta^{(3)},\C^{(t)}).$ Let $\epsilon^{(t)}_{2,1} = \widehat{u}_2(\theta^{(2)},\theta^{(2)},\C^{(t)})-\widehat{u}_2(\theta^{(2)},\theta^{(3)},\C^{(t)})$, and $\epsilon^{(t)}_{1,2} = \widehat{u}_2(\theta^{(3)},\theta^{(2)},\C^{(t)})-\widehat{u}_2(\theta^{(3)},\theta^{(3)},\C^{(t)}).$ We have the following Lemma. 
\begin{lemma}
\label{lem:epsilon_ttt}
For $j\in[2]$ and $i\in[2]$, we have $\epsilon^{(t)}_{j,i}$ and $\epsilon_i$  have the same sign. Moreover, $|\epsilon_{j,i}^{(t)}-\epsilon_{j,i}^{(1)}|\leq\frac{\xi_1}{4}$. 
\end{lemma}
\begin{proof}
Note that if $\epsilon_1<0$, we have 
\begin{equation}
    \begin{split}
    \label{eqn:sto:case:3111}
    \epsilon^{(t)}_{1,1} = {} &  \widehat{u}_1(\theta^{(2)},\theta^{(2)},\C^{(t)})-\widehat{u}_1(\theta^{(3)},\theta^{(2)},\C^{(t)})\leq {u}_1(\theta^{(2)},\theta^{(2)}) - {u}_1(\theta^{(3)},\theta^{(2)})+\frac{\xi_1}{8} = \epsilon_1  + \frac{\xi_1}{8} <0, 
    \end{split}
\end{equation}
and similarly 
\begin{equation}
    \begin{split}
    \label{eqn:sto:case:3222}
    \epsilon^{(t)}_{1,2} = {} &  \widehat{u}_1(\theta^{(2)},\theta^{(3)},\C^{(t)})-\widehat{u}_1(\theta^{(3)},\theta^{(3)},\C^{(t)})\geq {u}_1(\theta^{(2)},\theta^{(3)}) - {u}_1(\theta^{(3)},\theta^{(3)})+\frac{\xi_1}{8} = \epsilon_2  + \frac{\xi_1}{8} >0. 
    \end{split}
\end{equation}
That is, $\epsilon^{(t)}_{1,i}$ and $\epsilon_i$ have the same sign for $i\in[2]$. A similar argument can be applied to show $\epsilon^{(t)}_{2,i}$ and $\epsilon_i$ also have the same sign.  Moreover, based on Lemma \ref{lem:u-estimation} we have 
\begin{equation*}
\begin{split}
 \epsilon_{1,1}^{(t)} - \epsilon_{1,1}^{(1)} = {} &  \widehat{u}_1(\theta^{(2)},\theta^{(2)},\C^{(t)}) - \widehat{u}_1(\theta^{(2)},\theta^{(2)},\C^{(1)}) - \widehat{u}_1(\theta^{(3)},\theta^{(2)},\C^{(t)}) + \widehat{u}_1(\theta^{(3)},\theta^{(2)},\C^{(1)}) \\   
 \leq  {} &  {u}_1(\theta^{(2)},\theta^{(2)}) - {u}_1(\theta^{(2)},\theta^{(2)}) - {u}_1(\theta^{(3)},\theta^{(2)}) + {u}_1(\theta^{(3)},\theta^{(2)}) + 
 \frac{\xi_1}{4}\leq \frac{\xi_1}{4},
\end{split}
\end{equation*}
and 
\begin{equation*}
\begin{split}
 \epsilon_{1,2}^{(t)} - \epsilon_{1,2}^{(1)} = {} &  \widehat{u}_1(\theta^{(2)},\theta^{(3)},\C^{(t)}) - \widehat{u}_1(\theta^{(2)},\theta^{(3)},\C^{(1)}) - \widehat{u}_1(\theta^{(3)},\theta^{(3)},\C^{(t)}) + \widehat{u}_1(\theta^{(3)},\theta^{(3)},\C^{(1)}) \\
 \leq {} &  {u}_1(\theta^{(2)},\theta^{(3)}) - {u}_1(\theta^{(2)},\theta^{(3)}) - {u}_1(\theta^{(3)},\theta^{(3)}) + {u}_1(\theta^{(3)},\theta^{(3)}) + \frac{\xi_1}{4}
 \leq  \frac{\xi_1}{4}.
\end{split}
\end{equation*}
Similarly, we also have $\epsilon_{1,1}^{(1)}-\epsilon_{1,1}^{(t)}\leq \frac{\xi_1}{4}$ and  $\epsilon_{1,2}^{(1)}-\epsilon_{2,2}^{(t)}\leq \frac{\xi_1}{4}$. A similar argument can be used to show $|\epsilon_{2,i}^{(t)}-\epsilon_{2,i}^{(1)}|\leq\frac{\xi_1}{4}$ for $i\in[2]$.
\end{proof}

For this case, we set $p^{(1)}_{2,2}\epsilon^{(1)}_{1,1}+p^{(1)}_{2,3}\epsilon^{(1)}_{1,2}<-\Delta$,  $p^{(1)}_{1,2}\epsilon^{(1)}_{2,1}+p^{(1)}_{1,3}\epsilon^{(1)}_{2,2}>\Delta$, for some $\Delta>\frac{\xi_1}{2}$. We have for $t=1$,  
\begin{equation*}
    \begin{split}
\frac{p^{(2)}_{1,2}}{p^{(2)}_{1,3}} = \frac{p^{(1)}_{1,2}}{p^{(1)}_{1,3}}\exp\left(\eta\left(p^{(1)}_{2,2}\epsilon^{(1)}_{1,1} + p^{(1)}_{2,3}\epsilon^{(1)}_{1,2}\right)\right) \leq   \frac{p^{(1)}_{1,2}}{p^{(1)}_{1,3}}\exp(-\eta\Delta),
    \end{split}
\end{equation*}
Note that since $p^{(t)}_{1,2}+p^{(t)}_{1,3}=1$ for all $t$, the deceasing of the ratio means that $p^{(2)}_{1,2}< p^{(1)}_{1,2}$, and $p^{(2)}_{1,3}> p^{(1)}_{1,3}$. On the other hand, 
\begin{equation*}
    \begin{split}
\frac{p^{(2)}_{2,3}}{p^{(2)}_{2,2}} = \frac{p^{(1)}_{2,3}}{p^{(1)}_{2,2}}\exp\left(-\eta\left(p^{(1)}_{1,2}\epsilon^{(1)}_{2,1} + p^{(1)}_{1,3}\epsilon^{(1)}_{2,2}\right)\right) \leq   \frac{p^{(1)}_{2,3}}{p^{(1)}_{2,2}}\exp(-\eta \Delta). 
    \end{split}
\end{equation*}
Similar to our reasoning for Bank 1, this implies that $p^{(2)}_{2,3}<p^{(1)}_{2,3}$, and $p^{(2)}_{2,2}>p^{(1)}_{2,2}$. 
% Consequently, noting that $\epsilon_1 < 0$ and $\epsilon_2 > 0$ we have $p_{1,2}^{(2)}\epsilon_1 + p_{1,3}^{(2)} \epsilon_2 > p_{1,2}^{(1)} \epsilon_1 + p_{1,3}^{(1)} \epsilon_2$ and $p_{2,2}^{(2)} \epsilon_1 + p_{2,3}^{(2)} \epsilon_2 < p_{2,2}^{(1)} \epsilon_1 + p_{2,3}^{(1)} \epsilon_2$.
We now show that $p^{(t+1)}_{1,2}< p^{(1)}_{1,2}$,  $p^{(t+1)}_{1,3}> p^{(1)}_{1,3}$, $p^{(t+1)}_{2,3}<p^{(1)}_{2,3}$, and $p^{(t+1)}_{2,2}>p^{(1)}_{2,2}$ through an inductive argument.
Assume at $t\geq 2$, $p^{(t)}_{1,2}< p^{(1)}_{1,2}$,  $p^{(t)}_{1,3}> p^{(1)}_{1,3}$, $p^{(t)}_{2,3}<p^{(1)}_{2,3}$, and $p^{(t)}_{2,2}>p^{(1)}_{2,2}$.
Then, we have:
\begin{equation*}
    \begin{split}
       \frac{p^{(t+1)}_{1,2}}{p^{(t+1)}_{1,3}} = {} & \frac{p^{(t)}_{1,2}}{p^{(t)}_{1,3}}\exp\left(\eta\left(p^{(t)}_{2,2}\epsilon^{(t)}_{1,1} + p^{(t)}_{2,3}\epsilon^{(t)}_{1,2}\right)\right) \leq \frac{p^{(t)}_{1,2}}{p^{(t)}_{1,3}}\exp\left(\eta\left(p^{(1)}_{2,2}\epsilon^{(t)}_{1,1} + p^{(1)}_{2,3}\epsilon^{(t)}_{1,2}\right)\right)\\
       \leq {} & \frac{p^{(t)}_{1,2}}{p^{(t)}_{1,3}}\exp\left(\eta\left(p^{(1)}_{2,2}\epsilon^{(1)}_{1,1} + p^{(1)}_{2,3}\epsilon^{(1)}_{1,2}\right)+\frac{\xi_1\eta}{2}\right) \leq \frac{p^{(t)}_{1,2}}{p^{(t)}_{1,3}}\exp\left(\eta\left(-\Delta+\frac{\xi_1}{2}\right)\right).
    \end{split}
\end{equation*}
where the first and second inequalities is based on Lemma \ref{lem:epsilon_ttt}. Similarly, we also have 
\begin{equation*}
    \begin{split}
  \frac{p^{(t+1)}_{2,3}}{p^{(t+1)}_{2,2}} = {} & \frac{p^{(t)}_{2,3}}{p^{(t)}_{2,2}}\exp\left(-\eta\left(p^{(t)}_{1,2}\epsilon^{(t)}_{2,1} + p^{(t)}_{1,3}\epsilon^{(t)}_{2,2}\right)\right) \leq \frac{p^{(t)}_{2,3}}{p^{(t)}_{2,2}}\exp\left(-\eta\left(p^{(1)}_{1,2}\epsilon^{(1)}_{2,1} + p^{(1)}_{1,3}\epsilon^{(1)}_{2,2}\right)+\frac{\eta\xi_1}{2}\right)\\
  \leq {} & \frac{p^{(t)}_{2,3}}{p^{(t)}_{2,2}}\exp\left(\eta\left(-\Delta+\frac{\xi_1}{2}\right)\right),       
    \end{split}
\end{equation*}

which implies that we have $p^{(t+1)}_{1,2}< p^{(1)}_{1,2}$,  $p^{(t+1)}_{1,3}> p^{(1)}_{1,3}$, $p^{(t+1)}_{2,3}<p^{(1)}_{2,3}$, and $p^{(t+1)}_{2,2}>p^{(1)}_{2,2}$, which finishes the induction. This also implies that 
\[
p^{(t+1)}_{1,2}\leq p^{(t+1)}_{1,3}c_{ini}\exp\left(\eta\left(-\Delta+\frac{\xi_1}{2}\right)t\right), \quad \text{and} \quad p^{(t+1)}_{2,3}\leq p^{(t+1)}_{2,2}c_{ini}\exp\left(\eta\left(-\Delta+\frac{\xi_1}{2}\right)t\right).
\]





Then we have 
\[
p^{(t+1)}_{1,2}\geq \frac{1}{1+c_{ini}\exp\left(\eta\left(-\Delta+\frac{\xi_1}{2}\right)t\right)},\ \ \text{and}\ \ p^{(t+1)}_{2,3}\geq \frac{1}{1+c_{ini}\exp\left(\eta\left(-\Delta+\frac{\xi_1}{2}\right)t\right)},
\]
which implies that the algorithm converges to the asymmetric NE $((\tau_{\ell},\gamma_{h}),(\tau_{h},\gamma_{\ell}))$. 

Finally, consider the other case, i.e., $p^{(1)}_{2,2}\epsilon^{(1)}_{1,1}+p^{(1)}_{2,3}\epsilon^{(1)}_{1,2}>\Delta$, and $p^{(1)}_{1,2}\epsilon^{(1)}_{2,1}+p^{(1)}_{1,3}\epsilon^{(1)}_{2,2}<-\Delta$, for some constant $\Delta>\frac{\xi_1}{2}$. The proof proceeds very similarly as to the previous case, but the algorithm will converge to a different NE. We have for $t=1$: 
\begin{equation*}
    \begin{split}
\frac{p^{(2)}_{2,2}}{p^{(2)}_{2,3}} = \frac{p^{(1)}_{2,2}}{p^{(1)}_{2,3}}\exp\left(\eta\left(p^{(1)}_{1,2}\epsilon^{(1)}_{2,1} + p^{(1)}_{1,3}\epsilon^{(1)}_{2,2}\right)\right) \leq   \frac{p^{(1)}_{2,2}}{p^{(1)}_{2,3}}\exp(-\eta\Delta),  
    \end{split}
\end{equation*}
and 
\begin{equation*}
    \begin{split}
\frac{p^{(2)}_{1,3}}{p^{(2)}_{1,2}} = \frac{p^{(1)}_{1,3}}{p^{(1)}_{1,2}}\exp\left(-\eta\left(p^{(1)}_{2,2}\epsilon^{(1)}_{1,1} + p^{(1)}_{2,3}\epsilon^{(1)}_{1,2}\right)\right) \leq   \frac{p^{(1)}_{1,3}}{p^{(1)}_{1,2}}\exp(-\eta \Delta). 
    \end{split}
\end{equation*}
Therefore, $p_{2,2}^{(2)}< p_{2,2}^{(1)}$,  $p_{1,3}^{(2)}< p_{1,3}^{(1)}$. Assume for $t\geq 2$, $p_{2,2}^{(t)}< p_{2,2}^{(1)}$,  $p_{1,3}^{(t)}< p_{1,3}^{(1)}$,  $p_{2,3}^{(t)}> p_{2,3}^{(1)}$,  $p_{1,2}^{(t)}> p_{1,2}^{(1)}$. 
Then, we have:
\begin{equation*}
    \begin{split}
   \frac{p^{(t+1)}_{2,2}}{p^{(t+1)}_{2,3}} = {} & \frac{p^{(t)}_{2,2}}{p^{(t)}_{2,3}}\exp\left(\eta\left(p^{(t)}_{1,2}\epsilon^{(t)}_{2,1} + p^{(t)}_{1,3}\epsilon^{(t)}_{2,2}\right)\right) \leq \frac{p^{(t)}_{2,2}}{p^{(t)}_{2,3}}\exp\left(\eta\left(p^{(1)}_{1,2}\epsilon^{(1)}_{2,1} + p^{(1)}_{1,3}\epsilon^{(1)}_{2,2}\right)+\frac{\xi_1}{2}\right)\\
   \leq {} & \frac{p^{(t)}_{2,2}}{p^{(t)}_{2,3}} \exp\left(\eta\left(-\Delta+\frac{\xi}{2}\right)\right).
    \end{split}
\end{equation*}


and 
\begin{equation*}
    \begin{split}
        \frac{p^{(t+1)}_{1,3}}{p^{(t+1)}_{1,2}} = {} & \frac{p^{(t)}_{1,3}}{p^{(t)}_{1,2}}\exp\left(-\eta\left(p^{(t)}_{2,2}\epsilon^{(t)}_{1,1} + p^{(t)}_{2,3}\epsilon^{(t)}_{1,2}\right)\right) \leq \frac{p^{(t)}_{1,3}}{p^{(t)}_{1,2}}\exp\left(-\eta\left(p^{(1)}_{2,2}\epsilon^{(1)}_{1,1} + p^{(1)}_{2,3}\epsilon^{(1)}_{1,2}\right)\right)\\
        \leq {} & \frac{p^{(t)}_{1,3}}{p^{(t)}_{1,2}}\exp\left(\eta\left(-\Delta+\frac{\xi}{2}\right)\right).
    \end{split}
\end{equation*}

which implies that $p_{2,2}^{(t+1)}< p_{2,2}^{(1)}$,  $p_{1,3}^{(t+1)}< p_{1,3}^{(1)}$,  $p_{2,3}^{(t+1)}> p_{2,3}^{(1)}$,  $p_{1,2}^{(t+1)}> p_{1,2}^{(1)}$, and finishes the induction. This shows that that algorithm converges to the asymmetric NE $((\tau_{h},\gamma_{\ell}),(\tau_{\ell},\gamma_h))$.


%\noindent \textbf{a) Symmetric initialization:} Without loss of generality, assume $p^{(1)}_{2,2}\epsilon_1+p^{(1)}_{2,3}\epsilon_2<\Delta$, and $p^{(1)}_{1,2}\epsilon_1+p^{(1)}_{1,3}\epsilon_2<\Delta$, for some constant $\Delta<0$.  We have for $t=1$,  
\noindent \textbf{Case IV: $\epsilon_1>0$, $\epsilon_2<0$.} 
Similar to Case III, we assume $p^{(t)}_{1,1}=p^{(t)}_{1,4}=0$ for $t\geq1$. The proof here will be in a way symmetric to Case III.   We  consider the symmetric initialization case. Without loss of generality, assume $p^{(1)}_{2,2}\epsilon^{(1)}_{1,1}+p^{(1)}_{2,3}\epsilon^{(1)}_{1,2}<-\Delta$, and $p^{(1)}_{1,2}\epsilon^{(1)}_{2,1}+p^{(1)}_{1,3}\epsilon^{(1)}_{2,2}<-\Delta$, for some constant $\Delta>0$. We have for $t=1$, 
\begin{equation*}
    \begin{split}
\frac{p^{(2)}_{1,2}}{p^{(2)}_{1,3}} = \frac{p^{(1)}_{1,2}}{p^{(1)}_{1,3}}\exp\left(\eta\left(p^{(1)}_{2,2}\epsilon^{(1)}_{1,1} + p^{(1)}_{2,3}\epsilon^{(1)}_{1,2}\right)\right)\leq    \frac{p^{(1)}_{1,2}}{p^{(1)}_{1,3}}\exp(-\eta\Delta). 
    \end{split}
\end{equation*}
Note that since $p^{(t)}_{1,2}+p^{(t)}_{1,3}=1$ for all $t$, the deceasing of the ratio means that $p^{(2)}_{1,2}< p^{(1)}_{1,2}$, and $p^{(2)}_{1,3}> p^{(1)}_{1,3}$. Similarly, we have
\begin{equation*}
    \begin{split}
\frac{p^{(2)}_{2,2}}{p^{(2)}_{2,3}} = \frac{p^{(1)}_{2,2}}{p^{(1)}_{2,3}}\exp\left(\eta\left(p^{(1)}_{1,2}\epsilon^{(1)}_{2,1} + p^{(1)}_{1,3}\epsilon^{(1)}_{2,2}\right)\right) \leq   \frac{p^{(1)}_{2,2}}{p^{(1)}_{2,3}}\exp(-\eta \Delta). 
    \end{split}
\end{equation*}
Since $\Delta>0$, it also implies that $p^{(2)}_{2,2}<p^{(1)}_{2,2}$, and $p^{(2)}_{2,3}>p^{(1)}_{2,3}$. Assume at $t\geq 2$, $p^{(t)}_{1,2}< p^{(1)}_{1,2}$,  $p^{(t)}_{1,3}> p^{(1)}_{1,3}$, $p^{(t)}_{2,2}<p^{(1)}_{2,2}$, and $p^{(t)}_{2,3}>p^{(1)}_{2,3}$, then we have:
\begin{equation*}
    \begin{split}
        \frac{p^{(t+1)}_{1,2}}{p^{(t+1)}_{1,3}} = {} & \frac{p^{(t)}_{1,2}}{p^{(t)}_{1,3}}\exp\left(\eta\left(p^{(t)}_{2,2}\epsilon^{(t)}_{1,1} + p^{(t)}_{2,3}\epsilon^{(t)}_{1,2}\right)\right) \leq \frac{p^{(t)}_{1,2}}{p^{(t)}_{1,3}}\exp\left(\eta\left(p^{(1)}_{2,2}\epsilon^{(1)}_{1,1} + p^{(1)}_{2,3}\epsilon^{(1)}_{1,2}\right)+\frac{\xi_1\eta}{2}\right)
        \\
        \leq {} & \frac{p^{(t)}_{1,2}}{p^{(t)}_{1,3}}\exp\left(\eta\left(-\Delta+\frac{\xi}{2}\right)\right),  
    \end{split}
\end{equation*}
and 
\begin{equation*}
    \begin{split}
          \frac{p^{(t+1)}_{2,2}}{p^{(t+1)}_{2,3}} = {} & \frac{p^{(t)}_{2,2}}{p^{(t)}_{2,3}}\exp\left(\eta\left(p^{(t)}_{1,2}\epsilon^{(t)}_{2,1} + p^{(t)}_{1,3}\epsilon^{(t)}_{2,2}\right)\right) \leq \frac{p^{(t)}_{2,3}}{p^{(t)}_{2,2}}\exp\left(-\eta\left(p^{(1)}_{1,2}\epsilon^{(1)}_{2,1} + p^{(1)}_{1,3}\epsilon^{(1)}_{2,2}\right)\right)\\
          \leq {} & \frac{p^{(t)}_{2,3}}{p^{(t)}_{2,2}}\exp\left(\eta\left(-\Delta+\frac{\xi}{2}\right)\right),
    \end{split}
\end{equation*}
which implies that we have $p^{(t+1)}_{1,2}< p^{(1)}_{1,2}$,  $p^{(t+1)}_{1,3}> p^{(1)}_{1,3}$, $p^{(t+1)}_{2,2}<p^{(1)}_{2,2}$, and $p^{(t+1)}_{2,3}>p^{(1)}_{2,3}$, which finishes the induction. This also implies that 
\[
p^{(t+1)}_{1,2}\leq p^{(t+1)}_{1,3}c_{ini}\exp\left(\eta\left(-\Delta+\frac{\xi}{2}\right)t\right), \quad \text{and} \quad p^{(t+1)}_{2,2}\leq p^{(t+1)}_{2,3}c_{ini}\exp\left(\eta\left(-\Delta+\frac{\xi}{2}\right)t\right),
\]
i.e., the algorithm converges to the asymmetric NE $((\gamma_{\ell},\tau_{h}),(\gamma_{\ell},\tau_{h}))$. Following a similar argument, it can be shown that, if $p^{(1)}_{2,2}\epsilon^{(1)}_{1,1}+p^{(1)}_{2,3}\epsilon^{(1)}_{1,2}>\Delta$, and $p^{(1)}_{1,2}\epsilon^{(1)}_{2,1}+p^{(1)}_{1,3}\epsilon^{(1)}_{2,2}>\Delta$, for some $\Delta>0$.  The algorithm converges to $((\gamma_{h},\tau_{\ell}),(\gamma_{h},\tau_{\ell}))$, i.e., the other pure NE. \\

\subsection{Proof of Lemma \ref{lem:u-estimation}}
\label{Proof of u-estimation}
Note that since $y\in[0,1]$, we have $\widehat{u}_i(\theta_1, \theta_2, \mathcal{C}^{(t)})\in[-1,2]$. Therefore, based on Hoeffding's inequality, we have at round $t$, for a pair of fixed $\theta_1,\theta_2\in\{\theta^{(1)},\dots,\theta^{(4)}\}$, a fixed bank $i\in[2]$, $\forall \alpha>0$, 
$$ \P\left[\left|\widehat{u}_i(\theta_1, \theta_2, \mathcal{C}^{(t)})-{u}_i(\theta_1, \theta_2)\right|\geq \alpha\right] \leq 2\exp\left(-\frac{2\alpha^2n}{9}\right), $$
i.e., with a probability $1-\frac{\delta}{32T}$, we have 
$$ \left|\widehat{u}_i(\theta_1, \theta_2, \mathcal{C}^{(t)})-{u}_i(\theta_1, \theta_2)\right|\leq \sqrt{\frac{5}{n}\log\frac{64T}{\delta}}. $$
Taking a union bound over all 16 decisions pairs, both banks and all rounds $t\in[T]$, we have with probability at least  $1-\delta$, at each round $t\in[T]$, for all $\theta_1,\theta_2\in\{\theta^{(1)},\dots,\theta^{(4)}\}$, and $i\in[2]$, we have 
$$ \left|\widehat{u}_i(\theta_1, \theta_2, \mathcal{C}^{(t)})-{u}_i(\theta_1, \theta_2)\right|\leq \sqrt{\frac{5}{n}\log\frac{64T}{\delta}}\leq \frac{\xi_1}{16}. $$
where the last inequity is because $n\geq \frac{1280\log\frac{64T}{\delta}}{\xi_1^2}.$ 


\subsection{Proof of Lemma \ref{lem:1 and 4 p:stochastic}} 
\label{proof:Theoerm:stoc}
We now show the convergence of $p_{i,1}^{(t)}$ and $p_{i,4}^{(t)}$. The proof is similar to the proof of Lemma \ref{lem:1 and 4 p}. We have for $t>1$,
\begin{equation*}
\begin{split}
    \frac{p^{(t+1)}_{1,1}}{p^{(t+1)}_{1,3}}  = {} & \frac{p^{(t)}_{1,1}\exp\left(\eta \sum_{j=1}^4 p_{2,j}^{(t)}\widehat{u}_1\left(\theta^{(1)},\theta^{(j)},\C^{(t)}\right) \right)}{p^{(t)}_{1,3}\exp\left(\eta \sum_{j=1}^4 p_{2,j}^{(t)}\widehat{u}_1\left(\theta^{(3)},\theta^{(j)},\C^{(t)}\right) \right)}\\
    = {} & \frac{p^{(t)}_{1,1}}{p_{1,3}^{(t)}}\exp\left(\eta \sum_{j=1}^4 p_{2,j}^{(t)}\left(\widehat{u}_1\left(\theta^{(1)},\theta^{(j)},\C^{(t)}\right) - \widehat{u}_1\left(\theta^{(3)},\theta^{(j)},\C^{(t)}\right)\right) \right)\\
    \leq {} & \frac{p^{(t)}_{1,1}}{p_{1,3}^{(t)}}\exp\left(\eta \sum_{j=1}^4 p_{2,j}^{(t)}\left({u}_1\left(\theta^{(1)},\theta^{(j)}\right) - {u}_1\left(\theta^{(3)},\theta^{(j)}\right)\right) + \frac{\xi_1\eta}{2}\right)\\
    \leq {} & \frac{p^{(t)}_{1,1}}{p_{1,3}^{(t)}}\exp\left( -\eta\xi_1 +\frac{\xi_1\eta}{2} \right)\leq  \frac{p^{(t)}_{1,1}}{p_{1,3}^{(t)}}\exp\left( -\frac{\eta\xi_1}{2} \right)
\end{split}
\end{equation*}
where the first inequality is based on Lemma  \ref{lem:u-estimation}. Therefore, we have 
\[
{p^{(t+1)}_{1,1}}\leq p_{1,3}^{(t+1)} \frac{p^{(1)}_{1,1}}{p^{(1)}_{1,3}}\exp\left(-\frac{\eta\xi_1t}{2}\right)\leq  \frac{p^{(1)}_{1,1}}{p^{(1)}_{1,3}}\exp\left(-\frac{\eta\xi_1t}{2}\right)\leq c_{ini}\exp\left(-\frac{\eta\xi_1t}{2}\right),
\]
where the last inequality plugs in the definition of $c_{ini}=\max_{i\in[2]}\left\{\frac{\max_{j\in[4]}p^{(1)}_{i,j}}{\min_{i\in[4]}p^{(1)}_{i,j}}\right\}$.
Similarly, we also have 

\begin{equation*}
\begin{split}
    \frac{p^{(t+1)}_{2,1}}{p^{(t+1)}_{2,3}}  = {} & \frac{p^{(t)}_{2,1}\exp\left(\eta \sum_{j=1}^4 p_{1,j}^{(t)}\widehat{u}_2\left(\theta^{(j)},\theta^{(1)},\C^{(t)}\right) \right)}{p^{(t)}_{2,3}\exp\left(\eta \sum_{j=1}^4 p_{1,j}^{(t)}\widehat{u}_2\left(\theta^{(j)},\theta^{(3)},\C^{(t)}\right) \right)}\\
    = {} & \frac{p^{(t)}_{2,1}}{p_{2,3}^{(t)}}\exp\left(\eta \sum_{j=1}^4 p_{1,j}^{(t)}\left(\widehat{u}_2\left(\theta^{(j)},\theta^{(1)},\C^{(t)}\right) - \widehat{u}_2\left(\theta^{(j)},\theta^{(3)},\C^{(t)}\right)\right) \right)\\
    \leq {} & \frac{p^{(t)}_{2,1}}{p_{2,3}^{(t)}}\exp\left(\eta \sum_{j=1}^4 p_{1,j}^{(t)}\left({u}_2\left(\theta^{(j)},\theta^{(1)}\right) - {u}_2\left(\theta^{(j)},\theta^{(3)}\right)\right) + \frac{\xi_1\eta}{2}\right)\\
    \leq {} & \frac{p^{(t)}_{2,1}}{p_{2,3}^{(t)}}\exp\left( -\eta\xi_1 +\frac{\xi_1\eta}{2} \right)\leq  \frac{p^{(t)}_{2,1}}{p_{2,3}^{(t)}}\exp\left( -\frac{\eta\xi_1}{2} \right)
\end{split}
\end{equation*}
where the first inequality is based on Lemma \ref{lem:u-estimation}. It implies that 
\[
{p^{(t+1)}_{2,1}}\leq c_{ini}\exp\left(-\frac{\eta\xi_1t}{2}\right).
\]
Next, we know that $u_1(\theta^{(2)},\theta^{(1)})=u_1(\theta^{(4)},\theta^{(1)})=0$, but for all $j \neq 1$, we have $u_1\left(\theta^{(2)},\theta^{(j)}\right) - u_1\left(\theta^{(4)},\theta^{(j)}\right) \geq \xi_1$. Therefore, we have for all $t>1$,
\begin{equation*}
   \begin{split}
\frac{p^{(t+1)}_{1,4}}{p^{(t+1)}_{1,2}} = {} &     \frac{p^{(t)}_{1,4}}{p^{(t)}_{1,2}} \exp\left(\eta \sum_{j=2}^4 p^{(t)}_{2,j}\left(\widehat{u}_1\left(\theta^{(4)},\theta^{(j)},\C^{(t)}\right)-\widehat{u}_1\left(\theta^{(2)},\theta^{(j)},\C^{(t)}\right)\right) \right)\\
=  {} &   \frac{p^{(t)}_{1,4}}{p^{(t)}_{1,2}} \exp\left(\eta \sum_{j=2}^4 p^{(t)}_{2,j}\left(u_1\left(\theta^{(4)},\theta^{(j)}\right)-u_1\left(\theta^{(2)},\theta^{(j)}\right)\right) +\frac{\eta\xi_1}{2}\right)\leq \frac{p^{(t)}_{1,4}}{p^{(t)}_{1,2}} \exp\left(-\eta \xi_1\sum_{j=2}^4 p^{(t)}_{2,j}+\frac{\eta\xi_1}{2} \right).
   \end{split} 
\end{equation*}
Note that it implies that the ratio between $p_{1,4}^{(t)}$ and $p_{1,2}^{(t)}$ is non-increasing. Moreover, we have just shown that
\[
p_{2,1}^{(t)}\leq c_{ini}\exp\left(-\frac{\eta\xi_1(t-1)}{2}\right)
\]
For $t\geq t_s'=\max\left\{\left\lceil\frac{2\log 4c_{ini}}{\epsilon\eta}+1\right\rceil,2\right\}$, we have $p_{2,1}^{(t)}\leq \frac{1}{4}$, which implies that 
\[
\sum_{j=2}^4p_{2,j}^{(t)}= 1- p_{1,2}^{(t)}\geq\frac{3}{4}.
\] 
Thus, 
\begin{equation*}
    \begin{split}
 \frac{p^{(t+1)}_{1,4}}{p^{(t+1)}_{1,2}}\leq {} &  \frac{p^{(t)}_{1,4}}{p^{(t)}_{1,2}} \exp\left(-\eta \xi_1\sum_{j=2}^4 p^{(t)}_{2,j} + \frac{\eta\xi_1}{2}\right)\leq \frac{p^{(t'_s-1)}_{1,4}}{p^{(t_s'-1)}_{1,2}} \left(\exp\left(-\frac{1}{4}\eta \xi_1 \right)\right)^{t-t'_s+1}\exp\left(-\eta\xi_1\sum_{j=2}^4p_{2,j}^{(t_s'-1)}\right)\\
\leq {} &\frac{p^{(1)}_{1,4}}{p^{(1)}_{1,2}} \left(\exp\left(-\frac{1}{4}\eta \xi_1 \right)\right)^{t-t_s'+1}\leq c_{ini}\left(\exp\left(-\frac{1}{4}\eta \xi_1 (t-t_s'+1)\right)\right).       
    \end{split}
\end{equation*}

%Then the algorithm either converges to the pure NE in finite time, or mixed NE as $t\rightarrow\infty.$