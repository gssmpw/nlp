\begin{figure}[H]
    \centering
   {
    \begin{subfigure}{0.49\linewidth}
        \centering
        \includegraphics[width=0.9\linewidth]{Figures/2gamma/casepm/distance/truemat_Bank1_distance_to_NE,_known_matrix.pdf}
        % \caption{Bank1: Known utility matrix}
    \end{subfigure}
    \begin{subfigure}{0.49\linewidth}
        \centering
        \includegraphics[width=0.9\linewidth]{Figures/2gamma/casepm/distance/freshmat_Bank1_distance_to_NE,_fresh_estimate.pdf}
        % \caption{Bank1: Estimated utility matrix}
    \end{subfigure}

    \begin{subfigure}{0.49\linewidth}
        \centering
        \includegraphics[width=0.9\linewidth]{Figures/2gamma/casepm/distance/truemat_Bank2_distance_to_NE,_known_matrix.pdf}
    \end{subfigure}
    \begin{subfigure}{0.49\linewidth}
        \centering
        \includegraphics[width=0.9\linewidth]{Figures/2gamma/casepm/distance/freshmat_Bank2_distance_to_NE,_fresh_estimate.pdf}
    \end{subfigure}
    }
\caption{Case +\,-: Distance of the iterates of Bank 1 (top row) and Bank 2 (bottom row) to the Nash equilibrium. Left: known utility matrices. Right: estimated utilities using a single sample per round. $y \sim$ truncated Gaussian ($\mu=0.1, \sigma=0.2$), $\gamma_l = 0.4$, $\gamma_h = 0.8$. Mean(solid) and standard error (shaded) across 5 random initializations.\label{fig:dist-pm}}
\end{figure}
For the last iterate of Algorithm \ref{alg:Hedge} for both banks, i.e., \( p_1^{(T)} \) and \( p_2^{(T)} \), we identify the closest Nash equilibrium \( (\theta_1^*, \theta_2^*) \) from the set of Nash equilibria. We then compute the \( L_2 \) distance of each iterate along the trajectory to these equilibrium profiles, \( \|p_1^t - \theta_1^*\|_2 \) and \( \|p_2^t - \theta_2^*\|_2 \), and plot the mean and standard error across five random initializations. The mean is shown as a solid line, with the shaded region representing the standard error of the mean at each time step \( t \).  In Figure~\ref{fig:dist-pm}, we show the \( L_2 \) distance to the nearest Nash equilibrium for the \( \epsilon_1 > 0, \epsilon_2 < 0 \) case. Similar figures for other sign combinations of \( (\epsilon_1, \epsilon_2) \) are provided in Appendix~\ref{app-distNE}. We do note that when estimating the utility matrix using a single sample per round (Algorithm~\ref{alg:Hedge:stoc} with $n_{sample} = 1$), we observe slight deviations in the convergence path due to in-sample errors. However, even in this setting, the algorithm consistently converges to a Nash equilibrium, following a trajectory similar to that in the full-information case.


% For clarity, Table~\ref{tab:NE_distance} maps each figure in this section to the corresponding sign combination of \( \epsilon_1, \epsilon_2 \), simplifying the exposition.  Our findings are consistent across all figures: regardless of initialization, the dynamics converge to one of the Nash equilibria predicted by theory in Section~\ref{sec:understand NE}. We do note that when estimating the utility matrix using a single sample per round (Algorithm \ref{alg:Hedge:stoc}), we observe slight deviations in the convergence path due to in-sample errors. However, even in this setting, the algorithm consistently converges to a Nash equilibrium, following a trajectory similar to that in the full-information case.



