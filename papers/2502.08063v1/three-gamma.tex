\section{Representative Dynamics for the Bank Game with a Larger Action Set}
\label{app:3gamma}





\begin{algorithm}[!h]
\caption{Online Learning Dynamic through Exponential Weights with Larger Decision Set}
\begin{algorithmic}
\label{alg:Hedge:L} 
\STATE \textbf{Input}: step size $\eta>0$, time period $T$
    \STATE \textbf{Initialization: $p_1^{(1)},p_2^{(1)}\in\Delta_9$}
\FOR{$t=1,\dots,T$}
\STATE \emph{Decision:} Each player $i$ plays $\theta^{(j)}$ with probability $p_{i,j}^{(t)}$;
\STATE \emph{Update Rule:} Players update their probability distributions to $p_{1}^{(t+1)},~p_{2}^{(t+1)}$ as follows:
\begin{align*}
&p^{(t+1)}_{1,j}\propto p^{(t)}_{1,j} \cdot \exp\left(\eta \sum_{k=1}^9 p^{(t)}_{2,k} \cdot u_1\left(\theta^{(j)},\theta^{(k)}\right)\right),~~\forall k\in[9]\\
&p^{(t+1)}_{2,j}\propto p^{(t)}_{2,j} \cdot \exp\left(\eta \sum_{k=1}^9 p^{(t)}_{1,k} \cdot u_2\left(\theta^{(k)},\theta^{(j)} \right)\right),~~\forall k\in[9]
\end{align*}
% $\forall i\in[4]$, $p^{(t+1)}_{1,i}\propto p^{(t)}_{1,i}\exp\left(\eta \sum_{j=1}^4 p^{(t)}_{2,j}u_1\left(\theta^{(i)},\theta^{(j)}\right)\right)$
% \STATE $\forall i\in[4]$, $p^{(t+1)}_{2,i}\propto p^{(t)}_{2,i}\exp\left(\eta \sum_{j=1}^4 p^{(t)}_{1,j}u_2\left(\theta^{(i)},\theta^{(j)}\right)\right)$
\ENDFOR
\end{algorithmic}
\end{algorithm}
\begin{algorithm}[h]
\caption{Online Learning Dynamic through Exponential Weights with Fresh Samples and Larger Decision Set}
\begin{algorithmic}
\label{alg:Hedge:stoc:L} 
\STATE \textbf{Input}: step size $\eta>0$, time period $T$, number of fresh samples at each round $n_{samples}$
    \STATE \textbf{Initialization: $p_1^{(1)},p_2^{(1)}\in\Delta_9$}
\FOR{$t=1,\dots,T$}
\STATE \emph{Decision:} Each player $i$ plays $\theta^{(j)}$ with probability $p_{i,j}^{(t)}$;
\STATE \emph{Sampling:} Sample a batch of customers $\mathcal{C}^{(t)}=\{y_i\}_{i=1}^{n_{samples}}$, where $y_i\sim\D_y$, for $i\in[n_{samples}]$ 
\STATE \emph{Estimation:} Construct Estimated Utility function $\widehat{u}_1(\theta_1, \theta_2, \mathcal{C}^{(t)})$ and $\widehat{u}_2(\theta_1, \theta_2, \mathcal{C}^{(t)}) $ by \eqref{eq:utility:est1} and \eqref{eq:utility:est2}. 
\STATE \emph{Update Rule:} Players update their probability distributions to $p_{1}^{(t+1)},~p_{2}^{(t+1)}$ as follows:
\begin{align*}
&p^{(t+1)}_{1,j}\propto p^{(t)}_{1,j} \cdot \exp\left(\eta \sum_{k=1}^9 p^{(t)}_{2,k} \cdot \widehat{u}_1\left(\theta^{(j)},\theta^{(k)},\mathcal{C}^{(t)}\right)\right),~~\forall k\in[9]\\
&p^{(t+1)}_{2,j}\propto p^{(t)}_{2,j} \cdot \exp\left(\eta \sum_{k=1}^9 p^{(t)}_{1,k} \cdot \widehat{u}_2\left(\theta^{(k)},\theta^{(j)} ,\mathcal{C}^{(t)}\right)\right),~~\forall k\in[9]
\end{align*}
% $\forall i\in[4]$, $p^{(t+1)}_{1,i}\propto p^{(t)}_{1,i}\exp\left(\eta \sum_{j=1}^4 p^{(t)}_{2,j}u_1\left(\theta^{(i)},\theta^{(j)}\right)\right)$
% \STATE $\forall i\in[4]$, $p^{(t+1)}_{2,i}\propto p^{(t)}_{2,i}\exp\left(\eta \sum_{j=1}^4 p^{(t)}_{1,j}u_2\left(\theta^{(i)},\theta^{(j)}\right)\right)$
\ENDFOR
\end{algorithmic}
\end{algorithm}
In this section, we run the online learning dynamic with a larger action set. The deterministic version of the algorithm is given in Algorithm \ref{alg:Hedge:L}, and the stochastic version is in Algorithm \ref{alg:Hedge:stoc:L}. Specifically, we consider three interest rates: low, medium and high $\{\gamma_l, \gamma_m, \gamma_h\}$, along with the corresponding credit thresholds $\{\tau_l = \frac{1}{2+\gamma_h}, \tau_m = \frac{1}{2+\gamma_m}, \tau_h = \frac{1}{2+\gamma_l}\}$
Thus, the strategy for each bank is represented as a nine-dimensional vector in the probability simplex, $\Delta_9$, corresponding to the weight assignments on the following $9$ actions:  
$
(\tau_l, \gamma_l), (\tau_l, \gamma_m), (\tau_l, \gamma_h), (\tau_m, \gamma_l), (\tau_m, \gamma_m), (\tau_m, \gamma_h), (\tau_h, \gamma_l), (\tau_h, \gamma_m), (\tau_h, \gamma_h),
$ and we enumerate them as $\{\theta^{(1)},\dots,\theta^{(9)}\}.$ In Appendix \ref{app:3-gam-truncgauss}, we provide figures for various Truncated Gaussian distributions, and in Appendix \ref{app:3-gamma-puf}, we present figures for different piecewise uniform distributions. The Nash equilibria for each instance are computed numerically using vertex enumeration \cite{Nisan_Roughgarden_Tardos_Vazirani_2007}.  

We note here that the key insights from the two-interest-rate in the main body generally hold, with similar convergence of Algorithm \ref{alg:Hedge} and \ref{alg:Hedge:stoc} to Nash Equilibria. However, for carefully crafted piecewise uniform distributions, we observe cycling around the mixed Nash equilibrium. While rare, this phenomenon highlights that increasing the action space can, in specific cases, lead to non-convergent last iterates.

\subsection{Game instances with Truncated Gaussian distributions}
\label{app:3-gam-truncgauss}

The instances of the Bank game with the truncated Gaussian distribution, along with their corresponding Nash 
equilibria and dynamics, are summarized in Table \ref{tab:3-gamma-NE-TG}.

\vspace*{-11pt} % HACK

\begin{table}[!h]
    \centering
    \renewcommand{\arraystretch}{1.3}
    \begin{tabular}{|c | c | c | c|}
        \hline
        \textbf{Distribution} & $(\gamma_l, \gamma_m, \gamma_h)$ & \textbf{Nash Equilibria} & \textbf{Figure} \\
        \hline
        Truncated & $(0.1, 0.4, 0.8)$ & 1 pure: & Fig. \ref{fig:dyna-A1} \\
        Gaussian $(0.3, 0.1)$ &  & $((\tau_l, \gamma_h), (\tau_l, \gamma_h))$ & \\
        \hline
        Truncated & $(0.1, 0.4, 0.8)$ & 1 pure: & Fig. \ref{fig:dyna-A2-pure} \\
        Gaussian $(0.1, 0.3)$ &  & $((\tau_h, \gamma_l), (\tau_h, \gamma_l))$ & \\
        &  & 2 symmetric mixed: & Fig. \ref{fig:dyna-A2-mixed1} \\
        &  & $\left(  
        \begin{array}{c}  
        (0, 0, 0.057, 0, 0, 0, 0.943, 0, 0), \\  
        (0, 0, 0.057, 0, 0, 0, 0.943, 0, 0)  
        \end{array}  
        \right)$ & \\
        &  & $\left(  
        \begin{array}{c}  
        (0, 0, 0.319, 0, 0.16, 0, 0.521, 0, 0), \\  
        (0, 0, 0.319, 0, 0.16, 0, 0.521, 0, 0)  
        \end{array}  
        \right)$ & Fig. \ref{fig:dyna-A2-mixed2} \\
        \hline
        Truncated & $(0.1, 0.4, 0.8)$ & 2 pure: & Fig. \ref{fig:dyna-A3-pure1},\ref{fig:dyna-A3-pure2} \\
        Gaussian $(0.1, 0.2)$ &  & $((\tau_l, \gamma_h), (\tau_l, \gamma_h))$ & \\
        &  & $((\tau_m, \gamma_m), (\tau_m, \gamma_m))$ & \\
        &  & 1 symmetric mixed: & Fig. \ref{fig:dyna-A3-mixed} \\
        &  & $\left(  
        \begin{array}{c}  
        (0, 0, 0.619, 0, 0.381, 0, 0, 0, 0), \\  
        (0, 0, 0.619, 0, 0.381, 0, 0, 0, 0)  
        \end{array}  
        \right)$ & \\
        \hline
    \end{tabular}
    \caption{Truncated Gaussian parameters, $3$ interest rates, Nash equilibria of the game, and figures for dynamics of Algorithm \ref{alg:Hedge:L} and \ref{alg:Hedge:stoc:L}}
    \label{tab:3-gamma-NE-TG}
\end{table}

\begin{figure}[H]
    \centering
    \begin{subfigure}{0.49\linewidth}
        \centering
        \includegraphics[width=\linewidth]{Figures/3gamma/ms31_A1/pureNE/Known-MatrixBank1.pdf}
    \end{subfigure}
    \begin{subfigure}{0.49\linewidth}
        \centering
        \includegraphics[width=\linewidth]{Figures/3gamma/ms31_A1/pureNE/Fresh-EstimateBank1.pdf}
    \end{subfigure}

    \begin{subfigure}{0.49\linewidth}
        \centering
        \includegraphics[width=\linewidth]{Figures/3gamma/ms31_A1/pureNE/Known-MatrixBank2.pdf}
    \end{subfigure}
    \begin{subfigure}{0.49\linewidth}
        \centering
        \includegraphics[width=\linewidth]{Figures/3gamma/ms31_A1/pureNE/Fresh-EstimateBank2.pdf}
    \end{subfigure}
    \caption{$y \sim$  Truncated Gaussian ($\mu=0.3, \sigma=0.1$), Exponential weights dynamics for Bank 1 (top row) and Bank 2 (bottom row), converging to the unique pure NE $((\tau_l, \gamma_h),(\tau_l, \gamma_h))$. Left: known utility matrices. Right: estimated utilities using a single sample per round. Initial weights for both banks are $(1/9 \ldots 1/9)$ \label{fig:dyna-A1}}
\end{figure}


\begin{figure}[H]
    \centering
    \begin{subfigure}{0.49\linewidth}
        \centering
        \includegraphics[width=\linewidth]{Figures/3gamma/ms13_A2/pureNE/Known-MatrixBank1.pdf}
        % \caption{Bank1: Known utility matrix}
    \end{subfigure}
    \begin{subfigure}{0.49\linewidth}
        \centering
        \includegraphics[width=\linewidth]{Figures/3gamma/ms13_A2/pureNE/Fresh-EstimateBank1.pdf}
        % \caption{Bank1: Estimated utility matrix}
    \end{subfigure}

    \begin{subfigure}{0.49\linewidth}
        \centering
        \includegraphics[width=\linewidth]{Figures/3gamma/ms13_A2/pureNE/Known-MatrixBank2.pdf}
        % \caption{Bank2: Known utility matrix}
    \end{subfigure}
    \begin{subfigure}{0.49\linewidth}
        \centering
        \includegraphics[width=\linewidth]{Figures/3gamma/ms13_A2/pureNE/Fresh-EstimateBank2.pdf}
        % \caption{Bank2: Estimated utility matrix}
    \end{subfigure}
    \caption{$y \sim$ Truncated Gaussian ($\mu=0.1, \sigma=0.3$), Exponential weights dynamics for Bank 1 (top row) and Bank 2 (bottom row), converging to the unique pure NE $((\tau_h, \gamma_l),(\tau_h, \gamma_l))$. Left: known utility matrices. Right: estimated utilities using a single sample per round. Initial weights for both banks are $(1/9 \ldots 1/9)$ \label{fig:dyna-A2-pure}}

\end{figure}

\begin{figure}[H]
    \centering
    \begin{subfigure}{0.49\linewidth}
        \centering
        \includegraphics[width=\linewidth]{Figures/3gamma/ms13_A2/mixedNE1/Known-MatrixBank1.pdf}
        % \caption{Bank1: Known utility matrix}
    \end{subfigure}
    \begin{subfigure}{0.49\linewidth}
        \centering
        \includegraphics[width=\linewidth]{Figures/3gamma/ms13_A2/mixedNE1/Fresh-EstimateBank1.pdf}
        % \caption{Bank1: Estimated utility matrix}
    \end{subfigure}

    \begin{subfigure}{0.49\linewidth}
        \centering
        \includegraphics[width=\linewidth]{Figures/3gamma/ms13_A2/mixedNE1/Known-MatrixBank2.pdf}
        % \caption{Bank2: Known utility matrix}
    \end{subfigure}
    \begin{subfigure}{0.49\linewidth}
        \centering
        \includegraphics[width=\linewidth]{Figures/3gamma/ms13_A2/mixedNE1/Fresh-EstimateBank2.pdf}
        % \caption{Bank2: Estimated utility matrix}
    \end{subfigure}
    \caption{Truncated Gaussian ($\mu=0.1, \sigma=0.3$), Exponential weights dynamics for Bank 1 (top row) and Bank 2 (bottom row). Left: known utility matrices. Right: estimated utilities using a single sample per round. Both initialized at the mixed symmetric NE profile $(0, 0, 0.057, 0, 0, 0, 0.943, 0, 0)$. The dynamics with the known matrix are stationary at this NE, whereas those with the fresh estimate converge to the pure NE $((\tau_h, \gamma_l),(\tau_h, \gamma_l))$ \label{fig:dyna-A2-mixed1}}

\end{figure}


\begin{figure}[H]
    \centering
    \begin{subfigure}{0.49\linewidth}
        \centering
        \includegraphics[width=\linewidth]{Figures/3gamma/ms13_A2/mixedNE2/Known-MatrixBank1.pdf}
        % \caption{Bank1: Known utility matrix}
    \end{subfigure}
    \begin{subfigure}{0.49\linewidth}
        \centering
        \includegraphics[width=\linewidth]{Figures/3gamma/ms13_A2/mixedNE2/Fresh-EstimateBank1.pdf}
        % \caption{Bank1: Estimated utility matrix}
    \end{subfigure}

    \begin{subfigure}{0.49\linewidth}
        \centering
        \includegraphics[width=\linewidth]{Figures/3gamma/ms13_A2/mixedNE2/Known-MatrixBank2.pdf}
        % \caption{Bank2: Known utility matrix}
    \end{subfigure}
    \begin{subfigure}{0.49\linewidth}
        \centering
        \includegraphics[width=\linewidth]{Figures/3gamma/ms13_A2/mixedNE2/Fresh-EstimateBank2.pdf}
        % \caption{Bank2: Estimated utility matrix}
    \end{subfigure}
    \caption{$y \sim$ Truncated Gaussian ($\mu=0.1, \sigma=0.3$), Exponential weights dynamics for Bank 1 (top row) and Bank 2 (bottom row). Left: known utility matrices. Right: estimated utilities using a single sample per round. Both banks initialized at the mixed symmetric NE profile $(0, 0, 0.319, 0, 0.16, 0, 0.521, 0, 0)$. The dynamics with the known matrix are stationary at this NE, whereas those with the fresh estimate converge to the pure NE $((\tau_h, \gamma_l),(\tau_h, \gamma_l))$ \label{fig:dyna-A2-mixed2}}

\end{figure}

\begin{figure}[H]
    \centering
    \begin{subfigure}{0.49\linewidth}
        \centering
        \includegraphics[width=\linewidth]{Figures/3gamma/ms12_A3/pureNE1/Known-MatrixBank1.pdf}
    \end{subfigure}
    \begin{subfigure}{0.49\linewidth}
        \centering
        \includegraphics[width=\linewidth]{Figures/3gamma/ms12_A3/pureNE1/Fresh-EstimateBank1.pdf}
    \end{subfigure}

    \begin{subfigure}{0.49\linewidth}
        \centering
        \includegraphics[width=\linewidth]{Figures/3gamma/ms12_A3/pureNE1/Known-MatrixBank2.pdf}
    \end{subfigure}
    \begin{subfigure}{0.49\linewidth}
        \centering
        \includegraphics[width=\linewidth]{Figures/3gamma/ms12_A3/pureNE1/Fresh-EstimateBank2.pdf}
    \end{subfigure}
    \caption{$y \sim$ Truncated Gaussian ($\mu=0.1, \sigma=0.2$), Exponential weights dynamics for Bank1 (top row) and Bank2 (bottom row), converging to the pure NE $((\tau_l, \gamma_h),(\tau_l, \gamma_h))$. Left: known utility matrices. Right: estimated utilities using a single sample per round. Initial weights for both banks are $(1/9 \ldots 1/9)$ \label{fig:dyna-A3-pure1}}

\end{figure}


\begin{figure}[H]
    \centering
    \begin{subfigure}{0.49\linewidth}
        \centering
        \includegraphics[width=\linewidth]{Figures/3gamma/ms12_A3/pureNE2/Known-MatrixBank1.pdf}
    \end{subfigure}
    \begin{subfigure}{0.49\linewidth}
        \centering
        \includegraphics[width=\linewidth]{Figures/3gamma/ms12_A3/pureNE2/Fresh-EstimateBank1.pdf}
    \end{subfigure}

    \begin{subfigure}{0.49\linewidth}
        \centering
        \includegraphics[width=\linewidth]{Figures/3gamma/ms12_A3/pureNE2/Known-MatrixBank2.pdf}
    \end{subfigure}
    \begin{subfigure}{0.49\linewidth}
        \centering
        \includegraphics[width=\linewidth]{Figures/3gamma/ms12_A3/pureNE2/Fresh-EstimateBank2.pdf}
    \end{subfigure}
    \caption{$y \sim$ Truncated Gaussian ($\mu=0.1, \sigma=0.2$), Exponential weights dynamics for Bank 1 (top row) and Bank 2 (bottom row), converging to the pure NE $((\tau_m, \gamma_m),(\tau_m, \gamma_m))$. Left: known utility matrices. Right: estimated utilities using a single sample per round. Initial weights for both banks are $(0,0,0, 0,0.8,0.2, 0,0,0)$ \label{fig:dyna-A3-pure2}}
\end{figure}

\begin{figure}[H]
    \centering
    \begin{subfigure}{0.49\linewidth}
        \centering
        \includegraphics[width=\linewidth]{Figures/3gamma/ms12_A3/mixedNE/Known-MatrixBank1.pdf}
    \end{subfigure}
    \begin{subfigure}{0.49\linewidth}
        \centering
        \includegraphics[width=\linewidth]{Figures/3gamma/ms12_A3/mixedNE/Fresh-EstimateBank1.pdf}
    \end{subfigure}
    \begin{subfigure}{0.49\linewidth}
        \centering
        \includegraphics[width=\linewidth]{Figures/3gamma/ms12_A3/mixedNE/Known-MatrixBank2.pdf}
    \end{subfigure}
    \begin{subfigure}{0.49\linewidth}
        \centering
        \includegraphics[width=\linewidth]{Figures/3gamma/ms12_A3/mixedNE/Fresh-EstimateBank2.pdf}
    \end{subfigure}
    \caption{$y \sim$ Truncated Gaussian ($\mu=0.1, \sigma=0.2$), Exponential weights dynamics for Bank 1 (top row) and Bank 2 (bottom row). Left: known utility matrices. Right: estimated utilities using a single sample per round. Both banks initialized at the symmetric mixed NE profile $(0, 0, 0.619, 0, 0.381, 0, 0, 0, 0)$. The dynamics with the known matrix are stationary at this NE, whereas those with the fresh estimate converge to the pure NE $((\tau_l, \gamma_h),(\tau_l, \gamma_h))$ \label{fig:dyna-A3-mixed}} 
\end{figure}

\subsection{Game instances with piecewise uniform distributions}
\label{app:3-gamma-puf}
Most instances with piecewise uniform distributions and three interest rates converge to a Nash equilibrium. The instances in Table \ref{tab:3-gamma-NE-PU} are selected to highlight cases of non-convergence. The first instance has a unique symmetric mixed Nash equilibrium, with exponential weights dynamics cycling around it (Figure \ref{fig:dyna-puf-fastcycles}) for Algorithm \ref{alg:Hedge:L} and \ref{alg:Hedge:stoc:L}. The second instance, which has two asymmetric and one symmetric mixed Nash equilibrium, shows cycling around the asymmetric mixed NE (Figure \ref{fig:dyna-puf-slowcycles-ass}) and convergence to the symmetric mixed NE (Figure \ref{fig:dyna-puf-slowcycles-sym}), depending on the initialization. While rare, these examples highlight that increasing the action space can, in certain cases, lead to non-convergent last iterates.

\begin{table}[H]
    \centering
    \renewcommand{\arraystretch}{1.3}
    \begin{tabular}{|c | c | c | c|}
        \hline
        \textbf{Distribution} & $(\gamma_l, \gamma_m, \gamma_h)$ & \textbf{Nash Equilibria} & \textbf{Figure} \\
        \hline
        Piecewise & $(0.54, 0.61, 0.7)$ & 1 symmetric mixed: & Fig. \ref{fig:dyna-puf-fastcycles} \\
        Uniform (Eq. \eqref{eq:app-pu-dist1}) &  & $\left(  
        \begin{array}{c}  
        (0, 0.395, 0.32, 0, 0, 0, 0.286, 0, 0), \\  
        (0, 0.395, 0.32, 0, 0, 0, 0.286, 0, 0)
        \end{array}  
        \right)$ & \\
        \hline
        Piecewise & $(0.77, 0.81, 0.93)$ & 2 asymmetric mixed: & Fig. \ref{fig:dyna-puf-slowcycles-ass} \\
        Uniform (Eq. \eqref{eq:app-pu-dist2}) &  & $\left(  
        \begin{array}{c}  
        (0, 0, 0, 0, 0.315, 0, 0.685, 0, 0), \\  
        (0, 0, 0.851, 0, 0, 0, 0.149, 0, 0)  
        \end{array}  
        \right)$ & \\
        &  & $\left(  
        \begin{array}{c}  
        (0, 0, 0.851, 0, 0, 0, 0.149, 0, 0), \\  
        (0, 0, 0, 0, 0.315, 0, 0.685, 0, 0)  
        \end{array}  
        \right)$ & \\
        &  & 1 symmetric mixed: & Fig. \ref{fig:dyna-puf-slowcycles-sym} \\
        &  & $\left(  
        \begin{array}{c}  
        (0, 0, 0.825, 0, 0, 0, 0.175, 0, 0), \\  
        (0, 0, 0.825, 0, 0, 0, 0.175, 0, 0)  
        \end{array}  
        \right)$ & \\
        \hline
    \end{tabular}
    \caption{ Piecewise Uniform distributions, $3$ interest rates, Nash equilibria of the game, and figures for dynamics of Algorithm \ref{alg:Hedge} and \ref{alg:Hedge:stoc}}
    \label{tab:3-gamma-NE-PU}
\end{table}

Below, we provide the exact distribution for our two instances of the piecewise uniform distribution. Equation~\ref{eq:app-pu-dist1} corresponds to the case where $(\gamma_l,\gamma_m,\gamma_h) = (0.54,0.61,0.7)$.

\begin{equation}
\label{eq:app-pu-dist1}
y \sim  
\begin{cases}  
0.01 \times 2.61 & y \in [0, \frac{1}{2.61}] \\  
0.95 \times \frac{1}{(\frac{1}{2.54}- \frac{1}{2.61})}, & y \in (\frac{1}{2.61}, \frac{1}{2.54}) \\  
0.04 \times \frac{1}{(1 - \frac{1}{2.54})} & y \in [\frac{1}{2.54}, 1]
\end{cases}
\end{equation}

Equation~\ref{eq:app-pu-dist2} corresponds to the case where $(\gamma_l,\gamma_m,\gamma_h) = (0.77,0.81,0.93)$:

\begin{equation}
\label{eq:app-pu-dist2}
y \sim  
\begin{cases}  
0.01 \times 2.93 & y \in [0, \frac{1}{2.93}] \\  
0.95 \times \frac{1}{(\frac{1}{2.77}- \frac{1}{2.93})}, & y \in (\frac{1}{2.93}, \frac{1}{2.77}) \\  
0.04 \times \frac{1}{(1 - \frac{1}{2.77})} & y \in [\frac{1}{2.77}, 1]
\end{cases}
\end{equation}



\begin{figure}[H]
    \centering
    \begin{subfigure}{0.49\linewidth}
        \centering
        \includegraphics[width=\linewidth]{Figures/3gamma/pu_fastcycling/Known-MatrixBank1.pdf}
    \end{subfigure}
    \begin{subfigure}{0.49\linewidth}
        \centering
        \includegraphics[width=\linewidth]{Figures/3gamma/pu_fastcycling/Fresh-EstimateBank1.pdf}
    \end{subfigure}

    \begin{subfigure}{0.49\linewidth}
        \centering
        \includegraphics[width=\linewidth]{Figures/3gamma/pu_fastcycling/Known-MatrixBank2.pdf}
    \end{subfigure}
    \begin{subfigure}{0.49\linewidth}
        \centering
        \includegraphics[width=\linewidth]{Figures/3gamma/pu_fastcycling/Fresh-EstimateBank2.pdf}
    \end{subfigure}
\caption{$y \sim$ Piecewise uniform (Eq\eqref{eq:app-pu-dist1}), Exponential weights dynamics for Bank 1 (top row) and Bank 2 (bottom row). Left: known utility matrices. Right: estimated utilities using a single sample per round. The banks are initialized at the mixed asymmetric Nash equilibrium $(1/9 \ldots 1/9)$. The dynamics cycle around the mixed NE. \label{fig:dyna-puf-fastcycles}}
\end{figure}


\begin{figure}[H]
    \centering
    \begin{subfigure}{0.49\linewidth}
        \centering
        \includegraphics[width=\linewidth]{Figures/3gamma/pu_slowcycling/ass_mne_cycling/Known-MatrixBank1.pdf}
    \end{subfigure}
    \begin{subfigure}{0.49\linewidth}
        \centering
        \includegraphics[width=\linewidth]{Figures/3gamma/pu_slowcycling/ass_mne_cycling/Fresh-EstimateBank1.pdf}
    \end{subfigure}

    \begin{subfigure}{0.49\linewidth}
        \centering
        \includegraphics[width=\linewidth]{Figures/3gamma/pu_slowcycling/ass_mne_cycling/Known-MatrixBank2.pdf}
    \end{subfigure}
    \begin{subfigure}{0.49\linewidth}
        \centering
        \includegraphics[width=\linewidth]{Figures/3gamma/pu_slowcycling/ass_mne_cycling/Fresh-EstimateBank2.pdf}
    \end{subfigure}
\caption{$y \sim$ Piecewise uniform (Eq\eqref{eq:app-pu-dist2}) Exponential weights dynamics for Bank 1 (top row) and Bank 2 (bottom row). Left: known utility matrices. Right: estimated utilities using a single sample per round, The banks are initialized at $((0.1, 0.2, 0.0, 0.1, 0.1, 0.1, 0.2, 0, 0.2), (0.2, 0, 0.2, 0, 0, 0, 0.2, 0.2, 0.2))$. The dynamics cycle around the asymmetric mixed Nash equilibrium \label{fig:dyna-puf-slowcycles-ass}}
\end{figure}

\begin{figure}[H]
    \centering
    \begin{subfigure}{0.49\linewidth}
        \centering
        \includegraphics[width=\linewidth]{Figures/3gamma/pu_slowcycling/sym_me_conv/Known-MatrixBank1.pdf}
    \end{subfigure}
    \begin{subfigure}{0.49\linewidth}
        \centering
        \includegraphics[width=\linewidth]{Figures/3gamma/pu_slowcycling/sym_me_conv/Fresh-EstimateBank1.pdf}
    \end{subfigure}

    \begin{subfigure}{0.49\linewidth}
        \centering
        \includegraphics[width=\linewidth]{Figures/3gamma/pu_slowcycling/sym_me_conv/Known-MatrixBank2.pdf}
    \end{subfigure}
    \begin{subfigure}{0.49\linewidth}
        \centering
        \includegraphics[width=\linewidth]{Figures/3gamma/pu_slowcycling/sym_me_conv/Fresh-EstimateBank2.pdf}
    \end{subfigure}
\caption{$y \sim$ Piecewise uniform (Eq\eqref{eq:app-pu-dist2}) Exponential weights dynamics for Bank 1 (top row) and Bank 2 (bottom row). Left: known utility matrices. Right: estimated utilities using a single sample per round, The banks are initialized at $(1/9 \ldots 1/9)$. The dynamics converge to the symmetric mixed Nash equilibrium \label{fig:dyna-puf-slowcycles-sym}}
\end{figure}