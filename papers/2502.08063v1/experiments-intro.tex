In the previous section, we established theoretical convergence guarantees under two settings: when the utility matrix is exactly known (Section \ref{sec:det-hedge-convergence}) and when it is estimated via samples (Section \ref{sec:stoch-hedge-convergence}). In this section, we experimentally validate these results across the four possible sign combinations of the game-dependent functionals \(\epsilon_1, \epsilon_2 \in \{+\,+, +\,-, -\,+, -\,-\}\). As shown in Theorems \ref{thm:pureNE} and \ref{thm:mixed-NE}, the signs of the functionals $(\epsilon_1,\epsilon_2)$ determine the existence of pure and mixed Nash equilibria in the game.  

We provide representative figures for the strategies of both banks under Algorithm \ref{alg:Hedge} and \ref{alg:Hedge:stoc} in Section \ref{sec:exp-dyna}, illustrating their convergence to the respective Nash equilibria. Additionally, we plot the distance of the banks' strategies to their Nash equilibrium strategies, averaged over five random initializations, in Section \ref{sec:exp-dist}. 

To analyze game equilibria and dynamics in larger action spaces, we extend our experiments to a setting with three interest rates and credit score thresholds. The results, presented in Appendix \ref{app:3gamma}, show that the key insights from the two-interest-rate setting generally hold, with similar convergence to Nash Equilibria. However, for carefully crafted piecewise uniform distributions, we observe cycling around the mixed Nash equilibrium (Appendix \ref{app:3-gamma-puf}). While rare, this phenomenon highlights that increasing the action space can, in specific cases, lead to non-convergent last iterates. 
% To analyze game equilibria and dynamics in larger action spaces, we extend our experiments to a setting with three interest rates and three credit score thresholds. The results, presented in Appendix \ref{app:3gamma}, generally show similar convergence dynamics. However, we highlight two key differences not theoretically possible in the two-interest-rate case: (i) the emergence of multiple mixed Nash equilibria, and (ii) cases where the last iterates of the exponential weights dynamics do not converge to a Nash equilibrium, instead cycling around it, as seen in Appendix \ref{app:3-gamma-puf}. This contrasts with the guaranteed convergence observed in the two-interest-rate setting, emphasizing the increased complexity of learning dynamics in larger action spaces. 









