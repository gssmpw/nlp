\section{Related work}
\heading{Signal processing} The majority of studies investigating bowel sounds primarily focus on extracting and enhancing the BS signal using advanced signal processing techniques. Examples include the Wavelet Transform-Based Stationary-Non-Stationary Filter \cite{hadjileontiadis1999,liatsos2003}, Short Time Fourier Transformation \cite{ficek2021}, and Fractal Dimension \cite{dimoulas2007}. Similarly, machine learning techniques, such as the use of jitter and shimmer parameters \cite{kim2011}, neural networks \cite{yin2015,kim22011}, and support vector machines \cite{yin2018}, have been employed for BS signal processing and analysis.

\heading{BS identification and classification}While these methods provide robust tools for signal detection and enhancement, only a limited number of studies have focused on identifying and classifying BS patterns. The early work by Dimoulas et al \cite{dimoulas2003} introduced abdominal sound pattern analysis (ASPA), combining BS time-energy alterations with electrical or pressure signals. This study marked the first step toward understanding BS patterns in the context of gastrointestinal function.

Building on these foundations, further research aimed to autonomously classify BS patterns. For instance, one study employed a wavelet neural network to differentiate between two BS patterns and three types of interfering noises \cite{dimoulas2007}. Another significant study analyzed two hours of BS recordings from ten participants, identifying five distinct BS types based on waveform and spectrogram characteristics. This study also investigated inter-subject variations in the duration and proportion of these patterns \cite{du2018}. Similarly, research involving 1,140 BS recordings from 15 volunteers identified four unique BS patterns \cite{Zhang2024}.

More recently, advancements in deep learning have facilitated new approaches to BS analysis. A convolutional neural network (CNN)-based detector was developed to identify four types of BS and to investigate the acoustic effects of food consumption. This study proposed that total BS duration increases post-consumption, highlighting the relevance of studying BS patterns in relation to the digestion cycle \cite{wang2022}.
Pre-trained models have been introduced for the first time in BS identification by \cite{Baronetto2023}, where different pre-trained DNNs have been developed for BS event spotting. To the best of our knowledge, no prior work has examined transfer learning strategies for BS pattern classification.