\section{Results}
\label{sec:Results}

In this section, we present the main results of the proposed control framework in both high-fidelity simulation and on small-size humanoid hardware. 

\subsection{Validation Setup}
We perform the simulation validations through a custom MATLAB/Simulink-based simulator with Simscap Multi-body Library. The robot model has two variants: the 2-D 5-link bipedal robot consists of 7 degree-of-freedom (DoF) and the 3-D humanoid robot is custom-made with 24 DoF, 5 actuated joints on each leg, and 4 actuated joints on each arm. Both robots are miniature-sized and have a total leg length of $44 \unit{cm}$. 

The simulations are performed on a Ubuntu 20.04 machine with an AMD Ryzen 7900X CPU. The optimization problems are solved through CasADi \cite{Andersson2019} MATLAB interface with \texttt{sqpmethod} solver for SQP problems, and \texttt{qpoases} solver \cite{Ferreau2014} for QP subproblems. The humanoid hardware equips Intel NUC i5 mini-PC, which solves the sequential CMPCs in real-time with the \texttt{qpoases} solver in C++. The Gait-Net is lightweight and its inference is done with CPU computation.

\subsection{Numerical Analysis of Gait-Net}
We first compare the performance of the trained Gait-Net with (1) full feature space provided by simulation data, meaning the input space consists of the full CoM states (position $\bm p_c$, velocity $\dot{\bm p}_c$, Euler angles $\bm \Theta$, and angular velocity $\bm \omega$) and desired foot locations, totaling $\mathbb R^{16}$; (2) reduced feature space after performing PCA and selecting top features from 6 principle axes, shown in Fig. \ref{fig:PCA}. Table. \ref{tab:NetsolveTime} compares the normalized inference time of the Gait-Net between the two methods and their RMSE in predicting the MPC sampling time $dt$ with test sets. 

\begin{table}[h]
\setlength\extrarowheight{2pt}
\setlength{\tabcolsep}{3.5pt}
\vspace{0cm}
    \centering
    \caption{Gait-Net Performance Comparisons}
    \begin{tabular}{ | m{12em} | m{8em} | m{8em} | }
    \hline \hline
	 \makecell[c]{Feature Space}  & \makecell[c]{Full (baseline)} & \makecell[c]{\textbf{PCA Selection}}  \\
    \hline \hline
    Feature Size & \makecell[c]{$\mathbb {R}^{16}$}  & \makecell[c]{$\mathbb {R}^{6}$} \\ \hline
    Normalized Inference Time  & \makecell[c]{\underline{1.00}} & \makecell[c]{\textbf{0.74}} \\ \hline
    Prediction RMSE & \makecell[c]{$4.98\mathrm{e}{-3}$} & \makecell[c]{$6.79\mathrm{e}{-3}$}   \\
    \hline \hline
    \end{tabular}
    \label{tab:NetsolveTime}
    \vspace{-0.2cm}
\end{table}

\subsection{Comparison of Kino-dynamic MPC Solving Methods}
Next, we present a comparison of our proposed kino-dynamic MPC and its solving method along with a few other state-of-the-art methods, shown in Table. \ref{tab:mpcsolveTime}.

%%%%%%%%%%%%% solve time comparison  %%%%%%%%%%%%%%%%%%
\begin{table}[t]
\vspace{0.2cm}
\centering
\setlength\extrarowheight{1pt}
\setlength{\tabcolsep}{2.5pt}
\caption{Comparison with Solving Approaches (3-D)}
\label{tab:mpcsolveTime}
\begin{scriptsize}
  \begin{tabular}{ | m{6.5em} | m{3.5em}  | m{3.5em} | m{4.5em} | m{4.5em} | m{4.25em} | } 
  \hline
  \hline
  \makecell[c]{Method} & \multicolumn{3}{ c |}{SQP} & \makecell[c]{AD(3)} & \makecell[c]{\bf{Proposed}} \\ 
  \hline
  \hline
  MPC Model & \makecell[c]{WBD} &  \makecell[c]{Explicit\\KD} & \multicolumn{2}{ c |}{CD}& \makecell[c]{Implicit\\KD} \\
  \hline
  Optimization Variable Size & \makecell[c]{$\mathbb {R}^{78h}$} & \makecell[c]  {$\mathbb {R}^{60h}$} & \makecell[c]{$\mathbb {R}^{34h-22}$} & \makecell[c]{$\mathbb {R}^{27h-15}$ \\ $\mathbb {R}^{21h-9}$ \\$\mathbb {R}^{16h-4}$} &  \makecell[c]{$\mathbb {R}^{33h-21}$}\\ 
  \hline
  Normalized Solve Time & \makecell[c]{12.05} & \makecell[c]{7.28} & \makecell[c]{\underline{1.00}} & \makecell[c]{2.58} &  \makecell[c]{\textbf{0.76}}\\ 
  \hline
   Average Number of QPs & \makecell[c]{18.2} & \makecell[c]{17.7} & \makecell[c]{19.5} & \makecell[c]{27.5} &  \makecell[c]{\textbf{17.1}}\\ 
  \hline
  \hline
\end{tabular}
\end{scriptsize}
\vspace{0.1cm}
\begin{flushleft}
\footnotesize{\scriptsize{ \quad WBD - \textit{Whole-body dynamics}; \: CD - \textit{Centroidal dynamics}; \: KD - \textit{Kino-dynamics}.}}    
\end{flushleft}
\vspace{-0.2cm}
\end{table}

 The compared whole-body MPC (Sec. \ref{subsec:wbmpc}) and explicit kino-dynamic MPC (Section. \ref{subsec:cdmpc}) consider MPC sampling time as additional optimization variables.
 % The explicit kino-dynamic MPC is presented in Section. \ref{subsec:cdmpc}.
 The average solve time of each solving method is normalized by the baseline (CD-MPC solved with SQP). The average number of QPs refers to the QP (sequential) subproblems performed until convergence. These values are measured by running 1000 time steps of each solving mechanism. The alternating direction (AD) approach follows a similar algorithm in \cite{meduri2023biconmp} but with 3 alternating directions (\textit{a.k.a}., mountain-climbing method \cite{konno1976cutting}). All problems are solved until the same termination threshold is reached in this comparison. 


\begin{figure}[!t]
\vspace{0.2cm}
    \center
    \includegraphics[clip, trim=4.5cm 9cm 4.8cm 9cm, width=1\columnwidth]{figures/step_duration.pdf}
    \caption{{\bfseries Variable Step Durations under Unknown Disturbances.} Hardware experiment of (1) Push-recovery, Fig. \ref{fig:title}(a); (2) Carrying unknown load, Fig. \ref{fig:title}(b); (3) Locomotion over uneven terrain, Fig. \ref{fig:title}(c).}
    \label{fig:disturbance}
    \vspace{-0.2cm}
\end{figure}

Whole-body MPC is expected to be the most time-consuming variant when solved to full convergence, this aligns with the motivation for adopting a lighter kino-dynamic model for more efficient deployment while maintaining a reasonable trade-off in model fidelity. In CD-MPC, the AD approach requires numerous iterations to solve the 3 separate subproblems, resulting in a longer overall solve time. In contrast, our proposed method is more efficiently designed, achieving a faster convergence rate compared to the baseline approaches.

Then, we validate the significance of Gait-Net to determine the step duration (interpreted as MPC $dt$) within the sequential solving mechanism, rather than solving it as a part of the optimization. We compare locomotion performance in 2-D numerical simulation with a 5-link bipedal robot and discrete terrain gaps in the range of $[5,\;15]$ cm. Fig. \ref{fig:comp1} compares snapshots of our proposed kino-dynamic MPC with (1) fixed gait frequency, (2) variable gait frequency by solving step duration as an additional optimization variable, and (3) variable gait frequency by the proposed Gait-Net. With fixed step duration, the periodic gait is un-optimized especially near the discrete terrain boundary. Solving step duration in optimization is also not optimal, as the MPC $dt$ does not have a strong physical correlation with the step location, resulting in random/inconsistent solutions that are unintuitive to interpret in motion. With the proposed method, the step duration is highly dependent on the robot state and next foot location through Gait-Net, rendering more natural and whole-body-kinematic-aware walking behavior.

\begin{figure}[!t]
\vspace{0.2cm}
     \centering
     \begin{subfigure}[b]{0.5\textwidth}
         \centering
	   \includegraphics[clip, trim=0cm 7cm 8.7cm 0cm, width=1\columnwidth]{figures/pushing_comparison_snaps.pdf}
          \caption{Simulation Snapshots. Green and blue dashed lines represent the start and final location of the cart.}
          \vspace{0.2cm}
          \label{fig:comppush1}
     \end{subfigure}
     \begin{subfigure}[b]{0.475\textwidth}
        \includegraphics[clip, trim=4.5cm 8.9cm 4.5cm 8.7cm, width=1\columnwidth]{figures/pushing_comparison.pdf}
	\caption{CoM X position comparison and joint torque plots.}
        \vspace{0.2cm}
	\label{fig:comppush2}
     \end{subfigure}
     \caption{{\bfseries Comparison of Approaches in Pushing Carts with Unknown Payload.} The time instances of the robot hand's contact with the cart are synchronized in the comparison plot to enhance visualization.}
    \label{fig:compare_pushing}
    \vspace{-0.2cm}
\end{figure}

\subsection{3-D Locomotion over Discrete Terrain}
\label{subsec:discrete_terrain}

We showcase the capability of the proposed method in the 3-D environment with a line-foot humanoid robot. In simulation, we construct a stepping-stone course for the robot to navigate through with the proposed method. Given a forward velocity command, $\dot {\bm p}^\text{ref}_{c,x} = 0.75$ m/s, the control goal is to reach the end of the course while avoiding stepping outside of the stepping stone patches, illustrated in Fig. \ref{fig:footlocation}. Note that the green patches allow both feet to step on, the red patch and the blue patch only allow the corresponding colored foot to step on. The center locations of the feet are plotted in Fig. \ref{fig:footlocation}, demonstrating the capability of our variable-frequency walking method in traversing challenging terrains dynamically and the satisfaction of foot location constraints with the proposed method.

Associated plots of spatial momenta are also provided in Fig. \ref{fig:h_tracking}, where the scatter data represent the reference trajectories from MPC at the beginning of each footstep for better visualization purposes. The plots demonstrate a good prediction of spatial momenta evolution with the proposed Gait-Net-augmented sequential method. 

% \remark{By accurately predicting the evolution of spatial momentum and centroidal pose, the design of these trajectories can implicitly impose kinematic assurance in the MPC problem, ensuring that the resulting trajectories remain kinematically feasible. }


\begin{figure}[!t]
\vspace{0.2cm}
    \center
    \includegraphics[clip, trim=0cm 2cm 0cm 0cm, width=0.95\columnwidth]{figures/vx_tracking.pdf}
    \caption{{\bfseries Velocity X Tracking Plot.} In discrete terrain locomotion hardware experiment with a 20-cm gap. }
    \label{fig:vx}
    \vspace{-0.2cm}
\end{figure}

\subsection{Hardware Experiment Validation}
We conduct hardware experiments on a 3-D small-size humanoid robot with 5-DoF legs. The hardware experiment snapshots are presented in Fig. \ref{fig:title}. The readers are encouraged to watch the supplementary video for better visual aids. 

\subsubsection{Locomotion under Unknown Perturbations}
First, we demonstrate the robustness of the proposed control framework in handling unknown perturbations, including push-recovery, handling unknown payload, and uneven terrain locomotion, as illustrated in Fig. \ref{fig:title}(a-c). The controller dynamically adjusts step duration at each step to maintain balance while handling these unknown disturbances. Fig. \ref{fig:disturbance} presents the variable step durations from Gait-Net-augmented MPC. We clip the MPC $dt$ predictions to be within the bound of $[0.045,\:0.07]$ s to ensure hardware feasibility. On relatively flat terrain with minimal perturbations, the MPC optimizes $dt$ to the nominal value of 0.045 s, equivalent to a 0.225 s step duration. While under disturbances, the step durations are adjusted. Notably, foot positions are not the primary factors in predicting the MPC step duration; instead, the Gait-Net is trained to incorporate additional deterministic state features for improved adaptation.

\subsubsection{Baseline Comparison in Loco-manipulation}

In this comparison, the robot is tasked with negotiating and pushing a cart of a 35-kg unknown load (219$\%$ of the robot's mass) using two approaches: (1) SRBM MPC with a fixed gait frequency and (2) the proposed Gait-Net-augmented kino-dynamic MPC. Snapshots of the experiment and CoM position comparisons are presented in Fig. \ref{fig:compare_pushing}.
The results show that the SRBM approach struggles to handle such a large unknown load, whereas the proposed method successfully manages the disturbance while maintaining sufficient joint torque headroom for even more dynamic motions, as shown in Fig. \ref{fig:comppush2}. However, the observed performance difference is not solely due to gait frequency adaptation but rather benefited from the integration of the proposed techniques.


\subsubsection{Discrete Terrain Locomotion}

Following the validation objectives outlined in Sec. \ref{subsec:discrete_terrain}, we conduct dynamic locomotion experiments on discrete terrains with hardware. While the robot has prior knowledge of the terrain map, terrain discontinuity information is provided only as one-step preview.

We first present a locomotion experiment on flat ground featuring a 20-cm-wide virtual gap, marked by red tape, with a commanded velocity of 0.75 \unit{m/s}. Snapshots of this experiment are shown in Fig. \ref{fig:title}(c), and the velocity tracking performance is shown in Fig. \ref{fig:vx}. It is important to note that foot placement is defined by the center of the foot, allowing a small portion of the toe or heel to step into the red-taped area. This is feasible because the CWC constraint adapts to ensure compliance with the line-foot constraint along the edges, as long as the center of the foot is within the safe area. 

Next, we increase the complexity of the discrete terrain course by introducing a 10-cm-wide 3-cm-high obstacle, and a 15-cm-wide virtual gap, marked by red tape. However, the obstacle's height is unknown to the robot. Snapshots of this experiment are presented in Fig. \ref{fig:title}(d).
An interesting observation is that while traversing the obstacle, the robot deliberately leans to its right, allowing the right foot to step on flat ground while the left leg is optimized for a slightly longer step duration to clear the obstacle successfully.
