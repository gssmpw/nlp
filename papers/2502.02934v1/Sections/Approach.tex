\section{Proposed Approach}
\label{sec:approach}

In this section, we introduce the proposed approaches in this work, including the Gait-frequency Network, the Gait-Net-augmented 
kino-dynamics MPC formulation, and the sequential MPC solving mechanism.


\subsection{Gait-frequency Network}
\label{subsec:gaitnet}

Instead of relying on heuristics or solving an additional optimization problem to determine step frequency from the desired foot location, we propose a lightweight Gait-frequency Network that maps current state feedback and desired foot placement to the preferred step duration for the upcoming stride, seamlessly integrating with the MPC framework while co-optimizing the target variables. An illustration of the Gait-Net is shown in Fig.  \ref{fig:gaitnet}. 

\subsubsection{Data collection}
We begin by using a whole-body MPC as our baseline controller to collect variable-frequency walking data. Besides a more accurate representation of kinematics and dynamics of the robot model, this approach is chosen for two key reasons: first, whole-body MPC outperforms other simplified-model-based control methods under strong unknown disturbances \cite{dantec2024centroidal}. Second, we utilize a MATLAB-based high-fidelity simulation framework, bypassing real-time hardware constraints. This enables the use of a more precise and robust interior-point-method-based nonlinear programming solver at higher control frequencies. The primary objective of this stage is to gather high-resolution data using a robust control algorithm under controlled disturbances in simulation.

We ran 15 different sets of walking simulations that have a total period of 600 seconds of walking data with a small-size humanoid model in the simulation. In each set, we command the robot to walk at a constant velocity in the range of $[0,\:1]$ \unit{m/s}. At the beginning of each stride, we generate a randomized step duration between $[150,\:400]$ \unit{ms}. In the first half of the walking simulation, the robot is commanded to walk without any disturbances, while in the second half, we apply randomized external impulses to the CoM of the robot every 2 seconds with the magnitude of $[10,\:100]$ \unit{N} for a duration of 0.2 seconds. 

Despite the relatively small training dataset, the Gait-Net and MPC combination effectively handles various scenarios with predicted step durations, as demonstrated in Sec. \ref{sec:Results}.

\subsubsection{Latent space feature reduction}
The collected data initially includes the robot's floating-base state variables and world-frame foot locations for both legs ($\mathbb R^{16}$). However, not all features/states are equally deterministic in the CoM space to determine the output of the network. To streamline the feature space for more efficient inference in each iteration while maintaining accurate predictions of variable MPC sampling time, we apply Principal Component Analysis (PCA) to reduce the input features. Specifically, we select one feature with a high absolute loading from each of the top six principal axes, as these features are minimally correlated and capture the most significant variance in the feature space. Fig.  \ref{fig:PCA} illustrates the PCA loadings of all features across six principal axes.

%%%%%%%%%%%%%%%%%%%%%%%%%%%%%%%%%%%%%%%%%%%
\subsection{Gait-Net-augmented Kino-dynamic MPC}
\label{subsec:gaitnetmpc}

\subsubsection{Motivation}
Given a fixed periodic contact sequence in MPC, one can vary the duration of each swing phase by altering the MPC sampling time $dt$ for the entire swing duration of each footstep made up of $h'$ MPC time steps. Hence the swing time $\Delta t = h'dt$. To achieve a variable-frequency walking MPC, we need to optimize the contact wrenches, contact locations, and MPC $dt$ all together.
In the discrete-time CD at time step $k$, 
\begin{equation}
\begin{aligned}
\label{eq:cd_discrete}
    \begin{bmatrix}
        \bm l_{G,k+1} \\ \bm k_{G,k+1}
    \end{bmatrix}
     = \begin{bmatrix}
        \bm l_{G,k} \\ \bm k_{G,k}
    \end{bmatrix} +
    \left[\begin{array}{c} 
    \sum_{i = 0}^{n_i}\bm f_i \\
    \sum_{i = 0}^{n_i}(\bm p_{f,i}-\bm p_c) \times \bm f_i + \bm \tau_i
    \end{array} \right]dt_k.
\end{aligned}
\end{equation}
Nonlinearity arises from the bilinear and multi-linear terms formed by the (cross) products of the three optimization variables.

In this section, we introduce a novel NN-augmented solving mechanism inspired by the popular SQP approach, allowing the MPC $dt$ to be concurrently determined by the Gait-Net alongside the optimization of other variables. 

\subsubsection{Implicit kinematics assurance in trajectory reference}
By selecting spatial momenta $\bm h$ and their primitive $\bm H$ (centroidal pose) as optimization variables, CD-MPC avoids including joint angles in the optimization process. However, generating these spatial reference trajectories still requires a corresponding whole-body joint space trajectory, which is a significant part of ensuring the kinematics feasibility of the optimization results. 
\begin{align}
    \bm h^\text{ref}_k = \bm A_G(\mathbf q^\text{ref}_k)\dot{\mathbf q}^\text{ref}_k,
\end{align}
\begin{equation}
\begin{aligned}
        \bm H^\text{ref}_k
        \approx \bm A_G(\mathbf q^\text{ref}_k)\mathbf q^\text{ref}_k - \sum^{k-1}_{i = 0} \dot{\bm A}_G(\mathbf q^\text{ref}_i,\dot{\mathbf q}^\text{ref}_i)\mathbf q^\text{ref}_i\:dt.
\end{aligned}
\end{equation}

Additionally, after obtaining the local foot location and CoM position trajectory solutions in each iteration, we perform a fast analytical inverse kinematics (IK) $f_\text{IK}$ to compute the corresponding joint trajectories for spatial momenta updates. The complete analytical IK and the approximation of $\bm H$ are provided in the Appendix. 

\remark{The spatial momentum and centroidal pose reference trajectories are updated in each sequential iteration to align with foot position updates, implicitly enforcing kinematic consistency in the reference trajectory. }


\subsubsection{Convex MPC Subproblem}
\label{subsubsec:cmpc}
To address the weak correlation between foot location and swing duration in the CD formulation, we integrate the proposed Gait-Net (Sec. \ref{subsec:gaitnet}) to predict the MPC sampling time at the beginning of each step based on the local foot and CoM location solutions at each sequential iteration. This process continues until the convergence condition is met. This also translates $dt$ from the optimization variable space to the parameter space for computation efficiency. 

\begin{figure}[!t]
\vspace{0.2cm}
    \center
    \includegraphics[clip, trim=4.5cm 12cm 4.1cm 12cm, width=1\columnwidth]{figures/bilinear_surf.pdf}
    \caption{{\bfseries Bilinear Envelope Approximation by Neglecting Search Direction Product $\delta a\cdot\delta b$.}}
    \label{fig:bisurf}
    \vspace{-0.2cm}
\end{figure}

To linearize the bilinearly constrained dynamics constraint (\ref{eq:cd_discrete}) in the kino-dynamics MPC problem. We take inspiration from the SQP solving mechanism and solve the search direction $\delta$ of the bilinear variables, 
\begin{align}
    \bm f_i = \bm f^j_i +  \delta \bm f_i, \:\: \bm \tau_i = \bm \tau^j_i +  \delta \bm \tau_i, \nonumber\\
    \bm p_{f,i} = \bm p^j_{f,i} +\delta \bm p_{f,i}, \:\: \bm p_{c} = \bm p^j_{c} +\delta \bm p_{c},
\end{align}
where the $j$ superscript denotes the total solution from the previous sequential iteration $j$. 
The dynamics evolution of the angular momentum can  be simplified with the assumption:
\begin{assumption}
    \textit{With reasonable warm-start/initialization of the bilinear variables $a_0,\: b_0$, the bilinear product of the search directions are minimal and negligible as the sequential iteration $j$ increases, $\delta a^j \cdot \delta b^j \approx 0$, illustrated in Fig.  \ref{fig:bisurf}.} 
    \label{assump2}
\end{assumption}

Therefore, at iteration $j+1$, 
\begin{align}
\label{eq:lG}
    \bm l_{G,k+1} & = \bm l_{G,k} + \sum_{i = 0}^{n_i}\bm f^j_i dt_k +  \delta \bm f_i dt_k
\end{align}
\begin{equation}
\allowdisplaybreaks
\begin{aligned}
    \label{eq:kG}
    \bm k_{G,k+1} & = \bm k_{G,k} + \sum_{i = 0}^{n_i}(\bm p^j_{f,i} - \bm p^j_c+\delta \bm p_{f,i}-\delta \bm p_c) \\ & \quad 
    \times (\bm f^j_i +  \delta \bm f_i)dt_k  + (\bm \tau^j_i + \delta \bm \tau_i)dt_k \\
    & = \bm k_{G,k} + \sum_{i = 0}^{n_i} \delta \bm \tau_i dt_k  -\bm f^j_i \times(\delta \bm p_{f,i}-\delta \bm p_c) dt_k \\ & \quad  
    +(\bm p^j_{f,i} - \bm p^j_c)\times \delta \bm f_i dt_k + 
    \underbrace{(\delta \bm p_{f,i}-\delta \bm p_c) \times \delta \bm f_i dt_k}_{\approx 0} \\ & \quad
     + \underbrace{(\bm p^j_{f,i} - \bm p^j_c)\times \bm f^j_i dt_k + \bm \tau^j_i dt_k}_\text{const.} 
\end{aligned}
\end{equation}
\remark{The linearization w.r.t. search directions $\delta$ and with \textit{Assumption} \ref{assump2} arrives mathematically identical to 1st-order Taylor expansion of bilinear constraints with the SQP algorithm \cite{boggs1995sequential}. }


\paragraph{Optimization Variables}
We choose the state variable to be the spatial momentum vector $\bm h$ and centroidal pose $\bm H$, and the control input variables to be the search directions of ground contact wrench, foot location, and MPC sampling time over a finite horizon $h$,
\begin{align}
    \bm {x} = \bigl\{ \bm H_{k};\: {\bm h}_{k}\bigr\}^{h}_{k=0}, \quad\quad\quad\quad\quad\\
    \bm {u} = \bigl\{\{\delta \bm f_{i,k},\: \delta \bm \tau_{i,k},\: \delta \bm p_{f,i,k}\}^{n_i}_{i=0},\:\delta \bm p_{c} \bigr\}^{h-1}_{k=0},
\end{align}

\paragraph{Objective Function}
The linearized finite horizon optimization problem can be formulated as 
\begin{alignat}{3}
    \label{eq:CDMPCcost2}
    \underset{\bm x^j, \bm {u}^j}{\text{min}} \: & \sum_{k = 0}^{h-1} \Bigg\| 
    \begin{bmatrix}
        \bm h_{k} \\ \bm H_{k} \\ \bm p^{j-1}_{f,k} +\delta \bm p_{f,k} \\ \bm p^{j-1}_{c,k} +\delta \bm p_{c,k}
    \end{bmatrix} -
    \begin{bmatrix}
        \bm h_{k}^{\text{ref}} \\ \bm H_{k}^{\text{ref}} \\ \bm p_{f,k}^{\text{ref}}\\ \bm p_{c,k}^{\text{ref}}
    \end{bmatrix}
    \Bigg\|^2 _{\bm L_1} \\ \nonumber & \quad +
    \Bigg\| \begin{bmatrix} 
    \bm f^{j-1}_i + \delta \bm f_i \\ \bm \tau^{j-1}_i  + \delta \bm \tau_i 
    \end{bmatrix} \Bigg\|^2 _{\bm L_2}
\end{alignat}
The objectives are to track spatial trajectories, foot reference trajectory, and CoM position trajectory while minimizing ground reaction wrenches. Note that the foot reference is constructed based on a preferred foot location through Gait-Net with a nominal MPC $dt$. This trajectory will be updated with each sequential iteration $j$ until convergence.
% \subsubsection{Constraints}
\paragraph{Linearized dynamics}
With \textit{Assumption} \ref{assump2} and search direction linearization in equation (\ref{eq:lG}-\ref{eq:kG}), we obtain the CD as a linear form with respect to our choice of control input and state variables. At time step $k$, the discrete dynamics is 
\begin{align}
\label{eq:cdlinear}
    \bm x_{k+1} = \mathbf A_k \bm x_{k} + \mathbf B_k \bm u_k + \mathbf C_k,
\end{align}
where $\mathbf A_k$ and $\mathbf B_k$ are the state space matrices. $\mathbf C_k$ contains only the constant terms shown in equation (\ref{eq:lG}-\ref{eq:kG}) in terms of optimization results obtained from the last sequential iteration $j-1$. By including a constant $1$ at the end of the state variables, $\bm x'_{k} = [\bm x_{k};\:1]$, we can then obtain the discrete dynamics in a linearized state-space form, as similarly outlined in \cite{di2018dynamic,chen2024learning}.

\paragraph{Linear momentum and CoM position}
The dynamics evolution of CoM position $\bm p_c$ is embedded in the linear momentum equation as an equality constraint:
\begin{align}
    \bm p^{j-1}_{c,k+1} +\delta \bm p_{c,k+1}  = \bm p^{j-1}_{c,k} +\delta \bm p_{c,k} + \frac{\bm l_{G,k}}{m}\:dt_k.
\end{align}

\paragraph{Friction pyramid}
The friction pyramid is a conservative approximation of the friction cone when the pyramid is inscribed, such that the pyramid friction coefficient is then approximated as 
\begin{align}
    \mu_{\Box} = \frac{\sqrt{2}}{2}\mu,
\end{align}
where $\mu$ is the actual ground friction coefficient. The linear inequality constraint is
\begin{align}
    \nonumber 
    -\mu_{\Box}  ({f}_{i,z}^{j-1}+\delta f_{i,z}) \leq  \big\{({f}_{i,x}^{j-1}+\delta f_{i,x}), 
    \: ({f}_{i,y}^{j-1}+\delta f_{i,y})\bigr\} \\
    \leq \mu_{\Box}  ({f}_{i,x}^{j-1}+\delta f_{i,x}) \quad \quad
    \label{eq:frictionCons}
\end{align}

\paragraph{Contact-switching Constraints}
For periodic walking gait, we use a binary contact schedule $\bm \sigma \in \mathbb R^{n_i\times h}$ to describe the contact switches for each contact point $i$. Hence, the ground reaction wrench can be switched on or off based on whether a leg is in swing or stance, 
\begin{align}
    \label{eq:contactConstraint1}
     \sigma_{i,k}
    \begin{bmatrix}
        \bm f_\text{min} \\ \bm \tau_\text{min} 
    \end{bmatrix} \leq
    \begin{bmatrix}
        \bm f^{j-1}_i + \delta \bm f_i \\ \bm \tau^{j-1}_i  + \delta \bm \tau_i
    \end{bmatrix} \leq 
    \sigma_{i,k}
    \begin{bmatrix}
        \bm f_\text{max} \\ \bm \tau_\text{max} 
    \end{bmatrix}  
\end{align}
In addition to the control input saturation, we also enforce stationary foot location for each footstep while the contact schedule $\sigma_{i,k} = 1$ for contact point $i$.

\paragraph{Dynamic Range of Foot location}
With the future foot location as part of the optimization variables, the constraints can be directly applied with only one-step preview data in discrete terrain - the bounds of the foot location are dependent on the position of the robot,  
\begin{align}
    \bm p_{f,\text{min}}(\bm p_c) \leq \bm p^{j-1}_{f,k} +\delta \bm p_{f,k} \leq \bm p_{f,\text{max}}(\bm p_c)
    \label{eq:KDfoot}
\end{align}
A visualization of the constraint-triggering conditions is illustrated in Fig. \ref{fig:footlocation}. 


\section{\thename}
\subsection{End-to-End Driving Policy}
The overall framework of \thename{} is depicted in Fig.~\ref{fig:framework}. 
\thename{} takes multi-view image sequences as input, transforms the sensor data into scene token embeddings, outputs the probabilistic distribution of actions, and samples an action to control the vehicle. 

\boldparagraph{BEV Encoder.} 
We first employ a BEV encoder~\cite{li2022bevformer} to transform multi-view image features from the perspective view to the Bird's Eye View (BEV), obtaining a feature map in the BEV space. This feature map is then used to learn instance-level map features and agent features.

\boldparagraph{Map Head.} 
Then we utilize a group of map tokens~\cite{maptrv2, liao2022maptr, lanegap} to learn the vectorized map elements of the driving scene from the BEV feature map, including lane centerlines, lane dividers, road boundaries, arrows, traffic signals, \etc.

\boldparagraph{Agent Head.} 
Besides, a group of agent tokens~\cite{jiang2022pip} is adopted to predict the motion information of other traffic participants, including location, orientation, size, speed, and multi-mode future trajectories.

\boldparagraph{Image Encoder.} 
Apart from the above instance-level map and agent tokens, we also use an individual image encoder~\cite{vit,he2016resnet} to transform the original images into image tokens. These image tokens provide dense and rich scene information for planning, complementary to the instance-level tokens.

\begin{figure}[t]
\centering
\includegraphics[width=0.98\linewidth]{fig/post-training-2.pdf} 
\caption{\textbf{Post-training.}  $N$  workers parallelly run. The generated rollout data $(s_t,a_t, r_{t+1},s_{t+1},...)$ are recorded in a rollout buffer. Rollout data and human driving demonstrations are used in RL- and IL-training steps to fine-tune the AD policy synergistically.
}
\label{fig:post-training}
\end{figure}

\boldparagraph{Action Space.} 
To accelerate the convergence of RL training, we design a decoupled discrete action representation. 
We divide the action into two independent components: lateral action and longitudinal action. 
The action space is constructed over a short $0.5$-second time horizon, during which the vehicle's motion is approximated by assuming constant linear and angular velocities. 
Under this assumption, the lateral action $a^x$ and longitudinal action $a^y$ can be directly computed based on the current linear and angular velocities.
By combining decoupling with a limited temporal scope and simplified motion model, our approach effectively reduces the dimensionality of the action space, accelerating training convergence.


\boldparagraph{Planning Head.} 
We use $E_\text{scene}$ to denote the scene representation, which consists of map tokens, agent tokens, and image tokens. We initialize a planning embedding denoted as $E_\text{plan}$. A cascaded Transformer decoder $\phi$ takes the planning embedding $E_\text{plan}$ as the query and the scene representation $E_\text{scene}$ as both key and value.

The output of the decoder $\phi$ is then combined with navigation information $E_\text{navi}$ and ego state $E_\text{state}$ to output the probabilistic distributions of the lateral action $a^x$ and the longitudinal action $a^y$:
\begin{equation}
\begin{aligned}
     \pi(a^x\mid s) = & \text{softmax}(\text{MLP}(\phi(E_\text{plan}, E_\text{scene}) \\
    & + E_\text{navi} + E_\text{state})), \\
     \pi(a^y\mid s) = & \text{softmax}(\text{MLP}(\phi(E_\text{plan}, E_\text{scene}) \\
     & + E_\text{navi} + E_\text{state})),
\label{eq:action distribution}
\end{aligned}
\end{equation}
where $E_\text{plan}$, $E_\text{navi}$, $E_\text{state}$, and the output of $\text{MLP}$ are all of the same dimension ($1 \times D$).

The planning head also outputs the value functions $V_x(s)$ and $V_y(s)$, which estimate the expected cumulative rewards for the lateral and longitudinal actions, respectively: 
\begin{equation}
\begin{aligned}
    & V_x(s) = \text{MLP}(\phi(E_\text{plan}, E_\text{scene}) + E_\text{navi} + E_\text{state}), \\
    & V_y(s) = \text{MLP}(\phi(E_\text{plan}, E_\text{scene}) + E_\text{navi} + E_\text{state}).
\end{aligned}
\end{equation}
The value functions are used in RL training (Sec.~\ref{sec:optimization}).

\subsection{Training Paradigm}
We adopt a three-stage training paradigm: perception pre-training, planning pre-training, and reinforced post-training, as shown in Fig.~\ref{fig:framework}.

\boldparagraph{Perception Pre-Training.} 
Information in the image is sparse and low-level. In the first stage,  
the map head and the agent head explicitly output map elements and agent motion information, which are supervised with ground-truth labels. Consequently,  
map tokens and agent tokens implicitly encode the corresponding high-level information.  
In this stage, we only update the parameters of the BEV encoder, the map head, and the agent head.



\boldparagraph{Planning Pre-Training.} 
In the second stage, to prevent the unstable cold start of RL training, IL is first performed to initialize the probabilistic distribution of actions based on large-scale real-world driving demonstrations from expert drivers. In this stage, we only update the parameters of the image encoder and the planning head, while the parameters of the BEV encoder, map head, and agent head are frozen. The optimization objectives of perception tasks and planning tasks may conflict with each other. However, with the training stage and parameters decoupled, such conflicts are mostly avoided.

\boldparagraph{Reinforced Post-Training.} 
In the reinforced post-training, RL and IL synergistically fine-tune the distribution. RL aims to guide the policy to be sensitive to critical risky events and adaptive to out-of-distribution situations. IL serves as the regularization term to keep the policy's behavior similar to that of humans.

We select a large amount of risky dense-traffic clips from collected driving demonstrations. For each clip, we train an independent 3DGS model that reconstructs the clip and serves as a digital driving environment.  
As shown in Fig.~\ref{fig:post-training}, we set $N$ parallel workers.  
Each worker randomly samples a 3DGS environment and begins rollout, i.e., the AD policy controls the ego vehicle to move and iteratively interacts with the 3DGS environment. After the rollout process of this 3DGS environment ends, the generated rollout data $(s_t,a_t, r_{t+1},s_{t+1},...)$ are recorded in a rollout buffer, and the worker will sample a new 3DGS environment for another round of rollout.

As for policy optimization, we iteratively perform RL-training steps and IL-training steps. For RL-training steps, we sample data from the rollout buffer and follow the Proximal Policy Optimization (PPO) framework~\cite{PPO} to update the AD policy. For IL-training steps, we use real-world driving demonstrations to update the policy. After a fixed number of training steps, the updated AD policy is sent to every worker to replace the old one, to avoid a distribution shift between data collection and optimization.
We only update the parameters of the image encoder and the planning head. The parameters of the BEV encoder, the map head, and the agent head are frozen.  
The detailed RL design is presented below.

\subsection{Interaction Mechanism between AD Policy and 3DGS Environment}
In the 3DGS environment, the ego vehicle acts according to the AD policy. Other traffic participants act according to real-world data in a log-replay manner.  
A simplified kinematic bicycle model is employed to iteratively update the ego vehicle's pose at every $\Delta t$ seconds as follows:  
\begin{equation}
\begin{aligned}
x_{t+1}^{w} & = x_{t}^w + v_t \cos \left(\psi_{t}^w\right) \Delta t, \\
y_{t+1}^{w} & = y_{t}^w + v_t \sin \left(\psi_{t}^w\right) \Delta t, \\
\psi_{t+1}^{w} & = \psi_{t}^w + \frac{v_t}{L} \tan \left(\delta_t\right) \Delta t,
\label{equation:kinematic_model}
\end{aligned}
\end{equation}  
where $x_t^{w}$ and $y_t^{w}$ denote the position of the ego vehicle relative to the world coordinate; $\psi_t^w$ is the heading angle that defines the vehicle's orientation with respect to the world $x$-coordinate; $v_t$ is the linear velocity of the ego vehicle; $\delta_t$ is the steering angle of the front wheels; and $L$ is the wheelbase, i.e., the distance between the front and rear axles.

During the rollout process, the AD policy outputs actions $(a_t^x, a_t^y)$ for a $0.5$-second time horizon at time step $t$. We derive the linear velocity $v_t$ and steering angle $\delta_t$ based on $(a_t^x, a_t^y)$.  
Based on the kinematic model in Eq.~\ref{equation:kinematic_model},  
the pose of the ego vehicle in the world coordinate system is updated from ${p}_t = (x_{t}^w, y_{t}^w, \psi_{t}^w)$ to ${p}_{t+1} = (x_{t+1}^{w}, y_{t+1}^{w}, \psi_{t+1}^{w})$.  

Based on the updated ${p}_{t+1}$, the 3DGS environment computes the new ego vehicle's state $s_{t+1}$. The updated pose ${p}_{t+1}$ and state $s_{t+1}$ serve as the input for the next iteration of the inference process.

The 3DGS environment also generates rewards $\mathcal{R}$ (Sec.~\ref{sec:reward}) according to multi-source information (including trajectories of other agents, map information, the expert trajectory of the ego vehicle, and the parameters of Gaussians), which are used to optimize the AD policy (Sec.~\ref{sec:optimization}).

\begin{figure}[t]
\centering
\includegraphics[width=1.0\linewidth]{fig/reward.pdf} 
\caption{\textbf{Example diagram of four types of reward sources.}  (1): Collision with a dynamic obstacle ahead triggers a reward $r_{\text{dc}}$. (2): Hitting a static roadside obstacle incurs a reward $r_{\text{sc}}$. (3): Moving onto the curb exceeds the positional deviation threshold $d_{\text{max}}$, triggering a reward $r_{\text{pd}}$. (4): Drifting toward the adjacent lane exceeds the heading deviation threshold $\psi_{\text{max}}$, triggering a reward $r_{\text{hd}}$.
}
\label{fig: reward source}
\end{figure}
\subsection{Reward Modeling}
\label{sec:reward}
The reward is the source of the training signal, which determines the optimization direction of RL. The reward function is designed to guide the ego vehicle's behavior by penalizing unsafe actions and encouraging alignment with the expert trajectory. It is composed of four reward components: (1) collision with dynamic obstacles, (2) collision with static obstacles, (3) positional deviation from the expert trajectory, and (4) heading deviation from the expert trajectory:
\begin{equation}
\begin{aligned}
\mathcal{R} = \{r_{\text{dc}}, r_{\text{sc}}, r_{\text{pd}}, r_{\text{hd}}  \}. 
\end{aligned}
\end{equation}

As illustrated in Fig.~\ref{fig: reward source}, these reward components are triggered under specific conditions.  
In the 3DGS environment, dynamic collision is detected if the ego vehicle's bounding box overlaps with the annotated bounding boxes of dynamic obstacles, triggering a negative reward $r_{\text{dc}}$. Similarly, static collision is identified when the ego vehicle's bounding box overlaps with the Gaussians of static obstacles, resulting in a negative reward $r_{\text{sc}}$.  
Positional deviation is measured as the Euclidean distance between the ego vehicle's current position and the closest point on the expert trajectory. A deviation beyond a predefined threshold $d_{\text{max}}$ incurs a negative reward $r_{\text{pd}}$.  
Heading deviation is calculated as the angular difference between the ego vehicle's current heading angle $ \psi_t $ and the expert trajectory's matched heading angle $\psi_{\text{expert}}$. A deviation beyond a threshold $ \psi_{\text{max}}$ results in a negative reward $r_{\text{hd}}$.

Any of these events, including dynamic collision, static collision, excessive positional deviation, or excessive heading deviation, triggers immediate episode termination. Because after such events occur, the 3DGS environment typically generates noisy sensor data, which is detrimental to RL training.

\subsection{Policy Optimization}
\label{sec:optimization}
In the closed-loop environment, the error in each single step accumulates over time. The aforementioned rewards are not only caused by the current action but also by the actions of the preceding steps.  
The rewards are propagated forward with Generalized Advantage Estimation (GAE)~\cite{gae} to optimize the action distribution of the preceding steps.

Specifically, for each time step $t$, we store the current state $s_t$, action $a_t$, reward $r_t$, and the estimate of the value $V(s_t)$.  
Based on the decoupled action space, and considering that different rewards have different correlations to lateral and longitudinal actions, the reward $r_t$ is divided into lateral reward $r_t^x$ and longitudinal reward $r_t^y$:
\begin{equation}
\begin{aligned}
r_t^x &= r_t^{\text{sc}} + r_t^{\text{pd}} + r_t^{\text{hd}}, \\
r_t^y &= r_t^{\text{dc}}.
\label{eq:reward-decouple}
\end{aligned}
\end{equation}
Similarly, the value function $V(s_t)$ is decoupled into two components: $V_x(s_t)$ for the lateral dimension and $V_y(s_t)$ for the longitudinal dimension. These value functions estimate the expected cumulative rewards for the lateral and longitudinal actions, respectively. The advantage estimates $\hat{A}_t^x$ and $\hat{A}_t^y$ are then computed as follows:
\begin{equation}
\begin{aligned}
\delta_t^x &= r_t^x + \gamma V_x(s_{t+1}) - V_x(s_t), \\
\delta_t^y &= r_t^y + \gamma V_y(s_{t+1}) - V_y(s_t), \\
\hat{A}_t^x &= \sum_{l=0}^{\infty}(\gamma \lambda)^l \delta_{t+l}^x, \\
\hat{A}_t^y &= \sum_{l=0}^{\infty}(\gamma \lambda)^l \delta_{t+l}^y,
\label{eq:advantage}
\end{aligned}
\end{equation}
where $\delta_t^x$ and $\delta_t^y$ are the temporal difference errors for the lateral and longitudinal dimensions, $\gamma$ is the discount factor, and $\lambda$ is the GAE parameter that controls the trade-off between bias and variance.

To further clarify the relationship between the advantage estimates and the reward components, we decompose $\hat{A}_t^x$ and $\hat{A}_t^y$ based on the reward decomposition in Eq.~\ref{eq:reward-decouple} and the advantage estimation in Eq.~\ref{eq:advantage}. Specifically, we derive the following decomposition:
\begin{equation}
\begin{aligned}
\hat{A}_t^x &= \hat{A}_t^{\text{sc}} + \hat{A}_t^{\text{pd}} + \hat{A}_t^{\text{hd}}, \\
\hat{A}_t^y &= \hat{A}_t^{\text{dc}},
\end{aligned}
\end{equation}
where $\hat{A}_t^{\text{sc}}$ is the advantage estimate for avoiding static collisions, $\hat{A}_t^{\text{pd}}$ is the advantage estimate for minimizing positional deviations, $\hat{A}_t^{\text{hd}}$ is the advantage estimate for minimizing heading deviations, and $\hat{A}_t^{\text{dc}}$ is the advantage estimate for avoiding dynamic collisions.

These advantage estimates are used to guide the update of the AD policy $\pi_{\theta}$, following the PPO framework~\cite{PPO}. By leveraging the decomposed advantage estimates $\hat{A}_t^x$ and $\hat{A}_t^y$, we can independently optimize the lateral and longitudinal dimensions of the policy. This is achieved by defining separate objective functions $\mathcal{L}_x^{\text{CLIP}}(\theta)$ and $\mathcal{L}_y^{\text{CLIP}}(\theta)$ for each dimension,  as follows:
\begin{equation}
\begin{aligned}
\mathcal{L}_x^{\text{PPO}}(\theta) &= \mathbb{E}_t \left[ \min \left( \rho_t^x \hat{A}_t^x, \ \text{clip}(\rho_t^x, 1-\epsilon_x, 1+\epsilon_x) \hat{A}_t^x \right) \right], \\
\mathcal{L}_y^{\text{PPO}}(\theta) &= \mathbb{E}_t \left[ \min \left( \rho_t^y \hat{A}_t^y, \ \text{clip}(\rho_t^y, 1-\epsilon_y, 1+\epsilon_y) \hat{A}_t^y \right) \right], \\
\mathcal{L}^{\text{PPO}}(\theta) &= \mathcal{L}_x^{\text{PPO}}(\theta) + \mathcal{L}_y^{\text{PPO}}(\theta),
\end{aligned}
\end{equation}
where $\rho_t^x = \frac{\pi_{\theta}(a_t^x \mid s_t)}{\pi_{\theta_{\text{old}}}(a_t^x \mid s_t)}$ is the importance sampling ratio for the lateral dimension, $\rho_t^y = \frac{\pi_{\theta}(a_t^y \mid s_t)}{\pi_{\theta_{\text{old}}}(a_t^y \mid s_t)}$ is the importance sampling ratio for the longitudinal dimension, $\epsilon_x$ and $\epsilon_y$ are small constants that control the clipping range for the lateral and longitudinal dimensions, ensuring stable policy updates.

The clipped objective function $\mathcal{L}^{\text{PPO}}(\theta)$ prevents excessively large updates to the policy parameters $\theta$, thereby maintaining training stability.

\begin{table*}[ht]
    \centering
{
\begin{tabular}{lccccccccc}
    \toprule
    RL:IL & CR$\downarrow$ & DCR$\downarrow$ & SCR$\downarrow$ & DR$\downarrow$ & PDR$\downarrow$ & HDR$\downarrow$ &ADD$\downarrow$ & Long. Jerk$\downarrow$ & Lat. Jerk$\downarrow$ \\
    \midrule
     0:1  & 0.229 & 0.211 & 0.018 & 0.066 & 0.039 & 0.027  & 0.238 & 3.928 & 0.103\\
     1:0  & 0.143 & 0.128 & 0.015 &0.080 &0.065 &0.015 &0.345 &4.204 &0.085\\
     2:1 & 0.137 & 0.125 & 0.012 & 0.059 & 0.050 & 0.009  & 0.274 & 4.538 & 0.092\\
     4:1 & 0.089 & 0.080 & 0.009 & 0.063 & 0.042 & 0.021  & 0.257 & 4.495 & 0.082 \\
     8:1 & 0.125 & 0.116 & 0.009 & 0.084 & 0.045 & 0.039  & 0.323 & 5.285 & 0.115\\
    \bottomrule
\end{tabular}
}
    \caption{\textbf{Ablation on RL-to-IL step mixing ratios in the reinforced post-training stage.}}
    \label{tab:ratio}
\end{table*}

\subsection{Auxiliary Objective}
RL usually faces the challenge of sparse rewards, which makes the convergence process unstable and slow. To speed up convergence, we introduce auxiliary objectives that provide dense guidance to the entire action distribution.

The auxiliary objectives are designed to penalize undesirable behaviors by incorporating specific reward sources, including dynamic collisions, static collisions, positional deviations, and heading deviations. These objectives are computed based on the actions \( a_t^{x, \text{old}} \) and \( a_t^{y, \text{old}} \) selected by the old AD policy \( \pi_{\theta_{\text{old}}} \) at time step \( t \). To facilitate the evaluation of these actions, we separate the probability distribution of the action into four parts:
\begin{equation}
\begin{aligned}
\Delta \pi_y^{\text{dec}} &= \sum_{a_t^y < a_t^{y, \text{old}}} \pi_\theta(a_t^y \mid s_t), \\
\Delta \pi_y^{\text{acc}} &= \sum_{a_t^y > a_t^{y, \text{old}}} \pi_\theta(a_t^y \mid s_t), \\
\Delta \pi_x^{\text{left}} &= \sum_{a_t^x < a_t^{x, \text{old}}} \pi_\theta(a_t^x \mid s_t), \\
\Delta \pi_x^{\text{right}} &= \sum_{a_t^x > a_t^{x, \text{old}}} \pi_\theta(a_t^x \mid s_t).
\end{aligned}
\end{equation}
Here, \( \Delta \pi_y^{\text{dec}} \) represents the total probability of deceleration actions, \( \Delta \pi_y^{\text{acc}} \) represents the total probability of acceleration actions, \( \Delta \pi_x^{\text{left}} \) represents the total probability of leftward steering actions, and \( \Delta \pi_x^{\text{right}} \) represents the total probability of rightward steering actions.

\boldparagraph{Dynamic Collision Auxiliary Objective.}  
The dynamic collision auxiliary objective adjusts the longitudinal control action \(a_t^y\) based on the location of potential collisions relative to the ego vehicle. If a collision is detected ahead, the policy prioritizes deceleration actions (\(a_t^y < a_t^{y, \text{old}}\)); if a collision is detected behind, it encourages acceleration actions (\(a_t^y > a_t^{y, \text{old}}\)). To formalize this behavior, we define a directional factor \(f_\text{dc}\):
\begin{equation}
\begin{aligned}
f_\text{dc} = \begin{cases} 
1 & \text{if the collision is ahead}, \\
-1 & \text{if the collision is behind}.
\end{cases} 
\end{aligned}
\end{equation}

The auxiliary objective for dynamic collision avoidance is defined as:
\begin{equation}
\begin{aligned}
\mathcal{L}_\text{dc}(\theta_y) = \mathbb{E}_t \left[ 
    \hat{A}_t^\text{dc} \cdot f_\text{dc} \cdot (\Delta \pi_y^{\text{dec}} - \Delta \pi_y^{\text{acc}})
\right],
\end{aligned}
\end{equation}
where \(\hat{A}_t^\text{dc}\) is the advantage estimate for dynamic collision avoidance.

\boldparagraph{Static Collision Auxiliary Objective.}  
The static collision auxiliary objective adjusts the steering control action $a_t^x$ based on the proximity to static obstacles. If the static obstacle is detected on the left side, the policy promotes rightward steering actions ($a_t^x > a_t^{x,\text{old}}$); if the static obstacle is detected on the right side, it promotes leftward steering actions ($a_t^x < a_t^{x,\text{old}}$). To formalize this behavior, we define a directional factor $f_\text{sc}$:  
\begin{equation}
\begin{aligned}
f_\text{sc} = \begin{cases} 
1 & \text{if static obstacle is on the left}, \\
-1 & \text{if static obstacle is on the right}.
\end{cases} 
\end{aligned}
\end{equation}

The auxiliary objective for static collision avoidance is defined as:  
\begin{equation}
\begin{aligned}
\mathcal{L}_\text{sc}(\theta_x) = \mathbb{E}_t \left[ 
    \hat{A}_t^\text{sc} \cdot f_\text{sc} \cdot (\Delta \pi_x^{\text{right}} - \Delta \pi_x^{\text{left}})
\right],
\end{aligned}
\end{equation}  
where $\hat{A}_t^\text{sc}$ is the advantage estimate for static collision avoidance.  

\boldparagraph{Positional Deviation Auxiliary Objective.}  
The positional deviation auxiliary objective adjusts the steering control action $a_t^x$ based on the ego vehicle's lateral deviation from the expert trajectory. If the ego vehicle deviates leftward, the policy promotes rightward corrections ($a_t^x > a_t^{x,\text{old}}$); if it deviates rightward, it promotes leftward corrections ($a_t^x < a_t^{x,\text{old}}$). We formalize this with a directional factor $f_\text{pd}$:  
\begin{equation}
\begin{aligned}
f_\text{pd} = \begin{cases} 
1 & \text{if ego vehicle deviates leftward}, \\
-1 & \text{if ego vehicle deviates rightward}.
\end{cases} 
\end{aligned}
\end{equation}

The auxiliary objective for positional deviation correction is:
\begin{equation}
\begin{aligned}
\mathcal{L}_\text{pd}(\theta_x) = \mathbb{E}_t \left[ 
    \hat{A}_t^\text{pd} \cdot f_\text{pd} \cdot (\Delta \pi_x^{\text{right}} - \Delta \pi_x^{\text{left}})
\right],
\end{aligned}
\end{equation}  
where $\hat{A}_t^\text{pd}$ estimates the advantage of trajectory alignment.

\boldparagraph{Heading Deviation Auxiliary Objective.}  
The heading deviation auxiliary objective adjusts the steering control action $a_t^x$ based on the angular difference between the ego vehicle’s current heading and the expert’s reference heading. If the ego vehicle deviates counterclockwise, the policy promotes clockwise corrections ($a_t^x > a_t^{x,\text{old}}$); if it deviates clockwise, it promotes counterclockwise corrections ($a_t^x < a_t^{x,\text{old}}$). To formalize this behavior, we define a directional factor $f_\text{hd}$:  
\begin{equation}
\begin{aligned}
f_\text{hd} = \begin{cases} 
1 & \text{if ego vehicle deviates clockwise}, \\
-1 & \text{if ego vehicle deviates counterclockwise}.
\end{cases} 
\end{aligned}
\end{equation}

The auxiliary objective for heading deviation correction is then defined as:  
\begin{equation}
\begin{aligned}
\mathcal{L}_\text{hd}(\theta_x) = \mathbb{E}_t \left[ 
    \hat{A}_t^\text{hd} \cdot f_\text{hd} \cdot (\Delta \pi_x^{\text{right}} - \Delta \pi_x^{\text{left}})
\right],
\end{aligned}
\end{equation}  
where $\hat{A}_t^\text{hd}$ is the advantage estimate for heading alignment.  

\begin{table*}[ht]
\begin{center}
\centering
\resizebox{0.98\textwidth}{!}{
\begin{tabular}{cccccccccccccc}
\toprule
\multirow{2}{*}{ID} & Dynamic & Static & Position & Heading & \multirow{2}{*}{CR$\downarrow$} &\multirow{2}{*}{DCR$\downarrow$} &\multirow{2}{*}{SCR$\downarrow$} &\multirow{2}{*}{DR$\downarrow$} &\multirow{2}{*}{PDR$\downarrow$} &\multirow{2}{*}{HDR$\downarrow$} &\multirow{2}{*}{ADD$\downarrow$} &\multirow{2}{*}{Long. Jerk$\downarrow$} &\multirow{2}{*}{Lat. Jerk$\downarrow$}\\
& Collision & Collision & Deviation & Deviation & & & & & & & & & \\
\midrule
1 & \cmark  &  &  &  & 0.172 & 0.154 & 0.018 & 0.092 & 0.033 & 0.059  & 0.259 & 4.211 & 0.095 \\
2 &  & \cmark & \cmark & \cmark & 0.238 & 0.217 & 0.021 & 0.090 & 0.045 & 0.045  & 0.241 & 3.937 & 0.098 \\
3 & \cmark &  & \cmark & \cmark & 0.146 & 0.128 & 0.018 & 0.060 & 0.030 & 0.030  & 0.263 & 3.729 & 0.083\\
4 & \cmark & \cmark &  & \cmark & 0.151 & 0.142 & 0.009 & 0.069 & 0.042 & 0.027 & 0.303 & 3.938 & 0.079\\
5 & \cmark & \cmark & \cmark &  & 0.166 & 0.157 & 0.009 & 0.048 & 0.036 & 0.012 & 0.243 & 3.334 & 0.067\\
6 & \cmark & \cmark & \cmark & \cmark & 0.089 & 0.080 & 0.009 & 0.063 & 0.042 & 0.021 & 0.257 & 4.495 & 0.082 \\
\bottomrule
\end{tabular}
}
\end{center}
\vspace{-2mm}
\caption{\textbf{Ablation on reward sources.} The table shows the impact of different reward components on performance.}
\label{tab:reward_ablation}
\end{table*}

\begin{table*}[ht]
\begin{center}
\centering
\resizebox{0.98\textwidth}{!}{
\begin{tabular}{ccccccccccccccc}
\toprule
\multirow{2}{*}{ID} & \multirow{2}{*}{PPO Obj.}  & Dynamic Col. & Static Col. & Position Dev. & Heading Dev. & \multirow{2}{*}{CR$\downarrow$} & \multirow{2}{*}{DCR$\downarrow$}  & \multirow{2}{*}{SCR$\downarrow$} & \multirow{2}{*}{DR$\downarrow$} & \multirow{2}{*}{PDR$\downarrow$} & \multirow{2}{*}{HDR$\downarrow$} & \multirow{2}{*}{ADD$\downarrow$} & \multirow{2}{*}{Long. Jerk$\downarrow$} & \multirow{2}{*}{Lat. Jerk$\downarrow$} \\
& & Auxiliary Obj. & Auxiliary Obj. & Auxiliary Obj. & Auxiliary Obj. & & & & & & & & & \\
\midrule
1 &\cmark&  &  &  &  & 0.249 & 0.223 & 0.026 & 0.077 & 0.047 & 0.030  & 0.266 & 4.209 & 0.104 \\
2 &\cmark& \cmark &  &  &  & 0.178 & 0.163 & 0.015 & 0.151 & 0.101 & 0.050 & 0.301 & 3.906 & 0.085 \\
3 &\cmark&  & \cmark & \cmark & \cmark & 0.137 & 0.125 & 0.012 & 0.157 & 0.145 & 0.012 & 0.296 & 3.419 & 0.071 \\
4 &\cmark& \cmark &  & \cmark & \cmark & 0.169 & 0.151 & 0.018 & 0.075 & 0.042 & 0.033 & 0.254 & 4.450 & 0.098 \\
5 &\cmark& \cmark & \cmark &  & \cmark & 0.149 & 0.134 & 0.015 & 0.063 & 0.057 & 0.006 & 0.324 & 3.980 & 0.086 \\
6 &\cmark& \cmark & \cmark & \cmark & & 0.128 & 0.119  & 0.009 & 0.066 & 0.030 & 0.036  & 0.254 & 4.102 & 0.092 \\
7 &&\cmark  &\cmark  &\cmark  &\cmark  & 0.187 &0.175  &0.012 &0.077 &0.056  &0.021  &0.309  &5.014  &0.112  \\
8 &\cmark& \cmark & \cmark & \cmark & \cmark & 0.089 & 0.080 & 0.009 & 0.063 & 0.042 & 0.021  & 0.257 & 4.495 & 0.082 \\
\bottomrule
\end{tabular}
}
\end{center}
\vspace{-2mm}
\caption{\textbf{Ablation on auxiliary objectives.} The table shows the impact of different auxiliary objectives on performance.}
\label{tab:auxiliary_ablation}
\end{table*}

\boldparagraph{Overall Auxiliary Objectives.}  
The overall auxiliary objectives are a weighted sum of the individual objectives:
\begin{equation}
\begin{aligned}
\mathcal{L}_\text{aux}(\theta) = &\lambda_1 \mathcal{L}_\text{dc}(\theta_y) + \lambda_2 \mathcal{L}_\text{sc}(\theta_x)  + \\ 
&\lambda_3 \mathcal{L}_\text{pd}(\theta_x) +\lambda_4 \mathcal{L}_\text{hd}(\theta_x),
\end{aligned}
\end{equation}
where $\lambda_1$, $\lambda_2$, $\lambda_3$, and $\lambda_4$ are weighting coefficients that balance the contributions of each auxiliary objective.

\boldparagraph{Optimization Objective.}  
The final optimization objective combines the clipped PPO objective with the auxiliary objective:
\begin{equation}
\mathcal{L}(\theta) = \mathcal{L}^{\text{PPO}}(\theta) + \mathcal{L}_\text{aux}(\theta).
\end{equation}

\subsubsection{Gait-Net-augmented Sequential MPC Algorithm}
Algorithm \ref{alg:gaitMPC} outlines the procedure for solving the proposed kino-dynamic MPC with sequential CMPC subproblems and Gait-Net. 

In the initialization stage (1-4), $f_\text{j2m}$ describes the mapping from joint-space general-coordinate states to spatial momenta. $f_\text{ref}$ construct a reference trajectory in generalized coordinate with a nominal sampling duration $dt^0$. Within the sequential CMPC iterations (5-11), the iteration is terminated until reached max iteration $j_\text{max}$ or the search direction reaches the desired tolerance $\bm \eta$. In each iteration, the CMPC subproblem described in \ref{subsubsec:cmpc} is solved via QP; the MPC sampling time is updated through Gait-Net policy $\Pi_\text{GN}$. Subsequently, the reference trajectories are updated to reflect the latest foot location and MPC $dt$. 

%%%%%%%%%%%%%% performance comparison %%%%%%%%%%%%%%%%
\begin{figure}[!t]
\vspace{0.2cm}
     \centering
     \begin{subfigure}[b]{0.5\textwidth}
         \centering
	   \includegraphics[clip, trim=0cm 10.4cm 7.4cm 0cm, width=1\columnwidth]{figures/comparison1_3.pdf}
          \caption{Baseline 1: Fixed step duration for every step.}
          \vspace{0.2cm}
          \label{fig:comp1_3}
     \end{subfigure}
     \begin{subfigure}[b]{0.5\textwidth}
        \includegraphics[clip, trim=0cm 11cm 7.4cm 0cm, width=1\columnwidth]{figures/comparison1_1.pdf}
	\caption{Baseline 2: Solving step duration as part of optimization variables in NMPC.}
        \vspace{0.2cm}
	\label{fig:comp1_1}
     \end{subfigure}
     \begin{subfigure}[b]{0.5\textwidth}
         \centering
	   \includegraphics[clip, trim=0cm 10cm 7.4cm 0cm, width=1\columnwidth]{figures/comparison1_2.pdf}
          \caption{Proposed: Gait-Net-augmented Kino-dynamic MPC.}
          \vspace{0.4cm}
          \label{fig:comp1_2}
     \end{subfigure}
     \begin{subfigure}[b]{0.48\textwidth}
         \centering
	   \includegraphics[clip, trim=0cm 3cm 0cm 0cm, width=1\columnwidth]{figures/mpcdt_comparison.pdf}
          \caption{Comparison of MPC $dt$ (interpreted as step duration).}
          % \vspace{0.1cm}
          \label{fig:dt_comp}
     \end{subfigure}
     \caption{{\bfseries{Comparison of Discrete Terrain Locomotion Performance in 2D Simulation.}} }
        \label{fig:comp1}
\end{figure}

%%%%%%%%%%%%%%%% 3D simulation %%%%%%%%%%%%%
\begin{figure*}[!t]
\vspace{0.2cm}
		\center
		\includegraphics[clip, trim=0cm 12cm 0.2cm 0cm, width=2\columnwidth]{figures/foot_location_snap.pdf}
		\caption{{\bfseries Locomotion over 3-D Stepping-stone Terrain.} Simulation snapshots (left) and plot of measured foot locations (right). In the plot, only foot locations that are on the ground are visualized. The green dashed-line bounding box represents the CoM position threshold that triggers the foot location constraints for the corresponding stepping stone patch.}
		\label{fig:footlocation}
		\vspace{-0.2cm}
\end{figure*}
\begin{figure*}[!t]
\vspace{-0.1cm}
		\center
		\includegraphics[clip, trim=0cm 11.3cm 0.2cm 0cm, width=1.9\columnwidth]{figures/h_tracking2.pdf}
		\caption{{\bfseries Spatial Momenta Measurement vs. MPC Prediction along $l_{G,x},\:k_{G,y},\:$and $k_{G,z}$ of 3-D Stepping-stone Simulation Results.}}
		\label{fig:h_tracking}
		\vspace{-0.2cm}
\end{figure*}

\remark{As the sequential iteration progresses, the reference trajectories $\{ \bm x^\text{ref},\:\bm p_f^\text{ref}\}$ are continuously updated to match closely to the real spatial momentum and pose trajectories based on the latest kinematics results. This process inherently embeds an \textit{implicit kinematics assurance} within the framework.}

\remark{The Gait-Net-augmented kino-dynamic MPC is run at the beginning of each footstep to determine a local step duration in terms of MPC $dt$. The rest of this footstep will incorporate the same $dt$ without the inference of Gait-Net and solve only the contact location and wrenches.}