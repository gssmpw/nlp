
\section{Introduction}
\label{sec:Introduction}

%%%%%%%% Motivation %%%%%%%

%% contact planning as standalone approach
Contact/footstep planning is a fundamental problem in humanoid robot motion control. Due to the inherent instability of these robots, 
efficient integration of both contact planning and motion control is an essential step to enable dynamic and robust locomotion across diverse terrains.

\subsection{Contact Planning}
Footstep planning has traditionally been treated as a high-level, standalone problem solved before motion control execution \cite{deits2014footstep, bouyarmane2012humanoid, carpentier2016versatile, ponton2021efficient}. For instance, \citet{deits2014footstep} addresses 3D humanoid footstep planning using a highly efficient mixed-integer quadratically constrained quadratic program (MIQCQP) to compute trajectories in seconds. In addition, \citet{bouyarmane2012humanoid} proposes a best-first algorithm for collision-free multi-contact planning in humanoid loco-manipulation. However, these offline trajectories, tracked by whole-body control or inverse-kinematics-based controllers, often suffer from error accumulation over long trajectories due to their open-loop nature.

%% contact implicit planning
Contact-implicit model-predictive control (CI-MPC) has gained popularity for optimizing contact force, location, and timing within a unified framework by addressing the linear complementarity problem (LCP) between contact velocity and force \cite{kim2023contact, le2024fast, kong2023hybrid}. These frameworks integrate contact planning and motion control to generate optimal contact behaviors in real-time. However, CI-MPC involves solving highly complex, nonlinear problems with significant computational demands. While \citet{le2024fast} precompute LCP parameters offline to improve online speed, this process remains costly and must be repeated for each new setup.

%%%%%%%%%%%%% title figure %%%%%%%%%%%%%%%
\begin{figure}[!t]
\vspace{0cm}
    \center
    \includegraphics[clip, trim=0cm 4cm 17.5cm 0.1cm, width=1\columnwidth]{figures/Gaitnet_title.pdf}
    \caption{{\bfseries Gait-Net-augmented Kino-dynamic MPC.} Hardware Experiment Snapshots. (a). Push-recovery in locomotion (b). Locomotion while carrying an unknown 0.75 kg object; (c). Locomotion over unknown uneven terrain; (d). Dynamic walking over terrain with a 20 cm terrain gap at 0.75 m/s; (e). Dynamic walking over terrain gap and obstacle. Full experiment video: \url{https://youtu.be/UqLDYHGL5EA} }
    \label{fig:title}
    \vspace{-0.2cm}
\end{figure}
%%%%%%%%%%%%%%%%%%%%%%%%%%%%%%%%%%%%%%%%%%

\subsection{Humanoid Locomotion with Simplified Models}
Convex MPC (CMPC) is widely used in legged locomotion control, leveraging linearized dynamics and constraints, such as single rigid-body model (SRBM) \cite{di2018dynamic}, for fast and high-frequency control by optimizing ground reaction forces. However, due to the bilinear coupling of force and foot location vectors, predefined contact locations are required \cite{di2018dynamic, ding2022orientation} for linearization. Foot placement is typically determined separately using heuristics (\textit{e.g.}, Raibert heuristic \cite{raibert1986legged}), capture point methods \cite{pratt2006capture}, or optimizations with the linear inverted pendulum model \cite{gu2024walking}. These approaches embed predefined step durations (\textit{a.k.a.}, one-step gait duration), making it challenging to invert the problem and determine duration from foot location, especially in 3-D. Feed-forward step duration is also suboptimal for adaptive strategies on uneven terrain, where step duration should highly correlate with stride length and current foot actuation.

Centroidal dynamics (CD) is widely used in humanoid robot control for its simplified yet effective representation of whole-body dynamics, making it suitable for real-time planning and control \cite{orin2013centroidal}. CD-MPC and kino-dynamic MPC are often formulated as nonlinear MPC (NMPC) problems \cite{romualdi2022online, dai2014whole, elobaid2023online}. \citet{garcia2021mpc} leverages CMPC formulation and CD-augmented SRBM to include link inertia.  However, the framework still requires predefined gait schedules and the generation of footstep locations from separate modules. Additionally, Kino-dynamic MPC offers an advantage over CD-MPC by explicitly optimizing joint states, ensuring feasible whole-body motions \cite{dantec2024centroidal}. In contrast, CD-MPC lacks kinematic coupling between foot location and floating base states, often necessitating lower-level inverse kinematics (IK) motion generation \cite{meduri2023biconmp} or whole-body control \cite{wensing2016improved}.
In our work, the implicit kino-dynamic MPC aids kinematic assurance and eliminates the need to optimize joint states. 


% %%%%%%%%%%%% system architecture %%%%%%%%%%%
\begin{figure*}[!t]
\vspace{0.2cm}
		\center
		\includegraphics[clip, trim=0.5cm 0.3cm 0.5cm 0cm, width=1.8\columnwidth]{figures/controlArchi.pdf}
		\caption{{\bfseries Control System Architecture}}
		\label{fig:controlArchi}
		\vspace{-0.2cm}
\end{figure*}

%%%%% variable frequency walking:%% 
\subsection{Challenges in Variable Frequency Locomotion}
Variable gait frequency locomotion remains under-explored in humanoid robotics. 
Previous works have primarily addressed this problem at the footstep planning level \cite{khadiv2016step, griffin2017walking, nguyen2017dynamic, xiang2024adaptive}, optimizing only foot placement and timing.
While the MIT humanoid robot demonstrates impressive performance with whole-body MPC \cite{khazoom2024tailoring}, its gait frequency is fixed and determined by the MPC sampling time. \citet{li2023dynamic} optimize gait frequency offline for stepping stone terrain, but the open-loop approach lacks real-time adjustments based on state feedback. In contrast, our work enables variable-frequency bipedal walking by concurrently optimizing foot location, contact force, and gait frequency compactly and efficiently, addressing the motion control and variable foot-step planning together in one optimization.

Gait frequency can be integrated into the MPC framework by optimizing the sampling time per step with a fixed contact schedule. However, this introduces a multi-linear coupling of foot location, contact force, and sampling time, significantly increasing problem complexity and solving time. To address this, we propose a customized Neural-network-augmented sequential QP (SQP) solver that efficiently handles these multi-linear terms through iteratively solving QP problems.

\subsection{Learning-augmented Optimization-based Control}

Learning-based control methods, such as reinforcement learning (RL), have achieved significant success in legged locomotion \cite{li2021reinforcement, margolis2024rapid, krishna2022linear, bao2024deep}. At the same time, model-based approaches have also gained from learning-based techniques \cite{bang2024variable, el2024real, chen2024learning, romualdi2024online}. For example, pre-trained neural networks (NN) can approximate complex, computationally intensive nonlinear functions, as shown in \cite{bang2024variable}, where an NN predicts centroidal inertia evolution, eliminating the need for complex spatial momentum computations in optimization. In this work, we address the weak correlation between gait frequency and foot location, an inherent limitation of simplified dynamics models such as SRBM and CD, by introducing Gait-Net, an NN that is trained on data from variable-gait-frequency MPC with whole-body dynamics.


%%%%%%%%%%%%%%%%% Main Contributions %%%%%%%%%%%%%%%%%%
\subsection{Contributions}
The main contributions are twofold. Firstly, we propose a novel Gait-Net-augmented Implicit Kino-dynamic MPC framework that concurrently optimizes foot contact force, foot location, and footstep duration. In this framework, we introduce supervised learning in a sequential CMPC algorithm to efficiently solve for the proposed NMPC with multi-linear constraints. 

Secondly, we complement CD with a Gait-frequency Network (Gait-Net) to form an \textit{implicit} kino-dynamic MPC. In each sequential MPC iteration, Gait-Net (1) determines the preferred step duration in terms of MPC sampling time $dt$, transforming $dt$ from a decision variable to a parameter for efficiency; and (2) improves the estimation of reference spatial momentum and pose trajectories to mimic more closely as an explicit kino-dynamic approach while eliminating the joint angles as part of state variables.

Additionally, we validate the proposed approach through both simulation and hardware experiments on humanoid robots. Our controller demonstrates robustness against unknown disturbances, successfully handling uneven terrain, push recovery, and unknown loads. It enables the robot to push a 35 kg cart and traverse discrete terrains with gaps (discontinuities) up to 20 cm. 

The rest of the paper is organized as follows. Sec. \ref{sec:overview} presents the overview of the proposed control system architecture. Sec. \ref{sec:bg} outlines the background and preliminaries of the MPC-based locomotion control methods.  Sec. \ref{sec:approach} presents the main approaches, including the proposed kino-dynamics MPC, the Gait-Net, and the main algorithm of the Gait-Net-augmented sequential MPC. Sec. \ref{sec:Results} presents highlighted numerical and hardware validations.


