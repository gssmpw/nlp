\section{Preliminary}
\label{sec:bg}

In this section, we introduce the background of the whole-body MPC used during the variable-frequency walking data collection process and a general explicit kino-dynamics MPC formulation for humanoid robots.

\subsection{Whole-body Model Predictive Control}
\label{subsec:wbmpc}
Driven by the strong dynamics and kinematics correlation between the footstep location, duration, and whole-body coordination with Whole-body MPC on humanoid robots \cite{khazoom2024tailoring, dantec2024centroidal}, we leverage such a control paradigm to obtain high-fidelity humanoid locomotion results in simulation to serve as the training dataset for Gait-Net. We particularly focus on adapting this MPC to achieve variable MPC sampling times for each footstep (\textit{i.e.}, variable-frequency walking).
Notably, to achieve variable-frequency walking, we fix a periodic contact sequence of the locomotion to be every $h$ MPC time-steps. Therefore, we can allow variable step durations at each footstep by adjusting the sampling time $dt_k$ of every $h'=h/2$ time-steps, as a one-step gait duration. 


The optimization variable $\mathbf X^\text{wb}$ includes the robot states in the generalized coordinates $\mathbf q \in \mathbb R^{6+n_j}$, their rates of change $\dot{\mathbf q} \in \mathbb R^{6+n_j}$, joint torque $\bm \tau_j \in \mathbb R^{n_j}$, and the constraint forces $\bm \lambda = \{\bm f_i\in \mathbb R^{3};\: \bm \tau_i\in \mathbb R^{3}\}^{n_i}_{i=0} $ (\textit{i.e.}, spatial contact wrench), 
\begin{align}
    \mathbf X^\text{wb} = \{ \mathbf q_k,\: \dot{\mathbf q}_k,\: \bm \tau_{j,k},\: \bm \lambda_k \}^{h}_{k=0}
\end{align}

The nonlinear optimization problem can be formulated as

\begin{alignat}{3}
\label{eq:NMPCcost}
\underset{\mathbf X^\text{wb}}{\text{min}} \: & \sum_{k = 0}^{h-1} \big\| \mathbf q_k-  \mathbf q^{\text{ref}}_k\big\|^2 _{\bm Q_1} + \big\| \bm{\tau}_{j,k}  \big\|^2 _{\bm Q_2} + \big\| \bm \lambda_k  \big\|^2 _{\bm Q_3} + \big\| \dot{\mathbf q}_k  \big\|^2 _{\bm Q_4}\\ 
    \nonumber
    \textrm{subj} & \textrm{ect to:}
\end{alignat}
\vspace{-0.35cm}
\begin{subequations}
\label{eq: NMPCconstraints}
\allowdisplaybreaks
\setlength\abovedisplayskip{-3pt}
\begin{alignat}{3}
    \label{eq:mpcDynamics}
    \textrm{Dynamics: } \quad & \{\dot{{\mathbf q}}_{k+1},\: {{\mathbf q}}_{k+1} \}= f^{\mathrm{wb}}(\mathbf X^\text{wb}_{k}, dt_k),\\
    \label{eq:mpcequality}
    \textrm{Eq. Cons.:} \quad & g^\text{wb}_\mathrm{eq}(\mathbf X^\text{wb}_k) = \bm b^\text{wb}_\mathrm{eq}, \\
    \label{eq:mpcinequality}
    \textrm{Ineq. Cons.:} \quad  & \bm b^\text{wb}_\mathrm{lb} \leq g^\text{wb}_\mathrm{ineq}(\mathbf X^\text{wb}_k) \leq \bm b^\text{wb}_\mathrm{ub}.
\end{alignat}
\end{subequations}

Equation (\ref{eq:NMPCcost}) refers to the objective function of the NMPC, which is to track a predefined trajectory, while minimizing joint torque, contact wrench, and rate of change of the states. Each objective is weighted by the corresponding diagonal matrix $\bm Q$. Equation (\ref{eq:mpcDynamics}) describes the discrete-time whole-body dynamics in the spatial vector form, modified from
\begin{align}
    \arraycolsep=1.4pt\def\arraystretch{1.25}
    \left[\begin{array}{cc} 
    \bm M(\mathbf q) & -\bm J_i^\intercal(\mathbf q) \\
    -\bm J_i^\intercal(\mathbf q) & \mathbf 0
    \end{array} \right]
    \left[\begin{array}{c} 
    \ddot{\mathbf q}\\
    \bm \lambda_i
    \end{array} \right] = 
    \left[\begin{array}{c} 
    -\bm C(\mathbf q, \dot{\mathbf q}) + \bm S \bm \tau_j \\
    \bm J_i(\mathbf q)\dot{\mathbf q}
    \end{array} \right] 
\end{align}
where $\bm M$ and $\bm C$ are the mass matrix and combination of the Coriolis terms and the gravity vector. $\bm J_i$ represents the spatial contact Jacobian of $i$th contact point. $\bm S$ is an actuation selection matrix in joint space.  

The optimization problem is also subjected to additional equality (\ref{eq:mpcequality}) and inequality (\ref{eq:mpcinequality}) constraints. These constraints include general constraints to ensure humanoid locomotion, such as friction pyramid constraints, torque limits, joint limits, contact wrench saturation, and contact wrench cone (CWC) constraints \cite{caron2015stability}. The joint torque result from optimization can be directly applied as the motor command to control the robot. Hence, the deployment frequency of such a controller has a direct impact on the control performance (\textit{i.e.}, a higher hyper-sample rate will benefit faster reaction to disturbances). 


\subsection{Explicit Kino-dynamic Model Predictive Control}
\label{subsec:cdmpc}
Due to the heavy computation burden of the full-order dynamics model in whole-body MPC, many works that leverage this control method trade-off solution accuracy for high-frequency online deployment, such as constraining the number of iterations in SQP solvers \cite{khazoom2024tailoring, galliker2022planar} and leveraging Differential Dynamic Programming (DDP) \cite{dantec2022whole}. According to \cite{dantec2024centroidal}, the performance of such whole-body MPCs is still to be proven to outperform the relatively more computation-friendly Centroidal-dynamics MPC (CD-MPC) and Kino-dynamics MPC (full kinematics + CD) \cite{romualdi2022online, garcia2021mpc, he2024cdm, chignoli2021humanoid}. Hence, a well-constructed Kino-dynamics MPC can very well stand at the middle ground of computation intensity and control performance for online deployment.

The centroidal dynamics of a humanoid robot is:
\begin{align}
\label{eq:cd}
    \dot{\bm h}= \left[\begin{array}{c} 
    \dot{\bm l}_G\\
    \dot{\bm k}_G 
    \end{array} \right] = 
    \left[\begin{array}{c} 
    \sum_{i = 0}^{n_i}\bm f_i \\
    \sum_{i = 0}^{n_i}(\bm p_{f,i}-\bm p_c) \times \bm f_i + \bm \tau_i
    \end{array} \right],
\end{align}
where $\dot{\bm l}_G \in \mathbb R^3$ and $\dot{\bm k}_G \in \mathbb R^3$ are the rate of change of linear momentum and angular momentum. $(\bm p_{f,i}-\bm p_c)$ is the distance vector from the $i$th contact point $\bm p_{f,i}$ to robot CoM $\bm p_c$. 

Empirically, the \textit{explicit} kino-dynamics MPC requires joint states as part of the optimization variables, as the centroidal momentum matrix $\bm A_G$ \cite{orin2013centroidal} is dependent on the whole-body configuration, and
\begin{align}
\label{eq:hdot}
    \bm h  = \bm A_G(\mathbf q) \dot{\mathbf q}, \quad
    \dot {\bm h} =  \bm A_G(\mathbf q) \ddot{\mathbf q} + \dot{\bm A}_G(\mathbf q, \dot {\mathbf q}) \dot{\mathbf q}.
\end{align}

Hence the optimization variables of the explicit kino-dynamics MPC can be chosen as,
\begin{align}
    \mathbf X^\text{kd} = \{ \mathbf q,\: \dot{\mathbf q},\: \bm \lambda_k \}^{h}_{k=0},
\end{align}
 The finite horizon optimization problem with a prediction of $h$ steps can expressed as,
\begin{alignat}{3}
\label{eq:CDMPCcost}
\underset{\mathbf X^\text{kd}}{\text{min}} \: & \sum_{k = 0}^{h-1} 
\big\| \bm h_{k}^{\text{ref}} - \bm h_{k}  \big\|^2 _{\bm R_1}
+ \big\| \mathbf X_{k}^{\text{kd,ref}} - \mathbf X^\text{kd}_{k}  \big\|^2 _{\bm R_2} \\ 
\nonumber
\textrm{subject to:} & \quad
\end{alignat}
\vspace{-0.35cm}
\begin{subequations}
\label{eq: CDMPCconstraints}
\allowdisplaybreaks
\setlength\abovedisplayskip{-3pt}
\begin{alignat}{3}
    \label{eq:CDDynamics}
    \textrm{Dynamics: } \quad & \mathbf X^\text{kd}_{k+1} = f^{\mathrm{kd}}(\mathbf X^\text{kd}_{k}, dt_k),\\
    \label{eq:CDequality}
    \textrm{Eq. Cons.:} \quad & g^\mathrm{kd}_\mathrm{eq}(\mathbf X^\text{kd}_k) = \bm b^\mathrm{kd}_\mathrm{eq}, \\
    \label{eq:CDinequality}
    \textrm{Ineq. Cons.:} \quad  & \bm b^\mathrm{kd}_\mathrm{lb} \leq g^\mathrm{kd}_\mathrm{ineq}(\mathbf X^\text{kd}_k) \leq \bm b^\mathrm{kd}_\mathrm{ub}.
\end{alignat}
\end{subequations}
where the objective function (\ref{eq:CDMPCcost}) aims to track a predefined spatial momentum trajectory, state variable trajectory, and minimize ground reaction wrenches. The dynamics $f^{\text{kd}}$ in (\ref{eq:CDDynamics}) is the discrete-time dynamics of (\ref{eq:cd}-\ref{eq:hdot}). Additional constraints (\ref{eq:CDequality}-\ref{eq:CDinequality}) are similar to those of whole-body MPC.

\begin{figure}[!t]
\vspace{0.2cm}
    \center
    \includegraphics[clip, trim=1.2cm 0.5cm 1.2cm 0.3cm, width=1\columnwidth]{figures/Gait-Net.pdf}
    \caption{{\bfseries Illustration of the Gait-frequency Network}}
    \label{fig:gaitnet}
    \vspace{-0.5cm}
\end{figure}


It is worth noting that the kinematics aspect of the kino-dynamics MPC is computationally burdensome due to the direct optimization of full joint states. However, it may be embedded in the design of the MPC. As shown in \cite{dantec2024centroidal}, a CD-MPC chooses the spatial momentum vector and float-based states as the MPC states. This approach requires a somewhat accurate approximation of the evolution of the joint angles within the prediction horizon to construct meaningful spatial momentum trajectories. Hence,
\assumption{\label{assumption1}\textit{If the actual spatial momentum and pose evolution closely align with the designed trajectories, kinematic assurance can be implicitly embedded into the reference trajectory design rather than explicitly included in constraints. This enables an implicit kino-dynamic MPC approach.} }

\begin{figure*}[!t]
\vspace{0.2cm}
    \center
    \includegraphics[clip, trim=0.5cm 11.5cm 0.5cm 11.5cm, width=2\columnwidth]{figures/PCA.pdf}
    \caption{{\bfseries Feature Projection Bar Graphs along 6 Principle Axes.} The feature with the highest projection in each axis (red bar) is selected to be part of the new feature space. Note that along principle axes 1 and 2, both left and right legs are equally weighted with opposite signs, making the single Gait-Net suitable for both legs' prediction. }
    \label{fig:PCA}
    \vspace{-0.2cm}
\end{figure*}

In contrast, in BiConMP \cite{meduri2023biconmp}, the kinematic feasibility is guaranteed by a subsequent kinematics optimization after the MPC finds a feasible spatial momentum trajectory. In our work, we design and update these trajectories within each sequential CMPC subproblem and actively update the spatial reference based on the newly updated foot locations and float-based states within sequential iterations. We will discuss this key method in Sec. \ref{subsec:gaitnetmpc}. 