\section{System Overview}
\label{sec:overview}

In this section, we present the control system architecture of the proposed framework, shown in Fig. \ref{fig:controlArchi}. 
Empirically, humanoid kino-dynamics MPC explicitly optimizes the joint states through kinematics constraints \cite{gu2025humanoid}, while traditional centroidal-dynamics MPC often requires subsequent inverse kinematics solver or whole-body control for motion execution. Both approaches employ nonlinear approaches to solve the optimization problem. In our framework, we proposed a Gait-Net-augmented sequential CMPC algorithm that translates the original nonlinear problem into convex sequential subproblems. With the additional assistance of Gait-Net, we reduce the optimization variable and mimic a natural step duration decision in each iteration. 

The control framework converts user commands and contact sequence into joint space references $\{\mathbf q_k^\text{ref} \in \mathbb R^{6+n_j},\: \dot{\mathbf q}_k^\text{ref} \in \mathbb R^{6+n_j}\}^h_{k = 0}$ and foot location reference trajectory $\{\bm p_f^\text{ref}\in \mathbb R^{3n_i}\}^h_{k = 0}$, where $n_j$ is the number of joints, $n_i$ is the number of contact/foot, and $h$ is a finite number of horizon. These joint-space trajectories, along with joint-space feedback states, are then translated into spatial momenta $\bm h\in \mathbb R^6$ and their primitive, the centroidal pose $\bm H\in \mathbb R^6$, which are the state variables used in the Gait-Net-augmented kino-dynamic MPC. Within the MPC, we break down the nonlinear dynamics constraints into sequential CMPC subproblems that can be solved through QP solvers. In each sequential iteration $j$, the Gait-Net predicts and updates the MPC sampling time $dt$ towards convergence and enables variable-frequency walking.
The spatial momentum and pose trajectories are updated at each iteration to reflect the kinematic configuration based on the iterative solution of $dt$, CoM location $\bm p_c \in \mathbb R^3$, and foot locations $\bm p_f\in \mathbb R^{3n_i}$,
providing a kinematically feasible reference. Once the terminal condition is met in the custom sequential solver, the control inputs are then mapped to motor commands in low-level control, which incorporates standard techniques such as inverse kinematics, contact Jacobian mapping, and joint-PD swing leg control \cite{di2018dynamic}. Notably, the full Gait-Net-augmented Kino-dynamic MPC is run at the beginning of each footstep to determine the step duration, the rest of the duration will incorporate the kino-dynamic MPC with the same MPC $dt$ throughout this very footstep. 


 
