\section{Conclusion}\label{sec:conclusion}

%%% Summarize main points and give brief overview of results %%%
Model-based approaches for time series classification can be effectively utilized when a model of the underlying dynamical process is available~\cite{shen2017classification}. 
Using structural identifiability (SI) analysis, structurally identifiable parameter combinations of the dynamical model can be obtained. 
Individual time series observations may then be represented as point estimates in the original parameter space or in the space of structurally identifiable parameter combinations. 
We introduced a novel method \eco{dubbed} \textbf{S}tructural-\textbf{I}dentifiability \textbf{M}apping (\myMethod{}) and demonstrated that \myMethod{} improves classification performance for the classification of time series data when taking a model-based approach and the underlying dynamical model is structurally unidentifiable.

% improved performance on classification task
Furthermore, it has been shown on a set of relevant example systems that classification performance is significantly improved when learning with data represented in the space of structurally identifiable parameter combinations. 
The increase in performance also persists when time series data of varying quality \mjc{are} \eco{produced}: for all types of time grids (dense, sparse and irregular) as well as for \eco{varying levels of} the observation\eco{al} noise introduced, learning in the space of structurally identifiable parameter combinations outperforms learning in the space of the original model parameters.

This work presents a first success in incorporating SI analysis directly into the learning process for classification. 
The \myMethod{} approach is straightforward and can be applied whenever a SI analysis can be carried out. 
An explicit reparametrisation of a given dynamical model in terms of fewer, structurally identifiable parameters is not needed in order to benefit from SI analysis. 
\pt{This is especially important in situations where explicit expressions for structurally identifiable parameter combinations are available following a SI analysis, but suitable model reparametrizations are not possible.}

Finally, outcomes of the learning process stay interpretable: while interpretation in the space of structurally identifiable parameter combinations is not straightforward, any insight in this space may be translated back to the space of the original model parameters $g^{-1}(\boldsymbol{\Phi})$, which, in turn, are meaningful in the domain-specific context.
