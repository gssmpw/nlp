% If you have an EPS/PDF photo (graphicx package needed) extra braces are
% needed around the contents of the optional argument to biography to prevent
% the LaTeX parser from getting confused when it sees the complicated
% \includegraphics command within an optional argument. (You could create
% your own custom macro containing the \includegraphics command to make things
% simpler here.)

% or if you just want to reserve a space for a photo:
% [{\includegraphics[width=1in,height=1.25in,clip,keepaspectratio]{Figures/Biography/JNorden.jpg}}]

\def\myWidth{1in}\def\myHeight{1.25in}

\begin{IEEEbiography}[{\includegraphics[width=\myWidth,height=\myHeight,clip,keepaspectratio]{JanisNorden.jpg}}]{Janis Norden}

received the bachelor's and master's degrees in Mathematics from the University of Groningen, The Netherlands. There, he is currently working toward the PhD degree at the Faculty of Science and Engineering. His research interests include machine learning and dynamical systems.
\end{IEEEbiography}

\begin{IEEEbiography}[{\includegraphics[width=\myWidth,height=\myHeight,clip,keepaspectratio]{ElisaOostwal.png}}]{Elisa Oostwal} obtained a bachelor's degree in Physics as well as a bachelor's and master's degree in Computing Science at the University of Groningen. She is currently employed as a PhD student within the Intelligent Systems research group, which is part of Bernoulli Institute for Mathematics, Computer Science and Artificial Intelligence. While her previous work was on the statistical physics of learning, her current focus is on combining machine learning with interpretable models such as dynamical systems.
\end{IEEEbiography}

\begin{IEEEbiography}[{\includegraphics[width=\myWidth,height=\myHeight,clip,keepaspectratio]{MichaelChappell.jpg}}]{Mike Chappell}
Mike Chappell (BSc, MSc, PhD in Mathematics, University of Warwick) is a Professor within the School of Engineering at the University of Warwick where he heads the Systems and Information Engineering Discipline Stream. 
He co-leads the Biomedical Engineering Institute within the School of Engineering and co-leads the Technology in Health theme within the University of Warwick’s Health Research Spotlight. 
He also heads the University’s Gait Laboratory. 
He has research expertise in the systems modelling and data analysis of biomedical, pharmacokinetic/pharmacodynamic, biological and Quantitative and Systems Pharmacology (QSP) processes. 
Much of his work relates to the application to these fields of techniques in system dynamics, non-linear systems, data analytics, control and system identification with particular expertise in structural and practical identifiability analyses. 
He has had close research collaboration with industry and hospital-based research groups for over 30 years, in particular, major pharma. 
He led the Marie Curie funded European Industrial Doctorate training network on Innovative Modelling for Pharmaceutical Advances through Collaborative Training (IMPACT) in collaboration with AstraZeneca and is a member of the Management Committee for the EPSRC/MRC funded QSP-UK network. 
He is also a member of the International Federation for Automatic Control (IFAC) Technical Committee 8.2 on Biological and Medical Systems.
\end{IEEEbiography}

\begin{IEEEbiography}[{\includegraphics[width=\myWidth,height=\myHeight,clip,keepaspectratio]{PeterTino.png}}]{Peter Ti\v{n}o}
(M.Sc. Slovak University of Technology, Ph.D. Slovak Academy of Sciences) 
%was a Fulbright Fellow with the NEC Research Institute, Princeton, NJ, USA, and a Post-Doctoral Fellow with the Austrian Research Institute for AI, Vienna, Austria, and with Aston University, Birmingham, UK. Since 2003, he 
%has been 
is with the School of Computer Science, University of Birmingham, %Edgbaston, Birmingham, UK, 
%since 2003 
where he is currently a Full Professor-Chair in Complex and Adaptive Systems. His current research interests include dynamical systems, machine learning, probabilistic modelling of structured data, evolutionary computation, and fractal analysis. Peter was a recipient of the U.K.- Hong-Kong Fellowship for Excellence in 2008, three Outstanding Paper of the Year Awards from the IEEE Trans. on Neural Networks in 1998 and 2011 and the IEEE Trans. on Evolutionary Computation in 2010, and the Best Paper Award at ICANN 2002. 
%He serves on the editorial boards of several journals.
\end{IEEEbiography}

\begin{IEEEbiography}[{\includegraphics[width=\myWidth,height=\myHeight,clip,keepaspectratio]{KerstinBunte.jpg}}]{Kerstin Bunte} %graduated at the Faculty of Technology at the University of Bielefeld, Germany, She 
is an associate professor and Rosalind Franklin Fellow at the Bernoulli Institute for Mathematics, Computer Science and Artificial Intelligence (University of Groningen, The Netherlands). % since July 2016. 
%She received her Ph.D. in Computer Science in 2011 from the University of Groningen. 
%She had research visits at the University of Rochester (USA) and postdoctoral positions have taken her to Bielefeld University (Germany), Aalto University (Finland), and Université catholique de Louvain (Belgium). 
In 2015 she got a European Marie Sk{\l}odowska-Curie Fellowship (Project ID: 659104) at the University of Birmingham and is partner in the European ITN: SUNDIAL, Project ID: 721463. 
Her recent work has focused on the development of interpretable machine learning techniques for interdisciplinary data analysis and inclusion of expert knowledge in form of dynamic models with applications in medicine, astronomy and smart industry. %and visualization.  
Further information can be obtained from http://www.cs.rug.nl/$\sim$kbunte/.
\end{IEEEbiography}

% if you will not have a photo at all:
%\begin{IEEEbiographynophoto}{John Doe}
%Biography text here.
%\end{IEEEbiographynophoto}

% insert where needed to balance the two columns on the last page with
% biographies
%\newpage

% You can push biographies down or up by placing
% a \vfill before or after them. The appropriate
% use of \vfill depends on what kind of text is
% on the last page and whether or not the columns
% are being equalized.

%\vfill

% Can be used to pull up biographies so that the bottom of the last one
% is flush with the other column.
%\enlargethispage{-5in}