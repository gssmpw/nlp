\section{Related Work}
\label{section 5}
\noindent{\bf Graph filtering-based approaches.} Within the realm of graph signal processing, GCN is regarded as a parameterized graph convolutional filter \cite{DBLP:conf/iclr/KipfW17, shen2021powerful}. As a representative approach, NGCF \cite{wang2019neural} was introduced by learning appropriate LPFs while capturing high-order collaborative signals inherent in user--item interactions \cite{shen2021powerful}. LightGCN \cite{he2020lightgcn} achieved convincing performance by eliminating linear transformation and non-linear activation from the GCN layers in NGCF. By closing the gap between LightGCN and graph filtering methods alongside a closed-form solution for the infinite-dimensional LightGCN, GF-CF \cite{shen2021powerful} stood out for achieving accurate recommendation performance while significantly reducing time consumption with its parameter-free nature. PGSP \cite{liu2023personalized} made use of a mixed-frequency filter that combines a linear LPF with an ideal LPF. Turbo-CF \cite{park2024turbo} introduced polynomial graph filtering for CF, enabling diverse LPF designs without costly matrix decomposition. In a subsequent study, to show its efficacy in another recommendation scenario, Group-GF \cite{kim2025leveraging} was developed by presenting multi-view polynomial graph filtering that offers a holistic view of complex user--group dynamics in the group recommendation task. 

\noindent\textbf{MC recommender systems.} Efforts have consistently been made to incorporate MC ratings to enhance the accuracy of recommendations. In one of the initial endeavors, a support vector regression-based approach \cite{jannach2012accuracy} was introduced to assess the relative importance of individual criteria ratings. CFM \cite{fan2021predicting} was formulated by collectively employing matrix factorization for MC rating matrices. DTTD \cite{chen2021deep} was developed by integrating cross-domain knowledge alongside side information. Moreover, in light of the extensive adoption of deep learning, there has been a continuous endeavor to develop DNN-based MC recommender systems. For instance, ExtendedSAE \cite{tallapally2018user} was introduced to capture the relationship between MC ratings using stacked autoencoders. LatentMC \cite{li2019latent} was designed with variational autoencoders to map user reviews into latent vectors, constituting latent MC ratings. DMCF \cite{nassar2020novel} was devised for predicting MC ratings using a DNN, with the predicted ratings being aggregated by another DNN. AEMC \cite{shambour2021deep} employed deep autoencoders, capturing non-linear relationships between users' preferences for criteria. As pioneer work on integrating light graph convolution into MC recommender systems, CPA-LGC \cite{park2023criteria} was devised to capture complex semantics in MC ratings. 
\vspace{-3mm}