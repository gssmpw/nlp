\documentclass{article} % For LaTeX2e
\usepackage{iclr2025_conference,times}

% Optional math commands from https://github.com/goodfeli/dlbook_notation.
%%%%% NEW MATH DEFINITIONS %%%%%

\usepackage{amsmath,amsfonts,bm}
\usepackage{derivative}
% Mark sections of captions for referring to divisions of figures
\newcommand{\figleft}{{\em (Left)}}
\newcommand{\figcenter}{{\em (Center)}}
\newcommand{\figright}{{\em (Right)}}
\newcommand{\figtop}{{\em (Top)}}
\newcommand{\figbottom}{{\em (Bottom)}}
\newcommand{\captiona}{{\em (a)}}
\newcommand{\captionb}{{\em (b)}}
\newcommand{\captionc}{{\em (c)}}
\newcommand{\captiond}{{\em (d)}}

% Highlight a newly defined term
\newcommand{\newterm}[1]{{\bf #1}}

% Derivative d 
\newcommand{\deriv}{{\mathrm{d}}}

% Figure reference, lower-case.
\def\figref#1{figure~\ref{#1}}
% Figure reference, capital. For start of sentence
\def\Figref#1{Figure~\ref{#1}}
\def\twofigref#1#2{figures \ref{#1} and \ref{#2}}
\def\quadfigref#1#2#3#4{figures \ref{#1}, \ref{#2}, \ref{#3} and \ref{#4}}
% Section reference, lower-case.
\def\secref#1{section~\ref{#1}}
% Section reference, capital.
\def\Secref#1{Section~\ref{#1}}
% Reference to two sections.
\def\twosecrefs#1#2{sections \ref{#1} and \ref{#2}}
% Reference to three sections.
\def\secrefs#1#2#3{sections \ref{#1}, \ref{#2} and \ref{#3}}
% Reference to an equation, lower-case.
\def\eqref#1{equation~\ref{#1}}
% Reference to an equation, upper case
\def\Eqref#1{Equation~\ref{#1}}
% A raw reference to an equation---avoid using if possible
\def\plaineqref#1{\ref{#1}}
% Reference to a chapter, lower-case.
\def\chapref#1{chapter~\ref{#1}}
% Reference to an equation, upper case.
\def\Chapref#1{Chapter~\ref{#1}}
% Reference to a range of chapters
\def\rangechapref#1#2{chapters\ref{#1}--\ref{#2}}
% Reference to an algorithm, lower-case.
\def\algref#1{algorithm~\ref{#1}}
% Reference to an algorithm, upper case.
\def\Algref#1{Algorithm~\ref{#1}}
\def\twoalgref#1#2{algorithms \ref{#1} and \ref{#2}}
\def\Twoalgref#1#2{Algorithms \ref{#1} and \ref{#2}}
% Reference to a part, lower case
\def\partref#1{part~\ref{#1}}
% Reference to a part, upper case
\def\Partref#1{Part~\ref{#1}}
\def\twopartref#1#2{parts \ref{#1} and \ref{#2}}

\def\ceil#1{\lceil #1 \rceil}
\def\floor#1{\lfloor #1 \rfloor}
\def\1{\bm{1}}
\newcommand{\train}{\mathcal{D}}
\newcommand{\valid}{\mathcal{D_{\mathrm{valid}}}}
\newcommand{\test}{\mathcal{D_{\mathrm{test}}}}

\def\eps{{\epsilon}}


% Random variables
\def\reta{{\textnormal{$\eta$}}}
\def\ra{{\textnormal{a}}}
\def\rb{{\textnormal{b}}}
\def\rc{{\textnormal{c}}}
\def\rd{{\textnormal{d}}}
\def\re{{\textnormal{e}}}
\def\rf{{\textnormal{f}}}
\def\rg{{\textnormal{g}}}
\def\rh{{\textnormal{h}}}
\def\ri{{\textnormal{i}}}
\def\rj{{\textnormal{j}}}
\def\rk{{\textnormal{k}}}
\def\rl{{\textnormal{l}}}
% rm is already a command, just don't name any random variables m
\def\rn{{\textnormal{n}}}
\def\ro{{\textnormal{o}}}
\def\rp{{\textnormal{p}}}
\def\rq{{\textnormal{q}}}
\def\rr{{\textnormal{r}}}
\def\rs{{\textnormal{s}}}
\def\rt{{\textnormal{t}}}
\def\ru{{\textnormal{u}}}
\def\rv{{\textnormal{v}}}
\def\rw{{\textnormal{w}}}
\def\rx{{\textnormal{x}}}
\def\ry{{\textnormal{y}}}
\def\rz{{\textnormal{z}}}

% Random vectors
\def\rvepsilon{{\mathbf{\epsilon}}}
\def\rvphi{{\mathbf{\phi}}}
\def\rvtheta{{\mathbf{\theta}}}
\def\rva{{\mathbf{a}}}
\def\rvb{{\mathbf{b}}}
\def\rvc{{\mathbf{c}}}
\def\rvd{{\mathbf{d}}}
\def\rve{{\mathbf{e}}}
\def\rvf{{\mathbf{f}}}
\def\rvg{{\mathbf{g}}}
\def\rvh{{\mathbf{h}}}
\def\rvu{{\mathbf{i}}}
\def\rvj{{\mathbf{j}}}
\def\rvk{{\mathbf{k}}}
\def\rvl{{\mathbf{l}}}
\def\rvm{{\mathbf{m}}}
\def\rvn{{\mathbf{n}}}
\def\rvo{{\mathbf{o}}}
\def\rvp{{\mathbf{p}}}
\def\rvq{{\mathbf{q}}}
\def\rvr{{\mathbf{r}}}
\def\rvs{{\mathbf{s}}}
\def\rvt{{\mathbf{t}}}
\def\rvu{{\mathbf{u}}}
\def\rvv{{\mathbf{v}}}
\def\rvw{{\mathbf{w}}}
\def\rvx{{\mathbf{x}}}
\def\rvy{{\mathbf{y}}}
\def\rvz{{\mathbf{z}}}

% Elements of random vectors
\def\erva{{\textnormal{a}}}
\def\ervb{{\textnormal{b}}}
\def\ervc{{\textnormal{c}}}
\def\ervd{{\textnormal{d}}}
\def\erve{{\textnormal{e}}}
\def\ervf{{\textnormal{f}}}
\def\ervg{{\textnormal{g}}}
\def\ervh{{\textnormal{h}}}
\def\ervi{{\textnormal{i}}}
\def\ervj{{\textnormal{j}}}
\def\ervk{{\textnormal{k}}}
\def\ervl{{\textnormal{l}}}
\def\ervm{{\textnormal{m}}}
\def\ervn{{\textnormal{n}}}
\def\ervo{{\textnormal{o}}}
\def\ervp{{\textnormal{p}}}
\def\ervq{{\textnormal{q}}}
\def\ervr{{\textnormal{r}}}
\def\ervs{{\textnormal{s}}}
\def\ervt{{\textnormal{t}}}
\def\ervu{{\textnormal{u}}}
\def\ervv{{\textnormal{v}}}
\def\ervw{{\textnormal{w}}}
\def\ervx{{\textnormal{x}}}
\def\ervy{{\textnormal{y}}}
\def\ervz{{\textnormal{z}}}

% Random matrices
\def\rmA{{\mathbf{A}}}
\def\rmB{{\mathbf{B}}}
\def\rmC{{\mathbf{C}}}
\def\rmD{{\mathbf{D}}}
\def\rmE{{\mathbf{E}}}
\def\rmF{{\mathbf{F}}}
\def\rmG{{\mathbf{G}}}
\def\rmH{{\mathbf{H}}}
\def\rmI{{\mathbf{I}}}
\def\rmJ{{\mathbf{J}}}
\def\rmK{{\mathbf{K}}}
\def\rmL{{\mathbf{L}}}
\def\rmM{{\mathbf{M}}}
\def\rmN{{\mathbf{N}}}
\def\rmO{{\mathbf{O}}}
\def\rmP{{\mathbf{P}}}
\def\rmQ{{\mathbf{Q}}}
\def\rmR{{\mathbf{R}}}
\def\rmS{{\mathbf{S}}}
\def\rmT{{\mathbf{T}}}
\def\rmU{{\mathbf{U}}}
\def\rmV{{\mathbf{V}}}
\def\rmW{{\mathbf{W}}}
\def\rmX{{\mathbf{X}}}
\def\rmY{{\mathbf{Y}}}
\def\rmZ{{\mathbf{Z}}}

% Elements of random matrices
\def\ermA{{\textnormal{A}}}
\def\ermB{{\textnormal{B}}}
\def\ermC{{\textnormal{C}}}
\def\ermD{{\textnormal{D}}}
\def\ermE{{\textnormal{E}}}
\def\ermF{{\textnormal{F}}}
\def\ermG{{\textnormal{G}}}
\def\ermH{{\textnormal{H}}}
\def\ermI{{\textnormal{I}}}
\def\ermJ{{\textnormal{J}}}
\def\ermK{{\textnormal{K}}}
\def\ermL{{\textnormal{L}}}
\def\ermM{{\textnormal{M}}}
\def\ermN{{\textnormal{N}}}
\def\ermO{{\textnormal{O}}}
\def\ermP{{\textnormal{P}}}
\def\ermQ{{\textnormal{Q}}}
\def\ermR{{\textnormal{R}}}
\def\ermS{{\textnormal{S}}}
\def\ermT{{\textnormal{T}}}
\def\ermU{{\textnormal{U}}}
\def\ermV{{\textnormal{V}}}
\def\ermW{{\textnormal{W}}}
\def\ermX{{\textnormal{X}}}
\def\ermY{{\textnormal{Y}}}
\def\ermZ{{\textnormal{Z}}}

% Vectors
\def\vzero{{\bm{0}}}
\def\vone{{\bm{1}}}
\def\vmu{{\bm{\mu}}}
\def\vtheta{{\bm{\theta}}}
\def\vphi{{\bm{\phi}}}
\def\va{{\bm{a}}}
\def\vb{{\bm{b}}}
\def\vc{{\bm{c}}}
\def\vd{{\bm{d}}}
\def\ve{{\bm{e}}}
\def\vf{{\bm{f}}}
\def\vg{{\bm{g}}}
\def\vh{{\bm{h}}}
\def\vi{{\bm{i}}}
\def\vj{{\bm{j}}}
\def\vk{{\bm{k}}}
\def\vl{{\bm{l}}}
\def\vm{{\bm{m}}}
\def\vn{{\bm{n}}}
\def\vo{{\bm{o}}}
\def\vp{{\bm{p}}}
\def\vq{{\bm{q}}}
\def\vr{{\bm{r}}}
\def\vs{{\bm{s}}}
\def\vt{{\bm{t}}}
\def\vu{{\bm{u}}}
\def\vv{{\bm{v}}}
\def\vw{{\bm{w}}}
\def\vx{{\bm{x}}}
\def\vy{{\bm{y}}}
\def\vz{{\bm{z}}}

% Elements of vectors
\def\evalpha{{\alpha}}
\def\evbeta{{\beta}}
\def\evepsilon{{\epsilon}}
\def\evlambda{{\lambda}}
\def\evomega{{\omega}}
\def\evmu{{\mu}}
\def\evpsi{{\psi}}
\def\evsigma{{\sigma}}
\def\evtheta{{\theta}}
\def\eva{{a}}
\def\evb{{b}}
\def\evc{{c}}
\def\evd{{d}}
\def\eve{{e}}
\def\evf{{f}}
\def\evg{{g}}
\def\evh{{h}}
\def\evi{{i}}
\def\evj{{j}}
\def\evk{{k}}
\def\evl{{l}}
\def\evm{{m}}
\def\evn{{n}}
\def\evo{{o}}
\def\evp{{p}}
\def\evq{{q}}
\def\evr{{r}}
\def\evs{{s}}
\def\evt{{t}}
\def\evu{{u}}
\def\evv{{v}}
\def\evw{{w}}
\def\evx{{x}}
\def\evy{{y}}
\def\evz{{z}}

% Matrix
\def\mA{{\bm{A}}}
\def\mB{{\bm{B}}}
\def\mC{{\bm{C}}}
\def\mD{{\bm{D}}}
\def\mE{{\bm{E}}}
\def\mF{{\bm{F}}}
\def\mG{{\bm{G}}}
\def\mH{{\bm{H}}}
\def\mI{{\bm{I}}}
\def\mJ{{\bm{J}}}
\def\mK{{\bm{K}}}
\def\mL{{\bm{L}}}
\def\mM{{\bm{M}}}
\def\mN{{\bm{N}}}
\def\mO{{\bm{O}}}
\def\mP{{\bm{P}}}
\def\mQ{{\bm{Q}}}
\def\mR{{\bm{R}}}
\def\mS{{\bm{S}}}
\def\mT{{\bm{T}}}
\def\mU{{\bm{U}}}
\def\mV{{\bm{V}}}
\def\mW{{\bm{W}}}
\def\mX{{\bm{X}}}
\def\mY{{\bm{Y}}}
\def\mZ{{\bm{Z}}}
\def\mBeta{{\bm{\beta}}}
\def\mPhi{{\bm{\Phi}}}
\def\mLambda{{\bm{\Lambda}}}
\def\mSigma{{\bm{\Sigma}}}

% Tensor
\DeclareMathAlphabet{\mathsfit}{\encodingdefault}{\sfdefault}{m}{sl}
\SetMathAlphabet{\mathsfit}{bold}{\encodingdefault}{\sfdefault}{bx}{n}
\newcommand{\tens}[1]{\bm{\mathsfit{#1}}}
\def\tA{{\tens{A}}}
\def\tB{{\tens{B}}}
\def\tC{{\tens{C}}}
\def\tD{{\tens{D}}}
\def\tE{{\tens{E}}}
\def\tF{{\tens{F}}}
\def\tG{{\tens{G}}}
\def\tH{{\tens{H}}}
\def\tI{{\tens{I}}}
\def\tJ{{\tens{J}}}
\def\tK{{\tens{K}}}
\def\tL{{\tens{L}}}
\def\tM{{\tens{M}}}
\def\tN{{\tens{N}}}
\def\tO{{\tens{O}}}
\def\tP{{\tens{P}}}
\def\tQ{{\tens{Q}}}
\def\tR{{\tens{R}}}
\def\tS{{\tens{S}}}
\def\tT{{\tens{T}}}
\def\tU{{\tens{U}}}
\def\tV{{\tens{V}}}
\def\tW{{\tens{W}}}
\def\tX{{\tens{X}}}
\def\tY{{\tens{Y}}}
\def\tZ{{\tens{Z}}}


% Graph
\def\gA{{\mathcal{A}}}
\def\gB{{\mathcal{B}}}
\def\gC{{\mathcal{C}}}
\def\gD{{\mathcal{D}}}
\def\gE{{\mathcal{E}}}
\def\gF{{\mathcal{F}}}
\def\gG{{\mathcal{G}}}
\def\gH{{\mathcal{H}}}
\def\gI{{\mathcal{I}}}
\def\gJ{{\mathcal{J}}}
\def\gK{{\mathcal{K}}}
\def\gL{{\mathcal{L}}}
\def\gM{{\mathcal{M}}}
\def\gN{{\mathcal{N}}}
\def\gO{{\mathcal{O}}}
\def\gP{{\mathcal{P}}}
\def\gQ{{\mathcal{Q}}}
\def\gR{{\mathcal{R}}}
\def\gS{{\mathcal{S}}}
\def\gT{{\mathcal{T}}}
\def\gU{{\mathcal{U}}}
\def\gV{{\mathcal{V}}}
\def\gW{{\mathcal{W}}}
\def\gX{{\mathcal{X}}}
\def\gY{{\mathcal{Y}}}
\def\gZ{{\mathcal{Z}}}

% Sets
\def\sA{{\mathbb{A}}}
\def\sB{{\mathbb{B}}}
\def\sC{{\mathbb{C}}}
\def\sD{{\mathbb{D}}}
% Don't use a set called E, because this would be the same as our symbol
% for expectation.
\def\sF{{\mathbb{F}}}
\def\sG{{\mathbb{G}}}
\def\sH{{\mathbb{H}}}
\def\sI{{\mathbb{I}}}
\def\sJ{{\mathbb{J}}}
\def\sK{{\mathbb{K}}}
\def\sL{{\mathbb{L}}}
\def\sM{{\mathbb{M}}}
\def\sN{{\mathbb{N}}}
\def\sO{{\mathbb{O}}}
\def\sP{{\mathbb{P}}}
\def\sQ{{\mathbb{Q}}}
\def\sR{{\mathbb{R}}}
\def\sS{{\mathbb{S}}}
\def\sT{{\mathbb{T}}}
\def\sU{{\mathbb{U}}}
\def\sV{{\mathbb{V}}}
\def\sW{{\mathbb{W}}}
\def\sX{{\mathbb{X}}}
\def\sY{{\mathbb{Y}}}
\def\sZ{{\mathbb{Z}}}

% Entries of a matrix
\def\emLambda{{\Lambda}}
\def\emA{{A}}
\def\emB{{B}}
\def\emC{{C}}
\def\emD{{D}}
\def\emE{{E}}
\def\emF{{F}}
\def\emG{{G}}
\def\emH{{H}}
\def\emI{{I}}
\def\emJ{{J}}
\def\emK{{K}}
\def\emL{{L}}
\def\emM{{M}}
\def\emN{{N}}
\def\emO{{O}}
\def\emP{{P}}
\def\emQ{{Q}}
\def\emR{{R}}
\def\emS{{S}}
\def\emT{{T}}
\def\emU{{U}}
\def\emV{{V}}
\def\emW{{W}}
\def\emX{{X}}
\def\emY{{Y}}
\def\emZ{{Z}}
\def\emSigma{{\Sigma}}

% entries of a tensor
% Same font as tensor, without \bm wrapper
\newcommand{\etens}[1]{\mathsfit{#1}}
\def\etLambda{{\etens{\Lambda}}}
\def\etA{{\etens{A}}}
\def\etB{{\etens{B}}}
\def\etC{{\etens{C}}}
\def\etD{{\etens{D}}}
\def\etE{{\etens{E}}}
\def\etF{{\etens{F}}}
\def\etG{{\etens{G}}}
\def\etH{{\etens{H}}}
\def\etI{{\etens{I}}}
\def\etJ{{\etens{J}}}
\def\etK{{\etens{K}}}
\def\etL{{\etens{L}}}
\def\etM{{\etens{M}}}
\def\etN{{\etens{N}}}
\def\etO{{\etens{O}}}
\def\etP{{\etens{P}}}
\def\etQ{{\etens{Q}}}
\def\etR{{\etens{R}}}
\def\etS{{\etens{S}}}
\def\etT{{\etens{T}}}
\def\etU{{\etens{U}}}
\def\etV{{\etens{V}}}
\def\etW{{\etens{W}}}
\def\etX{{\etens{X}}}
\def\etY{{\etens{Y}}}
\def\etZ{{\etens{Z}}}

% The true underlying data generating distribution
\newcommand{\pdata}{p_{\rm{data}}}
\newcommand{\ptarget}{p_{\rm{target}}}
\newcommand{\pprior}{p_{\rm{prior}}}
\newcommand{\pbase}{p_{\rm{base}}}
\newcommand{\pref}{p_{\rm{ref}}}

% The empirical distribution defined by the training set
\newcommand{\ptrain}{\hat{p}_{\rm{data}}}
\newcommand{\Ptrain}{\hat{P}_{\rm{data}}}
% The model distribution
\newcommand{\pmodel}{p_{\rm{model}}}
\newcommand{\Pmodel}{P_{\rm{model}}}
\newcommand{\ptildemodel}{\tilde{p}_{\rm{model}}}
% Stochastic autoencoder distributions
\newcommand{\pencode}{p_{\rm{encoder}}}
\newcommand{\pdecode}{p_{\rm{decoder}}}
\newcommand{\precons}{p_{\rm{reconstruct}}}

\newcommand{\laplace}{\mathrm{Laplace}} % Laplace distribution

\newcommand{\E}{\mathbb{E}}
\newcommand{\Ls}{\mathcal{L}}
\newcommand{\R}{\mathbb{R}}
\newcommand{\emp}{\tilde{p}}
\newcommand{\lr}{\alpha}
\newcommand{\reg}{\lambda}
\newcommand{\rect}{\mathrm{rectifier}}
\newcommand{\softmax}{\mathrm{softmax}}
\newcommand{\sigmoid}{\sigma}
\newcommand{\softplus}{\zeta}
\newcommand{\KL}{D_{\mathrm{KL}}}
\newcommand{\Var}{\mathrm{Var}}
\newcommand{\standarderror}{\mathrm{SE}}
\newcommand{\Cov}{\mathrm{Cov}}
% Wolfram Mathworld says $L^2$ is for function spaces and $\ell^2$ is for vectors
% But then they seem to use $L^2$ for vectors throughout the site, and so does
% wikipedia.
\newcommand{\normlzero}{L^0}
\newcommand{\normlone}{L^1}
\newcommand{\normltwo}{L^2}
\newcommand{\normlp}{L^p}
\newcommand{\normmax}{L^\infty}

\newcommand{\parents}{Pa} % See usage in notation.tex. Chosen to match Daphne's book.

\DeclareMathOperator*{\argmax}{arg\,max}
\DeclareMathOperator*{\argmin}{arg\,min}

\DeclareMathOperator{\sign}{sign}
\DeclareMathOperator{\Tr}{Tr}
\let\ab\allowbreak



\usepackage{hyperref}
\usepackage{url}
\usepackage{array}  % For advanced table features
\usepackage{boldline}
\usepackage{booktabs}  % For professional-looking tables
\usepackage{xcolor}    % For custom colors
\usepackage{colortbl}  % For coloring table cells
\usepackage{siunitx}   % For aligning numbers
\usepackage{makecell}  % For line breaks in cells
\usepackage{caption}   % For customizing captions
\usepackage{graphicx}

\usepackage{amsmath,amsfonts,amssymb}       %
\usepackage{nicefrac}       %
\usepackage{microtype}      %
\usepackage{float}

\usepackage{algorithm,algorithmic}

\usepackage{wrapfig}

\usepackage{microtype}
\usepackage{graphicx}
\usepackage{subcaption}
\usepackage{chngcntr}
\definecolor{lightgray}{gray}{0.9}
\definecolor{midgray}{gray}{0.7}
\usepackage{enumitem}

\usepackage{amsmath}
\usepackage{amssymb}
\usepackage{mathtools}
\usepackage{amsthm}
\usepackage{textcomp}
\usepackage{newtxmath}
\newcommand{\std}[1]{\textcolor{black}{\scriptsize{$\pm #1$}}}
\newcommand{\highlight}[1]{\cellcolor{blue!10}{#1}}

% venues:
% https://delta-workshop.github.io/
% https://sites.google.com/view/fpiworkshop/about


\title{Shaping Inductive Bias in Diffusion Models through Frequency-Based Noise Control}

% Authors must not appear in the submitted version. They should be hidden
% as long as the \iclrfinalcopy macro remains commented out below.
% Non-anonymous submissions will be rejected without review.

\author{Thomas Jiralerspong \\%\thanks{ Use footnote for providing further information
% about author (webpage, alternative address)---\emph{not} for acknowledging
% funding agencies.  Funding acknowledgements go at the end of the paper.} \\
Mila - Quebec AI Institute and Université de Montréal, Quebec\\
\texttt{thomas.jiralerspong@mila.quebec} \\
\AND
Berton Earnshaw, Jason Hartford \\
Valence Labs \\
\texttt{\{berton.earnshaw,jason.hartford\}@recursion.com} \\
\AND
Yoshua  Bengio, Luca Scimeca \\%\thanks{ Use footnote for providing further information
% about author (webpage, alternative address)---\emph{not} for acknowledging
% funding agencies.  Funding acknowledgements go at the end of the paper.} \\
Mila - Quebec AI Institute and Université de Montréal, Quebec\\
\texttt{\{yoshua.bengio,luca.scimeca\}@mila.quebec} 
}

% The \author macro works with any number of authors. There are two commands
% used to separate the names and addresses of multiple authors: \And and \AND.
%
% Using \And between authors leaves it to \LaTeX{} to determine where to break
% the lines. Using \AND forces a linebreak at that point. So, if \LaTeX{}
% puts 3 of 4 authors names on the first line, and the last on the second
% line, try using \AND instead of \And before the third author name.

\newcommand{\fix}{\marginpar{FIX}}
\newcommand{\new}{\marginpar{NEW}}

\iclrfinalcopy % Uncomment for camera-ready version, but NOT for submission.
\begin{document}


\maketitle

\begin{abstract}
% placeholder abstract -- to be changed/modified
Diffusion Probabilistic Models (DPMs) are powerful generative models that have achieved unparalleled success in a number of generative tasks. In this work, we aim to build inductive biases into the training and sampling of diffusion models to better accommodate the target distribution of the data to model. For topologically structured data, we devise a frequency-based noising operator to purposefully manipulate, and set, these inductive biases. We first show that appropriate manipulations of the noising forward process can lead DPMs to focus on particular aspects of the distribution to learn. We show that different datasets necessitate different inductive biases, and that appropriate frequency-based noise control induces increased generative performance compared to standard diffusion. Finally, we demonstrate the possibility of ignoring information at particular frequencies while learning. We show this in an image corruption and recovery task, where we train a DPM to recover the original target distribution after severe noise corruption.
\end{abstract}


%%%%%%%%%%%%%%%%%%%%%%%%%%%%%%%%%%%%%%%%%%%%%%%%%%%%%%%%%%%%%%%%%%%%%%%%%%%%%%%%%%%%%%%%%%%%%%%%%%%%%%%%%%%%%%%%%%%%%%%%
%%%%%%%%%%%%%%%%%%%%%%%%%%%%%%%%%%%%%%%        INTRODUCTION         %%%%%%%%%%%%%%%%%%%%%%%%%%%%%%%%%%%%%%%%%%%%%%%%%%%%
%%%%%%%%%%%%%%%%%%%%%%%%%%%%%%%%%%%%%%%%%%%%%%%%%%%%%%%%%%%%%%%%%%%%%%%%%%%%%%%%%%%%%%%%%%%%%%%%%%%%%%%%%%%%%%%%%%%%%%%%

\section{Introduction}
\label{sec:intro}

Diffusion Probabilistic Models (DPMs) have recently emerged as powerful tools for approximating complex data distributions, finding applications across a variety of domains, from image synthesis to probabilistic modeling \citep{yang_diffusion_2024, ho_denoising_2020, sohl-dickstein_deep_2015, venkatraman2024amortizing, sendera2024diffusion}. These models operate by gradually transforming data into noise through a defined diffusion process and training a denoising model~\citep{vincent2008extracting,alain2014regularized} to learn to reverse this process, enabling the generation of samples from the desired distribution via appropriate scheduling. Despite their success, the inductive biases inherent in diffusion models remain largely unexplored, particularly in how these biases influence model performance and the types of distributions that can be effectively modeled.

Inductive biases are known to play a crucial role in deep learning models, guiding the learning process by favoring certain types of data representations over others \citep{geirhos2019imagenet, bietti2019inductive, tishby2015deep}. A well-studied example is the Frequency Principle (F-principle) or spectral bias, which suggests that neural networks tend to learn low-frequency components of data before high-frequency ones \citep{xu_training_2019, rahaman_spectral_2019}. Another related phenomenon is what is also known as the simplicity bias, or shortcut learning \citep{Geirhos2020, scimeca2021which, scimeca2023shortcut}, in which models are observed to preferentially pick up on simple, easy-to-learn, and often spuriously correlated features in the data for prediction. 
If left implicit, it is often unclear whether these biases will improve or hurt the performance of generative model on downstream task, and they could lead to flawed approximations\citep{scimeca2023leveraging}.
% goal 
 In this work, we aim to explicitly tailor the  
%benefits to building 
inductive biases of DPMs to better learn the target distribution of interest.% to approximate. 



Recent studies have begun to explore the inductive biases inherent in diffusion models. For instance, Kadkhodaie et al. (2023) analyze how the inductive biases of deep neural networks trained for image denoising contribute to the generalization capabilities of diffusion models. They demonstrate that these biases lead to geometry-adaptive harmonic representations, which play a crucial role in the models' ability to generalize beyond the training data~\citep{kadkhodaie2023generalization}.
Similarly, Zhang et al. (2024) investigate the role of inductive and primacy biases in diffusion models, particularly in the context of reward optimization. They propose methods to mitigate overoptimization by aligning the models' inductive biases with desired outcomes~\citep{zhang2024confronting}. Other methods, such as noise schedule adaptations~\citep{sahoo_diffusion_2024} and the introduction of non-Gaussian noise \citep{bansal_cold_2022} have shown promise in improving the performance of diffusion models on various tasks. However, the exploration of frequency domain techniques within diffusion models is a relatively new area of interest. One of the pioneering studies in this domain investigates the application of diffusion models to time series data, where frequency domain methods have shown potential for capturing temporal dependencies more effectively \citep{crabbé2024timeseriesdiffusionfrequency}. Similarly, the integration of spatial frequency components into the denoising process has been explored for enhancing image generation tasks \citep{Qian_2024_CVPR, yuan2023spatialfrequencyunetdenoisingdiffusion}, showcasing the importance of considering frequency-based techniques as a means of refining the inductive biases of diffusion models.

 
% hypothesis
In this work, we explore a new avenue, to build inductive biases in DPMs by frequency-based noise control. The main hypothesis in this paper is that the noising operator in a diffusion model has a direct influence on the model's representation of the data. Intuitively, the information erased by the noising process is the very information that the denoising model has pressure to learn, so that reconstruction is possible. Accordingly, we propose that by strategically manipulating the noising operation, we can effectively steer the model to learn particular aspects of the data distribution. We focus our attention to the generative learning of topologically structured data, and propose an approach that involves designing a frequency-based noise schedule that selectively emphasizes or de-emphasizes certain frequency components during the noising process. In this paper, we refer to our approach as \emph{frequency diffusion}. Because the Fourier transform of a Gaussian is just another Gaussian in the frequency domain, this approach allows us to maintain the Gaussian assumptions of the diffusion process while reorienting the noising operator within the frequency domain, enabling the generation of Gaussian noise at different frequencies and thereby influencing the model’s learning trajectory.

% proof 
We report several findings. First, we show that when the information content in the data lies more heavily in particular frequencies, frequency diffusion yields better samplers. Furthermore, we test this in several natural datasets, and show that depending on the dataset characteristic, different settings of our frequency diffusion approach yield optimal results, often with comparable or superior performance to standard diffusion. Finally, we show that through frequency-denoising we can recover complex distributions after severe noise corruption at particular frequencies, opening interesting venues for applications within the generative landscape. 
%(TODO ADD DISCUSSION SECTION AND REFER TO IT).

We summarize our contributions as follows:

\begin{enumerate}
\item We introduce a frequency-informed noising operator that can shape the inductive biases of diffusion models.
\item We empirically show that \emph{frequency diffusion} can steer models to better approximate information at particular frequencies of the underlying data distribution.
\item We provide empirical evidence that models trained with frequency-based noise schedules can outperform traditional diffusion schedules across multiple datasets.
\item We show that through frequency-denoising we can recover complex distributions after severe noise corruption at particular frequencies.
\end{enumerate}

% This paper is structured as follows. In \autoref{sec:related_work} we briefly discuss related work, in \autoref{sec:methods} we discuss the main methods in the paper, including an introduction to DPMS as used in this work in \autoref{sec:methods:dpms}, the frequency noise scheduling proposed in \autoref{sec:methods:freq_noise} and the frequency analysis used in \autoref{sec:methods:analysis}. We present the main results in \autoref{sec:results}, and finally, discuss and conclude in section \autoref{sec:discussion_and_conclusion}.


% \section{Related Work}
\label{sec:related_work}

% \paragraph{DPMs Inductive Biases}



% \paragraph{DPMs and the frequency domain}



% \paragraph{DPMs and ?}

% Differently than previous work, we incorporating frequency-informed noising operators into the diffusion process. This not only allows for more targeted control over the learned representations but also opens new avenues for optimizing the diffusion model’s performance across diverse datasets.

% \color{red}MORE (tbd)\color{black}



%%%%%%%%%%%%%%%%%%%%%%%%%%%%%%%%%%%%%%%%%%%%%%%%%%%%%%%%%%%%%%%%%%%%%%%%%%%%%%%%%%%%%%%%%%%%%%%%%%%%%%%%%%%%%%%%%%%%%%%%
%%%%%%%%%%%%%%%%%%%%%%%%%%%%%%%%%%%%%%%%%        METHODS         %%%%%%%%%%%%%%%%%%%%%%%%%%%%%%%%%%%%%%%%%%%%%%%%%%%%%%%
%%%%%%%%%%%%%%%%%%%%%%%%%%%%%%%%%%%%%%%%%%%%%%%%%%%%%%%%%%%%%%%%%%%%%%%%%%%%%%%%%%%%%%%%%%%%%%%%%%%%%%%%%%%%%%%%%%%%%%%%

\section{Methods} \label{sec:methods}

\subsection{Denoising Probabilistic Models (DPMs)}\label{sec:methods:dpms}

Denoising Probabilistic Models, are a class of generative models that learn to reconstruct complex data distributions by reversing a gradual noising process. DPMs are characterized by a \emph{forward} and \emph{backward} process. The \emph{forward process} defines how data is corrupted, typically by Gaussian noise, over time. Given a data point $\mathbf{x}_0$ sampled from the data distribution $q(\mathbf{x}_0)$, the noisy versions of the data $\mathbf{x}_1, \mathbf{x}_2, \ldots, \mathbf{x}_T$ are generated according to:

\begin{equation}
q(\mathbf{x}_t \mid \mathbf{x}_{t-1}) = \mathcal{N}(\mathbf{x}_t; \sqrt{\alpha_t} \mathbf{x}_{t-1}, (1 - \alpha_t)\mathbf{I})
\end{equation}

with $\alpha_t$ variance schedule.  The \emph{reverse process} models the denoising operation, attempting to recover $\mathbf{x}_{t-1}$ from $\mathbf{x}_t$:

\begin{equation}
p_\theta(\mathbf{x}_{t-1} \mid \mathbf{x}_t) = \mathcal{N}(\mathbf{x}_{t-1}; \mu_\theta(\mathbf{x}_t, t), \sigma^2_t \mathbf{I}),
\end{equation}

where $\mu_\theta(\mathbf{x}_t, t)$ is predicted by a neural network $f_\theta$, and the variance $\sigma^2_t$ is can be fixed, learned, or precomputed based on a schedule. We often train the denoising model by minimizing a variational bound on the negative log-likelihood:

\begin{equation}
L = \mathbb{E}_{t, \mathbf{x}_0, \mathbf{\epsilon}} \left[ \left\| \mathbf{\epsilon} - \mathbf{\epsilon}_\theta(\mathbf{x}_t, t) \right\|^2 \right]
\end{equation}

where $\mathbf{\epsilon}$ is the Gaussian noise added to $\mathbf{x}_0$, and $\mathbf{\epsilon}_\theta$ is the model’s prediction of this noise. To generate new samples, we sample from a Gaussian distribution and apply the learned reverse process iteratively, often starting from a sample drawn from a simple Gaussian noise distribution.

\subsection{Frequency Diffusion}\label{sec:methods:freq_noise}


\begin{figure*}
    \centering
    \includegraphics[width=\textwidth]{figures/frequency_diffusion.pdf}
    \caption{Frequency diffusion under a generalized framework.}
    \label{fig:freq_diffusion}
\end{figure*}

The objective of this section is to generate spatial Gaussian noise whose frequency content can be systematically manipulated according to an arbitrary weighting function. In \autoref{sec:methods:dpms}, we describe how $ \mathbf{x}_t $ is obtained from $ \mathbf{x}_{t-1} $ by adding Gaussian noise sampled from a normal distribution to the sample at time step $ t-1 $. Specifically, we can sample $ \mathbf{\epsilon}_t \sim \mathcal{N}(0, \mathbf{I}) $ and obtain $\mathbf{x}_t$ as:
\begin{equation}
\mathbf{x}_t = \sqrt{\alpha_t}\,\mathbf{x}_{t-1} \;+\; \sqrt{1 - \alpha_t}\,\mathbf{\epsilon}.
\end{equation}


Let us denote by \(\mathbf{x}\in\mathbb{R}^{H\times W}\) an image (or noise field) in the spatial domain, and by \(\mathcal{F}\) the two-dimensional Fourier transform operator. We let \(\mathbf{N}_\text{freq} \in \mathbb{C}^{H\times W}\) be a complex-valued random field whose real and imaginary parts are i.i.d.~Gaussian:
\begin{equation}
\mathbf{N}_\text{freq} 
\;=\; 
\mathbf{N}_\text{real} \;+\; i\,\mathbf{N}_\text{imag},
\quad
\mathbf{N}_\text{real}, \,\mathbf{N}_\text{imag} 
\;\sim\; 
\mathcal{N}\bigl(0, \mathbf{I}\bigr).
\end{equation}
where each pixel (or frequency bin) in \(\mathbf{N}_\text{real}\) and \(\mathbf{N}_\text{imag}\) is drawn independently from a standard normal distribution. We introduce a \emph{weighting function} \(w(f_x, f_y)\) that scales the amplitude of each frequency component.  Let \(\mathbf{f} = (f_x, f_y)\) denote coordinates in frequency space, where $f_x \;=\; \frac{k_x}{W}$, $f_y \;=\; \frac{k_y}{H}$, and $k_x, k_y$ are integer indices (ranging over the width and height), while \(H\) and \(W\) are the image dimensions. We define the frequency-controlled noise $\mathbf{N}_\text{freq}^{(w)}(\mathbf{f})$ as:
\begin{equation} \label{eq:general_weight}
\mathbf{N}_\text{freq}^{(w)}(\mathbf{f})
\;=\;
\mathbf{N}_\text{freq}(\mathbf{f})
\;\odot\;
w(\mathbf{f}),
\end{equation}
After applying \(w(\mathbf{f})\) in the frequency domain, we invert back to the spatial domain to obtain \(\mathbf{\epsilon}^{(w)}\), our \emph{frequency-shaped} noise:
\begin{equation} \label{eq:general_noise_spatial}
\mathbf{\epsilon}^{(w)} 
\;=\; 
\Re\Bigl(\mathcal{F}^{-1}\bigl(\mathbf{N}_\text{freq}^{(w)}\bigr)\Bigr),
\end{equation}
where \(\Re(\cdot)\) denotes the real part, ensuring that our final noise field is purely real. 

In summary, any frequency weighting can be represented in this unified framework:
\begin{equation*}
\mathbf{\epsilon}
\xrightarrow{\;\mathcal{F}\;}
\mathbf{N}_\text{freq}
\xrightarrow{\;w(\mathbf{f})\;}
\mathbf{N}_\text{freq}^{(w)}
\xrightarrow{\;\mathcal{F}^{-1}\;}
\mathbf{\epsilon}^{(w)}.
\end{equation*}
With this, we have a simple mechanism for generating noise whose power spectrum can purposefully controlled. Note that standard white Gaussian noise is a special case of this formulation, where \(w(\mathbf{f})=1\) for all \(\mathbf{f}\). In contrast, more sophisticated weightings allow one to emphasize, de-emphasize, or even remove specific bands of the frequency domain.



\subsection{Frequency Noise operators}
\label{supp:sec:freq_noise_operator}

In this work, the design of $w(\mathbf{f})$ is especially important. In this section, we propose several alternatives, while showing empirical results on a particular choice of $w(\mathbf{f})$. 

\subsubsection*{Power-Law Weighting.}
A natural alternative choice is the power-law weighting, expressed as:
\begin{equation}
w(\mathbf{f}) \;=\; \|\mathbf{f}\|^{\alpha},
\end{equation}
where \(\mathbf{f} = (f_x,f_y)\) denotes a frequency coordinate, and the exponent \(\alpha\) determines which frequencies are amplified or suppressed.
% \begin{itemize}
%     \item If \(\alpha > 0\), higher frequencies (\(\|\mathbf{f}\|\) large) are emphasized relative to lower ones, yielding more \emph{granular} textures in the spatial domain.
%     \item If \(\alpha < 0\), lower frequencies are emphasized, leading to \emph{larger} spatial variations.
%     \item If \(\alpha = 0\), we recover a flat weighting \(w(\mathbf{f}) = 1\), which corresponds to standard white Gaussian noise in the spatial domain.
% \end{itemize}
Power-law weighting is popular in the modeling if natural phenomena (e.g., fractal landscapes, turbulence) where the energy distribution often follows an approximate power spectrum \citep{van1996modelling}.

\subsubsection*{Exponential Decay Weighting}
Another alternative is an exponential decay function, defined as as:
\begin{equation}
w(\mathbf{f}) \;=\; \exp\!\bigl(-\beta\,\|\mathbf{f}\|^{2}\bigr),
\end{equation}
where \(\beta>0\), and frequencies with larger norms \(\|\mathbf{f}\|\) are exponentially suppressed. This weighting effectively imposes spatial correlations, e.g. for \(\beta\) close to 0 the function induces the retention of more high-frequency components, while for large \(\beta\), the function quickly damps out high frequencies, resulting in a smoothing of the spatial domain.

\subsubsection*{Band-Pass Masking and Two-Band Mixture}
Finally, a \emph{band-pass mask} can be viewed as a special case of a more general weighting function:
\begin{equation}
w(\mathbf{f}) \;\in\; \{0,1\}.
\end{equation}
In this case, the frequency domain is split into a set of permitted and excluded regions, or radial thresholds. We this, we can construct several types of filters, including a low-pass filter retaining only frequencies below a cutoff (e.g., \(\|\mathbf{f}\| \leq \omega_c\)) a high-pass filter keeping only frequencies above a cutoff, or more generally a filter restricting \(\|\mathbf{f}\|\) to lie between two thresholds \([a_{\mathrm{min}},b_{\mathrm{max}}]\).
We thus define a simple band pass filter as:
\begin{equation}
w(\mathbf{f}) \;=\; 
\mathbf{M}_{[a,b]}(f_x, f_y)
\;=\;
\begin{cases}
1, & \text{if } a \leq d(f_x, f_y) \leq b, \\
0, & \text{otherwise}.
\end{cases}
\end{equation}
Here, $d(f_x, f_y) = \sqrt{\left(f_x - \tfrac{1}{2}\right)^2 + \left(f_y - \tfrac{1}{2}\right)^2}$ measures the radial distance in frequency space. In this special case, $ w(\mathbf{f}) $ is simply a \emph{binary} mask, selecting only those frequencies within $ [a, b] $. 

For the experiments in this paper we formulate a simple two-band mixture, where, we limit ourselves to constructing noise as a simple linear combination of two band-pass filtered noise components. Specifically, as in the original band-based approach, we generate frequency-filtered noise $\mathbf{\epsilon}_f$ via:
\begin{equation} \label{eq:high_low_formulation}
\mathbf{\epsilon}_f 
\;=\; 
\gamma_l \,\mathbf{\epsilon}_{[a_l, b_l]}
\;+\;
\gamma_h \,\mathbf{\epsilon}_{[a_h, b_h]},
\end{equation}
where $ \gamma_l $ and $ \gamma_h $ denote the relative contributions of a low- and a high-frequency noise components, each filtering noise respectively in the ranges $[a_l, b_l]$ (low-frequency range) and $[a_h, b_h]$ (high-frequency range).
We uniquely refer to $\epsilon_{[a, b]}$ as the noise filtered in the $[a, b]$ frequency range following \autoref{eq:general_weight} and \autoref{eq:general_noise_spatial}. Standard Gaussian noise emerges as a particular instance (with $ \gamma_l = 0.5 $, $ \gamma_h = 0.5 $, $ a_l = 0 $, $ b_l = 0.5 $, $ a_h = 0.5 $, and $ b_h = 1 $) of this formulation.



\subsection{datasets} \label{suppl:sec:datasets}
For the experiments, we consider five datasets, namely: MNIST, CIFAR-10, Domainnet-Quickdraw, Wiki-Art and CelebA; providing examples of widely different visual distributions, scales, and domain-specific statistics.

\paragraph{MNIST:} MNIST consists of \(70,000\) grayscale images of handwritten digits (0-9)~\citep{dsprites17}. MNIST provides a simple test-bed to for the hypothesis in this work, as a well understood dataset with well structured, and visually coherent samples.

\paragraph{CIFAR-10:} CIFAR-10 contains \(60,000\) color images distributed across 10 object categories \citep{krizhevsky2009learning}. The dataset is highly diverse in terms of object appearance, backgrounds, and colors, with the wide-ranging visual variations across classes like animals, vehicles, and other common objects.

\paragraph{DomainNet-Quickdraw:} DomainNet-Quickdraw features \(120,750\) sketch-style images,
 These images, drawn in a minimalistic, abstract style, present a distribution that is drastically different from natural images, with sparse details and heavy visual simplifications. 


\paragraph{WikiArt:} WikiArt consists of over \(81,000\) images of artwork spanning a wide array of artistic styles, genres, and historical periods \citep{saleh2015large}. The dataset encompasses a rich and varied distribution of textures, color palettes, and compositions, making it a challenging benchmark for generative models, which must capture both the global structure and fine-grained stylistic variations that exist across different forms of visual art.

\paragraph{CelebA:} CelebA contains \(202,599\) images of celebrity faces, each \(178 \times 218\) pixels in resolution ~\citep{liu2015faceattributes}. The dataset presents a diverse distribution of human faces with variations in pose, lighting, and facial expressions. 
%%%%%%%%%%%%%%%%%%%%%%%%%%%%%%%%%%%%%%%%%%%%%%%%%%%%%%%%%%%%%%%%%%%%%%%%%%%%%%%%%%%%%%%%%%%%%%%%%%%%%%%%%%%%%%%%%%%%%%%%
%%%%%%%%%%%%%%%%%%%%%%%%%%%%%%%%%%%%%%%%%        RESULTS         %%%%%%%%%%%%%%%%%%%%%%%%%%%%%%%%%%%%%%%%%%%%%%%%%%%%%%%
%%%%%%%%%%%%%%%%%%%%%%%%%%%%%%%%%%%%%%%%%%%%%%%%%%%%%%%%%%%%%%%%%%%%%%%%%%%%%%%%%%%%%%%%%%%%%%%%%%%%%%%%%%%%%%%%%%%%%%%%


\section{Results} \label{sec:results}
All experiments involve separately training and testing DPMs with various \emph{frequency diffusion} schedules, as well as baseline standard denoising diffusion training. We use DDPM fast sampling \citep{ho2020denoising} to efficiently generate samples for all reported metrics. Across the experiments, we report FID and KID scores as similarity score estimate metrics of the generated samples with respect to a held-out set of data samples. In all relevant experiments, we compute the metrics on embeddings from block 768 of a pre-trained Inception v3 model.


\subsection{Improved Diffusion Sampling via Frequency-Based Noise Control}
% There is a relationship between frequency analysis of information loading and best hyperparameters… To Test. (can we predict the noise parameters from analysis?)
% LUCA TO DO
In the first set of experiments, we wish to test our main hypothesis, i.e. that appropriate manipulation of the frequency components of the noise can better support the learning of the distribution of interest. 
\begin{wrapfigure}[26]{r}{0.50\textwidth}
\centering
\vspace*{-.5em}
\includegraphics[width=\linewidth]{figures/frequency_diffusion_noising.pdf}
\caption{Power spectra and image visuals of the forward Process in standard diffusion, as compared to high and low-frequency noise settings of a two-band mixture noise parametrization.} 
\label{fig:frequency_diffusion_noising}
\end{wrapfigure} covering \(345\) object categories \citep{peng2019moment}.
We follow the formulation in \autoref{eq:high_low_formulation} to train and compare diffusion models with a noisy operator prioritizing different parts of the frequency distribution. In these experiments we fix $a_l = 0 $, $ b_2 = 1 $, and $ b_l = a_h = 0.5 $, while performing a linear sweep of the  $\gamma_l$ and $\gamma_h$ parameters by searching $\gamma_l \in [.1, .2, ..., .9]$ and $\gamma_h = 1-\gamma_l$.



% \begin{figure}[t]
% \centering
% \begin{minipage}[t]{0.6\linewidth} % Adjust width as needed
%     \centering
%     \includegraphics[width=.6\linewidth]{figures/fid.png}
%     \caption{FID of diffusion samplers trained with various combinations of frequency noise. The settings for $\gamma_l=0.5$ yields standard diffusion training.}
%     \label{fig:corrupt_noise_learning}
% \end{minipage}%
% \hfill % Adjust spacing
% \begin{minipage}[t]{0.29\linewidth} % Adjust width as needed
%     \vspace*{-11em} % Adjust vertical alignment if needed
%     % \centering
%     \resizebox{.65\linewidth}{!}{
%     \begin{tabular}{@{}cc@{}}
%     \toprule
%     $\gamma_l$ ($\rightarrow$) & FID ($\downarrow$) \\
%     \midrule
%     0.1 & $2.69 \pm 0.01$ \\
%     0.2 & $2.62 \pm 0.02$ \\
%     0.3 & $2.59 \pm 0.01$ \\
%     0.4 & $2.57 \pm 0.01$ \\
%     0.5 & $2.56 \pm 0.00$ \\
%     0.6 & $2.53 \pm 0.02$ \\
%     0.7 & $2.43 \pm 0.03$ \\
%     0.8 & $2.40 \pm 0.01$ \\
%     0.9 & $2.45 \pm 0.03$ \\
%     \bottomrule
% \end{tabular}
%     }
%     \captionof{table}{\looseness=-1 FID for all \emph{frequency diffusion} models.}
%     \label{tab:corrupt_noise_learning}
% \end{minipage}
% \end{figure}

\subsubsection{Qualitative Overview}


First, we show a qualitative example of a standard linear noising schedule forward operation in \autoref{fig:frequency_diffusion_noising}, as compared to two particular settings of our constant high and low-frequency linear schedules of the band-pass filter. With standard noise, information is uniformly removed from the image, with sample quality degrading evenly over time. In the high-frequency noising schedule, sharpness and texture are removed more prominently, while in the low-frequency noising schedule, general shapes and homogeneous pixel clusters are affected most, yielding qualitatively different information destruction operations. As discussed previously, we hypothesize that this will in turn purposely affect the statistics of the information learned by the denoiser model, effectively focusing the diffusion sampling process on different parts of the distribution. 

\begin{wrapfigure}[17]{r}{0.50\textwidth}
\includegraphics[width=.9\linewidth]{figures/fid.png}
\caption{FID of diffusion samplers trained with various combinations of frequency noise. The settings for $\gamma_l=0.5$ yields standard diffusion training.}
\label{fig:corrupt_noise_learning}
\end{wrapfigure}
\subsubsection{Learning Target Distributions from Frequency-Bounded Information}
% alternative title: Adapting Learning to Information in Specific Frequency Bands
We conduct experiments to learn the distribution of data where, by construction, the information content lies in the low frequencies. We use the CIFAR-10 dataset, and corrupt the original data with high-frequency noise $\mathbf{\epsilon}_{[.3, 1.]}$, thus erasing the high-frequency content while predominantly preserving the low-frequency details in the range $\mathbf{\epsilon}_{[0., .3]}$. We train 9 diffusion models, including a standard diffusion (\emph{baseline}) model, and 8 models trained with frequency-based noise control spanning 8 combinations of $\gamma_l$ ($\gamma_h=1-\gamma_l$). We repeat the experiment over three seeds and report the average FID and error in \autoref{fig:corrupt_noise_learning}. In the figure, we observe the DPMs trained with higher amounts of low-frequency noise (higher $\gamma_l$) to perform significantly better than both the baseline ($\gamma_l=0.5$), and higher frequency denoising models (lower $\gamma_l$). Furthermore, we see a mostly monotonically descending trend in FID for increasing values of lower frequency noise in the diffusion forward schedule, supporting the original intuition of how the frequency manipulation of the noising operator can directly steer the denoiser's learning trends, and therefor how progressively higher amount of low-frequency forward noise aid in the learning of samplers for data containing mostly low-frequency information. 




\subsubsection{Frequency-Based noise control in natural datasets} \label{sec:qualitative_freq_results}
% Different combinations of frequency noise perform best wrt to different datasets (Figure 2? + Table 1).

We further test our hypothesis by training 9 models for each of the datasets considered, inclusive of all $\gamma$-variations of our two-band mixture frequency-based noise schedule. We train these models on MNIST, CIFAR-10, Domainnet-Quickdraw, Wiki-Art and CelebA, and report the FID and KID metrics for all ablations in \autoref{tab:freq_qualitative}. In the table, we observe three out of five datasets to significantly benefit from frequency-controlled noising schedules, achieving the lowest FID and KID scores across all tested models. Interestingly, the performance trends are also mostly monotonic, which together with our previous experiments is indicative of where the learned information lies. For simple datasets, such as MNIST or CIFAR-10, most frequency denoising settings perform well, with balanced high-to-low-frequency schedules performing best overall. Denoisers for Domainnet-Quickdraw and CelebA yield better performance for slightly higher frequency noising schedules, suggesting higher frequency information content for good FID and KID approximations, while Wiki-Art shows slight biases towards lower frequency schedules. 

% we observe \emph{frequency diffusion} settings other then $\gamma_l=\gamma_h=0.5$ (i.e. standard diffusion), to outperform baseline DPM training on three different datasets, namely: CIFAR-10, Domainnet-Quickdraw, and Wiki-Art, while MNIST and CelebA achieved lowest FID for diffusion with standard noise.

% We show in \autoref{fig:bar_plot} the distributions of FID scores according to each setting of the \emph{frequency diffusion} constant scheduled considered. We observe monotonic trends with each datasets displaying low FID in specific, and different, areas of the distribution of $\gamma$ values. In particular, we observe the visual distributions over MNIST and Domainnet-Quickdraw to benefit from noising processes at medium to high low-frequency noise, suggesting learning benefits from concentrating on lower-frequency areas of the visual distribution; while CIFAR-10, Wiki-Art and CelebA are better approximated by diffusion models trained at medium-to-high high-frequency noise, suggesting more balanced information content within the data, and benefits for diffusion models to focus on both high- and low-frequency information, with slight preference towards high-frequency regions of the distribution.
% \subsection{State-of-the-art Frequency-noise diffusion }


% \begin{table}\label{tab:freq_qualitative}
% \centering
% \caption{Results for FID and KID across different datasets (mean $\pm$ standard error across best 10 epochs). The baseline runs are reported under $\gamma_l=\gamma_h=0.5$ }
% \label{tab:freq_qualitative}
% \resizebox{1\linewidth}{!}{
% \begin{tabular}{@{}lcccccccccc@{}}
% \toprule
% \textbf{Dataset} $\rightarrow$ & \multicolumn{2}{c}{\textbf{MNIST}} & \multicolumn{2}{c}{\textbf{CIFAR-10}} & \multicolumn{2}{c}{\textbf{Domainnet-Quickdraw}} & \multicolumn{2}{c}{\textbf{Wiki-Art}} & \multicolumn{2}{c}{\textbf{CelebA}} \\\cmidrule(lr){2-3}\cmidrule(lr){4-5}\cmidrule(lr){6-7}\cmidrule(lr){8-9}\cmidrule(lr){10-11}
% \textbf{Algo} $\downarrow$ \textbf{Metric} $\rightarrow$ & FID ($\downarrow$) & KID ($\downarrow$) & FID ($\downarrow$) & KID ($\downarrow$) & FID ($\downarrow$) & KID ($\downarrow$) & FID ($\downarrow$) & KID ($\downarrow$) & FID ($\downarrow$) & KID ($\downarrow$) \\\midrule
% $\ \ \ \ \ $baseline & \highlight{0.0452}\std{1.80e-02} & \highlight{1.11e-04}\std{4.81e-05} & 0.1447\std{7.70e-03} & 2.31e-04\std{1.68e-05} & 0.1106\std{2.20e-03} & 2.38e-04\std{6.04e-06} & 0.2058\std{6.10e-03} & 3.53e-04\std{1.38e-05} & \highlight{0.0852}\std{1.30e-03} & \highlight{1.50e-04}\std{2.53e-06}  \\
% $\gamma_l$=0.1, $\gamma_h$=0.9  & 1.7494\std{2.60e-01} & 6.02e-03\std{9.36e-04} & 0.4800\std{1.84e-02} & 8.58e-04\std{4.12e-05} & 1.3442\std{1.21e-02} & 3.98e-03\std{3.44e-05} & 0.3877\std{1.06e-02} & 6.70e-04\std{2.49e-05} & 0.3174\std{2.39e-02} & 7.09e-04\std{5.64e-05}  \\
% $\gamma_l$=0.2, $\gamma_h$=0.8 & 0.4798\std{2.11e-01} & 1.48e-03\std{7.15e-04} & 0.3398\std{1.41e-02} & 6.33e-04\std{2.69e-05} & 0.4190\std{1.32e-02} & 1.12e-03\std{3.74e-05} & 0.3043\std{1.53e-02} & 5.61e-04\std{3.81e-05} & 0.1487\std{8.60e-03} & 2.91e-04\std{2.12e-05}  \\
% $\gamma_l$=0.3, $\gamma_h$=0.7 & 0.1421\std{7.04e-02} & 3.98e-04\std{2.09e-04} & \highlight{0.1292}\std{3.00e-03} & \highlight{1.96e-04}\std{6.46e-06} & 0.2293\std{4.20e-03} & 5.59e-04\std{1.02e-05} & 0.2228\std{1.03e-02} & 3.84e-04\std{2.06e-05} & 0.0861\std{3.70e-03} & 1.63e-04\std{8.78e-06}  \\
% $\gamma_l$=0.4, $\gamma_h$=0.6 & 0.1174\std{4.05e-02} & 2.90e-04\std{1.08e-04} & 0.1525\std{7.70e-03} & 2.57e-04\std{1.97e-05} & 0.1427\std{5.30e-03} & 3.25e-04\std{1.37e-05} & \highlight{0.1814}\std{1.02e-02} & \highlight{3.07e-04}\std{2.48e-05} & 0.1067\std{3.40e-03} & 1.94e-04\std{8.16e-06}  \\
% $\gamma_l$=0.6, $\gamma_h$=0.4 & 0.1303\std{6.52e-02} & 3.94e-04\std{2.13e-04} & 0.1832\std{1.50e-02} & 2.98e-04\std{3.02e-05} & \highlight{0.0951}\std{2.80e-03} & \highlight{1.91e-04}\std{7.57e-06} & 0.2162\std{9.90e-03} & 3.67e-04\std{2.27e-05} & 0.1373\std{4.90e-03} & 2.73e-04\std{1.19e-05}  \\
% $\gamma_l$=0.7, $\gamma_h$=0.3 & 0.1305\std{6.70e-02} & 3.55e-04\std{2.11e-04} & 0.2948\std{3.22e-02} & 5.51e-04\std{7.24e-05} & 0.1068\std{3.00e-03} & 2.20e-04\std{9.02e-06} & 0.2506\std{1.80e-02} & 4.53e-04\std{4.31e-05} & 0.1325\std{6.20e-03} & 2.65e-04\std{1.76e-05}  \\
% $\gamma_l$=0.8, $\gamma_h$=0.2 & 0.1034\std{3.84e-02} & 2.84e-04\std{1.17e-04} & 0.3469\std{3.54e-02} & 6.59e-04\std{8.30e-05} & 0.1170\std{4.90e-03} & 2.36e-04\std{1.34e-05} & 0.2950\std{1.86e-02} & 5.44e-04\std{4.44e-05} & 0.1211\std{2.80e-03} & 2.23e-04\std{7.19e-06}  \\
% $\gamma_l$=0.9, $\gamma_h$=0.1 & 0.1456\std{5.08e-02} & 4.05e-04\std{1.59e-04} & 0.3314\std{3.26e-02} & 6.23e-04\std{7.79e-05} & 0.1456\std{1.50e-03} & 2.92e-04\std{6.02e-06} & 0.3492\std{1.11e-02} & 6.63e-04\std{2.12e-05} & 0.1273\std{4.00e-03} & 2.26e-04\std{9.83e-06}  \\
% \bottomrule
% \end{tabular}
% }
% \end{table}

\begin{table}
\centering
\caption{Results for FID and KID across different settings of $(\gamma_l, \gamma_h)$ for our frequency diffusion two-band mixture schedule across different datasets  (mean $\pm$ standard error across 3 seeds).  The baseline runs correspond to $\gamma_l=\gamma_h=0.5$. }
\label{tab:freq_qualitative}
\resizebox{1\linewidth}{!}{
\begin{tabular}{@{}lcccccccccc@{}}
\toprule
\textbf{Dataset} $\rightarrow$ & \multicolumn{2}{c}{\textbf{MNIST}} & \multicolumn{2}{c}{\textbf{CIFAR-10}} & \multicolumn{2}{c}{\textbf{Domainnet-Quickdraw}} & \multicolumn{2}{c}{\textbf{Wiki-Art}} & \multicolumn{2}{c}{\textbf{CelebA}} \\
\cmidrule(lr){2-3}\cmidrule(lr){4-5}\cmidrule(lr){6-7}\cmidrule(lr){8-9}\cmidrule(lr){10-11}
\textbf{Algo} $\downarrow$ \textbf{Metric} $\rightarrow$ & FID ($\downarrow$) & KID ($\downarrow$) & FID ($\downarrow$) & KID ($\downarrow$) & FID ($\downarrow$) & KID ($\downarrow$) & FID ($\downarrow$) & KID ($\downarrow$) & FID ($\downarrow$) & KID ($\downarrow$) \\
\midrule
$\ \ \ \ \ $baseline & \highlight{0.0168}\std{0.0010} & \highlight{0.0000}\std{0.0000} & \highlight{0.1055}\std{0.0042} & \highlight{0.0001}\std{0.0000} & 0.0875\std{0.0060} & 1.69e-04\std{1.61e-05} & 0.1622\std{0.0133} & 2.53e-04\std{1.80e-05} & 0.0863\std{0.0094} & \highlight{0.0001}\std{0.0000} \\
$\gamma_l=0.1, \gamma_h=0.9$ & 0.2624\std{0.2184} & 7.90e-04\std{6.85e-04} & 0.2648\std{0.0691} & 4.31e-04\std{1.30e-04} & 0.5250\std{0.3907} & 1.46e-03\std{1.21e-03} & 0.2673\std{0.0273} & 4.31e-04\std{4.56e-05} & 0.1555\std{0.0273} & 2.97e-04\std{6.93e-05} \\
$\gamma_l=0.2, \gamma_h=0.8$ & 0.0432\std{0.0187} & 1.10e-04\std{5.24e-05} & 0.2191\std{0.0223} & 3.86e-04\std{6.72e-05} & 0.1843\std{0.0723} & 4.20e-04\std{2.15e-04} & 0.2048\std{0.0063} & 3.43e-04\std{1.27e-05} & 0.1024\std{0.0045} & 1.85e-04\std{2.72e-06} \\
$\gamma_l=0.3, \gamma_h=0.7$ & 0.0267\std{0.0029} & 6.40e-05\std{8.63e-06} & 0.1506\std{0.0168} & 2.28e-04\std{3.34e-05} & 0.1248\std{0.0375} & 2.70e-04\std{1.13e-04} & 0.1865\std{0.0181} & 2.86e-04\std{2.46e-05} & \highlight{0.0838}\std{0.0107} & 1.44e-04\std{1.89e-05} \\
$\gamma_l=0.4, \gamma_h=0.6$ & 0.0224\std{0.0032} & 5.29e-05\std{8.15e-06} & 0.1131\std{0.0079} & 1.64e-04\std{2.15e-05} & \highlight{0.0799}\std{0.0166} & \highlight{0.0001}\std{0.0000} & 0.1597\std{0.0122} & 2.62e-04\std{3.23e-05} & 0.0875\std{0.0020} & 1.49e-04\std{1.71e-06} \\
$\gamma_l=0.6, \gamma_h=0.4$ & 0.0253\std{0.0039} & 5.81e-05\std{7.63e-06} & 0.1131\std{0.0074} & 1.56e-04\std{1.95e-05} & 0.1128\std{0.0174} & 2.57e-04\std{5.56e-05} & \highlight{0.1348}\std{0.0126} & \highlight{0.0002}\std{0.0000} & 0.1068\std{0.0039} & 2.04e-04\std{1.07e-05} \\
$\gamma_l=0.7, \gamma_h=0.3$ & 0.0363\std{0.0075} & 9.14e-05\std{2.04e-05} & 0.1432\std{0.0203} & 2.19e-04\std{3.66e-05} & 0.1353\std{0.0223} & 2.91e-04\std{6.08e-05} & 0.1561\std{0.0123} & 2.32e-04\std{2.46e-05} & 0.0990\std{0.0082} & 1.84e-04\std{2.12e-05} \\
$\gamma_l=0.8, \gamma_h=0.2$ & 0.0512\std{0.0119} & 1.36e-04\std{3.60e-05} & 0.1898\std{0.0095} & 2.88e-04\std{1.85e-05} & 0.2288\std{0.0737} & 5.85e-04\std{2.21e-04} & 0.2256\std{0.0096} & 3.86e-04\std{3.08e-05} & 0.1053\std{0.0185} & 1.95e-04\std{4.34e-05} \\
$\gamma_l=0.9, \gamma_h=0.1$ & 0.3403\std{0.1513} & 9.74e-04\std{4.47e-04} & 0.3226\std{0.0660} & 5.31e-04\std{1.20e-04} & 0.9827\std{0.4229} & 2.84e-03\std{1.29e-03} & 0.3250\std{0.0270} & 5.57e-04\std{3.22e-05} & 0.2291\std{0.0605} & 4.86e-04\std{1.52e-04} \\
\bottomrule
\end{tabular}
}
\end{table}

% \begin{table}\label{tab:freq_qualitative}
% \centering
% \caption{Results for FID and KID across different datasets (mean $\pm$ standard error across 3 seeds).  }
% \label{tab:freq_qualitative}
% \resizebox{1\linewidth}{!}{
% \begin{tabular}{@{}lcccccccccc@{}}
% \toprule
% \textbf{Dataset} $\rightarrow$ & \multicolumn{2}{c}{\textbf{MNIST}} & \multicolumn{2}{c}{\textbf{CIFAR-10}} & \multicolumn{2}{c}{\textbf{Domainnet-Quickdraw}} & \multicolumn{2}{c}{\textbf{Wiki-Art}} & \multicolumn{2}{c}{\textbf{CelebA}} \\\cmidrule(lr){2-3}\cmidrule(lr){4-5}\cmidrule(lr){6-7}\cmidrule(lr){8-9}\cmidrule(lr){10-11}
% \textbf{Algo} $\downarrow$ \textbf{Metric} $\rightarrow$ & FID ($\downarrow$) & KID ($\downarrow$) & FID ($\downarrow$) & KID ($\downarrow$) & FID ($\downarrow$) & KID ($\downarrow$) & FID ($\downarrow$) & KID ($\downarrow$) & FID ($\downarrow$) & KID ($\downarrow$) \\\midrule
% $\ \ \ \ \ $baseline & \highlight{0.0168}\std{0.0010} & \highlight{4.12e-05}\std{2.52e-06} & \highlight{0.1055}\std{0.0042} & \highlight{1.41e-04}\std{4.48e-06} & \highlight{0.0875}\std{0.0060} & \highlight{1.69e-04}\std{1.61e-05} & 0.1622\std{0.0133} & 2.53e-04\std{1.80e-05} & \highlight{0.0863}\std{0.0094} & \highlight{1.43e-04}\std{1.79e-05} \\
% $\gamma_l$=0.1, $\gamma_h$=0.9 & 0.2624\std{0.2184} & 7.90e-04\std{6.85e-04} & 0.2648\std{0.0691} & 4.31e-04\std{1.30e-04} & 0.5250\std{0.3907} & 1.46e-03\std{1.21e-03} & 0.2673\std{0.0273} & 4.31e-04\std{4.56e-05} & 0.1555\std{0.0273} & 2.97e-04\std{6.93e-05} \\
% $\gamma_l$=0.2, $\gamma_h$=0.8 & 0.0432\std{0.0187} & 1.10e-04\std{5.24e-05} & 0.2191\std{0.0223} & 3.86e-04\std{6.72e-05} & 0.1843\std{0.0723} & 4.20e-04\std{2.15e-04} & 0.2048\std{0.0063} & 3.43e-04\std{1.27e-05} & 0.1024\std{0.0045} & 1.85e-04\std{2.72e-06} \\
% $\gamma_l$=0.3, $\gamma_h$=0.7 & 0.0267\std{0.0029} & 6.40e-05\std{8.63e-06} & 0.1506\std{0.0168} & 2.28e-04\std{3.34e-05} & 0.1248\std{0.0375} & 2.70e-04\std{1.13e-04} & 0.1865\std{0.0181} & 2.86e-04\std{2.46e-05} & 0.0838\std{0.0107} & 1.44e-04\std{1.89e-05} \\
% $\gamma_l$=0.4, $\gamma_h$=0.6 & 0.0224\std{0.0032} & 5.29e-05\std{8.15e-06} & 0.1131\std{0.0079} & 1.64e-04\std{2.15e-05} & 0.0799\std{0.0166} & 1.45e-04\std{4.74e-05} & 0.1597\std{0.0122} & 2.62e-04\std{3.23e-05} & 0.0875\std{0.0020} & 1.49e-04\std{1.71e-06} \\
% $\gamma_l$=0.6, $\gamma_h$=0.4 & 0.0253\std{0.0039} & 5.81e-05\std{7.63e-06} & 0.1131\std{0.0074} & 1.56e-04\std{1.95e-05} & 0.1128\std{0.0174} & 2.57e-04\std{5.56e-05} & \highlight{0.1348}\std{0.0126} & \highlight{2.02e-04}\std{2.33e-05} & 0.1068\std{0.0039} & 2.04e-04\std{1.07e-05} \\
% $\gamma_l$=0.7, $\gamma_h$=0.3 & 0.0363\std{0.0075} & 9.14e-05\std{2.04e-05} & 0.1432\std{0.0203} & 2.19e-04\std{3.66e-05} & 0.1353\std{0.0223} & 2.91e-04\std{6.08e-05} & 0.1561\std{0.0123} & 2.32e-04\std{2.46e-05} & 0.0990\std{0.0082} & 1.84e-04\std{2.12e-05} \\
% $\gamma_l$=0.8, $\gamma_h$=0.2 & 0.0512\std{0.0119} & 1.36e-04\std{3.60e-05} & 0.1898\std{0.0095} & 2.88e-04\std{1.85e-05} & 0.2288\std{0.0737} & 5.85e-04\std{2.21e-04} & 0.2256\std{0.0096} & 3.86e-04\std{3.08e-05} & 0.1053\std{0.0185} & 1.95e-04\std{4.34e-05} \\
% $\gamma_l$=0.9, $\gamma_h$=0.1 & 0.3403\std{0.1513} & 9.74e-04\std{4.47e-04} & 0.3226\std{0.0660} & 5.31e-04\std{1.20e-04} & 0.9827\std{0.4229} & 2.84e-03\std{1.29e-03} & 0.3250\std{0.0270} & 5.57e-04\std{3.22e-05} & 0.2291\std{0.0605} & 4.86e-04\std{1.52e-04} \\
% \bottomrule
% \end{tabular}
% }
% \end{table}


% \newpage
% % \begin{figure}
% \begin{wrapfigure}[45]{r}{0.36\textwidth}
% % \begin{figure*}[htbp]
%     \centering
%     \includegraphics[width=0.35\textwidth]{figures/bar_plot_main.png}
%     \caption{FID for different settings of \emph{frequency diffusion} on all datasets}
%     \label{fig:bar_plot}
% % \end{figure*}
% % \end{figure}
% \end{wrapfigure}




\subsection{Selective Learning: Frequency-Based Noise Control to Omit Targeted Information}
% previous title: Distribution generative recovery under severe noisy corruptions
Following our original intuition, a denoising model has pressure to learn the very information that is erased by the forward noising operator to achieve successful reconstruction. Conversely, when the noising operator is crafted to leave parts of the original distribution intact, no such pressure exists, and the denoising model can effectively discard the left-out statistics during generation. 

In this section, we perform experiments whereby the original data is corrupted with noise at different frequency ranges. The objective is to manipulate the inductive biases of diffusion denoisers to avoid learning the corruption noise, while correctly approximating the relevant information in the data.  We formulate our corruption process as $\rvx'=A_c(\rvx)$, where: 

\begin{equation}
    A_c(\rvx) =  \rvx + \gamma_c \mathbf{\epsilon}_{f[a_c, b_c]}
\end{equation}

Here, $\mathbf{\epsilon}_{[a_c, b_c]}$ denotes noise in the $[a_c, b_c]$ frequency range. We default $\gamma_c=1.$ and show samples of the original and corrupted distributions in \autoref{fig:noise_removal}. For any standard DPM training procedure, the denoiser would make no distinction of which information to learn, and thus would approximate the corrupted distribution presented at training time. As such, the recovery of the original, noiseless, distribution would normally be impossible. Assuming knowledge of the corruption process, we frame the frequency diffusion learning procedures as a noiseless distribution recovery process, and set $a_l = 0 $, $ b_h = 1 $, $b_l = a_c$, and $a_h = b_c$. This formulation effectively allows for the forward frequency noising operator to omit the range of frequencies in which the noise lies. In line with our previous rationale, this would effectively put no pressure on the denoiser to learn the noise part of the distribution at hand, and focus instead on the frequency ranges where the true information lies.

We compare original and corrupted samples from MNIST, as well as samples from standard and frequency diffusion-trained models in \autoref{fig:noise_removal}. In line with our hypothesis, we observe frequency diffusion DPMs trained with an appropriate frequency noise operator to be able to discard the corrupting information and recover the original distribution after severe noisy corruption. We further measure the FID and KID of the samples generated by the baseline and frequency DPMs against the original (uncorrupted) data samples in \autoref{tab:corruption_fids}. 
We perform 8 ablation studies, considering noises at $0.1$ non-overlapping intervals in the $[0.1, .9]$ frequency range. We observe \emph{frequency diffusion} to outperform standard diffusion training across all tested ranges. Interestingly, we observe better performance (lower FID) for data corruption in the high-frequency ranges, and reduced performance for data corruptions in low-frequency ranges, suggesting a marginally higher information content in the low frequencies for the MNIST dataset.
% Consistent with our previous findings in \autoref{fig:bar_plot}, we observe a primary role of the low-frequency information in the visual distribution in MNIST, upon which destruction leads to reduced sample recoverability. 
% According to our result, data corruptions at high-frequency ranges present easier targets for recovery, as the majority of the true information content is still present in the data and the DPMs trained on \emph{frequency diffusion} schedules can extract the most important aspects of the visual distribution according to the Inception v3 embeddings used for FID computation.  


\begin{table} \label{tab:corruption}
\centering
\caption{Resulting FID and KID between standard diffusion and frequency diffusion DPMs trained on noise-corrupted data, with respect to samples from the true uncorrupted distribution (mean $\pm$ standard error across 3 seeds). We report eight ablation experiments across different non-overlapping corruption noise schemes.}
\label{tab:corruption_fids}
\resizebox{.8\linewidth}{!}{
\begin{tabular}{@{}lcccc@{}}
\toprule
\textbf{Dataset} $\rightarrow$ & \multicolumn{2}{c}{\textbf{Baseline}} & \multicolumn{2}{c}{\textbf{Ours}} \\
\cmidrule(lr){2-3}\cmidrule(lr){4-5}
\textbf{Corruption} $\downarrow$ & FID ($\downarrow$) & KID ($\downarrow$) & FID ($\downarrow$) & KID ($\downarrow$) \\
\midrule
$\epsilon_{[0.1,0.2]}$ & 3.2273\std{8.50e-03} & 0.0114\std{3.13e-05} & \highlight{2.7572\std{3.56e-02}} & \highlight{0.0095\std{1.47e-04}} \\
$\epsilon_{[0.2,0.3]}$ & 3.6601\std{4.43e-03} & 0.0132\std{1.67e-05} & \highlight{3.0416\std{4.47e-02}} & \highlight{0.0107\std{1.79e-04}} \\
$\epsilon_{[0.3,0.4]}$ & 3.4771\std{4.79e-03} & 0.0125\std{1.89e-05} & \highlight{2.9952\std{3.35e-02}} & \highlight{0.0106\std{1.23e-04}} \\
$\epsilon_{[0.4,0.5]}$ & 3.4281\std{5.46e-03} & 0.0123\std{1.98e-05} & \highlight{2.9218\std{2.54e-02}} & \highlight{0.0105\std{8.79e-05}} \\
$\epsilon_{[0.5,0.6]}$ & 3.3638\std{6.31e-03} & 0.0121\std{2.32e-05} & \highlight{2.8267\std{2.81e-02}} & \highlight{0.0102\std{9.32e-05}} \\
$\epsilon_{[0.6,0.7]}$ & 3.2444\std{7.10e-03} & 0.0116\std{2.55e-05} & \highlight{2.7026\std{3.90e-02}} & \highlight{0.0097\std{1.28e-04}} \\
$\epsilon_{[0.7,0.8]}$ & 3.0442\std{6.32e-03} & 0.0109\std{2.29e-05} & \highlight{2.5469\std{6.39e-02}} & \highlight{0.0091\std{2.00e-04}} \\
$\epsilon_{[0.8,0.9]}$ & 3.4660\std{7.90e-03} & 0.0124\std{2.96e-05} & \highlight{2.5138\std{9.63e-02}} & \highlight{0.0090\std{3.07e-04}} \\
\bottomrule
\end{tabular}
}
\end{table}

\begin{figure}
% \begin{figure*}[htbp]
    \centering
    \includegraphics[width=1\textwidth]{figures/corruption_experiment_main.png}
    \caption{Samples from the original data distribution, the degraded data distribution, a standard diffusion sampler trained on the degraded data distribution, and a \emph{frequency diffusion} sampler trained on the degraded data distribution. We generate noise for data corruption in the frequency range [$a_c=0.5$, $b_c=0.6)$].}
    \label{fig:noise_removal}
% \end{figure*}
\end{figure}
%%%%%%%%%%%%%%%%%%%%%%%%%%%%%%%%%%%%%%%%%%%%%%%%%%%%%%%%%%%%%%%%%%%%%%%%%%%%%%%%%%%%%%%%%%%%%%%%%%%%%%%%%%%%%%%%%%%%%%%%
%%%%%%%%%%%%%%%%%%%%%%%%%%%%%%%%%%        DISCUSSION/CONCLUSION         %%%%%%%%%%%%%%%%%%%%%%%%%%%%%%%%%%%%%%%%%%%%%%%%
%%%%%%%%%%%%%%%%%%%%%%%%%%%%%%%%%%%%%%%%%%%%%%%%%%%%%%%%%%%%%%%%%%%%%%%%%%%%%%%%%%%%%%%%%%%%%%%%%%%%%%%%%%%%%%%%%%%%%%%%

% \section{Discussion and Conclusion} \label{sec:discussion_and_conclusion}
% \begin{itemize}
%     \item Frequency Diffusion: We develop \emph{frequency diffusion}, manipulating the frequency components of the forward noising process to build inductive biases into DPMS and lead them to focus on specific areas of the distribution to approximate.
%     \item Forward noise manipulation as an inductive bias: We test and show that purposefully manipulating the nosing forward (destruction) operation can lead diffusion models to learn the statistics of particular. This has various implications in the field, beyond what has been explored in this paper. In fact, many different types of noise manipulations can be explored, each affecting diffusion differently. Within \emph{frequency diffusion}, for example, we can go beyond constant frequency schedules and shift the generational focus of denoisers from lower (general shapes) to higher frequency (sharp edges, texture) information during the backward process, more closely mimicking the generation process in humans (CITE). Other schedules might shift altogether thefocus on other noise manipulations beyond the frequency domain.
%     \item We show these inductive biases can dignificantly affect DPM training and sampling. we show this in a data corruption and recovery task whereby DPMS, trained with noise corrupted data, are meant to learn the noiseless statistics of the corrupted data to recover the original distribution. We show significant performance leaps in \emph{frequency diffusion}, which is effectively able disregard the noise while focusing on approximating the noiseless parts of the distribution to approximate.
%     \item A limitation lies within the complexity of the visual spatial distributions with respect to the same in frequency space. In fact, it is often hard to appropriately study and predict how the information content in the frequency domain translates to spatial visual perception. As such, analytical methods might be generally limited in their ability to predict which frequency scheules may be more suited for specific data distributions and validation alternatives might be necessary.  
    
% \end{itemize}

\section{Discussion and Conclusion} \label{sec:discussion_and_conclusion}

% \paragraph{Frequency Diffusion} 
In this work, we studied the potential to build inductive biases in the training and sampling of Diffusion Probabilistic Models by purposeful manipulation of the forward, noising, process. We introduced \emph{frequency diffusion}, an approach that enables us to guide DPMs toward learning specific statistics of the data distribution. We compare \emph{frequency diffusion} to DPS trained with standard gaussian noise on generative visual tasks set by several datasets, with significant varying structure and scales. We show several key findings. First, we show that appropriate manipulation of the forward noising process can serve as a stong inductive bias for diffusion models to better learn the information of the distribution at particular frequencies. Second, we show that this important characteristic can be readily used when training diffusion models on natural dataset, some of which may be better supported by appropriate frequency diffusion schedules, yielding higher sampling quality. Third, we show how this processes can be used to discard unwanted information at particular frequency ranges, yielding DPMs capable of extract noiseless signals from the remaining ranges. 

In our approach, we have limited the results to a simple two-band pass frequency filter. We propose in \autoref{supp:sec:freq_noise_operator} several other alternatives, which may serve as more flexible tools to inject useful inductive biases for similar tasks. Moreover, the approach can be extended beyond constant schedules. For instance, it may prove useful to introduce dynamic frequency noise strategies that shift the focus from low-frequency (general shapes) to high-frequency (sharp edges and textures) components over the time discretization of the sampling process. Such methods could more closely align with human visual processing, which progressively sharpens details over time, offering a more natural sampling process. Additionally, other domains of noise manipulation—outside of the frequency domain may also present new opportunities for further improving DPMs across various tasks.

Finally, a current limitation of this approach lies in the complexity of understanding the relationship between visual data in spatial and frequency domains. The perception of information in the frequency domain does not always translate straightforwardly to visual content, complicating the process of designing optimal noise schedules. As such, it is not trivial to design appropriate frequency schedules for a particular distribution. In practice, empirical validation may still be required to identify the best inductive biases for a given dataset. Future work could focus on refining analytical tools for frequency analysis or exploring alternative inductive bias mechanisms that extend beyond frequency-based manipulations.

% We show that depending on the distribution to learn, \emph{frequency diffusion} can more accurately and flexibly approximate the underlying data distribution than standard diffusion, providing a more data-aligned inductive bias for diffusion models. 

Overall, this work opens the door for more targeted and flexible diffusion generative modeling by building inductive biases through the manipulation of the forward nosing process. The ability to design noise schedules that align with specific data characteristics holds promise for advancing the state of the art in generative modeling.


% \subsubsection*{Author Contributions}
% If you'd like to, you may include  a section for author contributions as is done
% in many journals. This is optional and at the discretion of the authors.

\subsubsection*{Acknowledgments}
The authors acknowledge funding from CIFAR, and Recursion. The research was enabled in part by computational resources provided by the Digital Research
Alliance of Canada (\url{https://alliancecan.ca}), Mila (\url{https://mila.quebec}), and
NVIDIA.


\bibliography{iclr2025_conference}
\bibliographystyle{iclr2025_conference}

% -- additional by luca --

% \newpage
% \clearpage
% \setcounter{page}{1}
% \setcounter{section}{0}
% \renewcommand{\thesection}{S\arabic{section}}
% \setcounter{table}{0}
% \renewcommand{\thetable}{S\arabic{table}}%
% \setcounter{figure}{0}
% \renewcommand{\thefigure}{S\arabic{figure}}%
%%%%%%%%%%%%%%%%%%%%%%%%%
% \appendix
% \section{Appendix}


%%%%%%%%%%%%%%%%%%%%%%%%%%%%%%%%%%%%%%%%%%%%%%%%%%%%%%%%%%%%%%%%%%%%%%%%%%%%%%%%%%%%%%%%%%%%%%%%%%%%%%%%%%%%%%%%%%%%%%%%
%%%%%%%%%%%%%%%%%%%%%%%%%%%%%%%%%%%%%%        SUPPLEMENTARY         %%%%%%%%%%%%%%%%%%%%%%%%%%%%%%%%%%%%%%%%%%%%%%%%%%%%
%%%%%%%%%%%%%%%%%%%%%%%%%%%%%%%%%%%%%%%%%%%%%%%%%%%%%%%%%%%%%%%%%%%%%%%%%%%%%%%%%%%%%%%%%%%%%%%%%%%%%%%%%%%%%%%%%%%%%%%%
% \section{Supplementary Methods}



% Choosing a suitable weighting function depends on the target application and the desired visual or statistical properties of the noise. Power-law variants are often favored for modeling \emph{natural} random processes with scale-invariant behavior, whereas exponential decay is useful for enforcing smoother correlations. Band-pass masks, in turn, allow for \emph{hard} frequency cuts, ensuring that only specific bands appear in the final noise. By tuning parameters such as \(\alpha\), \(\beta\), or the band limits, one can flexibly control the level of detail, smoothness, or feature size in the spatial domain.

% In summary, the frequency noise operator described here provides a unifying view of these approaches: one first samples white noise in the frequency domain, applies the chosen weighting function, and then transforms back to the spatial domain. This simple pipeline enables the design of noise fields tailored to a wide variety of imaging and simulation tasks.



% \section{Additional Results}



% \begin{table}
% \centering
% \caption{Diversity across different datasets (mean $\pm$ standard error across best 10 epochs)}
% \begin{tabular}{lccccc}
% \hline
% Run & mnist & CIFAR-10 & domainnet-quickdraw & wiki-art & CelebA \\
% \hline
% $\gamma_l$=0.1 & 0.9220 $\pm$ 0.0011 & 0.8482 $\pm$ 0.0008 & \textbf{0.9235} $\pm$ 0.0008 & 0.8476 $\pm$ 0.0018 & 0.8575 $\pm$ 0.0008  \\
% $\gamma_l$=0.2 & 0.9310 $\pm$ 0.0040 & 0.8564 $\pm$ 0.0010 & 0.8996 $\pm$ 0.0002 & 0.8332 $\pm$ 0.0022 & 0.8546 $\pm$ 0.0007  \\
% $\gamma_l$=0.3 & 0.9414 $\pm$ 0.0028 & 0.8448 $\pm$ 0.0006 & 0.8904 $\pm$ 0.0006 & 0.8358 $\pm$ 0.0018 & 0.8557 $\pm$ 0.0004  \\
% $\gamma_l$=0.4 & 0.9392 $\pm$ 0.0020 & 0.8471 $\pm$ 0.0010 & 0.8914 $\pm$ 0.0009 & 0.8366 $\pm$ 0.0014 & 0.8544 $\pm$ 0.0005  \\
% $\gamma_l$=0.5 & \textbf{0.9457} $\pm$ 0.0006 & 0.8475 $\pm$ 0.0008 & 0.8909 $\pm$ 0.0007 & 0.8429 $\pm$ 0.0014 & 0.8568 $\pm$ 0.0004  \\
% $\gamma_l$=0.6 & \textbf{0.9437} $\pm$ 0.0016 & 0.8494 $\pm$ 0.0010 & 0.8941 $\pm$ 0.0004 & 0.8443 $\pm$ 0.0011 & \textbf{0.8594} $\pm$ 0.0007  \\
% $\gamma_l$=0.7 & 0.9410 $\pm$ 0.0025 & \textbf{0.8601} $\pm$ 0.0017 & 0.8962 $\pm$ 0.0004 & 0.8511 $\pm$ 0.0023 & \textbf{0.8588} $\pm$ 0.0008  \\
% $\gamma_l$=0.8 & \textbf{0.9445} $\pm$ 0.0011 & \textbf{\textbf{0.8619}} $\pm$ 0.0016 & 0.8997 $\pm$ 0.0002 & \textbf{0.8540} $\pm$ 0.0027 & 0.8579 $\pm$ 0.0005  \\
% $\gamma_l$=0.9 & \textbf{0.9447} $\pm$ 0.0019 & \textbf{\textbf{0.8619}} $\pm$ 0.0011 & 0.9008 $\pm$ 0.0003 & \textbf{0.8569} $\pm$ 0.0019 & 0.8576 $\pm$ 0.0003  \\
% \hline
% \end{tabular}
% \end{table}




\end{document}