\section{Related Work}
In this section, we review previous works that are relevant to the development of jade recognition systems, as well as mobile image recognition and collaborative model architectures. The advancements in these areas form the foundation for our proposed system, and this review highlights the key research trends, challenges, and solutions that have informed our approach.

\subsection{Jade and Artifact Recognition}

Jade recognition and artifact classification have been a significant area of research, particularly in the context of cultural heritage preservation and appraisal. Traditional methods for jade identification often rely on the expertise of appraisers, whose judgments can be subjective and prone to error. As a result, there has been an increasing interest in automating the jade recognition process using computer vision and machine learning techniques.

Sax et al. \cite{sax2004identification} proposed a method for jade identification that used texture analysis to extract features for classification. This early work laid the groundwork for applying image processing to jade artifact recognition. In more recent years, deep learning techniques have been applied to automate jade recognition. For example, Jaiswal et al. \cite{jaiswal2020survey} conducted a survey of various machine learning methods applied to cultural heritage artifact recognition, including jade. They found that while convolutional neural networks (CNNs) offer substantial improvements in accuracy, challenges such as variable lighting conditions and complex textures continue to make jade recognition difficult.

Other studies have proposed feature extraction techniques based on the physical characteristics of jade, such as its luster, texture, and transparency \cite{jiawei2022analysis}. These works have emphasized the importance of creating specialized datasets and fine-tuning models for jade-specific features. However, the high variability in jade artifacts, especially in shape and texture, has made it difficult to achieve consistent results across different jade types and environmental conditions.

\subsection{Mobile Image Recognition}

With the rise of mobile computing, there has been growing interest in deploying image recognition systems on mobile devices. While traditional deep learning models have achieved state-of-the-art performance in image classification tasks, they often require substantial computational resources, making them impractical for mobile devices. As a result, lightweight models and model compression techniques have been proposed to reduce the resource demands while maintaining acceptable accuracy.

A notable example is MobileNet, an efficient CNN architecture designed for mobile and edge devices. MobileNet employs depthwise separable convolutions to reduce computational complexity while preserving the performance of the model. Similarly, EfficientNet introduced a family of models that scale more efficiently, achieving high accuracy with fewer parameters and lower computational cost. These advancements have made it possible to run powerful image recognition models on mobile devices in real time.

Moreover, researchers have also explored edge computing as a solution for running more complex models while offloading computational tasks to cloud or edge servers. The use of edge computing for deep learning allows for faster processing times by leveraging the computational resources of nearby servers. In this way, the heavy lifting of image recognition can be handled by the server, while the mobile device focuses on user interaction and basic processing tasks. This architecture is ideal for real-time applications, such as jade recognition, where accuracy and performance are critical.

\subsection{Collaborative Models in Recognition Systems}

Collaborative or hybrid models that combine lightweight and large models for recognition tasks have gained attention in recent years, especially for mobile and edge applications. These systems leverage the strengths of both lightweight models, which provide real-time performance on mobile devices, and large models, which offer higher accuracy but require more computational resources.

Several works have adopted this collaborative model architecture for various applications. For instance, Xu et al. proposed a collaborative approach that combines a lightweight model for real-time image preprocessing and a more complex model for detailed analysis, effectively balancing accuracy and processing speed. The authors demonstrated that such hybrid systems could significantly improve both the efficiency and accuracy of object detection tasks. Similarly, in the field of medical image analysis, researchers have implemented collaborative models to offload resource-intensive image processing tasks to cloud-based servers, while keeping real-time response times on mobile devices.

These hybrid models have proven to be effective in situations where mobile devices must handle high-throughput recognition tasks, such as real-time object detection and classification, but are limited by hardware constraints. The integration of lightweight and large models in such systems enables them to overcome the trade-offs between performance and resource usage, making them well-suited for jade recognition applications on mobile devices.

\subsection{Challenges and Opportunities}

While the use of deep learning and collaborative models has significantly advanced jade recognition and mobile image recognition, several challenges remain. For jade recognition, the diversity of jade artifacts, variations in texture, lighting conditions, and the complexity of surface patterns all pose significant hurdles. Traditional single-model approaches often struggle to strike a balance between achieving high accuracy and maintaining real-time performance on mobile platforms.

In response, recent developments in model compression and adaptive systems have shown promise in overcoming these challenges. Techniques such as pruning, quantization, and knowledge distillation can help reduce the computational load of deep learning models, making them more suitable for mobile devices. However, these methods often require trade-offs in accuracy, and fine-tuning them to handle jade-specific characteristics remains a challenge.

Furthermore, the collaboration between lightweight and large models is still an emerging research area. While there have been successful implementations of hybrid models in other domains, the design of efficient collaboration mechanisms for jade recognition is less explored. The optimal task distribution, data exchange, and feature fusion between the two models are critical to ensuring that the system performs well under varying conditions and resource constraints.

In conclusion, the research on jade recognition, mobile image processing, and collaborative model architectures provides a solid foundation for our proposed system. By leveraging the strengths of both lightweight and large models, our approach aims to overcome the challenges faced by traditional jade recognition systems and bring about a more efficient and accurate solution for mobile jade identification.