\section{Related Work}
Human arm motion has been extensively studied within the context of a number of different fields including clinical rehabilitation and digital media.

In sports science, rehabilitation, and kinesiology, the study of arm motion usually focuses on monitoring, treatment, or rehabilitation of conditions with motor impacts such as stroke or cerebral palsy~\cite{yin2021discovering} but works identifying typical motor development are also common~\cite{nemanich2025age}. Traditional data collection methods include physical input devices, like a hand pedal~\cite{nemanich2025age}, or physical sensors, like IMUs~\cite{rudisch2018developmental}, to capture movement. The fewer pieces of physical equipment, the more natural the movement captured can be. Works which use immersive VR or augmented reality set-ups in game-like settings for capturing more natural movement to perform kinematic analysis or for rehabilitation training are more related to the work presented here~\cite{proencca2018serious}. Examples include the game \textit{SuperPop} which has been developed as a training intervention for children with cerebral palsy~\cite{chen2015effect} and~\textit{PigScape} which analyzes movement strategies in children with ADHD~\cite{remizova2023exploring}. Games have also been used to study movement patterns in people who have had a stroke~\cite{coias2022low}, people who are recovering from burns~\cite{lan2023use}, and people with speech impairments~\cite{alsebayel2024articumotion}.

\begin{figure*}
    \centering
    \includegraphics[width=0.9\textwidth]{figures/fig1.png}
    \caption{AR game data processing pipeline. (left) Reaching mini-game screenshot of participant bilaterally reaching towards virtual targets displayed on a screen. (center) Illustrated upper body skeleton overlaid on video frame. (right) Extracted skeleton.}
    \label{fig:game_screenshot}
\end{figure*}

Other work with a similar aim of characterizing how bimanual kinematic elements develop over childhood often measures the hand phasing, or the synchrony of the arm motions~\cite{de2012development} but more specific metrics such as velocity~\cite{gasser2010development} are also used. The rate at which different kinematic elements of bilateral motion develop in childhood is contradicted in literature~\cite{schneiberg2002development, lantero2009factors, gasser2010development, de2012development}. However some these works use different tasks in their studies which suggest that bilateral coordination and planning skills may display differently depending on if the task contains sequential movements~\cite{lantero2009factors}, or if the arms needed to be used symmetrically or asymmetrically~\cite{de2012development, rudisch2018developmental}. 

Outside of rehabilitation and healthcare, characteristics of motion kinematics are also studied in digital media fields who want to create realistic digital human characters on small~\cite{ferstl2021human,chan2020emotion,adkins2023important} and large~\cite{sprenger2019capturing} scales. Understanding and quantifying the characteristically variable patterns of these motions has been identified as a key element of making movements more human-like~\cite{hetherington2015believability}. Children's motion in particular changes dramatically as they age, leading to clear perceptual differences between child and adult motion~\cite{jain2016motion}. To address this need for more realistic child motions, previous work has considered automated ways to adapt animations from adult to child motion, e.g., via a neural network~\cite{dong2020adult2child}. 

In contrast to the above work, our aim is to characterize how specific kinematic elements develop over childhood, and characterize different motor strategies children use to adapt to an asymmetric sequential motor task without any physical input device to impair children's movements.