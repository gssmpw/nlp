\section{Related Work}
\label{sec:Related Work}

\paragraph{Robustness Evaluation in MRC} A typical approach to evaluate the robustness of MRC models is via test-time perturbation. This line of research develops different perturbation methods as attacks, such as
% Research on the robustness of MRC against test-time perturbations can be broadly classified based on the methodologies employed to construct the perturbed test set: either through synthetic perturbations applied to the input or by introducing naturally occurring real-world noises. Synthetic perturbation, commonly defined as the deliberate introduction of artificial modifications to input instances with the aim of simulating real-world variations, has been extensively investigated, with various perturbation strategies being devised, such as
adversarial distracting sentence addition \citep{jia-liang-2017-adversarial, tran-etal-2023-impacts}, low-level attacks~\citep{eger-benz-2020-hero}, word substitution \citep{wu-etal-2021-evaluating}, character swap \citep{si-etal-2021-benchmarking}, entity renaming \citep{yan-etal-2022-robustness} and paraphrasing \citep{gan-ng-2019-improving, lai-etal-2021-machine, wu-etal-2023-machine}. Our work also fits within the category of test-time perturbation, but differs from previous works in that we introduce perturbations that naturally occur in real-world scenarios, therefore contributing to a more practical robustness test. 


\begin{figure*}[t!]
    \centering
    \includegraphics[width=0.77\textwidth]{latex/figures/nversion/nppipeline.pdf}
    \caption{Process of generating naturally perturbed MRC test sets.}
    \label{fig:rm-updated}
\end{figure*}

\paragraph{Natural Perturbation for Robustness Assessment} Compared with deliberately crafting the perturbed instances, the study of natural perturbation is quite under-explored. In the computer vision domain, researchers find that real-world clean images without intentional modifications can confuse deep learning models as well, terming them as natural adversarial examples \citep{Hendrycks_2021_CVPR, Pedraza2022-yo}. Similarly, in the field of Natural Language Processing (NLP), naturally occurring perturbations extracted from human-written texts can also degrade model performance in tasks such as machine translation \citep{belinkov2018synthetic} and toxic comments detection \citep{le-etal-2022-perturbations}.
% In the field of computer vision, there has been exploration of natural adversarial examples--real-world images without intentional modifications that can fool deep learning models \citep{Hendrycks_2021_CVPR, Pedraza2022-yo}. Similarly, in NLP, \citet{belinkov2018synthetic} conducted natural perturbations by replacing words in the input source translation sentence with their counterparts from a lookup table created using naturally occurring errors from available corpora of edits.
Motivated by these, we attempt to harvest natural perturbations from available Wikipedia revision histories and utilise them to modify the original MRC instances.
% we adopt the same approach to mine natural perturbations from Wikipedia revision histories, but we do not restrict the edits to single-word replacements.
\uline{To the best of our knowledge, we are the first to investigate MRC model robustness under real natural perturbations.}