\section{MDGD: Modality-Decoupled Gradient Regularization and Descent}
Motivated by the visual forgetting problem caused by the degradation of multimodal encoding in Eq.~\eqref{eq:erank_degradation}, we introduce a modality-decoupling gradient regularization (\textbf{MDGD}) to approximate orthogonal gradients between visual understanding drift and downstream task optimization. Specifically, leveraging modality-decoupled gradients $\Bar{g}_\theta$ and $\Bar{g}_\phi$ derived from the current MLLM and a pre-trained MLLM respectively, we propose a gradient regularization term $\Tilde{g}_\theta$ for more efficient multimodal instruction tuning, which promotes the alignment of downstream tasks while mitigating visual forgetting \cite{zhu2024model}. Since MDGD requires the estimation of parameter gradients, we could not directly apply parameter-efficient fine-tuning methods (\emph{e.g.}, LoRA \cite{hu2021lora}). Thus, we alternatively formulate the regularization as a gradient mask $M_{\Tilde{g}_\theta}$, which allows efficient fine-tuning only on a subset of masked model parameters.

\subsection{Modality Decoupling}
Based on the information bottleneck objective in Eq.~\eqref{eq:ib_vision}, the objective encourages the model to maximize $I(y; Z)$ while compressing $I(X^v; Z)$ \cite{tishby2000information, alemi2016deep}. 
In practice, this compression may discard useful visual details, leading to visual forgetting. To mitigate such compression and preserve the pre-trained visual knowledge, we follow the KL divergence loss
$D_{\text{KL}}\Bigl(\mu_\phi(X^v) \,\Big\|\, \pi_\theta (X^v)\Bigr)$
to constrain the current model’s visual representation $\pi_\theta(X^v)$ to remain close to the pre-trained distribution $\mu_\phi(X^v)$, 
thereby preserving the mutual information $I(X^v; Z)$ that would otherwise be reduced by the compression \cite{hinton2015distilling, lopez2018information}. 
However, since MLLMs cannot directly track the distributions of image tokens, we instead introduce an auxiliary loss function
\begin{equation}\label{eq:visual_loss}
    \mathcal{L}_v(\phi,\theta) = \|\mu(X^v|\phi) - \pi(X^v|\theta)\|_1,
\end{equation}
which approximates the KL divergence loss \cite{zhu2022wdibs,zhu2017unpaired} by penalizing discrepancies between the pre-trained visual representation and that obtained during instruction tuning. 

In the MLLM instruction tuning, the visual output tokens (e.g., $\{z^{vl}_k\}_{k=1}^M$) are encoded as latent representations. 
Such visual encoding cannot be directly supervised by any learning objective but is learned through textual gradient propagation of the negative log-likelihood loss in downstream tasks. 
To approximate the visual optimization direction, we derive the gradients of $\mathcal{L}_v(\phi,\theta)$ for both the pre-trained MLLM $\pi_\phi$ and the current MLLM $\pi_\theta$:
\begin{align*}
    h_{\phi} &= \nabla_{\phi}\mathcal{L}_v(\phi) = \boldsymbol{\lambda}(\phi,\theta) \cdot \nabla_\phi \mu(X^v|\phi), \\
    h_{\theta} &= \nabla_{\theta}\mathcal{L}_v(\theta) = -\boldsymbol{\lambda}(\phi,\theta) \cdot \nabla_\theta \pi(X^v|\theta),
\end{align*}
where $\boldsymbol{\lambda}(\phi,\theta) = \text{sign}\left( \mu(X^v|\phi) - \pi(X^v|\theta) \right)$.
Intuitively, when the MLLM's visual understanding drift causes visual forgetting, we further derive the orthogonal task gradients $\Bar{g}_\phi$ and $\Bar{g}_\theta$:
\begin{align}\label{eq:orth}
    \Bar{g}_\phi &= \nabla_{\phi}\mathcal{L}_{vl}(\phi) - \frac{\nabla_{\phi}\mathcal{L}_{vl}(\phi)^\top h_{\phi}}{\|h_{\phi}\|^2} \cdot h_{\phi}, \\
    \Bar{g}_\theta &= \nabla_{\theta}\mathcal{L}_{vl}(\theta) - \frac{\nabla_{\theta}\mathcal{L}_{vl}(\theta)^\top h_{\theta}}{\|h_{\theta}\|^2} \cdot h_{\theta},
\end{align}
which enables \textbf{modality decoupling} of the downstream task loss gradient in Eq.\eqref{eq:task_loss} orthogonal to the visual understanding drift
for the pretrained MLLM $\Bar{g}_{\phi} \perp h_{\phi}$ and current MLLM $\Bar{g}_{\theta} \perp h_{\theta}$.

\begin{figure}
\centering

\definecolor{mPurple}{rgb}{0.58,0,0.82}
\lstset{
  basicstyle=\footnotesize\fontfamily{ttfamily}\selectfont, % set the font
  keywordstyle={\color{mPurple}}, % set the keyword style
  morecomment=[l]{//},
  commentstyle=\rmfamily\slshape,
  % commentstyle=\itshape, % set the comment style
  showstringspaces=false, % don't show spaces in strings
  columns=fullflexible, % use proportional spacing
  morekeywords={fun,func,let,val,in,end,case,of,SOME, NONE, and, structure, if, else, then, return, def}, % define additional keywords
  mathescape=true, % enable math mode
  escapechar={@},
  keepspaces=true,
  breaklines=true,
  numbers=left,
  numbersep=4pt,
  xleftmargin=2em}

\begin{minipage}[t]{0.32\textwidth}
\begin{lstlisting}
$\Omega$: int
$\oracle{}$: circuit $\rightarrow$ circuit
$\cost{}$: circuit $\rightarrow$ int
compact: circuit $\rightarrow$ circuit

def $\mathsf{\algname{}}(C)$:
  $C'$ = segopt(compact($C$))
  if $C' = C$:
    return $C$
  else:
    return $\mathsf{\algname{}}(C')$
\end{lstlisting}
\end{minipage}
\begin{minipage}[t]{0.32\textwidth}
\begin{lstlisting}[firstnumber=12]
def segopt($C$):
  $d$ = length$(C)$
  if $d \leq 2\Omega$:
    $C'$ = $\oracle{}(C)$
    if $\costof{C'} < \costof{C}$:@\label{line:guard1start}@
      return $C'$
    else:
      return $C$@\label{line:guard1stop}@
  else:
    $m$ = $\lfloor d/2 \rfloor$
    $C_1$ = $C[0 : m]$
    $C_2$ = $C[m : d]$
    $C_1'$ = segopt($C_1$)
    $C_2'$ = segopt($C_2$)
    return meld$(C_1', C_2')$
\end{lstlisting}
\end{minipage}
\begin{minipage}[t]{0.345\textwidth}
\begin{lstlisting}[firstnumber=27]
def meld$(C_1, C_2)$:
  $d_1$ = length($C_1$)
  $d_2$ = length($C_2$)
  $W$ = $C_1 [d_1 - \Omega : d_1] + C_2 [0 : \Omega]$@\label{line:combine}@
  $W'$ = $\oracle{}(W)$@\label{line:guard2start}@
  if $\costof{W'} = \costof{W}$:
    return $(C_1 ; C_2)$@\label{line:guard2stop}@  // concatenate
  else:
    $M$ = meld$(C_1[0 : d_1 - \Omega], W')$@\label{line:mrec}@
    return meld($M, C_2[\Omega : d_2]$)
\end{lstlisting}
\end{minipage}

% $\Omega$: int
% Oracle: circuit $\rightarrow$ circuit
% cost: circuit $\rightarrow$ int

% def meld$(C_1, C_2)$:
%   $P$, $W_1$ = splitLastLayers$(C_1, \Omega)$
%   $W_2$, $S$ = splitFirstLayers$(C_2, \Omega)$
%   $W = W_1 + W_2$
%   $W'$ = Oracle$(W)$
%   if $\costof{W'} = \costof{W}$:
%     return concat$(P, W, S)$
%   else:
%     // assert ($\costof{W'} < \costof{W}$)
%     return meld(meld$(P, W')$, $S$)

% $c_1 + c_2$ = c $\mathit{where}\ \sizeof{c_1} = \sizeof{c}/2$
% $p_1 + s_1$ = $c_1$  $\mathit{where}\ \sizeof{s_1} = \Omega$
% $p_2 + s_2$ = $c_2$  $\mathit{where}\ \sizeof{p_2} = \Omega$
% $\textit{wd} \gets s_1 + p_2$
% $\textit{wd}' \gets \mathcal{O}(\textit{wd})$



% \begin{lstlisting}
% fun meld (c1, c2) =
%   let
%     val (c1s, c1p) = C.splitEnd (c1, $\Omega$)
%     val c2' = C.merge (c1p, c2)
%   in
%     case opt_prefix (2*$\Omega$, c2') of
%       SOME c2'' => meld (c1s, c2'')
%     | NONE => C.merge (c1, c2')
%   end

% and fun opt_prefix (prefix_size, c) =
%   let
%     val (cp, cs) = C.split (c, prefix_size)
%   in
%     case $\mathcal{O}$(cp) of
%       SOME cp' => SOME (meld (cp', cs))
%     | NONE => NONE
%   end
% \end{lstlisting}

\caption{
Algorithm \algname{} produces locally optimal circuits with
respect to a given $\oracle{}$, $\cost{}$, and segment length $\Omega$.
%
To achieve local optimality,
\algname{} only uses the oracle on small segments of length $2\Omega$.
%
The algorithm repeatedly optimizes and compacts the circuit
until convergence.
%
The function \textsf{segopt}$()$ implements our optimization algorithm
and uses $\mathsf{meld}()$ to efficiently produce \wopttext{} circuits.
}
% \vspace{-1.5in}
\label{fig:lopt-code}
\end{figure}


\section{Main Algorithm}
\label{sec:main_alg}
We present our main result in this section and explain the algorithm in a top-down manner.  The algorithm is based on the localization framework of  
\cite{FKT20}; see Algorithm~\ref{alg:loacalizatioin} in the Appendix for details. The main result is stated formally below:
\begin{theorem}
\label{thm:main_result}
Under Assumptions~\ref{assum:lispchitz_smooth} and \ref{assump:dia_dominant}, suppose $\beta\le\frac{G}{D}(\frac{\sqrt{n}\epsilon}{\sqrt{m}\log(nmd/\delta)}+\frac{\sqrt{d\log(1/\delta)\log(nmd)}}{\sqrt{m}\epsilon})$, $\epsilon\le O(1),n\ge \log^2(nd/\delta)/\epsilon$ and $ m\le n^{O(\log\log n)}$.
Setting $\eta\le\frac{D}{G}\cdot \min\{ \frac{B\sqrt{m}}{\sqrt{n}} ,   \frac{\sqrt{m}\epsilon}{\sqrt{d\log(1/\delta)\log( nmd)}}\}$, $B=100\log(mnd/\delta)/\epsilon$, $\tau=O(G\log(nmd)/\sqrt{m})$ and $\upsilon=0.9B+\frac{2\log(T/\delta)}{\epsilon}$, Algorithm~\ref{alg:loacalizatioin}  is $(\epsilon,\delta)$-user-level-DP. 
When the $nm$ functions in dataset $\calD$ are i.i.d. drawn from the underlying distribution $\calP$, it takes $mn$ gradient computations and outputs $x_S$ such that
    \begin{align*}
        \E[F(x_S)-F(x^*)]\le \Tilde{O} \left(\frac{d}{\sqrt{nm}}+\frac{d^{3/2}}{n\epsilon^2\sqrt{m}} \right).
    \end{align*}
\end{theorem}

We briefly describe the localization framework.  
In the first phase, it runs (non-private) SGD using half of the dataset, and averages the iterates to get $\bx_1$.
Roughly speaking, the solution $\bx_1$ already provides a good approximation with a small population loss when the datasets are  drawn i.i.d. from the underlying distribution. However, to ensure privacy, we require a  
sensitivity bound on $\|\bx_1\|$ and add noise $\zeta_1$ correspondingly to privatize $\bx_1$, yielding the private solution $x_1 \leftarrow \bx_1 + \zeta_1$.  

A naive bound on the excess loss due to the privatization is given by  
\[
\E[F(x_1) - F(\bx_1)] \leq G\|\zeta_1\|_2,
\]  
but the magnitude of the noise $\|\zeta_1\|_2$ is typically too large  
to achieve a good utility guarantee. Nevertheless, this process yields  
a much better initial point $x_1$ compared to the original starting  
point $x_0$. 
As a result, a smaller dataset and a smaller step size are sufficient  
to find the next good solution $\bx_2$ in expectation, with smaller noise $\|\zeta_2\|_2$ added to privatize $\bx_2$.  

This process is repeated over $O(\log n)$ phases, where each subsequent solution $\bx_S$ is progressively refined, and the Gaussian noise  
$\|\zeta_S\|_2$ becomes negligible. Ultimately, this iterative refinement  
balances privacy and utility, as established in Theorem~\ref{thm:main_result}.  
The formal argument about the utility guarantee and proof can be found in Lemma~\ref{lm:localization}.  

Our main contribution is in Algorithm~\ref{alg:dpsgd},  
which uses a novel gradient estimation sub-procedure.

% Given some initial point $x$, we define the Projected Gradient Descent sequences $\{x_{t}^Z\}_{t\in[m/K]}$, denoted by $\PGD(x,\eta,Z,K)$ for each user $Z$ with batch size $K$ as follows:
% \begin{align}
% \label{eq:PGD_each_user}
%     x_t^Z= \Pi_{\calX}(x^Z_{t-1}-\eta\frac{1}{K}\sum_{j\in[K]}\nabla f(x^Z_{t-1};z_{t,j})),
% \end{align}
% where $\{z_{t,j}\}_{j\in[K]}$ is a set of unused item functions of size $K$.
% This is simply running GD for each user for $m/K$ iterations, with batch size $K$ for each iteration.

\begin{algorithm2e}
\caption{SGD for User-level DP-SCO}
\label{alg:dpsgd}
\textbf{ Input:} dataset $\calD$, privacy parameters $\epsilon,\delta$, other parameters $\eta,\tau,\upsilon,B$, initial point $x_0$\;
%\textbf{ Process:} \\
Divide $\calD$ into {B} disjoint subsets of equal size, denoted by $\{\calD_i\}_{i\in[B]}$,
$\calD_i=\{Z_{i,t}\}_{t\in[|\calD|/B]}$\; 
Set $T=|\calD|/B$\;
\For{Step $t=1,\ldots,T$}
{
For each $i\in[B]$, get $q_t(Z_{i,t}):=\frac{1}{m}\sum_{z\in Z_{i,t}}\nabla f(x_{t-1};z)$\;
Let $g_{t-1}$ be the output of Algorithm~\ref{alg:robust_gradient_est} with inputs $\{q_t(Z_{i,t})\}_{i\in[B]}$ and threshold $1/\tau$\;
$x_{t}=\Pi_\calX(x_{t-1}-\eta g_{t-1})$
}
\tcc{Concentration Test}
\tcc{Recall the query $q_t(Z_{i,t})$ for each $t\in[T], i\in[B]$ from above}
Run Algorithm~\ref{alg:out_rem} with query $\{q_t\}_{t\in[T]}$ and parameters $\calD_t,\epsilon,\frac{\delta}{2Tmnd},\tau,\upsilon$ to get answers $\{a_t\}_{t\in [T]}$ \;
\If{$a_t=\top,\forall t\in[T]$}
{
\textbf{ Return:} Average iterate $\bar{x}=\frac{1}{T}\sum_{t\in[T]}x_t$\;
}
\Else
{
\textbf{ Output:} Initial point $x_0$\;
}
\end{algorithm2e}

\paragraph{ Iteration Sensitivity of Algorithm~\ref{alg:dpsgd}:}
The contractivity of gradient descent plays a crucial role in the sensitivity analysis, for which we need the Hessians to be diagonally  dominant
(Assumption~\ref{assump:dia_dominant}). 

\begin{restatable}{lem}{contractivity}[Contractivity]
    \label{lm:contractivity}
Suppose $f:\calX\to\R$ is a convex and $\beta$-smooth function satisfying Assumption~\ref{assump:dia_dominant}, then for any two points $x,y\in \calX$, with step size $\eta\le 2/\beta$, we have
    \begin{align*}
        \|(x-\eta \nabla f(x)) - (y-\eta \nabla f(y))\|_\infty\le \|x-y\|_\infty.
    \end{align*}
\end{restatable}

Now, we discuss Algorithm~\ref{alg:dpsgd}.  
Given the dataset $\calD$, we proceed in $T = |\calD|/B$ steps.  
At the $t$th step, we draw $B$ users $\{Z_{i,t}\}_{i \in [B]}$ and compute the average gradient of each user. 
We then apply our gradient estimation algorithm (Algorithm~\ref{alg:robust_gradient_est}) and perform normal gradient descent for $T$ steps.  

In the second phase of Algorithm~\ref{alg:dpsgd}, we perform the concentration test  
(Algorithm~\ref{alg:out_rem}) on the $B$ gradients at each step based on $\AboTh$ (Algorithm~\ref{alg:mean_est_with_AT}).  
If the concentration test passes for all steps (i.e., $a_t = \top$  
for all $t \in [T]$), we output the average iterate. Otherwise, the  
algorithm fails and returns the initial point.  
As mentioned in the Introduction, the crucial novelty of Algorithm~\ref{alg:dpsgd}  
and Algorithm~\ref{alg:robust_gradient_est} lies in bounding the sensitivity  
of each solution $x_t$ by incorporating the (coordinate-wise) robust  
statistics into SGD.

% As discussed before, we apply the (coordinate-wise) geometric median into the SGD algorithm, and show that the iteration-sensitivity can always be bounded in terms of the $\ell_\infty$ norm when the number of ``bad'' users does not exceed the ``break point''.

% Our algorithm framework is based on SGD.
% For the $t$-th phase, we get solution $x_t$ and then take a batch of $B$ users, denoted by $\{Z_{i,t}\}_{i\in[B]}$.
% Each user owns $m$ functions and can run their own gradient descent freely with batch size $K$ from $[1,m]$.
% We take $K=m$ for simplicity; that is, each user takes the average of the $m$ gradients and does one descent step, and we get $\{x_1^{Z_{i,t}}\}_{i\in[B]}$.
% Then we let $x_{t+1}:=\arg\min_{x}\sum_{i\in[B]}\|x-x_1^{Z_{i,t}}\|_\infty$ be the geometric median over the $B$ points. 

\begin{algorithm2e}
\caption{Gradient Estimation based on Robust Statistics}
\label{alg:robust_gradient_est}
\textbf{ Input:} a set of $d$-dimensional vectors $\{X_i\}_{i\in[B]}$, threshold parameter $\varsigma>0$\;
%Initialize a zero vector $X_{est}=\mathbf{0}$\;
\For{Each dimension $j=1,\ldots,d$}
{
Compute the robust statistics $X_{\rs}[j]$, and the mean $\bx[j]$ over $\{X_{i}[j]\}_{i\in[B]}$\;
\If{$|X_{\rs}[j]-\bx[j]|\ge \varsigma$}
{
Set $X_{est}[j]=\Pi_{B(Y_j,\varsigma)}(\bx[j])$\;
}
\Else
{
Set $X_{est}[j]=\bx[j]$\;
}
}
\textbf{ Return $X_{est}$}
\end{algorithm2e}


We utilize robust statistics in the  
gradient estimation sub-procedure. 
We make the following assumptions regarding the robust statistics used:

\begin{assumption}
\label{assum:prop_geo_median}
    Given a set $\{X_i\}_{i \in [B]}$ of vectors, let $X_{\rs}$ be  
    any robust statistic satisfying the following properties:
    
    (i) For any $\rho \ge 0$, if there exists a point $X'$ such  
        that more than $B/2$ points lie within $B_\infty(X', \rho)$,  
        then $X_{\rs} \in B_\infty(X', \rho)$.
        
(ii) If we perturb each point $Y_i = X_i + \iota_i$ such that  
        $\|\iota_i\|_\infty \le \Delta$ for any $\Delta \ge 0$, and let  
        $Y_{\rs}$ be the robust statistic of $\{Y_i\}$, then  
        $\|X_{\rs} - Y_{\rs}\|_\infty \le \Delta$.
        
    (iii) For any real numbers $a$ and $b$, if $Z_i = aX_i + b$ for  
        each $i \in [B]$, and $Z_{\rs}$ is the corresponding robust  
        statistic of $\{Z_i\}_{i \in [B]}$, then $Z_{\rs} = aX_{\rs} + b$.  
\end{assumption}

\begin{remark}
    Common robust statistics, such as the (coordinate-wise) median and trimmed mean,  
    satisfy Assumption~\ref{assum:prop_geo_median}.
    %Pasin: I'm commenting the following out since I don't think all robust statistics are computed in coordinate-wise manner.
    %One can verify  
    %whether the robust statistic satisfies Assumption~\ref{assum:prop_geo_median}  
    %in one dimension, as robust statistics can be computed in a  
    %coordinate-wise manner.
\end{remark}
\vspace{-2mm}
In Algorithm~\ref{alg:robust_gradient_est}, we output means if they are close to the robust statistics to control the bias, and project the means onto the sphere around the robust statistics in a coordinate-wise manner when they are far apart.  
However, we still need to ensure that the sensitivity remains bounded when the projection is operated.  
The following technical lemma plays a crucial role in establishing iteration sensitivity to deal with the sensitivity with potential projection operations.
% Its proof can be found in the Appendix:
\vspace{-1mm}

\begin{restatable}{lem}{projmeantors}
\label{lm:proj_mean_to_rs}
Consider four real numbers $a,b,c,d$, such that $|a-b|\le 1$, and $|c-d|\le 1$.
Let $c'=\Pi_{B(a,r)}(c)$ and $d'=\Pi_{B(b,r)}(d)$ for any $r\ge 0$.
Then, we have $|c'-d'|\le 1.$
\end{restatable} 


Unfortunately, we are unaware of any robust statistic satisfying  
Assumption~\ref{assum:prop_geo_median} in high dimensions under the  
$\ell_2$-norm, and Lemma~\ref{lm:proj_mean_to_rs} does not hold in high  
dimensions either. These limitations restrict the applicability of our  
techniques in high-dimensional Euclidean spaces; see Section~\ref{sec:discussion}.  

Let $\{x_t\}_{t \in [T]}$ and $\{y_t\}_{t \in [T]}$ be two trajectories  
corresponding to neighboring datasets that differ by one user. The  
crucial technical novelty is that, for any $t \in [T]$, we can control  
$\|x_t - y_t\|_{\infty}$ as long as the number of ``bad'' users in each  
phase ($B$ in total) does not exceed the ``break point'', say $2B/3$.  
Without loss of generality, assume that $Z_{1,1} \neq Z_{1,1}'$ is the  
differing user in the neighboring dataset pairs $(\calD, \calD')$  
considered in the following proof.  

The first property of Assumption~\ref{assum:prop_geo_median} ensures that when the number of ``bad'' users in each phase does not exceed the  ``break point'' $2B/3$, the robust statistic remains close to most of the gradients, allowing us to control $\|x_1 - y_1\|_\infty$.  
To formalize this, we say that the neighboring dataset pair 
$(\calD, \calD')$ is $\rho$-\textit{aligned} if there exist points  
$X'$ and $Y'$ such that $|X_{\good}| \ge 2B/3$ and  
$|Y_{\good}| \ge 2B/3$, where  
\[
    X_{\good} = \{q_1(Z_{i,1}) : q_1(Z_{i,1}) \in B_{\infty}(X', \rho),  
    i \in [B]\},  \text{ and }
\]  
\[
    Y_{\good} = \{q_1'(Z_{i,1}') : q_1'(Z_{i,1}') \in B_{\infty}(Y', \rho),  
    i \in [B]\}.  
\]  
This definition essentially states that the number of ``bad'' users does  
not exceed the ``break point'' in either $\calD$ or $\calD'$, ensuring  
that most gradients remain well-aligned within a bounded region.

\begin{restatable}{lem}{itesensitivitybase}
    \label{lm:ite_sensitivity_base}
    For some (unknown) parameter $\rho > 0$, suppose $(\calD, \calD')$  
    is $\rho$-aligned. Then, by running Algorithm~\ref{alg:robust_gradient_est}  
    with threshold parameter $\varsigma \ge 0$, we have $\|x_1 - y_1\|_\infty \le \eta(4\rho + 2\varsigma)$.
\end{restatable}


% Now the sequential sensitivity can be bounded by induction, for which the base case, $\|x_1-y_1\|_\infty$ is already bounded.
% Say $\|x_{t-1}-y_{t-1}\|_\infty$ is bounded,
% then by Lemma~\ref{lm:contractivity}, we can show that $\|x_{j}^{Z_{i,t}}-y_{j}^{Z_{i,t}}\|_\infty\le \|x_{t-1}-y_{t-1}\|_\infty$.
% We then treat $x_{j}^{Z_{i,t}}-y_{j}^{Z_{i,t}}$ as the perturbation and apply the second property in Assumption~\ref{assum:prop_geo_median}, which leads to that $\|x_t-y_t\|_\infty\le \|x_{t-1}-y_{t-1}\|_\infty$.
% The formal statements can be found in Lemma~\ref{lm:iteration_sensitivity}.
The sequential sensitivity can be bounded using induction, with the base  
case $\|x_1 - y_1\|_\infty$ already established. The formal statement  
is provided in Lemma~\ref{lm:iteration_sensitivity}.  

\begin{algorithm2e}
\caption{Concentration Test}
\label{alg:out_rem}
\textbf{ Input:} Dataset $\calD=(Z_1,\ldots,Z_B)$, privacy parameters $\epsilon,\delta$, parameters $\tau,\upsilon$\;
\For{$t=1,\ldots,T$}
{ 
Receive a new concentration query $q_t:\calZ\to\R^d$\;
Define the concentration score
\begin{align}
\label{eq:concentration_score_def}
    \qcon_t(\calD,\tau):=\frac{1}{B}\sum_{Z\in\calD}\sum_{Z'\in \calD}\exp(-\tau\|q_t(Z)-q_t(Z')\|_\infty)\;
\end{align}
\textbf{ Return }$\AboTh(\qcon_t, \epsilon/2, \upsilon)$
}
\end{algorithm2e}


\begin{restatable}[Iteration Sensitivity]{lem}{iterationsensitivity}
\label{lm:iteration_sensitivity}  
    If we use a robust statistic satisfying Assumption~\ref{assum:prop_geo_median}  
    in Algorithm~\ref{alg:robust_gradient_est}, then for all $t \in [T]$, we have  $\|x_t - y_t\|_\infty \le \|x_1 - y_1\|_\infty$.
\end{restatable}

Lemmas~\ref{lm:ite_sensitivity_base} and \ref{lm:iteration_sensitivity}  
together establish the iteration sensitivity of Algorithm~\ref{alg:dpsgd}.

\paragraph{ Query Sensitivity of Concentration Test (Algorithm~\ref{alg:out_rem}):}
We have established iteration sensitivity for any aligned neighboring  
dataset pair $(\calD, \calD')$. Next, we analyze the influence of the  
concentration test, which we use to check if the number of ``bad'' users exceed the ``break point''.

To apply the privacy guarantee of $\AboTh$  
(Lemma~\ref{thm:Above_Threshold}), it suffices to bound the sensitivity  
of each query in the concentration test.  
Recall that we assume $Z_{1,1} \neq Z_{1,1}'$ in the neighboring datasets.  
Thus, by the definition (Equation~\eqref{eq:concentration_score_def}), it is straightforward to observe that  
\begin{align}
\label{eq:query_sensitivity_qcon_one}
    |\qcon_1(\calD, \tau) - \qcon_1(\calD', \tau)| \le 2.  
\end{align}  

Next, we consider the sensitivity of $\qcon_t$ for $t \ge 2$.  
The sensitivity is proportional to $\|x_t - y_t\|_\infty$, which we have  
already bounded by $\|x_1 - y_1\|_\infty$.  
Note that we can bound the iteration sensitivity if the neighboring  
datasets are aligned, meaning the number of ``bad'' users does not  
exceed the ``break point''. We first show that if the number of ``bad''  
users exceeds the ``break point'', the algorithm is likely to halt  
after the first step by failing the first test.


\begin{restatable}{lem}{sensitivitybase}
    \label{lm:sensitivity_base}
Suppose $B\ge \frac{100\log(T/\delta)}{\epsilon}, \epsilon\le O(1)$ and we set $\upsilon=0.9B+\frac{2\log(T/\delta)}{\epsilon}$.
Suppose for any point $Y$, we get $|X_{\good}|<B/3$ where $X_{\good}=\{q_1(Z_{i,1}):q_1(Z_{i,1})\in B_{\infty}(Y,1/\tau),i\in[B]\}$.
Then with probability at least $1-\delta/T\exp(\epsilon)$, the $\AboTh$ returns $a_1=\bot$.
\end{restatable}






% \begin{lemma}[Query Sensitivity: Part One]
% \label{lm:query_diff}
% Consider two initial points $x$ and $y$ such that $\|x-y\|_\infty \le \nu$, and get $\{x_{t}^Z\}_{t\in[m/K]}$ and $\{y_{t}^Z\}_{t\in[m/K]}$ respectively from running $\PGD(x,\eta,Z,K)$ and $\PGD(y,\eta,Z,K)$, defined from Equation~\eqref{eq:PGD_each_user}.
% Similarly, we get $\{x_t^{Z'}\}_{t\in[m/K]}$ and $\{y_t^{Z'}\}_{t\in[m/K]}$ for another user $Z'$.
% Then under Assumptions~\ref{assum:lispchitz_smooth}, with $\eta\beta\le1$, for any $t\in[m/K]$, we have
% \begin{align*}
%     \Bigg|\|x_t^Z-x_t^{Z'}\|_\infty-\|y_t^{Z}-y_t^{Z'}\|_\infty\Bigg|\le 2\nu \eta\beta.
% \end{align*}
% \end{lemma}

% \begin{proof}
%     % We prove the statement by induction.
%     % As for the basic case when $t=1$, we have
%     % \begin{align*}
%     %    & \|x_t^Z-x_t^{Z'}\|_\infty-\|y_t^Z-y_t^{Z'}\|_\infty\\
%     %    =& \|x_t^Z-x-(x_t^{Z'}-x)\|_\infty-\|y_t^Z-y-(y_t^{Z'}-y)\|_\infty\\
%     %    \le & \|x_t^Z-x-(y_t^Z-y)+(y_t^{Z'}-y)-(x_t^{Z'}-x)\|_\infty\\
%     %    \le & \|x_t^Z-x-(y_t^Z-y)\|_\infty+\|(y_t^{Z'}-y)-(x_t^{Z'}-x)\|_\infty\\
%     %    \le & 2\eta\nu\beta,
%     % \end{align*}
%     % where the last inequality comes from the assumption on smoothness.

%     % Now suppose the condition holds for any $t\le t'$, consider the case when $t=t'+1$.
%     Letting $x_0^Z=x_0^{Z'}=x$ and $y_0^Z=y_0^{Z'}$, notice that
%     \begin{align*}
%         &\|(x_t^Z-x_{t-1}^Z)-(y_t^Z-y_{t-1}^Z)\|_\infty \\
%         \le&  \eta/K\|\sum_{j\in[K]}\nabla f(x_{t-1}^Z;z_{t,j})-\nabla f(y_{t-1}^Z;z_{t,z})\|_\infty \\
%         \le & 2\eta\beta \|x_{t-1}-y_{t-1}^Z\|_\infty\\
%         \le & 2\eta\beta\nu.
%     \end{align*}
    
%  Hence, we have
%     \begin{align*}
%         & \|x_t^Z-x_t^{Z'}\|_\infty-\|y_t^Z-y_t^{Z'}\|_\infty\\
%        =& \|\sum_{i=1}^{t}(x_i^Z-x_{i-1}^Z)-\sum_{i=1}^t(x_i^{Z'}-x_{i-1}^{Z'})\|_\infty-\|\sum_{i=1}^{t}(y_i^Z-y_{i-1}^Z)-\sum_{i=1}^t(y_i^{Z'}-y_{i-1}^{Z'})\|_\infty\\
%        \le & \|\sum_{i=1}^{t} (x_i^Z-x_{i-1}^Z)-(y_i^Z-y_{i-1}^Z)\|_\infty+\|\sum_{i=1}^t(y_i^{Z'}-y_{i-1}^{Z'})-(x_i^{Z'}-x_{i-1}^{Z'})\|_\infty\\
%        \le & 2t\nu\eta\beta.
%     \end{align*}
%     This completes the proof.
% \end{proof}

We now analyze the query sensitivity between the aligned neighboring  
datasets.

\begin{restatable}[Query Sensitivity]{lem}{querysensitivity}
    \label{lm:query_sensitivity}
Suppose $6\beta\eta B\le1$.
Suppose $(\calD,\calD')$ is $(1/\tau)$-aligned and set threshold parameter $\varsigma=1/\tau$ in Algorithm~\ref{alg:mean_est_with_AT}, the sensitivity of the query is bounded by at most $2$.
That is,
\begin{align*}
    |\qcon_t(\calD,\tau)-\qcon_1(\calD',\tau)|\le 2, & & \forall t\ge 2.
\end{align*}
\end{restatable}



Equation~\eqref{eq:query_sensitivity_qcon_one} shows that the sensitivity is always bounded for $\qcon_1$.  
Lemma~\ref{lm:sensitivity_base} shows that if the number of ``bad''  
users exceeds the ``break point'', we obtain $a_1 = \bot$, and  
the query sensitivities of the subsequent queries do not need to be considered.  
Lemma~\ref{lm:query_sensitivity} establishes the query sensitivity  
in the concentration test when the neighboring datasets are aligned,  
and the number of "bad" users is below the threshold.

\paragraph{Privacy proof.}
% Consider the implementation on two neighboring datasets $\calD$ and $\calD'$.
% Without loss of generality, we assume that the different users appeared in the first batch, that is, $t=1$.

%Now, we can complete the proof of the privacy guarantee.

%The final privacy guarantee is stated below. 
The final privacy guarantee--stated formally below--now easily follows from the previous lemmas.
% Due to space constraint, 
The full proof is deferred to Appendix~\ref{app:privacy-proof}.

\begin{restatable}[Privacy Guarantee]{lem}{privacyguarantee}
    \label{lm:privacy_guarantee}
    Under Assumption~\ref{assum:lispchitz_smooth} and Assumption~\ref{assump:dia_dominant}, suppose $\epsilon\le O(1), B\ge\frac{100\log(T/\delta)}{\epsilon}$, then Algorithm~\ref{alg:loacalizatioin} is $(\epsilon,\delta)$-user-level-DP.
\end{restatable}





\paragraph{Utility proof.}
We apply the localization framework in private optimization to finish the utility argument.
We analyze the utility guarantee of Algorithm~\ref{alg:dpsgd} based on the classic convergence rate of SGD on smooth convex functions (Lemma~\ref{lm:sgd_smooth}) as follows:
% The following classic result states the convergence rate of SGD for smooth convex functions.



% \Daogao{Clean the notations..}
% We have the following lemma:

% \begin{lemma}[\cite{LLA24}]
% \label{lemma_tech_core}
% Assume $f(\cdot, z)$ is convex, $G$-Lipschitz, and $\beta$-smooth on $\calX$ with $\eta \leq 1/\beta$. Let $\tilde{x} \gets SGD(D, \eta, T, x_0,1)$ and $\tilde{y} \gets SGD(D', \eta, T, x_0,1)$ be two independent runs of projected SGD, where
% $D, D' \sim \calP^T$ are i.i.d. Then, with probability at least $1 - \zeta$, we have \[
% \|\tilde{x} - \tilde{y}\|_2 \lesssim \eta G\sqrt{T \log(dT/\zeta)}.
% \]
% \end{lemma}

% As may be noticed, the naive bound we can get is $\|\title{x}-\title{y}\|_2\le 2\eta LT$.
% Hence, the distributional assumption on $\calD$ and $\calD'$ improves the stability from $\Tilde{O}(\eta LT)$ to $\Tilde{O}(\eta L\sqrt{T})$, which is crucial in getting improved results in user-level setting.

% We generalize it into a batched version of SGD, that is the batch size of each iterate is captured by a parameter $K\ge 1$:
% \Daogao{Overuse notation $T$...}

% \begin{lemma}
% \label{lm:batched_tech_core}
%     Assume $f(\cdot, z)$ is convex, $G$-Lipschitz, and $\beta$-smooth on $\calX$ with $\eta \leq 1/\beta$. Let $\tilde{x} \gets SGD(D, \eta, T, x_0,K)$ and $\tilde{y} \gets SGD(D', \eta, T, x_0,K)$ be two independent runs of projected SGD, where
% $D, D' \sim \calP^{TK}$ are i.i.d. Then, with probability at least $1 - \zeta$, we have \[
% \|\tilde{x} - \tilde{y}\|_2 \lesssim \eta G\sqrt{T \log(dT/\zeta)/K}.
% \]
% \end{lemma}

% \begin{proof}
% Let $g_t:= \frac{1}{K}\sum_{i\in[K]}\nabla f(x_t, z_{t,i})$ for $\{z_{t,i}\}_{i\in[K]}$ drawn uniformly from $D$ without replacement and $g_t':=  \frac{1}{K}\sum_{i\in[K]}\nabla f(y_t, z'_{t,i})$ for $z'_{t,i}$ drawn uniformly from $D'$ without replacement. Let $F(x) := \E_{z \sim \calP}[f(x,z)]$. 

% We will prove that $\|x_t - y_t\| \lesssim \eta L\sqrt{T \log(dT/\zeta)/K}$ with probability at least $1 - \zeta/t$ for all $t \in [T]$. Note that this implies the lemma. We proceed by induction. The base case, when $t=0$, is trivially true since $x_0 = y_0$. For the inductive hypothesis, suppose there is an absolute constant $c > 0$ such that with probability at least $1-t\zeta/T$, we have 
% \begin{align*}
%     \|x_{i}-y_i\|\le  c  \eta L\sqrt{i\cdot \log(dT/\zeta)/K},
% \end{align*}
% $\forall i \le t$. Then, for the inductive step, we have by non-expansiveness of projection onto convex sets, that
% \begin{align}
% \label{eq: thingy}
%     \|x_{t+1}-y_{t+1}\|^2 \le &~ \|x_t-\eta g_t-(y_t-\eta g_t')\|^2 \nonumber \\
%     =& ~ \|x_t-\eta \nabla F(x_t)-(y_t-\eta \nabla F(y_t))-\eta (g_t- \nabla F(x_t)-g_t'+\nabla F(y_t)   )\|^2 \nonumber\\
%     =& ~ \|x_t-\eta\nabla F(x_t)-(y_t-\eta \nabla F(y_t))\|^2 \nonumber\\ 
%     &~-2\eta \langle  x_t-\eta\nabla F(x_t)-(y_t-\eta \nabla F(y_t)),g_t- \nabla F(x_t)-g_t'+\nabla F(y_t)\rangle \nonumber\\
%     &~+ \eta^2 \|g_t-\nabla F(x_t)-g_t'+\nabla F(y_t)\|^2 \nonumber \\
%     \stackrel{(i)}{\le} & ~ \|x_t-y_t\|^2-2\eta \langle  x_t-\eta\nabla F(x_t)-(y_t-\eta \nabla F(y_t)),g_t- \nabla F(x_t)-g_t'+\nabla F(y_t)\rangle \nonumber \\
%     &~+ \eta^2 \|g_t-\nabla F(x_t)-g_t'+\nabla F(y_t)\|^2, 
% \end{align}
% where $(i)$ follows from the non-expansive property of gradient descent on smooth convex function for $\eta \le 1/\beta$~\cite{hardt16}.

% For any $t\in T$, we have
% \begin{align*}
%  \Pr\Big[\eta^2 \|g_t-\nabla F(x_t)-g_t'+\nabla F(y_t)\|^2\ge 4\log(Td/\zeta)\eta^2L^2/K\Big]\le \zeta/T .  
% \end{align*}
% Conditional on this event in the following argument.

% Define $a_t:=-2\eta \langle  x_t-\eta\nabla F(x_t)-(y_t-\eta \nabla F(y_t)),g_t- \nabla F(x_t)-g_t'+\nabla F(y_t)\rangle$.
% By Inequality~\eqref{eq: thingy} and the inductive hypothesis, we obtain
% \begin{align*}
%     \|x_{t+1}-y_{t+1}\|^2\le 4t\log(Td/\zeta)\eta^2L^2/K+\sum_{i=1}^{t}a_t.
% \end{align*}
% It remains to bound $\sum_{i=1}^t a_i$.
% Note that $\E[a_i \mid a_1,\cdots,a_{i-1}]=0$ and $g_t-\nabla F(x_t)$ is $\nSG(\log(d/\zeta)/\sqrt{K})$. 
% By Lemma~\ref{lm:inner_product_nSG}, we know there is a constant $c' > 0$ such that $a_i$ is $\nSG(c' \eta L \|x_i-y_i\|/\sqrt{K})$ for all $i$.
% Hence by Theorem~\ref{thm:hoeffding_nSG}, we know
% \begin{align*}
% \mathbb{P}\left[\left|\sum_{i=1}^{t}a_i\right|\ge c'\eta L\sqrt{\log(dT/\zeta)\sum_{i\le t}\|x_i-y_i\|^2/K}\right]\le 1-\zeta/T.
% \end{align*}

% Conditional on the event that $\|x_{i}-y_{i}\|\le c\sqrt{\log(dT/\zeta)}\eta L\sqrt{i/K}$ for all $i \leq t$ (which happens with probability $1-t\zeta/T$ by the  inductive hypothesis), we know
% \begin{align*}
% \mathbb{P}\left[\left|\sum_{i=1}^{t}a_i\right|\ge (cc')tL^2\eta^2\log(dT/\zeta)/K \middle| \|x_i-y_i\|\le c\log(dT/\zeta)\eta L\sqrt{i/K},\forall i\le t\right]\le 1-\zeta/T.
% \end{align*}
% Hence, as long as $4t+cc't\le c^2(t+1)$, we know
% \begin{align*}
%     \mathbb{P}\left[\|x_{t+1}-y_{t+1}\|^2\ge c^2\log(dT/\zeta)\eta^2L^2(t+1)/K \middle|  \|x_i-y_i\|\le c\log(dT/\zeta)\eta L\sqrt{i/K},\forall i\le t \right]\le 1-\zeta/T.
% \end{align*}
% Combining the above elements completes the inductive step, showing that 
% \[\|x_{t+1}-y_{t+1}\|\le c\sqrt{(t+1)\log(dT/\zeta)/K}\eta L\]
% with probability at least $1-(t+1)\zeta/T$.
% This completes the proof.
% \end{proof}


% \begin{lemma}
% \label{lem:concentrated_grd}
% Under Assumption~\ref{assum:lispchitz_smooth},
% for any fixed $x$ and for each $Z_i$, if each item in $Z_i$ is drawn i.i.d. from $\calP$, with probability at least $1-\gamma/n$, we have
% \begin{align*}
%     \|\nabla F(x;Z_i)-\nabla F_\calP(x)\|_2\lesssim \frac{G\log(nd/\gamma)}{\sqrt{m}},
% \end{align*}
% \end{lemma}



% \subsection{Further Potential Improvements}
% It is interesting to see if we can generalize the results to the $\ell_2$ norm by utilizing other estimators.
% In the $\ell_2$ norm, we may use and generalize the following stability lemma from \cite{LLA24}.



\begin{restatable}{lem}{dgsgdutility}
    \label{lm:dpsgd_utility}
    Let $x\in\calX$ be any point in the domain.
    Suppose the data set $\calD$ of the users, whose size $|\calD|$ is larger than $\frac{100\log(T/\delta)}{\epsilon}$, is drawn i.i.d. from the distribution $\calP$.
    Setting $\tau=G\log(nmd/\omega)/\sqrt{m}$
    then the final output $\bar{x}$ of Algorithm~\ref{alg:dpsgd} satisfies that
    \begin{align*}
        \E[F(\bar{x})-F(x)]\lesssim \left(\beta+\frac{1}{\eta} \right)\frac{\E[\|x_0-x\|^2]}{T}+\frac{\eta G^2d}{Bm}+GDd\omega.
    \end{align*}
\end{restatable}



% \begin{proof}
% We reindex the iterates by $y_{i,(t-1)m/K+j}=x_{j}^{Z_{i,t}}$ when $j\neq m/K$, ane define $y_{i,tm/K}=x_{t}$ for $0\le t\le T$.

% Then the average iterate $\bar{x}=\frac{K}{mTB}\sum_{i\in[B],j\in[m/K],t\in[T]}x_j^{Z_{0,t}}=\frac{K}{mTB}\sum_{i\in[B],j\in[Tm/K]}y_{i,j}$.

% To prove the statement, without loss of generality, it suffices to show
% \begin{align*}
%     \E[F(\frac{K}{Tm}\sum_{j\in[Tm/K]}y_{1,j})-F(x)]\lesssim (\beta+\frac{1}{\eta})\frac{\E[\|x_0-x\|^2]}{Tm/K}+\frac{\eta(TG^2d m/K^2) }{Tm/K}.
% \end{align*}

% Let $g_j$ be the gradient estimate, that is
% \begin{align*}
%     y_{1,j}=\Pi_\calX(y_{1,j-1}-\eta g_j).
% \end{align*}

% By Lemma~\ref{lm:sgd_smooth}, 
% it suffices to bound $\sum_{j\in[Tm/K]}\E\|g_j-\nabla F(y_{1,j})\|_2^2\le \eta TG^2dm/K^2$.

% For any $j\mod (m/K)\neq 0$, we know $\E g_j=\nabla F(y_{1,j})$ and $\E\|g_j-\nabla F(1,j)\|^2_2\le G^2/K$.
% For the other case when $j\mod (m/K)\equiv 0$, define $\Tilde{g}_j$ be the gradient estimator for which 
% \begin{align*}
%     x_{m/K}^{Z_{1,jK/m}}=\Pi_{\calX}(x_{m/K-1}^{Z_{1,jK/m}}-\eta \Tilde{g}_j).
% \end{align*}

% We have
% \begin{align*}
%     \E\|g_j- \nabla F(y_{1,j})\|_2^2\le & 2\E\|g_j-\Tilde{g}_j\|_2^2+2\E\|\Tilde{g}_j-\nabla F(y_{1,j})\|_2^2.
% \end{align*}
% Similarly, we can bound $\E\|\Tilde{g}_j-\nabla F(y_{1,j})\|_2^2\le G^2/K$ by the assumptions on i.i.d. and Lipschitz.
% It remains to bound $\E\|g_j-\tg_j\|_2^2$.

% By Lemma~\ref{lm:batched_tech_core} and Lemma~\ref{lm:prop_geo_median}, we know that 
% $
% \|y_{1,j}-x_{m/K}^{Z_{1,jK/m}}\|_\infty\lesssim \eta G\sqrt{m\log(dm)/K^2}$,
% and hence we can bound 
% $\E\|g_j-\tg_j\|_2^2\le\|y_{1,j}-x_{m/K}^{Z_{1,jK/m}}\|_2^2/\eta^2+G^2\lesssim G^2md\log(dm)/K^2$.
% This completes the proof.
% \Daogao{We need a new lemma to handle the shift for the utility when $K$ is large...}
% \end{proof}

Now we apply the localization framework.
We set $\omega=1/(nmd)^3$ to make the term depending on it negligible.
The proof of the following lemma mostly follows from \cite{FKT20}.

\begin{restatable}[Localization]{lem}{Localization}
    \label{lm:localization}
Under Assumption~\ref{assum:lispchitz_smooth} and Assumption~\ref{assump:dia_dominant}, suppose $\beta\le\frac{G}{D}(\frac{\sqrt{n}\epsilon}{\sqrt{m}\log(nmd/\delta})$, $n\ge \log^2(nd/\delta)/\epsilon, \epsilon\le O(1)$ and $ m\le n^{O(\log\log n)}$.
Set $\eta\le\frac{D}{G}\cdot \min\{ \frac{B\sqrt{m}}{\sqrt{n}} ,  \frac{\sqrt{m}\epsilon}{\sqrt{d\log(1/\delta)\log( nmd)}}\}$, $B=100\log(mnd/\delta)/\epsilon$, $\tau=O(G\log(nmd)/\sqrt{m})$ and $\upsilon=0.9B+\frac{2\log(T/\delta)}{\epsilon}$.
If the dataset is drawn i.i.d. from the distribution $\calP$,
the final output $x_S$ for Algorithm~\ref{alg:loacalizatioin} satisfies
\begin{align*}
    \E[F(x_S)-F(x^*)]\le \Tilde{O}\Big(GD\Big(\frac{d}{\sqrt{mn}}+\frac{d^{3/2}}{n\epsilon^2\sqrt{m}}\Big)\Big).
\end{align*}
\end{restatable}

\noindent\textbf{ Main Result:}
Theorem~\ref{thm:main_result} directly follows from Lemma~\ref{lm:localization} and Lemma~\ref{lm:privacy_guarantee}.


% \begin{lemma}[Localization]
% \label{lm:localization}
% Under Assumption~\ref{assum:lispchitz_smooth} and Assumption~\ref{assump:dia_dominant},
% the final output $x_k$ for Algorithm~\ref{alg:loacalizatioin} satisfies that
% \begin{align*}
%     \E[F(x_k)-F(x^*)]\le \Tilde{O}(\frac{\sqrt{d}}{\sqrt{mn\epsilon}}+\frac{d}{n\epsilon\sqrt{m}}).
% \end{align*}
% \end{lemma}

% \begin{proof}
% We need $\eta\le\min\{ \frac{K}{\epsilon\sqrt{nmd}} ,   \frac{K\epsilon}{d\sqrt{m}}\}$.
% Let $\bx_0=x^*$ and $\zeta_0=x_0-x^*$.
% By the assumption, we know $\|\zeta_0\|_2\le D\sqrt{d}$.
% Recall that by definition $\eta\le \frac{D}{G}\cdot\frac{K\epsilon}{d\sqrt{m}}$, for all $t\ge 0$,
% \begin{align*}
%     \E[\|\zeta_t\|_2^2]=d\sigma_t^2=\frac{G^2d^2m}{K^2\epsilon^2}\cdot\frac{D^2K^2\epsilon^2}{mdG^2(\log m)^{2t}}\le (\frac{D}{\log^{-t} m})^2.
% \end{align*}
% Then by Lemma~\ref{lm:dpsgd_utility}, we have
% \begin{align*}
%     \E[F(x_k)]-F(x^*)=&\sum_{t=1}^{k}\E[F(\bx_{t}-\bx_{t-1})]+\E[F(x_k)-F(\bx_k)]\\
%     \le & \sum_{t=1}^{k}(\frac{\E[\|\zeta_{i-1}\|_2^2]}{\eta_i(T_im/K)}+\frac{\eta_iG^2d}{2K})+G\E[\|\zeta_k\|_2]\\
%     \le& \sum_{i=1}^{k}(\frac{\log m}{2})^{-i}(\frac{D^2}{\eta nm\epsilon/K}+\frac{\eta G^2d}{2K})+\frac{GD}{(\log m)^{\log n}}\\
%     \lesssim &   GD(\frac{\sqrt{d}}{\sqrt{nm}}+\frac{d}{n\sqrt{m}\epsilon}).
% \end{align*}
% \end{proof}








\subsection{Regularized Gradient Descent}
The auxiliary loss in Eq.~\eqref{eq:visual_loss} preserves the visual representation at a distribution level via the feature alignment auxiliary loss in Eq.~\eqref{eq:visual_loss}. 
However, the information bottleneck framework indicates that the gradient component compressing $I(X^v; Z)$ (\emph{i.e.}, $\nabla_\theta I(X^v; Z)$), 
can harm visual preservation by reducing the effective rank of the features \cite{achille2018information,lee2021compressive}.

To address this compression-induced drift, we incorporate an orthogonal gradient as a regularize. 
Motivated by multi-task orthogonal gradient optimization \cite{yu2020gradient, zhu2022gradient, dong2022gdod}, 
we leverage the gradient $\Bar{g}_\phi$ from the pre-trained model $\mu_\phi$, which reflects the accumulated visual drift and approximates a global orthogonal learning effect in the downstream task. 
We then project the current model’s gradient onto this direction:
\begin{equation}\label{eq:gd}
    \Tilde{g}_\theta = \frac{\Bar{g}_\theta^\top \Bar{g}_\phi}{\|\Bar{g}_\phi\|^2}\cdot \Bar{g}_\phi.
\end{equation}

In addition, to prevent discrepancies between the regularization and task gradients, we include the feature alignment auxiliary loss (Eq.~\eqref{eq:visual_loss}) in the overall objective. The final parameter update is:
\begin{equation}\label{eq:opt-gd}
    \pi_\theta \leftarrow \pi_\theta - \nabla_\theta\mathcal{L}_{vl}(\theta) - \nabla_\theta\mathcal{L}_v(\theta) - \Tilde{g}_\theta.
\end{equation}

\subsection{Enabling Parameter-efficient Fine-tuning of MDGD via Gradient Masking}
Parameter-efficient fine-tuning (PEFT) methods, such as adapters \cite{houlsby2019parameter} and LoRA \cite{hu2021lora}, aim to reduce the computational cost and memory usage when fine-tuning models on downstream tasks under practical constraints \cite{han2024parameter}. 
However, due to the requirement of directly estimating gradient directions on the pre-trained model parameters, MDGD cannot be directly applied to these PEFT methods, which introduce additional model parameters whose gradients are separate from the original model weights. 

To address this challenge, we propose a variant, MDGD-GM, by formulating the gradient regularization term in Eq.~\eqref{eq:gd} as gradient masking that selects model weights with efficient gradient directions. Specifically, we define the gradient mask as
\begin{equation}\label{eq:masking}
    M_{\Tilde{g}_\theta} = \mathbf{1}\left\{\frac{\Bar{g}_\theta^\top \Bar{g}_\phi}{\|\Bar{g}_\phi\| \|\Bar{g}_\theta\|} \geq T_\alpha \right\},
\end{equation}
where $T_\alpha$ is determined by a percentile $\alpha$ of trainable parameters with the highest similarity scores between $\Bar{g}_\theta$ and $\Bar{g}_\phi$. Consequently, the optimization in Eq.~\eqref{eq:opt-gd} is reformulated as
\begin{equation}\label{eq:opt-mask}
    \pi_\theta \leftarrow \pi_\theta - M_{\Tilde{g}_\theta} \cdot \left(\nabla_\theta\mathcal{L}_{vl}(\theta) + \nabla_\theta\mathcal{L}_v(\theta)\right).
\end{equation}
We summarize and illustrate the optimization process of MDGD and MDGD-GM in Algorithm~\ref{alg}.
