\section{Conclusion}
In this work, we addressed the challenge of visual forgetting in MLLMs during instruction tuning by introducing a novel modality-decoupled gradient descent (MDGD) approach. MDGD disentangles the gradient updates for visual representation learning from task-specific alignment, thereby preserving the effective rank of pre-trained visual features and mitigating the over-compression effects highlighted by the information bottleneck perspective. This decoupling enables MLLMs to retain rich visual knowledge while adapting robustly to new downstream tasks. Furthermore, our gradient masking variant, MDGD-GM, enhances memory efficiency and optimizes parameter usage, making fine-tuning both practical and scalable. Extensive experiments across various downstream tasks and backbone models demonstrate that MDGD not only effectively prevents visual forgetting but also outperforms existing strategies in achieving balanced multimodal representation learning and task adaptation. Our findings underscore the importance of preserving visual representations during instruction-tuning and offer a viable solution for efficient and effective multimodal learning in real-world scenarios.

