


\begin{abstract}
Recent MLLMs have shown emerging visual understanding and reasoning abilities after being pre-trained on large-scale multimodal datasets. Unlike pre-training, where MLLMs receive rich visual-text alignment, instruction-tuning is often text-driven with weaker visual supervision, leading to the degradation of pre-trained visual understanding and causing visual forgetting. Existing approaches, such as direct fine-tuning and continual learning methods, fail to explicitly address this issue, often compressing visual representations and prioritizing task alignment over visual retention, which further worsens visual forgetting. To overcome this limitation, we introduce a novel perspective leveraging effective rank to quantify the degradation of visual representation richness, interpreting this degradation through the information bottleneck principle as excessive compression that leads to the degradation of crucial pre-trained visual knowledge. Building on this view, we propose a modality-decoupled gradient descent (\textbf{MDGD}) method that regulates gradient updates to maintain the effective rank of visual representations while mitigating the over-compression effects described by the information bottleneck. By explicitly disentangling the optimization of visual understanding from task-specific alignment, MDGD preserves pre-trained visual knowledge while enabling efficient task adaptation. To enable lightweight instruction-tuning, we further develop a memory-efficient fine-tuning approach using gradient masking, which selectively updates a subset of model parameters to enable parameter-efficient fine-tuning (PEFT), reducing computational overhead while preserving rich visual representations. Extensive experiments across various downstream tasks and backbone MLLMs demonstrate that MDGD effectively mitigates visual forgetting from pre-trained tasks while enabling strong adaptation to new tasks.
\end{abstract}

