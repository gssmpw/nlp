\documentclass[12pt]{article}
\usepackage{authblk}
\usepackage{setspace}  % Package for line spacing
\usepackage{graphicx} % Required for inserting images
\usepackage{geometry}
\usepackage{xcolor}
\geometry{margin={70pt,70pt}}
\usepackage{comment}
\usepackage{fixltx2e}
\renewcommand{\baselinestretch}{1.5}

\usepackage{multirow}
\usepackage{graphicx} 
\usepackage{soul}

\title{Unlocking the Potential of Generative AI through Neuro-Symbolic Architectures – Benefits and Limitations}
\author[1]{Oualid Bougzime}
\author[2]{Samir Jabbar}
\author[2]{Christophe Cruz}
\author[1,3]{Frédéric Demoly}
\affil[1]{ICB UMR 6303 CNRS, Université Marie et Louis Pasteur, UTBM, 90010 Belfort Cedex, France}
\affil[2]{ICB UMR 6303 CNRS, Université Bourgogne Europe, 21078 Dijon, France}
\affil[3]{Institut universitaire de France (IUF), Paris, France}
\date{}
\renewcommand\Affilfont{\itshape\small}


%\date{September 2024}
\usepackage{amsmath}
\usepackage{tikz}
\usepackage{adjustbox}
\usepackage{caption} % Pour gérer les légendes
\usetikzlibrary{shapes.geometric, arrows, positioning}

% Define custom colors
\definecolor{neuralColor}{HTML}{1f77b4}
\definecolor{symbolicColor}{HTML}{ff7f0e}
\definecolor{rootColor}{HTML}{2ca02c}

% TikZ block styles
\tikzset{
  block/.style={rectangle, draw, thick, rounded corners, minimum height=0.5cm, minimum width=2cm, text centered, font=\footnotesize},
  neural/.style={block, fill=neuralColor!30},
  symbolic/.style={block, fill=symbolicColor!30},
  root/.style={block, fill=rootColor!30},
  arrow/.style={thick,->,>=stealth}
}


\usepackage{url} % Pour utiliser la commande \url{}
\usepackage{breakurl}
\usepackage{hyperref} % Pour rendre les liens cliquables avec \href{}
\begin{document}
\maketitle
\doublespacing
\section*{Abstract}
Neuro-symbolic artificial intelligence (NSAI) represents a transformative approach in artificial intelligence (AI) by combining deep learning's ability to handle large-scale and unstructured data with the structured reasoning of symbolic methods. By leveraging their complementary strengths, NSAI enhances generalization, reasoning, and scalability while addressing key challenges such as transparency and data efficiency. This paper systematically studies diverse NSAI architectures, highlighting their unique approaches to integrating neural and symbolic components. It examines the alignment of  contemporary AI techniques such as retrieval-augmented generation, graph neural networks, reinforcement learning, and multi-agent systems with NSAI paradigms. This study then evaluates these architectures against comprehensive set of criteria, including generalization, reasoning capabilities, transferability, and interpretability, therefore providing a comparative analysis of their respective  strengths and limitations. Notably, the Neuro → Symbolic ← Neuro model consistently outperforms its counterparts across all evaluation metrics. This result aligns with state-of-the-art research that highlight the efficacy of such architectures in harnessing advanced technologies like multi-agent systems. 


\vspace*{0.5cm}


\noindent \underline{Keywords}: Neuro-symbolic Artificial Intelligence, Neural Network, Symbolic AI, Generative AI, Retrieval-Augmented Generation (RAG), Reinforcement Learning (RL), Natural Language Processing (NLP), Explainable AI (XAI), Benchmark

\section{Introduction}
Neuro-symbolic artificial intelligence (NSAI) is fundamentally defined as the combination of deep learning and symbolic reasoning \cite{garcez2023neurosymbolic}. This hybrid approach aims to overcome the limitations of both symbolic and neural artificial intelligence (AI) systems while harnessing their respective strengths. Symbolic AI excels in reasoning and interpretability, whereas neural AI thrives in learning from vast amounts of data. By merging these paradigms, NSAI aspires to embody two  fundamental aspects of intelligent cognitive behavior: the ability to learn from experience and the capacity to reason based on acquired knowledge \cite{garcez2023neurosymbolic, valiant2003three}.

\vspace*{0.5cm}

The importance of NSAI has been increasingly recognized in recent years, especially after the 2019 Montreal AI Debate between Gary Marcus and Yoshua Bengio. This debate highlighted two contrasting perspectives on the future of AI: Marcus argued that “expecting a monolithic architecture to handle abstraction and reasoning is unrealistic,” emphasizing the limitations of current AI systems, while Bengio maintained that “sequential reasoning can be performed while staying in a deep learning framework” \cite{bengio2019ai}. This discussion brought attention to the strengths and weaknesses of neural and symbolic approaches, catalyzing a surge of interest in hybrid solutions. Bengio’s subsequent remarks at IJCAI 2021 underscored the importance of addressing out-of-distribution (OOD) generalization, stating that “we need a new learning theory” to tackle this critical challenge \cite{bengio2022system}. This aligns with the broader consensus within the AI community that combining neural and symbolic paradigms is essential to developing more robust and adaptable systems. Drawing on concepts like Daniel Kahneman’s dual-process theory of reasoning, which compares fast, intuitive thinking (System 1) to deliberate, logical thought (System 2), NSAI seeks to bridge the gap between learning from data and reasoning with structured knowledge \cite{marcus2019rebooting}. Despite ongoing debates about the optimal architecture for integrating these two paradigms, the 2019 Montreal AI Debate has played a pivotal role in shaping the trajectory of research in this promising field \cite{marcus2018deep, liu2022neural, zhang2021neural, lamb2020graph, von2021informed, belle2020symbolic}.

\vspace*{0.5cm}

NSAI offers a promising avenue for addressing limitations of purely symbolic or neural systems. For instance, while neural networks (NNs) often struggle with interpretability, symbolic AI systems are rigid and require extensive domain knowledge. By combining the adaptability of neural models with the explicit reasoning capabilities of symbolic methods, NSAI systems aim to provide enhanced generalization, interpretability, and robustness. These characteristics make NSAI particularly well-suited for solving complex, real-world problems where adaptability and transparency are critical \cite{hamilton2024neuro}. Several NSAI architectures have been proposed to integrate these paradigms effectively. Examples include Symbolic Neuro Symbolic systems, Symbolic[Neuro], Neuro[Symbolic], Neuro | Symbolic coroutines, Neuro\textsubscript{Symbolic}, and others \cite{kautz2022third}. Each architecture offers unique advantages but also poses specific challenges in terms of scalability, interpretability, and adaptability. A systematic evaluation of these architectures is imperative to understand their potential and limitations, guiding future research in this rapidly evolving field.

\vspace*{0.5cm}

Generative AI has witnessed remarkable advancements, encompassing a diverse range of technologies that address various challenges in data processing, reasoning, and decision-making. These advancements can be categorized into several major branches of AI. Natural language processing (NLP) \cite{vaswani2017attention} includes technologies such as retrieval-augmented generation (RAG) \cite{lewis2020retrieval}, sequence-to-sequence models \cite{sutskever2014sequence}, semantic parsing \cite{jiang2024survey}, named entity recognition (NER) \cite{marrero2013named}, and relation extraction \cite{zhao2024comprehensive}, which focus on understanding and generating human language. Reinforcement learning techniques, like reinforcement learning with human feedback (RLHF) \cite{christiano2017deep}, enable systems to learn optimal actions through interaction with their environment. Advanced NNs include innovations such as graph neural networks (GNNs) \cite{zhou2020graph} and generative adversarial networks (GANs) \cite{goodfellow2014generative}, which excel in handling structured data and generating realistic data samples, respectively. Multi-agent systems \cite{guo2024large, maldonado2024multi} explore the coordination and decision-making among multiple intelligent agents. Recent advances leverage mixture of experts (MoE) architectures to enhance scalability and specialization in collaborative frameworks. In MoE-based multi-agent systems, each expert operates as an autonomous agent, specializing in distinct sub-tasks or data domains, while a dynamic gating mechanism orchestrates their contributions \cite{he2024mixturemillionexperts, lo2024closerlookmixtureofexpertslarge}. Transfer Learning \cite{alyafeai2020survey}, including pre-training \cite{devlin2018bert}, fine-tuning \cite{howard2018universal}, and few-shot learning \cite{parnami2022learning}, allows AI models to adapt knowledge from one task to another efficiently. Explainable AI (XAI) \cite{arrieta2020explainable} focuses on making AI systems transparent and interpretable, while efficient learning techniques, such as model distillation \cite{hinton2015distilling}, aim to optimize resource usage. Reasoning and inference methods like chain-of-thought (CoT) \cite{wei2022chain} reasoning and link prediction enhance logical decision-making capabilities. Lastly, continuous learning \cite{chen2018lifelong} paradigms ensure adaptability over time. Together, these technologies form a comprehensive toolkit for tackling the increasingly complex demands of generative AI applications.

\vspace*{0.5cm}

The classification of generative AI technologies within the NSAI framework is crucial for several reasons. Firstly, it provides a structured approach to understanding how these diverse technologies relate to and enhance NSAI capabilities. By mapping these techniques to specific NSAI architectures, researchers and practitioners can better grasp their potential applications and limitations. This classification also facilitates the identification of synergies between different AI approaches, potentially leading to more robust and versatile hybrid systems. Furthermore, it aids in decision-making processes when selecting appropriate technologies for specific tasks, considering factors like interpretability, reasoning capabilities, and generalization. As AI continues to evolve, this systematic categorization becomes increasingly valuable for bridging the gap between cutting-edge research and practical implementation, ultimately driving the field towards more integrated and powerful AI solutions.

\vspace*{0.5cm}

Therefore, this research aims to explores the alignment of generative AI technologies with the core catergories of NASAI and examines the insights this classification provides regarding their strenghts and limitations. The proposed methodology is threefold: (i) to define and analyze existing NSAI architectures, (ii) to classify generative AI technologies within the NSAI framework to provide a unified perspective on their integration, and (iii) to develop a systematic framework for assessing NSAI architectures across various criteria.

\section{Neuro-Symbolic AI: Combining Learning and Reasoning to Overcome AI's Limitations}

\noindent NNs have been exemplary in handling unstructured forms of data, e.g., images, sounds, and textual data. The capacity of these networks to acquire sophisticated patterns and representations from voluminous datasets has provided major breakthroughs in a series of disciplines, from computer vision, speech recognition, to NLP \cite{kenton2019bert,vaswani2017attention}. One of the major benefits of NNs is that they learn and become better from raw data without requiring pre-coded rules or expert knowledge. This makes them highly scalable and efficient to utilize in applications with large raw data. However, despite these benefits, NNs also have some very well-documented disadvantages. One of the major ones of these might be that they are not transparent. Indeed, neural models pose interpretability challenges, making it difficult to understand the process by which they arrive at specific decisions or predictions. Such opacity causes problems for critical applications where explanation is necessary, such as in healthcare, finance, legal frameworks, and engineering. Additionally, NNs have a high requirement for data, requiring substantial amounts of labeled training data in order to operate effectively. This reliance on large data makes them ineffective when applied to data-scarce or data-costly environments. Neural models also struggle with reasoning and generalizing beyond their training data, which makes their performance less impressive when it comes to tasks in logical inference or commonsense reasoning. Specifically, tasks including understanding causality, sequential problem-solving, and decision-making relying on outside world knowledge.

\vspace*{0.5cm}

Symbolic AI is better at handling areas that are weaker for NNs. Symbolic systems function on explicit rules and structured representations, which enables them to achieve reasoning tasks related to complicated issues, such as mathematical proofs, planning, and expert systems. Symbolic AI is most important because it is transparent. Since symbolic methods are grounded in known rules and logical formalisms, decision-making processes are easy to interpret and explain. However, symbolic AI systems have some drawbacks. One of the biggest ones is that they are rigid and difficult to respond to new circumstances. They require rules to be manually defined and require structured input data, leading them difficult to apply to real-world situations where data might contain noise, incompleteness, or unstructured form. They are also susceptible to combinatorial explosions in handling big data or hard reasoning problems, which significantly slows down their performance at scale. Symbolic systems are also not well suited for perception tasks like image or speech recognition since they are unable to draw knowledge from raw data alone.

\vspace*{0.5cm}

While traditional NNs are strong at recognizing patterns in collections of data but falter when presented with new situations, symbolic reasoning provides a rational foundation for decision-making but is limited in the manner in which it can learn knowledge from new information and adapt in a dynamic process. The combination of these two approaches in NSAI effectively minimizes these limitations, producing a more flexible, explainable, and effective AI system.
Another distinguishing feature of NSAI is that it is able to generalize outside its training set. Traditional AI systems are prone to fail in novel situations; however, NSAI, because of its combination of learning and logical reasoning, works better in such cases. Such a feature is critical for real-world applications such as autonomous transport and medicine, where systems need to perform well in uncontrolled environments. Apart from that, in an interdisciplinary engineering context such as 4D printing, which brings together materials science, additive manufacturing, and engineering, NSAI holds significant promise for improving both the interpretability and reliability of design decisions on the actuation and mechanical performance, and printability. Although these advantages seem promising, they remain hypotheses requiring more extensive validation and industrial-scale testing. Ongoing research must demonstrate, through empirical studies and real-world implementations, how NSAI can reliably accelerate the discovery of smart materials and structures \cite{bougzime2025nsai4d}. The second key benefit point of NSAI is that it has a reduced need for big data sets. Traditional AI systems usually require a tremendous amount of data to operate, which might be very time- and resource-consuming. NSAI, however, is able to do better with a much smaller set of data required, due to its symbolic reasoning ability. This makes it a more sustainable and viable option, especially for small organizations or new research areas with limited resources. Along with the aforementioned data efficiency, NSAI models also have the exceptional transferability, i.e., their capacity for using knowledge learned from one task and applying it in another with less need for retraining. Such a property is highly desirable in situations where there is little data related to a new task.

\vspace*{0.5cm}

%The last interesting characteristic of NSAI lies in its improved reasoning capacity. Although, NNs are superior at recognizing correlations and concealed patterns in data sets, they can be deficient where there is a requirement for logical reasoning or problem-solving. NSAI fills this gap by being able to handle more complex problems effectively, for example, diagnosing disease, examining court cases, or generating scientific breakthroughs. NSAI is also extremely scalable and flexible, which allows it to handle increasingly complex problems in various fields. Its capacity to blend unstructured data processing and structured logic allows it to keep evolving and developing as AI technologies evolve, positioning NSAI at the center of future AI-based systems.

\section{Neuro-Symbolic AI Architectures}
This section provides an overview of various NSAI architectures, offering insights into their design principles, integration strategies, and unique capabilities. While Kautz’s classification \cite{kautz2022third} serves as a foundational framework, we extend it by incorporating additional architectural perspectives to capture the evolving landscape of NSAI systems. These approaches range from symbolic systems augmented by neural modules for specialized tasks to deeply integrated models where explicit reasoning engines operate within neural frameworks. This expanded categorization highlights the diversity of design strategies and the broad applicability of NSAI techniques, emphasizing their potential for more interpretable, robust, and data-efficient AI solutions.

\subsection{Sequential}
As part of the sequential NSAI, the \textit{Symbolic} $\to$ \textit{Neuro} $\to$ \textit{Symbolic} architecture involves systems where both the input and output are symbolic, with a NN acting as a mediator for processing (Figure~\ref{fig:sequential}a). Symbolic input, such as logical expressions or structured data, is first mapped into a continuous vector space through an encoding process. The NN operates on this encoded representation, enabling it to learn complex transformations or patterns that are difficult to model symbolically. Once the processing is complete, the resulting vector is decoded back into symbolic form, ensuring that the final output aligns with the structure and semantics of the input domain. This framework is especially useful for tasks that require leveraging the generalization capabilities of NNs while preserving symbolic interpretability \cite{lample2019deep, dimassi2021ontology}. A formulation of this architecture is presented below:

\begin{equation}
y = f_\text{neural}(x)   
\end{equation}

\noindent where $x$ is the symbolic input, $f_\text{neural}(x)$ represents the NN that processes the input, and $y$ is the symbolic output. 

\begin{figure}[h!]
    \centering
    \includegraphics[width=0.6\linewidth]{Figure1_sequential.pdf}
    \caption{Sequential architecture: (a) Principle and (b) application to knowledge graph construction.}
    \label{fig:sequential}
\end{figure}


This architecture can be used in a semantic parsing task, where the input is a sequence of symbolic tokens (e.g., words). Here, each token is mapped to a continuous embedding via word2vec, GloVe, or a similar method \cite{mikolov2013efficient, pennington2014glove}. The NN then processes these embeddings to learn compositional patterns or transformations. From this, the network’s output layer decodes the processed information back into a structured logical form (such as knowledge-graph triples), as illustrated in Figure~\ref{fig:sequential}b.



%\noindent Figure~\ref{fig:sequential} \hl{illustrates a sequential architecture for knowledge graph construction. In this process, an embedding model transforms a document into numerical vectors, which are subsequently projected into a vector space. These representations are then leveraged to generate a knowledge graph.}

\subsection{Nested}
The nested NSAI category is composed of two different architectures. The first -- \textit{Symbolic[Neuro]} -- places a NN as a subcomponent within a predominantly symbolic system (Figure~\ref{fig:nested}a). Here, the NN is used to perform tasks that require statistical pattern recognition, such as extracting features from raw data or making probabilistic inferences, which are then utilized by the symbolic system. The symbolic framework orchestrates the overall reasoning process, incorporating the neural outputs as intermediate results \cite{silver2016mastering}. This architecture can formally defined as follows:

\begin{equation}
y = g_\text{symbolic}(x, f_\text{neural}(z))  
\end{equation}

\noindent where $x$ represents the symbolic context, $z$ is the input passed from the symbolic reasoner to the NN, $f_\text{neural}(z)$ expresses the neural model processing the input, and $g_\text{symbolic}$  the symbolic reasoning engine that integrates neural outputs. A well-known instance of this architecture is AlphaGo \cite{silver2016mastering}, where a symbolic Monte-Carlo tree search orchestrates high-level decision-making, while a NN evaluates board states, providing a data-driven heuristic to guide the symbolic search process \cite{coulom2006efficient} (Figure~\ref{fig:nested}b). Similarly, in a medical diagnosis scenario, a rule-based engine oversees the core diagnostic process by applying expert guidelines to patient history, symptoms, and lab results. At the same time, a NN interprets unstructured radiological images, delivering key indicators such as tumor likelihood. The symbolic system then integrates these indicators into its final decision, combining transparent and rule-driven logic with robust pattern recognition.

The second architecture -- \textit{Neuro[Symbolic]} --  integrates a symbolic reasoning engine as a component within a neural system, allowing the network to incorporate explicit symbolic rules or relationships during its operation (Figure~\ref{fig:nested}c). The symbolic engine provides structured reasoning capabilities, such as rule-based inference or logic, which complement the NN’s ability to generalize from data. By embedding symbolic reasoning within the neural framework, the system gains interpretability and structured decision-making while retaining the flexibility and scalability of neural computation. This integration is particularly effective for tasks that require reasoning under constraints or adherence to predefined logical frameworks \cite{heule2016solving, madan2021fast}. This configuration can be described as follows:

\begin{equation}
 y = f_\text{neural}(x, g_\text{symbolic}(z))   
\end{equation}

\noindent where $x$ represents the input data to the neural system, $z$ is the input passed from the NN to the symbolic reasoner, $g_\text{symbolic}$ is the symbolic reasoning function, and $f_\text{neural}$ denotes the NN processing the combined inputs.


This architecture is currently applied in automated warehouse, where a robot navigates dynamically changing aisles. During normal operation, it relies on a neural policy to select routes based on learned patterns. When it encounters an unexpected obstacle, it offloads route computation to a symbolic solver (e.g., a pathfinding or constraint-satisfaction algorithm), which returns an alternative path. The solver’s output is then integrated back into the neural policy, and the robot resumes its usual pattern-based navigation. Over time, the robot also learns to identify which challenges call for the symbolic solver, effectively blending fast pattern recognition with precise combinatorial planning.

\begin{figure}[!h]
    \centering
    \includegraphics[width=0.75\linewidth]{Figure2_nested.pdf}
    \caption{Nested architectures: (a) \textit{Symbolic[Neuro]} principle and (b) its application to tree Search, (c) \textit{Neuro[Symbolic]} principle and (d) its application to maze-solving.}
    \label{fig:nested}
\end{figure}

\noindent Figure~\ref{fig:nested}d illustrates this framework, a symbolic reasoning engine processes structured data, such as a maze, to generate a solution path. A NN  encodes the problem into a latent representation and decodes it into a symbolic sequence of actions (e.g., forward, turn left, turn right).

\subsection{Cooperative}
As a cooperative framework, \textit{Neuro $|$ Symbolic} uses neural and symbolic components as interconnected coroutines, collaborating iteratively to solve a task (Figure~\ref{fig:cooperative}a). NNs process unstructured data, such as images or text, and convert it into symbolic representations that are easier to reason about. The symbolic reasoning component then evaluates and refines these representations, providing structured feedback to guide the NN’s updates. This feedback loop continues over multiple iterations until the system converges on a solution that meets predefined symbolic constraints or criteria. By combining the strengths of NNs for generalization and symbolic reasoning for interpretability, this approach achieves robust and adaptive problem-solving \cite{mao2019neuro}. This architecture can be described as follows:

\begin{equation}
z^{(t+1)} = f_\text{neural}(x, y^{(t)}), \quad y^{(t+1)} = g_\text{symbolic}(z^{(t+1)}), \quad \forall t \in \{0, 1, \dots, n\}    
\end{equation}

\noindent where $x$ represents non-symbolic data input, $z^{(t)}$ is the intermediate symbolic representation at iteration $t$, $y^{(t)}$ is the symbolic reasoning output at iteration $t$, $f_\text{neural}(x, y^{(t)})$ expresses the NN that processes the input $x$ and feedback from the symbolic output $y^{(t)}$, $g_\text{symbolic}(z^{(t+1)})$ is the symbolic reasoning engine that updates $y^{(t+1)}$ based on the neural output $z^{(t+1)}$, and $n$ is the maximum number of iterations or a convergence threshold. The hybrid reasoning halts when the outputs $y^{(t)}$ converge (e.g., $|y^{(t+1)} - y^{(t)}| < \epsilon$)), where $\epsilon$ is a small threshold denoting minimal change between successive outputs, or when the maximum iterations $n$ is reached.

For instance, this architecture can applied in autonomous driving systems, where a NN processes real-time images from vehicle cameras to detect and classify traffic signs. It identifies shapes, colors, and patterns to suggest potential signs, such as speed limits or stop signs. A symbolic reasoning engine then evaluates these detections based on contextual rules—like verifying if a detected speed limit sign matches the road type or if a stop sign appears in a logical position (e.g., near intersections). If inconsistencies are detected, such as a stop sign identified in the middle of a highway, the symbolic system flags the issue and prompts the neural network to re-evaluate the scene. This iterative feedback loop continues until the system reaches consistent, high-confidence decisions, ensuring robust and reliable traffic sign recognition, even in challenging conditions like poor lighting or partial occlusions (Figure~\ref{fig:cooperative}b).

\begin{figure}[!h]
    \centering
    \includegraphics[width=0.75\linewidth]{Figure4_cooperative.pdf}
    \caption{Cooperative architecture: (a) principle and (b) application to visual reasoning.}
    \label{fig:cooperative}
\end{figure}

%\noindent Figure~\ref{fig:cooperative} \hl{illustrates a cooperative neuro-symbolic architecture where an image is processed to extract visual embeddings, which are used to generate predictions. These predictions are then compared against a symbolic reasoning engine that holds ground-truth knowledge. In case of discrepancies, the reasoning engine provides feedback to retrain the neural model, ensuring continuous learning and improved accuracy.}

\subsection{Compiled}
As part of the compiled NSAI, \textit{Neuro\textsubscript{Symbolic\textsubscript{Loss}}} uses symbolic reasoning into the loss function of a NN (Figure~\ref{fig:compiled}a). The loss function is typically used to measure the discrepancy between the model's predictions and the true outputs. By incorporating symbolic rules or constraints, the network’s training process not only minimizes prediction error but also ensures that the output aligns with symbolic logic or predefined relational structures. This allows the model to learn not just from data but also from symbolic reasoning, helping to guide its learning process toward solutions that are both accurate and consistent with symbolic principles.

\begin{equation}
 \mathcal{L} = \mathcal{L}_\text{task}(y, y_\text{target}) + \lambda \cdot \mathcal{L}_\text{symbolic}(y)   
\end{equation}


\noindent where $y$ is the model prediction,$y_\text{target}$ represents the ground truth labels, $\mathcal{L}_\text{task}$ is the task-specific loss (e.g., cross-entropy), $\mathcal{L}_\text{symbolic}$ is the penalization for violating symbolic rules, $\lambda$ the Weight balancing the two loss components, and $\mathcal{L}$ the final loss, combining both the task-specific loss and the symbolic constraint penalty to guide model optimization. This architecture is typically useful in the field of 4D printing, where structures need to be optimized at the material level to achieve a target shape. In such a case, a NN predicts the  material distribution and geometric configuration that allows the structure to adapt under external stimuli. The training process incorporates a physics-informed loss function, where, in addition to minimizing the difference between predicted and desired mechanical behavior, the model is penalized whenever the predicted deformation violates symbolic mechanical constraints, such as equilibrium equations or the stress-strain relationship (Figure~\ref{fig:compiled}b). By embedding these symbolic equations directly into the loss function, the NN learns to generate designs that are not only data-driven but also physically consistent, ensuring that the final 4D-printed structure maintains the desired shape across different operational conditions.


A second compiled NSAI architecture, called \textit{Neuro\textsubscript{Symbolic\textsubscript{Neuro}}}, uses symbolic reasoning at the neuron level by replacing traditional activation functions with mechanisms that incorporate symbolic reasoning (Figure~\ref{fig:compiled}c). Rather than using standard mathematical operations like ReLU or sigmoid, the neuron activation is governed by symbolic rules or logic. This allows the NN to reason symbolically at a more granular level, integrating explicit reasoning steps into the learning process. This fusion of symbolic and neural operations enables more interpretable and constrained decision-making within the network, enhancing its ability to reason in a structured and rule-based manner while retaining the flexibility of neural computations. This architecture can be described as follows:

\begin{equation}
 y = g_\text{symbolic}(x)   
\end{equation}

\noindent where: $x$ represents the pre-activation input, $g_\text{symbolic}(x)$ is the symbolic reasoning-based activation function, and $y$ the final neuron. This architecture can find application in lean approval systems, where neural activations are driven by symbolic financial rules rather than traditional functions. One example is the collateral-based constraint neuron, which dynamically adjusts the risk score based on the value of the pledged collateral. When the collateral’s value falls below a predefined threshold relative to the loan amount, the neuron applies a strict penalty that substantially increases the risk score, effectively preventing the system from underestimating the associated financial risk. This symbolic constraint ensures that, regardless of favorable patterns identified in other data, the model consistently accounts for the critical impact of insufficient collateral, leading to more reliable and regulation-compliant credit decisions (Figure~\ref{fig:compiled}d).

Finally, the last compiled architecture, \textit{Neuro:Symbolic $\to$ Neuro}, uses a  symbolic reasoner to generate labeled data pairs \((x, y)\), where \(y\) is produced by applying symbolic rules or reasoning to the input \(x\) (Figure~\ref{fig:compiled}e). These pairs are then used to train a NN, which learns to map from the symbolic input \(x\) to the corresponding output \(y\). The symbolic reasoner acts as a supervisor, providing high-quality, structured labels that guide the NN’s learning process \cite{riegel2020logical}. This architecture can be governed as follows:

\begin{equation}
 \mathcal{D}_\text{train} = \{(x, g_\text{symbolic}(x)) \mid x \in \mathcal{X}\}   
\end{equation}

\noindent where $\mathcal{D}_\text{train}$ is the training dataset, $x$ denotes the unlabeled data, $g_\text{symbolic}(x)$ represents symbolic rules generating labeled data, and $\mathcal{X}$ the set of all input data (Figure~\ref{fig:compiled}b).

Figure~\ref{fig:compiled}f illustrates this architecture, where a reasoning engine is used to label unlabeled training data, transforming raw inputs into structured $(x,y)$ pairs, where symbolic rules enhance the data quality.

\begin{figure}[!h]
    \centering
    \includegraphics[width=1\linewidth]{Figure6_compiled.pdf}
    \caption{Compiled architectures: (a) \textit{Neuro\textsubscript{Symbolic\textsubscript{Loss}}} principle and (b) application to physics-informed learning; (c) \textit{Neuro\textsubscript{Symbolic\textsubscript{Neuro}}} principle and (d) application of symbolic reasoning in NNs; (e) \textit{Neuro:Symbolic $\rightarrow$ Neuro} principle and (f) application to data Llabeling.}
    \label{fig:compiled}
\end{figure}

%\noindent Figure~\ref{fig:compiled} illustrates various compiled neuro-symbolic architectures that integrate symbolic reasoning into different stages of neural network learning and training. In Neuro:Symbolic $\rightarrow$ Neuro, a reasoning engine is used to label unlabeled training data, transforming raw inputs into structured $(x, y)$ pairs, where symbolic rules enhance the data quality. Conversely, in Neuro\textsubscript{Symbolic\textsubscript{Neuro}}, a neural network is enhanced with symbolic reasoning, allowing it to incorporate logical rules and structured knowledge during inference and training. Finally, in Neuro\textsubscript{Symbolic\textsubscript{Loss}}, a neural network is trained using a physics-informed loss function, incorporating domain-specific symbolic knowledge into the backpropagation process to improve model training.

%\noindent \textbf{Example:} A rule-based system generates synthetic labeled examples for text classification, which are later used to train a neural model.

\subsection{Ensemble}
Another promising architecture, called \textit{Neuro $\to$ Symbolic $\leftarrow$ Neuro} uses multiple interconnected NNs  via a symbolic fibring function, which enables them to collaborate and share information while adhering to symbolic constraints (Figure~\ref{fig:ensemble}a). The symbolic function acts as an intermediary, facilitating communication between the networks by ensuring that their interactions respect predefined symbolic rules or structures. This enables the networks to exchange information in a structured manner, allowing them to jointly solve problems while benefiting from both the statistical learning power of NNs and the logical constraints imposed by the symbolic system \cite{garcez2004fibring}. This architecture can formally defined as follows:

\begin{equation}
 y = g_\text{fibring}(\{f_i\}_{i=1}^n)   
\end{equation}

\noindent where $f_i$ represents the individual NN, $g_\text{fibring}$ is the logic-aware aggregator that enforces symbolic constraints while unifying the outputs of multiple NNs, $n$ the umber of NNs, and $y$ is the combined output of interconnected NNs, produced through the symbolic fibring function $g_\text{fibring}$. For instance in smart cities and urban planning, multiple NNs can be employed, each handle a different urban data stream—such as real-time traffic flow, energy consumption, and air quality measurements. A symbolic fibring function then harmonizes these outputs, enforcing city-level constraints (e.g., ensuring pollution alerts match local environmental regulations, verifying that traffic predictions align with current road network rules). If one network forecasts a surge in vehicle congestion that would push pollution levels beyond acceptable thresholds, the symbolic aggregator identifies the conflict and directs all networks to converge on a coordinated strategy—such as adjusting traffic signals or advising public transport usage. By leveraging each network’s specialized insight within logical urban-planning constraints, the system delivers efficient, consistent decisions across the city’s complex infrastructure.

\begin{figure}[!h]
    \centering
    \includegraphics[width=0.5\linewidth]{Figure5_ensembe.pdf}
    \caption{Ensemble architecture: (a) principle and (b) application to NN collaboration.}
    \label{fig:ensemble}
\end{figure}

Figure~\ref{fig:ensemble}b illustrates this architecture, where two NNs (Neural Network 1 and Neural Network 2) communicate through activation states, which enables dynamic exchange of learned representations.

\section{Leveraging NSAI in AI Technologies}

Generative AI is advancing at a remarkable pace, addressing increasingly complex challenges through the integration of diverse methodologies. A key development is the combination of NNs with symbolic reasoning, resulting in hybrid systems that leverage both strengths. Recent studies have demonstrated the effectiveness of this approach in various applications, including design generation and enhancing instructability in generative models \cite{sheth2024neurosymbolic, jacobson2025integrating}.  This section aims to classify  contemporary AI techniques such as RAG, GNNs,  agent-based AI, and transfer learning within the NSAI framework. This classification clarifies how generative AI aligns with neuro-symbolic approaches, bridging cutting-edge research with established paradigms. It also reveals how generative AI increasingly embodies both neural and symbolic characteristics, moving beyond siloed methods.

Additionally, this classification enhances our understanding of these techniques’ roles in AI’s broader landscape, particularly in addressing challenges like interpretability, reasoning, and generalization. It identifies synergies between methods, fostering robust hybrid models that combine neural learning’s adaptability with symbolic reasoning’s precision. Lastly, it supports informed decision-making, guiding researchers and practitioners in selecting the most suitable AI techniques for specific tasks.

\subsection{Overview of Key AI Technologies}

One of the most significant advancements is RAG, which integrates information retrieval with generative models to perform knowledge-intensive tasks. By combining a retrieval mechanism to extract relevant external data with Seq2Seq models for generation \cite{yin2022seq2seq}, RAG excels in applications such as question answering and knowledge-driven conversational AI~\cite{yang2024rag}. Seq2Seq models themselves, built as encoder-decoder architectures, have been pivotal in machine translation, text summarization, and conversational modeling, providing the foundation for many generative AI systems. An extension of RAG is the GraphRAG approach \cite{edge2024local}, which incorporates graph-based reasoning into the retrieval and generation process. By leveraging knowledge graph (KGq) and ontologies structures to represent relationships between information elements, GraphRAG enhances query-focused summarization and reasoning tasks \cite{chen2020review, antoniou2009web}. This method has demonstrated success in producing coherent and contextually rich summaries by integrating local and global reasoning.

GNNs \cite{mavromatis2024gnn} represent a breakthrough in extending neural architectures to graph-structured data, enabling advanced reasoning over interconnected entities. Their ability to model relationships between entities makes them indispensable for a range of tasks, including link prediction, node classification, and recommendation systems, with notable success in KG reasoning. GNNs have also proven highly effective in named entity recognition (NER) \cite{roy2021recent}, where they can leverage graph representations to capture contextual dependencies and relationships between entities in text. This capability extends to relation extraction \cite{wu2024towards}, where GNNs identify and classify semantic relationships between entities, crucial for building and enhancing KG. 

Advances in agentic AI systems, which leverage Large Language Models (LLMs), have shown significant potential in enabling autonomous decision-making and task execution. These systems are designed to function independently, interacting with environments, coordinating with other agents, and adapting to dynamic situations without human intervention. A notable example is AutoGen \cite{wu2023autogen}, a framework that enables the creation of autonomous agents that can interact with each other to solve tasks and improve through continual interactions. Recent work has further enhanced these systems through MoE architectures, which integrate specialized sub-models (``experts") into multi-agent frameworks to optimize task-specific performance and computational efficiency. For instance, MoE-based coordination allows agents to dynamically activate subsets of experts based on context, enabling scalable specialization in complex environments \cite{shazeer2017outrageouslylargeneuralnetworks, lepikhin2020gshard}. Xie et al. \cite{xie2024large} explored the role of LLMs in these agentic systems, discussing their ability to facilitate autonomous cooperation and communication between agents in complex environments, and marking an important step toward scalable and self-sufficient AI. By combining MoE principles with multi-agent collaboration, systems can achieve hierarchical decision-making: LLMs act as meta-controllers, routing tasks to specialized agents (e.g., vision, planning, or language experts) while maintaining global coherence.

However, the growing autonomy of such systems underscores the importance of XAI \cite{ding2022explainability} to ensure transparency and trust. XAI has gained prominence as a means to enhance accountability and support ethical AI adoption. By providing insights into model behavior, XAI ensures that even highly autonomous systems remain interpretable and accountable, addressing concerns about their decisions and actions in sensitive and dynamic environments.


Recent advancements in AI have demonstrated the potential of integrating fine-tuning, distillation, and in-context learning to enhance model performance. Huang et al. \cite{huang2022context} introduced in-context learning distillation, a novel method that transfers few-shot learning capabilities from large pre-trained LLMs to smaller models. By combining in-context learning objectives with traditional language modeling, this approach allows smaller models to perform effectively with limited data while maintaining computational efficiency.

Transfer learning \cite{iman2023review} has similarly emerged as a foundational technique, enabling pre-trained models to adapt their extensive knowledge to new domains using minimal data. This capability is particularly advantageous in resource-constrained scenarios. Techniques such as feature extraction, where pre-trained model layers are repurposed for specific tasks, and fine-tuning, which involves adjusting the weights of the pre-trained model for new tasks, further illustrate its adaptability. 

Complementing these methods, prompt engineering empowers LLMs to perform task-specific functions through carefully designed prompts. Techniques such as CoT prompting \cite{wei2022chain}, zero-shot \cite{pourpanah2022review}, and few-shot prompting  enhance the ability of LLMs to reason and generalize across diverse tasks without extensive retraining \cite{reynolds2021prompt}. Additionally, knowledge distillation  plays a crucial role in optimizing AI models by transferring knowledge from larger, more complex models to smaller, efficient ones \cite{gou2021knowledge}. Variants of distillation, such as task-specific distillation, feature distillation, and response-based distillation, further streamline the process for edge computing and resource-limited environments.

Reinforcement learning and its variant RLHF \cite{dai2023safe}, focus on training agents to make sequential decisions in dynamic environments. RLHF further aligns agent behavior with human preferences, fostering ethical and adaptive AI systems. Finally, continuous learning, or lifelong learning, addresses the challenge of adapting AI systems to new data while retaining previously learned knowledge, ensuring AI remains effective in changing environments \cite{riseAI}.

These techniques represent the cutting edge of generative AI, each contributing to solving complex challenges across diverse applications. The classification of these methods within NSAI paradigm, explored in the following sections, offers a structured perspective on their synergies and practical relevance.

\subsection{Classification of AI Technologies within NSAI Architectures}
This section categorizes generative AI techniques within the eight distinct NSAI architectures, highlighting their underlying principles and practical applications. By classifying these approaches, we gain a clearer understanding of how neural and symbolic methods synergize to address diverse challenges in AI, as summarized in Figure~\ref{fig:archi}.

\begin{figure}[!h]
    \centering
    \includegraphics[width=0.7\linewidth]{Figure1.pdf}
    \caption{Classification of AI technologies into NSAI architectures.}
    \label{fig:archi}
\end{figure}

\subsubsection{The Sequential Paradigm: From Symbolic to Neural Reasoning}
Techniques like RAG, GraphRAG, and Seq2Seq models (including LLMs, e.g., GPT \cite{openai2024gpt4technicalreport}) align with this method due to their reliance on neural encodings of symbolic data (e.g., text or structured information) to perform complex transformations before outputting results in symbolic form. Similarly, semantic parsing benefits from this framework by leveraging NNs to uncover latent patterns in symbolic inputs and generating interpretable symbolic conclusions. For instance, RAG-Logic  proposes a dynamic example-based framework using RAG to enhance logical reasoning capabilities by integrating relevant, contextually appropriate examples \cite{anonymous2024raglogic}. It first encodes symbolic input (e.g., logical premises) into neural representations using the RAG knowledge base search module. Neural processing occurs through the translation module, which transforms the input into formal logical formulas. Finally, the fix module ensures syntactic correctness, and the solver module evaluates the logical consistency of the formulas, decoding the results back into symbolic output. This process maintains the interpretability of symbolic reasoning while leveraging the power of NNs to improve flexibility and performance.


\subsubsection{The Nested Paradigm: Embedding Symbolic Logic in Neural Systems}
In-context learning, such as few-shot learning and CoT reasoning, aligns with the \textit{Symbolic[Neuro]} approach by leveraging NNs for context-aware predictions, while symbolic systems facilitate higher-order reasoning. Similarly, XAI falls into this category, as it often combines neural models for extracting features with symbolic frameworks to produce explanations that are easily understood by humans.

Zhang et al. \cite{zhang2022impact} presented a framework in which symbolic reasoning is enhanced by NNs. CoT is used as a method to generate prompts that combine symbolic rules with neural reasoning. For example, the task of reasoning about relationships between entities, such as “Joseph’s sister is Katherine” is approached by generating a reasoning path through CoT. The reasoning path is structured using symbolic rules, such as $Sister(A,C) \leftarrow Brother(A,B) \land Sister(B,C)$, which define the relationships between entities. These rules are then used to form CoT prompts that guide the model through the reasoning steps. The NN processes these prompts, performing feature extraction and probabilistic inference, while the symbolic system (including the knowledge base and logic rules) orchestrates the overall reasoning process. In this approach, the symbolic framework is the primary system for structuring the reasoning task, and the NN acts as a subcomponent that processes raw data and interprets the symbolic rules in the context of the query.

Methods like GNNs, NER, link prediction, and relation extraction fit into the \textit{Neuro[Symbolic]} category. These methods often leverage symbolic relationships, such as ontologies or graphs, as integral components to enhance neural processing. In addition, they integrate symbolic reasoning subroutines to perform higher-order logical operations, enforce consistency, or derive insights from structured representations. RL and RLHF exemplify this approach, where symbolic reasoning is integrated into the reward shaping and policy optimization stages to enforce logical constraints, ensure decision-making consistency, and align neural outputs with human-like decision-making criteria. For instance, NeuSTIP \cite{singh2023neustip} exemplifies this approach by combining GNN-based neural processing with symbolic reasoning to tackle link prediction and time interval prediction in temporal knowledge graphs (TKGs). NeuSTIP employs temporal logic rules, extracted via ``all-walks" on TKGs, to enforce consistency and strengthen reasoning. By embedding symbolic reasoning subroutines into the neural framework, NeuSTIP demonstrates how such models can effectively derive structured insights and perform reasoning under constraints.

\subsubsection{The Cooperative Paradigm: Iterative Interaction Between Neural and Symbolic Modules}
GANs align with this paradigm as their iterative interplay mirrors a cooperative dynamic between two distinct components: the generator creates outputs, while the discriminator evaluates them against predefined criteria, providing structured feedback to improve the generator's performance. This iterative feedback loop exemplifies the \textit{Neuro $|$ Symbolic} framework, where neural networks and symbolic reasoning components collaborate to achieve robust and adaptive problem-solving while adhering to symbolic constraints or logical consistency. Moreover, this cooperative dynamic inherently facilitates continuous learning, a process in which both neural and symbolic modules undergo iterative refinement to enhance their performance over time. In this paradigm, NN continuously updates its internal representations and model parameters in response to feedback derived from the symbolic module’s logical inferences and constraint evaluations. This adaptive process enables the NN to generalize more effectively across diverse and evolving data distributions. Simultaneously, the symbolic module is not static; it dynamically revises its rule-based reasoning mechanisms and knowledge structures by integrating new information extracted from the NN’s learned representations.
An example of this approach in reinforcement learning is the detect-understand-act (DUA) framework \cite{mitchener2022detect}, where neural and symbolic components collaborate iteratively to solve tasks in a structured manner. In DUA, the detect module uses a traditional computer vision object detector and tracker to process unstructured environmental data into symbolic representations. The understand component, which integrates symbolic reasoning, processes this data using answer set programming (ASP) and inductive logic programming (ILP), ensuring that decisions align with symbolic rules and constraints. The act component, composed of pre-trained reinforcement learning policies, acts as a feedback loop to refine the symbolic representations, allowing the system to converge on solutions that meet predefined criteria.

\subsubsection{The Compiled Paradigm: Embedding Symbolic Reasoning Within Neural Computation}
Approaches such as model distillation, fine-tuning, pre-training, and transfer learning align with the \textit{Neuro\textsubscript{Symbolic}} approach by integrating symbolic constraints or objectives (e.g., logical consistency, relational structures) directly into the learning process of NNs, either through the loss function or at the neuron level via activation functions. This ensures that outputs adhere to predefined symbolic rules, enabling structured reasoning within the network. Consequently, all NN models can be modeled by this paradigm, by embedding symbolic logic into neural architectures, bridging data-driven learning with symbolic reasoning. Mendez-Lucero et al. \cite{mendez2024semantic} complemented this perspective by embedding logical constraints within the loss function. The authors propose a distribution-based method that incorporates symbolic logic, such as propositional formulas and first-order logic, into the learning process. These constraints are encoded as a distribution and incorporated into the optimization procedure using measures like the Fisher-Rao distance or Kullback-Leibler divergence, effectively guiding the NN to adhere to symbolic constraints. This integration of symbolic knowledge into the loss function ensures that the neural model not only learns from data but also incorporates predefined logical rules, reinforcing the connection between neural learning and symbolic reasoning in the context of model distillation, fine-tuning, pre-training, and transfer learning.

Data augmentation leverages the \textit{Neuro:Symbolic $\to$ Neuro} approach, which uses symbolic reasoning to generate synthetic examples, enabling effective data augmentation. By producing high-quality labeled data through logical inference, it enhances the training process of NNs. This method seamlessly integrates the precision and structure of symbolic logic with the scalability and adaptability of NNs, resulting in more robust and efficient learning.
Li et al. \cite{li2024neuro} proposed a methodological framework that exemplifies this approach. Their framework systematically generates labeled data pairs \((x, y)\), where \(y\) is derived from \(x\) through symbolic transformations based on formal logical rules. The process begins with the formalization of mathematical problems in a symbolic space using mathematical solvers, ensuring the logical validity of the generated instances. Subsequently, mutation mechanisms are applied to diversify the examples, including simplification strategies (reducing the complexity of expressions) and complication strategies (adding constraints or variables). Each transformation results in a new problem instance with its corresponding solution, forming labeled pairs \((x', y')\) that enrich the training corpus with controlled complexity levels.


\subsubsection{The Fibring Paradigm: Connecting Neural Models Through Symbolic Constraints}
Techniques such as multi-agent AI and MoE systems align with this paradigm by leveraging symbolic functions to facilitate communication and coordination between agents (i.e., neural models). Symbolic reasoning mediates interactions, enforces constraints, and ensures alignment with predefined rules, while neural components adapt and learn from collective behaviors. This interplay enables robust and scalable problem-solving in complex, multi-agent environments. Belle et al.  \cite{belle2023neuro} explored how the combination of symbolic reasoning and agents can enable the development of advanced systems that are closer to human-like intelligence. They discusses how symbolic reasoning can mediate communication between agents, ensuring that they adhere to predefined rules while allowing the neural components to learn and adapt from collective behaviors. This directly aligns with the fibring paradigm, where multiple NNs are interconnected via a symbolic fibring function, enabling them to collaborate and share information in a structured manner.

Similarly, the recent DeepSeek-R1  \cite{guo2025deepseek} framework employs a MoE architecture to enhance reasoning capabilities in large-scale AI systems. DeepSeek’s MoE approach activates only a subset of its parameters for each task, mimicking a team of specialized experts. These experts coordinate effectively using reinforcement learning rewards and symbolic constraints, enabling efficient resource utilization while ensuring adherence to reasoning rules. The symbolic constraints act as an intermediary layer, guiding the interactions between experts in a structured manner, aligning their individual outputs to form a cohesive solution.

Likewise, Mixtral 8x7B \cite{jiang2024mixtral} employs a sparse mixture-of-experts (SMoE) framework, where each layer selects specific expert groups to process input tokens. This architecture not only reduces computational costs but also ensures that the model specializes in handling different tasks through expert routing. Mixtral’s ability to adaptively select experts for tasks requiring mathematical reasoning or multilingual understanding exemplifies how MoE-based systems achieve scalability and specialization while maintaining efficiency. The symbolic mediator within Mixtral ensures that expert selection follows a structured process governed by logical rules, promoting an orderly exchange of information between the experts while adhering to predefined symbolic constraints. 




\section{Evaluation of NSAI Architectures}

Ensuring the reliability and practical applicability of NASAI architectures requires a systematic evaluation across multiple well-defined criteria. Such an evaluation not only identifies the strengths and limitations of the architectures but also fosters trust among stakeholders by emphasizing interpretability, transparency, and robustness—qualities essential in domains such as healthcare, finance, and autonomous systems. Moreover, a rigorous assessment provides benchmarks that can stimulate the development of next-generation models. The following sections delineate the key criteria for evaluating NSAI architectures. 

\subsection{Core Criteria}
The evaluation framework for NSAI architectures is built upon several fundamental criteria: generalization, scalability, data efficiency, reasoning, robustness, transferability, and interpretability. Each criterion is elaborated below.\\

\noindent \textbf{Generalization:}
Generalization is defined as the capability of a model to extend its learned representations beyond the training dataset to perform effectively in novel or unforeseen situations. This criterion is evaluated based on:

\begin{itemize}
    \item[--] \textit{Out-of-distribution (OOD) performance}: The ability to maintain performance on data that deviate from the training distribution.
    \item[--] \textit{Contextual flexibility}: The capacity to adapt seamlessly to changes in context or domain with minimal retraining.
    \item[--] \textit{Relational accuracy}: The capacity to identify and exploit relevant relationships in  data while mitigating the influence of spurious correlations.
\end{itemize}

\noindent \textbf{Scalability:}
Scalability assesses the performance of NSAI architecture under
increasing data volumes or computational demands. A scalable system should remain efficient and effective as it scales. Key aspects include:

\begin{itemize}
    \item[--] \textit{Large-scale adaptation}: The ability to process and derive insights from massive datasets.
    \item[--] \textit{Hardware efficiency}: Optimal utilization of available computational resources, enabling operation on both low-resource devices and high-performance infrastructures.
    \item[--] \textit{Complexity management}: The ability to accommodate increased architectural complexity without compromising speed or deployment feasability.
\end{itemize}

\noindent \textbf{Data Efficiency:}
Data efficiency measures how effectively an NSAI model learns from limited data, an important consideration in scenarios where labeled data are scarce or expensive to obtain. This criterion encompasses: 

\begin{itemize}
    \item[--] \textit{Data reduction}: Achieving high performance with a reduced amount of training data.
    \item[--] \textit{Data optimization}: Maximizing the utility of available data (both labeled and unlabeled), potentially through semi-supervised learning techniques.
    \item[--] \textit{Incremental adaptability}: The capacity to incorporate new data progressively without undergoing complete retraining.
\end{itemize}

\noindent \textbf{Reasoning:}
Reasoning reflects the model's ability to analyze data, extract insights, and draw logical conclusions. This criterion underscores the unique advantage of NSAI architectures, which combine neural learning with symbolic reasoning. This criterion evaluates:

\begin{itemize}
    \item[--] \textit{Logical reasoning}: The systematic application of explicit rules to derive precise and consistent inferences. 
    \item[--] \textit{Relational understanding}: The comprehension of complex relationships between entities within the data.
    \item[--] \textit{Cognitive versatility}: The integration of various reasoning paradigms (e.g., deductive, inductive, and abductive reasoning) to tackle diverse challenges. 
\end{itemize}

\paragraph{Robustness:}
Robustness measures the system’s reliability and resilience to disruptions, including noisy data, adversarial inputs, or dynamic environments. The evaluation considers:

\begin{itemize}
    \item[--] \textit{Resilience to perturbations/anomalies}: The ability to sustain stable performance despite the presence of noise or adversarial data.
    \item[--] \textit{Adaptive resilience}: The maintenance of functionality under changing or unpredictable conditions.
    \item[--] \textit{Bias resilience}: The effectiveness in detecting and correcting biases to ensure fairness and accuracy in predictions.
\end{itemize}

\noindent \textbf{Transferability:}
Transferability assesses the model’s ability in applying learned knowledge to new contexts, domains, or tasks. This is essential for reducing the effort and time required for model adaptation. Its evaluation involves:

\begin{itemize}
    \item[--] \textit{Multi-domain adaptation}: The capacity to generalize across diverse domains with minimal modifications.
    \item[--] \textit{Multi-task learning}: The capability to handle multiple tasks simultaneously through shared knowledge representations.
    \item[--] \textit{Personalization}: The adaptability of the model to meet specific user or application requirements with limited additional effort.
\end{itemize}

\noindent \textbf{}{Interpretability:}
Interpretability evaluates the model’s ability to explain its decisions, ensuring transparency and trust in NSAI systems. This criterion assesses:

\begin{itemize}
    \item[--] \textit{Transparency}: The clarity with which the internal mechanisms and decision processes of the model are revealed.
    \item[--] \textit{Explanation}: The ability to provide comprehensible justifications for predictions or decisions.
    \item[--] \textit{Traceability}: The capability to reconstruct the sequence of operations and factors that contributed to a given outcome.
\end{itemize}

\noindent By systematicaly addressing these criteria, researchers and practitioners can ensure that NSAI architectures are not only scientifically rigorous but also practical, adaptable, and ready for real-world applications. This evaluation framework not only facilitates continuous improvement and innovation but also supports the broad adoption of NSAI systems across various industries and application domains.

\subsection{Evaluation Methodology}

\noindent The evaluation of NSAI architectures was conducted using a systematic approach to ensure a robust and transparent assessment of their performance across multiple criteria. This process relied on three key sources: scientific literature, empirical findings, and an analysis of the design principles underlying each architecture. \textbf{Table~\ref{tab:references}} summarizes the relevant research works associated with the identified NSAI architectures in Section 3. %These sources provided a comprehensive foundation for assigning ratings such ``High", ``Medium", or ``Low" to each criterion.
 The scientific literature served as the primary source of qualitative insights, offering detailed analyses of the strengths and limitations of various architectures. Foundational research and state-of-the-art studies provided evidence of performance in areas such as scalability, reasoning, and interpretability, helping to guide the evaluation. Additionally, empirical results from experimental studies and benchmarks offered quantitative data, enabling objective comparisons across architectures. Metrics such as accuracy, adaptability, and efficiency were particularly valuable in validating the claims made in research papers. The design principles of each technology were also considered to understand how neural and symbolic components were integrated. This analysis provided insights into the inherent capabilities and constraints of each architecture, such as its suitability for handling complex reasoning tasks, scalability to large datasets, or adaptability to dynamic environments.

\vspace*{0.5cm}

\noindent For each criterion, the ratings were assigned as follows:

\begin{itemize}
    \item \textit{High:} Awarded to architectures that consistently demonstrated exceptional performance across multiple studies and benchmarks, showcasing clear advantages in the specific criterion.
    \item \textit{Medium:} Assigned to architectures with satisfactory performance, excelling in certain aspects but with notable limitations in others.
    \item \textit{Low:} Given to architectures with significant weaknesses, such as inconsistent results or an inability to effectively address the criterion.
\end{itemize}

 By combining insights from literature, empirical findings, and design analysis, this methodology ensures a balanced and evidence-based evaluation. It provides a clear understanding of the strengths and weaknesses of each architecture, enabling meaningful comparisons and guiding future advancements in NSAI research and applications.

\begin{table}[h]
    \centering
        \caption{Set of relevant published NSAI architectures considered in the proposed study.}

    \begin{tabular}{|l|p{10cm}|}
        \hline
        \textbf{Architecture} & \textbf{References} \\
        \hline
        \textit{Symbolic $\to$ Neuro $\to$ Symbolic} & \cite{kouris2021abstractive}, \cite{sutherland2019leveraging}, \cite{gu2019local}, \cite{cui2021sememes}, \cite{xu2019relation}, \cite{cowen2019neural}, \cite{bounabi2021new}, \cite{es2021sentence}, \cite{lima2019impact}, \cite{zhou2021relation}, \cite{gong2020hierarchical}, \cite{tato2019hybrid}, \cite{langton2021applied}, \cite{bracsoveanu2019semantic}, \cite{pinhanez2021using}, \cite{dehua2021bdcn}, \cite{fazlic2019novel}, \cite{d2019team}, \cite{ayyanar2019causal}, \cite{hu2021dialoguecrn}, \cite{chen2020question}, \cite{manda2020automated}, \cite{honda2019question}, \cite{schon2019corg}, \cite{amin2019cases} \\
        \hline
        \textit{Neuro[Symbolic]} & \cite{heule2016solving}, \cite{madan2021fast} \\
        \hline
        \textit{Symbolic[Neuro]} & \cite{silver2016mastering}, \cite{chen2021web}, \cite{chen2021neurallog}, \cite{pacheco2021modeling}, \cite{chaturvedi2019fuzzy}, \cite{qin2021neural} \\
        \hline
        \textit{Neuro $|$ Symbolic} & \cite{mao2019neuro}, \cite{yao2018learning}, \cite{shi2021neural}, \cite{vskrlj2021autobot}, \cite{wang2021variational}, \cite{lemos2020neural}, \cite{huang2019attentive} \\
        \hline
        \textit{Neuro $\to$ Symbolic $\leftarrow$ Neuro} & \cite{das2021case}, \cite{garcez2004fibring}, \cite{belle2023neuro}, \cite{guo2025deepseek}, \cite{jiang2024mixtral}, \cite{guo2024large}, \cite{maldonado2024multi}, \cite{he2024mixturemillionexperts}, \cite{lo2024closerlookmixtureofexpertslarge} \\
        \hline
        \textit{Neuro:Symbolic $\to$ Neuro} & \cite{lample2019deep}, \cite{yabloko2020ethan}, \cite{zhou2020temporal}, \cite{saveleva2021graph}, \cite{gupta2021neuro}, \cite{demeter2020just}, \cite{jiang2021lnn}, \cite{kogkalidis2020neural}, \cite{zhang2021noahqa}, \cite{sen2020learning}, \cite{huo2019graph}, \cite{jiang2020medical}, \cite{liu2021heterogeneous}, \cite{chaudhury2021neuro}, \cite{verga2020facts}, \cite{socher2013reasoning}\\
        \hline
        \textit{Neuro\textsubscript{Symbolic\textsubscript{Loss}}} & \cite{serafini2016logic}, \cite{raissi2019physics}, \cite{chen2020mapping}, \cite{graziani2019jointly}, \cite{altszyler2020zero}, \cite{hussain2018semi} \\
        \hline
        \textit{Neuro\textsubscript{Symbolic\textsubscript{Neuro}}} & \cite{smolensky2016basicreasoningtensorproduct} \cite{smolensky1990tensor} \\
        \hline
    \end{tabular}
    \label{tab:references}
\end{table}


\subsection{Results and Discussion}
\noindent \textbf{Figure~\ref{comparison}}  provides a comparative analysis of various NSAI architectures across seven main evaluation criteria and their respective sub-criteria. This comprehensive evaluation highlights the strengths and weaknesses of each architecture, offering insights into their performance, adaptability, and interpretability.



\begin{figure}
    \centering
    \includegraphics[width=1\linewidth]{Archi.png}
    \caption{Comparison of NSAI architectures based on various criteria and sub-criteria.}
    \label{comparison}
\end{figure}

For example, under the ``generalization" criterion, \textit{Neuro $\to$ Symbolic $\leftarrow$ Neuro} and \textit{Neuro $|$ Symbolic} perform well in generalization scenarios, demonstrating strong generalization capabilities, particularly in handling relational accuracy, making it suitable for complex, real-world applications. However, \textit{Neuro\textsubscript{Symbolic\textsubscript{Loss}}} and \textit{Neuro\textsubscript{Symbolic\textsubscript{Neuro}}} demonstrates notable shortcomings in continuous flexibility and OOD generalization, highlighting its difficulty in adapting to dynamic and evolving contexts without the need for extensive retraining. As for the ``scalability" criterion, \textit{Neuro $\to$ Symbolic $\leftarrow$ Neuro} and \textit{Neuro\textsubscript{Symbolic\textsubscript{Neuro}}} excel across all sub-criteria, including large-scale adaptation and hardware efficiency, demonstrating their capacity to handle industrial-scale applications. Conversely, \textit{Symbolic[Neuro]} achieves only medium performance in scalability, reflecting challenges in balancing its rule-based reasoning with the demands of large-scale or resource-intensive tasks. In particular, \textit{Neuro $|$ Symbolic}, rated low, struggles to maintain efficiency and adaptability when scaling to more complex systems, highlighting a need for improved coordination between its neural and symbolic components. 

\vspace*{0.5cm}

In terms of ``data efficiency", architectures such as \textit{Neuro $\to$ Symbolic $\leftarrow$ Neuro}, \textit{Symbolic Neuro Symbolic}, and \textit{Neuro\textsubscript{Symbolic\textsubscript{Neuro}}} consistently achieve high ratings, excelling in both data reduction and optimization. This indicates their ability to learn effectively with limited data. However, \textit{Symbolic[Neuro]} demonstrates only medium adaptability when incorporating incremental data updates. When evaluating the ``Reasoning" criterion, architectures such as \textit{Symbolic[Neuro]}, \textit{Neuro $\to$ Symbolic $\leftarrow$ Neuro}, and \textit{Neuro\textsubscript{Symbolic\textsubscript{Neuro}}} show strong capabilities in logical reasoning and relational understanding. However, \textit{Neuro:Symbolic $\to$ Neuro} displays lower versatility in combining diverse reasoning methods, reflecting limitations in solving complex problems. For ``Robustness", most architectures perform well, demonstrating high resilience to perturbations and effective bias handling. However, \textit{Symbolic[Neuro]} and \textit{Symbolic Neuro Symbolic} architectures exhibit weaknesses in adapting to dynamic environments and mitigating biases effectively. 

\vspace*{0.5cm}

Regarding ``Transferability", architectures like \textit{Neuro $\to$ Symbolic $\leftarrow$ Neuro}, \textit{Neuro\textsubscript{Symbolic\textsubscript{Loss}}}, and \textit{Neuro\textsubscript{Symbolic\textsubscript{Neuro}}} excel in multi-task learning and multi-domain adaptation, enabling effective reuse of knowledge across domains. In contrast, \textit{Symbolic Neuro Symbolic}, \textit{Neuro:Symbolic $\to$ Neuro}, and nested architectures demonstrate lower adaptability to personalized applications.
Lastly, in ``Interpretability", most architectures perform well, achieving high marks in transparency and traceability. \textit{Symbolic[Neuro]} also achieves commendable results in this criterion, demonstrating its ability to explain decisions effectively, which is essential for sensitive applications like healthcare and finance.

\vspace*{0.5cm}

Overall, the \textit{Neuro $\to$ Symbolic $\leftarrow$ Neuro} architecture emerges as the best-performing model, consistently achieving high ratings across all criteria. Its exceptional performance in generalization, scalability, and interpretability makes it highly suitable for real-world applications that demand reliability, adaptability, and transparency. While other architectures also perform well in specific areas, the versatility and robustness of \textit{Neuro $\to$ Symbolic $\leftarrow$ Neuro} set it apart as the most balanced and capable solution. This conclusion aligns with findings in the state of the art, which highlight the effectiveness of \textit{Neuro $\to$ Symbolic $\leftarrow$ Neuro} architectures in leveraging advanced AI technologies, such as multi-agent systems. Multi-agent systems are well-documented for their robustness, particularly in dynamic and distributed environments, where their ability to coordinate, adapt, and reason collectively enables superior performance. For instance, Subramanian et al. \cite{subramanian2024neuro} demonstrated that incorporating neuro-symbolic approaches into multi-agent RL enhances both interpretability and probabilistic decision-making. This makes such systems highly robust in environments with partial observability or uncertainties. Similarly, Keren et al.  \cite{keren2021collaboration} highlighted that collaboration among agents in multi-agent frameworks promotes group resilience, enabling these systems to adapt effectively to dynamic or adversarial conditions. These attributes are particularly valuable in \textit{Neuro $\to$ Symbolic $\leftarrow$ Neuro} architectures, as they address the critical need for transparency and robustness in complex real-world applications.

%\vspace*{0.5cm}

%In summary, the strong performance of the \textit{Neuro $\to$ Symbolic $\leftarrow$ Neuro} architecture across multiple criteria, supported by contemporary research on multi-agent systems, confirms its potential as a leading framework for solving challenging problems in AI. Its ability to combine collaborative reasoning, symbolic interpretability, and dynamic adaptability makes it a robust and transparent solution for addressing real-world challenges.

\section{Conclusion}
This study evaluates several NSAI architectures against a comprehensive set of criteria, including generalization, scalability, data efficiency, reasoning, robustness, transferability, and interpretability. The results highlight the strengths and weaknesses of each architecture, offering valuable insights into their capabilities for real-world applications. Among the architectures investigated, \textit{Neuro $\to$ Symbolic $\leftarrow$ Neuro} emerges as the most balanced and robust solution. It consistently demonstrates superior performance across multiple criteria, excelling in generalization, scalability, and interpretability. These results align with recent advancements in the field, which emphasize the role of multi-agent systems in enhancing robustness and adaptability. As shown in recent studies, multi-agent frameworks, when integrated with neuro-symbolic methods, provide significant advantages in handling uncertainty, fostering collaboration, and maintaining resilience in dynamic environments. This integration not only enables better decision-making but also ensures transparency and traceability, which are critical for sensitive applications.  Moreover, its ability to leverage advanced AI technologies, such as multi-agent systems, positions \textit{Neuro $\to$ Symbolic $\leftarrow$ Neuro} as a leading candidate for addressing the demands of generative AI applications.

\vspace*{0.5cm}

Future work will be focused on exploring the scalability of this architecture in even larger and more diverse environments. Additionally, advancing the integration of symbolic reasoning within multi-agent systems may further enhance their robustness and cognitive versatility. As the field evolves, \textit{Neuro $\to$ Symbolic $\leftarrow$ Neuro} architectures are likely to remain at the forefront of innovation, offering practical and scientifically grounded solutions to the most pressing challenges in AI.

\section*{CRediT authorship contribution statement}
\textbf{Oualid Bougzime:} Writing – original draft, Methodology, Investigation. \textbf{Samir Jabbar:} Writing – original draft, Methodology, Investigation. \textbf{Christophe Cruz:} Writing – review \& editing, Methodology, Supervision. \textbf{Fr\'ed\'eric Demoly:} Writing –
review \& editing, Methodology, Supervision, Funding acquisition, Project administration.

\section*{Declaration of competing interest}
The authors declare that they have no known competing financial interests or personal relationships that could have appeared to influence the work reported in this paper.

\section*{Acknowledgements}
This research was funded by the IUF, Innovation Chair on 4D Printing, the French National Research Agency under the “France 2030 Initiative” and the “DIADEM Program”, grant number 22-PEXD-0016 (“ARTEMIS”).

\bibliographystyle{unsrt}
\documentclass{MITstyle}

%\usepackage[table]{xcolor}
\usepackage{chngcntr}
\usepackage{hyperref}
\usepackage{microtype}

\title{A Lightweight and Extensible Cell Segmentation and Classification Model for Whole Slide Images}

\author{Nikita Shvetsov~$^{1, }$\footnote{Correspondence e-mail: nikita.shvetsov@uit.no}, Thomas K. Kilvaer~$^{2, 3}$, Masoud Tafavvoghi~$^{4}$, Anders Sildnes~$^{1}$, \\ Kajsa Møllersen~$^{4}$, Lill-Tove Rasmussen Busund~$^{5, 6}$, Lars Ailo Bongo~$^{1}$ \\
%
\vspace{1em} % Space between authors and afilliations
%
\normalfont{\small $^{1}$Department of Computer Science, UiT The Arctic University of Norway}\\
\normalfont{\small $^{2}$Department of Oncology, University Hospital of North Norway}\\
\normalfont{\small $^{3}$Department of Clinical Medicine, UiT The Arctic University of Norway}\\
\normalfont{\small $^{4}$Department of Community Medicine, UiT The Arctic University of Norway}\\
\normalfont{\small $^{5}$Department of Medical Biology, UiT The Arctic University of Norway} \\
\normalfont{\small $^{6}$Department of Clinical Pathology, University Hospital of North Norway} %\vspace{2em}
}

\begin{document}
\maketitle

\section*{Abstract}

% \begin{abstract}
% Developing clinically useful cell-level analysis tools in digital pathology remains challenging due to limitations in dataset granularity, inconsistent annotations, computational demands of advanced models, and difficulties in integrating new technologies into clinical workflows. To address these challenges, we propose a multi-faceted solution that enhances data quality, model performance, and usability to create a lightweight and extensible cell segmentation and classification model.

% First, we update data labels by employing a cross-relabeling process that refines the labels of two existing datasets, PanNuke and MoNuSAC, to create a new unified dataset with enhanced granularity, encompassing seven distinct cell types. Second, we leverage the H-Optimus foundation model as a fixed encoder to improve feature representation for simultaneous cell segmentation and classification tasks. Third, to address the computational demands of foundation models, we employ knowledge distillation to reduce model size and complexity while maintaining comparable performance. Finally, to facilitate integration into clinical workflows, we integrate the distilled model into the QuPath software, a widely used open-source platform in digital pathology.

% Our results demonstrate improvements in cell segmentation and classification performance using the H‑Optimus-based model compared to a CNN-based model. Specifically, the average $R^2$ improved from 0.575 to 0.871, and the average $PQ$ score improved from 0.450 to 0.492, indicating better alignment with actual cell counts and enhanced segmentation and classification quality. Furthermore, the distilled student model maintains performance comparable to the larger foundation model while reducing the parameter count by a factor of 48.
% Overall, by reducing computational complexity and integrating it into existing workflows, the proposed approach may significantly impact diagnostic processes, reduce the workload of pathologists, and contribute to improved patient outcomes. Though our approach shows potential enhancements in efficiency and usability of cell segmentation and classification models in digital pathology, extensive validation is needed to deploy these models in clinical practice.
% \end{abstract}

%%% shortened abstract
\begin{abstract}
Developing clinically useful cell-level analysis tools in digital pathology remains challenging due to limitations in dataset granularity, inconsistent annotations, high computational demands, and difficulties integrating new technologies into workflows. To address these issues, we propose a solution that enhances data quality, model performance, and usability by creating a lightweight, extensible cell segmentation and classification model. 

First, we update data labels through cross-relabeling to refine annotations of PanNuke and MoNuSAC, producing a unified dataset with seven distinct cell types. Second, we leverage the H-Optimus foundation model as a fixed encoder to improve feature representation for simultaneous segmentation and classification tasks. Third, to address foundation models' computational demands, we distill knowledge to reduce model size and complexity while maintaining comparable performance. Finally, we integrate the distilled model into QuPath, a widely used open-source digital pathology platform. 

Results demonstrate improved segmentation and classification performance using the H-Optimus-based model compared to a CNN-based model. Specifically, average $R^2$ improved from 0.575 to 0.871, and average $PQ$ score improved from 0.450 to 0.492, indicating better alignment with actual cell counts and enhanced segmentation quality. The distilled model maintains comparable performance while reducing parameter count by a factor of 48. By reducing computational complexity and integrating into workflows, this approach may significantly impact diagnostics, reduce pathologist workload, and improve outcomes. Although the method shows promise, extensive validation is necessary prior to clinical deployment.
\end{abstract}
\clearpage

\section{Introduction}
In digital pathology, accurate segmentation and classification of cells are crucial for many diagnostic, prognostic, and predictive analyses \cite{Jaber_Beziaeva_etal._2019,Lin_Pan_etal._2022,Park_Ock_etal._2022,Shen_Choi_etal._2024}. Nowadays, developments in computational pathology offer multiple solutions \cite{H._Qu_P._Wu_etal._2020,Javed_Mahmood_etal._2020} to utilize cell-level datasets to train machine learning models that solve these problems. The quality and specificity of training datasets are critical for robust and accurate models. Adhering to the principle of "garbage in, garbage out", it is essential to ensure that these datasets are extensively and accurately labeled with distinct classes that reflect the diverse biological characteristics of different cell types. Unfortunately, the number of open-source datasets comprising such high-quality annotations is limited. Existing cell segmentation datasets \cite{Gamper_Koohbanani_etal._2019,Graham_Vu_etal._2019,Verma_Kumar_etal._2021} may offer extensive annotations for certain cell types while providing more general labels for others. For example, in PanNuke, which is one of the largest open-source datasets comprising labeled cells, various types of morphologically and functionally different inflammatory cells like macrophages and lymphocytes are clustered in a broad "inflammatory" class. Consequently, these classes are frequently omitted from analyses or aggregated into broader meta-classes \cite{Gamper_Koohbanani_etal._2020} and likely interfere with other cell classes included in the dataset. This and similar inconsistencies in annotation granularity limit the ability of machine learning models to learn the comprehensive and nuanced features necessary for accurate cell segmentation and classification. To address these challenges, methods for refining and standardizing dataset annotations are essential to enhance the quality of training data.

A complementary approach to mitigate the absence of high-quality training data is the use of foundation models. Foundation models as encoders are defined as large-scale, versatile networks pre-trained on vast, diverse datasets using self-supervised learning, contrasting with convolutional neural network (CNN) pre-trained encoders that rely on supervised learning with labeled data. In practice, foundation models leverage enormous amounts of weakly or unlabeled data from millions of whole slide images (WSIs) and employ self-attention mechanisms to capture long-range dependencies and global context \cite{Chen_Ding_etal._2024,Saillard_Jenatton_etal._2024,Vorontsov_Bozkurt_etal._2024,Xu_Usuyama_etal._2024}. As a consequence, foundation models are able to produce transferable feature representations across different cell types and tissue environments. The feature representations can be leveraged by decoder networks to produce segmentation masks and pixel-level classifications. Because foundation models have comprehensive feature representations, they can be effectively fine-tuned using much smaller amounts of cell-level data compared to the large datasets needed to train models from scratch. Furthermore, foundation models incorporate adversarial training elements or contrastive learning \cite{Chen_Ding_etal._2024,Xu_Usuyama_etal._2024}, enhancing their resilience and adaptability by exposing them to challenging and varied scenarios during training. This may result in more generalizable models, often making them well-suited for diverse and complex tasks in digital pathology.

Despite the inherent advantages of foundation models, their deployment for practical use faces its own obstacles. In particular, they require substantial computational power, financial investments and rigorous testing to ensure reliability and efficacy for a given task \cite{Akkus_Dangott_etal._2022,Dragomir_Cocuz_etal._2022,Go_2022,Jafri_Farooqui_etal._2024}. Moreover, while foundation models enhance feature representation and performance, they depend on the quality of available annotations for decoder fine-tuning and, like any other model, cannot resolve existing inconsistencies or ambiguities in data labels. Therefore, there remains a critical need for solutions that address both data quality and practical deployment considerations.
Further, integrating new technologies into existing clinical workflows often encounters resistance, as it necessitates adjustments to established diagnostic processes. So, there is a need to develop solutions that could be integrated into current practices, minimizing the burden on medical professionals to adopt new tools \cite{King_Williams_etal._2023}.

Existing solutions \cite{Goldsborough_Philps_etal._2024,Hörst_Rempe_etal._2024}, while addressing some aspects of these challenges, fall short in providing a comprehensive approach. To address the data quality and clinical deployment issues, we propose a multi-faceted solution that encompasses data refinement, model optimization, and integration with existing pathology tools (\hyperref[fig:fig1]{Figure 1}). The outcome is a lightweight cell segmentation and classification model that can be integrated into digital pathology workflows for practical clinical use.

\begin{figure}[h!]
    \centering
    \includegraphics[width=\textwidth, height=0.82\textheight, keepaspectratio]{images/Figure_1.pdf}
    \caption{Overview of the proposed solution, including 1) Data refinement using cross-relabeling, 2) Teacher model development and fine tuning, 3) Student model optimization with knowledge distillation and 4) Student model and QuPath integration}
    \label{fig:fig1}
\end{figure}
\clearpage

Our approach begins with preparing the data for the fine-tuning and training of the machine learning models. We create a refined dataset, acquired via cross-relabeling two cell-level datasets, enhancing annotation specificity and consistency of the labeled data. Subsequently, we create a cell segmentation and classification model based on the foundation model. We leverage the foundation model as a fixed encoder and fine-tune a decoder using the refined dataset to improve generalization across diverse tissue- and cell types.
To ensure that the model remains lightweight and deployable in a possibly resource-constrained environment, we employ knowledge distillation to approximate the functionality of the foundation model. Finally, to facilitate the practical application of our model in digital pathology workflows, we integrate it with the QuPath \cite{Bankhead_Loughrey_etal._2017} application. Each methodological component contributes to the overarching goal of enhancing model performance, generalizability, and usability in clinical settings.

The primary contributions of this paper are:
\begin{enumerate}
    \item \textit{Data labels refinement through cross-relabeling:}
    
    We propose a new method for refining labels of cell-level datasets through cross-relabeling. This method employs classification models to re-label broad and ambiguous instances, resulting in a more diverse dataset. Our evaluation demonstrates that these classification models achieve high accuracy on test subsets, indicating the reliability of the method for label refinement.

    \item \textit{Enhanced model performance via foundation models:}
    
    We employ a foundation model as a feature extractor for the cell segmentation and classification task. In comparison with training a CNN model from scratch, the foundation model backbone only needs fine-tuning, which significantly reduces training time, computational resources and data requirements. We show that using a foundation model encoder leads to better performance in cell segmentation and classification networks than using a CNN-based encoder. This improvement may enable the model to generalize more effectively across various tissue types and imaging methods.
    
    \item \textit{Model optimization through knowledge distillation:}
    
    We show that a smaller student model trained using knowledge distillation on the refined dataset obtained via our cross-relabeling approach from a foundation model achieves comparable performance in cell segmentation and quantification tasks. As a result, this model is more suitable for deployment in environments without high-performance computing resources.
    
    \item \textit{Integration with QuPath:}
    
    We integrate the distilled cell segmentation and classification model into QuPath, a widely used open-source digital pathology platform, to accelerate clinical adaptation by enabling pathologists to more easily incorporate advanced computational tools into their existing workflows.
\end{enumerate}

Through these methodological steps, we aim to bridge the gap between advanced machine learning techniques and practical clinical applications, making accurate and efficient digital pathology accessible in a broader range of healthcare settings.

\section{Refining Existing Datasets Using Cross-Relabeling}
To address the limitations of sparse and ambiguous labeling of cell-level datasets, we propose a generalizable cross-relabeling strategy that can be applied to any dataset containing broadly categorized or imprecisely labeled cell types. This approach involves training and subsequently leveraging classification models to refine broad categories into more specific or biologically relevant classes.
When applied to cell-level data, the methodology includes extracting individual cell images from the dataset patches, preprocessing these images to standardize the size and accommodate partial cells, and then training deep learning classifiers capable of distinguishing between the finer cell subtypes within the coarser categories. 
To illustrate our approach, we focus on the PanNuke \cite{Gamper_Koohbanani_etal._2020, Gamper_Koohbanani_etal._2019} and MoNuSAC \cite{Verma_Kumar_etal._2021} datasets that we have used to train models for cell quantification in our previous works \cite{Shvetsov_Grønnesby_etal._2022,Shvetsov_Sildnes_etal._2024}. We find that for better cell differentiation we have to introduce more granular labels. PanNuke includes a broad classification of "inflammatory" cells, encompassing lymphocytes, macrophages, and neutrophils. Each cell type differs significantly in structure, function, and clinical relevance. Conversely, MoNuSAC uses the label "epithelial" for a class that comprises both benign epithelial cells and malignant neoplastic cells. This practice makes it challenging to differentiate between benign and malignant epithelial cells in the dataset, which is a critical distinction when identifying tumor areas within tissue samples. To address these issues, we implement a cross-relabeling strategy as shown in \hyperref[fig:fig2]{Figure 2}. The key components are two classification models: one is trained on singular cell images from PanNuke data to classify the epithelial meta-class into epithelial and neoplastic classes. The other is trained on MoNuSAC to refine the inflammatory class into lymphocytes, neutrophils, and macrophages.

\begin{figure}[h!]
    \centering
    \includegraphics[width=\textwidth]{images/Figure_2.pdf}
    \caption{Refined dataset generation via cross relabeling}
    \label{fig:fig2}
\end{figure}

The refining approach consists of three consecutive steps. The first is the preprocessing step, in which we extract individual cells from both datasets (\hyperref[fig:fig3]{Figure 3}). The specifics of PanNuke and MoNuSAC patch preparation before cell preprocessing are provided in \hyperref[chap:S1]{Appendix S1}.

\begin{figure}[h!]
    \centering
    \includegraphics[width=\textwidth]{images/Figure_3.pdf}
    \caption{Cell instances preprocessing including (1) cell map extraction, (2) bounding box delineation, (3) adjusting cell boxes and (4) cropping and resizing of cell images}
    \label{fig:fig3}
\end{figure}

During preprocessing, we extract cell type maps from the ground truth label mask and calculate bounding boxes around each cell instance. To accommodate partial cells at patch borders, a common issue in cropped patch images, we employ mirror padding and extend the field of view of the cell label by 15 pixels to capture adjacent cells. We then crop and resize the identified regions to $64 \times 64$ pixels using bicubic interpolation.

The preprocessed PanNuke dataset comprises 68,031 neoplastic and 23,207 epithelial cell images, while MoNuSAC comprises  33,104 lymphocytes, 1,252 neutrophils, and 1,695 macrophages, which we subsequently use in training cell classification models and classifying the cell image data \hyperref[fig:S2]{Appendix Figure S2 (1)}. 

The next step is to train two distinct ResNet50-based classifiers tailored to address the specific labeling challenges inherent in each dataset. We use ResNet50 for classification models due to its proven effectiveness for image classification tasks in histopathology \cite{pan2022reviewmachinelearningapproaches}, and its compatibility with small images. For the PanNuke dataset, we design the classifier, trained on MoNuSAC data, to disaggregate the heterogeneous "inflammatory" cell category into distinct subtypes: lymphocytes, macrophages, and neutrophils. Similarly, for the MoNuSAC dataset, the classifier is trained on PanNuke data and distinguishes between benign and malignant epithelial cells within the overarching "epithelial" label. By applying these targeted classifiers to their respective datasets, we assign more specific labels to individual cell instances, thus enabling us to create a unified dataset.
To ensure a balanced representation of classes, we train both models on datasets that had been equalized to match the size of the least represented class. Thus, we obtain datasets comprising 23,207 samples per class for PanNuke and 1,252 samples per class for MoNuSAC data. Next, we partition both of them into training (70\%), validation (20\%), and testing (10\%) subsets. To mitigate the risk of overfitting, we use a single dropout layer with a rate of p=0.5 in both models and data augmentation using randomized color perturbations, rotation, and horizontal and vertical flipping. We employ AdamW optimizer and the cross-entropy loss function for the training criterion.

To evaluate the two trained models, we measure the classification accuracy on the respective test subsets. The accuracies on the test subset for both classifiers are presented in \hyperref[tab:1]{Table 1}. The PanNuke model achieves an average accuracy of 93.57\%, with higher accuracy for neoplastic cells (96.06\%) compared to epithelial cells (86.26\%). The confusion matrix in Figure A3.1 shows that the model predominantly distinguishes accurately between epithelial and neoplastic tissues, with a substantial number of correct classifications and relatively few misclassifications. The MoNuSAC model demonstrates an average accuracy of 98.92\%, excelling in classifying lymphocytes (99.67\%) and macrophages (94.12\%), with lower performance for neutrophils (85.71\%). The confusion matrix in Figure A3.2 shows that the model identifies lymphocytes and performs reasonably well with macrophages and neutrophils.

\begin{table}[h!]
\renewcommand{\arraystretch}{1.5}
  \centering
  \caption{Cell classification results for PanNuke and MoNuSAC trained models (CI 95\%).}
  \label{tab:1}
  \begin{tabular}{|l|c|c|}
   \hline
   %\rowcolor{gray!30}
    Accuracy               & PanNuke model              & MoNuSAC model              \\
    \hline
    Average      & 0.936 (0.931--0.941)         & 0.989 (0.986--0.993)        \\
    \hline
    Neoplastic   & 0.961 (0.956--0.965)         & -                          \\
    \hline
    Epithelial   & 0.863 (0.849--0.877)         & -                          \\
    \hline
    Lymphocytes  & -                          & 0.997 (0.995--0.999)        \\
    \hline
    Neutrophils  & -                          & 0.857 (0.796--0.918)        \\
    \hline
    Macrophages  & -                          & 0.941 (0.906--0.976)        \\
    \hline
  \end{tabular}
\end{table}

Finally, during the last step, we use the model trained on PanNuke data for epithelial cells in MoNuSAC and the model trained on MoNuSAC for the inflammatory cells class in PanNuke. Specifically, we use classifier models to relabel epithelial cells in MoNuSAC and inflammatory cells in PanNuke data. Then we combine cells with refined labels and the rest of the cells in both datasets to create a refined dataset (\hyperref[fig:S2]{Appendix Figure S2 (2)}). The process of relabeling cells and visualizing them on a patch is shown in \hyperref[fig:fig4]{Figure 4}. The cell counts in the refined dataset are provided in \hyperref[tab:S4]{Appendix Table S4}.

\begin{figure}[h!]
    \centering
    \includegraphics[width=\textwidth, height=0.42\textheight, keepaspectratio]{images/Figure_4.pdf}
    \caption{Cell relabeling procedure for epithelial and inflammatory cell classes}
    \label{fig:fig4}
\end{figure}

%\hfill

Relabeling and combining datasets have been explored in a prior study \cite{Parulekar_Kanwat_etal._2023}, where consecutive fine-tuning on multiple datasets was employed to account for hierarchical class label structures. While the method presented in \cite{Parulekar_Kanwat_etal._2023} is intuitive, it often lacks consistency and requires multiple fine-tuning runs, which can be cumbersome and time-consuming. 
In contrast, cross-relabeling simplifies this process by using specialized classification models tailored to each dataset's specific labeling challenges. This approach provides better transparency and produces a unified dataset encompassing seven distinct cell types across multiple tissue samples, enhancing data diversity for further model training or fine-tuning.

Despite these improvements, cross-relabeling does not entirely resolve issues related to poor labeling quality or the amount of labeled data. Specifically, our results show lower accuracies persist for underrepresented classes, such as macrophages, which may stem from a limited sample availability and intrinsic challenges in distinguishing these cells based solely on H\&E staining. Furthermore, while our method enhances label specificity, it relies on the initial quality of the broad labels; thus, any fundamental inaccuracies in the original annotations can propagate through the relabeling process. Addressing the overall problem of limited data labels may require integrating additional data sources or utilizing complementary immunohistochemical staining methods.
Although the reported performance metrics are obtained from evaluations on the native test sets of each dataset, it is important to note that the primary application of these classifiers is to perform cross-relabeling, where a model trained on one dataset (e.g., PanNuke) is applied to another (e.g., MoNuSAC) and vice versa. We acknowledge that a more systematic evaluation of cross-dataset generalization is needed and could be performed in future work.

Overall, the refined dataset produced by our approach can enhance the supervised training or fine-tuning of cell segmentation and classification models, especially those that utilize pre-trained foundation models to improve feature extraction robustness. In addition, these models can detect nuanced classes that enable researchers to conduct more detailed analyses of biological processes in computational pathology.

\section{Foundation models for robust cell segmentation and classification}

Accurate cell segmentation and classification in digital pathology are hindered by limited labeled data and the fact that conventional CNNs are unable to capture global contextual information due to their local receptive field constraints \cite{Gheflati_Rivaz_2022,Yang_Marcus_etal.}. Traditional approaches in cell quantification have predominantly relied on CNN encoders, such as ResNet50, given their proven effectiveness in semantic segmentation tasks \cite{Deshmane_2023,Graham_Vu_etal._2019,Mukasheva_Koishiyeva_etal._2024,Stringer_Wang_etal._2021}. However, approaches that include fine-tuning of pretrained CNNs, data augmentation, and stain normalization to partially increase data variability and address staining differences often fail to achieve the necessary generalization and robustness across diverse tissue types and staining conditions \cite{G._Wang_W._Li_etal._2018,Gao_Bagci_etal._2018,Karim_El_Khoury_Martin_Fockedey_etal._2021}.

To overcome these challenges, we leverage an encoder-decoder network that uses a foundation model as the encoder and a CNN upsampling decoder (\hyperref[fig:fig5]{Figure 5}) for simultaneous cell segmentation and classification in 2D patches extracted from WSIs. Foundation models with transformer-based architectures are viable alternatives to CNN-based encoders \cite{Shamshad_Khan_etal._2023,Sourget_2023}. They enable the creation of more advanced architectures that can decode or transform learned features more effectively \cite{Chen_Duan_etal._2023,Cheng_Misra_etal._2022,Xie_Wang_etal._2021}.

\begin{figure}[h!]
    \centering
    \includegraphics[width=\textwidth]{images/Figure_5.pdf}
    \caption{UNETR-like model with foundational model as backbone}
    \label{fig:fig5}
\end{figure}

By utilizing a transformer-based encoder, we incorporate global contextual information into the feature extraction process, which is a key advantage of such architectures \cite{Chen_Lu_etal._2021}. This foundation model integration facilitates accurate pixel-wise segmentation and classification without the need for extensive encoder training, thereby potentially improving generalization across varied cellular structures and tissue types.
In our implementation, we employ a modified UNETR \cite{Hatamizadeh_Tang_etal._2021} architecture that combines a vision transformer (ViT) \cite{Dosovitskiy_Beyer_etal._2021} encoder with a CNN-based decoder. The encoder utilizes the pretrained H-Optimus foundation model, which contains 1.1 billion parameters and is trained on over 500,000 H\&E stained WSIs \cite{Saillard_Jenatton_etal._2024}. We extract outputs from four evenly spaced transformer blocks $Z_i$, where $i \in [1, 14, 26, 38]$, to serve as residual connections for the CNN decoder. We select these blocks based on our observation that features from non-adjacent levels of the encoder lead to better overall performance on the test subset.

The CNN decoder upsamples the feature representations, acquired from the transformer blocks, to generate an intermediate vector that is handled by two task-specific layers that generate cell segmentation and classification masks. The first task-specific layer is the ‘Cellpose head’,  which is used to delineate cell instances. The layer generates horizontal and vertical gradient maps to form vector fields that are refined through gradient tracking in a post-processing step using the Cellpose algorithm \cite{Stringer_Wang_etal._2021}, known for its efficacy in cell segmentation tasks and generalizability across multiple domains \cite{Pachitariu_Stringer_2022,Stringer_Pachitariu_2024}. The second task-specific layer is the "Cell type head", which assigns labels to individual pixels. In the post-processing step, we determine the output classification label of each segmented cell instance by majority voting over the labeled pixels that comprise the cell in the segmentation map.

To evaluate model performance and measure the impact of adding a foundation model as backbone, we compare it to a ResNet50-based model. ResNet50 is a widely used solution for encoders in segmentation architectures in the medical domain \cite{Deshmane_2023,Graham_Vu_etal._2019,Mukasheva_Koishiyeva_etal._2024,Stringer_Wang_etal._2021}. For the H-Optimus-based model, we utilize frozen weights for the encoder and only fine-tune the decoder to take advantage of the extensive pre-training of the foundation model. For the ResNet50-based model we start with ImageNet \cite{Deng_Dong_etal.} weights and train both encoder and decoder parts. Hyperparameters for the training step are set to be identical, where possible, for comparable evaluation. 
For this evaluation, we deliberately use the PanNuke dataset to provide a standardized and controlled comparison between the H‑Optimus and ResNet50-based models (\hyperref[fig:S2]{Appendix Figure S2 (3)}). Specifically, we use two of the default PanNuke dataset splits (66\%) for training and validation, and reserve the third split (33\%) for testing.

To address the challenge of cell class imbalance in the PanNuke dataset, which is a common characteristic in most cell-level H\&E patch datasets, both models’ training processes employ a weighted loss function comprising cross-entropy and focal loss \cite{Lin_Goyal_etal._2018}. The focal loss component is adjusted with coefficients derived from each cell class' instance frequency, emphasizing learning from underrepresented classes and enhancing the model's sensitivity to rare but significant cellular patterns. The cross-entropy loss is augmented with spectral decoupling regularization \cite{Pezeshki_Kaba_etal._2021,Pohjonen_Stürenberg_etal._2022} and spatially varying label smoothing \cite{Islam_Glocker_2021}, which potentially stabilizes training and improves generalization in case of complex tissue morphologies. For optimization, we employ the \textit{AdamW} \cite{Loshchilov_Hutter_2019} to counter unbalanced class scenarios, with cosine annealing learning rate scheduler.

We utilize the scikit-learn library \cite{Van_der_Walt_Schönberger_etal._2014} and HoVer-Net \cite{Graham_Vu_etal._2019} implementations of $R^2$ (the coefficient of determination) and $PQ$ (panoptic quality) to evaluate our experiments. Complete mathematical formulations and detailed explanations of these metrics are provided in \hyperref[chap:S5]{Appendix S5}. To compute confidence intervals, we use nonparametric bootstrapping, where after calculating the metric on the full sample, we generated 1000 bootstrap replicates by resampling with replacement and then determined the 95\% confidence intervals as the 2.5th and 97.5th percentiles of the resulting empirical distribution.

%\hfill

The model comparisons are summarized in \hyperref[tab:2]{Table 2}. The H‑Optimus-based model achieves higher $R^2$ across all cell classes compared to the ResNet50-based model, which means that its predictions are more closely aligned with the PanNuke cell counts, indicating a stronger correlation with the observed data. Notably, the improvement of $R^2_{dead}$ may be an indicator of better global contextual representations provided by the foundation model backbone. In terms of segmentation and classification quality combined, measured by the PQ score, the H‑Optimus-based model demonstrates notable improvements across most cell classes. Overall, the average $R^2$ improved from 0.575 to 0.871, while the average $PQ$ score improved from 0.450 to 0.492, demonstrating better performance of the H-Optimus-based model.

\begin{table}[h!]
\renewcommand{\arraystretch}{1.5}
  \centering
  \caption{Cell quantification metrics for baseline and proposed models (CI 95\%).}
  \label{tab:2}
  \begin{tabular}{|l|c|c|}
    \hline
    %\rowcolor{gray!30}
    Metric             & Resnet50-based            & H-optimus-based              \\
    \hline
    $R^2_{neoplastic}$    & 0.681 (0.576--0.769)       & \textbf{0.941 (0.917--0.960)} \\
    \hline
    $R^2_{inflammatory}$  & 0.863 (0.778--0.903)       & \textbf{0.949 (0.918--0.966)} \\
    \hline
    $R^2_{connective}$    & 0.600 (0.488--0.698)       & 0.609 (0.436--0.772)          \\
    \hline
    $R^2_{dead}$          & 0.097 (-11.389--0.669)     & 0.925 (0.404--0.982)          \\
    \hline
    $R^2_{epithelial}$    & 0.635 (0.490--0.747)       & \textbf{0.930 (0.886--0.964)} \\
    \hline
    $PQ_{neoplastic}$       & 0.517 (0.499--0.535)       & \textbf{0.589 (0.575--0.604)} \\
    \hline
    $PQ_{inflammatory}$     & 0.455 (0.429--0.482)       & \textbf{0.528 (0.507--0.549)} \\
    \hline
    $PQ_{connective}$       & 0.416 (0.400--0.431)       & \textbf{0.451 (0.436--0.465)} \\
    \hline
    $PQ_{dead}$             & 0.374 (0.342--0.408)       & 0.292 (0.209--0.365)          \\
    \hline
    $PQ_{epithelial}$       & 0.488 (0.460--0.519)       & \textbf{0.599 (0.579--0.618)} \\
    \hline
  \end{tabular}
\end{table}

Our results  show that integrating the H‑Optimus foundation model within the UNETR architecture enhances the model's ability to segment and classify cells across diverse tissues from PanNuke data. The pretrained transformer encoder provides robust feature representations, resulting in higher average $R^2$ and $PQ$ scores compared to the CNN-based model. This leads to more reliable cell quantification and more accurate downstream analysis. Additionally, the streamlined fine-tuning process reduces computational overhead and training time, making the model more adaptable for new data.

Despite these advancements, the foundation model-based approach does not fully resolve all challenges related to cell segmentation and classification. We observe lower metric scores for underrepresented classes in the training data. Furthermore, foundation models typically encompass billions of parameters, resulting in substantial computational and memory requirements. It therefore poses challenges for deployment in resource-constrained environments, limiting their practical applicability in certain clinical settings.

\section{Model optimization via Knowledge Distillation}

To address the limitations posed by the extensive size of foundation models, we implement knowledge distillation — a model compression technique that leverages the teacher-student paradigm \cite{Hinton_Vinyals_etal._2015}. By training a smaller, more efficient student model to replicate the output of a larger, pre-trained teacher model, we retain performance while significantly reducing the model's complexity and resource requirements (\hyperref[fig:fig6]{Figure 6}).

\begin{figure}[h!]
    \centering
    \includegraphics[width=\textwidth, height=0.45\textheight, keepaspectratio]{images/Figure_6.pdf}
    \caption{Knowledge distillation framework for training a student model using a pre-trained teacher}
    \label{fig:fig6}
\end{figure}

We employ knowledge distillation to compress the H‑Optimus-based teacher model into a more efficient student model. The teacher model is the modified UNETR architecture with the H‑Optimus foundation model described in the previous chapter. The student model is based on a UNet architecture augmented with residual connections and incorporates a smaller ViT encoder with 9 million parameters \cite{Steiner_Kolesnikov_etal._2022,Wightman_2019}. 

First, we fine-tune the teacher model using the refined dataset from the cross-relabeling procedure (Section 2). Initially we train the decoder of the teacher model while keeping the encoder weights frozen. We split the refined dataset into train (70\%), validation (20\%) and test (10\%) subsets (\hyperref[fig:S2]{Appendix Figure S2 (4)}). During fine-tuning, we use the train and validation subsets, while leaving the test subset for model evaluation. We set the training procedure and model hyperparameters to be identical to those that were used to demonstrate the utility of foundation models for the simultaneous cell segmentation and classification task.

Next, we perform knowledge distillation from teacher to student using the refined dataset used to fine-tune the teacher model. The student model is trained to replicate the teacher model's outputs. We utilize a specialized loss function that aligns the student's predicted probability distribution with the teacher's, incorporating the teacher's class probability distribution derived from the output. Following the methodology of Hinton et al. \cite{Hinton_Vinyals_etal._2015}, we experiment with various hyperparameter settings for the temperature ($T$) and the balancing coefficients ($\alpha$ and $\beta$) in the loss function. We vary $T$ from 1 to 20 and adjust $\alpha$ and $\beta$ to balance the distillation and student losses. Through iterative tuning and evaluation, we identify that setting $T=14$, $\alpha=0.3$, and $\beta=0.7$ yields a configuration that converges and closely approximates the teacher model's performance during training.

Finally, we assess the performance of both models using the $R^2$ and $PQ$ (defined in \hyperref[chap:S5]{Appendix S5}) on the test set of the refined dataset (\hyperref[tab:3]{Table 3}). We observe that the 95\% confidence intervals overlap for most cell types, so we cannot claim statistically significant performance differences between the teacher and student models. One exception appears in the neoplastic class. The teacher model produces an $R^2$ of 0.919, while the student model shows an $R^2$ of 0.852. In addition, the student model achieves higher $PQ$ values for the neoplastic and connective classes, though the confidence intervals show overlap.

\begin{table}[h!]
\renewcommand{\arraystretch}{1.5}
  \centering
  \caption{Cell quantification metrics for teacher and distilled student models (CI 95\%).}
  \label{tab:3}
  \begin{tabular}{|l|c|c|}
    \hline
    %\rowcolor{gray!30}
    Metric & Teacher & Student \\
    \hline
    $R^2_{neoplastic}$    & \textbf{0.919} (0.898--0.939) & 0.852 (0.800--0.891) \\
    \hline
    $R^2_{lymphocyte}$    & 0.969 (0.956--0.977)         & 0.969 (0.956--0.978) \\
    \hline
    $R^2_{connective}$    & 0.694 (0.548--0.809)         & 0.618 (0.469--0.741) \\
    \hline
    $R^2_{dead}$          & 0.755 (0.400--0.908)         & 0.424 (0.100--0.731) \\
    \hline
    $R^2_{epithelial}$    & 0.922 (0.870--0.958)         & 0.843 (0.738--0.917) \\
    \hline
    $R^2_{macrophage}$    & 0.384 (-0.369--0.724)        & 0.704 (0.352--0.859) \\
    \hline
    $R^2_{neutrofil}$     & 0.854 (0.578--0.929)         & 0.833 (0.502--0.925) \\
    \hline
    $PQ_{neoplastic}$       & 0.581 (0.569--0.593)         & 0.601 (0.588--0.613) \\
    \hline
    $PQ_{lymphocyte}$       & 0.536 (0.520--0.553)         & 0.563 (0.544--0.579) \\
    \hline
    $PQ_{connective}$       & 0.436 (0.421--0.451)         & 0.457 (0.441--0.474) \\
    \hline
    $PQ_{dead}$             & 0.272 (0.235--0.315)         & 0.279 (0.201--0.369) \\
    \hline
    $PQ_{epithelial}$       & 0.522 (0.500--0.545)         & 0.530 (0.506--0.555) \\
    \hline
    $PQ_{macrophage}$       & 0.524 (0.459--0.588)         & 0.474 (0.405--0.543) \\
    \hline
    $PQ_{neutrofil}$        & 0.541 (0.490--0.592)         & 0.565 (0.522--0.607) \\
    \hline
  \end{tabular}
\end{table}


We further decompose the $PQ$ metric into its $SQ$ and $DQ$ components (\hyperref[tab:S6]{Appendix Table S6}). Both models produce nearly identical $SQ$ values, which indicates that they predict instance boundaries with similar precision. Although the student model shows some improvement in $DQ$ scores for certain classes, the confidence intervals overlap and do not confirm a statistically significant difference.

We observe that the student and teacher models yield comparable detection performance despite the student model using a much smaller and simpler architecture. A model with fewer parameters reduces the risk of overfitting when training data are scarce relative to the model’s complexity \cite{Farias_Ludermir_etal._2022}. The knowledge distillation process also encourages the student model to focus on the most generalizable detection features learned from the teacher. These factors enable the student model to achieve similar detection performance across different cell types.

Additionally, considering the model sizes reported in \hyperref[tab:4]{Table 4}, the distilled model achieves a significant reduction compared to the teacher model, with a 48-fold decrease in parameter count and a 5.5-fold reduction in on-disk size. In inference mode, the teacher model requires 16 GB of VRAM for a batch size of 32, while the distilled model only needs 3 GB of VRAM for the same batch size. These reductions make the distilled model significantly more practical for fine-tuning and deployment in resource-constrained environments.

\begin{table}[h!]
\renewcommand{\arraystretch}{1.5}
  \centering
  \caption{Parameter counts and size of teacher and distilled model}
  \label{tab:4}
  \adjustbox{max width=\textwidth}{%
  \begin{tabular}{|l|c|c|c|}
    \hline
    %\rowcolor{gray!30}
    Metric & H-optimus-based (Teacher) & mobileViT-based (Student) & Magnitude of difference \\
    \hline
    Parameters count       & 1,158,917,906   & \textbf{24,093,393}   & \textbf{48x}  \\
    \hline
    Estimated Total Size (MB) & 87,912       & \textbf{15,935}    & \textbf{5.5x} \\
    \hline
  \end{tabular}%
}
\end{table}

%\hfill

With recent advancements in complex network architectures and the use of pretrained encoders to achieve state-of-the-art performance \cite{Baumann_Dislich_etal._2024,Hörst_Rempe_etal._2024} in cell segmentation and classification tasks, model size, computational complexity, and processing times have increased. This limits the scalability and accessibility of these models. As we demonstrate, this may be mitigated using knowledge distillation. Studies in the field of natural language processing have demonstrated the efficacy of knowledge distillation in retaining the capabilities of the teacher model while achieving significant reductions in size and complexity \cite{Huangpu_Gao_2024,Sun_Yu_etal.}. 

We demonstrate the feasibility of knowledge distillation in digital pathology, specifically for cell segmentation and classification tasks. Moreover, we achieve this performance while also significantly reducing the parameter count. In addressing the challenge of knowledge transfer, we found that distillation from a transformer-based model to a smaller transformer is more straightforward than attempting to map transformer features to CNN blocks. In our experiments, using a CNN-based network as a student results in worse cell quantification performance due to the structural constraints of CNN feature space dimensions. 

Although our primary approach relies on a transformer-based student model that performs well, it can be further optimized to incorporate advantages from CNN architectures. For example, employing alternative techniques such as using ViT adapters \cite{Chen_Duan_etal._2023} or $1 \times 1$ convolutions to adjust feature map sizes may be beneficial for harnessing CNN advantages like enhanced local feature extraction. Moreover, if additional performance improvements are desired, the process can be further enhanced by applying supplementary knowledge distillation techniques, such as self-distillation \cite{Zhang_Song_etal._2019} or online distillation \cite{Houyon_Cioppa_etal._2023}.

Despite these promising results, further validation on independent datasets is necessary to fully understand the model's limitations. Underrepresented classes may pose challenges when addressing complex cases. Pathologists need to validate these models to adopt them in clinical settings. While the distilled models are smaller and more deployable, a technological gap persists because pathologists traditionally rely on established methods for inspecting WSIs and diagnosing diseases. Addressing the complexities involved in deploying models for inference and supporting pathologists in adopting new tools is essential for integrating these models into clinical workflows.

\section{Model integration with QuPath}
Digital pathology tools with graphical user interfaces are essential for visualizing and analyzing WSIs. To make our student model useful in clinical pathology workflows, it needs to be integrated into a tool that enables inspecting regions, creating annotations, and providing quantitative analyses of biomarkers. Therefore, we integrate the trained student model from the previous chapter into the QuPath open‑source platform \cite{Bankhead_Loughrey_etal._2017}. QuPath provides the required annotation, visualization, and analysis tools to interpret complex histological data, including workflows for cell segmentation, classification, and quantification (\hyperref[fig:fig7]{Figure 7}). 

\begin{figure}[h!]
    \centering
    \includegraphics[width=\textwidth]{images/Figure_7.pdf}
    \caption{Visualization of model-generated cell quantification annotations (left) and the corresponding unannotated slide (right) in QuPath}
    \label{fig:fig7}
\end{figure}

To identify the regions in a WSI critical for prognosticating tumor development, such as specific tumor areas or border regions without overlapping healthy tissue, the pathologist uses QuPath to outline these regions. Then, the pathologist initiates a cell segmentation and classification script through the QuPath interface for the selected regions. The resulting annotations and quantified cell information are then directly overlaid onto the WSI in the QuPath interface. Additional design and implementation details are in \hyperref[chap:S7]{Appendix S7}. 

Two common approaches for integrating deep learning models into QuPath are Java‑based native QuPath extensions \cite{Goldsborough_Philps_etal._2024} and the execution of RESTful API requests to a model server coupled with handling the response via an extension, as demonstrated in the application of cell segmentation models applied to immunofluorescence images \cite{Sugawara_2023}. While the community is actively working on these integration strategies, there is currently no universal solution that fully addresses all integration and performance requirements.

Extensions may offer better integration with QuPath, allowing slightly improved performance and more widespread usage of the built-in QuPath models, but they lack the flexibility to customize models and modify their behavior. For example, the newest version of QuPath includes models such as StarDist \cite{Weigert_Schmidt} and InstanSeg \cite{Goldsborough_Philps_etal._2024} that can perform cell segmentation. Both models pose limitations when applied to simultaneous cell segmentation and classification. StarDist performs well only on convex, round shapes by design, whereas some neoplastic, inflammatory, and connective cells exhibit complex and non-convex shapes. InstanSeg provides only semantic segmentation without assigning classes to the segmented cells.

%\hfill

In contrast, our approach offers an alternative integration strategy. It utilizes the paquo library to directly interact with QuPath’s internal application programming interface from within Python. This enables data exchange and processing without the need for intermediate conversion steps and provides greater control over model customization, retraining, and the incorporation of custom processing steps.

The integration of our custom model with QuPath underscores its potential to significantly enhance the diagnostic process by reducing the time burden on pathologists and enabling them to focus on more complex interpretative tasks using familiar software. Leveraging a tool that is already well-established among pathologists increases the likelihood of its adoption into daily clinical workflows. The quantitative data generated through the automated workflow is critical for both clinical decision-making and research, facilitating more accurate biomarker analysis, enabling robust statistical evaluations, and supporting hypothesis generation and testing. Additionally, by streamlining cell segmentation and classification, the tool enhances the scalability and reproducibility of pathological assessments, ultimately contributing to improved diagnostic accuracy and patient outcomes.

\section{Conclusion and future work}

In this study, we address critical challenges in digital pathology and tackle the usability and deployment issues of the developed models in standard computing environments without the need for high-performance computing systems. Our multi-faceted approach encompasses data refinement through cross-relabeling, leveraging foundation models for robust cell segmentation and classification, optimizing model performance via knowledge distillation, and integrating the optimized model into the QuPath software for practical application. This approach is used to construct a capable, versatile, and adjustable model for cell segmentation and classification, with enhanced performance and usability.

\begin{sloppypar}
While our approach shows potential in the field of computational pathology, certain limitations persist. 
For example, our implementation currently exhibits lower performance in detecting macrophages. 
This serves as an instance of the broader challenge of accurately identifying complex cell types. In order to address this issue, extending our approach to incorporate additional data sources, exploring alternative modeling approaches, and integrating other imaging modalities such as immunohistochemical staining may help improve detection accuracy. Moreover, although the distilled model reduces computational demands, integrating advanced deep learning models into clinical practice requires addressing technological gaps and potential resistance to adopting new tools within established diagnostic processes.
\end{sloppypar}

Future work could focus on several key areas to refine the proposed approach and facilitate its adoption in clinical environments. Enhancing the cell-relabeling process with additional datasets \cite{Graham_Jahanifar_etal._2021} could improve the representation of underrepresented cell types and enhance overall model performance. Also, incorporating additional data sources, such as multi-modal imaging or complementary staining methods, may address limitations related to cell type differentiation and class imbalance. Exploring other foundation models \cite{Vorontsov_Bozkurt_etal._2024,Zimmermann_Vorontsov_etal._2024} or introducing additional modalities \cite{Ding_Wagner_etal._2024,Vaidya_Zhang_etal._2025} may provide alternative architectures better suited to specific tasks or offer improved efficiency. Implementing more complex knowledge distillation techniques \cite{Houyon_Cioppa_etal._2023,Zhang_Song_etal._2019} could further optimize the model's performance and adaptability. Additionally, deeper integration with QuPath or other digital pathology software could provide pathologists more control over cell quantification analysis directly within the QuPath interface, thereby increasing accessibility and usability. Such enhancements would not only refine model performance but also ensure greater adaptability and scalability within various clinical environments. Finally, extensive validation of the model by pathologists and benchmarking against independent datasets are essential steps toward establishing the model's reliability and fostering confidence in its clinical utility.

\section*{Acknowledgments} 
This work was funded in part by the Research Council of Norway grant no. 309439 SFI Visual Intelligence, and the North Norwegian Health Authority grant no. HNF1521-20.

\bibliographystyle{IEEEtran}
\begin{sloppypar}
\begin{thebibliography}{99}

\bibitem{chaplot2020neural} Chaplot, Devendra Singh, et al. "Neural topological slam for visual navigation." Proceedings of the IEEE/CVF conference on computer vision and pattern recognition. 2020.

\bibitem{maksymets2021thda} Maksymets, Oleksandr, et al. "Thda: Treasure hunt data augmentation for semantic navigation." Proceedings of the IEEE/CVF International Conference on Computer Vision. 2021.

\bibitem{mezghan2022memory} Mezghan, Lina, et al. "Memory-augmented reinforcement learning for image-goal navigation." 2022 IEEE/RSJ International Conference on Intelligent Robots and Systems (IROS). IEEE, 2022.

\bibitem{al2022zero} Al-Halah, Ziad, Santhosh Kumar Ramakrishnan, and Kristen Grauman. "Zero experience required: Plug \& play modular transfer learning for semantic visual navigation." Proceedings of the IEEE/CVF Conference on Computer Vision and Pattern Recognition. 2022.

\bibitem{ye2021auxiliary} Ye, Joel, et al. "Auxiliary tasks and exploration enable objectgoal navigation." Proceedings of the IEEE/CVF international conference on computer vision. 2021.

\bibitem{chaplot2020object} Chaplot, Devendra Singh, et al. "Object goal navigation using goal-oriented semantic exploration." Advances in Neural Information Processing Systems 33 (2020)

\bibitem{ramakrishnan2022poni} Ramakrishnan, Santhosh Kumar, et al. "Poni: Potential functions for objectgoal navigation with interaction-free learning." Proceedings of the IEEE/CVF Conference on Computer Vision and Pattern Recognition. 2022.

\bibitem{ramrakhya2022habitat} Ramrakhya, Ram, et al. "Habitat-web: Learning embodied object-search strategies from human demonstrations at scale." Proceedings of the IEEE/CVF Conference on Computer Vision and Pattern Recognition. 2022.

\bibitem{mousavian2019visual} Mousavian, Arsalan, et al. "Visual representations for semantic target driven navigation." 2019 International Conference on Robotics and Automation (ICRA). IEEE, 2019.

\bibitem{dhariwal2021diffusion} Dhariwal, Prafulla, and Alexander Nichol. "Diffusion models beat gans on image synthesis." Advances in neural information processing systems 34 (2021)

\bibitem{ho2022classifier} Ho, Jonathan, and Tim Salimans. "Classifier-free diffusion guidance." arXiv preprint arXiv:2207.12598 (2022).

\bibitem{nichol2021glide} Nichol, Alex, et al. "Glide: Towards photorealistic image generation and editing with text-guided diffusion models." arXiv preprint arXiv:2112.10741 (2021)

\bibitem{brooks2023instructpix2pix} Brooks, Tim, Aleksander Holynski, and Alexei A. Efros. "Instructpix2pix: Learning to follow image editing instructions." Proceedings of the IEEE/CVF Conference on Computer Vision and Pattern Recognition. 2023.

\bibitem{fu2023guiding} Fu, Tsu-Jui, et al. "Guiding instruction-based image editing via multimodal large language models." arXiv preprint arXiv:2309.17102 (2023).

\bibitem{geng2024instructdiffusion} Geng, Zigang, et al. "Instructdiffusion: A generalist modeling interface for vision tasks." Proceedings of the IEEE/CVF Conference on Computer Vision and Pattern Recognition. 2024.

\bibitem{zhou2024minedreamer} Zhou, Enshen, et al. "Minedreamer: Learning to follow instructions via chain-of-imagination for simulated-world control." arXiv preprint arXiv:2403.12037 (2024).

\bibitem{zhou2023esc} Zhou, Kaiwen, et al. "Esc: Exploration with soft commonsense constraints for zero-shot object navigation." International Conference on Machine Learning. PMLR, 2023.

\bibitem{yu2023l3mvn} Yu, Bangguo, Hamidreza Kasaei, and Ming Cao. "L3mvn: Leveraging large language models for visual target navigation." 2023 IEEE/RSJ International Conference on Intelligent Robots and Systems (IROS). IEEE, 2023.

\bibitem{gadre2023cows} Gadre, Samir Yitzhak, et al. "Cows on pasture: Baselines and benchmarks for language-driven zero-shot object navigation." Proceedings of the IEEE/CVF Conference on Computer Vision and Pattern Recognition. 2023.

\bibitem{shah2023navigation} Shah, Dhruv, et al. "Navigation with large language models: Semantic guesswork as a heuristic for planning." Conference on Robot Learning. PMLR, 2023.

\bibitem{cai2024bridging} Cai, Wenzhe, et al. "Bridging zero-shot object navigation and foundation models through pixel-guided navigation skill." 2024 IEEE International Conference on Robotics and Automation (ICRA). IEEE, 2024.

\bibitem{yu2023co} Yu, Bangguo, Hamidreza Kasaei, and Ming Cao. "Co-NavGPT: Multi-robot cooperative visual semantic navigation using large language models." arXiv preprint arXiv:2310.07937 (2023).

\bibitem{wu2024voronav} Wu, Pengying, et al. "Voronav: Voronoi-based zero-shot object navigation with large language model." arXiv preprint arXiv:2401.02695 (2024).

\bibitem{qin2023mp5} Qin, Yiran, et al. "Mp5: A multi-modal open-ended embodied system in minecraft via active perception." arXiv preprint arXiv:2312.07472 (2023).

\bibitem{du2024learning} Du, Yilun, et al. "Learning universal policies via text-guided video generation." Advances in Neural Information Processing Systems 36 (2024).

\bibitem{ajay2024compositional} Ajay, Anurag, et al. "Compositional foundation models for hierarchical planning." Advances in Neural Information Processing Systems 36 (2024).

\bibitem{liang2024skilldiffuser} Liang, Zhixuan, et al. "Skilldiffuser: Interpretable hierarchical planning via skill abstractions in diffusion-based task execution." Proceedings of the IEEE/CVF Conference on Computer Vision and Pattern Recognition. 2024.

\bibitem{heusel2017gans} Heusel, Martin, et al. "Gans trained by a two time-scale update rule converge to a local nash equilibrium." Advances in neural information processing systems 30 (2017).

\bibitem{zhang2018unreasonable} Zhang, Richard, et al. "The unreasonable effectiveness of deep features as a perceptual metric." Proceedings of the IEEE conference on computer vision and pattern recognition. 2018.

\bibitem{brown2020language} Brown, Tom B. "Language models are few-shot learners." arXiv preprint arXiv:2005.14165 (2020).

\bibitem{podell2023sdxl} Podell, Dustin, et al. "Sdxl: Improving latent diffusion models for high-resolution image synthesis." arXiv preprint arXiv:2307.01952 (2023).

\bibitem{brohan2022rt} Brohan, Anthony, et al. "Rt-1: Robotics transformer for real-world control at scale." arXiv preprint arXiv:2212.06817 (2022).

\bibitem{brohan2023rt} Brohan, Anthony, et al. "Rt-2: Vision-language-action models transfer web knowledge to robotic control." arXiv preprint arXiv:2307.15818 (2023).

\bibitem{li2024manipllm} Li, Xiaoqi, et al. "Manipllm: Embodied multimodal large language model for object-centric robotic manipulation." Proceedings of the IEEE/CVF Conference on Computer Vision and Pattern Recognition. 2024.

\bibitem{shah2023vint} Shah, Dhruv, et al. "ViNT: A foundation model for visual navigation." arXiv preprint arXiv:2306.14846 (2023).

\bibitem{liu2024visual} Liu, Haotian, et al. "Visual instruction tuning." Advances in neural information processing systems 36 (2024).

\bibitem{hu2021lora} Hu, Edward J., et al. "Lora: Low-rank adaptation of large language models." arXiv preprint arXiv:2106.09685 (2021).

\bibitem{qin2023supfusion} Qin, Yiran, et al. "SupFusion: Supervised LiDAR-camera fusion for 3D object detection." Proceedings of the IEEE/CVF International Conference on Computer Vision. 2023.

\bibitem{qin2024worldsimbench} Qin, Yiran, et al. "Worldsimbench: Towards video generation models as world simulators." arXiv preprint arXiv:2410.18072 (2024).

\bibitem{yu2025gamefactory} Yu, Jiwen, et al. "GameFactory: Creating New Games with Generative Interactive Videos." arXiv preprint arXiv:2501.08325 (2025).

\bibitem{zhou2024code} Zhou, Enshen, et al. "Code-as-Monitor: Constraint-aware Visual Programming for Reactive and Proactive Robotic Failure Detection." arXiv preprint arXiv:2412.04455 (2024).

\bibitem{zhang2024ad} Zhang, Zaibin, et al. "AD-H: Autonomous Driving with Hierarchical Agents." arXiv preprint arXiv:2406.03474 (2024).

\bibitem{wang2024toward} Wang, Chaoqun, et al. "Toward Accurate Camera-based 3D Object Detection via Cascade Depth Estimation and Calibration." arXiv preprint arXiv:2402.04883 (2024).

\bibitem{huang2024story3d} Huang, Yuzhou, et al. "Story3d-agent: Exploring 3d storytelling visualization with large language models." arXiv preprint arXiv:2408.11801 (2024).

\bibitem{savinov2018semi} Savinov, Nikolay, Alexey Dosovitskiy, and Vladlen Koltun. "Semi-parametric topological memory for navigation." arXiv preprint arXiv:1803.00653 (2018).

\bibitem{majumdar2022zson} Majumdar, Arjun, et al. "Zson: Zero-shot object-goal navigation using multimodal goal embeddings." Advances in Neural Information Processing Systems 35 (2022): 32340-32352.

\bibitem{yadav2023offline} Yadav, Karmesh, et al. "Offline visual representation learning for embodied navigation." Workshop on Reincarnating Reinforcement Learning at ICLR 2023. 2023.

\bibitem{yadav2023ovrl} Yadav, Karmesh, et al. "Ovrl-v2: A simple state-of-art baseline for imagenav and objectnav." arXiv preprint arXiv:2303.07798 (2023).

\bibitem{sun2024fgprompt} Sun, Xinyu, et al. "FGPrompt: fine-grained goal prompting for image-goal navigation." Advances in Neural Information Processing Systems 36 (2024).

\bibitem{zhu2017target} Zhu, Yuke, et al. "Target-driven visual navigation in indoor scenes using deep reinforcement learning." 2017 IEEE international conference on robotics and automation (ICRA). IEEE, 2017.

\bibitem{koh2024generating} Koh, Jing Yu, Daniel Fried, and Russ R. Salakhutdinov. "Generating images with multimodal language models." Advances in Neural Information Processing Systems 36 (2024).

\bibitem{krantz2022instance} Krantz, Jacob, et al. "Instance-specific image goal navigation: Training embodied agents to find object instances." arXiv preprint arXiv:2211.15876 (2022).

\bibitem{schulman2017proximal} Schulman, John, et al. "Proximal policy optimization algorithms." arXiv preprint arXiv:1707.06347 (2017).

\bibitem{anderson2018evaluation} Anderson, Peter, et al. "On evaluation of embodied navigation agents." arXiv preprint arXiv:1807.06757 (2018).

\bibitem{lin2024navcot} Lin, Bingqian, et al. "NavCoT: Boosting LLM-Based Vision-and-Language Navigation via Learning Disentangled Reasoning." arXiv preprint arXiv:2403.07376 (2024).

\bibitem{NavGPT} Zhou, Gengze, Yicong Hong, and Qi Wu. "Navgpt: Explicit reasoning in vision-and-language navigation with large language models." Proceedings of the AAAI Conference on Artificial Intelligence.

\bibitem{hahn2021no} Hahn, Meera, et al. "No rl, no simulation: Learning to navigate without navigating." Advances in Neural Information Processing Systems 34 (2021): 26661-26673.

\bibitem{li2025t2isafety} Li, Lijun, et al. "T2ISafety: Benchmark for Assessing Fairness, Toxicity, and Privacy in Image Generation." arXiv preprint arXiv:2501.12612 (2025).

\bibitem{an2024agfsync} An, Jingkun, et al. "AGFSync: Leveraging AI-Generated Feedback for Preference Optimization in Text-to-Image Generation." arXiv preprint arXiv:2403.13352 (2024).


\end{thebibliography}
\end{sloppypar}

\clearpage
\beginsupplement
\section*{Appendix}
\renewcommand{\thesubsection}{S\arabic{subsection}}

\subsection{\label{chap:S1}PanNuke and MoNuSAC preprocessing}
The PanNuke dataset comprises a set of 7,901 RGB patches, each with dimensions of $256 \times 256$ pixels, which we set as the standard patch size for our analysis. In contrast, the MoNuSAC dataset encompasses 294 images of heterogeneous dimensions. To standardize the MoNuSAC images with our experiments, we implement a standardization protocol. Specifically, for images exceeding the dimensions of $256 \times 256$ pixels, we segment them into equal-sized patches and apply mirror padding to the remaining portions to avoid information loss at the peripherals. Patches with dimensions less than $128 \times 128$ pixels are excluded from the dataset due to the insufficient resolution to capture relevant cellular details. For patches where either dimension falls between 128 and 256 pixels, we employ upsampling to achieve the standard patch size. As a result, we obtain a total of 2,823 RGB patches derived from the MoNuSAC dataset for subsequent analysis. For additional details on the MoNuSAC data preparation process, refer to the source code \cite{Shvetsov_2025a}.
\clearpage

\subsection{\label{chap:S2}Data usage for the methodology}

\counterwithin{figure}{subsection}
\renewcommand{\thefigure}{S\arabic{subsection}}

\begin{figure}[h!]
    \centering
    \includegraphics[width=\textwidth, height=0.85\textheight, keepaspectratio]{images/A2.pdf}
    \caption{Overview of the methodology for cross-labeling, dataset refinement, and model comparison. (1) Cross-relabeling - training and testing cell classification models, (2) Cross-relabeling - using cell classification models to create refined dataset, (3) Fine-tuning and training models for comparison, (4) Student knowledge distillation with refined dataset}
    \label{fig:S2}
\end{figure}
\clearpage

\subsection{\label{chap:S3}Confusion matrices for classification models}
\counterwithin{figure}{subsection}
\renewcommand{\thefigure}{S\arabic{subsection}.\arabic{figure}}

\begin{figure}[h!]
    \centering
    \includegraphics[width=\textwidth, height=0.4\textheight, keepaspectratio]{images/A3_1.pdf}
    \caption{Confusion matrix for PanNuke trained model}
    \label{fig:S3.1}
\end{figure}

\begin{figure}[h!]
    \centering
    \includegraphics[width=\textwidth, height=0.4\textheight, keepaspectratio]{images/A3_2.pdf}
    \caption{Confusion matrix for MoNuSAC trained model}
    \label{fig:S3.2}
\end{figure}

\clearpage

\subsection{\label{chap:S4}Datasets cell counts}

\counterwithin{table}{subsection}
\renewcommand{\thetable}{S\arabic{subsection}}

\begin{table}[h!]
\renewcommand{\arraystretch}{2.0}
\centering
\caption{\label{tab:S4}Cell counts for PanNuke, MoNuSAC and refined datasets. Numbers in parentheses indicate preprocessed cell counts for cell classifier models training and testing.}
%\adjustbox{max width=\textwidth}{%
\begin{tabular}{|l|c|c|c|}
\hline
%\rowcolor{gray!30}
Cell type & PanNuke & MoNuSAC & Refined \\
\hline
Neoplastic & 77,403 (68,031) & - & 105,451 \\
\hline
Epithelial & 26,572 (23,207) & - & 29,926 \\
\hline
Epithelial (benign and malignant) & - & 31,402 & - \\
\hline
Inflammatory & 32,276 & - & - \\
\hline
Lymphocytes & - & 37,045 (33,104) & 65,275 \\
\hline
Neutrophils & - & 1,355 (1,252) & 3,833 \\
\hline
Macrophage & - & 1,842 (1,695) & 3,410 \\
\hline
Dead & 2,908 & - & 2,908 \\
\hline
Connective & 50,585 & - & 50,585 \\
\hline
\end{tabular}
%
%}
\end{table}



\clearpage

\subsection{\label{chap:S5}Definition of validation metrics}
\counterwithin{equation}{subsection}
\renewcommand{\theequation}{\arabic{equation}}

\subsubsection{\label{chap:S5.1}R\textsuperscript{2}}
The coefficient of determination, denoted as $R^2$, is a statistical measure that represents the proportion of variance in the dependent variable that is predictable from the independent variables. In the context of cell quantification in pathology, $R^2$ is used to assess how well the predicted quantities of different cell types in a patch align with the actual quantities observed in the ground truth data, with higher values representing more accurate quantification. $R^2$ is defined as
\begin{equation*}
R^2 = 1 - \frac{\sum_{i=1}^n (y_i - \hat{y}_i)^2}{\sum_{i=1}^n (y_i - \bar{y})^2},
\end{equation*}
where $y_i$ represents the actual number of cells of a specific type in the $i$-th image, $\hat{y}_i$ represents the predicted number of cells of that type in the $i$-th image, $\bar{y}$ is the mean of the actual numbers across all images, and $n$ is the total number of images in the dataset.

The $R^2$ metric has a range of $(-\infty, 1]$. An $R^2$ of 1 indicates perfect prediction, where all predicted values exactly match the actual values. An $R^2$ of 0 suggests that the model explains none of the variability of the response data around its mean. If $R^2$ is negative, it indicates that the model performs worse than a model that simply predicts the mean of the actual values for all observations.

\subsubsection{\label{chap:S5.2}PQ}
Panoptic Quality ($PQ$) is a comprehensive metric used to evaluate the performance of segmentation models in tasks that require both instance segmentation and classification. $PQ$ provides a single score that encapsulates both the detection accuracy (i.e., how many objects were correctly identified) and the segmentation quality (i.e., how accurately the objects' boundaries were delineated). This metric is particularly useful in multiclass scenarios where each pixel is classified into distinct categories, such as different cell types in pathology images.

$PQ$ is calculated as the product of two terms: Detection Quality ($DQ$) and Segmentation Quality ($SQ$). It can be expressed as
\begin{equation*}
PQ = DQ \cdot SQ,
\end{equation*}
where
\begin{equation*}
DQ = \frac{TP}{TP + 0.5\, FP + 0.5\, FN},
\end{equation*}
\begin{equation*}
SQ = \frac{\sum_{(p, g) \in \mathcal{M}} IoU(p, g)}{TP}.
\end{equation*}
In these formulas, $TP$ denotes the number of correctly matched instances between ground truth and prediction, $FP$ denotes the predicted instances that have no corresponding ground truth, $FN$ denotes the ground truth instances that were not detected, $IoU(p, g)$ is the Intersection over Union for a pair of matched instances $p$ (prediction) and $g$ (ground truth), and $\mathcal{M}$ is the set of matched pairs.

The $PQ$ metric is calculated for each class and is averaged across classes to provide a global performance measure.

The $PQ$ score has a range of $[0, 1.0]$, where a higher score indicates better performance in both detecting and segmenting the instances correctly. A $PQ$ of 1 signifies perfect identification and segmentation of all instances, whereas a $PQ$ of 0 indicates that no instances were correctly identified and segmented.

\clearpage

\subsection{\label{chap:S6}Segmentation and Detection quality metrics for teacher and student models}

\begin{table}[h!]
\renewcommand{\arraystretch}{2.0}
\centering
\caption{Segmentation and detection quality for student and teacher models (CI 95\%)}
\label{tab:S6}
%\adjustbox{max width=\textwidth}{%
\begin{tabular}{|l|c|c|}
\hline
%\rowcolor{gray!30}
Metric & Teacher & Student \\
\hline
$SQ_{neoplastic}$ & 0.819 (0.815--0.823) & 0.824 (0.819--0.828) \\
\hline
$SQ_{lymphocyte}$ & 0.795 (0.788--0.802) & 0.790 (0.783--0.796) \\
\hline
$SQ_{connective}$ & 0.770 (0.762--0.776) & 0.780 (0.772--0.786) \\
\hline
$SQ_{dead}$ & 0.659 (0.623--0.688) & 0.657 (0.624--0.695) \\
\hline
$SQ_{epithelial}$ & 0.780 (0.770--0.790) & 0.788 (0.779--0.797) \\
\hline
$SQ_{macrophage}$ & 0.788 (0.760--0.810) & 0.757 (0.730--0.783) \\
\hline
$SQ_{neutrofil}$ & 0.782 (0.761--0.801) & 0.775 (0.759--0.792) \\
\hline
$DQ_{neoplastic}$ & 0.706 (0.692--0.719) & 0.727 (0.712--0.741) \\
\hline
$DQ_{lymphocyte}$ & 0.675 (0.656--0.698) & 0.713 (0.691--0.734) \\
\hline
$DQ_{connective}$ & 0.566 (0.546--0.584) & 0.583 (0.565--0.602) \\
\hline
$DQ_{dead}$ & 0.410 (0.361--0.465) & 0.435 (0.306--0.561) \\
\hline
$DQ_{epithelial}$ & 0.668 (0.639--0.694) & 0.673 (0.644--0.702) \\
\hline
$DQ_{macrophage}$ & 0.657 (0.583--0.727) & 0.615 (0.531--0.703) \\
\hline
$DQ_{neutrofil}$ & 0.691 (0.625--0.753) & 0.729 (0.679--0.778) \\
\hline
\end{tabular}
%
%}
\end{table}

\clearpage

\subsection{\label{chap:S7}QuPath integration method}
We adopt an integration strategy leveraging the paquo \cite{Bayer_AG} library, a Python package that enables direct interaction with QuPath’s internal API, thereby facilitating seamless data exchange without intermediate conversion steps. The data processing pipeline (\hyperref[fig:S7]{Appendix Figure S7}) begins with the acquisition of WSIs and their associated annotations from QuPath, which are represented as Shapely \cite{Gillies_Wel_etal._2024} polygons. Utilizing paquo, we directly read, create, and modify these annotations and detections within a QuPath project in the Python environment. Images are then cropped using these polygons and processed by cell segmentation and classification models employing standard vision processing toolkits such as OpenCV, pyvips, and PyTorch. Additionally, QuPath employs Groovy scripts to initiate a Python process that starts the entire pipeline from QuPath graphical interface: fetching polygons, extracting images from them, and running deep learning model inference on the cropped images. 
The results are returned to QuPath, leveraging paquo's Python bindings to manipulate QuPath data while minimizing the computational overhead typically associated with cross-environment communication.

\counterwithin{figure}{subsection}
\renewcommand{\thefigure}{S\arabic{subsection}}

\begin{figure}[h!]
    \centering
    \includegraphics[width=\textwidth]{images/A7.pdf}
    \caption{QuPath integration workflow using Python environment}
    \label{fig:S7}
\end{figure}

Compared to traditional workflows that involve exporting annotations as GeoJSON, classifying them in Python, and reimporting them into QuPath, our approach offers several advantages. We eliminate the need to switch between programming languages, providing a cohesive and streamlined development process entirely within QuPath software and removing the necessity to use other tools. Meanwhile, we avoid storing annotations as intermediate JSON files unless required for external use or archiving. By conducting the entire inference and post-processing workflow within the Python environment, we leverage the power and flexibility of Python libraries for image processing and machine learning. This approach also enables adjustments to any set of labels and models, thereby improving its applicability.

%\hfill

The distilled model and QuPath integration code are packaged into a Docker container, enabling streamlined execution with the Docker engine. Detailed integration code and deployment instructions can be found in the GitHub repository \cite{Shvetsov_2025b}.

Despite these benefits, we acknowledge that the paquo library is a proof‑of‑concept project in its early development stage and has not been tested across all versions of QuPath.

\clearpage

\subsection{\label{chap:S8}Data and code availability statement}
All datasets, models, and code used in this study are publicly available and can be obtained from the repositories listed below. 
The PanNuke \cite{Gamper_Koohbanani_etal._2019} and MoNuSAC \cite{Verma_Kumar_etal._2021} datasets are publicly accessible, and download information along with detailed descriptions can be found in their respective articles. Preprocessing scripts for PanNuke and MoNuSAC data, as well as individual cell extraction scripts, are available on GitHub \cite{Shvetsov_2025a}. The H-Optimus foundation model used in our experiments can be downloaded from the HuggingFace repository \cite{hoptimus2024}, and model information is available on GitHub \cite{Saillard_Jenatton_etal._2024}. In addition, the integration code for QuPath and the distilled model packaged in a Docker container are provided in the repository \cite{Shvetsov_2025b}, and paquo Python library is available from the authors GitHub repository \cite{Bayer_AG}.
\clearpage

\end{document}

\end{document}

