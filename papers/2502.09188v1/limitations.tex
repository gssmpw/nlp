\section{Limitations}
While our Persian corpus represents a significant step forward in providing high-quality data, there are several limitations to be noted:

\textbf{Sub-Document Level Redundancies:} Although we applied deduplication at the document level, we did not perform deduplication within documents, meaning there may be redundancies at the sentence or paragraph level. This limitation arises from the high memory and computational resources required to encode and compare sections of all documents. Unfortunately, we did not have the resources necessary to conduct this process at a finer granularity.

\textbf{Sensitive Content and Language:} Despite selecting Persian websites with minimal adult content and removing sensitive data from public datasets, some sensitive material and inappropriate language remain, particularly in social media data. We did not filter out offensive or explicit language, as it reflects real-world language use. However, researchers utilizing the dataset should be mindful of this content when applying it in their work.

\textbf{Residual Irrelevant Data:} While we inspected samples from all data sources and employed various heuristics and filtering functions to remove irrelevant content, such as links, hashtags, advertisements, and tags, some may have evaded our processes. These elements are generally considered noise given the large scale of the dataset but may need to be addressed for more specialized use cases.

These limitations highlight potential areas for improvement, especially for projects with specific needs regarding data quality and sensitivity.