\section{Matina Corpus}
% The Matina corpus is built from a variety of data sources, each of which is processed based on its specific content characteristics. While these sources are divided into three main categories, the general preprocessing steps remain consistent, as shown in Figure~\ref{fig:pipeline}, with variations mainly in hyperparameters. However, certain sources require extra cleaning steps, which we describe in detail in their respective sections. In this section, we outline the data collection process, the applied preprocessing techniques, and the rationale behind the decisions made during these steps.

The Matina corpus is built from a variety of data sources, each of which is processed based on its specific content characteristics. Although these sources are grouped into three main categories, the overall preprocessing pipeline remains consistent, as depicted in Figure~\ref{fig:pipeline}, with variations primarily in hyperparameters. Certain sources, however, demand additional cleaning steps, which are detailed in their respective sections.

Figure~\ref{fig:boxplot1} visualizes the distribution of token counts across documents from each source, using a box plot to illustrate the variance in document length. These three categories—web-based crawled data, crawled books and papers, and social media—form the core of our dataset, each with distinct preprocessing requirements. In this section, we describe the data collection process, the preprocessing techniques applied, and the rationale behind the decisions made throughout these steps.

\begin{figure}[t]
  \includegraphics[trim=2.5cm 4cm 2.2cm 1.8cm , clip,width=\columnwidth]{./processingPipeline.pdf}
  \caption{The overall stages of processing pipeline of Matina Corpus.}
  \label{fig:pipeline}

\end{figure} 

\subsection{Web-based Crawled Data}
Web crawling is a common and efficient method for collecting data in any language. Websites offer a vast range of valuable information and, given their structured nature and wide availability, can largely be crawled automatically. As a result, web data is frequently used as the primary source for constructing large-scale text datasets. However, while the bulk collection of web data is straightforward, extracting meaningful content from irrelevant elements such as metadata, advertisements, and embedded links remains challenging. Web pages often contain spam-like elements, which complicates the cleaning process and increases the likelihood of errors.

Most web-based datasets begin with basic steps such as text extraction and language detection, often followed by optional URL filtering to exclude content deemed inappropriate or irrelevant. Further preprocessing steps are applied, followed by deduplication to ensure data quality and minimize redundancy. We adopt a similar approach in preprocessing the web data collected for the Matina corpus.

Matina's web-based data is divided into two parts: data crawled by our team and data taken from two public databases using the Common Crawl \citep{commoncrawl} dataset. This dual-source strategy uses both proprietary and publically available data to increase the corpus's breadth and diversity.

In any language, certain domains are recognized for their reliability and high-quality information. We identified such domains in Persian and crawled them to extract relevant textual content. This step helped minimize the inclusion of irrelevant elements such as advertisements, tags, or comments. Text extracted from headings and paragraphs was merged to form unified documents, with additional informative fields (e.g., summaries or subheadings) incorporated as metadata, if available. Because these domains were manually selected, language detection and URL filtering were unnecessary. We also ensured that the selected URLs did not contain harmful, sensitive, or adult content.

For the public datasets, Madlad-400 \citep{kudugunta2024madlad} and CulturaX \citep{nguyen2023culturax}, the initial preprocessing steps—such as language detection, text extraction, and URL filtering—had already been completed by the dataset providers. These datasets also included filters for toxic or harmful content, which allowed us to directly proceed to the next stages of preprocessing. While both datasets applied generic filters—such as language mismatch detection, character ratio checks, and word/sentence length thresholds, these filters were not language-specific. Therefore, we processed data from these sources similarly to the web data we crawled ourselves. After applying the processing on data sourced from web and the public datasets, there remained 64.3B tokens with an average document length of 1,141.8 tokens. 

After inspecting samples from various domains, we defined heuristic functions to modify documents and remove those deemed irrelevant. These heuristics were inspired by preprocessing pipelines adopted in BLOOM \citep{le2023bloom}, MassiveText \citep{muennighoff2022massiveText}, and RefineWeb \citep{penedo2023refinedweb}, but we tailored them to the specific characteristics of our data and added multiple other processing functions. 

Our preprocessing pipeline for web-based data encompasses three primary stages: character-level processing, line and paragraph-level processing, and document-level processing. Each stage employs a series of targeted operations to enhance data quality, ensure linguistic consistency, and eliminate redundancies.  Appendix~\ref{sec:appendixA} provides a full explanation of each step in the preprocessing and deduplication procedures.  

\textbf{Character-level processing} involves normalizing Persian characters, mapping symbols and numbers to their Persian equivalents, limiting the occurrence of repeated characters, standardizing newline characters, and removing non-standard Unicode symbols. This stage ensures that the text adheres to consistent encoding standards and minimizes the presence of corrupted or irrelevant characters.

\textbf{Line and paragraph-level processing} focuses on the structural integrity of the text by removing HTML and JavaScript tags, handling custom structures specific to certain domains, filtering out lines with excessive special characters, and eliminating short or incomplete lines that do not contribute meaningful content.

\textbf{Document-level processing} entails a comprehensive evaluation of each document's relevance and quality. Documents are discarded based on criteria such as insufficient length, predominance of non-Persian content, excessive repetition of words, high proportion of short lines, and the presence of out-of-vocabulary (OOV) words. These filters ensure that only high-quality, relevant, and linguistically coherent documents are retained in the corpus.

After cleaning the documents, we apply a deduplication step to mitigate data redundancy, a crucial aspect of the preprocessing pipeline highlighted in several studies \citep{gao2020pile, penedo2023refinedweb, le2023bloom}.  Utilizing the MinHash algorithm \citep{broder1997minhash}, we efficiently identify and eliminate both exact and near-duplicate documents, thereby enhancing the corpus's uniqueness.

For two manually inspected domains, \href{https://virgool.io/}{Virgool}\footnote{\href{https://virgool.io/}{https://virgool.io/}} and \href{https://en.wikishia.net/}{WikiShia}\footnote{\href{https://en.wikishia.net/}{https://en.wikishia.net/}}, we adopted a tailored processing approach to account for domain-specific characteristics. Virgool's diverse blog posts required relaxed filtering criteria to preserve technical content, while WikiShia's recursive linking and bilingual content deemed for  specialized deduplication and language handling techniques to maintain content integrity and cultural relevance.
\begin{figure}[t]
  \includegraphics[width=\columnwidth]{./boxplot_all.png}
  \caption{Distribution of document length by source in the Matina Corpus. Length is determined by the log of the number of tokens using Llama3.1 \cite{dubey2024llama3.1} tokenizer.}
  \label{fig:boxplot1}
\end{figure}
\subsection{Crawled Books and Papers}
Data collected from the web alone does not provide sufficient factual or literary content. To enrich our dataset, we also sourced publicly accessible books and academic papers from websites and social media channels. As demonstrated in Figure~\ref{fig:boxplot1}, the box plot of document length distribution clearly shows that books and papers contain significantly longer texts compared to web and social media content, making them more informative and comprehensive. This length, along with the depth of the content, further justifies the inclusion of these sources in our corpus. 

Since most of these sources provide data in PDF format, additional steps were required to convert PDFs into usable text. However, the limited accuracy of Persian OCR systems introduces challenges, particularly when processing PDFs that contain scanned images.

We divided the data from books and papers into two groups, each requiring different processing steps based on the nature of the data: Text-based PDFs and Image-based PDFs (OCR). Just like the data from web, the processing of books and papers involved a combination of document-level, character-level, and line-level operations to ensure data quality, as outlined below.

\subsubsection{Text-based PDFs}
Text-based PDFs primarily include books and academic papers sourced from Telegram channels and Persian websites. The PDFs were converted into text using several Python libraries. To ensure quality, we tested various tools on sample documents and applied low-level heuristic filters to remove corrupted or irrelevant content.

The filtering process involves removing documents with insufficient Persian content, short text lengths, or an excessive use of symbols. This stage ensures that only relevant and high-quality documents are retained. Following this initial filtering, we apply a preprocessing pipeline to address document, character, and line-level inconsistencies, ensuring the text is properly structured. Additional technical details on these steps, including character normalization, watermark removal, and deduplication, are provided in the Appendix~\ref{sec:appendixB}.

\subsubsection{Image-based PDFs (OCR)}
Many papers in our dataset were converted to text using image-based OCR due to the unavailability of text-based PDFs. Given the limitations of Persian OCR, errors were introduced during text extraction. To address this, we filtered out low-quality documents, focusing on those with a high percentage of nonsensical tokens or merged words. As a result, the dataset was refined to include 321,244 documents. The documents were then processed using steps  similar to those applied to web-based crawled data, with additional procedures. Additional information on the OCR-specific filtering methods is provided in the Appendix.

\subsection{Social Media}

Although some books and blogs may include informal Persian text or dialogues, the overall proportion of such data is minimal. The data collected from web-based sources and books generally lacks unstructured or colloquial language. Social media, however, provides a rich source of unstructured and informal linguistic data. To capture this, we gathered Persian-language data from Twitter, as well as public channels and groups from Telegram and Eitaa (an Iranian chat application). After identifying relevant channels and groups, we crawled all associated messages and processed them using the pipeline described for web-based data, with thresholds tailored to social media content. Additional processing steps we applied are outlined below.

Upon examination, we found that shorter messages were mostly replies, often lacking substantive content or containing inappropriate language. These messages were filtered out. We also identified hashtags embedded within the text and at the end of messages. Hashtags within the text were retained to preserve context, while those at the end, frequently related to political or social topics and often irrelevant to the main content, were removed. We employed regular expressions (regex) to remove channel IDs and URLs, ensuring that irrelevant content was minimized.

A notable difference in processing social media data was the deduplication strategy. We observed that many messages from different sources differed only in date or pricing—typically for goods, gold, silver, or cryptocurrencies. To address this, we removed all numeric values and dates before deduplication. After identifying and eliminating duplicate entries, we restored the original content, including numbers and dates, for the final dataset. This method ensured that informative variations were preserved while content containing no new knowledge was removed. 

\subsection{Final Dataset}
Applying the outlined preprocessing steps, including deduplication, resulted in a significant reduction in the number of documents. As illustrated in Figure~\ref{fig:barplot}, the overall document count decreased by an average of \textbf{24\%} after preprocessing, with a further reduction of \textbf{18.83\%} following deduplication.

The largest reduction occurred in social media content, particularly from Twitter and Telegram. Many Twitter posts were short and lacked meaningful content, while Telegram messages were often redundant, brief, and became even less informative after hashtags and links were removed. The special deduplication method we applied also identified many of these messages as duplicates. Although only 1.6\% of social media documents remained after processing, these retained documents were significantly longer, accounting for around 10\% of the total token count from the initial data.

Image-based academic papers also experienced a considerable loss during processing. In this category, the number of documents was nearly halved, as we applied multiple criteria to remove poor-quality documents. In contrast, text-based papers saw minimal loss, with only 2\% of documents eliminated during preprocessing. However, papers in this category contained more duplicates, which contributed to the reduction.

Books had the lowest proportion of document elimination during both preprocessing and deduplication. This reflects the higher quality of book content and the effectiveness of the methods used to extract data from PDF files.

For the web-crawled data, deduplication had a bigger impact than the initial preprocessing, with more documents being removed in this step. Even though we carefully tried to avoid duplicates during crawling, the nature of web crawling—often involving nested links—led to the inclusion of duplicates. Additionally, many news websites repost the same content across different agencies, which shows just how important thorough deduplication is for web-sourced data.

An interesting observation from the bar plot is that, although CulturaX FA \citep{nguyen2023culturax} and Madlad-400 FA \citep{kudugunta2024madlad} claim to have already undergone processing and deduplication, our language-specific preprocessing steps and content-specific deduplication further reduced their size. In Madlad-400 FA, only 7\% of documents were discarded, whereas nearly 70\% of CulturaX FA documents did not meet the qualifications for proper Persian data. This emphasizes the importance of language-specific processing and careful evaluation by native speakers to ensure data quality.

\begin{figure*}[t]
  \includegraphics[width=15.5cm]
  {./barplot.pdf}
  \centering
  \caption{Data reduction during preprocessing and deduplication varies significantly across sources. Social media shows the most drastic drop, with just 1.6\% of documents remaining after deduplication, while other sources retain between 56.1\% and 93.3\%. The three bars for each source represent the percentage of documents left after each stage. Overall, about 14\% of the initial documents remain.}
  \label{fig:barplot}
\end{figure*}