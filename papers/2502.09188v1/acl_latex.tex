% This must be in the first 5 lines to tell arXiv to use pdfLaTeX, which is strongly recommended.
\pdfoutput=1
% In particular, the hyperref package requires pdfLaTeX in order to break URLs across lines.

\documentclass[11pt]{article}

% Change "review" to "final" to generate the final (sometimes called camera-ready) version.
% Change to "preprint" to generate a non-anonymous version with page numbers.
\usepackage[final]{acl}

% Standard package includes
\usepackage{times}
\usepackage{latexsym}

% For proper rendering and hyphenation of words containing Latin characters (including in bib files)
\usepackage[T1]{fontenc}
% For Vietnamese characters
% \usepackage[T5]{fontenc}
% See https://www.latex-project.org/help/documentation/encguide.pdf for other character sets

% This assumes your files are encoded as UTF8
\usepackage[utf8]{inputenc}

% This is not strictly necessary, and may be commented out,
% but it will improve the layout of the manuscript,
% and will typically save some space.
\usepackage{microtype}

% This is also not strictly necessary, and may be commented out.
% However, it will improve the aesthetics of text in
% the typewriter font.
\usepackage{inconsolata}
\usepackage[utf8]{inputenc}
\usepackage[arabic,farsi,english]{babel}
%Including images in your LaTeX document requires adding
%additional package(s)
\usepackage{graphicx}

% If the title and author information does not fit in the area allocated, uncomment the following
%
%\setlength\titlebox{<dim>}
%
% and set <dim> to something 5cm or larger.

\title{Matina: A Large-Scale 73B Token Persian Text Corpus}

% Author information can be set in various styles:
% For several authors from the same institution:
% \author{Author 1 \and ... \and Author n \\
%         Address line \\ ... \\ Address line}
% if the names do not fit well on one line use
%         Author 1 \\ {\bf Author 2} \\ ... \\ {\bf Author n} \\
% For authors from different institutions:
% \author{Author 1 \\ Address line \\  ... \\ Address line
%         \And  ... \And
%         Author n \\ Address line \\ ... \\ Address line}
% To start a separate ``row'' of authors use \AND, as in
% \author{Author 1 \\ Address line \\  ... \\ Address line
%         \AND
%         Author 2 \\ Address line \\ ... \\ Address line \And
%         Author 3 \\ Address line \\ ... \\ Address line}

\author{
  Sara Bourbour Hosseinbeigi \\
  Tarbiat Modares University \\
  \texttt{s.bourbour@modares.ac.ir} \\
  \And
  Fatemeh Taherinezhad \\
  University of Tehran \\
  \texttt{ftaherinezhad@ut.ac.ir} \\
  \And
  Heshaam Faili \\
  University of Tehran \\
  \texttt{hfaili@ut.ac.ir} \\
  \AND
  Hamed Baghbani \\
  University of Tehran \\
  \texttt{baghbani.hamed@ut.ac.ir} \\
  \And
  Fatemeh Nadi \\
  University of Tehran \\
  \texttt{fatemehnadi@ut.ac.ir} \\
  \And
  Mostafa Amiri \\
  University of Tehran \\
  \texttt{mostafa.amiri@ut.ac.ir}
}


% \author{
%  \textbf{Sara Bourbour Hosseinbeigi\textsuperscript{1}},
%  \textbf{Heshaam Faili\textsuperscript{2}},
%  \textbf{Fatemeh Taherinezhad\textsuperscript{2}},
%  \textbf{Hamed Baghbani\textsuperscript{2}},
% \\
%  \textbf{Fatemeh Nadi\textsuperscript{2}},
%  \textbf{Mostafa Amiri\textsuperscript{2}},
% \\
% \\
%  \textsuperscript{1}Tarbiat Modares University,
%  \textsuperscript{2}University of Tehran
% \\
 % \small{
 %   \textbf{Correspondence:} \href{mailto:ftaherinezhad@ut.ac.ir}{ftaherinezhad@ut.ac.ir}
 % }
% }


\begin{document}
\maketitle
\begin{abstract}
Text corpora are essential for training models used in tasks like summarization, translation, and large language models (LLMs). While various efforts have been made to collect monolingual and multilingual datasets in many languages, Persian has often been underrepresented due to limited resources for data collection and preprocessing. Existing Persian datasets are typically small and lack content diversity, consisting mainly of weblogs and news articles. This shortage of high-quality, varied data has slowed the development of NLP models and open-source LLMs for Persian. Since model performance depends heavily on the quality of training data, we address this gap by introducing the Matina corpus, a new Persian dataset of 72.9B tokens, carefully preprocessed and deduplicated to ensure high data quality. We further assess its effectiveness by training and evaluating transformer-based models on key NLP tasks. Both the dataset and preprocessing codes are publicly available\footnote{\href{https://github.com/FTaheriN/Matina-Text-Preprocessing}{https://github.com/FTaheriN/Matina-Text-Preprocessing}}, enabling researchers to build on and improve this resource for future Persian NLP advancements.
    \end{abstract}
    
\begin{table*}
  \centering
  \begin{tabular}{lll}
    \hline
    \textbf{Component}           & \textbf{Number of tokens} & \textbf{Mean document length} \\
    \hline
    Books           &    2,842,128,225   &   162,648.9   \\
    Papers          &    3,547,046,981   &   10,620.5    \\
    Social Media    &    2,143,415,349   &   351.6       \\
    Web Crawled     &    14,782,414,716     &   749.7        \\
    CulturaX FA     &    20,469,778,795     &   1,124.8       \\
    MADLAD-400 FA   &    29,131,569,264     &   1,352.96      \\
    \hline
    \textbf{Matina} &    72,916,353,330  &   1,106.5     \\
    \hline
  \end{tabular}
  \caption{\label{table:corpus-size}
    Overview of components in Matina Corpus. Tokens are counted by the Llama 3.1 \citep{dubey2024llama3.1} tokenizer. 
  }
\end{table*}


\documentclass[../main.tex]{subfiles}
\graphicspath{{../images/}}
\makeatletter
\def\input@path{{../images/}}
\makeatother
\begin{document}
\section{Introduction}
\begin{figure}
\centering
\begin{tikzpicture}
\node[inner sep=0pt] (ws) at (0, 0) {
\includegraphics[height=.4\textwidth, trim={10cm 0 10cm 0},clip]{world_space.png}};
\node[inner sep=0pt] (cs) at (6,0) {\includegraphics[height=.4\textwidth, trim={10cm 1cm 10cm 4cm},clip]{conf_space.png}};
\end{tikzpicture}
\vspace{-5pt}
\label{fig:pbrm_intro}
\caption{\textbf{Left}: Shows world space obstacles as grey spheres. Robots start and goal configuration is colored red and green, respectively. Configurations along the computed path are colored transparent blue. \textbf{Right:} Mapped world space scenario to configuration space. Obstacle region is the grey mesh. Red spheres are collision-free regions computed by the neural SCDF. The optimized shortest path in the convex corridor is the blue curve.}
\vspace{-25pt}
\end{figure}
Motion planning is the problem of finding a collision-free trajectory that connects a given start and goal configuration. The planning takes place in the configuration space of the robot. For single body robots, like mobile robots or drones, the configuration space and the world space are usually the same. This simplifies the planning, since explicit obstacle representations are available which enables geometrical tools like separating hyperplanes, smallest distance to obstacles etc., to be used when designing motion planning algorithms. For multi-body robots like manipulators, the situation is completely different. The world space obstacles are usually mapped to non-convex regions, and to make the problem even harder, the mapping is usually not known. Forming explicit representations of the obstacle region in the configuration space is usually too expensive or intractable. Despite all of this, sampling based planners are used with great success, which mainly is due to their use of implicit representations of the obstacle region. The basic idea is to construct a graph in the configuration space that covers and connects the collision-free region. From this graph, a path can be extracted that connects a given start and goal configuration. The approach is computationally expensive, since the graph is constructed with the smallest geometrical building block available, points, which represents a collision-check. Furthermore, the extracted paths from the graph are non-smooth and jagged due to the stochastic nature of the approach. This adds an additional post-processing step to the process, where the paths are shortcutted and smoothened, before the path can be used for tracking. Clearly a lot of time is invested to form this graph and produce smooth paths. Thus, if the obstacles start to move, then all of this work is done in no use, since all points that make up this graph need to be re-verified, which is simply too time consuming to be done in real time.
\\\\
In this work, we want to address the existing drawbacks of the sampling based planners. Our main contribution is an improved motion planner where each vertex in the graph covers a collision-free region in the form of a sphere instead of a point and where the edges are formed with neighboring intersecting spheres. This representation has the advantage of instead of returning piecewise linear paths, returning a sequence of overlapping spheres, i.e. a convex corridor, that connects a given start and goal configuration, illustrated in Figure \ref{fig:pbrm_intro}. This convex corridor allows us to use convex optimization to produce smooth trajectories, instead of computationally expensive post-processing methods. The representation further allows us to estimate the coverage of the collision-free space, which gives us awareness and feedback in the offline roadmap construction phase. Finally, our representation is simple to adapt to moving obstacles, simply requery for the new radii and recheck for intersections. 
\\\\
The spherical collision-free regions are formed using a signed distance function (SDF), which is a function that returns the smallest distance from an arbitrary point to the boundary of an obstacle. As the name implies, the distance is signed, thus if the point is inside the obstacle it is negative otherwise positive. If the distance is positive, a sphere with radius equal to the distance is guaranteed to cover a collision-free region. Using an SDF in motion planning is not new, but what is novel about our approach is that we express the distance in the configuration space instead of the world space and by doing so allows us to form these convex collision-free regions. We refer to the resulting SDF as a signed configuration distance function (SCDF). Computing an SCDF analytically is non-trivial, our approach is therefore to parameterize the SCDF with a deep neural network and learn the mapping by supervised learning. Our resulting neural SCDF can compute distances for different parameter values of obstacle shapes and we also show how multiple distances can be combined, thus making our approach flexible.
\section{Related work}
Motion planning algorithms can roughly be divided into three families, grid-based, sampling based and optimization based methods. Grid-based methods (GBM) discretize the planning space from which a graph is then compiled. A standard search method is A$^\star$ \citep{a_star}, which is classified as an \textit{informed} search method, since it employs a heuristic function to speed up the search. A$^\star$ guarantees to return an optimal path at the level of discretization used. GBMs usually discretize the planning space by a regular lattice and this limits the GBMs to problems with low dimensionality due to the curse of dimensionality. Thus, GBMs are usually limited to single-body robots where the degrees of freedom (DOF) are low. To overcome the inherent scaling problem with the GBMs, stochastic methods are usually used for multi-body robots. These methods are termed as sampling-based methods (SBM) and core members within this family are the rapidly-exploring random trees (RRT) \citep{rrt} and the probabilistic roadmap (PRM) \citep{prm}. RRT grows a tree from the start configuration and explores the collision-free region in a rapid way until it is able to connect to the goal region. RRT is usually improved by bi-directional planning \citep{rrt_connect}, i.e. an additional tree is grown from the goal configuration and the trees are tested for connection after any tree has been expanded. RRT is a single-query method, thus it searches for a path from scratch each time it is queried. Contrary to this, PRM is a multi-query method, which solves for multiple queries without starting from scratch. PRM does this by creating a roadmap (graph) that covers the collision-free space as an offline step. The graph is then used to solve for multiple queries. PRMs are used in cases where the environment does not change since the extra offline step is too computationally costly and needs to be re-done if the environment is changed. In our work, we address this inherent issue by using a different roadmap representation. Our vertices in the graph cover a collision-free region in the form of spheres and we form the edges by checking for intersecting spheres. If something in the environment changes, we recompute the spheres radii and recheck the intersections, without relying on collision detection. We use a trained neural network to compute the sphere radius, therefore querying for the radius can be done fast, hence our representation enables the PRM for dynamic environments.
\\\\
In the recent decades, optimization based methods (OBM) \citep{chomp, schulman, itomp, stomp} have been introduced as an alternative to SBM for multi-body robots. Like the SBM, the OBMs scale well to higher dimensional problems and produce smoother motion. It is common to use a SDF in the optimization since it is a smooth function, thus enabling gradient-based methods. However, the standard way of expressing the SDF is in world space. The distance therefore needs to be mapped to the configuration space by the forward kinematics. This mapping makes the optimization problem a non-linear program (NLP), which is computationally expensive to solve. Recently, a different approach has been proposed. In \cite{mp_gcs} motion planning is formulated as a convex optimization problem by using the graph of convex sets framework \citep{gcs}. The underlying idea is to decompose the collision-free space into intersecting convex sets from which a convex optimization problem is formulated. In cases where an explicit representation of the obstacles in the configuration space exists, like for single-body robots, creating collision-free convex regions can be done fast \citep{iris}. For multi-body robots, this is non-trivial. Existing work does this successfully \citep{iris_nlp, iris_c} by an optimization based approach, but the methods are still too time consuming to be used in the presence of moving obstacles. Our approach is instead to use deep learning to learn an SDF expressed in the configuration space. With this, we can query for shortest distances to the collision boundary, which allows us to expand spherical regions which are collision-free. Our approach is fast and therefore enables our suggested roadmap planner to be used in dynamic environments.
\\\\
Recent research has focused on learning collision detection \citep{fk_kernel_distance, diffco, graphdistnet} by predicting the signed distance between the robot links and the surrounding obstacles in the world space. The learned SDF is used in trajectory optimization but since the distance is expressed in the world space, the problem becomes an NLP and therefore takes a long time to solve. We take a novel approach and suggest to instead express the signed distance in the configuration space. This allows us to improve the PRM at the same time as it enables convex optimization for trajectory optimization, which runs faster and is more reliable than NLP solvers. In \cite{cspf} a learned signed distance function in the configuration space is proposed similar to our approach. However, their approach is restricted to point cloud representations, while we propose to represent the obstacles as parameterized geometric shapes, e.g. spheres. Furthermore, we also show how to use our learned SCDF to improve an existing roadmap planner.
\section{Problem formulation}
A robot is located in the world space, $\W \subset \R^3 $. The unique location of the robot is given by its configuration $\q \in \C$, where $\C$ is the configuration space. The set of points covered by the robots bodies at a certain configuration is expressed as $\B(\q) \subset \W$. The robot is surrounded by $\NrObst$ obstacles $\O = \bigcup_{i=1}^{\NrObst} \O_i$, where  $\O_i \subset \W$. The representation of the obstacle in the configuration space is the set $\C\O_i = \{\q \in \C \: |\: \B(\q) \cap \O_i \neq \emptyset \}$. The obstacle space is formed as $\Co = \bigcup_{i=1}^{\NrObst} \C \O_i$. The complement is referred to as the free space, $\Cf = \C \setminus \Co$. The path planning problem is a tuple, ($\Cf$, $\qStart$, $\qGoal$), where we want to connect a query pair, consisting of a start, $\qStart$, and goal configuration, $\qGoal$, with a geometric path, $\q(s): [0, 1] \mapsto \Cf$, such that $\q(0)=\qStart$ and $\q(1)=\qGoal$, or report correctly when such a path does not exist.
\end{document}


\section{Related work}

The literature related to our work can be classified into two
categories: general purpose DR techniques
(\autoref{sec:relatedWorkGeneralPurpose}) and topology-aware techniques
(\autoref{sec:relatedWorkTopology}).

\subsection{General purpose dimensionality reduction}
\label{sec:relatedWorkGeneralPurpose}

Numerous DR techniques have been proposed and the related literature has been
summarized in several books~\cite{borg97, dimensionReductionBook} and surveys
\cite{surveyDimensionReduction2, surveyDimensionReduction1, NonatoA19}.
Principal Component Analysis (PCA)~\cite{pearson1901liii} is by far the most
popular linear DR technique.
Although it is an indispensable tool for data analysis,
its linear nature does not always allow it to apprehend complex non-linear
phenomena. One of the first non linear DR methods is the multidimensional
scaling (MDS)~\cite{torgerson1952multidimensional}. It aims at preserving as far
as possible the pairwise distances in the high- and low-dimensional point
clouds.
Another approach, particularly related to our work,
consists in optimizing an autoencoder neural network~\cite{hinton_reducing_2006}.
The \textit{encoder} is used to represent the explicit projection map from the
high-dimensional input space to the low-dimensional representation
space, while the \textit{decoder} tries to reconstruct the input data
from its encoded representation.
We will refer to these methods as \emph{global} methods.

Global methods have been used successfully in many applications, but
they do not take into
account the possible distribution of the input points over some implicit,
unknown manifold. This may lead to the unwanted preservation of distances
between points that are close in the ambient space but far apart on this
manifold. \emph{Locally topology-aware} methods have therefore been
introduced to address this issue. For instance,
Isomap~\cite{tenenbaum_global_2000}
preserves geodesic distances on a captured manifold structure of the
input data.
%\remove{Because it suffers from computational
%inefficiencies, Isomap was sped up with the use of landmark points (L-Isomap
%\cite{silva2003global}).}
Other methods also feature neighborhood preservation objectives.
For example, Local Linear Embedding (LLE)~\cite{roweis2000nonlinear} relies
on linear reconstructions of local neighborhoods.
Other methods leverage additional landmarks~\cite{silva2003global} or user-provided
control points~\cite{joia:tvcg:2011}.
%Some local methods additionally support user
%constraints expressed as control points~\cite{joia:tvcg:2011}.

All these methods try to preserve local
Euclidean distances when projecting to a lower dimension.
However, this can sometimes lack relevance in the applications,
especially with high-dimensional datasets for which
the distribution of pairwise Euclidean distances tend to be uniform.
For such cases, local distance preservation fails at characterizing
well relevant local relations.
To alleviate this issue, SNE~\cite{hinton2002stochastic} and later
t\nobreakdash-SNE~\cite{van2008visualizing} use a conditional probability
formulation to represent similarities between points and try to
have similar distributions both in high- and low-dimension thanks to a
Kullback--Leibler divergence minimization.
More recently UMAP has been introduced~\cite{mcinnes2018umap} along a
theoretical foundation on category theory.
It provides results that are similar visually to t-SNE, but in a more
scalable way.
Variants were later introduced to better preserve the global structure in the embedding, such as TriMAP~\cite{amid2022trimap} that constrains the proximity order within triplets of points, or PaCMAP~\cite{wang_understanding_2021} that adds constraints on more distant point pairs.
Although these methods succeed in preserving the local topology, they are not
explicitly aware of the global structure
of the input, which can lead to the loss of noteworthy global,
topological features.

\subsection{Globally topology-aware dimensionality reduction}
\label{sec:relatedWorkTopology}

Topology-based methods have become popular over the last
two decades in data analysis and
visualization~\cite{heine16} and have been applied to various areas:
astrophysics~\cite{sousbie11, shivashankar2016felix},
biological imaging~\cite{beiBrain18, carr04, topoAngler},
quantum chemistry~\cite{chemistry_vis14,harshChemistry, D2CP05893F},
fluid dynamics~\cite{kasten_tvcg11, NauleauVBBT22},
material sciences~\cite{gyulassy_vis07, gyulassy_vis15, SolerPDPCT19},
turbulent combustion~\cite{gyulassy_ev14, laney_vis06}. They leverage tools that
define concise signatures of the data based on its topological properties and
that have been summarized in topological data analysis reference
books ~\cite{edelsbrunner_computational_2010, zomorodian_computational_2010}
and surveys~\cite{chazal_introduction_2021}.

Several DR methods have been proposed
by the visualization community to preserve specific topological signatures
of the input data. For instance, terrain metaphors have been
investigated for the visualization of an input high-dimensional scalar
field, in the form of a three-dimensional terrain, whose elevation yields an
identical contour tree~\cite{Weber:2007} or an identical set of separatrices
\cite{gerber2010, gerber2013}.
This framework has been extended to density
estimators~\cite{OesterlingHJS10,
OesterlingSTHKEW10, OHJSH11, Oesterling0WS13} for the support of
high-dimensional point clouds. However, such metaphors completely discard
the metric information of the input space~\cite{OesterlingHJS10}, possibly
placing next to each other points which are arbitrarily far
in the input space (and reciprocally). Yan et al.~\cite{abs-1806-08460}
introduced a DR approach driven by the Mapper structure~\cite{SinghMC07}, an
approximation of the Reeb graph~\cite{reeb46}, which can capture in practice
large handles in the data, however without guarantees, since the number of handles in the considered manifold is only an upper bound on the number of loops in the Reeb graph~\cite{edelsbrunner_computational_2010}.

To incorporate the metric information from the input data while
preserving at the same time some of their topological characteristics, several
approaches have focused on driving the projection by
the \emph{persistence diagram}
of the Rips filtration of the point cloud (see \autoref{sec:persistentHomology}
for  a technical description).
Carriere et al.~\cite{carriere2021optimizing} presented a generic persistence
optimization framework with an application to dimensionality reduction.
Their approach explicitly minimizes the Wasserstein distance
(\autoref{sec:persistentHomology}) between the $1$-dimensional persistence
diagrams in high and low dimensions. However, this approach solely focuses on
this penalization term. As a result,
although the number and persistence of cycles  may be well-preserved,
the solver tends to produce cycles in low dimensions which involve arbitrary
points (e.g., which were not necessarily located along the cycles in high
dimensions), which challenges visual interpretation, as later
detailed in \autoref{sec:results:analysis}.

To enforce a correspondence between the topological
features at the data point level, additional structures need to be preserved.
For the specific case of $0$-dimensional persistent homology (\PH{0}),
Doraiswamy et al. introduced \emph{TopoMap}~\cite{doraiswamy2020topomap}, an
algorithm which constructively preserves the \emph{persistence pairs}
(\autoref{sec:persistentHomology}) through the preservation of the minimum
spanning tree of the data. An accelerated version, with improved layouts, has
recently been proposed~\cite{guardieiro2024topomap++}.
Alternative approaches have considered the usage of an optimization framework
(typically based on an autoencoder neural network
\cite{hinton_reducing_2006}), with the integration of specific topology-aware
losses~\cite{moor2020topological,barannikov2021representation,
nelson2022topology,trofimov2023learning,schonenberger2020witness}. Among them,
a prominent approach is the \emph{Topological Autoencoders}
(TopoAE)~\cite{moor2020topological}. Its loss aims at preserving
the diameter of the simplices involved in
persistence pairs, when going from high to low dimensions and reciprocally.
However, the above techniques focused in practice on the
preservation of \PH{0} and did not, to our knowledge, report experiments
regarding the preservation of higher dimensional PH.
Specifically, we show in \autoref{sec:analysis} that, while a zero
TopoAE loss indeed implies a preservation of the persistence pairs for \PH{0},
it is not the case for higher dimensional PH. We provide a counter example for
\PH{1}, which is addressed by our novel, generalized loss.


\section{Matina Corpus}
% The Matina corpus is built from a variety of data sources, each of which is processed based on its specific content characteristics. While these sources are divided into three main categories, the general preprocessing steps remain consistent, as shown in Figure~\ref{fig:pipeline}, with variations mainly in hyperparameters. However, certain sources require extra cleaning steps, which we describe in detail in their respective sections. In this section, we outline the data collection process, the applied preprocessing techniques, and the rationale behind the decisions made during these steps.

The Matina corpus is built from a variety of data sources, each of which is processed based on its specific content characteristics. Although these sources are grouped into three main categories, the overall preprocessing pipeline remains consistent, as depicted in Figure~\ref{fig:pipeline}, with variations primarily in hyperparameters. Certain sources, however, demand additional cleaning steps, which are detailed in their respective sections.

Figure~\ref{fig:boxplot1} visualizes the distribution of token counts across documents from each source, using a box plot to illustrate the variance in document length. These three categories—web-based crawled data, crawled books and papers, and social media—form the core of our dataset, each with distinct preprocessing requirements. In this section, we describe the data collection process, the preprocessing techniques applied, and the rationale behind the decisions made throughout these steps.

\begin{figure}[t]
  \includegraphics[trim=2.5cm 4cm 2.2cm 1.8cm , clip,width=\columnwidth]{./processingPipeline.pdf}
  \caption{The overall stages of processing pipeline of Matina Corpus.}
  \label{fig:pipeline}

\end{figure} 

\subsection{Web-based Crawled Data}
Web crawling is a common and efficient method for collecting data in any language. Websites offer a vast range of valuable information and, given their structured nature and wide availability, can largely be crawled automatically. As a result, web data is frequently used as the primary source for constructing large-scale text datasets. However, while the bulk collection of web data is straightforward, extracting meaningful content from irrelevant elements such as metadata, advertisements, and embedded links remains challenging. Web pages often contain spam-like elements, which complicates the cleaning process and increases the likelihood of errors.

Most web-based datasets begin with basic steps such as text extraction and language detection, often followed by optional URL filtering to exclude content deemed inappropriate or irrelevant. Further preprocessing steps are applied, followed by deduplication to ensure data quality and minimize redundancy. We adopt a similar approach in preprocessing the web data collected for the Matina corpus.

Matina's web-based data is divided into two parts: data crawled by our team and data taken from two public databases using the Common Crawl \citep{commoncrawl} dataset. This dual-source strategy uses both proprietary and publically available data to increase the corpus's breadth and diversity.

In any language, certain domains are recognized for their reliability and high-quality information. We identified such domains in Persian and crawled them to extract relevant textual content. This step helped minimize the inclusion of irrelevant elements such as advertisements, tags, or comments. Text extracted from headings and paragraphs was merged to form unified documents, with additional informative fields (e.g., summaries or subheadings) incorporated as metadata, if available. Because these domains were manually selected, language detection and URL filtering were unnecessary. We also ensured that the selected URLs did not contain harmful, sensitive, or adult content.

For the public datasets, Madlad-400 \citep{kudugunta2024madlad} and CulturaX \citep{nguyen2023culturax}, the initial preprocessing steps—such as language detection, text extraction, and URL filtering—had already been completed by the dataset providers. These datasets also included filters for toxic or harmful content, which allowed us to directly proceed to the next stages of preprocessing. While both datasets applied generic filters—such as language mismatch detection, character ratio checks, and word/sentence length thresholds, these filters were not language-specific. Therefore, we processed data from these sources similarly to the web data we crawled ourselves. After applying the processing on data sourced from web and the public datasets, there remained 64.3B tokens with an average document length of 1,141.8 tokens. 

After inspecting samples from various domains, we defined heuristic functions to modify documents and remove those deemed irrelevant. These heuristics were inspired by preprocessing pipelines adopted in BLOOM \citep{le2023bloom}, MassiveText \citep{muennighoff2022massiveText}, and RefineWeb \citep{penedo2023refinedweb}, but we tailored them to the specific characteristics of our data and added multiple other processing functions. 

Our preprocessing pipeline for web-based data encompasses three primary stages: character-level processing, line and paragraph-level processing, and document-level processing. Each stage employs a series of targeted operations to enhance data quality, ensure linguistic consistency, and eliminate redundancies.  Appendix~\ref{sec:appendixA} provides a full explanation of each step in the preprocessing and deduplication procedures.  

\textbf{Character-level processing} involves normalizing Persian characters, mapping symbols and numbers to their Persian equivalents, limiting the occurrence of repeated characters, standardizing newline characters, and removing non-standard Unicode symbols. This stage ensures that the text adheres to consistent encoding standards and minimizes the presence of corrupted or irrelevant characters.

\textbf{Line and paragraph-level processing} focuses on the structural integrity of the text by removing HTML and JavaScript tags, handling custom structures specific to certain domains, filtering out lines with excessive special characters, and eliminating short or incomplete lines that do not contribute meaningful content.

\textbf{Document-level processing} entails a comprehensive evaluation of each document's relevance and quality. Documents are discarded based on criteria such as insufficient length, predominance of non-Persian content, excessive repetition of words, high proportion of short lines, and the presence of out-of-vocabulary (OOV) words. These filters ensure that only high-quality, relevant, and linguistically coherent documents are retained in the corpus.

After cleaning the documents, we apply a deduplication step to mitigate data redundancy, a crucial aspect of the preprocessing pipeline highlighted in several studies \citep{gao2020pile, penedo2023refinedweb, le2023bloom}.  Utilizing the MinHash algorithm \citep{broder1997minhash}, we efficiently identify and eliminate both exact and near-duplicate documents, thereby enhancing the corpus's uniqueness.

For two manually inspected domains, \href{https://virgool.io/}{Virgool}\footnote{\href{https://virgool.io/}{https://virgool.io/}} and \href{https://en.wikishia.net/}{WikiShia}\footnote{\href{https://en.wikishia.net/}{https://en.wikishia.net/}}, we adopted a tailored processing approach to account for domain-specific characteristics. Virgool's diverse blog posts required relaxed filtering criteria to preserve technical content, while WikiShia's recursive linking and bilingual content deemed for  specialized deduplication and language handling techniques to maintain content integrity and cultural relevance.
\begin{figure}[t]
  \includegraphics[width=\columnwidth]{./boxplot_all.png}
  \caption{Distribution of document length by source in the Matina Corpus. Length is determined by the log of the number of tokens using Llama3.1 \cite{dubey2024llama3.1} tokenizer.}
  \label{fig:boxplot1}
\end{figure}
\subsection{Crawled Books and Papers}
Data collected from the web alone does not provide sufficient factual or literary content. To enrich our dataset, we also sourced publicly accessible books and academic papers from websites and social media channels. As demonstrated in Figure~\ref{fig:boxplot1}, the box plot of document length distribution clearly shows that books and papers contain significantly longer texts compared to web and social media content, making them more informative and comprehensive. This length, along with the depth of the content, further justifies the inclusion of these sources in our corpus. 

Since most of these sources provide data in PDF format, additional steps were required to convert PDFs into usable text. However, the limited accuracy of Persian OCR systems introduces challenges, particularly when processing PDFs that contain scanned images.

We divided the data from books and papers into two groups, each requiring different processing steps based on the nature of the data: Text-based PDFs and Image-based PDFs (OCR). Just like the data from web, the processing of books and papers involved a combination of document-level, character-level, and line-level operations to ensure data quality, as outlined below.

\subsubsection{Text-based PDFs}
Text-based PDFs primarily include books and academic papers sourced from Telegram channels and Persian websites. The PDFs were converted into text using several Python libraries. To ensure quality, we tested various tools on sample documents and applied low-level heuristic filters to remove corrupted or irrelevant content.

The filtering process involves removing documents with insufficient Persian content, short text lengths, or an excessive use of symbols. This stage ensures that only relevant and high-quality documents are retained. Following this initial filtering, we apply a preprocessing pipeline to address document, character, and line-level inconsistencies, ensuring the text is properly structured. Additional technical details on these steps, including character normalization, watermark removal, and deduplication, are provided in the Appendix~\ref{sec:appendixB}.

\subsubsection{Image-based PDFs (OCR)}
Many papers in our dataset were converted to text using image-based OCR due to the unavailability of text-based PDFs. Given the limitations of Persian OCR, errors were introduced during text extraction. To address this, we filtered out low-quality documents, focusing on those with a high percentage of nonsensical tokens or merged words. As a result, the dataset was refined to include 321,244 documents. The documents were then processed using steps  similar to those applied to web-based crawled data, with additional procedures. Additional information on the OCR-specific filtering methods is provided in the Appendix.

\subsection{Social Media}

Although some books and blogs may include informal Persian text or dialogues, the overall proportion of such data is minimal. The data collected from web-based sources and books generally lacks unstructured or colloquial language. Social media, however, provides a rich source of unstructured and informal linguistic data. To capture this, we gathered Persian-language data from Twitter, as well as public channels and groups from Telegram and Eitaa (an Iranian chat application). After identifying relevant channels and groups, we crawled all associated messages and processed them using the pipeline described for web-based data, with thresholds tailored to social media content. Additional processing steps we applied are outlined below.

Upon examination, we found that shorter messages were mostly replies, often lacking substantive content or containing inappropriate language. These messages were filtered out. We also identified hashtags embedded within the text and at the end of messages. Hashtags within the text were retained to preserve context, while those at the end, frequently related to political or social topics and often irrelevant to the main content, were removed. We employed regular expressions (regex) to remove channel IDs and URLs, ensuring that irrelevant content was minimized.

A notable difference in processing social media data was the deduplication strategy. We observed that many messages from different sources differed only in date or pricing—typically for goods, gold, silver, or cryptocurrencies. To address this, we removed all numeric values and dates before deduplication. After identifying and eliminating duplicate entries, we restored the original content, including numbers and dates, for the final dataset. This method ensured that informative variations were preserved while content containing no new knowledge was removed. 

\subsection{Final Dataset}
Applying the outlined preprocessing steps, including deduplication, resulted in a significant reduction in the number of documents. As illustrated in Figure~\ref{fig:barplot}, the overall document count decreased by an average of \textbf{24\%} after preprocessing, with a further reduction of \textbf{18.83\%} following deduplication.

The largest reduction occurred in social media content, particularly from Twitter and Telegram. Many Twitter posts were short and lacked meaningful content, while Telegram messages were often redundant, brief, and became even less informative after hashtags and links were removed. The special deduplication method we applied also identified many of these messages as duplicates. Although only 1.6\% of social media documents remained after processing, these retained documents were significantly longer, accounting for around 10\% of the total token count from the initial data.

Image-based academic papers also experienced a considerable loss during processing. In this category, the number of documents was nearly halved, as we applied multiple criteria to remove poor-quality documents. In contrast, text-based papers saw minimal loss, with only 2\% of documents eliminated during preprocessing. However, papers in this category contained more duplicates, which contributed to the reduction.

Books had the lowest proportion of document elimination during both preprocessing and deduplication. This reflects the higher quality of book content and the effectiveness of the methods used to extract data from PDF files.

For the web-crawled data, deduplication had a bigger impact than the initial preprocessing, with more documents being removed in this step. Even though we carefully tried to avoid duplicates during crawling, the nature of web crawling—often involving nested links—led to the inclusion of duplicates. Additionally, many news websites repost the same content across different agencies, which shows just how important thorough deduplication is for web-sourced data.

An interesting observation from the bar plot is that, although CulturaX FA \citep{nguyen2023culturax} and Madlad-400 FA \citep{kudugunta2024madlad} claim to have already undergone processing and deduplication, our language-specific preprocessing steps and content-specific deduplication further reduced their size. In Madlad-400 FA, only 7\% of documents were discarded, whereas nearly 70\% of CulturaX FA documents did not meet the qualifications for proper Persian data. This emphasizes the importance of language-specific processing and careful evaluation by native speakers to ensure data quality.

\begin{figure*}[t]
  \includegraphics[width=15.5cm]
  {./barplot.pdf}
  \centering
  \caption{Data reduction during preprocessing and deduplication varies significantly across sources. Social media shows the most drastic drop, with just 1.6\% of documents remaining after deduplication, while other sources retain between 56.1\% and 93.3\%. The three bars for each source represent the percentage of documents left after each stage. Overall, about 14\% of the initial documents remain.}
  \label{fig:barplot}
\end{figure*}
\section{Assessing the Impact of the Matina Corpus}

A large-scale Persian corpus has numerous applications in NLP, including training transformer-based models for tasks such as summarization, sentiment analysis, emotion detection, question answering, sentence embeddings, and text retrieval. Additionally, such corpora play a crucial role in pretraining large language models (LLMs) and generating instructions for LLM post-training. To assess the effectiveness of the Matina Corpus, we conducted experiments on transformer-based model training and continued pretraining of LLMs. This section provides a detailed discussion of these experiments and their outcomes.

\subsection{Masked Language Model Training and Evaluation}
While LLMs have excelled in various NLP tasks such as sentiment analysis and named entity recognition (NER), there remains a need for lightweight models that can be easily fine-tuned for specific tasks and datasets. These models are typically built on transformer-based architectures, particularly masked language models trained on large-scale datasets. 

To address this need, we conducted continual pretraining of masked language models (MLMs), specifically XLM-RoBERTa Large \citep{xlmro}, on 54.69 billion tokens of our dataset. This extensive corpus facilitates the development of high-quality sentence embeddings, further refined by adapting the model into a Sentence-BERT architecture without Next Sentence Prediction (NSP). These enhancements yield more precise semantic representations, significantly improving Persian NLP tasks. By leveraging a well-curated dataset with rigorous preprocessing, our model effectively captures Persian linguistic nuances. 

To evaluate the effectiveness of Matina corpus in training transformer models, we benchmarked out Roberta-based model against existing models using datasets such as \textbf{Arman Emo}, \textbf{Pars-ABSA}, \textbf{PQUAD}, and \textbf{PEYMA}. As shown in Table \ref{tabelMLM}, our model demonstrates substantial performance gains, achieving \textbf{56.54} on \textbf{Arman Emo}, surpassing TookaBERT and AriaBERT, and \textbf{74.92} on \textbf{Pars-ABSA}, highlighting its robustness in aspect-based sentiment analysis. These results validate the impact of our dataset on enhancing Persian NLP performance, particularly within transformer-based architectures.

The success of our MLM underscores the crucial role of high-quality data in pretraining. By capturing Persian linguistic and cultural nuances, our model not only enhances task-specific performance but also advances the goal of developing inclusive and representative language technologies. This approach ensures that underrepresented languages like Persian receive the attention they deserve, fostering more equitable advancements in NLP.


\begin{table*}[t]
    \centering
    \caption{Results of Masked Language Models Evaluation.}
    \begin{tabular}{p{6.4cm}p{2cm}p{2cm}p{1.6cm}p{1.2cm}}
    \hline
        \textbf{Model} 
        & \rotatebox{0}{\textbf{Arman Emo}} 
        & \rotatebox{0}{\textbf{Pars-ABSA}} 
        % & \rotatebox{90}{\textbf{Taghche}} 
        % & \rotatebox{90}{\textbf{Snapp Food}} 
        % & \rotatebox{90}{\textbf{Digikala}} 
        % & \rotatebox{90}{\textbf{Digimag}} 
        % & \rotatebox{90}{\textbf{Persian News}} 
        & \rotatebox{0}{\textbf{PQUAD}} 
        % & \rotatebox{90}{\textbf{Reading Comp.}} 
        % & \rotatebox{90}{\textbf{DeepSentPars}} 
        & \rotatebox{0}{\textbf{PEYMA}} \\ \hline
        \textbf{XLM-RoBERTa (ours)} & \textbf{56.54}  & \textbf{74.92}  & 86.82 & \textbf{85.65} \\ 
        \textbf{TookaBERT} \citep{sadraeijavaheri2024tookabert} & 52.87 & 74.65  & 86.73 & 86.09 \\ 
        \textbf{AriaBERT} \citep{ghafouri2023ariabert} & 38.23 & 74.59  & 83.14 & 35.78 \\ 
        \textbf{XLM-RoBERTa} \citep{xlmro} & 32.48 & 74.18 & \textbf{87.6}  & 87.94 \\ 
        \textbf{mBERT} & 6.74 & 68.15 & 85.94 & 65.32 \\ \hline
    \end{tabular}
    \label{tabelMLM}
\end{table*}

\subsection{Large Language Model Pretraining and Evaluation}
\begin{table}[t]
  \centering
  \begin{tabular}{ll}
    \hline
    \textbf{Dataset}           & \textbf{Number of tokens} \\
    \hline
    Social and Politics           &    1.1 B   \\
    Cooking          &    15 M   \\
    \hline
  \end{tabular}
  \caption{\label{table:pretrainData}
    Number of tokens used for LLM continual pretraining. Tokens are counted by the Llama 3.1 \citep{dubey2024llama3.1} tokenizer. 
  }
\end{table}

\begin{figure}[b]
  \includegraphics[trim=0.4cm 0.9cm 0.8cm 0.3cm, width=\columnwidth]{./win_lose_tie_plot.pdf}
  \caption{Win rate of pretrained models over models without pretraining.}
  \label{fig:winLose}
\end{figure}

Pretraining is essential for transferring knowledge to LLMs, shaping their linguistic and factual understanding. However, multilingual LLMs often struggle with underrepresented languages like Persian and exhibit cultural biases favoring Western perspectives \cite{cao2023assessing, alkhamissi2024investigating} due to the dominance of English in their training data. This leads to diminished performance in other languages and cultures. Incorporating language-specific data during pretraining can help address this issue.

To evaluate the impact of our dataset on LLM training, we conducted the following experiment. We first tagged our dataset in an unsupervised manner using a procedure similar to InsTag \cite{lu2023instag}, categorizing it into multiple domains. From these, we selected two—social and politics and cooking—and extracted a subset of data from each domain. These domain-specific subsets were then used to train models. The token count for each domain is presented in Table \ref{table:pretrainData}. We then constructed large instruction datasets for these domains and fine-tuned LLaMA 3.2-Instruct 8B using two different approaches: (1) continued pretraining on the domain-specific data followed by instruction tuning, and (2) direct instruction tuning without additional pretraining. To evaluate model performance, we conducted a human evaluation, where annotators ranked model outputs in a win-lose format, indicating which model provided better responses to a held-out evaluation set derived from the instruction dataset.

The evaluation results, shown in Figure \ref{fig:winLose}, indicate that models benefit significantly from pretraining on even a relatively small dataset before instruction tuning. This effect is particularly noticeable in the cooking domain, where the pretrained model was preferred nearly twice as often as the model without pretraining. These findings highlight the effectiveness of the Matina Corpus in improving language models by providing high-quality, domain-specific data. Pretraining on a small, well-curated dataset not only enriches the model’s knowledge but also enhances its alignment with the target language and cultural context.



\section*{Conclusion}
This paper aims to enhance our understanding of the computational complexity of computing various Shapley value variants. We found that for various ML models --- including decision trees, regression tree ensembles, weighted automata, and linear regression --- both local and global interventional and baseline SHAP can be computed in polynomial time under HMM modeled distributions. This extends popular algorithms, such as TreeSHAP, beyond their empirical distributional scope. We also establish strict complexity gaps between the various SHAP variants (baseline, interventional, and conditional) and prove the intractability of computing SHAP for tree ensembles and neural networks in simplified scenarios. Overall, we present SHAP as a versatile framework whose complexity depends on four key factors: \begin{inparaenum}[(i)] \item model type, \item SHAP variant, \item distribution modeling approach, \item and local vs. global explanations\end{inparaenum}. We believe this perspective provides deeper insight into the computational complexity of SHAP, paving the way for future work.




%We believe that our framework provides a more intricate understanding of SHAP computation complexity across different models, distributions, and variants, paving the way for further research.

Our work opens promising directions for future research. First, expanding our computational analysis to other SHAP-related metrics, such as asymmetric SHAP~\citep{frye20} and SAGE~\citep{covert2020understanding}, would be valuable. Additionally, we aim to explore more expressive distribution classes and relaxed assumptions beyond those in Section \ref{sec:tractable} while maintaining tractable SHAP computation. Finally, when exact computation is intractable (Section \ref{sec:intractable}), investigating the approximability of SHAP metrics through approximation and parameterized complexity theory~\citep{downey2012parameterized} is an important direction.

%Our work opens several promising avenues for future research on the computational properties of explainable AI methods, with a particular focus on SHAP. First, it would be interesting to broaden the computational analysis conducted in this work to include other popular SHAP-related metrics in the literature, such as asymmetric SHAP \cite{frye20} and SAGE \cite{covert2020understanding}. Also, in the future, we aim to explore more expressive distribution classes and relaxed distributional assumptions—extending beyond those examined in Section \ref{sec:tractable} —that still yield tractable SHAP computation. Finally, when exact computation proves intractable (Section \ref{sec:intractable}), it is worthwhile to theoretically investigate the question of the approximability of computing the SHAP metrics across various configurations, through the lens of approximation and parametrized complexity theory \cite{arora2009computational}.

%This paper aims to deepen our understanding of the computational complexity involved in obtaining different Shapley value variants. We found that for a variety of ML models, including decision trees, tree ensembles for regression, weighted automata, and linear regression models — computing both local and global interventional and baseline SHAP can be done in polynomial time when distributions are modeled by HMMs. This extends the distributional scope of popular algorithms like TreeSHAP, which is limited to empirical distributions. Additionally, we demonstrate a strict complexity gap between SHAP variants, showing that interventional and baseline SHAP can be strictly easier to compute than conditional SHAP. Despite these positive results, we uncovered intractability for various SHAP variants in neural networks and tree ensembles. Finally, we provided generalized complexity relations across SHAP variants. We believe that our framework offers a deeper understanding of the complexity involved in computing SHAP across various variants, models, distributions, as well as in both local and global computations, laying the groundwork for future research.
\section{Limitations} 

In this work, we compared the effectiveness and interplay of SFT and RL-based methods, under fixed data constraints. In particular, we chose offline methods like DPO and KTO as the baseline implementation of the RL method because it eliminates the need for reward modeling or iterative finetuning. This means that the process of development is limited to collecting an offline dataset and fientuning it - making it the most fair comparable to SFT in terms of implementation effort, compute costs and annotation efforts. Since this baseline RL method shows optimal performance over SFT, we hope that this motivates future work to study more complex RL-based methods and their interplay with SFT. In addition, we used GPT4o annotation for synthetic data generation, and also for evaluating Summarization and Helpfulness, which could include potential biases inherited from the model. 

In addition, we limited the size of the model to under 10 Billion parameters, to keep the finetuning cost low enough to ignore as compared to the data annotation costs. In addition, it would be extremely compute resource intensive to run thousands of finetuning runs with larger model sizes like 70B parameters. We hope that future work would study the scaling trends of RL-based methods against different model sizes, and also study the compute-data trade-off in-depth.

\section*{Acknowledgments}
We thank Zhenjia Xu, Yizhou Zhao, Yu Fang for help with hardware.
We thank Pulkit Goyal, Hawkeye King, Peter Varvak and Haoru Xue for help with motion capture setup. 
We thank Ziyan Xiong, Yilin Wu and Xian Zhou for support on Genesis integration.
We thank Rui Chen, Yifan Sun and Kai Yun for help with G1 hardware. 
We thank Xuxin Cheng, Chong Zhang and Toru Lin for always being there to help with any problem and answer any question.
We thank Unitree Robotics for help with G1 support. 



% Bibliography entries for the entire Anthology, followed by custom entries
%\bibliography{anthology,custom}
% Custom bibliography entries only
\bibliography{acl_latex}

\appendix
\renewcommand{\thesection}{\Alph{section}}
\renewcommand{\thesubsection}{\Alph{section}.\arabic{subsection}}

\section*{Appendix}

% \documentclass{MITstyle}

%\usepackage[table]{xcolor}
\usepackage{chngcntr}
\usepackage{hyperref}
\usepackage{microtype}

\title{A Lightweight and Extensible Cell Segmentation and Classification Model for Whole Slide Images}

\author{Nikita Shvetsov~$^{1, }$\footnote{Correspondence e-mail: nikita.shvetsov@uit.no}, Thomas K. Kilvaer~$^{2, 3}$, Masoud Tafavvoghi~$^{4}$, Anders Sildnes~$^{1}$, \\ Kajsa Møllersen~$^{4}$, Lill-Tove Rasmussen Busund~$^{5, 6}$, Lars Ailo Bongo~$^{1}$ \\
%
\vspace{1em} % Space between authors and afilliations
%
\normalfont{\small $^{1}$Department of Computer Science, UiT The Arctic University of Norway}\\
\normalfont{\small $^{2}$Department of Oncology, University Hospital of North Norway}\\
\normalfont{\small $^{3}$Department of Clinical Medicine, UiT The Arctic University of Norway}\\
\normalfont{\small $^{4}$Department of Community Medicine, UiT The Arctic University of Norway}\\
\normalfont{\small $^{5}$Department of Medical Biology, UiT The Arctic University of Norway} \\
\normalfont{\small $^{6}$Department of Clinical Pathology, University Hospital of North Norway} %\vspace{2em}
}

\begin{document}
\maketitle

\section*{Abstract}

% \begin{abstract}
% Developing clinically useful cell-level analysis tools in digital pathology remains challenging due to limitations in dataset granularity, inconsistent annotations, computational demands of advanced models, and difficulties in integrating new technologies into clinical workflows. To address these challenges, we propose a multi-faceted solution that enhances data quality, model performance, and usability to create a lightweight and extensible cell segmentation and classification model.

% First, we update data labels by employing a cross-relabeling process that refines the labels of two existing datasets, PanNuke and MoNuSAC, to create a new unified dataset with enhanced granularity, encompassing seven distinct cell types. Second, we leverage the H-Optimus foundation model as a fixed encoder to improve feature representation for simultaneous cell segmentation and classification tasks. Third, to address the computational demands of foundation models, we employ knowledge distillation to reduce model size and complexity while maintaining comparable performance. Finally, to facilitate integration into clinical workflows, we integrate the distilled model into the QuPath software, a widely used open-source platform in digital pathology.

% Our results demonstrate improvements in cell segmentation and classification performance using the H‑Optimus-based model compared to a CNN-based model. Specifically, the average $R^2$ improved from 0.575 to 0.871, and the average $PQ$ score improved from 0.450 to 0.492, indicating better alignment with actual cell counts and enhanced segmentation and classification quality. Furthermore, the distilled student model maintains performance comparable to the larger foundation model while reducing the parameter count by a factor of 48.
% Overall, by reducing computational complexity and integrating it into existing workflows, the proposed approach may significantly impact diagnostic processes, reduce the workload of pathologists, and contribute to improved patient outcomes. Though our approach shows potential enhancements in efficiency and usability of cell segmentation and classification models in digital pathology, extensive validation is needed to deploy these models in clinical practice.
% \end{abstract}

%%% shortened abstract
\begin{abstract}
Developing clinically useful cell-level analysis tools in digital pathology remains challenging due to limitations in dataset granularity, inconsistent annotations, high computational demands, and difficulties integrating new technologies into workflows. To address these issues, we propose a solution that enhances data quality, model performance, and usability by creating a lightweight, extensible cell segmentation and classification model. 

First, we update data labels through cross-relabeling to refine annotations of PanNuke and MoNuSAC, producing a unified dataset with seven distinct cell types. Second, we leverage the H-Optimus foundation model as a fixed encoder to improve feature representation for simultaneous segmentation and classification tasks. Third, to address foundation models' computational demands, we distill knowledge to reduce model size and complexity while maintaining comparable performance. Finally, we integrate the distilled model into QuPath, a widely used open-source digital pathology platform. 

Results demonstrate improved segmentation and classification performance using the H-Optimus-based model compared to a CNN-based model. Specifically, average $R^2$ improved from 0.575 to 0.871, and average $PQ$ score improved from 0.450 to 0.492, indicating better alignment with actual cell counts and enhanced segmentation quality. The distilled model maintains comparable performance while reducing parameter count by a factor of 48. By reducing computational complexity and integrating into workflows, this approach may significantly impact diagnostics, reduce pathologist workload, and improve outcomes. Although the method shows promise, extensive validation is necessary prior to clinical deployment.
\end{abstract}
\clearpage

\section{Introduction}
In digital pathology, accurate segmentation and classification of cells are crucial for many diagnostic, prognostic, and predictive analyses \cite{Jaber_Beziaeva_etal._2019,Lin_Pan_etal._2022,Park_Ock_etal._2022,Shen_Choi_etal._2024}. Nowadays, developments in computational pathology offer multiple solutions \cite{H._Qu_P._Wu_etal._2020,Javed_Mahmood_etal._2020} to utilize cell-level datasets to train machine learning models that solve these problems. The quality and specificity of training datasets are critical for robust and accurate models. Adhering to the principle of "garbage in, garbage out", it is essential to ensure that these datasets are extensively and accurately labeled with distinct classes that reflect the diverse biological characteristics of different cell types. Unfortunately, the number of open-source datasets comprising such high-quality annotations is limited. Existing cell segmentation datasets \cite{Gamper_Koohbanani_etal._2019,Graham_Vu_etal._2019,Verma_Kumar_etal._2021} may offer extensive annotations for certain cell types while providing more general labels for others. For example, in PanNuke, which is one of the largest open-source datasets comprising labeled cells, various types of morphologically and functionally different inflammatory cells like macrophages and lymphocytes are clustered in a broad "inflammatory" class. Consequently, these classes are frequently omitted from analyses or aggregated into broader meta-classes \cite{Gamper_Koohbanani_etal._2020} and likely interfere with other cell classes included in the dataset. This and similar inconsistencies in annotation granularity limit the ability of machine learning models to learn the comprehensive and nuanced features necessary for accurate cell segmentation and classification. To address these challenges, methods for refining and standardizing dataset annotations are essential to enhance the quality of training data.

A complementary approach to mitigate the absence of high-quality training data is the use of foundation models. Foundation models as encoders are defined as large-scale, versatile networks pre-trained on vast, diverse datasets using self-supervised learning, contrasting with convolutional neural network (CNN) pre-trained encoders that rely on supervised learning with labeled data. In practice, foundation models leverage enormous amounts of weakly or unlabeled data from millions of whole slide images (WSIs) and employ self-attention mechanisms to capture long-range dependencies and global context \cite{Chen_Ding_etal._2024,Saillard_Jenatton_etal._2024,Vorontsov_Bozkurt_etal._2024,Xu_Usuyama_etal._2024}. As a consequence, foundation models are able to produce transferable feature representations across different cell types and tissue environments. The feature representations can be leveraged by decoder networks to produce segmentation masks and pixel-level classifications. Because foundation models have comprehensive feature representations, they can be effectively fine-tuned using much smaller amounts of cell-level data compared to the large datasets needed to train models from scratch. Furthermore, foundation models incorporate adversarial training elements or contrastive learning \cite{Chen_Ding_etal._2024,Xu_Usuyama_etal._2024}, enhancing their resilience and adaptability by exposing them to challenging and varied scenarios during training. This may result in more generalizable models, often making them well-suited for diverse and complex tasks in digital pathology.

Despite the inherent advantages of foundation models, their deployment for practical use faces its own obstacles. In particular, they require substantial computational power, financial investments and rigorous testing to ensure reliability and efficacy for a given task \cite{Akkus_Dangott_etal._2022,Dragomir_Cocuz_etal._2022,Go_2022,Jafri_Farooqui_etal._2024}. Moreover, while foundation models enhance feature representation and performance, they depend on the quality of available annotations for decoder fine-tuning and, like any other model, cannot resolve existing inconsistencies or ambiguities in data labels. Therefore, there remains a critical need for solutions that address both data quality and practical deployment considerations.
Further, integrating new technologies into existing clinical workflows often encounters resistance, as it necessitates adjustments to established diagnostic processes. So, there is a need to develop solutions that could be integrated into current practices, minimizing the burden on medical professionals to adopt new tools \cite{King_Williams_etal._2023}.

Existing solutions \cite{Goldsborough_Philps_etal._2024,Hörst_Rempe_etal._2024}, while addressing some aspects of these challenges, fall short in providing a comprehensive approach. To address the data quality and clinical deployment issues, we propose a multi-faceted solution that encompasses data refinement, model optimization, and integration with existing pathology tools (\hyperref[fig:fig1]{Figure 1}). The outcome is a lightweight cell segmentation and classification model that can be integrated into digital pathology workflows for practical clinical use.

\begin{figure}[h!]
    \centering
    \includegraphics[width=\textwidth, height=0.82\textheight, keepaspectratio]{images/Figure_1.pdf}
    \caption{Overview of the proposed solution, including 1) Data refinement using cross-relabeling, 2) Teacher model development and fine tuning, 3) Student model optimization with knowledge distillation and 4) Student model and QuPath integration}
    \label{fig:fig1}
\end{figure}
\clearpage

Our approach begins with preparing the data for the fine-tuning and training of the machine learning models. We create a refined dataset, acquired via cross-relabeling two cell-level datasets, enhancing annotation specificity and consistency of the labeled data. Subsequently, we create a cell segmentation and classification model based on the foundation model. We leverage the foundation model as a fixed encoder and fine-tune a decoder using the refined dataset to improve generalization across diverse tissue- and cell types.
To ensure that the model remains lightweight and deployable in a possibly resource-constrained environment, we employ knowledge distillation to approximate the functionality of the foundation model. Finally, to facilitate the practical application of our model in digital pathology workflows, we integrate it with the QuPath \cite{Bankhead_Loughrey_etal._2017} application. Each methodological component contributes to the overarching goal of enhancing model performance, generalizability, and usability in clinical settings.

The primary contributions of this paper are:
\begin{enumerate}
    \item \textit{Data labels refinement through cross-relabeling:}
    
    We propose a new method for refining labels of cell-level datasets through cross-relabeling. This method employs classification models to re-label broad and ambiguous instances, resulting in a more diverse dataset. Our evaluation demonstrates that these classification models achieve high accuracy on test subsets, indicating the reliability of the method for label refinement.

    \item \textit{Enhanced model performance via foundation models:}
    
    We employ a foundation model as a feature extractor for the cell segmentation and classification task. In comparison with training a CNN model from scratch, the foundation model backbone only needs fine-tuning, which significantly reduces training time, computational resources and data requirements. We show that using a foundation model encoder leads to better performance in cell segmentation and classification networks than using a CNN-based encoder. This improvement may enable the model to generalize more effectively across various tissue types and imaging methods.
    
    \item \textit{Model optimization through knowledge distillation:}
    
    We show that a smaller student model trained using knowledge distillation on the refined dataset obtained via our cross-relabeling approach from a foundation model achieves comparable performance in cell segmentation and quantification tasks. As a result, this model is more suitable for deployment in environments without high-performance computing resources.
    
    \item \textit{Integration with QuPath:}
    
    We integrate the distilled cell segmentation and classification model into QuPath, a widely used open-source digital pathology platform, to accelerate clinical adaptation by enabling pathologists to more easily incorporate advanced computational tools into their existing workflows.
\end{enumerate}

Through these methodological steps, we aim to bridge the gap between advanced machine learning techniques and practical clinical applications, making accurate and efficient digital pathology accessible in a broader range of healthcare settings.

\section{Refining Existing Datasets Using Cross-Relabeling}
To address the limitations of sparse and ambiguous labeling of cell-level datasets, we propose a generalizable cross-relabeling strategy that can be applied to any dataset containing broadly categorized or imprecisely labeled cell types. This approach involves training and subsequently leveraging classification models to refine broad categories into more specific or biologically relevant classes.
When applied to cell-level data, the methodology includes extracting individual cell images from the dataset patches, preprocessing these images to standardize the size and accommodate partial cells, and then training deep learning classifiers capable of distinguishing between the finer cell subtypes within the coarser categories. 
To illustrate our approach, we focus on the PanNuke \cite{Gamper_Koohbanani_etal._2020, Gamper_Koohbanani_etal._2019} and MoNuSAC \cite{Verma_Kumar_etal._2021} datasets that we have used to train models for cell quantification in our previous works \cite{Shvetsov_Grønnesby_etal._2022,Shvetsov_Sildnes_etal._2024}. We find that for better cell differentiation we have to introduce more granular labels. PanNuke includes a broad classification of "inflammatory" cells, encompassing lymphocytes, macrophages, and neutrophils. Each cell type differs significantly in structure, function, and clinical relevance. Conversely, MoNuSAC uses the label "epithelial" for a class that comprises both benign epithelial cells and malignant neoplastic cells. This practice makes it challenging to differentiate between benign and malignant epithelial cells in the dataset, which is a critical distinction when identifying tumor areas within tissue samples. To address these issues, we implement a cross-relabeling strategy as shown in \hyperref[fig:fig2]{Figure 2}. The key components are two classification models: one is trained on singular cell images from PanNuke data to classify the epithelial meta-class into epithelial and neoplastic classes. The other is trained on MoNuSAC to refine the inflammatory class into lymphocytes, neutrophils, and macrophages.

\begin{figure}[h!]
    \centering
    \includegraphics[width=\textwidth]{images/Figure_2.pdf}
    \caption{Refined dataset generation via cross relabeling}
    \label{fig:fig2}
\end{figure}

The refining approach consists of three consecutive steps. The first is the preprocessing step, in which we extract individual cells from both datasets (\hyperref[fig:fig3]{Figure 3}). The specifics of PanNuke and MoNuSAC patch preparation before cell preprocessing are provided in \hyperref[chap:S1]{Appendix S1}.

\begin{figure}[h!]
    \centering
    \includegraphics[width=\textwidth]{images/Figure_3.pdf}
    \caption{Cell instances preprocessing including (1) cell map extraction, (2) bounding box delineation, (3) adjusting cell boxes and (4) cropping and resizing of cell images}
    \label{fig:fig3}
\end{figure}

During preprocessing, we extract cell type maps from the ground truth label mask and calculate bounding boxes around each cell instance. To accommodate partial cells at patch borders, a common issue in cropped patch images, we employ mirror padding and extend the field of view of the cell label by 15 pixels to capture adjacent cells. We then crop and resize the identified regions to $64 \times 64$ pixels using bicubic interpolation.

The preprocessed PanNuke dataset comprises 68,031 neoplastic and 23,207 epithelial cell images, while MoNuSAC comprises  33,104 lymphocytes, 1,252 neutrophils, and 1,695 macrophages, which we subsequently use in training cell classification models and classifying the cell image data \hyperref[fig:S2]{Appendix Figure S2 (1)}. 

The next step is to train two distinct ResNet50-based classifiers tailored to address the specific labeling challenges inherent in each dataset. We use ResNet50 for classification models due to its proven effectiveness for image classification tasks in histopathology \cite{pan2022reviewmachinelearningapproaches}, and its compatibility with small images. For the PanNuke dataset, we design the classifier, trained on MoNuSAC data, to disaggregate the heterogeneous "inflammatory" cell category into distinct subtypes: lymphocytes, macrophages, and neutrophils. Similarly, for the MoNuSAC dataset, the classifier is trained on PanNuke data and distinguishes between benign and malignant epithelial cells within the overarching "epithelial" label. By applying these targeted classifiers to their respective datasets, we assign more specific labels to individual cell instances, thus enabling us to create a unified dataset.
To ensure a balanced representation of classes, we train both models on datasets that had been equalized to match the size of the least represented class. Thus, we obtain datasets comprising 23,207 samples per class for PanNuke and 1,252 samples per class for MoNuSAC data. Next, we partition both of them into training (70\%), validation (20\%), and testing (10\%) subsets. To mitigate the risk of overfitting, we use a single dropout layer with a rate of p=0.5 in both models and data augmentation using randomized color perturbations, rotation, and horizontal and vertical flipping. We employ AdamW optimizer and the cross-entropy loss function for the training criterion.

To evaluate the two trained models, we measure the classification accuracy on the respective test subsets. The accuracies on the test subset for both classifiers are presented in \hyperref[tab:1]{Table 1}. The PanNuke model achieves an average accuracy of 93.57\%, with higher accuracy for neoplastic cells (96.06\%) compared to epithelial cells (86.26\%). The confusion matrix in Figure A3.1 shows that the model predominantly distinguishes accurately between epithelial and neoplastic tissues, with a substantial number of correct classifications and relatively few misclassifications. The MoNuSAC model demonstrates an average accuracy of 98.92\%, excelling in classifying lymphocytes (99.67\%) and macrophages (94.12\%), with lower performance for neutrophils (85.71\%). The confusion matrix in Figure A3.2 shows that the model identifies lymphocytes and performs reasonably well with macrophages and neutrophils.

\begin{table}[h!]
\renewcommand{\arraystretch}{1.5}
  \centering
  \caption{Cell classification results for PanNuke and MoNuSAC trained models (CI 95\%).}
  \label{tab:1}
  \begin{tabular}{|l|c|c|}
   \hline
   %\rowcolor{gray!30}
    Accuracy               & PanNuke model              & MoNuSAC model              \\
    \hline
    Average      & 0.936 (0.931--0.941)         & 0.989 (0.986--0.993)        \\
    \hline
    Neoplastic   & 0.961 (0.956--0.965)         & -                          \\
    \hline
    Epithelial   & 0.863 (0.849--0.877)         & -                          \\
    \hline
    Lymphocytes  & -                          & 0.997 (0.995--0.999)        \\
    \hline
    Neutrophils  & -                          & 0.857 (0.796--0.918)        \\
    \hline
    Macrophages  & -                          & 0.941 (0.906--0.976)        \\
    \hline
  \end{tabular}
\end{table}

Finally, during the last step, we use the model trained on PanNuke data for epithelial cells in MoNuSAC and the model trained on MoNuSAC for the inflammatory cells class in PanNuke. Specifically, we use classifier models to relabel epithelial cells in MoNuSAC and inflammatory cells in PanNuke data. Then we combine cells with refined labels and the rest of the cells in both datasets to create a refined dataset (\hyperref[fig:S2]{Appendix Figure S2 (2)}). The process of relabeling cells and visualizing them on a patch is shown in \hyperref[fig:fig4]{Figure 4}. The cell counts in the refined dataset are provided in \hyperref[tab:S4]{Appendix Table S4}.

\begin{figure}[h!]
    \centering
    \includegraphics[width=\textwidth, height=0.42\textheight, keepaspectratio]{images/Figure_4.pdf}
    \caption{Cell relabeling procedure for epithelial and inflammatory cell classes}
    \label{fig:fig4}
\end{figure}

%\hfill

Relabeling and combining datasets have been explored in a prior study \cite{Parulekar_Kanwat_etal._2023}, where consecutive fine-tuning on multiple datasets was employed to account for hierarchical class label structures. While the method presented in \cite{Parulekar_Kanwat_etal._2023} is intuitive, it often lacks consistency and requires multiple fine-tuning runs, which can be cumbersome and time-consuming. 
In contrast, cross-relabeling simplifies this process by using specialized classification models tailored to each dataset's specific labeling challenges. This approach provides better transparency and produces a unified dataset encompassing seven distinct cell types across multiple tissue samples, enhancing data diversity for further model training or fine-tuning.

Despite these improvements, cross-relabeling does not entirely resolve issues related to poor labeling quality or the amount of labeled data. Specifically, our results show lower accuracies persist for underrepresented classes, such as macrophages, which may stem from a limited sample availability and intrinsic challenges in distinguishing these cells based solely on H\&E staining. Furthermore, while our method enhances label specificity, it relies on the initial quality of the broad labels; thus, any fundamental inaccuracies in the original annotations can propagate through the relabeling process. Addressing the overall problem of limited data labels may require integrating additional data sources or utilizing complementary immunohistochemical staining methods.
Although the reported performance metrics are obtained from evaluations on the native test sets of each dataset, it is important to note that the primary application of these classifiers is to perform cross-relabeling, where a model trained on one dataset (e.g., PanNuke) is applied to another (e.g., MoNuSAC) and vice versa. We acknowledge that a more systematic evaluation of cross-dataset generalization is needed and could be performed in future work.

Overall, the refined dataset produced by our approach can enhance the supervised training or fine-tuning of cell segmentation and classification models, especially those that utilize pre-trained foundation models to improve feature extraction robustness. In addition, these models can detect nuanced classes that enable researchers to conduct more detailed analyses of biological processes in computational pathology.

\section{Foundation models for robust cell segmentation and classification}

Accurate cell segmentation and classification in digital pathology are hindered by limited labeled data and the fact that conventional CNNs are unable to capture global contextual information due to their local receptive field constraints \cite{Gheflati_Rivaz_2022,Yang_Marcus_etal.}. Traditional approaches in cell quantification have predominantly relied on CNN encoders, such as ResNet50, given their proven effectiveness in semantic segmentation tasks \cite{Deshmane_2023,Graham_Vu_etal._2019,Mukasheva_Koishiyeva_etal._2024,Stringer_Wang_etal._2021}. However, approaches that include fine-tuning of pretrained CNNs, data augmentation, and stain normalization to partially increase data variability and address staining differences often fail to achieve the necessary generalization and robustness across diverse tissue types and staining conditions \cite{G._Wang_W._Li_etal._2018,Gao_Bagci_etal._2018,Karim_El_Khoury_Martin_Fockedey_etal._2021}.

To overcome these challenges, we leverage an encoder-decoder network that uses a foundation model as the encoder and a CNN upsampling decoder (\hyperref[fig:fig5]{Figure 5}) for simultaneous cell segmentation and classification in 2D patches extracted from WSIs. Foundation models with transformer-based architectures are viable alternatives to CNN-based encoders \cite{Shamshad_Khan_etal._2023,Sourget_2023}. They enable the creation of more advanced architectures that can decode or transform learned features more effectively \cite{Chen_Duan_etal._2023,Cheng_Misra_etal._2022,Xie_Wang_etal._2021}.

\begin{figure}[h!]
    \centering
    \includegraphics[width=\textwidth]{images/Figure_5.pdf}
    \caption{UNETR-like model with foundational model as backbone}
    \label{fig:fig5}
\end{figure}

By utilizing a transformer-based encoder, we incorporate global contextual information into the feature extraction process, which is a key advantage of such architectures \cite{Chen_Lu_etal._2021}. This foundation model integration facilitates accurate pixel-wise segmentation and classification without the need for extensive encoder training, thereby potentially improving generalization across varied cellular structures and tissue types.
In our implementation, we employ a modified UNETR \cite{Hatamizadeh_Tang_etal._2021} architecture that combines a vision transformer (ViT) \cite{Dosovitskiy_Beyer_etal._2021} encoder with a CNN-based decoder. The encoder utilizes the pretrained H-Optimus foundation model, which contains 1.1 billion parameters and is trained on over 500,000 H\&E stained WSIs \cite{Saillard_Jenatton_etal._2024}. We extract outputs from four evenly spaced transformer blocks $Z_i$, where $i \in [1, 14, 26, 38]$, to serve as residual connections for the CNN decoder. We select these blocks based on our observation that features from non-adjacent levels of the encoder lead to better overall performance on the test subset.

The CNN decoder upsamples the feature representations, acquired from the transformer blocks, to generate an intermediate vector that is handled by two task-specific layers that generate cell segmentation and classification masks. The first task-specific layer is the ‘Cellpose head’,  which is used to delineate cell instances. The layer generates horizontal and vertical gradient maps to form vector fields that are refined through gradient tracking in a post-processing step using the Cellpose algorithm \cite{Stringer_Wang_etal._2021}, known for its efficacy in cell segmentation tasks and generalizability across multiple domains \cite{Pachitariu_Stringer_2022,Stringer_Pachitariu_2024}. The second task-specific layer is the "Cell type head", which assigns labels to individual pixels. In the post-processing step, we determine the output classification label of each segmented cell instance by majority voting over the labeled pixels that comprise the cell in the segmentation map.

To evaluate model performance and measure the impact of adding a foundation model as backbone, we compare it to a ResNet50-based model. ResNet50 is a widely used solution for encoders in segmentation architectures in the medical domain \cite{Deshmane_2023,Graham_Vu_etal._2019,Mukasheva_Koishiyeva_etal._2024,Stringer_Wang_etal._2021}. For the H-Optimus-based model, we utilize frozen weights for the encoder and only fine-tune the decoder to take advantage of the extensive pre-training of the foundation model. For the ResNet50-based model we start with ImageNet \cite{Deng_Dong_etal.} weights and train both encoder and decoder parts. Hyperparameters for the training step are set to be identical, where possible, for comparable evaluation. 
For this evaluation, we deliberately use the PanNuke dataset to provide a standardized and controlled comparison between the H‑Optimus and ResNet50-based models (\hyperref[fig:S2]{Appendix Figure S2 (3)}). Specifically, we use two of the default PanNuke dataset splits (66\%) for training and validation, and reserve the third split (33\%) for testing.

To address the challenge of cell class imbalance in the PanNuke dataset, which is a common characteristic in most cell-level H\&E patch datasets, both models’ training processes employ a weighted loss function comprising cross-entropy and focal loss \cite{Lin_Goyal_etal._2018}. The focal loss component is adjusted with coefficients derived from each cell class' instance frequency, emphasizing learning from underrepresented classes and enhancing the model's sensitivity to rare but significant cellular patterns. The cross-entropy loss is augmented with spectral decoupling regularization \cite{Pezeshki_Kaba_etal._2021,Pohjonen_Stürenberg_etal._2022} and spatially varying label smoothing \cite{Islam_Glocker_2021}, which potentially stabilizes training and improves generalization in case of complex tissue morphologies. For optimization, we employ the \textit{AdamW} \cite{Loshchilov_Hutter_2019} to counter unbalanced class scenarios, with cosine annealing learning rate scheduler.

We utilize the scikit-learn library \cite{Van_der_Walt_Schönberger_etal._2014} and HoVer-Net \cite{Graham_Vu_etal._2019} implementations of $R^2$ (the coefficient of determination) and $PQ$ (panoptic quality) to evaluate our experiments. Complete mathematical formulations and detailed explanations of these metrics are provided in \hyperref[chap:S5]{Appendix S5}. To compute confidence intervals, we use nonparametric bootstrapping, where after calculating the metric on the full sample, we generated 1000 bootstrap replicates by resampling with replacement and then determined the 95\% confidence intervals as the 2.5th and 97.5th percentiles of the resulting empirical distribution.

%\hfill

The model comparisons are summarized in \hyperref[tab:2]{Table 2}. The H‑Optimus-based model achieves higher $R^2$ across all cell classes compared to the ResNet50-based model, which means that its predictions are more closely aligned with the PanNuke cell counts, indicating a stronger correlation with the observed data. Notably, the improvement of $R^2_{dead}$ may be an indicator of better global contextual representations provided by the foundation model backbone. In terms of segmentation and classification quality combined, measured by the PQ score, the H‑Optimus-based model demonstrates notable improvements across most cell classes. Overall, the average $R^2$ improved from 0.575 to 0.871, while the average $PQ$ score improved from 0.450 to 0.492, demonstrating better performance of the H-Optimus-based model.

\begin{table}[h!]
\renewcommand{\arraystretch}{1.5}
  \centering
  \caption{Cell quantification metrics for baseline and proposed models (CI 95\%).}
  \label{tab:2}
  \begin{tabular}{|l|c|c|}
    \hline
    %\rowcolor{gray!30}
    Metric             & Resnet50-based            & H-optimus-based              \\
    \hline
    $R^2_{neoplastic}$    & 0.681 (0.576--0.769)       & \textbf{0.941 (0.917--0.960)} \\
    \hline
    $R^2_{inflammatory}$  & 0.863 (0.778--0.903)       & \textbf{0.949 (0.918--0.966)} \\
    \hline
    $R^2_{connective}$    & 0.600 (0.488--0.698)       & 0.609 (0.436--0.772)          \\
    \hline
    $R^2_{dead}$          & 0.097 (-11.389--0.669)     & 0.925 (0.404--0.982)          \\
    \hline
    $R^2_{epithelial}$    & 0.635 (0.490--0.747)       & \textbf{0.930 (0.886--0.964)} \\
    \hline
    $PQ_{neoplastic}$       & 0.517 (0.499--0.535)       & \textbf{0.589 (0.575--0.604)} \\
    \hline
    $PQ_{inflammatory}$     & 0.455 (0.429--0.482)       & \textbf{0.528 (0.507--0.549)} \\
    \hline
    $PQ_{connective}$       & 0.416 (0.400--0.431)       & \textbf{0.451 (0.436--0.465)} \\
    \hline
    $PQ_{dead}$             & 0.374 (0.342--0.408)       & 0.292 (0.209--0.365)          \\
    \hline
    $PQ_{epithelial}$       & 0.488 (0.460--0.519)       & \textbf{0.599 (0.579--0.618)} \\
    \hline
  \end{tabular}
\end{table}

Our results  show that integrating the H‑Optimus foundation model within the UNETR architecture enhances the model's ability to segment and classify cells across diverse tissues from PanNuke data. The pretrained transformer encoder provides robust feature representations, resulting in higher average $R^2$ and $PQ$ scores compared to the CNN-based model. This leads to more reliable cell quantification and more accurate downstream analysis. Additionally, the streamlined fine-tuning process reduces computational overhead and training time, making the model more adaptable for new data.

Despite these advancements, the foundation model-based approach does not fully resolve all challenges related to cell segmentation and classification. We observe lower metric scores for underrepresented classes in the training data. Furthermore, foundation models typically encompass billions of parameters, resulting in substantial computational and memory requirements. It therefore poses challenges for deployment in resource-constrained environments, limiting their practical applicability in certain clinical settings.

\section{Model optimization via Knowledge Distillation}

To address the limitations posed by the extensive size of foundation models, we implement knowledge distillation — a model compression technique that leverages the teacher-student paradigm \cite{Hinton_Vinyals_etal._2015}. By training a smaller, more efficient student model to replicate the output of a larger, pre-trained teacher model, we retain performance while significantly reducing the model's complexity and resource requirements (\hyperref[fig:fig6]{Figure 6}).

\begin{figure}[h!]
    \centering
    \includegraphics[width=\textwidth, height=0.45\textheight, keepaspectratio]{images/Figure_6.pdf}
    \caption{Knowledge distillation framework for training a student model using a pre-trained teacher}
    \label{fig:fig6}
\end{figure}

We employ knowledge distillation to compress the H‑Optimus-based teacher model into a more efficient student model. The teacher model is the modified UNETR architecture with the H‑Optimus foundation model described in the previous chapter. The student model is based on a UNet architecture augmented with residual connections and incorporates a smaller ViT encoder with 9 million parameters \cite{Steiner_Kolesnikov_etal._2022,Wightman_2019}. 

First, we fine-tune the teacher model using the refined dataset from the cross-relabeling procedure (Section 2). Initially we train the decoder of the teacher model while keeping the encoder weights frozen. We split the refined dataset into train (70\%), validation (20\%) and test (10\%) subsets (\hyperref[fig:S2]{Appendix Figure S2 (4)}). During fine-tuning, we use the train and validation subsets, while leaving the test subset for model evaluation. We set the training procedure and model hyperparameters to be identical to those that were used to demonstrate the utility of foundation models for the simultaneous cell segmentation and classification task.

Next, we perform knowledge distillation from teacher to student using the refined dataset used to fine-tune the teacher model. The student model is trained to replicate the teacher model's outputs. We utilize a specialized loss function that aligns the student's predicted probability distribution with the teacher's, incorporating the teacher's class probability distribution derived from the output. Following the methodology of Hinton et al. \cite{Hinton_Vinyals_etal._2015}, we experiment with various hyperparameter settings for the temperature ($T$) and the balancing coefficients ($\alpha$ and $\beta$) in the loss function. We vary $T$ from 1 to 20 and adjust $\alpha$ and $\beta$ to balance the distillation and student losses. Through iterative tuning and evaluation, we identify that setting $T=14$, $\alpha=0.3$, and $\beta=0.7$ yields a configuration that converges and closely approximates the teacher model's performance during training.

Finally, we assess the performance of both models using the $R^2$ and $PQ$ (defined in \hyperref[chap:S5]{Appendix S5}) on the test set of the refined dataset (\hyperref[tab:3]{Table 3}). We observe that the 95\% confidence intervals overlap for most cell types, so we cannot claim statistically significant performance differences between the teacher and student models. One exception appears in the neoplastic class. The teacher model produces an $R^2$ of 0.919, while the student model shows an $R^2$ of 0.852. In addition, the student model achieves higher $PQ$ values for the neoplastic and connective classes, though the confidence intervals show overlap.

\begin{table}[h!]
\renewcommand{\arraystretch}{1.5}
  \centering
  \caption{Cell quantification metrics for teacher and distilled student models (CI 95\%).}
  \label{tab:3}
  \begin{tabular}{|l|c|c|}
    \hline
    %\rowcolor{gray!30}
    Metric & Teacher & Student \\
    \hline
    $R^2_{neoplastic}$    & \textbf{0.919} (0.898--0.939) & 0.852 (0.800--0.891) \\
    \hline
    $R^2_{lymphocyte}$    & 0.969 (0.956--0.977)         & 0.969 (0.956--0.978) \\
    \hline
    $R^2_{connective}$    & 0.694 (0.548--0.809)         & 0.618 (0.469--0.741) \\
    \hline
    $R^2_{dead}$          & 0.755 (0.400--0.908)         & 0.424 (0.100--0.731) \\
    \hline
    $R^2_{epithelial}$    & 0.922 (0.870--0.958)         & 0.843 (0.738--0.917) \\
    \hline
    $R^2_{macrophage}$    & 0.384 (-0.369--0.724)        & 0.704 (0.352--0.859) \\
    \hline
    $R^2_{neutrofil}$     & 0.854 (0.578--0.929)         & 0.833 (0.502--0.925) \\
    \hline
    $PQ_{neoplastic}$       & 0.581 (0.569--0.593)         & 0.601 (0.588--0.613) \\
    \hline
    $PQ_{lymphocyte}$       & 0.536 (0.520--0.553)         & 0.563 (0.544--0.579) \\
    \hline
    $PQ_{connective}$       & 0.436 (0.421--0.451)         & 0.457 (0.441--0.474) \\
    \hline
    $PQ_{dead}$             & 0.272 (0.235--0.315)         & 0.279 (0.201--0.369) \\
    \hline
    $PQ_{epithelial}$       & 0.522 (0.500--0.545)         & 0.530 (0.506--0.555) \\
    \hline
    $PQ_{macrophage}$       & 0.524 (0.459--0.588)         & 0.474 (0.405--0.543) \\
    \hline
    $PQ_{neutrofil}$        & 0.541 (0.490--0.592)         & 0.565 (0.522--0.607) \\
    \hline
  \end{tabular}
\end{table}


We further decompose the $PQ$ metric into its $SQ$ and $DQ$ components (\hyperref[tab:S6]{Appendix Table S6}). Both models produce nearly identical $SQ$ values, which indicates that they predict instance boundaries with similar precision. Although the student model shows some improvement in $DQ$ scores for certain classes, the confidence intervals overlap and do not confirm a statistically significant difference.

We observe that the student and teacher models yield comparable detection performance despite the student model using a much smaller and simpler architecture. A model with fewer parameters reduces the risk of overfitting when training data are scarce relative to the model’s complexity \cite{Farias_Ludermir_etal._2022}. The knowledge distillation process also encourages the student model to focus on the most generalizable detection features learned from the teacher. These factors enable the student model to achieve similar detection performance across different cell types.

Additionally, considering the model sizes reported in \hyperref[tab:4]{Table 4}, the distilled model achieves a significant reduction compared to the teacher model, with a 48-fold decrease in parameter count and a 5.5-fold reduction in on-disk size. In inference mode, the teacher model requires 16 GB of VRAM for a batch size of 32, while the distilled model only needs 3 GB of VRAM for the same batch size. These reductions make the distilled model significantly more practical for fine-tuning and deployment in resource-constrained environments.

\begin{table}[h!]
\renewcommand{\arraystretch}{1.5}
  \centering
  \caption{Parameter counts and size of teacher and distilled model}
  \label{tab:4}
  \adjustbox{max width=\textwidth}{%
  \begin{tabular}{|l|c|c|c|}
    \hline
    %\rowcolor{gray!30}
    Metric & H-optimus-based (Teacher) & mobileViT-based (Student) & Magnitude of difference \\
    \hline
    Parameters count       & 1,158,917,906   & \textbf{24,093,393}   & \textbf{48x}  \\
    \hline
    Estimated Total Size (MB) & 87,912       & \textbf{15,935}    & \textbf{5.5x} \\
    \hline
  \end{tabular}%
}
\end{table}

%\hfill

With recent advancements in complex network architectures and the use of pretrained encoders to achieve state-of-the-art performance \cite{Baumann_Dislich_etal._2024,Hörst_Rempe_etal._2024} in cell segmentation and classification tasks, model size, computational complexity, and processing times have increased. This limits the scalability and accessibility of these models. As we demonstrate, this may be mitigated using knowledge distillation. Studies in the field of natural language processing have demonstrated the efficacy of knowledge distillation in retaining the capabilities of the teacher model while achieving significant reductions in size and complexity \cite{Huangpu_Gao_2024,Sun_Yu_etal.}. 

We demonstrate the feasibility of knowledge distillation in digital pathology, specifically for cell segmentation and classification tasks. Moreover, we achieve this performance while also significantly reducing the parameter count. In addressing the challenge of knowledge transfer, we found that distillation from a transformer-based model to a smaller transformer is more straightforward than attempting to map transformer features to CNN blocks. In our experiments, using a CNN-based network as a student results in worse cell quantification performance due to the structural constraints of CNN feature space dimensions. 

Although our primary approach relies on a transformer-based student model that performs well, it can be further optimized to incorporate advantages from CNN architectures. For example, employing alternative techniques such as using ViT adapters \cite{Chen_Duan_etal._2023} or $1 \times 1$ convolutions to adjust feature map sizes may be beneficial for harnessing CNN advantages like enhanced local feature extraction. Moreover, if additional performance improvements are desired, the process can be further enhanced by applying supplementary knowledge distillation techniques, such as self-distillation \cite{Zhang_Song_etal._2019} or online distillation \cite{Houyon_Cioppa_etal._2023}.

Despite these promising results, further validation on independent datasets is necessary to fully understand the model's limitations. Underrepresented classes may pose challenges when addressing complex cases. Pathologists need to validate these models to adopt them in clinical settings. While the distilled models are smaller and more deployable, a technological gap persists because pathologists traditionally rely on established methods for inspecting WSIs and diagnosing diseases. Addressing the complexities involved in deploying models for inference and supporting pathologists in adopting new tools is essential for integrating these models into clinical workflows.

\section{Model integration with QuPath}
Digital pathology tools with graphical user interfaces are essential for visualizing and analyzing WSIs. To make our student model useful in clinical pathology workflows, it needs to be integrated into a tool that enables inspecting regions, creating annotations, and providing quantitative analyses of biomarkers. Therefore, we integrate the trained student model from the previous chapter into the QuPath open‑source platform \cite{Bankhead_Loughrey_etal._2017}. QuPath provides the required annotation, visualization, and analysis tools to interpret complex histological data, including workflows for cell segmentation, classification, and quantification (\hyperref[fig:fig7]{Figure 7}). 

\begin{figure}[h!]
    \centering
    \includegraphics[width=\textwidth]{images/Figure_7.pdf}
    \caption{Visualization of model-generated cell quantification annotations (left) and the corresponding unannotated slide (right) in QuPath}
    \label{fig:fig7}
\end{figure}

To identify the regions in a WSI critical for prognosticating tumor development, such as specific tumor areas or border regions without overlapping healthy tissue, the pathologist uses QuPath to outline these regions. Then, the pathologist initiates a cell segmentation and classification script through the QuPath interface for the selected regions. The resulting annotations and quantified cell information are then directly overlaid onto the WSI in the QuPath interface. Additional design and implementation details are in \hyperref[chap:S7]{Appendix S7}. 

Two common approaches for integrating deep learning models into QuPath are Java‑based native QuPath extensions \cite{Goldsborough_Philps_etal._2024} and the execution of RESTful API requests to a model server coupled with handling the response via an extension, as demonstrated in the application of cell segmentation models applied to immunofluorescence images \cite{Sugawara_2023}. While the community is actively working on these integration strategies, there is currently no universal solution that fully addresses all integration and performance requirements.

Extensions may offer better integration with QuPath, allowing slightly improved performance and more widespread usage of the built-in QuPath models, but they lack the flexibility to customize models and modify their behavior. For example, the newest version of QuPath includes models such as StarDist \cite{Weigert_Schmidt} and InstanSeg \cite{Goldsborough_Philps_etal._2024} that can perform cell segmentation. Both models pose limitations when applied to simultaneous cell segmentation and classification. StarDist performs well only on convex, round shapes by design, whereas some neoplastic, inflammatory, and connective cells exhibit complex and non-convex shapes. InstanSeg provides only semantic segmentation without assigning classes to the segmented cells.

%\hfill

In contrast, our approach offers an alternative integration strategy. It utilizes the paquo library to directly interact with QuPath’s internal application programming interface from within Python. This enables data exchange and processing without the need for intermediate conversion steps and provides greater control over model customization, retraining, and the incorporation of custom processing steps.

The integration of our custom model with QuPath underscores its potential to significantly enhance the diagnostic process by reducing the time burden on pathologists and enabling them to focus on more complex interpretative tasks using familiar software. Leveraging a tool that is already well-established among pathologists increases the likelihood of its adoption into daily clinical workflows. The quantitative data generated through the automated workflow is critical for both clinical decision-making and research, facilitating more accurate biomarker analysis, enabling robust statistical evaluations, and supporting hypothesis generation and testing. Additionally, by streamlining cell segmentation and classification, the tool enhances the scalability and reproducibility of pathological assessments, ultimately contributing to improved diagnostic accuracy and patient outcomes.

\section{Conclusion and future work}

In this study, we address critical challenges in digital pathology and tackle the usability and deployment issues of the developed models in standard computing environments without the need for high-performance computing systems. Our multi-faceted approach encompasses data refinement through cross-relabeling, leveraging foundation models for robust cell segmentation and classification, optimizing model performance via knowledge distillation, and integrating the optimized model into the QuPath software for practical application. This approach is used to construct a capable, versatile, and adjustable model for cell segmentation and classification, with enhanced performance and usability.

\begin{sloppypar}
While our approach shows potential in the field of computational pathology, certain limitations persist. 
For example, our implementation currently exhibits lower performance in detecting macrophages. 
This serves as an instance of the broader challenge of accurately identifying complex cell types. In order to address this issue, extending our approach to incorporate additional data sources, exploring alternative modeling approaches, and integrating other imaging modalities such as immunohistochemical staining may help improve detection accuracy. Moreover, although the distilled model reduces computational demands, integrating advanced deep learning models into clinical practice requires addressing technological gaps and potential resistance to adopting new tools within established diagnostic processes.
\end{sloppypar}

Future work could focus on several key areas to refine the proposed approach and facilitate its adoption in clinical environments. Enhancing the cell-relabeling process with additional datasets \cite{Graham_Jahanifar_etal._2021} could improve the representation of underrepresented cell types and enhance overall model performance. Also, incorporating additional data sources, such as multi-modal imaging or complementary staining methods, may address limitations related to cell type differentiation and class imbalance. Exploring other foundation models \cite{Vorontsov_Bozkurt_etal._2024,Zimmermann_Vorontsov_etal._2024} or introducing additional modalities \cite{Ding_Wagner_etal._2024,Vaidya_Zhang_etal._2025} may provide alternative architectures better suited to specific tasks or offer improved efficiency. Implementing more complex knowledge distillation techniques \cite{Houyon_Cioppa_etal._2023,Zhang_Song_etal._2019} could further optimize the model's performance and adaptability. Additionally, deeper integration with QuPath or other digital pathology software could provide pathologists more control over cell quantification analysis directly within the QuPath interface, thereby increasing accessibility and usability. Such enhancements would not only refine model performance but also ensure greater adaptability and scalability within various clinical environments. Finally, extensive validation of the model by pathologists and benchmarking against independent datasets are essential steps toward establishing the model's reliability and fostering confidence in its clinical utility.

\section*{Acknowledgments} 
This work was funded in part by the Research Council of Norway grant no. 309439 SFI Visual Intelligence, and the North Norwegian Health Authority grant no. HNF1521-20.

\bibliographystyle{IEEEtran}
\begin{sloppypar}
\begin{thebibliography}{99}

\bibitem{chaplot2020neural} Chaplot, Devendra Singh, et al. "Neural topological slam for visual navigation." Proceedings of the IEEE/CVF conference on computer vision and pattern recognition. 2020.

\bibitem{maksymets2021thda} Maksymets, Oleksandr, et al. "Thda: Treasure hunt data augmentation for semantic navigation." Proceedings of the IEEE/CVF International Conference on Computer Vision. 2021.

\bibitem{mezghan2022memory} Mezghan, Lina, et al. "Memory-augmented reinforcement learning for image-goal navigation." 2022 IEEE/RSJ International Conference on Intelligent Robots and Systems (IROS). IEEE, 2022.

\bibitem{al2022zero} Al-Halah, Ziad, Santhosh Kumar Ramakrishnan, and Kristen Grauman. "Zero experience required: Plug \& play modular transfer learning for semantic visual navigation." Proceedings of the IEEE/CVF Conference on Computer Vision and Pattern Recognition. 2022.

\bibitem{ye2021auxiliary} Ye, Joel, et al. "Auxiliary tasks and exploration enable objectgoal navigation." Proceedings of the IEEE/CVF international conference on computer vision. 2021.

\bibitem{chaplot2020object} Chaplot, Devendra Singh, et al. "Object goal navigation using goal-oriented semantic exploration." Advances in Neural Information Processing Systems 33 (2020)

\bibitem{ramakrishnan2022poni} Ramakrishnan, Santhosh Kumar, et al. "Poni: Potential functions for objectgoal navigation with interaction-free learning." Proceedings of the IEEE/CVF Conference on Computer Vision and Pattern Recognition. 2022.

\bibitem{ramrakhya2022habitat} Ramrakhya, Ram, et al. "Habitat-web: Learning embodied object-search strategies from human demonstrations at scale." Proceedings of the IEEE/CVF Conference on Computer Vision and Pattern Recognition. 2022.

\bibitem{mousavian2019visual} Mousavian, Arsalan, et al. "Visual representations for semantic target driven navigation." 2019 International Conference on Robotics and Automation (ICRA). IEEE, 2019.

\bibitem{dhariwal2021diffusion} Dhariwal, Prafulla, and Alexander Nichol. "Diffusion models beat gans on image synthesis." Advances in neural information processing systems 34 (2021)

\bibitem{ho2022classifier} Ho, Jonathan, and Tim Salimans. "Classifier-free diffusion guidance." arXiv preprint arXiv:2207.12598 (2022).

\bibitem{nichol2021glide} Nichol, Alex, et al. "Glide: Towards photorealistic image generation and editing with text-guided diffusion models." arXiv preprint arXiv:2112.10741 (2021)

\bibitem{brooks2023instructpix2pix} Brooks, Tim, Aleksander Holynski, and Alexei A. Efros. "Instructpix2pix: Learning to follow image editing instructions." Proceedings of the IEEE/CVF Conference on Computer Vision and Pattern Recognition. 2023.

\bibitem{fu2023guiding} Fu, Tsu-Jui, et al. "Guiding instruction-based image editing via multimodal large language models." arXiv preprint arXiv:2309.17102 (2023).

\bibitem{geng2024instructdiffusion} Geng, Zigang, et al. "Instructdiffusion: A generalist modeling interface for vision tasks." Proceedings of the IEEE/CVF Conference on Computer Vision and Pattern Recognition. 2024.

\bibitem{zhou2024minedreamer} Zhou, Enshen, et al. "Minedreamer: Learning to follow instructions via chain-of-imagination for simulated-world control." arXiv preprint arXiv:2403.12037 (2024).

\bibitem{zhou2023esc} Zhou, Kaiwen, et al. "Esc: Exploration with soft commonsense constraints for zero-shot object navigation." International Conference on Machine Learning. PMLR, 2023.

\bibitem{yu2023l3mvn} Yu, Bangguo, Hamidreza Kasaei, and Ming Cao. "L3mvn: Leveraging large language models for visual target navigation." 2023 IEEE/RSJ International Conference on Intelligent Robots and Systems (IROS). IEEE, 2023.

\bibitem{gadre2023cows} Gadre, Samir Yitzhak, et al. "Cows on pasture: Baselines and benchmarks for language-driven zero-shot object navigation." Proceedings of the IEEE/CVF Conference on Computer Vision and Pattern Recognition. 2023.

\bibitem{shah2023navigation} Shah, Dhruv, et al. "Navigation with large language models: Semantic guesswork as a heuristic for planning." Conference on Robot Learning. PMLR, 2023.

\bibitem{cai2024bridging} Cai, Wenzhe, et al. "Bridging zero-shot object navigation and foundation models through pixel-guided navigation skill." 2024 IEEE International Conference on Robotics and Automation (ICRA). IEEE, 2024.

\bibitem{yu2023co} Yu, Bangguo, Hamidreza Kasaei, and Ming Cao. "Co-NavGPT: Multi-robot cooperative visual semantic navigation using large language models." arXiv preprint arXiv:2310.07937 (2023).

\bibitem{wu2024voronav} Wu, Pengying, et al. "Voronav: Voronoi-based zero-shot object navigation with large language model." arXiv preprint arXiv:2401.02695 (2024).

\bibitem{qin2023mp5} Qin, Yiran, et al. "Mp5: A multi-modal open-ended embodied system in minecraft via active perception." arXiv preprint arXiv:2312.07472 (2023).

\bibitem{du2024learning} Du, Yilun, et al. "Learning universal policies via text-guided video generation." Advances in Neural Information Processing Systems 36 (2024).

\bibitem{ajay2024compositional} Ajay, Anurag, et al. "Compositional foundation models for hierarchical planning." Advances in Neural Information Processing Systems 36 (2024).

\bibitem{liang2024skilldiffuser} Liang, Zhixuan, et al. "Skilldiffuser: Interpretable hierarchical planning via skill abstractions in diffusion-based task execution." Proceedings of the IEEE/CVF Conference on Computer Vision and Pattern Recognition. 2024.

\bibitem{heusel2017gans} Heusel, Martin, et al. "Gans trained by a two time-scale update rule converge to a local nash equilibrium." Advances in neural information processing systems 30 (2017).

\bibitem{zhang2018unreasonable} Zhang, Richard, et al. "The unreasonable effectiveness of deep features as a perceptual metric." Proceedings of the IEEE conference on computer vision and pattern recognition. 2018.

\bibitem{brown2020language} Brown, Tom B. "Language models are few-shot learners." arXiv preprint arXiv:2005.14165 (2020).

\bibitem{podell2023sdxl} Podell, Dustin, et al. "Sdxl: Improving latent diffusion models for high-resolution image synthesis." arXiv preprint arXiv:2307.01952 (2023).

\bibitem{brohan2022rt} Brohan, Anthony, et al. "Rt-1: Robotics transformer for real-world control at scale." arXiv preprint arXiv:2212.06817 (2022).

\bibitem{brohan2023rt} Brohan, Anthony, et al. "Rt-2: Vision-language-action models transfer web knowledge to robotic control." arXiv preprint arXiv:2307.15818 (2023).

\bibitem{li2024manipllm} Li, Xiaoqi, et al. "Manipllm: Embodied multimodal large language model for object-centric robotic manipulation." Proceedings of the IEEE/CVF Conference on Computer Vision and Pattern Recognition. 2024.

\bibitem{shah2023vint} Shah, Dhruv, et al. "ViNT: A foundation model for visual navigation." arXiv preprint arXiv:2306.14846 (2023).

\bibitem{liu2024visual} Liu, Haotian, et al. "Visual instruction tuning." Advances in neural information processing systems 36 (2024).

\bibitem{hu2021lora} Hu, Edward J., et al. "Lora: Low-rank adaptation of large language models." arXiv preprint arXiv:2106.09685 (2021).

\bibitem{qin2023supfusion} Qin, Yiran, et al. "SupFusion: Supervised LiDAR-camera fusion for 3D object detection." Proceedings of the IEEE/CVF International Conference on Computer Vision. 2023.

\bibitem{qin2024worldsimbench} Qin, Yiran, et al. "Worldsimbench: Towards video generation models as world simulators." arXiv preprint arXiv:2410.18072 (2024).

\bibitem{yu2025gamefactory} Yu, Jiwen, et al. "GameFactory: Creating New Games with Generative Interactive Videos." arXiv preprint arXiv:2501.08325 (2025).

\bibitem{zhou2024code} Zhou, Enshen, et al. "Code-as-Monitor: Constraint-aware Visual Programming for Reactive and Proactive Robotic Failure Detection." arXiv preprint arXiv:2412.04455 (2024).

\bibitem{zhang2024ad} Zhang, Zaibin, et al. "AD-H: Autonomous Driving with Hierarchical Agents." arXiv preprint arXiv:2406.03474 (2024).

\bibitem{wang2024toward} Wang, Chaoqun, et al. "Toward Accurate Camera-based 3D Object Detection via Cascade Depth Estimation and Calibration." arXiv preprint arXiv:2402.04883 (2024).

\bibitem{huang2024story3d} Huang, Yuzhou, et al. "Story3d-agent: Exploring 3d storytelling visualization with large language models." arXiv preprint arXiv:2408.11801 (2024).

\bibitem{savinov2018semi} Savinov, Nikolay, Alexey Dosovitskiy, and Vladlen Koltun. "Semi-parametric topological memory for navigation." arXiv preprint arXiv:1803.00653 (2018).

\bibitem{majumdar2022zson} Majumdar, Arjun, et al. "Zson: Zero-shot object-goal navigation using multimodal goal embeddings." Advances in Neural Information Processing Systems 35 (2022): 32340-32352.

\bibitem{yadav2023offline} Yadav, Karmesh, et al. "Offline visual representation learning for embodied navigation." Workshop on Reincarnating Reinforcement Learning at ICLR 2023. 2023.

\bibitem{yadav2023ovrl} Yadav, Karmesh, et al. "Ovrl-v2: A simple state-of-art baseline for imagenav and objectnav." arXiv preprint arXiv:2303.07798 (2023).

\bibitem{sun2024fgprompt} Sun, Xinyu, et al. "FGPrompt: fine-grained goal prompting for image-goal navigation." Advances in Neural Information Processing Systems 36 (2024).

\bibitem{zhu2017target} Zhu, Yuke, et al. "Target-driven visual navigation in indoor scenes using deep reinforcement learning." 2017 IEEE international conference on robotics and automation (ICRA). IEEE, 2017.

\bibitem{koh2024generating} Koh, Jing Yu, Daniel Fried, and Russ R. Salakhutdinov. "Generating images with multimodal language models." Advances in Neural Information Processing Systems 36 (2024).

\bibitem{krantz2022instance} Krantz, Jacob, et al. "Instance-specific image goal navigation: Training embodied agents to find object instances." arXiv preprint arXiv:2211.15876 (2022).

\bibitem{schulman2017proximal} Schulman, John, et al. "Proximal policy optimization algorithms." arXiv preprint arXiv:1707.06347 (2017).

\bibitem{anderson2018evaluation} Anderson, Peter, et al. "On evaluation of embodied navigation agents." arXiv preprint arXiv:1807.06757 (2018).

\bibitem{lin2024navcot} Lin, Bingqian, et al. "NavCoT: Boosting LLM-Based Vision-and-Language Navigation via Learning Disentangled Reasoning." arXiv preprint arXiv:2403.07376 (2024).

\bibitem{NavGPT} Zhou, Gengze, Yicong Hong, and Qi Wu. "Navgpt: Explicit reasoning in vision-and-language navigation with large language models." Proceedings of the AAAI Conference on Artificial Intelligence.

\bibitem{hahn2021no} Hahn, Meera, et al. "No rl, no simulation: Learning to navigate without navigating." Advances in Neural Information Processing Systems 34 (2021): 26661-26673.

\bibitem{li2025t2isafety} Li, Lijun, et al. "T2ISafety: Benchmark for Assessing Fairness, Toxicity, and Privacy in Image Generation." arXiv preprint arXiv:2501.12612 (2025).

\bibitem{an2024agfsync} An, Jingkun, et al. "AGFSync: Leveraging AI-Generated Feedback for Preference Optimization in Text-to-Image Generation." arXiv preprint arXiv:2403.13352 (2024).


\end{thebibliography}
\end{sloppypar}

\clearpage
\beginsupplement
\section*{Appendix}
\renewcommand{\thesubsection}{S\arabic{subsection}}

\subsection{\label{chap:S1}PanNuke and MoNuSAC preprocessing}
The PanNuke dataset comprises a set of 7,901 RGB patches, each with dimensions of $256 \times 256$ pixels, which we set as the standard patch size for our analysis. In contrast, the MoNuSAC dataset encompasses 294 images of heterogeneous dimensions. To standardize the MoNuSAC images with our experiments, we implement a standardization protocol. Specifically, for images exceeding the dimensions of $256 \times 256$ pixels, we segment them into equal-sized patches and apply mirror padding to the remaining portions to avoid information loss at the peripherals. Patches with dimensions less than $128 \times 128$ pixels are excluded from the dataset due to the insufficient resolution to capture relevant cellular details. For patches where either dimension falls between 128 and 256 pixels, we employ upsampling to achieve the standard patch size. As a result, we obtain a total of 2,823 RGB patches derived from the MoNuSAC dataset for subsequent analysis. For additional details on the MoNuSAC data preparation process, refer to the source code \cite{Shvetsov_2025a}.
\clearpage

\subsection{\label{chap:S2}Data usage for the methodology}

\counterwithin{figure}{subsection}
\renewcommand{\thefigure}{S\arabic{subsection}}

\begin{figure}[h!]
    \centering
    \includegraphics[width=\textwidth, height=0.85\textheight, keepaspectratio]{images/A2.pdf}
    \caption{Overview of the methodology for cross-labeling, dataset refinement, and model comparison. (1) Cross-relabeling - training and testing cell classification models, (2) Cross-relabeling - using cell classification models to create refined dataset, (3) Fine-tuning and training models for comparison, (4) Student knowledge distillation with refined dataset}
    \label{fig:S2}
\end{figure}
\clearpage

\subsection{\label{chap:S3}Confusion matrices for classification models}
\counterwithin{figure}{subsection}
\renewcommand{\thefigure}{S\arabic{subsection}.\arabic{figure}}

\begin{figure}[h!]
    \centering
    \includegraphics[width=\textwidth, height=0.4\textheight, keepaspectratio]{images/A3_1.pdf}
    \caption{Confusion matrix for PanNuke trained model}
    \label{fig:S3.1}
\end{figure}

\begin{figure}[h!]
    \centering
    \includegraphics[width=\textwidth, height=0.4\textheight, keepaspectratio]{images/A3_2.pdf}
    \caption{Confusion matrix for MoNuSAC trained model}
    \label{fig:S3.2}
\end{figure}

\clearpage

\subsection{\label{chap:S4}Datasets cell counts}

\counterwithin{table}{subsection}
\renewcommand{\thetable}{S\arabic{subsection}}

\begin{table}[h!]
\renewcommand{\arraystretch}{2.0}
\centering
\caption{\label{tab:S4}Cell counts for PanNuke, MoNuSAC and refined datasets. Numbers in parentheses indicate preprocessed cell counts for cell classifier models training and testing.}
%\adjustbox{max width=\textwidth}{%
\begin{tabular}{|l|c|c|c|}
\hline
%\rowcolor{gray!30}
Cell type & PanNuke & MoNuSAC & Refined \\
\hline
Neoplastic & 77,403 (68,031) & - & 105,451 \\
\hline
Epithelial & 26,572 (23,207) & - & 29,926 \\
\hline
Epithelial (benign and malignant) & - & 31,402 & - \\
\hline
Inflammatory & 32,276 & - & - \\
\hline
Lymphocytes & - & 37,045 (33,104) & 65,275 \\
\hline
Neutrophils & - & 1,355 (1,252) & 3,833 \\
\hline
Macrophage & - & 1,842 (1,695) & 3,410 \\
\hline
Dead & 2,908 & - & 2,908 \\
\hline
Connective & 50,585 & - & 50,585 \\
\hline
\end{tabular}
%
%}
\end{table}



\clearpage

\subsection{\label{chap:S5}Definition of validation metrics}
\counterwithin{equation}{subsection}
\renewcommand{\theequation}{\arabic{equation}}

\subsubsection{\label{chap:S5.1}R\textsuperscript{2}}
The coefficient of determination, denoted as $R^2$, is a statistical measure that represents the proportion of variance in the dependent variable that is predictable from the independent variables. In the context of cell quantification in pathology, $R^2$ is used to assess how well the predicted quantities of different cell types in a patch align with the actual quantities observed in the ground truth data, with higher values representing more accurate quantification. $R^2$ is defined as
\begin{equation*}
R^2 = 1 - \frac{\sum_{i=1}^n (y_i - \hat{y}_i)^2}{\sum_{i=1}^n (y_i - \bar{y})^2},
\end{equation*}
where $y_i$ represents the actual number of cells of a specific type in the $i$-th image, $\hat{y}_i$ represents the predicted number of cells of that type in the $i$-th image, $\bar{y}$ is the mean of the actual numbers across all images, and $n$ is the total number of images in the dataset.

The $R^2$ metric has a range of $(-\infty, 1]$. An $R^2$ of 1 indicates perfect prediction, where all predicted values exactly match the actual values. An $R^2$ of 0 suggests that the model explains none of the variability of the response data around its mean. If $R^2$ is negative, it indicates that the model performs worse than a model that simply predicts the mean of the actual values for all observations.

\subsubsection{\label{chap:S5.2}PQ}
Panoptic Quality ($PQ$) is a comprehensive metric used to evaluate the performance of segmentation models in tasks that require both instance segmentation and classification. $PQ$ provides a single score that encapsulates both the detection accuracy (i.e., how many objects were correctly identified) and the segmentation quality (i.e., how accurately the objects' boundaries were delineated). This metric is particularly useful in multiclass scenarios where each pixel is classified into distinct categories, such as different cell types in pathology images.

$PQ$ is calculated as the product of two terms: Detection Quality ($DQ$) and Segmentation Quality ($SQ$). It can be expressed as
\begin{equation*}
PQ = DQ \cdot SQ,
\end{equation*}
where
\begin{equation*}
DQ = \frac{TP}{TP + 0.5\, FP + 0.5\, FN},
\end{equation*}
\begin{equation*}
SQ = \frac{\sum_{(p, g) \in \mathcal{M}} IoU(p, g)}{TP}.
\end{equation*}
In these formulas, $TP$ denotes the number of correctly matched instances between ground truth and prediction, $FP$ denotes the predicted instances that have no corresponding ground truth, $FN$ denotes the ground truth instances that were not detected, $IoU(p, g)$ is the Intersection over Union for a pair of matched instances $p$ (prediction) and $g$ (ground truth), and $\mathcal{M}$ is the set of matched pairs.

The $PQ$ metric is calculated for each class and is averaged across classes to provide a global performance measure.

The $PQ$ score has a range of $[0, 1.0]$, where a higher score indicates better performance in both detecting and segmenting the instances correctly. A $PQ$ of 1 signifies perfect identification and segmentation of all instances, whereas a $PQ$ of 0 indicates that no instances were correctly identified and segmented.

\clearpage

\subsection{\label{chap:S6}Segmentation and Detection quality metrics for teacher and student models}

\begin{table}[h!]
\renewcommand{\arraystretch}{2.0}
\centering
\caption{Segmentation and detection quality for student and teacher models (CI 95\%)}
\label{tab:S6}
%\adjustbox{max width=\textwidth}{%
\begin{tabular}{|l|c|c|}
\hline
%\rowcolor{gray!30}
Metric & Teacher & Student \\
\hline
$SQ_{neoplastic}$ & 0.819 (0.815--0.823) & 0.824 (0.819--0.828) \\
\hline
$SQ_{lymphocyte}$ & 0.795 (0.788--0.802) & 0.790 (0.783--0.796) \\
\hline
$SQ_{connective}$ & 0.770 (0.762--0.776) & 0.780 (0.772--0.786) \\
\hline
$SQ_{dead}$ & 0.659 (0.623--0.688) & 0.657 (0.624--0.695) \\
\hline
$SQ_{epithelial}$ & 0.780 (0.770--0.790) & 0.788 (0.779--0.797) \\
\hline
$SQ_{macrophage}$ & 0.788 (0.760--0.810) & 0.757 (0.730--0.783) \\
\hline
$SQ_{neutrofil}$ & 0.782 (0.761--0.801) & 0.775 (0.759--0.792) \\
\hline
$DQ_{neoplastic}$ & 0.706 (0.692--0.719) & 0.727 (0.712--0.741) \\
\hline
$DQ_{lymphocyte}$ & 0.675 (0.656--0.698) & 0.713 (0.691--0.734) \\
\hline
$DQ_{connective}$ & 0.566 (0.546--0.584) & 0.583 (0.565--0.602) \\
\hline
$DQ_{dead}$ & 0.410 (0.361--0.465) & 0.435 (0.306--0.561) \\
\hline
$DQ_{epithelial}$ & 0.668 (0.639--0.694) & 0.673 (0.644--0.702) \\
\hline
$DQ_{macrophage}$ & 0.657 (0.583--0.727) & 0.615 (0.531--0.703) \\
\hline
$DQ_{neutrofil}$ & 0.691 (0.625--0.753) & 0.729 (0.679--0.778) \\
\hline
\end{tabular}
%
%}
\end{table}

\clearpage

\subsection{\label{chap:S7}QuPath integration method}
We adopt an integration strategy leveraging the paquo \cite{Bayer_AG} library, a Python package that enables direct interaction with QuPath’s internal API, thereby facilitating seamless data exchange without intermediate conversion steps. The data processing pipeline (\hyperref[fig:S7]{Appendix Figure S7}) begins with the acquisition of WSIs and their associated annotations from QuPath, which are represented as Shapely \cite{Gillies_Wel_etal._2024} polygons. Utilizing paquo, we directly read, create, and modify these annotations and detections within a QuPath project in the Python environment. Images are then cropped using these polygons and processed by cell segmentation and classification models employing standard vision processing toolkits such as OpenCV, pyvips, and PyTorch. Additionally, QuPath employs Groovy scripts to initiate a Python process that starts the entire pipeline from QuPath graphical interface: fetching polygons, extracting images from them, and running deep learning model inference on the cropped images. 
The results are returned to QuPath, leveraging paquo's Python bindings to manipulate QuPath data while minimizing the computational overhead typically associated with cross-environment communication.

\counterwithin{figure}{subsection}
\renewcommand{\thefigure}{S\arabic{subsection}}

\begin{figure}[h!]
    \centering
    \includegraphics[width=\textwidth]{images/A7.pdf}
    \caption{QuPath integration workflow using Python environment}
    \label{fig:S7}
\end{figure}

Compared to traditional workflows that involve exporting annotations as GeoJSON, classifying them in Python, and reimporting them into QuPath, our approach offers several advantages. We eliminate the need to switch between programming languages, providing a cohesive and streamlined development process entirely within QuPath software and removing the necessity to use other tools. Meanwhile, we avoid storing annotations as intermediate JSON files unless required for external use or archiving. By conducting the entire inference and post-processing workflow within the Python environment, we leverage the power and flexibility of Python libraries for image processing and machine learning. This approach also enables adjustments to any set of labels and models, thereby improving its applicability.

%\hfill

The distilled model and QuPath integration code are packaged into a Docker container, enabling streamlined execution with the Docker engine. Detailed integration code and deployment instructions can be found in the GitHub repository \cite{Shvetsov_2025b}.

Despite these benefits, we acknowledge that the paquo library is a proof‑of‑concept project in its early development stage and has not been tested across all versions of QuPath.

\clearpage

\subsection{\label{chap:S8}Data and code availability statement}
All datasets, models, and code used in this study are publicly available and can be obtained from the repositories listed below. 
The PanNuke \cite{Gamper_Koohbanani_etal._2019} and MoNuSAC \cite{Verma_Kumar_etal._2021} datasets are publicly accessible, and download information along with detailed descriptions can be found in their respective articles. Preprocessing scripts for PanNuke and MoNuSAC data, as well as individual cell extraction scripts, are available on GitHub \cite{Shvetsov_2025a}. The H-Optimus foundation model used in our experiments can be downloaded from the HuggingFace repository \cite{hoptimus2024}, and model information is available on GitHub \cite{Saillard_Jenatton_etal._2024}. In addition, the integration code for QuPath and the distilled model packaged in a Docker container are provided in the repository \cite{Shvetsov_2025b}, and paquo Python library is available from the authors GitHub repository \cite{Bayer_AG}.
\clearpage

\end{document}

\section{Details of Web-Based Document Processing Pipeline}
\label{sec:appendixA}
\subsection{Character-level Processing}
Character-level processing is the initial step in our preprocessing pipeline, aimed at standardizing and cleaning the text at the most granular level. This stage involves several key operations:

\begin{enumerate}
    \item Unicode Normalization: We convert all characters to their Persian equivalents, and remove Arabic I'rab marks. We then normalize space and tab characters to the standard keyboard space, with exceptions made for the half-space character used in specific Persian words.
    
    \item Symbol and Number Mapping: We map symbols and numbers not belonging to the English, Arabic, or Persian character sets to their Persian equivalents using the \href{https://github.com/arushadev/piraye}{Piraye} library. This is to ensure language consistency in the dataset.
    
    \item Repeated Characters: We identify any character repeated more than three times in sequence, typically used for emphasis, and truncate it to three occurrences to maintain readability and consistency.
    
    \item Newline Normalization: We merge consecutive newlines, including those with spaces or tabs, to standardize line breaks across documents.
    
    \item Non-standard Unicode Removal: By taking multiple samples from the data we found that there are chracters within the text that are not standard. We then detect and remove these non-standard Unicode characters, such as special emojis or corrupted symbols (e.g., bordered question marks) based on our predefined criteria. 
\end{enumerate}

\subsection{Line and Paragraph-level Processing}
Once character-level normalization is complete, we focus on the structural elements of the text. This stage involves:
\begin{enumerate}
    \item HTML and JavaScript Tag Removal: We identify lines containing HTML or JavaScript tags and functions using regular expressions and replace them with newlines.

    \item Custom Structures Handling: We inspected that some domains include unique tag structures that do not follow the format of standard tags (JavaScript and HTML) which are not captured by regular expressions. We identify and remove these using structures. 

    \item Special Character Ratio Filtering: We calculate the ratio of special characters (e.g., emojis, symbols, numbers) to total characters in each line. Lines exceeding a 0.85 ratio are removed, particularly targeting lines corrupted during text extraction, such as tables or formulas.

    \item Short Line Removal: We inspected that certain sources contain incomplete or irrelevant information in the few short lines at the start of the content. We therefore remove lines shorter these specific sources.
\end{enumerate}

\subsection{Document-level Processing}
The final stage involves document-level processing. We treat documents as a whole and remove those that meet any of the following criteria: (we refer to words as space-separated text sequences that are neither a number nor a symbol)
\begin{itemize}
    \item Short Length Filtering: Documents shorter than 30 words are removed, as they are either corrupted or devoid of useful information.

    \item Non-Persian Content Removal: Documents where over 50\% of characters are non-Persian are eliminated to maintain linguistic consistency and relevance.

    \item Repeated Words Elimination: Documents where more than 50\% of the words are identical are eliminated, targeting pages that use SEO techniques or lack informative content.

    \item Short Lines Proportion Filtering: Documents with over 50\% of lines shorter than 15 words are discarded, as they typically consist of lists or content tables.

    \item Out-of-Vocabulary (OOV) Words Filtering: Specifically for the CulturaX \citep{nguyen2023culturax} dataset, documents containing more than 2.5\% OOV words are removed to exclude irrelevant content such as code fragments or corrupted text.
\end{itemize} 

Finally, we eliminate any repeated empty newlines resulted from the removal of lines or paragraphs to maintain the document's structural integrity.

\subsection{Deduplication Process}
To address data redundancy, we leverage the MinHash algorithm \citep{broder1997minhash}, a well-established technique for efficient similarity detection in large collections of text. The deduplication pipeline consists of the following steps:\citep{broder1997minhash}. The process involves several steps:
\begin{enumerate}
    \item Text Normalization: We normalize Text within all documents by unifying recurring elements like days of the week and removing numbers and symbols. This normalization step is particularly crucial for content from websites that repost similar material daily. By handling these elements, we aim to reduce semantic duplicates.

    \item Tokenization and Hashing: We tokenize each document into 13-grams, and hash values are computed using 128 distinct hashing functions to capture text patterns.

    \item LeanMeanHash Compression: We then segment the hash values into eight sliding windows and processed using the LeanMeanHash algorithm, which compresses the hash signatures for efficient storage and comparison.

    \item Graph-based Similarity Detection: Finally, we construct a graph in which each node represents a document, and edges connect nodes based on hash similarity. By identifying connected components within this graph, only one representative document per component is retained, effectively removing duplicates and near-duplicates.
\end{enumerate}

This deduplication strategy ensures a significant reduction in redundant data, enhancing the corpus's quality and uniqueness, and facilitating better model generalization by preventing overfitting on repeated content.

\subsection{Domain-specific Processing}
Since \href{https://virgool.io/}{Virgool} and \href{https://en.wikishia.net/}{WikiShia} domains contain highly relevant content related to Persian culture and religion, it is  necessary to modify our standard preprocessing pipeline to avoid information loss. We perform the following specialized preprocessings.

For Virgool, which primarily features blog posts on diverse topics, including programming languages and mathematical content, applying the standard preprocessing thresholds resulted in the removal of valuable content. To address this, we relaxed certain filtering criteria: 
\begin{itemize}
    \item By pass the removal of numbers and symbols to preserve technical content.
    \item Incorporate more complex regular expressions to accurately detect and remove residual HTML tags or functions that were not filtered out by the standard pipeline.
    \item Adjust the ratio of Persian stopwords to lower values, and the threshold for the proportion of short lines (in relation to the total number of lines) was increased, ensuring the retention of concise but informative posts.
    \item Employed a privacy-preserving step to remove any personal data found in public blogs, even though the blogs are publicly accessible. This aspect of our pipeline will be discussed in detail in the subsequent section.
\end{itemize}

Another unique challenge with WikiShia was the significant presence of Arabic text, particularly due to references to the Quran and Arabic scholarly sources. To address this, we adjusted our processing thresholds: we increased the tolerance for Arabic stopwords while simultaneously lowering the threshold for Persian stopwords. This adjustment allowed us to better capture the bilingual nature of the content.


For WikiShia which includes bilingual content and presents challenges related to content duplication, we performe the following:

\begin{itemize}
    \item Content Duplication: our recursive crawling process exposed a significant issue of content duplication. Multiple URLs often corresponded to the same page, differing only by a minor subheading. Additionally, the site includes detailed descriptions of events associated with specific dates, resulting in multiple unique URLs hosting nearly identical content tied to calendar events.
    To address this, we employed an exact-match deduplication strategy using MinHashLSH \citep{leskovec2020lsh}. Unlike our standard deduplication pipeline, we opted not to normalize or remove dates, numbers, or references to specific days of the week, as these elements are critical for preserving the chronological and cultural relevance of the content. By applying this approach, we were able to eliminate documents with a similarity threshold of 98\% or higher.

    \item Bilingual Content Handling: Another unique challenge with WikiShia was the significant presence of Arabic text, particularly due to references to the Quran and Arabic scholarly sources. To address this, we adjusted our processing thresholds. The tolerance for Arabic stopwords was increased, while the threshold for Persian stopwords was lowered, effectively capturing the bilingual nature of the content.
\end{itemize}

The boxplot in Figure~\ref{fig:baxplot3} illustrates the token count distribution across three document sources we had for web-based data. The results are in tokens by the Llama3.1 \citep{dubey2024llama3.1} and after the application of our comprehensive preprocessing pipeline and deduplication. Data crawled by the team, named Web-crawled, show the widest range, with a median around 1000 tokens and some documents extending beyond \(10^5\) tokens. Madlad exhibits a slightly narrower distribution but still maintains substantial variation. CulturaX demonstrates the most compact distribution, with a lower median and maximum token count. These distributions highlight the success of our preprocessing in maintaining diversity while standardizing document lengths. The presence of outliers, particularly in the Web-crawled and Madlad sources, indicates that our pipeline preserves some longer, potentially information-rich documents. This final data composition ensures a balance between consistency and variety, crucial for robust model training and generalization.
\begin{figure}[t]
  \includegraphics[width=\columnwidth]{./boxplot_web.png}
  \caption{Document Length Distribution For Web-based Crawled Data}
  \label{fig:baxplot3}
\end{figure}

\section{Details of Book and Paper Processing Pipeline}
For data extraction and OCR conversion, we utilized a range of Python libraries, including Selenium\footnote{\url{https://selenium-python.readthedocs.io/}}, BeautifulSoup\footnote{\url{https://beautiful-soup-4.readthedocs.io/en/latest/}}, and Pytesseract\footnote{\url{https://github.com/madmaze/pytesseract}}. Text-based PDFs were converted using lightweight tools such as pdf2image\footnote{\url{https://github.com/Belval/pdf2image}}, while image-based PDFs required more advanced processing with Pytesseract and Fitz\footnote{\url{https://github.com/pymupdf/PyMuPDF}}. To improve accuracy, we employed an iterative approach, applying multiple tools to the same documents and manually inspecting those with errors before refining the extraction process.

\label{sec:appendixB}
\subsection{Text-based PDFs: Detailed Processing}
After removing corrupted or non-Persian documents, we apply a 3-stage processing pipeline involving document-level, character-level and line-level processing. Unlike documents from web, we first apply the document-level processing to avoid redundant processing. 

\subsubsection{Document-level Processing}
In the first stage, we applied document-level processing, where a document was viewed holistically. If it met any of the following criteria, it was eliminated: 
\begin{itemize} 
    \item Documents with fewer than 150 space-separated words. 
    \item Documents containing less than 50\% Persian characters. 
    \item Documents with an average word length of fewer than 3 characters or greater than 10 characters. 
    \item Documents with a numeric or symbolic character ratio exceeding 0.8. 
    \item Documents where over 80\% of the lines were considered short, defined as containing fewer than four space-separated words. 
    \item Documents where fewer than 10\% of the words were Persian or Arabic stopwords. 
\end{itemize}

\subsubsection{Character-level Processing}
Given that many of the books contained long Arabic text, which needed to be preserved, we only normalized non-Arabic, non-English, and non-Persian characters and symbols to their Persian format. We did not remove I'rab (diacritics). Standard procedures, such as replacing consecutive repeated characters, normalizing newlines, and removing non-standard Unicode characters, were applied as in previous section, though with additional Unicode characters added to the filtering set. Furthermore, nonsensical patterns detected in the text, which added no value and increased noise, were removed. These patterns included: 
\begin{itemize} 
    \item Website links to the source of the document. 
    \item Repeated occurrences of the book's title at the top or bottom of pages. 
    \item Page numbers in various forms, such as \FR{صفحه۱}, \FR{صفحه ۱}, \FR{صفحه۱ از ۲۰۰}, \FR{ص۱}, \FR{صفحه ۱ از کتاب ...}, etc. 
    \item Tags related to cover pages. \item Errors or tags related to multimedia, such as \texttt{'Your browser does not support the audio tag.'} 
    \item Images or tables converted to 'UNKNOWN' strings. 
    \item Personal information, such as phone numbers, email addresses, account numbers, and credit card numbers (e.g., Shaba numbers), which were found at the end of some books and at the beginning of papers. 
\end{itemize}

\subsubsection{Line-level Processing}
Following character-level processing, we performed line-level processing to remove lines that contained formulas or tables that were corrupted during the conversion from PDF to text. As part of this stage, the following types of lines or paragraphs were removed: 
\begin{itemize} 
    \item Lines with a numeric character ratio exceeding 0.8. 
    \item Lines with a symbolic character ratio exceeding 0.8. 
    \item Lines that were repeated multiple times within the document, which often included hidden watermarks or the repeated mention of the book's title. 
\end{itemize}

\subsubsection{Deduplication}
To avoid redundancy, a deduplication process was applied using MinHash and Locality-Sensitive Hashing (LSH). We deduplicated documents within each source, ensuring that only unique documents were retained.

\subsection{Image-based PDFs (OCR): Detailed Processing}
For OCR-processed documents, the primary issue was the introduction of errors during text extraction. To mitigate this, we employed the following steps:
\begin{enumerate}
    \item Removed content preceding the keywords section, which was often corrupted, using regex patterns to detect specific document structures.
    \item Removed documents with more than 5\% out-of-vocabulary tokens.
    \item REmoved papers containing more than 10 words exceeding 15 characters, indicative of merged words.
\end{enumerate}
Although some OCR-generated text still contains minor issues, such as occasional word merging, these are manageable with model tokenizers and do not significantly affect overall context and understanding.

The boxplot in Figure~\ref{fig:baxplot2} shows the token count distribution across different document sources. Books have a notably higher median token count and broader range compared to papers. Both image-based and text-based papers display lower token counts with numerous outliers, indicating diverse token lengths. Text-based papers have a lower median as they contain paper summaries as well as internal papers. Image-based papers also contain high-quality and longer scientific documents. 

\begin{figure}[th]
  \includegraphics[width=\columnwidth]{./result_bp.pdf}
  \caption{Document Length Distribution For Crawled Books and Papers}
  \label{fig:baxplot2}
\end{figure}

\end{document}


