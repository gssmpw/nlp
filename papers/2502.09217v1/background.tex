%This section collects a few definitions used in the paper.
%, and introduces the PT net formalism.
%The reader may
%We refer to~\cite{ReisigPN}
%and~\cite{ARCDFH93}
%for a complete description of PT nets.
%and SN, respectively.

%\subsection{Multisets}
%\label{sec:bag}
This section provides a concise overview of the (stochastic) PT formalism and emphasizes the key aspects of the 	\texttt{Maude} framework. For exhaustive information, we direct readers to the reference papers.

A \textit{multiset} (or \textit{bag}) $b$ on a set $D$ is a map $b: D \rightarrow \Nat$, where $b(d)$ is the \emph{multiplicity} of $d$ in $b$. We denote by $Bag[D]$ the set of multisets on $D$.
%$d \in b$ if and only if $b(d) > 0$.
Standard relational and arithmetic operations can be applied to multisets on a component-by-component basis. 
%We represent a multiset as a weighted ``sum'' of its elements.
%The \textit{empty} multiset is denoted $nil_D$ ($nil$, if $D$ is implicit).
%Let 
%
%, and $b_1,b_2 \in Bag[D]$.
%$b_1+b_2$ and $b_1-b_2$ are $Bag[D]$ elements such that
%$\forall d \in D$:
%$b_1+b_2 (d) = b_1(d) + b_2(d)$;  $b_1-b_2 (d) = b_1(d)-b_2(d)$ if $b_1(d) \geq b_2(d)$, $0$ otherwise. Also, relational operators are component-wise:
%$b_1 < b_2$ if and only if $\forall d \in D \  b_1(d) < b_2(d)$.
%With $b_1 \ rop' \ b_2$ we mean the \emph{restriction} of $rop$ to $\{d | d \in b_1 \}$.
%\subsection{(Stochastic) Place/Transition (PT) Nets with Inhibitor Arcs}
%\label{sec:ptnets}
%A \emph{net} N is a directed bipartite graph $(P, T, F)$, where $P$ and $T$ are non-empty, finite, disjoint sets (the places and transitions of N, drawn as circles and bars, respectively) representing system state variables and events causing state changes; $F \subseteq (P \times T) \cup ( T \times P)$ is the flow relation (an arc $a \in F$ is drawn as an arrow).
%PT nets are nets with a richer flow relation.
A stochastic PT (or SPN) \emph{net}~\cite{ReisigPN,GSPN1993}
%enriched with inhibitor arcs
is a 6-tuple $(P,T,I,O,H,\lambda)$, where:
$P$, $T$ are finite, non-empty, disjoint sets holding the net's nodes (places and transitions, respectively);
$I,O,H:$ $T \ \rightarrow \bag{P}$ represent the transitions' \emph{input}, \emph{output}, and \emph{inhibitor} incidence matrices, respectively;
$\lambda : T \ \rightarrow \Rel^+$ assigns each transition a negative exponential firing rate. 
%such that $\forall t \in T: I(t) \neq O(t)$.
%$P \times T \rightarrow \mathbb{N}$, i.e., $\{I,O,H\} \subset Bag[P \times T]$
%\item every element of $P \cup T$ occurs in $I$, $O$, or $H$.
%$P$ and $T$ hold the net \emph{places} and \emph{transitions}. The former, drawn as circles, represent system state variables, whereas the latter, drawn as bars, represent events causing local state changes.
%\footnote{the condition set on transitions ensures this}.
A PT net \emph{marking} (or state) is a multiset $m \in Bag[P]$.
%The marking of $p$ is $m(p)$.
%A PT \emph{system} is a pair $(N,m)$
%A net is a directed, bipartite multi-graph with $P \cup T$ nodes. Maps 
%(\raisebox{-0.4em}{\includegraphics[height=1.3em]{figures/arctypes.pdf}}).
%Let $f \in \{I, O, H\}$: if $k = f(t)(p) > 0$, then a weight-$k$ edge of corresponding type links $p$ to $t$.
%We may use the notation $f(t,p)$ instead of $f(t)(p)$.
%We may assume that there are no \emph{isolated nodes}, i.e., nodes that no I/O edge is incident to. This is acceptable from the modelling point of view and plays an important role in a dynamic context,
%\footnote{it might be further constrained to $I, O$ edges} in an ordinary context.
%The set of input, output and inhibitor places of a transition $t$ are denoted by ${}^{\bullet}t,~t^{\bullet},$ and $~{}^{\circ}t$, respectively.
%We assume that there are no isolated nodes in a PT net.

The PT net dynamics
%in a given marking
is defined by the \emph{firing rule}:
$t\in T$ is \emph{enabled} in $m$ if  
$I(t) \leq m$ and $\forall p \in P:  H(t)(p) = 0 \vee H(t)(p) > m(p)$. 
%where $'>'$ is restricted to the elements of $H(t)$.   
If $t$ is enabled in $m$ it may fire, leading to marking $m^\prime = m + O(t) - I(t)$.
We denote this: $m [ t \rangle m'$.
%means that $t$ is enabled in $m$ and leads to $m'$.
A PT-\emph{system} is a pair $(N,m_0)$, where $N$ is a PT net and $m_0$ is a marking of $N$.
% and $\forall p \in m \ \ exist t \in T : p \in I(t) \vee p \in O(t) \vee p \in H(t)$, is called PT \emph{system}\footnote{The conditions set on places/transitions ensure that there are no isolated nodes.}.  
The interleaving semantics of $(N,m_0)$ is specified by the \emph{reachability graph} (RG), an edge-labelled, directed graph $(V, E)$  whose nodes are markings. It is defined inductively: $m_0 \in V$; if $m \in V$ and
$m [ t\rangle m'$
%$m' \neq m$,
then $m' \in V$, $m \xrightarrow{t} m' \in E$.
%We call $V$ reachability set.
%The set of reachable marking is called \emph{reachability set}.
%A desirable property of a system $(N,m_0)$ may be the absence of isolated places, i.e., places present in $m_0$ but not in $N$. In a highly dynamic context, however, we do not consider this a strict requirement.
The timed semantics of a stochastic PT system is a CTMC isomorphic to the RG:
For any two $m_i, m_j \in V$, the transition rate from $m_i$ to $m_j$ is $r_{i,j} := \sum_{t : m_i [ t \rangle m_j} \lambda(t)$.
%The CTMC infinitesimal generator is a $|V| \times |V|$ matrix $Q$ such that $Q[i,j] = r_{i,j}$ if $i \neq j$,
%$Q[i,i] = 1 - \sum_{j, j \neq i} r_{i,j}$.


%In Generalized Stochastic Petri Nets (GSPN) \cite{GSPN1993} transitions can be assigned a priority (the firing rule is extended accordingly): transitions with a priority greater than zero occur instantly, and the associated stochastic parameters (denoted by the function $\lambda$) are used to resolve potential conflicts probabilistically. Consequently, their timed semantics leads to a Continuous-Time Markov Chain (CTMC) that is isomorphic to the "reduced" RG, obtained by eliminating nonobservable markings. This paper focuses on Stochastic Petri Nets (SPN) even though our specification encompasses GSPN.

\vspace{-11pt}
\paragraph{Maude}
\texttt{Maude} \cite{maude07} is an expressive, purely declarative language characterized by a rewriting logic semantics \cite{rewlog03}. Statements consist of \emph{equations} and \emph{rules}. Each side of a rule or equation represents terms of a specific \emph{kind} that might include variables. Rules and equations have intuitive rewriting semantics, where instances of the left-hand side are substituted by corresponding instances of the right-hand side. The expressivity of \texttt{Maude} is realized through the use of matching modulo operator equational attributes, sub-typing, partiality, generic types, and reflection. A \emph{functional} module comprises only \emph{equations} and works as a functional program defining operations through equations, utilized as simplification rules. It details an \emph{equational theory} within membership equational logic \cite{membeqlog00}: Formally, a tuple $(\Sigma,E \cup A)$, with $\Sigma$ representing the signature, which includes the declaration of all sorts, subsorts, kinds\footnote{implicit equivalence classes defined by connected components of sorts (as per subsort partial order). Terms in a kind without a specific sort are \emph{error} terms.}, and operators; $E$ being the set of equations and membership axioms; and $A$ as the set of operator equational attributes (e.g., \texttt{assoc}). The model of $(\Sigma,E \cup A)$ is the \emph{initial algebra} $T_{\Sigma /E \cup A}$, which mathematically corresponds to the quotient of the ground-term algebra $T_{\Sigma}$. Provided that $E$ and $A$ satisfy nonrestrictive conditions, the final (or \emph{canonical}) values of ground terms form an algebra isomorphic to the initial algebra, that is, the mathematical and operational semantics coincide.

A \texttt{Maude} \emph{system module} includes \emph{rewrite rules} and, possibly, equations. Rules illustrate local transitions in a concurrent system. Specifically, a system module describes a generalized \emph{rewrite theory} \cite{rewlog03} $\mathcal{R}= (\Sigma,E \cup A,\phi,R)$, where $(\Sigma,E \cup A)$ constitutes a membership equation theory; $\phi$ identifies the frozen arguments for each operator in $\Sigma$; and $R$ contains a set of rewrite rules. 
%\footnote{Rewrite rules do not apply to frozen arguments.}.
A rewrite theory models a concurrent system: $(\Sigma,E \cup A)$ establishes the algebraic structure of the states, while $R$ and $\phi$ define the concurrent transitions of the system. The initial model of $\mathcal{R}$ assigns to each kind $k$ a labeled transition system (TS) where the states are the elements of $T_{\Sigma /E \cup A,k}$, and transitions occur as $ [t] \overset{[\alpha] }{\rightarrow} [t']$, with $[\alpha]$ representing \emph{equivalent} rewrites. The property of \emph{coherence} guarantees that a strategy that reduces terms to their canonical forms before applying the rules is sound and complete. A \texttt{Maude} system module is an executable specification of distributed systems. Given finite reachability, it enables the verification of invariant properties and the discovery of counterexamples.
%Moreover, it supports the verification of LTL formulas. 
%When the TS generated by a ground term becomes excessively large or infinite, bounded searches or abstractions might be employed.