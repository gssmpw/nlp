A rewritable PT \texttt{system}  generates a transition system (TS) isomorphic to a lumped CTMC. 
However, the TS produced via the \texttt{show search graph} command of \texttt{Maude} embodies a \emph{parametric} CTMC: in line with the rewriting logic semantics, state transitions denote \emph{classes} of equivalent rewrites, meaning, PT firings that result in identical normalized markings or net rewrites that lead to isomorphic PT systems.

%Consider, e.g., the marking $m_2$ of the previous example: %which matches a normalized form. 
%We can easily verify that this marking enables two equivalent transitions "line" ($ln_i$) and two equivalent transitions "fault" ($ft_i$), respectively, that lead to equivalent (automorphic) markings.
To obtain the CTMC generator, it is necessary to quantify instances that align with a specific state transition. Regrettably, the \texttt{Maude} system lacks a mechanism to determine the matches of a rewrite rule in the TS construction process.

%To achieve our goal,
%We have three possible strategies.
%\begin{enumerate}
%    \item Use a partial "edge unfolding" operator to calculate all possible rewrites from a normalized source state to a normalized target state of the TS. It requires post-processing the TS generated by \texttt{Maude}.
%    \item Use \texttt{Maude}'s meta-level to establish meta-rules that can identify potential matches of base rules.
%    \item 
  Our solution consists of first (automatically) generating a Transition System with states having a composite structure, which provides a detailed view of the equivalent rewrites that result in state transitions. Subsequently, using elementary parse to compute the cumulative rates of the lumped CTMC.
%\end{enumerate}
 %The second strategy, although elegant, has been shown far from efficient. Because efficiency is our primary concern, we have opted for a middle-ground solution: the third strategy.
 Despite a redundant state representation, this method incurs an acceptable time overhead because it only involves \emph{normalized} states. 
 %It relies on two operators. The first operator provides potential matches for each rule considering the subset of independent variables involved, and then the second emulates the rewriting process. It is possible to automatically define these two operators from the syntax of a rule. 
 %To better illustrate the idea, refer to the listing \ref{lst:l5} in Appendix \ref{apx:ap1}.

%The state structure defined by the mixfix constructor \verb|StateTranSys| comprises four fields. The initial pair describes the PT system, while the remaining two fields detail the state transitions caused by the \verb|firing| rule and other rewrites, in that order. We collect state transitions that share the target for calculating aggregated rates. Let us take the \verb|firing| rule as an example: it has two associated operators, namely \verb|enabled|, which gives the set of possible transitions, and \verb|fire|, which provides the reached markings,
%(in their normal form),
%each one prefixed by a multiset of transition tags. We proceed analogously for other rules.

%The operator \verb|toStateTran| converts the conventional state representation into a structured one. The firing rule is adapted easily.
When considering the term \verb|NPLsys(2,2,2)|, which aligns with the PT net at the top left of Figure \ref{fig:symPL-degarde2}, the resulting quotient TS comprises 295 states compared to the 779 states in the standard TS. State transitions often correspond to multiple matches: For instance, the initial state (the term above) includes two 'load' instances and four 'fault' instances that lead to markings with identical normal forms. Consequently, the combined rates are $2\cdot0.5$ and $4\cdot0.001$. 
%Equivalent rewrites of the net structure are observed when $N > 2$. 

\vspace{-10pt}
\paragraph{Experimental Evidence}
We showcase experimental validation of the method and a demonstration of standard performance indicators. Table \ref{tab:perf} displays the results of the \texttt{search} command to locate the final states.
We used Linux WSL on an 11th Gen. Intel Core i5 with 40GB RAM. The state spaces match those of the lumped CTMC. The analysis of large models is feasible solely by exploiting the model's symmetry.
Notice that the number of absorbing states in the TS quotient does not vary with $N$. 
%In \cite{CAPRA-TCS2024}, the effectiveness of modular RwPT was compared to that of Symmetric Nets (SN, previously known as well-formed nets) \cite{CHIOLA97}, a colored Petri net that generates a symbolic reachability graph equivalent %(in its stochastic extension)
%to a lumped CTMC. A parametric SN model was developed, whose unfolding coincides with \verb| NPLsys (N, K, M)|
%but without the degradation process.
%As the values of N and K increase, the degree of state aggregation in modular RwPT significantly exceeds that of SN. For instance, when N=10, K=3, and M=3, the degree of state aggregation is about 45 times that of SN, and when N=10, K=4, and M=3, it is about 200 times that of SN.

%\vspace{-3pt}
%As an example of performance metrics, 
Figure
%\ref{fig:PlotX} shows the system throughput, while 
\ref{fig:PlotRel} shows the system \emph{reliability} (the complement of Time to System Failure distribution).
%as a time function. 
As expected, %both 
is decreases with time; additionally, the scenario that involves more replicas demonstrates
%increased throughput and
enhanced reliability.
The inflexion point at around time 800
%in both curves
represents the system's reconfiguration time. The increased execution time of the job (not reported) is a result of a system failure.
The overall trend is also noticeable when we look at larger values of \verb|N|. As \verb|N| increases, both reliability and throughput curves show significant improvements.
However, we observe an asymptotic trend when \verb|N| is greater than 6. Our interpretation is that beyond a certain point, the benefit of using a higher number of replicas is outweighed by the higher fault rate and the increased configuration overhead.

\vspace{-0.5cm}
\begin{table}[htbp]
\centering\small
\caption{Ordinary vs Quotient TS of \texttt{NPLsys(N,2,2)} \hspace{2pt} ${}^\dag$ \texttt{search} timed out after 10 h} 
\label{tab:perf}
\begin{tabular}{ |p{0.5cm}||p{2.5cm}p{1.5cm}||p{2.5cm}p{1.5cm}| }
\hline
 & \multicolumn{2}{c||}{Ordinary} & \multicolumn{2}{|c|}{Quotient} \\
\verb|N|   & states(final)  & time (sec) &  states(final) & time (sec) \\
\hline
\hline
1   &    60(2) & 0 &      42(2) & 0\\
\hline
2   &    779(4) & 0.1 &     295(2) & 0.1\\
\hline
3  &   6101(6) & 4.8  &  1059(2) & 0.9\\
\hline
4 &    37934(8) & 69 & 2764(2) & 3.6\\
\hline
5 &   204362(10) & 818 & 5970(2) & 10 \\
\hline
6 &   1000187(12) & 13930 & 11367(2) & 27\\
\hline
7 &  - & ${}^\dag$ & 19775(2) & 65 \\
\hline
8 &  - & ${}^\dag$ &  32144(2) & 186 \\
\hline
9 & - & ${}^\dag$ &  49554(2) & 569 \\
\hline
10 & - & ${}^\dag$ & 73215(2) & 2450 \\ 
\hline
\end{tabular}
\end{table}
%\begin{figure}[htbp]
%   \begin{center}    
%    \includegraphics[clip,trim={2.1cm 2.7cm 2.2cm 2.2cm},width=0.85\columnwidth]{figures/PlotThroughput.pdf}
%    \end{center}
%    \caption{System Throughput.}
%    \label{fig:PlotX}
%\end{figure}

\begin{figure}[htbp]
   \begin{center}    
    \includegraphics[clip,trim={2.1cm 2.7cm 2.2cm 2.2cm},width=0.7\columnwidth]{figures/PlotReliability.pdf}
    \end{center}
    \caption{System Reliability.}
    \label{fig:PlotRel}
\end{figure}

%To evaluate the system's performance, Figure \ref{fig:PlotXdivRel} shows the throughput while the system is operational, which is the ratio between the graphs in Figures \ref{fig:PlotX} and \ref{fig:PlotRel}. It can be seen that the throughput approaches that of a single line, which, given the parameters, is $1/202.5 = 4.98E-03$.

%\begin{figure}[htbp]
%   \begin{center}    
%    \includegraphics[clip,trim={2.1cm 2.7cm 2.2cm 2.2cm},width=0.85\columnwidth]{figures/PlotConditionedThroughput.pdf}
%    \end{center}
%    \caption{System Throughput conditioned to its reliability.}
%    \label{fig:PlotXdivRel}
%\end{figure}
