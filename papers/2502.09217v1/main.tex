\documentclass[submission,copyright,creativecommons]{eptcs}
\providecommand{\event}{ICLP 2024} % Name of the event you are submitting to

\usepackage{iftex}

\ifpdf
  \usepackage{underscore}         % Only needed if you use pdflatex.
  \usepackage[T1]{fontenc}        % Recommended with pdflatex
\else
  \usepackage{breakurl}           % Not needed if you use pdflatex only.
\fi

\usepackage[utf8]{inputenc}
\usepackage[english]{babel}
\usepackage{url}
\usepackage{amssymb}
\usepackage{amsmath}
\usepackage{graphicx}
\usepackage{amsthm}

%%%%%%%%%%%%%%%%%%%%%%%%%%%%%%%%%%%%%%%%%%%%%%%%%%%%%%%
%%%%%%%%%%%%%%%    theorems %%%%%%%%%%%%%%%%%%%%%%%%%%%
%%%%%%%%%%%%%%%%%%%%%%%%%%%%%%%%%%%%%%%%%%%%%%%%%%%%%%%
% \usepackage{mdframed}
\usepackage{kantlipsum}

%%%%%%%%%%%%%%%%%%%%%%%%%%%%%%%%%%%%%%%%%%%%%%%%%%%%%%%
%%%%%%%%%%%%%%%    theorems %%%%%%%%%%%%%%%%%%%%%%%%%%%
%%%%%%%%%%%%%%%%%%%%%%%%%%%%%%%%%%%%%%%%%%%%%%%%%%%%%%%
\theoremstyle{plain}
\newtheorem{theorem}{Theorem}[section]
\newtheorem{proposition}[theorem]{Proposition}
\newtheorem{lemma}[theorem]{Lemma}
\newtheorem{example}[theorem]{Example}
\newtheorem{corollary}[theorem]{Corollary}
\theoremstyle{definition}
\newtheorem{definition}[theorem]{Definition}
\newtheorem{assumption}[theorem]{Assumption}
\theoremstyle{remark}
\newtheorem{remark}[theorem]{Remark}


% \titleformat{\subsection}[runin]% runin puts it in the same paragraph
%        {\normalfont\bfseries}% formatting commands to apply to the whole heading
%        {\thesubsection}% the label and number
%        {0.5em}% space between label/number and subsection title
%        {}% formatting commands applied just to subsection title
%        [.]% punctuation or other commands following subsection title


%%%%%%%%%%%%%%%%%%%%%%%%%%%%%%%%%%%%%%%%%%%%%%%%%%%%%%%
%%%%%%%%%%%%%%%  mathematical notations%%%%%%%%%%%%%%%%
% \usepackage[english]{babel}
% \usepackage[utf8]{inputenc}
% \usepackage[T1]{fontenc}

%% Figures, tables and lists
\usepackage[dvipsnames]{xcolor}
\usepackage{paralist}
\usepackage{graphicx}
\usepackage{subcaption}
\usepackage{longtable} 
\usepackage{multirow}
\usepackage{listings}
\usepackage{makecell}
\usepackage{array}
\usepackage{float}
\usepackage{dsfont}
\usepackage{rotating}
\usepackage{booktabs}
\usepackage{enumerate}
\usepackage{tikz}
\usepackage{pgf}
\usepackage{enumitem}
\usepackage{lipsum} % for generating filler text
\usepackage{titlesec}

%% Math
% \usepackage{amssymb, amsthm,bbm}
\usepackage{mathtools}
\usepackage{mathrsfs}
%% References and author info 
\mathtoolsset{showonlyrefs}
\usepackage{natbib}
\usepackage{authblk}
\usepackage{todonotes}
\usepackage{xr-hyper}


%%%%%%%%%%%%%%%%%%%%%%%%%%%%%%%%%%%%%%%%%%%%%%%%%%%%%%%
\newcommand{\R}{\mathbb R}
\newcommand{\EE}{\mathbb{E}}

\DeclareMathOperator{\Tr}{Tr}
\DeclareMathOperator*{\argmin}{argmin}
\DeclareMathOperator*{\argmax}{argmax}

\newcommand{\bs}[1]{\ensuremath{\boldsymbol{#1}}}
\newcommand{\mc}{\mathcal}
\newcommand{\opt}{^\star}


\newcommand{\diff}{\textnormal{d}}


\def \iid {\stackrel{\textnormal{i.i.d.}}{\sim}}
\def \iidtext {\textnormal{i.i.d.}}





%%%%%%%%%%%%%%%%%%%%%%%%%%%%%%%%%%%%%%%%%%%%%%%%%%%%%%%
%%%%%%%%%%%%%%%%%%%%% colors     %%%%%%%%%%%%%%%%%%%%%%
%%%%%%%%%%%%%%%%%%%%%%%%%%%%%%%%%%%%%%%%%%%%%%%%%%%%%%%
\definecolor{myblue}{rgb}{.8, .8, 1}
\definecolor{mathblue}{rgb}{0.2472, 0.24, 0.6} % mathematica's Color[1, 1--3]
\definecolor{mathred}{rgb}{0.6, 0.24, 0.442893}
\definecolor{mathyellow}{rgb}{0.6, 0.547014, 0.24}


% May add more in future.






% Highlighting for your Maude, Real-Time Maude, Maude files.
% 
% Source: http://www.github.com/garyyread
% Author: Gary Read
% Contact: garyyread@gmail.com
%
% Using this HI-Maude, Real-Time Maude, Maude listing:
% \lstinputlisting[language = maude]{SOURCE.maude}
%
% or
%
% \begin{lstlisting}[language=maude]
% ***
% *** Maude Source Code
% ***
% \end{lstlisting}

\usepackage{listings}
\usepackage{xcolor}

\definecolor{delimiterColor}{HTML}{B65E47}
\definecolor{numberColor}{HTML}{FF0000}
\definecolor{commentColor}{HTML}{008000}
\definecolor{keyColor}{HTML}{002BFF}

\lstdefinelanguage{maude}
{
	%numbers=left,
	breaklines=true,
	extendedchars=true,
	tabsize=2,
	%frame=shadowbox,
	columns=fullflexible,
	showtabs=false,
	showstringspaces=false,
	showspaces=false,
	showstringspaces=false,
	identifierstyle={\ttfamily},
	keywordstyle={\color{keyColor}},
	ndkeywordstyle={\color{keyColor}},
	stringstyle={\color{delimiterColor}},
	commentstyle={\color{commentColor}},
	ndkeywords={},
	keywords={pr, protecting, sort, sorts, op, ops, var, vars,eq, cq, ceq, crl, rl, mb, cmb, endfm, fmod, is, mod, endm, =, ==, =/=, ctor, ditto, Object, owise, Oid, prec, assoc, id, if, class, homod, endhom, eof, var, vars, eq, op, ops, pr, inc, protecting, including, ceq, is, tomod, endtom, sort, subsort, subsorts, to, endom, fmod, endfm, mod, endm, endtm, comm, gather, fth, endfth, format, metadata, memo},
	morecomment={[l]{***}},
	morecomment={[l]{---}},
}
%\usepackage[toc,page]{appendix}

\renewcommand\UrlFont{\color{blue}\rmfamily}

%\theoremstyle{definition}
\newtheorem{definition}{Definition}[section]

\newtheorem{lemma}{Lemma}[section]

\newtheorem{property}{Property}[section]


\begin{document}


%% The "title" command
\title{Modular Stochastic Rewritable Petri Nets}

\author{Lorenzo Capra
\institute{
	{Dipartimento di Informatica}\\ {Universit{\`a} degli Studi di Milano}, Italy
}
}

\def\titlerunning{Modular Stochastic Rewritable Petri Nets}
\def\authorrunning{Lorenzo Capra}

\maketitle

\begin{abstract}
Retrieval-Augmented Generation (RAG) is often used with Large Language Models (LLMs) to infuse domain knowledge or user-specific information. In RAG, given a user query, a retriever extracts chunks of relevant text from a knowledge base. These chunks are sent to an LLM as part of the input prompt. Typically, any given chunk is repeatedly retrieved across user questions. However, currently, for every question, attention-layers in LLMs fully compute the key values (KVs) repeatedly for the input chunks, as state-of-the-art methods cannot reuse KV-caches when chunks appear at arbitrary locations with arbitrary contexts. Naive reuse leads to output quality degradation.  This leads to potentially redundant computations on expensive GPUs and increases latency. In this work, we propose \sys, a system for managing and reusing precomputed KVs corresponding to the text chunks (we call \textit{chunk-caches}) in RAG-based systems. We present how to identify \hl{\textit{chunk-caches} that are reusable}, how to efficiently perform a small fraction of recomputation to \textit{fix} the cache to maintain output quality, and how to efficiently store and evict \textit{chunk-caches} in the hardware for maximizing reuse while masking any overheads. With real production workloads as well as synthetic datasets, we show that \sys reduces redundant computation by \textbf{51\%} over SOTA prefix-caching and \textbf{75\%} over full recomputation.
\hl{Additionally, with continuous batching on a real production workload, we get a \textbf{1.6$\times$} speedup in throughput and a \textbf{2$\times$} reduction in end-to-end response latency over prefix-caching while maintaining quality, for both the \llama-3-8B and \llama-3-70B models. 
}
\end{abstract}






\section{Introduction}
\label{sec:intro}
\section{Introduction}
\label{sec:intro}

\begin{figure*}[tb]
    \centering
    \includegraphics[width=0.848\linewidth]{figs/circuitnn.pdf} 
    \caption{Illustration of differentiable CircuitNN. CircuitNN is designed based on differentiable NAND gates. After DAS is guided by PI and PO pairs of the truth table, CircuitNN can get the precise circuit architecture logic equivalent to the truth table.}
    \label{fig:circuitnn}
\end{figure*}

% 1. Describe the importance of logic synthesis
% 2. Existing Problems
% (a) Neural Architecture Search: Unstable, Predefined Setting, etc.
% (b) Circuit Generation: Probabilistic Model, Logic Equivalence

With the rapid advancement of technology, the scale of integrated circuits (ICs) has expanded exponentially. 
This expansion has introduced significant challenges in chip manufacturing, particularly concerning power and area metrics.
A primary objective in IC design is achieving the same circuit function with fewer transistors, thereby reducing power usage and area occupancy.

Logic synthesis~\cite{hachtel2005logicsynth}, a critical step in electronic design automation (EDA), transforms behavioral-level circuit designs into optimized gate-level circuits, ultimately yielding the final IC layout. 
The primary goal of logic synthesis is to identify the physical implementation with the fewest gates for a given circuit function. 
This task constitutes a challenging NP-hard combinatorial optimization problem. 
Current logic synthesis tools~\cite{brayton2010abc, wolf2013yosys} rely on human-designed heuristics, often leading to sub-optimal outcomes.

Differentiable architecture search (DAS) techniques~\cite{liu2018darts, chu2020darts} offer novel perspectives on addressing challenges in this problem.
Circuit functions can be represented through truth tables, which map binary inputs to their corresponding outputs. 
Truth tables provide a precise representation of input-output relationships, ensuring the design of functionally equivalent circuits.
Inspired by this, researchers~\cite{deepmind2024ai4sys, wang2024tnet} have begun exploring the application of DAS to synthesize circuits directly from truth tables.
Specifically, \citet{deepmind2024ai4sys} proposed CircuitNN, a framework that learns differentiable connection structures with logic gates, enabling the automatic generation of logic circuits from truth tables.
This approach significantly reduces the complexity of traditional circuit generation. 
Building on this, \citet{wang2024tnet} introduced T-Net, a triangle-shaped variant of CircuitNN, incorporating regularization techniques to enhance the efficiency of DAS.

Despite these advancements, several challenges remain. 
The computational complexity of DAS grows quadratically with the number of gates, posing scalability issues.
Although triangle-shaped architecture~\cite{wang2024tnet} partially mitigates this problem, redundancy persists. 
%Additionally, DAS is susceptible to converging to local optima, limiting the ability to search architectures that satisfy the given truth tables~\cite{liu2018darts}. 
%Furthermore, hyperparameters (network depth and layer width) require extensive searches, introducing complexity and prolonging the synthesis process. 
Additionally, DAS is susceptible to converging to local optima~\cite{liu2018darts} and hyperparameters (network depth and layer width) require extensive searches. 
The challenges arise from the vast search space in DAS. 
% Even with predefined settings for CircuitNN, finding a configuration that meets the truth table requires extensive trial and error during the DAS process. 
Intuitively, limiting the search space through predefined parameters (network depth, gates per layer, and connection probabilities) can significantly reduce the complexity.

Recent advances~\cite{openai2023gpt4, abramson2024alphafold3, esser2024sd3, li2024mar} in conditional generative models have demonstrated remarkable performance across language, vision, and graph generation tasks. 
Motivated by these developments, we propose a novel approach to circuit generation that generates preliminary circuit structures to guide DAS in generating refined circuits matching specified truth tables. 
Firstly, we introduce CircuitVQ, a tokenizer with a discrete codebook for circuit tokenization. 
Built upon our Circuit AutoEncoder framework~\cite{hou2022graphmae,li2023maskgae,wu2025mgvga}, CircuitVQ is trained through a circuit reconstruction task. 
Specifically, the CircuitVQ encoder encodes input circuits into discrete tokens using a learnable codebook, while the decoder reconstructs the circuit adjacency matrix based on these tokens.
Subsequently, the CircuitVQ encoder serves as a circuit tokenizer for CircuitAR pretraining, which employs a masked autoregressive modeling paradigm~\cite{chang2022maskgit, li2023mage}. 
In this process, the discrete codes function as supervision signals. 
After training, CircuitAR can generate discrete tokens progressively, which can be decoded into initial circuit structures by the decoder of the CircuitVQ. 
These prior insights can guide DAS in producing refined circuits that match the target truth tables precisely.

Our key contributions can be summarized as follows:
\begin{itemize}
\item We introduce CircuitVQ, a circuit tokenizer that facilitates graph autoregressive modeling for circuit generation, based on our Circuit AutoEncoder framework;
\item Develop CircuitAR, a model trained using masked autoregressive modeling, which generates initial circuit structures conditioned on given truth tables;
\item Propose a refinement framework that integrates differentiable architecture search to produce functionally equivalent circuits guided by target truth tables;
\item Comprehensive experiments demonstrating the scalability and capability emergence of our CircuitAR and the superior performance of the proposed circuit generation approach.
\end{itemize}

% Motivation
% (a) Diffusion (Vision, Graph), Autoregressive (Language, Vision)
% (b) Circuit Generation for Predefined Setting
% (c) Neural Architecture Search for Strict Logic Equivalence

% Contribution
% (a) Circuit Tokenizer (new transformer arch, training strategy)
% (b) CircuitAR (train and gen strategies, post-ar strategy)
% (c) Extensive Evaluation including BitD (Bit Distance) for Scalability


\section{(Stochastic) PT nets, \texttt{Maude}, and demonstrative example}
\label{sec:backgr}
\section{Basic Background: Supervised Learning and the PAC Model}
\label{sec:background}

At this point almost everyone has heard of machine learning (ML). Anyone likely to stumble upon this article will have also heard of its most influential special case, supervised learning, and those theoretically inclined will also be familiar with the PAC model. Nonetheless, I will set the stage by  recapping the basics.

\subsection{Basics of Supervised Learning}%Let's set the stage in any case

\emph{Supervised Learning} is the task of ``coming up'' with a function $f: \X \to \Y$ to ``explain'' or ``fit'' a sequence of input/output examples   $(x_1,y_1), \ldots, (x_n,y_n)$, with $x_i \in \X$ and $y_i \in \Y$.  Here $\X$ is a \emph{data domain} consisting of \emph{datapoints} $x \in \X$, $\Y$ is a \emph{label set} consisting of \emph{labels} $y \in \Y$, and the sequence $(x_1,y_1),\ldots,(x_n,y_n)$ is the \emph{training data} consisting of \emph{labeled examples (a.k.a. samples)}~$(x_i,y_i)$.  I~will refer to the chosen function $f$ as a \emph{predictor}, and to $n$ as the \emph{sample size}. A \emph{learning algorithm} takes as input training data, and outputs (some representation of) a predictor $f \in \Y^\X$.\footnote{Note that this describes the usual \emph{batch}, a.k.a.~\emph{offline}, setting of supervised learning. I do not discuss other paradigms such as online or active learning in this article.} 



Success in supervised learning is defined as \emph{generalization} to  future examples: For a typical \emph{test example}  $(x_{\tst},y_{\tst})$, the predicted label $y'_{\tst}=f(x_{\tst})$ should ``equal'' $y_{\tst}$, perhaps approximately. We usually assume the test example is drawn from the same  ``source'' as the training data  --- commonly, i.i.d.~from the same distribution. The quality of the prediction is quantified by $\ell(y'_{\tst},y_{\tst})$, where $\ell:~\Y~\times~\Y \to \RR_{\geq 0}$ is a \emph{loss function} chosen as part of the problem definition. Common loss functions include the 0-1 loss $\ell_{0-1}(y',y) = [y' \neq y]$ for \emph{classification} problems,\footnote{The notation $[P]$ denotes $1$ when predicate $P$ is true, and denotes $0$ when $P$ is false.} as well as the absolute loss $|y'-y|$ or squared loss $(y'-y)^2$ for \emph{regression problems} featuring $\Y  \sse \RR$.

Nontrivial generalization properties are typically only possible if one assumes something about the data.\footnote{The need for such an assumption is formalized by the  \emph{no free lunch theorems} of supervised learning \cite{wolpert_connection_1992,wolpert_lack_1996,schaffer_conservation_1994}.} The Bayesian approach to  machine learning, common in many applications, assumes some parametric form for the distribution generating the data, and postulates a prior on the parameters. This is not the approach I will take in this article. Instead, I will focus on the frequentist --- and some would say ``worst-case'' or ``adversarial'' ---  approach that is common in the computational learning theory community, embodied by the PAC model. Here we assume that the (training and test) data can be explained, perhaps approximately, by a function in some ``simple enough to learn'' class of functions $\H \sse \Y^\X$, often called the \emph{hypotheses}. Equivalently, we  seek a predictor which explains the unseen data roughly  as well as the best hypothesis $h^* \in \H$, whether or not we assume that $h^*$ itself provides a perfect explanation.



 \paragraph{Common Algorithmic Templates.} Perhaps the best known general-purpose supervised learning algorithm is \emph{empirical risk minimization (ERM)}, which chooses as its predictor a hypothesis $f \in \H$ minimizing $\frac{1}{n} \sum_{i=1}^n \ell(f(x_i),y_i)$ --- a quantity called the \emph{training error}, \emph{empirical error}, or \emph{empirical risk} of $f$. %\footnote{When multiple hypotheses minimize the empirical risk, we assume ERM breaks ties arbitrarily.}
A common template for generalizing ERM involves adding a \emph{regularization term} $\psi(f)$ to the  objective function, typically chosen to measure some notion of ``hypothesis complexity.'' An algorithm instantiating this template is known as a \emph{structural risk minimizer (SRM)}, and chooses as its predictor the hypothesis $f \in \H$ minimizing the \emph{structural risk} $\frac{1}{n} \sum_{i=1}^n \ell(f(x_i),y_i) + \psi(f)$. Other well-known algorithms, such as gradient descent and its variations,  can frequently be interpreted as approximate implementations of ERM or SRM.


\paragraph{Proper vs Improper Learning.} A learning algorithm is said to be \emph{proper} if its predictor $f$ is always chosen from the hypothesis class, i.e., $f \in \H$, otherwise it is said to be \emph{improper}. ERM  is an example of a proper learning algorithm, as are SRM algorithms of the form described above.  In the \emph{proper regime} of learning, algorithms are required to be proper. This article will be concerned with the more flexible \emph{improper regime} (a.k.a \emph{representation-independent learning}), where no such constraint is placed on the learner. In other words, all we care about is predictive power at test time, rather than any insights derived from the functional form or representation of the predictor~itself.


\subsection{The PAC Model}
A standard mathematical setup for evaluation of supervised learning algorithms, at least in the theoretical computer science community, is Valiant's \emph{Probably Approximately Correct (PAC) model} of learning (see e.g.~\cite{kearns_introduction_1994,mohri_foundations_2018}). Here, we assume there is an unknown distribution $\D$ on $\X \times \Y$ from which training and test data are  drawn.  Specifically, the labeled datapoints of the training set  $(x_1,y_1), \ldots, (x_n,y_n)$, as well as the test data  $(x_\tst,y_\tst)$, are i.i.d.~from $\D$. Often it is assumed that $\D$ lies in some class of distributions of interest. The \emph{true expected loss}, or simply \emph{loss}, of a predictor $f: \X \to \Y$ is the expected loss it incurs on draws from $\D$, written $L_\D(f) = \Ex_{(x,y) \sim \D} \ell(f(x),y)$.


There are two main ``settings'' in PAC learning. The  \emph{realizable setting} only requires that the data be perfectly explained by some hypothesis in $\H$. More generally, the \emph{agnostic setting} makes no assumption relating the data to the hypotheses, but shifts the goalposts as necessary to allow nontrivial guarantees: the expected loss at test time is evaluated only ``relative'' to that of the best hypothesis $h^* \in \H$. There are other settings which make more nuanced assumptions, such as $\D$ being of a particular parametric form or its support living in some (unknown) lower-dimensional space, etc. I will mostly discuss the realizable and agnostic settings in this article, those being the simplest and most studied from a theoretical perspective. %TODO:We will briefly discuss other settings in Section ??

The PAC model demands high probability guarantees of learners, in the worst case over distributions of interest. Consider first the realizable setting, where $\D$ is such that $\min_{h \in \H} L_{\D}(h) = 0$. A PAC learner has \emph{error} $\epsilon=\epsilon(n)$ and \emph{confidence} $\delta=\delta(n)$ if, when training data consists of $n$ i.i.d~samples from a realizable distribution $\D$, it produces a predictor $f$  satisfying $L_\D(f) \leq \epsilon$ with probability at least $1-\delta$. In the agnostic setting, where $\D$ can be arbitrary, we require $L_\D(f) - \min_{h \in \H} L_\D(h) \leq \epsilon$ with probability $1-\delta$.

In both the realizable and agnostic settings, we look for PAC learners with small $\epsilon$ and $\delta$ as a function of the sample size $n$. An equivalent perspective looks at the sample complexity $m(\epsilon,\delta)$, which is the minimum sample size which guarantees error  at most $\epsilon$ with probability at least $1-\delta$. We say a problem is \emph{PAC learnable} if its PAC sample complexity is finite whenever $\epsilon,\delta > 0$.

For most PAC learning problems, learnability and sample complexity are characterized in terms of a  ``dimension'' of the hypothesis class. Most prominently this is the \emph{VC dimension} for binary classification, the \emph{fat shattering dimension} for agnostic regression, and the \emph{DS dimension} for multiclass classification (see \cite{anthony_neural_1999,daniely_optimal_2014,brukhim_characterization_2022}). Treatment of these is beyond the scope of this article. The unfamiliar reader need not worry, however,  as dimensions will feature only tangentially in our~discussion.




%\paragraph{Learning settings: Realizable, Agnostic, etc.} In learning theory, evaluating a supervised learning algorithm requires specifying a data model and an objective. We will leave the details of the data model flexible for now, to allow for both the PAC model and the adversarial transductive model. Nonetheless we will describe two variations, which we call ``settings'', which cut across different models. The  \emph{realizable setting}  requires only that the data be perfectly explained by some hypothesis $h \in \H$ --- i.e., there exists a hypothesis which is guaranteed to suffer a loss of $0$ on training and test data. The performance of the learning algorithm is its expected loss at test time for some ``worst case'' realizable instance. More generally, the \emph{agnostic setting} makes no assumption relating the data to the hypotheses, but shifts the goalposts as necessary to allow nontrivial guarantees: the expected loss at test time is evaluated only ``relative'' to that of the best hypothesis $h^* \in \H$, again for some ``worst case'' instance. There are other settings which make more nuanced assumptions about the data, such as it is drawn from a distribution of a particular parametric form, or that it lives in some (unknown) lower-dimensional space, etc. We will mostly discuss the realizable and agnostic settings, those being the simplest and most studied from a theoretical perspective.




%%% Local Variables:
%%% mode: latex
%%% TeX-master: "learning_matching"
%%% End:


\paragraph{Running example: fault-tolerant production line}
\label{sec:exe}
\section{Example}
\label{app:example}

Table~\ref{tab:example} shows a counseling session between a client agent and CAMI, with the client's profile shown in Table~\ref{tab:example profile}. During the session, CAMI explores Health and Relationships topics in the first 6 turns.  It subsequently performs Step Into operation from Relationships to Parenting topics in Turn 7, and from Parenting to Paternal Bond topics in Turn 9, before a Switch operation to Habituation topic in Turn 11.  The full exploration path is visualized in Figure~\ref{fig:exploration path}.

\begin{table*}[ht]
\centering
\begin{tabularx}{\textwidth}{c X}
\toprule
        {\bf Turn ID} & {\bf Utterances}                     \\ \midrule
1       & Counselor: Hello. How are you?                                                                                                                                                                                                                                                                                                                     \\ \hline
2       & Client: I am good. What about you?                                                                                                                                                                                                                                                                                                                 \\ \hline
3       & Counselor: I'm doing well, thanks for asking! Can you share how you've been feeling about your health lately, especially in relation to smoking?                                                                                                                                                                                                   \\ \hline
4       & Client: I've been pretty busy with work and juggling family stuff. Sometimes I just think about how nice it is to take a break with my coworkers outside, chatting and during those moments, life feels a bit more relaxed. You know how it is—it’s nice to unwind.                                                                                \\ \hline
5       & Counselor: It sounds like smoking is a comforting routine for you amidst your busy life with work and your daughter. I wonder how it might impact the time you spend with your family.                                                                                                                                                             \\ \hline
6       & Client: I get what you mean, but I really don't see my smoking as an issue. It helps me manage stress, and I don’t think it affects my time with my daughter. We still have good moments together, and I wouldn't want to change that.                                                                                                             \\ \hline
7       & Counselor: It sounds like you find comfort in smoking to manage stress, and that feels crucial to your parenting now. I wonder, though, how you might feel if your daughter started to see smoking as a way to unwind, too. What impact would that have on your relationship with her?                                                             \\ \hline
8       & Client: That’s a tough thought. I remember starting young and how hard it was to quit later. I wouldn’t want that for her. It makes me reflect on what I’m modeling.                                                                                                   \\ \hline
9       & Counselor: It's great that you're reflecting on your modeling for your daughter. Perhaps seeing smoking as something that may weaken your bond could help reshape how you approach stress—considering healthier ways to connect can actually strengthen your relationship and set a positive example for her.                                      \\ \hline
10      & Client: I appreciate that perspective. I do want to set a strong example for her. My experience with my own family and their struggles makes me cautious. I see the patterns and don’t want her to feel that smoking is a go-to solution for stress, like I did.                                                                                   \\ \hline
11      & Counselor: It sounds like your family’s struggles have really shaped your perspective on smoking. You want to protect your daughter from those same patterns, knowing that smoking can not only impact her stress management but also her overall learning and adaptation to challenges. That’s a powerful motivation.                             \\ \hline
12      & Client: Exactly. I want her to learn healthier coping strategies instead of following in my footsteps. I realize that I need to make a change for both our sakes. It’s not just about me; it’s about creating an environment where she can thrive without those habits. It's time to find better ways to manage stress that we can share together.   
\\ \bottomrule
\end{tabularx}
\caption{Counseling Session between CAMI and a Simulated Client.}
\label{tab:example}
\end{table*}



\begin{table*}[ht]
\centering
\begin{tabularx}{\textwidth}{l X}
\toprule
{\bf Profile Item}        & {\bf Description}                     \\ \midrule
Behavioral Problem  & Smoking \\ \hline
State of Mind       & Precontemplation \\ \hline
Personas            & You smoked when you were young.", "You have a 12-year-old daughter. \newline The friends of the your daughter have tried smoking. \newline When you are at work, you all go out and smoke and chit chat and talk about things. \newline You like to have a cigarette after eating. \\ \hline
Motivation          & You are motivated because of the risk of smoking in habituation for your daughter, as you don't want your daughter to think smoking is good thing.    \\ \hline
Beliefs             & You feel smoking enjoyable and relaxing. \newline You would be left out if you would hate to not smoke. \newline You use smoking as a way to manage weight and control eating habits. \newline You have been smoking for a long time and feels it's almost impossible to quit. \\ \hline
Motivation Topics          & Interpersonal Relationships, Parenting, Habituation
\\ \bottomrule
\end{tabularx}
\caption{Client's Profile in the Example.}
\label{tab:example profile}
\end{table*}


\begin{figure*}
    \centering
    \includegraphics[width=\linewidth]{figs/example_explore_path.pdf}
    \caption{Topic Exploration Path by the Counselor in the Example.}
    \label{fig:exploration path}
\end{figure*}




\section{Modular Rewritable Stochastic PN: Symmetries and Lumpability}
\label{sec:rewPT}
Rewritable stochastic PT nets (RwSPT) build upon the concept of \emph{modular} rewritable PT nets \cite{CAPRA-TCS2024} by linking negative exponential rates to the firing of PT transitions and the process of net rewrites.
%priorities and stochastic parameters. An RwSPT serves as an algebraic model of a Generalized Stochastic PN \cite{GSPN1993}, combining rewrite rules with the PT firing mechanism. We here concentrate on stochastic PN consisting of zero-priority transitions, accompanied by an 

The definition of RwSPT includes a hierarchy of \textbf{Maude} modules (e.g., \texttt{BAG}, \texttt{PT-NET}, \texttt{PT-SYSTEM}) described in  \cite{CAPRA-TCS2024}.
The \texttt{Maude} sources can be found in \url{https://github.com/lgcapra/rewpt/tree/main/modSPT}.
RwSPT uses structured annotations to underline the symmetry of the model. It features a concise place-based encoding to aid in state canonization and is based on the functional module \verb|BAG{X}|, which introduces multisets as a complex data type. The commutative and associative \verb|_+_| operator provides an intuitive way to describe a multiset as a weighted sum, for instance, \verb|3 . a + 1 . b|. The sort \verb|Pbag| contains multisets of places.
Each place label (a term of sort \verb|Plab|) is a nonempty list of pairs built of \verb|String| and a \verb|Nat|. Places are uniquely identified by their labels. These pairs represent a symmetric component within a nested hierarchy. Compositional operators annotate places incrementally from right to left: The label suffix represents the root of a hierarchy. 
For example, the 'assembly' place of line 1 in Production Line 2 would be encoded as:
%\begin{center}
\verb|p(< "a"; 0 > < "L"; 1 >)|.     
%\end{center}
We implicitly describe net transitions (terms \verb |Tran|) through their incidence matrix (a 3-tuple of terms \verb|Pbag|) and associated tags. A tag includes a descriptive \verb|String|
%, a \verb|Nat| (indicating a priority) 
and a \verb |Float| interpreted as a firing rate. 
%or a probabilistic weight, depending on whether the priority is zero or greater).
The syntax is:
\verb|[I,O,H] -> << S, R >>|. 

Using the associative composition operator \verb"_;_" and the subsort relation \verb"Tran < Net", we can easily construct PT nets in a modular way. For example, we can depict the subnet containing the load transition ($ld$) and a robot ($ln_0$ ) as the \verb"Net" term in the listing below.
%(the zero-arity operator \verb"nilP" represents an empty multi-set).
%in Appendix \ref{apx:ap1}.


{\small\begin{lstlisting}[frame=single,language=maude]
  [2 . p(< "s" ; 0 >), 1 . p(< "w" ; 0 >) + 1 . p(< "w" ; 1 >), nilP] |->  << "ld", 0.5 >> ;
  [1 . p(< "w" ; 0 >), 1 . p(< "a" ; 0 >), 1 . p(< "f" ; 0 > ] |-> << "ln", 0.1 >>
\end{lstlisting}}

A \verb|System| term is the empty juxtaposition (\verb|__|) of a \verb|Net| and a \verb|Pbag| representing the marking. 
The conditional rewrite rule \verb|firing| specifies the PT firing rule
%\footnote{
(notice the use of a matching equation :=).
%: The free variables T, N',
%are matched (:=) against the canonical ground term bound to the variable N.}, as shown in the listing \ref{lst:l2}.
%in Appendix \ref{apx:ap1}.

{\small\begin{lstlisting}[frame=single,language=maude]
 vars N N' : Net . vars T  : Tran . var M : Pbag .
 crl [firing] : N M => N firing(T, M) if T ; N' := N  /\ enabled(T, N M) . 
\end{lstlisting}}

%The predicate \verb|enabled| takes into account priority and 
%relies on \verb|hasConcession|, which determines 
%the 'topological' part of enabling.
%(listing \ref{lst:l2-bis}):

An RwSPT is defined by a system module that contains two constant operators, used as aliases:
\verb|op net : -> Net | and \verb|op m0 : -> Pbag |.
Two equations define their bindings.
This module includes a set $R$ of \verb|System| rewrite rules incorporating \verb|firing|.
We adopt interleaving semantics: Rewrites 
%take the same priority and 
have an exponential rate (specified in the rule label but for \verb|firing|), so that for the state transition system it holds ($\subseteq$ is the subgraph relation):
$TS($\verb|net m0|, $\{$\verb|firing|$\}) \subseteq TS($\verb|net m0|, $R)$.

%Transitioning between the \texttt{Maude} encoding of PT systems and the PNML format adopted by many PN tools is straightforward and reversible.
\vspace{-11pt}
\paragraph{Modularity, symmetries, and lumpability} We have provided net-algebra and net-rewriting operators \cite{CAPRA-TCS2024} with a twofold intention: to ease the modeler's task and to enable the construction and modification of large-scale models with nested components by implicitly highlighting their symmetry.
A compact \emph{quotient} TS is built using simple manipulation of node labels.
This approach outperforms that integrated into \texttt{Maude} \cite{Capra:RP22} and based on traditional graph canonization.

In this context, the identification of behavioral equivalences is reduced to a graph \emph{morphism}. PT system morphism must maintain the edges and the marking: In our encoding,
a \emph{morphism} between PT systems \verb|(N m)| and \verb|(N' m')| is a bijection $\phi \ :$ \verb|places(N)| $\rightarrow$ \verb|places(N')| such that, considering the homomorphic extension of $\phi$ on multisets, $\phi($\verb|N|$) = \ $\verb|N'| and $\phi($\verb|m|$) = \ $\verb|m'|.
Moreover, $\phi$ must retain the textual annotations of the place labels and the transition tags. If \verb|N'| = \verb|N| we speak of \emph{automorphism}, in which case $\phi$ is a permutation in the set of places.
We refer to a \emph{normal} form that principally involves identifying sets of automorphic (permutable) places:
Two markings \verb|m|, \verb|m'| of a net \verb|N| are said automorphic if there is an automorphism $\phi$ in \verb|N| that maps \verb|m| into \verb|m'|.
We denote this \verb|m| $\cong$ \verb|m'|. The equivalence relation $\cong$ is a congruence, that is, it preserves the transition firings and \emph{rates}.
%The next definition helps us simplify the process.
\begin{definition}[Symmetric Labeling]
\label{def:modsym}
A \verb|Net| term is symmetrically labeled if any two maximal sets of places whose labels have the same suffix (possibly empty), which is preceded by pairs with the same tag, are permutable. A \verb|System| term is  symmetrically labeled if its \verb|Net| subterm is.
\end{definition}

\noindent In other words, if a \verb|Net| term \verb|N| meets definition \ref{def:modsym}, then for any two maximal subsets of places matching:

$P := \{$\verb|p(L' < w ; i > L)|$\}, \quad P' := \{$\verb|p(L'' < w ; j > L)|$\}$,

where: \verb|L, L', L'' : Plab, w: String, i, j : Nat| 

\noindent there is an automorphism $\phi$ such that $\phi(P) = P'$, $\phi(P') = P$,
%is a  that maps one set to the other,
which is extended as an identity to the rest\footnote{According to the definition of PT morphism, the prefixes \texttt{L'} and \texttt{L''} are consistent in the textual component.}.

If a \verb|System| term adheres to the previous definition, it can be transformed into a 'normal' form by merely swapping indices on the place labels (e.g., i $\leftrightarrow$ j), while still complying with definition \ref{def:modsym}. This normal form is the most minimal according to a lexicographic order within the automorphism class ($\cong$) implicitly defined by \ref{def:modsym}. In contrast to general graph canonization, there is no need for any pruning strategy or backtracking: A monotone procedure is used where the sequence of index swaps does not matter (see \cite{CAPRA-TCS2024}). Efficiency is achieved as the normalized form of the subterm \verb | Net | is derived through basic ``name abstraction``, where at each hierarchical level the indices of structured place labels continuously span from $0$ to $k \in \Nat$.  

The strategy involves providing a concise set of operators that preserve nets' symmetric labelling. This set includes \emph{compositional} operators %(influenced by process algebra)
and operators for \emph{manipulating} nets.
%, such as adding/removing components.
Rewrite rules require these operators to manipulate \verb|System| terms defined in a modular manner. 
Furthermore, rules must adhere to parametricity conditions that limit the use of ground terms \cite{CAPRA-TCS2024}.
%\paragraph{Lumpability}
Under these assumptions, we get a \emph{quotient} TS from a \verb|System|
term that
retains reachability and
meets strong bisimulation. 

Let $t,t',u,u'$ be (final) terms of sort \verb|System|, 
%$\hat{t}$ the normal form of $t$,
$r$ a \verb|System| rule $r : \ s \Rightarrow s'$. The notation $t \overset{r(\sigma)}{\Rightarrow} t'$
%($t$ is rewritten into $t'$ through $r$)
means there is a ground substitution $\sigma$ of $r$'s variables such that $\sigma(s) = t$ and $\sigma(s')= t'$.
%which meets Definitions \ref{def:symr},\ref{prop:symbr}, 
%and $\hat{r} : \ s \Rightarrow normalize(s')$
%(if we refer to terms that meet Definition \ref{def:modsym}, $u \cong t$ is equivalent to $\hat{u} = \hat{t}$).
%If $\sigma$ is a ground substitution of $r$'s variables and $\phi$ is a morphism between $t$ and $t'$ (i.e., a map from the places of $t$ to the places of $t'$), then $\phi(\sigma)$ is the ground assignment induced by $\phi$ (we recall that \verb|Net| and \verb|System| terms are built of \verb|Place| terms). 
\begin{property}
\label{prop:transition-corr}
Let
%rule $r$ meet Definitions~\ref{def:symr},~\ref{prop:symbr} and 
$t$ meet Definition \ref{def:modsym}.
%\begin{itemize}
    %\item[-] If $t \overset{r(\sigma)}{\Rightarrow} t'$ then $ \exists  \sigma' : \hat{t} \overset{\hat{r}(\sigma')}{\Rightarrow} \hat{t}'$ ($t'$ meets Definition \ref{def:modsym})
    %, where $\phi$ is a morphism $t \rightarrow \hat{t}$.
%\item[-] If $\hat{t} \overset{\hat{r}(\sigma)}{\Rightarrow} \hat{t}'$ then $\forall u \cong t$ $\exists \sigma', u' \cong t' : \ u \overset{r(\sigma')}{\Rightarrow} u'$ ($u'$ meets Definition \ref{def:modsym})
%where $\phi$ is a morphism $t  \rightarrow  u$.
If $t \overset{r(\sigma)}{\Rightarrow} t'$, then $\forall u, \phi, t \cong_{\phi} u$: $ u \overset{r(\phi(\sigma))}{\Rightarrow} u'$, $ t' \cong u'$ 
%($u', t'$ meet the definition \ref{def:modsym})
%\end{itemize}
\end{property}

The TS quotient generated by a normal form $\hat{t}$ is obtained by applying the operator \verb|normalize| to the terms on the right side of the rewrite rules. When a \verb|System| undergoes a rewrite due to the \verb|firing| rule, the process only involves the marking subterm.
%This means applying \verb|normalize| (which is overloaded) to the 
%right-hand side of rule 
%subterm \verb | firing (T, M)| 
%in the listing 1.2.
According to property \ref{prop:transition-corr} (firing preservation), because the morphism $\phi$ preserves the transition rates and the rules are parameterized, it is feasible to map the TS quotient of $\hat{t}$ onto an isomorphic "lumped" CTMC: In a Markov process's state space, an equivalence relation is considered "strong lumpability" if the cumulative transition rates between any two states within a class to any other class remain consistent. Despite the possibility of establishing a more stringent condition, namely "exact lumpabability," we focus on the aggregated probability.
%if the cumulative transition rates from states of an equivalence class into each state of another (or the same class) are all equal.


%\noindent \textbf{Example}
%To demonstrate the aforementioned concepts, we will outline a model of a modular Production System with graceful degradation (Section \ref{sec:exe}). Initially, it is composed of $N$ Production Lines (PL) that share raw materials, with each PL split into $K$ interchangeable lines.
%We first define the net transitions and then build a PL using the \verb"repl&share" operator. The term \verb"PL(K)" represents a line with \verb"K" symmetric branches (Figure 1, top). The structure of the submodel is expressed by adding a pair with the tag \verb|"L"| to the place labels. For example, \verb|p(< "w" ; 0 > < "L" ; 1 >)| describes the "working" place of line 1 of the Production Line. We can choose to exclude places to share among replicas: In this case, we exclude those representing the "warehouse" (tag "s") and faults (tag "o"). Additionally, we can indicate transitions to share: For instance, "load" and "assembly" are shared.
%As a result, the incident input/output edges are weight-\verb|K|.

%The term \verb | Net | \verb | NPL(N, K)| consists of \verb|N| PLs, each of which contains \verb|K| branches. This net was generated using the \verb|repl&share| operator, which adds the \verb|"PL"| tag to place labels to indicate an additional nesting level. The sharing mechanism ensures each PL gathers \verb|K| raw pieces. The PT net represented by \verb|NPL(2,2)| can be seen in Figure \ref{fig:symPL-degarde2}, top-right. Furthermore, the term \verb | System | \verb | NPLsys (N, K, M) | is a PT system that holds \verb|K*M| tokens in the 'warehouse' place, with a single token in each place tagged with \verb|"o"| to trigger fault occurrences within a PL.
%We can build an identical model using the "symmetric" version of the process algebra ALT operator.
%The \verb|System| term generated using the above operators possesses symmetrical labeling (refer to definition \ref{def:modsym}), and its \verb|Net| subterm has already been normalized. 
%Consider, e.g., \verb|NPLsys(2, 2, 1)|.
%By triggering the conflicting transitions "load", which are initially enabled, the following two markings (essentially, subterms of the \verb|System| terms) can be obtained:

%\begin{itemize}
%\item[$m_1$] \verb|p(< "o"; 0 > < "PL"; 0 >) + p(< "o"; 0 > < "PL"; 1 >) +|

%\verb|p(< "w"; 0 > < "L"; 0 > < "PL"; 1 >) + p(< "w"; 0 > < "L"; 1 > < "PL"; 1 >)|

%\item[$m_2$] \verb|p(< "o"; 0 > < "PL"; 0 >) + p(< "o"; 0 > < "PL"; 1 >) +| 

%\verb|p(< "w"; 0 > < "L"; 0 > < "PL"; 0 >) + p(< "w"; 0 > < "L"; 1 > < "PL"; 0 >)|.
%\end{itemize}

%These are automorphic (one can be converted into the other by interchanging \verb|< "PL"; 1 >| \(\leftrightarrow\) \verb|< "PL"; 0 >|), but the second marking is the smallest in lexicographic order and hence corresponds to the normalized form.

%A rewrite rule encapsulates the self-adjustment of a PL with $K=2$ in response to a fault, allowing it to function at diminished capacity (Figure \ref{fig:symPL-degarde2}).
%This rule deviates slightly from \cite{CAPRA-TCS2024}, as it is locally activated by a breakdown, leading to a significantly larger TS. Skipping technical details, we point out that the rule meets the parameterity and only employs operators that uphold the definition \ref{def:modsym}, such as \verb | join |, \verb | detach |, \verb | setMark |. 
%A similar rule removes a faulty and degraded PL from the system. These rules are based on net-operators that retain symmetrical labeling (definition \ref{def:modsym}).
%With the model-checking facilities of \texttt{Maude} (command \texttt{search}), it is possible to formally demonstrate that for any given $N$, the quotient transition system has two absorbing states: Every state comprises a deteriorated PL that contains all $2 \cdot M$ parts (unprocessed, except possibly one).
%This is equivalent to the command below, which yields the same results as its unconditioned counterpart.
%\begin{verbatim}
%search NPLsys(N,2,M) =>! F:System such that 
%net(F:System) == faultyPL /\ B:Pbag := marking(F:System) /\ 
%| match(B:Pbag, "w") | + | match(B:Pbag, "a") | == %2 * M  .    
%\end{verbatim}


%%ARRIVATO QUI
 

\section{Getting the Lumped CTMC generator from RwSPT}
\label{sec:CTMC}
A rewritable PT \texttt{system}  generates a transition system (TS) isomorphic to a lumped CTMC. 
However, the TS produced via the \texttt{show search graph} command of \texttt{Maude} embodies a \emph{parametric} CTMC: in line with the rewriting logic semantics, state transitions denote \emph{classes} of equivalent rewrites, meaning, PT firings that result in identical normalized markings or net rewrites that lead to isomorphic PT systems.

%Consider, e.g., the marking $m_2$ of the previous example: %which matches a normalized form. 
%We can easily verify that this marking enables two equivalent transitions "line" ($ln_i$) and two equivalent transitions "fault" ($ft_i$), respectively, that lead to equivalent (automorphic) markings.
To obtain the CTMC generator, it is necessary to quantify instances that align with a specific state transition. Regrettably, the \texttt{Maude} system lacks a mechanism to determine the matches of a rewrite rule in the TS construction process.

%To achieve our goal,
%We have three possible strategies.
%\begin{enumerate}
%    \item Use a partial "edge unfolding" operator to calculate all possible rewrites from a normalized source state to a normalized target state of the TS. It requires post-processing the TS generated by \texttt{Maude}.
%    \item Use \texttt{Maude}'s meta-level to establish meta-rules that can identify potential matches of base rules.
%    \item 
  Our solution consists of first (automatically) generating a Transition System with states having a composite structure, which provides a detailed view of the equivalent rewrites that result in state transitions. Subsequently, using elementary parse to compute the cumulative rates of the lumped CTMC.
%\end{enumerate}
 %The second strategy, although elegant, has been shown far from efficient. Because efficiency is our primary concern, we have opted for a middle-ground solution: the third strategy.
 Despite a redundant state representation, this method incurs an acceptable time overhead because it only involves \emph{normalized} states. 
 %It relies on two operators. The first operator provides potential matches for each rule considering the subset of independent variables involved, and then the second emulates the rewriting process. It is possible to automatically define these two operators from the syntax of a rule. 
 %To better illustrate the idea, refer to the listing \ref{lst:l5} in Appendix \ref{apx:ap1}.

%The state structure defined by the mixfix constructor \verb|StateTranSys| comprises four fields. The initial pair describes the PT system, while the remaining two fields detail the state transitions caused by the \verb|firing| rule and other rewrites, in that order. We collect state transitions that share the target for calculating aggregated rates. Let us take the \verb|firing| rule as an example: it has two associated operators, namely \verb|enabled|, which gives the set of possible transitions, and \verb|fire|, which provides the reached markings,
%(in their normal form),
%each one prefixed by a multiset of transition tags. We proceed analogously for other rules.

%The operator \verb|toStateTran| converts the conventional state representation into a structured one. The firing rule is adapted easily.
When considering the term \verb|NPLsys(2,2,2)|, which aligns with the PT net at the top left of Figure \ref{fig:symPL-degarde2}, the resulting quotient TS comprises 295 states compared to the 779 states in the standard TS. State transitions often correspond to multiple matches: For instance, the initial state (the term above) includes two 'load' instances and four 'fault' instances that lead to markings with identical normal forms. Consequently, the combined rates are $2\cdot0.5$ and $4\cdot0.001$. 
%Equivalent rewrites of the net structure are observed when $N > 2$. 

\vspace{-10pt}
\paragraph{Experimental Evidence}
We showcase experimental validation of the method and a demonstration of standard performance indicators. Table \ref{tab:perf} displays the results of the \texttt{search} command to locate the final states.
We used Linux WSL on an 11th Gen. Intel Core i5 with 40GB RAM. The state spaces match those of the lumped CTMC. The analysis of large models is feasible solely by exploiting the model's symmetry.
Notice that the number of absorbing states in the TS quotient does not vary with $N$. 
%In \cite{CAPRA-TCS2024}, the effectiveness of modular RwPT was compared to that of Symmetric Nets (SN, previously known as well-formed nets) \cite{CHIOLA97}, a colored Petri net that generates a symbolic reachability graph equivalent %(in its stochastic extension)
%to a lumped CTMC. A parametric SN model was developed, whose unfolding coincides with \verb| NPLsys (N, K, M)|
%but without the degradation process.
%As the values of N and K increase, the degree of state aggregation in modular RwPT significantly exceeds that of SN. For instance, when N=10, K=3, and M=3, the degree of state aggregation is about 45 times that of SN, and when N=10, K=4, and M=3, it is about 200 times that of SN.

%\vspace{-3pt}
%As an example of performance metrics, 
Figure
%\ref{fig:PlotX} shows the system throughput, while 
\ref{fig:PlotRel} shows the system \emph{reliability} (the complement of Time to System Failure distribution).
%as a time function. 
As expected, %both 
is decreases with time; additionally, the scenario that involves more replicas demonstrates
%increased throughput and
enhanced reliability.
The inflexion point at around time 800
%in both curves
represents the system's reconfiguration time. The increased execution time of the job (not reported) is a result of a system failure.
The overall trend is also noticeable when we look at larger values of \verb|N|. As \verb|N| increases, both reliability and throughput curves show significant improvements.
However, we observe an asymptotic trend when \verb|N| is greater than 6. Our interpretation is that beyond a certain point, the benefit of using a higher number of replicas is outweighed by the higher fault rate and the increased configuration overhead.

\vspace{-0.5cm}
\begin{table}[htbp]
\centering\small
\caption{Ordinary vs Quotient TS of \texttt{NPLsys(N,2,2)} \hspace{2pt} ${}^\dag$ \texttt{search} timed out after 10 h} 
\label{tab:perf}
\begin{tabular}{ |p{0.5cm}||p{2.5cm}p{1.5cm}||p{2.5cm}p{1.5cm}| }
\hline
 & \multicolumn{2}{c||}{Ordinary} & \multicolumn{2}{|c|}{Quotient} \\
\verb|N|   & states(final)  & time (sec) &  states(final) & time (sec) \\
\hline
\hline
1   &    60(2) & 0 &      42(2) & 0\\
\hline
2   &    779(4) & 0.1 &     295(2) & 0.1\\
\hline
3  &   6101(6) & 4.8  &  1059(2) & 0.9\\
\hline
4 &    37934(8) & 69 & 2764(2) & 3.6\\
\hline
5 &   204362(10) & 818 & 5970(2) & 10 \\
\hline
6 &   1000187(12) & 13930 & 11367(2) & 27\\
\hline
7 &  - & ${}^\dag$ & 19775(2) & 65 \\
\hline
8 &  - & ${}^\dag$ &  32144(2) & 186 \\
\hline
9 & - & ${}^\dag$ &  49554(2) & 569 \\
\hline
10 & - & ${}^\dag$ & 73215(2) & 2450 \\ 
\hline
\end{tabular}
\end{table}
%\begin{figure}[htbp]
%   \begin{center}    
%    \includegraphics[clip,trim={2.1cm 2.7cm 2.2cm 2.2cm},width=0.85\columnwidth]{figures/PlotThroughput.pdf}
%    \end{center}
%    \caption{System Throughput.}
%    \label{fig:PlotX}
%\end{figure}

\begin{figure}[htbp]
   \begin{center}    
    \includegraphics[clip,trim={2.1cm 2.7cm 2.2cm 2.2cm},width=0.7\columnwidth]{figures/PlotReliability.pdf}
    \end{center}
    \caption{System Reliability.}
    \label{fig:PlotRel}
\end{figure}

%To evaluate the system's performance, Figure \ref{fig:PlotXdivRel} shows the throughput while the system is operational, which is the ratio between the graphs in Figures \ref{fig:PlotX} and \ref{fig:PlotRel}. It can be seen that the throughput approaches that of a single line, which, given the parameters, is $1/202.5 = 4.98E-03$.

%\begin{figure}[htbp]
%   \begin{center}    
%    \includegraphics[clip,trim={2.1cm 2.7cm 2.2cm 2.2cm},width=0.85\columnwidth]{figures/PlotConditionedThroughput.pdf}
%    \end{center}
%    \caption{System Throughput conditioned to its reliability.}
%    \label{fig:PlotXdivRel}
%\end{figure}



\section{Conclusion and Future Work}
We have developed a lumped Markov process for modular and rewritable Petri nets (RwPT), a flexible model of adaptive distributed systems.
%encoded in \texttt{Maude}.
RwPTs, which we construct and manipulate using a small set of algebraic operators, exhibit structural symmetries that result in an efficient quotient state-transition graph.
%Using an example of a gracefully degrading system,
We have outlined a semi-automatic procedure for deriving the CTMC infinitesimal generator from the RwPT quotient graph. Future efforts will focus, on the one hand, on exploring orthogonal structured solutions and, on the other, on fully implementing the process and integrating it into graphical editors.
%such as \texttt{DrawNET} (\url{https://www.draw-net.com/}). 
%Simultaneously,
We aim to broaden the approach
%on one side,
to derive a lumped Markov process 
%from rewritable GSPN, and on the other, to extract the infinitesimal CTMC generator
from any \texttt{Maude} specification.


%% Define the bibliography file to be used
\bibliographystyle{eptcs}
\bibliography{lc}

\end{document}
\newpage
\appendix
\section{Appendix: Figures and \texttt{Maude} excerpts}
\label{apx:ap1}

\begin{figure}[htbp]
   \begin{center}    
    \begin{tabular}{c}
        \fbox{\includegraphics[scale=0.5]{figures/PL}}\\
        $\Downarrow$\\

  \fbox{\includegraphics[scale=0.5]{figures/degrPL}}
    \end{tabular}
    \end{center}
    \caption{Production Line (PL) and adaptation following a fault.}
    \label{fig:FMS}
%\end{figure}

%\begin{figure}[htbp]
   \begin{center}    
    \begin{tabular}{rl}
        \fbox{\includegraphics[scale=0.33]{figures/2PL}}&
        $\Rightarrow$
        \fbox{\includegraphics[scale=0.33]{figures/2PL1F}}
        $\Rightarrow$\\
        $~$&$~$\\
        $\Rightarrow$
        \fbox{\includegraphics[scale=0.33]{figures/2PL2F}}&
        $\Rightarrow$
        \fbox{\includegraphics[scale=0.33]{figures/2PLfinal}}
    \end{tabular}
    \end{center}
    \caption{One of the possible paths of the Gracefully Degrading Production System.}
    \label{fig:symPL-degarde2}
\end{figure}

\noindent The following are the listings relevant to the procedures described in this paper.

{\small
\begin{lstlisting}[frame=single,caption=Topological enabling condition \label{lst:l2-bis},language=maude]
vars X Y Z X' : Pbag .
op hasConcession : Tmatrix  Pbag -> Bool .
eq hasConcession([X,Y,Z], X') = X <= X' and-then Z > X' .
\end{lstlisting}}

{\small
\begin{lstlisting}[frame=single,,caption=Modular Specification of a Fault Tolerant Production System \label{lst:l3},language=maude]
fmod FTPL is
 pr NET-OP{SPTlab} .
 ops PL PLA nomPL faultyPL NfaultyPL : NzNat -> Net . 
 op faultySys : NzNat -> System . 
 op NPL  : NzNat NzNat -> Net [memo]. 
 op NPLsys : NzNat NzNat NzNat -> System . 
 ops loadLab asLab failLab workLab : -> Tlab [memo] .
 eq loadLab = << "ld",0, 0.5 >> .
 eq asLab = << "as",0, 2.0 >> .
 eq workLab = << "ln",0, 0.1 >> .
 eq failLab = << "ft",0, 0.001 >> .
 var I  : Nat . 
 vars N K M : NzNat .
 eq line =  [1 . p(< "w" ; 0 >),1 . p(< "a" ; 0 >),1 . p(< "f" ; 0 >) ] |-> workLab .
 eq fault = [1 . p(< "o" ; 0 >) , 1 . p(< "f" ; 0 >), nilP ] |-> failLab .
 eq load  = [1 . p(< "s" ; 0 >) , 1 . p(< "w" ; 0 >) , nilP ] |-> loadLab .
 eq ass   = [1 . p(< "a" ; 0 >) , 1 . p(< "s" ; 0 >)  , nilP ]  |-> asLab .
 eq cycle = load ; line ; ass ; fault .
 eq PL(K) = repl&share(cycle, K, "L", p (< "o" ; 0 >) U p(< "s" ; 0 >), asLab U loadLab) .
 eq NPL(N, K) = repl&share(PL(K), N, "PL", p(< "s" ; 0 >), emptyStlab) . 
 eq NPLsys(N, K, M) = setMark(setMark(NPL(N, K), "o" "PL", 1), "s", K * M) .
 ...
endfm
\end{lstlisting}
}

{\small
\begin{lstlisting}[frame=single,caption=Rewrite rule of a PL (the label contains the rule's exponential rate) \label{lst:l4},language=maude,label=list:rewrules]
vars S S' S'' : Pbag .
vars I  J : Nat .
var Sys Sys' : System .
var L : Lab .
crl [r1-0.005] : N  S => normalize(join(Sys, setMark(setMark(Sys', "w" "fPL", | match(S', "w") |), "a" "fPL", | match(S', "a") |)))  
if  S'' + 1 . p(< "f" ; J > L < "PL" ; I >) := S /\ N' := nomPL(I) /\ dead (N' S) /\ S' := subag(S'', < "PL" ; I >) /\ Sys := detache(N, N') S'' - S' /\ Sys' := faultySys(notIn(N, "fPL"))  .
\end{lstlisting}
}

{\small
\begin{lstlisting}[frame=single,caption=TS encoding for the lumped CTMC \label{lst:l5} ,language=maude,label=list:rewrules]
var N : Net .
vars M M' : Pbag .  
var  TS    : TagSet .
var  FS   : Fset .
var  RS   : Rset .
op NET:_ M:_ FIRING:_REW:_ : Net Pbag Fset Rset -> StateTranSys [ctor] .
op toStateTran : System -> StateTranSys . 
eq toStateTran(N M) = NET: N  M: M FIRING: fire(enabled(N M), M) REW: allRew(N M) . 
***
rl [sfiring] : NET: N M: M FIRING: (TS --> M', FS) REW: RS => toStateTran(N M') .

\end{lstlisting}
}


%%
%% If your work has an appendix, this is the place to put it.
%\appendix

%\section{Appendix - Maude specification of rewritable PT nets}
%\section{Proofs from Section~\ref{sec:gammaok}} \label{app:gamma}

\subsection{On the girth of locally \texorpdfstring{$\gamma$}{gamma}-sparse graphs}
\begin{lemma}\label{lemma:girth_rev}
    Let $G = (V,E)$ be an undirected graph with girth $g(G)$.
    Then $G$ is \ok{0} if and only if $g(G) \geq 5$.
\end{lemma}
\begin{proof}
    We first prove that if $G$ is \ok{0} then $g(G)$ must be at least $5$.
    In order to prove that, we simply negate the statement and prove that if $G$ has girth $<5$ then $G$ can not be \ok{0}.
    Without loss of generality, assume that $g(G) = 4$ (the case $g(G) = 3$ is similar).
    Then there must exist a cycle $C = (u_1, u_2, u_3, u_4)$ of $4$ vertices.
    It is simple to see that $u_2,u_4 \in \lset_1(u_1)$ and $u_3 \in \lset_2(u_1)$.
    Since $u_3$ is a neighbor of both $u_2$ and $u_4$, the degree of $u_3$ in the subgraph $G\left[\lset_1(u_1) \cup \{u_4\} \right]$ is at least $2$, hence $G$ is not \ok{0} (see \Cref{subfig:girth1}).
    
    We now prove that if $g(G) \geq 5$ then $G$ must be \ok{0}.
    Again, we negate this statement and prove that if $G$ is not \ok{0} then the girth of $G$ must be less then $5$.
    Let us assume that $G$ is \gammaok, for any $\gamma > 0$, thus it is not \ok{0}.
    Since $G$ is not \ok{0} there exists a vertex $v \in V$ such that at least one of the following properties holds (see \Cref{subfig:girth2}):
    \begin{enumerate}
        \item $\exists u \in \lset_1(v)$ such that the degree of $u$ in $G\left[ \lset_1(v) \right]$ is greater then $0$, or;
        \item $\exists w \in \lset_2(v)$ such that the degree of $w$ in $G\left[ \lset_1(v) \cup \{ w \} \right]$ is greater then $1$.
    \end{enumerate}
    In the first case, we have a cycle of $3$ vertices, then $g(G) = 3$.
    In the second case, we have a cycle of $4$ vertices, then $g(G) = 4$.
    In both cases $g(G) < 5$.
\end{proof}
\begin{figure}[h]
    \centering
    \begin{subfigure}[b]{0.35\linewidth}
            \centering
            \includegraphics[width=\linewidth]{img/girth-1.pdf}
            \caption{}
            \label{subfig:girth1}
    \end{subfigure}
    \begin{subfigure}[b]{0.6\linewidth}
            \centering
            \includegraphics[width=\linewidth]{img/girth-2.pdf}
            \caption{}
            \label{subfig:girth2}
    \end{subfigure}%
    \caption{}
    \label{fig:example_girth}
\end{figure}

\subsection{Deterministic lazy-update on \texorpdfstring{$\gamma$}{gamma}-sparse graphs}\label{apx:gamma-ok-deterministic}

\begin{theorem}\label{lemma:gamma-ok-error-bound-balls}
    
Let $\varepsilon \in (0,1)$, and let $G^{(0)}$ be an initial graph. Consider any sequence of edge insertions that yields a final graph $G$. If $G$ is \gammaok, \lazyscheme$(\varphi = \frac{\varepsilon}{1 - \varepsilon},k=0)$ has an approximation ratio of  $\frac{\gamma + 1}{1-\varepsilon}$ and amortized update cost $O(1/\varepsilon)$. 
    
\end{theorem}
\begin{proof}
Recall that $\bd_u$ denotes the black degree of $u$, and that  \Cref{alg:det_thresh} guarantees that $\deg_u$ is at most $(1+\varphi)\bd_u$.
    Then, it is simple to give an upper bound to the size of $\ball_2(u)$, that is $\vert \ball_2(u) \vert \leq 1+ \sum_{v \in \lset_1(u)} (1 + \varphi)\bd_v$.Consider a vertex $v \in \lset_1(u)$. Since $G$ is \gammaok, the number of neighbors of $v$ belonging to $\lset_2(u)$ is at lest $\deg_v - (\gamma+1)$ of which $\bd_v - (\gamma+1)$ must belong to $\apxball_2(u)$. Moreover, a vertex in $\lset_2(u)$ has at most $\gamma+1$ neighbors in $\lset_1(u)$. Therefore: 
    \begin{align*}
    \vert \apxball_2(u) \vert
    &\geq  \bd_u + 1 + \frac{1}{\gamma + 1}\sum_{v \in \lset_1(u)}(\bd_v - (\gamma + 1))\\
    &= \bd_u + 1 + \frac{1}{\gamma + 1}\sum_{v \in \lset_1(u)}\bd_v - \underbrace{\frac{1}{\gamma + 1}\sum_{v \in \lset_1(u)}(\gamma + 1)}_{= \bd_u}\\
    &= 1+ \frac{1}{\gamma + 1}\sum_{v \in \lset_1(u)}\bd_v.
    \end{align*}
  
    As a consequence, $\vert \apxball_2(u) \vert/\vert \ball_2(u) \vert \ge \frac{1}{(1+\varphi)(\gamma+1)}$. By setting $\varphi = \frac{\varepsilon}{1 - \varepsilon}$, and by using \Cref{lm:amortized_det_alg},  the claim follows.
\end{proof}

\subsection{Proof of \Cref{le:gamma_ok_expect_lowerbound}}\label{apx:proof_gamma_ok_expect_lowerbound}
\begin{proof}
Let $e_1, \dots, e_{\ell_v}$ be the \emph{red edges} between $v$ and $\lset_2(u)$, and define the binary random variable $\lrdr_v(i)$ that is equal to $1$ if $e_i$ is a \emph{quasi-black edge} for $u$, $0$ otherwise, for $i = 1, \dots, \lrd_v$. Thus we can express $\lrdr_v = \sum_{i=1}^{\lrd_v} \lrdr_v(i)$, with expectation

\begin{equation}\label{eq:gamma_ok_lb_fact_eq_1}
\begin{aligned}
  \Expec{}{\lrdr_v} & = \sum_{i=1}^{\lrd_v}{\Prob{}{\lrdr_v(i)=1}} = \lrd_v - \sum_{i=1}^{\lrd_v} {\Prob{}{\lrdr_v(i)=0}}.
\end{aligned}
\end{equation}

Without loss of generality, assume that the edges $e_1, \dots, e_{\lrd_v}$ have been inserted at times $t_1 < \dots < t_{\lrd_v}$, respectively.
If $e_i$ is not a quasi-black edge for $u$, then it must be that $u$ is not selected by $v$ at \Cref{line:random_selection} of \Cref{alg:det_thresh}, at times $t_i, t_{i+1},\dots, t_{\lrd_v}$.
This holds with probability 
\begin{equation}\label{eq:gamma_ok_lb_fact_eq_2}
\begin{aligned} 
    &\Prob{}{\lrdr_v(i) = 0}
    \leq \prod_{j=i}^{\lrd_v} \left( 1-\frac{k}{\deg_v^{(t_j)}} \right)
    \leq \prod_{j=i}^{\lrd_v} \left( 1 - \frac{k}{\deg_{v}^{(t_{\lrd_v})}} \right) \\
    &\leq \left( 1-\frac{k}{\lbdd_v + \lrd_v + \gamma + 1}\right)^{\lrd_v - i + 1} 
    \leq \left(1-\frac{k}{2(\lbdd_v + \gamma + 1)}\right)^{\lrd_v - i}.
\end{aligned}
\end{equation}
The third inequality holds since the edges incident to $v$ having endpoints in $L_1(u)$ are at most $\gamma$, while those having endpoints in $L_2(u)$ are exactly $\lbdd_v+ \lrd_v$. Moreover, the last inequality holds because $\lrd_v \leq \rd_v \leq \bd_v \leq \lbdd_v + \gamma + 1$, given the assumption $\varphi = 1$.

By plugging in \eqref{eq:gamma_ok_lb_fact_eq_2} into   \eqref{eq:gamma_ok_lb_fact_eq_1} and we obtain
\begin{align*}
    &\Expec{}{\lrdr_v} \geq \lrd_v - \sum_{i=1}^{\lrd_v}\left( 1-\frac{k}{2(\lbdd_v + \gamma + 1)}\right)^{\lrd_v - i} \\
    &= \lrd_v - \sum_{i=0}^{\lrd_v-1} \left(1-\frac{k}{2(\lbdd_v + \gamma + 1)}\right)^i 
    \leq \lrd_v - \frac{1-\left(1-\frac{k}{2(\lbdd_v+\gamma+1)}\right)^{\lrd_v}}{1-\left(1-\frac{k}{2(\lbdd_v + \gamma + 1)}\right)} \\
    &\geq \lrd_v - \frac{1}{1-\left(1-\frac{k}{2(\lbdd_v + \gamma + 1)}\right)}
    \geq \lrd_v - \frac{2(\lbdd_v + \gamma + 1)}{k}.
\end{align*}
\end{proof}

%\section{Appendix - auxiliary Maude modules}
%\section{Declarations}
%=======================================================================
%
\subsection{Acknowledgements}
%
Hagen Holthusen and Tim Brepols gratefully acknowledge financial support of the projects 417002380 and 453596084 by the Deutsche Forschungsgemeinschaft.
In addition, Kevin Linka is supported by the Emmy Noether Grant 533187597 by the Deutsche Forschungsgemeinschaft.
This work was supported by the NSF CMMI Award 2320933 Automated Model Discovery for Soft Matter and by the ERC Advanced Grant 101141626 DISCOVER to Ellen Kuhl.
%
%=======================================================================
%
\subsection{Conflict of interest}
%
The authors of this work certify that they have no affiliations with or involvement in any organization or entity with any financial interest (such as honoraria; participation in speakers’ bureaus; membership, employment, consultancies, stock ownership, or other equity interest; and expert testimony or patent-licensing arrangements), or non-financial interest (such as personal or professional relationships, affiliations, knowledge or beliefs) in the subject matter or materials discussed in this manuscript.
%=======================================================================
%
\subsection{Availability of data and material}
%
Our data used for training and testing are accessible to the public at \url{https://doi.org/10.5281/zenodo.14894687}
%=======================================================================
%
\subsection{Code availability}
%
Our source code and examples of the iCANN implementation in JAX are accessible to the public at \url{https://doi.org/10.5281/zenodo.14894687}.
%=======================================================================
%
\subsection{Contributions by the authors}
%
\textbf{Hagen Holthusen:} Conceptualization, Methodology, Software, Validation, Formal analysis, Investigation, Data Curation, Writing - Original Draft, Writing - Review \& Editing, Visualization, Funding acquisition\\
\textbf{Kevin Linka:} Methodology, Writing - Original Draft, Writing - Review \& Editing, Funding acquisition\\
\textbf{Ellen Kuhl:} Methodology, Writing – Original draft, Writing – review \& editing, Funding acquisition\\
\textbf{Tim Brepols:} Methodology, Writing - Original Draft, Writing - Review \& Editing, Funding acquisition\\
%=======================================================================
%
\subsection{Statement of AI-assisted tools usage}
%
This document was prepared with the assistance of OpenAI's ChatGPT, an AI language model. ChatGPT was used for language refinement. The authors reviewed, edited, and take full responsibility for the content and conclusions of this work.
