As an illustrative example, we refer to the model of a distributed production system that gracefully degrades presented in \cite{CAPRA-TCS2024}. The system is composed of $N$ production lines (PL), each branching into $K$ fully interchangeable robots, that handle K raw items in parallel 
%(a multiple of $K$). 
%These branches ($\{w_i, ln_i, a_i\}$, $i:0\ldots K-1$) are fully interchangeable.
and assemble them into an artifact. In this study, $K = 2$. The items are loaded from a warehouse in an PL, $K$ at a time. 
%The initial count of pieces (tokens) in $s$ is $K\cdot M$, where $M \in \Nat^+$ is another parameter of the model. For each artifact produced, $K$ new pieces are introduced. 
A robot in a PL might fail, upon which
%(transitions $ft_i$). When that occurs, 
the PL restructures to continue functioning, but with reduced capacity.
%Simple static analysis can show that the PL system reaches a \emph{deadlock} after a failure.
 %, with half the pieces ($M$) on the broken branch and half (ready to be assembled) on the other.
 %The net at the bottom of Figure~\ref{fig:FMS} shows the development of the PL after a fault happens. %(considering scenario $K = 2$).
The reconfiguration process involves moving items from the faulted branch of a PL to the remaining branch(es) to maintain the production cycle. Traditional PNs
%(including High-Level PN variants) 
are unable to model this operation. 
When a second fault occurs in a degraded PL, the system disconnects the PL. The leftover items are then relocated to the warehouse.
%Items left on the faulty line
%(represented as place $w_1$ here) 
%are transferred to the remaining functional line:
%($w_0$): 
%The marking of the PT net at the bottom demonstrates the state after adaptation. We assume that a PL that fails twice is beyond repair.
%We will examine a situation in which $N$ PL replicas function simultaneously and degrade in a regulated fashion. 
Figure \ref{fig:symPL-degarde2} shows the evolution of a system starting with two PLs. This scenario can be extended to $N$ PLs, each operating $K$ parallel robots,
%lines 
that handle $K \cdot M$ items ($M$ being a third parameter of the model), 
denoted by the term \verb|NPLsys(N, K, M)|. 
%The evolution proceeds in two phases:
%\begin{itemize}
%    \item[s1] 
%    When a fault affects an PL, it autonomously adjusts to continue operating at diminished capacity.
 %   \item[s2] 
 %(the final step in Figure \ref{fig:symPL-degarde2}):
%\end{itemize}
\begin{figure}
   \begin{center}    
    \begin{tabular}{rl}
        \fbox{\includegraphics[scale=0.33]{figures/2PL}}&
        $\Rightarrow$
        \fbox{\includegraphics[scale=0.33]{figures/2PL1F}}
        $\Rightarrow$\\
        $~$&$~$\\
        $\Rightarrow$
        \fbox{\includegraphics[scale=0.33]{figures/2PL2F}}&
        $\Rightarrow$
        \fbox{\includegraphics[scale=0.33]{figures/2PLfinal}}
    \end{tabular}
    \end{center}
    \caption{One of the possible paths of the Gracefully Degrading Production System.}
    \label{fig:symPL-degarde2}
\end{figure}
