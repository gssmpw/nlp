Traditional formalisms such as Petri Nets, Automata, and Process Algebra do not make it easy for designers to define dynamic system changes. Several extensions inspired by the $\pi$-calculus and the Nets-within-Nets paradigm have been proposed, but they often lack suitable analysis techniques. Rewritable PT nets (RwPT) \cite{lcICDCIT22} is versatile formalism for analyzing adaptive distributed systems. RwPT is specified using the declarative language \texttt{Maude}, which adopts Rewriting Logic to offer both operational and mathematical semantics, creating a scalable model for self-adapting PT nets.
Unlike similar approaches (\cite{Barbosa11,RPN-Maude2016}),
%which translate a less complex type of PNs into \texttt{Maude},
the RwPT formalism provides data abstraction, is concise and efficient, and avoids the limitations posed by 'pushout' in Graph Transformation Systems. RwPT is an extension of GTS. Considering graph isomorphism (GI) when identifying equivalent states within the model dynamics is crucial to scaling up the model complexity.
%, particularly when incorporating a Stochastic Process into the model's state space.
Recent work has shown that GI is quasi-polynomial \cite{GIBabai}.
Graph Canonization (GC) involves finding a representative such that for any two graphs $G$ and $G'$, $G \simeq G' \Leftrightarrow can(G) = can(G')$. We developed a general canonization technique \cite{Capra:RP22} for use with RwPT, integrated into \texttt{Maude}. This technique works well for irregular models, but it is less effective for more realistic models.

In \cite{CAPRA-TCS2024}, we presented a technique for constructing comprehensive RwPT models using algebraic operators. The strategy is simple: leverage the modular characteristics of the models during analysis. By employing composite node labelling, we capture symmetries and sustain the hierarchical organization through net rewrites. A benchmark case study illustrates the effectiveness of our method.
%, with results that indicate a significant advantage over nonmodular techniques.
In this paper, we demonstrate a procedure to derive a lumped Continuous-Time Markov Chain (CTMC) from the quotient graph formed by an RwPT model, after introducing stochastic parameters into the framework.
%We provide background information in Section \ref{sec:backgr} and detail our example in Section \ref{sec:exe}. The modular RwPT formalism, enhanced with stochastic parameters, is detailed in Section \ref{sec:rewPT}. In Section \ref{sec:CTMC}, we describe the process of deriving a lumped CTMC from an RwPT model and offer experimental proof of its efficacy. Finally, we conclude by discussing the ongoing work.

The potential of \texttt{Maude} as functional logic programming framework was discussed in \cite{Escobar2014} and recently in \url{https://logicprogramming.org/2023/02/extensions-of-logic-programming-in-maude/}. Although we do not consider free variables in terms, our work can be seen as an example of \emph{symbolic reachability} in which we use term rewriting through pattern matching (modulo normalization) instead of narrowing through unification. 
The use of narrowing to get a quotient state-transition system deserves further study.
%specification language combines logical variables, computing with partial information, and constraint solving with reasoning modulo e

