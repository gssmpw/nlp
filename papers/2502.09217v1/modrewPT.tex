Rewritable stochastic PT nets (RwSPT) build upon the concept of \emph{modular} rewritable PT nets \cite{CAPRA-TCS2024} by linking negative exponential rates to the firing of PT transitions and the process of net rewrites.
%priorities and stochastic parameters. An RwSPT serves as an algebraic model of a Generalized Stochastic PN \cite{GSPN1993}, combining rewrite rules with the PT firing mechanism. We here concentrate on stochastic PN consisting of zero-priority transitions, accompanied by an 

The definition of RwSPT includes a hierarchy of \textbf{Maude} modules (e.g., \texttt{BAG}, \texttt{PT-NET}, \texttt{PT-SYSTEM}) described in  \cite{CAPRA-TCS2024}.
The \texttt{Maude} sources can be found in \url{https://github.com/lgcapra/rewpt/tree/main/modSPT}.
RwSPT uses structured annotations to underline the symmetry of the model. It features a concise place-based encoding to aid in state canonization and is based on the functional module \verb|BAG{X}|, which introduces multisets as a complex data type. The commutative and associative \verb|_+_| operator provides an intuitive way to describe a multiset as a weighted sum, for instance, \verb|3 . a + 1 . b|. The sort \verb|Pbag| contains multisets of places.
Each place label (a term of sort \verb|Plab|) is a nonempty list of pairs built of \verb|String| and a \verb|Nat|. Places are uniquely identified by their labels. These pairs represent a symmetric component within a nested hierarchy. Compositional operators annotate places incrementally from right to left: The label suffix represents the root of a hierarchy. 
For example, the 'assembly' place of line 1 in Production Line 2 would be encoded as:
%\begin{center}
\verb|p(< "a"; 0 > < "L"; 1 >)|.     
%\end{center}
We implicitly describe net transitions (terms \verb |Tran|) through their incidence matrix (a 3-tuple of terms \verb|Pbag|) and associated tags. A tag includes a descriptive \verb|String|
%, a \verb|Nat| (indicating a priority) 
and a \verb |Float| interpreted as a firing rate. 
%or a probabilistic weight, depending on whether the priority is zero or greater).
The syntax is:
\verb|[I,O,H] -> << S, R >>|. 

Using the associative composition operator \verb"_;_" and the subsort relation \verb"Tran < Net", we can easily construct PT nets in a modular way. For example, we can depict the subnet containing the load transition ($ld$) and a robot ($ln_0$ ) as the \verb"Net" term in the listing below.
%(the zero-arity operator \verb"nilP" represents an empty multi-set).
%in Appendix \ref{apx:ap1}.


{\small\begin{lstlisting}[frame=single,language=maude]
  [2 . p(< "s" ; 0 >), 1 . p(< "w" ; 0 >) + 1 . p(< "w" ; 1 >), nilP] |->  << "ld", 0.5 >> ;
  [1 . p(< "w" ; 0 >), 1 . p(< "a" ; 0 >), 1 . p(< "f" ; 0 > ] |-> << "ln", 0.1 >>
\end{lstlisting}}

A \verb|System| term is the empty juxtaposition (\verb|__|) of a \verb|Net| and a \verb|Pbag| representing the marking. 
The conditional rewrite rule \verb|firing| specifies the PT firing rule
%\footnote{
(notice the use of a matching equation :=).
%: The free variables T, N',
%are matched (:=) against the canonical ground term bound to the variable N.}, as shown in the listing \ref{lst:l2}.
%in Appendix \ref{apx:ap1}.

{\small\begin{lstlisting}[frame=single,language=maude]
 vars N N' : Net . vars T  : Tran . var M : Pbag .
 crl [firing] : N M => N firing(T, M) if T ; N' := N  /\ enabled(T, N M) . 
\end{lstlisting}}

%The predicate \verb|enabled| takes into account priority and 
%relies on \verb|hasConcession|, which determines 
%the 'topological' part of enabling.
%(listing \ref{lst:l2-bis}):

An RwSPT is defined by a system module that contains two constant operators, used as aliases:
\verb|op net : -> Net | and \verb|op m0 : -> Pbag |.
Two equations define their bindings.
This module includes a set $R$ of \verb|System| rewrite rules incorporating \verb|firing|.
We adopt interleaving semantics: Rewrites 
%take the same priority and 
have an exponential rate (specified in the rule label but for \verb|firing|), so that for the state transition system it holds ($\subseteq$ is the subgraph relation):
$TS($\verb|net m0|, $\{$\verb|firing|$\}) \subseteq TS($\verb|net m0|, $R)$.

%Transitioning between the \texttt{Maude} encoding of PT systems and the PNML format adopted by many PN tools is straightforward and reversible.
\vspace{-11pt}
\paragraph{Modularity, symmetries, and lumpability} We have provided net-algebra and net-rewriting operators \cite{CAPRA-TCS2024} with a twofold intention: to ease the modeler's task and to enable the construction and modification of large-scale models with nested components by implicitly highlighting their symmetry.
A compact \emph{quotient} TS is built using simple manipulation of node labels.
This approach outperforms that integrated into \texttt{Maude} \cite{Capra:RP22} and based on traditional graph canonization.

In this context, the identification of behavioral equivalences is reduced to a graph \emph{morphism}. PT system morphism must maintain the edges and the marking: In our encoding,
a \emph{morphism} between PT systems \verb|(N m)| and \verb|(N' m')| is a bijection $\phi \ :$ \verb|places(N)| $\rightarrow$ \verb|places(N')| such that, considering the homomorphic extension of $\phi$ on multisets, $\phi($\verb|N|$) = \ $\verb|N'| and $\phi($\verb|m|$) = \ $\verb|m'|.
Moreover, $\phi$ must retain the textual annotations of the place labels and the transition tags. If \verb|N'| = \verb|N| we speak of \emph{automorphism}, in which case $\phi$ is a permutation in the set of places.
We refer to a \emph{normal} form that principally involves identifying sets of automorphic (permutable) places:
Two markings \verb|m|, \verb|m'| of a net \verb|N| are said automorphic if there is an automorphism $\phi$ in \verb|N| that maps \verb|m| into \verb|m'|.
We denote this \verb|m| $\cong$ \verb|m'|. The equivalence relation $\cong$ is a congruence, that is, it preserves the transition firings and \emph{rates}.
%The next definition helps us simplify the process.
\begin{definition}[Symmetric Labeling]
\label{def:modsym}
A \verb|Net| term is symmetrically labeled if any two maximal sets of places whose labels have the same suffix (possibly empty), which is preceded by pairs with the same tag, are permutable. A \verb|System| term is  symmetrically labeled if its \verb|Net| subterm is.
\end{definition}

\noindent In other words, if a \verb|Net| term \verb|N| meets definition \ref{def:modsym}, then for any two maximal subsets of places matching:

$P := \{$\verb|p(L' < w ; i > L)|$\}, \quad P' := \{$\verb|p(L'' < w ; j > L)|$\}$,

where: \verb|L, L', L'' : Plab, w: String, i, j : Nat| 

\noindent there is an automorphism $\phi$ such that $\phi(P) = P'$, $\phi(P') = P$,
%is a  that maps one set to the other,
which is extended as an identity to the rest\footnote{According to the definition of PT morphism, the prefixes \texttt{L'} and \texttt{L''} are consistent in the textual component.}.

If a \verb|System| term adheres to the previous definition, it can be transformed into a 'normal' form by merely swapping indices on the place labels (e.g., i $\leftrightarrow$ j), while still complying with definition \ref{def:modsym}. This normal form is the most minimal according to a lexicographic order within the automorphism class ($\cong$) implicitly defined by \ref{def:modsym}. In contrast to general graph canonization, there is no need for any pruning strategy or backtracking: A monotone procedure is used where the sequence of index swaps does not matter (see \cite{CAPRA-TCS2024}). Efficiency is achieved as the normalized form of the subterm \verb | Net | is derived through basic ``name abstraction``, where at each hierarchical level the indices of structured place labels continuously span from $0$ to $k \in \Nat$.  

The strategy involves providing a concise set of operators that preserve nets' symmetric labelling. This set includes \emph{compositional} operators %(influenced by process algebra)
and operators for \emph{manipulating} nets.
%, such as adding/removing components.
Rewrite rules require these operators to manipulate \verb|System| terms defined in a modular manner. 
Furthermore, rules must adhere to parametricity conditions that limit the use of ground terms \cite{CAPRA-TCS2024}.
%\paragraph{Lumpability}
Under these assumptions, we get a \emph{quotient} TS from a \verb|System|
term that
retains reachability and
meets strong bisimulation. 

Let $t,t',u,u'$ be (final) terms of sort \verb|System|, 
%$\hat{t}$ the normal form of $t$,
$r$ a \verb|System| rule $r : \ s \Rightarrow s'$. The notation $t \overset{r(\sigma)}{\Rightarrow} t'$
%($t$ is rewritten into $t'$ through $r$)
means there is a ground substitution $\sigma$ of $r$'s variables such that $\sigma(s) = t$ and $\sigma(s')= t'$.
%which meets Definitions \ref{def:symr},\ref{prop:symbr}, 
%and $\hat{r} : \ s \Rightarrow normalize(s')$
%(if we refer to terms that meet Definition \ref{def:modsym}, $u \cong t$ is equivalent to $\hat{u} = \hat{t}$).
%If $\sigma$ is a ground substitution of $r$'s variables and $\phi$ is a morphism between $t$ and $t'$ (i.e., a map from the places of $t$ to the places of $t'$), then $\phi(\sigma)$ is the ground assignment induced by $\phi$ (we recall that \verb|Net| and \verb|System| terms are built of \verb|Place| terms). 
\begin{property}
\label{prop:transition-corr}
Let
%rule $r$ meet Definitions~\ref{def:symr},~\ref{prop:symbr} and 
$t$ meet Definition \ref{def:modsym}.
%\begin{itemize}
    %\item[-] If $t \overset{r(\sigma)}{\Rightarrow} t'$ then $ \exists  \sigma' : \hat{t} \overset{\hat{r}(\sigma')}{\Rightarrow} \hat{t}'$ ($t'$ meets Definition \ref{def:modsym})
    %, where $\phi$ is a morphism $t \rightarrow \hat{t}$.
%\item[-] If $\hat{t} \overset{\hat{r}(\sigma)}{\Rightarrow} \hat{t}'$ then $\forall u \cong t$ $\exists \sigma', u' \cong t' : \ u \overset{r(\sigma')}{\Rightarrow} u'$ ($u'$ meets Definition \ref{def:modsym})
%where $\phi$ is a morphism $t  \rightarrow  u$.
If $t \overset{r(\sigma)}{\Rightarrow} t'$, then $\forall u, \phi, t \cong_{\phi} u$: $ u \overset{r(\phi(\sigma))}{\Rightarrow} u'$, $ t' \cong u'$ 
%($u', t'$ meet the definition \ref{def:modsym})
%\end{itemize}
\end{property}

The TS quotient generated by a normal form $\hat{t}$ is obtained by applying the operator \verb|normalize| to the terms on the right side of the rewrite rules. When a \verb|System| undergoes a rewrite due to the \verb|firing| rule, the process only involves the marking subterm.
%This means applying \verb|normalize| (which is overloaded) to the 
%right-hand side of rule 
%subterm \verb | firing (T, M)| 
%in the listing 1.2.
According to property \ref{prop:transition-corr} (firing preservation), because the morphism $\phi$ preserves the transition rates and the rules are parameterized, it is feasible to map the TS quotient of $\hat{t}$ onto an isomorphic "lumped" CTMC: In a Markov process's state space, an equivalence relation is considered "strong lumpability" if the cumulative transition rates between any two states within a class to any other class remain consistent. Despite the possibility of establishing a more stringent condition, namely "exact lumpabability," we focus on the aggregated probability.
%if the cumulative transition rates from states of an equivalence class into each state of another (or the same class) are all equal.


%\noindent \textbf{Example}
%To demonstrate the aforementioned concepts, we will outline a model of a modular Production System with graceful degradation (Section \ref{sec:exe}). Initially, it is composed of $N$ Production Lines (PL) that share raw materials, with each PL split into $K$ interchangeable lines.
%We first define the net transitions and then build a PL using the \verb"repl&share" operator. The term \verb"PL(K)" represents a line with \verb"K" symmetric branches (Figure 1, top). The structure of the submodel is expressed by adding a pair with the tag \verb|"L"| to the place labels. For example, \verb|p(< "w" ; 0 > < "L" ; 1 >)| describes the "working" place of line 1 of the Production Line. We can choose to exclude places to share among replicas: In this case, we exclude those representing the "warehouse" (tag "s") and faults (tag "o"). Additionally, we can indicate transitions to share: For instance, "load" and "assembly" are shared.
%As a result, the incident input/output edges are weight-\verb|K|.

%The term \verb | Net | \verb | NPL(N, K)| consists of \verb|N| PLs, each of which contains \verb|K| branches. This net was generated using the \verb|repl&share| operator, which adds the \verb|"PL"| tag to place labels to indicate an additional nesting level. The sharing mechanism ensures each PL gathers \verb|K| raw pieces. The PT net represented by \verb|NPL(2,2)| can be seen in Figure \ref{fig:symPL-degarde2}, top-right. Furthermore, the term \verb | System | \verb | NPLsys (N, K, M) | is a PT system that holds \verb|K*M| tokens in the 'warehouse' place, with a single token in each place tagged with \verb|"o"| to trigger fault occurrences within a PL.
%We can build an identical model using the "symmetric" version of the process algebra ALT operator.
%The \verb|System| term generated using the above operators possesses symmetrical labeling (refer to definition \ref{def:modsym}), and its \verb|Net| subterm has already been normalized. 
%Consider, e.g., \verb|NPLsys(2, 2, 1)|.
%By triggering the conflicting transitions "load", which are initially enabled, the following two markings (essentially, subterms of the \verb|System| terms) can be obtained:

%\begin{itemize}
%\item[$m_1$] \verb|p(< "o"; 0 > < "PL"; 0 >) + p(< "o"; 0 > < "PL"; 1 >) +|

%\verb|p(< "w"; 0 > < "L"; 0 > < "PL"; 1 >) + p(< "w"; 0 > < "L"; 1 > < "PL"; 1 >)|

%\item[$m_2$] \verb|p(< "o"; 0 > < "PL"; 0 >) + p(< "o"; 0 > < "PL"; 1 >) +| 

%\verb|p(< "w"; 0 > < "L"; 0 > < "PL"; 0 >) + p(< "w"; 0 > < "L"; 1 > < "PL"; 0 >)|.
%\end{itemize}

%These are automorphic (one can be converted into the other by interchanging \verb|< "PL"; 1 >| \(\leftrightarrow\) \verb|< "PL"; 0 >|), but the second marking is the smallest in lexicographic order and hence corresponds to the normalized form.

%A rewrite rule encapsulates the self-adjustment of a PL with $K=2$ in response to a fault, allowing it to function at diminished capacity (Figure \ref{fig:symPL-degarde2}).
%This rule deviates slightly from \cite{CAPRA-TCS2024}, as it is locally activated by a breakdown, leading to a significantly larger TS. Skipping technical details, we point out that the rule meets the parameterity and only employs operators that uphold the definition \ref{def:modsym}, such as \verb | join |, \verb | detach |, \verb | setMark |. 
%A similar rule removes a faulty and degraded PL from the system. These rules are based on net-operators that retain symmetrical labeling (definition \ref{def:modsym}).
%With the model-checking facilities of \texttt{Maude} (command \texttt{search}), it is possible to formally demonstrate that for any given $N$, the quotient transition system has two absorbing states: Every state comprises a deteriorated PL that contains all $2 \cdot M$ parts (unprocessed, except possibly one).
%This is equivalent to the command below, which yields the same results as its unconditioned counterpart.
%\begin{verbatim}
%search NPLsys(N,2,M) =>! F:System such that 
%net(F:System) == faultyPL /\ B:Pbag := marking(F:System) /\ 
%| match(B:Pbag, "w") | + | match(B:Pbag, "a") | == %2 * M  .    
%\end{verbatim}


%%ARRIVATO QUI
 