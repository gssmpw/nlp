\section{Related Work}
% \todo{related work for identical, binary, and ternary valuations.}
	
	Computing fair and efficient allocations has recently emerged as a very
	prominent stream of research in the area of fair allocation of indivisible
	resources. Allocations with maximum Nash welfare are both Pareto optimal and EF1, but computing such allocations is
	NP-hard~\cite{CKMPSW19}. Likewise, computing an allocation with the highest
	utilitarian social welfare among all EF1 allocations is NP-hard even for two
	agents~\cite{AzizHMS23}. As discussed before, the main difference of our model
	is that we allow partial allocations, and consequently we consider envy-freeness
	instead of its relaxation EF1.
	
	
	Allowing partial allocations is an important approach to guarantee the existence
	of EFX allocations (the existence of EFX complete allocations is still an open question).  
	\citet{DBLP:conf/ec/CaragiannisGH19} showed that there always exists an EFX
	partial allocation with at least half of the maximum Nash welfare.
	\citet{DBLP:journals/siamcomp/ChaudhuryKMS21} showed that donating at most $n-1$
	resources can guarantee the existence of EFX allocation such that no agent prefers the
	donated resources to its own bundle, where $n$ denotes the number of agents.
	This bound was latter improved to $n-2$ in general and to $1$ for the case with
	four agents~\cite{BergerCFF22}.
	Besides existence, \citet{Bu22} studied the problem of computing partial
	allocations with the maximum utilitarian welfare among all EFX allocations.
	% since for EFX partial allocations may achieve higher utilitarian welfare than complete allocations, which is not true for EF1.
	Our work differs from this stream of research in that we focus on envy-freeness
	instead of EFX.
	
	\citet{AzizSW16} studied the problem of deleting (or adding) a minimum number of
	resources such that the resulting instance admits an envy-free allocation; which
	is equivalent to finding an envy-free allocation with the maximum size.
	However, they consider ordinal preferences whereas we consider cardinal preferences.
	Moreover, \citet{AzizSW16} considered the number of deleted resources, where the
	problem is NP-hard even if no resource can be deleted. In contrast, we consider
	the dual parameter the lower bound on the allocated resources to identify polynomial-time solvable cases.
	
	\citet{DBLP:journals/aamas/BoehmerBHKL24} studied the problem of transforming a given unfair allocation into an
	EF or EF1 allocation by donating few resources.
	In addition to upper bounds on the number of donated resources and the decrease
	on the utilitarian welfare, they also consider the lower bounds on the remaining
	allocated resources and the remaining utilitarian welfare.
	\citet[Chap 5]{DBLP:journals/algorithmica/DornHS21} studied the same problem but focused
	on a different fairness notion.
	The most prominent difference to our work is that in our model there is no given
	allocation.
	
	\citet{HosseiniSVWX20} introduced a fairness notion where agents can hide some
	of the resources in their own bundles such that no agent is envious assuming
	that the agents do not know the existence of the hidden resources in other agents'
	bundles.  Then the goal is to find a complete allocation and a minimum number of
	hidden resources such that no agent is envious. While the idea is similar to
	find an envy-free partial allocation with the maximum size, note that the hidden
	resources are not deleted; their owners get utility from them just like normal
	resources.
	
	A series of works \cite{GanSV19,BeynierCGHLMW19,KamiyamaMS21} studied the computational complexity of finding an envy-free house allocation when the number of houses is larger than the number of agents.
	This is equivalent to finding an envy-free (partial) allocation that allocates \emph{exactly} one resource to each agent. 
	Our model does not have this kind of upper bound on the number of resources allocated to each agent.
	\citet{aigner2022} studied the problem of finding an envy-free matching of maximum cardinality in a bipartite graph.
	Taking the bipartite graph as the representation of binary utilities of agents on one side towards resources on the other side, the problem studied by \citet{aigner2022} is equivalent to finding an envy-free (partial) allocation with the maximum size such that each agent gets at most one resource liked by it.
	Our model differs from it in that we do not add an upper bound for agents' bundles and we allow agents to receive resources with utility 0.
	Nevertheless, many of our algorithms for binary utilities use the structural
	properties of envy-free matchings %established
	by~\citet{aigner2022}.
	
	
	%%%%%%%%%%%%%%%%%%%%%%%%%%%%%%%%%%%%%%%%%%%%%%%%%%%%%%%%%%%%%%%%%%%%%%%%
	
	\begin{table*}[t]
		\caption{Summary of results. Columns denote different utility constraints and
			efficiency threshold~$t$ values.
			Rows represent different efficiency concepts~$\Efficiency$. The hardness
			results for~$t=1$ apply to every positive~$t$ as well.} \label{tab:results}
		\centering
		% \vskip 1.5em 
		\renewcommand{\arraystretch}{1.25}
		\setlength{\tabcolsep}{12pt}
		\begin{tabular}{rcccc}
			\toprule
			& Identical & \multicolumn{2}{c}{Binary} &	Ternary \\ 
			\cmidrule(lr){3-4}
			& $t=1$  & $t=1$ & $t$ & $t=1$ \\\midrule
			utilitarian social welfare ($\usw$) & \multirow{4}{*}{NP-h} & P & NP-h (FPT) & NP-h \\ 
			egalitarian social welfare ($\esw$) & ~ & P & P & NP-h \\ 
			\#resources allocated ($\size$) & ~ & P & NP-h (FPT) & NP-h  \\ 
			min-cardinality ($\mcar$) & ~ & P & NP-h & NP-h \\ \bottomrule
		\end{tabular}      
	\end{table*}