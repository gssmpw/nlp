\documentclass{article} % For LaTeX2e
%\usepackage{iclr2025, times}
\usepackage{arxiv,times}

%\usepackage{newtxtext} 
\usepackage{microtype}
\usepackage{graphicx}
\usepackage{subcaption}
\usepackage{booktabs}
\PassOptionsToPackage{dvipsnames}{xcolor}
\usepackage[colorlinks=true, linkcolor=BrickRed, urlcolor=DarkGreen, citecolor=DarkGreen, anchorcolor=DarkGreen,backref=page]{hyperref}
\graphicspath{ {../figures/} }
\usepackage{titletoc}
\usepackage{booktabs}
\usepackage{colortbl}
\usepackage{xcolor}
\usepackage{wrapfig}
\newcommand{\theHalgorithm}{\arabic{algorithm}}
\usepackage{amsmath}
\usepackage{amssymb}
\usepackage{mathtools}
\usepackage{amsthm}
\usepackage{natbib}
%\usepackage{macros}
\usepackage{mdframed}
\usepackage{tcolorbox}
\usepackage[capitalize,noabbrev]{cleveref}

\RequirePackage{algorithm}
\RequirePackage{algorithmic}
% Optional math commands from https://github.com/goodfeli/dlbook_notation.
%%%%% NEW MATH DEFINITIONS %%%%%

\usepackage{amsmath,amsfonts,bm}
\usepackage{derivative}
% Mark sections of captions for referring to divisions of figures
\newcommand{\figleft}{{\em (Left)}}
\newcommand{\figcenter}{{\em (Center)}}
\newcommand{\figright}{{\em (Right)}}
\newcommand{\figtop}{{\em (Top)}}
\newcommand{\figbottom}{{\em (Bottom)}}
\newcommand{\captiona}{{\em (a)}}
\newcommand{\captionb}{{\em (b)}}
\newcommand{\captionc}{{\em (c)}}
\newcommand{\captiond}{{\em (d)}}

% Highlight a newly defined term
\newcommand{\newterm}[1]{{\bf #1}}

% Derivative d 
\newcommand{\deriv}{{\mathrm{d}}}

% Figure reference, lower-case.
\def\figref#1{figure~\ref{#1}}
% Figure reference, capital. For start of sentence
\def\Figref#1{Figure~\ref{#1}}
\def\twofigref#1#2{figures \ref{#1} and \ref{#2}}
\def\quadfigref#1#2#3#4{figures \ref{#1}, \ref{#2}, \ref{#3} and \ref{#4}}
% Section reference, lower-case.
\def\secref#1{section~\ref{#1}}
% Section reference, capital.
\def\Secref#1{Section~\ref{#1}}
% Reference to two sections.
\def\twosecrefs#1#2{sections \ref{#1} and \ref{#2}}
% Reference to three sections.
\def\secrefs#1#2#3{sections \ref{#1}, \ref{#2} and \ref{#3}}
% Reference to an equation, lower-case.
\def\eqref#1{equation~\ref{#1}}
% Reference to an equation, upper case
\def\Eqref#1{Equation~\ref{#1}}
% A raw reference to an equation---avoid using if possible
\def\plaineqref#1{\ref{#1}}
% Reference to a chapter, lower-case.
\def\chapref#1{chapter~\ref{#1}}
% Reference to an equation, upper case.
\def\Chapref#1{Chapter~\ref{#1}}
% Reference to a range of chapters
\def\rangechapref#1#2{chapters\ref{#1}--\ref{#2}}
% Reference to an algorithm, lower-case.
\def\algref#1{algorithm~\ref{#1}}
% Reference to an algorithm, upper case.
\def\Algref#1{Algorithm~\ref{#1}}
\def\twoalgref#1#2{algorithms \ref{#1} and \ref{#2}}
\def\Twoalgref#1#2{Algorithms \ref{#1} and \ref{#2}}
% Reference to a part, lower case
\def\partref#1{part~\ref{#1}}
% Reference to a part, upper case
\def\Partref#1{Part~\ref{#1}}
\def\twopartref#1#2{parts \ref{#1} and \ref{#2}}

\def\ceil#1{\lceil #1 \rceil}
\def\floor#1{\lfloor #1 \rfloor}
\def\1{\bm{1}}
\newcommand{\train}{\mathcal{D}}
\newcommand{\valid}{\mathcal{D_{\mathrm{valid}}}}
\newcommand{\test}{\mathcal{D_{\mathrm{test}}}}

\def\eps{{\epsilon}}


% Random variables
\def\reta{{\textnormal{$\eta$}}}
\def\ra{{\textnormal{a}}}
\def\rb{{\textnormal{b}}}
\def\rc{{\textnormal{c}}}
\def\rd{{\textnormal{d}}}
\def\re{{\textnormal{e}}}
\def\rf{{\textnormal{f}}}
\def\rg{{\textnormal{g}}}
\def\rh{{\textnormal{h}}}
\def\ri{{\textnormal{i}}}
\def\rj{{\textnormal{j}}}
\def\rk{{\textnormal{k}}}
\def\rl{{\textnormal{l}}}
% rm is already a command, just don't name any random variables m
\def\rn{{\textnormal{n}}}
\def\ro{{\textnormal{o}}}
\def\rp{{\textnormal{p}}}
\def\rq{{\textnormal{q}}}
\def\rr{{\textnormal{r}}}
\def\rs{{\textnormal{s}}}
\def\rt{{\textnormal{t}}}
\def\ru{{\textnormal{u}}}
\def\rv{{\textnormal{v}}}
\def\rw{{\textnormal{w}}}
\def\rx{{\textnormal{x}}}
\def\ry{{\textnormal{y}}}
\def\rz{{\textnormal{z}}}

% Random vectors
\def\rvepsilon{{\mathbf{\epsilon}}}
\def\rvphi{{\mathbf{\phi}}}
\def\rvtheta{{\mathbf{\theta}}}
\def\rva{{\mathbf{a}}}
\def\rvb{{\mathbf{b}}}
\def\rvc{{\mathbf{c}}}
\def\rvd{{\mathbf{d}}}
\def\rve{{\mathbf{e}}}
\def\rvf{{\mathbf{f}}}
\def\rvg{{\mathbf{g}}}
\def\rvh{{\mathbf{h}}}
\def\rvu{{\mathbf{i}}}
\def\rvj{{\mathbf{j}}}
\def\rvk{{\mathbf{k}}}
\def\rvl{{\mathbf{l}}}
\def\rvm{{\mathbf{m}}}
\def\rvn{{\mathbf{n}}}
\def\rvo{{\mathbf{o}}}
\def\rvp{{\mathbf{p}}}
\def\rvq{{\mathbf{q}}}
\def\rvr{{\mathbf{r}}}
\def\rvs{{\mathbf{s}}}
\def\rvt{{\mathbf{t}}}
\def\rvu{{\mathbf{u}}}
\def\rvv{{\mathbf{v}}}
\def\rvw{{\mathbf{w}}}
\def\rvx{{\mathbf{x}}}
\def\rvy{{\mathbf{y}}}
\def\rvz{{\mathbf{z}}}

% Elements of random vectors
\def\erva{{\textnormal{a}}}
\def\ervb{{\textnormal{b}}}
\def\ervc{{\textnormal{c}}}
\def\ervd{{\textnormal{d}}}
\def\erve{{\textnormal{e}}}
\def\ervf{{\textnormal{f}}}
\def\ervg{{\textnormal{g}}}
\def\ervh{{\textnormal{h}}}
\def\ervi{{\textnormal{i}}}
\def\ervj{{\textnormal{j}}}
\def\ervk{{\textnormal{k}}}
\def\ervl{{\textnormal{l}}}
\def\ervm{{\textnormal{m}}}
\def\ervn{{\textnormal{n}}}
\def\ervo{{\textnormal{o}}}
\def\ervp{{\textnormal{p}}}
\def\ervq{{\textnormal{q}}}
\def\ervr{{\textnormal{r}}}
\def\ervs{{\textnormal{s}}}
\def\ervt{{\textnormal{t}}}
\def\ervu{{\textnormal{u}}}
\def\ervv{{\textnormal{v}}}
\def\ervw{{\textnormal{w}}}
\def\ervx{{\textnormal{x}}}
\def\ervy{{\textnormal{y}}}
\def\ervz{{\textnormal{z}}}

% Random matrices
\def\rmA{{\mathbf{A}}}
\def\rmB{{\mathbf{B}}}
\def\rmC{{\mathbf{C}}}
\def\rmD{{\mathbf{D}}}
\def\rmE{{\mathbf{E}}}
\def\rmF{{\mathbf{F}}}
\def\rmG{{\mathbf{G}}}
\def\rmH{{\mathbf{H}}}
\def\rmI{{\mathbf{I}}}
\def\rmJ{{\mathbf{J}}}
\def\rmK{{\mathbf{K}}}
\def\rmL{{\mathbf{L}}}
\def\rmM{{\mathbf{M}}}
\def\rmN{{\mathbf{N}}}
\def\rmO{{\mathbf{O}}}
\def\rmP{{\mathbf{P}}}
\def\rmQ{{\mathbf{Q}}}
\def\rmR{{\mathbf{R}}}
\def\rmS{{\mathbf{S}}}
\def\rmT{{\mathbf{T}}}
\def\rmU{{\mathbf{U}}}
\def\rmV{{\mathbf{V}}}
\def\rmW{{\mathbf{W}}}
\def\rmX{{\mathbf{X}}}
\def\rmY{{\mathbf{Y}}}
\def\rmZ{{\mathbf{Z}}}

% Elements of random matrices
\def\ermA{{\textnormal{A}}}
\def\ermB{{\textnormal{B}}}
\def\ermC{{\textnormal{C}}}
\def\ermD{{\textnormal{D}}}
\def\ermE{{\textnormal{E}}}
\def\ermF{{\textnormal{F}}}
\def\ermG{{\textnormal{G}}}
\def\ermH{{\textnormal{H}}}
\def\ermI{{\textnormal{I}}}
\def\ermJ{{\textnormal{J}}}
\def\ermK{{\textnormal{K}}}
\def\ermL{{\textnormal{L}}}
\def\ermM{{\textnormal{M}}}
\def\ermN{{\textnormal{N}}}
\def\ermO{{\textnormal{O}}}
\def\ermP{{\textnormal{P}}}
\def\ermQ{{\textnormal{Q}}}
\def\ermR{{\textnormal{R}}}
\def\ermS{{\textnormal{S}}}
\def\ermT{{\textnormal{T}}}
\def\ermU{{\textnormal{U}}}
\def\ermV{{\textnormal{V}}}
\def\ermW{{\textnormal{W}}}
\def\ermX{{\textnormal{X}}}
\def\ermY{{\textnormal{Y}}}
\def\ermZ{{\textnormal{Z}}}

% Vectors
\def\vzero{{\bm{0}}}
\def\vone{{\bm{1}}}
\def\vmu{{\bm{\mu}}}
\def\vtheta{{\bm{\theta}}}
\def\vphi{{\bm{\phi}}}
\def\va{{\bm{a}}}
\def\vb{{\bm{b}}}
\def\vc{{\bm{c}}}
\def\vd{{\bm{d}}}
\def\ve{{\bm{e}}}
\def\vf{{\bm{f}}}
\def\vg{{\bm{g}}}
\def\vh{{\bm{h}}}
\def\vi{{\bm{i}}}
\def\vj{{\bm{j}}}
\def\vk{{\bm{k}}}
\def\vl{{\bm{l}}}
\def\vm{{\bm{m}}}
\def\vn{{\bm{n}}}
\def\vo{{\bm{o}}}
\def\vp{{\bm{p}}}
\def\vq{{\bm{q}}}
\def\vr{{\bm{r}}}
\def\vs{{\bm{s}}}
\def\vt{{\bm{t}}}
\def\vu{{\bm{u}}}
\def\vv{{\bm{v}}}
\def\vw{{\bm{w}}}
\def\vx{{\bm{x}}}
\def\vy{{\bm{y}}}
\def\vz{{\bm{z}}}

% Elements of vectors
\def\evalpha{{\alpha}}
\def\evbeta{{\beta}}
\def\evepsilon{{\epsilon}}
\def\evlambda{{\lambda}}
\def\evomega{{\omega}}
\def\evmu{{\mu}}
\def\evpsi{{\psi}}
\def\evsigma{{\sigma}}
\def\evtheta{{\theta}}
\def\eva{{a}}
\def\evb{{b}}
\def\evc{{c}}
\def\evd{{d}}
\def\eve{{e}}
\def\evf{{f}}
\def\evg{{g}}
\def\evh{{h}}
\def\evi{{i}}
\def\evj{{j}}
\def\evk{{k}}
\def\evl{{l}}
\def\evm{{m}}
\def\evn{{n}}
\def\evo{{o}}
\def\evp{{p}}
\def\evq{{q}}
\def\evr{{r}}
\def\evs{{s}}
\def\evt{{t}}
\def\evu{{u}}
\def\evv{{v}}
\def\evw{{w}}
\def\evx{{x}}
\def\evy{{y}}
\def\evz{{z}}

% Matrix
\def\mA{{\bm{A}}}
\def\mB{{\bm{B}}}
\def\mC{{\bm{C}}}
\def\mD{{\bm{D}}}
\def\mE{{\bm{E}}}
\def\mF{{\bm{F}}}
\def\mG{{\bm{G}}}
\def\mH{{\bm{H}}}
\def\mI{{\bm{I}}}
\def\mJ{{\bm{J}}}
\def\mK{{\bm{K}}}
\def\mL{{\bm{L}}}
\def\mM{{\bm{M}}}
\def\mN{{\bm{N}}}
\def\mO{{\bm{O}}}
\def\mP{{\bm{P}}}
\def\mQ{{\bm{Q}}}
\def\mR{{\bm{R}}}
\def\mS{{\bm{S}}}
\def\mT{{\bm{T}}}
\def\mU{{\bm{U}}}
\def\mV{{\bm{V}}}
\def\mW{{\bm{W}}}
\def\mX{{\bm{X}}}
\def\mY{{\bm{Y}}}
\def\mZ{{\bm{Z}}}
\def\mBeta{{\bm{\beta}}}
\def\mPhi{{\bm{\Phi}}}
\def\mLambda{{\bm{\Lambda}}}
\def\mSigma{{\bm{\Sigma}}}

% Tensor
\DeclareMathAlphabet{\mathsfit}{\encodingdefault}{\sfdefault}{m}{sl}
\SetMathAlphabet{\mathsfit}{bold}{\encodingdefault}{\sfdefault}{bx}{n}
\newcommand{\tens}[1]{\bm{\mathsfit{#1}}}
\def\tA{{\tens{A}}}
\def\tB{{\tens{B}}}
\def\tC{{\tens{C}}}
\def\tD{{\tens{D}}}
\def\tE{{\tens{E}}}
\def\tF{{\tens{F}}}
\def\tG{{\tens{G}}}
\def\tH{{\tens{H}}}
\def\tI{{\tens{I}}}
\def\tJ{{\tens{J}}}
\def\tK{{\tens{K}}}
\def\tL{{\tens{L}}}
\def\tM{{\tens{M}}}
\def\tN{{\tens{N}}}
\def\tO{{\tens{O}}}
\def\tP{{\tens{P}}}
\def\tQ{{\tens{Q}}}
\def\tR{{\tens{R}}}
\def\tS{{\tens{S}}}
\def\tT{{\tens{T}}}
\def\tU{{\tens{U}}}
\def\tV{{\tens{V}}}
\def\tW{{\tens{W}}}
\def\tX{{\tens{X}}}
\def\tY{{\tens{Y}}}
\def\tZ{{\tens{Z}}}


% Graph
\def\gA{{\mathcal{A}}}
\def\gB{{\mathcal{B}}}
\def\gC{{\mathcal{C}}}
\def\gD{{\mathcal{D}}}
\def\gE{{\mathcal{E}}}
\def\gF{{\mathcal{F}}}
\def\gG{{\mathcal{G}}}
\def\gH{{\mathcal{H}}}
\def\gI{{\mathcal{I}}}
\def\gJ{{\mathcal{J}}}
\def\gK{{\mathcal{K}}}
\def\gL{{\mathcal{L}}}
\def\gM{{\mathcal{M}}}
\def\gN{{\mathcal{N}}}
\def\gO{{\mathcal{O}}}
\def\gP{{\mathcal{P}}}
\def\gQ{{\mathcal{Q}}}
\def\gR{{\mathcal{R}}}
\def\gS{{\mathcal{S}}}
\def\gT{{\mathcal{T}}}
\def\gU{{\mathcal{U}}}
\def\gV{{\mathcal{V}}}
\def\gW{{\mathcal{W}}}
\def\gX{{\mathcal{X}}}
\def\gY{{\mathcal{Y}}}
\def\gZ{{\mathcal{Z}}}

% Sets
\def\sA{{\mathbb{A}}}
\def\sB{{\mathbb{B}}}
\def\sC{{\mathbb{C}}}
\def\sD{{\mathbb{D}}}
% Don't use a set called E, because this would be the same as our symbol
% for expectation.
\def\sF{{\mathbb{F}}}
\def\sG{{\mathbb{G}}}
\def\sH{{\mathbb{H}}}
\def\sI{{\mathbb{I}}}
\def\sJ{{\mathbb{J}}}
\def\sK{{\mathbb{K}}}
\def\sL{{\mathbb{L}}}
\def\sM{{\mathbb{M}}}
\def\sN{{\mathbb{N}}}
\def\sO{{\mathbb{O}}}
\def\sP{{\mathbb{P}}}
\def\sQ{{\mathbb{Q}}}
\def\sR{{\mathbb{R}}}
\def\sS{{\mathbb{S}}}
\def\sT{{\mathbb{T}}}
\def\sU{{\mathbb{U}}}
\def\sV{{\mathbb{V}}}
\def\sW{{\mathbb{W}}}
\def\sX{{\mathbb{X}}}
\def\sY{{\mathbb{Y}}}
\def\sZ{{\mathbb{Z}}}

% Entries of a matrix
\def\emLambda{{\Lambda}}
\def\emA{{A}}
\def\emB{{B}}
\def\emC{{C}}
\def\emD{{D}}
\def\emE{{E}}
\def\emF{{F}}
\def\emG{{G}}
\def\emH{{H}}
\def\emI{{I}}
\def\emJ{{J}}
\def\emK{{K}}
\def\emL{{L}}
\def\emM{{M}}
\def\emN{{N}}
\def\emO{{O}}
\def\emP{{P}}
\def\emQ{{Q}}
\def\emR{{R}}
\def\emS{{S}}
\def\emT{{T}}
\def\emU{{U}}
\def\emV{{V}}
\def\emW{{W}}
\def\emX{{X}}
\def\emY{{Y}}
\def\emZ{{Z}}
\def\emSigma{{\Sigma}}

% entries of a tensor
% Same font as tensor, without \bm wrapper
\newcommand{\etens}[1]{\mathsfit{#1}}
\def\etLambda{{\etens{\Lambda}}}
\def\etA{{\etens{A}}}
\def\etB{{\etens{B}}}
\def\etC{{\etens{C}}}
\def\etD{{\etens{D}}}
\def\etE{{\etens{E}}}
\def\etF{{\etens{F}}}
\def\etG{{\etens{G}}}
\def\etH{{\etens{H}}}
\def\etI{{\etens{I}}}
\def\etJ{{\etens{J}}}
\def\etK{{\etens{K}}}
\def\etL{{\etens{L}}}
\def\etM{{\etens{M}}}
\def\etN{{\etens{N}}}
\def\etO{{\etens{O}}}
\def\etP{{\etens{P}}}
\def\etQ{{\etens{Q}}}
\def\etR{{\etens{R}}}
\def\etS{{\etens{S}}}
\def\etT{{\etens{T}}}
\def\etU{{\etens{U}}}
\def\etV{{\etens{V}}}
\def\etW{{\etens{W}}}
\def\etX{{\etens{X}}}
\def\etY{{\etens{Y}}}
\def\etZ{{\etens{Z}}}

% The true underlying data generating distribution
\newcommand{\pdata}{p_{\rm{data}}}
\newcommand{\ptarget}{p_{\rm{target}}}
\newcommand{\pprior}{p_{\rm{prior}}}
\newcommand{\pbase}{p_{\rm{base}}}
\newcommand{\pref}{p_{\rm{ref}}}

% The empirical distribution defined by the training set
\newcommand{\ptrain}{\hat{p}_{\rm{data}}}
\newcommand{\Ptrain}{\hat{P}_{\rm{data}}}
% The model distribution
\newcommand{\pmodel}{p_{\rm{model}}}
\newcommand{\Pmodel}{P_{\rm{model}}}
\newcommand{\ptildemodel}{\tilde{p}_{\rm{model}}}
% Stochastic autoencoder distributions
\newcommand{\pencode}{p_{\rm{encoder}}}
\newcommand{\pdecode}{p_{\rm{decoder}}}
\newcommand{\precons}{p_{\rm{reconstruct}}}

\newcommand{\laplace}{\mathrm{Laplace}} % Laplace distribution

\newcommand{\E}{\mathbb{E}}
\newcommand{\Ls}{\mathcal{L}}
\newcommand{\R}{\mathbb{R}}
\newcommand{\emp}{\tilde{p}}
\newcommand{\lr}{\alpha}
\newcommand{\reg}{\lambda}
\newcommand{\rect}{\mathrm{rectifier}}
\newcommand{\softmax}{\mathrm{softmax}}
\newcommand{\sigmoid}{\sigma}
\newcommand{\softplus}{\zeta}
\newcommand{\KL}{D_{\mathrm{KL}}}
\newcommand{\Var}{\mathrm{Var}}
\newcommand{\standarderror}{\mathrm{SE}}
\newcommand{\Cov}{\mathrm{Cov}}
% Wolfram Mathworld says $L^2$ is for function spaces and $\ell^2$ is for vectors
% But then they seem to use $L^2$ for vectors throughout the site, and so does
% wikipedia.
\newcommand{\normlzero}{L^0}
\newcommand{\normlone}{L^1}
\newcommand{\normltwo}{L^2}
\newcommand{\normlp}{L^p}
\newcommand{\normmax}{L^\infty}

\newcommand{\parents}{Pa} % See usage in notation.tex. Chosen to match Daphne's book.

\DeclareMathOperator*{\argmax}{arg\,max}
\DeclareMathOperator*{\argmin}{arg\,min}

\DeclareMathOperator{\sign}{sign}
\DeclareMathOperator{\Tr}{Tr}
\let\ab\allowbreak


\newtheorem{theorem}{Theorem}
\newtheorem{lemma}{Lemma}
\newtheorem*{theorem*}{Theorem}
\newtheorem{example}{Example}
\numberwithin{equation}{section}
\numberwithin{theorem}{section}



% Define colors
\definecolor{green}{HTML}{17891a}
\definecolor{DarkGreen}{HTML}{054802}
\definecolor{SmokeBlue}{HTML}{2F5E90}
\definecolor{purple}{HTML}{800080}
\definecolor{blue}{HTML}{0000FF}   
\definecolor{red}{HTML}{FF0000}    
\definecolor{orange}{HTML}{FFA500}
\definecolor{gray}{HTML}{808080}  
\definecolor{pastelcyan}{rgb}{0.5, 0.8, 0.8}
\definecolor{pastelyellow}{rgb}{0.9, 0.8, 0.6}



\newtcolorbox{empheqboxedblue}{colback=pastelcyan, 
 colframe=white,
 width=\linewidth,
 sharpish corners,
 top=1mm, % default value 2mm
 bottom=0pt,
 left=2pt,
 right=2pt
}
\newtcolorbox{empheqboxedyellow}{colback=pastelyellow, 
 colframe=white,
 width=\linewidth,
 sharpish corners,
 top=1mm, % default value 2mm
 bottom=0pt,
 left=2pt,
 right=2pt
}
\newtcolorbox{highlight}{colback=gray!20, 
 colframe=white,
 width=\linewidth,
 sharpish corners,
 top=1mm, % default value 2mm
 bottom=0pt,
 left=2pt,
 right=2pt
}
\newmdenv[
  leftline=true,
  topline=false,
  bottomline=false,
  rightline=false,
  linewidth=2pt,
  linecolor=darkgray,
  skipabove=\baselineskip,
  skipbelow=\baselineskip % Adjust spacing above
]{theorembox}


\pagestyle{fancy}
\fancyhf{}
\fancyhead[C]{Causal Lifting of Neural Representations: Zero-Shot Generalization for Causal Inferences}
\fancyfoot[C]{\thepage}

\title{Causal Lifting of Neural Representations: \\ 
Zero-Shot Generalization for Causal Inferences}

% Authors must not appear in the submitted version. They should be hidden
% as long as the \iclrfinalcopy macro remains commented out below.
% Non-anonymous submissions will be rejected without review.

\author{
    Riccardo Cadei$^1$,  
    Ilker Demirel$^2$\thanks{Equal contribution.},  
    Piersilvio De Bartolomeis$^3$\footnotemark[1],  
    Lukas Lindorfer$^1$,  
    \\
    \textbf{Sylvia Cremer}$^1$,  
    \textbf{Cordelia Schmid}$^4$,  
    \textbf{Francesco Locatello}$^1$ \\
    \\
    $^1$Institute of Science and Technology Austria (ISTA) \\
    $^2$Massachusetts Institute of Technology (MIT) \\
    $^3$Department of Computer Science, ETH Zurich \\
    $^4$INRIA, Ecole Normale Supérieure, CNRS, PSL Research University \\
    %\texttt{riccardo.cadei@ist.ac.at}
}


% \author{
%     Riccardo Cadei \\
%     Institute of Science and Technology, Austria (ISTA) \\
%     \texttt{riccardo.cadei@ist.ac.at} \\
%     \And
%     Ilker Demirel\thanks{Equal contribution.} \\
%     Massachusetts Institute of Technology (MIT) \\
%     \And
%     Piersilvio De Bartolomeis\footnotemark[1] \\
%     Department of Computer Science, ETH Zurich \\
%     \And
%     Lukas Lindorfer \\
%     Institute of Science and Technology, Austria (ISTA) \\
%     \And
%     Sylvia Cremer \\
%     Institute of Science and Technology, Austria (ISTA) \\
%     \And
%     Cordelia Schmid \\
%     INRIA, Ecole Normale Supérieure, CNRS, PSL Research University \\
%     \And
%     Francesco Locatello \\
%     Institute of Science and Technology, Austria (ISTA) \\
% }


% The \author macro works with any number of authors. There are two commands
% used to separate the names and addresses of multiple authors: \And and \AND.
%
% Using \And between authors leaves it to \LaTeX{} to determine where to break
% the lines. Using \AND forces a linebreak at that point. So, if \LaTeX{}
% puts 3 of 4 authors names on the first line, and the last on the second
% line, try using \AND instead of \And before the third author name.

\newcommand{\fix}{\marginpar{FIX}}
\newcommand{\new}{\marginpar{NEW}}

%\iclrfinalcopy % Uncomment for camera-ready version, but NOT for submission.
\begin{document}


\maketitle

\begin{abstract}
A plethora of real-world scientific investigations is waiting to scale with the support of trustworthy predictive models that can reduce the need for costly data annotations.  We focus on causal inferences on a target experiment with unlabeled factual outcomes, retrieved by a predictive model fine-tuned on a labeled \textit{similar} experiment. 
First, we show that factual outcome estimation via Empirical Risk Minimization (ERM) may fail to yield valid causal inferences on the target population, even in a randomized controlled experiment and infinite training samples. Then, we propose to leverage the observed experimental settings during training to empower generalization to downstream interventional investigations, ``\textit{Causal Lifting}'' the predictive model. We propose \textit{Deconfounded Empirical Risk Minimization} (DERM), a new simple learning procedure minimizing the risk over a fictitious target population, preventing potential confounding effects. We validate our method on both synthetic and real-world scientific data. Notably, for the first time, we zero-shot generalize causal inferences on ISTAnt dataset (without annotation) by causal lifting a predictive model on our experiment variant.
\end{abstract}

\section{Introduction}
Artificial Intelligence (AI) systems hold great promise for accelerating scientific discovery by providing flexible models capable of automating complex tasks. We already depend on deep learning predictions across various applications, including biology \citep{jumper2021highly, tunyasuvunakool2021highly, elmarakeby2021biologically, mullowney2023artificial}, sustainability \citep{castello2021quantification}, and the social sciences \citep{jerzak2022image, daoud2023using}.

While these models offer transformative potential for scientific research, their black-box nature poses new challenges. They can perpetuate hidden biases, which are difficult to detect and quantify, and risk invalidating conclusions drawn from their predictions for downstream experiments. 
Recent efforts have focused on combining capable black-box models with partially annotated data to power valid and efficient statistical inference  \citep{angelopoulos2023prediction,angelopoulos2023ppi++}. Drawing inspiration from there, we focus on enabling \textit{causal inference} on unlabeled experimental data via \textit{factual} predictions, developing methods that can leverage powerful AI models reliably in that endeavor, i.e., Prediction-Powered Causal Inference (PPCI). A key challenge in this setting is that small modeling biases can invalidate the causal conclusions, even in the simplest possible scenario, where the downstream experiment is a randomized controlled trial \citep{cadei2024smoke}. Secondly, we aim to retrieve the annotations even out-of-distribution, allowing for zero-shot generalization.
Yet, manual annotation of scientific experiments is costly, requiring experts to identify subtle signals, e.g., analyzing hours of videos to detect behavioral markers in experimental ecology. Automating the annotation process with machine learning models without any further training can alleviate this burden completely, tremendously accelerating the full pipeline. 

At the same time, in scientific applications, experimentalists often collect data through multiple experiments with similar designs, e.g., investigating the effect on the same outcome of interest under different treatment or environmental settings. While historical experiments may yield too little data to train a performant model from scratch, one can fine-tune a pre-trained foundational model to learn the patterns needed for annotating experiments. Despite being a promising direction, a critical hurdle to generalize across experiments without introducing bias remains. Fine-tuning foundational models is typically done via Empirical Risk Minimization (ERM), which tends to exploit any \textit{statistical association} in the training data to minimize prediction error. Therefore, one risks leveraging spurious associations between experiment-specific factors (e.g., equipment artifacts) and outcomes, leading to systematic prediction errors on the target experiment. 
To address the problem, we propose ``causal lifting'' such foundation models from potential confounding effects, suppressing the application-specific spurious correlations during fine-tuning.

We first discuss the challenges and feasibility of the problem, and in agreement with \citet{yao2024unifying} we show how the supervised objective has to be paired with a conditional independence constraint enforcing the model to not rely on spurious correlations in its class of experiments. We then propose a simple and tailored implementation for such constraint via a resampling approach, reweighting the samples in the empirical risk, i.e., Deconfounded Empirical Risk Minimization (DERM). 
We validate the full pipeline for Causal Lifting on both synthetic and real-world data. Notably, we leverage a new experiment (ours) \textit{similar} to ISTAnt \citep{cadei2024smoke} yet differing in several experimental and technical details, including lower-quality light conditions and diverse treatments.  For the first time, our method enables a foundational model to retrieve valid Causal Inference on ISTAnt dataset without annotation, i.e., 0-shot generalization of causal inferences on a completely unlabelled experiment.

In broader terms, this paper emphasizes the ``representation learning'' aspect of ``causal representation learning'', which has traditionally focused on identification. In~\citet{bengio2013representation}, good representations are defined as ones ``\textit{that make it easier to extract useful information when building classifiers or other predictors}.'' In a similar spirit, we focus on representations that make extracting causal information easier or at all possible with some downstream estimator. As we shall demonstrate, guaranteeing identification of the causal effect is not always possible depending on the distributional differences between the experiments and our modeling choices. Yet, we hope that our viewpoint can also offer benchmarking opportunities that are currently missing in the causal representation learning literature \citep{scholkopf2021toward} and have great potential, especially in the context of scientific discoveries.

Overall, our contributions are: 
\begin{enumerate}%[leftmargin=*]
    \item[i.] a \textbf{new problem} formulation, i.e., PPCI, reshaping the definition of Causal Representation Learning as Representation Learning for Causal Downstream Tasks beyond untestable identifiability results and enabling quantitative benchmarking,
    \item[ii.] a \textbf{new method}, i.e., DERM, for \textit{Causal Lifting} of foundational models unconfounding their representations from spurious correlations between the perceived experiment settings and the outcome of interest,
    \item[iii.] \textbf{first} valid and efficient \textbf{0-shot generalization} for PPCI on ISTAnt, \textit{Causal Lifting} DINOv2 on our lower-quality experiment.%, which \textbf{data}, i.e., recordings and annotations, we plan to \textbf{release} publicly upon acceptance (preview: \href{https://figshare.com/s/9a490b6f6eeebd73350b}{https://figshare.com/s/9a490b6f6eeebd73350b}).
\end{enumerate}



\section{Problem Formulation}
\label{sec:problem}

Let $\mathcal{E}$ a countable index set, and consider a class of Structural Causal Models (SCM) $\mathfrak{S}:=\{\mathcal{M}^e\}_{e \in \mathcal{E}}$, characterized by the following (universal) Structural Equations: 
\begin{equation}
    \left\{ 
\begin{aligned}
    \bm{Z} &:= n_{\bm{Z}} \\
    Y &:= f_Y(\bm{Z},n_Y) \\
    \bm{X} &:= f_{\bm{X}}(\bm{Z},Y,n_{\bm{X}})
\end{aligned}   
\right.
\end{equation}
and varying the exogenous variables distribution\footnote{Note that $\bm{Z}, Y$ and $\bm{X}$ distributions all depend on the environment $e$, but we omit the reference for simplicity of language by always explicit the considered distribution.}:
\begin{equation}
    n_\textbf{Z}, n_Y, n_\textbf{X} \sim  \mathbb{P}^e.
\end{equation}
We further assume that there exists a model $g^*$ retrieving $Y$ from $\bm{X}$ almost surely for the whole class, i.e.,
\begin{equation}
\label{eq:determinism}
    \exists g^*: \mathbb{P}^e(Y=g^*(\bm{X}))=1 \quad \forall e \in \mathcal{E}.
\end{equation}
Many variants of real-world experiments can be modeled via such a class of SCM, where:
\begin{itemize}
    \item $\bm{Z}$ is the universal set of (possible) experimental settings potentially affecting the outcome, i.e., all the ancestors of $\bm{X}$ and $Y$ (excluding noise),
    \item $Y$ is the outcome of interest,
    \item $\bm{X}$ is a fully informative high-dimensional observation of the experiment (e.g., video or text description), which, without machine learning, is analyzed by hand by human experts, also relying on the existence of an invariant model $g^*$.
\end{itemize}
This framework is particularly suitable for Causal Inference applications\footnote{We focus here only on \textit{causality in mean} \citep{pearl2018book}, ignoring counterfactual reasoning.}, where the outcome of interest is commonly not observed directly but extracted from a high-dimensional observation, and some experiment settings are naturally collected and potentially controlled.
In the following, we use $\bm{Z}^{e}$ to refer to the experiment settings actually observed in experiment $\mathcal{M}^e$, and $\bm{U}^e$ for the unobserved. We can further distinguish, within  $\bm{Z}^{e}$, between a treatment variable $T^e$ and observed pre-treatment variables $\bm{W}^e$. 
All together:
\begin{equation}
    \bm{Z} = {\underbrace{T^e \cup \bm{W}^e}_{\bm{Z}^e}} \cup \bm{U}^e \quad \forall e \in \mathcal{E}.
\end{equation}
Note that which variables $\bm{Z}^{e}\subseteq \bm{Z}$ are observed may change across experiments, in particular, the treatment of interest $T^e$ and the observed pre-treatment variables, $\bm{W}^e$, together with their distributions.

When a new experiment is performed, we collect observations $\bm{X}$ (and experimental conditions $T^{e}, \bm{W}^{e}$), from which the outcome $Y$ can be extracted. Instead of annotating $Y$ by hand for every new experiment, we wonder when we could leverage similar experiments, i.e., in the same class, to train or fine-tune a machine-learning system capable of supplying accurate predictions about the outcome of interest and obtain trustworthy confidence interval on a causal downstream task.
We refer to this problem as (\textit{factual}) Prediction-Powered Causal Inference (PPCI). In summary:
\begin{highlight}
    \centerline{
    \textbf{Prediction-Powered Causal Inference}}
    \vspace{0.3cm}
    \textbf{Sources: \ }
    \setlength{\leftmargini}{18pt}
    \begin{itemize}
        \item A random sample  $\mathcal{D}^{e_1}=\{(T^{e_1}_i, \bm{W}^{e_1}_i, Y_i, \bm{X}_i)\}_{i=1}^{n^{e_1}}$ 
        from a reference experiment $\mathcal{M}^{e_1} \in \mathfrak{S}$,
        \item A random sample  $\mathcal{D}^{e_2}=\{(T^{e_2}_i, \bm{W}^{e_2}_i, \_, \bm{X}_i)\}_{i=1}^{n^{e_2}}$ from a target experiment $\mathcal{M}^{e_2} \in \mathfrak{S}$, not observing the factual outcome of interest\footnote{The observed experiment settings are not necessarily shared between experiments.}.
    \end{itemize}
    \textbf{Assumption: \ } Existence of an invariant factual outcome model from the raw observations $\bm{X}$, i.e., 
    \begin{equation}
    \label{eq:determinismppci}
        \exists g^*: \mathbb{P}^e(Y=g^*(\bm{X}))=1 \quad \forall e \in \{e_1,e_2\}. 
    \end{equation}
    \textbf{Task: \ } Learn a factual outcome model estimator $\hat{g}$ conditionally unbiased on the target population, i.e,
    \begin{equation}
    \label{eq:unbiased}
        \mathbb{E}_{\mathbb{P}_{e_2}}[Y-\hat{g}(\bm{X}) |\bm{Z}]\overset{\text{a.s.}}{=} 0,
    \end{equation}
    enabling different downstream causal inferences.
\end{highlight}
Figure \ref{fig:causalmodels} illustrates the reference and target experiment using their causal models.
 \begin{figure}[h!] % Add [t] or [h] for float position control
    \centering
    \begin{subfigure}[b]{0.49\linewidth}
        \centering
        \includegraphics[width=0.7\linewidth]{figures/RE.png} 
        \caption{\footnotesize{Reference Experiment ($\mathcal{M}^{e_1}$)}}
        \label{fig:RE}
    \end{subfigure}
    %\hfill
    \begin{subfigure}[b]{0.49\linewidth}
        \centering
        \includegraphics[width=0.7\linewidth]{figures/TE.png} 
        \caption{\footnotesize{Target Experiment ($\mathcal{M}^{e_2}$)}}
        \label{fig:TE}
    \end{subfigure}
    \caption{Causal Model visualization of a Reference and Target Experiment from the same SCM class $\mathfrak{S}$. The observed variables are in light gray, and the unobserved in white.}
    \label{fig:causalmodels}
\end{figure}
Condition \ref{eq:unbiased} effectively means that the factual outcome estimator, $\hat{g}$, is unbiased under \textit{any} experimental setting $Z$ that can be observed in the \textit{target} distribution, ${\mathbb{P}}_{e_2}$. Once we have a factual outcome estimator that satisfies Condition~\ref{eq:unbiased}, we can use it to impute the missing outcome on the target sample and then estimate, e.g., Average Treatment Effect (ATE) via AIPW estimator~\citep{robins1994estimation,robins1995semiparametric}. As Theorem \ref{th:feasibility} formalizes, it ensures (asymptotically) valid confidence intervals---a key requirement for scientific research---on the ATE without any factual outcome observations (assuming the causal effect is identifiable). 

\begin{theorembox}
\begin{theorem}[Informal]
\label{th:feasibility}
    Given a PPCI problem and a factual outcome model $g$ conditionally unbiased on the target population, i.e., satisfying Eq. \ref{eq:unbiased}. Assume that the ATE would be identifiable in the target experiment with ground-truth labels of the effect. Then,  the AIPW estimator over the prediction-powered target sample provides an asymptotically valid confidence interval for the ATE. 
\end{theorem}
\end{theorembox}

See the formal proposition and proof in Appendix \ref{sec:proofs}.  Analogous results hold for interventional causal inferences on continuous treatment and heterogenous effect estimation, i.e., CATE estimation.


\subsection{Zero-Shot Generalization}
\label{ssec:challenges}

There are generally no guarantees for Condition \ref{eq:unbiased} to hold while training $\hat{g}$ on the reference experiment. Indeed,  due to interventions to the experimental settings $\bm{Z}$, and being $\bm{Z} \rightarrow \bm{X}$, also the high-dimensional observation $\bm{X}$ may shift out of support on target, leaving the factual outcome model not \textit{identifiable} even in the infinite sample setting. Foundational models pre-trained on extended corpus offer a promising solution to the issue, practically enabling to consistently extract all the useful information hidden in $\bm{X}$ to predict $Y$ (but not only).
This section discusses potential distribution shift issues in infinite and finite sample settings, motivating why standalone Empirical Risk Minimization (ERM) cannot mitigate any of these challenges.
Indeed, even if we assume access to an oracle encoder $\phi^*(\bm{X})=\begin{bmatrix}
\phi_{Y}(\bm{X}) \\
\psi(\bm{X})
\end{bmatrix}$, extracting from $\bm{X}$, and among other features:
\begin{itemize}
    \item all the information of $Y$, i.e., $H_{\mathbb{P}^{e}}[Y|\phi_{Y}(\bm{X})]=0$,
    \item disentangled from all the experiment setting  $\bm{Z}$, i.e., $I_{\mathbb{P}^{e}}(\phi_{Y}(\bm{X}), \bm{Z}|Y)=0$,
\end{itemize}
for all possible experiments $e \in \mathcal{E}$; we may still have trouble in learning a factual outcome classifier $h$ on top by ERM, 
since it could still perfectly minimize the empirical risk, but rely on spurious correlations between some experimental settings, e.g., retrievable from $\psi(\bm{X})$, and the outcome of interest.

\subsubsection{Issues in Infinite-Sample}
If a specific instance of observed experiment settings $\bm{z}$ is fully informative of the outcome on the reference population, e.g., $\text{Var}(Y|\bm{Z}^{e_1}=\bm{z})=0$, while varying on target for distribution shifts of the unobserved ones, standole ERM has no criteria to privilege an invariant solution to one relying on the retrieved experiment setting spurious correlations. More generally, it is enough that the outcome support is not full on the reference experiment, conditioning on some experiment settings, that ERM may privilege a model overfitting on such spurious correlation, \textit{stereotyping}.

\begin{example} \textit{Consider a hypothetical behavior classification task from videos where two different treatments with the same appearance are considered, respectively $T^{e_1}$ on the reference experiment and $T^{e_2}$ on the target experiment, e.g., two observable micro-particle applications on an ant with the same appearance as in the illustration in Figure \ref{fig:t1}\footnote{In ISTAnt dataset such effect is not applicable since the considered treatments are not visually distinguishable, but the discussion still apply to several other experiments also if two experimental settings have the same appearance but different effect on the outcome of interest, e.g., artificial light and sun light.}. Let's assume that in the reference experiment, a certain behavior $y$ is not happening if $T^{e_1}=1$, despite it being observed if $T^{e_1}=0$. A ``confounded'' model may retrieve $T^{e_1}$ appearance and simplify the classification when $T^{e_1}=1$. Such a short-cut may not hold at test time since $T^{e_2}$ has a different relation with the outcome of interest but looks like $T^{e_1}$ to the model. See Figure \ref{fig:e1re}-\ref{fig:e1te} for visualizing the reference and target causal models.
}
\end{example}

\begin{figure}[ht!]
    \centering
    % First subfigure
    \begin{subfigure}[b]{0.32\linewidth}
        \includegraphics[width=0.9\textwidth]{figures/ant.jpg} 
        \caption{Treatment}
        \label{fig:t1}
    \end{subfigure}
    \hfill
    % Second subfigure
    \begin{subfigure}[b]{0.32\linewidth}
        \includegraphics[width=0.9\textwidth]{figures/E1RE.png} 
        \caption{Reference}
        \label{fig:e1re}
    \end{subfigure}
    \hfill
    % Third subfigure
    \begin{subfigure}[b]{0.32\linewidth}
        \includegraphics[width=0.9\textwidth]{figures/E1TE.png} 
        \caption{Target}
        \label{fig:e1te}
    \end{subfigure}
    \caption{Illustration of a confounding effect due to different treatments of interest with the same appearance, e.g., a liquid drop, affecting ERM even in the infinite-sample regime.}
    \label{fig:example1}
\end{figure}

\subsubsection{Issues in Finite-Sample} 
In real-world applications, similar issues are due to weak overlap between the conditional outcome on the experimental settings and outcome distribution, still allowing a candidate model to leverage spurious correlations, even if not perfectly solving the task.

\begin{example}
    \textit{Consider a behavior classification task from videos with a few possible backgrounds considered and repeated varying other experimental settings. In ISTAnt dataset, for example, videos were recorded with nine possible backgrounds recognizable by some pen lines. See Figure \ref{fig:position} to visualize a batch example. In a too-small reference sample, a certain behavior $y$ may rarely appear in a certain position $p$, i.e., $\mathbb{P}_{e_1}(Y=y|P=p)\ll1$. In Figure \ref{fig:E2CM}, a visualization of the true causal model, and in Figure \ref{fig:E2PCM}, an intuitive visualization of the misleading causal model perceived in finite-sample.
    Although a human annotator is naturally agnostic to such position information for behavior classification, a model trained to minimize the empirical risk over such a sample may blindly rely on such spurious correlation and fail in generalization, avoiding predicting such behavior for that position. 
    }
\end{example}

\begin{figure}[h!]
    \centering
    % First subfigure
    \begin{subfigure}[b]{0.32\linewidth}
    \centering
        \includegraphics[width=0.74\textwidth]{figures/batch.png} 
        \caption{Position}
        \label{fig:position}
    \end{subfigure}
    \hfill
    % Second subfigure
    \begin{subfigure}[b]{0.32\linewidth}
        \includegraphics[width=0.9\textwidth]{figures/E2CM.png} 
        \caption{Causal Model}
        \label{fig:E2CM}
    \end{subfigure}
    \hfill
    % Third subfigure
    \begin{subfigure}[b]{0.32\linewidth}
        \includegraphics[width=0.9\textwidth]{figures/E2PCM.png} 
        \caption{Perceived Model}
        \label{fig:E2PCM}
    \end{subfigure}
    \caption{Illustration of a confounding effect due to positivity assumption violation in the finite-sample setting.}
    \label{fig:example2}
\end{figure}



\section{Deconfounded Empirical Risk Minimization (DERM)}
\label{sec:method}
As motivated in Section \ref{sec:problem}, directly minimizing the empirical risk of an expressive head on top of a (pre-trained) encoder may not be sufficient in generalization for PPCI, even in the infinite sample setting and relying on an oracle encoder. \textit{What should we enforce then, to enable} (or at least attempt) \textit{such desired causal generalization?}

The natural mitigation to prevent the described overfitting on spurious dependencies is to optimize for the model sufficiency while enforcing unconfoundness in the representation space, i.e., 
\begin{equation}
\label{eq:suff+unconf}
\begin{split}
    \min_{h,\phi} \quad &\mathbb{E}_{\mathbb{P}^{e_1}}[ \mathcal{L}\left(Y, h \circ \phi(\bm{X})\right) ]\\
    \text{s.t.} \quad &\phi(\bm{X})\indep_{\mathbb{P}^{e_1}} \bm{Z} | Y=y \quad \forall y \in \mathcal{Y}.
\end{split}
\end{equation}
The conditional independence constraint enforces that the mutual information between the representation and the experimental settings, conditioning on the outcome of interest, is null, i.e.,
\begin{equation}
\label{eq:mutualinfo}
    \text{I}_{\mathbb{P}^{e_1}}(\phi(\bm{X}), \bm{Z}|Y)=0,
\end{equation}
such that it cannot be used to leverage some conditional dependencies $\mathbb{P}^{e_1}_{Y|\bm{Z}^{e_1}=\bm{z}}$ potentially breaking on the target experiment since only $\mathbb{P}^e_{Y|\bm{Z}=\bm{z}}$ is invariant.
Note that such conditions can be enforced while learning $f=h\circ \phi$ from scratch or just fine-tuning a head $h$ on top of a pre-trained (oracle) encoder $\phi^*$. The former approach is exactly the problem described in Formulation \ref{eq:suff+unconf}, while in the latter case, the unconfoundness constraint, i.e., Eq. \ref{eq:mutualinfo}, has to be enforced on an internal representation of the head $h$, ideally isolating only the disentangled information of the outcome of interest (see $\phi_Y$ in Section \ref{sec:problem} discussion).


Formulation \ref{eq:suff+unconf} is a well-known problem in Representation Learning literature, and different approaches were developed, enforcing the unconfoundness constraint directly or indirectly. In Section \ref{sec:relatedwork}, we discuss a brief overview of the different paradigms. 
Among them we propose a resampling approach \citep{kirichenko2022last,li2019repair}, carefully designing a fictitious auxiliary experiment $\mathcal{M}^{e_1^{\perp\!\!\!\perp}}$ to sample from during training. It is obtained by manipulating the original reference sample so that its spurious correlations between $\bm{Z}^{e_1}$ and $Y$ are not observed, and confounding effects are prevented. In particular, we define such fictitious \textit{unconfounded} population intervening on the joint distribution $\mathbb{P}^{e_1}_{\bm{Z}^{e_1},Y}$ and enforcing independence, i.e., $\bm{Z}^{e_1} \perp\!\!\!\perp Y$, blocking any not-causal path from $\bm{X}$ to $Y$ during learning. Among the possible joint distribution guaranteeing independence, we define, for all $y \in \mathcal{Y}$ and $\bm{z}\in \mathcal{Z}^{e_1}$ in support of $\mathbb{P}^{e_1}_{\bm{Z}^{e_1},Y}$: 
\begin{equation}
    \label{eq:disentangleddist}
    \mathbb{P}^{e_1^{\perp\!\!\!\perp}}(Y=y, \bm{Z}^{e_1}=\bm{z}) := \frac{\text{Var}_{\mathbb{P}^{e_1}}(Y|\bm{Z}^{e_1}=\bm{z})}{\displaystyle\sum_{\bm{z}'\in \mathcal{Z}^{e_1}} \text{Var}_{\mathbb{P}^{e_1}}(Y|\bm{Z}^{e_1}=\bm{z}')},
\end{equation}
weighting more the least informative experimental settings for the outcome of interest (high conditional variance) and ignoring the fully informative ones (low or null variance) over the reference population. Indeed, let's observe that the marginal outcome given the observed experimental settings is constant, i.e., uniform distribution. If, for each not fully informative observed experimental setting $\bm{z}$\footnote{i.e.,  $\forall \bm{z}:H_{\mathbb{P}^{e_1}}(Y|\bm{Z}^{e_1}=\bm{z})>0$.},
\begin{equation}
\label{eq:support}
    \{y\in \mathcal{Y}:\mathbb{P}^{e_1}(Y=y|\bm{Z}^{e_1}=\bm{z})>0\} = \{y\in \mathcal{Y}:\mathbb{P}^{e_1}(Y=y)>0\},
\end{equation}
then the joint distribution described in Eq. \ref{eq:disentangleddist} trivially implies independence, i.e., $\bm{Z}^{e_1} \perp\!\!\!\perp Y$, and an unconfounded representation is enforced. It is important to observe that in such implementation, spurious solutions may still optimize the risk, but such solutions cannot be selected and preferred since there is no signal in the task to retrieve the experimental settings. 

We then propose to train/fine-tune the factual outcome model on such a fictitious population by ERM sampling from the reference population and reweighting the estimated joint distribution $\mathbb{P}^{e_1}_{\bm{Z}^{e_1},Y}$ to enforce the desired disentangled distribution described in Eq. \ref{eq:disentangleddist}. We refer to this approach as Deconfounded Empirical Risk Minimization (DERM). Such implementation is suitable for applications in Causal Inference where both the experimental settings and the outcome of interest are commonly low dimensional and discrete or anyway discretized for interpretability \citep{pearl2000models, rosenbaum2010design}. In Formula:

\begin{figure}[H]
    \centering
    \begin{minipage}{0.65\textwidth}
        \begin{highlight}
        \begin{center}
            \textbf{Deconfounded Empirical Risk Minimization:}
        \end{center}
        \begin{equation}
            \label{eq:DERM}
            \hat{g}:=\arg \min_{g \in \mathcal{G}} \sum_{i\in \mathcal{D}^{e_1}} \underbrace{w_i}_{\text{unconfoundness}}\cdot\underbrace{\mathcal{L}(g(\bm{x}_i), y_i)}_{\text{sufficiency}}
        \end{equation}
        where:
        \begin{equation*}
            w_i :=\underbrace{\frac{1}{\widehat{\mathbb{P}}^{e_1}(Y=y_i, \bm{Z}^{e_1}=\bm{z}_i)}}_{\text{reference distribution}} \cdot \overbrace{\frac{\widehat{\text{Var}}_{\mathbb{P}^{e_1}}(Y|\bm{Z}^{e_1}=\bm{z_i})}{\displaystyle\sum_{\bm{z}'\in \mathcal{Z}^{e_1}} \widehat{\text{Var}}_{\mathbb{P}^{e_1}}(Y|\bm{Z}^{e_1}=\bm{z}')}}^{\text{fictitious distribution s.t. $\bm{Z} \indep Y$}}
        \end{equation*}
        \noindent and the weights $w_i$ are computed una tantum before training, the joint distribution is estimated by frequency, and the conditional variances with the sample variance.
        \end{highlight}
    \end{minipage}
    \hfill
    \begin{minipage}{0.34\textwidth}
        \centering
        \includegraphics[width=\linewidth]{figures/HS.png} 
        \caption{Illustration of the factual model hypothesis space $\mathcal{G}$, the ERM solution set over the reference $\mathcal{G}^{e_1}$ and target sample $\mathcal{G}^{e_2}$, and an \textit{unconfounded} fictitious sample $\mathcal{G}^{e_1^{\perp\!\!\!\perp}}$.}
        \label{fig:hypspace} 
    \end{minipage}
\end{figure}

Let $\mathcal{G}$ be an expressive factual outcome model hypothesis space (containing an invariant factual outcome model $g^*$). The ERM solution set over the reference sample $\mathcal{G}^{e_1}$ and the target sample $\mathcal{G}^{e_2}$ overlap, while the ERM solution set $\mathcal{G}^{e_1^{\perp\!\!\!\perp}}$ over an \textit{unconfounded} fictitious sample from $\mathcal{M}^{e_1^{\perp\!\!\!\perp}}$ is included in such intersection and it still includes the invariant factual outcome model. In Figure \ref{fig:hypspace}, we illustrate the relations among these hypotheses and solution spaces. 

\looseness=-1\paragraph{Challenge: Beyond Full Support Assumption} When Condition \ref{eq:support} doesn't hold, it is not possible to retrieve a joint distribution enforcing such independence condition, i.e., Eq. \ref{eq:mutualinfo}, without setting $\mathbb{P}^{e_1^{\perp\!\!\!\perp}}(\bm{Z}^{e_1}=\bm{z})=0$ where the conditional outcome's support is strictly contained in the marginal outcome's support on the reference population. 
Our fictitious distribution still considers these samples while reducing their weight with respect to the predictivity of the observed experimental setting (approximately $\propto \text{Var}_{\mathbb{P}^{e_1}}(Y|\bm{Z}^{e_1}=\bm{z})$. Despite some spurious correlations may still be retrieved, it is a trade-off with ignoring a potentially substantial part of the reference sample. Tailored modification of our joint distribution can be proposed case-by-case. If necessary, alternative approaches enforcing Condition \ref{eq:mutualinfo} directly should be considered, not discarding any sample data, but computationally much more expensive and tricky to estimate. 


\subsection{Causal Lifting}
\looseness=-1An extensive supervised dataset to train a factual outcome model from scratch is rarely available in real-world applications. However, foundational models trained on extensive corpus may still be able to process complex data structures, e.g., images and text,  preserving sufficient information for the task while having never been supervised for it directly. We can then leverage such external sources to preprocess the data and train a deconfounded head on top via DERM on the available reference sample alone. We refer to this procedure as \textit{Causal Lifting} since enabling an expressive foundation model to filter only the invariant features for a task, thanks to a small fine-tuning on a single sample with additional supervision for the unconfoundness, i.e., the experiment settings information. Such procedure ideally
leads to a conditionally unbiased factual outcome estimator, enabling efficient Causal Inference on a prediction-powered target experiment\footnote{Either a Randomized Controlled Trial or Observational Study with observed confounders.} according to Theorem \ref{th:feasibility}. 
To summarize:

\begin{algorithm}[h!]
\caption{0-shot Generalization for PPCI (\textit{Causal Lifting})}
\label{alg:ppci}
\begin{algorithmic}[1] 
    \STATE \textbf{Input:} PPCI problem 
    \STATE \textbf{Output:} ATE inference on the target experiment
    \STATE \textbf{Procedure:} 
    \STATE \quad \textbf{Factual Outcome Model} 
    Extract representations from experiment observations via a foundational model and fine-tune its head factual outcome estimator using DERM.
    %\textit{Causal Lifting} of a foundational model\footnotemark{} via DERM on the reference experiment.
    \STATE \quad \textbf{Causal Inference} Via AIPW estimator on the prediction-powered target experiment.
\end{algorithmic}
\end{algorithm}
%\footnotetext{E.g., a pre-trained Vision Transformer.}
An in-detailed description of the procedure is reported in Appendix \ref{sec:algorithm}.






\section{Related Works} 
\label{sec:relatedwork}

\paragraph{Prediction-Powered Causal Inference}
The factual outcome estimation problem for causal inference from high-dimensional observation was first introduced by \citet{cadei2024smoke}. We extended the problem to generalization to a class of SCMs, also considering observational studies motivated by practitioners desiderata, e.g., experimental ecologists. In our paper, we formalize what they describe as ``encoder bias'' (see our discussion on Sufficiency and Unconfoundness in Section \ref{sec:method}), and our DERM is the first proposal to their call for ``\textit{new methodologies to mitigate this bias during adaptation}''.
\citet{demirel2024prediction} already attempted to discuss some generalization challenges in a PPCI problem but with unrealistic motivating assumptions. They ignored any high dimensional projection of the outcome of interest and assumed the experimental settings alone as sufficient for factual outcome estimation together with support overlapping, i.e., not generalization, making the model too application-specific and ignoring any connection with representation learning. Let's further observe that their framework is a special case of ours when $\bm{X}=\bm{Z}$ and low-dimensional, with the target experiment in-distribution. In contrast to the classic Prediction-Powered Inference~(PPI)~\citep{angelopoulos2023prediction,angelopoulos2023ppi++}, which improves estimation efficiency by imputing unlabeled in-distribution data via a predictive model, and recent causal inference extensions \citep{de2025efficient,poulet2025prediction} relying on counterfactual predictions, PPCI focuses on imputing missing \textit{factual} outcomes to generalize across unlabeled experiments, that have different and potentially non-overlapping ``covariate'' distributions, agnostic of the causal estimator. 

\paragraph{Unconfoundness} Learning representations invariant to certain attributes is a challenging and widely studied problem in different machine learning communities \citep{moyer2018invariant}. We wish to learn useful representations of $X$ and that can predict the outcome $Y$, but are invariant to the enviromental variables $Z$. In agreement with \citet{yao2024unifying}, we achieve Causal Representation Learning in DERM by combining a sufficiency objective with an invariance constraint enforcing unconfoundness in the representation space. Several alternative approaches can be considered to enforce such conditional independence constraints:  (i) Conditional Mutual Information Minimization \citep{song2019learning, cheng2020club, gupta2021controllable} (ii) Adversarial Independence Regularization such as \citet{louizos2015variational} which modifies variational autoencoder (VAE) architecture in \citet{kingma2014adam} to learn \textit{fair} representations that are invariant sensitive attributes, by training against an adversary that tries to predict those variables (iii) Conditional Contrastive Learning such as \citet{ma2021conditional} whereby one learns representations invariant to certain attributes by optimizing a conditional contrastive loss (iv) Variational Information Bottleneck methods where one learns useful and sufficient representations invariant to a specific \textit{domain} \cite{alemi2016deep, li2022invariant}.

\paragraph{Causal Representation Learning} In the broader context of causal representation learning methods \citep{scholkopf2021toward}, our proposal largely focuses on representation learning applications to causal inference: learning representations of data that make it possible to estimate causal estimands. We find this is in contrast with most recent works in causal representation learning, which uniquely focused on complete identifiability of all the variables or blocks, see \citet{yao2024unifying,varici2024general,von2024identifiable} for recent overviews targeting general settings. The main exceptions are \citet{yao2024unifying,yao2024marrying}. The former leverages domain generalization regularizers to debias treatment effect estimation in ISTAnt from selection bias. However, their proposal is not sufficient to prevent confounding when no data from the target experiment is given. The latter uses multi-view causal representation learning models to model confounding for adjustment in an observational climate application. In our paper, we also discuss conditions for identification, but we focus on a specific causal estimand, as opposed to block-identifiability of causal variables. Additionally, our perspective offers clear evaluation and benchmarking potential -- even in theoretically underspecified setting: the accuracy of the causal estimate. As opposed to virtually all existing work in causal representation learning, this can be empirically tested in \textit{real world} scientific experiments.


\section{Experiments}
\looseness=-1We empirically validate our approach for 0-shot generalization of PPCI solving a real-world scientific problem from experimental Behavioural Ecology. In particular, we considered ISTAnt dataset\footnote{So far the only benchmark dataset for scientifically motivated representation learning for a causal downstream task.} \citep{cadei2024smoke}, and we designed and collected a similar experimental dataset with lower filming quality and higher diversity in treatments to enable causal lifting of different pre-trained models for 0-shot generalization. We further validate our analysis on a synthetic causal manipulation of the MNIST dataset \citep{lecun1998mnist}.

\subsection{Zero-Shot Generalization on ISTAnt}
\begin{figure}[h!]
    \centering
    \begin{minipage}{0.62\linewidth}
        \centering
        \includegraphics[width=\linewidth]{figures/or_teb_ref.png}
        \caption{0-shot ATE Inference on ISTAnt dataset from our experiment, varying method and pre-trained encoder. 95\% confidence intervals estimated via AIPW asymptotic normality and baseline in black using AIPW on the ground truth outcome. Our approach applied to unsupervised backbones yield consistent estimates, unlike ERM or a general invariance regularization.}
        \label{fig:istantgeneralization}
    \end{minipage}
    \hfill
    \begin{minipage}{0.36\linewidth}
        \centering
        \begin{subfigure}[b]{0.48\linewidth}
        \centering
            \includegraphics[height=2.4cm]{figures/replicafs.jpg}
            \includegraphics[height=2.4cm]{figures/ISTantreplica.png} 
            \caption{Our experiment}
            \label{fig:ISTAntreplica}  
        \end{subfigure}
        \hfill
        \begin{subfigure}[b]{0.48\linewidth}
        \centering
            \includegraphics[height=2.4cm]{figures/istantfs.jpg} 
            \includegraphics[height=2.4cm]{figures/ISTAnt.png} 
            \caption{ISTAnt}
            \label{fig:ISTAnt}
        \end{subfigure}
        \caption{Filming box and example frame from our experiment and ISTAnt. The two datasets mainly differ in lighting quality, treatments considered, experimental nests (wall height) and color marking.}
        \label{fig:examples}
    \end{minipage}
\end{figure}


To empirically validate our methodology, we replicated a real-world scientific pipeline requiring generalization for PPCI and compared our performances with the existing approaches suitable for the task. We performed a similar experiment to the ISTAnt dataset with lower-quality filming conditions (in particular light conditions), slightly different ant coloring and diversified treatments with or without micro-particle application. We considered it the reference experiment and proposed to generalize to ISTAnt, the target experiment in the PPCI problem. Our experiment consists of 44 annotated videos, 30 minutes long, each randomly assigning one of three possible treatments (one of which does not entail usage of micro-particles and serves as a control) and considering the same outcome of interest as in ISTAnt, i.e., directional grooming, blindly annotated by a single expert (versus three different in ISTAnt). The pipeline reflects a common desiderata in experimental research: learning a model from previously annotated experiments able to cheaply and validly annotate new, out-of-distribution experiments. The new experiments commonly consider more experimental settings and rely on better or generally different data acquisition techniques.
In Figure \ref{fig:examples} we compare a random frame from ISTAnt dataset, and one from our experiment. A full description of the design and collection of the data is reported in Section \ref{ssec:replica}.
 
We started from the five best-performing vision transformers in \citet{cadei2024smoke} -- ViT-B \citep{dosovitskiy2020image}, ViT-L \citep{zhai2023sigmoid}, CLIP-ViT-B,-L \citep{radford2021learning}, DINOv2 \citep{oquab2023dinov2}. We trained several simple nonlinear heads, i.e., multi-layer perceptron, on top via (i) vanilla ERM, (ii) Variance Risk Extrapolation (vREx) \citet{krueger2021out} and (iii) DERM (ours) for Causal Lifting and used the model for 0-shot generalization for PPCI on the original ISTAnt dataset, via AIPW. For reference we considered the ATE Inference on ISTAnt by the AIPW estimator on the human-annotated factual outcomes (ground truth). On average, the treatment in ISTAnt increases the grooming time towards the focal ant by $\approx 40$ seconds.  Further details on the modeling choices, hyper-parameter, and fine-tuning are discussed in Appendix \ref{ssec:ISTAexp}

Figure \ref{fig:istantgeneralization} summarizes the results of our generalization experiment comparing the 95\% confidence intervals obtained by AIPW asymptotic normality. As expected, with vanilla ERM, there are no guarantees to Causal Lift any foundational model due to potential confounding effects, and the ATE estimates are consistently offset by underestimating/ignoring it. Similar results, using v-REx, as proposed by \citet{yao2024unifying}. Indeed, experiment setting performance invariance is not sufficient to prevent confounding effects on the target when certain association switches (see Example 2 in Section \ref{sec:problem}). DERM is the only method enabling 0-shot generalization for PPCI  with DINOv2 and (partially) with CLIP-based vision transformers. Interestingly enough, the most supervised encoder, i.e., ViT-based (trained on ImageNet \citep{deng2009imagenet}), struggles in the task, underestimating the effect, as opposed to the ones trained in a fully unsupervised fashion. We hypothesize that encoders pre-trained in a supervised fashion are more inclined to extract more entangled representations, more challenging to causal lift. 

\subsection{CausalMNIST}
\looseness=-1We replicated the analysis on colored manipulations of the MNIST dataset, enabling some fictitious PPCI problems, e.g., estimating the effect of the background color or pen color on the digit value, allowing complete control of the causal effects. While simpler, this experiment supplements the fact that obtaining ground-truth causal effect on real-world data is challenging, and one whole experiment only yields a single measurement of a target causal estimand.  

We test both on RCT and Observational Study experiments with observed confounders in either reference or target. A full description of the data-generating processes and analysis are reported in Appendix \ref{sec:CausalMNIST}. In Table \ref{tab:generalization}, we report the ATE inference as for ISTAnt, (i) on a target experiment $\mathcal{D}^{e_2}$ with a new treatment with the same appearance (see Example 1 in Section \ref{sec:problem}) and (ii) on a target experiment $\mathcal{D}^{e_3}$ strongly out-of-support. 
DERM is the unique method solving the problem on $\mathcal{D}^{e_2}$, and despite no method having guarantees on $\mathcal{D}^{e_3}$, it is still the least biased. 

\begin{table}[h]
\centering
\setlength{\tabcolsep}{8pt} % Adjust column spacing
\renewcommand{\arraystretch}{1.3} % Increase row height
    \begin{tabular}{cc|cc}
    Method & $\mathcal{D}^{e_1}$ & $\mathcal{D}^{e_2}$ & $\mathcal{D}^{e_3}$ \\ 
    & \footnotesize{($\text{ATE}=1.5$)} & \footnotesize{($\text{ATE}=0$)} & \footnotesize{($\text{ATE}=0$)} \\ \hline
    ERM & \textbf{0.00 $\pm$ 0.02} & 0.86 $\pm$ 0.14  & 1.05 $\pm$ 0.15 \\
    v-REx & 0.01 $\pm$ 0.03 & 0.83 $\pm$ 0.15 & 1.05 $\pm$ 0.14 \\
    Ours & 0.10 $\pm$ 0.07 & \textbf{0.14 $\pm$ 0.14} & \textbf{0.75 $\pm$ 0.05}
    \end{tabular}
    \medskip
\caption{ATE bias and standard deviation via AIPW on a reference trial $\mathcal{D}^{e_1}$ and two target samples $\mathcal{D}^{e_2}$-$\mathcal{D}^{e_3}$ of CausalMNIST not annotated and prediction-powered by a Convolutional Neural Network trained with different objectives.
Sample mean and standard deviation are computed over the same PPCI problem repeated 50 times, re-sampling both reference and target samples. ERM and v-REX yield biased estimates on the new population $\mathcal{D}^{e_2}$, unlike our approach.}
\label{tab:generalization}
\end{table}


\section{Conclusion}
We introduced Causal Lifting, a novel paradigm enabling zero-shot generalization of foundational models for prediction-powered causal inferences. Our concrete implementation in the Deconfounded Empirical Risk Minimization (DERM) leverages a sufficiency loss paired with an unconfoundness objective in the representation space to prevent overfitting on experiment-specific spurious correlation. Additionally, we thoroughly described in which settings causal lifting can yield unbiased estimates, unlike empirical risk minimization. Our framework is widely applicable to the analysis of experimental data, which we have empirically evaluated on the ISTAnt data set. Overall, this work offers a paradigm shift from the causal representation learning literature to learning representations that enable downstream causal estimates on real-world data, which we think is a critical component of representation learning to accelerate scientific discovery. 
The main limitation of this work is that via PPCIs we can rarely have guarantees a priori on the Causal Estimates, being Condition \ref{eq:unbiased} untestable without target annotations and Condition \ref{eq:support} potentially violated (on top of unobserved confounders issues). Model convergence is also not discussed, which is particularly interesting in the finite setting.
At the same time, we hope that more systematic (scientifically motivated) benchmarking will lead the progress of the field, e.g., challenging and comparing Causal Representation Learning identifiability results beyond their controlled assumptions. 


% \section*{Impact Statement}
% This paper presents work whose goal is to advance the field of Machine Learning. There are many potential societal consequences of our work, none of which we feel must be specifically highlighted here.


\section*{Acknowledgments} We thank the Causal Learning and Artificial Intelligence group at ISTA, and particularly Marco Fumero, for the continuous feedback and inspiring discussions during the last year. We thank the Social Immunity group at ISTA, particularly Jinook Oh, for the annotation program and Michaela Hoenigsberger for supporting our ecological experiment. We thank Irene Guerrieri for the illustration in Figure \ref{fig:t1}.
Riccardo Cadei is supported by a Google Research Scholar Award and a Google Initiated Gift to Francesco Locatello. 

\bibliographystyle{unsrtnat}
\bibliography{refs}
\clearpage
\appendix
\onecolumn
\subsection{Lloyd-Max Algorithm}
\label{subsec:Lloyd-Max}
For a given quantization bitwidth $B$ and an operand $\bm{X}$, the Lloyd-Max algorithm finds $2^B$ quantization levels $\{\hat{x}_i\}_{i=1}^{2^B}$ such that quantizing $\bm{X}$ by rounding each scalar in $\bm{X}$ to the nearest quantization level minimizes the quantization MSE. 

The algorithm starts with an initial guess of quantization levels and then iteratively computes quantization thresholds $\{\tau_i\}_{i=1}^{2^B-1}$ and updates quantization levels $\{\hat{x}_i\}_{i=1}^{2^B}$. Specifically, at iteration $n$, thresholds are set to the midpoints of the previous iteration's levels:
\begin{align*}
    \tau_i^{(n)}=\frac{\hat{x}_i^{(n-1)}+\hat{x}_{i+1}^{(n-1)}}2 \text{ for } i=1\ldots 2^B-1
\end{align*}
Subsequently, the quantization levels are re-computed as conditional means of the data regions defined by the new thresholds:
\begin{align*}
    \hat{x}_i^{(n)}=\mathbb{E}\left[ \bm{X} \big| \bm{X}\in [\tau_{i-1}^{(n)},\tau_i^{(n)}] \right] \text{ for } i=1\ldots 2^B
\end{align*}
where to satisfy boundary conditions we have $\tau_0=-\infty$ and $\tau_{2^B}=\infty$. The algorithm iterates the above steps until convergence.

Figure \ref{fig:lm_quant} compares the quantization levels of a $7$-bit floating point (E3M3) quantizer (left) to a $7$-bit Lloyd-Max quantizer (right) when quantizing a layer of weights from the GPT3-126M model at a per-tensor granularity. As shown, the Lloyd-Max quantizer achieves substantially lower quantization MSE. Further, Table \ref{tab:FP7_vs_LM7} shows the superior perplexity achieved by Lloyd-Max quantizers for bitwidths of $7$, $6$ and $5$. The difference between the quantizers is clear at 5 bits, where per-tensor FP quantization incurs a drastic and unacceptable increase in perplexity, while Lloyd-Max quantization incurs a much smaller increase. Nevertheless, we note that even the optimal Lloyd-Max quantizer incurs a notable ($\sim 1.5$) increase in perplexity due to the coarse granularity of quantization. 

\begin{figure}[h]
  \centering
  \includegraphics[width=0.7\linewidth]{sections/figures/LM7_FP7.pdf}
  \caption{\small Quantization levels and the corresponding quantization MSE of Floating Point (left) vs Lloyd-Max (right) Quantizers for a layer of weights in the GPT3-126M model.}
  \label{fig:lm_quant}
\end{figure}

\begin{table}[h]\scriptsize
\begin{center}
\caption{\label{tab:FP7_vs_LM7} \small Comparing perplexity (lower is better) achieved by floating point quantizers and Lloyd-Max quantizers on a GPT3-126M model for the Wikitext-103 dataset.}
\begin{tabular}{c|cc|c}
\hline
 \multirow{2}{*}{\textbf{Bitwidth}} & \multicolumn{2}{|c|}{\textbf{Floating-Point Quantizer}} & \textbf{Lloyd-Max Quantizer} \\
 & Best Format & Wikitext-103 Perplexity & Wikitext-103 Perplexity \\
\hline
7 & E3M3 & 18.32 & 18.27 \\
6 & E3M2 & 19.07 & 18.51 \\
5 & E4M0 & 43.89 & 19.71 \\
\hline
\end{tabular}
\end{center}
\end{table}

\subsection{Proof of Local Optimality of LO-BCQ}
\label{subsec:lobcq_opt_proof}
For a given block $\bm{b}_j$, the quantization MSE during LO-BCQ can be empirically evaluated as $\frac{1}{L_b}\lVert \bm{b}_j- \bm{\hat{b}}_j\rVert^2_2$ where $\bm{\hat{b}}_j$ is computed from equation (\ref{eq:clustered_quantization_definition}) as $C_{f(\bm{b}_j)}(\bm{b}_j)$. Further, for a given block cluster $\mathcal{B}_i$, we compute the quantization MSE as $\frac{1}{|\mathcal{B}_{i}|}\sum_{\bm{b} \in \mathcal{B}_{i}} \frac{1}{L_b}\lVert \bm{b}- C_i^{(n)}(\bm{b})\rVert^2_2$. Therefore, at the end of iteration $n$, we evaluate the overall quantization MSE $J^{(n)}$ for a given operand $\bm{X}$ composed of $N_c$ block clusters as:
\begin{align*}
    \label{eq:mse_iter_n}
    J^{(n)} = \frac{1}{N_c} \sum_{i=1}^{N_c} \frac{1}{|\mathcal{B}_{i}^{(n)}|}\sum_{\bm{v} \in \mathcal{B}_{i}^{(n)}} \frac{1}{L_b}\lVert \bm{b}- B_i^{(n)}(\bm{b})\rVert^2_2
\end{align*}

At the end of iteration $n$, the codebooks are updated from $\mathcal{C}^{(n-1)}$ to $\mathcal{C}^{(n)}$. However, the mapping of a given vector $\bm{b}_j$ to quantizers $\mathcal{C}^{(n)}$ remains as  $f^{(n)}(\bm{b}_j)$. At the next iteration, during the vector clustering step, $f^{(n+1)}(\bm{b}_j)$ finds new mapping of $\bm{b}_j$ to updated codebooks $\mathcal{C}^{(n)}$ such that the quantization MSE over the candidate codebooks is minimized. Therefore, we obtain the following result for $\bm{b}_j$:
\begin{align*}
\frac{1}{L_b}\lVert \bm{b}_j - C_{f^{(n+1)}(\bm{b}_j)}^{(n)}(\bm{b}_j)\rVert^2_2 \le \frac{1}{L_b}\lVert \bm{b}_j - C_{f^{(n)}(\bm{b}_j)}^{(n)}(\bm{b}_j)\rVert^2_2
\end{align*}

That is, quantizing $\bm{b}_j$ at the end of the block clustering step of iteration $n+1$ results in lower quantization MSE compared to quantizing at the end of iteration $n$. Since this is true for all $\bm{b} \in \bm{X}$, we assert the following:
\begin{equation}
\begin{split}
\label{eq:mse_ineq_1}
    \tilde{J}^{(n+1)} &= \frac{1}{N_c} \sum_{i=1}^{N_c} \frac{1}{|\mathcal{B}_{i}^{(n+1)}|}\sum_{\bm{b} \in \mathcal{B}_{i}^{(n+1)}} \frac{1}{L_b}\lVert \bm{b} - C_i^{(n)}(b)\rVert^2_2 \le J^{(n)}
\end{split}
\end{equation}
where $\tilde{J}^{(n+1)}$ is the the quantization MSE after the vector clustering step at iteration $n+1$.

Next, during the codebook update step (\ref{eq:quantizers_update}) at iteration $n+1$, the per-cluster codebooks $\mathcal{C}^{(n)}$ are updated to $\mathcal{C}^{(n+1)}$ by invoking the Lloyd-Max algorithm \citep{Lloyd}. We know that for any given value distribution, the Lloyd-Max algorithm minimizes the quantization MSE. Therefore, for a given vector cluster $\mathcal{B}_i$ we obtain the following result:

\begin{equation}
    \frac{1}{|\mathcal{B}_{i}^{(n+1)}|}\sum_{\bm{b} \in \mathcal{B}_{i}^{(n+1)}} \frac{1}{L_b}\lVert \bm{b}- C_i^{(n+1)}(\bm{b})\rVert^2_2 \le \frac{1}{|\mathcal{B}_{i}^{(n+1)}|}\sum_{\bm{b} \in \mathcal{B}_{i}^{(n+1)}} \frac{1}{L_b}\lVert \bm{b}- C_i^{(n)}(\bm{b})\rVert^2_2
\end{equation}

The above equation states that quantizing the given block cluster $\mathcal{B}_i$ after updating the associated codebook from $C_i^{(n)}$ to $C_i^{(n+1)}$ results in lower quantization MSE. Since this is true for all the block clusters, we derive the following result: 
\begin{equation}
\begin{split}
\label{eq:mse_ineq_2}
     J^{(n+1)} &= \frac{1}{N_c} \sum_{i=1}^{N_c} \frac{1}{|\mathcal{B}_{i}^{(n+1)}|}\sum_{\bm{b} \in \mathcal{B}_{i}^{(n+1)}} \frac{1}{L_b}\lVert \bm{b}- C_i^{(n+1)}(\bm{b})\rVert^2_2  \le \tilde{J}^{(n+1)}   
\end{split}
\end{equation}

Following (\ref{eq:mse_ineq_1}) and (\ref{eq:mse_ineq_2}), we find that the quantization MSE is non-increasing for each iteration, that is, $J^{(1)} \ge J^{(2)} \ge J^{(3)} \ge \ldots \ge J^{(M)}$ where $M$ is the maximum number of iterations. 
%Therefore, we can say that if the algorithm converges, then it must be that it has converged to a local minimum. 
\hfill $\blacksquare$


\begin{figure}
    \begin{center}
    \includegraphics[width=0.5\textwidth]{sections//figures/mse_vs_iter.pdf}
    \end{center}
    \caption{\small NMSE vs iterations during LO-BCQ compared to other block quantization proposals}
    \label{fig:nmse_vs_iter}
\end{figure}

Figure \ref{fig:nmse_vs_iter} shows the empirical convergence of LO-BCQ across several block lengths and number of codebooks. Also, the MSE achieved by LO-BCQ is compared to baselines such as MXFP and VSQ. As shown, LO-BCQ converges to a lower MSE than the baselines. Further, we achieve better convergence for larger number of codebooks ($N_c$) and for a smaller block length ($L_b$), both of which increase the bitwidth of BCQ (see Eq \ref{eq:bitwidth_bcq}).


\subsection{Additional Accuracy Results}
%Table \ref{tab:lobcq_config} lists the various LOBCQ configurations and their corresponding bitwidths.
\begin{table}
\setlength{\tabcolsep}{4.75pt}
\begin{center}
\caption{\label{tab:lobcq_config} Various LO-BCQ configurations and their bitwidths.}
\begin{tabular}{|c||c|c|c|c||c|c||c|} 
\hline
 & \multicolumn{4}{|c||}{$L_b=8$} & \multicolumn{2}{|c||}{$L_b=4$} & $L_b=2$ \\
 \hline
 \backslashbox{$L_A$\kern-1em}{\kern-1em$N_c$} & 2 & 4 & 8 & 16 & 2 & 4 & 2 \\
 \hline
 64 & 4.25 & 4.375 & 4.5 & 4.625 & 4.375 & 4.625 & 4.625\\
 \hline
 32 & 4.375 & 4.5 & 4.625& 4.75 & 4.5 & 4.75 & 4.75 \\
 \hline
 16 & 4.625 & 4.75& 4.875 & 5 & 4.75 & 5 & 5 \\
 \hline
\end{tabular}
\end{center}
\end{table}

%\subsection{Perplexity achieved by various LO-BCQ configurations on Wikitext-103 dataset}

\begin{table} \centering
\begin{tabular}{|c||c|c|c|c||c|c||c|} 
\hline
 $L_b \rightarrow$& \multicolumn{4}{c||}{8} & \multicolumn{2}{c||}{4} & 2\\
 \hline
 \backslashbox{$L_A$\kern-1em}{\kern-1em$N_c$} & 2 & 4 & 8 & 16 & 2 & 4 & 2  \\
 %$N_c \rightarrow$ & 2 & 4 & 8 & 16 & 2 & 4 & 2 \\
 \hline
 \hline
 \multicolumn{8}{c}{GPT3-1.3B (FP32 PPL = 9.98)} \\ 
 \hline
 \hline
 64 & 10.40 & 10.23 & 10.17 & 10.15 &  10.28 & 10.18 & 10.19 \\
 \hline
 32 & 10.25 & 10.20 & 10.15 & 10.12 &  10.23 & 10.17 & 10.17 \\
 \hline
 16 & 10.22 & 10.16 & 10.10 & 10.09 &  10.21 & 10.14 & 10.16 \\
 \hline
  \hline
 \multicolumn{8}{c}{GPT3-8B (FP32 PPL = 7.38)} \\ 
 \hline
 \hline
 64 & 7.61 & 7.52 & 7.48 &  7.47 &  7.55 &  7.49 & 7.50 \\
 \hline
 32 & 7.52 & 7.50 & 7.46 &  7.45 &  7.52 &  7.48 & 7.48  \\
 \hline
 16 & 7.51 & 7.48 & 7.44 &  7.44 &  7.51 &  7.49 & 7.47  \\
 \hline
\end{tabular}
\caption{\label{tab:ppl_gpt3_abalation} Wikitext-103 perplexity across GPT3-1.3B and 8B models.}
\end{table}

\begin{table} \centering
\begin{tabular}{|c||c|c|c|c||} 
\hline
 $L_b \rightarrow$& \multicolumn{4}{c||}{8}\\
 \hline
 \backslashbox{$L_A$\kern-1em}{\kern-1em$N_c$} & 2 & 4 & 8 & 16 \\
 %$N_c \rightarrow$ & 2 & 4 & 8 & 16 & 2 & 4 & 2 \\
 \hline
 \hline
 \multicolumn{5}{|c|}{Llama2-7B (FP32 PPL = 5.06)} \\ 
 \hline
 \hline
 64 & 5.31 & 5.26 & 5.19 & 5.18  \\
 \hline
 32 & 5.23 & 5.25 & 5.18 & 5.15  \\
 \hline
 16 & 5.23 & 5.19 & 5.16 & 5.14  \\
 \hline
 \multicolumn{5}{|c|}{Nemotron4-15B (FP32 PPL = 5.87)} \\ 
 \hline
 \hline
 64  & 6.3 & 6.20 & 6.13 & 6.08  \\
 \hline
 32  & 6.24 & 6.12 & 6.07 & 6.03  \\
 \hline
 16  & 6.12 & 6.14 & 6.04 & 6.02  \\
 \hline
 \multicolumn{5}{|c|}{Nemotron4-340B (FP32 PPL = 3.48)} \\ 
 \hline
 \hline
 64 & 3.67 & 3.62 & 3.60 & 3.59 \\
 \hline
 32 & 3.63 & 3.61 & 3.59 & 3.56 \\
 \hline
 16 & 3.61 & 3.58 & 3.57 & 3.55 \\
 \hline
\end{tabular}
\caption{\label{tab:ppl_llama7B_nemo15B} Wikitext-103 perplexity compared to FP32 baseline in Llama2-7B and Nemotron4-15B, 340B models}
\end{table}

%\subsection{Perplexity achieved by various LO-BCQ configurations on MMLU dataset}


\begin{table} \centering
\begin{tabular}{|c||c|c|c|c||c|c|c|c|} 
\hline
 $L_b \rightarrow$& \multicolumn{4}{c||}{8} & \multicolumn{4}{c||}{8}\\
 \hline
 \backslashbox{$L_A$\kern-1em}{\kern-1em$N_c$} & 2 & 4 & 8 & 16 & 2 & 4 & 8 & 16  \\
 %$N_c \rightarrow$ & 2 & 4 & 8 & 16 & 2 & 4 & 2 \\
 \hline
 \hline
 \multicolumn{5}{|c|}{Llama2-7B (FP32 Accuracy = 45.8\%)} & \multicolumn{4}{|c|}{Llama2-70B (FP32 Accuracy = 69.12\%)} \\ 
 \hline
 \hline
 64 & 43.9 & 43.4 & 43.9 & 44.9 & 68.07 & 68.27 & 68.17 & 68.75 \\
 \hline
 32 & 44.5 & 43.8 & 44.9 & 44.5 & 68.37 & 68.51 & 68.35 & 68.27  \\
 \hline
 16 & 43.9 & 42.7 & 44.9 & 45 & 68.12 & 68.77 & 68.31 & 68.59  \\
 \hline
 \hline
 \multicolumn{5}{|c|}{GPT3-22B (FP32 Accuracy = 38.75\%)} & \multicolumn{4}{|c|}{Nemotron4-15B (FP32 Accuracy = 64.3\%)} \\ 
 \hline
 \hline
 64 & 36.71 & 38.85 & 38.13 & 38.92 & 63.17 & 62.36 & 63.72 & 64.09 \\
 \hline
 32 & 37.95 & 38.69 & 39.45 & 38.34 & 64.05 & 62.30 & 63.8 & 64.33  \\
 \hline
 16 & 38.88 & 38.80 & 38.31 & 38.92 & 63.22 & 63.51 & 63.93 & 64.43  \\
 \hline
\end{tabular}
\caption{\label{tab:mmlu_abalation} Accuracy on MMLU dataset across GPT3-22B, Llama2-7B, 70B and Nemotron4-15B models.}
\end{table}


%\subsection{Perplexity achieved by various LO-BCQ configurations on LM evaluation harness}

\begin{table} \centering
\begin{tabular}{|c||c|c|c|c||c|c|c|c|} 
\hline
 $L_b \rightarrow$& \multicolumn{4}{c||}{8} & \multicolumn{4}{c||}{8}\\
 \hline
 \backslashbox{$L_A$\kern-1em}{\kern-1em$N_c$} & 2 & 4 & 8 & 16 & 2 & 4 & 8 & 16  \\
 %$N_c \rightarrow$ & 2 & 4 & 8 & 16 & 2 & 4 & 2 \\
 \hline
 \hline
 \multicolumn{5}{|c|}{Race (FP32 Accuracy = 37.51\%)} & \multicolumn{4}{|c|}{Boolq (FP32 Accuracy = 64.62\%)} \\ 
 \hline
 \hline
 64 & 36.94 & 37.13 & 36.27 & 37.13 & 63.73 & 62.26 & 63.49 & 63.36 \\
 \hline
 32 & 37.03 & 36.36 & 36.08 & 37.03 & 62.54 & 63.51 & 63.49 & 63.55  \\
 \hline
 16 & 37.03 & 37.03 & 36.46 & 37.03 & 61.1 & 63.79 & 63.58 & 63.33  \\
 \hline
 \hline
 \multicolumn{5}{|c|}{Winogrande (FP32 Accuracy = 58.01\%)} & \multicolumn{4}{|c|}{Piqa (FP32 Accuracy = 74.21\%)} \\ 
 \hline
 \hline
 64 & 58.17 & 57.22 & 57.85 & 58.33 & 73.01 & 73.07 & 73.07 & 72.80 \\
 \hline
 32 & 59.12 & 58.09 & 57.85 & 58.41 & 73.01 & 73.94 & 72.74 & 73.18  \\
 \hline
 16 & 57.93 & 58.88 & 57.93 & 58.56 & 73.94 & 72.80 & 73.01 & 73.94  \\
 \hline
\end{tabular}
\caption{\label{tab:mmlu_abalation} Accuracy on LM evaluation harness tasks on GPT3-1.3B model.}
\end{table}

\begin{table} \centering
\begin{tabular}{|c||c|c|c|c||c|c|c|c|} 
\hline
 $L_b \rightarrow$& \multicolumn{4}{c||}{8} & \multicolumn{4}{c||}{8}\\
 \hline
 \backslashbox{$L_A$\kern-1em}{\kern-1em$N_c$} & 2 & 4 & 8 & 16 & 2 & 4 & 8 & 16  \\
 %$N_c \rightarrow$ & 2 & 4 & 8 & 16 & 2 & 4 & 2 \\
 \hline
 \hline
 \multicolumn{5}{|c|}{Race (FP32 Accuracy = 41.34\%)} & \multicolumn{4}{|c|}{Boolq (FP32 Accuracy = 68.32\%)} \\ 
 \hline
 \hline
 64 & 40.48 & 40.10 & 39.43 & 39.90 & 69.20 & 68.41 & 69.45 & 68.56 \\
 \hline
 32 & 39.52 & 39.52 & 40.77 & 39.62 & 68.32 & 67.43 & 68.17 & 69.30  \\
 \hline
 16 & 39.81 & 39.71 & 39.90 & 40.38 & 68.10 & 66.33 & 69.51 & 69.42  \\
 \hline
 \hline
 \multicolumn{5}{|c|}{Winogrande (FP32 Accuracy = 67.88\%)} & \multicolumn{4}{|c|}{Piqa (FP32 Accuracy = 78.78\%)} \\ 
 \hline
 \hline
 64 & 66.85 & 66.61 & 67.72 & 67.88 & 77.31 & 77.42 & 77.75 & 77.64 \\
 \hline
 32 & 67.25 & 67.72 & 67.72 & 67.00 & 77.31 & 77.04 & 77.80 & 77.37  \\
 \hline
 16 & 68.11 & 68.90 & 67.88 & 67.48 & 77.37 & 78.13 & 78.13 & 77.69  \\
 \hline
\end{tabular}
\caption{\label{tab:mmlu_abalation} Accuracy on LM evaluation harness tasks on GPT3-8B model.}
\end{table}

\begin{table} \centering
\begin{tabular}{|c||c|c|c|c||c|c|c|c|} 
\hline
 $L_b \rightarrow$& \multicolumn{4}{c||}{8} & \multicolumn{4}{c||}{8}\\
 \hline
 \backslashbox{$L_A$\kern-1em}{\kern-1em$N_c$} & 2 & 4 & 8 & 16 & 2 & 4 & 8 & 16  \\
 %$N_c \rightarrow$ & 2 & 4 & 8 & 16 & 2 & 4 & 2 \\
 \hline
 \hline
 \multicolumn{5}{|c|}{Race (FP32 Accuracy = 40.67\%)} & \multicolumn{4}{|c|}{Boolq (FP32 Accuracy = 76.54\%)} \\ 
 \hline
 \hline
 64 & 40.48 & 40.10 & 39.43 & 39.90 & 75.41 & 75.11 & 77.09 & 75.66 \\
 \hline
 32 & 39.52 & 39.52 & 40.77 & 39.62 & 76.02 & 76.02 & 75.96 & 75.35  \\
 \hline
 16 & 39.81 & 39.71 & 39.90 & 40.38 & 75.05 & 73.82 & 75.72 & 76.09  \\
 \hline
 \hline
 \multicolumn{5}{|c|}{Winogrande (FP32 Accuracy = 70.64\%)} & \multicolumn{4}{|c|}{Piqa (FP32 Accuracy = 79.16\%)} \\ 
 \hline
 \hline
 64 & 69.14 & 70.17 & 70.17 & 70.56 & 78.24 & 79.00 & 78.62 & 78.73 \\
 \hline
 32 & 70.96 & 69.69 & 71.27 & 69.30 & 78.56 & 79.49 & 79.16 & 78.89  \\
 \hline
 16 & 71.03 & 69.53 & 69.69 & 70.40 & 78.13 & 79.16 & 79.00 & 79.00  \\
 \hline
\end{tabular}
\caption{\label{tab:mmlu_abalation} Accuracy on LM evaluation harness tasks on GPT3-22B model.}
\end{table}

\begin{table} \centering
\begin{tabular}{|c||c|c|c|c||c|c|c|c|} 
\hline
 $L_b \rightarrow$& \multicolumn{4}{c||}{8} & \multicolumn{4}{c||}{8}\\
 \hline
 \backslashbox{$L_A$\kern-1em}{\kern-1em$N_c$} & 2 & 4 & 8 & 16 & 2 & 4 & 8 & 16  \\
 %$N_c \rightarrow$ & 2 & 4 & 8 & 16 & 2 & 4 & 2 \\
 \hline
 \hline
 \multicolumn{5}{|c|}{Race (FP32 Accuracy = 44.4\%)} & \multicolumn{4}{|c|}{Boolq (FP32 Accuracy = 79.29\%)} \\ 
 \hline
 \hline
 64 & 42.49 & 42.51 & 42.58 & 43.45 & 77.58 & 77.37 & 77.43 & 78.1 \\
 \hline
 32 & 43.35 & 42.49 & 43.64 & 43.73 & 77.86 & 75.32 & 77.28 & 77.86  \\
 \hline
 16 & 44.21 & 44.21 & 43.64 & 42.97 & 78.65 & 77 & 76.94 & 77.98  \\
 \hline
 \hline
 \multicolumn{5}{|c|}{Winogrande (FP32 Accuracy = 69.38\%)} & \multicolumn{4}{|c|}{Piqa (FP32 Accuracy = 78.07\%)} \\ 
 \hline
 \hline
 64 & 68.9 & 68.43 & 69.77 & 68.19 & 77.09 & 76.82 & 77.09 & 77.86 \\
 \hline
 32 & 69.38 & 68.51 & 68.82 & 68.90 & 78.07 & 76.71 & 78.07 & 77.86  \\
 \hline
 16 & 69.53 & 67.09 & 69.38 & 68.90 & 77.37 & 77.8 & 77.91 & 77.69  \\
 \hline
\end{tabular}
\caption{\label{tab:mmlu_abalation} Accuracy on LM evaluation harness tasks on Llama2-7B model.}
\end{table}

\begin{table} \centering
\begin{tabular}{|c||c|c|c|c||c|c|c|c|} 
\hline
 $L_b \rightarrow$& \multicolumn{4}{c||}{8} & \multicolumn{4}{c||}{8}\\
 \hline
 \backslashbox{$L_A$\kern-1em}{\kern-1em$N_c$} & 2 & 4 & 8 & 16 & 2 & 4 & 8 & 16  \\
 %$N_c \rightarrow$ & 2 & 4 & 8 & 16 & 2 & 4 & 2 \\
 \hline
 \hline
 \multicolumn{5}{|c|}{Race (FP32 Accuracy = 48.8\%)} & \multicolumn{4}{|c|}{Boolq (FP32 Accuracy = 85.23\%)} \\ 
 \hline
 \hline
 64 & 49.00 & 49.00 & 49.28 & 48.71 & 82.82 & 84.28 & 84.03 & 84.25 \\
 \hline
 32 & 49.57 & 48.52 & 48.33 & 49.28 & 83.85 & 84.46 & 84.31 & 84.93  \\
 \hline
 16 & 49.85 & 49.09 & 49.28 & 48.99 & 85.11 & 84.46 & 84.61 & 83.94  \\
 \hline
 \hline
 \multicolumn{5}{|c|}{Winogrande (FP32 Accuracy = 79.95\%)} & \multicolumn{4}{|c|}{Piqa (FP32 Accuracy = 81.56\%)} \\ 
 \hline
 \hline
 64 & 78.77 & 78.45 & 78.37 & 79.16 & 81.45 & 80.69 & 81.45 & 81.5 \\
 \hline
 32 & 78.45 & 79.01 & 78.69 & 80.66 & 81.56 & 80.58 & 81.18 & 81.34  \\
 \hline
 16 & 79.95 & 79.56 & 79.79 & 79.72 & 81.28 & 81.66 & 81.28 & 80.96  \\
 \hline
\end{tabular}
\caption{\label{tab:mmlu_abalation} Accuracy on LM evaluation harness tasks on Llama2-70B model.}
\end{table}

%\section{MSE Studies}
%\textcolor{red}{TODO}


\subsection{Number Formats and Quantization Method}
\label{subsec:numFormats_quantMethod}
\subsubsection{Integer Format}
An $n$-bit signed integer (INT) is typically represented with a 2s-complement format \citep{yao2022zeroquant,xiao2023smoothquant,dai2021vsq}, where the most significant bit denotes the sign.

\subsubsection{Floating Point Format}
An $n$-bit signed floating point (FP) number $x$ comprises of a 1-bit sign ($x_{\mathrm{sign}}$), $B_m$-bit mantissa ($x_{\mathrm{mant}}$) and $B_e$-bit exponent ($x_{\mathrm{exp}}$) such that $B_m+B_e=n-1$. The associated constant exponent bias ($E_{\mathrm{bias}}$) is computed as $(2^{{B_e}-1}-1)$. We denote this format as $E_{B_e}M_{B_m}$.  

\subsubsection{Quantization Scheme}
\label{subsec:quant_method}
A quantization scheme dictates how a given unquantized tensor is converted to its quantized representation. We consider FP formats for the purpose of illustration. Given an unquantized tensor $\bm{X}$ and an FP format $E_{B_e}M_{B_m}$, we first, we compute the quantization scale factor $s_X$ that maps the maximum absolute value of $\bm{X}$ to the maximum quantization level of the $E_{B_e}M_{B_m}$ format as follows:
\begin{align}
\label{eq:sf}
    s_X = \frac{\mathrm{max}(|\bm{X}|)}{\mathrm{max}(E_{B_e}M_{B_m})}
\end{align}
In the above equation, $|\cdot|$ denotes the absolute value function.

Next, we scale $\bm{X}$ by $s_X$ and quantize it to $\hat{\bm{X}}$ by rounding it to the nearest quantization level of $E_{B_e}M_{B_m}$ as:

\begin{align}
\label{eq:tensor_quant}
    \hat{\bm{X}} = \text{round-to-nearest}\left(\frac{\bm{X}}{s_X}, E_{B_e}M_{B_m}\right)
\end{align}

We perform dynamic max-scaled quantization \citep{wu2020integer}, where the scale factor $s$ for activations is dynamically computed during runtime.

\subsection{Vector Scaled Quantization}
\begin{wrapfigure}{r}{0.35\linewidth}
  \centering
  \includegraphics[width=\linewidth]{sections/figures/vsquant.jpg}
  \caption{\small Vectorwise decomposition for per-vector scaled quantization (VSQ \citep{dai2021vsq}).}
  \label{fig:vsquant}
\end{wrapfigure}
During VSQ \citep{dai2021vsq}, the operand tensors are decomposed into 1D vectors in a hardware friendly manner as shown in Figure \ref{fig:vsquant}. Since the decomposed tensors are used as operands in matrix multiplications during inference, it is beneficial to perform this decomposition along the reduction dimension of the multiplication. The vectorwise quantization is performed similar to tensorwise quantization described in Equations \ref{eq:sf} and \ref{eq:tensor_quant}, where a scale factor $s_v$ is required for each vector $\bm{v}$ that maps the maximum absolute value of that vector to the maximum quantization level. While smaller vector lengths can lead to larger accuracy gains, the associated memory and computational overheads due to the per-vector scale factors increases. To alleviate these overheads, VSQ \citep{dai2021vsq} proposed a second level quantization of the per-vector scale factors to unsigned integers, while MX \citep{rouhani2023shared} quantizes them to integer powers of 2 (denoted as $2^{INT}$).

\subsubsection{MX Format}
The MX format proposed in \citep{rouhani2023microscaling} introduces the concept of sub-block shifting. For every two scalar elements of $b$-bits each, there is a shared exponent bit. The value of this exponent bit is determined through an empirical analysis that targets minimizing quantization MSE. We note that the FP format $E_{1}M_{b}$ is strictly better than MX from an accuracy perspective since it allocates a dedicated exponent bit to each scalar as opposed to sharing it across two scalars. Therefore, we conservatively bound the accuracy of a $b+2$-bit signed MX format with that of a $E_{1}M_{b}$ format in our comparisons. For instance, we use E1M2 format as a proxy for MX4.

\begin{figure}
    \centering
    \includegraphics[width=1\linewidth]{sections//figures/BlockFormats.pdf}
    \caption{\small Comparing LO-BCQ to MX format.}
    \label{fig:block_formats}
\end{figure}

Figure \ref{fig:block_formats} compares our $4$-bit LO-BCQ block format to MX \citep{rouhani2023microscaling}. As shown, both LO-BCQ and MX decompose a given operand tensor into block arrays and each block array into blocks. Similar to MX, we find that per-block quantization ($L_b < L_A$) leads to better accuracy due to increased flexibility. While MX achieves this through per-block $1$-bit micro-scales, we associate a dedicated codebook to each block through a per-block codebook selector. Further, MX quantizes the per-block array scale-factor to E8M0 format without per-tensor scaling. In contrast during LO-BCQ, we find that per-tensor scaling combined with quantization of per-block array scale-factor to E4M3 format results in superior inference accuracy across models. 



\end{document}
