\section{Related Work}
\label{sec:related}
% 
\textbf{Vision-Language Benchmarks.} Existing benchmarks for MLLMs cover various tasks, from college-level subject knowledge to advanced mathematical and multi-step reasoning.
%
Massive Multi-discipline Multimodal Understanding~\cite[MMMU, ][]{Yue_2024_CVPR} tests deliberate reasoning across 11.5K multimodal questions drawn from college exams, quizzes, and textbooks, spanning disciplines such as Art, Business, Science, Health, Social Science, and Engineering.
%
MathVista~\cite{lu2024mathvista} gauges mathematical reasoning within visual contexts, while GeomVerse~\cite{kazemi2024geomverse} evaluates geometry-based problem-solving in MLLMs.
%
ReMI~\cite{kazemi2024remi} focuses on multi-image reasoning, encompassing diverse domains, including math, physics, logic, code, table/chart comprehension, and spatio-temporal tasks.
%
Despite this breadth, none of these datasets specifically targets analogue clock and calendar interpretation.
%

\textbf{Analogue Clock and Dial Reading.}
%
Reading analogue clocks is a complex cognitive task that engages multiple mental processes~\cite{DBLP:journals/corr/abs-2410-11756}.
%
It involves several key cognitive components: visuospatial skills for understanding spatial relationships between clock elements, executive functioning for planning and reasoning, working memory to maintain mental representations of time concepts, and sustained attention to process the information accurately~\cite{DBLP:journals/corr/abs-2410-11756,5673a1d18644461d895f9b7c6ecfe2e1}.
%
\citet{yang2022its} provide a comprehensive framework for reading analogue clocks in natural images and videos, introducing a synthetic data generation pipeline which can produce a wide variety of clock images reflecting the challenges encountered in real-world scenes.
% 
Recently, \citet{deitke2024molmopixmoopenweights} introduced a large-scale synthetic dataset of analogue clocks comprising images rendered from different watch models.  However, the dataset's generation pipeline is not publicly available, making it difficult to reproduce.
% 
\citet{ghosal2024languagemodelspuzzleprodigies} propose a puzzle-based task that includes clock puzzles to identify when an event occurred or will occur, given a starting time and an elapsed or future duration. 
%
Another line of work focuses on automatic dial or gauge meter readings.
%
The solutions proposed for dial reading rely on neural models~\cite{Alexeev20}, projective transforms~\cite{bao19}, and virtual dataset generators~\cite{cai20, Howells_2021_CVPR}, which produce accurate results for gauges with known shape and style.
%
%