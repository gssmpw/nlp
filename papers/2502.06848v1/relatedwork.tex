\section{Related Work}
\label{related_work}

Graph Neural Networks (GNNs) have emerged as a powerful tool for simulating complex physical systems, particularly on unstructured meshes~\cite{meshgraphnets, GraphNetworkDiscontinuous, GraphNetworkConstraint, allen2022FIG, allen2022physical}. 
However, these methods predominantly rely on supervised training, which requires extensive annotated data. Common approaches involve generating data through analytical solvers like OpenFOAM~\cite{OpenFOAM} and ArcSim~\cite{arcsim}. Additionally, some works use real-world observations to train models~\cite{particle-based-RGB, allen2022FIG}.
Early work, such as \MGN ~\cite{meshgraphnets}, adapts the Encoder-Process-Decode architecture~\cite{GNS} to mesh data, with the Process module implemented as a GNN for effective message passing. 
Variants like EA-GNN and M-GNN~\cite{gnnunet} introduce enhancements such as virtual edges and multi-resolution graphs to improve efficiency and handle long-range interactions. 
Additionally, the transformer architecture has been explored in mesh-based physical simulations. Hybrid models like the GMR-Transformer-GMUS~\cite{TemporalAttention} and HCMT~\cite{HCMT} combine GNNs to learn local rules and transformers to capture global context and long-term dependencies over roll-out trajectories. 
Unlike most methods that directly predict future states from input data, C-GNS~\cite{GraphNetworkConstraint} employs a GNN to model system constraints and computes future states by solving an optimization problem based on these learned constraints. 

Transfer learning, which transfers knowledge from a source domain to a target domain, has gained prominence in deep learning for improving performance and reducing the need for annotated data ~\cite{TLResNet1, TLResNet2, GPT3, llama}.
Strategies typically involve parameter control, either by sharing parameters between models or enforcing similarity through techniques like $l^2$-norm penalties~\cite{TLsurvey, gouk2020distance, xuhong2018explicit}.
These approaches have proven effective in computer vision and natural language processing. However, the application of transfer learning to GNN-based physics simulations remains largely unexplored.