\documentclass{article}
\usepackage[utf8]{inputenc}
\usepackage{amsmath}
\usepackage{amssymb}
\usepackage{amsfonts}
\usepackage{amstext}
\usepackage{amsthm}
\usepackage{authblk}
\usepackage{xcolor} %For color commenting

\usepackage{tikz,lipsum,lmodern}
\usepackage[most]{tcolorbox}
\DeclareMathOperator{\Span}{span}
\newcommand{\R}{\mathbb{R}}
\newcommand{\Z}{\mathbb{Z}}
\newcommand{\Q}{\mathbb{Q}}
\newcommand{\N}{\mathbb{N}}
\newcommand{\ora}{\overrightarrow}
\newcommand{\mdd}[1]{{\color{red} MDD: #1}}

\title{Mesoscale Modeling of an Active Colloid’s Motion}
\author[$\dagger,*$]{Matthew Dobson}
\author[$*$]{David Masse}
\date{\today}
\affil[$*$]{\small{Department of Mathematics and Statistics, University of Massachusetts Amherst, USA}}
\affil[$\dagger$]{\small{Corresponding author: dobson@umass.edu}}
%\setlength{\parindent}{0pt}


\begin{document}

\maketitle

\begin{abstract}
This paper uses Cahn-Hilliard equations as a mesoscale model of the motion of active colloids.  The model attempts to capture the driving mechanisms and qualitative behavior of the isotropic colloids originally proposed by J. Decayeaux in 2021. We compare our model against the single colloid behavior presented in that work, as well as against multi-colloid systems. 
\end{abstract}

\section{Introduction}
Active colloids have received continued research interest due to their potential utility in a variety of fields including biomedical~\cite{Kaew1}, materials  science~\cite{Duan1}, and industry~\cite{Walt1}. Their dynamics have been explored both numerically and through derivation of equations of motion from first principles~\cite{Ram1,Lan1,Lieb1,Robert1}. The primary characteristic of active colloids is their enhanced diffusion compared to inactive ones. A variety of mechanisms exist to produce this effective propulsion, such as the miniature motors used by micro-organisms and the chemically induced propulsion of Janus spheres. These mechanisms continue to be studied both in experiment and simulation.~\cite{Bish1,Ebens1,Su1} A common theme among these mechanisms is that they require some asymmetry in the colloid or propulsion mechanism to produce the enhanced diffusion effect. 

Recently, a microscopic model of an isotropic colloid has also been demonstrated to achieve enhanced diffusion. This work by Decayeux et al.~\cite{Deca21,Deca2} demonstrates enhanced diffusion with numerical simulations and posits effective equations of motion. Their model has a large parameter space to explore and many potential extensions. Of particular interest is the number of colloids simulated, as their original work used only used a single colloid particle. Performing such simulations remain computationally costly, and so a mesoscale model demonstrating the same behavior is sought, allowing for quicker simulations with more colloids and at larger scale. Here we propose a model for these isotropic colloids using the Cahn-Hilliard equations. Results of the implementation of this model are presented, along with comparisons to the molecular dynamics model used in the literature.

\section{Problem Description}
Let us first give a brief summary of the physical setup and main results of~\cite{Deca21} along with the simulation parameters we use for the current work. Their setup consists of a number of identical solute particles and a single colloid particle immersed in a bath in a square two dimensional domain with periodic boundary conditions. All particles are subject to overdamped Langevin dynamics with interactions mediated by either a Weeks-Chandler-Anderson (WCA) or Lennard-Jones (LJ) potential. Therefore, if $\mathbf{r}_i$ is the position of particle $i$, 
$U(\mathbf{r}_i - \mathbf{r}_j)$ is the potential between two particles, and $\mathbf{\eta}_i$ the normally distributed random force acting on particle $i$, the overdamped Langevin dynamics of the system are written as~\cite{Deca21}:
\[
\mathbf{\dot{r}}_i(t) = -\frac{D_i}{k_B T}\sum_{i\neq j}\nabla U(\mathbf{r}_i - \mathbf{r}_j) +\sqrt{2D_i}\mathbf{\eta}_i (t)
\]
Using a forward Euler integration scheme, our simulation model is given by:
\[
\mathbf{r}_i(t + \Delta t) = \mathbf{r}_i(t) + \frac{D_i}{k_B T}\sum_{i\neq j}\nabla U(\mathbf{r}_i - \mathbf{r}_j)\Delta t +\sqrt{2D_i\Delta t}\mathbf{\eta}_i
\]
Here $D_i$ is the bare diffusion coefficient of the particle, representing its diffusivity if it were the only particle in the bath. The parameters $k_B$ and $T$ are the Boltzmann Constant and temperature respectively. The diffusion coefficient of the solute particles is taken as 1, while for the colloid it is 1/5. 

The pairwise inter-particle potential depends on the types of particles involved. Initially all solute particles are of type `A' and repulse each other. However, the colloid is endowed with a circular region of influence around it that transforms, at a chosen rate, normally repulsive solute particles into type 'B' particles which attract one another. Interactions between differing solute particle types are still repulsive, and both types repulse the colloid. Our present simulations will have multiple colloids, and these also repulse each other in the same manner. In all cases the repulsive forces arise from the WCA potential. Outside the region of influence, particles return to being type 'A' again at a chosen rate. The reaction rates are assumed to be identical, and are taken to be 10 per unit time, with reaction radius of 5 length units. The attractive force between two type 'B' solute particles is mediated by the LJ potential. We may write these potentials in the following forms:
\[
U_{WCA}(r_{ij}) = 4\epsilon' \left[\left(\frac{d_{ij}}{r_{ij}}\right)^{12} - \left(\frac{d_{ij}}{r_{ij}}\right)^6 \right] + \epsilon'
\]
\[
U_{LJ}(r_{ij}) = 4\epsilon \left[\left(\frac{d_{ij}}{r_{ij}}\right)^{12} - \left(\frac{d_{ij}}{r_{ij}}\right)^6 \right]
\]
Here the epsilons are parameters for the specific particles being simulated. We set $\epsilon'=10k_B T$ and $\epsilon=3k_B T$. The same values are used for all interactions. Finally $d_{ij} = (\sigma_i + \sigma_j)/2$, where $\sigma$ represents the particle diameter. In our simulations the solute particle diameter, also taken to be the characteristic length scale of the system, is 1, and for colloids it is 2. 

The region the simulation is taking place in is 70 length units in side length, with periodic boundary conditions. The characteristic time of the model is taken to be $\tau=\sigma_A^2/D_A$, with time steps $\Delta t=0.00003\tau$, and total time being $1500\tau.$ For our present simulations, the region is filled with 525 solute particles with randomly assigned positions.

The original study demonstrated that the colloid in this setup could experience enhanced diffusion compared to a non-reactive colloid so long as the above parameters were within a certain range. Furthermore, the effective equations of motion of the colloid matched those of a standard active Brownian particle~\cite{Deca21}.  The model has a number of parameters that one might want to experiment with.  For example, if the density of solute particles in the domain is high, then the colloid quickly becomes surrounded by a ring of solute particles and the enhanced diffusion effect is lost. In fact, the diffusion can be less than that of an inactive particle in an otherwise identical setup. Similarly if the reactive region of the colloid is very large, then solute particles will clump around each other but not necessarily in the region of the colloid, so their influence will be greatly diminished. See Appendix A for a comparison of a colloid with infinite reaction radius to one that is inactive. Another parameter not considered in the original paper is the number of colloids. While one colloid can be effectively modeled as an active Brownian particle, it would be surprising if a collection of colloids can be effectively modeled as a collection of independent active Brownian particles, given the complex geometrical arrangements of colloids and solute particles that can be generated. 

Testing all these possibilities is computationally costly for large numbers of particles, and as these are stochastic simulations, many trials are needed for each set of parameters. A natural question then is if there is a mesoscale model that can effectively capture the dynamics of this system over a wide range of parameters. Here we propose a model using the Cahn-Hilliard equations, which are well suited for capturing the separation of the solute particles into high and low density regions.

\section{Proposed Model}
The reactive region of the colloid has the effect of concentrating solute particles locally around it. Outside of this region the solute particles separate again and dissipate due to random motion. So the dynamics of our mesoscale model must be able to capture these concentrating and dissipating effects, and do so in relation to the changing position of the colloid particles. This focus on concentration inspires the choice of the Cahn-Hilliard equations to be a base for our model. Recent work has investigated interactions between active colloids and a system governed by Cahn-Hilliard dynamics;~\cite{Diaz1} the novelty of our model is how we use the Cahn-Hilliard dyanmics to drive the colloids themselves. The Cahn-Hilliard equations govern the phase separation of a binary mixture, and are typically written in terms of the concentration of one of the components. For our purposes we interpret this concentration as that of the solute particles. 

We will use a form of these governing equations from ~\cite{hill1,yama1}, which will allow us to change the behavior close to colloid in a particularly simple way. Let $c$ be the concentration of solute particles, $\mu$ be the diffusion potential, and $M_c$ be the diffusive mobility, then we have the following differential equation governing the time evolution of the concentration:
\[
\frac{\partial c}{\partial t} = \nabla\cdot\left(M_c\nabla\mu\right)
\]
Let us first focus on $\mu$, which is what we will manipulate to induce the desired behavior in our model. Here $\mu$ is given by:
\[
\mu=RT\left[ \log(c)-\log(1-c)\right]+L(1-2c)-a_c\nabla^2c
\]
Here $R=8.314\hspace{0.1cm}J/(mol K)$ is the Gas Constant, $T$ is the temperature, which for our numerical simulations is 673 K. The parameter $a_c$ is called the gradient energy coefficient, and in our simulations $a_c = 3\times 10^{-14}\hspace{0.1cm}Jm^2/mol$. 
$L$ is a parameter that depends on the constituent particles, referred to as the atomic interaction parameter or regular solution constant.~\cite{Diaz1}~\cite{yama1}
Typically this would be constant, but for our purposes $L$ will depend on the position of the colloid, and will be the source of the reactive effect in our mesoscale model. To see why a varying $L$ gives us a simple description of the effect we are trying to reproduce, let us consider the chemical potential of our binary mixture:
\[
g_{chem}=RT\left[c \log(c)-(1-c) \log(1-c)\right]+Lc(1-c)
\]
Note that the first 3 terms of $\mu$ are the partial derivative of $g_{chem}$ with respect to $c$. Let us define $L_0=14943\hspace{0.1cm}J/mol$ so that $L=L_0$ inside the colloid's region of influence, and $L=L_0/2$ outside of it. Calculating $g_chem$ with these values of $L$, we get the plots in Figure~\ref{fig:pot}, with $L=L_0$ in the left subplot and $L=L_0/2$ in the right subplot. The important feature here is that the left subplot has two stable regions that concentration can fall into, where as the right subplot has only one. By only varying $L$ we have gone from a system with two stable values of concentration to a system with only one.

\begin{figure}
\centerline{\includegraphics[scale=0.49,angle=0]{potentials.png}}
\caption{\label{fig:pot}\small Chemical potentials showing different stable regions. On the left, two stable regions exist, which drives the system to separation into high and low concentration regions. On the right, only one stable region exists, driving the system to a mixed state. The value of parameter L on the right is half of that for the figure on the left, so the dynamics of the system can be controlled by changing only this parameter.}
\end{figure}

 This gives us a natural tool to drive the concentration dynamics we want. In a circular region around the colloids, we simulate the Cahn-Hilliard equations with the $L=L_0$ to promote phase separation into localized regions of high and low concentration within a radius of 4 units of a colloid.  Outside this region at a distance greater than 5 units from a colloid, we use $L=L_0/2$ which drives the system back to a uniform concentration. We vary $L$ linearly from one value to the other near the intersection of these regions, the region between a radius of 4 and 5 from the center of the colloid. Finally we specify the form of $M_c$ as in ~\cite{yama1, Zhu1}, which is a function of $c$ and diffusivity of the constituent particles $D_A$ and $D_B$. $M_c$ is given by:
\[
M_c=\frac{D_A}{RT}\left[c+\frac{D_B}{D_A}(1-c)\right]c(1-c)
\]
In our model, $D_A=0.0001e^{-30000/(RT)} \approx 4.693\times 10^{-7} m^2/s$ where the value 30000 is typical for liquids. The absolute value of $D_A$ though is not of interest as it is canceled out in the characteristic scaling, but the ratio to $D_B$ does have some impact. We choose a fixed value $D_B=D_A/5$. Our model consists of only one particle type, the solute particles, so we may interpret one of these diffusivities as a free modeling parameter.

With this we have a mesoscale model of the solute particles and the concentration dynamics. The colloids are still exposed to random forces of the same scale as in the particle model. Colloids are repulsive up to the same distance as with the WCA potential, though this is accomplished with a simpler force model which allows for larger time steps. Let $d_{colloid} = 2^{1/6}\sigma_{colloid}$, then the force between two colloids is zero if $r_{ij} > d_{colloid}$, and otherwise is given by:
\[
\mathbf{F}_i = 0.15\sum_{i\neq j}\sqrt{\frac{d_{colloid} - r_{ij}}{d_{colloid}}}(\mathbf{r}_i - \mathbf{r}_j)
\]
Hence the force is zero at the cutoff radius, and rises sharply inside the cutoff radius as desired, but does not blow up as with the WCA potential.

To complete our model we need to specify the interaction of the colloid with the concentration field. This is accomplished with the following process. First we calculate the concentration in the region the colloid would move to according to the random force and the force applied by any neighboring colloids. If this concentration is $c < \gamma = 0.45$, the colloid will move accordingly. Otherwise, we implement a Metropolis algorithm, which discourages the movement of the colloid into areas of higher concentration. If the difference in concentration from current position to the pending position is given by $c_{\mbox{\tiny{diff}}}$, and $r$ is a uniformly generated random number, and $\beta$ is a parameter to be chosen, then the colloid moves if $r < e^{\beta c_{\mbox{\tiny{diff}}}}.$

In our simulations $\beta=7.5$. The value of $\gamma$ used for low concentration movement and the value of $\beta$ can be thought of as parameters controlling the coupling strength of the colloid to the concentration field. For example if we had chosen a value of $\gamma$ closer to 1, the particle essentially moves freely independent of the concentration field. We can think of $\beta$ as the ability of the colloid to penetrate dense particle clusters. A final note about the model is that while the colloids have a physical size for the purposes of collisions, the concentration field still exists in the region that the colloid occupies, and the Cahn-Hilliard dynamics behave as normal there. The simulation region is the same as in the particle model, and colloids have the same physical size of 2 units. The initial concentration field is 0.54 with random fluctuations on the order of 0.01. The total simulation time is again $1500\tau$, with time steps of $0.1\tau$

\section{Single Colloid Behavior}
\begin{figure}[ht]
\centerline{\includegraphics[scale=0.575,angle=0]{push_image.png}}
\caption{\label{fig:model1}\small High concentration regions effectively block the colloid from moving in that direction, so the otherwise random colloid motion becomes directed in the opposite direction. The left plot shows the particle setup, and the right shows the Cahn-Hilliard setup, both demonstrate the high concentration region that is the main driver of the enhanced diffusion.}
\end{figure}
Let us examine some simulation results that illustrate how our model captures the correct qualitative behavior of the system, particularly the mechanism of increased diffusion. In the left image of Figure~\ref{fig:model1}, we see a configuration of particles with the colloid in green, attractive solute particles in red, and repulsive solute particles in blue. Note the high density of red particles on one side of the colloid. As the colloid undergoes random motion, it will be obstructed from moving into this region by the inertia of the mass of red particles. This directs the mean motion of the colloid in the opposite direction, upwards in this case. Similarly in the right image of Figure~\ref{fig:model1}, the colloid also undergoes random motion but is discouraged from moving into the high concentration red region, so its mean motion is also directed upwards.

A quantitative indicator of increased diffusivity of the colloid is the mean squared displacement (MSD). In the left subplot of Figure~\ref{fig:MSD1} we compare colloid MSD of the particle and mesoscale models. MSD is computed from the average of 400 trials. The log-log scale plot shows the characteristic shape of an active Brownian particle, with good qualitative agreement, though the particle model has slightly higher average MSD by the end of the simulation time.
\begin{figure}
\centerline{\includegraphics[scale=0.36,angle=0]{msd_compare_1_4.png}}
\caption{\label{fig:MSD1}\small Colloid MSD comparison. On the left is a single colloid simulation, on the right is a four colloid simulation. Both demonstrate the characteristic shape of active brownian particles.}
\end{figure}
 
\section{Multiple Colloid Behavior}
So far we have seen that our model qualitatively captures the mesoscale behavior of a single colloid. Both the mesoscale model and particle model have no restriction to the number of colloids present, so a natural extension is to consider multiple colloids. To do so we first must specify what happens when regions of influence of the colloids overlap. In the particle case no extra considerations are needed, as we are only concerned if solute particles are inside or outside the region of influence. For the mesoscale model, the effect on the concentration field is governed by the closest colloid. 

With this in place, we can re-run our simulations for a varying number of colloids and compare results. The first point of comparison is also MSD. In the right subplot of Figure~\ref{fig:MSD1}, we see the results for 4 colloids.  In this case, we observe similarly good agreement between the mesoscale and particle model results as with the single colloid.

Since all simulations are run for the same duration, we can compare MSD at the final time across the range of colloid quantities. This is shown in figure~\ref{fig:MSD_All}. We see that in the Cahn-Hilliard model MSD falls off too quickly. This is the result of colloids getting confined too strongly in high concentration regions compared to the particle model. This confinement can be seen in the MSD plots for the 32 colloid case in Figure~\ref{fig:MSD_32}. 
\begin{figure}
\centerline{\includegraphics[scale=0.5,angle=0]{all_compare.png}}
\caption{\label{fig:MSD_All}\small MSD after a fixed time for 1, 2, 4, 8, 16, 32 colloids.}
\end{figure}
\begin{figure}
\centerline{\includegraphics[scale=0.44,angle=0]{MSD_32.png}}
\caption{\label{fig:MSD_32}\small MSD comparison for a 32 colloid simulation. Both log-log and linear time scales are presented, to illustrate inaccurate confinement that occurs in the Cahn-Hilliard case.}
\end{figure}
A final point of comparison is for very large numbers of colloids. Here the dynamics of the system change considerably as the colloids are very likely to interact with each other and become confined very quickly. In this regime, we should expect results that recover the Cahn-Hilliard dynamics directly, as the colloids will quickly cluster into regions of low concentration. So then if our model is a proper mesoscale model of the particle model, we should expect the particle model to exhibit behavior similar to the Cahn-Hilliard dynamics. That is, it should converge to a pseudo-steady state of large regions of confined colloids and large regions of low density solute particles. For this simulation we use a larger number of solute particles compared to the original simulations, 1050, to better illustrate the confinement and separation of particle types that occurs. We use 200 colloids in the simulations, with other parameters unchanged. The results are shown in Figure~\ref{fig:many_coll}. The particle model converges quite closely to a Cahn-Hilliard like arrangement, as would be expected based on how it is defined. The particle model shows strong confinement of colloids and phase separation, but lacks the sinuous appearance of the Cahn-Hilliard dynamics.
\begin{figure}
\centerline{\includegraphics[scale=0.45,angle=0]{large_count_1.png}}
\caption{\label{fig:many_coll}\small Comparison involving large numbers of colloids. On the left is a Cahn-Hilliard simulation with no colloids for comparison. In the center is the Cahn-Hilliard colloid model, which we see recovers the characteristic form of the original Cahn-Hilliard equations. On the right is a particle simulation with the same number of colloids. While phase separation is apparent, it is not in the same form as the Cahn-Hilliard model, though perhaps it would have a similar appearance at a different length scale.}
\end{figure}

\section{Conclusions and Future Work}
We  have presented a mesoscale model of an isotropic colloid based on Cahn-Hilliard dynamics. Comparisons of this model against the original particle based model show good qualitative as well as quantitative agreement for low numbers of colloids, up to approximately 8 per simulation area. Deficiencies in our model start to manifest for larger numbers of colloids, where colloids are confined far too quickly compared to the particle model. Still, the phase separation that takes place in the particle model for large numbers of colloids is promising evidence that the Cahn-Hilliard model is a reasonable approach. More work is needed to refine the model to better capture the dynamics. An additional challenge is matching parameters between the two models, as there is not a one-to-one correspondence.
\section{Acknowledgments and Data Availability}
The authors would like to thank Pierre Illien for helpfully providing the original code used in ~\cite{Deca21}, which served as a useful guide in developing our own code base for this work. The data used in this work and the code used to derive it is available upon request.
\section{Appendix A}
Here we present mean square displacement results for a system with a single inactive colloid, as well as for a single colloid with a reaction radius that is effectively infinite. All other parameters are identical. Compared to a system where the colloid has reaction radius 5, as used in our other simulations, we see the final MSD is less than 1/3 with no enhanced diffusion, and that the characteristic form of an active Brownian particle is lost.
\begin{figure}[h]
\centerline{\includegraphics[scale=0.45,angle=0]{baseline.png}}
\caption{\label{fig:Inactive}\small Comparison of MSD for an inactive colloid and one with infinite reaction area, and one with reaction radius 5. This demonstrates that the model parameters must be carefully controlled to achieve the increased diffusion effect.}
\end{figure}
% \bibliographystyle{siam}
% \bibliography{sources}

\begin{thebibliography}{10}

\bibitem{Bish1}
{\sc K.~J. Bishop, S.~L. Biswal, and B.~Bharti}, {\em Active colloids as models, materials, and machines}, Annual Review of Chemical and Biomolecular Engineering, 14 (2023), pp.~1--30.

\bibitem{hill1}
{\sc J.~W. Cahn and J.~E. Hilliard}, {\em {Free Energy of a Nonuniform System. I. Interfacial Free Energy}}, The Journal of Chemical Physics, 28 (1958), pp.~258--267.

\bibitem{Deca21}
{\sc J.~Decayeux, V.~Dahirel, M.~Jardat, and P.~Illien}, {\em Spontaneous propulsion of an isotropic colloid in a phase-separating environment}, Phys. Rev. E, 104 (2021), p.~034602.

\bibitem{Deca2}
{\sc J.~Decayeux, M.~Jardat, P.~Illien, and V.~Dahirel}, {\em Conditions for the propulsion of a colloid surrounded by a mesoscale phase separation}, The European Physical Journal E, 45 (2022), p.~96.

\bibitem{Diaz1}
{\sc J.~Diaz and I.~Pagonabarraga}, {\em Activity-driven emulsification of phase-separating binary mixtures}, Phys. Rev. Let.,  (to appear).

\bibitem{Duan1}
{\sc Y.~Duan, X.~Zhao, M.~Sun, and H.~Hao}, {\em Research advances in the synthesis, application, assembly, and calculation of janus materials}, Industrial \& Engineering Chemistry Research, 60 (2021), pp.~1071--1095.

\bibitem{Ebens1}
{\sc S.~Ebbens}, {\em Active colloids: Progress and challenges towards realising autonomous applications}, Current Opinion in Colloid \& Interface Science, 21 (2016), pp.~14--23.

\bibitem{Ram1}
{\sc R.~Golestanian}, {\em Anomalous diffusion of symmetric and asymmetric active colloids}, Phys. Rev. Lett., 102 (2009), p.~188305.

\bibitem{Kaew1}
{\sc C.~Kaewsaneha, P.~Tangboriboonrat, D.~Polpanich, M.~Eissa, and A.~Elaissari}, {\em Janus colloidal particles: Preparation, properties, and biomedical applications}, ACS Applied Materials \& Interfaces, 5 (2013), pp.~1857--1869.
\newblock PMID: 23394306.

\bibitem{Lan1}
{\sc Y.~Lan, M.~Xu, J.~Xie, Y.~Yang, and H.~Jiang}, {\em Spontaneous symmetry-breaking of the active cluster drives the directed movement and self-sustained oscillation of symmetric rod-like passive particles}, Soft Matter, 19 (2023), pp.~3222--3227.

\bibitem{Lieb1}
{\sc B.~Liebchen and A.~K. Mukhopadhyay}, {\em Interactions in active colloids}, Journal of Physics: Condensed Matter, 34 (2021), p.~083002.

\bibitem{yama1}
{\sc Y.~research group}, {\em Two-dimensional phase-field model for conserved order parameter ({Cahn-Hilliard} equation)}, Self Published,  (2019).

\bibitem{Robert1}
{\sc B.~Robertson, J.~Schofield, P.~Gaspard, and R.~Kapral}, {\em {Molecular theory of Langevin dynamics for active self-diffusiophoretic colloids}}, The Journal of Chemical Physics, 153 (2020), p.~124104.

\bibitem{Su1}
{\sc H.~Su, C.-A. {Hurd Price}, L.~Jing, Q.~Tian, J.~Liu, and K.~Qian}, {\em Janus particles: design, preparation, and biomedical applications}, Materials Today Bio, 4 (2019), p.~100033.

\bibitem{Walt1}
{\sc A.~Walther and A.~H.~E. M{\"u}ller}, {\em Janus particles: Synthesis, self-assembly, physical properties, and applications}, Chemical Reviews, 113 (2013), pp.~5194--5261.
\newblock PMID: 23557169.

\bibitem{Zhu1}
{\sc J.~Zhu, L.-Q. Chen, J.~Shen, and V.~Tikare}, {\em Coarsening kinetics from a variable-mobility cahn-hilliard equation: Application of a semi-implicit fourier spectral method}, Phys. Rev. E, 60 (1999), pp.~3564--3572.

\end{thebibliography}


\end{document}