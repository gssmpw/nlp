Software developers balance a variety of tasks in a workweek, yet the allocation of time often differs from what they consider ideal. Identifying and addressing these deviations is crucial for organizations aiming to enhance both productivity and employee well-being.
In this paper we present the findings from a survey that aims to identify the key differences between how software engineers allocate their time during an ideal workweek versus their actual workweek. Our analysis reveals significant deviations between a developers' ideal workweek and their actual workweek, with a clear correlation: as the gap between these two workweeks widens, we observe a decline in in both productivity and satisfaction. By examining these deviations in specific activities, we assess their direct impact on the developers' satisfaction and productivity. Additionally, given the growing adoption of AI tools in software engineering, both in the industry and academia, we identify specific tasks and areas that could be strong candidates for automation. 
Our work makes three key contributions: 1) We quantify the impact of workweek deviations on developer productivity and satisfaction 2) We identify individual tasks that disproportionately affect satisfaction and productivity. 3) We provide actual data-driven insights to guide future AI automation efforts in software engineering, aligning them with the developers' requirements and ideal workflows to maximize productivity, satisfaction and impact.  