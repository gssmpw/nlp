\subsection{Which tasks do developers want to automate?}

In the survey, we asked the developers to answer the question \textit{``Which of your current tasks would you most like to see automated? Which processes do you think could be enhanced to minimize repetitive work?"} 

We received 242 open-ended responses from the developers. To analyze the responses, we utilized the GPT-4 model to identify an initial set of task categories highlighted in the data. The model identified 17 distinct categories. To ensure comprehensive coverage, we uniformly sampled 60 responses based on their length. Four annotators independently reviewed these responses using a closed-coding approach. 
They assigned labels to each response based on the initial categories suggested by GPT-4, while also identifying any new categories that emerged during the process. 
The inter-rater agreement, measured using the Jaccard similarity metric, was 81.5\%, indicating a substantial level of agreement among annotators. Additionally, through discussions, the annotators agreed to introduce one new category (``Cloud Infrastructure Maintenance''), resulting in a list of 18 categories of tasks that developers would like to automate using AI. 
The remaining 182 responses were distributed equally amongst the four annotators and manually labelled using the extended taxonomy to identify the frequency of each category.

The final categories are summarized in \cref{tab:AutomationTasks}. For each category, we provide the name and a short description.
The table also illustrates the results from the manual annotation process and the frequency of each category. 
Documentation emerged as the most frequent category in our analysis of tasks that could be automated with AI. Developers suggested the use of AI to create and update documentation, maintain team knowledge bases, and also to efficiently understand existing code and systems from the available information. For example, D302 highlighted the importance of efficient documentation by saying: \textit{``I would love to see automated documentation of infrastructure and code. A lot of time is lost trying to understand the current resources that go into securing, building, and deploying our products and trying to understand how the different parts of our code base interact and depend on each other. It'd really lessen the cognitive load if this was well-documented in a way that is automatically kept up-to-date."} Similarly, developer D165 expresses the growing need for automating code documentation to keep pace with the rapid development cycles: \textit{``I would love it if there were some way to automatically generate architectural documentation or either a data flow or call stack diagram based on the structure of the actual code in the repository. While tools like CodeFlow exist for VSCode and Visual Studio, there's a learning curve and the lack of reward for better internal documentation means creating and updating documentation is on a volunteer basis. This, and the rapid pace of development means that onboarding any new engineer or existing engineer to a new area is a time-consuming manual process of last-minute documentation updates, or passing down tribal knowledge verbally. Would be nice if there were some way to tie internal documentation to a particular code base, so that every PR also `refreshes' the internal documentation or diagrams."} 

Following this, several developers emphasized the need for automating tasks associated with the setup and maintenance of development environments, which often involve complex processes such as configuring SSH keys, installing software dependencies, and initializing new development instances. Developer D81 points out the current challenges, stating, \textit{``Setting up the Dev. environment could be way simpler. Right now it is somewhat involved, time consuming, and brittle."} 
Similarly, Developer D403 emphasizes the significant impact of automation on productivity, noting, \textit{``Setting up a development environment takes up a huge amount of time in the development cycle, hence I feel that enhancing the same will greatly boost developer productivity"}


 Following closely are tasks related to  `Write/maintain tests', `Task Tracking \& Backlog Management', `Security \& Compliance', `Incident/Customer Issue Management', and `Communication'. This suggests that developers would like to use AI for tasks that support their core work of writing code.

Following are a few selected responses from developers discussing the activities which they want to see automated using AI. Developer D327 provides an elaborate description of the tasks to automate: \textit{`` b) A lot of my time goes into setting up Dev. environment to start working on a new PR or update old PR and merging the changes and building takes a lot of time (sometimes 2 hours) which hinders my productivity and sometimes is the reason why I procrastinate. c) We have weekly OCE (On-Call Engineering) rotations in our team where we have to create reports of the incidents we receive. I want the report generation to be automated. It’ll be such a big boost in our productivity."}.
