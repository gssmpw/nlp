%% bare_conf.tex
%% V1.4b
%% 2015/08/26
%% by Michael Shell
%% See:
%% http://www.michaelshell.org/
%% for current contact information.
%%
%% This is a skeleton file demonstrating the use of IEEEtran.cls
%% (requires IEEEtran.cls version 1.8b or later) with an IEEE
%% conference paper.
%%
%% Support sites:
%% http://www.michaelshell.org/tex/ieeetran/
%% http://www.ctan.org/pkg/ieeetran
%% and
%% http://www.ieee.org/

%%*************************************************************************
%% Legal Notice:
%% This code is offered as-is without any warranty either expressed or
%% implied; without even the implied warranty of MERCHANTABILITY or
%% FITNESS FOR A PARTICULAR PURPOSE! 
%% User assumes all risk.
%% In no event shall the IEEE or any contributor to this code be liable for
%% any damages or losses, including, but not limited to, incidental,
%% consequential, or any other damages, resulting from the use or misuse
%% of any information contained here.
%%
%% All comments are the opinions of their respective authors and are not
%% necessarily endorsed by the IEEE.
%%
%% This work is distributed under the LaTeX Project Public License (LPPL)
%% ( http://www.latex-project.org/ ) version 1.3, and may be freely used,
%% distributed and modified. A copy of the LPPL, version 1.3, is included
%% in the base LaTeX documentation of all distributions of LaTeX released
%% 2003/12/01 or later.
%% Retain all contribution notices and credits.
%% ** Modified files should be clearly indicated as such, including  **
%% ** renaming them and changing author support contact information. **
%%*************************************************************************


% *** Authors should verify (and, if needed, correct) their LaTeX system  ***
% *** with the testflow diagnostic prior to trusting their LaTeX platform ***
% *** with production work. The IEEE's font choices and paper sizes can   ***
% *** trigger bugs that do not appear when using other class files.       ***                          ***
% The testflow support page is at:
% http://www.michaelshell.org/tex/testflow/



\documentclass[10pt,conference]{IEEEtran} 
% Some Computer Society conferences also require the compsoc mode option,
% but others use the standard conference format.
%
% If IEEEtran.cls has not been installed into the LaTeX system files,
% manually specify the path to it like:
% \documentclass[conference]{../sty/IEEEtran}





% Some very useful LaTeX packages include:
% (uncomment the ones you want to load)


% *** MISC UTILITY PACKAGES ***
%
\usepackage{hyperref}
\usepackage{amsmath}
\usepackage{adjustbox}

\def\BibTeX{{\rm B\kern-.05em{\sc i\kern-.025em b}\kern-.08em
    T\kern-.1667em\lower.7ex\hbox{E}\kern-.125emX}}
    \makeatletter
\newcommand{\linebreakand}{%
  \end{@IEEEauthorhalign}
  \hfill\mbox{}\par
  \mbox{}\hfill\begin{@IEEEauthorhalign}
}
% Heiko Oberdiek's ifpdf.sty is very useful if you need conditional
% compilation based on whether the output is pdf or dvi.
% usage:
% \ifpdf
%   % pdf code
% \else
%   % dvi code
% \fi
% The latest version of ifpdf.sty can be obtained from:
% http://www.ctan.org/pkg/ifpdf
% Also, note that IEEEtran.cls V1.7 and later provides a builtin
% \ifCLASSINFOpdf conditional that works the same way.
% When switching from latex to pdflatex and vice-versa, the compiler may
% have to be run twice to clear warning/error messages.






% *** CITATION PACKAGES ***
%
%\usepackage{cite}
% cite.sty was written by Donald Arseneau
% V1.6 and later of IEEEtran pre-defines the format of the cite.sty package
% \cite{} output to follow that of the IEEE. Loading the cite package will
% result in citation numbers being automatically sorted and properly
% "compressed/ranged". e.g., [1], [9], [2], [7], [5], [6] without using
% cite.sty will become [1], [2], [5]--[7], [9] using cite.sty. cite.sty's
% \cite will automatically add leading space, if needed. Use cite.sty's
% noadjust option (cite.sty V3.8 and later) if you want to turn this off
% such as if a citation ever needs to be enclosed in parenthesis.
% cite.sty is already installed on most LaTeX systems. Be sure and use
% version 5.0 (2009-03-20) and later if using hyperref.sty.
% The latest version can be obtained at:
% http://www.ctan.org/pkg/cite
% The documentation is contained in the cite.sty file itself.






% *** GRAPHICS RELATED PACKAGES ***
%
\ifCLASSINFOpdf
  % \usepackage[pdftex]{graphicx}
  % declare the path(s) where your graphic files are
  % \graphicspath{{../pdf/}{../jpeg/}}
  % and their extensions so you won't have to specify these with
  % every instance of \includegraphics
  % \DeclareGraphicsExtensions{.pdf,.jpeg,.png}
\else
  % or other class option (dvipsone, dvipdf, if not using dvips). graphicx
  % will default to the driver specified in the system graphics.cfg if no
  % driver is specified.
  % \usepackage[dvips]{graphicx}
  % declare the path(s) where your graphic files are
  % \graphicspath{{../eps/}}
  % and their extensions so you won't have to specify these with
  % every instance of \includegraphics
  % \DeclareGraphicsExtensions{.eps}
\fi
% graphicx was written by David Carlisle and Sebastian Rahtz. It is
% required if you want graphics, photos, etc. graphicx.sty is already
% installed on most LaTeX systems. The latest version and documentation
% can be obtained at: 
% http://www.ctan.org/pkg/graphicx
% Another good source of documentation is "Using Imported Graphics in
% LaTeX2e" by Keith Reckdahl which can be found at:
% http://www.ctan.org/pkg/epslatex
%
% latex, and pdflatex in dvi mode, support graphics in encapsulated
% postscript (.eps) format. pdflatex in pdf mode supports graphics
% in .pdf, .jpeg, .png and .mps (metapost) formats. Users should ensure
% that all non-photo figures use a vector format (.eps, .pdf, .mps) and
% not a bitmapped formats (.jpeg, .png). The IEEE frowns on bitmapped formats
% which can result in "jaggedy"/blurry rendering of lines and letters as
% well as large increases in file sizes.
%
% You can find documentation about the pdfTeX application at:
% http://www.tug.org/applications/pdftex

\usepackage{graphicx}
\usepackage{dcolumn}


% *** MATH PACKAGES ***
%
%\usepackage{amsmath}
% A popular package from the American Mathematical Society that provides
% many useful and powerful commands for dealing with mathematics.
%
% Note that the amsmath package sets \interdisplaylinepenalty to 10000
% thus preventing page breaks from occurring within multiline equations. Use:
%\interdisplaylinepenalty=2500
% after loading amsmath to restore such page breaks as IEEEtran.cls normally
% does. amsmath.sty is already installed on most LaTeX systems. The latest
% version and documentation can be obtained at:
% http://www.ctan.org/pkg/amsmath





% *** SPECIALIZED LIST PACKAGES ***
%
%\usepackage{algorithmic}
% algorithmic.sty was written by Peter Williams and Rogerio Brito.
% This package provides an algorithmic environment fo describing algorithms.
% You can use the algorithmic environment in-text or within a figure
% environment to provide for a floating algorithm. Do NOT use the algorithm
% floating environment provided by algorithm.sty (by the same authors) or
% algorithm2e.sty (by Christophe Fiorio) as the IEEE does not use dedicated
% algorithm float types and packages that provide these will not provide
% correct IEEE style captions. The latest version and documentation of
% algorithmic.sty can be obtained at:
% http://www.ctan.org/pkg/algorithms
% Also of interest may be the (relatively newer and more customizable)
% algorithmicx.sty package by Szasz Janos:
% http://www.ctan.org/pkg/algorithmicx




% *** ALIGNMENT PACKAGES ***
%
%\usepackage{array}
% Frank Mittelbach's and David Carlisle's array.sty patches and improves
% the standard LaTeX2e array and tabular environments to provide better
% appearance and additional user controls. As the default LaTeX2e table
% generation code is lacking to the point of almost being broken with
% respect to the quality of the end results, all users are strongly
% advised to use an enhanced (at the very least that provided by array.sty)
% set of table tools. array.sty is already installed on most systems. The
% latest version and documentation can be obtained at:
% http://www.ctan.org/pkg/array


% IEEEtran contains the IEEEeqnarray family of commands that can be used to
% generate multiline equations as well as matrices, tables, etc., of high
% quality.




% *** SUBFIGURE PACKAGES ***
%\ifCLASSOPTIONcompsoc
%  \usepackage[caption=false,font=normalsize,labelfont=sf,textfont=sf]{subfig}
%\else
%  \usepackage[caption=false,font=footnotesize]{subfig}
%\fi
% subfig.sty, written by Steven Douglas Cochran, is the modern replacement
% for subfigure.sty, the latter of which is no longer maintained and is
% incompatible with some LaTeX packages including fixltx2e. However,
% subfig.sty requires and automatically loads Axel Sommerfeldt's caption.sty
% which will override IEEEtran.cls' handling of captions and this will result
% in non-IEEE style figure/table captions. To prevent this problem, be sure
% and invoke subfig.sty's "caption=false" package option (available since
% subfig.sty version 1.3, 2005/06/28) as this is will preserve IEEEtran.cls
% handling of captions.
% Note that the Computer Society format requires a larger sans serif font
% than the serif footnote size font used in traditional IEEE formatting
% and thus the need to invoke different subfig.sty package options depending
% on whether compsoc mode has been enabled.
%
% The latest version and documentation of subfig.sty can be obtained at:
% http://www.ctan.org/pkg/subfig




% *** FLOAT PACKAGES ***
%
%\usepackage{fixltx2e}
% fixltx2e, the successor to the earlier fix2col.sty, was written by
% Frank Mittelbach and David Carlisle. This package corrects a few problems
% in the LaTeX2e kernel, the most notable of which is that in current
% LaTeX2e releases, the ordering of single and double column floats is not
% guaranteed to be preserved. Thus, an unpatched LaTeX2e can allow a
% single column figure to be placed prior to an earlier double column
% figure.
% Be aware that LaTeX2e kernels dated 2015 and later have fixltx2e.sty's
% corrections already built into the system in which case a warning will
% be issued if an attempt is made to load fixltx2e.sty as it is no longer
% needed.
% The latest version and documentation can be found at:
% http://www.ctan.org/pkg/fixltx2e


%\usepackage{stfloats}
% stfloats.sty was written by Sigitas Tolusis. This package gives LaTeX2e
% the ability to do double column floats at the bottom of the page as well
% as the top. (e.g., "\begin{figure*}[!b]" is not normally possible in
% LaTeX2e). It also provides a command:
%\fnbelowfloat
% to enable the placement of footnotes below bottom floats (the standard
% LaTeX2e kernel puts them above bottom floats). This is an invasive package
% which rewrites many portions of the LaTeX2e float routines. It may not work
% with other packages that modify the LaTeX2e float routines. The latest
% version and documentation can be obtained at:
% http://www.ctan.org/pkg/stfloats
% Do not use the stfloats baselinefloat ability as the IEEE does not allow
% \baselineskip to stretch. Authors submitting work to the IEEE should note
% that the IEEE rarely uses double column equations and that authors should try
% to avoid such use. Do not be tempted to use the cuted.sty or midfloat.sty
% packages (also by Sigitas Tolusis) as the IEEE does not format its papers in
% such ways.
% Do not attempt to use stfloats with fixltx2e as they are incompatible.
% Instead, use Morten Hogholm'a dblfloatfix which combines the features
% of both fixltx2e and stfloats:
%
% \usepackage{dblfloatfix}
% The latest version can be found at:
% http://www.ctan.org/pkg/dblfloatfix


% Define the column type for decimal alignment
\newcolumntype{d}[1]{D{.}{.}{#1}}

% *** PDF, URL AND HYPERLINK PACKAGES ***
%
%\usepackage{url}
% url.sty was written by Donald Arseneau. It provides better support for
% handling and breaking URLs. url.sty is already installed on most LaTeX
% systems. The latest version and documentation can be obtained at:
% http://www.ctan.org/pkg/url
% Basically, \url{my_url_here}.




% *** Do not adjust lengths that control margins, column widths, etc. ***
% *** Do not use packages that alter fonts (such as pslatex).         ***
% There should be no need to do such things with IEEEtran.cls V1.6 and later.
% (Unless specifically asked to do so by the journal or conference you plan
% to submit to, of course. )


% correct bad hyphenation here
\hyphenation{op-tical net-works semi-conduc-tor}


\begin{document}
%
% paper title
% Titles are generally capitalized except for words such as a, an, and, as,
% at, but, by, for, in, nor, of, on, or, the, to and up, which are usually
% not capitalized unless they are the first or last word of the title.
% Linebreaks \\ can be used within to get better formatting as desired.
% Do not put math or special symbols in the title.
\title{The Introduction of README and CONTRIBUTING Files in Open Source Software Development}
%Readme and Contributing: The origin and impacts of project governance documents
%Ready to commit? The impact of software contributing guidelines
%What are Readme and Contributing Files Good For?
%What happens when software projects publish readme and contributing files?
% Introducing governance documentation toward encouraging OSS contributions




% author names and affiliations
% use a multiple column layout for up to three different
% affiliations
%\author{Anonymized for Review}

%\author{\IEEEauthorblockN{Matthew Gaughan}
%\IEEEauthorblockA{Northwestern University\\
%Email: gaughan@u.northwestern.edu}
%\and
%\IEEEauthorblockN{Kaylea Champion}
%\IEEEauthorblockA{University of Washington\\
%Email: kaylea@uw.edu}
%\linebreakand
%\IEEEauthorblockN{Sohyeon Hwang}
%\IEEEauthorblockA{Princeton %University\\
%Email: %sohyeon@u.princeton.edu}
%\and
%\IEEEauthorblockN{Aaron Shaw}
%\IEEEauthorblockA{Northwestern %University\\
%Email: %aaronshaw@northwestern.edu}}

% conference papers do not typically use \thanks and this command
% is locked out in conference mode. If really needed, such as for
% the acknowledgment of grants, issue a \IEEEoverridecommandlockouts
% after \documentclass

% for over three affiliations, or if they all won't fit within the width
% of the page, use this alternative format:
% 
\author{\IEEEauthorblockN{Matthew Gaughan\IEEEauthorrefmark{1},
Kaylea Champion\IEEEauthorrefmark{2},
Sohyeon Hwang\IEEEauthorrefmark{3}, 
Aaron Shaw\IEEEauthorrefmark{1}} 

\IEEEauthorblockA{\IEEEauthorrefmark{1}Northwestern University, Evanston, Illinois, USA\\
 Email: gaughan@u.northwestern.edu}

\IEEEauthorblockA{\IEEEauthorrefmark{2}University of Washington, Seattle, Washington, USA}

\IEEEauthorblockA{\IEEEauthorrefmark{3}Princeton University, Princeton, New Jersey, USA}}


% NEW OUTLINE
% 


\maketitle

% As a general rule, do not put math, special symbols or citations
% in the abstract
\begin{abstract}
%TODO
\texttt{README} and \texttt{CONTRIBUTING} files can serve as the first point of contact for potential contributors to free/libre and open source software (FLOSS) projects. Prominent open source software organizations such as Mozilla, GitHub, and the Linux Foundation advocate that projects provide community-focused and process-oriented documentation early to foster recruitment and activity. In this paper we investigate the introduction of these documents in FLOSS projects, including whether early documentation conforms to these recommendations or explains subsequent activity. We use a novel dataset of FLOSS projects packaged by the Debian GNU/Linux distribution and conduct a quantitative analysis to examine \texttt{README} (n=4226) and \texttt{CONTRIBUTING} (n=714) files when they are first published into projects' repositories. We find that projects create minimal \texttt{README}s proactively, but often publish \texttt{CONTRIBUTING} files following an influx of contributions. The initial versions of these files rarely focus on community development, instead containing descriptions of project procedure for library usage or code contribution.  The findings suggest that FLOSS projects do not create documentation with community-building in mind, but rather favor brevity and standardized instructions. 
%rarely set out to build contributor communities, despite the community-first recommendations organizations provide for the development of these documents. The apparent gap between putative understandings of the role that these documents play in project development and the reality of their publication underscores the need for further research into the balance of community and software development in FLOSS projects. 
\end{abstract}

% no keywords




% For peer review papers, you can put extra information on the cover
% page as needed:
% \ifCLASSOPTIONpeerreview
% \begin{center} \bfseries EDICS Category: 3-BBND \end{center}
% \fi
%
% For peerreview papers, this IEEEtran command inserts a page break and
% creates the second title. It will be ignored for other modes.
\IEEEpeerreviewmaketitle

\section{Introduction}
In the early stages of FLOSS project development, the sustainability of the project depends on the recruitment of experienced, committed contributors \cite{xiao_how_2023}. Open source guides propose that public documentation---such as \texttt{README} and \texttt{CONTRIBUTING} files---provides a key entry point, recruitment resource, and mechanism for establishing project goals, norms, and procedures. For example, GitHub notes that \texttt{README} files help maintainers communicate expectations for their projects and manage contributions.\footnote{\href{https://docs.github.com/en/repositories/managing-your-repositorys-settings-and-features/customizing-your-repository/about-readmes}{https://docs.github.com/en/repositories/managing-your-repositorys-settings-and-features/customizing-your-repository/about-readmes}} Prominent community leaders also argue that well-constructed documents facilitate scaling.\footnote{\href{https://github.com/hackergrrl/art-of-readme}{https://github.com/hackergrrl/art-of-readme}} The Mozilla Open Leadership Training site recommends that projects provide explicit contributor guidelines (usually, \texttt{CONTRIBUTING} files) ``that explain how new contributors can help out on your project,''\footnote{\href{https://mozilla.github.io/open-leadership-training-series/articles/building-communities-of-contributors/write-contributor-guidelines/}{https://mozilla.github.io/open-leadership-training-series/articles/building-communities-of-contributors/write-contributor-guidelines/}} and the Cloud Native Computing Foundation claims that such guidelines improve contribution quality.\footnote{\href{https://contribute.cncf.io/maintainers/templates/contributing/}{https://contribute.cncf.io/maintainers/templates/contributing/}} These recommendations from leaders in the FLOSS ecosystem assume that documentation files catalyze growth and contribution activity by attracting and building contributor community.

However, despite prevailing wisdom that ``good'' documentation contributes to successful, sustainable FLOSS projects, empirical study of these documents and their relationship to subsequent project growth and contributions remains scarce. Widespread belief in the value of documentation may lead projects to create files in a ritualistic, hollow way simply because doing so has become the norm. %By evaluating the timing and content of these files at the time of introduction, we can infer FLOSS communities' foundational beliefs of the value of process documentation; though projects evolve throughout development, these perspectives underlie subsequent organizational decision-making.
Alternatively, documentation may only arrive after projects have already become established, emerging as a consequence rather than a cause of growth or success along other dimensions. Further, claims that specific qualities of the content of documentation shape subsequent project activity merit empirical investigation. Understanding the significance of documentation more broadly entails examining both the introduction of documentation as well as whether documentation attributes help explain project outcomes. 
%Alternatively, documentation may not precede the growth of contributions or community, but instead be created in response to influxes of activity. 
%Successful project scaling can require organizational and interpersonal labor to manage user and contributor activity \cite{geiger_labor_2021}, including subsequent maintenance of process documentation files themselves (especially if initial files are poorly formulated). Understanding how, when, and with what kind of content process documentation is first introduced to FLOSS projects therefore offers insight into the significance of these files to projects. 
%Furthermore, prior work on FLOSS project life cycles indicates that only a small proportion of projects attract active contributor bases \cite{schweik_internet_2012}; the marginal effect of documentation may be negligible for most. 

%We are unaware of prior studies that adjudicate between these potential explanations of the relationship between project documentation and life cycles.
In this study, we evaluate the introduction of \texttt{README} and \texttt{CONTRIBUTING} files---two of the most common types of documentation about community procedures---in a large heterogeneous sample of FLOSS projects with diverse approaches to documentation. From an initial population of over 21,000 projects packaged in the Debian GNU/Linux operating system distribution, we identify 4,226 upstream projects with \texttt{README} files and 714 projects with \texttt{CONTRIBUTING} files. We parse the contribution logs for these projects and construct an original panel dataset to examine when they introduce these documents, document characteristics at the point of introduction, and the correlations of those initial characteristics with variations in subsequent project activity. We conduct a mixed methods documentation analysis and multilevel longitudinal analysis to describe how projects introduce documentation. While previous research has analyzed project documentation files, we believe that this is one of the first empirical studies evaluating the initial publication of \texttt{README} and \texttt{CONTRIBUTING} files in open source software.

Our results indicate that while \texttt{README} documents are standard early-stage additions, \texttt{CONTRIBUTING} documents are often published later, following a flurry of contributions. The initial versions of most \texttt{README} documents are very short, less than 160 words and requiring only a few seconds to read. Though the majority of \texttt{CONTRIBUTING} documents are still relatively short (less than 250 words) they tend to be longer than \texttt{README} files. In terms of content, we find that initial versions of the files lack community-related language and are more technical in focus. Regarding claims that introducing documentation will catalyze growth via subsequent contributor activity, we find no evidence to support such causality, though we do find associations between \texttt{CONTRIBUTING} files focusing on code style and subsequent contribution activity.
%In short: project contributors do not use these files as recommended. 


%
%These results contribute a first empirical assessment of the introduction of project documentation in a large sample of FLOSS projects. 
%Critically, these findings do not support the prevailing wisdom that documentation spurs contribution activity. In fact, document publication is often followed by periods of diminished contribution activity.
Critically, we find that document publication often follows periods of increased contribution activity and precedes periods of diminished activity. Thus, our study indicates that projects neither develop \texttt{README} and \texttt{CONTRIBUTING} files consistent with prevailing wisdom nor in a way consistent with a causal relationship to project growth or contribution activity. 

%Multiple explanations may account for the pattern of findings. For example, our life cycle findings suggest that early-stage projects may not seek contributor community growth; instead, project growth and attention may create the need for documentation (rather than the other way around). Further research should test this and prior mentioned explanations of how documentation and community management evolves over project life cycles.

% recommendations for community building sometimes argue that documentation should lead and that project development should follow
% across software development, documentation is the post-hoc getting the house in order, our results show that this trend holds for process documentation - or something



\section{Background \& Related Works}

\subsection{Documentation in FLOSS Development Processes}
Documentation serves as a key means by which FLOSS projects communicate with prospective contributors about the features, goals, and organization of a project. %FLOSS projects, like other instances of commons-based peer production, require self-organized and diversely motivated volunteers to manage their own work \cite{benkler_coases_2002, benkler_peer_2017}. 
The aims and processes that projects pursue, such as formal code reviews, have important consequences for the quality of software that results \cite{bosu_peer_2013}. Projects must balance tensions between reducing barriers to contribution (facilitating more activity from more volunteers) and maintaining the quality of the code that gets contributed \cite{de_stefano_impacts_2022, meneely_secure_2009, meneely_strengthening_2010}. Trustworthy and transparent projects better recruit and retain volunteers \cite{chakraborti_we_2023, gallivan_striking_2001, guizani_attracting_2022}. %
%% Sohyeon's re-arrangement of the previous paragraph, now commented out
As a result, recommendations that projects create friendly, readable, public documentation in order to facilitate newcomer contributions are widespread \cite{abdalla_template_2016}. Previous work has found that even out of date or poorly structured documentation can support continued engagement with the project \cite{forward_relevance_2002} and may help newcomers overcome barriers to entry \cite{forte_defining_2013}. 

%Documentation oriented towards prospective contributors may play a critical role shaping projects. 
\texttt{README} and \texttt{CONTRIBUTING} files have become the overwhelmingly common methods of communicating how a project can be accessed, used, and contributed to \cite{oakley_hackergrrlart--readme_2024}. 
Recommendations that projects create and use \texttt{README} and \texttt{CONTRIBUTING} files early in their life cycles to recruit and retain contributors have become commonplace throughout FLOSS ecosystems and other peer production systems. We provide specific examples and elaborate details of such recommendations below, but they usually entail two key assumptions: first, that project leaders want to grow and sustain a community of contributors; and second, that the files will support sustained growth by offering accessible, well-crafted information about how to get involved, establishing and demonstrating a welcoming community culture. Such claims gain support from prior work illustrating how clear communication of peer production project goals and ways of working may get large, collaborative endeavors off the ground in the first place \cite{hill_essays_2013}.

% While general project governance often exists in \texttt{GOVERNANCE.md} files or external project websites \cite{yan_github_2023}, 

Several recent studies cast doubt on these assumptions. Founders of peer production projects do not always seek to grow large communities \cite{foote_starting_2017}. Likewise, people do not always decide to join communities due to their size or growth \cite{hwang_why_2021}. Among software projects, \texttt{README} files contain more content about the usage of the project than assisting prospective contributors \cite{prana_categorizing_2019}, even though developers evaluate \texttt{README} files when deciding where to submit pull requests \cite{qiu_signals_2019}. While \texttt{CONTRIBUTING} documents often address would-be developers, they rarely focus on the barriers encountered by contributors \cite{fronchetti_contributing_2023}; when they do, contributors rarely adhere to documents' prescribed processes~\cite{elazhary_as_2019}. The apparent gap between the prevailing wisdom about the value of documentation files (i.e., guiding contributions toward building a robust community of contributors) and the empirical analyses of files raises questions about how projects create \texttt{README} and \texttt{CONTRIBUTING} files with regards to the broader prevailing wisdom.

The initial versions of \texttt{README} and \texttt{CONTRIBUTING} files in FLOSS projects offers unique insight into how projects approach such documentation. Prior work has conducted cross-section analyses of the latest versions of documentation files \cite{prana_categorizing_2019, tan_how_2024, fronchetti_contributing_2023}. However, the content and purpose of documentation may change over a project's life cycle. The latest versions cannot explain how or when projects introduced such files, which offers signals as to why projects adopt documentation. The point of introduction of \texttt{README} and \texttt{CONTRIBUTING} files is also significant because the first-version files likely have an out-sized role in shaping later document versions and subsequent organizational processes: prior literature in political science \cite{pierson_path_2000} and management \cite{Sydow_Schreyogg_Koch_2009} illustrates how early actions determine later organizational decisions via path dependence. %The first version of \texttt{README} and \texttt{CONTRIBUTING} files likely has an out-sized role in shaping later document versions and subsequent organizational processes. 

We focus on \texttt{README} and \texttt{CONTRIBUTING} files because these two kinds of documents are well-established, widely used, and the subject of explicit recommendations from both prior research and leading FLOSS organizations. We elaborate on these points below as we describe the two in greater depth to motivate our specific research questions and analyses. Additionally, because \texttt{README} and \texttt{CONTRIBUTING} files are so widespread, they enable analysis at a scale that supports inferences across a large and heterogeneous sample of FLOSS projects. We further discuss this when we introduce our research setting, sample, and analysis approach in the Methods section.  


\subsection{\texttt{README} Documents}

\texttt{README} files have been used in software projects since at least the mid-1970s to centralize project information for users and developers.\footnote{An example from a PDP 10 in 1974 may be found at: \href{http://pdp-10.trailing-edge.com/decus_20tap3_198111/01/decus/20-0079/readme.txt.html}{http://pdp-10.trailing-edge.com/decus\_20tap3\_198111/01/decus/20-0079/readme.txt.html}} 
%Recommendations to publish \texttt{README} files are common.%\footnote{\href{https://www.gnu.org/prep/standards/html_node/Releases.html}{https://www.gnu.org/prep/standards/html\_node/Releases.html}} 
A recent study found nearly ubiquitous adoption of \texttt{README} files across both continuously-active and deprecated projects \cite{coelho_why_2017}. As the ``project's first impression,'' recommendations suggest that \texttt{README} file touch on a wide range of topics, from routing help requests to testing.\footnote{For example, see \href{https://github.com/cfpb/open-source-project-template/blob/main/README.md}{https://github.com/cfpb/open-source-project-template/blob/main/README.md} and \href{https://gitlab.com/tgdp/templates/-/blob/main/readme/guide-readme.md}{https://gitlab.com/tgdp/templates/-/blob/main/readme/guide-readme.md}} 

%Prior empirical work has not assessed how or when these documents are published by their project maintainers. We ask:

%\textbf{RQ1: }\textit{When are \texttt{README} documents introduced to a project?}

%The contents of \texttt{README} files thus matter. 
Open source guides suggest that \texttt{README} documents should be structured, parsimonious, and detailed, providing enough documentation for the reader to adeptly make use of the software and to draw in prospective contributors.\footnote{See: \href{https://www.makeareadme.com/}{https://www.makeareadme.com/}} Prior empirical research finds that projects seem to follow at least some of these recommendations. Cross-sectional analysis has found that though documents rarely follow the structures recommended by GitHub, project contributors eventually develop lengthy \texttt{README} files \cite{liu_how_2022}. Many \texttt{README} files contain functional information regarding how to use the project and what the project accomplishes, but contain little information on community or contributions \cite{prana_categorizing_2019}. In addition, many \texttt{README} files provide sufficient detail to support automated project builds \cite{hassan_mining_2017}; and  documentation of engineering procedure that contains detailed usage examples facilitates higher developer productivity \cite{sohan_study_2017}. In some cases, \texttt{README} documents support contributor recruitment and retention \cite{qiu_signals_2019} and identify project contribution policies \cite{tan_how_2024}. However, while \texttt{README} files may evolve into useful documents that support multiple goals in a software development community, prior work does not explore the contents of \texttt{README} files when they are first created. %Yet a well-crafted \texttt{README} from the start is likely to be impactful. In addition to examining when \texttt{README} files are introduced to projects, we evaluate what content is included in their first versions. %We ask: 

%\textbf{RQ2:} \textit{What do maintainers include in the first versions of their \texttt{README} documents?}

\subsection{\texttt{CONTRIBUTING} Guidelines}

In contrast to the recommendations that \texttt{README} documents act as comprehensive guides to projects, open source guides recommend using \texttt{CONTRIBUTING} documents as focused tools to recruit and structure community contributions.\footnote{For example, contrast \href{https://github.com/cfpb/open-source-project-template/blob/main/CONTRIBUTING.md}{https://github.com/cfpb/open-source-project-template/blob/main/CONTRIBUTING.md} with \href{https://github.com/cfpb/open-source-project-template/blob/main/README.md}{https://github.com/cfpb/open-source-project-template/blob/main/README.md}} %GitHub also prompts repository owners to create contributing guidelines to develop a community.\footnote{\href{https://docs.github.com/en/communities/setting-up-your-project-for-healthy-contributions/setting-guidelines-for-repository-contributors}{https://docs.github.com/en/communities/setting-up-your-project-for-healthy-contributions/setting-guidelines-for-repository-contributors}} %As with \texttt{README} documents, we ask: 
%\textbf{RQ3: }\textit{When are \texttt{CONTRIBUTING} documents introduced to a project?}
CONTRIBUTING documents define the rules and processes of how contributions---primarily code commits---are made to the project.% the primary event in software development and maintenance 
\cite{gousios_exploratory_2014, hassan_studying_2006}. Previous studies have identified useful traits of contributing guidelines for community development, such as the convenient presentation of information~\cite{vincent_deeper_2021}, description of recruitment contributor recruitment processes \cite{fronchetti_contributing_2023}, and description of community mentorship programs \cite{balali_newcomers_2018}. However, although these guidelines may make first impressions for prospective project contributors \cite{qiu_signals_2019}, previous research has found that \texttt{CONTRIBUTING} files often neither address common issues of project contributors nor are adhered to by those contributors \cite{fronchetti_contributing_2023, elazhary_as_2019}. As with \texttt{README} files, we focus on \texttt{CONTRIBUTING} files' introduction to projects because the timing and content of initial versions can offer insight into how projects approach and use such files. %Extending our focus on the early versions of document creation and introduction, we ask: 

%\textbf{RQ4:} \textit{What do maintainers include in the first versions of their \texttt{CONTRIBUTING}  guidelines?}

Finally, for FLOSS projects, a primary governance concern is ensuring that the diverse motivations of project contributors align with volunteering time and energy towards a given project \cite{di_tullio_governance_2013}. The recruitment and management of code contributions are a primary topic of FLOSS project rules \cite{crowston_coordination_2005}. The governance of development processes is crucial to the successful recruitment of volunteer labor: just as the management of code commits can engage contributors, so too can projects demotivate further contributions through governance that is inequitable or overly restrictive \cite{alami_how_2020, steinmacher_almost_2018}. Recommendations for the development of \texttt{README} and \texttt{CONTRIBUTING} files often imply that the introduction of a document will cause the project to attract additional contributions.\footnote{\textit{e.g.}.~, see href{https://gitlab.com/tgdp/templates/-/blob/main/contributing-guide/guide-contributing-guide.md}{https://gitlab.com/tgdp/templates/-/blob/main/contributing-guide/guide-contributing-guide.md}} By analyzing contribution activity that follows the introduction of these files, we offer an initial evaluation of such claims. 
%To offer an initial evaluation of such claims, we ask: 

%\textbf{RQ5:} \textit{How do characteristics of initial \texttt{README} and \texttt{CONTRIBUTING} files files correlate to subsequent community activity?}

\section{Study Design}

We structure our analysis around five research questions to understand the introduction of governance documents: 
\begin{itemize}
    \item \textbf{RQ1: }\textit{When are \texttt{README} documents published to a project?}
    \item \textbf{RQ2:} \textit{What do maintainers include in the first versions of their \texttt{README} documents?}
    \item \textbf{RQ3: }\textit{When are \texttt{CONTRIBUTING} documents published to a project?}
    \item \textbf{RQ4:} \textit{What do maintainers include in the first versions of their \texttt{CONTRIBUTING} guidelines?}
    \item \textbf{RQ5:} \textit{How do characteristics of initial \texttt{README} and \texttt{CONTRIBUTING} files relate to subsequent activity?}
\end{itemize}

Our research questions motivate a large-scale analysis comparing documentation and contribution activity across a heterogeneous sample of FLOSS projects. To pursue this, we created a novel dataset of activity and document data that we collected from FLOSS projects packaged in the Debian GNU/Linux distribution.\footnote{Our data and code are available on the Harvard Dataverse: [\href{https://doi.org/10.7910/DVN/LEFZKR}{https://doi.org/10.7910/DVN/LEFZKR}]} These projects' inclusion in the Debian package distribution indicates that they are all widely used libraries that underpin broader FLOSS ecosystems. We investigate our research questions through three analyses: a longitudinal analysis of document introduction, a descriptive analysis of initial document characteristics, and an analysis of how initial document characteristics relate to later project contribution activity.

\subsection{Setting and Data Collection}

The Debian GNU/Linux ecosystem has a substantial history as a setting for academic research into software engineering communities and their governance \cite{champion_underproduction_2021, claes_historical_2015, ververs_influences_2011, krafft_how_2016}. The Debian distribution is also a source of packages for Ubuntu and several other GNU/Linux distributions. By studying the upstream projects of packages included in Debian, we are studying a sample of FLOSS projects that are developed by heterogeneous communities and organizations and that are important to the broader software ecosystem. Debian also does not specify rigid project governance or documentation requirements, resulting in diverse documentation practices. This is in contrast to other potential samples of projects (used in other prior work) drawn from a given umbrella foundation or ecosystem (e.g. Apache or Eclipse), which must adhere to specific, explicitly defined organizational norms in their development and maintenance. %Sampling within Debian has the additional benefit of enabling us to study communities that use a variety of version control and hosting tools.

We restricted our data set to packages hosted on Debian's Salsa platform and with an upstream project that used git as the primary version control system, regardless of where their upstream repository was hosted (e.g. GitHub, GitLab). The Debian project recommends that all packages within their distribution maintain information regarding upstream project location in the package files themselves, in the \texttt{upstream/package/metadata} directory.\footnote{\href{https://dep-team.pages.debian.net/deps/dep12/}{https://dep-team.pages.debian.net/deps/dep12/} (Archived:\href{https://perma.cc/9J2H-S8VP}{https://perma.cc/9J2H-S8VP})} Nevertheless, compliance with this recommendation remains low across bundled packages, and only one third of eligible Debian packages specified their upstream repository. Moreover we experienced some initial data loss as certain packages' upstream projects required developer credentials, resulting in access limitations. We drew from an initial list of the 21,902 Debian packages studied in Champion and Hill \cite{champion_underproduction_2021} to define an intermediary data set consisting of 5,092 upstream projects.  

Our initial sampling identified projects from the intermediary data set that contained \texttt{README} or \texttt{CONTRIBUTING} files in the root directory of their main branches on March 16, 2024. We collected project commit history with the GitPython library.\footnote{\href{https://gitpython.readthedocs.io/en/stable/intro.html}{https://gitpython.readthedocs.io/en/stable/intro.html}} From our intermediary data set, we were unable to clone 706 projects due to SSH git cloning errors.
Of the 4,386 projects that we were able to access, we identified 85 that lacked either a \texttt{README} or \texttt{CONTRIBUTING} file in their root directory. Using GitPython, we then found the first commit in each project's history which contained either a \texttt{README} or \texttt{CONTRIBUTING} document, irrespective of file type. This step generated a small amount of errors in downloading the document at commit time.
%This sampling generated a data set of 2694 \texttt{README.md} files and 528 \texttt{CONTRIBUTING.md} files. We then collected project commit history using the GitPython library\footnote{\href{https://gitpython.readthedocs.io/en/stable/intro.html}{https://gitpython.readthedocs.io/en/stable/intro.html}} and the CHAOSS GrimoireLab Perceval tool \cite{duenas_perceval_2018}. Using GitPython, we found each commit for which the files of the commit contained either a \texttt{README} or \texttt{CONTRIBUTING} document, irrespective of file type. Thus, though we collect first-version documentation with a range of file types, all project eventually contain documentation with the \texttt{.md} file type suffix; this sampling error is further discussed in Section \ref{sec:limitations}. We used Perceval to search the git commit tree for the first-version of the file to download to our research computer. Errors returned by Perceval in tree traversal led us to drop an additional 414 \texttt{README} files and 76 \texttt{CONTRIBUTING} files that could not be found. %During the file search process, Perceval would occasionally return KeyErrors for the file search stating that the file was not found in the commit tree though the existence of the documentation file is how commit hashes were identified. While we ran multiple iterations of these file searches, these errors resulted in data loss. 
Our final dataset consisted of \texttt{README} (n=4226) and \texttt{CONTRIBUTING} (n=714) files from 4247 projects (693 projects were represented in both datasets) that all use the git VCS, are widely used through Debian packaging, and have accessible upstream project source code. 

%GitPython and Perceval produced contradictory first commit data for 75 \texttt{README} files and three \texttt{CONTRIBUTING} files. As such, we removed these projects from descriptive analysis of the age of projects at document publication. 
%These projects were not removed from either the interrupted time series analysis or the document characteristic analyses. 
For both datasets, we collected weekly commit data for the six months before and after document introduction, as well as new contributor data for the two months before and after document introduction. Preliminary analysis showed large amounts of documentation commit activity in the hours following file introduction. To capture a more stable first version of the document, we collected the contents of the \texttt{README} and \texttt{CONTRIBUTING} file versions that existed one week after file initialization. 

\subsection{Modeling Document Introduction}
\label{sec: mlm-methods}
We construct multilevel longitudinal models to further understand the timing of document introduction and document attributes at the point of introduction (\textbf{RQ1}, \textbf{RQ3}), as well as the relationship of document attributes to subsequent contribution activity (\textbf{RQ5}). To understand when documents are typically introduced into projects, we evaluated distributions of weekly project contribution counts and age around the point at which documents were published. The average distribution of weekly project contributions surrounding document publication is shown in Figure \ref{fig:draft_averages}. Then, we cast our data as weekly contribution counts nested within projects. This results in a hierarchical data structure that requires a mixed effects time series approach \cite{steele_multilevel_2008} in order to account for the dependencies between errors as well as the variance. The mean and variance for weekly commit counts (per project) were 4.31 and 207.76 for projects in our \texttt{README} data set. For projects in our \texttt{CONTRIBUTING} data set, the mean and variance weekly commit counts were 8.37 and 349.06. To account for the shape of the data, we logarithmically transformed our weekly commit counts and used the \texttt{lme4} package in R to fit a negative binomial regression model~\cite{bates_fitting_2015}. 

    \begin{figure}[t!]
    \centering
        \includegraphics[width=\linewidth]{cr-020725-gam-introduction.png}
        \caption[Plot of average (log-transformed) weekly contribution counts over time around the point of document introduction (weeks offset from document publication date) for \texttt{README} (red) and \texttt{CONTRIBUTING} (blue) files. The Y-axis has been scaled to real count values.]{Plot of average (log-transformed) contribution counts over time around the point of document introduction (weeks offset from document publication date) for \texttt{README} (red) and \texttt{CONTRIBUTING} (blue) files.The Y-axis has been scaled to real count values. \footnotemark}\label{fig:draft_averages}
    \end{figure}
\footnotetext{Figure color palettes have been selected to support readability \cite{crameri_scientific_2023}.}

\subsubsection{Multilevel Time Series Model Specifications}
To characterize patterns of project activity around the introduction of the documents, we adopt a modeling strategy derived from Regression Discontinuity (RD) methods \cite{murnane_methods_2011}.\footnote{RD methods are most often used to identify causal effects. However, the assumptions necessary to support causal inference in an RD framework are not satisfied in our setting, where the timing of document introduction could have many causes related to the underlying trend in contribution activity. See \cite{murnane_methods_2011} for more on RD methods.} We identified the optimal bandwidth of 10 weeks for our time series analysis through using the Imbens-Kalyalaraman test through the \texttt{rdd} package \cite{imbens_optimal_2009}.\footnote{\href{https://cran.r-project.org/web/packages/rdd/rdd.pdf}{https://cran.r-project.org/web/packages/rdd/rdd.pdf}} The regression formula for the model of project contribution activity as a function of document publication appears in Equation \ref{eq:readme_rdd}, where for each project $j$, the log transformed commit count ($Y$) for each week ($i$) is a function of the intercept ($\beta_{0j}$), the weekly ($W$) growth rate of contributions ($\beta_{1j}$), the impact ($\beta_{2j}$) of document introduction ($D$), document publication impact ($\beta_{3j}$) on the weekly growth rate ($I:W$) and project age ($A_j$). The formula is the same for the modeling of both \texttt{README} and \texttt{CONTRIBUTING} data. While project age remains fixed, the other variables are unique to each weekly data point in our two and a half month window. We evaluated models' autocorrelation function (ACF) plots and variance inflation factors (VIF) to identify possible violations in our model. For both data sets, the ACF plots revealed autocorrelation in small lag values, with slight autocorrelation over larger lag values; our data show clusters of commit activity over a handful of weeks. Models' VIF scores showed multi-colinearity (VIF values above 5) between the $W$ and $D:W$ variables; given both variables' usage of the same weekly index variable, this is expected and not a threat to model validity. 
\begin{equation}
\centering
            Y_{ij} = \beta_{0j} + \beta_{1j}(W) + \beta_{2j}(D) + \beta_{3j}(D:W) + A_j + r_{ij}
\label{eq:readme_rdd} 
\end{equation}
Each $\beta_{ij}$ can be decomposed into its constituent parts of the true effect $\gamma_{ij}$ and error $u_{ij}$, as shown in Equation \ref{eq:beta_distillation}.   
\begin{equation}
\centering
        \beta_{ij} = \gamma_{ij} + u_{ij}
\label{eq:beta_distillation} 
\end{equation}
%The skew of our weekly contribution count data was abnormal enough to preclude using a traditional linear regression model. We used the \texttt{lme4} package in R fitted with a negative binomial regression model~\cite{bates_fitting_2015}. We log-transformed weekly count data due to strong skew in $Y$, and used a negative binomial model because the variance in $Y$ was still greater than its mean.

%\subsubsection{\texttt{CONTRIBUTING} Documents}
%Our regression equation of \texttt{CONTRIBUTING} documents is the same to that of \texttt{README} files with the introduction of the contributing document represented by $C$. The formula is shown in simplified form in Equation \ref{eq:contributing_rdd}. Like that of \texttt{README} documents, we used the \texttt{lme4} package to fit a mixed-effects negative binomial regression model. 
%\begin{center}
%\begin{equation}
%    Y_{ij} = \beta_{0j} + \beta_{1j}(W) + \beta_{2j}(C) + \beta_{3j}(C:W) + A_j + r_{ij}
%\label{eq:contributing_rdd} 
%\end{equation}
%\end{center}
\subsection{Descriptive Document Analysis}
\label{sec: ta-methods}
To understand document contents at the point of introduction (\textbf{RQ2}, \textbf{RQ4}), we performed computational grounded theory \cite{nelson_computational_2020} to study the first versions of \texttt{README} and \texttt{CONTRIBUTING} files. Given that prevailing sentiment (as articulated in FLOSS community guides) supposes that reader-friendly documents are useful for recruiting contributor activity, we calculated readability metrics for document text. To study the subject matter of documents, we used Latent-Dirichlet Allocation (LDA) topic models. 

We selected three readability metrics to evaluate document form: the Flesch reading ease score, a well-established general setting readability metric \cite{flesch_new_1948}; the linsear write formula, developed for the evaluation of technical documents \cite{klare_assessing_1974}; and the McAlpine EFLAW scoring system \cite{mcalpine_global_1997}, which evaluates the readability of English-language documents for readers whose primary language is not English. These widely used readability metrics provide a general assessment of document characteristics. We used the Python \texttt{textstat} library to calculate all three metrics.\footnote{\href{https://github.com/textstat/textstat}{https://github.com/textstat/textstat}}

To evaluate documents' subject matter, we tuned an LDA model implemented with the Gensim topic modeling library\cite{rehurek_software_2010}. For document stopwords, we used the Natural Language Toolkit English stopwords extended with the 20 most common terms across documents\footnote{These are: ["http", "com", "www", "org", "file", "code", "time", "software", "use", "user", "set", "line", "run", "source", "github", "lineno", "python", "php", "ruby", "api"].} \cite{bird_natural_2009} and stripped any presentation styling (e.g. Markdown, HTML). We used 
Greene et al.'s \cite{greene_how_2014} approach to select LDA topic counts through term-centric stability analysis; producing optimal topic counts of 9 for \texttt{README} documents and 5 for \texttt{CONTRIBUTING} documents. For other LDA model parameters (e.g. learning decay, batch size), we implemented a grid search to optimize other variables for model interpretability.

We labeled LDA topics with natural language descriptors through inductive thematic coding \cite{creswell_research_2009}. Three coauthors independently coded themes from modeled topics and reached consensus on topic descriptors. From these analyses we report topic labels, top 10 keywords from each topic, along with the 25th percentile, median, and 75th percentile of intra-document topic coefficient distributions. 


\subsection{Modeling Document Characteristics and Project Outcomes}
Using the results from our descriptive analysis of process documents as well as from our multilevel time series models, we test the relationships between document characteristics and project outcomes such as new contributor activity and commit activity in the eight weeks following \texttt{README} and \texttt{CONTRIBUTING} file publication through negative binomial regression models and descriptive analysis of document groupings. 

\subsubsection{Grouping by Coefficient Estimates}
    \begin{figure}[t!]
    \centering
        \includegraphics[width=\linewidth]{cr-0207-contributing-blup.png}
        \caption{Plot of project-level fitted random effects coefficients and estimated 95\% confidence intervals (Y-axis) sorted by coefficient rank (X-axis) from the model of project activity as a function of \texttt{CONTRIBUTING} file introduction. Colors correspond to groupings of projects based on whether the 95\% CI for the random effects coefficient estimate is less than (dark blue), overlapping with (teal), or greater than (lime) zero.} \label{fig:blup_grouping}
    \end{figure}

Mixed-effects models provide point estimates of the coefficient values for each project's random effects; we follow prior work in zoology and ecology and use these coefficients to identify project-level relationships \cite{robinson_that_1991}. Following prior recommendations for caution in using these coefficient estimates \cite{hadfield_misuse_2010, houslay_avoiding_2017}, we use the confidence intervals of our random effect coefficients from our multilevel time series models to categorize projects according to whether they experience activity decline, stability, or growth. We do this by calculating the coefficient estimates for the interaction effect of the document introduction on the weekly growth rate of commit activity. This random effect indicates the relative weekly change in contributions from before and after documents are introduced; it shows how projects sustain week-over-week contribution growth across the introduction of a given file. By using the 95\% confidence intervals of these estimates, we can specifically evaluate the characteristics of projects that have positive post-introduction growth changes ($CI > 0$), negative post-introduction growth changes ($CI < 0$) and null post-introduction growth changes ($0 \in CI$). This categorization is illustrated for our \texttt{CONTRIBUTING} document projects in Figure \ref{fig:blup_grouping}. The singular fit of both our mixed effects models indicate low variance across random effects coefficients. As such, the majority of projects in both data sets are categorized as experiencing null post-introduction growth changes ($0 \in CI$).


\subsubsection{Modeling Project Activity as a Function of Document Characteristics}

To further investigate the relationships between document contents and project outcomes (\textbf{RQ5}), we model the association between topic distributions with the total sums of commits in the eight weeks following document publication. To estimate these relationships, we again use a negative binomial model, as our outcome (project activity) is a count variable with a leftward skew. For fitting purposes, we removed our model intercept. Thus, we interpret our topic coefficients as the direct effects of a given topic on zero commits following document introduction. Given sparse data for new contributors, we did not find any significant relationships between document traits and contributor recruitment.

We evaluated the relationships between project outcomes ($Y$) and the distribution of our LDA model topics within a project's document. For \texttt{CONTRIBUTING} files, the model formula is shown in Equation \ref{eq:contributing_topic_outcome} where $TC1$ - $TC5$ represent the distribution of the corresponding topic (All topic numbers correspond to those provided in Table \ref{tab:lda-grouped-topics}). We used a similar formula for \texttt{README} files, where $TR1$ -$TR9$ represented the distributions of corresponding topics. 

\begin{equation}
\centering
    Y = \beta_{1}(TC1) + \beta_{2}(TC2) + \beta_{3}(TC3) + \beta_{4}(TC4) + \beta_{5}(TC5) + r
\label{eq:contributing_topic_outcome}
\end{equation}

%\subsubsection{Modeling the Introduction of Well-Developed Documents}

%Our research identified that in our data set of firs-version governance files, 40.45\% of project \texttt{README} documents and 13.05\% of project \texttt{CONTRIBUTING} documents would take less than 5 seconds to read. While Section TKTK and TKTK look at the relationship between underdevelopment and projects' contributions, we modeled project activity changes for the subset of projects whose initial governance documents who were developed adhering to recommended substantiality. We evaluated substantial length as the media reading time lengths of the three recommended projects from GitHub's \texttt{CONTRIBUTING} guideline page both document types\footnote{The three projects are \href{https://github.com/github/docs/blob/main}{GitHub Docs}, \href{https://github.com/rails/rails/blob/main}{Rails}, and \href{https://github.com/opengovernment/opengovernment/blob/master}{OpenGovernment}}; for \texttt{README} files, this was 48 seconds, for \texttt{CONTRIBUTING} files, this value was 59 seconds. To further adhere to institutional recommendations, we bounded our sample to projects who addresses the normative topic of the documetns (TR8 and TR4). Excluding projects whose documents did not meet these criteria resulted in new data sets of 116 \texttt{README} documents (5.09\% of the original data set) and 17 \texttt{CONTRIBUTING} documents (3.76\% of the original data set). Then, we used the same statsitical modeling approaches implemented for the larger data set (described in \ref{sec:tktk}. 



%The model formula appears in Equation \ref{eq:readability_outcome}, where a project's logarithmically transformed count of either new contributors or total commit activity for the eight weeks following document introduction ($Y$) relates to, reading time ($RT$), Linsear Write score ($LR$), Flesch Reading Ease score ($FRE$), McAlpine EFLAW score ($ME$), and the project document's word count ($WC$). 

%\begin{center}
%\begin{equation}
%Y = \beta_{0} + \beta_{1}(RT) + \beta_{2}(LR) + \beta_{3}(FRE) + \beta_{4}(ME)  + \beta_{5}(WC) + r
%\label{eq:readability_outcome}  
%\end{equation}
%\end{center}

%We evaluated the relationships between project outcomes ($Y$) and the distribution of our LDA model topics within a project's document. For \texttt{CONTRIBUTING} files, the model formula is shown in Equation \ref{eq:contributing_topic_outcome} where $T1$ - $T4$ represent the distribution of the corresponding topic (All topic numbers correspond to those provided in Table \ref{tab:lda-grouped-topics}). We used a similar formula for \texttt{README} files, where $T1$ - $T4$ and $T6$-$T8$ represented the distributions of corresponding topics; due to the similarity of Topic 5 and Topic 6 across \texttt{README} files in our data set, Topic 5 was withheld to account for the auto-correlation of topic distributions in model fitting. 

%\begin{center}
%\begin{equation}
%    Y = \beta_{0} + \beta_{1}(T1) + \beta_{2}(T2) + \beta_{3}(T3) + \beta_{4}(T4) + r
%\label{eq:contributing_topic_outcome}
%\end{equation}
%\end{center}

% TODO provide any useful details re: how we report/summarize model results, plots, etc.
% Traits are reported with medians
\section{Results}
%TODO, just to reflect the the Background/Methods
%start with descriptive analysis of documents 
% then when/how they're utilized 
%then their associations and relationships 

\subsection{Modeling Activity Around Document Introduction}

%\begin{table}[t]
%\caption{Quartiles for key project age and activity variables. The project age and age at document introduction variables are in years.}
%\centering
%\begin{tabular}{l | d{1} d{1} d{0} }
% & 25\% &50\% &75\%\\
%\hline
%\textbf{\texttt{README}} &   \\
%Age on 06/24/2024 & 10.57 & 13.32 & 16.50\\
%Age at Document Introduction & 0.00  & 0.00 & 0.59 \\
%Average Weekly Commits & 1.16 & 3.29 & 8.33\\
%New Contributors Before &1&1&1\\
%New Contributors After &1&1&3\\
%\hline
%\textbf{\texttt{CONTRIBUTING}} &   \\
%Age on 06/24/2024& 10.43 & 12.83 & 15.87\\
%Age at Document Introduction & 1.91 & 4.84 &  9.44\\
%Average Weekly Commits & 2.10 & 5.10 & 13.18\\
%New Contributors Before &0&2&5\\
%New Contributors After &1&3&7\\
%\end{tabular}
%  \label{tab:project_characteristics}
%\end{table}


\subsubsection{When do projects usually publish \texttt{README} and \texttt{CONTRIBUTING} files?}

\begin{table}[t!]
\caption{Results for two multilevel negative binomial longitudinal models regressing project activity on project age, time, and the introduction of \texttt{README} (first column) and \texttt{CONTRIBUTING} (second column) files. For each variable in both models, we report fitted coefficient values as well as corresponding 95\% confidence intervals (in brackets).}
\setlength{\tabcolsep}{0.4\tabcolsep}
\centering
\begin{tabular}{l r r}
\hline
 & \texttt{README} &\texttt{CONTRIBUTING}\\
\hline
(Intercept)                                           & $0.193^{*}$         & $0.472^{*}$         \\
                                                     & $ [ 0.130;  0.257]$ & $ [ 0.394;  0.550]$ \\
Introduction                                        & $0.473^{*}$         & $0.250^{*}$         \\
                                                     & $ [ 0.409;  0.538]$ & $ [ 0.172;  0.329]$ \\
Week (Time)                                           & $0.273^{*}$         & $0.168^{*}$         \\
                                                     & $ [ 0.246;  0.299]$ & $ [ 0.144;  0.191]$ \\
Project Age                                          & $-0.032^{*}$        & $-0.010$            \\
                                                     & $ [-0.051; -0.013]$ & $ [-0.053;  0.033]$ \\
Introduction:Week                                     & $-0.684^{*}$        & $-0.407^{*}$        \\
                                                      & $ [-0.714; -0.653]$ & $ [-0.443; -0.371]$ \\
\hline
AIC                                                 & $74892.750$         & $20808.134$         \\
BIC                                                  & $75026.103$         & $20919.005$         \\
Log Likelihood                                      & $-37430.375$        & $-10388.067$        \\
Num. obs.                                            & $30778$ \phantom{.000}             & $7551$\phantom{.000}              \\
Num. groups: Project                    & $4226$\phantom{.000}              & $714$\phantom{.000}               \\
%Var: project\_id (Intercept)                                 & $0.361$             & $0.331$             \\
%Var: project\_id before\_after                               & $0.128$             & $0.058$             \\
%Var: project\_id relative\_week                              & $0.046$             & $0.018$             \\
%Var: project\_id before\_after:relative\_week                & $0.212$             & $0.106$             \\
%Cov: project\_id (Intercept) before\_after                   & $-0.212$            & $-0.138$            \\
%Cov: project\_id (Intercept) relative\_week                  & $-0.099$            & $-0.066$            \\
%Cov: project\_id (Intercept) before\_after:relative\_week    & $0.277$             & $0.185$             \\
%Cov: project\_id before\_after relative\_week                & $0.067$             & $0.029$             \\
%Cov: project\_id before\_after before\_after:relative\_week  & $-0.163$            & $-0.078$            \\
%Cov: project\_id relative\_week before\_after:relative\_week & $-0.077$            & $-0.041$            \\
\hline
\multicolumn{3}{l}{\scriptsize{$^*$ Null hypothesis value outside the confidence interval.}}
\end{tabular}
\label{tab:contribution changes}
\end{table}

%Table \ref{tab:project_characteristics} summarizes project age and activity attributes around the point of document introduction. 
Our results show that \texttt{README} and \texttt{CONTRIBUTING} files are introduced at different times in project life cycles. %Projects introduced both document types following local maxima in contribution activity. 
The median time to \texttt{README} file publication was the on the same day of a project's initial commit, but there is a wide range of when projects publish, with a standard deviation over three years. In contrast, \texttt{CONTRIBUTING} files were introduced far later in project lifespans: the median time to \texttt{CONTRIBUTING} file publication (1806 days) was over four years into project development. 

Multilevel time series models of project activity around the point of document introduction suggest that both \texttt{README} and \texttt{CONTRIBUTING} documents are published around local maxima in contribution activity. For both types of files, project contribution activity declines after their introduction; however, subsequent decreases in commit activity may be due to when the documents are published in the project's life cycle (e.g., right before a release) rather than any effect of the files themselves. Table \ref{tab:contribution changes} shows regression model results for both document types. We summarize the results separately here.

\subsubsection{\texttt{README} Introduction}

For \texttt{README} files, the model results in the first column of Table \ref{tab:contribution changes} indicate that following file publication, weekly commit activity declines (the negative coefficient on the interaction of \textit{Introduction:Week}). When we exponentially transform our model values, our results show that the prototypical \texttt{README} publication follows a flurry of initial commits, file publication is then subsequently followed by an average decline of over six commits every week (\textbf{RQ1}). 

\subsubsection{\texttt{CONTRIBUTING} introduction}

Similarly, for \texttt{CONTRIBUTING} files (the second column of Table \ref{tab:contribution changes}), we observe an uptick in activity during the three weeks prior to introduction. Though activity continues to increase in the week of document introduction (the positive coefficient for \textit{Introduction},) we find a continued decline in contribution activity in the subsequent month (the negative coefficient on the interaction of \textit{Introduction:Week}). Our results show that a few years into project development, contributors typically publish \texttt{CONTRIBUTING} guides following a flurry of activity (\textbf{RQ3}), at which point contribution activity diminishes. 

We can put these results into more concrete terms by describing the model fitted values corresponding to contribution activity for a prototypical (mean activity level) project. Five weeks prior to \texttt{CONTRIBUTING} file publication, contributions to the project are steady week over week. Then, projects typically see contribution growth of 3.90 commits per week in the three weeks prior to document publication. In the week of document publication, projects experience a further increase in contributing activity by about 2.61 commits. Commit activity then tapers at a rate of 3.49 commits per week. %With these results, we find no strong causal effect of contributing guideline introduction, resulting in our inconclusive response to \textbf{RQ5}.

%We attribute this to the fact that though the median project in our dataset was created in 2012, contributing guidelines are more recent recommendations than \texttt{README} files. The later introduction of contributing documents in project life cycles offers us more contextual information about how and when projects publish their first set of contributing guidelines. Across\texttt{CONTRIBUTING}document introduction there is a significant uptick in the four weeks prior to document introduction (Effect size: 0.44 log-transformed contributions/week; $p < 0.001$). Similar to our analyses of \texttt{README} documents, in the two months after the introduction of contributing guidelines, contribution activity declined. This result leads us to hypothesize that contributing guidelines are often published in response to increased project activity, not in anticipation or solicitation of such labor. We also found a negative relationship between when the project introduces the document and the project's overall amount of contribution activity; contribution activity decreases the further it is from project initialization. 

%Using our model's fixed effects along with some of our cross-level and descriptive analyses, we can provide a description of the prototypical initial publication of projects' contribution guidelines. Our logarithmic transformation of commit data is on a scale between 0 and 6.745. Commit contributions to the project increase week over week by 0.18 units per week, before an accelerated increase of 0.44 units per week. Project maintainers then publish their contribution guide, a 200 word long document that primarily focuses on structuring the commit contribution process. After publication, contribution activity decreases week over week for the following two months at a rate of -0.061 units per week. 
%\subsubsection{Population changes}Our analysis of how project community populations changed with the introduction of governance documents were largely inconclusive. We were not able to find any significant relationship between the introduction of project \texttt{CONTRIBUTING} documents and new collaborator or contributor activity. We found significant negative relationships ($p<0.01$) for the same population changes for \texttt{README} documents, with the eight weeks post-introduction related to a decrease of -0.32 collaborators and -0.27 contributors; however, with \texttt{README} introductions largely occuring within the first TKTK days of project initialization, these results may reflect context more than document characteristics. 


\subsection{Descriptive Analysis of Document Contents}

%table right here
    \begin{figure}[t!]
    \centering
        \includegraphics[width=\linewidth]{cr-0207-wc-density.png}
        \caption{ Plot of the kernel density of document word counts for first-version \texttt{README} (red) and \texttt{CONTRIBUTING} (blue) files.} \label{fig:document_size}
    \end{figure}
%TODO: table of summary statistics for README introductions and CONTRIBUTING introductions
% To answer RQ1/ RQ3
% Get the averages of when the documents were introduced, etc. 
\subsubsection{Document Readability and Length}
Our quantitative analysis of initial \texttt{README} and \texttt{CONTRIBUTING} files\footnote{As noted earlier in Study Design, we examine the version one week after file initialization to capture a more stable initial version of the document} found that both document types had similar readability and length, though differed on subject matter. The initial versions of both files shared a similar average reading level around a high-school level of difficulty. \texttt{README} files in our data set had median scores of 16.50 for McAlpine EFLAW, 7.14 for Linsear Write, and 50.86 for Flesch Reading Ease. \texttt{CONTRIBUTING} files in our data set had median scores of 18.15 for McAlpine EFLAW, 7.38 for Linsear Write, and 53.10 for Flesch Reading Ease.

For both types of process documents, initial files were fairly short, the median reading time for \texttt{README} files was 14.79s while the median reading time for \texttt{CONTRIBUTING} files was 19.73s. However, more files in our \texttt{README} data set were empty than files in our \texttt{CONTRIBUTING} data set. Figure \ref{fig:document_size} plots the density of the distribution of word count for \texttt{README} and \texttt{CONTRIBUTING} documents. Both distributions exhibit a substantial left-skew with the bulk of each massed below 200 words. Indeed, the plot shows that most \texttt{README} documents are initially published with little to no content at all (fewer than 100 words). 

\begin{table*}[t]
  \centering
    \caption{Manually labeled topics from our fitted LDA model (left column), the top 10 keywords for each topic (second-left column), as well as the interquartile range for intra-document topic distributions for each topic (rightmost columns). The median coefficient distribution values are bolded. }
  \begin{tabular}{l l r r r }
  Topic (Manually Labeled) & Top 10 Keywords & 25\% & 50 \% & 75\%\\
  \hline
    \textbf{\texttt{README}} &   \\
       R1. Usage (graphics) & image, data, key, file, color, option, support, format, default, mode&  0.001 & \textbf{0.004} & 0.090\\
       R2. Usage (architecture) &   data, test, library, object, implementation, support, packet, used, byte, class & 0.001 & \textbf{0.006} & 0.117\\
       R3. Legal &  license, copyright, perl, gnu, free, version, module, public, general, warranty&  0.001 & \textbf{0.006} & 0.108\\
       R4. Code snippets & test, value, function, return, method, class, string, type, object, example & 0.001 & \textbf{0.004} & 0.102 \\
       R5. Configuration instructions & http, git, server, install, client, request, test, version, project, command& 0.001 & \textbf{0.028} & 0.201\\
       R6. Usage (parsing)& json, node, require, string, parser, var, object, parse, function, font &  0.001 & \textbf{0.004} & 0.072\\
       R7. Usage (command-line) &  command, output, option, process, make, program, script, tool, file, linux & 0.001 & \textbf{0.005} & 0.111\\
       R8. Usage (structured text) &  table, html, tag, text, django, xml, example, path, template, default& 0.001 & \textbf{0.004} & 0.081\\
       R9. Installation instructions &   install, make, build, library, version, directory, file, package, window, project& 0.002 & \textbf{0.148}  & 0.491\\
    \textbf{\texttt{CONTRIBUTING}}&   \\
    C1. General contribution instructions& http, project, git, bug, scipy, contributing, request, pull, issue, contribute
 &  0.001 & \textbf{0.002} & 0.022\\
    C2. Code style guide &  build, function, style, make, file, command, test, used, option, variable
 & 0.001 & \textbf{0.005} & 0.115\\
    C3. Contributing procedure & issue, request, pull, change, bug, feature, branch, git, project, make
  & 0.010 & \textbf{0.542} & 0.823\\
    C4. Contributor agreement &  license, contribution, patch, project, submit, test, open, sign, agreement, & 0.001 & \textbf{0.004} & 0.079\\
    C5. Build instructions &    test, git, make, install, version, doc, http, change, release, commit& 0.002 & \textbf{0.034}& 0.315
  \end{tabular}
  \label{tab:lda-topics}
\end{table*}

\subsubsection{Document Topics}

Our LDA topic models help illustrate the subject matter that project contributors include in the initial versions of both file types. Topic descriptions, prominent keywords, and average intra-document topic distributions are shown in Table \ref{tab:lda-topics}. Similar to the cross-sectional findings of Prana et al. \cite{prana_categorizing_2019}, we find that while many first versions of \texttt{README} documents are sparse, the average \texttt{README} document focuses primarily on technical installation and usage instructions with little attention to community development (\textbf{RQ2}). Shown in $TR1$, $TR2$, $TR6$, $TR7$, and $TR8$, given the prevalence of usage instructions in \texttt{README} files, the LDA model differentiated usage instructions by library functions. Moreover, while \texttt{CONTRIBUTING} documents include a lot of information about the technical processes of building a development environment ($TC5$) and the organizational processes of submitting a code change ($TC3$,) we do not find topics about building community (\textbf{RQ4}). 

\subsection{Document Characteristics and Contribution Activity}

%\begin{center}
%    \begin{figure*}[t!]
%        \includegraphics[width=\linewidth]{0715_readability_plots.png}
%        \caption{Plot showing the density for calculated Reading Ease and reading time metrics for both \texttt{CONTRIBUTING} and \texttt{README} documents.} 
%\label{fig:readability_metrics}
%    \end{figure*}
%\end{center}

    \begin{figure}[t!]
    \centering
        \includegraphics[width=\linewidth]{0207-blup-readability-plot.png}
        \caption{ Plot of the kernel densities for Flesch Reading Ease (left-column) and reading time (right-column) metrics for \texttt{README} (bottom row) and \texttt{CONTRIBUTING} (top row) documents. Colors correspond to groupings of projects based on whether the 95\% CI for the random effects coefficient estimate is less than (dark blue), overlapping with (teal), or greater than (lime) zero.} 
        \label{fig:readability_metrics}
    \end{figure}

For both document types, Figure \ref{fig:readability_metrics} shows the kernel density of readability and reading time grouped by random effects coefficient estimates from our multilevel time series models. Substantial variations in readability or file reading time produces divergent density distributions, which are mapped to the area of each color in the plots. Across coefficient groupings, documents were largely similar in form, with most documents short in length and at a high school to college reading level. However, the Flesch Reading Ease density plot for both files show that projects which published simple documents tended to experience post-publication declines in commit activity. 

Table \ref{tab:lda-grouped-topics} presents the manually labeled LDA model topics with mean intra-document topic distributions grouped by projects' random effects coefficient estimates as previously introduced. Each column shows the prototypical topic distribution of the introduced file of projects that experienced relative post-introduction contribution decline ($CI<0$), stability ($CI \in 0$), or increase ($CI > 0$). Projects that experienced a relative increase in contribution activity ($CI > 0$) following \texttt{README} introduction had 
\texttt{README}s focused more on installation instructions ($TR9$). Projects that experienced increased contribution activity following \texttt{CONTRIBUTING} introduction had \texttt{CONTRIBUTING} documents focused on code style guides ($TC2$) and less information about contributor agreements ($TC4$).% than the files of projects who experienced post-introduction activity declines. 

\begin{table}[t]
  \centering
    \caption{Manually labeled topics from the fitted LDA topic models (left column) with the mean intra-document topic distribution grouped by random effects coefficient groupings introduced in Figure \ref{fig:blup_grouping} (right columns).}
  \begin{tabular}{l  r r r}
  Topic (Manually Labeled) & $CI<0$ & $0 \in CI$ & $CI>0$  \\
  \hline
    \textbf{\texttt{README}} &   \\
       1. Usage (graphics) &0.072 & 0.073& 0.073 \\
       2. Usage (architecture) & 0.117& 0.094& 0.111\\
       3. Legal & 0.094 & 0.102 & 0.076\\
       4. Code snippets & 0.056& 0.084 & 0.050\\
       5. Configuration instructions & 0.148& 0.144 & 0.130\\
       6. Usage (parsing) & 0.069& 0.076 & 0.059 \\
       7. Usage (command-line) & 0.124 & 0.093 & 0.093\\
       8. Usage (structured text) & 0.057 & 0.074 & 0.067\\
       9. Installation instructions & 0.262 & 0.259 & 0.339\\
    \textbf{\texttt{CONTRIBUTING}}&   \\
    1. General contribution instructions &  0.090& 0.112 & 0.094 \\
    2. Code style guide &  0.052 &  0.099 & 0.126\\
    3. Contributing procedure & 0.501& 0.451 & 0.535 \\
    4. Contributor agreement & 0.184  & 0.115 & 0.073 \\
    5. Build instructions & 0.173  & 0.222 & 0.172 \\
  \end{tabular}
  \label{tab:lda-grouped-topics}
\end{table}

Fitting negative binomial models on our data, we found significant relationships between document subject matter and subsequent project activity. Table \ref{tab:topics-outcomes} shows the results of our models for both \texttt{README} and \texttt{CONTRIBUTING} data sets.

We found that projects with \texttt{README} files that contain more content pertaining to configuration ($TR5$) and installation ($TR9$) instructions experience more contribution activity in the weeks following document publication than those that do not. Similarly, when \texttt{CONTRIBUTING} files focus on code style guidelines ($TC2$), projects experience more subsequent contributions. Given the variety of possible antecedents of contribution activity, these models do not support causal interpretation. Rather, these results indicate how relative emphasis on certain topics at the point of document introduction is related to higher amounts of contribution activity in the weeks that follow.

\begin{table}
  \centering
    \caption{Fitted negative binomial models regressing project activity (new commits in the eight weeks following document publication) on manually-labeled topics from LDA topic models fitted on \texttt{README} and \texttt{CONTRIBUTING} files. We report 95\% confidence intervals in brackets.}
\begin{tabular}{l r r}
\hline
Topic (Manually Labeled) & \texttt{README} & \texttt{CONTRIBUTING} \\
\hline
\textbf{\texttt{README}} \\
1. Usage (graphics)     & $1.107^{*}$    &   \\
               & $ [0.993; 1.221]$ & \\
2. Usage (architecture)     & $1.221^{*}$      & \\
               & $ [1.130; 1.312]$ & \\
3. Legal       & $0.950^{*}$    &   \\
               & $ [0.863; 1.036]$ & \\
4. Code snippets        & $1.000^{*}$     &  \\
               & $ [0.888; 1.112]$  &\\
5. Configuration instructions        & $1.211^{*}$     &  \\
               & $ [1.141; 1.281]$  &\\
6. Usage (parsing)       & $1.069^{*}$     &  \\
               & $ [0.962; 1.176]$ & \\
7. Usage (command-line)      & $1.096^{*}$      & \\
               & $ [0.998; 1.193]$ & \\
8. Usage (structured text)        & $1.094^{*}$      & \\
               & $ [0.983; 1.205]$  &\\
9. Installation instructions        & $1.154^{*}$      & \\
               & $ [1.109; 1.199]$ & \\
\textbf{\texttt{CONTRIBUTING}} \\
1. General contribution instructions     &   & $1.196^{*}$       \\
             &  & $ [1.040; 1.351]$ \\
2. Code style guide    &  & $1.434^{*}$       \\
             &  & $ [1.254; 1.615]$ \\
3. Contributing procedure    &  & $1.270^{*}$       \\
             &  & $ [1.201; 1.338]$ \\
4. Contributor agreement     &  & $1.012^{*}$       \\
             &  & $ [0.848; 1.176]$ \\
5. Build instructions    &  & $1.243^{*}$       \\
             &  & $ [1.123; 1.362]$ \\
\hline
AIC        	& $14648.140$     	&  $2545.991$     	\\
BIC        	& $14711.630$   	&  $2573.417$      	\\
Log Likelihood & $-7314.070$    	& $-1266.996$    	\\
Deviance   	&  $2385.415$   	& $361.955$     	\\
Num. obs.  	& $4226$\phantom{.000}        	& $714$\phantom{.000}         	\\
\hline
\multicolumn{3}{l}{\scriptsize{$^*$ Null hypothesis value outside the confidence interval.}}
\end{tabular}
  \label{tab:topics-outcomes}
\end{table}



\section{Discussion}
%% The gap between maintainers just trying to publish code projects and the community-centric 
%% results tell us contributing readme files don't function to onboard newcomers to cultivate community 
%% discussion focuses on why is that 
%% contents as indicative of the intentions 
%% large organizations have incentive 
%% things are not doing the things that they assume that they should or would
%% 7/31 Matt: I think the subsections work here as a structure for the updated arguments and what we want to say,

\subsection{Project lifecycles and governance}
The analysis we report contributes the first large-scale empirical study of initial documentation in FLOSS projects. Our results show that FLOSS projects introduce both \texttt{README} and \texttt{CONTRIBUTING} files in ways that contrast with the prevailing consensus about the utility of these documents in attracting contributors to a project. \texttt{README} files are published very early on, while \texttt{CONTRIBUTING} files follow (instead of precede) growth in commit activity. Many \texttt{README} files start off short and focus primarily on project usage and installation. Initial versions of \texttt{CONTRIBUTING} files are longer, and tend to provide technical instructions and specify desired contributions. However, \texttt{CONTRIBUTING} documents also deviate from suggestions and do not include characterizations of contributor community. While some features of \texttt{README} and \texttt{CONTRIBUTING} files relate to variations in subsequent activity, such relationships appear tenuous and we find no clear pattern of specific documentation features and subsequent contribution activity.%of results to support crafting documentation in a specific way so as to optimize subsequent contributions.

Both types of files were relatively short: 1658 \texttt{README} files (39\% of our dataset) would take less than 10 seconds to read, while 187 \texttt{CONTRIBUTING} files (26\% of our dataset) are just as short. Such brevity suggests that at initial publication, maintainers either do not view the contents of these files as valuable tools in the development of their project or simply lack the information to populate the documents. Either scenario implies that when these documents are published, they lack content that prospective contributors look for such as thorough project descriptions or comprehensive contributing guidelines \cite{qiu_signals_2019}. 

Publishing a \texttt{README} file at the very beginning of a project's life may be a widespread norm. Platform incentives like GitHub's ``community standards'' checklist may reward the publication of \texttt{README} files as a signifier of project legitimacy, rather than as the community development tools that the platform wants to promote \cite{kerr_folly_1975}. Yet \texttt{README}s created for such reasons may have diminished returns. Projects that publish short, simple \texttt{README} files tend to experience a \textit{steeper} decline in post-publication commit activity than those who publish more longer, more complex files. Further support from practitioner communities and repository platforms (e.g. tools suggesting document content or structure) may be a helpful intervention in provoking more thoughtful and meaningful documentation earlier in project development. 

We also found that projects introduce \texttt{CONTRIBUTING} files in ways that contradict prevailing wisdom. Crucially, the publication of \texttt{CONTRIBUTING} files often came \textit{after} a rise in commit activity. This may imply that publishing a \texttt{CONTRIBUTING} document after an influx of activity reflects a desire to ``get the house in order,'' formalizing already implemented practices. Our topic analysis found that these files emphasized the functional instructions and rules of contributing procedure, rather than addressing dimensions of community or contributor conduct. 

%Our LDA topics emphasized functional instructions of contributing procedure, rather than dimensions of the community. %Our analysis cannot explain why this pattern emerges or whether other approaches to \texttt{CONTRIBUTING} files would produce different outcomes. 

\subsection{Governance as hygiene, governance as catalyst}

The pattern of our findings suggests that FLOSS projects may produce initial \texttt{README} and \texttt{CONTRIBUTING} documentation as a matter of project hygiene rather than as tools for growing a contributor community. On one hand, this might indicate that FLOSS projects are under-utilizing documentation. Future work examining the evolution of documentation files (e.g., common changes in structure, length, content) would offer valuable insight into whether there are opportunities to support earlier-stage writing and development of these files with tooling and suggestions.  On the other, this might imply that the production of FLOSS is not heavily dependent on documentation for initial recruitment of communities of contributors. Deeper investigation into activity preceding and following introduction of documentation, e.g., the burst of attention before \texttt{CONTRIBUTING} files are initially adopted, would help clarify the relationship of these files to contributor recruitment. 

%\subsection{Developing Towards Software, Not Community}

Our results indicate that the projects in our sample did not create \texttt{README} or \texttt{CONTRIBUTING} files in a manner consistent with the common recommendations that such documentation should be designed to draw in future contributions and community engagement. This is not necessarily harmful to project sustainability. For example, projects may not aim to develop contributor communities to begin with. Prior interview studies have identified multiple FLOSS maintainers who do not accept pull requests and generally seek to keep development work confined to themselves \cite{asparouhova_working_2020}. For these maintainers, open source projects are a way of sharing code that they have written, not a way of collaborating with others. An overwhelming research focus on community may misconstrue the actions of project contributors who are disinterested in collaboration. 
%This is not unique to FLOSS projects among online CBPP communities; many wiki founders do not set out to create expansive contributor communities~\cite{foote_starting_2017}. While we lack insight into the long-term perspectives of the projects in our sample, our results are a reminder that there are many ``open sources.''An overwhelming research focus on community may misconstrue the actions of project contributors who are disinterested in collaboration. 

However, because the projects in our sample are part of the Debian GNU/Linux distribution, they are widely used and there will be downstream consequences if maintenance decays. %\footnote{Debian packages support installations of both \href{https://www.debian.org/users/}{Debian} and \href{https://ubuntu.com/blog/ubuntu-is-everywhere}{Ubuntu}, two widely used operating system distributions.} 
Our results show that community creation and maintenance may not have been a priority at the time documentation was created; even though, at present, these widely-used projects can benefit from the risk management of active contributor communities \cite{champion_sources_2024, champion_underproduction_2021, walden_impact_2020}. These results illustrate a tension at the heart of FLOSS development: for small projects, community management may be burdensome or counterproductive (especially early in project lifecycles). Nevertheless, the work of community management can become critical for the long term success of the project, especially if it happens to grow to larger scale. %The short, technical files that characterize the introduction of documentation in our dataset of now-influential projects may reflect this tension.

Project maturity may be an important factor in shaping \texttt{README} and \texttt{CONTRIBUTING} files. Our findings showed that developers initialized files early in a project's life, with little content (sometimes, blank). Universal recommendations of documentation development may lead projects to publish underdeveloped documents in a rote manner, which could result in path dependencies that influence the characteristics of later versions. Notably, recommendations advocating for the creation of \texttt{README} and \texttt{CONTRIBUTING} files rarely specify the exact moment or phase of a project's life where community-building documentation would be most useful. 
Although we focus on two examples of early governance documentation, open source projects make many early decisions---including decisions about language, architecture, license, bug and feature requests, versioning and releases, project ownership, dependencies, vulnerability reporting, and communication flows with end users. Advice on any of these topics may be ignored or even be counterproductive if the goals of project creators and the realities of project growth do not align with the assumptions of those offering advice about best practices. 
Further specificity as well as empirical evaluation are necessary to support best practices for projects of different age and maturity.

%Advice on any of these topics may be ignored or even be counterproductive if the goals of the project creators are not aligned with the assumptions of those offering advice about best practices.  
%Further specificity to the context-dependence of FLOSS community interventions is needed to generate meaningful best-practices that support sustainable library development for projects of any age or maturity. 


\section{Limitations and Future Work}
\label{sec:limitations}

%generalizeability
%Our study contains several limitations that suggest future areas of research. 
Although the projects we analyze are an accurate snapshot of an important FLOSS ecosystem in current use, the median age of a project in our data set is over twelve years old. 
More recently developed projects or projects created in other ecosystems may introduce \texttt{README} and \texttt{CONTRIBUTING} files differently.
%robustness
%Our data collection methods limit our ability to observe specific details of project activity history. 
Projects can change version control tools, however our data collection did not include any project history prior to the adoption of git. Moreover, we evaluate commit activity in terms of pure size; the projects in our data set did not have many merge commits. However, previous research has identified approaches to contribution classification \cite{fang_novelty_2024}. Further research should identify the impact that contribution guidelines and additional project governance processes have on different kinds of commit activity. 

%validity
Though topic counts were established using term-centric stability analysis, at least one topic generated from our LDA model appears to be ill-defined. Prototypical documents for Topic 7 for the \texttt{README} document data set sometimes lacked topic keywords or semantic themes for the descriptive label of functionality summaries for command-line tools. Moreover, we did not analyze document structure (e.g. headers, outlines) and the possible relationships of those file characteristics to project activity. Given a range of document file types, we did not analyze the use of specific markup languages(e.g. Markdown); the standardized syntax of markup documents may facilitate further insights into document structure. 

%future work
Finally, this work makes novel contributions to the understanding of a critical event in FLOSS project life cycles: the initial publication of two kinds of project documents. However, projects continue to evolve as they age, including with the recent popularity of \texttt{GOVERNANCE} documents \cite{chakraborti_we_2023, yan_github_2023} and codes of conduct. Further research is necessary to understand what kinds of decisions are most influential at which points in a project's life cycle. 

% An example of a floating figure using the graphicx package.
% Note that \label must occur AFTER (or within) \caption.
% For figures, \caption should occur after the \includegraphics.
% Note that IEEEtran v1.7 and later has special internal code that
% is designed to preserve the operation of \label within \caption
% even when the captionsoff option is in effect. However, because
% of issues like this, it may be the safest practice to put all your
% \label just after \caption rather than within \caption{}.
%
% Reminder: the "draftcls" or "draftclsnofoot", not "draft", class
% option should be used if it is desired that the figures are to be
% displayed while in draft mode.
%
%\begin{figure}[!t]
%\centering
%\includegraphics[width=2.5in]{myfigure}
% where an .eps filename suffix will be assumed under latex, 
% and a .pdf suffix will be assumed for pdflatex; or what has been declared
% via \DeclareGraphicsExtensions.
%\caption{Simulation results for the network.}
%\label{fig_sim}
%\end{figure}

% Note that the IEEE typically puts floats only at the top, even when this
% results in a large percentage of a column being occupied by floats.


% An example of a double column floating figure using two subfigures.
% (The subfig.sty package must be loaded for this to work.)
% The subfigure \label commands are set within each subfloat command,
% and the \label for the overall figure must come after \caption.
% \hfil is used as a separator to get equal spacing.
% Watch out that the combined width of all the subfigures on a 
% line do not exceed the text width or a line break will occur.
%
%\begin{figure*}[!t]
%\centering
%\subfloat[Case I]{\includegraphics[width=2.5in]{box}%
%\label{fig_first_case}}
%\hfil
%\subfloat[Case II]{\includegraphics[width=2.5in]{box}%
%\label{fig_second_case}}
%\caption{Simulation results for the network.}
%\label{fig_sim}
%\end{figure*}
%
% Note that often IEEE papers with subfigures do not employ subfigure
% captions (using the optional argument to \subfloat[]), but instead will
% reference/describe all of them (a), (b), etc., within the main caption.
% Be aware that for subfig.sty to generate the (a), (b), etc., subfigure
% labels, the optional argument to \subfloat must be present. If a
% subcaption is not desired, just leave its contents blank,
% e.g., \subfloat[].


% An example of a floating table. Note that, for IEEE style tables, the
% \caption command should come BEFORE the table and, given that table
% captions serve much like titles, are usually capitalized except for words
% such as a, an, and, as, at, but, by, for, in, nor, of, on, or, the, to
% and up, which are usually not capitalized unless they are the first or
% last word of the caption. Table text will default to \footnotesize as
% the IEEE normally uses this smaller font for tables.
% The \label must come after \caption as always.
%
%\begin{table}[!t]
%% increase table row spacing, adjust to taste
%\renewcommand{\arraystretch}{1.3}
% if using array.sty, it might be a good idea to tweak the value of
% \extrarowheight as needed to properly center the text within the cells
%\caption{An Example of a Table}
%\label{table_example}
%\centering
%% Some packages, such as MDW tools, offer better commands for making tables
%% than the plain LaTeX2e tabular which is used here.
%\begin{tabular}{|c||c|}
%\hline
%One & Two\\
%\hline
%Three & Four\\
%\hline
%\end{tabular}
%\end{table}


% Note that the IEEE does not put floats in the very first column
% - or typically anywhere on the first page for that matter. Also,
% in-text middle ("here") positioning is typically not used, but it
% is allowed and encouraged for Computer Society conferences (but
% not Computer Society journals). Most IEEE journals/conferences use
% top floats exclusively. 
% Note that, LaTeX2e, unlike IEEE journals/conferences, places
% footnotes above bottom floats. This can be corrected via the
% \fnbelowfloat command of the stfloats package.




\section{Conclusion}
%In this study we examined how and when FLOSS projects first use process documentation, evaluating whether projects adhere to the popular recommendations for the files' use. 
%Through our mixed methods analysis of a novel data set of software libraries packaged with the Debian GNU/Linux distribution, 
We found that projects largely introduce \texttt{README} files at the beginning of their life cycle, but create \texttt{CONTRIBUTING} documents later, following increased project activity. Moreover, we found that documents are largely focused on the functional processes of usage and contribution, not on the community development for which these files are often recommended. %This gap between widespread advice and general practice deserves further investigation.
Efforts to promote FLOSS project sustainability often emphasize the development of communities. Yet, recommendations that assume that project founders want to create communities (and that process documents can catalyze this) may fail to address the realities of early-lifecycle projects. 
Too much emphasis on community-building could run counter to project founders' and contributors' diverse motives, generating a sense of bureaucracy, complexity, and weighty commitment. This, in turn, could discourage participation. At the same time, our findings suggest that more substantive project documentation may be associated with subsequent project growth. Recommendations for best practices and governance should evaluate these possibilities directly. We hope that our data and findings will bring more research attention to how projects use documentation and how these files impact project health. 



% conference papers do not normally have an appendix


% use section* for acknowledgment
\section*{Acknowledgment}


This work is indebted to the volunteers producing FLOSS who have made their work available for inspection. We also gratefully acknowledge support from the Ford/Sloan Digital Infrastructure Initiative (Sloan Award 2018-113560) and the National Science Foundation (Grant IIS-2045055).  This work was conducted using the Hyak supercomputer at the University of Washington as well as research computing resources at Northwestern University.





% trigger a \newpage just before the given reference
% number - used to balance the columns on the last page
% adjust value as needed - may need to be readjusted if
% the document is modified later
%\IEEEtriggeratref{8}
% The "triggered" command can be changed if desired:
%\IEEEtriggercmd{\enlargethispage{-5in}}

% references section

% can use a bibliography generated by BibTeX as a .bbl file
% BibTeX documentation can be easily obtained at:
% http://mirror.ctan.org/biblio/bibtex/contrib/doc/
% The IEEEtran BibTeX style support page is at:
% http://www.michaelshell.org/tex/ieeetran/bibtex/
%\bibliographystyle{IEEEtran}
% argument is your BibTeX string definitions and bibliography database(s)
%\bibliography{IEEEabrv,../bib/paper}
%
% <OR> manually copy in the resultant .bbl file
% set second argument of \begin to the number of references
% (used to reserve space for the reference number labels box)
\bibliographystyle{IEEEtran}
\bibliography{debgov_bib}




% that's all folks
\end{document}


