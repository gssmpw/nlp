\section{Related work}
\label{sec:Related_work}

\subsection{General Crowd Object Localisation}

In recent years, a range of methods have been developed for addressing small object localisation through an object counting task, where most of them were related to people. These approaches produce a heatmap mask as their output, which can subsequently be subjected to certain thresholding to achieve object localisation~(\cite{liu2019context,li2018csrnet,wang2020distribution}). More recent methods have shifted the focus to crowd counting through object localisation. Authors of~(\cite{song2021rethinking}) proposed a purely point-based framework for joint counting and individual localisation in crowds, employing the matching of a large set of predefined proposals. A similar but more direct approach was presented in~(\cite{liang2022end}), in which the End-to-End transformer model utilises a direct regression approach and produces point coordinates and confidence scores. A different approach, called Crowd Hat, was proposed in~(\cite{wu2023boosting}). The authors use the advantages of feature maps generated by detection-based methods and propose a plug-and-play module that takes the features and generates a response. The most recent approach is STEERER~(\cite{han2023steerer}), a state-of-the-art method for many ground-view object counting and localisation benchmarks. It addresses scale variations caused by peoples' relative positions to the camera by utilising selectively discriminative features from different scales, adopting feature selection, and segregating discriminative and undiscriminative components of lower-scale features. Regrettably, these methods mainly focus on developing a mechanism to deal with this kind of scale variation, which is not present in top-down UAV imagery. Furthermore, object counting, rather than locating objects, is a prior task in these applications.

\subsection{UAV-based Crowd Object Localisation}

Subsequently, with the release of the DroneCrowd dataset~(\cite{wen2021detection}), the initial approaches to detect crowds from drone imagery emerged. The authors of STNNet~(\cite{wen2021detection}) introduced an approach aimed at addressing density map estimation and object localisation within densely populated scenes captured by drones. In their method, the localisation subnetwork encompasses both classification and regression components. For object localisation, they distribute object proposals across pixels. The classification branch predicts the likelihood that a proposal represents an object, while the regression branch focuses on determining the positions of the positive proposals. In MFA~(\cite{asanomi2023multi}), an advanced technique for crowd localisation is introduced, representing the state-of-the-art in this domain. The authors explore two distinct methods for generating feature maps. The first approach involves heatmap generation, which employs UNet~(\cite{unet}) to estimate object positions' heatmap; each peak in this heatmap signifies the object's location. The second method, Motion and Position Map (MPM) encapsulates position and movement direction through sequential frames and behavioural patterns within the sequence data. These methods employ a sliding-window approach, which can result in the loss of the image's global context and lead to lower performance while also increasing the time needed for processing a single image.

\subsection{Aerial Tiny Object Detection}

Numerous tiny object detection methods dedicated to remote sensing tasks have been proposed, especially after the release of the AI-TOD dataset~(\cite{wang2021tiny}) composed of tiled aerial images with the resolution of $800\times800$ pixels containing tiny objects. The authors of~(\cite{wang2110normalized}) introduced the Normalized Wasserstein Distance (NWD) as a novel metric for bounding box similarity, replacing the conventional Intersection over Union (IoU) and significantly enhancing detection results. Subsequently, the Gaussian Receptive Field-based Label Assignment (RFLA)~(\cite{xu2022rfla}) was introduced to achieve better assignment of object bounding boxes. By leveraging Kullback-Leibler Divergence (KLD) and treating bounding boxes as 2D Gaussian distributions, the authors demonstrated improved scale generalisation of objects compared to NWD. Lastly, Swin-Deformable DEtection TRansformer (SD DETR)~(\cite{liao2023transformer}) was introduced to address very tiny (less than four pixels) objects, becoming state-of-the-art on the AI-TOD dataset. Although these methodologies were originally developed for object detection, they can be adapted for localisation tasks.

\subsection{Focal Loss in Object Localisation}
In this section, we also direct the focus on crowd-localisation approaches that incorporate the utilisation of Focal Loss~(\cite{lin2017focal}) within their frameworks, given its inclusion as an integral component of the loss function proposed in this paper. The authors of~(\cite{liang2022focal}) introduce a novel approach that involves the Focal Inverse Distance Transform (FIDT) map for label generation, together with the proposal of an I-SSIM loss for local Maxima detection. Compared to conventional density maps, their FIDT maps offer precise depictions of individuals' positions, particularly in dense regions, mitigating issues related to overlapping. Additionally, the application of the Independent SSIM loss addresses adverse effects, including blurring and the degradation of local structural details. The study most closely aligned with our work is outlined in~(\cite{zhong2022mask}). In this research, the authors introduce the Mask Focal Loss for crowd people counting, highlighting its effectiveness in enhancing the network's ability to discern head regions and predict heatmaps. They identify the versatility of this loss as a cohesive framework applicable to classification losses rooted in heat maps and binary feature maps.