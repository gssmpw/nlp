
\begin{tikzpicture}[every node/.style={draw, minimum width=1cm, minimum height=0.7cm}]

  \begin{table}[h]
    \setlength{\tabcolsep}{3pt}

    \centering
    \begin{tabular}{c|cccl}
    Slice No. & $\mathcal{|S|}$ & $FNR$ & \makecell{Avg. Perf. \\ Degra.} & Slice Description \\
        \hline

    $\mathcal{S}_1$     & 851	& 0.7086	& -0.0643 & \messagebubble{\makecell[l]{\textbf{age}: young \\
        \textbf{clothing-color}: dark-color}} \\
    \hline

    $\mathcal{S}_2$     & 338	& 0.7485	& -0.1042 & \messagebubble{\makecell[l]{\textbf{age}: young \\      
    \textbf{blurry}: true}} \\
    \hline   
    $\mathcal{S}_3$     & 368	& 0.731	    & -0.0867 & \messagebubble{\makecell[l]{\textbf{gender}: female \\
        \textbf{age}: young}} \\
    \hline

    % $\mathcal{S}_4$     & 59	& 0.2373	&-0.0789 & \messagebubble{\makecell[l]{\textbf{gender}: female \\
    %     \textbf{clothing-color}: bright-color}}\\
    % \hline

    % $\mathcal{S}_5$     & 23	& 0.3043	& -0.1460 & \messagebubble{\makecell[l]{\textbf{gender}: female \\
    %     \textbf{hair-color}: blond \\
    %     \textbf{clothing-color}: bright-color \\
    %     \textbf{construction-worker}: false}}\\
    \end{tabular}
    \caption{Quantitative analysis of the top-5 weak slices for the Panoptic-FCN model trained and evaluated on RailSem19 dataset. With the considered cut-off threshold, only 3 slices were found for this experiment.}
    \label{tab:appendix:railsem}
\end{table}
\end{tikzpicture}
