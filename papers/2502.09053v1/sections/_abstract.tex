\begin{abstract}
  Game theory establishes a fundamental framework for analyzing strategic interactions among rational decision-makers. The rapid advancement of large language models (LLMs) has sparked extensive research exploring the intersection of these two fields. Specifically, game-theoretic methods are being applied to evaluate and enhance LLM capabilities, while LLMs themselves are reshaping classic game models. This paper presents a comprehensive survey of the intersection of these fields, exploring a bidirectional relationship from three perspectives: (1) Establishing standardized game-based benchmarks for evaluating LLM behavior; (2) Leveraging game-theoretic methods to improve LLM performance through algorithmic innovations; (3) Characterizing the societal impacts of LLMs through game modeling. Among these three aspects, we also highlight how the equilibrium analysis for traditional game models is impacted by LLMs' advanced language understanding, which in turn extends the study of game theory. Finally, we identify key challenges and future research directions, assessing their feasibility based on the current state of the field. By bridging theoretical rigor with emerging AI capabilities, this survey aims to foster interdisciplinary collaboration and drive progress in this evolving research area. 
    % By synthesizing insights from computational game theory and contemporary AI research, this work aims to stimulate interdisciplinary collaboration and inform the development of robust frameworks for AI-driven strategic decision-making.
\end{abstract}


% Game theory have been employed to boost development of large language models (LLMs) in both theoretical and technical fields. In the mean time LLMs as new game subjects have been put into game scenarios to play and be analyzed. The promotion of games and large language models (LLMs) is bidirectional. Recent studies analyze LLMs in game with numerous dimensions, including evaluation of LLMs' behavioral performance LLMs struggle in matrix games, methods to enhance LLMs' game performance, and how LLMs can serve beyond as a game player. In parallel, game theory, known for its advantages in addressing complex equilibrium problems, game-theoretic issues, and the integration of diverse perspectives, offers a promising guidance for phenomenological understanding LLMs and stimulaing LLM algorithms. More than that, with the deepening of interaction of LLMs and game, there are also original game models that are born LLM related. In this survey, we aim to comprehensively assess the ralationship between currect game and LLMs development. Besides, we propose a new taxonomy of game for LLMs and LLMs for game to systematically categorize related works in this emerging field. Our analysis includes novel frameworks and definitions, highlighting potential research directions and challenges at this intersection. Through this study, we aim to stimulate targeted advancements with game theory and LLM together.