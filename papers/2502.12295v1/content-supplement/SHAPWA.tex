\section{The Tractability of computing SHAP for the class of WAs: Proofs of Intermerdiary Results} \label{app:shapwa}

In this section of the appendix, we provide detailed proofs of intermediary mathematical statements to prove the tractability of computing different SHAP variants on the class of WAs (Theorem \ref{thm:shapwa}). Specifically, we shall provide proofs of three intermediary results: 
\begin{enumerate}
    \item Proposition \ref{prop:efficentoperations} that states the computational efficiency of implementing the projection operation. 
    \item Lemma~\ref{lemma:shapasoperations}, which demonstrates how the computation of both local and global Interventional and Basleline SHAP for the family of WAs under distributions modeled by HMMs can be reduced to performing operations over N-Alphabet WAs.
    \item Proposition \ref{prop:nletterwaconstruction}, which asserts that the construction of N-Alphabet WAs can be achieved in polynomial time, thus enabling the polynomial-time algorithmic construction for both $\texttt{LOC-I-SHAP}(\texttt{WA}, \texttt{HMM})$ and $\texttt{GLO-I-SHAP}(\texttt{WA}, \texttt{HMM})$.
\end{enumerate}

\subsection{Terminology and Technical Background}
The proof of Theorem \ref{thm:shapwa} will rely on certain technical tools that were not introduced in the main paper. This initial section is devoted to providing the technical background upon which the proofs of various results presented in the rest of this section rely.
%will rely on some technical tools that weren't introduced in the main article. This first segment is dedicated to provide the technical background on which relies the proofs of different results presented in the remainder of this sections.

\paragraph{The Kronecker product.} \label{app:sec:ter} 
The Kronecker product between $A \in \mathbb{R}^{n \times m}$ and $B \in \mathbb{R}^{k \times l}$, denoted $A \otimes B$, is a matrix in $\mathbb{R}^{(n \cdot k) \times (m \cdot l)}$  
constructed as follows 
 $$A \otimes B = \begin{bmatrix}
     a_{1,1} \cdot B & a_{1,2} \cdot B & \dots & a_{1,m} \cdot B] \\ 
     a_{2,1} \cdot B & a_{2,2} \cdot B & \dots &  a_{2,m} \cdot B] \\
     \vdots & \vdots & \vdots & \vdots \\ 
     a_{n,1} \cdot B & a_{n,2} \cdot B & \dots & a_{n,m} \cdot B
 \end{bmatrix}$$
 where, for $(i,j) \in [n] \times [m]$, $a_{i,j}$ corresponds to the element in the $i$-th row and the $j$-th column of $A$. A property of the Kronecker product of matrices that will be utilized in several proofs in the appendix is the \emph{mixed-product} property:

 \begin{property} \label{app:eq:mixedproduct}
  Let $A, B, C, D$ be four matrices with compatible dimensions. We have that:
  $$(A \cdot B) \otimes (C \cdot D) = (A \otimes C) \cdot (B \otimes D)$$
 \end{property}

\paragraph{N-Alphabet Deterministic Finite Automata (N-Alphabet DFAs).} In subsection \ref{prop:nletterwaconstruction}, we shall employ a sub-class of N-Alphabet WAs more adapted to model binary functions (i.e. functions whose output domain is $\{0,1\}$)in the proof of Proposition \ref{prop:nletterwaconstruction}. The class of N-Alphabet DFAs can be seen as a generalization of the classical family of Deterministic Finite Automata for the multi-alphabet case. N-Alphabet DFAs are formally defined as follows:
\begin{definition} \label{app:def:naldfa}
A N-Alphabet DFA $A$ is represented by a tuple $\langle Q, q_{init}, \delta, F \rangle$ where:
\begin{itemize}
    \item $Q$ is a finite set corresponding to the state space,
    \item $q_{init} \in Q$ is called the initial state.
    \item $\delta$, called the transition function, is a partial map from $Q \times \Sigma_{1} \times \ldots \times \Sigma_{N}$ to $Q$
    \item $F \subseteq Q$ is called the final state set
\end{itemize}
\end{definition}

Figure \ref{app:fig:nalphabetdfa} illustrates the graphical representation of some N-Alphabet DFAs. Analgous to N-Alphabet WAs, we shall use the terminology \textit{DFA}, instead of 1-Alphabet DFA for $N=1$.

To show how N-Alphabet DFAs compute (binary) functions, we need to introduce the notion of a \textit{path}. For a N-Alphabet DFA $A = \langle Q, q_{init}, \delta, F \rangle$ over $\Sigma_{1} \times \ldots \times \Sigma_{N}$, a valid path in $A$ is a sequence $P = (q_{1} ,\sigma_{1}^{(1)}, \ldots \sigma_{1}^{(N)}) \ldots (q_{L} ,\sigma_{1}^{(L)}, \ldots \sigma_{L}^{(N)}) q_{L+1}$ in $ (Q \times \Sigma_{1} \times \ldots \times  \Sigma_{N})^{*} \times Q$ such that for any $i \in [L]$, $\delta(q_{i} ,(\sigma_{i}^{(1)}, \ldots \sigma_{i}^{(N)})) = q_{i+1}$. Given this definition of a valid path, the N-Alphabet $A$ for a given tuple of sequences $(w^{(1)}, \ldots , w^{(N)}) \in \Sigma_{1}^{*} \times \ldots \times \Sigma_{N}^{*}$ such that $|w^{(1)}| = \ldots = |w^{(N)}|= L $ if and only if there exists a valid path  $(q_{1}, w_{1}^{(1)}, \ldots w_{1}^{(N)}) (q_{2}, w_{2}^{(1)}, \ldots , w_{2}^{(N)}) \ldots (q_{L}, w_{L}^{(1)}, \ldots, w_{L}^{(N)}) q_{L+1}$ such that $q_{1} = 1$ and $q_{L+1} \in F$. For instance, the sequence $abab$ of the 1-Alphabet DFA in Figure \ref{app:fig:nalphabetdfa} is labeled by $1$. Indeed, the valid path $(q_{init}, a) (q_{init},b)(q_{1},a)(q_{2},b)q_{1}$ satisfies these conditions.

\begin{figure}
  \begin{minipage}[t]{0.3\linewidth}
        \centering
        
        \begin{tikzpicture}[shorten >=1pt, node distance=2cm, on grid, auto]
            % States
            \node[state]   (q0)                {$q_{init}$};
            \node[state, accepting]            (q1) [right=of q0]  {$q_{1}$};
            \node[state] (q2) [right=of q1]  {$q_{2}$};
            
            % Transitions
            \path[->]
                (q0) edge [loop above]   node {$a $}
                (q0)
                     edge []    node {$b $} (q1)
                 (q1) edge [loop above]   node {$c$}
                (q1) edge [bend left]    node {$a$} 
                (q2)
                     
                (q2) edge [loop above]   node {$a $} (q2)
                     edge [bend left]    node {$b$} (q1);
        \end{tikzpicture} \\
        \textbf{(a) A 1-Alphabet DFA: $\Sigma_{1} = \{a,b,c\}$}
    \end{minipage}%
    \hfill
    \begin{minipage}[t]{0.3\linewidth}
        \centering 
        \begin{tikzpicture}[shorten >=1pt, node distance=2cm, on grid, auto]
            % States
            \node[state]   (q0)                {$q_{init}$};
            \node[state,accepting]            (q1) [right=of q0]  {$q_{1}$};
            \node[state] (q2) [right=of q1]  {$q_{2}$};
            
            % Transitions
            \path[->]
                (q0) edge [loop above]   node {$(a,0)$} (q0)
                     edge []    node {$(b,1)$} (q1)
                (q1) edge [bend left]    node {$(a,0)$} (q2)
                     edge [loop above]   node {$(b,1)$} (q1)
                (q2) edge [loop above]   node {$(c,1)$} (q2)
                     edge [bend left]    node {$(b,1)$} (q1);
        \end{tikzpicture} \\
         \textbf{(b) A 2-Alphabet DFA: $\Sigma_{1} = \{a,b,c\},~\Sigma_{2} = \{0,1\}$}
    \end{minipage}
    \hfill
    \begin{minipage}[t]{0.3\linewidth}
        \centering 
        \begin{tikzpicture}[shorten >=1pt, node distance=2cm, on grid, auto]
            % States
            \node[state]   (q0)                {$q_{init}$};
            \node[state, accepting]            (q1) [right=of q0]  {$q_{1}$};
            \node[state] (q2) [right=of q1]  {$q_{2}$};
            
            % Transitions
            \path[->]
                (q0) edge [loop above]   node {$(a,0,x)$} (q0)
                     edge [bend left]    node {$(b,1,y)$} (q1)
                (q0)
                     edge [bend right, below]    node {$(b,1,x)$} (q1)
                (q1) edge [bend left]    node {$(a,0,x) $} (q2)
                     edge [loop above]   node {$(b,1,y)$} (q1)
                (q2) edge [loop above]   node {$(c,1,x)$} (q2)
                     edge [bend left]    node {$(b,1,x) $} (q1);
        \end{tikzpicture} \\
         \textbf{(b) A 3-Alphabet DFA : $\Sigma_{1} = \{a,b,c\},~\Sigma_{2} = \{0,1\},~\Sigma_{2} = \{x,y\}$}
    \end{minipage}
    \caption{A graphical representation of N-Alphabet DFAs. Nodes corresponding to final states are represented by double circles.}
  \label{app:fig:nalphabetdfa}
\end{figure}
 
 \paragraph{Linear Algebra operations over N-Alphabet WAs.} 
 The development of polynomial-time algorithms for computing various SHAP variants for the class of WAs in section \ref{sec:tractable} of the main paper is based on two operations: the projection and the Kronecker product operations. In the following section of the appendix, we will provide a proof of the tractability of their construction.
 
%The construction of a polynomial-time algorithm for computing different SHAP variants for the class of WAs in section \ref{sec:tractable} of the main paper are based on two operations: The projection and the kronecker product operations. In the next section of the appendix, we shall provide a proof of the tractability of their construction.

In addition to these two operators, linear algebra operations over N-Alphabet WAs have also been implicitly utilized in the construction. The closure of 1-Letter WAs under linear algebra operations is a well-established result in the WA literature \citep{droste10}. For completeness, we offer a brief discussion below on how linear algebra operations can be extended to handle multiple alphabets, including their construction and the associated complexity results:

%Besides these two operators, linear algebra operations over N-Alphabet WAs has also been implicitly used in the construction. The closure of 1-Letter WAs under Liner Algebra operations is a classical result in the literature of WAs \citep{droste10}. For the sake of completeness, we provide below a brief discussion on how linear algebra operations over WAs can be generalized to handle multiple alphabets, how they are constructed as well as their associated complexity results:
     \begin{itemize}
         \item \emph{The addition operation:} Given two N-Alpphabet WAs $T = \langle\alpha, \{A_{\sigma_{1},\ldots, \sigma_{N}}\}_{(\sigma_{1}, \ldots, \sigma_{N}) \in \Sigma_{1} \times \ldots \times \Sigma_{N}}, \beta\rangle$ and $T' = \langle\alpha', \{A'_{\sigma_{1},\ldots, \sigma_{N}}\}_{(\sigma_{1}, \ldots, \sigma_{N}) \in \Sigma_{1} \times \ldots \times \Sigma_{N}}, \beta'\rangle$  over $\Sigma_{1} \times \ldots \times \Sigma_{N}$, the N-Alphabet WA, denoted $T + T'$, that computes the function:
         $$f_{T + T'}(w^{(1)}, \ldots, w^{(N)}) = f_{T}(w^{(1)}, \ldots, w^{(N)}) + f_{T'}(w^{(1)}, \ldots, w^{(N)})$$
         is parametrized as follows:
          $$\langle\begin{pmatrix}
              \alpha \\ \alpha'
          \end{pmatrix} , \{\begin{pmatrix}
               A_{\sigma_{1}, \ldots , \sigma_{N}} & \mathbf{O} \\
               \mathbf{O} & A'_{\sigma_{1}, \ldots, \sigma_{N}}
          \end{pmatrix} \}_{(\sigma_{1}, \ldots, \sigma_{N}) \in \Sigma_{1} \times \ldots \times \Sigma_{N}}, \begin{pmatrix}
              \beta \\ \beta' 
          \end{pmatrix} \rangle$$
          The running time of the addition operation is $O(|\Sigma|^{N} \cdot (\texttt{size}(T) + \texttt{size}(T') ))$. The size of the resulting N-Alphabet WA is equal to $O(\texttt{size}(T) + \texttt{size}(T'))$.
          \item \emph{Multiplication by a scalar.} Let $T = \langle\alpha, \{A_{\sigma_{1},\ldots,\sigma_{N}}\}_{(\sigma_{1}, \ldots, \sigma_{N}) \in \Sigma_{1} \times \ldots \times \Sigma_{N}}, \beta\rangle$ be an N-Alphabet WA over $\Sigma_{1} \times \ldots \times \Sigma_{N}$, and a real number $C> 0$, the N-Alphabet WA, denoted $C \cdot T$ that computes the function $f_{C \cdot T}(w^{(1)}, \ldots, w^{(N)}) = C \cdot f_{T}(w^{(1)}, \ldots, w^{(N)})$ is parametrized as: $\langle C \cdot \alpha, \{ A_{\sigma_{1},  \ldots, \sigma_{N}}\}_{(\sigma_{1}, \ldots, \sigma_{N}) \in \Sigma_{1} \times \ldots \times \Sigma_{N}}, \beta \rangle$. It is easy to see that the construction of the N-Alphabet WA $C \cdot T$ runs in $O(1)$ time, and has size equal to the size of $T$. 
     \end{itemize}
  \begin{table}[ht]
  \footnotesize
  	\setlength{\tabcolsep}{0.8em}
    \centering
            \caption{Operations on N-Alphabet WAs, along with their time complexity and output size. The ``In 1'' and ``In 2'' (respectively ``Out'') columns indicate the number of alphabets in the input N-Alphabet WAs for each operation. The ``Time'' column specifies the time complexity of executing the operation, and the ``Output size'' column denotes the size of the resulting N-Alphabet WA after the operation is applied. By convention, a value of $0$ in the ``Output'' and ``Output size'' columns indicates a scalar result.
            %Operations over N-Alphabet WAs, associated with their running time complexity and the size of their outputs. In 1 and In 2 (resp. Out) columns correspond to the number of alphabets of the input N-Alphabet WAs for each operation. The column ``Time'' corresponds to the running time complexity of performing the operation, and the output size refers to the size of the output N-Alphabet WA after applying the operation. By convention, $0$ in the columns 'Output' and ``Output size'' refers to a scalar.
            }
    \begin{tabular}{|c|c|c|c|c|c|}
        \hline
         & \textbf{In 1} & \textbf{In 2} & \textbf{Out} & \textbf{Time} & \textbf{Output size} \\ \hline
        Addition ($+$) & $N$ & $N$ & $N$ & $O(\max\limits_{i \in [N]} |\Sigma_{i}|^{N} \cdot (\texttt{size}(\text{in}_{1}) + \texttt{size}(\text{in}_{2}))$ & $O(\texttt{size}(\text{in}_{1}) + \texttt{size}(\text{in}_{2}))$ \\ \hline
        Scalar Multiplication & $N$ & $0$ & $N$ & $O(1)$ & $O(\texttt{size}(\text{in}_{1})$  \\ \hline
        $\Pi_{0}$ & $1$ & - & $0$ & $O(|\Sigma_{1}| \cdot \texttt{size}(\text{in}_{1})^{2} \cdot n)$ & $0$ \\ \hline
        $\Pi_{1}$ & $1$ & $1$ & $0$ & $O(|\Sigma_{1}| \cdot (| \texttt{size}(\text{in}_{1}) \cdot \texttt{size}(\text{in}_{2}))^{2} \cdot n)$ \footnote{We assume that the operations $\Pi_{0}$ and $\Pi_{1}$ are applied to WAs whose support is equal to $\Sigma^{n}$. This assumption is sufficient for the purpose of our work. In this case, the running time complexity of $\Pi_{0}$ and $\Pi_{1}$ depends on the parameter $n$.} & $0$ \\ \hline
        $\Pi_{i}$ ($i \geq 2$) & $1$ & $N$ & $N-1$ &$O(\max\limits_{i \in [N]} |\Sigma_{i}|^{N} \cdot \texttt{size}(\text{in}_{1}) \cdot \texttt{size}(\text{in}_{2}))$ &  $O(\texttt{size}(\text{in}_{1}) \cdot \texttt{size}(\text{in}_{2}))$ \\ \hline
        $\otimes$ & $N$ & $N$ & $N$ &$O(\max\limits_{i \in [N]} |\Sigma_{i}|^{N} \cdot \texttt{size}(\text{in}_{1}) \cdot \texttt{size}(\text{in}_{2}))$ &  $O(\texttt{size}(\text{in}_{1}) \cdot \texttt{size}(\text{in}_{2}))$ \\ \hline
    \end{tabular}
    \label{app:fig:operationswas}
\end{table}

     A summary of the running time complexity and the size of outputted WAs by all operations over $N$-Alphabet encountered in this work can be found in Table \ref{app:fig:operationswas}. 

\subsection{Proof of proposition \ref{prop:efficentoperations}}
Recall the statement of Proposition \ref{prop:efficentoperations}:

\begin{unumberedproposition}
       Assume that $N = O(1)$. Then, the projection and the Kronecker product operations between $N$-Alphabet WAs can be computed in polynomial time.
\end{unumberedproposition}
 The following result provides an implicit construction of these two operators which implicitly induces the result of Proposition \ref{prop:efficentoperations}: 

\begin{proposition}
Let $N$ be an integer, and $\{\Sigma_{i}\}_{i \in [N]}$, a collection of finite alphabets. We have: 
\begin{enumerate}
 \item The projection operation: Fix an integer $i \in [N]$. Let $A = \langle \alpha, \{A_{\sigma}\}_{\sigma \in \Sigma}, \beta \rangle$ be a WA over $\Sigma_{i}$,  $T = (\alpha', \{A'_{\sigma_{1},\ldots, \sigma_{N}}\}_{(\sigma_{1}, \ldots, \sigma_{N}) \in \Sigma_{1} \times \ldots \times \Sigma_{N}}, \beta')$ be an N-Alphabet WA over $\Sigma_{1} \times \ldots \times \Sigma_{N}$, and $A$ be a WA over $\Sigma_{i}$. The projection of $A$ over $T$ at index $i$, denoted $\Pi_{i}(A,T)$, is parametrized as:
     \begin{align*}
     \Pi_{i}(A,T) :=&  \langle\Sigma_{1} \times \ldots \times \Sigma_{i-1} \times \Sigma_{i} \times \ldots \times \Sigma_{N}, \alpha \otimes \alpha' , \\
     & \{ \sum\limits_{\sigma_{i} \in \Sigma_{i}} A_{\sigma_{i}} \otimes A'_{\sigma_{1}, \ldots, \sigma_{i-1}, \sigma_{i+1}, \ldots, \sigma_{N}} \}_{(\sigma_{1}, \ldots, \sigma_{i-1},\sigma_{i}, \ldots, \sigma_{N}) \in \Sigma_{1} \times \ldots \times \Sigma_{i-1} \times \Sigma_{i+1} \times \ldots \times \Sigma_{N}} ,\beta \otimes \beta'\rangle
     \end{align*}

  \item The Kronecker product operation: Let $T = \langle\alpha, \{A_{\sigma_{1},\ldots, \sigma_{N}}\}_{(\sigma_{1}, \ldots, \sigma_{N}) \in \Sigma_{1} \times \ldots \times \Sigma_{N}}, \beta\rangle$ and $T' = \langle\alpha', \{A'_{\sigma_{1},\ldots, \sigma_{N}}\}_{(\sigma_{1}, \ldots, \sigma_{N}) \in \Sigma_{1} \times \ldots \times \Sigma_{N}}, \beta'\rangle$ be two N-Alphabet WAs over $\Sigma_{1} \times \ldots \times \Sigma_{N}$. The Kronecker product between $T$ and $T'$, $T \otimes T'$, is parametrized as:
  $$T \otimes T' = \langle\alpha \otimes \alpha', \sum\limits_{\sigma \in \Sigma} A_{\sigma_{1}, \ldots, \sigma_{N}} \otimes A'_{\sigma_{1}, \ldots, \sigma_{N}}, \beta \otimes \beta'\rangle$$
\end{enumerate}
\end{proposition}

\begin{proof}
    Let $N$ be an integer, and $\{\Sigma_{i}\}_{i \in [N]}$, a collection of finite alphabets.
    \begin{enumerate}
     \item For the projection operation: Fix $i \in [N]$. Let $T = \langle\alpha', \{A'_{\sigma_{1},\ldots, \sigma_{N}}\}_{(\sigma_{1}, \ldots, \sigma_{N}) \in \Sigma_{1} \times \ldots \times \Sigma_{N}}, \beta'\rangle$ be an N-Alphabet WA over $\Sigma_{1} \times \ldots \times \Sigma_{N}$, and let $A$ be a WA over $\Sigma_{i}$. Let there be some $(w^{(1)}, \ldots, w^{(N)}) \in \Sigma_{1}^{*} \times \ldots \Sigma_{N}^{*}$, such that $|w^{(1)}| = \ldots = |w^{(N)}| = L$. We have:

    \begin{align*}
        f_{\Pi_{i}(A,T)}(w^{(1)}, \ldots, w^{(i-1)}, w^{(i+1)}, w^{(N)}) &= \sum\limits_{w \in \Sigma_{i}^{L}} f_{A}(w) \cdot f_{T}(w^{(1)}, \ldots, w^{(i-1)}, w, w^{(i+1)}, w^{(N)}) \\
        &= \sum\limits_{w \in \Sigma_{i}^{L}} \left( \alpha^{T} \cdot \prod\limits_{j=1}^{L} A_{w_{j}} \cdot \beta \right) \cdot \left( \alpha'^{T} \cdot \prod\limits_{j=1}^{L} A'_{w_{j}^{(1)}, \ldots, w_{j}^{(i)}, \ldots w_{j}^{(N)}} \cdot \beta' \right)  \\
        &= \sum\limits_{w \in \Sigma_{i}^{L}} (\alpha \otimes \alpha')^{T} \cdot \left[ \prod\limits_{j=1}^{L} A_{w_{j}} \otimes A'_{w_{j}^{(1)}, \ldots, w_{j}^{(i)} \ldots w_{j}^{(N)}} \right] \cdot (\beta \otimes \beta') \\
        &= (\alpha \otimes \alpha')^{T} \cdot \prod\limits_{j=1}^{L} \left( \sum\limits_{\sigma \in \Sigma_{i}} A_{\sigma} \otimes A'_{w_{j}^{(1)}, \ldots, \sigma, w_{j}^{(N)}} \right) \cdot (\beta \otimes \beta')
    \end{align*}
    where the third equality is obtained using the mixed-product property of the Kronecker product between matrices.
    \item The Kronecker product operation: Let $T = \langle\alpha, \{A_{\sigma_{1},\ldots, \sigma_{N}}\}_{(\sigma_{1}, \ldots, \sigma_{N}) \in \Sigma_{1} \times \ldots \times \Sigma_{N}}, \beta\rangle$ and $T' = \langle\alpha', \{A'_{\sigma_{1},\ldots, \sigma_{N}}\}_{(\sigma_{1}, \ldots, \sigma_{N}) \in \Sigma_{1} \times \ldots \times \Sigma_{N}}, \beta'\rangle$ be two N-Alphabets WA over $\Sigma_{1} \times \ldots \times \Sigma_{N}$. Let $(w^{(1)}, \ldots, w^{(N)}) \in \Sigma_{1}^{*} \times \ldots \Sigma_{N}^{*}$ such that $|w^{(1)}| = \ldots = |w^{(N)}| = L$. We have:
    \begin{align*}
        f_{T \otimes T'}(w^{(1)}, \ldots, w^{(N)}) &= f_{T}(w^{(1)}, \ldots, w^{(N)}) \cdot f_{T'}(w^{(1)}, \ldots, w^{(N)}) \\
        &= (\alpha^{T} \cdot \prod\limits_{j=1}^{L} A_{w_{j}^{(1)}, \ldots w_{j}^{N}} \cdot \beta) \cdot (\alpha'^{T} \cdot \prod\limits_{j=1}^{L} A'_{w_{j}^{(1)}, \ldots w_{j}^{N}} \cdot \beta') \\
        &=  (\alpha \otimes \alpha')^{T} \cdot \prod\limits_{j=1}^{L} \left( A_{w_{j}^{(1)}, \ldots w_{j}^{N}} \otimes A'_{w_{j}^{(1)}, \ldots w_{j}^{N}} \right) \cdot (\beta \otimes \beta')
    \end{align*}
    where the last equality is obtained using the mixed-product property of the Kronecker product between matrices.
    \end{enumerate}
\end{proof}

\subsection{Proof of Lemma \ref{lemma:shapasoperations}}

In this segment, we provide the proof of the main lemma of section \ref{sec:tractable}:

\begin{unumberedlemma}
Fix a finite alphabet $\Sigma$. Let $f$ be a WA over $\Sigma$, and consider a sequence $(w, w^{\text{reff}}) \in \Sigma^{*} \times \Sigma^{}$ (representing an input and a basline $\x, \x^{\text{reff}} \in \mathcal{X}$) such that $|w| = |w^{\text{reff}}|$. Let $i \in [|w|]$ be an integer, and $\mathcal{D}_P$ be a distribution modeled by an HMM over $\Sigma$. Then:
%Fix a finite alphabet $\Sigma$. Let $f$ be a WA over $\Sigma$, and consider a sequence $(w, w^{\text{reff}}) \in \Sigma^{*} \times \Sigma^{*}$ (representing an input $\x \in \mathcal{X}$ and an auxiliary baseline $\x^{\text{reff}} \in \mathcal{X}$), such that $|w| = |w^{\text{reff}}|$. Additionally, let $i \in [|w|]$ be an integer, and $\mathcal{D}_P$ be a distribution modeled by an HMM over $\Sigma$. We have that:
 %Fix a finite alphabet $\Sigma$. Let $f$ be a WA over $\Sigma$, a sequence $(w,w^{\text{ref}}) \in \Sigma^{*} \times \Sigma^{*}$ (representing an input $\x\in\mathcal{X}$ and an auxialry basline $\x^{\text{ref}}\in\mathcal{X}$) such that $|w| = |w^{\text{ref}}|$, an integer $i \in [|w|]$, and a distribution $\mathcal{D}_P$ modeled by an HMM over $\Sigma$. We have:
         {\small 
            \begin{align*}
         %\emph{\texttt{LOC-I-SHAP}}
         \phi_i
         (f,w,i,\mathcal{D}_P) = \quad\quad\quad\quad\quad\quad\quad\quad\quad\quad\quad\quad\quad\quad\quad\quad \\ \Pi_{1} (A_{w,i}, \Pi_{2}(\mathcal{D}_P, \Pi_{3}(f,T_{w,i}) 
          - \Pi_{3}(f,T_{w}) ) ); \quad\quad \\
            %\end{align*}
            %}
            %{\small 
            %\begin{align*}  
            \Phi_i(f,i,n,\mathcal{D}_P) = \quad\quad\quad\quad\quad\quad\quad\quad\quad\quad\quad\quad\quad\quad\quad\quad \\ 
            \Pi_{0} ( \Pi_{2}(\mathcal{D}_P, A_{i,n} \otimes \Pi_{2}(\mathcal{D}_P, 
             \Pi_{3}(f,T_{i}) - \Pi_{3}(f,T)))); \quad\\
            %\end{align*}
            %}
            %{\small 
            %\begin{align*}
            \phi_b(f,w,i,w^{\text{reff}}) = \quad\quad\quad\quad\quad\quad\quad\quad\quad\quad\quad\quad\quad\quad\quad\quad \\ \Pi_{1} (A_{w,i}, \Pi_{2}(f_{w^{\text{reff}}}, \Pi_{3}(f,T_{w,i}) 
          - \Pi_{3}(f,T_{w})));\quad \\
            %\end{align*}    
            %}
            % {\small 
            %  \begin{align*}
            \Phi_b(f,i,n,w^{\text{reff}},\mathcal{D}_P) = \quad\quad\quad\quad\quad\quad\quad\quad\quad\quad\quad\quad\quad\quad \\ \Pi_{0} ( \Pi_{2}(\mathcal{D}_P, A_{i,n} \otimes \Pi_{2}(f_{w^{\text{reff}}} ,
            \Pi_{3}(f,T_{i}) - \Pi_{3}(f,T)) ))
               \end{align*}
             } where:
    \begin{itemize}
        \item $A_{w,i}$ is a 1-Alphabet WA over $\Sigma_{\#}$ implementing the uniform distribution over coalitions excluding the feature $i$ (i.e., $f_{A_{w,i}} = \mathcal{P}_{i}^{w}$);
        \item $T_{w}$ is a 3-Alphabet \emph{WA} over $\Sigma_{\#} \times \Sigma \times \Sigma$ implementing the function: $ g_{w}(p,w',u) := I(\texttt{do}(p,w',w) = u)$.
        %\begin{equation} \label{eq:Tw}
        %    g_{w}(p,w',u) = I(\texttt{do}(p,w',w) = u)
        %\end{equation}
        \item $T_{w,i}$ is a 3-Alphabet \emph{WA} over $\Sigma_{\#} \times \Sigma \times \Sigma$ implementing the function: $            g_{w,i}(p,w',u) := I(\texttt{do}(\texttt{swap}(p,w_{i},i),w',w) = u)$.
        %\begin{equation} \label{eq:Twi}
        %    g_{w,i}(p,w',u) = I(\texttt{do}(\texttt{swap}(p,w_{i},i),w',w) = u)
        %\end{equation}
        \item $T$ is a 4-Alphabet \emph{WA} over $\Sigma_{\#} \times \Sigma \times \Sigma \times \Sigma$ given as: $g(p,w',u,w) := g_{w}(p,w',u)$.
        %\begin{equation} \label{eq:T}
        %    g(p,w',u,w) = g_{w}(p,w',u)
        %\end{equation}
        \item $T_{i}$ is a 4-Alphabet \emph{WA} over $\Sigma_{\#} \times \Sigma \times \Sigma \times \Sigma$ given as: $ g_{i}(p,w',u,w) := g_{w,i}(p,w',u)$.
        %\begin{equation} \label{eq:Ti}
        %    g_{i}(p,w',u,w) = g_{w,i}(p,w',u)
        %\end{equation}
        \item $A_{i,n}$ is a 2-Alphabet \emph{WA} over $\Sigma_{\#} \times \Sigma$ implementing the function:
         $g_{i,n}(p,w) := I(p \in \mathcal{L}_{i}^{w}) \cdot \mathcal{P}_{i}^{w}(p)$,
         where $|w| = |p| = n$.
       \item $f_{w^{\text{reff}}}$ is an \emph{HMM} such that the probability of generating $w^{\text{reff}}$ as a prefix is equal to $1$.
    \end{itemize}

\end{unumberedlemma}

\begin{proof}

We will prove the complexity results specifically for the cases involving either local or global \emph{Interventional} SHAP. The corresponding proof for local and global Baseline SHAP can be derived by following the same approach as in the interventional case, with the sole modification of replacing $\mathcal{D}_{p}$ with the HMM that models the empirical distribution induced by the reference instance $w^{\text{reff}}$.


%We shall prove the complexity results only for the scenarios considering either local and global \emph{Interventional} SHAP. The same proof for local and global Baseline SHAP can be obtained by mimicking the proof of the interventional case by simply replacing $\mathcal{D}_{p}$ with the HMM modeling the empirical distribution induced by the reference instance $\mathbf{x}^{ref}$.

%Fix a finite alphabet $\Sigma$. Let $M$ be a WA over $\Sigma$, a sequence $w \in \Sigma^{*}$, a integer $i \in [|w|]$, and $P$ a HMM. Let $A_{w,i}$ be a WA and $T_{w},~T_{w,i}$ be two 3-Alphabet WAs as defined in the statement of the lemma. We split the following proof into two segments. The first segment is for the local interventional SHAP version, while the second segment is for the global one.


Fix a finite alphabet $\Sigma$. Let $f$ be a WA over $\Sigma$, $w \in \Sigma^{*}$ a sequence, $i \in [|w|]$ an integer, and $\mathcal{D}_{P}$ an HMM. Define $A_{w,i}$ as a WA and $T_{w},~T_{w,i}$ as two 3-Alphabet WAs as stated in the lemma. We divide the following proof into two parts. The first part addresses the local interventional SHAP version, while the second part covers the global version.







    \begin{enumerate}
        \item For \emph{local Interventional SHAP} we have that:
        \begin{align}
            \phi_i(f,w,i,\mathcal{D}_P) &= \mathbb{E}_{p \sim \mathcal{P}_{i}^{w}} \left[ V_{I}(w,\texttt{swap}(p,w_{i},i), \mathcal{D}_P) - 
 V_{I}(w,p,\mathcal{D}_P) \right] \nonumber \\ 
 &= \sum\limits_{p \in \Sigma_{\#}^{|w|}} f_{A_{w,i}}(p) \left[ \sum\limits_{w' \in \Sigma^{w}} \mathcal{D}_P(w') \cdot \left[ f(\texttt{do}(\texttt{swap}(p,w'_{i},i),w',w)) - f(\texttt{do}(p,w',w)\right] \right] \label{eq:locishap}
        \end{align}
        
 Note that for any $p \in \Sigma_{\#}^{|w|}$ and $(w',u) \in \Sigma^{|w|} \times \Sigma^{|w|}$, we have:
 \begin{equation} \label{eq:obswi}
     f(\texttt{do}(\texttt{swap}(p,w'_{i},i),w',w) = \sum\limits_{u \in \Sigma^{|w|}} f(u) \cdot g_{w,i}(p,w',u) = f_{\Pi_{3}(f,T_{w,i})}(p,w')  
 \end{equation}
 and,
 \begin{equation} \label{eq:obsw}
     f(\texttt{do}(p,w',w)) = \sum\limits_{u \in \Sigma^{|w|}} f(u) \cdot g_{w}(p,w',w) = f_{\Pi_{3}(f,T_{w})}(p,w')
 \end{equation}
 where $g_{w,i}$ and $g_{w}$ are defined implicitly in the body of the lemma statement.

 By plugging equations \eqref{eq:obswi} and \eqref{eq:obsw} in Equation \eqref{eq:locishap}, we obtain:
 \begin{align*}
     \phi_i(f,w,i, \mathcal{D}_P) &=  \sum\limits_{p \in \Sigma_{\#}^{|w|}} f_{A_{w,i}}(p) \left[ \sum\limits_{w' \in \Sigma^{|w|}} \mathcal{D}_P(w') \cdot [ f_{\Pi_{3}(M,T_{w,i})}(p,w') - f_{\Pi_{3}}(M, T_{w}) (p,w') ] \right]
 \end{align*}
 To ease exposition, we employ the symbol $\Tilde{T}$ to refer to the intermediary 2-Alphabet WA over $\Sigma_{\#} \times \Sigma$ defined as:
 $$\Tilde{T} \myeq \Pi_{3}(M,T_{w,i}) - \Pi_{3}(M,T_{w})$$
 Then, we have:
\begin{align*}
    \phi_i(f,w,i, \mathcal{D}_P) &= \sum\limits_{p \in \Sigma_{\#}^{|w|}} f_{A_{w,i}}(p) \left[ \mathcal{D}_P(w') \cdot f_{\Tilde{T}}(p,w') \right] \\
    &= \sum\limits_{p \in \Sigma_{\#}^{|w|}} f_{A_{w,i}}(p) \cdot f_{\Pi_{2}(\mathcal{D}_P,\Tilde{T})}(p) \\
    &= \Pi_{1}(A_{w,i}, \Pi_{2}(\mathcal{D}_P,\Tilde{T}))
\end{align*}
  \item For \emph{Global Interventional SHAP}, the proof follows the same structure as that of the Local version. We hence have that:
 \begin{align}
     \phi_i(f,i,n,P) &= \sum\limits_{w \in \Sigma^{n}} \mathcal{D}_P(w) \cdot \phi_i(f,w,i, \mathcal{D}_P) \nonumber \\
     &= \sum\limits_{w \in \Sigma^{n}}  \mathcal{D}_P(w) \sum\limits_{p \in \Sigma_{\#}}^{|w|} f_{A_{w,i}}(p) \left[ \sum\limits_{w' \in \Sigma^{w}} \mathcal{D}_P(w') \cdot \left[ f(\texttt{do}(\texttt{swap}(p,w'_{i},i),w',w)) - f(\texttt{do}(p,w',w)\right] \right] \label{eq:gloishapwa}
 \end{align}

   Note that for any $p \in \Sigma_{\#}^{n}$ and $(w',u,w) \in \Sigma^{n} \times \Sigma^{n} \times \Sigma^{n}$, we have:
 \begin{equation} \label{eq:obsi}
     f(\texttt{do}(\texttt{swap}(p,w'_{i},i),w',w) = \sum\limits_{u \in \Sigma^{n}} f(u) \cdot g_{i}(p,w',u,w) = f_{\Pi_{3}(M,T_{i})}(p,w',w)  
 \end{equation}
 and,
 \begin{equation} \label{eq:obs}
     f(\texttt{do}(p,w',w)) = \sum\limits_{u \in \Sigma^{|w|}} f(u) \cdot g(p,w',u,w) = f_{\Pi_{3}(M,T)}(p,w',w)
 \end{equation}
 where $g_{i}$ and $g$ are functions defined implicitly in the lemma statement.

 By, again, plugging equations \eqref{eq:obsi} and \eqref{eq:obs} into the equation \ref{eq:gloishapwa}, we obtain:
 \begin{align*}
     \phi_i(f,i,n,P) &= \sum\limits_{w \in \Sigma^{n}} \mathcal{D}_P(w) \cdot \sum\limits_{p \in \Sigma_{\#}^{n}}  f_{A_{i,n}}(p,w) \left[ \sum\limits_{w' \in \Sigma^{n}} \mathcal{D}_P(w') \cdot f_{\Tilde{T}}(p,w',w) \right] \\
     &= \sum\limits_{w \in \Sigma^{n}}  \mathcal{D}_P(w) \cdot \sum\limits_{p \in \Sigma_{\#}^{n}} f_{A_{i,n}}(p,w) \cdot f_{\Pi_{2}(\mathcal{D}_P,\Tilde{T})}(p,w) \\
     &= \sum\limits_{w \in \Sigma^{n}}  \mathcal{D}_P(w) \sum\limits_{p \in \Sigma_{\#}^{n}} f_{A_{i,n} \otimes \Pi_{2}(\mathcal{D}_P,\Tilde{T})}
     (p,w) \\
     &= \sum\limits_{p \in \Sigma_{\#}^{n}} f_{\Pi_{2}(\mathcal{D}_P, A_{i,n} \otimes \Pi_{2}(\mathcal{D}_P,\Tilde{T}))}(p) \\
     &= \Pi_{0} \left( \Pi_{2}(\mathcal{D}_P, A_{i,n} \otimes \Pi_{2}(\mathcal{D}_P,\Tilde{T}))\right)     
  \end{align*} 
    \end{enumerate}
\end{proof}

\subsection{Proof of proposition \ref{prop:nletterwaconstruction}}
In this segment, we shall provide a constructive proof of all machines defined implicitly in Lemma \ref{lemma:shapasoperations}. Formally, we shall prove the following:

\begin{unumberedproposition}
 The N-Alphabet WAs $A_{w,i}$, $T_{w}$, $T_{w,i}$, $T$, $T_{i}$, $A_{i,n}$ and the HMM $f_{w^{\text{reff}}}$ can be constructed in polynomial time with respect to $|w|$ and $|\Sigma|$.
\end{unumberedproposition}
\begin{table}[ht]
    \centering
        \caption{The summary of complexity results (in terms of the length of the sequence to explain $w$ and the size of the alphabet $|\Sigma|$) for the construction of models in proposition \ref{prop:nletterwaconstruction}}
    \begin{tabular}{|c|c|c|c|}
        \hline
         & \textbf{Running Time complexity} & \textbf{Output Size} & \textbf{Output's Alphabet size ($N$)}  \\ \hline
        $A_{w,i}$ & $O(|w|^{3})$ & $O(|w|^{3})$ & $1$  \\ \hline
        $T_{w,i}$ &  $O(|\Sigma|^{3} \cdot |w|)$ & $|w|$ & $3$  \\ \hline
        $T_{w}$ &$O(|\Sigma|^{3} \cdot |w|)$ & $|w|$ & $3$  \\ \hline
        $T_{i}$ & $O(|w|)$ & $O(|w|)$ & $4$  \\ \hline
        $T$ & $O(1)$ & $O(1)$ & $4$ \\ \hline
        $A_{i,n}$ &  $O(|\Sigma|^{2} \cdot |w|^{4})$ & $O(|w|^{4})$ & $2$  \\ \hline
        $f_{w^{reff}}$ & $O(|w|)$ & $O(|w|)$ & $1$  \\ \hline
    \end{tabular}
    \label{app:fig:prop6}
\end{table}
We split the proof of this proposition into four sub-sections. The first sub-section is dedicated to the construction of $A_{w,i}$ and $A_{i,n}$. The second sub-section is dedicated to the construction of $T_{w,i}$ and $T_{i}$. The third sub-section is dedicated to the construction of $T_{i}$ and $T$. The final subsection treats the construction of $f_{w^{reff}}$. The running time of all these constructions as well as the size of their respective outputs are summarized in Table \ref{app:fig:prop6}.  

\subsubsection{The construction of $A_{w,i}$ and $A_{i,n}$} 
Recall $A_{w,i}$ is a WA over $\Sigma_{\#}$ that implements the probability distribution: 
\begin{equation} \label{app:eq:piw}
\mathcal{P}_{i}^{(w)}(p) \myeq \frac{1}{|w|} \sum\limits_{k=1}^{|w|} \mathcal{P}_{i,k}^{(w)}(p)
\end{equation}
where $\mathcal{P}_{i,k}^{(w)}$ is the uniform distribution over patterns belonging to the following set:
$$\mathcal{L}_{i,k}^{(w)} \myeq \{ p \in \Sigma_{\#}^{|w|}: ~ w \in L_{p} \land |p|_{\#} = k \land p_{i} = \# \}$$
The 2-Alphabet WA $A_{i,n}$ can be seen as a global version of $A_{w,i}$ where the sequence $w$ becomes a part of the input of the automaton. Next, we shall provide the construction for $A_{w,i}$. The construction of $A_{i,n}$ is somewhat similar to that of $A_{w,i}$ and will be discussed later on in this section.

\paragraph{The construction of $A_{w,i}$.} Algorithm \ref{app:alg:Awi} provides the pseudo-code for constructing $A_{w,i}$. By noting that the target probability distribution $\mathcal{P}_{i}^{(w)}$ is a linear combination of the set of functions $ \mathcal{F} = \{\mathcal{P}_{i,k}^{(w)}\}$ (Equation \ref{app:eq:piw}), the strategy of our construction  consists at iteratively constructing a sequence of WAs $\{A_{i,k}^{(w)}\}_{k \in [|w|]}$, each of which implements a function $F \in \mathcal{F}$. Thanks to the closure of WAs under linear algebra operations (addition, and multplication with a scalar) and the tractability of implementing them, one can recover the target WA $A_{w,i}$ using linear combinations of the collection of WAs $ \mathcal{A} = \{A_{i,k}^{(w)}\}_{k \in [|w|]}$. 

\begin{algorithm}
\caption{Construction of $A_{w,i}$}
\label{app:alg:Awi}
\begin{algorithmic}[1]
\REQUIRE A sequence $w \in \Sigma^{*}$, An integer $i \in [|w|]$
\ENSURE A WA $A_{w,i}$  
\STATE Initialize $A_{w,i}$ as the empty WA
\FOR{$k = 1 \ldots |w|$}{
  \STATE Construct a DFA $\bar{A}_{i,k}^{|w|}$ that accepts the language $\mathcal{L}_{i,k}^{(w)}$
   \STATE $A_{w,i} \leftarrow A_{w,i} + \frac{1}{|w| \cdot |\mathcal{L}_{i,k}^{|w|}} \cdot \bar{A}_{i,k}^{|w|}$
}
\ENDFOR
\RETURN $A_{w,i}$
\end{algorithmic}
\end{algorithm}


The missing link to complete the construction is to show how each element in the set $\mathcal{A}$ is constructed. For a given $(i,k) \in [|w|]^{2}$, the strategy consists at constructing a DFA that accepts the language $\mathcal{L}_{i,k}^{(w)}$ (denoted $\bar{A}_{i,k}^{|w|}$ in Line 3 of Algorithm~\ref{app:alg:Awi})). Since the function implemented by $A_{i,k}^{(w)}$ represents the uniform probability distribution over patterns in $\mathcal{L}_{i,k}^{(w)}$, then $A_{i,k}^{(w)}$ can be recovered by simply normalizing the DFA $\bar{A}_{i,k}^{(w)}$ by the quantity $\frac{1}{|\mathcal{L}_{i,k}^{(w)}|}$.  

Given a sequence $w \in \Sigma^{*}$ and $(i,k) \in [|w|]^{2}$, the construction of the DFA, $\bar{A}_{i,k}^{(w)}$, is given as follows:
\begin{itemize}
    \item \textbf{The state space:} $Q = [|w|+1] \times \{0, \ldots, |w|\}$ (For a given state $(l,e) \in [|w|] \times [|w|]$, the element $l$ tracks the position of the running pattern, and $e$ computes the number of $\{\#\}$ symbols of the running pattern.
    \item \textbf{Initial State:} $(1,0)$
    \item \textbf{The transition function:} For a given state $(l,e) \in [|w|] \times \{0, \ldots, |w|\}$ and a symbol $\sigma \in \Sigma_{\#}$, we have:
    $$\delta((l,e),\sigma) = \begin{cases}
     (l+1, e+1) & \text{if} ~~\sigma = \# \\
     (l+1,e) & \text{if} ~~\sigma \in \Sigma \land w_{l} = \sigma \land l \neq i
    \end{cases}
    $$
    \item \textbf{The final state:} $F = (|w|+1 , k)$
\end{itemize}

\paragraph{Complexity.} The complexity of constructing $\bar{A}_{i,k}^{(w)}$ for a given $k \in [|w|]$ is equal to $O(|w|^{2})$. Consequently, taking into account the iterative procedure in Algorithm \ref{app:alg:Awi}, this latter algorithm runs in $O(|w|^{3})$. In addition, the size of the resulting $A_{w,i}$ is also $O(|w|^{3})$. 

\paragraph{The construction of $A_{i,n}$.} As previously noted, the 2-Alphabet WA can be interpreted as the global equivalent of $A_{w,i}$. Formally, for an integer $n > 0$ and $i \in [n]$, the 2-Alphabet WA $A_{i,n}$ realizes the function $g_{i,n}$ over $\Sigma_{\#} \times \Sigma$ as defined below:
%As mentioned earlier, the 2-Alphabet WA can be construed as the global counterpart of $A_{w,i}$. Formally, given an integer $n > 0$, and $i \in [n]$, the 2-Alphabet WA $A_{i,n}$ implements the function $g_{i,n}$ over $\Sigma_{\#} \times \Sigma$ given as follows:
\begin{equation} \label{app:eq:gin}
g_{i,n}(p,w) = I(p \in \mathcal{L}_{i}^{w}) \cdot \mathcal{P}_{i}^{w}(p)
\end{equation}

In light of Equation \eqref{app:eq:gin}, the construction of $A_{i,n}$ aligns to the following three step procedure:
\begin{enumerate}
    \item Construct a 2-Alphabet DFA $A'_{w,i}$ over $\Sigma_{\#} \times \Sigma$ that accepts the language $I(p \in \mathcal{L}_{i}^{w})$
    \item Construct a 2-Alphabet WA $\Tilde{A}_{w,i}$ over $\Sigma_{\#} \times \Sigma$ such that: $f_{\Tilde{A}_{w,i}}(p,w) = f_{A_{w,i}}(p)$ (Note that $f_{\Tilde{A}_{w,i}}$ is independent of its second argument)
    \item Return $A'_{w,i} \otimes A_{w,i}$ 
\end{enumerate}

The derivation of the 2-Alphabet WA $\Tilde{A}_{w,i}$ from $A_{w,i}$, whose construction is detailed in Algorithm \ref{app:alg:Awi}, is straightforward. It remains to demonstrate how to construct the 2-Alphabet DFA $A'_{w,i}$ (Step 1).

\paragraph{The construction of $A'_{w,i}$.} Recall that $\mathcal{L}_{i}^{(w)} \myeq \bigcup\limits_{k=1}^{|w|} \mathcal{L}_{i,k}^{(w)}$. Constructing a 2-Alphabet DFA that accepts the language $I(p \in \mathcal{L}_{i}^{w})$ is relatively simple: It involves verifying the conditions $w \in L_{p}$ and $p_{i} = \#$ for the running sequence $(p,w) \in \Sigma_{\#}^{*} \times \Sigma^{*}$. The construction is provided as follows:

%Recall that $\mathcal{L}_{i}^{(w)} \myeq \bigcup\limits_{k=1}^{|w|} \mathcal{L}_{i,k}^{(w)}$. The construction of a 2-Alphabet DFA that accepts the language $I(p \in \mathcal{L}_{i}^{w})$ is relatively straightforward: The conditions $w \in L_{p}$ and $p_{i} = \#$ need to be checked for the running sequence $(p,w) \in \Sigma_{\#}^{*} \times \Sigma^{*}$. It is given as follows:
\begin{itemize}
    \item \textbf{The state space:} $Q = [|w|]$
    \item \textbf{The initial state:} The state $1$
    \item \textbf{The transition function:} For a state $q \in Q$ and $(\sigma,\sigma') \in \Sigma_{\#} \times \Sigma$, then we have that $\delta(q, (\sigma, \sigma')) = q+1$ holds if and only if the predicate: 
     \begin{equation} \label{app:eq:predicate}
      \left(q \neq i \land (\sigma = \#) \lor (\sigma = \sigma') \right) \lor (q = i \land \sigma = \# )
     \end{equation}
     is true.


     
In essence, the predicate defined in Equation~\eqref{app:eq:predicate} captures the idea that, for a given position $q$ in the current pair of sequences $(p,w) \in \Sigma_{\#}^{*} \times \Sigma^{*}$, either $w_{q} \in L_{p_{q}}$ when $q \neq i$ (ensuring the condition $w \in L_p$), or $p_q = \#$ when $q = i$ (ensuring the condition $p_i = \#$).
     
     %Basically, the predicate \eqref{app:eq:predicate} formalizes the fact that for a given position $q$ in the running pair of sequences $(p,w) \in \Sigma_{\#}^{*} \times \Sigma^{*}$, we have that either $w_{q} \in L_{p_{q}}$ if $q \neq i$ (Ensuring the condition $w \in L_{p}$), or $p_{q} = \#$ if $q = 1$ (ensuring the condition $p_{i} = \#$).
    \item \textbf{The final state:} The state $|w|$
\end{itemize}

\paragraph{Complexity.} The construction of $A'_{w,i}$ requires $O(|w|)$ running time, and its size is equal to $O(|w|)$. The running time complexity for the construction of $A_{i,n}$ is dictated by the application of the Kronecker product between the 2-Alphabet WAs $A'_{w,i} \otimes A_{w,i}$ (Step 3 of the procedure outlined above). Given the complexity of computing the Kronecker product between N-Alphabet WAs (see Table \ref{app:fig:operationswas}) and the sizes of $A'_{w,i}$ and $A_{w,i}$, the overall time complexity is equal to $O(|\Sigma|^{2} \cdot |w|^{4})$. The size of $A_{i,n}$ is equal to $O(|w|^{4})$.  

\subsubsection{The construction of $T_{w,i}$ and $T_{w}$} 

The constructions of $T_{w,i}$ and $T_{w}$ are highly similar. Consequently, this section will mainly concentrate on the complete construction of $T_{w,i}$, as it introduces an additional challenge with the inclusion of the \texttt{swap} operation in the function implemented by this 3-Alphabet WA. A brief discussion on the construction of $T_{i}$ will follow at the end of this section, based on the approach used for $T_{w,i}$.

%The construction of $T_{w,i}$ and $T_{w}$ are closely similar. Therefore, this section will primarily focus on the full construction of $T_{w,i}$ as it presents an additional challenge due to the inclusion of the \texttt{swap} operation in the function implemented by this 3-Alphabet WA. A brief note on the construction of $T_{i}$ will be provided at the end of this segment in light of the approach used for $T_{w,i}$.

Recall that $T_{w,i}$ is a 3-Alphabet DFA over $\Sigma_{\#} \times \Sigma \times \Sigma$ that implements the function: 

$$g_{w,i}(p,w',u) = I(\texttt{do}(\texttt{swap}(p,w_{i},i),w',w)= u)$$
for a triplet $(p,w',u) \in \Sigma_{\#}^{*} \times \Sigma^{*} \times \Sigma^{*}$, for which $|p| = |w'| = |u|$.

To ease exposition, we introduce the following predicate:

\begin{equation} \label{app:eq:predicateTwi}
 \Phi(\sigma_{1}, \sigma_{2}, \sigma_{3}, \sigma_{4}) \myeq  \left( \sigma_{1} = \# \land \sigma_{3} = \sigma_{2} \right) \lor \left(\sigma_{1} \neq \# \land \sigma_{3} = \sigma_{4} \right) 
\end{equation}

where $(\sigma_{1}, \sigma_{2}, \sigma_{3}, \sigma_{4}) \in \Sigma_{\#} \times \Sigma \times \Sigma \times \Sigma$. 

The construction of $T_{w,i}$ is given as follows:
\begin{itemize}
    \item \textbf{The state space:} $Q = [|w| + 1]$
    \item \textbf{The initial state:} $q_{init} = 1$
    \item \textbf{The transition function:} For a state $q \in Q$, and a tuple of symbols $(\sigma_{1}, \sigma_{2}, \sigma_{3}) \in \Sigma_{\#} \times \Sigma \times \Sigma$, we have $\delta(q, (\sigma_{1}, \sigma_{2}, \sigma_{3})) = q+1$ if and only if the predicate:
    \begin{equation} \label{app:eq:Twi}
    \left[q \neq i \land \Phi(\sigma_{1},\sigma_{2}, \sigma_{3}, w_{q}) \right] \lor \left[q = i \land \sigma_{3} = w_{q} \right]
    \end{equation}
    is true.
    \item \textbf{The final state:} $|w| + 1$.
\end{itemize}

\textbf{Note.} As previously mentioned, the construction of the 3-Alphabet WA, $T_{w}$, is quite similar to that of $T_{w,i}$. To obtain a comparable construction of $T_{w}$, one can easily adjust the predicate defined in equation~\ref{app:eq:Twi} to $\Phi(\sigma, w_{q}, \sigma_{3}, \sigma_{4})$.
%As mentioned earlier, the construction of the 3-Alphabet WA, $T_{w}$, is closely similar to that $T_{w,i}$. To derive an analogous construction of $T_{w}$, one can simply modify the predicate \ref{app:eq:Twi} to $\Phi(\sigma, w_{q}, \sigma_{3}, \sigma_{4})$.

\subsubsection{The construction of $T_{i}$ and $T$.} 
The 4-Alphabet WAs $T_{i}$ and $T$ can be seen as the global counterparts of $T_{w,i}$ and $T_{w}$, respectively. In addition, their constructions are somewhat simpler than the latter.

\paragraph{The construction of $T_{i}$.} For a given $i \in \mathbb{N}$, the 4-Alphabet WA $T_{i}$ is of size $i$, and constructed as follows: 
\begin{itemize}
     \item \textbf{The state space:} $Q = [i+1]$
    \item \textbf{The initial state:} $q_{init} = 1$
    \item \textbf{The transition function:} For a state $q \in Q$, and $(\sigma_{1}, \sigma_{2}, \sigma_{3}, \sigma_{4}) \in \Sigma_{\#} \times \Sigma \times \Sigma \times \Sigma$, we have:
   $$\delta(q, (\sigma_{1}, \sigma_{2}, \sigma_{3}, \sigma_{4})) = 
   \begin{cases}
       q+1 & \text{if} ~~ [q < i \land \Phi(\sigma_{1}, \sigma_{2}, \sigma_{3}, \sigma_{4})] \lor [q=i \land \sigma_{3} = \sigma_{4}] \\
        i+1 &  \text{if} ~~ q = i+1 \land \Phi(\sigma_{1}, \sigma_{2}, \sigma_{3}, \sigma_{4})
   \end{cases}
   $$
   \item \textbf{The final state:} $i+1$
\end{itemize}


\paragraph{The construction of $T$.} The 4-Alphabet WA $T$ is an automaton with a single state, formally defined as follows:
\begin{itemize}
    \item \textbf{The state space:} $Q = \{1\}$,
    \item \textbf{The initial state:} $q_{init} = 1$.
    \item \textbf{The transition function:} For $(\sigma_{1}, \sigma_{2}, \sigma_{3}, \sigma_{4}) \in \Sigma_{\#} \times \Sigma \times \Sigma \times \Sigma$, we have that:
    $\delta(1, (\sigma_{1}, \sigma_{2}, \sigma_{3}, \sigma_{4})) = 1$ holds, if and only if the predicate $\Phi(\sigma_{1}, \sigma_{2}, \sigma_{3}, \sigma_{4})$ is true. 
\end{itemize}

\subsubsection{The construction of $f_{w^{reff}}$}
Given a sequence $w^{ref} \in \Sigma^{*}$, the machine $f_{w^{reff}}$ is defined as an HMM such that the probability of generating the sequence $w^{ref}$ as a prefix is equal to 1. The set of HMMs that meet this condition is infinite. In our case, we opt for an easy construction of an HMM, denoted $f_{w^{reff}}$, belonging to this set.

The state space of $f_{w^{reff}}$ is given as $Q = [|w_{ref}| + 1]$ states. Similarly to the constructions above, each state is associated with a position within the emitted sequence of the HMM. Its functioning mechanism can be described using the following recursive procedure: 
\begin{itemize}
    \item The HMM starts from the state $q = 1$ with probability $1$. The probability of emitting the symbol $w^{ref}_{1}$ and transitioning to the state $q= 2$ is equal to $1$.
    \item For $q \in [|w^{ref}|]$, the probability of generating the symbol $w^{ref}_{q}$ and transitioning to the state $q+1$ is equal to $1$.
\end{itemize}
When the HMM reaches the state $|w^{ref}| + 1$, it generates a random symbol in $\Sigma$ and remains at state $|w^{ref}| + 1$ with probability 1. One can readily verify (by a straightforward induction argument) that this HMM generates infinite sequences prefixed by the sequence $w^{ref}$ with probability 1. 

\paragraph{Complexity.} The running time of this construction is equal to $O(|w^{ref}|)$, and the size of the obtained HMM $f_{w^{reff}}$ is equal to $O(|w^{ref}|)$.