\section{Generalized Complexity Relations of SHAP Variants (Proof of Proposition~\ref{prop:hardnessrelation})} \label{app:sec:generalized}

In this section, we present the proof for Proposition~\ref{prop:hardnessrelation} from the main paper. Let us first restate the proposition:

%In this section, we provide the proof of proposition \ref{prop:hardnessrelation} in the main article. Recall the statement of this proposition:

\begin{unumberedproposition}
    Let $\mathcal{M}$ be a class of models and $\mathcal{P}$ a class of probability distributions such that $\texttt{\emph{EMP}} \preceq_{P}  \mathcal{P}$. %\footnote{The symbol $\preceq_{P}$ signifies "polynomially reducible to"}. Then, 
    Then, \texttt{\emph{LOC-B-SHAP}}($\mathcal{M}$)  $\preceq_{P}$ \texttt{\emph{GLO-B-SHAP}}($\mathcal{M}$, $\mathcal{P}$) and \texttt{\emph{LOC-B-SHAP}}($\mathcal{M}$)  $\preceq_{P}$   \texttt{\emph{LOC-I-SHAP}}( $\mathcal{M}, \mathcal{P}$) .
\end{unumberedproposition}
\begin{proof} 
    (1) \texttt{LOC-B-SHAP}($\mathcal{M}$) $\preceq_{P}$ \texttt{GLO-B-SHAP}($\mathcal{M}$, $\mathcal{P}$). 
    
This result is derived directly by observing that:
      $$\mathbf{\Phi}_{b}(f,i,x^{ref},P_{x}) = \Phi_{b}(f,i,x,x^{ref})$$
      where $P_{x}$ is the empirical distribution induced by the input instance $x$.
    
    
      (2) \texttt{LOC-B-SHAP}($\mathcal{M}$) $\preceq_{P}$ \texttt{LOC-I-SHAP}($\mathcal{M}$, $\mathcal{P}$):
         The result is obtained straightforwardly by noting that:
         $$\phi_{i}(f,i,x,P_{x^{ref}}) = \phi_{b}(f,i, x, x^{ref})$$
         where $P_{x^{ref}}$ is the empirical distribution induced by the reference instance $x^{ref}$.
\end{proof}