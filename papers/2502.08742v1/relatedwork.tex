\section{Related Work}
\label{Related-Work}
The scientific community has proposed several models for monitoring CPS to ensure their proper operation. These models can be broadly categorized into three main types:

\begin{enumerate}
 \item \textbf{Polling Models}: They periodically query the network's monitoring services to assess the status of individual nodes.
 \item \textbf{Trust Management Models}: They facilitate the exchange and sharing of trust and reputation information based on nodes' behavior within the network, such as their communication and cooperation with other nodes.
 \item \textbf{Community Models}: They assign roles to participating nodes, each governed by distinct policies that define permitted and prohibited actions. This approach enables the formation of mission-oriented, dynamically self-managing communities tailored to application-specific requirements.
\end{enumerate}

\subsection{Polling Models}
Big Brother (BB) \cite{MacGuire1997} was the first monitoring system to use the Internet as an interconnection point for users. This system allows users to gain a holistic view of network operations in a simplified manner. BB employs a client-server model and incorporates both push and pull data transmission mechanisms. Clients periodically send status updates to the monitoring server, while the network is assessed by polling all available monitoring services. Results are displayed in a central management unit, with each reference including an expiration date to indicate its validity. Typically, network monitoring models require dedicated software installation, limiting administrative flexibility. A distinguishing feature of BB is its web-based interface, which provides real-time status visualization accessible from any internet-connected device.

Zenoss \cite{Badger2008} is an open-source platform that utilizes the Simple Network Management Protocol (SNMP) \cite{Fedor1990,Stallings1993} to monitor networks, servers, applications, and services. The primary advantage of Zenoss is its open architecture, which allows for extensive customization. Since SNMP serves as the underlying protocol, the notification mechanism supports both polling and traps. Additionally, Zenoss enables the storage of historical data, facilitating performance analysis and insight into past operations.

\subsection{Trust Management Models}
The Agent-Based Trust and Reputation Management (ATRM) model \cite{Boukerch2007} is designed for wireless sensor networks (WSNs), where trust and reputation are managed locally with minimal overhead in terms of messaging and time delays. However, as mobile agents traverse the network and execute on remote nodes, they must be initiated by trusted entities.

Another agent-based trust model for WSNs \cite{Chen2007} employs a watchdog mechanism to observe node behavior and broadcast trust ratings. Sensor nodes receive these ratings from agent nodes, which are responsible for trust information collection, processing, and dissemination.

The reputation-based scheme DRBTS \cite{Srinivasan2006} enables beacon nodes (BNs) to monitor each other and provide trustworthiness information to sensor nodes (SNs) based on a voting mechanism. To validate a BN's information, a sensor node must receive trust votes from at least half of its common neighbors.

BTRM-WSN \cite{Marmol2011} is a bio-inspired trust and reputation model for WSNs that identifies the most reliable path to the most reputable node offering a particular service. Each node maintains a trace—analogous to pheromones in biological systems—for each neighbor.

The CONFIDANT model \cite{Buchegger2002} extends routing protocols with a reputation-based mechanism to isolate nodes that deviate from expected behavior. Each node monitors the behavior of its next-hop neighbor, with trust relationships and routing decisions based on observed, experienced, or reported routing and forwarding actions.

The SORI scheme \cite{He2004} incentivizes packet forwarding and discourages selfish behavior. A node's reputation is quantified using objective measures, and reputation propagation is secured via a hash chain-based authentication mechanism.

\subsection{Community Models}
The abstract model \cite{Schaeffer2008} for policy-based collaboration relies on task-oriented roles within sensor communities. Each role is associated with two policy classes:
\begin{enumerate}
 \item \textbf{Obligations}: Define adaptive actions that a role must perform in response to specific events (e.g., device failures or contextual changes).
 \item \textbf{Authorizations}: Define permitted or prohibited actions between roles. The framework supports a unified specification mechanism for both application-specific and management roles.
\end{enumerate}

Node role assignment is constrained to ensure integrity, confidentiality, and availability. Cardinality constraints specify the minimum and maximum number of nodes per role, while separation-of-duty constraints prevent conflicting role assignments. This role-based model explicitly distributes management responsibilities across community participants, enhancing scalability and robustness by avoiding single points of failure.

\subsection{Comparison of the Existing Models}
Both Zenoss and BB have been extensively used in industry. Due to their wide applicability and ease of installation, they remain integral to many monitoring frameworks \cite{Gupta2015}. Comparative analysis indicates that Zenoss is superior to BB for several reasons. Notably, Zenoss supports IPv6, which is crucial for managing the vast number of devices in a CPS. Furthermore, Zenoss leverages SNMP features to collect remote device information, making it a robust monitoring solution.

Trust management models play a crucial role in safeguarding large-scale CPSs against malicious attacks. These mechanisms foster cooperation among distributed entities, identify unreliable sources, and aid decision-making. The ATRM model mitigates network congestion by optimizing communication latency. DRBTS is well-suited for dense networks and can adapt to specific security needs. BTRM-WSN meets CPS requirements for security, scalability, and resource efficiency, consistently delivering robust and accurate results in dynamic, variable, and static networks. Even when malicious servers exceed 60\%, reliable server selection remains above 90\%, with a standard deviation of approximately 7.5\%. The CONFIDANT model effectively detects, warns, and responds to data forwarding and routing attacks. While introducing slight communication delays, it remains scalable and performs well even when 60\% of CPS nodes are compromised. The SORI model efficiently identifies and addresses problematic nodes using three key mechanisms:

\begin{enumerate}
 \item Objective quantification of node reputation.
 \item Secure reputation propagation via an authentication hash chain.
 \item Localized reputation propagation to minimize communication delays.
\end{enumerate}

Community models facilitate autonomous system collaboration through dynamic community formation. Their structural models, task distribution, and communication frameworks offer promising avenues for designing large-scale CPSs. Architectural models abstract sensor functions, enhancing interoperability and reuse. Collaborative CPS architectures can retrospectively implement new structures, streamlining large-scale CPS composition.

While polling, trust management, and community models offer distinct advantages, each has limitations. Polling models are vulnerable to node cloning, leading to false results. Trust management models enhance security but remain insufficient in addressing emerging threats. Community models rely on predefined hierarchies, limiting adaptability. Given evolving security threats \cite{Nur2016,Miao2017} and the emergence of new vulnerabilities \cite{Mitchell2016,Lu2016}, more effective and secure models are required to meet contemporary CPS security demands.