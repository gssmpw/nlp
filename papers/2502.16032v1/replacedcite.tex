\section{Related Work}
% \subsection{Residual Learning}
% In the field of deep learning, the Residual Network (ResNet)____ proposed by He Kaiming et al. in 2015, has successfully addressed the issue of performance degradation with increasing network depth by introducing the framework of residual learning. The core innovation of ResNet lies in the residual connections, which allow signals within the network to bypass one or more layers and directly transmit, thereby maintaining the flow of gradients and feature propagation within the network. With further research, various variants of ResNet, such as ResNeXt____ and DenseNet____, have emerged, enhancing feature representation capabilities in different ways. Moreover, the concept of residual connections has been further expanded and understood, such as counterfactual generation____ or through skip connections to strengthen the reuse of features and the flow of information____. 

% \subsection{Pre- and Post-contrast MRI Sequences in Clinical Practice}
% In clinical practice, utilizing both pre- and post-contrast MRI images or subtraction sequences for diagnosis is a common approach. According to ____, MRI provides excellent soft tissue contrast, which helps differentiate parathyroid adenomas from other neck structures. Comparing pre- and post-contrast sequences aids in distinguishing parathyroid adenomas from thyroid nodules and lymph nodes due to their different enhancement patterns. ____ mentions that in the diagnosis of glioblastoma, the contrast information from enhanced images allows for better quantitative analysis of tumor size or volume, even when vascular permeability is reduced. References____, which discuss the diagnosis of testicular tumors, gastrointestinal stromal tumors, breast cancer, and endometriosis respectively, offer similar descriptions. Therefore, we can conclude that in clinical diagnostics, the contrast information between pre- and post-enhancement MRI images is crucial for the detection and diagnosis of many diseases.

\vspace{-0.2cm}