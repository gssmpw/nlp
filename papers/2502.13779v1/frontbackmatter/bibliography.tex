%********************************************************************
% Bibliography
%*******************************************************
% work-around to have small caps also here in the headline
\documentclass[../../main.tex]{subfiles}
\begin{document}

\manualmark%
\markboth{\spacedlowsmallcaps{\bibname}}{\spacedlowsmallcaps{\bibname}} % work-around to have small caps also
%\phantomsection 
\refstepcounter{dummy}
\addtocontents{toc}{\protect\vspace{\beforebibskip}} % to have the bib a bit from the rest in the toc
\addcontentsline{toc}{chapter}{\tocEntry{\bibname}}
\label{app:bibliography}
{%
  \emergencystretch=1em%
  \printbibliography%
}


\newcommand{\bibstyleheader}[1]{%
  \section*{\normalsize #1}  % Adjust the font size to normalsize (or another size if desired)
  \markright{#1} % Update the header with the section name if desired}
  }
\bibstyleheader{Generative AI Used}
The following tools have been used during the writing of this dissertation. Prompts were, or in similar spirit to, "Correct the grammar and syntax [paragraph]", or "Give feedback on [paragraph]." Furthermore, they have been used to translate the English abstract into German. 
\begin{enumerate}
    \item \fullcite{openai2023chatgpt}
    \item \fullcite{perplexityai}
\end{enumerate}



\end{document}