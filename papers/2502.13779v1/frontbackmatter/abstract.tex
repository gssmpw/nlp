%*******************************************************
% Abstract
%*******************************************************
%\renewcommand{\abstractname}{Abstract}

\documentclass[../main.tex]{subfiles}
\begin{document}


\pdfbookmark[1]{Abstract}{Abstract}
\begingroup
\let\clearpage\relax
\let\cleardoublepage\relax
\let\cleardoublepage\relax

\chapter*{Abstract}
The balance between user agency and system automation in interactive intelligent systems is crucial for intuitive and efficient interactions. While fully automated systems could potentially offer greater efficiency and demonstrably improved performance, making them perfect is notoriously hard. The inevitable shortcomings of automated systems diminish usability and overall experience, thereby compromising users' perceived self-determination. Conversely, tools, systems that rely entirely on user agency and have no level of automation, though offering full control to the user, can be inefficient and fail to enhance the user's capabilities. Hence, for effective human-AI interactions, we need to find a balance between user agency and system automation. The question we address in this dissertation is "How can we balance user agency and system automation for the interaction with intelligent systems?"

We approach this challenge through four main contributions. First, we introduce a novel spherical electromagnet capable of generating adjustable forces on an untethered tool, allowing users to feel grounded forces while maintaining full agency. Second, we develop an integrated sensing and actuation system that tracks a passive magnetic tool in 3D space while simultaneously delivering haptic feedback, eliminating the need for external tracking. Third, we propose an optimal control method for electromagnetic haptic guidance systems that balances user input and system control, allowing users to adjust trajectories and speed as needed. Finally, we present a model-free reinforcement learning approach for adaptive user interfaces that learns interface adaptations without relying on heuristics or real user data.

Our findings, based on simulations and user studies, suggest that the shared control of intelligent systems has the potential to significantly outperform naive control strategies. Thus, we contribute methodologies that find an agency-automation trade-off and pave the way for more interaction with intelligent systems. Our research demonstrates that integrating models of human behavior, either explicitly or implicitly, into control strategies enables intelligent systems to better account for user agency. We show that the trade-off between user agency and system automation is not solely an algorithmic problem but must also be considered in the engineering of physical devices and interface design. We advocate for an integrated end-to-end approach to interaction with intelligent systems that incorporates algorithmic, engineering, and design perspectives.
\endgroup

\cleardoublepage%

\begingroup
\let\clearpage\relax
\let\cleardoublepage\relax
\let\cleardoublepage\relax

\begin{otherlanguage}{ngerman}
\pdfbookmark[1]{Zusammenfassung}{Zusammenfassung}
\chapter*{Zusammenfassung}
Das Gleichgewicht zwischen der Handlungsfreiheit des Benutzers und der Systemautomatisierung in interaktiven intelligenten Systemen ist entscheidend für intuitive und effiziente Interaktionen. Während vollautomatisierte Systeme potenziell größere Effizienz und nachweislich verbesserte Leistung bieten könnten, ist es bekanntermaßen schwierig, sie perfekt zu machen. Die unvermeidbaren Mängel automatisierter Systeme verringern die Benutzerfreundlichkeit und das Gesamterlebnis und beeinträchtigen dadurch die wahrgenommene Selbstbestimmung der Benutzer. Umgekehrt können Werkzeuge, d.h. Systeme, die sich vollständig auf die Benutzerautonomie stützen und kein Automatisierungsniveau aufweisen, zwar die volle Kontrolle für den Benutzer bieten, aber ineffizient sein und die Fähigkeiten des Benutzers nicht erweitern. Daher müssen wir für effektive Mensch-KI-Interaktionen ein Gleichgewicht zwischen Benutzerautonomie und Systemautomatisierung finden. Die Frage, die wir in dieser Dissertation behandeln, lautet: "Wie können wir Benutzerautonomie und Systemautomatisierung für die Interaktion mit intelligenten Systemen ausbalancieren?"

Wir nähern uns dieser Herausforderung durch vier Hauptbeiträge. Erstens stellen wir einen neuartigen sphärischen Elektromagneten vor, der in der Lage ist, einstellbare Kräfte auf ein kabelloses Werkzeug auszuüben und es den Benutzern ermöglicht, geerdete Kräfte zu spüren, während sie volle Autonomie behalten. Zweitens entwickeln wir ein integriertes Erfassungs- und Betätigungssystem, das ein passives magnetisches Werkzeug im 3D-Raum verfolgt und gleichzeitig haptisches Feedback liefert, wodurch die Notwendigkeit einer externen Verfolgung entfällt. Drittens schlagen wir eine optimale Kontrollmethode für elektromagnetische haptische Führungssysteme vor, die Benutzereingaben und Systemsteuerung ausbalanciert und es den Benutzern ermöglicht, Trajektorien und Geschwindigkeit nach Bedarf anzupassen. Schließlich präsentieren wir einen modellfreien Ansatz des verstärkenden Lernens für adaptive Benutzeroberflächen, der Schnittstellenanpassungen erlernt, ohne sich auf Heuristiken oder reale Benutzerdaten zu stützen.

Unsere Ergebnisse, basierend auf Simulationen und Benutzerstudien, deuten darauf hin, dass die geteilte Kontrolle intelligenter Systeme das Potenzial hat, naive Kontrollstrategien deutlich zu übertreffen. Somit tragen wir Methoden bei, die einen Kompromiss zwischen Autonomie und Automatisierung finden und den Weg für mehr Interaktion mit intelligenten Systemen ebnen. Unsere Forschung zeigt, dass die Integration von Modellen menschlichen Verhaltens, sei es explizit oder implizit, in Kontrollstrategien es intelligenten Systemen ermöglicht, die Benutzerautonomie besser zu berücksichtigen. Wir zeigen, dass der Kompromiss zwischen Benutzerautonomie und Systemautomatisierung nicht nur ein algorithmisches Problem ist, sondern auch bei der Entwicklung physischer Geräte und dem Interface-Design berücksichtigt werden muss. Wir befürworten einen integrierten End-to-End-Ansatz für die Interaktion mit intelligenten Systemen, der algorithmische, technische und Design-Perspektiven einbezieht.
\end{otherlanguage}

\endgroup

\vfill

\end{document}
