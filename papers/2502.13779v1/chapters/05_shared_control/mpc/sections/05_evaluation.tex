\section{Evaluation}
We first evaluated if our optimization scheme is beneficial for users in providing haptic guidance compared to a no-feedback baseline.
In a second experiment, we compared our method with an open-loop and a closed-loop approach.

\subsection{Experiment 1 - Haptic feedback}
We compared our MPCC formulation with a no-feedback baseline to gather insights on task performance and user perception.
Users were asked to draw several shapes (see \figref{fig:usertest_examples}) and we evaluated accuracy and subjective feedback. 

\begin{figure}[!t]
    \centering
    \includegraphics[width=\columnwidth]{\dir/figures//study-conditions.pdf}   
     \caption{Shapes of our user tests. 
    The drawing surface only contained sparse visual references (shown in blue) and starting positions (orange).}
    \label{fig:usertest_examples}
    \vspace{-1em}
\end{figure}

\subsubsection{Procedure and tasks}
We recruited 12 participants from the local university, all without professional drawing experience. 
Users were given an introduction to the system functionality and got to experience the system in a self-timed training phase.
During the experiment we asked each participant to draw six basic shapes, each with and without haptic feedback.
The presentation order of shapes and interface condition was counterbalanced. 
The drawing surface (white piece of paper) only contained a starting point and, in the case of more complex shapes, limited additional visual guidance (see \figref{fig:usertest_examples}).
Furthermore, the participants were shown a scaled version during task execution (scaled to prevent 1:1 copying). 
After the full experiment, users completed a questionnaire on their preference.

\subsubsection*{Quantitative Results}
\label{sc:quantitative_results}
We compute the Hausdorff-like distance \cite{rockafellar2009variational} between the drawn path and the reference path as error metric. 
To make the metric robust to drawing speed, we re-sample both paths equidistantly.
To ensure fairness we also compute the inverse distance (reference to drawn path). 
A Kolmogorov-Smirnov test \cite{kolmogorov1933sulla} indicated that the set of uni-directional distances is not significantly different from the set of inverse distances.
We therefore only report uni-directional distances. 
% 
The quantitative results for each target averaged over all participants are summarized in Table \ref{tab:accuracy}.
Our method on average resulted in a 66\% ($\pm$ 24.5\%) lower error, \ie it improve the average point-to-path difference by \unit[1.54]{mm}.
A two-way ANOVA on the mean error revealed a main effect for the feedback type (F=46.187, p<.001) and for the shapes (F=11.771, p < .001). 
Post-hoc analysis revealed that the line was statistically significantly different from all other shapes.
This indicates that our method is beneficial for non-trivial shapes.

\begin{table}[!t]
\caption{Mean accuracy ($mm$). * indicates statistical significance ($\alpha .05$).}
\vspace{.5em}
    \begin{tabular}{l|cc|cc|c}
    %\toprule
    \multicolumn{1}{c}{} &\multicolumn{2}{c}{With}&\multicolumn{2}{c}{Without}&\multicolumn{1}{c}{}\\
    \midrule
Scenario& Mean & SD & Mean & SD & Err $\%$ \\
 \midrule
 Circle* & \textbf{2.19} & 0.90 & 4.26 & 2.39 & 0.51 \\
 Line  & 1.18 & 0.80 & \textbf{1.03} & 0.84 & 1.15  \\
 Spiral* & \textbf{2.55} & 0.75 & 4.38 & 1.64 & 0.58  \\
 Sinus*  & \textbf{2.53} & 0.70 & 5.08 & 2.19 & 0.50  \\
 Dog*  & \textbf{2.31} & 0.54 & 3.81 & 1.32 & 0.60 \\
 Ellipse*  & \textbf{2.40} & 0.56 & 3.84 & 1.22 & 0.62 \\
 %\bottomrule
\end{tabular}
\label{tab:accuracy}
\end{table}

\subsubsection*{Qualitative Results}
A brief exit interview shows that users subjectively rate the system favorably, on a 5-point Likert scale (5 is best), in terms of accuracy ($4.33 \pm0.62$), speed ($4.00\pm0.91$), force ($3.50\pm0.86$) and overall performance ($4.50\pm0.9$).
While we acknowledge that this might be in part due to novelty effects, we believe that the quantitative results indicate that our system is beneficial for users in general.
The ratings indicate that participants generally see benefit in our approach and are not disturbed or hindered when using our approach.

% ##########################################################################
% ##########################################################################

\subsection{Experiment 2: Comparison of control strategies}
\label{Sc:preliminary_user_evaluation}
In this second experiment, we compared our time-free closed-loop optimization strategy to a simpler MPC variant and our implementation of dePENd \cite{yamaoka2013depend}, denoted as dePENd$^{*}$. 

\subsubsection{Procedure and tasks}
We asked twelve new participants (students and staff from a local university) to draw one complex shape (dog in \figref{fig:usertest_examples}) in three different conditions: \emph{dePENd$^{*}$}, time-dependent closed loop (\emph{MPC}), and time-free closed loop (\emph{Ours}), counterbalanced using a latin square. 
\add{The speed of the system in the time-dependent cases was decided empirically based on pre-tests to work well at regular drawing speeds.}
After receiving instructions and a brief training, participants completed the three drawings.
Participants were also encouraged to provide comments during the individual conditions.

\subsubsection{Data collection}
We analyze three measures: 
1) the mean distance from the pen to the path, 2) the mean distance from the pen position projected onto the path and the setpoint along the path, denoted as $d(pen, \mathbf{s(\theta)})$, and 3) the mean distance from the pen to the electromagnet. 
By taking the mean of the error terms over subjects we ensured equal numbers of datapoints, accounting for differences in speed. 
Participants were instructed to draw at roughly constant speed.
This was done to ensure fairness in comparing our approach with the open-loop approach, which would deteriorate if the variability of the drawing speed were to high.
Note that our approach does generally not require this assumption.
%
% \begin{figure}[h]
%     \centering
%     \includegraphics[width=.8\columnwidth]{Figures/control-strategies-metrics.pdf}
%     \caption{Overview of the different metrics for the preliminary user evaluation of the different control strategies.}
%     \label{fig:control-strategies-metrics}
% \end{figure}

%}\footnote{\edt{In our full implementation and haptic feedback experiments this assumption is not necessary. However, the metric used here assumes time-independent datapoints.}} \edt{ We also gathered qualitative feedback in the form of a semi-structured interview. A one-way ANOVA with Kruskal-Wallis test was performed for data analysis.}

\subsubsection{Quantitative results}

\begin{table}[!t]
    \centering
    \caption{Mean distances in $mm$ for Experiment 2). }
    \begin{tabular}{l|ccc}
         &|pen-path|& d(pen, $s(\theta)$) & |pen-em| \\
         \midrule
         \emph{dePENd$^{*}$} & $4.1(\pm 0.7)$ & $38.0(\pm 56.9)$  &$38.2(\pm 25.1)$ \\ 
         MPC & $3.9(\pm 1.3)$& $45.0(\pm 50.8)$ & $8.6(\pm 1.6)$ \\ 
         \textit{Ours} & \textbf{2.0}$(\pm 0.6 )$& \textbf{6.2}$(\pm 0.8)$ & \textbf{4.6}$(\pm 0.9)$\ 
    \end{tabular}
    \label{tab:strategy_results}
\end{table}

Table \ref{tab:strategy_results} summarizes our quantitative findings. 
Not surprisingly, the distance from the electromagnet to the pen and  $d(pen, \mathbf{s(\theta)})$ for \emph{dePENd$^{*}$} is quite large. 
Since the force exerted on the pen falls-off quadratically with distance, participants often lost any haptic guidance early on, confirmed via user comments such as ``I don't feel anything'' (P3) and ``Is the system on?'' (P6). 

A Kruskal-Wallis test revealed that our approach has the highest accuracy compared to \emph{dePENd$^{*}$} and MPC (H(2)=20.76, p<.001).
Furthermore, the setpoint $\mathbf{s}(\theta)$ (H(2)=7.362, p<.05) and the electromagnet (H(2)=27.12, p <.001) are closest to the pen using our approach. 
Thus our time-free formulation overcomes both problems of wrong setpoints (\emph{MPC}) and a run-away electromagnet (\emph{dePENd$^{*}$}).
 Figure \ref{fig:single_user_control} shows one typical example of a user. 
 Both the distance along the path and the pen-magnet distance fluctuate strongly for \emph{dePENd$^{*}$} and  \emph{MPC} control strategies.
 Our approach yielded a continuously low error.

While \emph{MPC} reduces the distance from the pen to the magnet, it does not optimize for the progress along the path and hence may pull the pen into undesired directions. 
Furthermore, we saw that \emph{MPC} produced extreme corner cutting behavior to catch up to the advancing setpoint.
Both \emph{dePENd$^{*}$} and \emph{MPC} also produce results with high standard deviations.
This is likely due to the absence of direct coupling between user feedback and path progress, which makes it possible for the user to lag behind the setpoint significantly (albeit at the cost of reduced force feedback). 
In our approach, the path progress is adjusted to the user's drawing speed, drastically reducing the standard deviation and in consequence ensuring delivery of force feedback throughout the drawn path. 
 
\begin{figure}[!t]%
	\centering
    \includegraphics[width=1\columnwidth]{\dir/figures//1_path_distance_pen_stheta-02.pdf}%
    \caption{Comparison of error (path distance pen-$\mathbf{s}(\theta)$) over time for a single participant (P1). The inverse u-shape illustrates that the setpoint $\mathbf{s}(\theta$) moves at a different speed than the user for \emph{dePENd$^{*}$} and MPC. The data is smoothed to increase readability.} 
    \label{fig:single_user_control}
    \vspace{-1em}
\end{figure}

\subsubsection{Qualitative results} 
From our observations we saw that $\mathbf{s}(\theta)$ was either in front or behind the user for MPC. 
This was also confirmed in our interview, where people especially commented on the MPC strategy: ``The system tries to push me in the wrong direction'' (P2) and ``It is counteracting me'' (P11). 
In contrast with our formulation the magnet remains always slightly ahead of the pen, resulting in users commenting on our approach as the most preferred condition. 
In the words of one subject this is: ``since I still had control'' (P9). 

In summary, theses initial results indicate that our approach outperforms existing open-loop and time-dependent closed-loop approaches. 
\emph{dePENd$^{*}$} causes numerous problems, including users not perceiving any haptic feedback. 
This is especially troublesome in settings where autonomy is desired. 
In MPC the haptic feedback is perceived, but can be erroneous. 
This is especially evident when users do not conform to the expected behavior. 
We plan to perform more in-depth experiments to investigate, for which applications our approach can be especially beneficial, and for which levels of autonomy.

% \vspace{5em}