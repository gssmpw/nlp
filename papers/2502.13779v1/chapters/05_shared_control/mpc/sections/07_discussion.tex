\section{Discussion}
Our experiments indicate that the proposed approach indeed increases accuracy in drawing tasks and that users perceive the system favorably. 
While our system increased users' accuracy for complex shapes, it did not yield any improvements for the straight line. 
This limitation can be attributed to the maximum speed of the linear stage, as indicated by user feedback.
In our interviews, some users indicated that they had the feeling that their drawings without feedback were more accurate once they experienced the haptic guidance.
This suggests the potential for short-term muscle memory development when using our haptic guidance system.
Long-term learning is a very interesting area to explore for our approach and haptic guidance systems in general. We plan to conduct such experiments in the future.
 
Our experiments currently focused on drawing and sketching applications.
We believe our control strategy can be beneficial for a wide range of applications.
We started exploring the usage of our approach with a tablet and digital stylus.
Further experiments are necessary to determine the levels of complexity at which our approach is most beneficial.
Allowing users to adjust their input on-demand is crucial, particularly since systems typically lack complete knowledge of the users' target paths.
Our approach is a first step in the direction of balancing user input and system control for haptic guidance systems, and can be extended to other devices beyond electromagnetic systems if appropriate force models are provided.

In terms of hardware, the speed of the linear stage was a limiting factor. Future improvements may involve faster, more compact linear stages or a matrix of stationary electromagnets. The former requires no changes to our formulation, while the latter necessitates modifications to the EM model and dynamics model. A matrix of electromagnets could pave the way for a thinner form-factor design, which is an exciting research direction.

Addressing the hardware-induced speed limitation will open new avenues for efficient closed-loop control strategies, since faster pen motion would also tighten the latency and accuracy budget. 
Incorporating a mechanism to reconstruct the tilt of the pen would enhance sensing capabilities.
This could be achieved for example via an accelerometer built into the pen or via a grid of hall-sensors underneath the surface. 
Information on the pen tilt could then be combined with the angle dependent formulation of our EM model. 
Furthermore, we believe there are many research opportunities in combining our approach with ink beautification approaches (\eg~ \cite{simo2016learning,simo2018mastering,xing2015autocomplete}). 
Particularly interesting would be to leverage fully predictive models for non-drawing applications (\eg DeepWriting \cite{Aksan:2018:DeepWriting}). 

Future work could explore combining our approach with various types of haptic feedback, either environment mounted or body-worn, and different form factors such as spherical electromagnets \cite{Langerak:2020:Omni, zarate2020contact}. 
Electromagnetic feedback in combination with spatial actuation maybe interesting in other settings. 
For example, a magnet mounted to a robotic arm could deliver contact-less feedback in VR scenarios. 
It would also be interesting to investigate how to best exploit the system capabilities in the context of motor memory and learning.
All these scenarios make it necessary that a system interactively reacts to user input.
Our approach enables such applications, and can generalize to such systems that go beyond 2D haptic guidance systems.

Finally, our optimization is subject to system dynamics, including user influence on pen position. Explicitly incorporating user behavior into system dynamics is challenging due to its non-linear nature. In the next chapter, we will explore a model-free learning-based approach that can learn underlying task structures and human behavior without explicit system dynamics.

% \section{Conclusion}
We have proposed a novel optimization scheme for electromagnetic haptic guidance systems based on the MPCC framework.
Our approach strikes a balance between user input and system control, allowing users to adjust their trajectory and speed on-demand.
It optimizes system states and inputs over a receding horizon by solving a stochastic optimal control problem at each timestep.
Our formulation provides dynamically adjustable forces and automatically regulates magnet position and strength.
It can be evaluated analytically and is hence suitable for iterative, real-time optimization approaches. 
We implemented our approach on a prototype hardware platform and experimentally demonstrated that the feedback is well-received by users and offers higher accuracy compared to open-loop and time-dependent closed-loop approaches.
We believe our approach offers a principled method for haptic guidance, enabling users to retain agency while receiving unobtrusive assistance. This approach has broad applications in areas such as drawing, sketching, writing, and guidance via virtual haptic tools.
%
%
%We have proposed MagPen, a system that delivers dynamically adjustable guidance in drawing and sketching tasks. We have detailed our hardware setup and discussed a novel model of the electromagnetic interactions in the system. 
%The proposed model can be evaluated analytically and is hence suitable for iterative, real-time optimization approaches. 
%We have furthermore demonstrated that the assumptions of dipole magnets and an upright pen lead only to a small approximation error. 
%However, we have included an angle dependent formulation that maybe used in future work, where pen-tilt information is available. 
%
%Our experiments have shown that the \hl{proposed} hardware-software solution is effective in improving accuracy and in guiding users in a variety of drawing tasks, without taking away agency and control from the user. We believe this is an interesting first step towards many exciting applications of electromagnetic haptic feedback. In order to foster future research we will release all hardware schematics and software source code to the public. 




%Two aspects from the exit interviews are noteworthy. 
%First, there is a high standard deviation in how users rated the perceived force. 
%We hypothesize that this is due to the way users operate the pen, with some leveraging the full arm and others rely more on the wrist. 
%We note that our palm rejection implementation is simple and may have contributed to this. 
%Furthermore, some users indicated that they had the feeling that their drawings without feedback were more accurate once they experienced the haptic guidance, indicating the possibility of short-term muscle memory. 
%Long-term learning however is difficult to evaluate experimentally and goes beyond the scope of this paper. 
%
%Finally, during our experiments we noticed a tendency to cut corners, well illustrated in the case of the sinusoidal. 
%To unpack this issue further, we performed an experiment in simulation, with the user model from Eq. \ref{eq:user_sim}, tracing references with increasingly sharp angles (see \figref{fig:angle_error}). 
%The plot clearly shows an increase in error with increase in curvature. 
%It has been shown that humans trade-off speed and accuracy in tracing tasks and that they slow down when tracing high-curvature paths \cite{accot1997beyond}. 
%Currently our implementation does not take curvature of the reference into account but it would be straightforward to penalize the progress $\theta$ along the reference according to its curvature.  
%
%\begin{figure}[t]
%    \centering
%    \includegraphics[width=\columnwidth]{Figures/angle.png}
%    \caption{Curvature dependent error. Insets increasingly sharp cornered references (dotted lines) and the resulting pen trajectory (in simulation).  x-axis is normalized for cord length, so that all angles can be directly compared. Bottom: error over 9 different levels of curvature (in degrees).}
%    \label{fig:angle_error}
%\end{figure}
