% !TEX root = ../Main.tex

%%%%\JZ{Hardware evaluation moved into hardware}
\section{Applications}
To further demonstrate the potential of our approach we illustrate possible usage-scenarios including calligraphy, outlining and inking. 
Finally, we combine the haptic feedback mechanism with a simple digital drawing application to initially explore the possibility of dynamic references.    

\subsubsection*{Calligraphy}
\figref{fig:caligraphy} illustrates writing of flourished characters, with only minimal visual guidance (single starting point). 
Our system takes the character as input, users can then draw at their desired speed. 
Although an offset from the reference path remains, the lines are smooth and the overall shape is close to the desired characters. 

\begin{figure}[!t]
    \centering
        \includegraphics[width=\columnwidth]{\dir/figures//apps-03-caligraphy.jpg}    
        \caption{Our approach can be used as a guidance system for calligraphy, where users either follow a target path very closely, or deviate if desired.}
    \label{fig:caligraphy}
\end{figure}
%
%\subsubsection*{Drawing teaching aid:}
%Connect-the-dots exercises are often used to teach children motor skills as well as stroke ordering. \figref{fig:dots} shows results from a similar exercise performed with our system, albeit using much fewer dots for visual guidance than a paper version.
%

\subsubsection*{Outlining \& inking}
\figref{fig:dragon} illustrates the effect of two core capabilities of the proposed approach. 
Here we first outline the proportions of the dragon head (gray guidance lines) and then use different pens to ink-in the details. 
Note that the system provides haptic guidance but allows the user to draw the shape in different styles (\eg the ears of the two upper dragons) and with varying high-frequency detail, while maintaining similarity to the reference shape. 
This is a direct consequence of using time-free closed loop control approach.
In this case, all four variants were drawn without changes to the system or desired path. 
\begin{figure}[!t]
    \centering
        \includegraphics[width=\columnwidth]{\dir/figures//apps-01-drawing.pdf}
    \caption{Different variants of the same dragon, drawn with identical system settings by a novice. Each pair of drawings used with different tools. First a pencil for proportions and a fine-liner (top) or pencil (bottom) to ink-in details. Multi-stroke lines are achieved by approaching each separate instance as a new drawing.}
    \label{fig:dragon}
\end{figure}


\subsubsection*{Virtual tools}
Using a digital tablet with capacitive display (\figref{fig:tablet}) we explore integrating dynamically changing references. 
In a sketching application, artist select different virtual tools, and position and configure these anywhere. 
The canvas and the haptic feedback system then pull the stylus towards these virtual guides. 
In \figref{fig:tablet}, the user has selected a tool that helps them when drawing an ellipse that snaps to a previous part of the drawing, both visually and in terms of haptics.

\begin{figure}[!t]
    \vspace{-.5em}
    \centering
    \includegraphics{\dir/figures//apps-02-digital.jpg}    \caption{Virtual tools can be used to dynamically construct a reference path combining haptic and visual feedback. Here  a simple drawing application combines freeform sketching with different virtual rules and guides that can be felt by the user.}
        \label{fig:tablet}
    \vspace{-.5em}
\end{figure}

% \subsubsection*{Topography Exploration}
% \figref{fig:topo} shows our in system can be used to explore a landscape. In this implementation the user feels drag on an ascending slope and gets pulled forward along a descending slope. For this application the desired path is updated online according to the slope of the terrain. Note, that therefore the path is unknown to the algorithm and changes with every iteration. 

% \subsubsection*{Pong}
% \figref{fig:pong} illustrates how our system can be used in the context of games. In this case, it is the classic Pong. However, it can be extended to a large variety. In this example we pull the player to the same y-position as the current ball position. For this purpose we set $w_{\dot{\theta}}$ to $0$ and thereby eliminate the forward pull. 


