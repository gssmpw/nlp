\section{Implementation} \label{sec:Implementation}
In this section, we detail our electromagnetic force model used in our optimal control scheme as well as the implementation of our hardware prototype.

\subsection{Electromagnetic force model}\label{sc:em_model}
Our approach requires a model of the interaction between a variable-strength electromagnet (EM) and the permanent magnet in the stylus that is sufficiently accurate and can be evaluated in real time.
Accurately modeling the non-linear EM field of the electromagnet's core is typically done through finite element analysis (FEA), which cannot be performed in real time.
Similarly, precomputing the volumetric map of the EM field $\mathbf{B_m}$ via FEA for all levels of electrical current is not computationally feasible.
We therefore contribute a novel fast approximate yet accurate electromagnetic model that provides a good balance between speed and accuracy to enable haptic guidance in applications such as writing or sketching.

In general, we aim at finding the actuation force on the pen $\mathbf{F_p}$, which is given by integrating over the volume of the permanent magnet in the pen:
\begin{equation}
    \mathbf{F_p} = \iiint \nabla \left( \mathbf{M_p} \cdot  \mathbf{B_m}(\cdot)\right) dxdydz , \label{eq:gradB2}
\end{equation}

\noindent where $\mathbf{M_p}$ is the magnetization of pen magnet and $\mathbf{B_m(\cdot)}$ is the EM field evaluated at the pen position, which is too costly to evaluate in real time.
Our model approximates this underlying physical phenomena, can be efficiently evaluated at every iteration of our optimization procedure and provides a very good fit to empirical data.
In this section, we briefly discuss the most important aspects of our model, for a full derivation and analysis we refer readers to the Appendix \ref{ap:em_model}.

We make the following two assumptions in our derivation:
\begin{inparaenum}[1)]
    \item the electromagnet and the permanent magnet can be approximated as dipoles (\ie oriented point magnets), and
    \item for the smaller dipole (the permanent magnet in the pen) the out-of-plane vector component is much larger than the in-plane counterpart. This allows us to use only the vertical component in the calculation of the force.
\end{inparaenum}

%
\begin{figure}[!t]
    \centering
    \includegraphics[width=\columnwidth]{\dir/figures//dipole-dipole.pdf} \\
    \caption{Illustration of the model to compute the force $\mathbf{F_p}$ on dipole $\mpBold$ due to dipole $\mmBold$. }
    \label{fig:dipole_dipole}
    \vspace{-1em}
\end{figure}

The first assumptions allows us to use the standard model by Yung~\etal~\cite{yung1998analytic} to compute the force exerted by the electromagnet $\mmBold$ onto the pen $\mpBold$ (see \figref{fig:dipole_dipole}) as:
%
\begin{multline}
   \mathbf{F_p} = {\dfrac  {3\mu _{0}}{4\pi \Rmagtopen^{5}}}
   \left [ \left(\langle\mpBold,\RmagtopenBold\rangle \right) \mmBold + 
   \left(\langle\mmBold,\RmagtopenBold\rangle\right) \mpBold \right . +
   \\
   \left(\langle\mpBold,\mmBold\rangle\right) \RmagtopenBold - 
    \left . {\dfrac{5\left(\langle\mpBold,\RmagtopenBold\rangle\right)
    \left(\langle\mmBold,\RmagtopenBold\rangle\right)}{\Rmagtopen^{2}}} \RmagtopenBold \right ] \ , \label{eq:F21-dip}
\end{multline}
%
where $\mu_0$ is a constant (vacuum permeability $4\pi \ 10^{-7}$ [H/m]) and  $\RmagtopenBold$ is the 3D vector between the centers of the electromagnet and pen dipoles.
In contrast to FEA, this expression is analytic and differentiable, thus suited for iterative optimization.
\figref{fig:dipole_dipole} shows all quantities needed to compute the total magnetic force exerted on the pen.
The expression does, however, lead to an actuation force $\mathbf{F_p}$ that depends on the tilt of the pen.
In pre-tests, we found that users can not perceive a difference in strength when tilting the pen in-place.
We therefore leverage our second assumption, which reduces the EM model from 6 DOF to 3 DOF, to avoid this computation. 

Based on the second assumption, we can retrieve the two vertical force vectors of the electromagnet $\mathbf{m_m}$ and the pen $\mpBoldt$.
The vector between the two centers can now be computed as $\RmagtopenBold$.
We then project this vector onto the plane, yielding the final vector $d$ between the pen tip and the in-plane projection of the actuator dipole.
The total force acting on the pen (Eq. \ref{eq:F21-dip}) can now be decomposed as:
\begin{equation}
    \mathbf{F_p} = F_a \ \mathbf{e_d} + F_z  \ \mathbf{e_z} \ . \label{eq:Fp_decomp}
\end{equation}

Here $\mathbf{F_a} = F_a \ \mathbf{e_d}$ represents the in-plane quantity we seek to control, as it is the magnitude $F_a$ of the force vector $\mathbf{F_a}$ in the direction of a unit vector $\mathbf{e_d}$ along $d$.
$F_z$ is the vertical force components which pulls the pen downwards. 
During our experiments there was no significant change in perceived friction when comparing the drawings with and without electromagnet (\ie with or without $F_z$). 
For this reason we do not actively account for $F_z$ in our optimization and only maintain the in-plane force contribution ($\mathbf{e_d}$ direction).
%
The actuation force as function of pen-magnet separation is obtained as:
%
\begin{equation}
    \mathbf{F_a} = \alpha \ F_0 \ \left( \frac{d \left(4 - \frac{d^2}{h^2}\right)}{h \left(1 + \frac{d^2}{h^2}\right)^\frac{7}{2}} \right)  \ \mathbf{e_d} , \label{eq:Fa}
\end{equation}
%
where  $\alpha \in \left[0,1\right]$ is a dimensionless scalar to control the desired strength of the force that should be felt by users, \add{$h$ is the center-to-center distance between both magnets projected on to the z-axis} (\figref{fig:em_model}) and $F_0$ is a constant force parameter given by the expression,
\begin{equation}
 F_0 = \frac{3 \ \mu_0 \ m_p \ m_m}{4 \ \pi \ h^4} \ . \label{eq:F0}
\end{equation}
%
The actuation force $F_a$ is zero if the two magnets are aligned with one another ($d=0$), it has a maximum $F_a^{max} = 0.9 \ F_0$ at $d=0.39h$, and we can assume there is no more attraction for distances $d>2h$. 
Note that the second assumption (only use in-plane component) lead only to a small approximation error
Compared to an angle dependent formulation (see Appendix \ref{sc:ap.angle-dipole}), a tilt of up to $\angt = 30^{\circ}$ leads to a max error in our model (Equation \ref{eq:Fa}) equivalent to shifting the distance $d$ by $\pm$ \unit[3]{mm}. 
This uncertainty in $d$ is comparable with the in-plane positioning error (dispersion) of our hardware prototype.
An angle dependent formulation of our model (\ie 6 DOF) can be found in Appendix \ref{sc:ap.angle-dipole} for future use in cases where the pen angle is tracked. 
This model remains valid for other hardware implementations involving a single moving electromagnet or can be easily extended onto a grid of fix electromagnets.

% ########################################################################################
% ########################################################################################
% ########################################################################################


\subsection{Hardware prototype}
\label{sc:hardware}
We have developed one possible hardware instance that utilizes our optimization scheme for an in-plane haptic guidance system (see \figref{fig:hardware}). 
Our system consists of 3 main components: 1) a moving platform that controls the 2D location of the electromagnetic actuator, 2) an input device such as a stylus, and 3) an output device such as a digital tablet or digitizer used in combination with a non-digital drawing surface. 

\subsubsection{Motion platform}
The motion platform consists of a controllable electromagnet on a bi-axial linear stage directly underneath the drawing plane. 
The linear stage has two orthogonal \unit[12]{mm} lead screws, which are driven by two \unit[24]{V}, \unit[4.0]{A} NEMA23 high-torque stepper motors. 
We control the motors with a DQ542MA stepper driver and an Arduino UNO. 
As electromagnet, we use an Intertec ITS-MS-5030-12VDC magnet (\unit[5]{cm} diameter, \unit[3]{cm} height, \unit[12]{V}), controlled via pulse-width modulation.
It can deliver up to \unit[488]{mN} of lateral force at 11 W. 
\add{
% We chose this magnet to balance power consumption, size, force and commercial availability.
% Using FFE
We used FEM analysis to select this magnet from a range of commercially available magnets \cite{comsol}.
It provides a balance between power consumption, size, and force, \ie it delivers a strong perceivable force while having a small footprint relative to our hardware.
% The choice of this magnet was based on power consumption, size and force. 
% The size was found through FEM analysis since the diameter and height change the perceived force. 
% This magnet was an off-the-shelf magnet that closely full-filled all criteria
}

To measure the positional dispersion of the motion platform, we moved the electromagnet at full strength ($\alpha=1$) to 300 random locations and then always back to the center of the drawing surface. 
During theses trials, a user held the pen upright and followed the magnet passively.
By measuring the difference in target and actual position, we found that our system yields \unit[2.8]{mm} ($\pm$ \unit[0.8]{mm}) of point dispersion.
We believe this is sufficient for most applications and our experiments.
One of the factors that contribute to this dispersion is the vanishing of the actuation force $F_a$ as $d\rightarrow0$. 
This can lead to the pen motion stopping slightly before it reaches the target.

\subsubsection{Software}
Our software runs on a standard PC (Intel Core i7-4770 CPU 4 cores at \unit[3.40]{GHz}) in all our experiments. 
The solver is implemented in FORCES Pro \cite{forcespro}, which produces efficient C-code. 
The following weight values are used for our control scheme \add{(Equation \ref{eq:J_k})}:

%\begin{table}[!htb]
%  \caption{List of variables and optimization weights used in this work.}
%  \label{tab:var}
  \begin{tabular}{ccccccccc}
    \toprule
    $w_l$&$w_c$&$w_\theta$&$w_{\dot{\theta}}$&$w_f$&$w_d$&$w_\alpha$&$w_v$&$w_m$\\
    \midrule
    1.5&1.5&10.&0.1&10.&0.05&7.&1.&1.\\
   \bottomrule
   \end{tabular}
  \vspace{5pt}
  
%  \emph{cont.}\\
%  \begin{tabular}{c|cc}
%    \toprule
%    Variable\\
%    \midrule
%    Value \\
%   \bottomrule
%\end{tabular}
%\end{table}

Due to the steepness of the electromagnetic force $\mathbf{F_p}$ and the potentially fast pen motion, runtime and latency are crucial performance metrics. 
The optimization algorithm contributes to both, whereas latency is dominated by the hardware and I/O. 
The mean solve time for a problem instance is \unit[7.4]{ms} ($\pm$ \unit[3.0]{ms}). 
This can be expected to be mostly constant since we do not manipulate the system state space and the only measured input comes from the pen. 
The hardware and overall system latency adds to the solve time. 
We use a high-speed camera (\unit[1000]{fps}) to establish the motion (pen) to motion (magnet) latency. 
This yields an approximate latency of \textasciitilde{}\unit[10]{ms}. 
Given the combined latency envelope of \textasciitilde{}\unit[20]{ms}, we did not experience any abrupt pen snapping in our experiments.

\subsubsection{Input and Output Devices}
Our primary input and output devices for our user experiments consist of a 3D printed ballpoint pen with a permanent ring magnet mounted in the shaft (see \figref{fig:hardware}) and a piece of paper. 
The strokes are recorded by a Sensel Morph pressure sensitive touch pad \cite{morph}. 
\add{If the system cannot locate the pen (\eg when it is lifted) the last known position is used.}
We chose the Sensel for its high spatial resolution (\unit[6502]{DPI}), high speed (\unit[500]{Hz}) and low latency (\unit[2]{ms}), according to specification, and since the sketching surface does not interfere with the input recognition.
% 
Additionally, we developed an all digital input/output system with a multi-touch tablet + stylus. 
We use a Galaxy Tab A tablet with capacitive input and an off-the-shelf stylus with a permanent magnet placed near the tip. 
This magnet is slightly bigger (\unit[12]{mm} radius, \unit[12]{mm} height) to compensate for the increase in tablet thickness. 