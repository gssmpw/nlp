\section{Method Overview}
The goal of our online optimal control scheme is to allow users to maintain control and agency over the input device (e.g., pen, stylus), while experiencing dynamic guidance from the system. Importantly, it leverages \textit{time-free references}, and thus the dynamics are entirely driven by the pen position over time, which is different from approaches such as MPC.

The proposed optimization scheme allows us to adjust the magnet position and strength such that it gently pulls the pen tip towards a desired stroke, while allowing users to draw at their desired speed and without fully taking over control. The algorithm is generally hardware agnostic and works for devices with electromagnetic actuators underneath an interaction surface. This can be implemented via bi-axial linear stage as in our prototype (see \figref{fig:hardware}) or via a matrix of electromagnets which would lend itself better to miniaturization. Furthermore, the algorithm requires a reference trajectory over the optimization horizon. This can be defined a priori, such as a known shape to be traced, or may be provided dynamically, e.g., the output of a predictive model (e.g., Aksan et al. \cite{Aksan:2018:DeepWriting}).

At each time step, we minimize a cost functional over a receding time horizon to find optimized values for system states $\mathbf{x}$ and inputs $\mathbf{u}$.
As a high-level abstraction, the cost function
\begin{equation}
    \underset{\mathbf{x},\mathbf{u}}{\text{minimize}} \sum 
    \underbrace{\mathcal{C}{\text{force}}(\mathbf{x},\mathbf{u})}{\text{Eq. \ref{eq:err_F}, \ref{eq:err_d} \& \ref{eq:erralpha} }} + 
    \underbrace{\mathcal{C}{ \text{path}}(\mathbf{x},\mathbf{u})}_{\text{Eq.  \ref{eq:errLC}}} +
    \underbrace{\mathcal{C}{\text{progress}}(\mathbf{x},\mathbf{u})}_{\text{Eq. \ref{eq:errtheta}}}
      \label{eq:min1},
\end{equation}
serves three main purposes: 1) ensuring that the user perceives haptic feedback of dynamically adjustable force ($\mathcal{C}{\text{force}}$), 2) stays close to the desired path but does not rigidly prescribe it ($\mathcal{C}{\text{path}}$), and 3) makes progress along it ($\mathcal{C}_{\text{progress}}$) but allows the user to vary drawing speed freely.
  
  
 \begin{table}[!t]  \label{tab:control_params}
  \caption{Overview control parameters and values}
  \begin{tabular}{cp{0.25\columnwidth}p{0.5\columnwidth}}
    \toprule
    Name & Range / Value & Description\\
    \midrule
     $\posp$   			& $\mathbb{R}^2$ 						& Position of pen
     \\     
     $\posm$  			& $\mathbb{R}^2$ 						& Position of electromagnet
     \\
     $\mathbf{F_a}$  			& $\mathbb{R}^3$ 						& Electromagnetic force vector
     \\
     $\alpha$ 				&$\left[0,1\right]$	 & Electromagnetic intensity
     \\
     $\mathbf{s}$  		& $\theta \in[0,L]$ 						& Target trajectory of length $L$
     \\
    $\mathbf{x}$ 		& $[\mathbf{p}_{m},\dot{\mathbf{p}}_{m}, \alpha, \theta]$ & System states 
    \\
    $\mathbf{u}$ 		& $[\ddot{\mathbf{p}}_{m}, \dot{\alpha}, \dot{\theta}] $	& System inputs 
    \\
     \bottomrule
\end{tabular}
\end{table}
