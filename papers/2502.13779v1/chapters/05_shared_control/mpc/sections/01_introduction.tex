\section{Introduction}
In this dissertation, we have presented pen-based haptic control systems, including a novel spherical electromagnet (\chapref{ch:shared:contact}) and a sensing algorithm (\chapref{ch:shared:volumetric}). However, while these systems aim to control the pen, they inadvertently reduce user agency. A haptic system should guide the user while preserving their autonomy.

Pen-based interactive systems featuring \textit{haptic guidance} support users in various applications such as drawing, sketching, writing, and CAD design. The goal of such systems is to enable users to draw higher-complexity shapes with less effort and higher accuracy, or to support users through virtual haptic tools such as rulers or guides. Crucially, such systems aim to strike a balance between giving users a strong sense of control and agency, while providing feedback unobtrusively. This is enabled through embedding controllable electromagnets in the system, which can then guide input devices such as a pen, or provide feedback to users about the positions and boundaries of virtual objects.

Existing systems such as dePenD \cite{yamaoka2013depend} typically employ an \textit{open-loop control approach}. The magnet that drives the pen is set to a predefined trajectory, and users then must closely follow the movement of the system. In this case, it is not possible for users to adjust the trajectory, since it would lead to a loss of haptic guidance. This effectively leads to a decrease in control of users, and arguably a loss of user agency.

Alternatively, it is possible to extend haptic guidance systems with traditional closed-loop approaches, \eg by implementing a proportional–integral–derivative (PID) controller \cite{aastrom1995pid} (and as we did in \omniHap and \omniUIST) or based on heuristics \cite{Lopes16}. Such systems adjust to users' movement but are, \add{usually}, based on a timed reference, effectively dictating users' drawing speed. This can lead to unintended behavior such as snapping whenever the pen is too close to the actuator, a problem that is exacerbated for magnetic systems due to the non-linear nature of the magnetic force over distance.

We propose a real-time closed-loop control approach that allows users to retain agency and control while being assisted by an electromagnetic haptic guidance system. Our approach enables users to draw at their desired speed and adjust their target trajectory continuously.
It then adapts and complies with such modifications while giving corrective feedback. Our algorithm positions and regulates a variable-strength electromagnet such that it provides dynamically adjustable in-plane magnetic forces to the pen tip.

We contribute a novel optimization scheme for electromagnetic-based haptic guidance systems, \ie models and control algorithm, that enables formalizing this problem in the established Model Predictive Contour Control (MPCC) framework \cite{lam2010model}, which has previously only been employed in contexts such as RC-racing \cite{Liniger2014} or drone cinematography \cite{Naegeli:2017:MultiDroneCine}.

We provide an accurate system model, parameters, and an appropriate cost function alongside a method to optimize the model parameters given user inputs. Modeling the non-linear interaction of an electromagnetic force field typically makes use of the finite element method (FEM), which is not applicable for real-time scenarios. To overcome this challenge, we additionally contribute a novel approximate yet accurate model of the electromagnetic force field that can be evaluated analytically in real time. Compared to simpler control schemes such as Model Predictive Control (MPC) \cite{Faulwasser:2009} and \add{many implementations of} PID control, our approach does not require a timed reference and hence allows users to draw at their desired speed.

Furthermore, our optimization scheme allows for error-correcting force feedback, gently pulling the user back to the desired trajectory rather than pushing or pulling the pen to a continuously advancing setpoint on the trajectory. With our approach, the reference path can be updated at every timestep, thus allowing users to continuously change their desired trajectories. This enables applying the algorithm to fully dynamic references, for example virtual tools such as rulers or programmable French curves. 

To assess the proposed control algorithm, we developed a proof-of-concept hardware implementation (see Figure~\ref{fig:hardware}), leveraging an electromagnet that moves underneath the drawing surface or display on a bi-axial linear stage. The magnet provides variable strength guidance onto the tip of a minimally instrumented pen or stylus via an electromagnet positioned directly below a drawing surface, guided by our proposed approach.

We demonstrate the feasibility of our approach with a set of applications, specifically drawing guidance on conventional paper for sketching and writing, and a digital sketching application that features virtual haptic guides and rulers.

To evaluate our approach, we performed two experiments with twelve participants each.
We first compared free-hand drawing of shapes with varying complexity with and without our feedback system. Results showed that the haptic guidance using our approach improved the accuracy across shapes by up to 50\% to $\unit[1.87]{mm}$. We then compared our approach to our implementation of dePENd (open-loop) and a simple MPC-based closed-loop control scheme. Our approach showed significantly higher accuracy and was preferred by users.

In summary, we contribute
\begin{itemize}
\setlength\itemsep{0em}
\item A novel MPCC-based optimization scheme for electromagnetic haptic guidance systems including models, parameters, cost function and control algorithm.
\item A novel real-time approximate model for electromagnets that generalizes beyond our hardware implementation.
\item Evaluations showing the improved accuracy of our method.
\end{itemize}
\begin{figure}[!t]
\includegraphics[width=\columnwidth]{\dir/figures//hardware-02.jpg}
\caption{We implement our proposed guidance system using a two-axis linear stage equipped with an electromagnet. All technical and user evaluations were completed using the Pressure Sensitive Tablet (\textit{right}). We additionally developed an all-digital implementation using a multi-touch tablet with display (\textit{left}).}
\label{fig:hardware}
\end{figure}