\chapter{Optimal control for electromagnetic haptic guidance systems}
\label{ch:control:optimal}
\contribution{
In \chapref{ch:shared:contact} and \chapref{ch:shared:volumetric}, we introduced novel actuator and sensing techniques for electromagnetic haptic devices. However, the challenge of effectively controlling the actuator given the sensor input remains unresolved. In this chapter, we address this by exploring an optimal control method for electromagnetic haptic guidance systems. Our approach assists users in pen-based tasks such as drawing, sketching, and designing, while ensuring that user agency is maintained.
%
Traditional methods force the stylus to follow a continuously advancing setpoint on a target trajectory, often resulting in loss of haptic guidance or unintended snapping. In contrast, our control approach gently pulls users towards the target trajectory, allowing for spontaneous adaptation and drawing at their own speed. To achieve this flexible guidance, we iteratively predict the motion of an input device (such as a pen) and dynamically adjust the position and strength of an underlying electromagnetic actuator.
%
To enable real-time computation, we introduce a novel, fast approximate model of an electromagnet. We validate our approach on a prototype hardware platform featuring an electromagnet on a bi-axial linear stage and demonstrate its effectiveness through various applications. Experimental results indicate that our method is more accurate and preferred by users compared to open-loop and time-dependent closed-loop approaches.
}


\begin{figure}
    \centering
  \includegraphics[width=\textwidth]{\dir/figures/teaser_new-02.pdf}
  \caption{We propose an optimal control scheme for electromagnetic guidance systems (\textit{left}). A target trajectory is provided, on which users are guided. They can always adapt the trajectory, our optimization then guides users back to the target (\textit{offset between target and drawing for illustration purposes}).}
  \label{fig:teaser_mpc}
\end{figure}

\begin{figure}[ht]
    \centering
    \includegraphics[width=0.8\linewidth]{graphs/greater_than_naive.pdf}
    \vspace{0.5cm}
    \includegraphics[width=0.8\linewidth]{graphs/p1_bottom.png}
    \vspace{-5pt}
    \caption{\textcolor{positional}{Positional} vs.\ \textcolor{nonpositional}{non-positional} circuits. In a \textcolor{nonpositional}{non-positional} circuit, the same edges must be included at all positions. A \textcolor{positional}{positional} circuit can distinguish between the same edge at different positions. This specificity yields better trade-offs between circuit size and faithfulness. It can also increase both precision and recall.}
    \label{fig:p1}
    \vspace{-5pt}
\end{figure}

\section{Introduction}

\looseness=-1
A primary goal of interpretability research is to characterize the internal mechanisms in language models (LMs) and other NLP models. 
A core approach in this area is \textbf{circuit discovery}---identifying the minimal subgraph within the model's computation graph that performs a specific task \citep{olah2021framework,olah-mech}.
Typically, the nodes of a circuit represent model components (e.g., attention heads, neurons, or layers).
While manual circuit discovery methods can yield position-specific insights \citep{wanginterpretability,goldowskydill2023localizingmodelbehaviorpath}, \emph{automatic methods often overlook positional information}, treating components as uniformly relevant across all input token positions \citep{conmytowards,syed2023attribution}. 
For instance, if an attention head is included in a circuit, it is assumed to contribute equally to the computation for every position in the input sequence.
The assumption that circuits are position-invariant ignores the fact that different positions often require distinct computations.
By ignoring positions, current methods limit their ability to capture mechanisms that operate across positions, such as interactions between attention heads across positions.

In this study, we start by demonstrating that positional agnosticism is a significant limitation (\S\ref{sec:motivating}). Then, to address these limitations, we introduce a new approach: position-aware edge attribution patching (PEAP; \S\ref{sec:full_circ_discovery}; Figure~\ref{fig:p1}). Current approaches  assume that if an edge is in a circuit, then the same edge will be in the circuit at all positions, thus leading to low precision. It is also assumed that an edge's importance should be aggregated across positions before deciding whether it should be included in the circuit; this can lead to cancellation effects, and thus low recall. PEAP instead allows us to compute the importance of cross-positional edges, and separately evaluates edge importance at each position. We show that this leads to smaller and more accurate circuits; see Figure~\ref{fig:p1}.

Incorporating positional information into circuit discovery is straightforward when inputs have the same length and structure across examples.

However, realistic datasets are not nearly this templatic.
How, then, can we incorporate positional information into automatic circuit discovery?
To address this challenge, we propose \textbf{schemas} (\S\ref{sec:schema}). 
Schemas assign semantic labels to spans of tokens, enabling information aggregation across examples even when the spans differ in length.

For example, in the input ``The \textcolor{positional}{war} lasted from 1453 to 14\underline{\hspace{1em}},'' the span ``\textcolor{positional}{war}'' could be labeled as ``\emph{Subject}''.
This enables handling spans with varying lengths: the phrase ``\textcolor{positional}{Black Plague}'' in another example can be treated as a single positional span with the same role as ``\textcolor{positional}{war}''.
In experiments with two LMs and three tasks, we find that circuits discovered using schemas achieve a better trade-off between circuit size and faithfulness to the model's behavior than position-agnostic circuits.
Importantly, position-aware circuits offer a more precise representation of the underlying mechanisms, providing a more concise foundation for mechanistic explanations.

We also present a fully automated pipeline for schema generation and application (\S\ref{sec:schema-generation}) using large language models (LLMs). 
We evaluate the quality of the generated schemas and their utility in discovering position-aware circuits (\S\ref{sec:schema-eval}).
Notably, circuits derived using automatically generated and applied schemas achieve comparable faithfulness scores to circuits discovered with human-designed and manually applied schemas.

We summarize our contributions as follows:
\begin{itemize}[noitemsep,leftmargin=*,topsep=1pt,parsep=1pt]
    \item Introduce a position-aware circuit discovery method, which obtains better faithfulness than position-agnostic discovery.  
    \item Introduce dataset schemas,  facilitating positional circuit discovery in more naturalistic settings. 
    \item Develop an automated schema generation and application pipeline with LLMs, yielding schemas that are comparable to manually-annotated ones.
\end{itemize}

\section{Method Overview}
The goal of our online optimal control scheme is to allow users to maintain control and agency over the input device (e.g., pen, stylus), while experiencing dynamic guidance from the system. Importantly, it leverages \textit{time-free references}, and thus the dynamics are entirely driven by the pen position over time, which is different from approaches such as MPC.

The proposed optimization scheme allows us to adjust the magnet position and strength such that it gently pulls the pen tip towards a desired stroke, while allowing users to draw at their desired speed and without fully taking over control. The algorithm is generally hardware agnostic and works for devices with electromagnetic actuators underneath an interaction surface. This can be implemented via bi-axial linear stage as in our prototype (see \figref{fig:hardware}) or via a matrix of electromagnets which would lend itself better to miniaturization. Furthermore, the algorithm requires a reference trajectory over the optimization horizon. This can be defined a priori, such as a known shape to be traced, or may be provided dynamically, e.g., the output of a predictive model (e.g., Aksan et al. \cite{Aksan:2018:DeepWriting}).

At each time step, we minimize a cost functional over a receding time horizon to find optimized values for system states $\mathbf{x}$ and inputs $\mathbf{u}$.
As a high-level abstraction, the cost function
\begin{equation}
    \underset{\mathbf{x},\mathbf{u}}{\text{minimize}} \sum 
    \underbrace{\mathcal{C}{\text{force}}(\mathbf{x},\mathbf{u})}{\text{Eq. \ref{eq:err_F}, \ref{eq:err_d} \& \ref{eq:erralpha} }} + 
    \underbrace{\mathcal{C}{ \text{path}}(\mathbf{x},\mathbf{u})}_{\text{Eq.  \ref{eq:errLC}}} +
    \underbrace{\mathcal{C}{\text{progress}}(\mathbf{x},\mathbf{u})}_{\text{Eq. \ref{eq:errtheta}}}
      \label{eq:min1},
\end{equation}
serves three main purposes: 1) ensuring that the user perceives haptic feedback of dynamically adjustable force ($\mathcal{C}{\text{force}}$), 2) stays close to the desired path but does not rigidly prescribe it ($\mathcal{C}{\text{path}}$), and 3) makes progress along it ($\mathcal{C}_{\text{progress}}$) but allows the user to vary drawing speed freely.
  
  
 \begin{table}[!t]  \label{tab:control_params}
  \caption{Overview control parameters and values}
  \begin{tabular}{cp{0.25\columnwidth}p{0.5\columnwidth}}
    \toprule
    Name & Range / Value & Description\\
    \midrule
     $\posp$   			& $\mathbb{R}^2$ 						& Position of pen
     \\     
     $\posm$  			& $\mathbb{R}^2$ 						& Position of electromagnet
     \\
     $\mathbf{F_a}$  			& $\mathbb{R}^3$ 						& Electromagnetic force vector
     \\
     $\alpha$ 				&$\left[0,1\right]$	 & Electromagnetic intensity
     \\
     $\mathbf{s}$  		& $\theta \in[0,L]$ 						& Target trajectory of length $L$
     \\
    $\mathbf{x}$ 		& $[\mathbf{p}_{m},\dot{\mathbf{p}}_{m}, \alpha, \theta]$ & System states 
    \\
    $\mathbf{u}$ 		& $[\ddot{\mathbf{p}}_{m}, \dot{\alpha}, \dot{\theta}] $	& System inputs 
    \\
     \bottomrule
\end{tabular}
\end{table}

\section{Method}
Our main contributions are models and a control strategy that enables using the MPCC framework \cite{lam2013model} for electromagnetic haptic guidance.
MPCC is a closed-loop \emph{time-independent} control strategy that minimizes a cost function over a fixed receding horizon. 
There are several advantages in using our formulation over open-loop (as used in dePENd~\cite{yamaoka2013depend}) or time-dependent strategies (\eg MPC). 
%
First, closed-loop control allows to react to user-input, whereas open-loop control removes all user agency. 
Both MPC and MPCC are closed-loop control strategies. 
However, MPC tracks a timed reference, requiring a fixed velocity by users. 
MPCC follows a time-free trajectory, which allows the user to progress at their own speed. 
\figref{fig:control} illustrates the expected behavior for the different strategies, given that the user slows down or stops moving the pen. 
The desired behavior here would be that the algorithm essentially ``waits'', \ie provides guidances towards a slowly or no-longer advancing setpoint.
In this situation, open-loop approaches would lead to lost haptic guidance. 
Closed-loop time-dependent approaches would guide the pen towards a constantly advancing setpoint (although users do no longer move), which can lead to problems such as the user being guided backwards (\eg timestep $t=3$ is in front of $t=2$).

\begin{figure}[!t]
    \centering
    \vspace{-.5em}
    \includegraphics[width=\columnwidth]{\dir/figures//control_strategies-02.pdf}
    \caption{
    Overview of different control strategies on a target trajectory (\textit{green}), with constant pen position.
    For open-loop, the position of the electromagnet is identical to the constantly advancing setpoint, leading to loss of haptic guidance.
    For MPC, although the pen is static, the guidance changes at every timestep since the setpoint advances.
    In our approach, the setpoint is also based on the pen position, therefore remains stationary in this case and guides the user towards the target trajectory.}
    \label{fig:control}
    \vspace{-1em}
\end{figure}

Our method is designed to exert a force $\mathbf{F}_\theta$ of desired strength onto the pen to guide the user towards the target trajectory $\mathbf{s}$. 
The path $\mathbf{s}$ of length $L$ is parametrized by $\theta \in[0,L]$. 
Note that we do not prescribe how fast users draw and hence for each given pen position $\posp$ we first need to establish the closest position on the path parameterized by $\mathbf{s}(\theta)$.
The vector between the pen position and $\mathbf{s}(\theta)$ is defined as $\Rtheta$.
We leverage a receding horizon optimization strategy and the global reference can hence be adjusted or replaced entirely at every iteration. 
The path $\mathbf{s}$ is then a local fit to the global reference.
Furthermore, we seek to find optimized values for the electromagnet intensity $\alpha$ and the in-plane electromagnet position $\mathbf{p}_{m}$. 
Solving the error functional given in Eq. (\ref{eq:J_k}) at each timestep yields optimized values for system states $\mathbf{x}$ and inputs $\mathbf{u}$.   
 
As common in MPC(C), the system is initialized from measurements at $t=0$. 
The system state is then propagated over the horizon with the dynamics model $f(\mathbf{x},\mathbf{u})$. 
The system state vector $\mathbf{x}$ contains variables that are controlled by the algorithm (magnet intensity and position, current path progress). 
The first of the optimized inputs ($u_0$) is then applied to the physical system, transitioning the system state to $x_1$, before iteratively repeating the process to allow for correcting modeling errors. 

% ####################################################
% ####################################################
% ####################################################

\begin{figure}[!t]
% \vspace{-1em}
    \centering
    \includegraphics[width=.8\columnwidth]{\dir/figures//cost-force-02.pdf} 
    \caption{Illustration of actuation force $\mathbf{F_a}$, desired force $\mathbf{F_{\theta}}$, and the force cost-term $\Cost_f$ associate with the difference between those two forces. 
    }
    \label{fig:em_model}
% \vspace{-1em}
\end{figure}

\subsection{Haptics model: controlling the force of the electromagnet} \label{sc:em_costs}
The main goals of our approach is that users can move freely in terms of position and speed, and that the actuator continuously pulls them towards an advancing setpoint $\mathbf{s}(\theta)$ on the target trajectory $\mathbf{s}$.
At any time, the magnet exhibits an actuation force $\mathbf{F_a}$ on the pen,  given by our electromagnetic force model (see \nameref{sec:Implementation} section).
Therein lies the challenge, illustrated in \figref{fig:em_model}. 
The setpoint is continuously advancing based on the movement of the pen to ensure progress.
The actuator needs to pull the pen towards the setpoint by exhibiting force $\mathbf{F_\theta}$, but currently exhibits $\mathbf{F_a}$.
The two forces only align if the pen is exactly at the setpoint, which is rarely the case.
To overcome this challenge, we propose modeling this interaction by a spring-like behavior that ``pulls'' $\mathbf{F_a}$ towards  $\mathbf{F_\theta}$.
In this way, the magnet continuously guides the pen towards the setpoint, and the force linearly increases with distance between the pen and the target setpoint denoted as:
\begin{equation}
     \mathbf{F}_{\theta} (\Rtheta) = c \ F_0 \ \mathbf{r_{\theta}} \ \mathbf{e_{r_{\theta}}} \ . \label{eq:Fd}
\end{equation}
Here $\mathbf{e_{r_{\theta}}}$ is a unit vector in the direction of $\mathbf{r_{\theta}}$, $c$ is a scalar that regulates the stiffness of the spring (in our case $c=5/h$), $F_0$ a scaling of the EM force (\ie the force felt by users) and $h$ the distance between dipoles in $z$ (see Fig. \ref{fig:em_model}). 
Although simple, this formulation ensures that the haptic guidance is strong under large deviation from the path while vanishing as the user approaches the target path ($r_{\theta} \to 0$). 
Note that Eq. \ref{eq:Fd} is a design choice. 
Different formulations can be used to achieve different user experiences. 
Furthermore, replacing our hardware prototype and force-model would allow for adaptation of the remainder of the method to different actuation principles.
%Note that the EM force saturates at $F_a^{max}$.

The above haptics model serves as basis for our problem formulation of electromagnetic guidance in the MPCC framework.
Using the vectors of the current actuation force $\mathbf{F_a}$ and desired force $\mathbf{F_\theta}$, we formulate a quadratic cost term to penalize the difference between desired force and actual force as:
\begin{equation}\label{eq:err_F}
    \Cost_f(\posm, \posp, \alpha) = \norm{ \ \mathbf{F}_{\theta}(\Rtheta) \ - \ \mathbf{F_a}(\mathbf{d}) \ }^2. 
    \end{equation}
%
where $\mathbf{d}$ is the in-plane vector between the magnet and the pen.
Since the actuation force $\mathbf{F_a}$ declines rapidly with distance $\mathbf{d}$, the gradient of $\Cost_f$ goes to 0 for large values of $\mathbf{d}$ causing the optimization to become unstable. 
To counterbalance this issue we encourage the electromagnet to stay close to the pen:
\begin{equation}
    \Cost_d(\posm,\posp) = d^2. \label{eq:err_d}
\end{equation}

Finally, we prioritize proximity between the magnet and the pen rather than increasing its force by penalizing excessive use of magnetic intensity $\alpha$:
\begin{equation}
    \Cost_{\alpha}(\alpha) = \alpha^2. \label{eq:err_alpha}
\end{equation}
%
%
\subsection{Controlling the position of the electromagnet} 
We continuously optimize the position of the electromagnet with the goal of keeping the distance between the desired path and the pen minimal. 
To give the user freedom in deciding their drawing speed we first need to find the reference point $\mathbf{s}(\theta)$ on the target trajectory $\mathbf{s}$. 
Finding the closest point on the path is an optimization problem itself and hence can not be used within our optimization. 
Similar to recent work in robot trajectory generation \cite{Naegeli:2017:MultiDroneCine, Gebhardt:2018}, we decompose the distance to the closest point into a contouring and lag error, as shown in Figure~\ref{fig:elc}. 
%
$\Rtheta$ is the vector between the pen $\posp$ and a point $\mathbf{s}(\theta)$ on the spline, and $\mathbf{n}$ as the normalized tangent vector to the spline at that point, which is defined as $\mathbf{n} = \frac{\partial \mathbf{s} (\theta)}{\partial\theta}$.
%
The vector $\Rtheta$ can now be decomposed into a lag error and a contour error (\figref{fig:elc}). 
The lag-error $\Cost_l$ is computed as the projection of $\Rtheta$.
The contour-error $\Cost_c$ is the component of $\Rtheta$ orthogonal to the normal:
%
\begin{equation}
    \begin{aligned}\label{eq:errL_C} 
\Cost_l (\posp, \theta) &= \norm{\langle\Rtheta,\mathbf{n}\rangle}^2 , \\
\Cost_c (\posp, \theta) &= \norm{ \Rtheta - \left( \langle\Rtheta,\mathbf{n}\rangle \right) \mathbf{n} }^2.
\end{aligned}
\end{equation}
%
Separating lag from contouring error allows us, for example, to differentiate how we penalize a deviation from the path ($\Cost_c$), versus encouraging the user to progress ($\Cost_l$). %This also ensures that the $\posst$ is not influenced to a large extent by the cost function and it is at a close position on the desired path. 
We furthermore include cost terms to ensure that the magnet stays ahead of the pen ($\Cost_{\theta}$) and to encourage smooth progress ($\Cost_{\dot{\theta}}$) computed as
% 
\begin{equation} \label{eq:err_theta}
  \begin{aligned}
\Cost_{\theta}(\theta) &= - \theta ,\\
\Cost_{\dot{\theta}}(\dot{\theta}) &= (\dot{\theta}_t-\dot{\theta}_{t-1})^2 .
\end{aligned}  
\end{equation}

\begin{figure}[!t]
    \centering
    \includegraphics[width=.9\columnwidth]{\dir/figures//cost-lag-contour-02.pdf}
    \caption{Illustration of lag- and contouring error decomposition.}
    \label{fig:elc}
    \vspace{-1em}
\end{figure}

% ################################################
% ################################################
\subsection{Dynamics model}
% Standard mass point model
To phrase electromagnetic haptic guidance in the MPCC framework, we contribute a a dynamics model $f(\mathbf{x},\mathbf{u})$ describing the system dynamics given its states $\mathbf{x}$ and inputs $\mathbf{u}$.  
\begin{equation}
\begin{gathered}
\label{eq:model}
    \dot{\mathbf{x}} = f(\mathbf{x}, \mathbf{u}) ~\text{with}\\
    \mathbf{x} = [\mathbf{p}_{m},\dot{\mathbf{p}}_{m}, \alpha, \theta] \in \mathbb{R}^6
    ~\text{and} ~
    \mathbf{u} = [\ddot{\mathbf{p}}_{m}, \dot{\alpha}, \dot{\theta}] \in \mathbb{R}^4.
\end{gathered}
\end{equation}

The system state $\mathbf{x}$ consists of the position of the electromagnet $\mathbf{p}_{m} \in  \mathbb{R}^2$ and its velocity $\dot{\mathbf{p}}_{m}$, the magnet intensity $\alpha$ and the current path progress $\theta$.
The inputs to the system $\mathbf{u}$ consist of the in-plane electromagnet accelerations $\ddot{\mathbf{p}}_{m}$, and velocities $\dot{\alpha}$ and $\dot{\theta}$ for magnet intensity and the spline progress respectively.
Note that we empirically found that magnet accelerations yield smoother motion than using velocities. 
% 
The system model is given by the non-linear ordinary differential equations using first and second derivatives as inputs:
\begin{equation}
  \ddot{\mathbf{p}}_{m} = v_{m}, \quad \dot{\alpha} = v_{\alpha} \quad \text{and} \quad \dot{\theta} = v_{\theta} ,  
\end{equation}
where $v_{\left(\cdot\right)}$ are the external inputs. 
The continuous dynamics model $\dot{\mathbf{x}} = f(\mathbf{x}, \mathbf{u})$ is discretized using a standard forward Euler approach: $\mathbf{x}_{t+1} = f(\mathbf{x}_t, \mathbf{u}_t)$ \cite{gibbs2011advanced}.

In our hardware implementation, we derive the sets of admissible states $\boldsymbol{\chi}$ and inputs $\boldsymbol{\zeta}$ empirically to conform to the physical hardware constraints of the linear stage (\eg max x,y-position) and EM specifications (\eg max voltage). 
These are used in the constrained optimization problem solved in Eq. \ref{eq:mpcc-formulation}.
%
The pen position is propagated via a standard linear Kalman filter \cite{gibbs2011advanced}. 
While not an accurate user model, it works well in practice since the states are recalculated at every timestep. 


% ################################################
% ################################################

\begin{table}[!t]
	\begin{tabular}{clc}
    \toprule
    Term & Description of cost & Eq.\\
    \midrule
     $\Cost_f$  		& Decreases difference in magnetic force		& \ref{eq:err_F}
     \\
    $\Cost_d$ 		& Decreases distance between magnet and pen		& \ref{eq:err_d}
    \\
    $\Cost_{\alpha}$ 		& Encourages close distance over large force		& \ref{eq:err_alpha}
    \\
     $\Cost_l$   				& Decreases lag to path contour		& \ref{eq:errL_C} 
     \\     
     $\Cost_c$  			& Decreases distance to path contour		& \ref{eq:errL_C} 
     \\
     $\Cost_{\theta}$  				& Magnet stays ahead of pen		& \ref{eq:err_theta}
     \\
     $\Cost_{\dot{\theta}}$ 		& Ensures smooth progress		& \ref{eq:err_theta}
    \\
     \bottomrule
\end{tabular}
\caption{Summary of costs terms used in optimization.}
\label{tab:costs}
\end{table}

\subsection{Optimization}
We combine the cost terms (Table \ref{tab:costs}) to control the force and position of the actuator to form the final stage cost:
%
\begin{align}\label{eq:J_k}
J_k= \quad 
     & w_f \Cost_f(\mathbf{p}_{m,k}, \mathbf{p}_{p,k}, \alpha_k, \theta_k) + \nonumber \\
     & w_d \Cost_d(\mathbf{p}_{m,k}, \mathbf{p}_{p,k}) + w_\alpha \Cost_{\alpha}(\alpha_k)+ \nonumber \\
	 &	 w_l \Cost_l(\mathbf{p}_{p,k}, \theta_k) +  w_c \Cost_c(\mathbf{p}_{p,k}, \theta_k) + \nonumber \\
     & w_{\theta} \Cost_{\theta}(\theta_k) +  w_{\dot{\theta}} \Cost_{\dot{\theta}}(\dot{\theta}_k),
\end{align}
%
where the scalar weights $w_l,w_c,w_{\theta},w_{\dot{\theta}},w_f,w_d, w_{\alpha}>0$ control the influence of the different cost terms. 
The values used in our experiments and applications can be found in the \nameref{sec:Implementation} section.
The system states and inputs are computed by solving the $N$-step finite horizon constrained non-linear optimization problem at time instance $t$. 

The final objective therefore is:
\begin{align}
\label{eq:mpcc-formulation}
\underset{\mathbf{x}, \mathbf{u}, \theta}{\text{minimize}}\quad & \sum_{k=0}^{N} w_k\left ( J_k + \mathbf{u}_k^T \mathbf{R} \mathbf{u_k} \right ) && \\
\text{Subject to:}\quad & \mathbf{x} _{k+1} = f(\mathbf{x_k}, \mathbf{u_k}) & \text{(System Model)} \nonumber\\
                        & \mathbf{x}_0 = \hat{\mathbf{x}}(t) & \text{(Initial State)} \nonumber \\
                        & \theta_0 = \hat{\theta}(t) & \text{(Initial Progress)} \nonumber \\
                        & \theta_{k+1} = \theta_k + \dot{\theta}_k dt & \text{(Progress along path)} \nonumber \\
                        & 0 \leq \theta_k \leq L& \text{(Path Length)} \nonumber \\
                        & \mathbf{x}_k \in \boldsymbol{\chi} & \text{(State Constraints)} \nonumber \\
                        & \mathbf{u}_k \in \boldsymbol{\zeta} & \text{(Input Constraints)} \nonumber
\end{align}

Here $k$ indicates the horizon stage and the additional weight $w_k$ reduces over the horizon, so that the current timestep has more importance than later timesteps. 
$\mathbf{R}\in\mathbb{S}_+^{n_u}$ is a positive definite penalty matrix avoiding excessive use of the control inputs. 
In our implementation we use a horizon length of $N=10$. 
Experimentally we found that this is sufficient to yield robust solutions to problem instances and longer horizons did not improve results, yet linearly increases computation time. 

% \section{The general case: Proof of \texorpdfstring{\Cref{thm:main-decomp}}{Theorem 1.6}}\label{sec:algo}

First, we show that data structure of \Cref{l:max_min_query} can be used to compute distances witnessed by shortest paths that pass through a constant-size separator.

\begin{lemma}\label{l:single_adhesion}
Fix a constant $k \in \mathbb{N}$. There exists an algorithm which as the input receives an edge-weighted graph $G$ on $n$ vertices and $m$ edges together with a partition of its vertices into three sets $A, B, C$ such that $|B| \leq k$ and there are no edges between $A$ and $C$, and as the output computes $\max_{c \in C} \dist(a, c)$ for every $a \in A$. The running time is $\Oh(m \log n + n \log^{k - 1} n)$.
\end{lemma}

\begin{proof}
Let $B = \{b_1, \ldots, b_k\}$. For any $a \in A, c \in C$, we have $\dist(a, c) = \min_{i \in [k]} \dist(a, b_i) + \dist(c, b_i)$. First, we run Dijkstra's algorithm from every vertex in $B$ to find $\dist(v, b_i)$ for every $v \in V(G)$ and $i \in [k]$. Next, we use \Cref{l:max_min_query} to construct a data structure $\mathbb{D}$ for the point set $\{(\dist(c, b_1), \dots, \dist(c, b_k))\colon c\in C\}\subseteq \mathbb{R}^k$. Now, the value $\max_{c \in C} \dist(a, c)$ for any given $a$ is equal to the answer of $\mathbb{D}$ to the query with argument $(\dist(a, b_1), \dots, \dist(a, b_k))$.
\end{proof}

After computing the distances over a constant-size separator, we will use the following observation to simplify one of the sides of the separation.

\begin{lemma}\label{l:inserting_paths}
Let $G$ be a edge-weighted connected graph and let $A, B, C$ be a partition of its vertices such that there are no edges between $A$ and $C$. For every pair of vertices $u, v \in B$, let $P_{u, v}$ be any shortest path from $u$ to $v$ with all internal vertices in $C$ (assuming such a path exists).

Let $G'$ denote a graph obtained from $G[A \cup B]$ by adding an edge from $u$ to $v$ of weight equal to the length of $P_{u, v}$, for all $u, v \in B$ for which $P_{u, v}$ exists. Then,  $$\dist_G(s, t) = \dist_{G'}(s, t)\qquad\textrm{for all }s,t\in A\cup B.$$
\end{lemma}
\begin{proof}
Let $G''$ be the graph obtained by adding new edges of $G'$ to $G$.
Fix any $s, t \in A \cup B$ and let $P$ denote the shortest path from $s$ to $t$ in $G''$ which minimizes the number of vertices from $C$ visited. Naturally, the weight of $P$ is equal $\dist_G(s, t)$. Assume that such path visits at least one vertex of $C$. Then, the path $P$ is of the form $s \xrightarrow{P_1} x \xrightarrow{P_2} y \xrightarrow{P_3} t$, where $x, y \in B$ and all the internal vertices of $P_2$ are in $C$. By the construction of $G'$, $P_2$ can be replaced with a direct edge from $x$ to $y$ of the same weight. We obtain a same weight path with a smaller number of vertices of $C$ visited, which is a contradiction. Therefore, $P$ is entirely contained in $A \cup B$, hence it exists in $G'$. This shows that $\dist_G(s, t) = \dist_{G'}(s, t)$.
\end{proof}


The next lemma encapsulates the main algorithmic content of the proof of \Cref{thm:main-decomp}. The algorithm will split the tree decomposition provided on input into smaller parts for which the eccentricities are easier to calculate. We use the following lemma to handle a single such part.
\begin{lemma}\label{l:star}
Fix constants $k, g \in \mathbb{N}, 0 < \delta < \frac{1}{54}$. Assume we are given $n \in \mathbb{N}$, an edge-weighted graph $G$ on at most $n$ vertices with a weight function $w \colon E(G) \to \mathbb{N}$, a vertex subset $A$ and a collection of non-empty vertex subsets $V_0, V_1, \dots, V_\ell$ satisfying the following conditions:
\begin{itemize}[nosep]
	\item The sum of weights of all the edges in $G$ is bounded by $\Oh(n)$.
	\item $V(G) \setminus A = V_0 \cup V_1 \cup \dots \cup V_\ell$.
	\item $|A| \leq k$.
	\item For every $i \in [\ell]$, $G[V_i \setminus V_0]$ is connected, $N_G(V_i \setminus V_0) = V_i \cap V_0$, $|V_i| = \Oh(n^\delta)$, and $|V_0 \cap V_i| \leq 4$.
	\item For all $i, j \in [\ell], i \neq j$, $V_i \setminus V_0$ and $V_j \setminus V_0$ are disjoint and non-adjacent in $G$.
	\item Every edge $uv \in E(G)$ with $u, v \not\in A$ is contained in $G[V_i]$ for some $i\in \{0,1,\ldots,\ell\}$.
	\item The graph obtained by taking $G[V_0]$ and adding a clique on $V_0 \cap V_i$ for every $i \in [\ell]$ has Euler genus bounded by $g$.
\end{itemize}
Then, we can compute the eccentricity of every vertex of $G$ in time $\Oh \left( n^{1 + \frac{150 + 54 \delta}{151}} \log^k n \right)$.
\end{lemma}

\begin{proof}
Fix $\delta' = \frac{1 + 97 \delta}{151}$; we have $\delta' - \delta = \frac{1 - 54\delta}{151} > 0$.
Let $E_i$ denote the set of edges with one endpoint in $V_i$ and the other endpoint in $V_i \setminus V_0$. For $i \in [\ell]$, we shall say that $V_i$ is {\em{heavy}} if the sum of weights of $E_i$ is larger than $n^{\delta'}$. Since the sets $E_i$ are pairwise disjoint and the total sum of weights of all the edges is bounded by $\Oh(n)$, the number of heavy subsets is bounded by $\Oh(n^{1 - \delta'})$. Without loss of generality, we may assume that $V_{\ell' + 1}, \dots, V_\ell$ are heavy and $V_1, \dots, V_{\ell'}$ are not, for some $\ell'\in \{0,\ldots,\ell\}$.


For any source vertex $s$, we can calculate distances from $s$ to every vertex of $G$  using breadth first search in time $\Oh(\sum_{e \in E(G)} w(e)) = \Oh(n)$.
In particular, for every $\ell' < i \leq \ell$, we can compute the distances from every vertex of $V_i$ to every vertex of $G$ in total time $\Oh(n^{2 - \delta' + \delta})$, because $$|V_{\ell'+1}\cup \ldots\cup V_{\ell}|\leq n^{1-\delta'}\cdot \Oh(n^\delta)=\Oh(n^{1-\delta'+
\delta}).$$
Additionally, we calculate distances $\dist_G(a, v)$ for every $a \in A, v \in V(G)$ in time $O(n)$.

For every $i \in [\ell]$ and $u,v \in V_0 \cap V_i$, there exists a shortest path $P_{i,u,v}$ from $u$ to $v$ with all internal vertices belonging to $V_i - V_0$ due to the assumption that $G[V_i - V_0]$ is connected and $N_G(V_i - V_0) = V_i \cap V_0$. Therefore, the distance from $u$ to $v$ is bounded by the sum of weights of edges in $E_i$. In particular, for $i \in [\ell']$, $\dist_G(u, v) \leq n^{\delta'}$.

We define $\widetilde{G}$ to be the graph obtained by taking $G[A \cup V_0 \cup \dots \cup V_{\ell'}]$ and applying the following operation for every $i \in \{\ell' + 1, \dots, \ell\}$:
for each pair of vertices $u, v \in A \cup (V_0 \cap V_i)$, add an edge in $\widetilde{G}$ between $u$ and $v$ with weight equal to the total weight of $P_{i,u,v}$. For a fixed $i, u$, we can find $P_{i, u, v}$ for all $v$ using breadth first search in time $\Oh(n)$. Taking a sum over all $i, u$, we get that $\tilde{G}$ can be computed in total time $\Oh(n^{2 - \delta'})$.


\begin{claim}\label{cl:wG}
The sum of the edge weights in $\widetilde{G}$ is $\Oh(n)$. Moreover, for all $u, v \in V(\widetilde{G})$, we have $\dist_{\widetilde{G}}(u, v) = \dist_{G}(u, v)$.
\end{claim}

\begin{proof}
Consider $i \in \{\ell' + 1, \dots, \ell\}$ and any $u, v \in A \cup (V_0 \cap V_i)$ for which we added an edge. Its weight is bounded by the sum of weights of edges in $E_i$. Therefore, the total weight of all edges added is at most
$$
\sum_{i \in \{\ell' + 1, \dots, \ell\}} \left( |A \cup (V_0 \cap V_i)|^2 \sum_{e \in E_i} w(e) \right) \leq (4 + k)^2 \sum_{e \in E(G)} w(e) = \Oh(n).
$$
This proves the first part of the claim.

For the second part of the claim, consider any $i \in \{\ell' + 1, \dots, \ell \}$ and observe that by our assumptions, $A \cup (V_0 \cap V_i)$ separates $(V_0 \cup \dots \cup V_{\ell'} \cup V_{i + 1} \cup \dots \cup V_\ell) \setminus V_i$ from $V_i \setminus V_0$. Hence it suffices to repeatedly apply \Cref{l:inserting_paths}.
\end{proof}

For every $u \in V(\widetilde{G})$, we have $\ecc_G(u) = \max(\ecc_{\widetilde{G}}(v), \max_{v \in V(G) \setminus V(\widetilde{G})} \dist_G(u, v))$. Note, that we already know all the distances $\dist_G(u, v)$ for $v \in V(G) \setminus V(\widetilde{G})$. Similarly, we can already compute $\ecc_G(u)$ for every $u \in V(G) \setminus V(\widetilde{G})$. Therefore, it remains to compute $\ecc_{\widetilde{G}}(v)$ for each $v \in V(\widetilde{G})$. Our goal is to show that this can be done efficiently using \Cref{l:main_ecc}.

Now, let $G'$ be the graph obtained from $\tilde{G}$ by replacing every edge $e$ non-indicent to $A$ with $w(e)\geq 2$ with a path of length $w(e)$ consisting of unit-weight edges. This operation again preserves the distances. Since the sum of edge weights in $\tilde{G}$ is of $\Oh(n)$, the total number of vertices in $G'$ is of $\Oh(n)$. For $0 \leq i \leq \ell'$, we write $V'_i$ to denote the set $V_i$ together with all the vertices added as a part of a path between two endpoints in $V_i$.
As $V_i$ is not heavy for each $i\in [\ell']$, we have
$$
|V'_i \setminus V'_0| \leq |V_i| + \sum_{e \in E_i} w(e) = \Oh(n^{\delta'})\qquad \textrm{for all }i\in [\ell'].
$$

Let $G_0$ denote the graph $G'[V'_0]$ and let $G_0^*$ denote the graph $G'- A$ with $V'_i - V'_0$ contracted to a single vertex $v_i^*$, for each $i \in [\ell']$; note that, all edges of $G_0$ and $G_0^*$ have unit weight.

\begin{claim}
	The graph $G_0^*$ is does not contain $K_{t}$ as a minor, where $t = \Oh(\sqrt{g})$.
\end{claim}

\begin{proof}
Let $\bar{G}_0$ denote the graph obtained by taking $G_0$ and adding a clique on $V_0 \cap V_i$ for every $i \in [\ell']$.
By lemma assumptions and the fact that subdividing edges does not increase the Euler genus, $\bar{G}_0$ has Euler genus at most $g$. In particular, $\bar{G}_0$ is $K_{t'}$-minor-free for some $t' = \Oh(\sqrt{g})$, because the Euler genus of $K_{t'}$ is $\Omega({t'}^2)$.

Similarly, let $\bar{G}_0^*$ be the graph obtained by taking $G_0^*$ and adding a clique on each $V_0 \cap V_i$.
Note, that $\bar{G}_0^* - \{v_1^*, \dots, v_{\ell'}^*\}$ is precisely $\bar{G}_0$. Let $t = \max(t', 6)$.
Recall that a minor model of a clique $K_t$ consists of $t$ pairwise vertex-disjoint connected subgraphs, called
branch sets, such that there is at least one edge between each pair of the branch sets.
Consider a minor model $\varphi$ of $K_{t}$ in $\bar{G}^*_0$.
Note that $\varphi$ cannot contain any singleton branch set of the form $\{v^*_i\}$, for the degree of $v^*_i$ in $\bar{G}^*_0$ is at most $4 < t - 1$. Furthermore, since $N_{\bar{G}^*_0}(v^*_i) = V_0 \cap V_i$, any branch set containing $v^*_i$ and at least one other vertex contains some $u \in V_0 \cap V_i$, and $N_{\bar{G}^*_0}(v^*_i)\subseteq N_{\bar{G}^*_0}(u)$, hence removing $v^*_i$ from this branch set preserves the model. Therefore, we can assume without loss of generality that all branch sets of $\varphi$ are disjoint from $\{v^*_1, \dots, v^*_{\ell'}\}$, hence $\varphi$ is a minor model of $K_{t}$ in $\bar{G}_0$. This is a contradiction, as $t \geq t'$ and $\bar{G}_0$ is $K_{t'}$-minor-free. Therefore, $\bar{G}_0^*$ is $K_t$-minor-free, hence $G_0^*$ also.
\end{proof}

Let $\rho' = \frac{2 - 108 \delta}{151} > 0$. The graph $G^*_0$ is a unit-weight graph and is $K_{t}$-minor-free.
Hence, by applying \Cref{t:r_division} to $G^*_0$ (with $\varepsilon = \rho'/2$)
we obtain an $n^{\rho'}$-division $\mathcal{R}_0$ in time $\Oh(n^{1 + \rho'})$.
We extend it to $G' - A$ by mapping every contracted vertex $v^*_i$ to $N_{G' - A}[V'_i - V'_0] = (V'_i - V'_0) \cup (V_0 \cap V_i)$. Formally, we put $V''_i \coloneqq N_{G' - A}[V'_i - V'_0]$ and 
$$
\mathcal{R} \coloneqq \left\{ (R_0 \cap V'_0) \cup \bigcup_{i \colon v^*_i \in R_0} V''_i \colon R_0 \in \mathcal{R}_0 \right\}.
$$

Now, we argue that $\mathcal{R}$ is a reasonable division of $G' - A$. Clearly, all sets $R \in \mathcal{R}$ are connected in $G' - A$. Pick any $R \in \mathcal{R}$ and let $R_0$ be its corresponding set in $\mathcal{R}_0$.
Every vertex $v^*_i$ is mapped to a set of size $\Oh(n^{\delta'})$, therefore
$$|R| \leq |R_0| \cdot \Oh(n^{\delta'}) = \Oh(n^{\rho' + \delta'}).$$

By our construction, for every $i \in [\ell']$, $R$ is either disjoint from $V'_i - V'_0$ or contains whole $N_{G' - A}[V'_i - V'_0]$. This means that no vertex belonging to any $V'_i - V'_0$ can be in $\partial R$, hence $\partial R \subseteq V'_0$.

Pick any $u \in \partial R \cap R_0$. Assume that $u \not\in \partial R_0$. Then every vertex of $N_{G_0^*}(u)$ must be in $R_0$, hence $N_{G - A'}(u) \subseteq R$, which is a contradiction. This means that $\partial R \cap R_0 \subseteq \partial R_0$.

Pick any $u \in \partial R - R_0$. Then, $u \in V_0 \cap V_i$ for some $i \in [\ell']$ such that $v_i^* \in R_0$. Moreover, $v_i^* \in \partial R_0$ and is adjacent to $u$ in $G_0^*$. The number of such $u$ is bounded by $4 |\partial R_0 \cap \{ v_1^*, \dots, v_{\ell'}^* \}|$.

Putting two cases together, we obtain:
$$
\sum_{R \in \mathcal{R}} |\partial R| = \sum_{R \in \mathcal{R}} \left(|\partial R \cap R_0| + |\partial R - R_0|\right) \leq \sum_{R_0 \in \mathcal{R}_0} \left(|\partial R_0| + 4 |\partial R_0 \cap \{ v_1^*, \dots, v_{\ell'}^* \}|\right) = \Oh(n^{1 - \frac{1}{2}\rho'}).
$$

It remains to show the following claim.

\begin{claim}
Pick any $R \in \mathcal{R}, s_R \in R$. The number of different distance profiles on $R$ relative to $s_R$ in $G' - A$ is of $\Oh(n^{48\rho' + 54\delta'})$.
\end{claim}
\begin{proof}
We look at every vertex $v \in V(G') \setminus A$ and consider three cases: $v \in R$, $v \in V'_0$, and $v \in V'_i \setminus (V'_0 \cup R)$ for some $i \in [\ell']$. By our construction, $R \cap V'_0$ is non-empty, hence w.l.o.g. we can assume that $s_R \in V'_0$ as whether two vertices have the same profile on $R$ is independent of the choice of the pivot vertex.

In the first case, there are at most $|R| = \Oh(n^{\rho' + \delta'})$ such vertices, hence they realise at most that many profiles.

In the second case, we want to observe that profile of any vertex $v \in V'_0$ on $R$ depends only on its profile on $R \cap V'_0$ (relative to $s_R$). Pick any $t \in R - V'_0$. Then $t \in V'_i - V'_0$ for some $i \in [\ell']$, $V_i \cap V_0 \subseteq R \cap V'_0$, and every path from $v$ to $t$ intersects $V_i \cap V_0$. In particular, distances from $v$ to vertices of $V_i \cap V_0$ determine its distance to $t$, which proves the observation.

Let $\tilde{G}_0$ denote the graph obtained by taking $G'[V'_0]$ and for every $i \in [\ell'], u, v \in V_0 \cap V_i$ adding a disjoint path from $u$ to $v$ of length $\dist(u, v)$. Let $P_i$ denote the vertex set of paths added between $V_0 \cap V_i$. For every $t \in V'_0$ we have $\dist_{G' - A}(v, t) = \dist_{\tilde{G}_0}(v, t)$, so it suffices to bound the number of profiles on $R \cap V'_0$ in $\tilde{G}_0$. By our assumptions, $\tilde{G}_0$ has Euler genus bounded by $g$ and all $P_i$ are of size $\Oh(n^{\delta'})$.

Let $R_0$ be the set of $\mathcal{R}_0$ corresponding to $R$. Let $\tilde{R}_0$ denote the set $(R \cap V'_0) \cup \bigcup_{i : v^*_i \in R_0} P_i$. Such set is connected in $\tilde{G}_0$. Moreover, similarly to $R$, its size is $\Oh(n^{\rho' + \delta'})$. Applying \Cref{thm:distprofiles}, we get that the number of distance profiles on $\tilde{R}_0$ in $\tilde{G}_0$ is $\Oh(n^{12(\rho' + \delta')})$, which also bounds the number of profiles on $R$ in $G' - A$ realised by $V'_0$.

For the third case, assume $v \in V'_i \setminus (V'_0 \cup R)$ for some $i\in [\ell']$. Every path from $v$ to any vertex of $R$ in $G' - A$ intersects $V_i \cap V_0$. Let $v_1, \dots v_p$ be the vertices of $V_i \cap V_0$, where $p \leq 4$. The profile of $v$ on $R$ is then determined by the following:
\begin{itemize}[nosep]
 \item[(a)] the profile of each $v_j$ on $R$,
 \item[(b)] $\dist_{G' - A}(v, v_j) - \dist_{G' - A}(v, v_1)$ for each $2 \leq j \leq p$, and
 \item[(c)] $\dist_{G' - A}(s_R, v_j) - \dist_{G' - A}(s_R, v_1)$ for each $2 \leq j \leq p$ where $s_R$ is some pivot vertex of $R$.
\end{itemize}
By the previous case, the number of distance profiles of each $v_j$ is $\Oh(n^{12(\rho' + \delta')})$. The distances between $v$ and $v_j$ are bounded by $|V'_i|$, hence each quantity described in (b) can take $\Oh(n^{\delta'})$ different possible values. Similarly, since $v_1$ and $v_j$ are connected via $V'_i$, $|\dist_{G' - A}(s_R, v_j) - \dist_{G' - A}(s_R, v_1)| \leq \Oh(n^{\delta'})$. The number of different possible profiles of such $v$ is therefore bounded by $\Oh(n^{48(\rho' + \delta') + 6\delta'}) = \Oh(n^{48\rho' + 54\delta'})$. This finishes the proof of the claim.
\end{proof}

Now we can apply \Cref{l:main_ecc} to graph $G'$ with apex set $A$, $X = V(\widetilde{G})$, and the following constants: $$\rho = \rho' + \delta',\qquad \gamma = 1 - \frac{1}{2}\rho',\quad \textrm{and}\quad \alpha = 48\rho' + 54 \delta'.$$ This allows us to calculate all $V(\widetilde{G})$-eccentricities in $G'$ in time
$$
\Oh \left( \left(
	n^{ 2 - \frac{1}{2} \rho' } +
	n^{ 1 + 48\rho' + 54 \delta' }
\right) \log^k n \right) =
\Oh \left( n^{1 + \frac{150 + 54 \delta}{151}} \log^k n \right).
$$
Since for each $v\in V(\widetilde{G})$ we have $\ecc_{\widetilde{G}}(v) = \max_{u \in V(\widetilde{G})} \dist_{\widetilde{G}}(v, u) = \max_{u \in V(\widetilde{G})} \dist_{G'}(v, u)$, this means that we have successfully computed all the eccentricities in $\widetilde{G}$; and as we argued, this is enough to compute all the eccentricities in $G$ as well.

Finally, the total running time of the algorithm is
$$
\Oh \left( n^{1 + \frac{150 + 54 \delta}{151}} \log^k n + n^{2 - \delta' + \delta} \right) =
\Oh \left( n^{1 + \frac{150 + 54 \delta}{151}} \log^k n \right).
$$\qedhere
\end{proof}


\begin{lemma}\label{l:star2}
Fix constants $k, g \in \mathbb{N}, 0 < \delta < \frac{1}{54}$. Assume we are given $n \in \mathbb{N}$, an edge-weighted graph $G$ on at most $n$ vertices with a weight function $w \colon E(G) \to \mathbb{N}$, a vertex subset $A$ and a collection of non-empty vertex subsets $V_0, V_1, \dots, V_\ell$ satisfying the same conditions as in \Cref{l:star} with the following differences:
\begin{itemize}
	\item we don't require $G[V_i - V_0]$ to be connected and $V_i - V_0$ to be adjacent to whole $V_i \cap V_0$;
	\item instead of $|V_0 \cap V_i| \leq 4$, we require $|V_0 \cap V_i| \leq k$.
\end{itemize}
Then, we can compute the eccentricity of every vertex of $G$ in time $\Oh \left( n^{1 + \frac{150 + 54 \delta}{151}} \log^{k + 5g} n \right)$.
\end{lemma}

\begin{proof}
We will reduce our input to one which will satisfy the conditions of \Cref{l:star}. We start by addressing the adhesions $V_0 \cap V_i$ containing too many vertices.

Let $G_0$ denote the graph $G[V_0]$ with cliques placed at $V_0 \cap V_i$ for every $i \in [\ell]$.
For every $i \in [\ell]$ we repeat the following procedure: while $|V_0 \cap V_i| > 4$,
remove arbitrary $5$ vertices from $V_0 \cap V_i$. Since $|V_0 \cap V_i| \leq k$ for each $i\in [\ell]$,
this procedure can be implemented in total time $\Oh(n)$. As a result, at the end we have $|V_0 \cap V_i| \leq 4$ for all $i \in [\ell]$. Let $M$ be the set of all the removed vertices. By our assumptions, $G_0$ has Euler genus bounded by $g$, hence it cannot contain $g + 1$ pairwise disjoint copies of $K_5$
(as the Euler genus of a graph is the sum of the Euler genera of its 2-connected components~\cite{StahlB77} and $K_5$ is not planar). Each removed quintiple of vertices induces a $K_5$ in $G_0$, hence we have $|M| \leq 5g$. We set $A' = A \cup M$ and may thus assume that $V_i$ is disjoint from $A'$ for all $0 \leq i \leq \ell$.

Now, fix $i \in [\ell]$. Let $C^i_1, \dots, C^i_{r_i}$ denote the connected components of $V_i - V_0$ in $G - A'$. We define $W^i_j := N_{G - A'}[C^i_j]$ for every $j \in [r_i]$. Clearly, all $W^i_j$ induce a connected subgraph of $G$ and satisfy $N_{G - A'}(W^i_j - V_0) = W^i_j \cap V_0$. We put $V'_0 := V_0$ and enumerate
$$
\{V'_1, V'_2, \dots V'_{\ell'}\} := \{ W^i_j \colon i \in [\ell], j \in [r_i] \}.
$$
It is easy to verify that the sets $A'$ and $V'_0, V'_1, \dots, V'_{\ell'}$ satisfy the conditions of \Cref{l:star}. We apply said lemma to calculate the eccentricity of every vertex of $G$ in the desired time.
\end{proof}



The next statement is a reformulation of \Cref{thm:main-decomp}.

\begin{theorem}
Fix constants $k, g \in \mathbb{N}$. Assume we are given a graph $G$ on $n$ vertices together with its tree decomposition $(T, \beta)$ and a set of private apices $A_t \subseteq \beta(t)$ for each node $t\in V(T)$ such that the following conditions hold:
\begin{itemize}[nosep]
 \item For every node $t \in V(T)$, we have $|A_t| \leq k$.
 \item For every edge $st \in E(T)$,  we have $|\beta(v) \cap \beta(u)|\leq k$.
 \item For every node $t \in V(T)$, graph obtained by taking $G[\beta(t)] - A_t$ and turning  $(\beta(t) \cap \beta(s))\setminus A_t$ into a clique for every edge $st \in E(T)$ has Euler genus bounded by $g$.
\end{itemize}
Then, we can compute the eccentricity of every vertex of $G$ in time $\Oh \left( n^{1 + \frac{355}{356}} \log^{k + 5g} n \right)$.
\end{theorem}

\begin{proof}
We may assume that $|V(T)|\leq n$, for every tree decomposition with no two bags comparable by inclusion has this property; and adjacent comparable bags can be merged by contracting the edge between them.

For a node $t\in V(T)$, by the {\em{weight}} of $t$ we mean the size of the corresponding bag, that is, $|\beta(t)|$. For any subset of nodes $S \subseteq V(T)$, we define $\beta(S) \coloneqq \bigcup_{t \in S} \beta(t)$ By the {\em{weight}} of $S$, we mean the total weight of the elements of $S$, that is, $\sum_{t\in S} |\beta(t)|$. 

\begin{claim}\label{cl:weight-T}
The weight of $V(T)$ is of $\Oh(n)$.
\end{claim}

\begin{proof}
The sets $\beta'(t) := \beta(t) - \bigcup_{s \in N_T(t)} \beta(s)$ are pairwise disjoint. We have
$$
\sum_{t \in V(T)} |\beta(t)| =
\sum_{t \in V(T)} |\beta'(t)| + 2 \cdot \sum_{st \in E(T)} |\beta(s) \cap \beta(t)| \leq
|V(T)| + 2k|E(T)| = \Oh(n).
$$
\end{proof}

Since every bag induces a graph of bounded Euler genus, the number of edges contained in a bag is linear in its size. In particular, this implies that the total number of edges of $G$ is also bounded by $\Oh(n)$.

We set $$\delta \coloneqq \frac{1}{356}\qquad\textrm{and}\qquad \Delta \coloneqq \frac{355}{356}.$$ Root the tree $T$ in an arbitrarily chosen node; this naturally imposes an ancestor-descendant relation in $T$ (for convenience, every node is considered its own ancestor and descendant).

We start by partitioning $T$ into connected subtrees using the following procedure.
We proceed bottom-up over $T$, processing nodes in any order so that a node is processed after all its strict descendants have been processed. Along the way, we mark some nodes and split the edges of $T$ into heavy and light. Let $t \in V(T)$ be the currently processed non-root node of $T$ and let $e \in E(T)$ be the edge connecting $t$ with its parent. If the total weight of all the unmarked nodes that are descendants of $t$ is at least $n^\delta$ (recall that this includes $t$ itself as well), then we declare $e$ heavy and mark all the descendants of $t$ that were unmarked so far. Otherwise, the edge $e$ is declared light and the procedure proceeds to further nodes of $T$.

Observe that
removing all heavy edges splits $T$ into connected subtrees, say $T'_1, \cdots T'_m$. All of the subtrees, except for possibly the subtree containing the root node, are of weight at least $n^\delta$. In particular, the number of subtrees $m$, and therefore the number of heavy edges, is  bounded by $\Oh(n^{1 - \delta})$. Moreover, in every subtree $T'_i$, removing the node closest to the root splits $T'_i$ into smaller components, each of weight less than $n^\delta$.

Fix a heavy edge $e$ and let $T^e_1$ and $T^e_2$ be the two subtrees into which $T$ splits after removing~$e$. Let $X^e_i = \beta(T^e_i)$ for $i \in \{1, 2\}$. Put $A_e = X^e_1 \setminus X^e_2$, $C_e = X^e_2 \setminus X^e_1$, and $B_e = X^e_1 \cap X^e_2$. By the properties of tree decompositions, such choice of $A_e, B_e, C_e$ satisfies the conditions of \Cref{l:single_adhesion}, hence in time $\Oh(n \log^{k - 1} n)$ we can compute $\max_{v \in X^e_2} \dist_G(u,v)$ for every $u \in X^e_1$, and $\max_{u \in X^e_1} \dist_G(u,v)$ for every $v \in X^e_2$. Computing this for every heavy edge $e$ takes total time $\Oh(n^{2 - \delta} \log^{k - 1} n)$.

Fix any subtree $T'=T'_j$. Let $e_1 = t^{e_1}_1t^{e_1}_2, e_2 = t^{e_2}_1 t^{e_2}_2, \dots, e_\ell = t^{e_\ell}_1 t^{e_\ell}_2$ denote the heavy edges incident to $T'$, where $t^{e_i}_1 \in V(T')$ and $V(T') \subseteq V(T_1^{e_i})$ for every $i \in [\ell]$.
For a vertex $v \in \beta(T')$, let
$$d_0(v) = \max_{u \in \beta(T')} \dist_G(v, u)\qquad\textrm{and}\qquad d_i(v) = \max_{u \in X_2^{e_i}}\dist_G(v,u),\quad\textrm{for } i \in [\ell].$$ We have $\ecc(v) = \max \{ d_i(v)\colon i\in \{0,1,\ldots,\ell\}\}$.The values of $d_i(v)$ are already calculated for all $i\in [\ell]$, hence it remains to compute $d_0(v)$.

For every $i \in [\ell]$ and every pair of vertices $u, v \in \beta(t^{e_i}_1) \cap \beta(t^{e_i}_2)$ we find a shortest path between $u$ and $v$ with all internal vertices inside $X^{e_i}_2$ (or determine that it doesn't exist). For a fixed $u, v$ this can be done in time $\Oh(n)$. Since in total we perform this step at most $2k^2$ times per heavy edge, it takes $\Oh(n^{2 - \delta})$ time in total. Let $P_{i, u, v}$ denote such path, assuming it exists.

Let $G'$ denote the graph obtained from $G[\beta(T')]$ by taking every $i, u, v$ for which $P_{i, u, v}$ exists and adding an edge between $u$ and $v$ of weight equal to the total weight of $P_{i, u, v}$.
The weight of every edge inserted in $\beta(t^{e_i}_1) \cap \beta(t^{e_i}_2)$ is bounded by $|X^{e_i}_2|+1$. The total weight of all edges inserted is therefore at most
$$
\sum_{i \in [\ell]} |\beta(t^{e_i}_1) \cap \beta(t^{e_i}_2)|^2 \cdot (|X^{e_i}_2|+1) \leq
k^2 \sum_{i \in [\ell]} (|X^{e_i}_2|+1) = \Oh(n),
$$
where the last equality follows from the fact that all the trees $T^{e_i}_2$ are pairwise disjoint.
By \Cref{l:inserting_paths}, we have $\dist_{G'}(u, v) = \dist_G(u, v)$ for each $u, v \in \beta(T')$. Hence, computing $d_0(v)$ for every $v \in \beta(T')$ is equivalent to computing the eccentricity of every vertex in $G'$.

If the size of $\beta(T')$ is smaller than $n^\Delta$, we compute the eccentricities naively in time $\Oh(|\beta(T')|^2)$, 
noting that $G'$ has $\Oh(|\beta(T')|)$ edges (thanks to Claim~\ref{cl:weight-T} and bounded genus assumption 
of the last bullet of the theorem statement). Otherwise, we argue that we can use the algorithm in \Cref{l:star} as follows.

Let $t$ be the node of $T'$ closest to the root. Let $s_1, \dots, s_p$ be the children of $t$ in $T$ and let $T''_i$ denote the connected component of $T' - \{t\}$ containing $s_i$. Set $V_0 = \beta(t)$ and $V_i = \beta(T''_i)$ for $i \in [p]$.

It is now easy to verify that $G'$ and sets $A, \{V_i\colon 0\leq i\leq p\}$ selected this way satisfy the assumptions of \Cref{l:star2}. This allows us to use it to compute the eccentricities in $G'$ in time
$$
\Oh \left( n^{1 + \frac{150 + 54\delta}{151}} \log^{k + 5g} n \right) =
\Oh \left( n^{1 + \frac{354}{356}} \log^{k + 5g} n \right).
$$
As we argued, from these eccentricities, we may easily compute all the eccentricities in $G$.

Now, let us analyse the total running time of the whole algorithm. We invoke \Cref{l:star} $\Oh(n^{1 - \Delta})$ times, since we apply it only to subtrees $T'_i$ of size at least $n^\Delta$. The total running time of those applications is hence
$$
\Oh \left( n^{2 + \frac{354}{356} - \Delta} \log^{k + 5g} n \right) =
\Oh \left( n^{1 + \frac{355}{356}} \log^{k + 5g} n \right).
$$
We compute the eccentricities naively for subtrees smaller than $n^\Delta$, hence the total running time of this computation is
$$
\sum_{i \in [m] \colon |\beta(T'_i)| \leq n^\Delta} |\beta(T'_i)|^2 \leq
n^\Delta \cdot \sum_{i \in m} |\beta(T'_i)| = \Oh(n^{1 + \Delta})=\Oh\left(n^{1+\frac{355}{356}}\right).
$$
The rest of computation can be done in $\Oh(n^{2 - \delta} \log^k n)$. Therefore, the whole algorithm runs in time $\Oh \left( n^{1 + \frac{355}{356}} \log^{k + 5g} n \right)$.
\end{proof}

\section{Decomposition Box}





% Feature 1: Inferring Users' Thought Process from Existing Code
% Feature 2: Independent Problem Decomposition through Step Trees and Natural Language.
% Feature 3: Step Tree Node Status Evaluation with Preservation of Original Structure
% Feature 4: Progressive Hint for Idea Formation
% Feature 5: Converting the Step Tree into Comments
% Feature 6: Validating Code Implementation against the Step Tree
% Feature 7: Progressive Hint for Idea Implementation


% Challenge 1: Excessive Help Hindering Active Learning
% Challenge 2: Difficulty in Utilizing ChatGPT for Effective Learning
% Challenge 3: Misalignment Between Provided Solutions and Learners’ Approaches
% Challenge 4: Disconnect Between Existing Code and Provided Solutions.
% Challenge 5: Lack of a Structured Problem-Solving Approach
% Challenge 6: Insufficient Fine-Grained Feedback on Learner Progress


% Goal 1: Scaffolding for Active Learning and Independent Thinking
% Goal 2: Personalization to Individual Problem-Solving Styles
% Goal 3: Connection and Structured Solution Presentation
% Goal 4: Fine-Grained Evaluation and Feedback

% \renewcommand{\arraystretch}{1.2}
% \begin{table*}[tp]  

% \centering  
% \fontsize{8}{8}\selectfont  

% \caption{A mapping between the identified challenges from our formative, the design goals, and the features of Decomposition Box}\label{table:mapping}
% \label{intention_themes}
% \begin{tabular}{m{5cm}<{\centering}m{4.5cm}<{\centering}m{4.5cm}<{\centering}}
% \toprule
% \textbf{Challenge}&\textbf{Design Goal}&\textbf{DBox Feature}\\
% \midrule

% Challenge 1: Excessive Help Hindering Active Learning & \multirow{3}*{\shortstack{Goal 1: Scaffolding for Active \\Learning and Independent Thinking}}& \multirow{3}*{\shortstack{Feature 1\&2 (Stage 1), \\Feature 4\&7 (Stage 1\&2)}}\\

% Challenge 2: Difficulty in Utilizing ChatGPT for Effective Learning &  & \\


% \midrule
% Challenge 3: Misalignment Between Provided Solutions and Learners’ Approaches & Goal 2: Personalization to Individual Problem-Solving Styles & Feature 1 (Stage 1), Feature 3 (Stage 1)\\


% \midrule
% Challenge 4: Disconnect Between Existing Code and Provided Solutions & \multirow{3}*{\shortstack{Goal 3: Connection and Structured\\ Solution Presentation}} & \multirow{3}*{\shortstack{Feature 1 (Stage 1), Feature 2 (Stage 1),\\ Feature 5 (Stage 2), Feature 6 (Stage 2)}}\\

% Challenge 5: Lack of a Structured Problem-Solving Approach &  & \\

% \midrule
% Challenge 6: Insufficient Fine-Grained Feedback on Learner Progress & Goal 4: Fine-Grained Evaluation and Feedback & Feature 3 (Stage 1), Feature 6 (Stage 2)\\

% \bottomrule
% \end{tabular}
% \end{table*}



% Based on our design goals, we developed Decomposition Box (DBox) accordingly. 
% % Table \ref{table:mapping} shows the mapping between challenges identified in our formative study, the design goals, and DBox’s features. 
% Figure \ref{fig:interface} shows the interface.

\subsection{Overview} \label{overview}




\begin{figure*}[htbp]
	\centering 
	\includegraphics[width=\linewidth]{figures/UI_new.pdf}
	\caption{The interface of Decomposition Box. The top row displays the full interface in the solution formation stage (the solution implementation stage is similar, but with different status indicators). The middle row demonstrates a learner's solution formation stage showing basic DBox features. The bottom row illustrates a learner's solution implementation stage. An overview of the DBox interface and workflow is provided in Sec. \ref{overview}, and an illustrative example is described in Sec. \ref{illustrative}. To save space, the second row omits the problem description and editor area, and the third row excludes the problem description area.}
	\label{fig:interface}
        \Description{}
\end{figure*}

As shown in Figure \ref{fig:interface} (top row), DBox's interface has three main parts. The Problem Description (Figure \ref{fig:interface}.A)) and Solution Code Editor (Figure \ref{fig:interface}.B) are similar to the LeetCode platform.
The Interactive Step Tree Widget (Figure \ref{fig:interface}.D) enables users to refine their thought process and receive feedback via an interactive step tree. 
Three buttons—``From Editor to Step Tree'', ``Check Match'', and ``Copy to Comments''—connect the editor and step tree. 
The ``Check Step Tree'' button provides feedback on the step tree's status, categorized into five types (Figure \ref{fig:interface}.D). 
Clicking ``Hint'' offers progressive guidance based on learners' existing attempts (Figure \ref{fig:interface}.E). 
Hovering over steps shows buttons for editing the step tree (Figure \ref{fig:interface}.F).
% \begin{itemize}
%     \item Problem Description (Figure \ref{fig:interface} A): Displays the current problem, with a "Problem List" button , similar to traditional platforms.
%     \item Code Editor (Figure \ref{fig:interface} B): Allows users to write and run code. Output and test case results are shown in Figure \ref{fig:interface} C.
%     \item Interactive Step Tree Widget (Figure \ref{fig:interface} D): Users refine their thought process and receive feedback via an interactive step tree. Three buttons—“From Editor to Step Tree”, “Check Match”, and “Copy to Comments”—connect the editor and step tree. The “Check Step Tree” button provides feedback on the step tree's status, categorized into five types (Figure \ref{fig:interface} D). Clicking ``Hint'' offers progressive guidance based on learners' existing attempts (Figure \ref{fig:interface} E). Hovering over steps shows buttons for editing the step tree (Figure \ref{fig:interface} F).
% \end{itemize}
Figure \ref{fig:workflow} shows two key stages in DBox's workflow:
\begin{itemize}
    \item \textbf{Solution Formation}: 
    The step tree starts as an empty box where students can freely add steps in either coding mode (directly writing code) or description mode (building a step tree using natural language). 
They can evaluate their progress with the ``From Editor to Step Tree'' or ``Check Step Tree'' buttons. The tree contains steps and substeps labeled as \emph{Correct}, \emph{Incorrect}, \emph{System Generated}, \emph{Missing}, or \emph{Can be Divided}. 
Layouts adjust dynamically based on the hierarchy. 
Steps that can be further divided are marked with dashed outlines, serving as a reminder, though students can decide whether further division is necessary. 
    \item \textbf{Solution Implementation}: Students can convert the step tree into comments with “Copy to Comments” or verify alignment by clicking “Check Match”. Nodes in the step tree are labeled as Implemented, Incorrectly Implemented, or To Be Coded. In this stage, DBox also offers progressive hints. When all nodes are implemented, students can test their solution by clicking “Run” button against the provided test cases.
\end{itemize}






\begin{figure*}[htbp]
	\centering 
	\includegraphics[width=\linewidth]{figures/pipeline.pdf}
	\caption{The DBox workflow supports learners through solution formation and implementation stages. During solution formation, (A) students can input ideas by either coding or using natural language to build a step tree. (B) By clicking ``From Editor to Step Tree'' or ``Check Step Tree'', (C) DBox renders the step tree and identifies node statuses (e.g., correct, incorrect, missing). Students can iteratively refine their code or step tree, receiving progressive hints, (D) until the step tree is fully correct. In the solution implementation stage, (E) students can convert the step tree into code comments or (F) check the alignment between their code and the step tree. Each node displays one of three statuses, and students can refine their work with ongoing hints until (G) all nodes are marked as ``implemented''.
 Finally, students can test if their code passes all test cases.}
	\label{fig:workflow}
        \Description{}
\end{figure*}



% Based on our design goals, we developed Decomposition Box (DBox). Table \ref{table:mapping} shows a mapping between the identified challenges from our formative study, the design goals, and the features of DBox. Figure \ref{fig:interface} shows the interface. Below, we provide an overview of the interface, followed by a detailed explanation of the key features and backend design.

% \subsection{Overview}

% As shown in Figure \ref{fig:interface} (top row), the interface of DBox is strategically divided into three distinct sections, from left to right, designed to enhance the user experience in algorithm programming:

% \begin{itemize}
%     \item Left Section - Problem Description (Figure \ref{fig:interface} (A)): This area allows users to view the problem they are currently working on. By clicking the ``Problem List'' button, users can easily switch between different problems they wish to practice. This setup is similar to traditional programming exercise platforms, providing a familiar layout for users.
%     \item Middle Section - Code Editor (Figure \ref{fig:interface} (B)): Central to the user interface, this section features an integrated CodeMirror editor where users can write and edit code in their chosen programming language. The ``Run'' button allows users to execute their code, with the output and any error messages (such as failed test cases or syntax errors) displayed (Figure \ref{fig:interface} C). If the code passes all test cases, it is marked as accepted, providing immediate feedback on the correctness of the solution.
%     \item Right Section - Decomposition Box (Figure \ref{fig:interface} (D)): This core area is where users can refine their thought processes and receive specific feedback and reminders related to their coding approach. Positioned between the code editor and this section are three interactive buttons that facilitate seamless communication between the editor and the Decomposition Box, enhancing the integration of feedback into the coding process.
% \end{itemize}

% As shown in Figure \ref{fig:workflow}, DBox is designed to support two key stages in students' algorithmic programming practice: \textbf{Idea Formation} and \textbf{Idea Implementation}. In the Idea Formation stage, students can input their thoughts in two modes (Figure \ref{fig:workflow} (A)): one is the coding mode, where they directly write code, and the other is the description mode, where they build a step tree interactively using natural language to describe each node. Based on the selected input mode, students can click different buttons to check the status of their step tree (Figure \ref{fig:workflow} (B)). The ``From Editor to Step Tree'' button converts incomplete code into a step tree, while the ``Check Step Tree'' button evaluates the status of each node in the tree. The step tree can include steps, sub-steps, and even sub-sub-steps, with DBox identifying the status of each node, which could be one of five types: Correct, Incorrect, AI Generated, Missing, or Divisible (Figure \ref{fig:workflow} (C)). Students can refine their code or step tree and click the corresponding buttons to check the updated status of the tree. Throughout this process, DBox provides progressive hints, divided into three levels. After several refinements, the step tree becomes fully correct (Figure \ref{fig:workflow} (D)), allowing students to proceed to the Idea Implementation stage.

% At this stage, students can use the ``Check Match'' button to verify whether the code in the editor aligns with the nodes in the step tree, or they can click the ``Copy to Comments'' button to convert the step tree into comments in the editor to assist with coding (Figure \ref{fig:workflow} (E)). After clicking ``Check Match,'' the nodes in the step tree will be labeled with one of three statuses: Implemented, Incorrectly Implemented, or To Be Coded (Figure \ref{fig:workflow} (F)). DBox continues to offer progressive hints during this phase as well. After modifying the code based on feedback, the step tree will eventually display all nodes as implemented (Figure \ref{fig:workflow} (G)). At this point, the student can click the ``Run'' button to validate the solution against the test cases.





% The interface is crafted to be easily integrated as a plugin into existing programming tools and platforms, such as online environments like LeetCode or offline editors like VSCode, offering flexibility and adaptability across different learning and development settings.






% \subsection{Overview of DBox Interface}
% A learner named Alice wants to practice her algorithmic programming using a problem ``Search in Rotated Sorted Array". After reading the problem description, Alice comes up with some initial thoughts in her mind. DBox provides two input modes, one is through directly writing codes, the other is building a step tree via natural languege description at the right-side of the interface. (1) Alice chooses to add two steps, Step 1 and Step 2, to construct an initial step tree. (2) After clicking the ``Check Step Tree'' button, DBox identifies Step 1 as correct and Step 2 as incorrect, also indicating a missing Step 3. Alice then clicks the hint button on Step 2 to access general and detailed hints. (3) If Alice makes two consecutive errors at Step 2, a new hint triggers: ``reveal step.'' (4) Clicking this button reveals a crucial sub-step (Step2-3) and leaves Step 2-1 and Step 2-2 for Alice to complete. (5) Once Alice completes these, she clicks ``Check Step Tree'' again and finds all of Step 2 correct. (6) Alice also correctly constructs Step 3. \textbf{The third row displays Alice's process of implementing the solution}. (8) First, Alice clicks the ``Copy to Comments'' button, and DBox converts the step tree into code comments, inserting them at the corresponding positions in the editor. (9) After writing some code, Alice uses the ``Check Match'' button to identify steps that are not correctly implemented, noting that Step 2-3 and Step 3 are incomplete. (10) Guided by DBox, Alice writes the corresponding code. Upon clicking ``Check Match'' again, all steps turn green to indicate they are implemented correctly. When hovering over a step, the corresponding code line is highlighted. Finally, Alice hits the Run button, passes all test cases, and successfully solves the problem. It's important to note that this figure only shows just one of many possible interactions. To save space, the second row of images displays only the step tree on the right side of the interface, while the third row shows both the middle editor and the step tree.






% \subsection{Overview of DBox Interface}
% Alice, a learner, is practicing algorithmic programming with the problem ``Search in Rotated Sorted Array.'' After reading the problem description, she formulates some initial ideas in her mind. DBox offers two input modes: directly writing code or constructing a step tree using natural language descriptions on the right side of the interface. (1) Alice opts to build a step tree and adds two initial steps, Step 1 and Step 2. (2) Upon clicking the ``Check Step Tree'' button, DBox identifies Step 1 as correct, flags Step 2 as incorrect, and highlights a missing Step 3. Alice clicks the hint button on Step 2 to access both general and detailed hints. (3) If she makes two consecutive errors in Step 2, DBox triggers an additional hint: ``reveal step.'' (4) Clicking this reveals a crucial sub-step (Step 2-3 in this case), while leaving Step 2-1 and Step 2-2 for Alice to complete. (5) After completing these sub-steps, (6) Alice checks the step tree again and finds all of Step 2 marked as correct. (7) She then constructs Step 3 successfully. Now, all nodes in the step tree are correct. (8) Alice then clicks the ``Copy to Comments'' button, and DBox converts the step tree into code comments, automatically inserting them into the editor. (9) After writing some code, she uses the ``Check Match'' button, which highlights steps that are not properly implemented, indicating that Step 2-3 and Step 3 are incomplete. (10) Following DBox's guidance, Alice writes the necessary code, and upon rechecking, all steps turn green, indicating correct implementation. Note that when Alice hovers over a step, the corresponding line of code is highlighted. Finally, she clicks the Run button, passes all test cases, and successfully solves the problem.



\subsection{Target Users and A System Walkthrough} \label{illustrative}
\ms{DBox is designed for learners who understand basic algorithm concepts but struggle to apply them to solve practical problems. Using a scaffolding approach, DBox emphasizes independent thinking by offering only essential support. It assumes students are motivated, self-regulated, and actively engaging with the tool to improve their decomposition skills. If a student is less motivated or prefers a quicker solution, they may bypass DBox to search for answers online.}
Next, we present an example walkthrough (Figure \ref{fig:interface}) of such a self-regulated student Alice: 
% usage example from our pilot study, as shown in Figure \ref{fig:interface}. While this illustrates one specific interaction, users can follow various workflows based on their preferences. Below is a detailed walkthrough of the scenario.

Alice, a learner tackling the ``Search in Rotated Sorted Array'' problem, begins by organizing her thoughts in the solution formation stage. DBox offers two options: she can either start coding or build a step tree using natural language. She opts for the latter and adds two initial steps (Figure \ref{fig:interface}.1). To check her progress, Alice clicks ``Check Step Tree'' button. DBox flags Step 1 as correct, Step 2 as incorrect, and highlights a missing Step 3 (Figure \ref{fig:interface}.2). She clicks the hint button on Step 2, receiving general and detailed guidance, but after another failed attempt, DBox offers another option for revealing a substep (Figure \ref{fig:interface}.3). Alice clicks ``Reveal (Sub)Step'', uncovering a sub-step 2-3 while leaving sub-steps 2-1 and 2-2 for her to solve (Figure \ref{fig:interface}.4). Inspired by the hints, Alice figures out how to break down and fills in these sub-steps (Figure \ref{fig:interface}.5). After checking again, Step 2 is marked correct (Figure \ref{fig:interface}.6). Alice adds the missing Step 3 (Figure \ref{fig:interface}.7), and finally, after checking, all steps turn to correct (Figure \ref{fig:interface}.8).

Next, Alice moves to the solution implementation stage. She clicks ``Copy to Comments'', and DBox converts her step tree into code comments (Figure \ref{fig:interface}.9). As Alice writes her code, she uses the ``Check Match'' button to identify incorrectly implemented and unimplemented steps. Step 2 is identified as unimplemented and Step 3 is identified as incorrectly implemented (Figure \ref{fig:interface}.10). Following DBox's guidance, she revises the code, and after another check, all steps turn to be correctly implemented (Figure \ref{fig:interface}.11). Satisfied with her progress, Alice clicks ``Run'' and successfully passes all test cases, solving the problem.

Note that we have presented only a simple walkthrough here, whereas the steps in a student's actual problem-solving process are more complex and dynamic (as shown later in Sec. \ref{actual_use}). Next, we introduce the specific features aligned with the four design goals as described in Sec. \ref{designgoal}.
% \textbf{D1}: Scaffolding for Active Learning and Independent Thinking; \textbf{D2}: Personalization to Individual Problem-Solving Styles; \textbf{D3}: Connection and Structured Solution Presentation; and \textbf{D4}: Fine-Grained Evaluation and Feedback. Each feature is tailored to different stages of the user's algorithmic programming journey.




% \subsection{An Illustrative Example} \label{illustrative}
% We now present a usage example observed during our user study, as shown in Figure \ref{fig:workflow}. While this illustrates one specific interaction, users can follow various workflows depending on their preferences. To conserve space, the second row in Figure \ref{fig:workflow} omits the problem description and editor, while the third row excludes the problem description. Below, we provide a detailed walkthrough of the scenario depicted.

% Alice, a algorithmic programming learner, is tackling the ``Search in Rotated Sorted Array'' problem. After reading the description, she starts to map out her approach. DBox offers two ways to proceed: Alice can either dive straight into coding or take a more structured route by building a step tree through natural language descriptions. She chooses the latter, organizing her thoughts by adding two initial steps—Step 1 and Step 2—to the step tree (Figure \ref{fig:workflow} (1)). Eager to check her progress, Alice clicks the ``Check Step Tree'' button. DBox instantly provides feedback: Step 1 is correct, but Step 2 is flagged as incorrect, with a missing Step 3 also highlighted (Figure \ref{fig:workflow} (2)). Alice clicks on the hint button for Step 2, receiving both general and detailed guidance. But she still experiences one more failed attempt on Step 2. At this time, DBox suggests a new hint strategy: revealing part of the solution (Figure \ref{fig:workflow} (3)). She clicks ``reveal step,'' uncovering a crucial sub-step (Step 2-3), while leaving her to figure out sub-steps 2-1 and 2-2 on her own (Figure \ref{fig:workflow} (4)). Alice continues working through the sub-steps (Figure \ref{fig:workflow} (5)). Finally, Step 2 is fully correct (Figure \ref{fig:workflow} (6)), and she successfully adds the missing Step 3. Now, the entire step tree is complete (Figure \ref{fig:workflow} (7)).

% With the structure in place, Alice moves on to the coding phase. She clicks the ``Copy to Comments'' button, and DBox seamlessly transforms her step tree into code comments, which are automatically inserted into the editor (Figure \ref{fig:workflow} (8)). She starts writing her code, and as she works, the ``Check Match'' button becomes her guide, highlighting which steps are incorrectly implemented. It's clear that Step 2-3 and Step 3 need further attention ("to be coded") (Figure \ref{fig:workflow} (9)). Following DBox's prompts, Alice revises the code, and after another check, all steps turn green, signaling success. As she hovers over each step in the tree, the corresponding line of code is highlighted, helping her stay aligned. Satisfied with her progress, Alice hits the Run button. Her code passes all test cases, and she successfully solves the problem (Figure \ref{fig:workflow} (10)). 



% Next, we will explore the specific features tailored to each stage of the user's journey in algorithmic programming.





% \subsection{Stage 1: Idea Formation}

% \subsubsection{Feature 1: Inferring Users' Thought Process from Existing Code.}
% When users click the “From Editor to Step Tree” button, DBox analyzes the current problem and the user’s incomplete code to infer their thought process, which is then displayed as a step tree on the right side. It’s important to note that this feature uses the user’s existing code as the primary input and does not take into account any pre-existing step tree, which will be overwritten. Once the step tree is generated, hovering over any step node highlights the corresponding code lines in the editor, helping users easily connect the step tree with their existing code. This feature is particularly useful when users have written some code and are stuck or when they want to check for errors in their existing code.

\subsection{Stage 1: Solution Formation}
\subsubsection{Two Input Modes (D1, D2, D3)}
DBox offers users the flexibility to develop their solutions through two distinct input modes: by writing code directly or by constructing a step tree using natural language descriptions, without needing to start with code. In the latter mode, users begin with a blank step tree and can click ``Add'' to insert nodes or ``Split'' to create sub-steps for more granular detail. Each node contains a text input field where users can articulate their thought process. Steps and sub-steps can be rearranged or deleted, allowing learners to iteratively and interactively refine and structure their mental model.

\subsubsection{Inferring Users' Thought Process from Existing Code (D1, D3)}
The ``From Editor to Step Tree'' function in DBox infers a learner’s intended solution and thought process based on their incomplete code. When activated, the system analyzes the code and problem, presenting the inferred steps as a tree on the right-hand side of the interface. Hovering over each node highlights the corresponding lines in the code editor, linking the inferred steps directly to the code. This feature assists users in diagnosing errors and identifying potential issues, especially when they are unsure how to proceed.


% \subsubsection{Feature 1: Inferring Users' Thought Process from Existing Code} The ``From Editor to Step Tree'' button in DBox leverages the user’s existing, incomplete code to infer their thought process. When activated, the system analyzes the problem and code, displaying the inferred steps as a tree on the right side of the interface. It's important to note that this feature prioritizes the current code over any pre-existing step tree, which will be overwritten. As the step tree populates, hovering over any node will highlight the corresponding lines in the code editor, linking conceptual steps directly to the code. This feature is invaluable for users who are stuck or wish to identify errors in their existing code, enhancing their ability to diagnose and resolve coding issues efficiently.



% \subsubsection{Feature 2: Decomposing A Problem via Step Tree via Natural Language Description.}
% In addition to generating a step tree from the user’s existing code, we allow users to develop their thought process directly through natural language descriptions, without needing to write code initially. We use an interactive visual step tree to help users organize their problem-solving ideas. Initially, the step tree area is blank. Users can click the Add button to create a node representing a step. They can add as many steps as needed, dividing the space of the step tree area. For each step node, users can click the Split button to add sub-steps. Each step and sub-step includes a blank text input area where users can describe their thought process in simple natural language. Users can also delete or rearrange any step or sub-step as they wish. Such a step tree is useful for helping learners build a structured mental model of the problem-solving strategy.


% \subsubsection{Feature 2: Independent Problem Decomposition through Step Trees and Natural Language}

% Our tool enhances problem-solving by allowing users to independently construct a step tree using natural language descriptions, without initially requiring code. The step tree area starts blank, and users can click the ``Add'' button to insert nodes representing individual steps, organizing their thought process visually. For more detailed breakdowns, the ``Split'' button enables users to add sub-steps under each main step. Each node in the step tree features a blank text input area where users can articulate their thought process in straightforward natural language. Users have the flexibility to delete or rearrange steps and sub-steps as needed, facilitating the development of a structured mental model for tackling complex problems. This method supports learners in systematically building and refining their problem-solving strategies.


% \subsubsection{Feature 2: Independent Problem Decomposition through Step Trees and Natural Language}

% DBox allows users to independently construct a step tree using natural language descriptions, without requiring initial code. Starting with a blank step tree, users can click ``Add'' to insert nodes representing steps, and ``Split'' to add sub-steps for more detailed breakdowns. Each node includes a text input area for users to articulate their thought process. Steps and sub-steps can be deleted or rearranged, helping users build a structured mental model. This feature supports learners in interactively developing and refining their problem-solving strategies.





% \subsubsection{Feature 2: Step Tree Node Status Evaluation.}
% Our tool provides a fine-grained evaluation of the user's thought process. For each step or sub-step, the system categorizes it into one of the following states: (1) Correct, indicating that the step is appropriate for the current problem-solving approach; (2) Incorrect, indicating an error in the thought process or specific details; (3) Missing, indicating a step that is necessary for the complete solution but is absent from the user’s incomplete code; (4) Can Be Further Divided, indicating that the step is complex and can be broken down into sub-steps; and (5) AI Suggested, where the system offers a description of the step in a blue box if the user triggers the most detailed hint level. Users have complete freedom to decide whether to further divide steps, ensuring flexibility in their problem-solving approach. 

% We’ve made a special design choice here: except when the system identifies missing steps and adds a blank missing node, the structure and content of the step tree remain as constructed by the user. The system does not override the user’s current step tree with what GPT considers to be the correct steps and descriptions. Instead, we provide feedback and guidance on the status of each node in the step tree, ensuring that users can continue to advance along their problem-solving approach.

% \subsubsection{Feature 3: Step Tree Node Status Evaluation with Preservation of Original Structure} Once the learner clicks ``From Editor to Step Tree'' or ``Check Step Tree'' button, DBox will conduct a detailed evaluation of each node in the user's step tree, categorizing each step or sub-step into one of five states: (1) \textbf{Correct}: The step is suitable for the current problem-solving approach. (2) \textbf{Incorrect}: There is an error in the thought process or specific details. (3) \textbf{Missing}: A necessary step is absent from the user's code. (4) \textbf{Can Be Further Divided}: The step is complex and could be broken into smaller, more manageable sub-steps. (5) \textbf{System Generated}: The system provides a step description in a blue box if the user activates the most detailed hint level. Users retain complete control over whether to sub-divide steps, allowing them to tailor their problem-solving approach flexibly. Importantly, except for adding a blank node when steps are identified as missing, the system preserves the user’s original step tree without replacing it with AI-determined correct steps. Feedback and guidance are provided on the status of each node, supporting users as they refine and advance their problem-solving strategies.


\subsubsection{Step Tree Node Status Evaluation with Preservation of Original Structure (D2, D4)}
When the learner clicks ``From Editor to Step Tree'' or ``Check Step Tree'', DBox evaluates each node, assigning one of five statuses:
(1) \textbf{Correct}: The step aligns with the learner's intended approach.
(2) \textbf{Incorrect}: Errors are identified in the step.
(3) \textbf{Missing}: A required step is absent.
(4) \textbf{Can Be Divided}: The step is complex and can be broken into sub-steps, indicated by dashed borders. Users decide whether to subdivide. This status can coexist with other statuses.
(5) \textbf{System Generated}: Step content is created by the system. This status is triggered only when the learner requests to reveal a (sub)step after repeated failures.
During the ``Check Step Tree'' process, DBox preserves the original step tree (both structure and contents), only adding blank nodes for missing steps, ensuring scaffolding while respecting the learner's thought process.





% \subsubsection{Feature 4: Progressive Hint.}
% We’ve designed a detailed scaffolding process that provides only the necessary guidance, encouraging users to think independently before offering more specific hints as needed. The first level is a hint presented as a question. When users know a step is incorrect or missing but still lack ideas, they can click the C button to receive initial guidance. This guidance does not directly reveal the answer but steers users in the right direction, such as: “Before you convert the string variable to an array, what should you do first?” The second level offers more specific guidance, including more concrete clues, if the user still struggles after the first hint. The third level of guidance is triggered if a user repeatedly fails to correct their thought process for a specific step. This level provides an option to view the AI-suggested correct steps. Although this third level is triggered, it is not displayed by default; users must click the View button to see it. The third-level feedback can appear in two scenarios: (1) If the step has no sub-steps, the tool directly presents the correct description of the step; (2) If the step includes sub-steps, the tool highlights one key sub-step and marks other sub-steps as missing, reminding the user to complete the remaining sub-steps. It is important to note that users can choose not to view any of these three levels of feedback to solve the problem independently.


% We also provide multi-level guidance for users who incorrectly implement or have not yet implemented their thought process, aiming to help them complete their code independently as much as possible. The first level is a simple hint that offers basic guidance, such as: “How should you correctly update variable A?” or “Consider updating the index before the loop ends.” The second level is a pseudocode hint, offering a more specific guide if the user is still unsure after viewing the basic hint. At this point, the user only needs to convert the pseudocode into actual code. The third level of guidance is triggered if the user incorrectly implements a step twice in a row. This level provides an option to view the correct code for that step. Although the third level is triggered, it is not displayed by default; users must click the View button to see it. It’s important to note that users can choose not to view any of these three levels of feedback to independently implement their ideas.

% \subsubsection{Feature 4: Progressive Hint for Idea Formation}

% We have implemented a scaffolding feature that progressively delivers guidance, fostering independent problem-solving while providing specific hints as needed. This feature is structured into three levels. (1) \textbf{Initial Hint (Question-Based)}: This is the first level of assistance where users receive a hint framed as a question, prompting them to think about the next step without giving away the solution. For instance, if a user is uncertain about the initial steps, they might see a hint like, “Before you convert the string variable to an array, what should you do first?” (2) \textbf{Detailed Guidance}: If the user continues to struggle, a second, more specific hint is provided. This could include more direct clues that guide the user closer to the solution but still require them to apply their reasoning. (3) \textbf{Recommendation for A (Sub)Step}: Triggered by repeated difficulties in correcting a specific step, this level offers the option to view the AI-suggested correct steps. For steps without sub-steps, it directly presents the correct description. For steps that include sub-steps, it highlights one key sub-step and marks the remaining as missing, prompting users to complete them. This level is optional and can be accessed by clicking a ``Reveal Step'' button, allowing users to decide whether to see the full solution or continue working independently. These layers of hints are designed to support users in building their problem-solving skills gradually while enabling them to maintain control over their learning process.

\subsubsection{Progressive Hints for Solution Formation (D1)}

DBox provides progressive hints to scaffold learners' problem-solving in three levels: (1) \textbf{General Hint} (Question-Based): Prompts learners' critical thinking without revealing solutions, e.g., ``Before converting the string to an array, what should you do first?'' (2) \textbf{Detailed Hint}: Offers more specific clues while requiring reasoning, e.g., ``Think about how you can traverse each character in the string.'' (3) \textbf{Reveal (Sub)Step} ((Sub)Step Recommendation): For repeated errors, the AI can suggest a substep within a larger step when users click the ``Reveal (Sub)Step'' button. This reveals one key substep while leaving the remaining steps for the learner to complete. Notably, students can choose not to trigger this hint. These progressive hints support problem-solving development while allowing learners to maintain independence and control.

Once the step tree is complete and all nodes are correct, learners proceed to the solution implementation stage.


% \subsection{Stage 2: Idea Implementation} After constructing a complete and correct step tree, users advance to the idea implementation stage.

% \subsubsection{Feature 5: Converting the Step Tree into Comments} This feature transforms each node of the step tree into code comments. By clicking ``Copy to Comments", these comments are pasted directly into the corresponding section of the editor, guiding students to systematically complete their code implementation.

% \subsubsection{Feature 6: Validating Code Implementation against the Step Tree} The tool evaluates the alignment between the user’s code and the problem-solving approach defined in the step tree. Using the ``Check Match'' button, the step tree's status updates to reflect: (1) \textbf{Implemented}: The code accurately implements the described step. (2) \textbf{Incorrectly Implemented}: There is an error in how the step is coded.
% (3) \textbf{To be Coded}: The step has yet to be coded. Highlighting in the editor indicates the correlation of code lines with the step tree, showing whether each step is correctly or incorrectly matched.

% \subsubsection{Feature 7: Progressive Hint for Idea Implementation.} Additionally, for users who have incorrectly implemented or have yet to implement their thought process, multi-level guidance is available. (1) \textbf{Basic Hint}: Offers straightforward suggestions to nudge the user in the right direction, like “How should you correctly update variable A?” or “Consider updating the index before the loop ends.” (2) \textbf{Pseudocode Hint}: If the user remains unsure after the basic hint, this level provides a pseudocode hint, clarifying what needs to be done, which the user then translates into actual code. (3) \textbf{Recommended Implementation}: This is activated if a user misimplements a step twice consecutively. It allows them to view the recommended code for that step by clicking a ``View'' button, although, like the previous levels, viewing is optional.

% Note that DBox does not teach specific algorithm concepts or knowledge, as our target users are learners who have a basic understanding of algorithms but want to improve their ability to apply algorithms to solve real-world problems.


\subsection{Stage 2: Solution Implementation}

\subsubsection{Converting the Step Tree into Comments (D3)} This feature converts each node of the step tree into code comments. When students click ``Copy to Comments'', the system intelligently inserts these comments into the appropriate sections of the code editor. This guides learners to implement their solutions within the corresponding parts of their code, ensuring a smooth transition from planning to coding while reinforcing their structured approach.

\subsubsection{Validating Code Implementation against the Step Tree (D3, D4)} The ``Check Match'' button evaluates the alignment between the code and the step tree. Steps are categorized and color-coded as: (1) \textbf{Implemented}, (2) \textbf{Incorrectly Implemented}, and (3) \textbf{To Be Coded}. Hovering over a step highlights the corresponding lines in the code, providing a direct mapping between the step tree and the code to help users efficiently debug their implementation.


\begin{figure*}[htbp]
	\centering 
	\includegraphics[width=\linewidth]{figures/processing_new.pdf}
	\caption{An illustration of DBox's data processing workflow highlights its core function—creating a step tree with node statuses from student inputs. The LLM processes learners' incomplete code or a step tree they’ve constructed. It outputs a structured JSON object containing steps, sub-steps (and sub-sub-steps, etc.), each with several attributes. Then the JSON object is rendered to the interface, preserving the original structure and only adding nodes for any missing steps. Each node keeps the student's original input, without directly revealing the correct solution. DBox encodes the status of each step with colors and provides progressive hints.}
 
	\label{fig:processing}
        \Description{}
\end{figure*}

\subsubsection{Progressive Hints for Solution Implementation (D1)}
For steps that are incorrectly implemented or yet to be coded, multi-level hints are available: (1) \textbf{General Hint}: Shows simple thought-provoking prompts/suggestions, e.g., ``How should you correctly iterate until the second last character?'' (2) \textbf{Detailed Hint} (Pseudocode): Provides simplified pseudocode to guide the user. (3) \textbf{Reveal Code} (Recommended Implementation): This option is activated only after two failed attempts. Clicking the ``Reveal Code'' button displays the recommended code implementation for the specific step.




\subsection{Backend Design}
DBox's backend is primarily powered by Large Language Models (the GPT-4o model specifically), with four distinct interactions corresponding to four buttons in the interface:
\begin{itemize}
    \item \textbf{From Editor to Step Tree}: This button sends the problem description and the user’s code to the LLM, which generates a step tree with nodes labeled as correct, incorrect, missing, or divisible.
    \item \textbf{Check Step Tree}: Clicking this button inputs the problem description and the user-constructed step tree into the LLM, which returns a labeled step tree with node statuses such as correct, incorrect, missing, or divisible.
    \item \textbf{Copy to Comments}: This button sends the problem description, current step tree, and user’s code to the LLM, generating a mapping of step tree nodes to corresponding lines of code.
    \item \textbf{Check Match}: Pressing this button sends the problem description, step tree, and user’s code to the LLM, which outputs a tree categorizing nodes as implemented, incorrectly implemented, or to be coded.
\end{itemize}

% To further illustrate, we present the data processing workflow behind two core functions: ``From Editor to Step Tree'' and ``Check Step Tree.'' Figure \ref{fig:processing} shows the workflow, where DBox processes student inputs in two modes: code or a constructed step tree. The LLM adapts based on the input. If only code is provided, it generates steps and substeps for the step tree, evaluates node status, and provides hints. If the student supplies a step tree, the LLM assesses the status of each node without altering the structure, only adding steps where needed. The LLM outputs a structured JSON format that organizes steps, substeps, and subdivisions. Each node includes the student’s input, its status (correctness, completeness, potential for subdivision), a general hint, a detailed hint, and the correct content. The step tree is \emph{selectively} displayed on the interface, primarily \emph{retaining the originally student-created structure} while highlighting missing steps. Node statuses are encoded in colors, with corresponding hints accessible through designated buttons.

To illustrate, we present the data processing workflow for two core functions: ``From Editor to Step Tree'' and ``Check Step Tree''. Figure \ref{fig:processing} shows how DBox processes student inputs in two modes: coding mode and language description mode (step tree input). Prompts adapt based on the input type. When only code is provided (\textcolor{darkblue}{dark blue} lines in Figure \ref{fig:processing}), the system first calls \colorbox{darkblue}{\textcolor{white}{[prompt from code]}} to generate a step tree, mapping each code line to a corresponding (sub)step based on the code’s meaning. It then uses the generated step tree to invoke \colorbox{orange}{\textcolor{white}{[prompt from step tree]}} (\textcolor{orange}{orange} lines) to evaluate node statuses, add missing nodes, and generate multi-level hints for incorrect or missing nodes. If the input is a natural language step tree, the LLM directly calls \colorbox{orange}{\textcolor{white}{[prompt from step tree]}} while preserving the original structure. 

The LLM outputs a JSON containing steps, sub-steps, and subdivisions, each with attributes like student original input, status (e.g., correctness, completeness), LLM-validated content, and hints. The JSON is rendered conditionally, preserving the student’s original structure while highlighting missing or incorrect steps. Even if a student's input differs from what the LLM considers correct, the original content is preserved and marked with a status color. Hints are provided and triggered progressively, offering ``just the right'' level of guidance.



\ms{
\subsection{Implementation}
For the front-end, we used native HTML, JavaScript, and jQuery. On the back-end, we deployed the application with the Flask\footnote{https://flask.palletsprojects.com/en/stable/} framework on our university's server. The code editor utilized CodeMirror\footnote{https://codemirror.net/} integrated with pyodide.js\footnote{https://pyodide.org/en/stable/} for running Python code. We employed OpenAI's GPT-4o model with a temperature of 0.8 to maintain flexibility during scaffolding. To align better with the front-end's step tree, we set the response format by setting the parameter response\_format={``type'': ``json\_object''}, restricting the LLM's output. Prompts designed with the chain-of-thought (CoT) technique \cite{wei2022chain} are detailed in the supplementary materials.
}
\begin{table*}[h]
\small
\centering
\setlength{\tabcolsep}{4pt}
\begin{threeparttable}
  \caption{Evaluation of \lumyn on SRE scenarios}
  \label{tab:sreagent-eval}
  \begin{tabular}{@{}lcccccc@{}}
    \toprule
    \multirow{2}{*}{\textbf{Models}}
      & \multicolumn{4}{c}{\textbf{Diagnosis}}
      & \multicolumn{2}{c}{\textbf{Mitigation}} \\
    \cmidrule(lr){2-5}\cmidrule(lr){6-7}
    & \textbf{pass@1 (\%)$\uparrow$}
    & \textbf{FL (NTAM)$\uparrow$}
    & \textbf{FPC (NTAM)$\uparrow$}
    & \textbf{MTTD (s)$\downarrow$}
    & \textbf{pass@1 (\%)$\uparrow$}
    & \textbf{MTTR (s)$\downarrow$}\\
    \midrule
    \textbf{granite-3.1-8B-instruct} &
    \cellcolor[gray]{0.97} $3.57 \pm 0.94$ &
    \cellcolor[gray]{0.96} $0.16 \pm 0.02$ &
    \cellcolor[gray]{0.94} $0.19 \pm 0.02$ &
    $259.92 \pm 65.01$ &
    $0.24 \pm 0.25$ &
    $845.50 \pm \text{---}$ \\
    \textbf{llama-3.1-8B-instruct} &
    $0.99 \pm 0.51$ &
    $0.07 \pm 0.01$ &
    $0.08 \pm 0.01$ &
    \cellcolor[gray]{0.85} $\textbf{57.50} \pm 2.05$ &
    \cellcolor[gray]{0.98} $1.98 \pm 0.68$ &
    \cellcolor[gray]{0.85} $\textbf{245.13} \pm 40.66$ \\
    \textbf{llama-3.3-70B-instruct} &
    \cellcolor[gray]{0.98} $3.10 \pm 0.84$ &
    \cellcolor[gray]{0.96} $0.16 \pm 0.02$ &
    \cellcolor[gray]{0.95} $0.16 \pm 0.02$ &
    \cellcolor[gray]{0.95} $191.85 \pm 31.34$ &
    \cellcolor[gray]{0.96} $3.33 \pm 0.90$ &
    \cellcolor[gray]{0.98} $776.27 \pm 252.87$ \\
    \textbf{gpt-4o} &
    \cellcolor[gray]{0.85} $\textbf{13.81} \pm 1.67$ &
    \cellcolor[gray]{0.85} $\textbf{0.39} \pm 0.05$ &
    \cellcolor[gray]{0.85} $\textbf{0.34} \pm 0.03$ &
    \cellcolor[gray]{0.86} $72.44 \pm 4.71$ &
    \cellcolor[gray]{0.85} $\textbf{11.43} \pm 1.52$ &
    \cellcolor[gray]{0.86} $282.47 \pm 30.04$ \\
    \bottomrule
  \end{tabular}
  \begin{tablenotes}
    \scriptsize
    \item[1] 42 scenarios (21 scenarios with traces and 21 without traces).
    \item[2] 10 runs per scenario per model.
    \item[3] pass@1 values are shown as percentages. `\text{---}' indicates missing data. 
    \item[4] std error for each metric is listed.
    \item[5] \textbf{FL (NTAM)} = Normalized topology-aware metric for root cause, 
          \textbf{FPC (NTAM)} = Normalized topology-aware metric for fault propagation chain (value between 0 and 1.0), 
          \textbf{MTTD} = Mean time to diagnosis (seconds), 
          \textbf{MTTR} = Mean time to repair (seconds). \textbf{Bold}: the best performance.
    \item[6] Details of NTAM are available in \Cref{appx:ntam}
  \end{tablenotes}
\end{threeparttable}
\end{table*}

\section{Results}
\label{sec:results}

\subsection{Evaluation Setup}
 

To understand the impact of reasoning and planning capabilities of LLMs on \bench scenarios, we instantiate our agents using different LLM models, both for natural language reasoning and code generation. 
Specifically, we employ GPT-4o\footnote{Checkpoint version 2024-11-20}, Llama-3.3-70B-instruct, Llama-3.1-8B-instruct, and Granite-3.1-8B-instruct for tasks that rely on natural language understanding and reasoning. For code-focused use cases, we utilize GPT-4o-mini, Llama-3.1-405b-instruct, and Mixtral-8x7b-instruct. 
All models use a context window of 128K tokens, enabling them to process more extensive input sequences.

We conduct our experiments primarily on AWS EC2 instances (m4.xlarge), although \bench can also be readily deployed on a consumer-grade laptop using a pseudo-cluster, thus making it easier to develop AI agents (Appendix \ref{appx:sre:exp_setup})

Below, we provide an overview of our baseline agents’ performance across \bench scenarios for SRE, CISO, and FinOps. Our findings indicate that both open-source and proprietary models often struggle with real-world tasks, underscoring the importance of benchmarks that push the limits of reasoning and planning in foundation models. For more comprehensive results and detailed scenario-level discussions, please refer to Appendix~\ref{appx:sre} (SRE), Appendix~\ref{appx:ciso} (CISO), and Appendix~\ref{appx:finops} (FinOps).

\begin{table*}[h]
\small
\centering
\setlength{\tabcolsep}{4pt}
\begin{threeparttable}
  \caption{Evaluation of CISO Compliance Assessment Agent on CISO scenarios}
  \label{tab:cisoagent-eval}
  \begin{tabular}{@{}lcccccc@{}}
    \toprule
    \multirow{2}{*}{\textbf{Models}}
      & \multicolumn{4}{c}{\textbf{Scenario pass@1 (\%) $\uparrow$}}
      & \multirow{2}{*}{\textbf{O/A pass@1 (\%) $\uparrow$}} 
      & \multirow{2}{*}{\textbf{TTP (s) $\downarrow$}} \\
    \cmidrule(lr){2-5}
    & \textbf{kyverno}
    & \textbf{k8s-opa}
    & \textbf{rhel-opa}
    & \textbf{kyverno-update} \\
    \midrule
    \textbf{granite-3.1-8B-instruct} &
    \cellcolor[gray]{0.99} $7.84 \pm 3.84$ &
    \cellcolor[gray]{1.00} $0.00 \pm 0.00$ &
    \cellcolor[gray]{1.00} $0.00 \pm 0.00$ &
    \cellcolor[gray]{1.00} $1.59 \pm 1.58$ &
    \cellcolor[gray]{1.00} $1.71 \pm 0.76$ &
    \cellcolor[gray]{1.00} $197.03 \pm 2.52$ \\
    \textbf{mixtral-8x7B-instruct} &
    \cellcolor[gray]{1.00} $7.35 \pm 3.19$ &
    \cellcolor[gray]{1.00} $1.43 \pm 1.42$ &
    \cellcolor[gray]{1.00} $0.00 \pm 0.00$ &
    \cellcolor[gray]{1.00} $1.29 \pm 4.34$ &
    \cellcolor[gray]{0.99} $3.94 \pm 1.03$ &
    \cellcolor[gray]{0.88} $120.63 \pm 3.77$ \\
    \textbf{llama-3.1-8B-instruct} &
    \cellcolor[gray]{0.99} $8.57 \pm 3.37$ &
    \cellcolor[gray]{1.00} $0.00 \pm 0.00$ &
    \cellcolor[gray]{1.00} $0.00 \pm 0.00$ &
    \cellcolor[gray]{0.94} $7.46 \pm 3.23$ &
    \cellcolor[gray]{0.99} $3.59 \pm 1.07$ &
    \cellcolor[gray]{0.88} $121.49 \pm 3.00$ \\
    \textbf{llama-3.3-70B-instruct} &
    \cellcolor[gray]{0.95} $18.46 \pm 4.94$ &
    \cellcolor[gray]{1.00} $0.00 \pm 0.00$ &
    \cellcolor[gray]{0.99} $1.43 \pm 2.88$ &
    \cellcolor[gray]{0.94} $8.06 \pm 3.50$ &
    \cellcolor[gray]{0.95} $9.32 \pm 1.67$ &
    \cellcolor[gray]{0.99} $189.61 \pm 2.71$ \\
    \textbf{mistral-large-2} &
    \cellcolor[gray]{1.00} $6.56 \pm 3.20$ &
    \cellcolor[gray]{0.92} $22.73 \pm 5.32$ &
    \cellcolor[gray]{0.96} $7.23 \pm 2.88$ &
    \cellcolor[gray]{0.92} $10.45 \pm 3.77$ &
    \cellcolor[gray]{0.94} $11.55 \pm 1.95$ &
    \cellcolor[gray]{0.95} $167.98 \pm 3.42$ \\
    \textbf{llama-3.1-405B-instruct} &
    \cellcolor[gray]{0.96} $16.22 \pm 4.32$ &
    \cellcolor[gray]{0.93} $20.83 \pm 4.86$ &
    \cellcolor[gray]{0.96} $8.75 \pm 3.26$ &
    \cellcolor[gray]{0.98} $3.17 \pm 2.22$ &
    \cellcolor[gray]{0.93} $12.46 \pm 1.98$ &
    \cellcolor[gray]{0.97} $178.89 \pm 3.37$ \\
    \textbf{gpt-4o-mini} &
    \cellcolor[gray]{0.96} $16.18 \pm 4.54$ &
    \cellcolor[gray]{0.85} $\textbf{43.10} \pm 6.99$ &
    \cellcolor[gray]{0.85} $\textbf{30.38} \pm 5.43$ &
    \cellcolor[gray]{0.93} $9.43 \pm 4.08$ &
    \cellcolor[gray]{0.85} $\textbf{25.19} \pm 2.80$ &
    \cellcolor[gray]{0.85} $102.40 \pm 3.70$ \\
    \textbf{gpt-4o} &
    \cellcolor[gray]{0.85} $\textbf{40.28} \pm 5.99$ &
    \cellcolor[gray]{0.86} $39.34 \pm 6.55$ &
    \cellcolor[gray]{0.96} $7.61 \pm 2.81$ &
    \cellcolor[gray]{0.85} $\textbf{17.74} \pm 4.92$ &
    \cellcolor[gray]{0.85} $24.74 \pm 2.64$ &
    \cellcolor[gray]{0.85} $\textbf{101.29} \pm 3.81$ \\
    \bottomrule
  \end{tabular}
  \begin{tablenotes}
    \scriptsize
    \item[1] 50 scenarios.
    \item[2] 8 runs per scenario per model.
    \item[3] pass@1 values are shown as percentages.
    \item[4] TTP Time to process (seconds).\\
    \item[5] \textbf{kyverno} = New K8s CIS-benchmarks on Kyverno, easy scenario class; 
          \textbf{k8s-opa} = New K8s CIS-benchmarks on OPA, medium scenario class;
          \textbf{rhel-opa} = New RHEL9 CIS-benchmarks on Ansible-OPA, medium scenario class;
          \textbf{kyverno-update} = Update K8s CIS-benchmarks on Kyverno, hard scenario class.
  \end{tablenotes}
  \vspace{-10pt}
\end{threeparttable}
\end{table*}


\begin{table*}[h]
\small
\centering
\begin{threeparttable}
  \caption{Evaluation of FinOpsAgent on FinOps scenarios.}
  \label{tab:finopsagent-eval}
  \begin{tabular}{@{}lccp{1.85cm}p{1.85cm}p{1.85cm}p{1.85cm}@{}}
    \toprule
    \multirow{2}{*}{\textbf{Models}} 
      & \multicolumn{1}{c}{\textbf{Diagnosis}}
      & \multicolumn{5}{c}{\textbf{Mitigation}} \\
    \cmidrule(lr){2-2}\cmidrule(lr){3-7}
     & \textbf{pass@1 (\%) $\uparrow$} 
     & \textbf{pass@1 (\%) $\uparrow$} 
     & \textbf{Proximity to Optimal CPU Cost $\uparrow$} 
     & \textbf{Proximity to Optimal Memory Cost $\uparrow$} 
     & \textbf{Proximity to Optimal CPU Efficiency $\uparrow$} 
     & \textbf{Proximity to Optimal Memory Efficiency $\uparrow$} \\
    \midrule
    \textbf{granite-3.1-8B-instruct} 
      & 0 
      & 0 
      & $0.47 \pm 0.01$ 
      & \cellcolor[gray]{0.94} $0.48 \pm 0.06$ 
      & $0.53 \pm 0.04$ 
      & \cellcolor[gray]{0.93} $0.94 \pm 0.01$ \\
    \textbf{llama-3.1-8B-instruct} 
      & 0 
      & 0 
      & \cellcolor[gray]{0.85} $\textbf{0.49} \pm 0.01$ 
      & $0.46 \pm 0.07$ 
      & \cellcolor[gray]{0.95} $0.56 \pm 0.08$ 
      & \cellcolor[gray]{0.85} $0.96 \pm 0.02$ \\
    \textbf{llama-3.3-70B-instruct} 
      & \cellcolor[gray]{0.92} 16.6
      & 0 
      & $0.47 \pm 0.01$ 
      & \cellcolor[gray]{0.91} $0.49 \pm 0.05$ 
      & $0.53 \pm 0.03$ 
      & \cellcolor[gray]{0.85} $0.96 \pm 0.02$ \\
    \textbf{gpt-4o} 
      & \cellcolor[gray]{0.85} \textbf{33} 
      & 0 
      & \cellcolor[gray]{0.93} $0.48 \pm 0.01$ 
      & \cellcolor[gray]{0.85} $0.51 \pm 0.02$ 
      & \cellcolor[gray]{0.85} $\textbf{0.63} \pm 0.07$ 
      & $0.92 \pm 0.08$ \\
    \bottomrule
  \end{tabular}
  \begin{tablenotes}
    \scriptsize
    \item pass@1 values are shown as percentages. 
    \item Proximity values shows how close the observed values to optimal values. 
    One represents achieving optimal and any deviations from 1 represents sub-optimal performance.
  \end{tablenotes}
\end{threeparttable}
\end{table*}

\subsection{Overall Results}
\Cref{tab:sreagent-eval}, \Cref{tab:cisoagent-eval} and \Cref{tab:finopsagent-eval} show the performance of SRE-agent, CISO-agent, and FinOps-agent respectively. 

\textbf{SRE.} 
We measure the efficiency of \lumyn on its ability to diagnose and mitigate production incidents (e.g., ``a high error rate on frontend service'').


Diagnosis efficiency is measured using pass@1\cite{chen2021evaluating} (i.e., identifying the cause as mentioned in ground truth), NTAM (Normalized Topology-Aware Metric) for root cause and fault propagation chain, and time to diagnosis\footnote{NTAM is Normalized topology-aware metric that measures the quality of the predicted root cause and fault propagation chains using a system and application topology. Refer to \Cref{appx:ntam}.}.
Mitigation efficiency is measured in terms of pass@1 (i.e., whether the alert was cleared) and mean time to repair.

As shown in \Cref{tab:sreagent-eval}, across all SRE scenarios, GPT-4o consistently outperforms the other models, achieving the highest pass@1 scores for diagnosis (13.81\%) and mitigation (11.43\%), as well as the highest score on NTAM (FL and FPC) metrics. 
Llama-3.3-70B ranks second overall, trailing GPT-4o on most metrics.
The 8B models have lower mitigation success rate. 
Surprisingly, Granite-3.1-8B (without any specialized finetuning) achieves higher accuracy than Llama-3.1-70B on the diagnosis task. 

Removing trace data can drastically reduce success rates (see \Cref{tab:appx:sre:traces} and \Cref{tab:appx:sre:disabled} in Appendix). For instance, GPT-4o's pass@1 in diagnosis falls from 13.81\% with traces to 9.52\% without them, and mitigation plummets to 2.86\%. This highlights the critical role of system observability in SRE, which \bench can evaluate under varying conditions. As there is no perfect observability in practice, how to guide SRE-agents to collect new observability data and to help SRE-agents reason about failures with incomplete observability is an important but open problem.

\textbf{CISO.}
We measure the efficacy of our agents across the four scenario classes introduced in \Cref{tab:bench_scenarios}. Each \textit{scenario\_class} imposes a distinct set of CIS-benchmarks requirements (e.g., ``minimize the admission of containers wishing to share the host network namespace''), each class has a specific level of complexity (e.g., Easy, Medium, or Hard), and generates scenario-specific code artifacts. 

The efficacy of CISO-agents is measured based on the ability to detect artifact misconfigurations (aka non-compliance, e.g., no minimum count of containers sharing namespace, or the count is above the threshold), or confirm proper configurations (aka compliance), within the varied environments of the scenario classes randomly injected with misconfigurations. 
Notably, GPT-based models dominate on both pass@1 and Time to Process metrics. The pass@1 is nearly 2x better than second-best models (alternating between llama-3.1-405b-instruct and mistral-large-2), while the TTP shows a handling of the scenarios in the minimal time across our scenario classes.

\textbf{FinOps.}
We measure the effectiveness of FinOps-agent on its ability to diagnose and mitigate the origin of cost alert (e.g., `increase in cost by 20\%'). 
Diagnosis effectiveness is measured using pass@1 (i.e., identifying the cause).
Mitigation effectiveness is measured in terms of proportional proximity to optimal cost of running, and efficiency that can be achieved for that workload.

GPT-4o consistently outperforms all other models, achieving a 33\% pass rate for diagnosing the origin of the cost increase alert. 
Performance on additional metrics related to cost and workload efficiency remains comparable across all models, with none attaining optimal CPU and memory cost or delivering high CPU efficiency. 



\subsection{Impact of Scenario Complexity}
\textbf{SRE.}
    We categorize scenarios as Easy, Medium, or Hard based on factors such as fault propagation chain length, number of resolution steps, and the diversity of technologies involved, as described in \Cref{ss:bench-sre-eq-task-complexity}. 
    Our results show that success rates (pass@1) clearly decline as the \textit{scenario\_complexity} increases.
    For example, GPT-4o (the best performing model) diagnosed only 36\%, 7.73\% and 5.0\% of the Easy, Medium, and Hard scenarios, respectively (refer to \Cref{tab:sre:diag_pass1}).
    Similarly, GPT-4o (the best performing model) successfully mitigated only 21\%, 12.27\% and 0.0\% of Easy, Medium, and Hard scenarios, respectively 
    (refer to \Cref{tab:sre:repair_pass1}). 
    
    None of the models could mitigate the hard scenarios in any of the runs, whereas over half of the Easy scenarios see successful mitigation. 
    Notably, GPT-4o is the only model that successfully diagnosed multiple ``Hard'' scenarios. 
      

\textbf{CISO.}
The complexity of the CISO scenarios is directly mapped to scenario classes. For example, \textit{scenario\_complexity} of Kyverno scenarios is Easy, \textit{scenario\_complexity} of k8s-opa and rhel-opa is Medium, while \textit{scenario\_complexity} of Kyverno-update scenarios is Hard. 
All models struggle, as expected, as the difficulty of the scenarios increases from the Easy \textit{kyverno} class to the Hard \textit{kyverno-upadate} class. 

\textbf{FinOps.}
Currently, \bench only has two FinOps scenarios, \textit{scenario\_complexity} of one is Easy and the other is Hard. None of the models, could diagnose (except for GPT-4o) or mitigate the hard scenario. 

This spectrum of complexity in \bench ensures that evaluations capture both straightforward and highly intricate problems across personas.

\subsection{Inherent Non-determinism in the Environment} 
GPT-4o remains the top performer across all evaluated personas (SRE, CISO, and FinOps), yet it still exhibits notable variability in scenario outcomes. 
For example, the SRE-agent with GPT-4o struggles to maintain deterministic behavior despite hyperparameter tuning aimed at ensuring consistency. 
SRE-agent with GPT-4o diagnosed the problem only in 6 out of 10 runs for scenario 13, 1 out of 10 runs for scenario 8, and 8 out of 10 runs for scenario 21, respectively (refer to \Cref{fig:sre:trace_on_diagnosis_pass1} for details on all scenarios).
Similarly, it mitigated 6 out of 10 runs for scenario 16, 2 out of 10 runs for scenario 8, and 5 out of 10 runs for scenario 21, respectively (refer to \Cref{fig:sre:trace_on_repair_pass1} for details on all scenarios).
This inherent non-determinism was observed with FinOps and CISO scenarios as well. 

These fluctuations arise from minor real-time telemetry changes, which can alter the large language model’s token generation. By tracking such dynamic behavior over multiple runs, \bench provides crucial insights into each agent’s robustness and reliability.





% !TEX root = ../Main.tex

%%%%\JZ{Hardware evaluation moved into hardware}
\section{Applications}
To further demonstrate the potential of our approach we illustrate possible usage-scenarios including calligraphy, outlining and inking. 
Finally, we combine the haptic feedback mechanism with a simple digital drawing application to initially explore the possibility of dynamic references.    

\subsubsection*{Calligraphy}
\figref{fig:caligraphy} illustrates writing of flourished characters, with only minimal visual guidance (single starting point). 
Our system takes the character as input, users can then draw at their desired speed. 
Although an offset from the reference path remains, the lines are smooth and the overall shape is close to the desired characters. 

\begin{figure}[!t]
    \centering
        \includegraphics[width=\columnwidth]{\dir/figures//apps-03-caligraphy.jpg}    
        \caption{Our approach can be used as a guidance system for calligraphy, where users either follow a target path very closely, or deviate if desired.}
    \label{fig:caligraphy}
\end{figure}
%
%\subsubsection*{Drawing teaching aid:}
%Connect-the-dots exercises are often used to teach children motor skills as well as stroke ordering. \figref{fig:dots} shows results from a similar exercise performed with our system, albeit using much fewer dots for visual guidance than a paper version.
%

\subsubsection*{Outlining \& inking}
\figref{fig:dragon} illustrates the effect of two core capabilities of the proposed approach. 
Here we first outline the proportions of the dragon head (gray guidance lines) and then use different pens to ink-in the details. 
Note that the system provides haptic guidance but allows the user to draw the shape in different styles (\eg the ears of the two upper dragons) and with varying high-frequency detail, while maintaining similarity to the reference shape. 
This is a direct consequence of using time-free closed loop control approach.
In this case, all four variants were drawn without changes to the system or desired path. 
\begin{figure}[!t]
    \centering
        \includegraphics[width=\columnwidth]{\dir/figures//apps-01-drawing.pdf}
    \caption{Different variants of the same dragon, drawn with identical system settings by a novice. Each pair of drawings used with different tools. First a pencil for proportions and a fine-liner (top) or pencil (bottom) to ink-in details. Multi-stroke lines are achieved by approaching each separate instance as a new drawing.}
    \label{fig:dragon}
\end{figure}


\subsubsection*{Virtual tools}
Using a digital tablet with capacitive display (\figref{fig:tablet}) we explore integrating dynamically changing references. 
In a sketching application, artist select different virtual tools, and position and configure these anywhere. 
The canvas and the haptic feedback system then pull the stylus towards these virtual guides. 
In \figref{fig:tablet}, the user has selected a tool that helps them when drawing an ellipse that snaps to a previous part of the drawing, both visually and in terms of haptics.

\begin{figure}[!t]
    \vspace{-.5em}
    \centering
    \includegraphics{\dir/figures//apps-02-digital.jpg}    \caption{Virtual tools can be used to dynamically construct a reference path combining haptic and visual feedback. Here  a simple drawing application combines freeform sketching with different virtual rules and guides that can be felt by the user.}
        \label{fig:tablet}
    \vspace{-.5em}
\end{figure}

% \subsubsection*{Topography Exploration}
% \figref{fig:topo} shows our in system can be used to explore a landscape. In this implementation the user feels drag on an ascending slope and gets pulled forward along a descending slope. For this application the desired path is updated online according to the slope of the terrain. Note, that therefore the path is unknown to the algorithm and changes with every iteration. 

% \subsubsection*{Pong}
% \figref{fig:pong} illustrates how our system can be used in the context of games. In this case, it is the classic Pong. However, it can be extended to a large variety. In this example we pull the player to the same y-position as the current ball position. For this purpose we set $w_{\dot{\theta}}$ to $0$ and thereby eliminate the forward pull. 




\section{Discussion}


In this paper, we adopted a learner-centered design approach, beginning with a formative study to identify students' challenges with existing tools. Based on these insights, we developed DBox, a tool that scaffolds students in breaking problems into smaller parts and provides personalized, adaptive support. Our user study demonstrated that DBox improved learners' performance on similar algorithmic problems, increased perceived learning gains, and fostered greater cognitive engagement, achievement, and satisfaction. In this section, we discuss design implications and generalizability based on our key findings.


\ms{
\subsection{Chaining Learners' Thoughts with Visualized Structured UI Components}

Decomposition requires students to effectively organize their thoughts. While visual elements are known to promote structured thinking and support mental model construction \cite{mcdougall2001effects, liu2010mental}, our formative and user studies revealed shortcomings in existing tools like LeetCode and ChatGPT, which rely on textual representations without adequately supporting structured mental models. In contrast, DBox uses an interactive step tree to visually organize learners' thoughts. This feature was praised by 22 of 24 participants for enhancing algorithmic thinking, serving as a progress tracker, and providing value even without AI assistance.

DBox's interactive step tree and tree-based scaffolding demonstrate the broader potential of intelligent tutoring systems (ITS) to promote active learning and self-regulated problem-solving in fields requiring problem decomposition. Similar principles could benefit STEM education, such as physics or engineering, by externalizing abstract concepts and facilitating multi-step problem-solving. Additionally, progress-tracking visual components may inspire designs for professional training tools in areas like medical diagnostics or software engineering.

\subsection{Promoting Independent Thinking and Active Decomposition Learning}

\subsubsection{\textbf{Transforming Learners from Passive Readers to Active Thinkers}}

Many coding tools provide direct answers or solutions \cite{kazemitabaar2023novices, phung2023generating}, which, while efficient, often bypass opportunities to develop critical problem-solving skills. In contrast, DBox cultivates students' decomposition abilities through structured scaffolding, fostering critical thinking and self-regulated learning in line with learning by doing \cite{anzai1979theory} and constructivist principles \cite{tobias2009constructivist}.

To strengthen decomposition skills, DBox first encourages students to develop their own decomposition strategies by coding or building a step tree from scratch. While DBox can generate parts of a step tree from a student's existing code, these steps are derived from the learner's own reasoning, with DBox acting solely as a modality converter. Besides, DBox provides feedback on tree node statuses, identifying potential errors or missing steps without directly showing the correct answer, challenging students to critically evaluate and refine their decomposition plans.


DBox's scaffolded hint system further supports decomposition skill development by providing adaptive guidance tailored to the student’s progress without overwhelming them. All hints are based on the learner's current decomposition skeleton, with the most detailed hint—``reveal substep''—triggered only after repeated attempts and struggles. Notably, even the most detailed hints prompt only one substep, requiring students to complete the rest independently. As shown in Sec \ref{hintusage}, only 19\% of hints are this detailed, with students primarily relying on simpler, thought-provoking question hints. This scaffolded support system balances guidance and independent thinking, keeping students engaged during challenges without compromising their ability to independently decompose problems \cite{kinnunen2006students}.

Based on these findings, we recommend fostering active problem-solving by shifting students from passive content consumption to active solution creation. Designers could adopt layered scaffolding, starting with minimal guidance and increasing support as needed, to help students progressively master decomposition skills while maintaining confidence and avoiding frustration. Additionally, adaptive learning techniques, such as real-time feedback and progress tracking, can further tailor the support to individual decomposition barriers, encouraging deeper engagement with decomposition tasks. Moreover, designers could integrate metacognitive strategies, such as encouraging students to articulate or reflect on their decomposition approaches, to further enhance critical thinking and foster habits of independent thinking.




\subsubsection{\textbf{Choice of Scaffolding: Balancing Independent Problem-Solving and Efforts}}

Scaffolding involves providing tailored support to help learners accomplish tasks they cannot yet complete independently \cite{kim2011scaffolding, tobias2009constructivist}. Broadly, scaffolding strategies fall into two categories \cite{van2010scaffolding}: (1) gradually reducing assistance as learners gain proficiency, and (2) encouraging independent problem-solving while offering incremental support to address challenges. DBox adopts the second approach, emphasizing independent thinking and encouraging learners to actively decompose problems \cite{zimmerman2013theories}. While our scaffolding strategies successfully enhanced critical thinking, satisfaction, and perceived usefulness, they also led to increased cognitive effort (Sec. \ref{Effects_on_UX}). This tradeoff underscores the importance of carefully balancing cognitive effort with the promotion of independent thinking.

Future designs could incorporate adaptive scaffolding that adjusts support dynamically based on learner proficiency, reducing unnecessary effort in areas where students have demonstrated competence. Additionally, while incremental scaffolding was effective for algorithmic problem-solving, tailoring strategies to different educational contexts could enhance their applicability in diverse domains. Such adaptive, context-specific approaches could further optimize the balance between support and independence in learning environments.


\subsection{Supporting Personalized Algorithmic Programming Learning}

\subsubsection{\textbf{Prioritizing Learners' Own Solutions Over Optimality}}

Algorithmic problems often have multiple solutions with varying time and space complexities. DBox prioritizes independent exploration by supporting learners' strategies rather than steering them toward a single ``optimal'' solution. Using LLM-driven prompts, it evaluates and guides each step based on the learner's reasoning, preserving their step decomposition and respecting their input—even when errors occur. While some solutions may not be the most efficient, this approach fosters autonomy by aligning feedback with learners’ thought processes instead of enforcing rigid standards.

Our user study showed that this approach improves learning outcomes and is well-received by students. We recommend designing systems that respect personalized problem-solving strategies by aligning feedback with learners' reasoning while allowing for diverse approaches. Designers should balance flexibility and rigor, using prompts and interfaces that support varied strategies while gently guiding learners toward effective solutions.


\subsubsection{\textbf{Catering to Individual Learning Styles and Contextual Needs}}

DBox accommodates diverse problem-solving approaches with two input modes: coding and natural language descriptions. Each mode offers distinct advantages tailored to different learners, stages, and situations. Learners can switch seamlessly between modes, with progress automatically synced across the interface. Features such as verifying code-step alignment ensure strong integration between modes.

Our findings reveal that this flexibility enhances user experience. Participant interaction logs and interviews revealed three usage patterns, highlighting that each mode fits different needs: code mode works well for students with a clear and detailed problem-solving plan already, while the step tree with natural language descriptions helps less experienced students with only a basic idea who are not ready to write code directly, boosting their confidence.


We argue there is no universal “best” mode for programming education—each has unique benefits depending on the learner habits, expertise, and context. Future tools should provide flexibility, like DBox, or use adaptive algorithms to recommend modes based on user needs and context. This flexibility highlights the importance of designing educational tools that accommodate varying levels of expertise and problem-solving styles, which can be generalized to other domains requiring personalized learning \cite{bernacki2021systematic}.

\subsection{Appropriate Usage of LLMs for Supporting Algorithmic Programming Learning}

\subsubsection{\textbf{Caution About LLM Errors}}

Although LLMs have shown strong performance in coding tasks \cite{finnie2023my, leinonen2023using}, they remain prone to errors. Our technical evaluation and user study revealed that even with comprehensive context—such as problem statements, user code, and natural language steps—LLM sometimes misinterprets user descriptions. These errors likely arise from discrepancies between the natural language used by students and the formal, precise language the LLM was trained on, which is primarily sourced from web-based code and comments \cite{liu2023wants}.

Such misinterpretations can hinder learning by causing confusion or frustration. While future improvements to training data and GPT versions may mitigate these issues, design strategies can help address them. \textbf{First}, LLMs should avoid giving direct solutions and instead focus on fostering active problem-solving through explanations and hints. \textbf{Second}, feedback could be paired with interactive features, like a ``Run Code'' option, allowing students to validate their reasoning. \textbf{Third}, simple tutorials could teach users how to phrase their descriptions more clearly, improving LLM's understanding. Additionally, future tools could integrate a ``Language Enhancement'' feature to suggest improvements or assess the clarity of descriptions, aiding LLM in accurately capturing user intent. Most importantly, we recommend designers prioritize technical feasibility, such as conducting rigorous evaluations like ours, before fully integrating LLMs into programming learning tools.
}



\subsubsection{\textbf{Learner-LLM Co-Decomposition of Solutions: Learner as Leader, LLM as Aid}}

A central feature of DBox is the construction of a step tree, where students break solutions into steps and sub-steps. The LLM supports this by mapping code to step descriptions, evaluating them, and offering hints. However, students maintain full control, deciding how to decompose problems and define each step, fostering independent thinking. The LLM acts solely as an aid, using a scaffolding approach to support the development of learners' Zone of Proximal Development (ZPD) \cite{chaiklin2003zone}. Unlike tools like ChatGPT or Copilot that dominate problem-solving, DBox fosters deeper cognitive engagement. Students reported greater accomplishment and found this approach more effective for learning.

This contrasts with existing human-AI collaboration paradigms in non-educational scenarios where AI usually suggest options, leaving final decisions to users \cite{dang2023choice, gao2024collabcoder, gebreegziabher2023patat, ma2019smarteye, ma2022glancee}, such as in human-AI decision-making \cite{ma2023should, ma2024towards, ma2024you}. Some educational tools, like Jin et al. \cite{jin2024teach}, use LLMs to generate solutions for students to evaluate, which aids in syntax learning but such ``LLM-generate then learner-evaluate'' approach is less effective for algorithmic problem-solving, where constructing solutions is key. Just evaluating LLM-generated contents can place a cognitive anchor on learners \cite{furnham2011literature}, limiting independent thinking and creativity. Thus, task allocation between humans and AI should align with the educational context (e.g., whether it is basic knowledge/concept learning or higher-level creative thinking). Future LLM-based educational tools should carefully define the division of roles between LLMs and learners, tailoring it to specific learning contexts and goals.




% \subsubsection{Human-LLM Co-Decomposition of Solution: AI Should Judge Instead of Recommending}

% A core interaction in DBox is the construction of a step tree, where the entire solution is broken down into a series of steps and sub-steps. We refer to this as the human-LLM co-decomposition process. In this process, the LLM behind DBox plays three roles: First, it maps the student's written code into step descriptions. Second, it evaluates the status of each step and sub-step (whether they are correct, incorrect, missing, or need further decomposition). Third, it provides hints for incorrect or missing steps or sub-steps. However, the actual construction of the step tree—such as dividing the solution into steps and sub-steps and determining the content of each node—remains primarily the student's responsibility.

% This division of labor maximizes student engagement in independent thinking and problem-solving. The LLM does not provide any suggestions for decomposition nor directly recommend content for specific steps, aligning with the scaffolding educational approach, where guidance is provided appropriately, but the main task of forming the solution is left to the students.

% In contrast, when students directly seek help from an LLM, such as asking questions in ChatGPT or using Copilot for code completion, the LLM takes too much initiative by directly offering ideas or code. In our co-decomposition design, however, students demonstrated higher cognitive engagement and more active critical thinking. Furthermore, students reported that constructing solutions in this way gave them a greater sense of achievement and made them feel the process was more beneficial for learning, leading to higher satisfaction with the experience.

% Related work has proposed similar approaches. For instance, XXX, in the context of problem-solving, uses the "learning by teaching" concept, where students take on the tasks of judging and teaching, while the LLM generates most of the solutions. Compared to our approach, their division of labor between the student and the LLM is reversed. This method works well in introductory programming, where the focus is on mastering syntax. Having students guide the LLM to generate code or evaluate potentially incorrect code produced by the LLM is an effective way to quiz them. However, in our work, which focuses on algorithmic programming, the key step is constructing a solution from scratch. If the LLM builds the solution, leaving students only to judge it, it hampers their independent thinking.

% Thus, when designing LLM-based educational tools in the future, it is crucial to consider the specific context to effectively allocate tasks between the student and the LLM, ensuring that students derive the maximum benefit from the co-decomposition process.


% \subsection{Future Design Opportunities}

% \emph{Providing Appropriate Generative Assistance:} While DBox promotes independent problem-solving, some users showed interest in features like auto-completion for trivial coding tasks. Future versions could balance promoting independence with targeted assistance by enabling adjustable difficulty levels and offering contextual suggestions when appropriate.

% \emph{Covering All Stages of Algorithmic Programming:} DBox currently lacks a focus on foundational algorithm instruction and problem comprehension. Future iterations could include features like generating distractor solutions, input-output tests, and step-by-step rephrasing to help students grasp key concepts and understand the coding problem.

% \emph{Combining Step Trees with Dialogue:} Users can currently describe their thought processes but cannot ask questions. Adding a dialogue system to the step tree would allow students to share challenges and ask follow-up questions. GPT could then provide guided feedback without giving direct answers, supporting independent problem-solving.





% \emph{Other Important Features.} DBox could offer more control by allowing users to select specific parts of their code for targeted evaluation and guidance. A ``review'' feature could also help students reflect on key stumbling points, understand where their thought process went wrong, and how they eventually solved the problem.


% \subsection{Future Design Opportunities}

% \emph{Providing Appropriate Generative Assistance.} Our tool primarily focuses on encouraging users to create the step tree and write the code independently, with the system mainly serving as a judge. However, users expressed a desire for some intelligent completion features, particularly for repetitive or simple code, allowing them to focus their efforts on learning the key parts. Future improvements should strike a balance between fostering independent thinking and providing appropriate assistance. One approach could be designing basic rules where the tool offers intelligent suggestions and completions for parts unrelated to the core logic, while maintaining the current level of independence for key learning areas. Additionally, the system could offer different modes, allowing users to choose the level of assistance, from basic judgment-only feedback to a combination of judgment, guidance, necessary completions, and even on-demand suggestions.

% \emph{Covering All Stages of Algorithmic Programming.} Currently, our system does not cover the basic teaching of algorithms or the problem comprehension stage. In the future, to address the diversity and uncertainty in solutions and help students grasp multiple approaches, we could expand assistance during the idea formation phase. For example, GPT could generate multiple potential solutions with distractors, prompting students to identify the one that meets the problem's complexity requirements. We could also introduce specialized algorithm training, where students select a specific algorithm, and the system’s guidance focuses solely on that algorithm. To assist with problem comprehension, we could incorporate input-output tests to check students' understanding of the problem and step-by-step rephrasing to help them grasp more complex problems.

% \emph{Combining Interactive Step Trees with Dialogue Boxes.} Sometimes users want to describe their difficulties, and currently, we ask them to outline their thought processes. Additionally, users may want to ask follow-up questions. In the future, we could combine the structured step tree with a small dialogue box. The primary goal would still be to construct the step tree, but users could engage in a conversation with GPT in the context of the current step tree or a specific step. Importantly, GPT should guide the user without revealing direct answers.

% \emph{Other Important Features.} First, DBox could offer learners more control, such as allowing users to select specific parts of the code for targeted evaluation and guidance. We could also introduce a summary feature for key stumbling points, helping students reflect on the challenges they faced, where their thought process went wrong, and how they eventually overcame the problem.




\subsection{Limitations and Future Work}

This study has several limitations. \emph{First}, we tested DBox's effectiveness on only two problem types; future work should examine a broader range of algorithms. \emph{Second}, participants engaged in just one learning session per condition due to time constraints, whereas mastering algorithmic problems typically requires extended practice. Longitudinal studies should explore how DBox supports skill development over time, including changes in mental models and skill retention. \emph{Third}, we assessed learning gains based on correctness in a test session using similar learning and test problems. Future research should evaluate knowledge transfer to less similar problems. Due to time constraints, we conducted a single post-test rather than a pre-post comparison. While pre-test expertise filtering and randomization minimized prior familiarity effects, a more rigorous pre-post design would yield more accurate learning gain measurements. Looking ahead, we plan to release DBox as a Chrome plugin for integration with existing coding platforms, enabling large-scale field studies. This will allow for the collection of long-term usage data and periodic surveys to identify usage patterns and learning experiences over time.



% This study has several limitations. First, in our within-subject design, we selected two types of algorithm problems—Greedy and Binary Search—and randomly assigned them to two conditions (DBox and baseline). However, selection bias may still exist, as some participants might naturally excel at one type of algorithm. Although we addressed this by filtering participants' proficiency through a pre-test and using a Latin Square design, further validation across a broader range of algorithms is needed in future work.

% Second, students experienced only one learning session per condition before the test session. While this allowed for a fair comparison, mastering algorithmic problems typically requires extended practice. Future work should explore how DBox supports students' long-term improvement in algorithmic skills. Longitudinal studies could provide insights into changes in learners' mental models, allowing students more time to deepen their understanding and refine their decomposition methods. Additionally, retention tests could assess whether students can still apply learned problem-solving methods after a time gap.

% We measured learning gains through correctness scores in the test session, with relatively similar learning and test problems. Future work should explore students' ability to transfer their knowledge to problems with lower similarity. Due to time constraints, we opted for a single post-test rather than a pre-post comparison. While we minimized prior familiarity effects by filtering participants and randomizing problem assignments, future studies could adopt a more rigorous pre-post test design for better measurement of learning gains.

% Looking ahead, we plan to release DBox as a Chrome plugin for integration with existing online coding platforms and large-scale real-world testing. In such settings, where students may be more motivated (e.g., preparing for algorithm interviews), we can gather long-term usage data while ensuring privacy. We also plan to conduct periodic surveys to track changes in students' usage patterns and learning experiences over time.



% \subsection{Limitations and Future Work}

% This study has several limitations. First, in our within-subjects study, we selected two types of algorithm problems, Greedy and Binary Search, and randomly assigned them to two conditions, DBox and the baseline. However, there may still be selection bias, where some participants were naturally better at one type of algorithm. While we mitigated this issue to a large extent by filtering participants' proficiency through a pre-test and employing a Latin Square design to randomize the problem-condition assignment, there is still room for improvement. Future work should validate DBox's effectiveness across a broader range of problem types.

% Second, in our experiment, students only experienced one learning session in each condition before moving on to the test session. Although this comparison was fair (as both conditions had only one learning session), mastering an algorithmic problem often requires extended practice. Future work should explore how DBox can help students gradually improve their algorithmic programming skills over time. Longitudinal studies may reveal significant changes in learners' mental models, providing more time for them to understand a specific algorithm and enhance their decomposition methods. Additionally, future studies could include retention tests to measure whether students can still effectively apply previously learned problem-solving methods after a period of time.

% Furthermore, when objectively measuring students' learning gains, we calculated their correctness score in the test session. On the one hand, the learning session and test session problems had a relatively high degree of similarity. Future work should investigate whether students can transfer what they have learned to solve problems of the same algorithm type with lower similarity. On the other hand, due to time constraints, we did not include a pre-post test comparison, opting for a single post-test instead. This result might be influenced by students' pre-existing familiarity with the problems. Although we mitigated this issue by filtering for familiarity (ensuring participants were not too familiar with the problems) and randomizing the problem assignments, future work could include a more rigorous pre-post test design to better calculate students' learning gains.

% Moreover, DBox is currently only applied in algorithmic programming, specifically solving algorithm problems. However, this decomposition-based computational thinking approach could be extended to other learning scenarios, such as project-based learning. Future work could explore how to adapt DBox to broader educational contexts outside of algorithmic programming.

% Looking forward, we aim to deploy DBox in real-world algorithm courses. Since algorithms are a core required subject in undergraduate computer science curricula, we hope to investigate how students who have just learned algorithm concepts use DBox to develop their problem-solving skills. Additionally, we plan to convert DBox into a Chrome plugin and release it in the Chrome Web Store for real-world testing. This would allow DBox to seamlessly integrate with existing online coding platforms, enabling large-scale experiments. In such settings, students' motivation may be stronger (e.g., a graduate preparing for an algorithm interview), leading to more realistic usage patterns. Students could use DBox to tackle a wide variety of algorithm problems. We hope to collect long-term (e.g., six-month) usage data from real-world users while ensuring privacy, and use periodic surveys to capture changes in students' usage patterns and learning experiences over time.





\section{Conclusion}
% In this paper, we introduced Decomposition Box (DBox), a novel tool designed to scaffold learners in decomposing problems during algorithmic programming learning. Based on insights from a formative study, we identified key design goals to address the limitations of existing tools in algorithmic programming education. DBox supports two critical stages of the programming process: idea formation and idea implementation. By offering two modes (code mode and language mode), it encourages users to independently develop their solution strategies. The interactive, visual step tree helps students break down problems and build a structured mental model. DBox provides fine-grained, step-level feedback, enabling students to quickly identify issues, while its multi-level guidance offers targeted support without undermining independent thinking.

% Our user study demonstrated that DBox led to significantly higher learning gains, cognitive engagement, and critical thinking. Students reported a stronger sense of achievement and found the assistance both appropriate and effective for their learning. We identified three main usage patterns, underscoring the importance of respecting students' problem-solving habits and offering them autonomy. The learner-LLM co-decomposition model we designed promotes independent thinking while allowing the LLM to contribute meaningfully, even with occasional imperfections. 

% We hope the formative study, design goals, features, technical evaluation, and key findings from this work will inspire future research on developing educational tools for broader programming learning.
In this paper, we introduced DBox, an interactive tool designed to help learners decompose algorithmic programming problems by supporting both solution formation and implementation. Featuring an intuitive tree-like box widget, DBox accepts input in both code and natural language, fostering independent problem-solving while its step tree structure helps learners develop structured mental models. It provides step-level feedback and layered guidance without compromising learner autonomy.
Our user study showed that DBox significantly improved learning outcomes, cognitive engagement, and critical thinking, with students reporting a greater sense of achievement and finding the support highly effective. Additionally, we identified three key usage patterns, highlighting the importance of accommodating individual problem-solving styles. Moreover, our findings suggest that the learner-LLM co-decomposition approach fosters independent thinking while providing meaningful guidance, even with occasional imperfections.
We hope the insights from our system design will inspire future research on integrating LLMs into educational tools for programming learning.