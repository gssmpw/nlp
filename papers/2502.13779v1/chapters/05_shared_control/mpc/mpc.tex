\chapter{Optimal control for electromagnetic haptic guidance systems}
\label{ch:control:optimal}
\contribution{
In \chapref{ch:shared:contact} and \chapref{ch:shared:volumetric}, we introduced novel actuator and sensing techniques for electromagnetic haptic devices. However, the challenge of effectively controlling the actuator given the sensor input remains unresolved. In this chapter, we address this by exploring an optimal control method for electromagnetic haptic guidance systems. Our approach assists users in pen-based tasks such as drawing, sketching, and designing, while ensuring that user agency is maintained.
%
Traditional methods force the stylus to follow a continuously advancing setpoint on a target trajectory, often resulting in loss of haptic guidance or unintended snapping. In contrast, our control approach gently pulls users towards the target trajectory, allowing for spontaneous adaptation and drawing at their own speed. To achieve this flexible guidance, we iteratively predict the motion of an input device (such as a pen) and dynamically adjust the position and strength of an underlying electromagnetic actuator.
%
To enable real-time computation, we introduce a novel, fast approximate model of an electromagnet. We validate our approach on a prototype hardware platform featuring an electromagnet on a bi-axial linear stage and demonstrate its effectiveness through various applications. Experimental results indicate that our method is more accurate and preferred by users compared to open-loop and time-dependent closed-loop approaches.
}


\begin{figure}
    \centering
  \includegraphics[width=\textwidth]{\dir/figures/teaser_new-02.pdf}
  \caption{We propose an optimal control scheme for electromagnetic guidance systems (\textit{left}). A target trajectory is provided, on which users are guided. They can always adapt the trajectory, our optimization then guides users back to the target (\textit{offset between target and drawing for illustration purposes}).}
  \label{fig:teaser_mpc}
\end{figure}

\section{Introduction}
\label{section:introduction}

% redirection is unique and important in VR
Virtual Reality (VR) systems enable users to embody virtual avatars by mirroring their physical movements and aligning their perspective with virtual avatars' in real time. 
As the head-mounted displays (HMDs) block direct visual access to the physical world, users primarily rely on visual feedback from the virtual environment and integrate it with proprioceptive cues to control the avatar’s movements and interact within the VR space.
Since human perception is heavily influenced by visual input~\cite{gibson1933adaptation}, 
VR systems have the unique capability to control users' perception of the virtual environment and avatars by manipulating the visual information presented to them.
Leveraging this, various redirection techniques have been proposed to enable novel VR interactions, 
such as redirecting users' walking paths~\cite{razzaque2005redirected, suma2012impossible, steinicke2009estimation},
modifying reaching movements~\cite{gonzalez2022model, azmandian2016haptic, cheng2017sparse, feick2021visuo},
and conveying haptic information through visual feedback to create pseudo-haptic effects~\cite{samad2019pseudo, dominjon2005influence, lecuyer2009simulating}.
Such redirection techniques enable these interactions by manipulating the alignment between users' physical movements and their virtual avatar's actions.

% % what is hand/arm redirection, motivation of study arm-offset
% \change{\yj{i don't understand the purpose of this paragraph}
% These illusion-based techniques provide users with unique experiences in virtual environments that differ from the physical world yet maintain an immersive experience. 
% A key example is hand redirection, which shifts the virtual hand’s position away from the real hand as the user moves to enhance ergonomics during interaction~\cite{feuchtner2018ownershift, wentzel2020improving} and improve interaction performance~\cite{montano2017erg, poupyrev1996go}. 
% To increase the realism of virtual movements and strengthen the user’s sense of embodiment, hand redirection techniques often incorporate a complete virtual arm or full body alongside the redirected virtual hand, using inverse kinematics~\cite{hartfill2021analysis, ponton2024stretch} or adjustments to the virtual arm's movement as well~\cite{li2022modeling, feick2024impact}.
% }

% noticeability, motivation of predicting a probability, not a classification
However, these redirection techniques are most effective when the manipulation remains undetected~\cite{gonzalez2017model, li2022modeling}. 
If the redirection becomes too large, the user may not mitigate the conflict between the visual sensory input (redirected virtual movement) and their proprioception (actual physical movement), potentially leading to a loss of embodiment with the virtual avatar and making it difficult for the user to accurately control virtual movements to complete interaction tasks~\cite{li2022modeling, wentzel2020improving, feuchtner2018ownershift}. 
While proprioception is not absolute, users only have a general sense of their physical movements and the likelihood that they notice the redirection is probabilistic. 
This probability of detecting the redirection is referred to as \textbf{noticeability}~\cite{li2022modeling, zenner2024beyond, zenner2023detectability} and is typically estimated based on the frequency with which users detect the manipulation across multiple trials.

% version B
% Prior research has explored factors influencing the noticeability of redirected motion, including the redirection's magnitude~\cite{wentzel2020improving, poupyrev1996go}, direction~\cite{li2022modeling, feuchtner2018ownershift}, and the visual characteristics of the virtual avatar~\cite{ogawa2020effect, feick2024impact}.
% While these factors focus on the avatars, the surrounding virtual environment can also influence the users' behavior and in turn affect the noticeability of redirection.
% One such prominent external influence is through the visual channel - the users' visual attention is constantly distracted by complex visual effects and events in practical VR scenarios.
% Although some prior studies have explored how to leverage user blindness caused by visual distractions to redirect users' virtual hand~\cite{zenner2023detectability}, there remains a gap in understanding how to quantify the noticeability of redirection under visual distractions.

% visual stimuli and gaze behavior
Prior research has explored factors influencing the noticeability of redirected motion, including the redirection's magnitude~\cite{wentzel2020improving, poupyrev1996go}, direction~\cite{li2022modeling, feuchtner2018ownershift}, and the visual characteristics of the virtual avatar~\cite{ogawa2020effect, feick2024impact}.
While these factors focus on the avatars, the surrounding virtual environment can also influence the users' behavior and in turn affect the noticeability of redirection.
This, however, remains underexplored.
One such prominent external influence is through the visual channel - the users' visual attention is constantly distracted by complex visual effects and events in practical VR scenarios.
We thus want to investigate how \textbf{visual stimuli in the virtual environment} affect the noticeability of redirection.
With this, we hope to complement existing works that focus on avatars by incorporating environmental visual influences to enable more accurate control over the noticeability of redirected motions in practical VR scenarios.
% However, in realistic VR applications, the virtual environment often contains complex visual effects beyond the virtual avatar itself. 
% We argue that these visual effects can \textbf{distract users’ visual attention and thus affect the noticeability of redirection offsets}, while current research has yet taken into account.
% For instance, in a VR boxing scenario, a user’s visual attention is likely focused on their opponent rather than on their virtual body, leading to a lower noticeability of redirection offsets on their virtual movements. 
% Conversely, when reaching for an object in the center of their field of view, the user’s attention is more concentrated on the virtual hand’s movement and position to ensure successful interaction, resulting in a higher noticeability of offsets.

Since each visual event is a complex choreography of many underlying factors (type of visual effect, location, duration, etc.), it is extremely difficult to quantify or parameterize visual stimuli.
Furthermore, individuals respond differently to even the same visual events.
Prior neuroscience studies revealed that factors like age, gender, and personality can influence how quickly someone reacts to visual events~\cite{gillon2024responses, gale1997human}. 
Therefore, aiming to model visual stimuli in a way that is generalizable and applicable to different stimuli and users, we propose to use users' \textbf{gaze behavior} as an indicator of how they respond to visual stimuli.
In this paper, we used various gaze behaviors, including gaze location, saccades~\cite{krejtz2018eye}, fixations~\cite{perkhofer2019using}, and the Index of Pupil Activity (IPA)~\cite{duchowski2018index}.
These behaviors indicate both where users are looking and their cognitive activity, as looking at something does not necessarily mean they are attending to it.
Our goal is to investigate how these gaze behaviors stimulated by various visual stimuli relate to the noticeability of redirection.
With this, we contribute a model that allows designers and content creators to adjust the redirection in real-time responding to dynamic visual events in VR.

To achieve this, we conducted user studies to collect users' noticeability of redirection under various visual stimuli.
To simulate realistic VR scenarios, we adopted a dual-task design in which the participants performed redirected movements while monitoring the visual stimuli.
Specifically, participants' primary task was to report if they noticed an offset between the avatar's movement and their own, while their secondary task was to monitor and report the visual stimuli.
As realistic virtual environments often contain complex visual effects, we started with simple and controlled visual stimulus to manage the influencing factors.

% first user study, confirmation study
% collect data under no visual stimuli, different basic visual stimuli
We first conducted a confirmation study (N=16) to test whether applying visual stimuli (opacity-based) actually affects their noticeability of redirection. 
The results showed that participants were significantly less likely to detect the redirection when visual stimuli was presented $(F_{(1,15)}=5.90,~p=0.03)$.
Furthermore, by analyzing the collected gaze data, results revealed a correlation between the proposed gaze behaviors and the noticeability results $(r=-0.43)$, confirming that the gaze behaviors could be leveraged to compute the noticeability.

% data collection study
We then conducted a data collection study to obtain more accurate noticeability results through repeated measurements to better model the relationship between visual stimuli-triggered gaze behaviors and noticeability of redirection.
With the collected data, we analyzed various numerical features from the gaze behaviors to identify the most effective ones. 
We tested combinations of these features to determine the most effective one for predicting noticeability under visual stimuli.
Using the selected features, our regression model achieved a mean squared error (MSE) of 0.011 through leave-one-user-out cross-validation. 
Furthermore, we developed both a binary and a three-class classification model to categorize noticeability, which achieved an accuracy of 91.74\% and 85.62\%, respectively.

% evaluation study
To evaluate the generalizability of the regression model, we conducted an evaluation study (N=24) to test whether the model could accurately predict noticeability with new visual stimuli (color- and scale-based animations).
Specifically, we evaluated whether the model's predictions aligned with participants' responses under these unseen stimuli.
The results showed that our model accurately estimated the noticeability, achieving mean squared errors (MSE) of 0.014 and 0.012 for the color- and scale-based visual stimili, respectively, compared to participants' responses.
Since the tested visual stimuli data were not included in the training, the results suggested that the extracted gaze behavior features capture a generalizable pattern and can effectively indicate the corresponding impact on the noticeability of redirection.

% application
Based on our model, we implemented an adaptive redirection technique and demonstrated it through two applications: adaptive VR action game and opportunistic rendering.
We conducted a proof-of-concept user study (N=8) to compare our adaptive redirection technique with a static redirection, evaluating the usability and benefits of our adaptive redirection technique.
The results indicated that participants experienced less physical demand and stronger sense of embodiment and agency when using the adaptive redirection technique. 
These results demonstrated the effectiveness and usability of our model.

In summary, we make the following contributions.
% 
\begin{itemize}
    \item 
    We propose to use users' gaze behavior as a medium to quantify how visual stimuli influences the noticebility of redirection. 
    Through two user studies, we confirm that visual stimuli significantly influences noticeability and identify key gaze behavior features that are closely related to this impact.
    \item 
    We build a regression model that takes the user's gaze behavioral data as input, then computes the noticeability of redirection.
    Through an evaluation study, we verify that our model can estimate the noticeability with new participants under unseen visual stimuli.
    These findings suggest that the extracted gaze behavior features effectively capture the influence of visual stimuli on noticeability and can generalize across different users and visual stimuli.
    \item 
    We develop an adaptive redirection technique based on our regression model and implement two applications with it.
    With a proof-of-concept study, we demonstrate the effectiveness and potential usability of our regression model on real-world use cases.

\end{itemize}

% \delete{
% Virtual Reality (VR) allows the user to embody a virtual avatar by mirroring their physical movements through the avatar.
% As the user's visual access to the physical world is blocked in tasks involving motion control, they heavily rely on the visual representation of the avatar's motions to guide their proprioception.
% Similar to real-world experiences, the user is able to resolve conflicts between different sensory inputs (e.g., vision and motor control) through multisensory integration, which is essential for mitigating the sensory noise that commonly arises.
% However, it also enables unique manipulations in VR, as the system can intentionally modify the avatar's movements in relation to the user's motions to achieve specific functional outcomes,
% for example, 
% % the manipulations on the avatar's movements can 
% enabling novel interaction techniques of redirected walking~\cite{razzaque2005redirected}, redirected reaching~\cite{gonzalez2022model}, and pseudo haptics~\cite{samad2019pseudo}.
% With small adjustments to the avatar's movements, the user can maintain their sense of embodiment, due to their ability to resolve the perceptual differences.
% % However, a large mismatch between the user and avatar's movements can result in the user losing their sense of embodiment, due to an inability to resolve the perceptual differences.
% }

% \delete{
% However, multisensory integration can break when the manipulation is so intense that the user is aware of the existence of the motion offset and no longer maintains the sense of embodiment.
% Prior research studied the intensity threshold of the offset applied on the avatar's hand, beyond which the embodiment will break~\cite{li2022modeling}. 
% Studies also investigated the user's sensitivity to the offsets over time~\cite{kohm2022sensitivity}.
% Based on the findings, we argue that one crucial factor that affects to what extent the user notices the offset (i.e., \textit{noticeability}) that remains under-explored is whether the user directs their visual attention towards or away from the virtual avatar.
% Related work (e.g., Mise-unseen~\cite{marwecki2019mise}) has showcased applications where adjustments in the environment can be made in an unnoticeable manner when they happen in the area out of the user's visual field.
% We hypothesize that directing the user's visual attention away from the avatar's body, while still partially keeping the avatar within the user's field-of-view, can reduce the noticeability of the offset.
% Therefore, we conduct two user studies and implement a regression model to systematically investigate this effect.
% }

% \delete{
% In the first user study (N = 16), we test whether drawing the user's visual attention away from their body impacts the possibility of them noticing an offset that we apply to their arm motion in VR.
% We adopt a dual-task design to enable the alteration of the user's visual attention and a yes/no paradigm to measure the noticeability of motion offset. 
% The primary task for the user is to perform an arm motion and report when they perceive an offset between the avatar's virtual arm and their real arm.
% In the secondary task, we randomly render a visual animation of a ball turning from transparent to red and becoming transparent again and ask them to monitor and report when it appears.
% We control the strength of the visual stimuli by changing the duration and location of the animation.
% % By changing the time duration and location of the visual animation, we control the strengths of attraction to the users.
% As a result, we found significant differences in the noticeability of the offsets $(F_{(1,15)}=5.90,~p=0.03)$ between conditions with and without visual stimuli.
% Based on further analysis, we also identified the behavioral patterns of the user's gaze (including pupil dilation, fixations, and saccades) to be correlated with the noticeability results $(r=-0.43)$ and they may potentially serve as indicators of noticeability.
% }

% \delete{
% To further investigate how visual attention influences the noticeability, we conduct a data collection study (N = 12) and build a regression model based on the data.
% The regression model is able to calculate the noticeability of the offset applied on the user's arm under various visual stimuli based on their gaze behaviors.
% Our leave-one-out cross-validation results show that the proposed method was able to achieve a mean-squared error (MSE) of 0.012 in the probability regression task.
% }

% \delete{
% To verify the feasibility and extendability of the regression model, we conduct an evaluation study where we test new visual animations based on adjustments on scale and color and invite 24 new participants to attend the study.
% Results show that the proposed method can accurately estimate the noticeability with an MSE of 0.014 and 0.012 in the conditions of the color- and scale-based visual effects.
% Since these animations were not included in the dataset that the regression model was built on, the study demonstrates that the gaze behavioral features we extracted from the data capture a generalizable pattern of the user's visual attention and can indicate the corresponding impact on the noticeability of the offset.
% }

% \delete{
% Finally, we demonstrate applications that can benefit from the noticeability prediction model, including adaptive motion offsets and opportunistic rendering, considering the user's visual attention. 
% We conclude with discussions of our work's limitations and future research directions.
% }

% \delete{
% In summary, we make the following contributions.
% }
% % 
% \begin{itemize}
%     \item 
%     \delete{
%     We quantify the effects of the user's visual attention directed away by stimuli on their noticeability of an offset applied to the avatar's arm motion with respect to the user's physical arm. 
%     Through two user studies, we identified gaze behavioral features that are indicative of the changes in noticeability.
%     }
%     \item 
%     \delete{We build a regression model that takes the user's gaze behavioral data and the offset applied to the arm motion as input, then computes the probability of the user noticing the offset.
%     Through an evaluation study, we verified that the model needs no information about the source attracting the user's visual attention and can be generalizable in different scenarios.
%     }
%     \item 
%     \delete{We demonstrate two applications that potentially benefit from the regression model, including adaptive motion offsets and opportunistic rendering.
%     }

% \end{itemize}

\begin{comment}
However, users will lose the sense of embodiment to the virtual avatars if they notice the offset between the virtual and physical movements.
To address this, researchers have been exploring the noticing threshold of offsets with various magnitudes and proposing various redirection techniques that maintain the sense of embodiment~\cite{}.

However, when users embody virtual avatars to explore virtual environments, they encounter various visual effects and content that can attract their attention~\cite{}.
During this, the user may notice an offset when he observes the virtual movement carefully while ignoring it when the virtual contents attract his attention from the movements.
Therefore, static offset thresholds are not appropriate in dynamic scenarios.

Past research has proposed dynamic mapping techniques that adapted to users' state, such as hand moving speed~\cite{frees2007prism} or ergonomically comfortable poses~\cite{montano2017erg}, but not considering the influence of virtual content.
More specifically, PRISM~\cite{frees2007prism} proposed adjusting the C/D ratio with a non-linear mapping according to users' hand moving speed, but it might not be optimal for various virtual scenarios.
While Erg-O~\cite{montano2017erg} redirected users' virtual hands according to the virtual target's relative position to reduce physical fatigue, neglecting the change of virtual environments. 

Therefore, how to design redirection techniques in various scenarios with different visual attractions remains unknown.
To address this, we investigate how visual attention affects the noticing probability of movement offsets.
Based on our experiments, we implement a computational model that automatically computes the noticing probability of offsets under certain visual attractions.
VR application designers and developers can easily leverage our model to design redirection techniques maintaining the sense of embodiment adapt to the user's visual attention.
We implement a dynamic redirection technique with our model and demonstrate that it effectively reduces the target reaching time without reducing the sense of embodiment compared to static redirection techniques.

% Need to be refined
This paper offers the following contributions.
\begin{itemize}
    \item We investigate how visual attractions affect the noticing probability of redirection offsets.
    \item We construct a computational model to predict the noticing probability of an offset with a given visual background.
    \item We implement a dynamic redirection technique adapting to the visual background. We evaluate the technique and develop three applications to demonstrate the benefits. 
\end{itemize}



First, we conducted a controlled experiment to understand how users perceived the movement offset while subjected to various distractions.
Since hand redirection is one of the most frequently used redirections in VR interactions, we focused on the dynamic arm movements and manually added angular offsets to the' elbow joint~\cite{li2022modeling, gonzalez2022model, zenner2019estimating}. 
We employed flashing spheres in the user's field of view as distractions to attract users' visual attention.
Participants were instructed to report the appearing location of the spheres while simultaneously performing the arm movements and reporting if they perceived an offset during the movement. 
(\zhipeng{Add the results of data collection. Analyze the influence of the distance between the gaze map and the offset.}
We measured the visual attraction's magnitude with the gaze distribution on it.
Results showed that stronger distractions made it harder for users to notice the offset.)
\zhipeng{Need to rewrite. Not sure to use gaze distribution or a metric obtained from the visual content.}
Secondly, we constructed a computational model to predict the noticing probability of offsets with given visual content.
We analyzed the data from the user studies to measure the influence of visual attractions on the noticing probability of offsets.
We built a statistical model to predict the offset's noticing probability with a given visual content.
Based on the model, we implement a dynamic redirection technique to adjust the redirection offset adapted to the user's current field of view.
We evaluated the technique in a target selection task compared to no hand redirection and static hand redirection.
\zhipeng{Add the results of the evaluation.}
Results showed that the dynamic hand redirection technique significantly reduced the target selection time with similar accuracy and a comparable sense of embodiment.
Finally, we implemented three applications to demonstrate the potential benefits of the visual attention adapted dynamic redirection technique.
\end{comment}

% This one modifies arm length, not redirection
% \citeauthor{mcintosh2020iteratively} proposed an adaptation method to iteratively change the virtual avatar arm's length based on the primary tasks' performance~\cite{mcintosh2020iteratively}.



% \zhipeng{TO ADD: what is redirection}
% Redirection enables novel interactions in Virtual Reality, including redirected walking, haptic redirection, and pseudo haptics by introducing an offset to users' movement.
% \zhipeng{TO ADD: extend this sentence}
% The price of this is that users' immersiveness and embodiment in VR can be compromised when they notice the offset and perceive the virtual movement not as theirs~\cite{}.
% \zhipeng{TO ADD: extend this sentence, elaborate how the virtual environment attracts users' attention}
% Meanwhile, the visual content in the virtual environment is abundant and consistently captures users' attention, making it harder to notice the offset~\cite{}.
% While previous studies explored the noticing threshold of the offsets and optimized the redirection techniques to maintain the sense of embodiment~\cite{}, the influence of visual content on the probability of perceiving offsets remains unknown.  
% Therefore, we propose to investigate how users perceive the redirection offset when they are facing various visual attractions.


% We conducted a user study to understand how users notice the shift with visual attractions.
% We used a color-changing ball to attract the user's attention while instructing users to perform different poses with their arms and observe it meanwhile.
% \zhipeng{(Which one should be the primary task? Observe the ball should be the primary one, but if the primary task is too simple, users might allocate more attention on the secondary task and this makes the secondary task primary.)}
% \zhipeng{(We need a good and reasonable dual-task design in which users care about both their pose and the visual content, at least in the evaluation study. And we need to be able to control the visual content's magnitude and saliency maybe?)}
% We controlled the shift magnitude and direction, the user's pose, the ball's size, and the color range.
% We set the ball's color-changing interval as the independent factor.
% We collect the user's response to each shift and the color-changing times.
% Based on the collected data, we constructed a statistical model to describe the influence of visual attraction on the noticing probability.
% \zhipeng{(Are we actually controlling the attention allocation? How do we measure the attracting effect? We need uniform metrics, otherwise it is also hard for others to use our knowledge.)}
% \zhipeng{(Try to use eye gaze? The eye gaze distribution in the last five seconds to decide the attention allocation? Basically constructing a model with eye gaze distribution and noticing probability. But the user's head is moving, so the eye gaze distribution is not aligned well with the current view.)}

% \zhipeng{Saliency and EMD}
% \zhipeng{Gaze is more than just a point: Rethinking visual attention
% analysis using peripheral vision-based gaze mapping}

% Evaluation study(ideal case): based on the visual content, adjusting the redirection magnitude dynamically.

% \zhipeng{(The risk is our model's effect is trivial.)}

% Applications:
% Playing Lego while watching demo videos, we can accelerate the reaching process of bricks, and forbid the redirection during the manipulation.

% Beat saber again: but not make a lot of sense? Difficult game has complicated visual effects, while allows larger shift, but do not need large shift with high difficulty



\section{Method Overview}
The goal of our online optimal control scheme is to allow users to maintain control and agency over the input device (e.g., pen, stylus), while experiencing dynamic guidance from the system. Importantly, it leverages \textit{time-free references}, and thus the dynamics are entirely driven by the pen position over time, which is different from approaches such as MPC.

The proposed optimization scheme allows us to adjust the magnet position and strength such that it gently pulls the pen tip towards a desired stroke, while allowing users to draw at their desired speed and without fully taking over control. The algorithm is generally hardware agnostic and works for devices with electromagnetic actuators underneath an interaction surface. This can be implemented via bi-axial linear stage as in our prototype (see \figref{fig:hardware}) or via a matrix of electromagnets which would lend itself better to miniaturization. Furthermore, the algorithm requires a reference trajectory over the optimization horizon. This can be defined a priori, such as a known shape to be traced, or may be provided dynamically, e.g., the output of a predictive model (e.g., Aksan et al. \cite{Aksan:2018:DeepWriting}).

At each time step, we minimize a cost functional over a receding time horizon to find optimized values for system states $\mathbf{x}$ and inputs $\mathbf{u}$.
As a high-level abstraction, the cost function
\begin{equation}
    \underset{\mathbf{x},\mathbf{u}}{\text{minimize}} \sum 
    \underbrace{\mathcal{C}{\text{force}}(\mathbf{x},\mathbf{u})}{\text{Eq. \ref{eq:err_F}, \ref{eq:err_d} \& \ref{eq:erralpha} }} + 
    \underbrace{\mathcal{C}{ \text{path}}(\mathbf{x},\mathbf{u})}_{\text{Eq.  \ref{eq:errLC}}} +
    \underbrace{\mathcal{C}{\text{progress}}(\mathbf{x},\mathbf{u})}_{\text{Eq. \ref{eq:errtheta}}}
      \label{eq:min1},
\end{equation}
serves three main purposes: 1) ensuring that the user perceives haptic feedback of dynamically adjustable force ($\mathcal{C}{\text{force}}$), 2) stays close to the desired path but does not rigidly prescribe it ($\mathcal{C}{\text{path}}$), and 3) makes progress along it ($\mathcal{C}_{\text{progress}}$) but allows the user to vary drawing speed freely.
  
  
 \begin{table}[!t]  \label{tab:control_params}
  \caption{Overview control parameters and values}
  \begin{tabular}{cp{0.25\columnwidth}p{0.5\columnwidth}}
    \toprule
    Name & Range / Value & Description\\
    \midrule
     $\posp$   			& $\mathbb{R}^2$ 						& Position of pen
     \\     
     $\posm$  			& $\mathbb{R}^2$ 						& Position of electromagnet
     \\
     $\mathbf{F_a}$  			& $\mathbb{R}^3$ 						& Electromagnetic force vector
     \\
     $\alpha$ 				&$\left[0,1\right]$	 & Electromagnetic intensity
     \\
     $\mathbf{s}$  		& $\theta \in[0,L]$ 						& Target trajectory of length $L$
     \\
    $\mathbf{x}$ 		& $[\mathbf{p}_{m},\dot{\mathbf{p}}_{m}, \alpha, \theta]$ & System states 
    \\
    $\mathbf{u}$ 		& $[\ddot{\mathbf{p}}_{m}, \dot{\alpha}, \dot{\theta}] $	& System inputs 
    \\
     \bottomrule
\end{tabular}
\end{table}

\section{Method}
Our main contributions are models and a control strategy that enables using the MPCC framework \cite{lam2013model} for electromagnetic haptic guidance.
MPCC is a closed-loop \emph{time-independent} control strategy that minimizes a cost function over a fixed receding horizon. 
There are several advantages in using our formulation over open-loop (as used in dePENd~\cite{yamaoka2013depend}) or time-dependent strategies (\eg MPC). 
%
First, closed-loop control allows to react to user-input, whereas open-loop control removes all user agency. 
Both MPC and MPCC are closed-loop control strategies. 
However, MPC tracks a timed reference, requiring a fixed velocity by users. 
MPCC follows a time-free trajectory, which allows the user to progress at their own speed. 
\figref{fig:control} illustrates the expected behavior for the different strategies, given that the user slows down or stops moving the pen. 
The desired behavior here would be that the algorithm essentially ``waits'', \ie provides guidances towards a slowly or no-longer advancing setpoint.
In this situation, open-loop approaches would lead to lost haptic guidance. 
Closed-loop time-dependent approaches would guide the pen towards a constantly advancing setpoint (although users do no longer move), which can lead to problems such as the user being guided backwards (\eg timestep $t=3$ is in front of $t=2$).

\begin{figure}[!t]
    \centering
    \vspace{-.5em}
    \includegraphics[width=\columnwidth]{\dir/figures//control_strategies-02.pdf}
    \caption{
    Overview of different control strategies on a target trajectory (\textit{green}), with constant pen position.
    For open-loop, the position of the electromagnet is identical to the constantly advancing setpoint, leading to loss of haptic guidance.
    For MPC, although the pen is static, the guidance changes at every timestep since the setpoint advances.
    In our approach, the setpoint is also based on the pen position, therefore remains stationary in this case and guides the user towards the target trajectory.}
    \label{fig:control}
    \vspace{-1em}
\end{figure}

Our method is designed to exert a force $\mathbf{F}_\theta$ of desired strength onto the pen to guide the user towards the target trajectory $\mathbf{s}$. 
The path $\mathbf{s}$ of length $L$ is parametrized by $\theta \in[0,L]$. 
Note that we do not prescribe how fast users draw and hence for each given pen position $\posp$ we first need to establish the closest position on the path parameterized by $\mathbf{s}(\theta)$.
The vector between the pen position and $\mathbf{s}(\theta)$ is defined as $\Rtheta$.
We leverage a receding horizon optimization strategy and the global reference can hence be adjusted or replaced entirely at every iteration. 
The path $\mathbf{s}$ is then a local fit to the global reference.
Furthermore, we seek to find optimized values for the electromagnet intensity $\alpha$ and the in-plane electromagnet position $\mathbf{p}_{m}$. 
Solving the error functional given in Eq. (\ref{eq:J_k}) at each timestep yields optimized values for system states $\mathbf{x}$ and inputs $\mathbf{u}$.   
 
As common in MPC(C), the system is initialized from measurements at $t=0$. 
The system state is then propagated over the horizon with the dynamics model $f(\mathbf{x},\mathbf{u})$. 
The system state vector $\mathbf{x}$ contains variables that are controlled by the algorithm (magnet intensity and position, current path progress). 
The first of the optimized inputs ($u_0$) is then applied to the physical system, transitioning the system state to $x_1$, before iteratively repeating the process to allow for correcting modeling errors. 

% ####################################################
% ####################################################
% ####################################################

\begin{figure}[!t]
% \vspace{-1em}
    \centering
    \includegraphics[width=.8\columnwidth]{\dir/figures//cost-force-02.pdf} 
    \caption{Illustration of actuation force $\mathbf{F_a}$, desired force $\mathbf{F_{\theta}}$, and the force cost-term $\Cost_f$ associate with the difference between those two forces. 
    }
    \label{fig:em_model}
% \vspace{-1em}
\end{figure}

\subsection{Haptics model: controlling the force of the electromagnet} \label{sc:em_costs}
The main goals of our approach is that users can move freely in terms of position and speed, and that the actuator continuously pulls them towards an advancing setpoint $\mathbf{s}(\theta)$ on the target trajectory $\mathbf{s}$.
At any time, the magnet exhibits an actuation force $\mathbf{F_a}$ on the pen,  given by our electromagnetic force model (see \nameref{sec:Implementation} section).
Therein lies the challenge, illustrated in \figref{fig:em_model}. 
The setpoint is continuously advancing based on the movement of the pen to ensure progress.
The actuator needs to pull the pen towards the setpoint by exhibiting force $\mathbf{F_\theta}$, but currently exhibits $\mathbf{F_a}$.
The two forces only align if the pen is exactly at the setpoint, which is rarely the case.
To overcome this challenge, we propose modeling this interaction by a spring-like behavior that ``pulls'' $\mathbf{F_a}$ towards  $\mathbf{F_\theta}$.
In this way, the magnet continuously guides the pen towards the setpoint, and the force linearly increases with distance between the pen and the target setpoint denoted as:
\begin{equation}
     \mathbf{F}_{\theta} (\Rtheta) = c \ F_0 \ \mathbf{r_{\theta}} \ \mathbf{e_{r_{\theta}}} \ . \label{eq:Fd}
\end{equation}
Here $\mathbf{e_{r_{\theta}}}$ is a unit vector in the direction of $\mathbf{r_{\theta}}$, $c$ is a scalar that regulates the stiffness of the spring (in our case $c=5/h$), $F_0$ a scaling of the EM force (\ie the force felt by users) and $h$ the distance between dipoles in $z$ (see Fig. \ref{fig:em_model}). 
Although simple, this formulation ensures that the haptic guidance is strong under large deviation from the path while vanishing as the user approaches the target path ($r_{\theta} \to 0$). 
Note that Eq. \ref{eq:Fd} is a design choice. 
Different formulations can be used to achieve different user experiences. 
Furthermore, replacing our hardware prototype and force-model would allow for adaptation of the remainder of the method to different actuation principles.
%Note that the EM force saturates at $F_a^{max}$.

The above haptics model serves as basis for our problem formulation of electromagnetic guidance in the MPCC framework.
Using the vectors of the current actuation force $\mathbf{F_a}$ and desired force $\mathbf{F_\theta}$, we formulate a quadratic cost term to penalize the difference between desired force and actual force as:
\begin{equation}\label{eq:err_F}
    \Cost_f(\posm, \posp, \alpha) = \norm{ \ \mathbf{F}_{\theta}(\Rtheta) \ - \ \mathbf{F_a}(\mathbf{d}) \ }^2. 
    \end{equation}
%
where $\mathbf{d}$ is the in-plane vector between the magnet and the pen.
Since the actuation force $\mathbf{F_a}$ declines rapidly with distance $\mathbf{d}$, the gradient of $\Cost_f$ goes to 0 for large values of $\mathbf{d}$ causing the optimization to become unstable. 
To counterbalance this issue we encourage the electromagnet to stay close to the pen:
\begin{equation}
    \Cost_d(\posm,\posp) = d^2. \label{eq:err_d}
\end{equation}

Finally, we prioritize proximity between the magnet and the pen rather than increasing its force by penalizing excessive use of magnetic intensity $\alpha$:
\begin{equation}
    \Cost_{\alpha}(\alpha) = \alpha^2. \label{eq:err_alpha}
\end{equation}
%
%
\subsection{Controlling the position of the electromagnet} 
We continuously optimize the position of the electromagnet with the goal of keeping the distance between the desired path and the pen minimal. 
To give the user freedom in deciding their drawing speed we first need to find the reference point $\mathbf{s}(\theta)$ on the target trajectory $\mathbf{s}$. 
Finding the closest point on the path is an optimization problem itself and hence can not be used within our optimization. 
Similar to recent work in robot trajectory generation \cite{Naegeli:2017:MultiDroneCine, Gebhardt:2018}, we decompose the distance to the closest point into a contouring and lag error, as shown in Figure~\ref{fig:elc}. 
%
$\Rtheta$ is the vector between the pen $\posp$ and a point $\mathbf{s}(\theta)$ on the spline, and $\mathbf{n}$ as the normalized tangent vector to the spline at that point, which is defined as $\mathbf{n} = \frac{\partial \mathbf{s} (\theta)}{\partial\theta}$.
%
The vector $\Rtheta$ can now be decomposed into a lag error and a contour error (\figref{fig:elc}). 
The lag-error $\Cost_l$ is computed as the projection of $\Rtheta$.
The contour-error $\Cost_c$ is the component of $\Rtheta$ orthogonal to the normal:
%
\begin{equation}
    \begin{aligned}\label{eq:errL_C} 
\Cost_l (\posp, \theta) &= \norm{\langle\Rtheta,\mathbf{n}\rangle}^2 , \\
\Cost_c (\posp, \theta) &= \norm{ \Rtheta - \left( \langle\Rtheta,\mathbf{n}\rangle \right) \mathbf{n} }^2.
\end{aligned}
\end{equation}
%
Separating lag from contouring error allows us, for example, to differentiate how we penalize a deviation from the path ($\Cost_c$), versus encouraging the user to progress ($\Cost_l$). %This also ensures that the $\posst$ is not influenced to a large extent by the cost function and it is at a close position on the desired path. 
We furthermore include cost terms to ensure that the magnet stays ahead of the pen ($\Cost_{\theta}$) and to encourage smooth progress ($\Cost_{\dot{\theta}}$) computed as
% 
\begin{equation} \label{eq:err_theta}
  \begin{aligned}
\Cost_{\theta}(\theta) &= - \theta ,\\
\Cost_{\dot{\theta}}(\dot{\theta}) &= (\dot{\theta}_t-\dot{\theta}_{t-1})^2 .
\end{aligned}  
\end{equation}

\begin{figure}[!t]
    \centering
    \includegraphics[width=.9\columnwidth]{\dir/figures//cost-lag-contour-02.pdf}
    \caption{Illustration of lag- and contouring error decomposition.}
    \label{fig:elc}
    \vspace{-1em}
\end{figure}

% ################################################
% ################################################
\subsection{Dynamics model}
% Standard mass point model
To phrase electromagnetic haptic guidance in the MPCC framework, we contribute a a dynamics model $f(\mathbf{x},\mathbf{u})$ describing the system dynamics given its states $\mathbf{x}$ and inputs $\mathbf{u}$.  
\begin{equation}
\begin{gathered}
\label{eq:model}
    \dot{\mathbf{x}} = f(\mathbf{x}, \mathbf{u}) ~\text{with}\\
    \mathbf{x} = [\mathbf{p}_{m},\dot{\mathbf{p}}_{m}, \alpha, \theta] \in \mathbb{R}^6
    ~\text{and} ~
    \mathbf{u} = [\ddot{\mathbf{p}}_{m}, \dot{\alpha}, \dot{\theta}] \in \mathbb{R}^4.
\end{gathered}
\end{equation}

The system state $\mathbf{x}$ consists of the position of the electromagnet $\mathbf{p}_{m} \in  \mathbb{R}^2$ and its velocity $\dot{\mathbf{p}}_{m}$, the magnet intensity $\alpha$ and the current path progress $\theta$.
The inputs to the system $\mathbf{u}$ consist of the in-plane electromagnet accelerations $\ddot{\mathbf{p}}_{m}$, and velocities $\dot{\alpha}$ and $\dot{\theta}$ for magnet intensity and the spline progress respectively.
Note that we empirically found that magnet accelerations yield smoother motion than using velocities. 
% 
The system model is given by the non-linear ordinary differential equations using first and second derivatives as inputs:
\begin{equation}
  \ddot{\mathbf{p}}_{m} = v_{m}, \quad \dot{\alpha} = v_{\alpha} \quad \text{and} \quad \dot{\theta} = v_{\theta} ,  
\end{equation}
where $v_{\left(\cdot\right)}$ are the external inputs. 
The continuous dynamics model $\dot{\mathbf{x}} = f(\mathbf{x}, \mathbf{u})$ is discretized using a standard forward Euler approach: $\mathbf{x}_{t+1} = f(\mathbf{x}_t, \mathbf{u}_t)$ \cite{gibbs2011advanced}.

In our hardware implementation, we derive the sets of admissible states $\boldsymbol{\chi}$ and inputs $\boldsymbol{\zeta}$ empirically to conform to the physical hardware constraints of the linear stage (\eg max x,y-position) and EM specifications (\eg max voltage). 
These are used in the constrained optimization problem solved in Eq. \ref{eq:mpcc-formulation}.
%
The pen position is propagated via a standard linear Kalman filter \cite{gibbs2011advanced}. 
While not an accurate user model, it works well in practice since the states are recalculated at every timestep. 


% ################################################
% ################################################

\begin{table}[!t]
	\begin{tabular}{clc}
    \toprule
    Term & Description of cost & Eq.\\
    \midrule
     $\Cost_f$  		& Decreases difference in magnetic force		& \ref{eq:err_F}
     \\
    $\Cost_d$ 		& Decreases distance between magnet and pen		& \ref{eq:err_d}
    \\
    $\Cost_{\alpha}$ 		& Encourages close distance over large force		& \ref{eq:err_alpha}
    \\
     $\Cost_l$   				& Decreases lag to path contour		& \ref{eq:errL_C} 
     \\     
     $\Cost_c$  			& Decreases distance to path contour		& \ref{eq:errL_C} 
     \\
     $\Cost_{\theta}$  				& Magnet stays ahead of pen		& \ref{eq:err_theta}
     \\
     $\Cost_{\dot{\theta}}$ 		& Ensures smooth progress		& \ref{eq:err_theta}
    \\
     \bottomrule
\end{tabular}
\caption{Summary of costs terms used in optimization.}
\label{tab:costs}
\end{table}

\subsection{Optimization}
We combine the cost terms (Table \ref{tab:costs}) to control the force and position of the actuator to form the final stage cost:
%
\begin{align}\label{eq:J_k}
J_k= \quad 
     & w_f \Cost_f(\mathbf{p}_{m,k}, \mathbf{p}_{p,k}, \alpha_k, \theta_k) + \nonumber \\
     & w_d \Cost_d(\mathbf{p}_{m,k}, \mathbf{p}_{p,k}) + w_\alpha \Cost_{\alpha}(\alpha_k)+ \nonumber \\
	 &	 w_l \Cost_l(\mathbf{p}_{p,k}, \theta_k) +  w_c \Cost_c(\mathbf{p}_{p,k}, \theta_k) + \nonumber \\
     & w_{\theta} \Cost_{\theta}(\theta_k) +  w_{\dot{\theta}} \Cost_{\dot{\theta}}(\dot{\theta}_k),
\end{align}
%
where the scalar weights $w_l,w_c,w_{\theta},w_{\dot{\theta}},w_f,w_d, w_{\alpha}>0$ control the influence of the different cost terms. 
The values used in our experiments and applications can be found in the \nameref{sec:Implementation} section.
The system states and inputs are computed by solving the $N$-step finite horizon constrained non-linear optimization problem at time instance $t$. 

The final objective therefore is:
\begin{align}
\label{eq:mpcc-formulation}
\underset{\mathbf{x}, \mathbf{u}, \theta}{\text{minimize}}\quad & \sum_{k=0}^{N} w_k\left ( J_k + \mathbf{u}_k^T \mathbf{R} \mathbf{u_k} \right ) && \\
\text{Subject to:}\quad & \mathbf{x} _{k+1} = f(\mathbf{x_k}, \mathbf{u_k}) & \text{(System Model)} \nonumber\\
                        & \mathbf{x}_0 = \hat{\mathbf{x}}(t) & \text{(Initial State)} \nonumber \\
                        & \theta_0 = \hat{\theta}(t) & \text{(Initial Progress)} \nonumber \\
                        & \theta_{k+1} = \theta_k + \dot{\theta}_k dt & \text{(Progress along path)} \nonumber \\
                        & 0 \leq \theta_k \leq L& \text{(Path Length)} \nonumber \\
                        & \mathbf{x}_k \in \boldsymbol{\chi} & \text{(State Constraints)} \nonumber \\
                        & \mathbf{u}_k \in \boldsymbol{\zeta} & \text{(Input Constraints)} \nonumber
\end{align}

Here $k$ indicates the horizon stage and the additional weight $w_k$ reduces over the horizon, so that the current timestep has more importance than later timesteps. 
$\mathbf{R}\in\mathbb{S}_+^{n_u}$ is a positive definite penalty matrix avoiding excessive use of the control inputs. 
In our implementation we use a horizon length of $N=10$. 
Experimentally we found that this is sufficient to yield robust solutions to problem instances and longer horizons did not improve results, yet linearly increases computation time. 

% \begin{algorithm}[ht!]
\caption{\textit{NovelSelect}}
\label{alg:novelselect}
\begin{algorithmic}[1]
\State \textbf{Input:} Data pool $\mathcal{X}^{all}$, data budget $n$
\State Initialize an empty dataset, $\mathcal{X} \gets \emptyset$
\While{$|\mathcal{X}| < n$}
    \State $x^{new} \gets \arg\max_{x \in \mathcal{X}^{all}} v(x)$
    \State $\mathcal{X} \gets \mathcal{X} \cup \{x^{new}\}$
    \State $\mathcal{X}^{all} \gets \mathcal{X}^{all} \setminus \{x^{new}\}$
\EndWhile
\State \textbf{return} $\mathcal{X}$
\end{algorithmic}
\end{algorithm}

\section{Decomposition Box}





% Feature 1: Inferring Users' Thought Process from Existing Code
% Feature 2: Independent Problem Decomposition through Step Trees and Natural Language.
% Feature 3: Step Tree Node Status Evaluation with Preservation of Original Structure
% Feature 4: Progressive Hint for Idea Formation
% Feature 5: Converting the Step Tree into Comments
% Feature 6: Validating Code Implementation against the Step Tree
% Feature 7: Progressive Hint for Idea Implementation


% Challenge 1: Excessive Help Hindering Active Learning
% Challenge 2: Difficulty in Utilizing ChatGPT for Effective Learning
% Challenge 3: Misalignment Between Provided Solutions and Learners’ Approaches
% Challenge 4: Disconnect Between Existing Code and Provided Solutions.
% Challenge 5: Lack of a Structured Problem-Solving Approach
% Challenge 6: Insufficient Fine-Grained Feedback on Learner Progress


% Goal 1: Scaffolding for Active Learning and Independent Thinking
% Goal 2: Personalization to Individual Problem-Solving Styles
% Goal 3: Connection and Structured Solution Presentation
% Goal 4: Fine-Grained Evaluation and Feedback

% \renewcommand{\arraystretch}{1.2}
% \begin{table*}[tp]  

% \centering  
% \fontsize{8}{8}\selectfont  

% \caption{A mapping between the identified challenges from our formative, the design goals, and the features of Decomposition Box}\label{table:mapping}
% \label{intention_themes}
% \begin{tabular}{m{5cm}<{\centering}m{4.5cm}<{\centering}m{4.5cm}<{\centering}}
% \toprule
% \textbf{Challenge}&\textbf{Design Goal}&\textbf{DBox Feature}\\
% \midrule

% Challenge 1: Excessive Help Hindering Active Learning & \multirow{3}*{\shortstack{Goal 1: Scaffolding for Active \\Learning and Independent Thinking}}& \multirow{3}*{\shortstack{Feature 1\&2 (Stage 1), \\Feature 4\&7 (Stage 1\&2)}}\\

% Challenge 2: Difficulty in Utilizing ChatGPT for Effective Learning &  & \\


% \midrule
% Challenge 3: Misalignment Between Provided Solutions and Learners’ Approaches & Goal 2: Personalization to Individual Problem-Solving Styles & Feature 1 (Stage 1), Feature 3 (Stage 1)\\


% \midrule
% Challenge 4: Disconnect Between Existing Code and Provided Solutions & \multirow{3}*{\shortstack{Goal 3: Connection and Structured\\ Solution Presentation}} & \multirow{3}*{\shortstack{Feature 1 (Stage 1), Feature 2 (Stage 1),\\ Feature 5 (Stage 2), Feature 6 (Stage 2)}}\\

% Challenge 5: Lack of a Structured Problem-Solving Approach &  & \\

% \midrule
% Challenge 6: Insufficient Fine-Grained Feedback on Learner Progress & Goal 4: Fine-Grained Evaluation and Feedback & Feature 3 (Stage 1), Feature 6 (Stage 2)\\

% \bottomrule
% \end{tabular}
% \end{table*}



% Based on our design goals, we developed Decomposition Box (DBox) accordingly. 
% % Table \ref{table:mapping} shows the mapping between challenges identified in our formative study, the design goals, and DBox’s features. 
% Figure \ref{fig:interface} shows the interface.

\subsection{Overview} \label{overview}




\begin{figure*}[htbp]
	\centering 
	\includegraphics[width=\linewidth]{figures/UI_new.pdf}
	\caption{The interface of Decomposition Box. The top row displays the full interface in the solution formation stage (the solution implementation stage is similar, but with different status indicators). The middle row demonstrates a learner's solution formation stage showing basic DBox features. The bottom row illustrates a learner's solution implementation stage. An overview of the DBox interface and workflow is provided in Sec. \ref{overview}, and an illustrative example is described in Sec. \ref{illustrative}. To save space, the second row omits the problem description and editor area, and the third row excludes the problem description area.}
	\label{fig:interface}
        \Description{}
\end{figure*}

As shown in Figure \ref{fig:interface} (top row), DBox's interface has three main parts. The Problem Description (Figure \ref{fig:interface}.A)) and Solution Code Editor (Figure \ref{fig:interface}.B) are similar to the LeetCode platform.
The Interactive Step Tree Widget (Figure \ref{fig:interface}.D) enables users to refine their thought process and receive feedback via an interactive step tree. 
Three buttons—``From Editor to Step Tree'', ``Check Match'', and ``Copy to Comments''—connect the editor and step tree. 
The ``Check Step Tree'' button provides feedback on the step tree's status, categorized into five types (Figure \ref{fig:interface}.D). 
Clicking ``Hint'' offers progressive guidance based on learners' existing attempts (Figure \ref{fig:interface}.E). 
Hovering over steps shows buttons for editing the step tree (Figure \ref{fig:interface}.F).
% \begin{itemize}
%     \item Problem Description (Figure \ref{fig:interface} A): Displays the current problem, with a "Problem List" button , similar to traditional platforms.
%     \item Code Editor (Figure \ref{fig:interface} B): Allows users to write and run code. Output and test case results are shown in Figure \ref{fig:interface} C.
%     \item Interactive Step Tree Widget (Figure \ref{fig:interface} D): Users refine their thought process and receive feedback via an interactive step tree. Three buttons—“From Editor to Step Tree”, “Check Match”, and “Copy to Comments”—connect the editor and step tree. The “Check Step Tree” button provides feedback on the step tree's status, categorized into five types (Figure \ref{fig:interface} D). Clicking ``Hint'' offers progressive guidance based on learners' existing attempts (Figure \ref{fig:interface} E). Hovering over steps shows buttons for editing the step tree (Figure \ref{fig:interface} F).
% \end{itemize}
Figure \ref{fig:workflow} shows two key stages in DBox's workflow:
\begin{itemize}
    \item \textbf{Solution Formation}: 
    The step tree starts as an empty box where students can freely add steps in either coding mode (directly writing code) or description mode (building a step tree using natural language). 
They can evaluate their progress with the ``From Editor to Step Tree'' or ``Check Step Tree'' buttons. The tree contains steps and substeps labeled as \emph{Correct}, \emph{Incorrect}, \emph{System Generated}, \emph{Missing}, or \emph{Can be Divided}. 
Layouts adjust dynamically based on the hierarchy. 
Steps that can be further divided are marked with dashed outlines, serving as a reminder, though students can decide whether further division is necessary. 
    \item \textbf{Solution Implementation}: Students can convert the step tree into comments with “Copy to Comments” or verify alignment by clicking “Check Match”. Nodes in the step tree are labeled as Implemented, Incorrectly Implemented, or To Be Coded. In this stage, DBox also offers progressive hints. When all nodes are implemented, students can test their solution by clicking “Run” button against the provided test cases.
\end{itemize}






\begin{figure*}[htbp]
	\centering 
	\includegraphics[width=\linewidth]{figures/pipeline.pdf}
	\caption{The DBox workflow supports learners through solution formation and implementation stages. During solution formation, (A) students can input ideas by either coding or using natural language to build a step tree. (B) By clicking ``From Editor to Step Tree'' or ``Check Step Tree'', (C) DBox renders the step tree and identifies node statuses (e.g., correct, incorrect, missing). Students can iteratively refine their code or step tree, receiving progressive hints, (D) until the step tree is fully correct. In the solution implementation stage, (E) students can convert the step tree into code comments or (F) check the alignment between their code and the step tree. Each node displays one of three statuses, and students can refine their work with ongoing hints until (G) all nodes are marked as ``implemented''.
 Finally, students can test if their code passes all test cases.}
	\label{fig:workflow}
        \Description{}
\end{figure*}



% Based on our design goals, we developed Decomposition Box (DBox). Table \ref{table:mapping} shows a mapping between the identified challenges from our formative study, the design goals, and the features of DBox. Figure \ref{fig:interface} shows the interface. Below, we provide an overview of the interface, followed by a detailed explanation of the key features and backend design.

% \subsection{Overview}

% As shown in Figure \ref{fig:interface} (top row), the interface of DBox is strategically divided into three distinct sections, from left to right, designed to enhance the user experience in algorithm programming:

% \begin{itemize}
%     \item Left Section - Problem Description (Figure \ref{fig:interface} (A)): This area allows users to view the problem they are currently working on. By clicking the ``Problem List'' button, users can easily switch between different problems they wish to practice. This setup is similar to traditional programming exercise platforms, providing a familiar layout for users.
%     \item Middle Section - Code Editor (Figure \ref{fig:interface} (B)): Central to the user interface, this section features an integrated CodeMirror editor where users can write and edit code in their chosen programming language. The ``Run'' button allows users to execute their code, with the output and any error messages (such as failed test cases or syntax errors) displayed (Figure \ref{fig:interface} C). If the code passes all test cases, it is marked as accepted, providing immediate feedback on the correctness of the solution.
%     \item Right Section - Decomposition Box (Figure \ref{fig:interface} (D)): This core area is where users can refine their thought processes and receive specific feedback and reminders related to their coding approach. Positioned between the code editor and this section are three interactive buttons that facilitate seamless communication between the editor and the Decomposition Box, enhancing the integration of feedback into the coding process.
% \end{itemize}

% As shown in Figure \ref{fig:workflow}, DBox is designed to support two key stages in students' algorithmic programming practice: \textbf{Idea Formation} and \textbf{Idea Implementation}. In the Idea Formation stage, students can input their thoughts in two modes (Figure \ref{fig:workflow} (A)): one is the coding mode, where they directly write code, and the other is the description mode, where they build a step tree interactively using natural language to describe each node. Based on the selected input mode, students can click different buttons to check the status of their step tree (Figure \ref{fig:workflow} (B)). The ``From Editor to Step Tree'' button converts incomplete code into a step tree, while the ``Check Step Tree'' button evaluates the status of each node in the tree. The step tree can include steps, sub-steps, and even sub-sub-steps, with DBox identifying the status of each node, which could be one of five types: Correct, Incorrect, AI Generated, Missing, or Divisible (Figure \ref{fig:workflow} (C)). Students can refine their code or step tree and click the corresponding buttons to check the updated status of the tree. Throughout this process, DBox provides progressive hints, divided into three levels. After several refinements, the step tree becomes fully correct (Figure \ref{fig:workflow} (D)), allowing students to proceed to the Idea Implementation stage.

% At this stage, students can use the ``Check Match'' button to verify whether the code in the editor aligns with the nodes in the step tree, or they can click the ``Copy to Comments'' button to convert the step tree into comments in the editor to assist with coding (Figure \ref{fig:workflow} (E)). After clicking ``Check Match,'' the nodes in the step tree will be labeled with one of three statuses: Implemented, Incorrectly Implemented, or To Be Coded (Figure \ref{fig:workflow} (F)). DBox continues to offer progressive hints during this phase as well. After modifying the code based on feedback, the step tree will eventually display all nodes as implemented (Figure \ref{fig:workflow} (G)). At this point, the student can click the ``Run'' button to validate the solution against the test cases.





% The interface is crafted to be easily integrated as a plugin into existing programming tools and platforms, such as online environments like LeetCode or offline editors like VSCode, offering flexibility and adaptability across different learning and development settings.






% \subsection{Overview of DBox Interface}
% A learner named Alice wants to practice her algorithmic programming using a problem ``Search in Rotated Sorted Array". After reading the problem description, Alice comes up with some initial thoughts in her mind. DBox provides two input modes, one is through directly writing codes, the other is building a step tree via natural languege description at the right-side of the interface. (1) Alice chooses to add two steps, Step 1 and Step 2, to construct an initial step tree. (2) After clicking the ``Check Step Tree'' button, DBox identifies Step 1 as correct and Step 2 as incorrect, also indicating a missing Step 3. Alice then clicks the hint button on Step 2 to access general and detailed hints. (3) If Alice makes two consecutive errors at Step 2, a new hint triggers: ``reveal step.'' (4) Clicking this button reveals a crucial sub-step (Step2-3) and leaves Step 2-1 and Step 2-2 for Alice to complete. (5) Once Alice completes these, she clicks ``Check Step Tree'' again and finds all of Step 2 correct. (6) Alice also correctly constructs Step 3. \textbf{The third row displays Alice's process of implementing the solution}. (8) First, Alice clicks the ``Copy to Comments'' button, and DBox converts the step tree into code comments, inserting them at the corresponding positions in the editor. (9) After writing some code, Alice uses the ``Check Match'' button to identify steps that are not correctly implemented, noting that Step 2-3 and Step 3 are incomplete. (10) Guided by DBox, Alice writes the corresponding code. Upon clicking ``Check Match'' again, all steps turn green to indicate they are implemented correctly. When hovering over a step, the corresponding code line is highlighted. Finally, Alice hits the Run button, passes all test cases, and successfully solves the problem. It's important to note that this figure only shows just one of many possible interactions. To save space, the second row of images displays only the step tree on the right side of the interface, while the third row shows both the middle editor and the step tree.






% \subsection{Overview of DBox Interface}
% Alice, a learner, is practicing algorithmic programming with the problem ``Search in Rotated Sorted Array.'' After reading the problem description, she formulates some initial ideas in her mind. DBox offers two input modes: directly writing code or constructing a step tree using natural language descriptions on the right side of the interface. (1) Alice opts to build a step tree and adds two initial steps, Step 1 and Step 2. (2) Upon clicking the ``Check Step Tree'' button, DBox identifies Step 1 as correct, flags Step 2 as incorrect, and highlights a missing Step 3. Alice clicks the hint button on Step 2 to access both general and detailed hints. (3) If she makes two consecutive errors in Step 2, DBox triggers an additional hint: ``reveal step.'' (4) Clicking this reveals a crucial sub-step (Step 2-3 in this case), while leaving Step 2-1 and Step 2-2 for Alice to complete. (5) After completing these sub-steps, (6) Alice checks the step tree again and finds all of Step 2 marked as correct. (7) She then constructs Step 3 successfully. Now, all nodes in the step tree are correct. (8) Alice then clicks the ``Copy to Comments'' button, and DBox converts the step tree into code comments, automatically inserting them into the editor. (9) After writing some code, she uses the ``Check Match'' button, which highlights steps that are not properly implemented, indicating that Step 2-3 and Step 3 are incomplete. (10) Following DBox's guidance, Alice writes the necessary code, and upon rechecking, all steps turn green, indicating correct implementation. Note that when Alice hovers over a step, the corresponding line of code is highlighted. Finally, she clicks the Run button, passes all test cases, and successfully solves the problem.



\subsection{Target Users and A System Walkthrough} \label{illustrative}
\ms{DBox is designed for learners who understand basic algorithm concepts but struggle to apply them to solve practical problems. Using a scaffolding approach, DBox emphasizes independent thinking by offering only essential support. It assumes students are motivated, self-regulated, and actively engaging with the tool to improve their decomposition skills. If a student is less motivated or prefers a quicker solution, they may bypass DBox to search for answers online.}
Next, we present an example walkthrough (Figure \ref{fig:interface}) of such a self-regulated student Alice: 
% usage example from our pilot study, as shown in Figure \ref{fig:interface}. While this illustrates one specific interaction, users can follow various workflows based on their preferences. Below is a detailed walkthrough of the scenario.

Alice, a learner tackling the ``Search in Rotated Sorted Array'' problem, begins by organizing her thoughts in the solution formation stage. DBox offers two options: she can either start coding or build a step tree using natural language. She opts for the latter and adds two initial steps (Figure \ref{fig:interface}.1). To check her progress, Alice clicks ``Check Step Tree'' button. DBox flags Step 1 as correct, Step 2 as incorrect, and highlights a missing Step 3 (Figure \ref{fig:interface}.2). She clicks the hint button on Step 2, receiving general and detailed guidance, but after another failed attempt, DBox offers another option for revealing a substep (Figure \ref{fig:interface}.3). Alice clicks ``Reveal (Sub)Step'', uncovering a sub-step 2-3 while leaving sub-steps 2-1 and 2-2 for her to solve (Figure \ref{fig:interface}.4). Inspired by the hints, Alice figures out how to break down and fills in these sub-steps (Figure \ref{fig:interface}.5). After checking again, Step 2 is marked correct (Figure \ref{fig:interface}.6). Alice adds the missing Step 3 (Figure \ref{fig:interface}.7), and finally, after checking, all steps turn to correct (Figure \ref{fig:interface}.8).

Next, Alice moves to the solution implementation stage. She clicks ``Copy to Comments'', and DBox converts her step tree into code comments (Figure \ref{fig:interface}.9). As Alice writes her code, she uses the ``Check Match'' button to identify incorrectly implemented and unimplemented steps. Step 2 is identified as unimplemented and Step 3 is identified as incorrectly implemented (Figure \ref{fig:interface}.10). Following DBox's guidance, she revises the code, and after another check, all steps turn to be correctly implemented (Figure \ref{fig:interface}.11). Satisfied with her progress, Alice clicks ``Run'' and successfully passes all test cases, solving the problem.

Note that we have presented only a simple walkthrough here, whereas the steps in a student's actual problem-solving process are more complex and dynamic (as shown later in Sec. \ref{actual_use}). Next, we introduce the specific features aligned with the four design goals as described in Sec. \ref{designgoal}.
% \textbf{D1}: Scaffolding for Active Learning and Independent Thinking; \textbf{D2}: Personalization to Individual Problem-Solving Styles; \textbf{D3}: Connection and Structured Solution Presentation; and \textbf{D4}: Fine-Grained Evaluation and Feedback. Each feature is tailored to different stages of the user's algorithmic programming journey.




% \subsection{An Illustrative Example} \label{illustrative}
% We now present a usage example observed during our user study, as shown in Figure \ref{fig:workflow}. While this illustrates one specific interaction, users can follow various workflows depending on their preferences. To conserve space, the second row in Figure \ref{fig:workflow} omits the problem description and editor, while the third row excludes the problem description. Below, we provide a detailed walkthrough of the scenario depicted.

% Alice, a algorithmic programming learner, is tackling the ``Search in Rotated Sorted Array'' problem. After reading the description, she starts to map out her approach. DBox offers two ways to proceed: Alice can either dive straight into coding or take a more structured route by building a step tree through natural language descriptions. She chooses the latter, organizing her thoughts by adding two initial steps—Step 1 and Step 2—to the step tree (Figure \ref{fig:workflow} (1)). Eager to check her progress, Alice clicks the ``Check Step Tree'' button. DBox instantly provides feedback: Step 1 is correct, but Step 2 is flagged as incorrect, with a missing Step 3 also highlighted (Figure \ref{fig:workflow} (2)). Alice clicks on the hint button for Step 2, receiving both general and detailed guidance. But she still experiences one more failed attempt on Step 2. At this time, DBox suggests a new hint strategy: revealing part of the solution (Figure \ref{fig:workflow} (3)). She clicks ``reveal step,'' uncovering a crucial sub-step (Step 2-3), while leaving her to figure out sub-steps 2-1 and 2-2 on her own (Figure \ref{fig:workflow} (4)). Alice continues working through the sub-steps (Figure \ref{fig:workflow} (5)). Finally, Step 2 is fully correct (Figure \ref{fig:workflow} (6)), and she successfully adds the missing Step 3. Now, the entire step tree is complete (Figure \ref{fig:workflow} (7)).

% With the structure in place, Alice moves on to the coding phase. She clicks the ``Copy to Comments'' button, and DBox seamlessly transforms her step tree into code comments, which are automatically inserted into the editor (Figure \ref{fig:workflow} (8)). She starts writing her code, and as she works, the ``Check Match'' button becomes her guide, highlighting which steps are incorrectly implemented. It's clear that Step 2-3 and Step 3 need further attention ("to be coded") (Figure \ref{fig:workflow} (9)). Following DBox's prompts, Alice revises the code, and after another check, all steps turn green, signaling success. As she hovers over each step in the tree, the corresponding line of code is highlighted, helping her stay aligned. Satisfied with her progress, Alice hits the Run button. Her code passes all test cases, and she successfully solves the problem (Figure \ref{fig:workflow} (10)). 



% Next, we will explore the specific features tailored to each stage of the user's journey in algorithmic programming.





% \subsection{Stage 1: Idea Formation}

% \subsubsection{Feature 1: Inferring Users' Thought Process from Existing Code.}
% When users click the “From Editor to Step Tree” button, DBox analyzes the current problem and the user’s incomplete code to infer their thought process, which is then displayed as a step tree on the right side. It’s important to note that this feature uses the user’s existing code as the primary input and does not take into account any pre-existing step tree, which will be overwritten. Once the step tree is generated, hovering over any step node highlights the corresponding code lines in the editor, helping users easily connect the step tree with their existing code. This feature is particularly useful when users have written some code and are stuck or when they want to check for errors in their existing code.

\subsection{Stage 1: Solution Formation}
\subsubsection{Two Input Modes (D1, D2, D3)}
DBox offers users the flexibility to develop their solutions through two distinct input modes: by writing code directly or by constructing a step tree using natural language descriptions, without needing to start with code. In the latter mode, users begin with a blank step tree and can click ``Add'' to insert nodes or ``Split'' to create sub-steps for more granular detail. Each node contains a text input field where users can articulate their thought process. Steps and sub-steps can be rearranged or deleted, allowing learners to iteratively and interactively refine and structure their mental model.

\subsubsection{Inferring Users' Thought Process from Existing Code (D1, D3)}
The ``From Editor to Step Tree'' function in DBox infers a learner’s intended solution and thought process based on their incomplete code. When activated, the system analyzes the code and problem, presenting the inferred steps as a tree on the right-hand side of the interface. Hovering over each node highlights the corresponding lines in the code editor, linking the inferred steps directly to the code. This feature assists users in diagnosing errors and identifying potential issues, especially when they are unsure how to proceed.


% \subsubsection{Feature 1: Inferring Users' Thought Process from Existing Code} The ``From Editor to Step Tree'' button in DBox leverages the user’s existing, incomplete code to infer their thought process. When activated, the system analyzes the problem and code, displaying the inferred steps as a tree on the right side of the interface. It's important to note that this feature prioritizes the current code over any pre-existing step tree, which will be overwritten. As the step tree populates, hovering over any node will highlight the corresponding lines in the code editor, linking conceptual steps directly to the code. This feature is invaluable for users who are stuck or wish to identify errors in their existing code, enhancing their ability to diagnose and resolve coding issues efficiently.



% \subsubsection{Feature 2: Decomposing A Problem via Step Tree via Natural Language Description.}
% In addition to generating a step tree from the user’s existing code, we allow users to develop their thought process directly through natural language descriptions, without needing to write code initially. We use an interactive visual step tree to help users organize their problem-solving ideas. Initially, the step tree area is blank. Users can click the Add button to create a node representing a step. They can add as many steps as needed, dividing the space of the step tree area. For each step node, users can click the Split button to add sub-steps. Each step and sub-step includes a blank text input area where users can describe their thought process in simple natural language. Users can also delete or rearrange any step or sub-step as they wish. Such a step tree is useful for helping learners build a structured mental model of the problem-solving strategy.


% \subsubsection{Feature 2: Independent Problem Decomposition through Step Trees and Natural Language}

% Our tool enhances problem-solving by allowing users to independently construct a step tree using natural language descriptions, without initially requiring code. The step tree area starts blank, and users can click the ``Add'' button to insert nodes representing individual steps, organizing their thought process visually. For more detailed breakdowns, the ``Split'' button enables users to add sub-steps under each main step. Each node in the step tree features a blank text input area where users can articulate their thought process in straightforward natural language. Users have the flexibility to delete or rearrange steps and sub-steps as needed, facilitating the development of a structured mental model for tackling complex problems. This method supports learners in systematically building and refining their problem-solving strategies.


% \subsubsection{Feature 2: Independent Problem Decomposition through Step Trees and Natural Language}

% DBox allows users to independently construct a step tree using natural language descriptions, without requiring initial code. Starting with a blank step tree, users can click ``Add'' to insert nodes representing steps, and ``Split'' to add sub-steps for more detailed breakdowns. Each node includes a text input area for users to articulate their thought process. Steps and sub-steps can be deleted or rearranged, helping users build a structured mental model. This feature supports learners in interactively developing and refining their problem-solving strategies.





% \subsubsection{Feature 2: Step Tree Node Status Evaluation.}
% Our tool provides a fine-grained evaluation of the user's thought process. For each step or sub-step, the system categorizes it into one of the following states: (1) Correct, indicating that the step is appropriate for the current problem-solving approach; (2) Incorrect, indicating an error in the thought process or specific details; (3) Missing, indicating a step that is necessary for the complete solution but is absent from the user’s incomplete code; (4) Can Be Further Divided, indicating that the step is complex and can be broken down into sub-steps; and (5) AI Suggested, where the system offers a description of the step in a blue box if the user triggers the most detailed hint level. Users have complete freedom to decide whether to further divide steps, ensuring flexibility in their problem-solving approach. 

% We’ve made a special design choice here: except when the system identifies missing steps and adds a blank missing node, the structure and content of the step tree remain as constructed by the user. The system does not override the user’s current step tree with what GPT considers to be the correct steps and descriptions. Instead, we provide feedback and guidance on the status of each node in the step tree, ensuring that users can continue to advance along their problem-solving approach.

% \subsubsection{Feature 3: Step Tree Node Status Evaluation with Preservation of Original Structure} Once the learner clicks ``From Editor to Step Tree'' or ``Check Step Tree'' button, DBox will conduct a detailed evaluation of each node in the user's step tree, categorizing each step or sub-step into one of five states: (1) \textbf{Correct}: The step is suitable for the current problem-solving approach. (2) \textbf{Incorrect}: There is an error in the thought process or specific details. (3) \textbf{Missing}: A necessary step is absent from the user's code. (4) \textbf{Can Be Further Divided}: The step is complex and could be broken into smaller, more manageable sub-steps. (5) \textbf{System Generated}: The system provides a step description in a blue box if the user activates the most detailed hint level. Users retain complete control over whether to sub-divide steps, allowing them to tailor their problem-solving approach flexibly. Importantly, except for adding a blank node when steps are identified as missing, the system preserves the user’s original step tree without replacing it with AI-determined correct steps. Feedback and guidance are provided on the status of each node, supporting users as they refine and advance their problem-solving strategies.


\subsubsection{Step Tree Node Status Evaluation with Preservation of Original Structure (D2, D4)}
When the learner clicks ``From Editor to Step Tree'' or ``Check Step Tree'', DBox evaluates each node, assigning one of five statuses:
(1) \textbf{Correct}: The step aligns with the learner's intended approach.
(2) \textbf{Incorrect}: Errors are identified in the step.
(3) \textbf{Missing}: A required step is absent.
(4) \textbf{Can Be Divided}: The step is complex and can be broken into sub-steps, indicated by dashed borders. Users decide whether to subdivide. This status can coexist with other statuses.
(5) \textbf{System Generated}: Step content is created by the system. This status is triggered only when the learner requests to reveal a (sub)step after repeated failures.
During the ``Check Step Tree'' process, DBox preserves the original step tree (both structure and contents), only adding blank nodes for missing steps, ensuring scaffolding while respecting the learner's thought process.





% \subsubsection{Feature 4: Progressive Hint.}
% We’ve designed a detailed scaffolding process that provides only the necessary guidance, encouraging users to think independently before offering more specific hints as needed. The first level is a hint presented as a question. When users know a step is incorrect or missing but still lack ideas, they can click the C button to receive initial guidance. This guidance does not directly reveal the answer but steers users in the right direction, such as: “Before you convert the string variable to an array, what should you do first?” The second level offers more specific guidance, including more concrete clues, if the user still struggles after the first hint. The third level of guidance is triggered if a user repeatedly fails to correct their thought process for a specific step. This level provides an option to view the AI-suggested correct steps. Although this third level is triggered, it is not displayed by default; users must click the View button to see it. The third-level feedback can appear in two scenarios: (1) If the step has no sub-steps, the tool directly presents the correct description of the step; (2) If the step includes sub-steps, the tool highlights one key sub-step and marks other sub-steps as missing, reminding the user to complete the remaining sub-steps. It is important to note that users can choose not to view any of these three levels of feedback to solve the problem independently.


% We also provide multi-level guidance for users who incorrectly implement or have not yet implemented their thought process, aiming to help them complete their code independently as much as possible. The first level is a simple hint that offers basic guidance, such as: “How should you correctly update variable A?” or “Consider updating the index before the loop ends.” The second level is a pseudocode hint, offering a more specific guide if the user is still unsure after viewing the basic hint. At this point, the user only needs to convert the pseudocode into actual code. The third level of guidance is triggered if the user incorrectly implements a step twice in a row. This level provides an option to view the correct code for that step. Although the third level is triggered, it is not displayed by default; users must click the View button to see it. It’s important to note that users can choose not to view any of these three levels of feedback to independently implement their ideas.

% \subsubsection{Feature 4: Progressive Hint for Idea Formation}

% We have implemented a scaffolding feature that progressively delivers guidance, fostering independent problem-solving while providing specific hints as needed. This feature is structured into three levels. (1) \textbf{Initial Hint (Question-Based)}: This is the first level of assistance where users receive a hint framed as a question, prompting them to think about the next step without giving away the solution. For instance, if a user is uncertain about the initial steps, they might see a hint like, “Before you convert the string variable to an array, what should you do first?” (2) \textbf{Detailed Guidance}: If the user continues to struggle, a second, more specific hint is provided. This could include more direct clues that guide the user closer to the solution but still require them to apply their reasoning. (3) \textbf{Recommendation for A (Sub)Step}: Triggered by repeated difficulties in correcting a specific step, this level offers the option to view the AI-suggested correct steps. For steps without sub-steps, it directly presents the correct description. For steps that include sub-steps, it highlights one key sub-step and marks the remaining as missing, prompting users to complete them. This level is optional and can be accessed by clicking a ``Reveal Step'' button, allowing users to decide whether to see the full solution or continue working independently. These layers of hints are designed to support users in building their problem-solving skills gradually while enabling them to maintain control over their learning process.

\subsubsection{Progressive Hints for Solution Formation (D1)}

DBox provides progressive hints to scaffold learners' problem-solving in three levels: (1) \textbf{General Hint} (Question-Based): Prompts learners' critical thinking without revealing solutions, e.g., ``Before converting the string to an array, what should you do first?'' (2) \textbf{Detailed Hint}: Offers more specific clues while requiring reasoning, e.g., ``Think about how you can traverse each character in the string.'' (3) \textbf{Reveal (Sub)Step} ((Sub)Step Recommendation): For repeated errors, the AI can suggest a substep within a larger step when users click the ``Reveal (Sub)Step'' button. This reveals one key substep while leaving the remaining steps for the learner to complete. Notably, students can choose not to trigger this hint. These progressive hints support problem-solving development while allowing learners to maintain independence and control.

Once the step tree is complete and all nodes are correct, learners proceed to the solution implementation stage.


% \subsection{Stage 2: Idea Implementation} After constructing a complete and correct step tree, users advance to the idea implementation stage.

% \subsubsection{Feature 5: Converting the Step Tree into Comments} This feature transforms each node of the step tree into code comments. By clicking ``Copy to Comments", these comments are pasted directly into the corresponding section of the editor, guiding students to systematically complete their code implementation.

% \subsubsection{Feature 6: Validating Code Implementation against the Step Tree} The tool evaluates the alignment between the user’s code and the problem-solving approach defined in the step tree. Using the ``Check Match'' button, the step tree's status updates to reflect: (1) \textbf{Implemented}: The code accurately implements the described step. (2) \textbf{Incorrectly Implemented}: There is an error in how the step is coded.
% (3) \textbf{To be Coded}: The step has yet to be coded. Highlighting in the editor indicates the correlation of code lines with the step tree, showing whether each step is correctly or incorrectly matched.

% \subsubsection{Feature 7: Progressive Hint for Idea Implementation.} Additionally, for users who have incorrectly implemented or have yet to implement their thought process, multi-level guidance is available. (1) \textbf{Basic Hint}: Offers straightforward suggestions to nudge the user in the right direction, like “How should you correctly update variable A?” or “Consider updating the index before the loop ends.” (2) \textbf{Pseudocode Hint}: If the user remains unsure after the basic hint, this level provides a pseudocode hint, clarifying what needs to be done, which the user then translates into actual code. (3) \textbf{Recommended Implementation}: This is activated if a user misimplements a step twice consecutively. It allows them to view the recommended code for that step by clicking a ``View'' button, although, like the previous levels, viewing is optional.

% Note that DBox does not teach specific algorithm concepts or knowledge, as our target users are learners who have a basic understanding of algorithms but want to improve their ability to apply algorithms to solve real-world problems.


\subsection{Stage 2: Solution Implementation}

\subsubsection{Converting the Step Tree into Comments (D3)} This feature converts each node of the step tree into code comments. When students click ``Copy to Comments'', the system intelligently inserts these comments into the appropriate sections of the code editor. This guides learners to implement their solutions within the corresponding parts of their code, ensuring a smooth transition from planning to coding while reinforcing their structured approach.

\subsubsection{Validating Code Implementation against the Step Tree (D3, D4)} The ``Check Match'' button evaluates the alignment between the code and the step tree. Steps are categorized and color-coded as: (1) \textbf{Implemented}, (2) \textbf{Incorrectly Implemented}, and (3) \textbf{To Be Coded}. Hovering over a step highlights the corresponding lines in the code, providing a direct mapping between the step tree and the code to help users efficiently debug their implementation.


\begin{figure*}[htbp]
	\centering 
	\includegraphics[width=\linewidth]{figures/processing_new.pdf}
	\caption{An illustration of DBox's data processing workflow highlights its core function—creating a step tree with node statuses from student inputs. The LLM processes learners' incomplete code or a step tree they’ve constructed. It outputs a structured JSON object containing steps, sub-steps (and sub-sub-steps, etc.), each with several attributes. Then the JSON object is rendered to the interface, preserving the original structure and only adding nodes for any missing steps. Each node keeps the student's original input, without directly revealing the correct solution. DBox encodes the status of each step with colors and provides progressive hints.}
 
	\label{fig:processing}
        \Description{}
\end{figure*}

\subsubsection{Progressive Hints for Solution Implementation (D1)}
For steps that are incorrectly implemented or yet to be coded, multi-level hints are available: (1) \textbf{General Hint}: Shows simple thought-provoking prompts/suggestions, e.g., ``How should you correctly iterate until the second last character?'' (2) \textbf{Detailed Hint} (Pseudocode): Provides simplified pseudocode to guide the user. (3) \textbf{Reveal Code} (Recommended Implementation): This option is activated only after two failed attempts. Clicking the ``Reveal Code'' button displays the recommended code implementation for the specific step.




\subsection{Backend Design}
DBox's backend is primarily powered by Large Language Models (the GPT-4o model specifically), with four distinct interactions corresponding to four buttons in the interface:
\begin{itemize}
    \item \textbf{From Editor to Step Tree}: This button sends the problem description and the user’s code to the LLM, which generates a step tree with nodes labeled as correct, incorrect, missing, or divisible.
    \item \textbf{Check Step Tree}: Clicking this button inputs the problem description and the user-constructed step tree into the LLM, which returns a labeled step tree with node statuses such as correct, incorrect, missing, or divisible.
    \item \textbf{Copy to Comments}: This button sends the problem description, current step tree, and user’s code to the LLM, generating a mapping of step tree nodes to corresponding lines of code.
    \item \textbf{Check Match}: Pressing this button sends the problem description, step tree, and user’s code to the LLM, which outputs a tree categorizing nodes as implemented, incorrectly implemented, or to be coded.
\end{itemize}

% To further illustrate, we present the data processing workflow behind two core functions: ``From Editor to Step Tree'' and ``Check Step Tree.'' Figure \ref{fig:processing} shows the workflow, where DBox processes student inputs in two modes: code or a constructed step tree. The LLM adapts based on the input. If only code is provided, it generates steps and substeps for the step tree, evaluates node status, and provides hints. If the student supplies a step tree, the LLM assesses the status of each node without altering the structure, only adding steps where needed. The LLM outputs a structured JSON format that organizes steps, substeps, and subdivisions. Each node includes the student’s input, its status (correctness, completeness, potential for subdivision), a general hint, a detailed hint, and the correct content. The step tree is \emph{selectively} displayed on the interface, primarily \emph{retaining the originally student-created structure} while highlighting missing steps. Node statuses are encoded in colors, with corresponding hints accessible through designated buttons.

To illustrate, we present the data processing workflow for two core functions: ``From Editor to Step Tree'' and ``Check Step Tree''. Figure \ref{fig:processing} shows how DBox processes student inputs in two modes: coding mode and language description mode (step tree input). Prompts adapt based on the input type. When only code is provided (\textcolor{darkblue}{dark blue} lines in Figure \ref{fig:processing}), the system first calls \colorbox{darkblue}{\textcolor{white}{[prompt from code]}} to generate a step tree, mapping each code line to a corresponding (sub)step based on the code’s meaning. It then uses the generated step tree to invoke \colorbox{orange}{\textcolor{white}{[prompt from step tree]}} (\textcolor{orange}{orange} lines) to evaluate node statuses, add missing nodes, and generate multi-level hints for incorrect or missing nodes. If the input is a natural language step tree, the LLM directly calls \colorbox{orange}{\textcolor{white}{[prompt from step tree]}} while preserving the original structure. 

The LLM outputs a JSON containing steps, sub-steps, and subdivisions, each with attributes like student original input, status (e.g., correctness, completeness), LLM-validated content, and hints. The JSON is rendered conditionally, preserving the student’s original structure while highlighting missing or incorrect steps. Even if a student's input differs from what the LLM considers correct, the original content is preserved and marked with a status color. Hints are provided and triggered progressively, offering ``just the right'' level of guidance.



\ms{
\subsection{Implementation}
For the front-end, we used native HTML, JavaScript, and jQuery. On the back-end, we deployed the application with the Flask\footnote{https://flask.palletsprojects.com/en/stable/} framework on our university's server. The code editor utilized CodeMirror\footnote{https://codemirror.net/} integrated with pyodide.js\footnote{https://pyodide.org/en/stable/} for running Python code. We employed OpenAI's GPT-4o model with a temperature of 0.8 to maintain flexibility during scaffolding. To align better with the front-end's step tree, we set the response format by setting the parameter response\_format={``type'': ``json\_object''}, restricting the LLM's output. Prompts designed with the chain-of-thought (CoT) technique \cite{wei2022chain} are detailed in the supplementary materials.
}

% \subsubsection{RQ1: Can our approach generate meaningful steps based on the current code solution provided by the users?}

% \begin{itemize}
%     \item If users' solution is correct, can our approach generate correct steps to map each code unit?
%     \item If users' solution is correct but incomplete, can our approach identify the correct steps as well as missing steps?
%     \item If users' solution contains incorrect code lines, can our approach identify the correct steps as well as missing steps?
% \end{itemize}
\section{User Study}




We conducted a user study to evaluate the effects of DBox, focusing on three questions:
\begin{itemize}
    \item \textbf{Q1}: How does DBox support algorithmic programming learning?
    \item \textbf{Q2}: How does DBox affect learners' perceptions and user experience?
    \item \textbf{Q3}: How do learners interact with DBox and perceive the usefulness of different features?
\end{itemize}

% \ms{
% \subsection{Target User Group}
% DBox targets highly motivated, self-regulated students. Using a scaffolding approach, it prioritizes learners' independent thinking by providing only essential support. DBox assumes students actively engage with the tool to improve decomposition skills rather than passively completing tasks. However, we acknowledge that users seeking shortcuts may bypass this by searching for answers online.
% }




\subsection{Conditions}
We conducted a within-subjects design to control for individual differences in programming abilities. Participants experienced two conditions in a randomly assigned order:


\begin{itemize}
    \item \textbf{DBox}: Participants solved problems using the proposed DBox.
    \item \textbf{Baseline}: Participants freely used any available tools (e.g., ChatGPT, Copilot, search engines, LeetCode, QA platforms) to reflect their real-world learning habits, with no restrictions on tool usage or combinations.
\end{itemize}






\subsection{Task and Materials}
In this experiment, participants solve problems from two distinct algorithm types. Each type includes a learning problem, where participants use DBox or baseline tools, and a test problem, solved independently without assistance.

We selected problems from the LeetCode problem bank based on several criteria: First, all problems were of medium difficulty, with an acceptance rate between 40\% and 50\% to ensure sufficient challenge. Second, GPT performs well on these problems. Third, the two algorithm types are distinctly different to avoid learning effects. Finally, the learning and test problems within each algorithm type require similar programming skills to avoid unfair comparisons due to differences in additional coding skills needed for each problem. Based on these criteria, we chose two algorithm types: Greedy and Binary Search. For Greedy, we selected ``Jump Game''\footnote{https://leetcode.com/problems/jump-game/description/} and ``Jump Game II''\footnote{https://leetcode.com/problems/jump-game-ii/description/}; for Binary Search, we selected ``Search in Rotated Sorted Array''\footnote{https://leetcode.com/problems/search-in-rotated-sorted-array/description/} and ``Search in Rotated Sorted Array II''\footnote{https://leetcode.com/problems/search-in-rotated-sorted-array-ii/description/}.

\ms{To help participants become familiar with or recall the algorithms used in the study, we provide them with lecture materials prior to the start of the study. The lecture materials for the two types of algorithms were sourced from GeeksforGeeks\footnote{Greedy: https://www.geeksforgeeks.org/introduction-to-greedy-algorithm-data-structures-and-algorithm-tutorials/}\footnote{Binary Search: https://www.geeksforgeeks.org/binary-search/}. These materials include an introduction to each algorithm, illustrated figures, and practical examples.}


\begin{figure*}[htbp]
	\centering 
	\includegraphics[width=0.9\linewidth]{figures/procedure.pdf}
	\caption{The procedure of our user study. \ms{To avoid learning effects, we used a counterbalanced design with four combinations: (1) DBox-Type1 → Baseline-Type2, (2) Baseline-Type1 → DBox-Type2, (3) DBox-Type2 → Baseline-Type1, and (4) Baseline-Type2 → DBox-Type1. For each combination, six participants were randomly assigned.}}
	\label{procedure}
        \Description{}
\end{figure*}

\subsection{Procedure}
\ms{As shown in Figure \ref{procedure}, obtaining participants' consent, we begin by explaining the study's objectives and procedure. We then familiarize participants with the algorithms they will be practicing using the provided lecture materials. Note that the lecture material was designed to help participants recap the key concepts of these two algorithms. Although the two problems are labeled as Binary Search and Greedy on the LeetCode platform, participants were not restricted to using these two specific approaches to solve the problems}. Afterwards, we administer a pre-test problem to assess their expertise. We also have participants rate their confidence in solving the problem on a 7-point Likert scale following \cite{jin2024teach}. \ms{After participants completed the task or indicated they could not proceed, the two authors (as experiment operators) assessed their solutions against pre-verified answers (with multiple solutions). Participants who solved the problem correctly were excluded, as it suggested higher expertise in the tested problem type. Following \cite{jin2024teach}, we also excluded participants who rated their confidence at 6 or above, as high self-confidence likely indicates less need for additional support. While self-reported confidence may not always align with actual ability, this criterion helps focus the study on the intended user group for our tool. We then provided participants with an interactive tutorial to familiarize them with DBox. The tutorial guides them through each view, button, and functionality of the tool. After the tutorial, they can explore the tool by solving an exercise problem, different from the two problem types in the main study.

Next, we assigned experimental conditions and problem types to participants with a counterbalanced design.} Within each problem type, one problem is randomly assigned in the learning session and the other in the testing session. In the learning session, participants use either DBox or baseline tools, during which they must successfully pass all test cases to proceed. They then complete an in-task survey about their perceptions and experience with the tool they just used. In the test session, participants solve problems without any tool assistance. After completing both problem types, they fill out a post-task survey, followed by a semi-structured interview.





\subsection{Participants}

We conducted a power analysis using G*Power \cite{faul2009statistical} for our two-condition within-subjects design. Assuming an effect size of $f = 0.6$ (moderate), $\alpha = 0.05$, and power of 0.8, we estimated a required sample size of 24 participants.

After IRB approval, we recruited 24 participants via emails and social media at local universities (10 female, 14 male, average age 23.5, SD = 1.7). The group included 16 undergraduates and eight graduate students, with majors in computer science (17), data science (3), electrical engineering (3), and mathematics (1). Most participants (18) coded weekly, six coded monthly, and 23 had used platforms like LeetCode. The 90-minute study compensated participants with 20 USD, equating to 13 USD/hour.






\subsection{Measurements}
Our measurements are summarized in Table \ref{tab:measurement}. To address \textbf{Q1} (effects on learning outcomes), we assessed correctness in the testing session \cite{kazemitabaar2023studying}, perceived learning gain \cite{zhou2021does}, confidence in solving similar problems \cite{hendriana2018role}, improvement in algorithmic thinking \cite{yaugci2019valid}, and self-efficacy \cite{tsai2019improving, yildiz2018digital}. For \textbf{Q2} (effects on perceptions and user experience), we measured cognitive engagement \cite{pitterson2016measuring}, critical thinking \cite{kamin2001measuring}, sense of achievement \cite{wiedenbeck2004factors}, feeling of cheating \cite{kazemitabaar2023studying}, perceived help appropriateness, usefulness \cite{davis1989perceived}, mental demand, effort, frustration \cite{hart2006nasa}, ease of use, satisfaction \cite{bangor2008empirical}, and future use intention \cite{holden2010technology}. For \textbf{Q3} (usage patterns and perceptions of DBox), we analyzed usage logs (e.g., clicks, edits, help-seeking), post-task feature ratings, and conducted semi-structured interviews to delve into participants' underlying reasons behind their perceptions, usage patterns, and reactions to AI errors. All questionnaires used a 7-point Likert scale.

% Most questions, aside from those specific to DBox features, were adapted from established sources such as the NASA Task Load Index \cite{hart2006nasa}, System Usability Scale \cite{brooke2013sus}, and Technology Acceptance Model \cite{holden2010technology}.








% \begin{itemize}
%     \item \textbf{Correctness Score}: The accuracy of learners' solutions in the test tasks. Following the approach used in \cite{kazemitabaar2023studying}, two independent researchers graded each submitted solution using a straightforward and consistent rubric, deducting 25\% for each major issue or missing component in the final submission, resulting in scores of 0\%, 25\%, 50\%, 75\%, or 100\%. The two graders fully agreed on 87.5\% of the submissions. For the remaining 12.5\% where disagreements arose, the graders discussed to resolve conflicts.
%     \item \textbf{Perceived Learning Gain}: "I have learned how to solve this type of problem." 
%     \item \textbf{Confidence in Solving Similar Problems}: "After solving this problem with the tool's help, I feel confident in tackling similar problems." 
%     \item \textbf{Algorithmic Thinking Improvement}: "This tool improved my ability to break down complex problems into smaller, manageable parts." 
%     \item \textbf{Self-Efficacy}: "I have mastered the problem-solving skills necessary for this type of problem." 
% \end{itemize}

% To address \textbf{RQ2: How does Decomposition Box affect learners’ perceptions and user experience?}, we measured the following factors:
% \begin{itemize}
%     \item \textbf{Cognitive Engagement}: "I was cognitively engaged in the programming exercises." 
%     \item \textbf{Critical Thinking}: "The learning process challenged me to think critically." 
%     \item \textbf{Sense of Achievement}: "I feel a sense of accomplishment/achievement when I complete the programming task." 
%     \item \textbf{Sense of Cheating}: "Using this tool feels like cheating." 
%     \item \textbf{Perceived Appropriateness of Help}: "I felt I received the right amount of help when needed—neither too much nor too little." 
%     \item \textbf{Perceived Usefulness}: "This tool is useful for learning how to solve specific problems." 
%     \item \textbf{Mental Demand}: "How mentally demanding was the task?" 
%     \item \textbf{Effort}: "How hard did you have to work to achieve your level of performance?" 
%     \item \textbf{Frustration}: "How insecure, discouraged, irritated, stressed, and annoyed were you?" 
%     \item \textbf{Ease of Use}: "I find this tool easy to use for learning algorithms." 
%     \item \textbf{Satisfaction}: "I am satisfied with the overall learning experience using this tool." 
%     \item \textbf{Future Use}: "I would like to use this tool in my future programming learning." 
% \end{itemize}




% 1. I have learned how to solve this type of problems.
% 2. I was cognitively engaged in the programming exercises.
% 3. The learning process challenged me to think critically.
% 4. After solving this problem with the help, I feel confident in dealing with similar problems.
% 5. I feel a sense of accomplishment/achievement when I complete the programming task.
% 6. Using this tool feels like cheating
% 7. I feel this task was NOT done by myself.
% 8. I felt I received the right amount of help when needed, which is not too much or too few.
% 9. This tool is useful for learning how to solve certain problems. 
% 10. I have mastered the problem-solving skills necessary for this problem.
% 11. This tool improved my ability to break down complex problems into smaller, manageable parts. 
% 12. How mentally demanding was the task?
% 13. How hard did you have to work to accomplish your level of performance?
% 14. How insecure, discouraged, irritated, stressed, and annoyed were you?
% 15. I find this tool easy to use for learning algorithm.
% 16. I am satisfied with the overall learning experience of using this tool.
% 17. I would like to use this tool in my future programming learning.



\renewcommand{\arraystretch}{1.5}
\begin{table*}[htp]  

\centering  
\fontsize{8}{8}\selectfont  

\caption{Measurements used in our user study. For the questionnaire items (within the quotation marks), a 7-point Likert scale was used, with 1 indicating ``Strongly disagree/Very low'' and 7 indicating ``Strongly agree/Very high''.}\label{tab:measurement}

\begin{tabular}{m{0.5cm}<{\centering}m{3cm}<{\centering}m{10.5cm}}
\toprule
\textbf{}&\textbf{Metrics}&\textbf{Detailed Meaning and Questions}\\ \hline\hline


\multirow{6}*{\shortstack{\textbf{Q1}}}&Correctness Score & The correctness of learners' test task solutions was evaluated using a consistent rubric from \cite{kazemitabaar2023studying}. Two authors independently graded submissions, deducting 25\% for each major issue or missing component, yielding scores of 0\%, 25\%, 50\%, 75\%, or 100\%. They agreed on 87.5\% of submissions, resolving disagreements through discussion for the rest.\\
\cline{2-3}
&\cellcolor{gray!15}Perceived Learning Gain &\cellcolor{gray!15} "I have learned how to solve this type of problem." \\ 
\cline{2-3}
&Confidence in Solving Similar Problems & "After solving this problem with the tool's help, I feel confident in tackling similar problems." \\ 
\cline{2-3}
&\cellcolor{gray!15}Perceived Algorithmic Thinking Improvement & \cellcolor{gray!15}"This tool improved my ability to break down complex problems into smaller, manageable parts." \\ 
\cline{2-3}
&Self-Efficacy & "I have mastered the problem-solving skills necessary for this type of problem." \\ 
\hline

\multirow{12}*{\shortstack{\textbf{Q2}}}&\cellcolor{gray!15}Cognitive Engagement &\cellcolor{gray!15} "I was cognitively engaged in the programming exercises." \\
\cline{2-3}
&Critical Thinking & "The learning process challenged me to think critically." \\
\cline{2-3}
&\cellcolor{gray!15}Sense of Achievement &\cellcolor{gray!15} "I feel a sense of accomplishment/achievement when I complete the programming task." \\
\cline{2-3}
&Sense of Cheating & "Using this tool feels like cheating." \\
\cline{2-3}
&\cellcolor{gray!15}Perceived Appropriateness of Help &\cellcolor{gray!15} "I felt I received the right amount of help when needed—neither too much nor too little." \\
\cline{2-3}
&Perceived Usefulness & "This tool is useful for learning how to solve specific problems." \\
\cline{2-3}
&\cellcolor{gray!15}Mental Demand &\cellcolor{gray!15} "How mentally demanding was the task?" \\
\cline{2-3}
&Effort & "How hard did you have to work to achieve your level of performance?" \\
\cline{2-3}
&\cellcolor{gray!15}Frustration &\cellcolor{gray!15} "How insecure, discouraged, irritated, stressed, and annoyed were you?" \\
\cline{2-3}
&Ease of Use & "I find this tool easy to use for learning algorithms." \\
\cline{2-3}
&\cellcolor{gray!15}Satisfaction &\cellcolor{gray!15} "I am satisfied with the overall learning experience using this tool." \\
\cline{2-3}
&Future Use & "I would like to use this tool in my future programming learning." \\
\hline


\multirow{6}*{\shortstack{\textbf{Q3}}}&\cellcolor{gray!15}Button Clicking &\cellcolor{gray!15} (with timestamp) From Editor to Step Tree, Check Step Tree, Check Match, From Step Tree to Comments, Run Code\\
\cline{2-3}
&Editing & (with timestamp) Code edit, step tree edit\\
\cline{2-3}
&\cellcolor{gray!15}Help-Seeking &\cellcolor{gray!15} (with timestamp) see general hint, see detailed hint, and reveal step/code\\
\cline{2-3}
&Usefulness Rating &Participants' ratings on the usefulness of various features in DBox using a 7-point Likert scale\\
\cline{2-3}
&\cellcolor{gray!15}Interviews &\cellcolor{gray!15} Participants' detailed reasons for their perceptions, reactions to AI errors, and self-reported usage patterns, etc.\\

\bottomrule
\end{tabular}
\end{table*}




\subsection{Data Analysis}
\ms{To eliminate the unfair comparison caused by the learning effect of participants using both tools to solve the same type of problem, we selected two distinct problem types. We implemented a randomization procedure to ensure that each participant used either DBox or the baseline tool in a random order, with a randomly assigned problem type for each tool. As a result, each participant used only one tool to solve one problem.

For the quantitative analysis, we employed a linear mixed effects model to analyze the data. The dependent variables (DVs) were our outcome measures (e.g., scores or questionnaire ratings). First, we analyzed the main effect of the two different tools (the coefficient and p-value were reported based on this analysis). Then, we examined the interaction effects between the learning tool and problem type (\emph{Tool*Problem Type}), as well as the interaction effect between the learning tool and the order of tool usage (\emph{Tool*Order}). The fixed effects in the models included the learning tool, problem type, and the order of tool usage, while the random effect accounted for individual differences between participants.

}

For the qualitative analysis of our semi-structured interview data, grounded in the designed questions, we conducted a thematic analysis \cite{hsieh2005three}. Two authors independently coded the data, developed a codebook, and reached a consensus through discussion. In the results, we present key themes supported by representative participant quotes.


% For the quantitative analysis of the questionnaire data and correctness scores, we first conducted a Shapiro–Wilk test to assess normality. Since most of the questionnaire data did not follow a normal distribution, we applied the Wilcoxon signed-rank test, reporting the test statistic ($Z$), significance levels ($p$), and effect sizes ($r$). Additionally, descriptive statistics were used to summarize participants' usage logs.

\section{Applications}
We implemented a series of applications to showcase Omni's potential in supporting spatial interaction with virtual objects that is supported by strong haptic sensations as shown in Figure~\ref{fig:usecases}. While \textit{Omni} can support traditional desktop interaction with haptic cues (\eg free-form tool-based gesture input), we focus our applications on mixed and virtual reality scenarios that are inherently spatial.

Specifically, we demonstrate scenarios in MR, which benefit most from \textit{Omni}'s walk-up-and-use nature to track and haptically actuate an untethered, small magnet in the space around \textit{Omni}'s base.

We implemented our applications using a video pass-through Mixed Reality device (Varjo XR-1), shown in Figure~\ref{fig:photo_device_ar}.
\textit{Videos and photos were recorded live through Varjo's software.}
\begin{figure}[!t]
\centering
\includegraphics[width=\columnwidth]{\dir/sensing/figures/user-applications-live.pdf}
\caption{We combine \textit{Omni} with a video see-through MR device (Varjo XR-1) to showcase the applications.}
\label{fig:photo_device_ar}
\end{figure}

\begin{figure*}[!t]
\centering
\includegraphics[width=\textwidth]{\dir/sensing/figures/applications-02.jpg}
\caption{We present possible use cases for \textit{Omni}. \textit{Left} shows the possibilities in 3D CAD design, in this case sculpting. \textit{Center} shows a user exploring and manipulating an augmented reality object. For both applications, users can feel the shape of the outer hull of the objects. \textit{Right} shows a racing game. Once the car collides with the wall, the pen gets pushed to the base. \textit{Arrows indicate movement and are drawn on top of the photo to increase clarity}. }
\label{fig:usecases}
\vspace{-1em}
\end{figure*}

\subsection{Sculpting}
Figure~\ref{fig:usecases} illustrates how \textit{Omni} haptically supports 3D sculpting and CAD design.

Here, the user finely selects locations on a 3D base object for extrusion by means of the stylus, which is tracked through \textit{Omni}. When extruding individual bumps from the starting configuration, the user can probe and feel the compliance of the material, rendered through attractive and repulsive forces.
Having extruded several bumps from the original shape, the user may inspect the 3D object visually as well as haptically, as \textit{Omni} renders collisions with the tool through tangential actuation.

Following this 3D interaction scenario, \textit{Omni}'s haptic capabilities could be scaled to common 3D editing techniques such as grid snapping, guided object rotation, and 3D transformation.

\subsection{Non-rigid object exploration}
\textit{Omni}'s tracking and actuation also lends itself to haptically rendering geometric objects that are non-rigid and may have anisotropic material properties, such as geographical surfaces, enlarged microscopic surfaces, or other complex geometries.
We demonstrate how \textit{Omni} generates haptic feedback while touching and poking a virtual dragon that is configured to simulate rubber-like material properties.
Here, the force \textit{Omni} renders increases with the amount of object deformation, which portrays the \textit{physical behavior} more accurately than would be possible to experience through mere visual feedback.

\subsection{Gaming}
Finally, we demonstrate how \textit{Omni} can be used for enhancing the experience of gameplay.
Using the magnet-equipped tool as a joystick, we demonstrate how users can steer a car in an AR racing game.
The combination of \textit{Omni}'s haptic feedback and visual control over the car increases the level of immersion provided by the system, as haptic and visual sensations render a coherent experience.
For example, \textit{Omni} renders car collisions with forces whose magnitude depends on collision speeds.
\section{Discussion}
\omniUIST is capable of tracking a passive tool with an accuracy of roughly 6.9 mm and, at the same time, deliver a maximum force of up to 2 N to the tool. This is enabled by our novel gradient-based approach in 3D position reconstruction that accounts for the force exerted by the electromagnet. 

Over extended periods of time, \omniUIST can comfortably produce a force of 0.615 N without the risk of overheating. In our applications, we show that \omniUIST has the potential for a wide range of usage scenarios, specifically to enrich AR and VR interactions.

\omniUIST is, however, not limited to spatial applications. We believe that \omniUIST can be a valuable addition to desktop interfaces, e.g., navigating through video editing tools or gaming. We plan to broaden \omniUIST's usage scenarios in the future.

The overall tracking performance of \omniUIST suffices for interactive applications such as the ones shown in this paper. The accuracy could be improved by adding more Hall sensors, or optimizing their placement further (e.g., placing them on the outer hull of the device).
Furthermore, a spherical tip on the passive tool that more closely resembles the dipole in our magnetic model could further improve \omniUIST's accuracy. We believe, however, that the design of \omniUIST represents a good balance of cost and complexity of manufacturing, and accuracy.

Our current implementation of \omniUIST and the accompanying tracking and actuation algorithms assumes the presence of a single passive tool. Our method, however, potentially generalizes to tracking multiple passive tools by accounting for the presence of multiple permanent magnets. This poses another interesting challenge: the magnets of multiple tools will interact with each other, i.e., attract and repel each other.The electromagnet will also jointly interact with those tools, leading to challenges in terms of computation and convergence. We believe that our gradient-based optimization can account for such interactions and plan to investigate this in the future.

In developing and testing our applications, we found that \omniUIST's current frame rate of 40 Hz suffices for many interactive scenarios. The frame rate is a trade-off between speed and accuracy. In our tests, decreasing the desired accuracy in our optimization doubled the frame rate, while resulting in errors in the 3D position estimation of more than 1 cm, however. Finding the sweet spot for this trade-off depends on the application. While our applications worked well with 40 Hz and the current accuracy, more intricate actions such as high-precision sculpting might benefit from higher frame rates \textit{and} precision.
Reducing the latency of several system components (e.g., sensor latency, convergence time) is another interesting direction of future research. 

Furthermore, the control strategy we used was fairly naïve, as it only takes the current tool position into account. A model predictive strategy could account for future states, user intent, and optimize to reduce heating. We will explore in the next chapter how model predictive approaches can be used for haptic systems.

Overall, the main benefits of \omniUIST lie in the high accuracy and large force it can produce. It does so without mechanically moving parts, which would be subject to wear.
Such wear is not the case for our device, because it is exclusively based on electromagnetic force. We believe that different form factors of \omniUIST (e.g., body-mounted, larger size) can present interesting directions of future research. \add{A body-mounted version could be interesting for VR applications in which the user moves in 3D space. The larger size could result in more discernible points.}

Additionally, the influence of strength on user perception and factors such as just-noticeable-difference will allow us to characterize the benefits and challenges of \omniUIST, and electromagnetic haptic devices in general.
We believe that \omniUIST opens interesting directions for future research in terms of novel devices, and magnetic actuation and tracking.