\section{Dipole-Dipole Model}
\subsection{In-Plane}
    In this section we describe the derivation of the dipole-dipole model for the in-plane actuation force, as well as the case of considering a pen tilt $\angt$ of the pen. Please refer to the schematic Figure \ref{fig:dipole_dipole} for vector notations we use in this section. 
    
    The coordinate system is given by,
    \begin{eqnarray}
     \ed &=& \frac{\posm - \posp}{||\posm - \posp||} \\
     \ez &=& [0,0,1]^T \\
     \et &=& \ed \times \ez
    \end{eqnarray}
    \noindent with $\ed$ the in-paper-plane distance from the pen contact point to the electromagnet center projection, $\ez$ the vertical out-of-plane direction and $\et$ the orthogonal vector to the former two.
    
    The dipole-dipole expression for the force acting on $\mpBold$ due to $\mmBold$ and separated by $\RmagtopenBold$ is:
    \begin{multline}
       \mathbf{F_p} = {\dfrac  {3\mu _{0}}{4\pi \Rmagtopen^{5}}}
       \left [ \left(\langle\mpBold,\RmagtopenBold\rangle \right) \mmBold + 
       \left(\langle\mmBold,\RmagtopenBold\rangle\right) \mpBold \right . +
       \\
       \left(\langle\mpBold,\mmBold\rangle\right) \RmagtopenBold - 
        \left . {\dfrac{5\left(\langle\mpBold,\RmagtopenBold\rangle\right)
        \left(\langle\mmBold,\RmagtopenBold\rangle\right)}{\Rmagtopen^{2}}} \RmagtopenBold \right ] \ , \label{eq:ap.F21-dip}
    \end{multline}
    
    The two dipoles and the vector distance between them can be expressed in the proposed coordinate system as,
    \begin{eqnarray}
     \mmBold &=& \alpha \ m_m \ \ez \label{ap.mm}\\
     \mpBold &=& - (m_p \stheta \cphi) \ \ed \nonumber \\
              && + (m_p \stheta \sphi) \ \et \nonumber \\
              && + (m_p \ctheta) \ \ez \label{ap.mp} \\
     \RmagtopenBold &=& - (d+h_p \stheta \cphi) \ \ed \nonumber \\
              && + (h_p \stheta \sphi) \ \et \nonumber \\
              && + (h - (1-\ctheta) h_p) \ \ez \label{ap.rmp}
    \end{eqnarray}
    
    \noindent and the three scalar products of equation \ref{eq:ap.F21-dip},
    \begin{eqnarray}
     \langle\mmBold,\RmagtopenBold\rangle&=& \alpha \ m_m [h-(1-\ctheta)h_p] \label{eq:ap.mm.r.1}\\
     \langle\mpBold,\RmagtopenBold\rangle&=&mp \ [-\stheta \cphi (d + h_p \stheta \cphi) \nonumber\\
     &&+\stheta^2 \sphi^2 h_p +  \nonumber\\
     &&\ctheta (h - h_p(1-\ctheta))] \ \  \label{eq:ap.mp.r.1} \\
     \langle\mmBold,\mpBold\rangle&=& \alpha m_m m_p \ctheta \label{ap:mm.mp.1}
    \end{eqnarray}

\subsubsection{Position-aware dipole-dipole model}
    \label{sc:ap.position-dipole}
    We first derive the position-aware dipole-dipole model\del{ (3 DOF)}, before continuing to the full \del{6 DOF }\add{position-aware and angle-aware} model.
    We rewrite Eq. \ref{eq:ap.F21-dip} with an equivalent pen dipole $\mpBoldt$, obtained by applying the small tilting angle approximation ($\ctheta \simeq 1$ and $\stheta \simeq 0$) to Eq. \ref{ap.mp}, %We first denote with:}
    \begin{equation}
        \mpBoldt = m_p \ \mathbf{e_z} \label{eq:m1t} \ ,
    \end{equation} 
    \noindent where the scalar magnetization is given by $m_p = B_r V/\mu_0$. $B_r$ is the residual magnetization of the permanent magnet and $V$ its volume and  $\mathbf{e_z}$ is the $z$-unit vector. This approximation removes the requirement for tracking the pen tilt. More importantly it drastically simplifies the force equation since both dipoles now only have a $z$ component and thus the actuation only depends on the distance $d$ between pen and magnet (not on $\angt$ nor $\angp$).
    %(see Table \ref{tab:em_model}). 
    This provides a simplified version of the 3D distance vector,
    \begin{equation}
        \RmagtopenBoldt = - d  \ \ed + h  \ \mathbf{e_z} , \label{eq:r21b} 
    \end{equation}
    \noindent where the vertical distance, $h = h_m + h_p$, is constant. Note that the in-plane distance $d = \norm{\posp - \posm}$ is one of the variables we seek to control, given the projections of the pen position ($\posm$) and the electromagnet position ($\posp$) onto the sketching plane. %The unity vector between these two position is $\ed = \frac{\posp - \posm}{\norm{\posp - \posm}}$.(see Figure \ref{fig:em_model}). 
    
    The electromagnet dipole ($\mmBold$) is mounted in a fixed upright position. Therefore it can be expressed via Eq. \ref{ap.mm}, without incurring any approximation error.
    The magnetization value of the full-strength dipole $m_m$, which approximates the electromagnet, can be derived experimentally. For this purpose we scan the magnetic field generated by the electromagnet, setting $\alpha = 1$ and using a hall sensor and adjust the parameters of EM field equation to give a good fit, as explained below in section \textit{Electromagnet dipole equivalent}. Table \ref{tab:ap:em_model} reports the values of $m_m$, $m_p$ and $h$ that were used in our experiments.
    
    \begin{table}[tb]
      \caption{List of electromagnet model and hardware parameters}
      \label{tab:ap:em_model}
      \begin{tabular}{lll}
        \toprule
        Name&Value&Description\\
        \midrule
        $\mu_0$ & $4\pi \ 10^{-7}$ [H/m] & Vacuum permeability \\
        $B_r$ & 1.3 [T] & Pen magnet type (NIB N42) \\
        $V$ & 0.66 [cm$^3$] & Pen magnet volume \\
        $m_p$ & 0.683 [A m$^2$]& pen dipole ($=B_r V / \mu_0$)\\
        $m_m$ & 1.286 [A m$^2$]& electromagnet dipole\\
        $h$ & 2.71 [cm] & z-distance $\mmBold$ to $\mpBold$ \\
        $h_p$ & 1.40 [cm] & height pen-tip to magnet \\
        $h_m$ & 1.31 [cm] & z-distance from plane to $\mmBold$.\\  
        $F_0$ & 0.488 [N] & force factor in Eq. \ref{eq:Fa} \\
      \bottomrule
    \end{tabular}
    \end{table}
    
    The total force acting on the pen (Eq. \ref{eq:ap.F21-dip}) can now be decomposed into the in-plane and vertical force components: 
    \begin{equation}
        \mathbf{F_p} = F_a \ \mathbf{e_d} + F_z  \ \mathbf{e_z} \ . \label{eq:ap:Fp_decomp}
    \end{equation}
    
    \noindent Here $\mathbf{F_a} = F_a \ \mathbf{e_d}$ represents the quantity we seek to control. 
    By substituting the results form Eq. \ref{ap.mm}, \ref{eq:m1t} and \ref{eq:r21b} into Eq. \ref{eq:ap.F21-dip} and maintaining only the in-plane contributions ($\mathbf{e_d}$ direction), we obtain the expression for the actuation force as function of pen-magnet separation:
    \begin{equation}
        \mathbf{F_a} = \alpha \ F_0 \ \left( \frac{d \left(4 - \frac{d^2}{h^2}\right)}{h \left(1 + \frac{d^2}{h^2}\right)^\frac{7}{2}} \right)  \ \mathbf{e_d} , \label{eq:Fa}
    \end{equation}
    where $F_0$ is a constant force parameter given by the expression,
    \begin{equation}
     F_0 = \frac{3 \ \mu_0 \ m_p \ m_m}{4 \ \pi \ h^4} \ . \label{eq:F0}
    \end{equation}
    
    Fig. \ref{fig:ap:approx_error} illustrates how the dimensionless ratio within parentheses in Eq. \ref{eq:Fa} governs the force strength as function of distance $d=\norm{\mathbf{r_d}}$. 
    The actuation force $F_a$ is zero if the two magnets are aligned with one another ($d=0$), it has a maximum $F_a^{max} = 0.9 F_0$ at $d=0.39h$, and we can assume there is no more attraction for distances $d>2h$. In Table \ref{tab:ap:em_model} we report the value of $F_0$ we obtained for our prototype.
    
    Note that these simplifications lead only to a small approximation error.
    Compared to an angle dependent formulation, a tilt of up to $\angt = 30^{\circ}$ leads to a max error in our model (Eq. \ref{eq:Fa}) equivalent to shifting the distance $d$ by $\pm 3$ [mm] (Figure \ref{fig:ap:approx_error}). 
    
    \begin{figure}[!t]
        \centering
        \includegraphics[width=0.6\columnwidth]{chapters/05_shared_control/mpc/figures/F-d-30.pdf} \\
        \caption{In-plane magnetic force as function of position. The horizontal displacement between curves (each denoting a different pen-tilt) is the approximation error induced by the upright pen (purple) assumption (angles defined in \protect\figref{fig:dipole_dipole}).}
        \label{fig:ap:approx_error}
    \end{figure}

\subsubsection{Angle-aware dipole-dipole model}
\label{sc:ap.angle-dipole}
    In this section, we derive the \del{angle-aware EM model, that can be used for 6 DOF actuation}\add{complete EM model, using both, the pen position and its tilting angle as free variables}.
    We continue the deduction of $\mathbf{F_p}$ by substituting Eq. \ref{ap.mm}---\ref{ap:mm.mp.1} into the main expression Eq. \ref{eq:ap.F21-dip}. However, by following that path we wouldn't necessarily attain information on how strong the actuation force depends on the tilting angles $\angt$ and $\angp$. Here we take a different path. Based on the geometry of our system, we consider the cases where the pen is tilted by only a small angle\del{ ($\angt < 30 ^o$)}. We introduce this small-angle approximation by keeping only the first order terms in $\angt$,
    \begin{eqnarray}
     \stheta &\approx& \angt \ \ \ \ \text{(with} \ \angt \ \text{in \ radians)} \label{ap.appsin} \\
     \ctheta &\approx& 1 \label{ap.appcos}
    \end{eqnarray}
    \noindent As an indication of what this approximation means, for an angle $\angt = 30 ^\circ$, the difference between using $\stheta$ or $\ctheta$ or their approximations forms (Eq. \ref{ap.appsin} and \ref{ap.appcos}) is 5\% and 15 \%, respectively. Under the small-$\angt$ approximation, the dipoles' vectors are,
    \begin{eqnarray}
     \mmBold &=& \alpha \ m_m \ \ez \label{ap.mm2}\\
     \mpBold &\simeq& -m_p \angt \cphi \ed + m_p \angt \sphi \et + m_p \ \ez \label{ap.mp2}
    \end{eqnarray}
    \noindent and the distance between dipoles,
    % \begin{eqnarray}
    % %  \RmagtopenBold \simeq  && - (d+h_p \angt \cphi) \ \ed \nonumber \\
    % %             && + (h_p \angt \sphi) \ \et \nonumber \\
    % %             && + (h -) \ \ez
    % \RmagtopenBold \simeq -(d+h_p \angt \cphi) \ed h_p \angt \sphi \et h \ez
    % \end{eqnarray}
    \begin{equation}
    \RmagtopenBold \simeq -(d+h_p \angt \cphi) \ \ed + h_p \angt \sphi \ \et + h \ \ez
    \end{equation}
    \noindent with the length of that distance, at first order on $\angt$,
    \begin{equation}
        \Rmagtopen \simeq d^2 + h^2 + 2 d h_p \angt \cphi \label{eq:ap.rapp}
    \end{equation}
    
    In turn, the scalar products (Eq. \ref{eq:ap.mm.r.1}---\ref{ap:mm.mp.1}) can be written as,
    \begin{eqnarray}
     \langle\mmBold,\RmagtopenBold\rangle &\simeq& \alpha \ m_m h \label{eq:ap.mm.r.2}\\
     \langle\mpBold,\RmagtopenBold\rangle &\simeq& mp \ [-\angt \cphi d + h] \label{eq:ap.mp.r.2} \\
     \langle\mmBold,\mpBold\rangle &\simeq&\alpha \ m_m m_p \label{ap:mm.mp.2}
    \end{eqnarray}
    
    We can now substitute these expressions into the main force equation \ref{eq:ap.F21-dip}. As we do in \del{Section \nameref{sc:em_model}}\add{previous section}, we consider only the terms that contribute to the component $\ed$ of the force. Keeping only these terms that contain $\angt$ up to the first order,
     \begin{eqnarray}
     \mathbf{F_{p}^{(d)}} =&& \frac{3\mu _{0} \alpha m_m m_p}{4\pi \Rmagtopen^{5}} \left[-d + \frac{5 d h^2}{\Rmagtopen^2} - h \angt \cphi - h_p \angt \cphi + \right.\nonumber \\
     && \left. + \frac{5 h^2 h_p \angt \cphi}{\Rmagtopen^2} - \frac{5 h d^2 \angt \cphi}{\Rmagtopen^2} \right] \ \ed \\
     =&& \frac{3\mu _{0} \alpha m_m m_p}{4\pi (h^2+d^2)^{5/2}} \left[ \frac{-d (d^2+h^2) +5 d h^2}{(h^2+d^2)} + \right. \nonumber \\
     && \left. \angt \cphi \left( -h -h_p + \frac{5(h^2 h_p - h d^2)}{(h^2+d^2)} - \frac{5 d^2 h^2 h_p}{(h^2+d^2)^2}  \right) \right] \ed \nonumber \\
     && \label{eq:ap.Fp2} \\
      \mathbf{F_{p}^{(d)}} =&& \alpha \ F_0 \ \left[ \ f_0(d) \ + \ \angt \cphi \ f_1(d) \ \right] \ \ed \label{eq:ap.Fd}
     \end{eqnarray}
    %  \noindent and finally split in the form,
    %  \begin{equation}
    %      \mathbf{F_{p}^{(d)}} = F_0 \left[ f_0(d) + \angt \cphi f_1(d) \right] \ed \label{eq:ap.Fd}
    %  \end{equation}
     \noindent where we define, 
    \begin{eqnarray}
     F_0 &=&  \frac{3 \ \mu_0 \ m_p \ m_em}{4 \ \pi \ h^4} \ . \label{eq:ap.F0}\\
     f_0(d) &=& \frac{d \left(4 - \frac{d^2}{h^2}\right)}{h \left(1 + \frac{d^2}{h^2}\right)^\frac{7}{2}} \label{ap.f_0}\\
     f_1(d) &=& \frac{1 + \frac{h_p}{h}}{\left(1 + \frac{d^2}{h^2} \right)^\frac{5}{2}} 
      + \frac{5 \left(\frac{h_p}{h} + \frac{d^2}{h^2}\right)}{\left(1 + \frac{d^2}{h^2} \right)^\frac{7}{2}} 
      - \frac{5 \left(\frac{h_p}{h}\right) \left(\frac{d^2}{h^2}\right)}{\left(1 + \frac{d^2}{h^2} \right)^\frac{9}{2}} 
    \end{eqnarray}
    
    Note that by considering the case $\angt = 0$ in Eq. \ref{eq:ap.Fd}, we recover what we obtain \del{in the Sec. \nameref{sc:em_model}} for $\mathbf{F_a}$ \add{ as calculated in Eq. \ref{eq:Fa}}. 
    That means that the equation for $\mathbf{F_{p}^{(d)}}$\del{ we obtained in this Appendix} subsumes the cases of the pen being tilted by a small angles, and it can be used in future EM actuated systems which may be able to track $\angt$ and $\angp$.