\chapter{Introduction}
\label{ch:shared:introduction}
% Shared Variable
Contextual intelligent systems often work with and alongside humans. Their contextual nature allows them to make decisions based on more variables than just explicit user input (i.e., context). This loop of observing the world, decision-making, and acting in the world is similar to human actions. Since both humans (natural agents) and intelligent systems (artificial agents) can operate in the real world simultaneously, this world becomes a shared medium of communication between both agents (see Fig. \ref{fig:enter-label}). The user can change something that impacts the perception and decision-making of the artificial agent, and vice versa. However, not all contextual variables in the world are shared between both agents, due to differences in perceptual capabilities, tasks, or interests. Hence, a subset of all contextual variables is a shared variable, meaning a variable that both agents can perceive and that influences the decision-making process.

\begin{figure}
    \centering
    \includegraphics[width=\textwidth]{example-image-a}
    \caption{Caption}
    \label{fig:enter-label}
\end{figure}

% --> Haptics
A specific instance of a shared variable interface is a subset of haptic interfaces. Haptic interfaces are particularly interesting because they operate in the physical world, where the challenge of variable ownership is much more impactful. Additionally, two agents exerting force on a single object is an intuitive way to consider shared variable interfaces. For a haptic device to have a shared variable, it generally needs to fulfill two prerequisites. First, as the term interface indicates, the haptic device must function as both an input and output device. While many haptic devices function primarily as output devices (such as Dextres \cite{hinchet2018dextres}), some can also serve as input devices. Second, the haptic feedback it provides should be kinesthetic rather than (vibro-)tactile. Haptic interfaces that fulfill both criteria are characterized by their ability to affect a shared variable through interaction, either by restricting hand movement or by actively moving a tool. For physically based interactions, as in haptics, this means delivering a force.

% Content
%% 1
We investigate a haptic shared variable interface through a use case, presented in two chapters. In the first chapter (\ref{}), we discuss the design and fabrication of a contact-free volumetric haptic feedback device. Our device is a novel spherical electromagnetic system that delivers grounded, yet untethered, forces on a tool. The device consists of a single 60 mm diameter spherical electromagnet and can render attractive and repulsive forces onto permanent magnets embedded in pointing tools such as a stylus or magnets directly worn on the user’s fingertip. Leveraging a dipole-dipole approximation of the electromagnet-magnet interaction, our system is capable of calculating and controlling the forces exerted on the permanent magnet in real-time while dynamically adjusting the force perceived by the user. The system can deliver perceptible forces up to 1N in a thin volume above the surface. Furthermore, we demonstrate that users can distinguish at least 25 different set-points separated by $18^\circ$ on the surface of the sphere. Our device builds on three key components that represent our contributions in this work: i) A computational model based on magnetic dipole-dipole interaction to produce force maps that allow for designing and generating location-dependent feedback, ii) the design and implementation of a 3 degree-of-freedom (DoF) spherical electromagnet prototype, and iii) a control strategy that translates desired high-level forces into low-level input signals (currents/voltages) for the coils, fast enough for interactive use.

%% 2
In the second chapter (\ref{}), we present Omni, a self-contained 3D haptic feedback system capable of sensing and actuating an untethered, passive tool containing only a small embedded permanent magnet. The electromagnetic actuation and sensing interfere since two physical phenomena (e.g., tool motion and electromagnet control) create a single, superimposed magnetic field. To decompose the resulting superposition of magnetic fields, we introduce a novel 3D positioning algorithm. Our formulation reconstructs the magnet’s 3D position by minimizing the residual between the predicted magnetic field strength at a given location and the actual measurement picked up by the Hall sensors. The formulation is amenable to gradient-based optimization, and we show that the problem can be solved via a quasi-Newton solver in real-time. We make the following contributions in this project: i) a fully self-contained system, combining actuation and tracking components, ii) a novel algorithm for the reconstruction of the 3D tool position under electromagnetic actuation, iii) a thorough technical evaluation of the algorithm, yielding an accuracy of 5 mm without actuation and 7 mm with the electromagnet on, iv) assessment of the force generation capabilities, showing up to 2 N of haptic force in any direction, and v) an exploration of use cases afforded by Omni.

% Implications
The implications of our investigation into haptic shared variable interfaces are two-fold. First, we show that shared variable interfaces open up novel applications in areas such as gaming, object exploration, and design. Second, we show that manipulating the same device is natural for the user. Sensing and actuating the same device is necessary for this. Doing this via the same modality, electromagnetism in this case, is a cost-effective and efficient solution.

% Summary
In summary, we introduce two novel electromagnetic platforms that actuate a permanent magnet in the space around it. In the first project, we focus on the design and fabrication of a spherical electromagnet. We used this to inform the second project, Omni. Omni integrates 3D magnetic sensing using Hall sensors and magnetic actuation through radial and tangential forces produced by three orthogonal electromagnetic coils integrated into a single sphere. Omni’s capabilities allow spatial interaction systems to integrate 3D tracking and actuation of untethered, free-ranging tools, simply by embedding a small permanent magnet. We can use our findings as a starting point for investigating shared variable interfaces. 