\chapter{Summary \& Insights}
\label{ch:shared:conclusion}

\paragraph{Summary}
We introduced two novel electromagnetic platforms designed to actuate a permanent magnet in the surrounding space. The first project, "\omniHapTitle" (\chapref{ch:shared:contact}), focused on designing and fabricating a spherical electromagnet capable of generating adjustable radial and tangential forces using three orthogonal electromagnetic coils integrated into a single sphere. This system targets a handheld tool, such as a stylus, and allows users to feel relatively large forces despite the tool being untethered. A user experiment with six participants characterized the force delivery and perceived precision, revealing a resolution of at least 25 discernible locations for repelling forces distributed across the hemisphere. However, \omniHap required external tracking of the tool, which was cumbersome and expensive.

In "\omniUISTTitle" (\chapref{ch:shared:volumetric}), we addressed the tracking limitation of \omniHap by integrating 3D magnetic sensing using Hall sensors with magnetic actuation through an improved spherical electromagnet. This integration allows spatial interaction systems to incorporate 3D tracking and actuation of untethered tools by embedding a small permanent magnet. The core contribution of \omniUIST is resolving the interference caused by simultaneous magnetic tracking and actuation using a novel gradient-based optimization method. This method provides 3D tracking capabilities with a mean error of 6.9 mm during actuation forces of up to 2N. We demonstrated several example applications showcasing \omniUIST's capabilities.

\paragraph{Implications}
In our context, both the user and the system interact with a shared variable, such as car acceleration or a graphical menu. Kinesthetic haptic feedback is a special case where this variable (e.g., joystick position) is both acted upon and perceived through the same modality, like the user's hands. Previous research on kinesthetic haptic feedback generally focuses on complex mechanical devices (e.g., \cite{Massie94, Stamper1997, VanDerLinde2002, Araujo2016, zoller2019assessment, Sinclair2019Capstan}). Extensions of these systems include exoskeletons~\cite{Gu2016, Choi2016}, gloves~\cite{Cybergrasp, hinchet2018dextres}, and tilt-platforms~\cite{Prattichizzo2013, Kim2016}. However, these often necessitate user instrumentation and introduce system friction that users can always feel, even when the system is off, thereby limiting full user control.

In contrast, our work provides grounded yet untethered forces onto a tool, thereby overcoming the limitations of not covering the full spectrum from user agency to system automation. Our work lays the foundation for investigating haptic devices from a new perspective, that of shared variable interfaces. We have taken the first step towards understanding these interfaces, demonstrating that haptic shared variable interfaces open up novel applications in gaming, object exploration, and design. By eliciting forces on the tool manipulated by the user, we enable haptic feedback in the same modality as the user's exertion, resulting in natural interactions. This integration of sensing and actuating the tool through electromagnetism is both cost-effective and efficient.

\paragraph{Limitations}
Both \omniHap and \omniUIST have limitations within the context of interfaces with shared variables.

First, while our work suggests that interacting with a shared variable haptic interface is natural for users, it remains uncertain to what extent our findings generalize to other interface types, such as graphical interfaces. Further investigation is needed to understand the interaction implications beyond the haptic paradigm.

Second, we have not compared our results to haptic feedback without a shared variable, such as vibrotactile feedback. Comparing against a broader range of haptic devices would allow us to draw more comprehensive conclusions regarding usability improvements.

Finally, we have not explored an optimal control strategy for haptic feedback. Haptic feedback involves not only the device itself but also how the device is controlled. Currently, we provide haptic feedback based on heuristics and the current tool position. An adaptive strategy incorporating a predictive user model might be more effective. Such a control method could balance autonomy and automation, and account for user intent, making the device more dynamic.