\chapter{Volumetric sensing and actuation of passive magnetic tools for dynamic
haptic feedback}
\label{ch:shared:volumetric}
\chaptermark{Volumemetric Sensing and Actuation of Passive Magnetic Tools}

\contribution{
    In this chapter, we present \omniUIST, a self-contained 3D haptic feedback system capable of sensing and actuating an untethered, passive tool containing only a small embedded permanent magnet. \omniUIST overcomes the limitations of \omniHap in two significant ways. First, it eliminates the need for external tracking by utilizing a novel gradient-based method to reconstruct the 3D position of the permanent magnet in midair. This is achieved using measurements from eight off-the-shelf Hall sensors integrated into the base. Second, \omniUIST features an improved 3 DoF spherical electromagnet capable of delivering increased forces due to intertwined coils. The fully integrated \omniUIST system, with no moving parts and no need for external tracking, is easy and affordable to fabricate. We detail \omniUIST’s hardware implementation, our 3D reconstruction algorithm, and provide an in-depth evaluation of its tracking and actuation performance. Finally, we demonstrate its capabilities through a set of interactive usage scenarios.
}


\begin{figure}
\centering
\includegraphics[width=\textwidth]{\dir/sensing/figures/teaser-03.jpg}
    \caption{We present Omni, a device that can simultaneously actuate and sense the position of a passive handheld tool. This is enabled through integrated hall effect sensors and our novel gradient-based optimization scheme. Omni can for example be used in 3D applications such as MR sculpting.}
\label{fig:teaser_sensing}
\end{figure}

\begin{figure}[ht]
    \centering
    \includegraphics[width=0.8\linewidth]{graphs/greater_than_naive.pdf}
    \vspace{0.5cm}
    \includegraphics[width=0.8\linewidth]{graphs/p1_bottom.png}
    \vspace{-5pt}
    \caption{\textcolor{positional}{Positional} vs.\ \textcolor{nonpositional}{non-positional} circuits. In a \textcolor{nonpositional}{non-positional} circuit, the same edges must be included at all positions. A \textcolor{positional}{positional} circuit can distinguish between the same edge at different positions. This specificity yields better trade-offs between circuit size and faithfulness. It can also increase both precision and recall.}
    \label{fig:p1}
    \vspace{-5pt}
\end{figure}

\section{Introduction}

\looseness=-1
A primary goal of interpretability research is to characterize the internal mechanisms in language models (LMs) and other NLP models. 
A core approach in this area is \textbf{circuit discovery}---identifying the minimal subgraph within the model's computation graph that performs a specific task \citep{olah2021framework,olah-mech}.
Typically, the nodes of a circuit represent model components (e.g., attention heads, neurons, or layers).
While manual circuit discovery methods can yield position-specific insights \citep{wanginterpretability,goldowskydill2023localizingmodelbehaviorpath}, \emph{automatic methods often overlook positional information}, treating components as uniformly relevant across all input token positions \citep{conmytowards,syed2023attribution}. 
For instance, if an attention head is included in a circuit, it is assumed to contribute equally to the computation for every position in the input sequence.
The assumption that circuits are position-invariant ignores the fact that different positions often require distinct computations.
By ignoring positions, current methods limit their ability to capture mechanisms that operate across positions, such as interactions between attention heads across positions.

In this study, we start by demonstrating that positional agnosticism is a significant limitation (\S\ref{sec:motivating}). Then, to address these limitations, we introduce a new approach: position-aware edge attribution patching (PEAP; \S\ref{sec:full_circ_discovery}; Figure~\ref{fig:p1}). Current approaches  assume that if an edge is in a circuit, then the same edge will be in the circuit at all positions, thus leading to low precision. It is also assumed that an edge's importance should be aggregated across positions before deciding whether it should be included in the circuit; this can lead to cancellation effects, and thus low recall. PEAP instead allows us to compute the importance of cross-positional edges, and separately evaluates edge importance at each position. We show that this leads to smaller and more accurate circuits; see Figure~\ref{fig:p1}.

Incorporating positional information into circuit discovery is straightforward when inputs have the same length and structure across examples.

However, realistic datasets are not nearly this templatic.
How, then, can we incorporate positional information into automatic circuit discovery?
To address this challenge, we propose \textbf{schemas} (\S\ref{sec:schema}). 
Schemas assign semantic labels to spans of tokens, enabling information aggregation across examples even when the spans differ in length.

For example, in the input ``The \textcolor{positional}{war} lasted from 1453 to 14\underline{\hspace{1em}},'' the span ``\textcolor{positional}{war}'' could be labeled as ``\emph{Subject}''.
This enables handling spans with varying lengths: the phrase ``\textcolor{positional}{Black Plague}'' in another example can be treated as a single positional span with the same role as ``\textcolor{positional}{war}''.
In experiments with two LMs and three tasks, we find that circuits discovered using schemas achieve a better trade-off between circuit size and faithfulness to the model's behavior than position-agnostic circuits.
Importantly, position-aware circuits offer a more precise representation of the underlying mechanisms, providing a more concise foundation for mechanistic explanations.

We also present a fully automated pipeline for schema generation and application (\S\ref{sec:schema-generation}) using large language models (LLMs). 
We evaluate the quality of the generated schemas and their utility in discovering position-aware circuits (\S\ref{sec:schema-eval}).
Notably, circuits derived using automatically generated and applied schemas achieve comparable faithfulness scores to circuits discovered with human-designed and manually applied schemas.

We summarize our contributions as follows:
\begin{itemize}[noitemsep,leftmargin=*,topsep=1pt,parsep=1pt]
    \item Introduce a position-aware circuit discovery method, which obtains better faithfulness than position-agnostic discovery.  
    \item Introduce dataset schemas,  facilitating positional circuit discovery in more naturalistic settings. 
    \item Develop an automated schema generation and application pipeline with LLMs, yielding schemas that are comparable to manually-annotated ones.
\end{itemize}


\section{Related work}


Recent advances in single-image animatable head avatar generation can be categorized into mainly 2D-based and 3D-based approaches. 

\paragraph{\bf Image to 2D Animatable Avatar.}
2D-based methods, leveraging the power of convolutional neural networks (CNNs)~\cite{DBLP:conf/cvpr/KarrasLAHLA20,DBLP:conf/cvpr/IsolaZZE17,DBLP:conf/nips/GoodfellowPMXWOCB14}, often employ generative adversarial networks (GANs)~\cite{DBLP:conf/cvpr/StyleGAN} for direct image synthesis. Early approaches~\cite{DBLP:conf/cvpr/WangDYSW23,DBLP:conf/cvpr/BurkovPGL20,DBLP:conf/iccv/ZakharovSBL19} focus on injecting expression and pose features into the generator network, often utilizing architectures like U-Net or StyleGAN~\cite{DBLP:conf/cvpr/StyleGAN}.
Some other 2D methods~\cite{DBLP:journals/corr/abs-2407-03168,DBLP:conf/cvpr/ZhangQZZW0CW023,DBLP:conf/cvpr/HongZS022,DBLP:conf/mm/DrobyshevCKILZ22,DBLP:conf/cvpr/BurkovPGL20,DBLP:conf/nips/SiarohinLT0S19} represent expressions and poses as warping fields applied to the source image. 
Benefiting from advances in image and video diffusion networks, more recent 2D-based works~\cite{DBLP:journals/corr/abs-2410-07718,DBLP:journals/corr/abs-2406-08801,DBLP:conf/eccv/TianWZB24} get improved results with diffusion techniques. 
However, these methods still face challenges related to long generation times and significant computational resource demands. Audio-driven 2D control methods~\cite{DBLP:conf/cvpr/ZhangCWZSGSW23,DBLP:journals/corr/abs-2211-12368,DBLP:conf/iccv/GuoCLLBZ21} are easy to use but cannot explicitly control facial expressions and poses. 2D-based techniques often struggle with large pose or expression variations due to the lack of an explicit 3D structure, sometimes producing unrealistic distortions or identity changes. While some 2D methods~\cite{SadTalker,StyleHEAT,Pirenderer,DBLP:conf/cvpr/WangM021,MegaPortraits} incorporate 3D Morphable Models (3DMMs)~\cite{DBLP:conf/fgr/GerigMBELSV18,DBLP:journals/tog/LiBBL017,DBLP:conf/avss/PaysanKARV09,DBLP:conf/siggraph/BlanzV99} to mitigate these issues, they typically cannot achieve free-viewpoint rendering. 

\vspace{-0.1in}

\begin{figure*}[h]
    \centering
    \includegraphics[width=0.9\linewidth]{images/framework.pdf}
    \caption{\textbf{Overall Framework.} Our framework utilizes learnable query features attached to FLAME vertices to perform cross-attention with the extracted multi-level image features. The extracted features are then decoded to reconstruct the Gaussian avatar in the canonical space, which can be animated utilizing standard linear blend skinning (LBS) and corrective blendshapes as the FLAME model did and rendered in real-time on various platforms.}
    \label{fig:framework}
\end{figure*}

\paragraph{\bf Image to 3D Animatable Avatar.}
3D-aware methods offer improved geometric consistency and free-viewpoint rendering capabilities. Early 3D approaches~\cite{DBLP:conf/eccv/KhakhulinSLZ22,DBLP:conf/cvpr/XuYCWDJT20} utilize 3DMMs for head avatar reconstruction. With the advent of Neural Radiance Fields (NeRFs)~\cite{DBLP:conf/eccv/MildenhallSTBRN20}, many recent methods~\cite{DBLP:conf/siggraph/YuFZWYBCSWSW23,DBLP:conf/cvpr/MaZQLZ23,DBLP:conf/cvpr/LiZWZ0CZWB023,GPAvatar,ye2024real3d,deng2024portrait4d,deng2024portrait4d2,DBLP:conf/eccv/KiMC24,DBLP:conf/cvpr/BaiFWZSYS23,PointAvatar,Nerfies,INSTA} have adopted this representation for higher fidelity, particularly in modeling fine details like hair. However, NeRF-based~\cite{DBLP:conf/cvpr/ZhangZLHLWGCL024,HAvatar,DBLP:conf/cvpr/BaiTHSTQMDDOPTB23,AD-NeRF,DBLP:journals/tog/GaoZXHGZ22,DBLP:journals/tog/ParkSHBBGMS21,DBLP:conf/cvpr/AtharXSSS22,DBLP:journals/corr/abs-2112-05637,DBLP:conf/iccv/TretschkTGZLT21,DBLP:conf/cvpr/GafniTZN21,DBLP:conf/eccv/KiMC24,DBLP:conf/cvpr/BaiFWZSYS23,PointAvatar,Nerfies,DBLP:conf/siggraph/YuFZWYBCSWSW23,DBLP:conf/cvpr/MaZQLZ23,DBLP:conf/cvpr/LiZWZ0CZWB023} approaches often require extensive training data, including multi-view or single-view videos, raising privacy concerns and limiting generalization to unseen identities. Some methods~\cite{DBLP:conf/cvpr/SunWWLZZL23,DBLP:conf/3dim/ZhuangMKS22,DBLP:journals/pami/SunWZHWL24,DBLP:journals/tvcg/TangZYZCMW24,DBLP:conf/iclr/XuZLZBFS23} bypass this data requirement by training generators with random noise and then inverting them for identity-specific reconstruction, but inversion accuracy remains a challenge. Test-time optimization offers another alternative, but its computational cost limits practical applications. Several recent works~\cite{goha2023,hidenerf2023,gpavatar2024,ye2024real3d,ma2024cvthead,deng2024portrait4d,deng2024portrait4d2,GGHead} have explored one-shot 3D head reconstruction to address the limitations of data requirements and computational cost. These methods employ various techniques, such as tri-plane features, deformation fields, point-based expression fields, and vertex-feature transformers. Despite these advancements, NeRF-based methods often struggle with real-time rendering. 
Recently, 3D Gaussian Splatting~\cite{GaussianSplatting} has emerged as a promising alternative, offering both high-quality results and fast rendering speeds. However, existing Gaussian Splatting methods~\cite{GaussianAvatar,DBLP:conf/cvpr/XuCL00ZL24} typically rely on video data for training for each person, limiting their ability to generalize to new identities. Instead, the most recent work, GAGAvatar~\cite{GAGAvatar}, proposes a one-shot 3D Gaussian-based head avatar generation method. However, it still relies heavily on complex 2D neural post-processing to achieve optimal animation outcomes, thus it is not a pure 3D solution and the extra neural network hinders its application on various platforms. In contrast, our work generates Gaussian heads that are immediately animatable and renderable without additional networks or post-processing steps, enabling seamless integration into existing rendering pipelines for real-time animation and rendering across a wide range of platforms, including mobile phones. 
\begin{figure*}[!t]
    \centering
    \includegraphics[width=.75\textwidth]{\dir/sensing/figures/long_hardware.pdf}
   \caption{Overview of our system. A 3D printed base contains the 3 DoF intertwined coils and the circular PCB with an array of eight hall sensors. Arbitrarily shapes tools can be 3D printed and augmented with a permanent magnet, to interact with \omniUIST.}
    \label{fig:hardware}
\end{figure*}

\section{System Overview}
%\section{Method}

\omniUIST is a self-contained haptic feedback system that simultaneously integrates 3D tracking and actuation using the same modality (see Figure~\ref{fig:hardware}).
Through actuating an untethered, contact-free tool by means of a magnetic field, our device supports rendering precise haptic attractive and repulsive forces as well as accurate tracking without the need for any external infrastructure or markers. Our system allows for rich interactions with and haptic perception of dynamic virtual surfaces.

Our goal is to enrich AR/VR and other 3D applications via \omniUIST and a minimally instrumented tool. Simplicity of the haptic prop and a walk-up-and-use experience were important design goals of our work. Furthermore, to create rich immersive experiences, such a system must be able to deliver different types of high-fidelity haptic forces and precisely sense user input without requiring any external tracking. Moreover, we aim for a self-contained device that is affordable and easy to manufacture.

The actuation mechanism used in \omniUIST is based on the working principle proposed in \cite{zarate2020contact}. We build up a hemispherical shell base whose core houses a symmetric omnidirectional electromagnetic actuator. Three interwoven and mutually orthogonal coils generate the haptic forces.

By controlling the current in each coil, we can precisely configure the exerted force onto an external magnet, such as the one inside the 3D stylus tool. As the tool approaches the sphere, \omniUIST is able to provide independently controlled radial and tangential forces. Our design contributes several important improvements over \cite{zarate2020contact} that allow us to provide twice the amount of force ($2$ N in either direction) and for much longer periods without suffering from self (over-)heating. %Together, these improvements enable its use under high-stress and continuous usage scenarios in AR/VR.

Although the tool is contact-free, the haptic force that the user perceives has its reaction force on the support base. In this sense, the user perceives grounded forces even if there is no mechanical link to the base.

In our current implementation, \omniUIST rests on a surface (\eg table), though it is compact and could be mounted on a robotic end effector to deliver large-scale 3D feedback. \add{This would allow for haptic feedback in a large volume, which would be beneficial for VR applications. In this case, however, geomagnetism should be taken into account more strongly.}

Beyond the improved actuator, our main contribution lies in the integration of the \omniUIST actuator with a fully self-contained real-time tracking method of the tool. To this end, eight Hall sensors are distributed below the interactive sphere. Each sensor reads a combination of the magnetic field generated by the tool, superimposed by the electromagnetic field generated by the actuator.We use a gradient-based optimization method to locate the magnet's 3D position based on the Hall sensors' readings, running at an interactive rate of 40 Hz.

%Below, we explain the main elements of our system: 1) The system description and our novel computational model; 2) our algorithm for tracking a permanent magnet in 3D space during simultaneous actuation with an electromagnet; and 3) the control of the electromagnet given desired forces onto the permanent magnet. 

\label{sec:method}

In this section, we introduce the method used to conduct the investigation on a set of \pc papers that discuss relevant bias issues.
Specifically, to construct the initial set of relevant work, we search the keywords ``bias" or ``fair" in the title of papers from NeurIPS, ICML, ICLR and FAccT published before February 2025. 
We include papers that discuss bias issues whose manifestation aligns with either Type I Bias or Type II Bias (we will detail the unification in~\cref{sec:unifying}).
We exclude papers that address other bias issues such as inductive bias~\cite{baxter2000model,zietlow2021demystifying}, implicit bias~\cite{fitzgerald2017implicit,camuto2021asymmetric}, selection bias~\cite{hernan2004structural,akbari2021recursive}, sampling bias~\cite{winship1992models,xu2022alleviating}, spectral bias~\cite{fang2024addressing}, exposure bias~\cite{li2024alleviating} or bias-variance~\cite{ha2024fine, chen2024on}.
Furthermore, to ensure we do not overlook any relevant papers without these keywords or from other prominent conferences such as CVPR, ICCV, and ECCV, we manually traversal the citation graph of the paper in the initial set and append the relevant papers that are either cited by or cite the papers in the initial set.






Once we identify the scope of the investigated papers, we read these papers to determine which type of bias they address by examining two aspects: problem statement and evaluation protocol.
We will elaborate on the criterion for categorizing papers into our definitions in~\cref{sec:unifying}.
To accommodate the recent emerging direction of addressing unlabeled and unknown bias, we enrich the taxonomy with an additional dimension about the status of attribute $A$.
As shown in~\cref{tab:taxonomy}, we count the number of papers in each category. 
Note that the total number is not equal to \pc since some papers address both types of biases.
We present the categorization list of all \pc investigated papers in Appendix.


\begin{table}[htbp]
\caption{The taxonomy of bias issues based on \pc papers.}
\label{tab:taxonomy}
\centering
\resizebox{0.45\textwidth}{!}{%

\begin{tabular}{lcccc}
\toprule
\multirow{2}{*}{Type of Bias} & \multicolumn{2}{c}{Attribute $A$} & \multirow{2}{*}{Papers} & \multirow{2}{*}{Examples}                                                   \\
\cmidrule(lr){2-3} 
                              & Known           & Labeled         &                         &                                                                             \\
                              \midrule
\multirow{3}{*}{Type I Bias}  & \cmark          & \cmark          & 253                     & \cite{DebFace,GAC,RL_RBN}                                                   \\
                              & \cmark          & \xmark          & -                       & -                                                                           \\
                              & \xmark          & \xmark          & -                       & -                                                                           \\
                              \midrule
\multirow{3}{*}{Type II Bias} & \cmark          & \cmark          & 246                     & \cite{learn_not_to_learn_Colored_MNIST,CSAD,End}                            \\
                              & \cmark          & \xmark          & 8                       & \cite{HEX_texture_bias1, ReBias_texture_bias2,rubi} \\
                              & \xmark          & \xmark          & 30                      & \cite{LfF_CelebA_Bias_conflicting,ECS,UBNet}                               \\
                              \midrule
Survey                        & -               & -               & 25                       & \cite{MLbias_survey,prediciton_quality_disparity,discussion_on_DP_EO}      \\
\bottomrule
\end{tabular}
}

\end{table}


\section{Evaluation}
\textit{Omni}'s capability of delivering convincing haptic sensations relies on the performance of two main components: tracking of the tool position and in-air actuation.
We performed technical evaluations on both aspects.

In summary, \textit{Omni} is able to reconstruct the position of the tool with an accuracy of $6.9 (\pm 3.2)$ mm and can deliver peak forces of up to $\pm$ 2 N, and 0.615 N continuously.
Besides this technical evaluation, we demonstrate \textit{Omni's} interactive capabilities in the application section.  We refer readers to Zarate \etal\cite{zarate2020contact} for a psychophysics evaluation of a comparable underlying actuation mechanism, showing that users can discriminate at least 25 discrete force locations.

\subsection{Tracking evaluation}
To evaluate \textit{Omni}'s tracking accuracy, we compared our position estimates to those of a 10-camera Optitrack setup, capturing a tracking space of 1.2 $\times$ 0.8 m with submillimeter accuracy at 100 Hz. We configured \textit{Omni} to run in precision mode at a frame rate of 40 Hz. For both tracking methods, we recorded the position and rotation angles.
We evaluated the accuracy of \textit{Omni} in two conditions: \emph{no actuation} and \emph{actuation}. In the \emph{no actuation} condition, no current was sent to the coils. In the \emph{actuation} condition, the coils were actuated using a sawtooth function sending current between -4 and 4 Amperes for each axis. 

For each condition, the pen was moved around the center of \textit{Omni} at a distance of up to 10 cm, covering the area around the device. We collected 1600 samples for the \emph{no actuation} condition and 2600 samples for the \emph{actuation} condition, both at roughly 5 Hz.

\subsubsection{Results}
We found that the average difference between the two tracking systems is $e_{r_p} = 4.9 (\pm 1.8)$ mm in the \emph{no actuation} condition and $e_{r_p} = 6.9 (\pm 3.2)$ mm in the \emph{actuation} condition. Analyzing each axis separately, we found that $\mathbf{e_{r_p}} = [3.4;\ 3.1;\ 2.7]$ mm for the tracking errors and \emph{no actuation} and $\mathbf{e_{r_p}} = [4.4;\ 5.2;\ 3.3]$ mm in the \emph{actuation} condition. The results are summarized in Figure~\ref{fig:optitrack_eval}.

Finally, we tested our formulation with and without the orientation estimation of the magnet.
While we found that including these additional optimization variables improves the accuracy of the position estimates, these estimates are unstable and not yet useful for interactive applications.

Intuitively, it makes sense that including the orientation in the model fitting improves position estimation since the orientation of the magnet does influence the magnetic field. Furthermore, it is known that the model we leverage \cite{yung1998analytic} works best for spherical magnets (\eg point estimates of positions) and hence the models' approximation error may be a source of noise in the orientation estimates. We leave an extension of the reconstruction method to 5-DoF for future work.
\begin{figure}[!t]
\centering
\includegraphics[width=\columnwidth]{\dir/sensing/figures/error-optitrack.pdf}
\caption{Distribution of tracking error with and without current applied to the electromagnet.}
\label{fig:optitrack_eval}
\end{figure}

\subsection{Actuator evaluation}
\textit{Omni}'s 3 DoF spherical electromagnet produces a force on the permanent magnet in the tool by dynamically adjusting the magnetic field through currents in the orthogonal coils.
To quantify this actuation, we measured the radial and tangential forces at different locations around the electromagnet in \textit{Omni}'s spherical base. We placed a 3D-printed hemisphere over the electromagnet (see Figure~\ref{fig:eval_actuator}). The hemisphere has three slots to place a test magnet (\emph{S-30-07-N, Supermagnete}, same as in tool) and two force sensors (\emph{FSAGPNXX1.5LCAC5, Honeywell}). The force sensors were attached between the electromagnet and the test magnet to measure radial force, and to the side of the test magnet to measure tangential forces.
\begin{figure}[!t]
\centering
\includegraphics[width=1\columnwidth]{\dir/sensing/figures/force-test-01.pdf}
\caption{Setup for actuator evaluation. An additional 3D printed hemisphere is placed on top of \textit{Omni} to hold the force sensors.}
\label{fig:eval_actuator}
\vspace{-1em}
\end{figure}

\subsubsection{Results}
We generally observed a linear response of actuation with respect to the applied current.
On top of the electromagnet, we measured a maximum vertical repulsive (radial) force of 1.95 N at $I_z = 14.6$ A and a maximum attractive force of -3.04 N at $I_z = -14.6$ A, shown in Figure~\ref{fig:Fz_vs_iz}.

When \textit{Omni} applies $I_z = +3.7 $ A, it compensates for the weight and snapping and the magnet starts to levitate\footnote{In this paper we use the term \emph{levitation} in the sense of \emph{compensate its weight completely}. A complete levitation would need to control the actuation in all three axes to keep the magnet floating in place.}. Note that at this position, the force is the sum of the electromagnetic actuation, the snapping to the core, and the gravitational attraction.

The weight of the tool produces a force of $F_r = -370$ mN (38 gr), while ferromagnetic snapping yielded additional 170 mN of force, combined these produce an attracting radial force of $F_r = -540$ mN without actuation ($I_z = 0$ A).All those components contribute to users' perception of force.

On top of the sphere, the weight and snapping are orthogonal to the $x$-axis and $y$-axis and do not influence the radial forces along those axes. Consistently, we measured a linear response on those axes of the form $Fr_{x-axis} = 0.122 ~[N/A] ~I_{x}$ and $Fr_{y-axis} = 0.142 ~[N/A] ~I_{y}$. For the other locations in our test setup (horizontal to vertical), we observed forces in the range of $\pm 2$ N at $\pm 15$ A with the corresponding corrections for weight and snapping. Note that the forces have been measured when the tool was in contact with \textit{Omni}'s hemisphere.

The force intensity decays with $1/(d_0 + g)^4$, where $g$ is the air-gap between the tool and the sphere. The parameter $d_0 = 41$ mm is the center-to-center distance between the electromagnet and the permanent magnet when the tool touches the sphere. For example, for $g = 10$ mm, the reachable range of forces drops to $\pm 835$ mN. \add{At 30mm from the hull this reduces to 0.2N}
\begin{figure}[!t]
\centering
\includegraphics[width=1\columnwidth]{\dir/sensing/figures/force-amps-plot.pdf}
\caption{Radial and tangential forces on the permanent magnet as a function of coil actuation $I_x$, $I_y$ and $I_z$, for the magnet located on top of the sphere (Fig~\ref{fig:eval_actuator}). Force was collected with a compression-like force sensor.}
\label{fig:Fz_vs_iz}
\end{figure}

\subsubsection{Evaluation of EM heating}
To test the stability of the generated forces and the thermal capabilities of our system, we ran two experiments. First, we set the $y-$axis coils to maximum actuation current $I_y = 15$ A for 25 seconds and let it cool down afterwards to test the system under \emph{peak-force} conditions. Second, we set the same coil to 1/3 of the maximum actuation and we let it run for 15 minutes, to test under \emph{constant-force} conditions. Figure \ref{fig:heating} shows the evolution of the generated force and the temperature of the coil for both conditions.

During \emph{peak-force}, the system delivers a force of $2.04 \pm 0.04$ N. Starting from room temperature (24 °C), the actuator heats up to 39 °C but only 40 seconds after the actuation has been turned off, showing the system's thermal inertia. The $\Delta T = 15$ °C during this intense actuation peak shows that our system is capable of thermally buffering and dissipating the heat generated by intense forces even during tens of seconds.

In our \emph{constant-force} experiment (Figure \ref{fig:heating}, \textit{bottom}), the force remained constant within the limits $0.615 \pm 0.015$ N and for a duration of 15 min, even when the temperature of the coils (and their resistances) significantly changed.
In addition to compensating for the actuation drifts, we used the coils' resistance changes over time as the limiting factor to avoid overheating of the coils and the 3D printed parts, in case the system is required to apply maximum forces for minutes.
\begin{figure}[!t]
\centering
\includegraphics[width=\columnwidth]{\dir/sensing/figures/force-heat-time-plot.pdf}
\caption{Temporal evolution of the self-heating of the coils for two different types of actuation. Top: a \emph{peak-force} of 2 N ($I_y = 15$ A) during 25 seconds. Bottom: a \emph{constant-force} of 600 mN ($I_y = 5$ A) during 15 minutes.}
\label{fig:heating}
\end{figure}
% !TEX root = ../Main.tex

%%%%\JZ{Hardware evaluation moved into hardware}
\section{Applications}
To further demonstrate the potential of our approach we illustrate possible usage-scenarios including calligraphy, outlining and inking. 
Finally, we combine the haptic feedback mechanism with a simple digital drawing application to initially explore the possibility of dynamic references.    

\subsubsection*{Calligraphy}
\figref{fig:caligraphy} illustrates writing of flourished characters, with only minimal visual guidance (single starting point). 
Our system takes the character as input, users can then draw at their desired speed. 
Although an offset from the reference path remains, the lines are smooth and the overall shape is close to the desired characters. 

\begin{figure}[!t]
    \centering
        \includegraphics[width=\columnwidth]{\dir/figures//apps-03-caligraphy.jpg}    
        \caption{Our approach can be used as a guidance system for calligraphy, where users either follow a target path very closely, or deviate if desired.}
    \label{fig:caligraphy}
\end{figure}
%
%\subsubsection*{Drawing teaching aid:}
%Connect-the-dots exercises are often used to teach children motor skills as well as stroke ordering. \figref{fig:dots} shows results from a similar exercise performed with our system, albeit using much fewer dots for visual guidance than a paper version.
%

\subsubsection*{Outlining \& inking}
\figref{fig:dragon} illustrates the effect of two core capabilities of the proposed approach. 
Here we first outline the proportions of the dragon head (gray guidance lines) and then use different pens to ink-in the details. 
Note that the system provides haptic guidance but allows the user to draw the shape in different styles (\eg the ears of the two upper dragons) and with varying high-frequency detail, while maintaining similarity to the reference shape. 
This is a direct consequence of using time-free closed loop control approach.
In this case, all four variants were drawn without changes to the system or desired path. 
\begin{figure}[!t]
    \centering
        \includegraphics[width=\columnwidth]{\dir/figures//apps-01-drawing.pdf}
    \caption{Different variants of the same dragon, drawn with identical system settings by a novice. Each pair of drawings used with different tools. First a pencil for proportions and a fine-liner (top) or pencil (bottom) to ink-in details. Multi-stroke lines are achieved by approaching each separate instance as a new drawing.}
    \label{fig:dragon}
\end{figure}


\subsubsection*{Virtual tools}
Using a digital tablet with capacitive display (\figref{fig:tablet}) we explore integrating dynamically changing references. 
In a sketching application, artist select different virtual tools, and position and configure these anywhere. 
The canvas and the haptic feedback system then pull the stylus towards these virtual guides. 
In \figref{fig:tablet}, the user has selected a tool that helps them when drawing an ellipse that snaps to a previous part of the drawing, both visually and in terms of haptics.

\begin{figure}[!t]
    \vspace{-.5em}
    \centering
    \includegraphics{\dir/figures//apps-02-digital.jpg}    \caption{Virtual tools can be used to dynamically construct a reference path combining haptic and visual feedback. Here  a simple drawing application combines freeform sketching with different virtual rules and guides that can be felt by the user.}
        \label{fig:tablet}
    \vspace{-.5em}
\end{figure}

% \subsubsection*{Topography Exploration}
% \figref{fig:topo} shows our in system can be used to explore a landscape. In this implementation the user feels drag on an ascending slope and gets pulled forward along a descending slope. For this application the desired path is updated online according to the slope of the terrain. Note, that therefore the path is unknown to the algorithm and changes with every iteration. 

% \subsubsection*{Pong}
% \figref{fig:pong} illustrates how our system can be used in the context of games. In this case, it is the classic Pong. However, it can be extended to a large variety. In this example we pull the player to the same y-position as the current ball position. For this purpose we set $w_{\dot{\theta}}$ to $0$ and thereby eliminate the forward pull. 




\section{Discussion}


In this paper, we adopted a learner-centered design approach, beginning with a formative study to identify students' challenges with existing tools. Based on these insights, we developed DBox, a tool that scaffolds students in breaking problems into smaller parts and provides personalized, adaptive support. Our user study demonstrated that DBox improved learners' performance on similar algorithmic problems, increased perceived learning gains, and fostered greater cognitive engagement, achievement, and satisfaction. In this section, we discuss design implications and generalizability based on our key findings.


\ms{
\subsection{Chaining Learners' Thoughts with Visualized Structured UI Components}

Decomposition requires students to effectively organize their thoughts. While visual elements are known to promote structured thinking and support mental model construction \cite{mcdougall2001effects, liu2010mental}, our formative and user studies revealed shortcomings in existing tools like LeetCode and ChatGPT, which rely on textual representations without adequately supporting structured mental models. In contrast, DBox uses an interactive step tree to visually organize learners' thoughts. This feature was praised by 22 of 24 participants for enhancing algorithmic thinking, serving as a progress tracker, and providing value even without AI assistance.

DBox's interactive step tree and tree-based scaffolding demonstrate the broader potential of intelligent tutoring systems (ITS) to promote active learning and self-regulated problem-solving in fields requiring problem decomposition. Similar principles could benefit STEM education, such as physics or engineering, by externalizing abstract concepts and facilitating multi-step problem-solving. Additionally, progress-tracking visual components may inspire designs for professional training tools in areas like medical diagnostics or software engineering.

\subsection{Promoting Independent Thinking and Active Decomposition Learning}

\subsubsection{\textbf{Transforming Learners from Passive Readers to Active Thinkers}}

Many coding tools provide direct answers or solutions \cite{kazemitabaar2023novices, phung2023generating}, which, while efficient, often bypass opportunities to develop critical problem-solving skills. In contrast, DBox cultivates students' decomposition abilities through structured scaffolding, fostering critical thinking and self-regulated learning in line with learning by doing \cite{anzai1979theory} and constructivist principles \cite{tobias2009constructivist}.

To strengthen decomposition skills, DBox first encourages students to develop their own decomposition strategies by coding or building a step tree from scratch. While DBox can generate parts of a step tree from a student's existing code, these steps are derived from the learner's own reasoning, with DBox acting solely as a modality converter. Besides, DBox provides feedback on tree node statuses, identifying potential errors or missing steps without directly showing the correct answer, challenging students to critically evaluate and refine their decomposition plans.


DBox's scaffolded hint system further supports decomposition skill development by providing adaptive guidance tailored to the student’s progress without overwhelming them. All hints are based on the learner's current decomposition skeleton, with the most detailed hint—``reveal substep''—triggered only after repeated attempts and struggles. Notably, even the most detailed hints prompt only one substep, requiring students to complete the rest independently. As shown in Sec \ref{hintusage}, only 19\% of hints are this detailed, with students primarily relying on simpler, thought-provoking question hints. This scaffolded support system balances guidance and independent thinking, keeping students engaged during challenges without compromising their ability to independently decompose problems \cite{kinnunen2006students}.

Based on these findings, we recommend fostering active problem-solving by shifting students from passive content consumption to active solution creation. Designers could adopt layered scaffolding, starting with minimal guidance and increasing support as needed, to help students progressively master decomposition skills while maintaining confidence and avoiding frustration. Additionally, adaptive learning techniques, such as real-time feedback and progress tracking, can further tailor the support to individual decomposition barriers, encouraging deeper engagement with decomposition tasks. Moreover, designers could integrate metacognitive strategies, such as encouraging students to articulate or reflect on their decomposition approaches, to further enhance critical thinking and foster habits of independent thinking.




\subsubsection{\textbf{Choice of Scaffolding: Balancing Independent Problem-Solving and Efforts}}

Scaffolding involves providing tailored support to help learners accomplish tasks they cannot yet complete independently \cite{kim2011scaffolding, tobias2009constructivist}. Broadly, scaffolding strategies fall into two categories \cite{van2010scaffolding}: (1) gradually reducing assistance as learners gain proficiency, and (2) encouraging independent problem-solving while offering incremental support to address challenges. DBox adopts the second approach, emphasizing independent thinking and encouraging learners to actively decompose problems \cite{zimmerman2013theories}. While our scaffolding strategies successfully enhanced critical thinking, satisfaction, and perceived usefulness, they also led to increased cognitive effort (Sec. \ref{Effects_on_UX}). This tradeoff underscores the importance of carefully balancing cognitive effort with the promotion of independent thinking.

Future designs could incorporate adaptive scaffolding that adjusts support dynamically based on learner proficiency, reducing unnecessary effort in areas where students have demonstrated competence. Additionally, while incremental scaffolding was effective for algorithmic problem-solving, tailoring strategies to different educational contexts could enhance their applicability in diverse domains. Such adaptive, context-specific approaches could further optimize the balance between support and independence in learning environments.


\subsection{Supporting Personalized Algorithmic Programming Learning}

\subsubsection{\textbf{Prioritizing Learners' Own Solutions Over Optimality}}

Algorithmic problems often have multiple solutions with varying time and space complexities. DBox prioritizes independent exploration by supporting learners' strategies rather than steering them toward a single ``optimal'' solution. Using LLM-driven prompts, it evaluates and guides each step based on the learner's reasoning, preserving their step decomposition and respecting their input—even when errors occur. While some solutions may not be the most efficient, this approach fosters autonomy by aligning feedback with learners’ thought processes instead of enforcing rigid standards.

Our user study showed that this approach improves learning outcomes and is well-received by students. We recommend designing systems that respect personalized problem-solving strategies by aligning feedback with learners' reasoning while allowing for diverse approaches. Designers should balance flexibility and rigor, using prompts and interfaces that support varied strategies while gently guiding learners toward effective solutions.


\subsubsection{\textbf{Catering to Individual Learning Styles and Contextual Needs}}

DBox accommodates diverse problem-solving approaches with two input modes: coding and natural language descriptions. Each mode offers distinct advantages tailored to different learners, stages, and situations. Learners can switch seamlessly between modes, with progress automatically synced across the interface. Features such as verifying code-step alignment ensure strong integration between modes.

Our findings reveal that this flexibility enhances user experience. Participant interaction logs and interviews revealed three usage patterns, highlighting that each mode fits different needs: code mode works well for students with a clear and detailed problem-solving plan already, while the step tree with natural language descriptions helps less experienced students with only a basic idea who are not ready to write code directly, boosting their confidence.


We argue there is no universal “best” mode for programming education—each has unique benefits depending on the learner habits, expertise, and context. Future tools should provide flexibility, like DBox, or use adaptive algorithms to recommend modes based on user needs and context. This flexibility highlights the importance of designing educational tools that accommodate varying levels of expertise and problem-solving styles, which can be generalized to other domains requiring personalized learning \cite{bernacki2021systematic}.

\subsection{Appropriate Usage of LLMs for Supporting Algorithmic Programming Learning}

\subsubsection{\textbf{Caution About LLM Errors}}

Although LLMs have shown strong performance in coding tasks \cite{finnie2023my, leinonen2023using}, they remain prone to errors. Our technical evaluation and user study revealed that even with comprehensive context—such as problem statements, user code, and natural language steps—LLM sometimes misinterprets user descriptions. These errors likely arise from discrepancies between the natural language used by students and the formal, precise language the LLM was trained on, which is primarily sourced from web-based code and comments \cite{liu2023wants}.

Such misinterpretations can hinder learning by causing confusion or frustration. While future improvements to training data and GPT versions may mitigate these issues, design strategies can help address them. \textbf{First}, LLMs should avoid giving direct solutions and instead focus on fostering active problem-solving through explanations and hints. \textbf{Second}, feedback could be paired with interactive features, like a ``Run Code'' option, allowing students to validate their reasoning. \textbf{Third}, simple tutorials could teach users how to phrase their descriptions more clearly, improving LLM's understanding. Additionally, future tools could integrate a ``Language Enhancement'' feature to suggest improvements or assess the clarity of descriptions, aiding LLM in accurately capturing user intent. Most importantly, we recommend designers prioritize technical feasibility, such as conducting rigorous evaluations like ours, before fully integrating LLMs into programming learning tools.
}



\subsubsection{\textbf{Learner-LLM Co-Decomposition of Solutions: Learner as Leader, LLM as Aid}}

A central feature of DBox is the construction of a step tree, where students break solutions into steps and sub-steps. The LLM supports this by mapping code to step descriptions, evaluating them, and offering hints. However, students maintain full control, deciding how to decompose problems and define each step, fostering independent thinking. The LLM acts solely as an aid, using a scaffolding approach to support the development of learners' Zone of Proximal Development (ZPD) \cite{chaiklin2003zone}. Unlike tools like ChatGPT or Copilot that dominate problem-solving, DBox fosters deeper cognitive engagement. Students reported greater accomplishment and found this approach more effective for learning.

This contrasts with existing human-AI collaboration paradigms in non-educational scenarios where AI usually suggest options, leaving final decisions to users \cite{dang2023choice, gao2024collabcoder, gebreegziabher2023patat, ma2019smarteye, ma2022glancee}, such as in human-AI decision-making \cite{ma2023should, ma2024towards, ma2024you}. Some educational tools, like Jin et al. \cite{jin2024teach}, use LLMs to generate solutions for students to evaluate, which aids in syntax learning but such ``LLM-generate then learner-evaluate'' approach is less effective for algorithmic problem-solving, where constructing solutions is key. Just evaluating LLM-generated contents can place a cognitive anchor on learners \cite{furnham2011literature}, limiting independent thinking and creativity. Thus, task allocation between humans and AI should align with the educational context (e.g., whether it is basic knowledge/concept learning or higher-level creative thinking). Future LLM-based educational tools should carefully define the division of roles between LLMs and learners, tailoring it to specific learning contexts and goals.




% \subsubsection{Human-LLM Co-Decomposition of Solution: AI Should Judge Instead of Recommending}

% A core interaction in DBox is the construction of a step tree, where the entire solution is broken down into a series of steps and sub-steps. We refer to this as the human-LLM co-decomposition process. In this process, the LLM behind DBox plays three roles: First, it maps the student's written code into step descriptions. Second, it evaluates the status of each step and sub-step (whether they are correct, incorrect, missing, or need further decomposition). Third, it provides hints for incorrect or missing steps or sub-steps. However, the actual construction of the step tree—such as dividing the solution into steps and sub-steps and determining the content of each node—remains primarily the student's responsibility.

% This division of labor maximizes student engagement in independent thinking and problem-solving. The LLM does not provide any suggestions for decomposition nor directly recommend content for specific steps, aligning with the scaffolding educational approach, where guidance is provided appropriately, but the main task of forming the solution is left to the students.

% In contrast, when students directly seek help from an LLM, such as asking questions in ChatGPT or using Copilot for code completion, the LLM takes too much initiative by directly offering ideas or code. In our co-decomposition design, however, students demonstrated higher cognitive engagement and more active critical thinking. Furthermore, students reported that constructing solutions in this way gave them a greater sense of achievement and made them feel the process was more beneficial for learning, leading to higher satisfaction with the experience.

% Related work has proposed similar approaches. For instance, XXX, in the context of problem-solving, uses the "learning by teaching" concept, where students take on the tasks of judging and teaching, while the LLM generates most of the solutions. Compared to our approach, their division of labor between the student and the LLM is reversed. This method works well in introductory programming, where the focus is on mastering syntax. Having students guide the LLM to generate code or evaluate potentially incorrect code produced by the LLM is an effective way to quiz them. However, in our work, which focuses on algorithmic programming, the key step is constructing a solution from scratch. If the LLM builds the solution, leaving students only to judge it, it hampers their independent thinking.

% Thus, when designing LLM-based educational tools in the future, it is crucial to consider the specific context to effectively allocate tasks between the student and the LLM, ensuring that students derive the maximum benefit from the co-decomposition process.


% \subsection{Future Design Opportunities}

% \emph{Providing Appropriate Generative Assistance:} While DBox promotes independent problem-solving, some users showed interest in features like auto-completion for trivial coding tasks. Future versions could balance promoting independence with targeted assistance by enabling adjustable difficulty levels and offering contextual suggestions when appropriate.

% \emph{Covering All Stages of Algorithmic Programming:} DBox currently lacks a focus on foundational algorithm instruction and problem comprehension. Future iterations could include features like generating distractor solutions, input-output tests, and step-by-step rephrasing to help students grasp key concepts and understand the coding problem.

% \emph{Combining Step Trees with Dialogue:} Users can currently describe their thought processes but cannot ask questions. Adding a dialogue system to the step tree would allow students to share challenges and ask follow-up questions. GPT could then provide guided feedback without giving direct answers, supporting independent problem-solving.





% \emph{Other Important Features.} DBox could offer more control by allowing users to select specific parts of their code for targeted evaluation and guidance. A ``review'' feature could also help students reflect on key stumbling points, understand where their thought process went wrong, and how they eventually solved the problem.


% \subsection{Future Design Opportunities}

% \emph{Providing Appropriate Generative Assistance.} Our tool primarily focuses on encouraging users to create the step tree and write the code independently, with the system mainly serving as a judge. However, users expressed a desire for some intelligent completion features, particularly for repetitive or simple code, allowing them to focus their efforts on learning the key parts. Future improvements should strike a balance between fostering independent thinking and providing appropriate assistance. One approach could be designing basic rules where the tool offers intelligent suggestions and completions for parts unrelated to the core logic, while maintaining the current level of independence for key learning areas. Additionally, the system could offer different modes, allowing users to choose the level of assistance, from basic judgment-only feedback to a combination of judgment, guidance, necessary completions, and even on-demand suggestions.

% \emph{Covering All Stages of Algorithmic Programming.} Currently, our system does not cover the basic teaching of algorithms or the problem comprehension stage. In the future, to address the diversity and uncertainty in solutions and help students grasp multiple approaches, we could expand assistance during the idea formation phase. For example, GPT could generate multiple potential solutions with distractors, prompting students to identify the one that meets the problem's complexity requirements. We could also introduce specialized algorithm training, where students select a specific algorithm, and the system’s guidance focuses solely on that algorithm. To assist with problem comprehension, we could incorporate input-output tests to check students' understanding of the problem and step-by-step rephrasing to help them grasp more complex problems.

% \emph{Combining Interactive Step Trees with Dialogue Boxes.} Sometimes users want to describe their difficulties, and currently, we ask them to outline their thought processes. Additionally, users may want to ask follow-up questions. In the future, we could combine the structured step tree with a small dialogue box. The primary goal would still be to construct the step tree, but users could engage in a conversation with GPT in the context of the current step tree or a specific step. Importantly, GPT should guide the user without revealing direct answers.

% \emph{Other Important Features.} First, DBox could offer learners more control, such as allowing users to select specific parts of the code for targeted evaluation and guidance. We could also introduce a summary feature for key stumbling points, helping students reflect on the challenges they faced, where their thought process went wrong, and how they eventually overcame the problem.




\subsection{Limitations and Future Work}

This study has several limitations. \emph{First}, we tested DBox's effectiveness on only two problem types; future work should examine a broader range of algorithms. \emph{Second}, participants engaged in just one learning session per condition due to time constraints, whereas mastering algorithmic problems typically requires extended practice. Longitudinal studies should explore how DBox supports skill development over time, including changes in mental models and skill retention. \emph{Third}, we assessed learning gains based on correctness in a test session using similar learning and test problems. Future research should evaluate knowledge transfer to less similar problems. Due to time constraints, we conducted a single post-test rather than a pre-post comparison. While pre-test expertise filtering and randomization minimized prior familiarity effects, a more rigorous pre-post design would yield more accurate learning gain measurements. Looking ahead, we plan to release DBox as a Chrome plugin for integration with existing coding platforms, enabling large-scale field studies. This will allow for the collection of long-term usage data and periodic surveys to identify usage patterns and learning experiences over time.



% This study has several limitations. First, in our within-subject design, we selected two types of algorithm problems—Greedy and Binary Search—and randomly assigned them to two conditions (DBox and baseline). However, selection bias may still exist, as some participants might naturally excel at one type of algorithm. Although we addressed this by filtering participants' proficiency through a pre-test and using a Latin Square design, further validation across a broader range of algorithms is needed in future work.

% Second, students experienced only one learning session per condition before the test session. While this allowed for a fair comparison, mastering algorithmic problems typically requires extended practice. Future work should explore how DBox supports students' long-term improvement in algorithmic skills. Longitudinal studies could provide insights into changes in learners' mental models, allowing students more time to deepen their understanding and refine their decomposition methods. Additionally, retention tests could assess whether students can still apply learned problem-solving methods after a time gap.

% We measured learning gains through correctness scores in the test session, with relatively similar learning and test problems. Future work should explore students' ability to transfer their knowledge to problems with lower similarity. Due to time constraints, we opted for a single post-test rather than a pre-post comparison. While we minimized prior familiarity effects by filtering participants and randomizing problem assignments, future studies could adopt a more rigorous pre-post test design for better measurement of learning gains.

% Looking ahead, we plan to release DBox as a Chrome plugin for integration with existing online coding platforms and large-scale real-world testing. In such settings, where students may be more motivated (e.g., preparing for algorithm interviews), we can gather long-term usage data while ensuring privacy. We also plan to conduct periodic surveys to track changes in students' usage patterns and learning experiences over time.



% \subsection{Limitations and Future Work}

% This study has several limitations. First, in our within-subjects study, we selected two types of algorithm problems, Greedy and Binary Search, and randomly assigned them to two conditions, DBox and the baseline. However, there may still be selection bias, where some participants were naturally better at one type of algorithm. While we mitigated this issue to a large extent by filtering participants' proficiency through a pre-test and employing a Latin Square design to randomize the problem-condition assignment, there is still room for improvement. Future work should validate DBox's effectiveness across a broader range of problem types.

% Second, in our experiment, students only experienced one learning session in each condition before moving on to the test session. Although this comparison was fair (as both conditions had only one learning session), mastering an algorithmic problem often requires extended practice. Future work should explore how DBox can help students gradually improve their algorithmic programming skills over time. Longitudinal studies may reveal significant changes in learners' mental models, providing more time for them to understand a specific algorithm and enhance their decomposition methods. Additionally, future studies could include retention tests to measure whether students can still effectively apply previously learned problem-solving methods after a period of time.

% Furthermore, when objectively measuring students' learning gains, we calculated their correctness score in the test session. On the one hand, the learning session and test session problems had a relatively high degree of similarity. Future work should investigate whether students can transfer what they have learned to solve problems of the same algorithm type with lower similarity. On the other hand, due to time constraints, we did not include a pre-post test comparison, opting for a single post-test instead. This result might be influenced by students' pre-existing familiarity with the problems. Although we mitigated this issue by filtering for familiarity (ensuring participants were not too familiar with the problems) and randomizing the problem assignments, future work could include a more rigorous pre-post test design to better calculate students' learning gains.

% Moreover, DBox is currently only applied in algorithmic programming, specifically solving algorithm problems. However, this decomposition-based computational thinking approach could be extended to other learning scenarios, such as project-based learning. Future work could explore how to adapt DBox to broader educational contexts outside of algorithmic programming.

% Looking forward, we aim to deploy DBox in real-world algorithm courses. Since algorithms are a core required subject in undergraduate computer science curricula, we hope to investigate how students who have just learned algorithm concepts use DBox to develop their problem-solving skills. Additionally, we plan to convert DBox into a Chrome plugin and release it in the Chrome Web Store for real-world testing. This would allow DBox to seamlessly integrate with existing online coding platforms, enabling large-scale experiments. In such settings, students' motivation may be stronger (e.g., a graduate preparing for an algorithm interview), leading to more realistic usage patterns. Students could use DBox to tackle a wide variety of algorithm problems. We hope to collect long-term (e.g., six-month) usage data from real-world users while ensuring privacy, and use periodic surveys to capture changes in students' usage patterns and learning experiences over time.





\section{Conclusion}
% In this paper, we introduced Decomposition Box (DBox), a novel tool designed to scaffold learners in decomposing problems during algorithmic programming learning. Based on insights from a formative study, we identified key design goals to address the limitations of existing tools in algorithmic programming education. DBox supports two critical stages of the programming process: idea formation and idea implementation. By offering two modes (code mode and language mode), it encourages users to independently develop their solution strategies. The interactive, visual step tree helps students break down problems and build a structured mental model. DBox provides fine-grained, step-level feedback, enabling students to quickly identify issues, while its multi-level guidance offers targeted support without undermining independent thinking.

% Our user study demonstrated that DBox led to significantly higher learning gains, cognitive engagement, and critical thinking. Students reported a stronger sense of achievement and found the assistance both appropriate and effective for their learning. We identified three main usage patterns, underscoring the importance of respecting students' problem-solving habits and offering them autonomy. The learner-LLM co-decomposition model we designed promotes independent thinking while allowing the LLM to contribute meaningfully, even with occasional imperfections. 

% We hope the formative study, design goals, features, technical evaluation, and key findings from this work will inspire future research on developing educational tools for broader programming learning.
In this paper, we introduced DBox, an interactive tool designed to help learners decompose algorithmic programming problems by supporting both solution formation and implementation. Featuring an intuitive tree-like box widget, DBox accepts input in both code and natural language, fostering independent problem-solving while its step tree structure helps learners develop structured mental models. It provides step-level feedback and layered guidance without compromising learner autonomy.
Our user study showed that DBox significantly improved learning outcomes, cognitive engagement, and critical thinking, with students reporting a greater sense of achievement and finding the support highly effective. Additionally, we identified three key usage patterns, highlighting the importance of accommodating individual problem-solving styles. Moreover, our findings suggest that the learner-LLM co-decomposition approach fosters independent thinking while providing meaningful guidance, even with occasional imperfections.
We hope the insights from our system design will inspire future research on integrating LLMs into educational tools for programming learning.



In this study, we performed the first large-scale analysis of data leakage across 83 software engineering (SE) benchmarks, covering three popular programming languages—Python, Java, and C/C++. By combining an efficient near-duplicate detection algorithm with extensive manual labeling, we ensured the accurate identification of leaked data.



Our findings show that while data leakage is generally low, with average leakage ratios of 4.8\%, 2.8\%, and 0.7\% for Python, Java, and C/C++ benchmarks respectively, some benchmarks exhibit higher leakage that requires attention. We identified four main causes of leakage: direct inclusion of benchmark data in pre-training datasets, overlap between source repositories, reliance on platforms like LeetCode, and shared data sources such as GitHub issues.
We also found that automatic detection methods, like Perplexity-based metrics, struggle to distinguish between leaked and non-leaked samples. Additionally, our experiments reveal that data leakage inflates evaluation metrics, with models performing significantly better on leaked samples. For instance, StarCoder-7b achieved a Pass@1 score 4.9 times higher on leaked samples, underlining the need to address leakage to ensure fair evaluations.
This study offers insights into data leakage status in SE benchmarks and its impact on LLM evaluation.


In the future, we aim to expand the analysis to additional benchmarks and explore new methods to prevent or further reduce data leakage.





\vspace{0.2cm}
\noindent \textbf{Acknowledgement.}  This research / project is supported by the National Research Foundation, under its Investigatorship Grant (NRF-NRFI08-2022-0002). Any opinions, findings and conclusions or recommendations expressed in this material are those of the author(s) and do not reflect the views of National Research Foundation, Singapore.

