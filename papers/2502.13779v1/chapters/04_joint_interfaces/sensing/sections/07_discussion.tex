\section{Discussion}
\omniUIST is capable of tracking a passive tool with an accuracy of roughly 6.9 mm and, at the same time, deliver a maximum force of up to 2 N to the tool. This is enabled by our novel gradient-based approach in 3D position reconstruction that accounts for the force exerted by the electromagnet. 

Over extended periods of time, \omniUIST can comfortably produce a force of 0.615 N without the risk of overheating. In our applications, we show that \omniUIST has the potential for a wide range of usage scenarios, specifically to enrich AR and VR interactions.

\omniUIST is, however, not limited to spatial applications. We believe that \omniUIST can be a valuable addition to desktop interfaces, e.g., navigating through video editing tools or gaming. We plan to broaden \omniUIST's usage scenarios in the future.

The overall tracking performance of \omniUIST suffices for interactive applications such as the ones shown in this paper. The accuracy could be improved by adding more Hall sensors, or optimizing their placement further (e.g., placing them on the outer hull of the device).
Furthermore, a spherical tip on the passive tool that more closely resembles the dipole in our magnetic model could further improve \omniUIST's accuracy. We believe, however, that the design of \omniUIST represents a good balance of cost and complexity of manufacturing, and accuracy.

Our current implementation of \omniUIST and the accompanying tracking and actuation algorithms assumes the presence of a single passive tool. Our method, however, potentially generalizes to tracking multiple passive tools by accounting for the presence of multiple permanent magnets. This poses another interesting challenge: the magnets of multiple tools will interact with each other, i.e., attract and repel each other.The electromagnet will also jointly interact with those tools, leading to challenges in terms of computation and convergence. We believe that our gradient-based optimization can account for such interactions and plan to investigate this in the future.

In developing and testing our applications, we found that \omniUIST's current frame rate of 40 Hz suffices for many interactive scenarios. The frame rate is a trade-off between speed and accuracy. In our tests, decreasing the desired accuracy in our optimization doubled the frame rate, while resulting in errors in the 3D position estimation of more than 1 cm, however. Finding the sweet spot for this trade-off depends on the application. While our applications worked well with 40 Hz and the current accuracy, more intricate actions such as high-precision sculpting might benefit from higher frame rates \textit{and} precision.
Reducing the latency of several system components (e.g., sensor latency, convergence time) is another interesting direction of future research. 

Furthermore, the control strategy we used was fairly naïve, as it only takes the current tool position into account. A model predictive strategy could account for future states, user intent, and optimize to reduce heating. We will explore in the next chapter how model predictive approaches can be used for haptic systems.

Overall, the main benefits of \omniUIST lie in the high accuracy and large force it can produce. It does so without mechanically moving parts, which would be subject to wear.
Such wear is not the case for our device, because it is exclusively based on electromagnetic force. We believe that different form factors of \omniUIST (e.g., body-mounted, larger size) can present interesting directions of future research. \add{A body-mounted version could be interesting for VR applications in which the user moves in 3D space. The larger size could result in more discernible points.}

Additionally, the influence of strength on user perception and factors such as just-noticeable-difference will allow us to characterize the benefits and challenges of \omniUIST, and electromagnetic haptic devices in general.
We believe that \omniUIST opens interesting directions for future research in terms of novel devices, and magnetic actuation and tracking.