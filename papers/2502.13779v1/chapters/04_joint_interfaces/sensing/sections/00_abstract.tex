\begin{abstract}
% JZ:
% - method for tracking magnets in the 3D space
% - 
% 
% In this paper we propose a gradient-based method to track a permanent magnet in 3D space, while simultaneously exerting electromagnetic forces onto the permanent magnet. Being able to track the magnet, embedded in a tool, enables more compact, accessible haptic feedback devices that are self-contained. As a proof of concept a circular array of hall sensors and a spherical electromagnet are combined to create a novel haptic feedback device. This device is thoroughly technically evaluated from both an actuating as well as sensing perspective. We find that we actuate with a force of at least ..N on ... different points in a shell above the electromagnet. In the same shell, we can track the pen with an accuracy of ...mm, this equates in ... number of zones in which we can disinguish the pen position. Finally, we show the merit of our method by showcasing a variety of applications, including exploring the viscosity of liquids, designing 3d shapes and playing games.  



% In this paper, we proposed a self-contained haptic feedback system that is capable of sensing, and actuating on, a permanent magnet.  
% Being able to track a permanent magnet in 3D space, while simultaneously exerting an electromagnetic force, has been a long-standing problem. We propose a gradient-based tracking method, in combination with eight affordable off-the-shelf hall sensors, to sense the position of a permanent magnet embedded in 3D printed tool. The hall sensors are located in a circular array around a spherical electromagnet. The electromagnet is capable of exerting significant forces with a large accuracy in a volumetric space around the system. Since the system is self-contained, and therefore not reliant on external systems,  it is easy and affordable to fabricate, making it accessible to a large audience. While providing haptic feedback, the tracking is accurate enough for most use-cases. Next to a thorough technical evaluation, we demonstrate a large variety of applications.

% #######################
% ABSTRACT OTMAR & CHRISTIAN, MERGED BY DL
% ####################### 



% We present Omni, a self-contained 3D haptic feedback system that is capable of \emph{sensing} and \emph{actuating} an untethered, passive tool containing only a small embedded permanent magnet.
% %
% Omni enriches VR, AR and desktop applications by providing an active haptic experience using a simple apparatus centered around an electromagnetic base. 
% %
% The spatial haptic capabilities of Omni are enabled by a novel gradient-based method to estimate the 3D position of the permanent in midair through eight small off-the-shelf hall sensors that are integrated into the base.
% %
% Omni's 3 DoF spherical electromagnet \textit{simultaneously} exerts dynamic and precise radial and tangential forces in a volumetric space around the device. 
% %
% %%%Notably, the amount of exerted force depends solely on the volume of the magnet inserted in the tool, and not its distance to the base of the device. 
% %
% Since our system is fully integrated and requires no external tracking, it is easy and affordable to fabricate, and because it contains no moving parts it can be very robust.
% %
% We describe Omni's implementation, its software control, and evaluate its tracking and actuation quality, as well as a series of use-cases and applications demonstrating its capabilities.
%
% #######################
% ABSTRACT OTMAR
% ####################### 

% In this paper, we propose a self-contained haptic feedback system that is capable of \emph{sensing} and \emph{actuating} an untethered, passive  tool containing only a permanent magnet, without requiring any external tracking equipment and thus admitting for a compact form-factor.  
% %
% The system can sense rich interaction and provide haptic feedback in the context of VR, AR or desktop applications in order to increase immersion.
% %
% To locate the handheld tool, the measurements of eight off-the-shelf hall sensors are integrated in a gradient-based optimization scheme to accurately track the position of a permanent magnet that moves in the midair.
% %
% In turn, a 3 DoF spherical electromagnet provides precise and dynamic force feedback in a volumetric space around the device. 
% The amount of exerted force depends on the volume of the magnet inserted in the tool, for which we provide thorough analysis and design guidelines.
% %
% Since the system is self-contained, and therefore not reliant on external systems, it is easy and affordable to fabricate, making it accessible to a broad audience. 
% %
% While providing haptic feedback, the tracking is fast and accurate enough for most use-cases. Next to a thorough technical evaluation, we demonstrate a large variety of applications.



% #######################
% ABSTRACT CHRISTIAN
% #######################

% We present NAME, a self-contained haptic feedback system that can track and actuate freely movable tools around it in 3D aided by a small embedded permamagnet.
% NAME complements VR, AR and desktop applications with convincing haptic experiences using a simple apparatus centered around an electromagnet base.
% What enables NAME's spatial haptic capabilities is our novel gradient-based method to estimate the 3D position of the permanent in midair through eight small hall sensors.
% % CH: is it really force feedback or actually an actuation? E.g., is it in response to an input force? I think it's even more general than just 'pushing back'
% Through the same modality, NAME's 3 DoF spherical electromagnet exerts dynamic and precise radial and tangential forces that, notably, are independent of the permamagnet's distance up to a certain boundary. 
% Our system is fully integrated and requires no external tracking, which makes fabrication quick and affordable and will allow NAME to be an accessible haptic feedback apparatus for a broad audience.

% We describe NAME's implementation, its software control, and evaluate the tracking and actuation qualitney depending on used magnet volumes.
% For example, NAME can track a (4\,mm)$^3$ permamagnet accurately within $\pm$ X\,mm accuracy at X\,fps. 
% We demonstrate a series of use-cases of our system and conclude with design guidelines for future applications.

\end{abstract}