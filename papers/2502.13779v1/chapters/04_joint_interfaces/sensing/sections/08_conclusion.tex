% \section{Conclusion}

In this chapter, we introduced \omniUIST, a novel electromagnetic platform that simultaneously tracks and actuates a permanent magnet in the space around it. Our self-contained base assembly integrates 3D magnetic sensing using Hall sensors and magnetic actuation through radial and tangential forces produced by three orthogonal electromagnetic coils within a single sphere.

Our core contribution lies in decomposing the natural interference caused by simultaneous magnetic tracking and actuation. This is enabled by our novel gradient-based optimization method, which minimizes the difference between estimated and observed magnetic fields. This approach affords 3D tracking capabilities with a mean error of 6.9 mm during actuation forces of up to 2 N.

\omniUIST's capabilities allow spatial interaction systems to integrate 3D tracking and actuation of untethered, free-ranging tools simply by embedding a small permanent magnet. We demonstrated a series of example applications leveraging \omniUIST's capabilities, showcasing its potential in various interactive scenarios.

However, an open question remains on how to control the system to enable user autonomy rather than guidance. Addressing this challenge will be a key focus for the next part of this dissertation, aiming to further enhance the usability and functionality of haptic devices in practical applications.
