\section{Method}
\subsection{Hardware}

\begin{figure}
    \centering
    \includegraphics[width=\columnwidth]{figures/hardware_try.png}
    \caption{Hardware Overview}
    \label{fig:hardware}
\end{figure}
electronics, size, FEM, informed design, 


\begin{figure*}[!h]
    \centering
    \includegraphics[width=\textwidth]{figures/hardware2.pdf}
    \caption{Hardware Overview}
    \label{fig:hardware}
\end{figure*}

\subsection{Input Sensing}
\begin{figure}
    \centering
    \includegraphics{sections/coordinates.pdf}
    \caption{Coordinate System Used}
    \label{fig:coordinates}
\end{figure}

In general, it is straight-forward to compute the magnetic field at an arbitrary position $\mathbf{B}(\mathbf{r})$:
\begin{equation}
    \mathbf {B} (\mathbf {r} )={\frac {\mu _{0}}{4\pi }}{\frac {3\mathbf {\hat {r}} (\mathbf {\hat {r}} \cdot \mathbf {m} )-\mathbf {m} }{|\mathbf {r} |^{3}}}.
    \label{eq:basic_B}
\end{equation}
given that $r$ is the vector from the center of the magnet to the arbitrary location, $\mu_0$ is the relative permeability of a vacuum and $m$ is the magnetic moment we can be expressed for both permanent magnets as well as electromagnets. 

However, it is impossible to inverse this function. Making a closed-form solution to find a magnet position (expressed in $r$) given $\mathbf{B}(\mathbf{r})$ and $\mathbf{m}$ intractable. However, we are interested in finding position $\mathbf{P}$ of a permanent magnet, a further complication is that the actuation of our spherical electromagnet also influences the magnetic field. 

In order to solve this problem we place 16 hall sensors around the input area and try to minimize the following cost-term:
\begin{equation}
    \arg\underset{\mathbf{P}, \mathbf{I}}{\min} \|\sum_i \mathbf{w}_{i} \left[(\mathbf{B}_{p,i}(\mathbf{P})+\mathbf{B}_{em,i}(\mathbf{I}) + \mathbf{B}_{n,i}) - \Tilde{\mathbf{B}}_i\right]^2 \|
    \label{eq:eneregy}
\end{equation}

\begin{equation}
    \mathbf{w}_{i} = (\Tilde{\mathbf{B}}_i/ \Tilde{\mathbf{B}}_{max})^2 \ \  
                \text{if}\ \ (\Tilde{\mathbf{B}}_i/ \Tilde{\mathbf{B}}_{max})^2 < 1,\ \ \text{else}\ \ 0
\end{equation}

where $\mathbf{B}_i$ is the magnetic field measurement in $x,y,z$-direction at hall sensor $i$. Specifically, $\mathbf{B}_p$ is the magnetic field caused by the permanent magnet in the pen as a function of its position $\mathbf{P}$ (defined as a 3-tuple: $(x,y,z)$), $\mathbf{B}_em$ by the electromagnet as a function of its input current $\mathbf{I}$, $\mathbf{B}_n$ is the noise in the environment. Given that magnetic fields obey the principle of super-position we can simply sum them. Then we try to minimize the difference between the hypothetical measurements and the actual measurements $\Tilde{\mathbf{B}}$ for each hall sensor. The hypothetical readings are based on Equation \ref{eq:basic_B}. 

We make several assumptions here. The first assumption is that all magnets behave as a dipole. The second assumption is that the tool, and therefore the permanent magnet, always points towards the center of the electromagnet. From the latter assumption it follows that the orientation of the permanent magnetic in spherical coordinates can be expressed as:

\begin{align}
    \theta &= \arctan \left( \frac{\sqrt{x^2+y^2}}{z}\right)\\
    \phi &= \arccos \left( \frac{x}{\sqrt{x^2+y^2}} \right)
\end{align}
from this we can define $\mathbf{m}_p$, which is the magnetic moment of the permanent magnetic:
\begin{align}
    \mathbf{m} &= m * \begin{bmatrix}\sin(\theta)*\cos(\phi)\\\sin(\theta)\sin(\phi)\\\cos(\theta)\end{bmatrix}\\
    m &= \frac{1}{\mu_0}*Br*V
\end{align}
where, $V$ is the volume of the permanent magnet and $Br$ is the magnetic flux density in Teslas. Given a 3-tuple for the location of the hall sensor $\mathbf{s} := \{s_x, s_y, s_z\}$ we can define $\mathbf{r}_p$:
\begin{equation}
    \mathbf{r}_p = \mathbf{s} - \mathbf{P}
\end{equation}

Given Equation \ref{eq:basic_B} and our definitions of $\mathbf{m}_p$ and $\mathbf{r}_p$ it is now trivial to compute $\mathbf{B}_p$ for each hall sensor:

\begin{equation}
        \mathbf {B}_p (\mathbf {r}_p )={\frac {\mu _{0}}{4\pi }}{\frac {3\mathbf {\hat {r}_p} (\mathbf {\hat {r}}_p \cdot \mathbf {m}_p )-\mathbf {m}_p }{|\mathbf {r}_p |^{3}}}.
        \label{eq:Bp}
\end{equation}

We can do a similar calculation for $\mathbf{B}_{em}$, since this is magnet is spherical the orientation does not matter, and since the magnet is centered around the origin of our global coordinate system $\mathbf{r}_em = \mathbf{s}$. Since we build the spherical electromagnet ourselves there are bound to be impurities and inconsistencies. Therefore, we refrain from using a theoretical model to calculate $\mathbf{m}_{em}$ and rely on a calibration matrix $C \in \mathbb{R}^{3\times3}$, which we assume linearly scales with the input current $\mathbf{I}$ so that:

\begin{equation}\label{eq:me_vs_I}
    \mathbf{m}_e =  \frac{2*\pi*I_t^{(s)}*\mathbf{d}^3}{\mathbf{c} *\mu_0}
\end{equation}

therefore,
\begin{equation}
        \mathbf {B}_{em} (\mathbf {r}_{em} )={\frac {\mu _{0}}{4\pi }}{\frac {3\mathbf {\hat {r}_{em}} (\mathbf {\hat {r}}_{em} \cdot \mathbf {m}_{em} )-\mathbf {m}_{em} }{|\mathbf {r}_{em} |^{3}}}.
        \label{eq:Be}
\end{equation}
Finally, $\mathbf{B}_n$ simply comes from a calibration and remains a constant throughout a session.

We minimize Equation \ref{eq:eneregy} in a gradient approach, where we update the six free variables found in the position of the permanent magnet and the current input in the electromagnet. In theory it is possible to directly use the input current and/or the measured current for the latter. However, in practice this infeasible due to latency and thermal drift (which alters the resistance of the coils and therefore the current supplied). Our optimization approach can be found in algorithm \ref{algo:postion}. We use pytorch due to its autograd and ready-implemented optimization schemes, which reduce overhead. Specifically we use Limited-memory BFGS, which is a quasi-newton method. It is an approximation of the BFGS, which works well on non-smooth optimization instances. 

\begin{algorithm}
\SetAlgoLined
\SetKwInOut{Output}{\textbf{Output}}
\SetKwInOut{Input}{\textbf{Input}}
\KwResult{$x,y,z$-position of permanent magnet, $x,y,z$-current of the electromagnet}

 $\mathbf{P} \leftarrow [0,0,h]$\tcp*{$h$ = distance from p to em}
 $\mathbf{I} \leftarrow [0,0,0]$\;
 $\mathbf{B}_n \leftarrow$ Calibration\;
 Loss() $\leftarrow \| (\mathbf{B}_{p}(\mathbf{P})+\mathbf{B}_{em}(\mathbf{I}) + \mathbf{B}_{n}) - \Tilde{\mathbf{B}}\|^2$\;
 \While{interaction session}{
  \Input{$\mathbf{I}_{\text{measured}}$}
  \Input{$\Tilde{\mathbf{B}}$}
  LBFGS.init($\mathbf{P}, \mathbf{I}_{\text{measured}}$)\;
  $\mathbf{P}, \mathbf{I} \leftarrow$ LBFGS.optim$(\Tilde{\mathbf{B}}, \mathbf{B}_n, \mathbf{B}_p(\mathbf{P}), \mathbf{B}_{em}(\mathbf{I}), Loss())$\;
 \Output{$\mathbf{P}$}
 }
 \caption{Permanent Magnet Position}
 \label{algo:postion}
\end{algorithm}

\begin{figure}
    \centering
    \def\svgwidth{\columnwidth}
    \includegraphics{figures/cost.pdf}
    \caption{Visual representation of the solver results. X,Y-axis are spherical coordinates, for representation purposes we omitted $r$. The Z-axis is the loss. This is one solver update, which is done in roughly 40ms and 20 iterations.}
    \label{fig:cost}
\end{figure}

Following the dipole-dipole assumption it is also straightforward to compute the forces exerted on the tool, in the coordinates of the tool:

\begin{eqnarray}
    \mathbf{F_{r}} &=& - \frac{3 \mu_0 \ m_{e} \ m_{p}}{2 \pi \ d^4} \cos(\alpha) \ \mathbf{e_{r}} \label{eq:Fr}\ ,\\ 
    \mathbf{F_{t}} &=& - \frac{3 \mu_0 \ m_{e} \ m_{p}}{4 \pi \ d^4} \sin(\alpha) \ \mathbf{e_{t}} \label{eq:Ft}\ ,
\end{eqnarray}
where $\me=|\meBold|$, $\mpp=|\mpBold|$. In the above expression, $\FrBold$ is the force in the radial direction $\RpenBold = d \ \er$ from the center of the device to the tool. Likewise, $\FtBold$ is the force in the tangential direction $\et$ that tends to align the location of the two dipoles along $\er$. Given this formulation there are several interesting observations to make. First of all, $\FrBold$ and $\FtBold$ are attractive when $\alpha < \pi / 2$. Secondly the forces decline rapidly with the magnet-magnet distance: $1/d^4$. Given this large decay, actuation forces are limited to a shell around the device of roughly $d_{min} \leq d \lesssim d_{min} + 4cm$. $d_{min}=41mm$, since the radius of the sphere enclosure is $35.5mm$, the bottom layer in the tool is $2mm$ and the total volume height is $7mm$. 

\subsection{Calibration}
\begin{figure}
    \centering
    \includegraphics[width=\columnwidth]{figures/calibrator.pdf}
    \caption{The calibration points used. The points are mirrored on the occluded side of the device. The total number of points is 36 of which 4 were discarded due to saturing an axis on a sensor. The device was a 3D printed shell, with holes at the highlighted locations.}
    \label{fig:my_label}
\end{figure}

In order for the cost term (Equation \ref{eq:eneregy} to work. The real world and model world hall sensor coordinates need to be sufficiently close. To answer the question: "how close is close enough?" we ran an experiment in which we evaluated the error of Algorithm \ref{algo:postion} over different errors in sensor position. Details can be found in Appendix \ref{appendix:analyse:calibration}. The result, shown in figure \ref{fig:calibration_accuracy_needed} seems to suggest that we need to know the position with less than 1mm accuracy. In order to achieve this, next to well designed hardware we need a calibration procedure. 

\begin{figure}
    \centering
    \includegraphics[width=\columnwidth]{figures/calibration_needed.pdf}
    \caption{Analyses on how precise we need to know the hall sensor location in 3D space. for different noise levels on sensor readings with regard to solve speed and accuracy. We use Alg. \ref{algo:postion} in simulation, however to the sensor reading $\mathbf{\Tilde{B}}$ are taken from a different initialization on sensor location in the model.}
    \label{fig:calibration_accuracy_needed}
\end{figure}

Another factor to consider is the gain of each sensor. This is the factor with which a reading needs to be multiplied, to get the real world value. The gain can be different between sensors, and also between axis on a sensor.

Our calibration procedure is similar, and based on the same foundation ,as the algorithm and physics we use for the tracking of a permanent magnet. In the tracking case, the sensor position is known and the magnet position is regressed to, in the calibration this is vice versa: the magnet position is known, but the sensors are not. Another difference is that we use ADAM to do update steps on our sensor positions, rather than the LBFGS algorithm. This is because solve time does not matter for calibration, and ADAM provided empirically better results.

We used 32 points on the surface of the sphere as calibration points. Theoretical a single point would suffice, however this decreases the noise in placement and in practice yields better results. For each magnet position we took a 1000 samples. From these samples we discarded the top and bottom 5\% and then used the mean of this sample, this was done per axis per sensor. This is done without the core present. 

\begin{algorithm}
\SetAlgoLined
\SetKwInOut{Output}{\textbf{Output}}
\SetKwInOut{Input}{\textbf{Input}}
\SetKwFunction{Alternate}{\textbf{Alternate}}
\KwResult{calibrated $x,y,z$-position of hall sensors}

 $\mathbf{S}\leftarrow \mathbb{R}^{n \times 3}$\tcp*{Initial estimate of hall sensor locations.}
 $\mathbf{C}\leftarrow x,y,z\text{-calibration positions} $\;
 $\mathbf{B}\leftarrow \text{sensor readings}$\;
 
 $\text{optimizer} \leftarrow \text{ADAM}$\;
 \While{Not Converged}{
  $\text{optimizer.params} \leftarrow \mathbf{S}$\;

     \While{Not Converged}{
        $loss \leftarrow 0$\;
        \For{$\mathbf{c},\mathbf{b} \in \text{zip}(\mathbf{C}, \mathbf{B}$)}{
            $\mathbf{b}_{\text{model}} \leftarrow \text{model}(\mathbf{c})$\;
            $loss += sum((\mathbf{b}_{\text{model}} - \mathbf{b})^2)$
        }
        $\text{loss.backprop()}$\;
        $\mathbf{S} \leftarrow \text{optimizer.step()}$\tcp*{take an update step on sensor location}
     }
 }
 \Output{$\mathbf{S}$}
 \caption{Hall Sensor Calibration}
 \label{algo:hall_Calibration}
\end{algorithm}

\subsection{Control}

From Eq \ref{eq:me_vs_I} follows:
\begin{equation}\label{eq:I_vs_me}
    \mathbf{I} = \mathbf{c} * \frac{\mathbf{m}_e * \mu_0}{2*\pi*\mathbf{d}^3} \quad ,
\end{equation}

\noindent here $\mathbf{c}$ is a constant coming from a calibration procedure; that, with the help of five hall sensors, maps input current to $\mathbf{m}_e$. $\mu_0=4*\pi*10^{-7}$ is the relative permeability of air and $d$ is the distance from the core to the hall sensors used for calibration ($0.055$ meter)


The magnetic field, and therefore the perceived force, follow Biot-Savart's law. This means that the magnetic is directly proportional to the current. We control the current Ohms Law:
\begin{equation}
    I = \frac{V}{R}
\end{equation}
If we know the resistance of the wire and, with Pulse-Width Modulation (PWM), control the voltage the current output directly follow.

Since we are using large amounts of current the whole system gets hot. This heat changes the resistance in the coils and we need to adapt our method accordingly. Previous worked use a PID controller to compensate for this overtime, however tuning PID controllers is cumbersome and even then would either lag in current supplied or oscillate. 

In order to overcome the unknown resitance of the coils we use sensors that can measure both current and voltage (INA260), again using Ohms law (and rewriting it), the resitance directly follows from both measurements. There we have an almost instantanously updating resistance of the coils. 
