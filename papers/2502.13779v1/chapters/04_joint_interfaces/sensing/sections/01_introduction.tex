\section{Introduction}
This chapter builds on \omniHapTitle (\chapref{ch:shared:contact}), by presenting \omniUIST, a self-contained 3D haptic feedback system. We enhance \omniHap in two significant ways: first, by improving the actuator to enable increased peak and continuous forces, and second, by integrating sensing to overcome the limitations of expensive and cumbersome external tracking.

One of the core challenges in enabling high-fidelity free-space haptic interactions is to reliably track the input device in space while \emph{simultaneously} exerting forces. While optical tracking systems can be used for 3D localization, they require significant instrumentation of the user's environment, making them impractical for real-world deployments.

To address this challenge, we propose \omniUIST, an integrated system that can locate the tool in free space and deliver finely controllable haptic feedback via the same modality. Our system consists of an omnidirectional spherical electromagnet that delivers attractive and repulsive forces in radial and tangential directions onto a small, handheld magnet (\figref{fig:teaser_sensing}).

To track the tool's location, we leverage eight integrated Hall sensors positioned in two separate XY-planes relative to the electromagnet's coils. These sensor readings allow us to reconstruct the tool's position based on the distortion of the magnetic field caused by the tool's motion. Unlike previous approaches that rely on passive magnets \cite{ashbrook2011nenya, liang2012gausssense}, our setting is more challenging due to the use of an active electromagnet. The electromagnetic actuation and sensing interfere, creating a superimposed magnetic field.

To decompose this superposition of magnetic fields, we introduce a novel 3D positioning algorithm. Our formulation reconstructs the magnet's 3D position by minimizing the residual between the predicted magnetic field strength at a given location and the actual measurements from the Hall sensors. This approach is amenable to gradient-based optimization, and we demonstrate that the problem can be solved in real-time using a quasi-Newton solver, achieving an accuracy of 6.9mm after a one-shot calibration procedure.

We demonstrate the capabilities of our approach through a series of interactive use cases, some shown in \figref{fig:usecases}, leveraging the ability to track the handheld tool in 3D while simultaneously delivering dynamically adjustable haptic feedback. The magnet fits inside passive tools, such as a 3D-printed stylus, supporting fine-grained interaction tasks and alternative uses as a joystick. We showcase applications in gaming, accessibility, mixed reality, and 3D CAD design that utilize \omniUIST's 3D tracking and actuation.

We detail our software and hardware implementation and thoroughly assess \omniUIST's sensing and actuation capabilities. Our findings indicate that \omniUIST can continuously deliver up to 0.6N in any direction near the sphere, with the electromagnetic coils heating up to a sustainable 47°C. The system's peak force reaches 1.8N for repelling and -3.1N for attractive forces to the sphere, and ±2N in the tangential plane. \omniUIST can actuate the magnet at 100Hz and sense and estimate its 3D position at 40Hz.

In summary, this chapter makes the following contributions:
\begin{itemize}
\item A fully self-contained system combining actuation and tracking components,
\item A novel algorithm for the reconstruction of the 3D tool position under electromagnetic actuation,
\item A thorough technical evaluation of the algorithm, achieving an accuracy of 5mm without actuation and 7mm with the electromagnet on,
\item An assessment of the force generation capabilities, demonstrating up to 2N of haptic force in any direction,
\item An exploration of use cases enabled by \omniUIST.
\end{itemize}
Our method can be extended to any apparatus involving a permanent magnet, integrating spatial position reconstruction and spatial actuation into a single device.