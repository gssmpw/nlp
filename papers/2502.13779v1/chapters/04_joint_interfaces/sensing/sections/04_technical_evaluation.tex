\section{Evaluation}
\textit{Omni}'s capability of delivering convincing haptic sensations relies on the performance of two main components: tracking of the tool position and in-air actuation.
We performed technical evaluations on both aspects.

In summary, \textit{Omni} is able to reconstruct the position of the tool with an accuracy of $6.9 (\pm 3.2)$ mm and can deliver peak forces of up to $\pm$ 2 N, and 0.615 N continuously.
Besides this technical evaluation, we demonstrate \textit{Omni's} interactive capabilities in the application section.  We refer readers to Zarate \etal\cite{zarate2020contact} for a psychophysics evaluation of a comparable underlying actuation mechanism, showing that users can discriminate at least 25 discrete force locations.

\subsection{Tracking evaluation}
To evaluate \textit{Omni}'s tracking accuracy, we compared our position estimates to those of a 10-camera Optitrack setup, capturing a tracking space of 1.2 $\times$ 0.8 m with submillimeter accuracy at 100 Hz. We configured \textit{Omni} to run in precision mode at a frame rate of 40 Hz. For both tracking methods, we recorded the position and rotation angles.
We evaluated the accuracy of \textit{Omni} in two conditions: \emph{no actuation} and \emph{actuation}. In the \emph{no actuation} condition, no current was sent to the coils. In the \emph{actuation} condition, the coils were actuated using a sawtooth function sending current between -4 and 4 Amperes for each axis. 

For each condition, the pen was moved around the center of \textit{Omni} at a distance of up to 10 cm, covering the area around the device. We collected 1600 samples for the \emph{no actuation} condition and 2600 samples for the \emph{actuation} condition, both at roughly 5 Hz.

\subsubsection{Results}
We found that the average difference between the two tracking systems is $e_{r_p} = 4.9 (\pm 1.8)$ mm in the \emph{no actuation} condition and $e_{r_p} = 6.9 (\pm 3.2)$ mm in the \emph{actuation} condition. Analyzing each axis separately, we found that $\mathbf{e_{r_p}} = [3.4;\ 3.1;\ 2.7]$ mm for the tracking errors and \emph{no actuation} and $\mathbf{e_{r_p}} = [4.4;\ 5.2;\ 3.3]$ mm in the \emph{actuation} condition. The results are summarized in Figure~\ref{fig:optitrack_eval}.

Finally, we tested our formulation with and without the orientation estimation of the magnet.
While we found that including these additional optimization variables improves the accuracy of the position estimates, these estimates are unstable and not yet useful for interactive applications.

Intuitively, it makes sense that including the orientation in the model fitting improves position estimation since the orientation of the magnet does influence the magnetic field. Furthermore, it is known that the model we leverage \cite{yung1998analytic} works best for spherical magnets (\eg point estimates of positions) and hence the models' approximation error may be a source of noise in the orientation estimates. We leave an extension of the reconstruction method to 5-DoF for future work.
\begin{figure}[!t]
\centering
\includegraphics[width=\columnwidth]{\dir/sensing/figures/error-optitrack.pdf}
\caption{Distribution of tracking error with and without current applied to the electromagnet.}
\label{fig:optitrack_eval}
\end{figure}

\subsection{Actuator evaluation}
\textit{Omni}'s 3 DoF spherical electromagnet produces a force on the permanent magnet in the tool by dynamically adjusting the magnetic field through currents in the orthogonal coils.
To quantify this actuation, we measured the radial and tangential forces at different locations around the electromagnet in \textit{Omni}'s spherical base. We placed a 3D-printed hemisphere over the electromagnet (see Figure~\ref{fig:eval_actuator}). The hemisphere has three slots to place a test magnet (\emph{S-30-07-N, Supermagnete}, same as in tool) and two force sensors (\emph{FSAGPNXX1.5LCAC5, Honeywell}). The force sensors were attached between the electromagnet and the test magnet to measure radial force, and to the side of the test magnet to measure tangential forces.
\begin{figure}[!t]
\centering
\includegraphics[width=1\columnwidth]{\dir/sensing/figures/force-test-01.pdf}
\caption{Setup for actuator evaluation. An additional 3D printed hemisphere is placed on top of \textit{Omni} to hold the force sensors.}
\label{fig:eval_actuator}
\vspace{-1em}
\end{figure}

\subsubsection{Results}
We generally observed a linear response of actuation with respect to the applied current.
On top of the electromagnet, we measured a maximum vertical repulsive (radial) force of 1.95 N at $I_z = 14.6$ A and a maximum attractive force of -3.04 N at $I_z = -14.6$ A, shown in Figure~\ref{fig:Fz_vs_iz}.

When \textit{Omni} applies $I_z = +3.7 $ A, it compensates for the weight and snapping and the magnet starts to levitate\footnote{In this paper we use the term \emph{levitation} in the sense of \emph{compensate its weight completely}. A complete levitation would need to control the actuation in all three axes to keep the magnet floating in place.}. Note that at this position, the force is the sum of the electromagnetic actuation, the snapping to the core, and the gravitational attraction.

The weight of the tool produces a force of $F_r = -370$ mN (38 gr), while ferromagnetic snapping yielded additional 170 mN of force, combined these produce an attracting radial force of $F_r = -540$ mN without actuation ($I_z = 0$ A).All those components contribute to users' perception of force.

On top of the sphere, the weight and snapping are orthogonal to the $x$-axis and $y$-axis and do not influence the radial forces along those axes. Consistently, we measured a linear response on those axes of the form $Fr_{x-axis} = 0.122 ~[N/A] ~I_{x}$ and $Fr_{y-axis} = 0.142 ~[N/A] ~I_{y}$. For the other locations in our test setup (horizontal to vertical), we observed forces in the range of $\pm 2$ N at $\pm 15$ A with the corresponding corrections for weight and snapping. Note that the forces have been measured when the tool was in contact with \textit{Omni}'s hemisphere.

The force intensity decays with $1/(d_0 + g)^4$, where $g$ is the air-gap between the tool and the sphere. The parameter $d_0 = 41$ mm is the center-to-center distance between the electromagnet and the permanent magnet when the tool touches the sphere. For example, for $g = 10$ mm, the reachable range of forces drops to $\pm 835$ mN. \add{At 30mm from the hull this reduces to 0.2N}
\begin{figure}[!t]
\centering
\includegraphics[width=1\columnwidth]{\dir/sensing/figures/force-amps-plot.pdf}
\caption{Radial and tangential forces on the permanent magnet as a function of coil actuation $I_x$, $I_y$ and $I_z$, for the magnet located on top of the sphere (Fig~\ref{fig:eval_actuator}). Force was collected with a compression-like force sensor.}
\label{fig:Fz_vs_iz}
\end{figure}

\subsubsection{Evaluation of EM heating}
To test the stability of the generated forces and the thermal capabilities of our system, we ran two experiments. First, we set the $y-$axis coils to maximum actuation current $I_y = 15$ A for 25 seconds and let it cool down afterwards to test the system under \emph{peak-force} conditions. Second, we set the same coil to 1/3 of the maximum actuation and we let it run for 15 minutes, to test under \emph{constant-force} conditions. Figure \ref{fig:heating} shows the evolution of the generated force and the temperature of the coil for both conditions.

During \emph{peak-force}, the system delivers a force of $2.04 \pm 0.04$ N. Starting from room temperature (24 °C), the actuator heats up to 39 °C but only 40 seconds after the actuation has been turned off, showing the system's thermal inertia. The $\Delta T = 15$ °C during this intense actuation peak shows that our system is capable of thermally buffering and dissipating the heat generated by intense forces even during tens of seconds.

In our \emph{constant-force} experiment (Figure \ref{fig:heating}, \textit{bottom}), the force remained constant within the limits $0.615 \pm 0.015$ N and for a duration of 15 min, even when the temperature of the coils (and their resistances) significantly changed.
In addition to compensating for the actuation drifts, we used the coils' resistance changes over time as the limiting factor to avoid overheating of the coils and the 3D printed parts, in case the system is required to apply maximum forces for minutes.
\begin{figure}[!t]
\centering
\includegraphics[width=\columnwidth]{\dir/sensing/figures/force-heat-time-plot.pdf}
\caption{Temporal evolution of the self-heating of the coils for two different types of actuation. Top: a \emph{peak-force} of 2 N ($I_y = 15$ A) during 25 seconds. Bottom: a \emph{constant-force} of 600 mN ($I_y = 5$ A) during 15 minutes.}
\label{fig:heating}
\end{figure}