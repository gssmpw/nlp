\begin{figure*}[!t]
    \centering
    \includegraphics[width=.75\textwidth]{\dir/sensing/figures/long_hardware.pdf}
   \caption{Overview of our system. A 3D printed base contains the 3 DoF intertwined coils and the circular PCB with an array of eight hall sensors. Arbitrarily shapes tools can be 3D printed and augmented with a permanent magnet, to interact with \omniUIST.}
    \label{fig:hardware}
\end{figure*}

\section{System Overview}
%\section{Method}

\omniUIST is a self-contained haptic feedback system that simultaneously integrates 3D tracking and actuation using the same modality (see Figure~\ref{fig:hardware}).
Through actuating an untethered, contact-free tool by means of a magnetic field, our device supports rendering precise haptic attractive and repulsive forces as well as accurate tracking without the need for any external infrastructure or markers. Our system allows for rich interactions with and haptic perception of dynamic virtual surfaces.

Our goal is to enrich AR/VR and other 3D applications via \omniUIST and a minimally instrumented tool. Simplicity of the haptic prop and a walk-up-and-use experience were important design goals of our work. Furthermore, to create rich immersive experiences, such a system must be able to deliver different types of high-fidelity haptic forces and precisely sense user input without requiring any external tracking. Moreover, we aim for a self-contained device that is affordable and easy to manufacture.

The actuation mechanism used in \omniUIST is based on the working principle proposed in \cite{zarate2020contact}. We build up a hemispherical shell base whose core houses a symmetric omnidirectional electromagnetic actuator. Three interwoven and mutually orthogonal coils generate the haptic forces.

By controlling the current in each coil, we can precisely configure the exerted force onto an external magnet, such as the one inside the 3D stylus tool. As the tool approaches the sphere, \omniUIST is able to provide independently controlled radial and tangential forces. Our design contributes several important improvements over \cite{zarate2020contact} that allow us to provide twice the amount of force ($2$ N in either direction) and for much longer periods without suffering from self (over-)heating. %Together, these improvements enable its use under high-stress and continuous usage scenarios in AR/VR.

Although the tool is contact-free, the haptic force that the user perceives has its reaction force on the support base. In this sense, the user perceives grounded forces even if there is no mechanical link to the base.

In our current implementation, \omniUIST rests on a surface (\eg table), though it is compact and could be mounted on a robotic end effector to deliver large-scale 3D feedback. \add{This would allow for haptic feedback in a large volume, which would be beneficial for VR applications. In this case, however, geomagnetism should be taken into account more strongly.}

Beyond the improved actuator, our main contribution lies in the integration of the \omniUIST actuator with a fully self-contained real-time tracking method of the tool. To this end, eight Hall sensors are distributed below the interactive sphere. Each sensor reads a combination of the magnetic field generated by the tool, superimposed by the electromagnetic field generated by the actuator.We use a gradient-based optimization method to locate the magnet's 3D position based on the Hall sensors' readings, running at an interactive rate of 40 Hz.

%Below, we explain the main elements of our system: 1) The system description and our novel computational model; 2) our algorithm for tracking a permanent magnet in 3D space during simultaneous actuation with an electromagnet; and 3) the control of the electromagnet given desired forces onto the permanent magnet. 
