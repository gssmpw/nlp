% Use this section to set the ACM copyright statement (e.g. for
% preprints).  Consult the conference website for the camera-ready
% copyright statement.



% Use this command to override the default ACM copyright statement
% (e.g. for preprints).  Consult the conference website for the
% camera-ready copyright statement.

%% HOW TO OVERRIDE THE DEFAULT COPYRIGHT STRIP --
%% Please note you need to make sure the copy for your specific
%% license is used here!
% \toappear{
% Permission to make digital or hard copies of all or part of this work
% for personal or classroom use is granted without fee provided that
% copies are not made or distributed for profit or commercial advantage
% and that copies bear this notice and the full citation on the first
% page. Copyrights for components of this work owned by others than ACM
% must be honored. Abstracting with credit is permitted. To copy
% otherwise, or republish, to post on servers or to redistribute to
% lists, requires prior specific permission and/or a fee. Request
% permissions from \href{mailto:Permissions@acm.org}{Permissions@acm.org}. \\
% \emph{UIST '20},  October 20--23, 2020, Minneapolis, MN, USA \\
% ACM xxx-x-xxxx-xxxx-x/xx/xx\ldots \$15.00 \\
% DOI: \url{http://dx.doi.org/xx.xxxx/xxxxxxx.xxxxxxx}
% }

% Arabic page numbers for submission.  Remove this line to eliminate
% page numbers for the camera ready copy
%  \pagenumbering{arabic}

% Load basic packages
\usepackage{balance}       % to better equalize the last page
\usepackage{graphics}      % for EPS, load graphicx instead 
\usepackage[T1]{fontenc}   % for umlauts and other diaeresis
\usepackage{txfonts}
\usepackage{mathptmx}
\usepackage[pdflang={en-US},pdftex]{hyperref}
\usepackage{color}
\usepackage{booktabs}
\usepackage{textcomp}
\usepackage{graphicx,import}
\usepackage[caption=false]{subfig}
\usepackage[normalem]{ulem}

\usepackage{units}

\usepackage{textcomp}

% Some optional stuff you might like/need.
\usepackage{microtype}        % Improved Tracking and Kerning
% \usepackage[all]{hypcap}    % Fixes bug in hyperref caption linking
\usepackage{ccicons}          % Cite your images correctly!
% \usepackage[utf8]{inputenc} % for a UTF8 editor only

% If you want to use todo notes, marginpars etc. during creation of
% your draft document, you have to enable the "chi_draft" option for
% the document class. To do this, change the very first line to:
% "\documentclass[chi_draft]{sigchi}". You can then place todo notes
% by using the "\todo{...}"  command. Make sure to disable the draft
% option again before submitting your final document.
\usepackage{todonotes}
\usepackage[ruled,vlined]{algorithm2e}

% Paper metadata (use plain text, for PDF inclusion and later
% re-using, if desired).  Use \emtpyauthor when submitting for review
% so you remain anonymous.

% TITLE WORDS
% Volume
% Nonplanar
% Electromagnet
% Sphere
% Tracking
% Sensing
% Actuating
% untethered

%  Other paper: "Optimal Control for Electromagnetic Haptic Guidance Systems"

% \def\plaintitle{\systemname: Non-Planar Electromagnetic Haptic Feedback}
% \def\plaintitle{Electromagnetic Haptic Feedback Sensing and Actuating}
% \def\plaintitle{3 Degrees of Freedom Sensing and Actuation for Electromagnetic Haptic Feedback}
% \def\plaintitle{Volumetric Sensing and Actuation for Spherical Electromagnetic Haptic Feedback}
% \def\plaintitle{Sensing and Actuation for Electromagnetic Haptic Feedback}
% \def\plaintitle{Sensing and Actuation for Volumetric Electromagnetic Haptic Feedback}
% \def\plaintitle{Untethered Sensing and Actuation for Volumetric Electromagnetic Haptic Feedback}
% \def\plaintitle{Stand-Alone Sensing and Actuation for Electromagnetic Haptic Feedback}
% \def\plaintitle{Stand-Alone Sensing and Actuation for Electromagnetic Systems}

\def\plaintitle{Omni: Volumetric Sensing and Actuation of Passive Magnetic Tools for Dynamic Haptic Feedback}


\def\plainauthor{Thomas Langerack, Juan Jose Zarate, David Lindlbauer, Christian Holz, Otmar Hilliges}
\def\emptyauthor{}
\def\plainkeywords{Haptic feedback; Electromagnetic sensing; Electromagnets; Mixed Reality}
\def\plaingeneralterms{Documentation, Standardization}

% llt: Define a global style for URLs, rather that the default one
\makeatletter
\def\url@leostyle{%
  \@ifundefined{selectfont}{
    \def\UrlFont{\sf}
  }{
    \def\UrlFont{\small\bf\ttfamily}
  }}
\makeatother
\urlstyle{leo}

% To make various LaTeX processors do the right thing with page size.
% \def\pprw{8.5in}
% \def\pprh{11in}
% \special{papersize=\pprw,\pprh}
% \setlength{\paperwidth}{\pprw}
% \setlength{\paperheight}{\pprh}
% \setlength{\pdfpagewidth}{\pprw}
% \setlength{\pdfpageheight}{\pprh}

% Make sure hyperref comes last of your loaded packages, to give it a
% fighting chance of not being over-written, since its job is to
% redefine many LaTeX commands.
\definecolor{linkColor}{RGB}{6,125,233}
\hypersetup{%
  pdftitle={\plaintitle},
% Use \plainauthor for final version.
%  pdfauthor={\plainauthor},
  pdfauthor={\plainauthor},
  pdfkeywords={\plainkeywords},
  pdfdisplaydoctitle=true, % For Accessibility
  bookmarksnumbered,
  pdfstartview={FitH},
  colorlinks,
  citecolor=black,
  filecolor=black,
  linkcolor=black,
  urlcolor=linkColor,
  breaklinks=true,
  hypertexnames=false
}