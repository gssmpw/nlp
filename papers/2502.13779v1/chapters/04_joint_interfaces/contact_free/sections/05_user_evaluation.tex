\section{User Evaluation}
To assess the efficacy of our proposed approach, we validate the prototype in a perceptual study with 6 participants in order to 1) determine how well users can differentiate between different set-points, and 2) how accurate and precise users are with finding a set-point.

\noindent\textbf{Procedure}: Based on a pilot study, we predetermined 25 evenly separated set-points (Figure \ref{fig:confusion_circle_error} right). We randomly selected a set-point, asked the user to find it, and report the corresponding number. Upon reporting, we also measured the Euclidean distance to the actual set-point. Every set-point was prompted exactly twice, resulting in 50 data points per user (300 in total). Only repulsive forces were tested. We used the same mapping parameters as in Figure \ref{fig:magnetic_model}.
\begin{figure}[!t]
    \centering
    \begin{tabular}{cc}
     \includegraphics[width=0.4\columnwidth]{\dir/contact_free/figures/confusion_fix.pdf} &
    \includegraphics[width=0.3\columnwidth]{\dir/contact_free/figures/errors_user.png}
    \end{tabular}
    \caption{\emph{Left:} confusion matrix of the 25 set-points, averaged over all users. High values on the diagonal indicate little confusion and the ability to differentiate between different set-points. \emph{Right:} Set-points used in the study. The opacity directly correlates with the percentage of correct identifications by the users. Arrows are drawn when 33\% or more of the \emph{wrong} answers were attributed to set-point that the arrow points to.}
    \label{fig:confusion_circle_error}
\end{figure}

\noindent\textbf{Location accuracy}: Figure \ref{fig:confusion_circle_error} depicts the resulting confusion matrix between set-points. It can be seen that users accurately perceive discrete actuation points. For those actuation points that do cause incorrect answers, users tend to pick the neighboring location (typically higher on the sphere). This effect is pronounced along the meridian arc facing away from the user, whereas the orthogonal meridian produces fewer erroneous detections. This could be due to the position of the hand and arm and differences in muscle groups that are involved in actuating the wrist versus the whole hand. The difference in coil diameters could be another contributing factor.
\begin{figure}[!t]
    \centering
    \medskip
    \includegraphics[width=0.51\columnwidth]{\dir/contact_free/figures/acc_prec_fix.pdf}
    \caption{Euclidean distance between the true set-point position and the user reported position as a function of the azimuth ($\theta$), measured from the top of the sphere and averaged over all angles and users.}
    \label{fig:error_vs_angle}
\end{figure}

\noindent\textbf{Precision}: We report the precision with respect to the angle $\theta$. Figure \ref{fig:error_vs_angle} shows that the error increases as a function of the angle. A potential contributing factor here is that gravity has more impact on the pen the further down it moves on the hemisphere. This may make it more difficult for users to differentiate between the EM-actuation force and gravity. The mean errors of $2.5 mm \pm 1.4$, $5.7 mm \pm 4.6$, $6.5 mm \pm 5.2$, and $7.2 mm \pm 5.1$ are relatively small across the device.