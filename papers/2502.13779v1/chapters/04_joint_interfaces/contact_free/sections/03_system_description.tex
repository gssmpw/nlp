\section{System Description}

We introduce a haptic feedback system that enables dynamic interactions with virtual surfaces through an untethered, contact-free tool.
Our device is a hemispherical shell. The core consists of three coils with mutually orthogonal axes. By controlling the current flow through the coils, we can shape the magnetic field around the device. This, in turn, enables the device to exert controlled electromagnetic forces on the permanent magnet located inside a handheld tool such as a stylus. Despite being contact-free, the forces perceived by the user are ultimately grounded to the support onto which the device is mounted, allowing for comparatively strong feedback.
We now detail the main components that make up our contribution: 1) a computational model of the electromagnet-magnet interactions; 2) the prototypical hardware design; and 3) a real-time control algorithm.

\subsection{Haptic force mapping}
\begin{figure}[t]
\centering
\medskip
\includegraphics[width=\columnwidth]{\dir/contact_free/figures/magnetic_lines_v3.png}
\caption{Schematic of the main quantities necessary to compute desired radial and tangential forces (a). Insets show: force map of a permanent magnet (b). Adjustable force map generated by our approach (c). Here $\rsBold = d_{min} \ez$, $\theta_1 = \pi/10$ and $\theta_2 = 3\pi/10$. Example virtual surface that can be felt by the user (d).}
\label{fig:magnetic_model}
\end{figure}{}

To enable the envisioned interactive experiences, we must be able to dynamically adjust the haptic feedback. We therefore require a model for the magnetic interaction between device and tool that is \emph{1}) precise enough to predict forces with sufficient accuracy and \emph{2}) fast enough to run at the feedback rates required for haptic interaction.

Computing the magnetic field around, and resulting interaction between, arbitrarily-shaped objects is a challenging and computationally expensive task. However, even though the magnetic field can be very complex in the direct vicinity of an object, this complexity rapidly decays with increasing distance and approaches a simple dipole field. This fact has been exploited in previous work to construct fast, approximate models based on dipole-dipole interaction \cite{thomaszewski2008magnets}. Instead of solving the Maxwell equations on a discretization of ambient space, this approximate model only requires the magnitude and orientation of the magnetic moment of each dipole, leading to drastically reduced computation times.

In adopting this approach, we model both the electromagnet of the device and the tool as a single dipole (see Figure \ref{fig:magnetic_model}.a). Let $\mpBold, \meBold \in \mathbb{R}^3$ denote the magnetic moments of the permanent magnet in the tool and the electromagnet in the device, respectively. The forces exerted on the tool, expressed in local coordinates, are obtained as:
\begin{eqnarray}
    \FrBold &=& - \frac{3 \mu_0 \ \me \ \mpp}{2 \pi \ d^4} \cos(\alpha) \ \er \label{eq:Fr}\ ,\\ 
    \FtBold &=& - \frac{3 \mu_0 \ \me \ \mpp}{4 \pi \ d^4} \sin(\alpha) \ \et \label{eq:Ft}\ ,
\end{eqnarray}
\noindent where $\me=|\meBold|$, $\mpp=|\mpBold|$. In the above expression, $\FrBold$ is the force in the radial direction $\RpenBold = d \ \er$ from the center of the device to the tool. Likewise, $\FtBold$ is the force in the tangential direction $\et$ that tends to align the location of the two dipoles along $\er$. Assuming that the tool is in contact with the shell, both force components depend only on the relative angle $\alpha$ between the dipoles. Furthermore, $\FrBold$ and $\FtBold$ are attractive (negative) when the two dipoles have the same sign and $\alpha<\pi/2$. Conversely, the forces become repulsive (positive) when the dipoles have opposite orientations (see Figure \ref{fig:magnetic_model}.a).

The interaction forces decay quickly, as $1/d^4$, with increasing magnet-magnet distance. The maximum force $\mathbf{F}\mathrm{r,max}$ is obtained when the tool is in contact with the device ($d = d\mathrm{min}$). In our case, $d_{min} = 50$ mm, since the outer case radius is 30 mm and inside the tool, the magnet center is 20 mm away from the tool tip. Our proposed geometry ensures that the distance $d$ will remain constant across the working surface as long as the tool is kept in contact with the surface, allowing for much simpler control of the force. However, it is worth noting that moving the tool 1 cm away in the radial direction makes the force fall to approximately $\mathbf{F}\mathrm{r,max}/2$, another extra centimeter results in a force $\mathbf{F}\mathrm{r,max}/4$. This rapid decay of the interaction forces can, to some extent, be mitigated by increasing the intensity of the magnetic field. However, to maintain power consumption and thermal effects within reasonable bounds, we constrain our interactions to a volumetric shell ($d_{min} \leq d \lesssim d_{min} + 2cm$) above the device's surface.

Equations \ref{eq:Fr} and \ref{eq:Ft} also reveal the comparatively weak variation of force magnitude with respect to angle that one would expect when two magnets interact: switching from attractive to repulsive forces requires a change in orientation of $\alpha = \pi$; see Figure \ref{fig:magnetic_model}.b. This weak force variation is inherent to permanent magnets: whereas the far-field interaction is dominated by torque (which decays only as $1/d^3$), the near-field force interaction is governed by the location of the dipoles, not their orientation. In our setting, this property would translate into weak angular resolution with a permanent magnet. To address this problem, we introduce the concept of a \emph{force map} that uses magnetic pole transformation to take advantage of the spherical symmetry and that is compliant with the physics of the system. Our system can generate force maps equivalent to multiple alternating pole regions, having sharper repulsive domes and attractive valleys. The force map is defined by four parameters:
\begin{itemize}
\item The center $\rsBold$ of the potential. When rendering a mountain-like dome, for instance, $\rsBold$ is the summit.
\item The height of the dome is measured as the maximum magnetic moment intensity $m_{e0}$.
\item The angle $(\theta_1)$ (i.e., the location of the tool in polar coordinates with respect to $\rsBold$) where the radial force vanishes for the first time. In our example, $(\theta_1)$ is the angle from the summit to the base. % of the dome.
\item The cut-off angle $\theta_2$ after which the potential is set to be zero. Having such a cut-off mechanism allows us to control how many individual potentials can be combined into one force map without mutual interference.
\end{itemize}

%%%%% algorithm %%%%%%
\begin{figure}[!t]
    \bigskip
    \textbf{Algorithm to calculate desired forces}\\
    \rule[5.0pt]{\columnwidth}{0.75pt}
    \vspace{-15pt}
    \begin{algorithmic}
    %%%%
    \STATE \% \emph{To compute} $\meBold$ \emph{given the tool position and the force map.}
    \REQUIRE $calc\_Me \ (\RpenBold, \rsBold, m_{e0}, \theta_1, \theta_2)$:
    \STATE $\RpenBold \rvert_{\rsBold} = \mathbb{T}_{\mathbf{r} \to \rsBold} \cdot  \RpenBold $
    \STATE $\mathbf{F} \rvert_{\rsBold} = calc\_F \ (\RpenBold\rvert_{\rsBold}, \rsBold, m_{e0}, \theta_1, \theta_2)$
    \STATE $\mathbf{F} = (\mathbb{T}_{\mathbf{r} \to \rsBold})^{-1} \cdot  \mathbf{F} \rvert_{\rsBold} $
    \STATE $\mathbf{F} \rvert_{\RpenBold} = \mathbb{T}_{\mathbf{r} \to \RpenBold} \cdot \mathbf{F}$
    \STATE $\meBold \rvert_{\RpenBold} = \frac{4 \pi d^4}{3 \mu_0} \ [1,1,-1/2] \cdot \mathbf{F} \rvert_{\RpenBold}$
    \STATE $\meBold = (\mathbb{T}_{\mathbf{r} \to \RpenBold})^{-1} \cdot \meBold \rvert_{\RpenBold}$
    \RETURN $\meBold$
    \end{algorithmic}
    %%%%
    \vspace{5pt}
    \begin{algorithmic}
    \STATE \% \emph{To compute the actuation force  in the} $\rvert_{\rsBold}$ \emph{coordinates.}
    \REQUIRE $calc\_F \ (\RpenBold\rvert_{\rsBold}, \rsBold, m_{e0}, \theta_1, \theta_2)$:
    \STATE $F_r = 0$
    \STATE $F_t = 0$
    \IF{$d < d_{max}$ \AND $\alpha < \theta_2$}
    \STATE $F_r = 2 F_0 \ cos(\alpha\frac{2\theta_1}{\pi}) \ \left(\frac{||\rsBold||}{||\RpenBold||}\right)^4$
    \STATE $F_t = F_0 \ sin(\alpha\frac{2\theta_1}{\pi}) \ \left(\frac{||\rsBold||}{||\RpenBold||}\right)^4$
    \ENDIF 
    \STATE $\mathbf{F} \rvert_{\rsBold} = [F_r, F_t, 0]$
    \RETURN $\mathbf{F} \rvert_{\rsBold}$
    \end{algorithmic}
    %%%
    \rule[0.0pt]{\columnwidth}{0.75pt}
    \caption{Pseudo-code of our force calculation algorithm. Note that $\mathcal{T}_{r_i \to r_j}$ is the rotation matrix that maps from coordinate system $r_i$ to $r_j$, and that $\mathbb{T}_{r_j \to r_i} = (\mathbb{T}_{r_i \to r_j})^{-1} = (\mathbb{T}_{r_i \to r_j})^{T}$.}
    \label{fig:algorithm}
\end{figure}
%%%%%%%%% 

Figure \ref{fig:algorithm} summarizes our algorithm to calculate the actuation vector $\meBold$ given the tool position and force map as input.
For simplicity and efficiency, we perform the different calculations in their natural coordinate system: the Cartesian system $\mathbf{r} = [x,y,z]$, the spherical system relative to the map's center $\rsBold$, and the spherical system centered around the tool position $\RpenBold$.

The force calculation incorporates the angular scaling by using $(\alpha\frac{2\theta_1}{\pi})$ as an argument for the trigonometric functions in Equations \ref{eq:Fr} and \ref{eq:Ft}. Note that if $\theta_1 = \pi/2$, we recover a permanent magnet. In Figure \ref{fig:magnetic_model}.c, we show an example where the center of the potential (\emph{red}) is on the north pole of the sphere, the first vanishing region (\emph{white}) appears at 18$^\circ$ and the forces are cut off at 54$^\circ$ (\emph{blue}).

Using the algorithm described in Figure \ref{fig:algorithm}, we obtain at each time step an actuation input $\meBold = (m_{e-x}, m_{e-y}, m_{e-z})^T$ given the tool position. Depending on the requirements of the application, the potential parameters (center position, intensity, angular variation, and cut-off) may also change as a function of tool position. For example, the force map for the terrain example can be dynamically adapted to emulate changes in landscape over time.

%%%%%%%%%%%%%%%%
\subsection{Spherical electromagnetic actuator}
%%%%%%%%%%%%%%%%
%%% Objective: explain why the spherical EM is important, and how we designed and fabricated
Having laid out the computational model for generating haptic feedback based on dipole interactions, we now describe hardware and implementation aspects for rendering these forces on our device (Fig. \ref{fig:syst_overview}).
Our device renders haptic forces by controlling the magnetic field generated by a spherical electromagnet. Compared to other alternatives, this approach has several advantages. First, there are no mechanically moving parts in the actuator, reducing complexity and eliminating wear. Changing the orientation of the resulting force on the tool is accomplished by adapting the currents in each coil such as to rotate the induced dipole in the core as desired; see also Figure \ref{fig:hardware_labels}. The underlying physical principle is that, in the presence of linear and isotropic materials, the magnetic field $\mathbf{B(\mathbf{r})}$ at any given point $\mathbf{r}$ can be calculated as the sum over all contributions of all magnetic sources \cite{petruska2014omnimagnet}. Under this linearity property of $\mathbf{B}$, the magnetic field produced by the three orthogonal coils is the superposition of the fields generated by each coil individually. Finally, we insert a magnetic core with isotropic (i.e., spherical) geometry and material at the center of the coils and operate it in the linear regime (i.e, $\me << m_{saturation}$), linearity is maintained such that $\mathbf{B(\mathbf{r})}$ can be computed by summing up each coil's contributions.
\begin{figure}[!t]
\centering
\medskip
\includegraphics[width=0.7\columnwidth]{\dir/contact_free/figures/hardware2.png}
\caption{3D cross-section of the proposed hardware setup. The device measures $15 \times 15$ cm across the base. Three coils are placed, orthogonal to each other and surrounding the iron core. Forces can be rendered onto a permanent magnet moving above the device. Hall sensors are used for calibration. A plastic cover isolates the coils from the user thermally and electrically. Active cooling is provided via several fans mounted in the base.}
\label{fig:hardware_labels}
\end{figure}{}

In order for the previous statement to remain valid, two assumptions have to be made. First, hysteresis effects can be neglected: the lower the coercivity and remanence of the core material, the lower the effect of past states of the electromagnet on the current one. The second assumption is that the distance $d$ between dipoles is large enough such that the core magnetization due to $\mpp$ is small compared with the effect of the coils. This will not be true if, for example, the tool snaps to the sphere with no electrical current in the coils. In our setting, however, a spherical cap around the coils prevents too close approach of the tool and, at the same time, provides the grounding required for generating sufficiently strong interaction forces.

For the standard low-carbon steel core, we did not observe any hysteresis effects for the update of $\meBold$ at 50 Hz refresh rate. To avoid the undesired self-magnetization of the core due to the tool, we tuned the size of the permanent magnet and the coil parameters using FEM simulations, followed by minor design adjustments informed by real-world tests.

The design choices for the hardware of our prototype are motivated by our goal to develop a device that is affordable and easy to manufacture. In particular, we use off-the-shelf electronic components but custom-wound coils. FEM simulations in Comsol Multiphysics are used to assist in the exploration of the design space. In Figure \ref{fig:hardware_labels} we show a 3D CAD rendering of our device. The external dimensions are 150 mm by 150 mm by 95 mm. The structure is built out of laser-cut acrylic glass and 3D-printed parts. The three orthogonal coils are arranged around the 30 mm steel core. All coils have a resistance of roughly 0.6 
$\Omega$ at room temperature. We use the 12V line of a standard CPU power supply to drive the coils, meaning a maximum electrical current of 24A per coil at full strength. The electrical current in each coil is controlled by a high-power motor driver (Pololu 18v17). The PWM signals are generated by a 12-bit driver (PCA9685) that allows for easy tuning of the carrier frequency and the duty cycle with 12-bit resolution. To be able to accurately control the electrical current and compensate for thermal drifts, we use INA219 current sensors in each coil with a 0.01 $\Omega$ shunt resistor. Finally, an Arduino board creates the bridge between the I2C components and the PC. The hardware is completed by 9 Hall sensors arranged collinearly with the axes and diagonals of the coils. Six fan coolers below the coils provide active cooling.

\subsection{Control Strategy}
\begin{figure}[!t]
\centering
\medskip
\includegraphics[width=0.95\columnwidth]{\dir/contact_free/figures/flowchart_v2.png}
\caption{Schematic overview of the software pipeline. Given the desired force map at time $t$, and the tool position provided by an external tracking system, we calculate the input value $\meBold$ using the algorithm of Fig. \ref{fig:algorithm}. Then the system inputs are computed using Eq. \ref{eq:control-law}, and finally a temperature compensation step corrects the system inputs.}
\label{fig:control}
\end{figure}

The main objective of the actuator control loop is to generate a stable and controllable force on the haptic tool. Although the mathematical principles are straightforward, the practical implementation poses some problems. Since the magnetic field is directly proportional to the current (Fig. \ref{fig:current_m}), controlling the latter is sufficient to determine the state of the system. If the resistance is known, controlling the voltage is equivalent to controlling the current via Ohm's law,
\begin{equation}\label{eq:ohms}
I = V/R \ ,
\end{equation}
\noindent and the voltage in turn can be controlled via Pulse-Width Modulation (PWM). Therefore, the input to our system is the PWM frequency. The complete control loop is shown in Figure \ref{fig:control}.
However, significant heating occurs due to the necessary power that in turn increases the resistance. Therefore, the PWM duty cycle (i.e., voltage) needs to be adjusted to maintain a constant current. Measuring the current allows determination of the resistance via inversion of Eq \ref{eq:ohms}. A simple controller then computes an input $\mathbf{u} \in [-1,1]$ at time $t$, corresponding to the PWM duty cycle. This depends on the desired current in Ampere ($\mathbf{I}^{(s)}t$), the resistance in Ohm ($\mathbf{R}t$), and the maximum voltage in the system, $V_0=12$:
\begin{equation}\label{eq:control-law}
\mathbf{u}{t} = \frac{\mathbf{I}^{(s)}{t}}{V_0}*\mathbf{R}t \quad ,
\end{equation}
\noindent where $I_t^{(s)}$ is based on the desired magnetization, $\mathbf{m}\text{e}$, computed via the algorithm presented in Fig. \ref{fig:algorithm} and can be determined via Biot-Savart Law (adapted for our purpose):
\begin{equation}\label{eq:I_vs_me}
I_t^{(s)} = \mathbf{c} * \frac{\mathbf{m}_e * \mu_0}{2*\pi*\mathbf{d}^3} \quad ,
\end{equation}
\noindent here $\mathbf{c}$ is a constant coming from a calibration procedure that, with the help of five Hall sensors, maps input current to $\mathbf{m}_e$ (Fig. \ref{fig:current_m}). $\mu_0=4*\pi*10^{-7}$ is the relative permeability of air and $d$ is the distance from the core to the Hall sensors used for calibration (0.055 meter).
Due to the thermal effects, $\mathbf{R_t}$ however is not a constant but depends on the measured current ($\mathbf{I}t^{(m)}$) computed and averaged over a sliding window:
\begin{equation}
\mathbf{R}t = \frac{V{0} * \frac{1}{N}\sum^N \mathbf{u}{t-i}}{\frac{1}{N}\sum^N\mathbf{I}^{(m)}_{t-i} } \quad .
\end{equation}