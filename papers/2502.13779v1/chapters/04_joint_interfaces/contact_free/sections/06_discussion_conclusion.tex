% !TeX root = ../main.tex

\section{Discussion}
In this chapter, we presented a novel contact-free volumetric haptic feedback device. This device uses a symmetric electromagnet combined with a dipole magnet model and a simple control law to deliver dynamically adjustable forces onto a handheld tool, such as a stylus. The tool only requires an embedded permanent magnet, allowing it to be completely untethered. Despite being contact-free, the force is grounded via the electromagnet, enabling the user to feel relatively large forces.

While our proposed method offers many advantages, it also has some drawbacks. Heat generation limits the number of interactions possible within a certain time frame. Specifically, when operating at full power, continuous interaction is limited to 5 seconds.

It is also important to note that the interaction between magnets involves both forces and torques. In this work, we focused on controlling the three force components via the 3 degrees of freedom (DoFs) of the electromagnet, allowing the torque values to adapt accordingly. However, the same procedure can be applied to control a specific torque map, leaving the force values unconstrained, or to manage a combination of force and torque. In future work, we aim to explore the dynamic capabilities of our approach, including advanced control schemes to continuously shape the force map.

Finally, our current control method relies on knowing the location of the tool, achieved through external cameras for optical tracking. However, this setup is cumbersome, expensive, and requires line-of-sight. In the next chapter, we will address this limitation by tracking the permanent magnet embedded in the tool using Hall sensors and a gradient-based method.