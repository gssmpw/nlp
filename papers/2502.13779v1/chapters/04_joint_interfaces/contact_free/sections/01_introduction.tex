% !TeX root = ../main.tex

\section{Introduction}
Many emerging computing paradigms such as virtual and augmented reality (VR/AR) rely on haptic feedback as an additional information channel to improve the user experience. For example, in VR, haptic feedback increases the sense of presence and immersion by rendering collisions, shapes, and forces between the user and virtual objects.

Existing approaches either rely on vibro-tactile actuators that are embedded into handheld controllers, displays or worn on the body. Such actuators can only render coarse, non-localized haptic sensations. More complex setups such as articulated arms and exoskeletons can render both large-force haptic feedback and can operate in three-dimensional space, but typically require force anchoring in the environment and require complex, and often bulky mechanisms, which prevents walk-up-and-use scenarios, thus hindering user uptake.  

To address this challenge, we propose an approach to deliver contact-free, volumetric haptic feedback via an omni-directional electromagnet. The device consists of a single 60 mm diameter spherical electromagnet and can render attractive and repulsive forces onto permanent magnets embedded in pointing tools such as a stylus or magnets directly worn on the user's fingertip. Leveraging a dipole-dipole approximation of the electromagnet-magnet interaction, our system is capable of calculating and controlling the forces exerted onto the permanent magnet in real-time while dynamically adjusting the force that is perceived by the user. The system can deliver perceptible forces up to 1N in a thin volume above the surface. Furthermore we demonstrate that users can distinguish at least 25 different set-points separated by 18$^\circ$ on the surface of the sphere. %In this work we focus on the control of 3 DoF forces. However, the interaction between magnets does not only involve forces but also torques which could be controlled in the same manner as presented here. 

\begin{figure}[t]
    \centering
    \includegraphics[width=0.7\columnwidth]{\dir/contact_free/figures/teaser_v2.png}
    \caption{We introduce a novel contact-free mechanism to render haptic feedback onto a tracked stylus via a hemispherical electromagnet. An approximate model of the magnet interaction and a computationally efficient control strategy allow for the dynamic rendering of  attracting and repulsive forces, for example, allowing users to explore virtual surfaces in a thin shell surrounding the device (inset).}
    \label{fig:syst_overview}
\end{figure}{} 

To demonstrate the efficacy of our approach we designed a functional prototype comprising of an iron core and three custom wound copper coils. The electromagnet is encased in a plastic dome upon which tools can come into contact and move about its surface (see \figref{fig:syst_overview}). The prototypical system can render radial (along the vector from the magnet to the tool) and tangential forces, both in the attractive and repulsive polarity. The system can furthermore dynamically adjust the opening angle and steepness of the electromagnetic potential to
%form perceivable hills and valleys respectively (see \figref{fig:syst_overview}, inset). For example,
 gently guide the user towards a desired set-point in the thin volume above the device.    

Modulating the magnetic field as a function of tool position opens the door to many different interactive applications. In a virtual terrain exploration, the tool can be repelled when moved along mountains and attracted to valleys while descending (see \figref{fig:syst_overview}, inset). As another example, the sensation of stirring a viscous liquid may be created by emulating the drag of the fluid on the tool. To enable these interactive experiences, our device builds on three key components that represent our contributions in this work:
\begin{itemize}
    \item A computational model based on magnetic dipole-dipole interaction to produce force maps that allow for designing and generating location-dependent feedback, 
    \item The design and implementation of a 3 degree-of-freedom (DoF) spherical electromagnet prototype,
    \item A control strategy that translates desired high-level forces into low-level input signals (currents/voltages) for the coils, fast enough for interactive use.
\end{itemize}

To assess the efficacy of the proposed design we characterize the system properties experimentally and report findings from a perceptual study which explores the thresholds for perception and localization capabilities of the electromagnetic actuation approach. Results from these early user tests indicate that users can perceive at least 25 different spatial locations with high precision.  



%%%%%%%%%%%%%%%%%%%%%%
%%%%%%%%%%%%%%%%%%%%%%%
%Many VR systems leverage vibro-tactile actuators, embedded in hand-held controllers (e.g., HTC Vive),  displays \cite{Wellman1995} or worn on the body \cite{Cybertouch, Gloveone}. This feedback modality  can only offer coarse, non-localized haptic sensation. More complex setups often involving articulated arms or external braking mechanisms \cite{Massie94, Stamper1997, VanDerLinde2002, Araujo2016} can reproduce higher fidelity haptics and render both tactile and kinesthetic feedback. While such approaches can produce large forces, they are not suitable for mobile scenarios due to their mechanical constraints.

%The sense of touch is vital for interacting with real-world objects. It not only allows us to judge materials, physical properties and constraints; it also allows us to manipulate the objects precisely. However, digital objects lack the physical presence and therefore cannot be touched, perceived or manipulated to a satisfactory and accurate level. This lack greatly reduces the immersion of digital environments. Especially with the rise of Virtual- and Augmented-Reality the haptic perception of virtual objects has gained tremendous importance. 

%Due to this, it is not surprising that a large variety of haptic interfaces exist. [name some]. However, these often operate on a plane surface [cite, cite, cite]. This limits the possible applications, especially since many real-world objects are not constrained to a planar surface. Another branch of haptic research focuses on volumetric haptic devices, which are capable of operating in a three dimensional space [some more citations]. However, these are often constrained by tools integrated with the device; thereby eliminating the potential of swapping tools on the fly and decreasing the immersion for the user. 

%To address these limitations, we introduce \devicename{}, a stand-alone haptic device capable of contact-free volumetric input sensing and rendering of nonplanar haptic forces. Our novel approach is based on the core principles of electromagnetism. We introduce a spherical electromagnet, consisting of three orthogonal coils. The advantage of using an electromagnet is that they are accurately controllable and do not require a physical link to exert force on an object. By controlling the current in each coil separately we can precisely render a force towards the center of the magnet over a distance. The second electromagnetic principle we employ is the hall effect to measure the magnetic field. We add a simple permanent magnet to a tool. By accurately modelling the electromagnetic field of the actuator and subtracting this from our sensor readings, we can triangulate the remaining field and thereby determine the position of the tool (permanent magnet). 

%Next to our hardware contribution, we layout an analytical model for calculating the force of the electromagnet on a permanent magnet given relative orientation. Similarly we introduce a non-linear problem to determine the position of a magnet in space. We solve this problem using an off-the-shelf solver in sub millisecond time. Finally, we introduce a mapping of desired forces with regard to current tool position and desired tool position. 

%We characterize the positioning, resolution and perceived forces of our system in a just-noticable-difference study (JND). We also show possible application domains, such as ..., ... and ....

%\devicename{} is easy and afforable to fabricate. To encourage adaption and exploration of this system. We make all the source code, blueprints and fabrication instructions freely available. 

% Problem:
% \begin{itemize}
%     \item ...
% \end{itemize}


% Hardware:
% \begin{itemize}
% \item Novel Hardware
%   \item Stand alone system
%   \item user-in-the-loop haptic feedback
%   \item easy and affordable to fabricate
%   \item Off-the-shelf electronics
%   \item hot-swappable 
% \end{itemize}

% Modelling:
% \begin{itemize}
%     \item Model of B, force and torque given actuation/current
%     \item Non-linear optimization method for determine permanent magnet in 3D space given N sensors
% \end{itemize}

% Evaluation:
% \begin{itemize}
%     \item Just Noticeable Difference (JND)
% \end{itemize}