% !TeX root = ../main.tex

%%%%%%%%%%%%%%%%
\section{System Evaluation}
%%%%%%%%%%%%%%%%

\begin{figure}[!t]
    \centering
    \medskip
    \includegraphics[width=0.8\columnwidth]{\dir/contact_free/figures/plot_temp_c2_v1_label.png}
    \caption{Thermal characterization of one of the coils as function of time. During the first 3 minutes the $y$-coil is driven with PWM=30\%, and then we let it cool over the remaining 3 minutes. $T_{in}$ is calculated by taking the thermally caused resistance variations into account while the current $I_y$ is `on', and $T_{out}$ is measured.}
    \label{fig:temp}
\end{figure}{}

%%%% Major problem is Joule heating. 
One of the main physical limitations of EM-based systems is thermal effects due to Joule heating, to obtain large forces \cite{esmailie2017thermal}. The temperature is directly proportional to the actuation power ($P$) and the thermal dissipation obtained by the active and/or passive cooling. We evaluated the thermal behavior of our system for different power values. In this experiment, we set the current to on' for three minutes and then let the device cool down. Figure \ref{fig:temp} shows data from the middle coil actuated at PWM = 30\%. $T_{out}$ is the temperature measured at the coil boundary, measured with a \emph{Dallas DS18B20} sensor. $T_{in}$ is the average temperature of the copper wire obtained via the variation in resistance. We also plot the electrical current $I_y$ that drops as the coil heats up and the resistance increases. Note that no temperature compensation was used for building these thermal calibration curves. Each coil is able to accumulate some heat during the actuation and continuously dissipates it by the forced air circulation. Our system has a thermal time ($\tau_T$) on the order of minutes, in which it reaches the asymptotic temperature. The average power in the past $\tau_T$ seconds must be maintained within a safe value $P_{ave}$. Based on this plot, we choose $P_{ave} = 17 W$ per coil for our system. However, each coil can absorb peaks up to $15*P_{ave}$ for a few seconds.

Within this safe range, we calibrate the values of $\me$ for each axis as a function of the current in each coil with the hall sensors around the sphere (see Figure \ref{fig:hardware_labels}) and with Eq. \ref{eq:I_vs_me}. Figure \ref{fig:current_m} shows the experimentally attained magnetization in the core $\meBold$ as a function of the current. For reference, applying a power $P_0 = 100$ W to each coil ($I_i = 12.9$ A), the equivalent dipole is $\meBold = [2.52; 2.7; 2.82]$ Am$^2$. We also obtain non-zero terms away from the diagonal since the coils are not perfectly orthogonal, and we use the calibration data to correct the PWM duty cycles.

\begin{figure}[!t]
    \centering
    \medskip
    \includegraphics[width=0.675\columnwidth]{\dir/contact_free/figures/m_current2_fix.pdf}
    \caption{Electromagnet induced magnetization in each axis, $\meBold = (m_{e-x}, m_{e-y}, m_{e-z})$, as a function of the applied current settings ($I_x$, $I_y$, $I_z$). The magnetic field values are measured with hall sensors placed co-linear with each coil, and then transformed into $M$ values.}
    \label{fig:current_m}
\end{figure}

Finally, values for the force acting on the permanent magnet can be attained via setting the magnetic dipole of the tool and Eq. \ref{eq:Fr} and \ref{eq:Ft}. In our experiments, we use a ring-shaped neodymium magnet (12 mm outside diameter, 5 mm inside diameter, 24 mm high). For any tool with this particular magnet, with a center-to-center distance between dipoles of 5 cm, we obtain a ratio of force per electrical current of 48 mN/A. This means the device can handle an averaged constant force of $F_r=258$ mN ($P = 17$ W) with a peak force of up to $F_r = 959$ mN ($P = 230$ W) at full strength (using PWM control). This force value can be increased by increasing the volume of the tool magnet, with the trade-off of losing angular resolution and adding weight to the tool.