\section{Introduction}
A specific instance of a shared variable interface is a subset of haptic interfaces. Haptic interfaces are particularly interesting because they operate in the physical world, where the challenge of variable ownership is more impactful. Additionally, two agents exerting force on a single object intuitively exemplify shared variable interfaces. Furthermore, many emerging computing paradigms, such as virtual and augmented reality (VR/AR), rely on haptic feedback as an additional information channel to enhance the user experience. For example, in VR, haptic feedback increases the sense of presence and immersion by rendering collisions, shapes, and forces between the user and virtual objects.

For a haptic device to qualify as having a shared variable interface, it must function as both an input and output device. Many haptic devices primarily function as output devices, such as vibrotactile actuators embedded in handheld controllers \cite{whitmire2018haptic} or worn on the body \cite{hinchet2018dextres}. Vibrotactile feedback offers one-way communication and can only render coarse, non-localized haptic sensations. In contrast, our device allows for bi-directional communication on a shared variable. While complex setups such as articulated arms and exoskeletons can render large-force haptic feedback in three-dimensional space, they typically require force anchoring in the environment and involve cumbersome, bulky mechanisms, which hinder user uptake in walk-up-and-use scenarios.

To address the limitations of vibrotactile and large complex devices, we propose a contact-free, volumetric haptic feedback approach via an omnidirectional electromagnet. The device consists of a single 60 mm diameter spherical electromagnet capable of rendering attractive and repulsive forces onto permanent magnets embedded in pointing tools, such as a stylus or magnets worn on the user's fingertip. Leveraging a dipole-dipole approximation of the electromagnet-magnet interaction, our system can calculate and control the forces exerted on the permanent magnet in real-time, dynamically adjusting the force perceived by the user. The system can deliver perceptible forces up to 1N in a thin volume above the surface. Furthermore, we demonstrate that users can distinguish at least 25 different set-points separated by 18° on the surface of the sphere.

\begin{figure}[t]
\centering
\includegraphics[width=0.7\columnwidth]{\dir/contact_free/figures/teaser_v2.png}
\caption{We introduce a novel contact-free mechanism to render haptic feedback onto a tracked stylus via a hemispherical electromagnet. An approximate model of the magnet interaction and a computationally efficient control strategy allow for the dynamic rendering of attracting and repulsive forces, for example, allowing users to explore virtual surfaces in a thin shell surrounding the device (inset).}
\label{fig:syst_overview}
\end{figure}

To demonstrate the efficacy of our approach, we designed a functional prototype comprising an iron core and three custom-wound copper coils. The electromagnet is encased in a plastic dome upon which tools can come into contact and move about its surface (see \figref{fig:syst_overview}). The prototypical system can render radial (along the vector from the magnet to the tool) and tangential forces, both in attractive and repulsive polarities. The system can dynamically adjust the opening angle and steepness of the electromagnetic potential to gently guide the user towards a desired set-point in the thin volume above the device.

Modulating the magnetic field based on tool position opens the door to various interactive applications. For example, in virtual terrain exploration, the tool can be repelled when moved along mountains and attracted to valleys while descending (see \figref{fig:syst_overview}, inset). Another example is creating the sensation of stirring a viscous liquid by emulating the fluid's drag on the tool. To enable these interactive experiences, our device builds on three key components representing our contributions in this chapter:
\begin{itemize}
\item A computational model based on magnetic dipole-dipole interaction to produce force maps that allow for designing and generating location-dependent feedback,
\item The design and implementation of a 3 degree-of-freedom (DoF) spherical electromagnet prototype,
\item A control strategy that translates desired high-level forces into low-level input signals (currents/voltages) for the coils, fast enough for interactive use.
\end{itemize}
To assess the efficacy of the proposed design, we experimentally characterized the system properties and conducted a perceptual study exploring the thresholds for perception and localization capabilities of the electromagnetic actuation approach. Results from these early user tests indicate that users can perceive at least 25 different spatial locations with high precision.