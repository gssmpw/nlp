% !TeX root = ../main.tex

% \subsection{Related work}
% VR and wearable computing have seen rapid adaptation of haptics in recent years. Many such systems leverage vibro-tactile actuators for feedback. These are often embedded into hand-held controllers (e.g., HTC Vive), directly into displays \cite{Wellman1995} or are worn on the body \cite{Cybertouch, Gloveone}. Vibro-tactile actuation however can only render coarse, non-localized sensation. More complex setups often involving articulated arms or external braking mechanisms \cite{Massie94, Stamper1997, VanDerLinde2002, Araujo2016, zoller2019assessment} can reproduce higher fidelity haptics and render both tactile and kinesthetic feedback. Similarly, exoskeletons and gloves \cite{Cybergrasp, Gu2016, Choi2016}, or tilt platforms \cite{Prattichizzo2013, Kim2016} can produce large forces. These type of grounded approaches however require anchoring of the force in the environment requiring complex mechanical structures and adding bulk. As a result, they are mostly limited to use in high-end niches such as robotic surgery and tele-operation. 

% Recently much work has focused on providing rich, yet contact free haptic feedback, overcoming the need for expensive and complex robotic-arm like elements \cite{brink2014factors}. Many different actuation principles have been explored, including active motion control of the tip of a hand-held stylus \cite{kianzad2018harold}, ultra-sound pressure waves \cite{hoshi2010noncontact}, and even drone-based delivery of haptics \cite{heo2018thor}. However, by far the most practical way to provide contact-free haptics is via the use of magnetism. In the simplest case this can be achieved via integration of passive magnets into interactive objects, for example via 3D printing \cite{zheng2019mechamagnets, ogata2018magneto}, sometimes such approaches can be combined with sensing capabilities \cite{kuo2016gaussmarbles}. However, relying on permanent magnets does not allow any dynamic control over the perceived forces.  
% %\cite{yasu2019magnetact}

% %%%% Magnetically active haptics
% Electromagnets (EM) allow for computational control of the forces (and sometimes torques) and have been used to create planar EM arrays to interactively attract or repulse magnets embedded into styluses or directly worn on the user's finger \cite{weiss2011fingerflux, zhang2016magnetic, yamaoka2013depend, adel2019magnetic, berkelman2012co, berkelman2013interactive}. Electromagnetism has also been exploited to deliver contact-free vibration onto a magnet in a 3D pointing device \cite{McIntosh2019}. Moreover, leveraging the Lorentz force to actuate a coil between two permanent magnets can deliver precise and large mechanically grounded forces onto a joystick \cite{berkelman2009extending}. However, range-of-motion is limited and the handheld grip has to be mechanically connected to the powered coil, rendering  contact-free haptics infeasible.

% %%%% Especial mention to Omnimanget, but square
% Possibly the closest related work to ours are the Omnimagnet by Petruska et al. \cite{petruska2014omnimagnet} and its variants \cite{berkelman2018electromagnetic}. Like ours, the system generates an omni-directional magnetic field in the surroundings of the actuator. However, the design is composed of 2-3 nested cuboid coils, which causes rapid force decay as the user moves along the surface of the device. Furthermore, the construction complicates heat dissipation and thus limits the maximal strength and duration of generated forces \cite{esmailie2017thermal}, making it most suitable for rendering of vibrotactile stimuli onto a stylus in a fixed position \cite{zhang2018six}. Furthermore, these devices rely on a cubical design. That means the center-to-center distance ($d$) between the two magnets inevitably must vary as the user explores the surface. This translates into high variance of the forces due to the quartic decay with distance (i.e., $F \propto 1/d^4$). Our design is spherical, symmetric and, to the best of our knowledge, for the first time demonstrates rendering of symmetric, contact-free continuous forces inside a partial spherical shell, of $\pm 60^{\circ}$. We also propose a control algorithm that allows for dynamic shaping of the perceived force depending on the hand-held tool's position in 3D space.   




%%%% OLD TEXT - TO BE REWORKED %%%%

% \oh{Some leftover bits that I don't know / don't know what to do with:}
% In the other end, 1D haptic kits for for have been developed and release, en general for education proposes \cite{gassert2012physical, morimoto2014d81}. 

%  Grounded \emph{handed-tool} haptic devices are an in-between approach: like deformable surfaces, they are grounded at one end, providing for large reaction forces (kinesthetic haptics is possible). However, instead of requiring an array of actuators waiting to be touched, a \emph{tool} provides the right force at the right position, i.e., where the hand is.


%%%%    * Applications
%%%%   @Otmar: maybe you can put some of these ref in the intro
% Opportunities in shape changing interfaces \cite{alexander2018grand} 
% Sphere display \cite{benko2008sphere}
% Inflatable spherical display \cite{stevenson2011inflatable}
% Haptic training in medicine \cite{tong2017magnetic}


