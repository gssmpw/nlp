\begin{figure}[h]
 \center
  \includegraphics[width=0.8\columnwidth]{chapters/02_related_work/figures/hapticloop.pdf}
  \caption{The Haptic-Loop allows for bi-directional exchange of information between machine and user (\citet{hayward2004haptic}).}
  \label{fig:haptic-loop} 
\end{figure}
Human haptic perception, one of our primary senses, is spread throughout the body in the skin, muscles, and joints. It allows us to perceive our environment, both natural and simulated, through two distinct modes: tactile and kinesthetic feedback \cite{Culbertson18}. Tactile sensing provides information about pressure, shear forces, and vibrations via mechanoreceptors in the skin. Kinesthetic feedback, closely tied to proprioception, enables us to sense the position of our limbs, body posture, and large-scale forces acting on our hands and body through muscle and joint activations. Haptic perception is unique in that it actively involves mechanical interaction - we manipulate the environment through touch and force, using the resulting feedback to continuously adjust our actions.

A haptic interface facilitates a two-way flow of information, allowing the user and machine to exchange data simultaneously, setting it apart from haptic displays and related graphical displays that only provide one-way information transfer. The haptic interface provides haptic feedback based on the user's input, creating a "haptic loop." For example, a standard mouse (left in \figref{fig:haptic-loop}) can be enhanced to provide haptic feedback (right in \figref{fig:haptic-loop}) that resists movement when the user approaches a disabled on-screen action. \citet{hayward2004haptic} further describe this information exchange as an exchange of energy, a concept that can be applied when using physics-based optimization techniques to develop haptic interfaces.

Haptic interfaces are especially interesting as a use case in our settings of shared variables. First, haptic technologies have found renewed interest in the human-computer interaction community due to the recent reemergence of Augmented and Virtual Reality systems. Second, the physical nature of haptic interaction makes variable ownership more impactful.

\subsection{Devices}
\subsubsection{Mechanical haptic feedback}
The most common form of rendering haptic feedback has been vibrotactile actuators.
They are popular in commercial devices, such as the controllers of game consoles, AR and VR controllers, and mobile phones. They can also be embedded directly into displays \cite{Wellman1995} or in clothing \cite{Cybertouch, Gloveone}.

Vibrotactile actuators usually produce coarse and global feedback, especially when used in handheld controllers (\eg rendering touch contact in VR~\cite{Benko2016,Choi2018}). Researchers have investigated how to overcome this limitation, such as by rendering interpolations between several such motors~\cite{israr2011tactile} or strategically distributing them across a controller (\eg placing them under the fingers to render local grasp feedback~\cite{Lee2019Torc}).

Alternatives to vibrotactile feedback often involve more complex articulated haptic elements, such as arms and braking mechanisms (\eg \cite{Massie94, Stamper1997, VanDerLinde2002, Araujo2016, zoller2019assessment, Sinclair2019Capstan}). In general, these types of systems can provide local haptic feedback at higher levels of fidelity and can render both tactile and kinesthetic feedback. Further extensions of such systems are exoskeletons~\cite{Gu2016, Choi2016}, gloves~\cite{Cybergrasp, hinchet2018dextres}, and tilt-platforms~\cite{Prattichizzo2013, Kim2016}. These platforms are usually (rigidly) anchored and can therefore supply large forces. These approaches, except DextrES \cite{hinchet2018dextres}, rely on mechanical structures and anchoring to the environment.
Therefore, their use is mostly limited to high-end applications such as teleoperation.

\subsubsection{Contact-free haptic feedback}
A second line of research focuses on contact-free haptics, which provide rich and strong feedback, and overcome the need for expensive and complex mechanical setups \cite{brink2014factors}. 

Within the contact-free domain, many different actuation devices have been explored, \eg ultra-sound pressure waves \cite{hoshi2010noncontact}, active control of stylus motions \cite{kianzad2018harold}, and drones \cite{heo2018thor}. The most popular and practical actuators in this domain use magnetism. The simplest form of magnetism is delivered by passive magnets that are embedded into interactive objects (\eg~\cite{yamaoka2013depend}). The recent advance of consumer 3D printing has allowed this approach to actuate objects with arbitrary shape and function~\cite{zheng2019mechamagnets, ogata2018magneto}. A big shortcoming of passive magnets, however, is the lack of dynamic control over them and thus the forces users perceive during interaction.

\subsubsection{Electromagnetic haptic feedback}
The shortcoming of passive magnets can be addressed using electromagnetism and computational control of magnetic forces. Two-dimensional arrays of electromagnets can be combined with passive magnets that are worn \cite{weiss2011fingerflux, zhang2016magnetic, yamaoka2013depend, adel2019magnetic, berkelman2012co, berkelman2013interactive, berkelman2018electromagnetic} or embedded in tools and interactive objects \cite{ju2002origami, weiss2010madgets, Langerak:2021:Hedgehog}. Pangaro et al.\cite{pangaro2002actuated} model the force-field of each electromagnet and combine these using standard aliasing techniques, allowing directed movement of multiple objects on the surface. Similarly, Yoshida et al.\cite{yoshida2006proactive} use linear induction motors to control objects on a tabletop. Strasnick et al.\cite{strasnick2017shiftio} use electromagnets to control an object on a mobile phone case. Suzuki et al.\cite{suzuki2018reactile} combine these two works and use a grid of electromagnetic cores to move objects on a tabletop. The actuation area can be increased by attaching an electromagnet to a biaxial linear stage \cite{langerak2020magpen, langerak2019demonstration}.

Similarly, by leveraging the electromagnetic forces in a coil between two permanent magnets, large and grounded forces can be delivered onto a joystick \cite{berkelman2009extending}. The main drawback of this approach is the requirement for a mechanical connection, limiting the range of motion and impeding contact-free haptics. Senkal et al.\cite{senkal2009spherical, senkal2011haptic} and Li et al.\cite{li20072} propose to use magnetorheological fluids in joysticks. With the help of an electromagnetic field, the internal friction can be significantly increased. This allows for a large breaking force; however, it does not allow adding energy to the system.

Perhaps the most well-known haptic interface that builds on Lorentz forces is the Butterfly Haptics Maglev \cite{berkelman1996design, berkelman2000lorentz}. The design of this system consists of a flotor bowl with six integrated coils to which an interaction handle is rigidly attached. This flotor bowl is levitated between magnets assemblies that are part of a stator bowl. Due to the Lorentz levitation, haptic feedback can be achieved in a degree of rotary movement and also a small translation. Due to the small movements allowed, the device is mainly suitable for small-scale manipulations, e.g., where only the fingers are used.

The design of our system is fundamentally different. In contrast, \omniHap and \omniUIST use a single omnidirectional electromagnet, thereby largely increasing the rotary capabilities. Furthermore, by omitting the levitating flotor bowl, our devices greatly reduce the complexity and therefore make it easier to fabricate, thereby more likely to foster adaptation. Finally, our designs allow for contact-free haptics, which is not possible in the Butterfly system.

Closely related work to \omniHap and \omniUIST also includes Omnimagnet by Petruska et al. \cite{petruska2014omnimagnet} and its variants \cite{iqbal2019design}. Similar to our devices, their system generates an omnidirectional magnetic field. Their design differs by using three nested cuboid-shaped coils, causing force decay as the user moves along the surface of the device as well as an obstruction of heat dissipation. This limits the maximal strength and duration of actuation~\cite{esmailie2017thermal}, and also makes the device only suitable for rendering vibrotactile stimuli \cite{zhang2018six}. Due to the cuboid shape, the center-to-center distance between the electromagnet and the permanent magnet is not constant among the surface, which causes high variance in the forces perceived. In contrast, \textit{Omni}'s design is spherical, symmetrical, and has intertwined coils. This results in better heat dissipation and less variance in the force, thereby arguably improving the user experience.

\subsubsection{Haptic Guidance Devices}
Providing haptic guidance to users can provide benefits for learning \cite{teranishi2018combining} and short-term performance (cf. Abbink et al.\cite{Abbink2012}). Teranishi et al.\cite{teranishi2018combining} demonstrate that participants showed improved learning for handwriting skills when receiving guidance through a 3-DOF Phantom Omni device. Mullin et al.~\cite{mullins2005haptic} use a similar device as a handwriting aid for rehabilitation. Forsyth and MacLean \cite{Forsyth2006} show that force cues are beneficial in navigation tasks. The focus of these works is that users receive tight guidance (i.e., they are supposed to follow the system as closely as possible).

There exists a large range of devices and systems that aim at providing guidance to users. Comp*Pass \cite{Nakagaki14} uses pantograph-like devices to assist users in drawing, while I-Draw \cite{Fernando14} is a motorized drawing assistant. Lin et al. \cite{Lin16} use a magnet mounted on a small robotic arm to retain the correspondence between the pen and a portable base. Digital rubbing employs a comparable system using a solenoid for tracing over digital images on real paper \cite{kim2008digital}. While users handle larger-scale motions of the devices, they generally aim at having full control over the resulting drawing. Users can take back this control; however, these systems do not provide a way to guide users back to the target trajectory. Besides the aforementioned systems, several works aid users in the process of crafting and manufacturing (cf. Zoran et al.\cite{zoran2014wise}). Free-D \cite{zoran2013freed} and D-Coil \cite{peng2015d} assist users in sculpting physical artifacts by guiding them on a predefined 3D shape. Shilkrot et al.\cite{Shilkrot2014} propose an augmented paintbrush to assist users in painting. While users can override these systems to deviate from the target shape, they have no mechanism that guides users back to the target.

\subsection{Control of Haptic Devices}
To effectively control haptic devices, a closed loop needs to be formed. This means that the user needs to be sensed, and based on this inference, an actuation decision needs to be made. We first discuss magnetic sensing and then the control for haptic guidance devices.

\subsubsection{Magnetic Sensing}
Permanent magnets have been used for tracking objects in 3D, ranging from styli and other interactive objects \cite{liang2012gausssense, kuo2016gaussmarbles}, jewelry \cite{ashbrook2011nenya} all the way to fingers \cite{han2007wearable}, joints, and other biological tissues \cite{bhadra2002implementation, tarantino2017myokinetic}.

Ample research exists on tracking permanent magnets. Most of the existing literature uses isotropic (i.e., spherically) shaped permanent magnets, because the dipole model most accurately resembles these \cite{jackson2007classical}. However, some work also uses electromagnets attached to fingertips \cite{chen2016finexus}. Due to this, the fingers can be tracked individually. One of the biggest challenges is that a closed-form solution of the magnet states (e.g., position) is unlikely to exist in most scenarios. Therefore, optimization-based methods are commonly used \cite{schlageter2001tracking, taylor2019low} and more recently also neural networks \cite{russel2017neural}. However, these methods were often employed offline, suffered from large latency, or converged to local minima. 

A key related work is Magnetips by McIntosh et al.\cite{McIntosh2019}. They use a permanent magnet attached to a fingertip to interact with a watch. The permanent magnet is tracked around the watch and also used to provide vibrotactile haptic feedback. Magnetips multiplexes actuation and sensing. However, this causes significant delays (2 ms for every swap between tracking and actuating), which may pose an issue for scenarios that require continuous interactions.

The work that most resembles our work from an algorithmic point of view is \cite{taylor2019low}. They track multiple spherical magnets online using an analytical Jacobian, allowing solving with a quasi-Newton method. In contrast, \omniUIST tracks a single magnet while compensating for drastically changing electromagnetic fields, rather than tracking multiple permanent magnets in a static environment. We do this by adjusting the dipole model so that it is suitable for electromagnetism. We also propose a novel formulation of the position reconstruction problem and an implementation in PyTorch that can leverage the framework's auto-grad capabilities, thus avoiding the need for an analytical Jacobian.

\subsubsection{Haptic Guidance Control}
% General guidance
Closest to \magpen in terms of hardware is dePENd by Yamaoka et al.~\cite{yamaoka2013depend}.
They move a permanent neodymium magnet on a two-axis setup to control the metal tip of a ballpoint pen. The neodymium magnet "drags" the input pen around a predefined path, similar to a plotter. dePENd employs an open-loop strategy to control the magnet, which means users cannot deviate from the predefined path without risking losing haptic guidance. In contrast, in \magpen we propose a mathematical model and optimal control strategy that allows users to move at their own pace through a drawing, for example, and reacts in real-time to user input by altering the position and strength of the magnet. We show that our approach provides better results than their open-loop approach, as well as existing closed-loop approaches.

Kianzad et al.~\cite{Kianzad20} use a ballpoint drive to assist users in sketching. They employ a proportional-derivative (PD) control loop, which allows users to deviate from the target to a certain extent. We show in our experiments that our optimization scheme outperforms such existing closed-loop approaches. Muscle-Plotter \cite{Lopes16} proposes active guidance for users based on electrical muscle stimulation. Their control strategy is based on heuristics for users to share control with the system. Our approach could be applied to their work if the electromagnetic force model is replaced by a model of the interaction between the muscle stimulation and the force users produce.

Optimal reference following given real-world influences is studied in depth in the control theory literature. Methods like Model Predictive Control (MPC) \cite{Faulwasser:2009} optimize the reference path and the actuator inputs simultaneously based on the system state.
MPC is widely applied in many robotics (e.g., to control quadcopters, Mueller et al.\cite{Mueller2013}) and graphics applications (e.g., for human motion prediction, Da Silva et al.\cite{dasilva:2008:mpc}). However, Aguiar et al.~\cite{AGUIAR2008} show that the tracking error for following timed trajectories can be larger than if following a geometric path only.

To address this issue, Lam et al. propose Model Predictive Contouring Control (MPCC) \cite{lam2013model} to follow a time-free reference, optimizing system control inputs for time-optimal progress. MPCC has been successfully applied in industrial contouring \cite{lam2013model}, RC racing \cite{Liniger2014}, and in drone cinematography \cite{Naegeli:2017:MultiDroneCine}.

We also pose our optimization problem in this well-established framework. However, to the best of our knowledge, we are the first to apply it for haptic guidance systems where one has to consider both a controllable (i.e., the electromagnetic force) and non-controllable (i.e., the user) system. We contribute a formulation of the problem including models and control algorithms to enable employing MPCC in this context.
