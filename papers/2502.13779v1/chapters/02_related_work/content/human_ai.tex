Research on human-AI collaboration has revealed that integrated teams of humans and AI systems are more effective than either working independently \cite{bansal2019beyond, bansal2019updates, demir2018conceptual}. Research explores various aspects of this collaboration, such as conceptual architectures and frameworks \cite{johnson2019ai, madni2018architectural, oneill2020human, prada2014human}, and performance metrics \cite{bansal2019beyond}.

To gain insights into human-machine collaboration, researchers have applied psychological theories of human teamwork \cite{devisser2018automation, mou2017media}. These theories highlight core principles for bidirectional communication, trust, goals, situational awareness, language, intentions, and decision-making between humans and AI systems \cite{demir2017team, shively2017human}. This approach marks a shift from the traditional one-way communication model in conventional HCI contexts \cite{xu2023transitioning}, emphasizing a reciprocal relationship. In this dissertation, we embed this reciprocal relationship by modeling human-AI interaction as a multi-agent problem. Furthermore, having a shared variable inherently forms a bidirectional communication channel.

There are foundational concepts in interpersonal teaming that require adaptation for application to human-machine teams. These include Autonomy (compared to Automation), Trust (compared to Reliability), and Teaming (compared to the Use of Automation) \cite{greenberg2023foundational}. This adaptation has become particularly important in the context of autonomous car research \cite{li2016trolley, awad2018moral} and military applications \cite{chen2014human}. \citet{sycara2004integrating} identified three general roles that machines can support within teams: assisting individuals with their tasks, acting as equal team members, or supporting the team as a whole. This dissertation focuses on the first role, supporting individuals. Usually, this role is investigated in the context of decision support systems \cite{lyons2021humanautonomy}. However, in contrast to decision support systems, our focus is on optimizing the control of interfaces to best support user interactions.