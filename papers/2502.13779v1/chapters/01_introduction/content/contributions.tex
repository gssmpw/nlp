\chapter{Contributions}
We summarize the four main contributions of our dissertation.
\section*{1. \omniHapTitle}
Many kinesthetic haptic feedback devices rely on mechanical contraptions to provide forces. However, even when unactuated, these devices limit user agency, as the mechanical components restrict movement. To investigate the agency-automation trade-off, we need to design an \emph{untethered} haptic feedback device capable of delivering grounded forces. To that end, \textbf{we introduce \omniHap (\chapref{ch:shared:contact}), a novel contact-free volumetric kinesthetic haptic feedback device.} This system combines a spherical electromagnet with a dipole magnet model and a simple control law to deliver dynamically adjustable forces onto a handheld tool. We conducted a user experiment with 6 participants to characterize the force delivery aspects and perceived precision of our system.
\section*{2. \omniUISTTitle}
Many haptic devices (including \omniHap) require external optical tracking, which is often expensive, cumbersome, and requires a clear line of sight. To overcome these limitations, we developed \omniUIST (\chapref{ch:shared:volumetric}). \omniUIST is a haptic feedback device with integrated spatial tracking. The spatial tracking capabilities of \omniUIST are enabled by \textbf{a novel gradient-based method, which reconstructs the 3D position of the permanent magnet in mid-air using measurements from eight off-the-shelf Hall sensors integrated into the base.} Furthermore, following improvements to the actuator design, \omniUIST delivers over twice the forces compared to \omniHap. We detail \omniUIST's hardware implementation and our 3D reconstruction algorithm, providing an in-depth evaluation of its tracking performance. \omniUIST shows how integrating sensing and actuation provides natural interaction that encourages user agency. Both \omniHap and \omniUIST open up new applications in AR/VR, particularly in design, gaming, and object exploration.
\section*{3. \magpenTitle}
Existing haptic devices typically employ open-loop control, which does not account for user feedback. Alternatively, they use proportional–integral–derivative (PID) control or heuristics, which are usually based on timed references, limiting user agency. \textbf{We propose \magpen (\chapref{ch:control:optimal}), a time-independent closed-loop control strategy that allows users to retain agency while receiving haptic guidance.} Our real-time approach assists in pen-based tasks such as drawing, sketching, or designing. By iteratively predicting the motion of an input device, such as a pen, and adjusting the position and strength of an underlying dynamic electromagnetic actuator, our method provides flexible guidance without diminishing user control. Experimental results demonstrate that our approach is more accurate and preferred by users compared to open-loop and time-dependent closed-loop methods.
\section*{4. \marluiTitle}
Most control strategies rely on known system dynamics. However, in many HCI scenarios, users are integral to the system, and their behavior is complex and not easily modeled. To address this, we treat HCI as a multi-agent problem where an agent learns system, task, and user dynamics through interaction with a synthetic user agent. Specifically, \textbf{we introduce \marlui (\chapref{ch:control:multi}), a multi-agent reinforcement learning approach for adaptive user interfaces (AUIs).} In our formulation, a \useragent mimics real user interactions with the UI, while an \interfaceagent learns to adapt the UI to maximize the \useragent's performance. Our method captures the underlying task structure, system dynamics, and user behavior. Experiments show that the learned policies generalize well to real users and achieve performance comparable to data-driven supervised learning baselines.