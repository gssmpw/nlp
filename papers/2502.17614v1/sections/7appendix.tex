% \clearpage
\appendix
\section{Proof of Theorems}
\label{app:propositions}

% \subsection{Detailed Proof of Theorem \ref{theo:1}}
% \label{app:theo1}
% % \begin{proof}[Proof of Theorem \ref{theo:1}]
%  The test error can be expressed as:
%     \[
%     \| \mathbf{Y} - \hat{\mathbf{Y}}' \| = \| \mathbf{Y} - \mathbf{F} \mathbf{W}' \|
%     \]
    
%     By adding and subtracting \( \mathbf{F} \mathbf{W} \), we have
%     \[
%     \| \mathbf{Y} - \mathbf{F} \mathbf{W} + \mathbf{F} \mathbf{W} - \mathbf{F} \mathbf{W}' \| \leq \| \mathbf{Y} - \mathbf{F} \mathbf{W} \| + \| \mathbf{F} (\mathbf{W} - \mathbf{W}') \|
%     \]

%     This demonstrates that the test error of graph condensation is bounded by the conventional GNN training error (first term) and the additional parameter matching error introduced by condensation (second term).
% \end{proof}



\subsection{Proof of Theorems~\ref{theo:training_stage}}
\label{app:proof_training_stage}
\begin{theorem-nonumber}
    The prediction distance in the training stage is bounded by the sum of representation distance and parameter distance.
    \begin{equation}
    \| \mathcal{K}(\hat{\mathbf{Y}}) - \hat{\mathbf{Y}}' \| 
    \leq  \| \mathcal{K}(\mathbf{F}) - \mathbf{F}' \| \cdot \| \mathbf{W}' \| + \| \mathbf{F} \| \cdot \| \mathbf{W} - \mathbf{W}' \|
    \end{equation}
    where $\|\cdot\|$ denotes the L2 norm and $\mathcal{K}(\cdot)$ can be any projection function that aligns the dimensions of $\hat{{\bf Y}}$ and $\hat{{\bf Y}}'$ or ${\bf F}$ and ${\bf F}'$.
\end{theorem-nonumber}

\begin{proof}
To preserve the training data information to maintain the performance of GNNs, we focus on matching the model predictions on the original graph $G$ and its condensed counterpart $G'$. Since $\mathcal{K}(\cdot)$ only aligns the first dimensions, we have $\mathcal{K}(\hat{\mathbf{Y}}) = \mathcal{K}(\mathbf{F})\mathbf{W}$. Therefore, the expression becomes:
\begin{equation}
    \begin{aligned}
    &\| \mathcal{K}(\hat{\mathbf{Y}}) - \hat{\mathbf{Y}}' \| \\
    = &\| \mathcal{K}(\mathbf{F})\mathbf{W} - \mathbf{F}'\mathbf{W}' \| \\
    = & \| \mathcal{K}(\mathbf{F}) \mathbf{W} - \mathcal{K}(\mathbf{F}) \mathbf{W}' + \mathcal{K}(\mathbf{F}) \mathbf{W}' - \mathbf{F}’ \mathbf{W}' \| \\
    \leq &\| \mathcal{K}(\mathbf{F}) (\mathbf{W} - \mathbf{W}') \| + \| (\mathcal{K}(\mathbf{F}) -  \mathbf{F}')\mathbf{W}' \| \\
    \leq & \| \mathcal{K}(\mathbf{F}) \| \cdot \| \mathbf{W} - \mathbf{W}' \| + \| \mathcal{K}(\mathbf{F}) - \mathbf{F}' \| \cdot \| \mathbf{W}' \|
\end{aligned}
\end{equation}
The objective in training stage is to minimizing $\| \mathcal{K}(\hat{\mathbf{Y}}) - \hat{\mathbf{Y}}' \|$, which can be formulated as:
\begin{equation}
    \begin{aligned}
    &\arg \min_{\hat{\mathbf{Y}}'} \| \mathcal{K}(\hat{\mathbf{Y}}) - \hat{\mathbf{Y}}' \|\\
    =&\arg \min_{\hat{\mathbf{Y}}'} \| \mathcal{K}(\mathbf{F}) \| \cdot \| \mathbf{W} - \mathbf{W}' \| + \| \mathcal{K}(\mathbf{F}) - \mathbf{F}' \| \cdot \| \mathbf{W}' \|\\
    \end{aligned}
\end{equation}
Note that $\|\mathbf{F}\|$ is a constant, and the weight matrix $\| \mathbf{W}' \|$ is naturally constrained due to regularization techniques during model optimization to control its magnitude. Then, we have:
\begin{equation}
    \begin{aligned}
    &\arg \min_{\hat{\mathbf{Y}}'} \| \mathcal{K}(\hat{\mathbf{Y}}) - \hat{\mathbf{Y}}' \|\\
    \approx&\arg \min_{\mathbf{F}'}  \underbrace{\| \mathbf{W} - \mathbf{W}' \|}_{\text{Parameter Distance}} + \underbrace{\| \mathcal{K}(\mathbf{F}) - \mathbf{F}' \|}_{\text{Representation Distance}}
    \end{aligned}
\end{equation}




Therefore, Theorem~\ref{theo:training_stage} indicates that by minimizing the \textbf{representation and parameter distances}, the predictions derived from the condensed graph can be close to those of the original graph.

This completes the proof. 
\end{proof}

% To achieve the objective, mainstream GC methods employ distribution matching, kernel ridge regression (KRR)-based matching, trajectory matching, and gradient matching~\cite{wu2020comprehensive}. However, KRR-based matching has been shown to be less effective when the condensed graph is evaluated using Graph Neural Networks (GNNs)~\cite{gcsntk, gong2024gc4nc}. Consequently, we exclude KRR-based approaches from our analysis. In contrast, both trajectory matching and gradient matching inherently aim to minimize the discrepancy between the model parameters trained on the condensed graph and those trained on the original training set. This objective can be succinctly characterized as parameter matching. Therefore, we categorize these two methodologies under the parameter matching framework. In conclusion, our theoretical framework emphasizes \textbf{distribution matching} and \textbf{parameter matching}, as we believe these approaches encompass the majority of existing GC methods.


\subsection{Proof of Theorems~\ref{theo:test_stage}}
\label{app:proof_test_stage}
\begin{theorem-nonumber}
    The test prediction error of the GNN trained of $G'$ is bounded by the test prediction error of the GNN trained on $G$ plus the parameter distance, as formularized by 
    \begin{equation*}
        \| \mathbf{Y} - \hat{\mathbf{Y}}'' \| \leq \| \mathbf{Y} - \mathbf{F}\mathbf{W} \| + \|\mathbf{F}\| \cdot \| \mathbf{W}- \mathbf{W}'\|
    \end{equation*}
\end{theorem-nonumber}

\begin{proof}
At the test stage, our condensation goal is to ensure that the GNN model, trained on condensed training data $G'$, generalizes effectively to the test graph, i.e., achieving a low prediction error on the test data. 
\begin{equation}
\begin{aligned}
    &\| \mathbf{Y} - \hat{\mathbf{Y}}'' \|  \\
    = &\| \mathbf{Y} - \mathbf{F} \mathbf{W}' \|\\
    = & \| \mathbf{Y} - \mathbf{F} \mathbf{W} + \mathbf{F} \mathbf{W} - \mathbf{F} \mathbf{W}' \| \\
    \leq &\| \mathbf{Y} - \mathbf{F} \mathbf{W} \| + \| \mathbf{F} (\mathbf{W} - \mathbf{W}') \| \\
    \leq &\| \mathbf{Y} - \mathbf{F} \mathbf{W} \| + \| \mathbf{F}\| \cdot \|\mathbf{W} - \mathbf{W}' \|
\end{aligned}
\end{equation}
This inequality incorporates both the original test prediction error and the parameter distance. The objective in testing stage is to minimizing $\| \mathbf{Y} - \hat{\mathbf{Y}}''\|$, which can be formulated as:
\begin{equation}
    \begin{aligned}
    &\arg \min_{\hat{\mathbf{Y}}''} \| \mathbf{Y} - \hat{\mathbf{Y}}'' \|\\
    \approx&\arg \min_{\hat{\mathbf{Y}}'} \| \mathbf{Y} - \mathbf{F} \mathbf{W} \| + \| \mathbf{F}\| \cdot \|\mathbf{W} - \mathbf{W}' \|\\
    \approx&\arg \min_{\hat{\mathbf{Y}}'} \underbrace{\|\mathbf{W} - \mathbf{W}'\|}_{\text{Parameter Distance}} \\
    \end{aligned}
\end{equation}
It indicates that by reducing the \textbf{parameter distance} $\| \mathbf{W}- \mathbf{W}'\|$, the test prediction error becomes more tightly bounded, assuming that the original test prediction error \( \| \mathbf{Y} - \mathbf{F}\mathbf{W}\| \) and the propagated feature matrix \( \mathbf{F} \) remain constant. 

This completes the proof. 
\end{proof}


% \begin{proof}
% We start our theoretical analysis with formulating a simple but new objective for GC. 
% GC can be seen as a process of minimizing the loss by a GNN model trained in the synthetic graph ${G'}$. The objective function can reformulated as follows: 
% \begin{equation}
% {G'}=\underset{{G'}}{\arg \min } \ \mathcal{L}(\mathbf{Y}, \operatorname{GNN}({G'})). \label{eq:graphReduction}
% \end{equation}  
%  Besides, without loss of generality, we employ a regression loss computed over the entire graph to evaluate performance during the testing phase, corresponding to a transductive setting. In an inductive setting, the feature matrix \( \mathbf{F} \) can be replaced by \( \mathbf{F}_t \), where \( \mathbf{F}_t \) represents the propagated features in the test graph.

% After training on the original training set, the prediction of the SGC model is given by:
% \begin{equation}
%     \hat{\mathbf{Y}} = \mathbf{F} \mathbf{W}
% \end{equation}
% where \( \mathbf{F} = \mathbf{A}^K \mathbf{X}\) is the propagated feature matrix of the entire graph. 
% \( \mathbf{W} \in \mathbf{R}^{d \times c} \) is the weight matrix of the trained SGC model.

% Conversely, the prediction of the model trained on the synthetic (condensed) dataset is expressed as:
% \begin{equation}
%     \hat{\mathbf{Y}}' = \mathbf{F} \mathbf{W}'
% \end{equation}
% where \( \mathbf{W}' \in \mathbf{R}^{d \times c} \) denotes the weight matrix obtained from training on the synthetic dataset. 

% The loss function $\mathcal{L}$ during the testing phase in Equ.~\ref{eq:graphReduction} can be computed using the regression loss
% \begin{equation}
%     \mathcal{L} = \| \mathbf{Y} - \hat{\mathbf{Y}}' \|
% \end{equation}
% where \( \mathbf{Y} \in \mathbf{R}^{n \times c} \) represents the true labels.
%  The test error can be expressed as:
%     \[
%     \| \mathbf{Y} - \hat{\mathbf{Y}}' \| = \| \mathbf{Y} - \mathbf{F} \mathbf{W}' \|
%     \]
    
%     By adding and subtracting \( \mathbf{F} \mathbf{W} \), we have
%     \[
%     \| \mathbf{Y} - \mathbf{F} \mathbf{W} + \mathbf{F} \mathbf{W} - \mathbf{F} \mathbf{W}' \| \leq \| \mathbf{Y} - \mathbf{F} \mathbf{W} \| + \| \mathbf{F} (\mathbf{W} - \mathbf{W}') \|
%     \]
% Finally, the regression loss can be bounded by decomposing the error into two components:
% \begin{equation}
%     \| \mathbf{Y} - \hat{\mathbf{Y}}' \| \leq \| \mathbf{Y} - \mathbf{F}\mathbf{W} \| + \| \mathbf{F} (\mathbf{W} - \mathbf{W}') \|
% \end{equation}
% This completes the proof.
% \end{proof}

\subsection{Proof of Theorem \ref{theo:bound_of_parameter}}
\label{app:proof_bound_of_parameter}
\begin{theorem-nonumber}
    The parameter distance can be bounded by the following inequality:
    \begin{equation}
        \| \mathbf{W} - \mathbf{W}' \| \leq \mathcal{C}(\max (\text{diag}(\mathbf{\mathbf{P}^\top\mathbf{P}})))^2,
    \end{equation}
    where $\mathcal{C}=\frac{\|\mathbf{F} \| \cdot \| \mathbf{Y}\|({\lambda_{\min}(\mathbf{F}^\top\mathbf{F})}+\|\mathbf{F} \|)}{{\lambda^2_{\min}(\mathbf{F}^\top\mathbf{F})}}$ is a constant, $\mathbf{P}^\top\mathbf{P} \in \mathbb{R}^{N' \times N'}$ is a diagonal matrix with each diagonal entry corresponding to how many original nodes are assigned to each synthetic node. 
\end{theorem-nonumber}

\begin{proof}
We aim to establish that clustering effectively bounds the error introduced by parameter matching and representation difference. The proofproceeds as follows:
\paragraph{Bounding the Parameter Matching Error \( \| \mathbf{W} - \mathbf{W}' \| \):}
Consider the weight matrices for the SGC and clustering-based methods
\[
\mathbf{W} = (\mathbf{F}^\top \mathbf{F})^{-1} \mathbf{F}^\top \mathbf{Y}, \quad \mathbf{W}' = (\mathbf{C}^\top \mathbf{C})^{-1} \mathbf{C}^\top \mathbf{Y}'
\]
where
\[
\mathbf{C} = (\mathbf{P}^\top\mathbf{P})^{-1} \mathbf{P}^\top \mathbf{F}, \quad \mathbf{Y}' = (\mathbf{P}^\top\mathbf{P}))^{-1} \mathbf{P}^\top \mathbf{Y}
\]

The difference between \( \mathbf{W} \) and \( \mathbf{W}' \) is
\[
\| \mathbf{W} - \mathbf{W}' \| = \left\| (\mathbf{F}^\top \mathbf{F})^{-1} \mathbf{F}^\top \mathbf{Y} - (\mathbf{C}^\top \mathbf{C})^{-1} \mathbf{C}^\top \mathbf{Y}' \right\|
\]

By substituting $\mathbf{C}$ and $\mathbf{Y}'$, we can express \( \mathbf{W}' \) in terms of \( \mathbf{F} \) and \( \mathbf{Y} \)
\[
\mathbf{W}' = \left( \mathbf{F}^\top \mathbf{P} (\mathbf{P}^\top\mathbf{P})^{-2} \mathbf{P}^\top \mathbf{F} \right)^{-1} \mathbf{F}^\top \mathbf{P} (\mathbf{P}^\top\mathbf{P})^{-2} \mathbf{P}^\top \mathbf{Y}
\]

Let \( \mathbf{A} = \mathbf{F}^\top \mathbf{F} \) and \( \mathbf{B} = \mathbf{F}^\top \mathbf{P} (\mathbf{P}^\top\mathbf{P})^{-2} \mathbf{P}^\top \mathbf{F} \), then:
\[
\mathbf{W} = \mathbf{A}^{-1} \mathbf{F}^\top \mathbf{Y}, \quad \mathbf{W}' = \mathbf{B}^{-1} \mathbf{F}^\top \mathbf{P} (\mathbf{P}^\top\mathbf{P})^{-2} \mathbf{P}^\top \mathbf{Y}
\]

The difference becomes
\[
\mathbf{W} - \mathbf{W}' = \mathbf{A}^{-1} \mathbf{F}^\top \mathbf{Y} - \mathbf{B}^{-1} \mathbf{F}^\top \mathbf{P} (\mathbf{P}^\top\mathbf{P})^{-2} \mathbf{P}^\top \mathbf{Y}
\]

Similar to the above two proofs, we add and subtract a term \( \mathbf{B}^{-1} \mathbf{F}^\top \mathbf{Y} \) and rewrite the difference by
\begin{align*}
        \mathbf{W} - \mathbf{W}' = &\left( \mathbf{A}^{-1} - \mathbf{B}^{-1} \right) \mathbf{F}^\top \mathbf{Y} \\
&+ \mathbf{B}^{-1} \mathbf{F}^\top \left( \mathbf{I} - \mathbf{P} (\mathbf{P}^\top\mathbf{P})^{-2} \mathbf{P}^\top \right) \mathbf{Y}
\end{align*}

Considering the norms, we have
    \begin{align*}
\| \mathbf{W} - \mathbf{W}' \| \leq &\| \mathbf{A}^{-1} - \mathbf{B}^{-1} \| \cdot \| \mathbf{F}^\top \mathbf{Y} \| \\
&+ \| \mathbf{B}^{-1} \| \cdot \| \mathbf{F}^\top (\mathbf{I} - \mathbf{P} (\mathbf{P}^\top\mathbf{P})^{-2} \mathbf{P}^\top) \mathbf{Y} \|
\end{align*}

We will now bound each term independently.

\textbf{Bounding the First Term}
\begin{align*}
        \| \mathbf{A}^{-1} - \mathbf{B}^{-1} \| \cdot \| \mathbf{F}^\top \mathbf{Y} \|
\end{align*}


Based on the norms of matrix inequality, we have
\[
\| \mathbf{A}^{-1} - \mathbf{B}^{-1} \| \leq \| \mathbf{A}^{-1} \| \cdot \| \mathbf{B} - \mathbf{A} \| \cdot \| \mathbf{B}^{-1} \|
\]
Then, according to
\[
\| \mathbf{A}^{-1} \| = \frac{1}{\lambda_{\min}(\mathbf{A})}, \quad \| \mathbf{B}^{-1} \| = \frac{1}{\lambda_{\min}(\mathbf{B})}
\]
and assuming \( \lambda_{\min}(\mathbf{B}) \geq \frac{1}{(\max_k (\mathbf{P}^\top\mathbf{P})_{kk})^2} \lambda_{\min}(\mathbf{A}) \), we have
\[
\| \mathbf{A}^{-1} \| \cdot \| \mathbf{B}^{-1} \| \leq \frac{(\max_k (\mathbf{P}^\top\mathbf{P})_{kk})^2}{\lambda_{\min}(\mathbf{A})^2}
\]

Bounding \( \| \mathbf{B} - \mathbf{A} \| \):
\begin{align*}
    \mathbf{B} - \mathbf{A} &= \mathbf{F}^\top \mathbf{P} (\mathbf{P}^\top\mathbf{P})^{-2} \mathbf{P}^\top \mathbf{F} - \mathbf{F}^\top \mathbf{F} \\
    &= -\mathbf{F}^\top (\mathbf{I} - \mathbf{P} (\mathbf{P}^\top\mathbf{P})^{-2} \mathbf{P}^\top) \mathbf{F}
\end{align*}
Taking norms:
\begin{align*}
\| \mathbf{B} - \mathbf{A} \| &= \| \mathbf{F}^\top (\mathbf{P} (\mathbf{P}^\top\mathbf{P})^{-2} \mathbf{P}^\top - \mathbf{I}) \mathbf{F} \| \\
&\leq \| \mathbf{F} \|^2 \cdot \| \mathbf{I} - \mathbf{P} (\mathbf{P}^\top\mathbf{P})^{-2} \mathbf{P}^\top \|
\end{align*}
Since \( \| \mathbf{I} - \mathbf{P} (\mathbf{P}^\top\mathbf{P})^{-2} \mathbf{P}^\top \| \leq 1 \), the inequality can be further simplified to
\(\| \mathbf{B} - \mathbf{A} \| \leq \| \mathbf{F} \|^2\).

Combining the above:
\[
\| \mathbf{A}^{-1} - \mathbf{B}^{-1} \| \cdot \| \mathbf{F}^\top \mathbf{Y} \| \leq \frac{(\max_k (\mathbf{P}^\top\mathbf{P})_{kk})^2}{\lambda_{\min}(\mathbf{A})^2} \cdot \| \mathbf{F} \|^2 \cdot \| \mathbf{Y} \|
\]


\textbf{Bounding the Second Term}
\begin{align*}
    \| \mathbf{B}^{-1} \| \cdot \| \mathbf{F}^\top (\mathbf{I} - \mathbf{P} (\mathbf{P}^\top\mathbf{P})^{-2} \mathbf{P}^\top) \mathbf{Y} \|
\end{align*}


Based on the proof above, we first have
\[
\| \mathbf{B}^{-1} \| = \frac{1}{\lambda_{\min}(\mathbf{B})} \leq \frac{(\max_k (\mathbf{P}^\top\mathbf{P})_{kk})^2}{\lambda_{\min}(\mathbf{A})}
\]

 Since \( \| \mathbf{I} - \mathbf{P} (\mathbf{P}^\top\mathbf{P})^{-2} \mathbf{P}^\top \| \leq 1 \), 
\[
\| \mathbf{F}^\top (\mathbf{I} - \mathbf{P} (\mathbf{P}^\top\mathbf{P})^{-2} \mathbf{P}^\top) \mathbf{Y} \| \leq \| \mathbf{F} \| \cdot \| \mathbf{Y} \|
\]

Combining these bounds:
\begin{align*}
        &\| \mathbf{B}^{-1} \| \cdot \| \mathbf{F}^\top (\mathbf{I} - \mathbf{P} (\mathbf{P}^\top\mathbf{P})^{-2} \mathbf{P}^\top) \mathbf{Y} \| \\
        &\leq \frac{(\max_k (\mathbf{P}^\top\mathbf{P})_{kk})^2}{\lambda_{\min}(\mathbf{A})} \cdot \| \mathbf{F} \| \cdot \| \mathbf{Y} \|
\end{align*}


\paragraph{Combining Both Terms:}
Adding the bounds for both terms, we obtain:
\begin{align*}
\| \mathbf{W} - \mathbf{W}' \| \leq & \frac{(\max_k (\mathbf{P}^\top\mathbf{P})_{kk})^2}{\lambda_{\min}(\mathbf{A})^2} \cdot \| \mathbf{F} \|^2 \cdot \| \mathbf{Y} \| \\
&\quad + \frac{(\max_k (\mathbf{P}^\top\mathbf{P})_{kk})^2}{\lambda_{\min}(\mathbf{A})} \cdot \| \mathbf{F} \| \cdot \| \mathbf{Y} \| \\
=&\max_k (\mathbf{P}^\top\mathbf{P})_{kk}) \cdot (\frac{\| \mathbf{F} \|^2 \cdot \| \mathbf{Y} \|}{\lambda_{\min}(\mathbf{A})^2} + \frac{\| \mathbf{F} \| \cdot \| \mathbf{Y} \|}{\lambda_{\min}(\mathbf{A})}) \\
=&\mathcal{C}(\max (\text{diag}(\mathbf{\mathbf{P}^\top\mathbf{P}})))^2
\end{align*}
where $\mathcal{C}=\frac{\|\mathbf{F} \| \cdot \| \mathbf{Y}\|({\lambda_{\min}(\mathbf{F}^\top\mathbf{F})}+\|\mathbf{F} \|)}{{\lambda^2_{\min}(\mathbf{F}^\top\mathbf{F})}}$ is a constant, $\mathbf{P}^\top\mathbf{P} \in \mathbb{R}^{N' \times N'}$ is a diagonal matrix with each diagonal entry corresponding to how many original nodes are assigned to each synthetic node. 

Our objective is to minimize the parameter distance, which can be reformulated by:
\begin{equation}
    \begin{aligned}
    &\arg \min_{\mathbf{W}'} \| \mathbf{W} - \mathbf{W}' \|\\
    \approx&\arg \min_{\mathbf{P}} \mathcal{C}(\max (\text{diag}(\mathbf{\mathbf{P}^\top\mathbf{P}})))^2\\
    \end{aligned}
\end{equation}

This completes the proof.   
\end{proof}
    
% \paragraph{Conclusion.} \textbf{First}, larger \( \max((\mathbf{P}^\top\mathbf{P})_{kk}) \) increases the bound on parameter matching error. \textbf{Second}, lower \( \sigma_{ck}^2 \) leads to smaller representation matching error, enhancing the overall condensation quality. These two factors are directly related to clustering quality. \textbf{Third}, a larger \( \lambda_{\min}(\mathbf{A}) \) decreases the bound, indicating that better-conditioned covariance matrices lead to tighter error bounds. \textbf{Finally}, higher \( \| \mathbf{F} \| \) and \( \| \mathbf{Y} \| \) contribute to larger error bounds. These two are predefined by the original dataset.

% \section*{Incremental vs.\ Batch-wise (Streaming) kmeans++}

% We have a dataset $\mathcal{X} = \{\mathbf{x}_1, \dots, \mathbf{x}_n\} \subset \mathbb{R}^d$ 
% and want $k$ centroids. Two procedures are common:

% \begin{enumerate}
%   \item \textbf{Standard kmeans++:} Select the first centroid $\mathbf{c}_1$ uniformly at random; 
%   for each subsequent centroid $\mathbf{c}_{t+1}$, 
%   use the distance-based probabilities 
%   $P(\mathbf{x}_i) = D(\mathbf{x}_i)^2 / \sum_{\mathbf{x}_j} D(\mathbf{x}_j)^2,$
%   where 
%   $D(\mathbf{x}_i) = \min_{1 \le s \le t}\|\mathbf{x}_i - \mathbf{c}_s\|.$
%   \vspace{0.2em}
%   \item \textbf{Incremental (or Streaming) kmeans++:} 
%   Split $\mathcal{X}$ into batches or handle data in an online manner. 
%   In each batch, we pick a subset of centroids using a similar \emph{kmeans++-like} rule, 
%   then \emph{merge} these partial centroids at the end, possibly running an extra final kmeans++ step on the merged set.
% \end{enumerate}

% \subsection*{1.\ Incremental Selection (One Centroid at a Time): Exact Equivalence}

% \paragraph{Algorithmic Description.} Instead of computing $D(\mathbf{x}_i)$ from scratch each time, we keep a \emph{running} (incremental) record of the distances:

% \begin{itemize}
%   \item Pick the first centroid $\mathbf{c}_1$ uniformly at random from $\mathcal{X}$.
%   \item Maintain for each point $\mathbf{x}_i$ a distance $D(\mathbf{x}_i) = \min_{1 \le s \le t}\|\mathbf{x}_i - \mathbf{c}_s\|^2$,
%         updated whenever a new centroid $\mathbf{c}_{t+1}$ is chosen:
%   \[
%     D(\mathbf{x}_i) \;=\; 
%       \min\Bigl(D(\mathbf{x}_i), \|\mathbf{x}_i - \mathbf{c}_{t+1}\|^2\Bigr).
%   \]
%   \item Sample the next centroid $\mathbf{c}_{t+1}$ with probability 
%   \[
%     P(\mathbf{x}_i) \;=\; 
%       \frac{D(\mathbf{x}_i)}{\sum_{\mathbf{x}_j \in \mathcal{X}} D(\mathbf{x}_j)},
%   \]
%   \item Repeat until we have $k$ centroids.
% \end{itemize}

% \paragraph{Proof of Equivalence.} 
% At each centroid-selection step $t$, the \emph{distance values} $\{D(\mathbf{x}_i)\}$ in the incremental method match exactly the corresponding distances in standard kmeans++, because in both cases 
% \[
%   D(\mathbf{x}_i) \;=\;
%   \min_{1 \le s \le t}\|\mathbf{x}_i - \mathbf{c}_s\|^2.
% \]
% Hence, both algorithms \emph{sample the next centroid} from \emph{the same probability distribution} over $\mathcal{X}$. 
% By induction on $t=1,\dots,k$, the two methods select the \emph{same} (or identically distributed) set of $k$ centroids. 
% Thus, \emph{incremental kmeans++} (one-at-a-time) is \textbf{exactly the same} as standard kmeans++ if they use the same random seed.

% \subsection*{2.\ Batch-wise (Streaming) kmeans++: Approximate Equivalence}

% In a streaming or large-scale setting, we may instead:
% \begin{enumerate}
%   \item Partition $\mathcal{X}$ into batches $\mathcal{X}_1, \mathcal{X}_2, \dots, \mathcal{X}_T$ (e.g.\ chunks or time windows).
%   \item On each batch $\mathcal{X}_t$, run a local kmeans++ to extract $k_t$ ``partial centroids'' 
%         $\{\mathbf{p}_1^{(t)}, \dots, \mathbf{p}_{k_t}^{(t)}\}$.
%   \item Merge all partial centroids $\{\mathbf{p}_j^{(t)}\}$ into one set, 
%         optionally weighting each centroid by the number of data points in its batch-cluster,
%         and run a final round of kmeans++ on this \emph{compressed} or merged set to extract $k$ global centroids.
% \end{enumerate}

% \paragraph{Why this is approximate.} In standard kmeans++, the distance probabilities are computed against \emph{all points} from \(\mathcal{X}\) at each centroid selection. 
% In the batch-wise approach, we only compute local distances within each batch and then later refine by clustering the merged centroids. 
% Hence, we generally cannot guarantee \emph{exact} equivalence: once data is partitioned, the cross-batch interactions are not fully accounted for in each local selection step. 

% However, one can show that if each batch-level selection is performed by kmeans++ and we preserve the relative weighting of partial centroids appropriately, 
% the final global clustering has a provable approximation factor relative to running a \emph{single} standard kmeans++ on all of \(\mathcal{X}\). 
% For instance, a common result in the coreset and streaming literature is that the final error (cost) of the merging approach is within a small constant factor (or $O(\log n)$ in the worst case) of the cost of a global optimum or a global kmeans++ initialization. 

% \paragraph{Sketch of Approximation Argument.}
% Suppose each batch $\mathcal{X}_t$ is of size $n_t$, with $\sum_{t=1}^\top n_t = n$. 
% Running local kmeans++ with $k_t$ partial centroids on batch $t$ yields a partial solution whose cost is close to the best $k_t$-clustering cost on that batch. 
% Collecting these partial centroids (with weights proportional to $n_t$ or the cluster sizes) forms a \emph{weighted coreset} \(\mathcal{P}\). 
% A final kmeans++ on \(\mathcal{P}\) yields $k$ global centroids. 
% By standard coreset arguments (and the properties of kmeans++ on weighted datasets), the final solution is guaranteed to be within a \emph{bounded factor} of the global kmeans++ cost on the full dataset. 
% Exact approximation ratios depend on how many centroids $k_t$ we extract per batch, the total number of batches, and the structure of the data. 
% Typical results show that for sufficiently large $k_t$ (relative to $k$) or well-chosen merges, 
% the final solution approximates a global kmeans++ solution to within constant or polylogarithmic factors.

% \subsection*{Summary}

% \begin{itemize}
%   \item \textbf{One-at-a-time incremental kmeans++ is exactly equivalent} to standard kmeans++ 
%   because the distance-based sampling distribution at each step is identical.

%   \item \textbf{Batch-wise (streaming) kmeans++ is generally an approximation} 
%   because each batch makes centroid choices without full knowledge of data in other batches. 
%   However, coreset/merging techniques plus a final global pass can ensure the final clustering 
%   is within a guaranteed approximation factor of the single-batch (standard) kmeans++ solution.
% % \end{itemize}
\section{Method Pipeline}
\begin{figure}[h]
    \centering
    \includegraphics[width=0.9\linewidth]{figs/main.pdf}
    \caption{Illustration of the GECC framework. The lower part of the figure depicts the two evolving settings: in the \textbf{inductive} setting, the graph structure evolves over time, while in the \textbf{transductive} setting, only the training nodes (labels) change, with the graph structure and non-training nodes remaining unchanged.}

    \label{fig:main}
\end{figure}

The proposed GECC framework is illustrated in Figure~\ref{fig:main}. GECC takes the current graph \( G_t = (\mathbf{X}_t, \mathbf{A}_t) \) along with the centroids from the previous time step to perform condensation and generate the condensed graph \( G'_t = (\mathbf{I}, \mathbf{C}'_t) \). The lower part of the figure also depicts the two evolving settings: in the \textbf{inductive} setting, the graph structure evolves over time, while in the \textbf{transductive} setting, only the training nodes (labels) change, whereas the graph structure and non-training nodes remain unchanged.



\section{Experimental Details}\label{app:exp}
\section{Dataset Generation}
\label{sec:dataset}
\revise{
To train the proposed GNN, we constructed a dataset of building structures and a subset of these structures were subjected to fire simulations using FEA. The dataset generation process is illustrated in \figref{fig:dataset_generation_procedure}. Initially, a total of 33,000 building structures with geometrical details, material properties, and gravity loads were created. Due to randomness in generating these structures, a filter is applied to remove unreasonable data after gravity load simulation, which included 15,377 structures. A trade-off between computational feasibility and model performance is made among the remaining 17,623 structures. As further labeling structures with MIDR requires resource-intensive fire simulations via OpenSeesRT, a large proportion of 16,050 structures is selected as unlabeled dataset. On the other hand, each of the other 1,573 structures was further subjected to 30 different fire simulations, forming the labeled dataset containing $1,573\times 30 = 47,190$ fire cases.} This section details the step-by-step process for generating the dataset, including geometry creation, material property assignment, and simulations due to gravity loads and fire scenarios. 
% To train the proposed neural network, we constructed a dataset comprising building structure data and a subset of fire scenario data. The dataset generation process is illustrated in \figref{fig:dataset_generation_procedure}. 
% A total of 33,000 building structures with geometric details, material properties, and gravity loads were initially created. Out of these, 3,000 structures were selected as labeled data, and the remaining 30,000 were designated as unlabeled data. Further, about half of them filtered out due to instability under gravity loads only. 
\begin{figure*}[h!]
    \centering
    \includegraphics[width=0.8\linewidth]{figures/dataset_filter_procedure.pdf}
    \caption{Workflow for dataset generation (geometry, material property, gravity loads, and fire scenarios).}
    \label{fig:dataset_generation_procedure}
\end{figure*}

\subsection{Geometry Generation}
\label{subsec:geometry_generation}
The geometry of the building structures forms the foundation of the dataset. Regular 
\revise{3D structures} resembling multi-story parking structures or shopping malls were generated, with parameters such as building floor dimensions and story heights selected randomly. Each building structure is composed of multiple rooms, which serve as the basic unit in this study. A room herein is a cuboid space defined by specific length, width, and height. Within a structure, rooms of the same dimensions are uniformly arranged along the length, width, and height, corresponding to the $x$-, $y$-, and $z$-axes, respectively. Structures vary in room size and number of rooms along each axis. Specifically, the room length, width, and height are independently sampled from a uniform distribution within the interval $[2, 5]$ meters along the three directions of the structure. Similarly, the room number along each axis is uniformly sampled independently as an integer within the interval $[2, 7]$, i.e., the maximum number of stories of the buildings simulated in this study is 7.

To introduce variability and simulate real-world scenarios, approximately $8\%$ of structural elements (beams or columns) are randomly removed after initial geometry creation. 
\revise{Such removal is not fire-induced damage, but reflects functional diversity often observed in real buildings, such as open spaces designed for activities in shopping malls, e.g., ice skating rinks. Examples of the generated geometries are illustrated in \figref{fig:example_generated_geometry}, showcasing the diversity and realism of the dataset. This element removal does not affect the definition of room's geometry in the structure and nor does it affect the number of considered fire scenarios.} 

\revise{A range of coefficient of variation values ($3.3\%$ to $17.5\%$) was derived from prior studies that investigated the statistics of geometrical and material properties of structural components of buildings (e.g., \cite{mirza1979variations, lee2004probabilistic}). These studies provide empirical data on the natural variability in parameters such as Young's modulus, yield strength, and dimensions of structural elements due to manufacturing tolerances and material inconsistencies. By selecting $8\%$ for the removal of structural elements in our database, we aimed to maintain a level of variability that is representative of real-world uncertainties while ensuring computational feasibility. This choice ensures that the database captures realistic deviations without introducing extreme cases that may not be commonly encountered in practice.}

\begin{figure*}[h!]
    \centering
    \includegraphics[width=\linewidth]{figures/example_generated_geometry.pdf}
    \caption{Examples of generated structural geometry of different sizes (all dimensions in meters).}
    \label{fig:example_generated_geometry} 
\end{figure*}

{\blockRevise

In this study, we opted for a deterministic square, dimension of $0.1$ m, solid cross-sectional steel elements due to their simplicity in modeling and analysis. Square sections exhibit uniform geometrical properties in all directions, simplifying the computation of structural responses and avoiding complications associated with more complex shapes, such as wide-flange sections, facilitating the computational efficiency and scalability to generate a large dataset. This choice also helps to mitigate issues related to stress concentrations and facilitates a more straightforward representation of structural behavior under thermal loads. 

\textit{Remark:} The selected cross-section provides a comparable flexural rigidity to a $W 130 \times 130 \times 28.1$ wide-flange section (metric units), albeit with significantly higher axial rigidity. This cross-section is acceptable for gravity-load-designed frames under service loading conditions where the models assume fully rigid, moment-resisting beam-column connections for the evaluation of the IDR under thermal loading. This assumption is reasonable in this computational study where the primary interest is to understand the global deformation response of frames under fire conditions. The selection of uniform square cross-sections for both beams and columns, rather than adherence to standard capacity design principles, was made here primarily for computational efficiency and to reduce design parameters in the database generation process. This choice allows for simplified and scalable approach to analyze the fire-induced response of generic steel frames without the need for large section variations, where this study mainly focuses on the fire vulnerability assessment using ML-based predictions. However, if additional loading conditions, e.g., seismic or wind loads, were to be considered, larger sections, strong-column/weak-beam principle, and ductile detailing would be required in the generated buildings for realistic structural behavior under combined loading conditions. Future studies may also consider investigating the influence of variable cross-sectional dimensions and semi-rigid connections on the structural performance under fire conditions. 
} % blockRevise

\subsection{Material Properties}
Steel is chosen as the material for the structures. To reflect real-world variations, we randomly assign one of five slightly different steel material types to each structural element. \revise{
The ranges of material properties are provided in \tabref{tab:material_property_ranges} and the properties are sampled from uniform distributions of the corresponding ranges. These variations simulate differences arising from manufacturing batches or regional material properties. That these properties are at ambient temperature and change when the temperature rises due to a fire. The selection of materials with varying properties is aimed at increasing the diversity of the data. Our goal is to represent as wide a range of data as possible with a limited amount of building structure data, thereby enhancing the generalization ability of the GNN. Our assumed material property ranges are expected to be wider than the real-world conditions based on findings in \cite{mirza1979variations, lee2004probabilistic}. Therefore, we are essentially tackling a more challenging and general task. If we can solve this problem, we are confident that our method will perform equally well or even better in real-world scenarios.
}
\begin{table}[h!]
    \centering
    \caption{Material properties ranges for considered steel structures.}
    \begin{tabular}{lc}
        \toprule
        Property & Range \\
        \midrule
        Young's modulus & [168, 252] GPa \\
        Yield strength & [220, 330] MPa \\
        Strain-hardening ratio & [0.8, 1.2] \% \\
        \bottomrule
    \end{tabular}
    \label{tab:material_property_ranges}
\end{table}

\subsection{Gravity Loads}
Gravity loads are applied to columns and beams based on their \revise{influence (tributary) areas as typically conducted in structural analysis. The considered ``service'' load conditions include the column self-weight and the additional loads directly supported on the beams from their self-weight and weights of the reinforced concrete slabs, people as live load, and building content. An edge beam typically carries approximately half the gravity load supported by a parallel interior beam}. The ranges of gravity loads are listed in \tabref{tab:gravity_load_ranges}. \revise{The loads are sampled from uniform distributions of the corresponding ranges.} Structures that failed to meet an MIDR threshold of $1\%$ under gravity loads were deemed unacceptable designs and filtered out, as such configurations of randomly chosen geometry, material, and gravity load combinations were considered unrealistic from a regulatory and practicality points of view.
\begin{table}[h!]
    \centering
    \caption{Gravity load ranges for considered beams and columns.}
    \begin{tabular}{lc}
        \toprule
        Element & Range (kN/m)  \\
        \midrule
        Column & [0.5, 1.0]  \\
        Edge beam & [1.5, 4.5]  \\
        Interior beam & [3.0, 7.5]  \\
        \bottomrule
    \end{tabular}
    \label{tab:gravity_load_ranges}
\end{table} 

\subsection{Rule-based Thermal Load Generation}
\label{subsec:thermal_load_generation}
To evaluate a building's structural response during a fire event, we employed a simplified rule-based approach for thermal load generation. 
% Previous studies \cite{nan_structuralfire_2023} have demonstrated that steel structures rapidly equilibrate with surrounding gases temperatures due to efficient heat exchange. Consequently, gas temperatures can be directly used as inputs for FEA tools, e.g., OpenSees, simplifying the process of modeling thermal loads. 
% Accurately simulating temperature fields in fire scenarios poses significant challenges. Advanced thermodynamic simulations, such as those performed using Fire Dynamics Simulator (FDS) \cite{mcgrattan_fire_2000}, provide precise temperature predictions. However, these methods are hindered by high computational costs, prolonging execution times, and limited scalability, making them impractical for generating large datasets. Additionally, real-world fire loads often display substantial spatial variability across different rooms \cite{dundar_fire_2023}, resulting in scenario-specific temperature fields with limited generalizability. For example, studies on bridge fires \cite{he_study_2024} have demonstrated that environmental factors, such as wind speeds, can significantly influence temperature distributions. Furthermore, even within identical scenarios, variations in fire modeling methodologies can produce distinctly different temperature fields \cite{zhang_temperature_2020, du_new_2012}. These challenges emphasize the need for efficient and adaptable methods to generate fire temperature data.
% To address these issues, we adopted a rule-based approach to model temperature variations. 
According to \cite{spearpoint_fire_2008}, a typical fire development follows a predictable pattern. During the {\em{growth stage}}, the temperature rises slowly and approximately linearly after ignition. This is followed by the {\em{flashover stage}}, where temperatures increase rapidly to peak values. After reaching the peak, the temperature either stabilizes or continues to rise slowly until the {\em{decay stage}} begins. Inspired by this fire development pattern, we describe the temperature evolution in time, $t$, prior to the decay stage in two distinct stages:
\begin{enumerate}
    \item {\bf{Initial linear increase stage}}: For $t \in [0, t_1)$, temperature increases gradually and linearly as the fire spreads through the building. This stage represents the time before the fire directly affects a structural element.  
    \item {\bf{ISO 834 fire curve stage}}: For $t \in [t_1, t_{\thre}]$, temperature rises rapidly following the ISO 834 curve \cite{ISO834}, modeling the direct impact of the fire on the structural element. 
\end{enumerate}
The slope of the linear temperature increase, $c$, and the transition time, $t_1$, are influenced by the spatial relationship between the fire source and the structural element. For the second stage of temperature evolution, we utilize the ISO 834 curve, a widely accepted standard for fire resistance testing. This standardized fire curve describes the temperature rise over time, enabling rapid and consistent thermal fields across various scenarios. The duration of fire simulation in this study is set to $t_{\thre}=60$ minutes. This value represents the upper limit for the temperature evolution of each structural element, providing a consistent basis for analyzing the structural response to fire.

Let $(x, y, z)$ represents the midpoint of a structural element and $(x_{\subfire}, y_{\subfire}, z_{\subfire})$ the fire source point. \revise{Integer parameters $h$ and $h_{\subfire}$ correspond to the respective floor levels of the element and the fire source}. The temperature evolution for each element is expressed as follows:
\begin{enumerate}
    \item Linear increase stage ($0 < t < t_1$):
    \begin{equation}
    T(t) = c \cdot t,
    \end{equation}
    where $c$, the rate of temperature increase ($^\circ\mathrm{C}/\mathrm{min}$), depends on the height difference between the element, $h$, and the fire source, $h_{\subfire}$:
    \begin{equation}
        c = 
        \begin{cases} 
        5\left/\left(h - h_{\subfire} + 1\right)\right., & h \geq h_{\subfire}, \\
        2\left/\left(h_{\subfire} - h\right)\right., & h < h_{\subfire}.
        \end{cases}
    \end{equation}
     \item ISO 834 stage ($t \geq t_1$):
\begin{equation}
    T(t) = c \cdot t_1 + 345 \log_{10} \left(8 \left(t - t_1\right) + 1\right).
\end{equation}
\end{enumerate}

The transition (arrival) time $t_1$, marking the end of the linear stage, depends on the spatial distance between the fire source and the element. We define the following two Euclidean distances $L_p$ in the $xy$ plane and $L_s$ in the $xyz$ space:
\begin{eqnarray}
L_p & \triangleq & \sqrt{(x - x_{\subfire})^2 + (y - y_{\subfire})^2}, \\
\label{eq:Lp}
L_s & \triangleq & \sqrt{(x - x_{\subfire})^2 + (y - y_{\subfire})^2 + (z - z_{\subfire})^2}.
\label{eq:Ls}
\end{eqnarray}
Accordingly, the transition time, $t_1$, is expressed as follows:
\begin{equation}
    t_1 = 
    \begin{cases}
    \beta_{1} \cdot \left(1 - \exp\left\{- L_s\left/\alpha_{1}\right.\right\}\right), & h > h_{\subfire}, \\
    \beta_{2} \cdot \left(1 - \exp\left\{- L_p\left/\alpha_{2}\right.\right\}\right), & h = h_{\subfire}, \\
    \beta_{3} \cdot \left(1 - \exp\left\{- L_s\left/\alpha_{3}\right.\right\}\right), & h < h_{\subfire} .
    \end{cases}
    \label{eq:t1}
\end{equation}
The parameters $\beta_i$ and $\alpha_i$ for determining $t_1$ are summarized in Table~\ref{tab:fire_spread_parameters}. In this study, we take $r_{\mathrm{up}}=0.95$ and $r_{\mathrm{down}}=0.97$.
\begin{table}[ht]
    \centering
    \caption{Fire spread parameters for $t_1$ calculations.}
    \begin{tabular}{lcc}
        \toprule
        Case  & $\beta_i$ & $\alpha_i$  \\
        \midrule
        $i=1$, Upward spread & $16 \left.\left(1-r_{\mathrm{up}}^{\left|h-h_{\subfire}\right|}\right)\right/\left(1-r_{\mathrm{up}}\right)$ & $10$  \\
        $i=2$, Horizontal spread & $18$ & $18$  \\
        $i=3$, Downward spread & $30 \left.\left(1-r_{\mathrm{down}}^{\left|h-h_{\subfire}\right|}\right)\right/\left(1-r_{\mathrm{down}}\right)$ & $5$  \\
        \bottomrule
    \end{tabular}
    \label{tab:fire_spread_parameters}
\end{table}

\figref{fig:t1_curve} illustrates the $t_1$ curves for various fire scenarios: (1) fire originating on the lower floor, $h-h_{\subfire}=1$ with rapid upward spread, (2) fire on the same floor, $h=h_{\subfire}$ with the fastest spread, and (3) fire on the upper floor, $h_{\subfire}-h=1$ with slow downward spread. The exponential decay in $t_1$ reflects the accelerating fire propagation speed as the distance increases. \figref{fig:t1_curve} also indicates that the employed simplified model is consistent with the Markov chain-based dynamic model given by \cite{cheng_dynamic_2011}, where the rooms at the same floor of the fire point start flashover slightly before the corresponding upper floors. Additionally, $\beta_{1}$ and $\beta_{3}$ are the summation of a geometric sequence, where story level $h$ is the index. The common ratios $r_{\mathrm{up}}<1$ in $\beta_{1}$ and $r_{\mathrm{down}}<1$ in $\beta_{3}$ indicate that the fire speeds up to spread through the next story, which is consistent with the real-world fire spread mechanism given in \cite{hokugo_mechanism_2000}. The temperature profile within the range $t \in [0, t_{\thre}]$ is subsequently used as the thermal load in OpenSeesRT simulations to compute displacements at each structural node at time $t_{\thre}$.
\begin{figure}[h!]
    \centering
    \includegraphics[width=0.8\linewidth]{figures/m204_t1_curve.pdf}
    \caption{Three examples for the $t_1$ curve.}
    \label{fig:t1_curve}
\end{figure}

\revise{
\textit{Remark:} The effects of structural elements, such as concrete floor slabs and partitions, are not explicitly modeled in our approach. Instead, their influence is implicitly captured through the careful selection of the parameters $ \alpha, \beta, r_\mathrm{up} $, and $ r_\mathrm{down} $. This parameterization provides a unified framework for generating temperature fields. Indeed, fire propagation is governed by a multitude of factors and remains an open research question. For instance, if the fire resistance of a floor slab is enhanced by fire protective coating, the corresponding model can account for this by decreasing $\alpha_1$ \& $\alpha_3$, increasing $\beta_1$ \& $\beta_3$, and adopting larger values for $r_\mathrm{up}$ \& $r_\mathrm{down}$, which collectively slow down the vertical spread of fire. Conversely, scenarios involving higher amounts of combustible materials would warrant the opposite adjustments. This flexible and integrated approach avoids the need to design separate models for different fire propagation scenarios while still capturing the essential effects.
}

\revise{
In conclusion, our rule-based approach is a computationally efficient method for approximating fire temperature fields, enabling large-scale dataset generation to train predictive models. By combining ISO 834 fire curves with spatial considerations and embedding structural effects through parameter calibration, the method achieves a balanced trade-off between accuracy and scalability, making it a practical solution for thermal load modeling in fire scenarios. After generating the temperature of each beam or column according to the middle point, the temperature is applied as uniform thermal load to the elements of the structure in question using OpenSeesRT. 
}

% In conclusion, this rule-based approach is a computationally efficient method to approximate fire temperature fields, enabling large-scale dataset generation to train predictive models. By combining ISO 834 fire curves with spatial considerations, the method balances accuracy and scalability, making it a practical solution for thermal load modeling in fire scenarios.

% \subsection{Interstory Drift Ratio}
\subsection{OpenSeesRT Simulation}
\label{subsec:opensees_simulation}

The thermal and mechanical responses of 3D frame structures under combined fire and gravity loads are simulated using OpenSeesRT \cite{perez2024openseesrt}. \revise{In the simulation, the IDR of each node at $t_{\thre}$ is computed using the computed nodal displacements. Each structural model features six degrees of freedom per node (3 translational  and 3 rotational), with linear geometrical transformations (\texttt{geomTransf: Linear}) defining how the element local coordinate systems are mapped to the global coordinate system and assuming small displacements and rotations. Although OpenSeesRT allows a variety of options for modeling finite deformations, in the present simulations and mainly for simplicity, we did not consider large deformations. All bottom nodes (nodes on the ground) are fully constrained in all six degrees of freedom, while degrees of freedom os all other nodes are free.} Material behavior is temperature-dependent and modeled with \texttt{Steel01Thermal}, while fiber-based sections (\texttt{FiberThermal}) capture nonlinear interactions between thermal and mechanical responses at the cross-section level. \revise{Structural elements are represented as displacement-based Euler-Bernoulli beam-columns (\texttt{dispBeamColumnThermal}). This element  formulation accounts for thermal strains (temperature gradients) in the section, which is discretized into fibers. Numerical integration is used along the length of each element using three integration (Gauss) points, one at each end and the third in the middle of the element.}

{\revise{Thermal expansion of steel members plays a crucial role in IDR development. In reality, reinforced concrete floor slabs heat at a different rate than steel members due to their higher thermal mass and lower thermal conductivity. This differential heating can lead to restrained thermal expansion, introducing axial compression in beams and affecting the overall structural response. In this study, explicit {\em{composite action}} between steel members and concrete slabs is not modeled. Instead, our approach focuses on isolating the response of the steel structural frame, which is often the critical load-bearing component in fire scenarios. This assumption aligns with prior studies \cite{Possidente_2024} demonstrating that steel structures reach thermal equilibrium with surrounding gases quickly, allowing the use of uniform thermal loading in fire analysis. Future work could enhance this framework by incorporating slab-beam interaction effects, through a refined FEA for an extended dataset where constraints imposed by floor slabs are explicitly considered.}

The analysis begins with the application of gravity loads, followed by incremental thermal loads simulating the fire exposure. A static nonlinear solver using  \texttt{ExpressNewton} algorithm ensures convergence, while the \texttt{NormDispIncr} test maintains accuracy. An incremental \texttt{LoadControl} scheme with small step sizes is employed to guarantee numerical stability, using 10\% for gravity loads and 1\% for thermal loads. 

\revise{
In the thermal load analysis, uniform thermal load is applied to each beam or column, i.e., the temperature of each element is set to be that at the middle point, according to \secref{subsec:thermal_load_generation}. The \texttt{Steel01Thermal} material allows the properties (e.g., Young's modulus and yield strength) to be adjusted at increasing temperatures according to \cite{EN1993} using its Table 3.1: Reduction factors for the stress-strain relationship of carbon steel at elevated temperatures. For example, if the Young’s modulus at ambient temperature is $E_0$, then as the temperature ($T$) increases, the modulus changes as $E(T) = \eta (T) \times E_0$. \cite{EN1993} directly provides the values of $\eta(T) \in \left[0,1\right] $ at every $100 ^\circ\mathrm{C}$ interval and recommends using linear interpolation to obtain $\eta(T)$ for intermediate values of $T$.
} OpenSeesRT documentation \cite{OpenSeesThermalExamples} provides several examples of thermal analyses.

This modeling framework accommodates variations in material properties, cross-sectional geometries, and temperature profiles, providing robust simulations of structural behavior under fire conditions. The primary settings and configurations for the OpenSeesRT simulations are summarized in \tabref{tab:ops_detail}.
\begin{table}[h!]
    \centering
        \caption{Key settings of OpenSeesRT simulations.}
    \begin{tabular}{l|>{\raggedright\arraybackslash}p{0.6\linewidth}} %
    \toprule
    Modeling Aspect     & Details \\
    \midrule
    Geometry            & 3D models; 6 degrees of freedom per node \\
    Transformation      & geomTransf: Linear \\ 
    Material            & Steel01Thermal \\
    Section             & FiberThermal; Cross-section: $0.1$ m $\times$ $0.1$ m \\ 
    Element type        & {dispBeamColumnThermal} \\ 
    Loading             & Gravity loads: {beamUniform}; Thermal loads: {beamThermal} \\
    Integration scheme  & Incremental {LoadControl}; Step size: $10\%$ (gravity analysis), $1\%$ (thermal analysis) \\
    Nonlinear solver    & {ExpressNewton} algorithm; {UmfPack} solver; Convergence test: {NormDispIncr} tolerance: $10^{-8}$; Maximum \# iterations per step: $1000$. \\ 
    \bottomrule
    \end{tabular}
    \label{tab:ops_detail}
\end{table}

For each structure in the labeled dataset, 30 fire points are selected using a dual-granularity approach, \revise{i.e., two-stage sampling strategy,} to ensure they are well-distributed. Specifically, rooms are sequentially selected, with one fire point randomly chosen within each selected room. If a building is large and contains more than 30 rooms, we randomly select 30 rooms without replacement, i.e., ensuring that no more than one fire point is located in the same room. Conversely, if the building is small and has fewer than 30 rooms, all rooms are initially selected, with one fire point randomly assigned to each room. Additionally, rooms are then selected with replacement until a total of 30 fire points are assigned. \revise{The room-level sampling prioritizes selecting distinct rooms to avoid spatial clustering of fire points, while the point-level sampling ensures intra-room variability. This approach aligns with stratified sampling principles commonly used for efficient spatial representation, where multi-stage sampling strategies optimize coverage and variability, e.g., \cite{arunachalam_generalized_2023}, and enables a more comprehensive characterizing of how the structures respond under fire conditions.}
% This selection method prevents fire points from clustering too closely while maintaining an element of randomness. By distributing fire points in this manner, the 30 fire scenarios are effectively utilized, enabling a more comprehensive characterizing of how the structures respond under fire conditions.

\subsection{Summary of the Dataset Generation}
As discussed in this section and related to  \figref{fig:dataset_generation_procedure}, three key steps were considered in the development of the dataset: 
\begin{enumerate}
    \item {\bf{Filtering process}}: Structures with MIDR exceeding $1\%$ under gravity loads were excluded,  resulting in $1,573$ labeled structures retained for fire simulation and $16,050$ unlabeled structures for training the MFSP predictor.
    \item {\bf{Fire simulations}}: For each retained labeled structure, 30 fire scenarios were simulated using OpenSeesRT, yielding $47,190$ fire cases.
    \item {\bf{Data distribution check}}: MIDR distributions for labeled and unlabeled data under gravity loads were highly similar, because both datasets were generated using the same method. Under fire conditions, the MIDR distribution shifted, reflecting significant structural deformation with values reaching a maximum of about 6\%, an average of 1.70\%, and a standard deviation of 1.12\%. This step ensured a diverse and comprehensive dataset for the proposed predictive framework.
\end{enumerate}
The statistical distribution histograms for MIDR (after applying the $1\%$ filtering threshold \revise{for gravity load responses}) under different loading conditions are plotted in \figref{fig:histogram_mdr}. Figures \ref{fig:histogram_mdr}(a) and \ref{fig:histogram_mdr}(b) show the MIDR distributions of the labeled and unlabeled data, respectively, under gravity loads only. \figref{fig:histogram_mdr}(c) shows the MIDR distribution of the labeled data under the combined effects of gravity and fire loads. Fire load causes the structures to significantly deform, leading to a noticeably \revise{right-skewed} MIDR distribution.

\begin{figure*}[h!]
    \centering
    \includegraphics[width=\linewidth]{figures/histogram_mdr.pdf}
    \caption{Histograms of MIDR for labeled and unlabeled structures with gravity loads and fire cases.}
    \label{fig:histogram_mdr}
\end{figure*}

\revise{
This dataset provides the basis for training and testing the performance of the GNN-based framework. Although we employed a simplified rule-based thermal load generation method compared with conventional CFD-based simulations, the temperature field, the changes of the material properties, and the response of the structures, are all still highly nonlinear and complex. Therefore, it is still a challenging task for the NN to predict the MIDRs based on this dataset.
}
\subsection{Dataset Statistics}
\label{app:statistics}
In line with most GC studies, we utilize seven datasets in total: five transductive datasets—\textit{Citeseer}, \textit{Cora} \citep{kipf2016semi}, Pubmed \citep{namata2012query}, \textit{Ogbn-arxiv}, and \textit{Ogbn-products} \citep{hu2020open}—and two inductive datasets, \textit{Flickr} and \textit{Reddit} \citep{zeng2019graphsaint}. Each graph is randomly split, ensuring a consistent class distribution. The details of the datasets statistics are shown in Table \ref{tab:statistics}. We list all evolving information in Table~\ref{tab:split_reduction}, rows above the midline correspond to smaller datasets, and rows below it correspond to larger ones.
Reduction rate $r$ is defined as (\#nodes in synthetic set)/(\#nodes in training set) while $r_w$ is (\#nodes in synthetic set)/(\#nodes of whole graph visible in training stage). The whole graph visible in training stage means the full graph dataset for transductive setting but only the training graph for inductive setting.
\begin{table}[ht!]
\caption{Split and reduction rate information. The ``\# Train Nodes'' and ``\# Syn Nodes'' columns denote the number of newly added training nodes and synthetic nodes at each time step, respectively.}
\label{tab:split_reduction}
\resizebox{0.47\textwidth}{!}{
\begin{tabular}{lrrrr}
\toprule
\textbf{Dataset} & 
\textbf{\# Train Nodes} & 
\textbf{\# Syn Nodes} & 
\textbf{$r$ (Train)} & 
\textbf{$r_w$ (Whole)} \\
\midrule
\textit{Citeseer}      & 24             & 12  & 0.5    & 1.80  \\
\textit{Cora}          & 28             & 14  & 0.5    & 2.60  \\
\textit{Pubmed}        & 12             & 6   & 0.5    & 0.15  \\
\midrule
\textit{Flickr}        & ~8{,}920 & 90  & 0.01   & 1.00  \\
\textit{Ogbn-arxiv}    & ~18{,}190 & 182 & 0.01   & 0.50  \\
\textit{Ogbn-products} & ~39{,}330 & 394 & 0.01   & 0.08  \\
\textit{Reddit}        & ~30{,}790 & 31  & 0.001  & 0.10  \\
\bottomrule
\end{tabular}}
\end{table}

\subsection{Platform and Hardware Information}
To efficiently execute the clustering algorithm, we run it on Intel(R) Xeon(R) Platinum 8260 CPUs @ 2.40GHz using NumPy~\cite{numpy}, while the downstream GNN evaluations are conducted on a cluster equipped with a mix of Tesla A100 40GB/V100 32GB GPUs for large datasets and K80 12GB GPUs for small datasets. All GNN models are implemented using the PyG package~\cite{pyg}.

\subsection{Baselines Selection}
To establish a fair benchmark, we selected recent state-of-the-art GC methods that emphasize both effectiveness and efficiency. Some recent methods, such as MCond, CGC, and GCPA, were excluded due to the unavailability of their code at the time of paper writing. For the selected approaches, we chose the best representatives from each category: GCondX for gradient matching, GCDM and SimGC for distribution matching, and GEOM for trajectory matching. We implemented these methods using the latest GraphSlim package\footnote{\url{https://github.com/Emory-Melody/GraphSlim/tree/main}}, except for SimGC\footnote{\url{https://github.com/BangHonor/SimGC}} and GEOM\footnote{\url{https://github.com/NUS-HPC-AI-Lab/GEOM/tree/main}}, for which we used their original source code. We specifically included SimGC because it is the only model-based GC method that can run on Ogbn-products without requiring any modifications.


\subsection{Implementation Details for Variants of GCond}
As mentioned in Section~\ref{sec:intro} and illustrated in Table~\ref{tab:preliminary}, adapting GCond to an evolving setting is challenging. We employ the structure-free variant of GCond, i.e., GCondX, for easier adaptation, as designing a specific growth mechanism for the condensed graph is nontrivial and requires significant effort.
In addition, to manifests the convergence speed difference between GCond and GCond-Init, we implement an early stopping criterion with a patience of 3 during intermediate evaluations. 
If no improvement in validation performance is observed over 3 consecutive evaluations, the condensation process is terminated.
\subsection{Hyperparameters}\label{app:hyper}
Compared to existing work and benchmarks in GC, we perform a moderate hyperparameter search on validation set, as detailed in Section~\ref{sec:hyper}. The final results are presented in Table~\ref{tab:hyper}. During hyperparameter optimization (HPO), we observe that inheriting clustering centroids results in an approximate 1\% absolute performance drop for \textit{Flickr} and \textit{Ogbn-arxiv}. Therefore, we also treat the use of incremental $k$-means++ as a tunable hyperparameter. Additionally, some datasets do not perform well with a single hyperparameter configuration. To address this, we employ two distinct hyperparameter sets tuned on the first and last time steps, respectively, and select the better-performing one during the evolution process. These two sets are represented using a "/" separator and are indicated as "if Dual" when this technique is applied. For all baselines, we use the best hyperparameters reported in their respective papers, as implemented in GC4NC~\cite{gong2024gc4nc}.

The optimal hyperparameters also offer meaningful insights. \textbf{First}, during the early evolution stage, graphs exhibit higher heterophily compared to later stages. For example, on the \textit{Cora} dataset, $\alpha_1=-0.3$ in the early phase contrasts with $\alpha_1=0.9$ later. This pattern likely arises because, in the early stages of a graph, groups have not yet formed; links appear more randomly, making it challenging for nodes to link to similar counterparts.
\textbf{Second}, it is noteworthy that some datasets do not rely on second-hop information. This observation is contrary to previous studies~\cite{wu2019simplifying,luo_classic_2024} that recommend using at least 2-hop propagation. We conjecture that the representation clustering process itself acts as an additional step of feature propagation.
\textbf{Finally}, weight decay emerges as a critical factor for the performance of downstream models, suggesting that future work should pay closer attention to its optimization.
\begin{table*}[ht!]
\centering
\caption{The test accuracy of GC methods on various datasets.
"Non-Evolving" displays the test accuracy at the final time step (largest possible graph).
"Evolving" shows the average test accuracy over five time-steps.
Each result includes the mean accuracy $\pm$ standard deviation (Std.) from 10 runs. The "Whole" column refers to the results obtained by running standard GCN training and testing. "OOM" indicates an Out-of-Memory error during the computation. The best results are marked in \textbf{bold}. The runner-up results are \underline{underlined}.}
% \vskip -1em
\resizebox{\textwidth}{!}{%
\begin{tabular}{lc|ccc|cccccc|c}
\toprule
\textbf{Dataset} & \textbf{Setting} 
  & \textbf{Random} 
  & \textbf{Herding} 
  & \textbf{Kcenter} 
  & \textbf{GCondX} 
  & \textbf{GCond} 
  & \textbf{GCDM} 
  & \textbf{SimGC}
  & \textbf{GEOM} 
  & \textbf{GECC} 
  & \multicolumn{1}{c}{\textbf{Whole}} \\
  \midrule
\multirow{2}{*}{CiteSeer} 
  &Non-Evolving& 62.62$\pm$0.63 & 66.66$\pm$0.54 & 59.04$\pm$0.90 & 68.38$\pm$0.45 & 69.35$\pm$0.82 & 72.08$\pm$0.19 & 66.40$\pm$0.15 &  \textcolor{red}{\underline{73.03$\pm$0.31}} & \textcolor{red}{\textbf{73.25$\pm$0.15}} & 72.11 \\
  &Evolving& 50.65$\pm$1.55 & 53.47$\pm$0.98 & 47.99$\pm$1.81 & 50.85$\pm$3.00 & 60.51$\pm$0.86 & \textcolor{blue}{\underline{61.51$\pm$0.53}}& 57.42$\pm$0.21 & 58.95$\pm$0.67 & \textcolor{blue}{\textbf{65.48$\pm$0.76}} & 63.57 \\
\midrule
\multirow{2}{*}{Cora} 
  &Non-Evolving& 72.24$\pm$0.59 & 73.77$\pm$0.93 & 70.55$\pm$1.35 & 78.60$\pm$0.31 & 80.54$\pm$0.67 & 80.68$\pm$0.27 & 79.60$\pm$0.11 & \textcolor{red}{\underline{82.82$\pm$0.17}} & \textcolor{red}{\textbf{82.99$\pm$0.27}} & 81.23 \\
  &Evolving& 58.00$\pm$1.48 & 63.07$\pm$1.43 & 59.90$\pm$1.41 & 67.18$\pm$1.73 & \textcolor{blue}{\underline{77.14$\pm$0.55}} & 74.54$\pm$0.59 & 64.42$\pm$0.19 & 72.56$\pm$0.88 & \textcolor{blue}{\textbf{77.36$\pm$0.41}} & 76.34 \\
\midrule
\multirow{2}{*}{Pubmed} 
  &Non-Evolving& 71.84$\pm$0.66 & 75.53$\pm$0.44 & 74.00$\pm$0.19 & 71.97$\pm$0.53 & 76.46$\pm$0.48 & 77.48$\pm$0.46 & 76.80$\pm$0.23 & \textcolor{red}{\underline{78.49$\pm$0.24}} & \textcolor{red}{\textbf{80.24$\pm$0.27}} & 78.65 \\
  &Evolving& 66.37$\pm$1.25 & 66.31$\pm$1.34 & 64.38$\pm$1.25 & 62.65$\pm$1.20 & 74.26$\pm$0.84 & \textcolor{blue}{\underline{74.49$\pm$0.56}} & 71.38$\pm$0.21 & 70.25$\pm$0.78 &  \textcolor{blue}{\textbf{76.74$\pm$0.27}} & 76.18 \\
\midrule
\multirow{2}{*}{Flickr} 
  &Non-Evolving& 44.68$\pm$0.55 & 45.12$\pm$0.39 & 43.53$\pm$0.59 & 46.58$\pm$0.14 & \textcolor{red}{\textbf{46.99$\pm$0.12}} & 45.88$\pm$0.10 & 41.01$\pm$0.23 & 46.13$\pm$0.22 & \textcolor{red}{\underline{46.63$\pm$0.23}} & 47.53 \\
  &Evolving& 44.70$\pm$0.46 & 44.66$\pm$0.43 & 44.33$\pm$0.49 & \textcolor{blue}{\underline{45.63$\pm$0.78}}& 45.52$\pm$0.49 & 44.98$\pm$0.34 & 41.94$\pm$0.22 & 45.43$\pm$0.39 &  \textcolor{blue}{\textbf{45.78$\pm$0.38}} & 46.97 \\
  \midrule
  \multirow{2}{*}{Ogbn-arxiv} 
  &Non-Evolving& 60.19$\pm$0.52 & 57.70$\pm$0.24 & 58.66$\pm$0.36 & 59.93$\pm$0.54 & 64.23$\pm$0.16 & 60.71$\pm$0.68 & 65.26$\pm$0.26 &  \textcolor{red}{\textbf{69.59$\pm$0.24}} & \textcolor{red}{\underline{66.71$\pm$0.10}} & 69.01 \\
  &Evolving& 56.04$\pm$0.67 & 57.57$\pm$0.48 & 56.21$\pm$0.73 & 60.73$\pm$0.53 & 62.50$\pm$0.36 & 59.98$\pm$0.48 & 64.97$\pm$0.20 & \textcolor{blue}{\textbf{66.30$\pm$0.39}} & \textcolor{blue}{\underline{65.42$\pm$0.14}} & 70.40 \\
\midrule
\multirow{2}{*}{Ogbn-products} 
  &Non-Evolving& 60.19$\pm$0.52 & 57.70$\pm$0.24 & 58.66$\pm$0.36 & OOM & OOM & OOM & \textcolor{red}{\underline{61.71$\pm$0.25}} & OOM & \textcolor{red}{\textbf{66.32$\pm$0.23}} & 73.40 \\
  &Evolving& 41.36$\pm$0.48 & 44.26$\pm$0.61 & 38.93$\pm$0.82 & OOM & OOM & OOM & \textcolor{blue}{\underline{61.93$\pm$0.20}} & OOM & \textcolor{blue}{\textbf{64.03$\pm$0.30}} & 73.88 \\
\midrule
\multirow{2}{*}{Reddit} 
  &Non-Evolving& 55.73$\pm$0.50 & 59.34$\pm$0.70 & 48.28$\pm$0.73 & 88.25$\pm$0.30 & 89.82$\pm$0.10 & 89.96$\pm$0.05 & 90.78$\pm$0.25 & \textcolor{red}{\underline{91.33$\pm$0.13}} & \textcolor{red}{\textbf{91.37$\pm$0.04}} & 93.70 \\
  &Evolving& 51.31$\pm$0.90 & 48.94$\pm$0.70 & 48.53$\pm$1.37 & 79.02$\pm$0.73 & 87.93$\pm$0.22 & 82.68$\pm$0.21 &\textcolor{blue}{\underline{89.85$\pm$0.25}} & 67.91$\pm$0.57 & \textcolor{blue}{\textbf{90.02$\pm$0.07}} & 93.92 \\
\bottomrule
\end{tabular}}
\label{tab:main_app}
\end{table*}
\begin{table*}[ht!]
\centering
\caption{\textbf{Average Runtime (seconds) Across Evolving Times.} The reported reduction time is rigorously computed by excluding the overhead of the data loading and evaluation processes.}
% \vskip -1em
\label{tab:time}
\resizebox{\textwidth}{!}{
\begin{tabular}{lccc|cccccc|c}
\toprule
\textbf{Dataset} & \textbf{Random} & \textbf{Herding} & \textbf{KCenter} & \textbf{GCondX}& \textbf{GCond}& \textbf{GCDM} & \textbf{SimGC}& \textbf{GEOM}& \textbf{GECC} & \textbf{Whole}\\
\midrule
\textit{Citeseer}
  & 0.04
  & 5.73
  & 5.84
  & 505.62
& 654.32
& 217.99
  & 1,680.02
& 1,362.40
& 1.65
& 3.98 \\

\textit{Cora}
  & 0.01
  & 4.20
  & 4.80
  & 331.53
& 1,190.65
& 142.82
  & 1,643.76
& 1,331.43
& 1.72
& 2.10 \\

\textit{Pubmed}
  & 0.02
  & 9.00
  & 7.18
  & 246.68
& 502.12
& 311.37
  & 1,654.23
& 995.21
& 1.42
& 5.76 \\ \midrule

\textit{Flickr}
  & 0.02
  & 11.53
  & 10.56
  & 609.98
& 1,446.76
& 353.51
  & 7,486.65
& 757.75
& 7.10
& 8.57 \\

\textit{Ogbn-arxiv}
  & 0.02
  & 14.36
  & 14.05
  & 2,895.06
& 6,076.18
& 686.12
  & 2,687.45
& 1,685.18
& 9.96
& 12.45 \\

\textit{Ogbn-products}
  & 0.02
  & 517.95
  & 513.36
  & OOM
& OOM
& OOM
  & 71,489.00
& OOM
& 146.82
& 542.61 \\

\textit{Reddit}
  & 0.02
  & 24.40
  & 24.84
  & 2,672.85& 6,130.46& 337.15
  & 6,610.70& 1,815.77& 4.91& 11.50 \\

\bottomrule
\end{tabular}}
\end{table*}
\begin{table*}[]
\centering
\resizebox{\textwidth}{!}{%
\begin{tabular}{@{}lcccccc@{}}
\toprule[1.5pt]
 & Model & Learning Rate  & Batch Size & KL Coefficient&Max Length & Training Epochs \\ 
\midrule[1pt]
& Llama-3.1-8B-Instruct & 5e-6  & 32 & 0.1&8000& 3\\
& Qwen2-7B-Instruct & 5e-6 & 32 & 0.1 &6000& 3 \\
& Qwen2.5-Math-7B & 5e-6  & 32 & 0.01&8000& 3 \\ 
\bottomrule[1.5pt]
\end{tabular}%
}
\caption{Model Training Hyperparameter Settings (SFT)}
\label{tab:hyper_sft}
\end{table*}

\begin{table*}[]
\centering
\resizebox{\textwidth}{!}{%
\begin{tabular}{@{}lccccccccc@{}}
\toprule[1.5pt]
 & Model & Learning Rate  & \makecell[c]{Training\\Batch Size} & \makecell[c]{Forward\\Batch Size} & KL Coefficient&Max Length & \makecell[c]{Sampling\\Temperature} &Clip Range &Training Steps \\ 
\midrule[1pt]
& Llama-3.1 &5e-7  & 64& 256 & 0.05&8000& 0.7&0.2&500\\
& Qwen2-7B-Instruct & 5e-7&  64& 256 & 0.05 &6000&0.7 &0.2&500\\\
& Qwen2.5-Math-7B & 5e-7 & 64& 256 & 0.01&8000&0.7 &0.2&500 \\ 
\bottomrule[1.5pt]
\end{tabular}%
}
\caption{Model Training Hyperparameter Settings (RL)}
\label{tab:hyper_rl}
\end{table*}

\section{Additional Results}
\subsection{Performance and Efficiency}
For simplicity, Table~\ref{tab:main} omits the standard error and running time of coreset selection methods. we provide the full results here.

Figure~\ref{fig:accuracy_vs_time} presents the accuracy vs. time trade-off for Reddit. For the remaining three large datasets, we provide the corresponding results in Figure~\ref{fig:accuracy_vs_time_large_vertical}. 
The results align with our main findings, further confirming that GECC surpasses the baselines in both efficiency and scalability. It consistently maintains stable performance while effectively managing computational resources throughout graph evolution. Notably, on the large-scale Ogbn-products dataset, which contains over one million nodes, most GC methods fail, whereas GECC remains robust and continues to operate successfully.


\begin{figure*}[htbp]
  \centering
  \begin{subfigure}[b]{0.85\linewidth}
    \centering
    \includegraphics[width=0.8\linewidth]{figs/ogbn-arxiv_time_vs_accuracy-cropped.pdf}
    % \caption{Flickr Caption}
    \label{fig:flickr}
  \end{subfigure}
  % \vspace{-1em}
  \begin{subfigure}[b]{0.85\linewidth}
    \centering
    \includegraphics[width=0.8\linewidth]{figs/Ogbn-products_time_vs_accuracy-cropped.pdf}
    % \caption{Ogbn-products Caption}
    \label{fig:ogbnproducts}
  \end{subfigure}
  % \vspace{-1em}
  \begin{subfigure}[b]{0.85\linewidth}
    \centering
    \includegraphics[width=0.8\linewidth]{figs/Reddit_time_vs_accuracy-cropped.pdf}
    % \caption{Reddit Caption}
    \label{fig:redproducts}
  \end{subfigure}
  % \vskip -2em
  \caption{Test accuracy vs. condensation time on large datasets (top-left is better).}
  \label{fig:accuracy_vs_time_large_vertical}
\end{figure*}


% \begin{figure*}[htbp]
%   \centering
%   \begin{subfigure}[b]{0.33\linewidth}
%     \centering
%     \includegraphics[width=\linewidth]{figs/ogbn-arxiv_time_vs_accuracy-cropped.pdf}
%     % \caption{Flickr Caption}
%     \label{fig:flickr}
%   \end{subfigure}%
%   \hfill
%   \begin{subfigure}[b]{0.33\linewidth}
%     \centering
%     \includegraphics[width=\linewidth]{figs/Ogbn-products_time_vs_accuracy-cropped.pdf}
%     % \caption{Ogbn-products Caption}
%     \label{fig:ogbnproducts}
%   \end{subfigure}%
%   \hfill
%   \begin{subfigure}[b]{0.33\linewidth}
%     \centering
%     \includegraphics[width=\linewidth]{figs/Reddit_time_vs_accuracy-cropped.pdf}
%     % \caption{}
%     \label{fig:redproducts}
%   \end{subfigure}
%   \vskip -2em
%   \caption{Test accuracy vs. condensation time on large datasets (top-left is better).}
%   \label{fig:accuracy_vs_time_large}
% \end{figure*}