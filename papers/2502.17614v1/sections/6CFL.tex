\section{Conclusion and Outlook}
In this study, we address the challenge of evolving graph condensation. We observe that a universal clustering framework can naturally optimize the assignment matrix, thereby achieving the common objectives of existing GC methods. Additionally, we propose a novel \emph{balanced SSE} metric that further tightens the upper bound of these objectives. In the evolving setting, we find that our clustering approach can be readily adapted to an incremental version, termed \emph{incremental \(k\)-means++}. Experimental results demonstrate that balanced SSE improves the performance of clustering-based GC, and incremental \(k\)-means++ significantly reduces the number of iterations, thereby enhancing efficiency in evolving environments. Future work includes developing more efficient and scalable clustering techniques, especially soft clustering algorithms for larger graph datasets and adaptively optimizing multi-hop weights, which could be beneficial when the graph keeps evolving over time.


% Despite these advances, our proposed method \textsc{GECC} has some limitations, suggesting several avenues for future work:
% \textbf{First}, 
% \textbf{First}, soft clustering is challenging to implement on large datasets, yet it has been proven effective for smaller datasets. This highlights the need for a more scalable and efficient soft clustering algorithm.  
% \textbf{Second}, while clustering has demonstrated significant benefits for GC, its impact on \emph{independent} data domains (e.g., images) remains theoretically underexplored \citep{zhu2021graphheterophily}. Further research is needed to understand its effectiveness in such contexts.  
% \textbf{Finally}, developing a more effective strategy for weighting multi-hop information (e.g., \(\alpha_i\)) is crucial, particularly as evolving graphs may demand deeper message passing or broader contextual aggregation.
% Although the training-free approach shows advantages in this study, we do not rule out the potential for a better optimization solution that can adaptively determine the weights.



