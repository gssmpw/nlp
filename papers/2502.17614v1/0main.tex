%%
%% This is file `sample-sigconf.tex',
%% generated with the docstrip utility.
%%
%% The original source files were:
%%
%% samples.dtx  (with options: `all,proceedings,bibtex,sigconf')
%% 
%% IMPORTANT NOTICE:
%% 
%% For the copyright see the source file.
%% 
%% Any modified versions of this file must be renamed
%% with new filenames distinct from sample-sigconf.tex.
%% 
%% For distribution of the original source see the terms
%% for copying and modification in the file samples.dtx.
%% 
%% This generated file may be distributed as long as the
%% original source files, as listed above, are part of the
%% same distribution. (The sources need not necessarily be
%% in the same archive or directory.)
%%
%%
%% Commands for TeXCount
%TC:macro \cite [option:text,text]
%TC:macro \citep [option:text,text]
%TC:macro \citet [option:text,text]
%TC:envir table 0 1
%TC:envir table* 0 1
%TC:envir tabular [ignore] word
%TC:envir displaymath 0 word
%TC:envir math 0 word
%TC:envir comment 0 0
%%
%% The first command in your LaTeX source must be the \documentclass
%% command.
%%
%% For submission and  of your manuscript please change the
%% command to \documentclass[manuscript, screen, review]{acmart}.
%%
%% When submitting camera ready or to TAPS, please change the command
%% to \documentclass[sigconf]{acmart} or whichever template is required
%% for your publication.
%%
%%
\documentclass[sigconf]{acmart}
% \documentclass[sigconf,anonymous,review]{acmart}
\usepackage{paralist}
\usepackage{multirow}
\usepackage{graphicx}
\usepackage{subcaption}
\usepackage{booktabs}
\usepackage{amsthm}
\usepackage{makecell} % for line breaks in cells
% \newtheorem{theorem}{Theorem}
\usepackage{amsmath}
\newtheorem*{theorem-nonumber}{Theorem}
%%
%% \BibTeX command to typeset BibTeX logo in the docs
\AtBeginDocument{%
  \providecommand\BibTeX{{%
    Bib\TeX}}}

%% Rights management information.  This information is sent to you
%% when you complete the rights form.  These commands have SAMPLE
%% values in them; it is your responsibility as an author to replace
%% the commands and values with those provided to you when you
%% complete the rights form.
\setcopyright{acmlicensed}
\copyrightyear{2018}
\acmYear{2018}
\acmDOI{XXXXXXX.XXXXXXX}
%% These commands are for a PROCEEDINGS abstract or paper.
\acmConference[Conference acronym 'XX]{Make sure to enter the correct
  conference title from your rights confirmation email}{June 03--05,
  2018}{Woodstock, NY}
%%
%%  Uncomment \acmBooktitle if the title of the proceedings is different
%%  from ``Proceedings of ...''!
%%
%%\acmBooktitle{Woodstock '18: ACM Symposium on Neural Gaze Detection,
%%  June 03--05, 2018, Woodstock, NY}
\acmISBN{978-1-4503-XXXX-X/2018/06}


%%
%% Submission ID.
%% Use this when submitting an article to a sponsored event. You'll
%% receive a unique submission ID from the organizers
%% of the event, and this ID should be used as the parameter to this command.
%%\acmSubmissionID{123-A56-BU3}

%%
%% For managing citations, it is recommended to use bibliography
%% files in BibTeX format.
%%
%% You can then either use BibTeX with the ACM-Reference-Format style,
%% or BibLaTeX with the acmnumeric or acmauthoryear sytles, that include
%% support for advanced citation of software artefact from the
%% biblatex-software package, also separately available on CTAN.
%%
%% Look at the sample-*-biblatex.tex files for templates showcasing
%% the biblatex styles.
%%

%%
%% The majority of ACM publications use numbered citations and
%% references.  The command \citestyle{authoryear} switches to the
%% "author year" style.
%%
%% If you are preparing content for an event
%% sponsored by ACM SIGGRAPH, you must use the "author year" style of
%% citations and references.
%% Uncommenting
%% the next command will enable that style.
%%\citestyle{acmauthoryear}

\usepackage{xcolor}
\newcommand{\mo}[1]{\textcolor{red}{@Mo:~#1@}}
\newcommand{\gsb}[1]{\textcolor{blue}{@Gsb:~#1@}}
\newcommand{\wei}[1]{\textcolor{orange}{\#Wei:~#1\#}}
\newcommand{\revised}[1]{\textcolor{brown}{@Revised:~#1@}}
\newcommand{\juntong}[1]{\textcolor{green}{@Juntong:~#1@}}
%%
%% end of the preamble, start of the body of the document source.
\begin{document}

%%
%% The "title" command has an optional parameter,
%% allowing the author to define a "short title" to be used in page headers.
\title{Scalable Graph Condensation with Evolving Capabilities}

%%
%% The "author" command and its associated commands are used to define
%% the authors and their affiliations.
%% Of note is the shared affiliation of the first two authors, and the
%% "authornote" and "authornotemark" commands
%% used to denote shared contribution to the research.
% \author{Anonymous Authors}
\author{Shengbo Gong}
\authornote{Both authors contributed equally to this research.}
\affiliation{%
  \institution{Emory University}
  \city{Atlanta}
  \state{Georgia}
  \country{USA}
  }
\email{shengbo.gong@emory.edu}

\author{Mohammad Hashemi}
\authornotemark[1]
\affiliation{%
  \institution{Emory University}
  \city{Atlanta}
  \state{Georgia}
  \country{USA}
  }
\email{mohammad.hashemi@emory.edu}  

\author{Juntong Ni}
\affiliation{%
  \institution{Emory University}
  \city{Atlanta}
  \state{Georgia}
  \country{USA}
}
\email{juntong.ni@emory.edu}

\author{Carl Yang}
\affiliation{%
  \institution{Emory University}
  \city{Atlanta}
  \state{Georgia}
  \country{USA}
}
\email{j.carlyang@emory.edu}

\author{Wei Jin}
\affiliation{%
  \institution{Emory University}
  \city{Atlanta}
  \state{Georgia}
  \country{USA}
}
\email{wei.jin@emory.edu}

%%
%% The abstract is a short summary of the work to be presented in the
%% article.
\begin{abstract}
Graph data has become a pivotal modality due to its unique ability to model relational datasets. However, real-world graph data continues to grow exponentially, resulting in a quadratic increase in the complexity of most graph algorithms as graph sizes expand. Although graph condensation (GC) methods have been proposed to address these scalability issues, existing approaches often treat the training set as static, overlooking the evolving nature of real-world graph data. This limitation leads to inefficiencies when condensing growing training sets.
In this paper, we introduce GECC (\underline{G}raph \underline{E}volving \underline{C}lustering \underline{C}ondensation), a scalable graph condensation method designed to handle large-scale and evolving graph data. GECC employs a traceable and efficient approach by performing class-wise clustering on aggregated features. Furthermore, it can inherits previous condensation results as clustering centroids when the condensed graph expands, thereby attaining an evolving capability. This methodology is supported by robust theoretical foundations and demonstrates superior empirical performance. Comprehensive experiments show that GECC achieves better performance than most state-of-the-art graph condensation methods while delivering an around 1,000$\times$ speedup on large datasets.
\end{abstract}




%%
%% The code below is generated by the tool at http://dl.acm.org/ccs.cfm.
%% Please copy and paste the code instead of the example below.
%%

% \begin{CCSXML}
% <ccs2012>
%    <concept>
%        <concept_id>10010147.10010257.10010293.10010294</concept_id>
%        <concept_desc>Computing methodologies~Neural networks</concept_desc>
%        <concept_significance>500</concept_significance>
%        </concept>
%  </ccs2012>
% \end{CCSXML}

% \ccsdesc[500]{Computing methodologies~Neural networks}


% \begin{CCSXML}
% <ccs2012>
% <concept>
% <concept_id>10010147.10010257.10010293.10010294</concept_id>
% <concept_desc>Computing methodologies~Neural networks</concept_desc>
% <concept_significance>500</concept_significance>
% </concept>
% <concept>
% <concept_id>10010147.10010257.10010258.10010259.10010263</concept_id>
% <concept_desc>Computing methodologies~Supervised learning by classification</concept_desc>
% <concept_significance>500</concept_significance>
% </concept>
% <concept>
% <concept_id>10002951.10003260.10003277</concept_id>
% <concept_desc>Information systems~Web mining</concept_desc>
% <concept_significance>500</concept_significance>
% </concept>
% </ccs2012>
% \end{CCSXML}

% \ccsdesc[500]{Computing methodologies~Neural networks}
% \ccsdesc[500]{Computing methodologies~Supervised learning by classification}
% \ccsdesc[500]{Information systems~Web mining}



%%
%% Keywords. The author(s) should pick words that accurately describe
%% the work being presented. Separate the keywords with commas.
\keywords{Graph Neural Networks, Data-Efficient Learning}
%% A "teaser" image appears between the author and affiliation
%% information and the body of the document, and typically spans the
%% page.
% \begin{teaserfigure}
%   \includegraphics[width=\textwidth]{sampleteaser}
%   \caption{Seattle Mariners at Spring Training, 2010.}
%   \Description{Enjoying the baseball game from the third-base
%   seats. Ichiro Suzuki preparing to bat.}
%   \label{fig:teaser}
% \end{teaserfigure}

\received{20 February 2007}
\received[revised]{12 March 2009}
\received[accepted]{5 June 2009}

%%
%% This command processes the author and affiliation and title
%% information and builds the first part of the formatted document.
\maketitle

\section{Introduction}

In today’s rapidly evolving digital landscape, the transformative power of web technologies has redefined not only how services are delivered but also how complex tasks are approached. Web-based systems have become increasingly prevalent in risk control across various domains. This widespread adoption is due their accessibility, scalability, and ability to remotely connect various types of users. For example, these systems are used for process safety management in industry~\cite{kannan2016web}, safety risk early warning in urban construction~\cite{ding2013development}, and safe monitoring of infrastructural systems~\cite{repetto2018web}. Within these web-based risk management systems, the source search problem presents a huge challenge. Source search refers to the task of identifying the origin of a risky event, such as a gas leak and the emission point of toxic substances. This source search capability is crucial for effective risk management and decision-making.

Traditional approaches to implementing source search capabilities into the web systems often rely on solely algorithmic solutions~\cite{ristic2016study}. These methods, while relatively straightforward to implement, often struggle to achieve acceptable performances due to algorithmic local optima and complex unknown environments~\cite{zhao2020searching}. More recently, web crowdsourcing has emerged as a promising alternative for tackling the source search problem by incorporating human efforts in these web systems on-the-fly~\cite{zhao2024user}. This approach outsources the task of addressing issues encountered during the source search process to human workers, leveraging their capabilities to enhance system performance.

These solutions often employ a human-AI collaborative way~\cite{zhao2023leveraging} where algorithms handle exploration-exploitation and report the encountered problems while human workers resolve complex decision-making bottlenecks to help the algorithms getting rid of local deadlocks~\cite{zhao2022crowd}. Although effective, this paradigm suffers from two inherent limitations: increased operational costs from continuous human intervention, and slow response times of human workers due to sequential decision-making. These challenges motivate our investigation into developing autonomous systems that preserve human-like reasoning capabilities while reducing dependency on massive crowdsourced labor.

Furthermore, recent advancements in large language models (LLMs)~\cite{chang2024survey} and multi-modal LLMs (MLLMs)~\cite{huang2023chatgpt} have unveiled promising avenues for addressing these challenges. One clear opportunity involves the seamless integration of visual understanding and linguistic reasoning for robust decision-making in search tasks. However, whether large models-assisted source search is really effective and efficient for improving the current source search algorithms~\cite{ji2022source} remains unknown. \textit{To address the research gap, we are particularly interested in answering the following two research questions in this work:}

\textbf{\textit{RQ1: }}How can source search capabilities be integrated into web-based systems to support decision-making in time-sensitive risk management scenarios? 
% \sq{I mention ``time-sensitive'' here because I feel like we shall say something about the response time -- LLM has to be faster than humans}

\textbf{\textit{RQ2: }}How can MLLMs and LLMs enhance the effectiveness and efficiency of existing source search algorithms? 

% \textit{\textbf{RQ2:}} To what extent does the performance of large models-assisted search align with or approach the effectiveness of human-AI collaborative search? 

To answer the research questions, we propose a novel framework called Auto-\
S$^2$earch (\textbf{Auto}nomous \textbf{S}ource \textbf{Search}) and implement a prototype system that leverages advanced web technologies to simulate real-world conditions for zero-shot source search. Unlike traditional methods that rely on pre-defined heuristics or extensive human intervention, AutoS$^2$earch employs a carefully designed prompt that encapsulates human rationales, thereby guiding the MLLM to generate coherent and accurate scene descriptions from visual inputs about four directional choices. Based on these language-based descriptions, the LLM is enabled to determine the optimal directional choice through chain-of-thought (CoT) reasoning. Comprehensive empirical validation demonstrates that AutoS$^2$-\ 
earch achieves a success rate of 95–98\%, closely approaching the performance of human-AI collaborative search across 20 benchmark scenarios~\cite{zhao2023leveraging}. 

Our work indicates that the role of humans in future web crowdsourcing tasks may evolve from executors to validators or supervisors. Furthermore, incorporating explanations of LLM decisions into web-based system interfaces has the potential to help humans enhance task performance in risk control.






\section{Related Work}
\label{sec:relatedworks}

% \begin{table*}[t]
% \centering 
% \renewcommand\arraystretch{0.98}
% \fontsize{8}{10}\selectfont \setlength{\tabcolsep}{0.4em}
% \begin{tabular}{@{}lc|cc|cc|cc@{}}
% \toprule
% \textbf{Methods}           & \begin{tabular}[c]{@{}c@{}}\textbf{Training}\\ \textbf{Paradigm}\end{tabular} & \begin{tabular}[c]{@{}c@{}}\textbf{$\#$ PT Data}\\ \textbf{(Tokens)}\end{tabular} & \begin{tabular}[c]{@{}c@{}}\textbf{$\#$ IFT Data}\\ \textbf{(Samples)}\end{tabular} & \textbf{Code}  & \begin{tabular}[c]{@{}c@{}}\textbf{Natural}\\ \textbf{Language}\end{tabular} & \begin{tabular}[c]{@{}c@{}}\textbf{Action}\\ \textbf{Trajectories}\end{tabular} & \begin{tabular}[c]{@{}c@{}}\textbf{API}\\ \textbf{Documentation}\end{tabular}\\ \midrule 
% NexusRaven~\citep{srinivasan2023nexusraven} & IFT & - & - & \textcolor{green}{\CheckmarkBold} & \textcolor{green}{\CheckmarkBold} &\textcolor{red}{\XSolidBrush}&\textcolor{red}{\XSolidBrush}\\
% AgentInstruct~\citep{zeng2023agenttuning} & IFT & - & 2k & \textcolor{green}{\CheckmarkBold} & \textcolor{green}{\CheckmarkBold} &\textcolor{red}{\XSolidBrush}&\textcolor{red}{\XSolidBrush} \\
% AgentEvol~\citep{xi2024agentgym} & IFT & - & 14.5k & \textcolor{green}{\CheckmarkBold} & \textcolor{green}{\CheckmarkBold} &\textcolor{green}{\CheckmarkBold}&\textcolor{red}{\XSolidBrush} \\
% Gorilla~\citep{patil2023gorilla}& IFT & - & 16k & \textcolor{green}{\CheckmarkBold} & \textcolor{green}{\CheckmarkBold} &\textcolor{red}{\XSolidBrush}&\textcolor{green}{\CheckmarkBold}\\
% OpenFunctions-v2~\citep{patil2023gorilla} & IFT & - & 65k & \textcolor{green}{\CheckmarkBold} & \textcolor{green}{\CheckmarkBold} &\textcolor{red}{\XSolidBrush}&\textcolor{green}{\CheckmarkBold}\\
% LAM~\citep{zhang2024agentohana} & IFT & - & 42.6k & \textcolor{green}{\CheckmarkBold} & \textcolor{green}{\CheckmarkBold} &\textcolor{green}{\CheckmarkBold}&\textcolor{red}{\XSolidBrush} \\
% xLAM~\citep{liu2024apigen} & IFT & - & 60k & \textcolor{green}{\CheckmarkBold} & \textcolor{green}{\CheckmarkBold} &\textcolor{green}{\CheckmarkBold}&\textcolor{red}{\XSolidBrush} \\\midrule
% LEMUR~\citep{xu2024lemur} & PT & 90B & 300k & \textcolor{green}{\CheckmarkBold} & \textcolor{green}{\CheckmarkBold} &\textcolor{green}{\CheckmarkBold}&\textcolor{red}{\XSolidBrush}\\
% \rowcolor{teal!12} \method & PT & 103B & 95k & \textcolor{green}{\CheckmarkBold} & \textcolor{green}{\CheckmarkBold} & \textcolor{green}{\CheckmarkBold} & \textcolor{green}{\CheckmarkBold} \\
% \bottomrule
% \end{tabular}
% \caption{Summary of existing tuning- and pretraining-based LLM agents with their training sample sizes. "PT" and "IFT" denote "Pre-Training" and "Instruction Fine-Tuning", respectively. }
% \label{tab:related}
% \end{table*}

\begin{table*}[ht]
\begin{threeparttable}
\centering 
\renewcommand\arraystretch{0.98}
\fontsize{7}{9}\selectfont \setlength{\tabcolsep}{0.2em}
\begin{tabular}{@{}l|c|c|ccc|cc|cc|cccc@{}}
\toprule
\textbf{Methods} & \textbf{Datasets}           & \begin{tabular}[c]{@{}c@{}}\textbf{Training}\\ \textbf{Paradigm}\end{tabular} & \begin{tabular}[c]{@{}c@{}}\textbf{\# PT Data}\\ \textbf{(Tokens)}\end{tabular} & \begin{tabular}[c]{@{}c@{}}\textbf{\# IFT Data}\\ \textbf{(Samples)}\end{tabular} & \textbf{\# APIs} & \textbf{Code}  & \begin{tabular}[c]{@{}c@{}}\textbf{Nat.}\\ \textbf{Lang.}\end{tabular} & \begin{tabular}[c]{@{}c@{}}\textbf{Action}\\ \textbf{Traj.}\end{tabular} & \begin{tabular}[c]{@{}c@{}}\textbf{API}\\ \textbf{Doc.}\end{tabular} & \begin{tabular}[c]{@{}c@{}}\textbf{Func.}\\ \textbf{Call}\end{tabular} & \begin{tabular}[c]{@{}c@{}}\textbf{Multi.}\\ \textbf{Step}\end{tabular}  & \begin{tabular}[c]{@{}c@{}}\textbf{Plan}\\ \textbf{Refine}\end{tabular}  & \begin{tabular}[c]{@{}c@{}}\textbf{Multi.}\\ \textbf{Turn}\end{tabular}\\ \midrule 
\multicolumn{13}{l}{\emph{Instruction Finetuning-based LLM Agents for Intrinsic Reasoning}}  \\ \midrule
FireAct~\cite{chen2023fireact} & FireAct & IFT & - & 2.1K & 10 & \textcolor{red}{\XSolidBrush} &\textcolor{green}{\CheckmarkBold} &\textcolor{green}{\CheckmarkBold}  & \textcolor{red}{\XSolidBrush} &\textcolor{green}{\CheckmarkBold} & \textcolor{red}{\XSolidBrush} &\textcolor{green}{\CheckmarkBold} & \textcolor{red}{\XSolidBrush} \\
ToolAlpaca~\cite{tang2023toolalpaca} & ToolAlpaca & IFT & - & 4.0K & 400 & \textcolor{red}{\XSolidBrush} &\textcolor{green}{\CheckmarkBold} &\textcolor{green}{\CheckmarkBold} & \textcolor{red}{\XSolidBrush} &\textcolor{green}{\CheckmarkBold} & \textcolor{red}{\XSolidBrush}  &\textcolor{green}{\CheckmarkBold} & \textcolor{red}{\XSolidBrush}  \\
ToolLLaMA~\cite{qin2023toolllm} & ToolBench & IFT & - & 12.7K & 16,464 & \textcolor{red}{\XSolidBrush} &\textcolor{green}{\CheckmarkBold} &\textcolor{green}{\CheckmarkBold} &\textcolor{red}{\XSolidBrush} &\textcolor{green}{\CheckmarkBold}&\textcolor{green}{\CheckmarkBold}&\textcolor{green}{\CheckmarkBold} &\textcolor{green}{\CheckmarkBold}\\
AgentEvol~\citep{xi2024agentgym} & AgentTraj-L & IFT & - & 14.5K & 24 &\textcolor{red}{\XSolidBrush} & \textcolor{green}{\CheckmarkBold} &\textcolor{green}{\CheckmarkBold}&\textcolor{red}{\XSolidBrush} &\textcolor{green}{\CheckmarkBold}&\textcolor{red}{\XSolidBrush} &\textcolor{red}{\XSolidBrush} &\textcolor{green}{\CheckmarkBold}\\
Lumos~\cite{yin2024agent} & Lumos & IFT  & - & 20.0K & 16 &\textcolor{red}{\XSolidBrush} & \textcolor{green}{\CheckmarkBold} & \textcolor{green}{\CheckmarkBold} &\textcolor{red}{\XSolidBrush} & \textcolor{green}{\CheckmarkBold} & \textcolor{green}{\CheckmarkBold} &\textcolor{red}{\XSolidBrush} & \textcolor{green}{\CheckmarkBold}\\
Agent-FLAN~\cite{chen2024agent} & Agent-FLAN & IFT & - & 24.7K & 20 &\textcolor{red}{\XSolidBrush} & \textcolor{green}{\CheckmarkBold} & \textcolor{green}{\CheckmarkBold} &\textcolor{red}{\XSolidBrush} & \textcolor{green}{\CheckmarkBold}& \textcolor{green}{\CheckmarkBold}&\textcolor{red}{\XSolidBrush} & \textcolor{green}{\CheckmarkBold}\\
AgentTuning~\citep{zeng2023agenttuning} & AgentInstruct & IFT & - & 35.0K & - &\textcolor{red}{\XSolidBrush} & \textcolor{green}{\CheckmarkBold} & \textcolor{green}{\CheckmarkBold} &\textcolor{red}{\XSolidBrush} & \textcolor{green}{\CheckmarkBold} &\textcolor{red}{\XSolidBrush} &\textcolor{red}{\XSolidBrush} & \textcolor{green}{\CheckmarkBold}\\\midrule
\multicolumn{13}{l}{\emph{Instruction Finetuning-based LLM Agents for Function Calling}} \\\midrule
NexusRaven~\citep{srinivasan2023nexusraven} & NexusRaven & IFT & - & - & 116 & \textcolor{green}{\CheckmarkBold} & \textcolor{green}{\CheckmarkBold}  & \textcolor{green}{\CheckmarkBold} &\textcolor{red}{\XSolidBrush} & \textcolor{green}{\CheckmarkBold} &\textcolor{red}{\XSolidBrush} &\textcolor{red}{\XSolidBrush}&\textcolor{red}{\XSolidBrush}\\
Gorilla~\citep{patil2023gorilla} & Gorilla & IFT & - & 16.0K & 1,645 & \textcolor{green}{\CheckmarkBold} &\textcolor{red}{\XSolidBrush} &\textcolor{red}{\XSolidBrush}&\textcolor{green}{\CheckmarkBold} &\textcolor{green}{\CheckmarkBold} &\textcolor{red}{\XSolidBrush} &\textcolor{red}{\XSolidBrush} &\textcolor{red}{\XSolidBrush}\\
OpenFunctions-v2~\citep{patil2023gorilla} & OpenFunctions-v2 & IFT & - & 65.0K & - & \textcolor{green}{\CheckmarkBold} & \textcolor{green}{\CheckmarkBold} &\textcolor{red}{\XSolidBrush} &\textcolor{green}{\CheckmarkBold} &\textcolor{green}{\CheckmarkBold} &\textcolor{red}{\XSolidBrush} &\textcolor{red}{\XSolidBrush} &\textcolor{red}{\XSolidBrush}\\
API Pack~\cite{guo2024api} & API Pack & IFT & - & 1.1M & 11,213 &\textcolor{green}{\CheckmarkBold} &\textcolor{red}{\XSolidBrush} &\textcolor{green}{\CheckmarkBold} &\textcolor{red}{\XSolidBrush} &\textcolor{green}{\CheckmarkBold} &\textcolor{red}{\XSolidBrush}&\textcolor{red}{\XSolidBrush}&\textcolor{red}{\XSolidBrush}\\ 
LAM~\citep{zhang2024agentohana} & AgentOhana & IFT & - & 42.6K & - & \textcolor{green}{\CheckmarkBold} & \textcolor{green}{\CheckmarkBold} &\textcolor{green}{\CheckmarkBold}&\textcolor{red}{\XSolidBrush} &\textcolor{green}{\CheckmarkBold}&\textcolor{red}{\XSolidBrush}&\textcolor{green}{\CheckmarkBold}&\textcolor{green}{\CheckmarkBold}\\
xLAM~\citep{liu2024apigen} & APIGen & IFT & - & 60.0K & 3,673 & \textcolor{green}{\CheckmarkBold} & \textcolor{green}{\CheckmarkBold} &\textcolor{green}{\CheckmarkBold}&\textcolor{red}{\XSolidBrush} &\textcolor{green}{\CheckmarkBold}&\textcolor{red}{\XSolidBrush}&\textcolor{green}{\CheckmarkBold}&\textcolor{green}{\CheckmarkBold}\\\midrule
\multicolumn{13}{l}{\emph{Pretraining-based LLM Agents}}  \\\midrule
% LEMUR~\citep{xu2024lemur} & PT & 90B & 300.0K & - & \textcolor{green}{\CheckmarkBold} & \textcolor{green}{\CheckmarkBold} &\textcolor{green}{\CheckmarkBold}&\textcolor{red}{\XSolidBrush} & \textcolor{red}{\XSolidBrush} &\textcolor{green}{\CheckmarkBold} &\textcolor{red}{\XSolidBrush}&\textcolor{red}{\XSolidBrush}\\
\rowcolor{teal!12} \method & \dataset & PT & 103B & 95.0K  & 76,537  & \textcolor{green}{\CheckmarkBold} & \textcolor{green}{\CheckmarkBold} & \textcolor{green}{\CheckmarkBold} & \textcolor{green}{\CheckmarkBold} & \textcolor{green}{\CheckmarkBold} & \textcolor{green}{\CheckmarkBold} & \textcolor{green}{\CheckmarkBold} & \textcolor{green}{\CheckmarkBold}\\
\bottomrule
\end{tabular}
% \begin{tablenotes}
%     \item $^*$ In addition, the StarCoder-API can offer 4.77M more APIs.
% \end{tablenotes}
\caption{Summary of existing instruction finetuning-based LLM agents for intrinsic reasoning and function calling, along with their training resources and sample sizes. "PT" and "IFT" denote "Pre-Training" and "Instruction Fine-Tuning", respectively.}
\vspace{-2ex}
\label{tab:related}
\end{threeparttable}
\end{table*}

\noindent \textbf{Prompting-based LLM Agents.} Due to the lack of agent-specific pre-training corpus, existing LLM agents rely on either prompt engineering~\cite{hsieh2023tool,lu2024chameleon,yao2022react,wang2023voyager} or instruction fine-tuning~\cite{chen2023fireact,zeng2023agenttuning} to understand human instructions, decompose high-level tasks, generate grounded plans, and execute multi-step actions. 
However, prompting-based methods mainly depend on the capabilities of backbone LLMs (usually commercial LLMs), failing to introduce new knowledge and struggling to generalize to unseen tasks~\cite{sun2024adaplanner,zhuang2023toolchain}. 

\noindent \textbf{Instruction Finetuning-based LLM Agents.} Considering the extensive diversity of APIs and the complexity of multi-tool instructions, tool learning inherently presents greater challenges than natural language tasks, such as text generation~\cite{qin2023toolllm}.
Post-training techniques focus more on instruction following and aligning output with specific formats~\cite{patil2023gorilla,hao2024toolkengpt,qin2023toolllm,schick2024toolformer}, rather than fundamentally improving model knowledge or capabilities. 
Moreover, heavy fine-tuning can hinder generalization or even degrade performance in non-agent use cases, potentially suppressing the original base model capabilities~\cite{ghosh2024a}.

\noindent \textbf{Pretraining-based LLM Agents.} While pre-training serves as an essential alternative, prior works~\cite{nijkamp2023codegen,roziere2023code,xu2024lemur,patil2023gorilla} have primarily focused on improving task-specific capabilities (\eg, code generation) instead of general-domain LLM agents, due to single-source, uni-type, small-scale, and poor-quality pre-training data. 
Existing tool documentation data for agent training either lacks diverse real-world APIs~\cite{patil2023gorilla, tang2023toolalpaca} or is constrained to single-tool or single-round tool execution. 
Furthermore, trajectory data mostly imitate expert behavior or follow function-calling rules with inferior planning and reasoning, failing to fully elicit LLMs' capabilities and handle complex instructions~\cite{qin2023toolllm}. 
Given a wide range of candidate API functions, each comprising various function names and parameters available at every planning step, identifying globally optimal solutions and generalizing across tasks remains highly challenging.



\section{Preliminaries and Notations} 

 Given a node set $\mathcal{V}$ and an edge set $\mathcal{E}$, a graph is denoted as $G=({\mathcal{V},\mathcal{E}})$. In the case of attributed graphs, where nodes are associated with features, the graph can be represented as $G=({\bf{X}, \bf{A}})$, where ${\bf X}=[{\bf x}_1,{\bf x}_2,...,{\bf x}_N]$ denotes the node attributes, and ${\bf A}$ shows the adjacency matrix. The graph Laplacian matrix is $\bf L=\bf D-\bf A$, where $\bf D$ is a diagonal degree matrix with ${\bf D}_{ii}=\sum_j {\bf A}_{ij}$. Let $N=|\mathcal{V}|$ and $E=|\mathcal{E}|$ represent the number of nodes and edges, respectively.

\vskip 0.3em \noindent\textbf{Graph Condensation Formulation.}
GC aims to condense a smaller synthetic graph \( G' = (\mathbf{X}', \mathbf{A}') \), where \( \mathbf{X}' \in \mathbb{R}^{N' \times d} \), \( \mathbf{A}' \in \{0, 1\}^{N' \times N'} \), and \( N' \ll N \), from the original large graph \( G = (\mathbf{X}, \mathbf{A}) \). The objective is to ensure that GNNs trained on \( G' \) achieve performance comparable to those trained on \( G \), thereby significantly accelerating GNN training \citep{jin2021graph}. 
% A variation of GC, known as structure-free graph condensation, focuses exclusively on the node features \( \mathbf{X}' \) without utilizing the adjacency matrix \( \mathbf{A}' \) \citep{zheng2024structure}. 
The large-scale graph \( G_t = (\mathbf{X}_t, \mathbf{A}_t) \) serves as the original graph of our GC framework. Each node is associated with one of \( c \) classes, encoded as numeric labels \( \mathbf{y}_t \in \{1, \dots, c\}^{N_t} \) and one-hot labels \( \mathbf{Y}_t \in \mathbb{R}^{N_t \times c} \). Any GC method focuses on generating a condensed graph \( G'_t = (\mathbf{X}'_t, \mathbf{A}'_t) \) from the original graph \( G_t = (\mathbf{X}_t, \mathbf{A}_t) \), preserving the key structural and feature information required for downstream tasks. 
% As our method is structure-free, the adjacency matrix of the generated condensed graph is defined as \( \mathbf{A}'_t = \mathbf{I} \). 

\vskip 0.3em \noindent\textbf{Evolving Graph Condensation Formulation.}\label{sec:evolving_form} In the evolving graph scenario, we consider a sequential stream of graph batches \(\{B_1, B_2, \dots, B_m\}\), where \(m\) represents the total number of time steps. Each graph batch \(B_i = (\mathbf{X}_i, \mathbf{A}_i)\) contains newly added nodes along with their associated edges in \textit{inductive} graphs, while in \textit{transductive} graphs, it contains newly labeled nodes but retains the entire graph structure~\cite{su2023robustincremental}. 
% The differences between these two kind of evolving are illustrated in Figure~\ref{fig:main}. 
These graph batches are progressively integrated into the existing graph over time, constructing a series of incremental graphs \(\{G_1, G_2, \dots, G_t\}\). At time step \(t\), the snapshot graph \(G_t = (\mathbf{X}_t, \mathbf{A}_t)\) encompasses all nodes and edges that have appeared up to that point, with \(G_t = \bigcup_{i=1}^t B_i\), and importantly, we preserve the distribution of classes across splits as the graph evolves. At each time step \(t\), during the GC phase, we aim to generate a condensed graph \(G'_t = (\mathbf{X}'_t, \mathbf{A}'_t)\) from the snapshot graph \(G_t\). This condensed graph \(G'_t\), which retains the essential structural and feature information of \(G_t\), is then used to train GNNs efficiently for any downstream tasks. During deployment, the GNN trained on \(G'_t\) is applied to classify the nodes in \(G_t\). As new graph batches are added at time step \(t+1\), the graph expands to \(G_{t+1}\). Unlike previous approaches that require repeating the GC process from scratch, our method can effectively inherit the condensed graph from the previous time step with minimal computational overhead, ensuring that \(G'_{t+1}\) effectively represents the expanded graph \(G_{t+1}\) while maintaining high performance on the growing dataset.






\begin{figure*}[t]
    \centering
    \includegraphics[width=0.98\textwidth]{figures/Method.pdf}
    \vskip -1.5em
    \caption{Overall framework of \method{}, which distills knowledge from a teacher model to a student MLP using (a) Multi-Scale Distillation and (b) Multi-Period Distillation at both feature and prediction levels. (a) Multi-Scale Distillation involves downsampling the original time series into multiple coarser scales and aligning these scales between the student and teacher. (b) Multi-Period Distillation applies FFT to transform the time series into a spectrogram, followed by matching the period distributions after applying softmax.}
    \label{fig:method}
    \vskip -1em
\end{figure*}

\vskip -2em
\section{Methodology}

% In alignment with our intuition about preserving multi-scale and multi-period pattern knowledge, we introduce a novel distillation framework named \method{}. Unlike conventional approaches that emphasize matching predictions or time series representations directly, \method{} focuses on transferring knowledge of multi-scale patterns in the temporal domain and multi-period patterns in the frequency domain. 
% To efficiently distill this knowledge from the teacher model to the MLP, we propose two specific distillation objectives: \textit{multi-scale distillation} and \textit{multi-period distillation}, which we will detail next. The overall framework of \method{} is shown in Figure~\ref{fig:method}.

% \wei{it could be good to introduce an overall framework for KD (using equations) in time series and the following subsections describe how \method{} address it? }

% \wei{Standardize the notations: (1) bold upper-case letters for matrices; (2) bold lower-case letters for vectors; (3) regular letters for scalars (typically lower cases while upper cases are fine). You can take a look at the itransformer paper}

Motivated by our preliminary studies, we propose a novel \textit{KD} framework \method{} for time series to transfer the knowledge from a fixed, pretrained teacher model \(f_t\) to a student MLP model \(f_s\). The student produces predictions \(\mathbf{\hat{Y}}_s \in \mathbb{R}^{S \times C}\) and internal features \(\mathbf{H}_s\in \mathbb{R}^{D \times C}\). The teacher model produces predictions \(\mathbf{\hat{Y}}_t \in \mathbb{R}^{S \times C}\) and internal features \(\mathbf{H}_t\in \mathbb{R}^{D_t \times C}\). Our general objective is:
\begin{equation}\label{eq:kd_obj}
    \min\nolimits_{\theta_s} \mathcal{L}_{sup}(\mathbf{Y}, \mathbf{\hat{Y}}_s) + \mathcal{L}_{\mathrm{KD}}^\mathbf{Y}(\mathbf{\hat{Y}}_t, \mathbf{\hat{Y}}_s) + \mathcal{L}_{\mathrm{KD}}^\mathbf{H}(\mathbf{H}_t, \mathbf{H}_s),
\end{equation}
where \(\theta_s\) is the parameter of the student; \(\mathcal{L}_{sup}\) is the supervised loss (e.g., MSE) between predictions and ground truth; \(\mathcal{L}_{\mathrm{KD}}^\mathbf{Y}\) and \(\mathcal{L}_{\mathrm{KD}}^\mathbf{H}\) are the distillation loss terms that encourage the student model to learn knowledge from the teacher on both \textbf{prediction level}~\cite{hinton2015distilling} and \textbf{feature level}~\cite{romero2014fitnets}, respectively. Unlike conventional approaches that emphasize matching model predictions, \method{} integrates key time-series patterns including multi-scale and multi-period knowledge. The overall framework of \method{} is shown in Figure~\ref{fig:method}. 
% In the following, we detail each component of our approach.



% In alignment with our intuition about preserving multi-scale and multi-period pattern knowledge, we introduce a novel distillation framework named \method{}. Unlike conventional approaches that emphasize matching predictions directly, \method{} focuses on transferring knowledge of multi-scale patterns in the temporal domain and multi-period patterns in the frequency domain. 
% The overall framework of \method{} is shown in Figure~\ref{fig:method}. In the following, we provide details on each component of our approach.To efficiently distill this knowledge from the teacher model to the MLP, we propose two specific distillation objectives: \textbf{multi-scale distillation} and \textbf{multi-period distillation}. The overall framework of \method{} is shown in Figure~\ref{fig:method}. In the following, we provide details on each component of our approach.

\subsection{Multi-Scale Distillation}
One key component of \method{} is multi-scale distillation, where ``multi-scale'' refers to representing the same time series at different sampling rates. This enables MLP to effectively capture both coarse-grained and fine-grained patterns. By distilling at both the prediction level and the feature level, we ensure that MLP not only replicates the teacher's multi-scale predictions but also aligns with its internal representations from the intermediate layer.

\vspace{-0.5em}
\paragraph{Prediction Level.}
At the prediction level, we directly transfer multi-scale signals from the teacher’s outputs to guide the MLP’s predictions. We first produce multi-scale predictions by downsampling the original predictions from the teacher \(\mathbf{\hat{Y}}_t \in \mathbb{R}^{S \times C}\) and the MLP \(\mathbf{\hat{Y}}_s \in \mathbb{R}^{S \times C}\), where \(S\) is the prediction length and \(C\) is the number of variables. The predictions at \textit{Scale 0} are equal to the original predictions, that is, \(\mathbf{\hat{Y}}_t^0=\mathbf{\hat{Y}}_t\) and \(\mathbf{\hat{Y}}_s^0=\mathbf{\hat{Y}}_s\). We then downsample these predictions across \(M\) scales using convolutional operations with a stride of 2, generating multi-scale prediction sets \(\mathcal{Y}_t = \{\mathbf{\hat{Y}}_t^0, \mathbf{\hat{Y}}_t^1,\cdots,\mathbf{\hat{Y}}_t^M\}\) and \(\mathcal{Y}_s = \{\mathbf{\hat{Y}}_s^0, \mathbf{\hat{Y}}_s^1,\cdots,\mathbf{\hat{Y}}_s^M\}\), where \(\mathbf{\hat{Y}}_t^M, \mathbf{\hat{Y}}_s^M \in \mathbb{R}^{\lfloor S/2^M \rfloor \times C}\). The downsampling is defined as: 
\begin{equation}
    \mathbf{\hat{Y}}_x^m = \mathrm{Conv}(\mathbf{\hat{Y}}_x^{m-1}, \mathrm{stride}=2),
    \label{eq:multiscale_downsample}
\end{equation}
where \(x \in \{t, s\}\), \(m \in \{1, \cdots, M\}\), $\mathrm{Conv}$ denotes a 1D-convolutional layer with a temporal stride of 2. The predictions at the lowest level \(\mathbf{\hat{Y}}_x^0=\mathbf{\hat{Y}}_x\) maintain the original temporal resolution, while the highest-level predictions \(\mathbf{\hat{Y}}_x^M\) represent coarser patterns. We define the multi-scale distillation loss at the prediction level as:
\begin{equation}
    \mathcal{L}_{scale}^\mathbf{Y} = \textstyle\sum_{m=0}^M ||\mathbf{\hat{Y}}_t^m - \mathbf{\hat{Y}}_s^m||^2 /(M+1).
\end{equation}
% \wei{if Y is a vector or matrix, we should not use () but $| |$} 
Here we use MSE loss to match the MLP’s predictions to the teacher’s predictions at multiple scales.

\vspace{-0.5em}
\paragraph{Feature Level.} 
At the feature level, we align MLP’s intermediate features with teacher’s multi-scale representations, enabling MLP to form richer internal structures that support more accurate forecasts.
Let \(\mathbf{H}_s \in \mathbb{R}^{D \times C}\) and \(\mathbf{H}_t \in \mathbb{R}^{D_t \times C}\) denote MLP and teacher features with feature dimensions \(D\) and \(D_t\), respectively. As their dimensions can be different, we first use a parameterized regressor (e.g. MLP) to align their feature dimensions: 
% \wei{what does the regressor mean?? no explanation for this operator: what is the detailed operator; what is the purpose}
\begin{equation}
    \mathbf{H}'_t = \text{Regressor}(\mathbf{H}_t),
\end{equation}
where \(\mathbf{H}'_t \in \mathbb{R}^{D \times C}\).  
% We then downsample both sets of features across multiple scales:
% \begin{equation}
%     \mathbf{H}_x^m = \mathrm{Conv}(\mathbf{H}_x^{m-1}, \mathrm{stride}=2),
% \end{equation}
% where \(x \in \{t, s\}\), \(m \in \{1, \cdots, M\}\), and \(\mathbf{H}_s^0 = \mathbf{H}_s\), \(\mathbf{H}_t^0 = \mathbf{H}'_t\). 
Similar to the prediction level, we compute $\mathbf{H}_x^m$ by downsampling $\mathbf{H}_s$ and $\mathbf{H}'_t$ across multiple scales using the same approach as in Equation~\ref{eq:multiscale_downsample}. We define the multi-scale distillation loss at the feature level as:
\begin{equation}
    \mathcal{L}_{scale}^\mathbf{H} = \textstyle\sum_{m=0}^M ||\mathbf{H}_t^m - \mathbf{H}_s^m||^2 /(M+1).
\end{equation}

\subsection{Multi-Period Distillation}
% \wei{seems that the prediction level and feature level are basically using the same equations? then we probably do not so much space to describe them given the redundancy}
% \wei{we need some transitions: in addition to multi-scale ...temporal domain...}  
{In addition to multi-scale distillation in the temporal domain, we further
propose multi-period distillation to help MLP learn complex periodic patterns in the frequency domain.} By aligning periodicity-related signals from the teacher model at both the prediction and feature levels, the MLP can learn richer frequency-domain representations and ultimately improve its forecasting performance.

\paragraph{Prediction Level.}
For the predictions from the teacher \(\mathbf{\hat{Y}}_t \in \mathbb{R}^{S \times C}\) and the MLP \(\mathbf{\hat{Y}}_s \in \mathbb{R}^{S \times C}\), we first identify their periodic patterns. We perform this in the frequency domain using the Fast Fourier Transform (FFT):
\begin{equation}
    \mathbf{A}_x = \text{Amp}(\text{FFT}(\mathbf{\hat{Y}}_x)),
    \label{eq:multiperiod_spectrograms}
\end{equation}
where \(x \in \{t, s\}\) and spectrograms \(\mathbf{A}_x \in \mathbb{R}^{\frac{S}{2} \times C}\). Here, \(\text{FFT}(\cdot)\) denotes the FFT operation and \(\text{Amp}(\cdot)\) calculates the amplitude. We remove the direct current (DC) component from \(\mathbf{A}_x\). For certain variable \(c\), the \(i\)-th value \(\mathbf{A}_x^{i,c}\) indicates the intensity of the frequency-\(i\) component, corresponding to a period length \(\lceil S/i\rceil\). Larger amplitude values indicate that the associated frequency (period) is more significant.

To reduce the influence of minor frequencies and avoid noise introduced by less meaningful frequencies~\cite{timesnet, fedformer}, we propose a distribution-based matching scheme. We use a softmax function with a colder temperature to highlight the most significant frequencies:
\begin{equation}
    \mathbf{Q}_x^\mathbf{Y} = {\exp\bigl(\mathbf{A}_x^i / \tau\bigr)}/{\sum\nolimits_{j=1}^{S/2} \exp\bigl(\mathbf{A}_x^j /\tau\bigr)},
    \label{eq:multiperiod_distribution}
\end{equation}
where \(\mathbf{Q}_x^\mathbf{Y} \in \mathbb{R}^{\frac{S}{2} \times C}\) and \(\tau\) is a temperature parameter that controls the sharpness of the distribution. We set \(\tau=0.5\) by default. The period distribution \(\mathbf{Q}_x^\mathbf{Y}\) represents the multi-period pattern in the prediction time series, which we want the MLP to learn from the teacher. We use KL divergence to match these distributions~\cite{hinton2015distilling}. We define the multi-period distillation loss at the prediction level as:
\begin{equation}
    \mathcal{L}_{period}^\mathbf{Y} = \text{KL}\bigl(\mathbf{Q}_t^\mathbf{Y}, \mathbf{Q}_s^\mathbf{Y}\bigr).
\end{equation}
% where KL denotes the Kullback--Leibler divergence~\cite{hinton2015distilling}, a common metric to measure distribution difference.

% \paragraph{Feature Level.}
% Similar to the prediction level, we also apply multi-period distillation at the feature level. We compute:
% \begin{equation}
%     \mathbf{B}_x = \text{Amp}(\text{FFT}(\mathbf{\hat{H}}_x)),
% \end{equation}
% \begin{equation}
%     \mathbf{Q}_x^\mathbf{H} = \frac{\exp\bigl(\mathbf{B}_x^i / \tau\bigr)}{\sum_{j=1}^{D/2} \exp\bigl(\mathbf{B}_x^j /\tau\bigr)},
% \end{equation}
% and define:
% \begin{equation}
%     \mathcal{L}_{period}^\mathbf{H} = \text{KL}\bigl(\mathbf{Q}_t^\mathbf{H}, \mathbf{Q}_s^\mathbf{H}\bigr),
% \end{equation}
% where \(\mathbf{B}_x, \mathbf{Q}_x^\mathbf{H} \in \mathbb{R}^{\frac{D}{2} \times C}\). These feature-level distributions represent the multi-period pattern in feature space, enabling the MLP to learn periodic structure from the teacher at the feature level.

\vspace{-0.5em}
\paragraph{Feature Level.}
Similar to the prediction level, we apply multi-period distillation at the feature level. For the features \(\mathbf{H}'_t \in \mathbb{R}^{D \times C}\) and \(\mathbf{H}_s \in \mathbb{R}^{D \times C}\), we compute the spectrograms and the corresponding period distributions \(\mathbf{Q}_x^\mathbf{H}\) using the same approach as in Equations~\ref{eq:multiperiod_spectrograms} and~\ref{eq:multiperiod_distribution}. Multi-period distillation loss at feature level is then defined as:
\begin{equation}
    \mathcal{L}_{period}^\mathbf{H} = \text{KL}\bigl(\mathbf{Q}_t^\mathbf{H}, \mathbf{Q}_s^\mathbf{H}\bigr).
\end{equation}

\subsection{Overall Optimization and Theoretical Analaysis}
During the training of \method{}, we jointly optimize both the multi-scale and multi-period distillation losses at both the prediction and feature levels, together with the supervised ground-truth label loss:
\begin{equation}
    \mathcal{L}_{sup} = ||\mathbf{Y} - \mathbf{\hat{Y}}_s||^2,
\end{equation}
where \(\mathcal{L}_{sup}\) is the ground-truth loss (for example, MSE loss) used to train MLP directly. Thus, the overall training loss for the student MLP is defined as:
\begin{equation}
    \mathcal{L} = \mathcal{L}_{sup} + \alpha \cdot \bigl(\mathcal{L}_{scale}^\mathbf{Y} + \mathcal{L}_{period}^\mathbf{Y}\bigr) + \beta \cdot \bigl(\mathcal{L}_{scale}^\mathbf{H} + \mathcal{L}_{period}^\mathbf{H}\bigr),
    \label{eq:overall_optimization}
\end{equation}
where \(\alpha\) and \(\beta\) are hyper-parameters that control the contributions of the prediction-level and feature-level distillation loss terms, respectively. The teacher model is pretrained and remains frozen throughout the training process of MLP.


\underline{\textbf{Theoretical Interpretations.}} We provide a theoretical understanding of multi-scale and multi-period distillation loss from \textbf{a novel data augmentation perspective}. We further show that the proposed distillation loss can be interpreted as training with augmented samples derived from a special \textit{mixup}~\cite{mixup} strategy. The distillation process augments data by blending ground truth with teacher predictions, analogous to label smoothing in classification, and provides several benefits for time series forecasting:
\textit{\textbf{(1)} Enhanced Generalization:} It enhances generalization by exposing the student model to richer supervision signals from augmented samples, thus mitigating overfitting, especially with limited or noisy data.
{\textit{\textbf{(2)} Explicit Integration of Patterns:} The augmented supervision signals explicitly incorporate patterns across multiple scales and periods, offering insights that are not immediately evident in the raw ground truth.}
\textit{\textbf{(3}) Stabilized Training Dynamics:} The blending of targets softens the supervision signals, which diminishes the model’s sensitivity to noise and leads to more stable training phases. This will in turn support smoother optimization dynamics and fosters improved convergence. For clarity, our discussion is centered at the prediction level.  We present the following theorem:  
% \wei{in eq. 12, we used alpha and beta; we wanna avoid abusing notations. you may use $\lambda_1$ and $\lambda_2$ in eq. 12 or change the threom}
% \begin{theorem} \label{thm:multiscale}
% Let $(x, y)$ denote original input data pairs and $(x, y^t)$ represent corresponding teacher data pairs. Consider a data augmentation function $\mathcal{A}(\cdot)$ applied to $(x, y)$, generating augmented samples $(x', y')$. Define the training loss on these augmented samples as $\mathcal{L}_{aug} = \textstyle\sum_{(x',y') \in \mathcal{A}(x,y)} |f_s(x') - y'|^2$. Then, the following inequality holds: 
% $\mathcal{L}_{sup} + \lambda \mathcal{L}_{scale} \geq \mathcal{L}_{aug}$
% % \begin{equation}
% %    \mathcal{L}_{sup} + \alpha \mathcal{L}_{scale} \geq \mathcal{L}_{aug}
% % \end{equation}
% when $\mathcal{A}(\cdot)$ is instantiated as a mixup function~\cite{mixup} that interpolates between the original input data $(x,y)$ and teacher data $(x,y^t)$ with a mixing coefficient $\lambda \in (0,1)$, i.e. $y' = \lambda y^t + (1-\lambda) y$.
% \end{theorem}
\begin{theorem} \label{thm:multiscale}
Let $(x, y)$ denote original input data pairs and $(x, y^t)$ represent corresponding teacher data pairs. Consider a data augmentation function $\mathcal{A}(\cdot)$ applied to $(x, y)$, generating augmented samples $(x', y')$. Define the training loss on these augmented samples as $\mathcal{L}_{aug} = \sum_{(x',y') \in \mathcal{A}(x,y)} |f_s(x') - y'|^2$. Then, the following inequality holds: 
$
   \mathcal{L}_{sup} + \eta \mathcal{L}_{scale} \geq \mathcal{L}_{aug},
$
when $\mathcal{A}(\cdot)$ is instantiated as a mixup function~\cite{mixup} that interpolates between the original input data $(x,y)$ and teacher data $(x,y^t)$ with a mixing coefficient $\lambda=\frac{1}{1+\eta}$, i.e. $y' = \lambda y + (1-\lambda) y^t$.
\end{theorem}
We provide proof of Theorem~\ref{thm:multiscale} in Appendix~\ref{app:theory}.  Theorem~\ref{thm:multiscale} suggests that optimizing multi-scale distillation loss \(\mathcal{L}_{\text{scale}}\) jointly with supervised loss \(\mathcal{L}_{\text{sup}}\) is equivalent to minimizing an upper bound on a special \textit{mixup} augmentation loss. In particular, we mix multi-scale teacher predictions \(\{\mathbf{\hat{Y}}_t^{(m)}\}_{m=0}^M\) with ground truth \(\mathbf{Y}\), thereby allowing MLP to learn more informative time series temporal pattern. Similarly, we present a theorem for understanding $\mathcal{L}_{period}$.

% \begin{theorem} \label{thm:multiperiod}
% Define the training loss on the augmented samples using KL divergence as $\mathcal{L}_{aug} = \textstyle\sum_{(x',y') \in \mathcal{A}(x,y)} \text{KL}\big(y', \mathcal{X}(f_s(x'))\big)$, where $\mathcal{X}(\cdot) = \text{Softmax}(\text{Amp}(\text{FFT}(\cdot)))$. Then, the following inequality holds: 
% $\mathcal{L}_{sup} + \lambda \mathcal{L}_{period} \geq \mathcal{L}_{aug}$
% % \begin{equation}
% %    \mathcal{L}_{sup} + \alpha \mathcal{L}_{period} \geq \mathcal{L}_{aug}
% % \end{equation}
% where $\mathcal{A}(\cdot)$ is instantiated as a mixup function that interpolates between the period distribution of original input data $(x,\mathcal{X}(y))$ and teacher data $(x,\mathcal{X}(y^t))$ with a mixing coefficient $\lambda \in (0,1)$, i.e. $y' = \lambda \mathcal{X}(y^t) + (1-\lambda) \mathcal{X}(y)$.
% \end{theorem}
\begin{theorem} \label{thm:multiperiod}
Define the training loss on the augmented samples using KL divergence as $\mathcal{L}_{aug} = \sum_{(x',y') \in \mathcal{A}(x,y)} \text{KL}\big(y', \mathcal{X}(f_s(x'))\big)$, where $\mathcal{X}(\cdot) = \text{Softmax}(\text{FFT}(\cdot))$. Then, the following inequality holds: 
$
   \mathcal{L}_{sup} + \eta\mathcal{L}_{period} \geq \mathcal{L}_{aug},
$
where $\mathcal{A}(\cdot)$ is instantiated as a mixup function that interpolates between the period distribution of original input data $(x,\mathcal{X}(y))$ and teacher data $(x,\mathcal{X}(y^t))$ with a mixing coefficient $\lambda=\eta$, i.e. $y' =  \mathcal{X}(y) + \lambda \mathcal{X}(y^t)$.
\end{theorem}
The proof can be found in Appendix~\ref{app:theory}. Theorem~\ref{thm:multiperiod} shows that optimizing the multi-period distillation loss \(\mathcal{L}_{\text{period}}\) jointly with the supervised loss \(\mathcal{L}_{\text{sup}}\) is equivalent to minimizing an upper bound on the KL divergence between the student period distribution \(\mathcal{X}(f_s(x'))\) (or \(\mathbf{Q}_s\)) and a \emph{mixed} period distribution \(y'\) (or \(\mathbf{Q}_y + \lambda\,\mathbf{Q}_t\)). 
% \wei{we probably do not need (or need to rephrase) the following because it is not very related to the theorem (data augmentation); we need to describe the benefit from the data augmtantion perspective like you did for the above paragraph} This helps the model learn multi-period frequency patterns by incorporating the teacher’s period distribution, thereby identifying and modeling cyclic behaviors with overlapping or multiple periodicities.

% \begin{theorem}\label{thm:multiperiod}
% Optimizing the multi-period distillation loss $\mathcal{L}_{\text{period}}$ 
% is equivalent to minimizing an upper bound on the KL divergence between the student distribution \(\mathbf{Q}_s\) and a \emph{mixed} label distribution \(\alpha\,\mathbf{Q}_y + (1-\alpha)\,\mathbf{Q}_t\).
% Formally, for $\alpha \in [0,1]$, the following inequality holds:
% \[
% \begin{aligned}
% &\alpha\,\mathrm{KL}\bigl(\mathbf{Q}_y, \mathbf{Q}_s\bigr)
% \;+\;
% (1-\alpha)\,\mathrm{KL}\bigl(\mathbf{Q}_t, \mathbf{Q}_s\bigr)\\
% &\ge
% \mathrm{KL}\Bigl(\alpha\,\mathbf{Q}_y + (1-\alpha)\,\mathbf{Q}_t
% \;,\;\mathbf{Q}_s\Bigr).
% \end{aligned}
% \]
% \end{theorem}

% \wei{please number these benefits to enhance readability} These two theorems provide a theoretical understanding of multi-scale and multi-period distillation loss from a novel data augmentation perspective. The distillation process augments data by blending ground truth with teacher predictions, analogous to label smoothing in classification, and provides several benefits for time series forecasting. 
% It enhances generalization by exposing the student model to richer supervision signals, mitigating overfitting, especially with limited or noisy data, and 
% capturing trends or patterns not immediately apparent in the ground truth. 
% Furthermore, the softened targets from this blending reduce sensitivity to noise, stabilize training, and facilitate better convergence by ensuring smoother optimization dynamics.

% \wei{this paragraph is very important; please carefully rewrite; the goal is to highglight the novelty of this theoretical perspective and provide benefits of why our distillation framework can benefit the forecasting process} 


\section{Additional Experimental Results and Analysis}
\label{app:exp}

\subsection{Evaluation of the fastText Filter}
To evaluate the precision of the fastText classifier in filtering general text from web retrieval data, we leverage \texttt{Claude-3-Sonnet} to annotate 20K samples. We then compare the predictions from the fastText filter against these annotated ground-truth labels. The evaluation results are presented in Table~\ref{tab:filter}.
The results indicate that the fastText filter achieves an accuracy of approximately 88\%, suggesting that the filtering outcomes are reliable and trustworthy. Moreover, the higher recall score indicates that the filtered data encompasses most agent-relevant information from the retrieval.
% However, we observed some limitations in the fastText approach. For instance, consider the following example of an advertisement on a radio channel, which should be categorized as general text:
% [Insert example text here]
% In this case, the fastText model incorrectly categorized the text as agent-relevant data. This misclassification likely occurred because fastText relies on gram frequency analysis, and the presence of multiple high-tech terms (e.g., iOS, App, Google Play) in the paragraph may have misled the model.

\begin{table}[ht]
\centering
\fontsize{8}{10}\selectfont\setlength{\tabcolsep}{0.3em}
\begin{tabular}{@{}l>{}cccc}
\toprule
\textbf{Model ($\downarrow$)} & \textbf{Accuracy} & \textbf{F-1} & \textbf{Precision} & \textbf{Recall}\\\midrule
fastText & 87.46 & 87.20 & 83.42 & 91.33 \\\bottomrule
\end{tabular}
\caption{Classification results of the fastText filter.
}\label{tab:filter}
% \vspace{-1ex}
\end{table}
% {'accuracy': 0.8746, 'f1': 0.8720, 'precision': 0.8342, 'recall': 0.9133}

\subsection{Evaluation of Base Models}
As base models often struggle to follow instructions to solve problems, existing works evaluate these models using few-shot prompting~\cite{wei2022chain, shao2024deepseekmath} or by assessing the negative log-likelihood of the final answer~\cite{dubey2024llama} (e.g., selecting the correct choice). However, these evaluation methods are not suitable for agent environments for the following reasons: (1) \textbf{Task Complexity.} Agent environment tasks are significantly more complex than multiple-choice questions, requiring the generation of long sequences of actions rather than selecting a single answer. (2) \textbf{Contextual Task Requirements.} Task requirements are often intricately embedded within the context, leaving insufficient space for few-shot exemplars.
To this end, we evaluate \method-Base on three agent benchmarks (Nexus~\cite{srinivasan2023nexusraven}, API-Bank~\cite{li2023api}, and API-Bench~\cite{patil2023gorilla}) and one general benchmark (MMLU)~\cite{hendrycks2020measuring}, reporting the benchmark loss in Figure~\ref{fig:val_loss}. 

% Evaluating base models on complex problem-solving tasks presents challenges, as these models often struggle to follow instructions directly. Previous studies have addressed this by employing few-shot prompting \cite{wei2022chain, shao2024deepseekmath} or by measuring the negative log-likelihood of the final answer \cite{dubey2024llama} (e.g., the correct choice in multiple-choice questions).
% However, these evaluation methods are inadequate for agent environments due to:

% Task complexity: Agent tasks require generating a long sequence of correct actions, rather than simply selecting a single choice.
% Context constraints: Well-crafted task requirements within the context often leave insufficient space for few-shot exemplars.

% To address these challenges, we evaluate \method-Base on three agent-specific benchmarks (API-Bank, API-Bench, NexusRaven) and one general benchmark (MMLU). Figure \ref{fig:val_loss} presents the benchmark loss results.


\subsection{Main Experimental Results on BFCL-v2}
\begin{table*}[h]
\centering
\fontsize{7}{9}\selectfont\setlength{\tabcolsep}{0.3em}
\begin{tabular}{@{}lcc>{}ccccccc>{}cccccc>{}c@{}}
\toprule
\textbf{Datasets ($\rightarrow$)} & \multicolumn{8}{c}{\textbf{AST}} & \multicolumn{7}{c}{\textbf{Exec}} & \textbf{BFCL-v2}\\
\cmidrule(lr){2-9} \cmidrule(lr){10-16} \cmidrule(lr){17-17}
\textbf{Models ($\downarrow$)} & \textbf{OA} & \textbf{Simple} & \textbf{Python} &  \textbf{Java} & \textbf{JS} & \textbf{MF} & \textbf{PF} & \textbf{PM} & \textbf{OA} & \textbf{Simple} & \textbf{Python} & \textbf{REST} & \textbf{MF} & \textbf{PF} &\textbf{PM} & \textbf{OA} \\\midrule
% \textbf{Models ($\downarrow$)$\slash$Datasets ($\rightarrow$)} & \textbf{Overall} & \textbf{OS} & \textbf{DB} &  \textbf{AlfWorld} & \textbf{KG} & \textbf{Mind2Web} & \textbf{Webshop}\\\midrule
% \textbf{Weights} & 1.0000 & 10.8 & 13.0 & 13.0 & 13.9 & 11.6 & 30.7 \\\midrule
\multicolumn{16}{l}{\emph{Base LLMs}}  \\ \midrule
LLaMA-3-8B~\cite{dubey2024llama} & 0.94 & 1.3 & 1.0 & 2.0 & 1.5 & 0.5 & 0.5 & 0.5 & 0.40 & 2.0 & 1.0 & 1.0 & 0.0 & 0.0 & 0.0 & 17.77 \\
LLaMA-3.1-8B~\cite{dubey2024llama}  & 6.05 & 10.2 & 12.0 & 5.0 & 6.0 & 4.0 & 7.5 & 2.5 & 0.43 & 1.7 & 2.0 & 1.4 & 0.0 & 0.0 & 0.0 & 21.10\\
\rowcolor{teal!12} \method-8B-Base & \textbf{15.4} & \textbf{12.2} & \textbf{15.0} & \textbf{4.0} & \textbf{6.0} & \textbf{25.0} & \textbf{11.5} & \textbf{13.0} & \textbf{2.24} & \textbf{2.9} & \textbf{2.0} & \textbf{4.3} & \textbf{6.0} & \textbf{0.0} & \textbf{0.0} & \textbf{25.18}\\\midrule
\multicolumn{16}{l}{\emph{Open-Source Instruction Fine-Tuned LLMs (Small)}}  \\ \midrule
% LLaMA-2-7B-Chat~\cite{touvron2023llama} & 7B & OS & 0.36 & 4.2 & 8.0 & 0.0 & 2.1 & 7.0 & 11.6 & - & - & - & - & - & - \\
% Vicuna-7B-v1.5~\cite{chiang2023vicuna} & 7B & OS & 0.43 & 9.7 & 8.7 & 0.0 & 2.5 & 9.0 & 2.2 & - & - & - & - & - & -\\
% CodeLLaMA-7B-Instruct~\cite{roziere2023code} & 7B & OS& 0.65 & 4.9 & 12.7 & 0.0 & 8.2 & 12.0 & 25.2 & - & - & - & - & - & - \\
% CodeLLaMA-13B-Instruct~\cite{roziere2023code} & 13B & OS & 0.74 & 3.5 & 9.7 & 0.0 & 10.4 & 14.0 & 43.8 & - & - & - & - & - & - \\
% LLaMA-2-13B-Chat~\cite{touvron2023llama} & 13B & OS & 0.66 & 4.2 & 11.7 & 6.0 & 3.6 & 13.0 & 25.3 & - & - & - & - & - & - \\
% Vicuna-13B-v1.5~\cite{chiang2023vicuna} &13B & OS & 0.86 & 10.4 & 6.7 & 8.0 & 9.4 & 12.0 & 41.7 & - & - & - & - & - & - \\
% Groq-8B-Tool-Use~\cite{groq} & 8B & OS & 1.27 & 15.3 & 11.7 & 4.0 & 17.6 & \underline{23.0} & 53.4 & 30.44 & 42.8 & 35.5 & 45.5 & 0.0 & \textbf{89.06} \\
LLaMA-3-8B-Instruct~\cite{dubey2024llama} & 60.47 & 58.3 & 65.5 & 38.0	& 42.0 & 76.5 & 58.0 & \textbf{49.0} & 68.88 & 44.5 & 89.0 & 55.7 & 86.0 & \textbf{78.0} & \textbf{55.0} & 59.57\\
LLaMA-3.1-8B-Instruct~\cite{dubey2024llama} & 58.38 & 60.0 & 68.8 & 32.0 & 46.0 & 66.5 & 65.0 & 42.0 & \textbf{72.60} & 83.7 & 87.0 & 77.1 & 83.0 & 76.0 & 52.5 & 61.39\\
% % Qwen-1.5-7B-Chat~\cite{} & 1.5922	& 11.1	& 41	& 14	& 18.4	& 13	& 56.8 \\\midrule
LLaMA-3-8B-IFT & 47.43 & 66.7 & 75.5 & 37.0 & \textbf{56.0} & 45.5 & 54.0 & 23.5 & 63.41 & \textbf{87.7} & 93.0 & \textbf{80.0} & 68.0 & 58.0 & 40.0 & 62.12\\
\rowcolor{teal!12} \method-8B-IFT & \textbf{66.39} & \textbf{72.5} & \textbf{81.8} & \textbf{45.0} & 54.0 & \textbf{79.5} & \textbf{70.5} & 43.0 & 69.82 & 85.3 & \textbf{95.0} & 71.4 & \textbf{88.0} & 66.0 & 40.0 & \textbf{70.78}\\\midrule
\multicolumn{16}{l}{\emph{For Reference: Open-Source Instruction Fine-Tuned LLMs (Medium to Large) and API-based Commercial LLMs}}  \\ \midrule
% LLaMA-2-70B-Chat~\cite{touvron2023llama} & 70B & OS & 0.66 & 9.7 & 13.0 & 2.0 & 8.0 & 19.0 & 5.6 & - & - & - & - & - & - \\
% CodeLLaMA-34B-Instruct~\cite{roziere2023code} & 34B & OS & 1.13 & 2.8 & 14.0 & 4.0 & 23.5 & 20.0 & 52.1 & - & - & - & - & - & - \\
Gemini-1.5-Flash~\cite{reid2024gemini} & 77.44 & 67.3 & 92.8 & 55.0 & 54.0 & 94.0 & 71.5 & 77.0 & 73.23 & 57.9 & 93.0 & 22.9 & 86.0 & 74.0 & 75.0 & 70.75\\
% text-davinci-003~\cite{ouyang2022training} & - & API & 1.90 & 20.1 & 16.3 & 20.0 & 34.9 & 26.0 & 61.7 & - & - & - & - & - & - \\
% DeepSeek-v2~\cite{liu2024deepseek} & 236B & OS & 1.97	& 20.8	& 21.7	& 38.0	& 21.7	& 22.0	& 57.4 & - & - & - & - & - & -\\
Mixtral-8x22B~\cite{jiang2024mixtral} & 57.92 & 67.2 & 87.5 & 54.0 & 60.0 & 82.0 & 50.5 & 32.0 & 63.59 & 71.9 & 88.0 & 55.7 & 74.0 & 56.0 & 52.5 & 63.26\\
gpt-3.5-turbo-0125~\cite{chatgpt} & 66.31 & 63.8 & 75.3 & 50.0 & 66.0 & 78.0 & 68.0 & 55.5 & 65.88 & 44.5 & 89.0 & 0.0 & 86.0 & 78.0 & 55.0  & 66.53 \\
Claude-3-Haiku~\cite{claude-3} & 62.52 & 77.6 & 95.8 & 63.0 & 74.0 & 93.0 & 47.5 & 32.0 & 60.73 & 89.4& 96.0 & 82.9 & 94.0 & 32.0 & 27.5 & 55.47 \\
Command-R-Plus-FC~\cite{command-r-plus} & 77.65 & 69. 6 & 85.8 & 61.0 & 62.0 & 88.0 & 82.5 & 70.5 & 77.41 & 89.1 & 94.0 & 84.3 & 86.0 & 82.0 & 52.5 & 76.29\\
LLaMA-3-70B-Instruct~\cite{dubey2024llama} & 87.90 & 75.6 & 94.8 & 60.0 & 72.0 & 94.0 & 93.0 &89.0 & 88.04 & 94.1 & 94.0 & 94.3 & 94.0 & 84.0 & 80.0  & 84.95\\
gpt-4-0613~\cite{achiam2023gpt} & 91.92 & 81.2 & 95.5 & 68.0 & 80.0 & 96.0 & 96.0 & 94.5 & 87.57 & 98.3 & 98.0 & 98.6 & 96.0 & 86.0 & 70.0 & {89.26} \\\bottomrule
\end{tabular}
\caption{Main experiment results on BFCL-v2. 
% \wrj{Is 89.06 of Groq-8B-Tool-Use on BFCL-v2 an abnormal value? Or the best acc to be highlighted in bold? Groq-8B-Tool-Use should be second best on Agent-Bench WB? LLaMa-3.1-Instruct should be second best on AgentBench OS? }
}\label{tab:bfcl-v2}
% \vspace{-1ex}
\end{table*}

Table~\ref{tab:bfcl-v2} displays detailed experimental results on BFCL-v2, covering AST and Execution, two aspects in evaluation of function calling capabilities.
Aside from the notations across the other tables, ``JS'' indicates ``JavaScript''; ``MF'', ``PF'', and ``PM'' refer to ``multiple functions'', ``parallel functions'', ``parallel multiple functions''.
The superior performance of \texttt{\method-3-8B} in AST evaluations indicates that the pre-training stage successfully introduced syntax knowledge of function calling into the model, which also contributes to improvements in the Execution aspect. However, the performance gain in Execution evaluations is less pronounced. This is because, lacking access to the instruction fine-tuning data used for \texttt{LLaMA-3-8B}, our \texttt{\method-8B-IFT} demonstrates limited instruction-following capabilities compared to \texttt{LLaMA-3-8B-Instruct} and \texttt{LLaMA-3.1-8B-Instruct}. Consequently, it is more challenging to follow instructions to generate executable functions.
% The superior performance in AST of \method-8B indicates that the pre-training stage successfully introduce syntax knowledge of function calling into the model, which also brings improvement in Execution aspect.
% However, the performance gain in Execution evaluations are less pronounced.
% This is because, as we have no access to the instruction fine-tuning data used on \texttt{LLaMA-3-8B}, our \method-8B-IFT shows limited instruction-following capabilities, compared with \texttt{LLaMA-3-8B-Instruct} and \texttt{LLaMA-3.1-8B-Instruct}.
% This leads to failure in following instructions to generate executable functions.

\subsection{Effect of Backbone LLMs}
\begin{table}[ht]
\centering
\fontsize{8}{10}\selectfont\setlength{\tabcolsep}{0.4em}
\begin{tabular}{@{}l>{}cccccc>{}c@{}}
\toprule
\textbf{Datasets ($\rightarrow$)} & \multicolumn{7}{c}{\textbf{AgentBench}} \\
\cmidrule(lr){2-8}
\textbf{Models ($\downarrow$)} & \textbf{OA} & \textbf{OS} & \textbf{DB} &  \textbf{HH} & \textbf{KG} & \textbf{WB} & \textbf{WS} \\\midrule
Mistral-7B-v0.3-Base~\cite{jiang2023mistral} & 0.40 & 7.6	& 0.7	& 0.0 & 8.9	& 11.0 &	1.4 \\
\rowcolor{teal!12} \method-7B-Base (Mistral) & \textbf{1.46} & \textbf{18.3} & \textbf{21.0} & \textbf{24.0} & \textbf{12.7} & \textbf{14.0} & \textbf{46.2}\\\midrule
Mistral-7B-v0.3-Instruct~\cite{jiang2023mistral} & 1.10 & 18.1 & 15.0	& 4.0 & 	8.9	& 18.0	& 39.6 \\
Mistral-7B-v0.3-IFT & 1.32	& 17.4	& \textbf{18.0}	& 8.0	& 15.9	& 20.0	& 45.1\\
\rowcolor{teal!12} \method-7B-IFT (Mistral) & \textbf{1.72} & \textbf{17.4} & 11.7 & \textbf{30.0} & \textbf{20.1} & \textbf{25.0} & \textbf{55.4}\\\bottomrule
\end{tabular}
\caption{Experimental results of \method-7B (Mistral) with \texttt{Mistral-7B-v0.3} as backbone LLM on AgentBench.
}\label{tab:mistral}
% \vspace{-1ex}
\end{table}

Table~\ref{tab:mistral} reports the performance of \method and the baselines using \texttt{Mistral-7B-v0.3} as backbone LLM on AgentBench. 
% We compare both the base models and instruction-tuned models on AgentBench.
Notably, there exist consistent gains in terms of the average performance on both base model and instruction-tuned model ($1.06$ on base model and $0.4$ on IFT model), justifying the advantage of pre-training on \dataset across different LLM types and architectures.



\section{Conclusion and Outlook}
In this study, we address the challenge of evolving graph condensation. We observe that a universal clustering framework can naturally optimize the assignment matrix, thereby achieving the common objectives of existing GC methods. Additionally, we propose a novel \emph{balanced SSE} metric that further tightens the upper bound of these objectives. In the evolving setting, we find that our clustering approach can be readily adapted to an incremental version, termed \emph{incremental \(k\)-means++}. Experimental results demonstrate that balanced SSE improves the performance of clustering-based GC, and incremental \(k\)-means++ significantly reduces the number of iterations, thereby enhancing efficiency in evolving environments. Future work includes developing more efficient and scalable clustering techniques, especially soft clustering algorithms for larger graph datasets and adaptively optimizing multi-hop weights, which could be beneficial when the graph keeps evolving over time.


% Despite these advances, our proposed method \textsc{GECC} has some limitations, suggesting several avenues for future work:
% \textbf{First}, 
% \textbf{First}, soft clustering is challenging to implement on large datasets, yet it has been proven effective for smaller datasets. This highlights the need for a more scalable and efficient soft clustering algorithm.  
% \textbf{Second}, while clustering has demonstrated significant benefits for GC, its impact on \emph{independent} data domains (e.g., images) remains theoretically underexplored \citep{zhu2021graphheterophily}. Further research is needed to understand its effectiveness in such contexts.  
% \textbf{Finally}, developing a more effective strategy for weighting multi-hop information (e.g., \(\alpha_i\)) is crucial, particularly as evolving graphs may demand deeper message passing or broader contextual aggregation.
% Although the training-free approach shows advantages in this study, we do not rule out the potential for a better optimization solution that can adaptively determine the weights.





%%
%% The next two lines define the bibliography style to be used, and
%% the bibliography file.
% \clearpage
\bibliographystyle{ACM-Reference-Format}
\bibliography{0kdd2025}

\cleardoublepage

\appendix
% \section{Notation Table}

\def \TabNotation{
\begin{table}[]
\centering
\resizebox{!}{0.18\textwidth}{
\begin{tabular}{@{}lc@{}}
\toprule
\textbf{Notation} & \textbf{Explanation} \\ 
\midrule
$\mathcal{M}$ & Language model \\ 
$\Sigma$ & Vocabulary \\
$\xInput$ & Input prompt \\ 
$\predSeq$ & Output responds \\ 
$A= \{ a_i,...a_m\}$ & Set of reference answers \\ 
$C_{\mathcal{M}}(\xInput,\predSeq)$ & Confidence score of $\predSeq$ \\ 
$f(\predSeq,\xInput)$ & Correctness function \\
$sim(\predSeq,A)$ & Similarity score \\
$\tau$ & LLM predefined threshold \\
$n$ & Number of open-form responses \\
$o_i$ & Options in QA dataset \\
$K$ & Number of options \\
\bottomrule
\end{tabular}}
\vspace{-1mm}
\caption{The notation used in this paper}
\vspace{-5mm}
\label{tab:notations}
\end{table}
}
%\TabNotation

% \section{Example Appendix}
% \label{sec:appendix}

\section{Experiments Details}\label{appendix:sec:exp_imp}

\subsection{Dataset Description}\label{sec:datasetDes}


\begin{itemize}
    \item \textbf{C-QA} A multiple-choice dataset designed for commonsense question answering. Each question requires world knowledge and reasoning to determine the correct answer from 5 given choices. The dataset consists of 1,221 test questions.
    
    \item \textbf{QASC} A multiple-choice commonsense reasoning dataset with 8 answer choices per question. Compared to C-QA, QASC presents a higher level of difficulty. While the dataset was originally designed for multi-hop reasoning, our focus is not on evaluating the reasoning capabilities of LLMs. Therefore, we do not provide the supporting facts to the model and instead present only the question. For our experiments, we use the original validation set, which includes 926 questions.
    
    \item \textbf{MedQA} A multiple-choice dataset with 5 options for answers, specifically designed for medical QA. 
    It covers three languages: English, simplified Chinese, and traditional Chinese, and contains 12,723, 34,251, and 14,123 questions for the three languages, respectively.
    The questions are sourced from professional medical board exams, making this dataset particularly challenging due to its reliance on specialized medical knowledge. 
    For our experiments, we randomly selected the first 1,000 questions from the English dataset.
    
    \item \textbf{RACE-m and RACE-h} used in this paper are derived from the RACE (\textbf{R}e\textbf{A}ding \textbf{C}omprehension dataset from \textbf{E}xaminations) dataset, a large-scale machine reading comprehension dataset introduced by Lai et al~\cite{lai2017race}. 
    RACE comprises 27,933 passages and 97,867 questions collected from English examinations for Chinese students aged 12–18. 
    These datasets evaluate a model’s ability to comprehend complex passages and answer questions based on contextual reasoning. 
    Each question is accompanied by four answer choices, with only one correct option. 
    For our experiments, we randomly sampled 1,000 questions from the entire dataset using a fixed random seed of 42 to ensure reproducibility.
\end{itemize}

% \textbf{RACE datasets:} The RACE-h and RACE-m datasets used in this paper are derived from the RACE (\textbf{R}e\textbf{A}ding \textbf{C}omprehension dataset from \textbf{E}xaminations) dataset, a large-scale machine reading comprehension dataset introduced by Lai et al~\cite{lai2017race}. 
% RACE comprises 27,933 passages and 97,867 questions collected from English examinations for Chinese students aged 12–18. 
% The dataset is split into two subsets: RACE-M, which includes 28,293 questions from middle school exams, and RACE-H, containing 69,574 questions from high school exams. Each question in the dataset is paired with four candidate answers, only one of which is correct. Unlike other machine reading comprehension datasets generated through heuristics or crowdsourcing, RACE's questions are designed by domain experts to test human reading and comprehension skills, making it a unique resource for evaluating large language understanding of models. For our specific study, since collecting responses and conducting evaluations is relatively time-consuming, so we conducted a random sample of 1,000 questions extracted from the entire dataset using a random seed of 42 to ensure reproducibility.


% \begin{table}[t!]
%     \centering
%     \begin{tabular}{lcc}
%         \toprule
%         Method & Description & \\
%         \midrule
%         \multicolumn{3}{c}{\textbf{Black-Box Methods}} \\
%         \midrule
%         Ecc(C) & \multicolumn{1}{c}{..} \\ 
%         Deg(C) & \multicolumn{1}{c}{..} \\ 
%         Ecc(E) & \multicolumn{1}{c}{..} \\ 
%         Deg(E) & \multicolumn{1}{c}{..} \\ 
%         Ecc(J) & \multicolumn{1}{c}{..} \\ 
%         Deg(J) & \multicolumn{1}{c}{..} \\ 
%         \midrule
%         \multicolumn{3}{c}{\textbf{White-Box Methods}} \\
%         \midrule
%         P(true) & \multicolumn{1}{c}{..} \\ 
%         CSL & \multicolumn{1}{c}{..} \\ 
%         CSL-next & \multicolumn{1}{c}{..} \\ 
%         SL & \multicolumn{1}{c}{..} \\ 
%         SL(norm) & \multicolumn{1}{c}{..} \\ 
%         TokenSAR & \multicolumn{1}{c}{..} \\ 
%         \bottomrule
%     \end{tabular}
%     \caption{All the baseline methods}
%     \vspace{-5mm}
%     \label{tab:similarity_matrix_stat}
% \end{table}
%\section{Implement Confidence Estimation Methods}


\subsection{Prompt Details}
\label{sec:appendix_prompt}
\begin{itemize}
    \item We use the following prompt to collect open-form responses for each of the 5 datasets separately.
    
\includegraphics[width=.9\columnwidth]{figures/generate_prompt.png}

    \item We use the following prompt to elicit P(True) confidence score.
    The ``Possible Answer'' is an option from the multiple-choice dataset.
    
\includegraphics[width=.9\columnwidth]{figures/ptrue_prompt.png}
\end{itemize}






\subsection{Computation Resources}
To efficiently process multiple queries, we used vLLM~\cite{kwon2023efficientvllm} for parallel inference.
All experiments were conducted on a Linux server running Ubuntu, equipped with an A100 80GB GPU.


\subsection{Response Generation }
For black-box methods, we mostly adopt the experimental configurations from~\citet{lin2024generating}. 
Sampling-based black-box confidence measures use $n=20$ open-form responses per question. 
The temperature settings for different LLMs are kept at their default values.


\section{Additional Experiments Results}\label{sec:full_results}

\subsection{Full Results of Different Evaluation Metrics}
In the main text, due to space constraints, we only show a subset of the AUROC results.
Here, \cref{appendix:tab:bb:auc,appendix:tab:wb:auc} show the AUROC and AUARC for black-box and white-box confidence measures, respectively. 
Similarly, \cref{appendix:tab:bb:calib,appendix:tab:wb:calib} present RCE and ECE results.
Note that all ECE are based on \textit{calibrated} confidence measures for fair comparisons, as some original confidence measures are not even constrained to $[0,1]$.
For the calibration step, we applied histogram binning method~\cite{KDD_HistogramBinning} on all methods.%, and compute the adaptive calibration error (ACE)~\cite{Nixon_2019_CVPR_Workshops}.


% full table
%%%%%%%%%%%%%%%%%%%%%%%%%%%%%%%%%%%%%%%%%%%%%%%%%%%%%%%%%%%%%%%%%%%%%%%%%%%%%%%%%%%%%%%%%%%%%%%%%%%%%%%%%%%%
%white
%%%%%%%%%%%%%%%%%%%%%%%%%%%%%%%%%%%%%%%%%%%%%%%%%%%%%%%%%%%%%%%%%%%%%%%%%%%%%%%%%%%%%%%%%%%%%%%%%%%%%%%%%%%%%%%%
\begin{table*}[h!]
\centering
\resizebox{\textwidth}{!}{%
\begin{tabular}{llcccccccccccc}
\toprule
\multirow{2}{*}{\textbf{Dataset}} & \multirow{2}{*}{\textbf{Model}} & \multicolumn{6}{c}{\textbf{AUROC $\Uparrow$}} & \multicolumn{6}{c}{\textbf{AUARC} $\Uparrow$} \\ 
\cmidrule(lr){3-8} \cmidrule(lr){9-14}
 &  & Ecc(C) & Deg(C) & Ecc(E) & Deg(E) & Ecc(J) & Deg(J) & Ecc(C) & Deg(C) & Ecc(E) & Deg(E) & Ecc(J) & Deg(J) \\ 
\midrule
\multirow{4}{*}{C-QA}
 & Llama2-7b   & 60.981 & 66.651 & 78.629 & 72.771 & 67.081 & 71.668 & 29.386 & 33.266 & 38.221 & 35.858 & 34.681 & 36.915 \\
 & Llama3-8b   & 57.590 & 62.592 & 80.004 & 73.734 & 65.886 & 76.583 &32.062 & 33.232 & 38.414 & 32.648 & 37.150 & 38.596\\
 & Phi4        & 67.879 & 68.413 & 80.712 & 69.976 & 71.447 & 75.278 & 32.123 & 31.596 & 19.294 & 30.032 & 28.570 & 24.739 \\
 & Qwen2.5-32b & 71.409 & 73.931 & 81.885 & 77.087 & 69.473 & 74.645 & 34.775 & 37.399 & 39.926 & 37.964 & 36.776 & 38.808 \\
\midrule
\multirow{4}{*}{QASC}
 & Llama2-7b   & 58.949 & 61.978 & 73.221 & 69.200 & 61.659 & 66.877 & 17.509 & 19.628 & 25.724 & 23.556 & 21.469 & 23.251 \\
 & Llama3-8b   & 55.121 & 55.446 & 74.912 & 72.033 & 64.124 & 72.657 & 15.785 & 15.952 & 25.199 & 24.163 & 23.198 & 25.786 \\
 & Phi4        & 65.100 & 65.553 & 76.980 & 67.692 & 68.496 & 71.209 & 20.297 & 21.063 & 26.740 & 21.422 & 24.067 & 24.308 \\
 & Qwen2.5-32b & 62.218 & 61.611 & 74.546 & 71.702 & 64.658 & 69.131 & 19.522 & 19.830 & 25.695 & 24.306 & 23.182 & 24.510  \\
\midrule
\multirow{4}{*}{MedQA}
 & Llama2-7b   & 53.683 & 54.129 & 52.076 & 52.963 & 53.137 & 53.778 & 21.956 & 23.105 & 21.160 & 22.863 & 23.454 & 23.371 \\
 & Llama3-8b   & 52.824 & 53.971 & 51.641 & 53.523 & 55.257 & 59.552 & 21.125 & 22.103 & 20.390 & 22.164 & 25.598 & 26.617 \\
 & Phi4        & 60.055 & 59.512 & 54.945 & 55.261 & 57.815 & 65.067 & 25.081 & 25.410 & 22.077 & 22.940 & 27.573 &29.201 \\
 & Qwen2.5-32b & 60.071 & 61.737 & 54.727 & 58.454 & 61.564 & 63.783 & 24.998 & 28.045 & 22.246 & 26.331 & 29.848 & 30.054 \\
\midrule
\multirow{4}{*}{RACE-m}
 & Llama2-7b  & 65.473 & 64.304 & 61.022 & 59.245 & 67.480 & 67.760 & 34.147 & 36.637 & 32.570 & 33.994 & 38.844 & 38.904 \\
 & Llama3-8b & 62.385 & 63.351 & 61.872 & 58.711 & 68.391 & 73.267 & 30.774 & 35.054 & 31.639 & 32.491 & 41.231 & 43.055 \\
 & Phi4     & 66.461 & 64.344 & 64.492 & 58.981 & 68.124 & 72.304 & 34.312 & 35.355 & 32.903 & 32.232 & 41.311 & 41.895  \\
 & Qwen2.5-32b & 65.425 & 67.627 & 60.268 & 61.309 & 75.420 & 75.746 & 34.393 & 37.409 & 32.092 & 34.850 & 44.281 & 44.585 \\
\midrule
\multirow{4}{*}{RACE-h}
 & Llama2-7b    & 58.991 & 53.597 & 57.178 & 54.037 & 59.300 & 59.856 & 34.147 & 36.637 & 32.570 & 33.994 & 38.844 & 38.904 \\
 & Llama3-8b  & 56.372 & 53.560 & 58.456 & 54.004 & 57.488 & 63.788 & 27.959 & 28.483 & 29.120 & 27.823 & 33.912 & 36.139 \\
 & Phi4    & 60.550 & 53.867 & 61.263 & 54.442 & 59.639 & 64.385 & 30.733 & 28.641 & 31.411 & 28.157 & 34.519 & 35.710    \\
 & Qwen2.5-32b   & 60.012 & 54.781 & 55.984 & 55.657 & 64.985 & 66.130 & 31.049 & 29.180 & 30.459 & 28.921 & 37.620 & 37.734 \\
\bottomrule
\end{tabular}%
}
\caption{AUROC and AUARC for black-box methods, across different models and datasets}
\label{appendix:tab:bb:auc}
\end{table*}


% only roc
% \begin{table*}[h!]
% \centering
% \resizebox{0.5\textwidth}{!}{%
% \begin{tabular}{llcccccc}
% \toprule
% \multirow{2}{*}{\textbf{Dataset}} & \multirow{2}{*}{\textbf{Model}} & \multicolumn{6}{c}{\textbf{AUROC $\Uparrow$}} \\
% \cmidrule(lr){3-8}
%  &  & Ecc(C) & Deg(C) & Ecc(E) & Deg(E) & Ecc(J) & Deg(J) \\
% \midrule
% \multirow{4}{*}{C-QA}
%  & Llama2-7b   & 60.981 & 66.651 & 78.629 & 72.771 & 67.081 & 71.668 \\
%  & Llama3-8b   & 57.590 & 62.592 & 80.004 & 73.734 & 65.886 & 76.583 \\
%  & Phi4        & 67.879 & 68.413 & 80.712 & 69.976 & 71.447 & 75.278 \\
%  & Qwen2.5-32b & 71.409 & 73.931 & 81.885 & 77.087 & 69.473 & 74.645 \\
% \midrule
% \multirow{4}{*}{QASC}
%  & Llama2-7b   & 58.949 & 61.978 & 73.221 & 69.200 & 61.659 & 66.877 \\
%  & Llama3-8b   & 55.121 & 55.446 & 74.912 & 72.033 & 64.124 & 72.657 \\
%  & Phi4        & 65.100 & 65.553 & 76.980 & 67.692 & 68.496 & 71.209 \\
%  & Qwen2.5-32b & 62.218 & 61.611 & 74.546 & 71.702 & 64.658 & 69.131 \\
% \midrule
% \multirow{4}{*}{MedQA}
%  & Llama2-7b   & 53.683 & 54.129 & 52.076 & 52.963 & 53.137 & 53.778 \\
%  & Llama3-8b   & 52.824 & 53.971 & 51.641 & 53.523 & 55.257 & 59.552 \\
%  & Phi4        & 60.055 & 59.512 & 54.945 & 55.261 & 57.815 & 65.067 \\
%  & Qwen2.5-32b & 60.071 & 61.737 & 54.727 & 58.454 & 61.564 & 63.783 \\
% \midrule
% \multirow{4}{*}{RACE-m}
%  & Llama2-7b   & 65.473 & 64.304 & 61.022 & 59.245 & 67.480 & 67.760 \\
%  & Llama3-8b   & 62.385 & 63.351 & 61.872 & 58.711 & 68.391 & 73.267 \\
%  & Phi4        & 66.461 & 64.344 & 64.492 & 58.981 & 68.124 & 72.304 \\
%  & Qwen2.5-32b & 65.425 & 67.627 & 60.268 & 61.309 & 75.420 & 75.746 \\
% \midrule
% \multirow{4}{*}{RACE-h}
%  & Llama2-7b   & 58.991 & 53.597 & 57.178 & 54.037 & 59.300 & 59.856 \\
%  & Llama3-8b   & 56.372 & 53.560 & 58.456 & 54.004 & 57.488 & 63.788 \\
%  & Phi4        & 60.550 & 53.867 & 61.263 & 54.442 & 59.639 & 64.385 \\
%  & Qwen2.5-32b & 60.012 & 54.781 & 55.984 & 55.657 & 64.985 & 66.130 \\
% \bottomrule
% \end{tabular}%
% }
% \caption{Black Box Methods Performance Metrics Across Different Models and Datasets (AUROC)}
% \label{tab:metrics_table}
% \end{table*}



% \begin{table*}[h!]
% \centering

%%%%%%%%%%%%%%%%%%%%%%%%%%%%%%%%%%%%%%%%%%%%%%%%%%%%%%%%%%%%%%%%%%%%%%%%%%%%%%%%%%%%%%%%%%%%%%%%%%%%%%%%%%%%
%black
%%%%%%%%%%%%%%%%%%%%%%%%%%%%%%%%%%%%%%%%%%%%%%%%%%%%%%%%%%%%%%%%%%%%%%%%%%%%%%%%%%%%%%%%%%%%%%%%%%%%%%%%%%%%%%%%


\begin{table*}[h!]
\centering
\resizebox{\textwidth}{!}{%
\begin{tabular}{llcccccccccccc}
\toprule
\multirow{2}{*}{\textbf{Dataset}} & \multirow{2}{*}{\textbf{Model}} & \multicolumn{6}{c}{\textbf{AUROC} $\Uparrow$} & \multicolumn{6}{c}{\textbf{AUARC} $\Uparrow$} \\ \cmidrule(lr){3-8} \cmidrule(lr){9-14}
 &  & P(true) & CSL & CSL-next & SL & Perplexity & TokenSAR & P(true) & CSL & CSL-next & SL & Perlexity & TokenSAR \\ \midrule
\multirow{4}{*}{C-QA}
 & Llama2-7b   & 62.278 & 78.253 & 74.799 & 81.390 & 76.958 & 77.888 &  28.401 & 38.231 & 36.213 & 40.178 & 37.579 & 37.450 \\
 & Llama3-8b   & 82.760 & 78.423 & 73.068 & 81.731 & 75.503 & 75.385 & 40.235 & 38.191 & 35.096 & 40.152 & 36.368 & 35.453  \\
 & Phi4        & 86.184 & 77.382 & 73.477 & 78.903 & 75.471 & 75.722 & 42.447 & 37.984 & 35.749 & 38.452 & 36.928 & 36.630  \\
 & Qwen2.5-32b & 89.892 & 82.486 & 77.087 & 82.143 & 78.964 & 79.064 & 45.449 & 40.802 & 38.003 & 40.596 & 38.674 & 38.215\\%[1ex]
 \midrule
\multirow{4}{*}{QASC}
 & Llama2-7b   & 66.198 & 77.535 & 76.053 & 79.589 & 77.637 & 77.696 &  19.815 & 25.986 & 25.494 & 27.632 & 26.324 & 25.921\\
 & Llama3-8b   & 86.069 & 77.970 & 73.090 & 80.718 & 74.531 & 75.006 &  30.127 & 26.215 & 24.251 & 28.253 & 24.442 & 24.308 \\
 & Phi4        & 84.478 & 77.556 & 74.596 & 78.661 & 75.678 & 76.222 & 29.977 & 26.068 & 25.246 & 27.064 & 25.463 &25.307 \\
 & Qwen2.5-32b & 88.998 & 79.324 & 73.895 & 78.598 & 74.485 & 75.175 & 32.992 & 26.810 & 24.608 & 27.387 & 24.069 & 23.992  \\%[1ex]
  \midrule
\multirow{4}{*}{MedQA}
 & Llama2-7b   & 54.660 & 55.144 & 55.852 & 54.766 & 55.766 & 55.703 & 22.414 & 24.437 & 24.888 & 24.246 & 24.848 & 24.795  \\
 & Llama3-8b   & 77.493 & 57.384 & 57.894 & 57.919 & 57.592 & 57.530 & 36.884 & 24.072 & 25.225 & 25.879 & 24.973 & 24.803 \\
 & Phi4 & 86.888 & 65.550 & 64.284 & 63.287 & 65.588 & 65.696 &42.615 & 31.671 & 31.050 & 30.888 &31.752 & 31.775  \\
 & Qwen2.5-32b & 80.131 & 63.264 & 63.712 & 63.109 & 62.564 & 62.164 &40.197 & 27.495 & 27.754 & 29.382 & 27.440 & 27.221  \\%[1ex]
  \midrule
\multirow{4}{*}{RACE-m}
 & Llama2-7b  & 63.965 & 69.194 & 70.819 & 67.568 & 71.823 & 71.984 & 35.543 & 38.429 & 39.404 & 38.870 & 40.030 & 40.133 \\
 & Llama3-8b   & 82.118 & 67.317 & 70.875 & 69.321 & 69.851 & 70.029 & 47.145 & 36.953 & 40.206 & 40.508 & 39.144 & 39.232 \\
 & Phi4        & 90.543 & 68.334 & 69.5354 & 68.8049 & 69.025 & 69.188 & 52.457 & 36.638 & 38.717 & 40.314 & 37.972 & 38.057 \\
 & Qwen2.5-32b  & 56.049 &67.294 & 69.102 & 73.267 & 69.147 & 69.279 & 29.283 & 34.913 & 36.873 & 42.373 & 36.220 & 36.318 \\%[1ex]
  \midrule
\multirow{4}{*}{RACE-h}
 & Llama2-7b   & 61.265 & 61.905 & 62.481 & 59.889 & 63.486 & 63.465 & 35.543 & 38.429 & 39.404 & 38.870 & 40.030 & 40.133 \\
 & Llama3-8b    & 79.466 & 60.775 & 63.868 & 61.253 & 64.134 & 64.146 & 44.910 & 31.300 & 34.086 & 33.463 & 33.973 & 33.974 \\
 & Phi4       & 87.172 & 62.253 & 62.680 & 60.178 & 63.391 & 63.383 &  50.250 & 32.395 & 33.484 & 33.243 & 33.547 & 33.537 \\
 & Qwen2.5-32b   & 52.811 & 61.837 & 64.047 & 63.555 & 64.050 & 64.024 & 27.605 & 31.279 & 32.714 & 34.462 & 32.462 & 32.458 \\
\bottomrule
\end{tabular}%
}
\caption{AUROC and AUARC for white-box methods, across different models and datasets}
\label{appendix:tab:wb:auc}
\end{table*}

% \begin{table*}[h!]
% \centering
% \resizebox{0.5\textwidth}{!}{%
% \begin{tabular}{llcccccc}
% \toprule
% \multirow{2}{*}{\textbf{Dataset}} & \multirow{2}{*}{\textbf{Model}} & \multicolumn{6}{c}{\textbf{AUROC} $\Uparrow$} \\
% \cmidrule(lr){3-8}
%  &  & P(true) & CSL & CSL-next & SL & Perplexity & TokenSAR \\ 
% \midrule
% \multirow{4}{*}{C-QA}
%  & Llama2-7b   & 62.278 & 78.253 & 74.799 & 81.390 & 76.958 & 77.888 \\
%  & Llama3-8b   & 82.760 & 78.423 & 73.068 & 81.731 & 75.503 & 75.385 \\
%  & Phi4        & 86.184 & 77.382 & 73.477 & 78.903 & 75.471 & 75.722 \\
%  & Qwen2.5-32b & 89.892 & 82.486 & 77.087 & 82.143 & 78.964 & 79.064 \\
% \midrule
% \multirow{4}{*}{QASC}
%  & Llama2-7b   & 66.198 & 77.535 & 76.053 & 79.589 & 77.637 & 77.696 \\
%  & Llama3-8b   & 86.069 & 77.970 & 73.090 & 80.718 & 74.531 & 75.006 \\
%  & Phi4        & 84.478 & 77.556 & 74.596 & 78.661 & 75.678 & 76.222 \\
%  & Qwen2.5-32b & 88.998 & 79.324 & 73.895 & 78.598 & 74.485 & 75.175 \\
% \midrule
% \multirow{4}{*}{MedQA}
%  & Llama2-7b   & 54.660 & 55.144 & 55.852 & 54.766 & 55.766 & 55.703 \\
%  & Llama3-8b   & 77.493 & 57.384 & 57.894 & 57.919 & 57.592 & 57.530 \\
%  & Phi4        & 86.888 & 65.550 & 64.284 & 63.287 & 65.588 & 65.696 \\
%  & Qwen2.5-32b & 80.131 & 63.264 & 63.712 & 63.109 & 62.564 & 62.164 \\
% \midrule
% \multirow{4}{*}{RACE-m}
%  & Llama2-7b   & 63.965 & 69.194 & 70.819 & 67.568 & 71.823 & 71.984 \\
%  & Llama3-8b   & 82.118 & 67.317 & 70.875 & 69.321 & 69.851 & 70.029 \\
%  & Phi4        & 90.543 & 68.334 & 69.5354 & 68.8049 & 69.025 & 69.188 \\
%  & Qwen2.5-32b & 56.049 & 67.294 & 69.102 & 73.267 & 69.147 & 69.279 \\
% \midrule
% \multirow{4}{*}{RACE-h}
%  & Llama2-7b   & 61.265 & 61.905 & 62.481 & 59.889 & 63.486 & 63.465 \\
%  & Llama3-8b   & 79.466 & 60.775 & 63.868 & 61.253 & 64.134 & 64.146 \\
%  & Phi4        & 87.172 & 62.253 & 62.680 & 60.178 & 63.391 & 63.383 \\
%  & Qwen2.5-32b & 52.811 & 61.837 & 64.047 & 63.555 & 64.050 & 64.024 \\
% \bottomrule
% \end{tabular}%
% }
% \caption{White Box Methods Performance Metrics Across Different Models and Datasets (AUROC)}
% \label{tab:metrics_table}
% \end{table*}
%%%%%%%%%%%%%%%%%%%%%%%%%%%%%%%%%%%%%%%%%%%%%%%%%%%%%%%%%%%%%%%%%%%%%%%%%%%%%%%%%%%%%%%%
%next
%%%%%%%%%%%%%%%%%%%%%%%%%%%%%%%%%%%%%%%%%%%%%%%%%%%%%%%%%%%%%%%%%%%%%%%%%%%%%%%%%%%%%%%%
%%%%%%%%%%%%%%%%%%%%%%%%%%%%%%%%%%%%%%%%%%%%%%%%%%%%%%%%%%%%%%%%%%%%%%%%%%%%%%%%%%%%%%%%

\begin{table*}[t]
\centering
\resizebox{\textwidth}{!}{%
\begin{tabular}{llcccccccccccc}
\toprule
\multirow{2}{*}{\textbf{Dataset}} & \multirow{2}{*}{\textbf{Model}} & \multicolumn{6}{c}{\textbf{RCE}} & \multicolumn{6}{c}{\textbf{Calibration ECE}} \\ \cmidrule(lr){3-8} \cmidrule(lr){9-14}
 &  & Ecc(C) & Deg(C) & Ecc(E) & Deg(E) & Ecc(J) & Deg(J) & Ecc(C) & Deg(C) & Ecc(E) & Deg(E) & Ecc(J) & Deg(J) \\ \midrule
\multirow{4}{*}{C-QA} 
 & Llama2-7b    & 0.2857  & 0.143722 & 0.117486 & 0.084357 & 0.271789 & 0.198744 & 0.014457 & 0.064792 & 0.025161 & 0.009014 & 0.009546 & 0.031801 \\
 & Llama3-8b    & 0.28071 & 0.15255  & 0.06311  & 0.041246 & 0.362527 & 0.153761 & 0.013865 & 0.044074 & 0.031566 & 0.016865 & 0.008845 & 0.060919 \\
 & Phi4         & 0.18881 & 0.115068 & 0.067507 & 0.038771 & 0.225698 & 0.218135 & 0.017734 & 0.059135 & 0.040364 & 0.024237 & 0.019987 & 0.056875 \\
 & Qwen2.5-32b  & 0.16192 & 0.114378 & 0.080021 & 0.055613 & 0.278165 & 0.198222 & 0.0111   & 0.087857 & 0.043406 & 0.016647 & 0.014439 & 0.051092 \\%[1ex]
  \midrule
\multirow{4}{*}{QASC} 
 & Llama2-7b    & 0.25132 & 0.162559 & 0.193186 & 0.121908 & 0.331258 & 0.252667 & 0.013984 & 0.020481 & 0.019263 & 0.012321 & 0.003108 & 0.022164 \\
 & Llama3-8b    & 0.28697 & 0.231308 & 0.083146 & 0.057512 & 0.401264 & 0.230094 & 0.003117 & 0.005336 & 0.004844 & 0.009398 & 0.010951 & 0.022145 \\
 & Phi4         & 0.19064 & 0.104986 & 0.066258 & 0.063753 & 0.23061  & 0.225091 & 0.004181 & 0.015734 & 0.012447 & 0.01108  & 0.003271 & 0.026654 \\
 & Qwen2.5-32b  & 0.25004 & 0.142512 & 0.091264 & 0.084393 & 0.31938  & 0.272657 & 0.010503 & 0.020774 & 0.012144 & 0.009716 & 0.004127 & 0.023387 \\%[1ex]
  \midrule
\multirow{4}{*}{MedQA} 
 & Llama2-7b    & 0.19817 & 0.188788 & 0.231296 & 0.243174 & 0.263178 & 0.213793 & 0.005909 & 0.006271 & 0.006057 & 0.01008  & 0.007157 & 0.008915 \\
 & Llama3-8b    & 0.21067 & 0.190038 & 0.286932 & 0.194414 & 0.290058 & 0.146904 & 0.006035 & 0.006757 & 0.006424 & 0.006872 & 0.01166  & 0.007277 \\
 & phi4         & 0.09127 & 0.09877  & 0.208792 & 0.132527 & 0.308812 & 0.087518 & 0.008327 & 0.018021 & 0.0156   & 0.008231 & 0.020912 & 0.016443 \\
 & Qwen2.5-32b  & 0.09064 & 0.089393 & 0.194414 & 0.087518 & 0.234422 & 0.118149 & 0.006312 & 0.01598  & 0.011337 & 0.021417 & 0.014092 & 0.021119 \\%[1ex]
  \midrule
\multirow{4}{*}{RACE-m} 
 & Llama2-7b  & 0.09876 & 0.31881 & 0.17315 & 0.27630 & 0.14502 & 0.16065 & 0.04523 & 0.07009 & 0.01980 & 0.01965  & 0.00778 & 0.01433 \\
 & Llama3-8b  & 0.10252 & 0.32068 & 0.12877 & 0.27005 & 0.21254 & 0.04500 & 0.00939 & 0.08513 & 0.00962 & 0.04675 & 0.025705  & 0.03261 \\
 & phi4          & 0.06001 & 0.31756  & 0.11252 & 0.26817 & 0.150655 & 0.07501 & 0.01699 & 0.07599 & 0.03366   & 0.01936 & 0.016385 &  0.01542 \\
 & Qwen2.5-32b  & 0.19378 &  0.32756 &  0.18253 & 0.27505 & 0.09689 & 0.1187 & 0.024623 & 0.10445  & 0.02922 & 0.05540 & 0.01300 & 0.02171 \\%[1ex]
  \midrule
\multirow{4}{*}{RACE-h} 
 & Llama2-7b  & 0.12565 & 0.36069 & 0.22441 & 0.40383 & 0.29568 & 0.30881 & 0.01702 & 0.06116 & 0.01635 & 0.01577  & 0.020679 & 0.01569 \\
 & Llama3-8b   & 0.20316 & 0.37007 & 0.18816 & 0.42070 & 0.26192 & 0.05938 & 0.01754 & 0.06838 & 0.01672 & 0.02324 & 0.02597  & 0.02622\\
 & phi4        & 0.09751 & 0.36757  & 0.14627 & 0.38820 &  0.26880 & 0.15878 & 0.01928 & 0.06393 & 0.021709   & 0.02191 & 0.02294 & 0.02502 \\
 & Qwen2.5-32b  & 0.11564 & 0.35069 & 0.21441 & 0.35569 & 0.28505 & 0.205666 & 0.01679 & 0.06562  & 0.01794 &0.02833 & 0.015137 & 0.01438 \\[1ex]
\bottomrule
\end{tabular}%
}
\caption{RCE and (calibrated) ECE for black-box methods, across different models and datasets}
\label{appendix:tab:bb:calib}
\end{table*}




\begin{table*}[t]
\centering
\resizebox{\textwidth}{!}{%
\begin{tabular}{llcccccccccccc}
\toprule
\multirow{2}{*}{\textbf{Dataset}} & \multirow{2}{*}{\textbf{Model}} & \multicolumn{6}{c}{\textbf{RCE}} & \multicolumn{6}{c}{\textbf{Calibration ECE}} \\ \cmidrule(lr){3-8} \cmidrule(lr){9-14}
 &  & P(true) & CSL & CSL-next & SL & SL(norm) & TokenSAR & P(true) & CSL & CSL-next & SL & SL(norm) & TokenSAR \\ \midrule
\multirow{4}{*}{C-QA} 
 & Llama2-7b    & 0.084386 & 0.0506   & 0.041895  & 0.041267     & 0.038126 & 0.034997 & 0.0102   & 0.035637  & 0.041958  & 0.023881        & 0.04454 & 0.027278 \\
 & Llama3-8b    & 0.040614 & 0.03563  & 0.068102  & 0.031902     & 0.057489 & 0.038742 & 0.01871  & 0.034739  & 0.050008  & 0.022294        & 0.04291 & 0.026352 \\
 & Phi4         & 0.043731 & 0.04626  & 0.046892  & 0.043771     & 0.041858 & 0.03501  & 0.0583   & 0.034232  & 0.055943  & 0.019535        & 0.04302 & 0.030655 \\
 & Qwen2.5-32b  & 0.058105 & 0.02999  & 0.044359  & 0.032513     & 0.044363 & 0.059406 & 0.0369   & 0.022175  & 0.046935  & 0.021905        & 0.03671 & 0.021438 \\%[1ex]
  \midrule
\multirow{4}{*}{QASC} 
 & Llama2-7b    & 0.077448 & 0.04685  & 0.078796  & 0.051258     & 0.043136 & 0.045007 & 0.01119  & 0.024505  & 0.037871  & 0.023245        & 0.0326  & 0.023127 \\
 & Llama3-8b    & 0.030627 & 0.04811  & 0.117522  & 0.050664     & 0.08503  & 0.043753 & 0.00894  & 0.020665  & 0.038958  & 0.025687        & 0.03274 & 0.020785 \\
 & Phi4         & 0.082518 & 0.04437  & 0.116905  & 0.066942     & 0.088115 & 0.049376 & 0.02122  & 0.021401  & 0.0415    & 0.028242        & 0.02548 & 0.033083 \\
 & Qwen2.5-32b  & 0.11997  & 0.04878  & 0.062505  & 0.081237     & 0.073773 & 0.041861 & 0.03096  & 0.014358  & 0.040047  & 0.025111        & 0.02665 & 0.023483 \\%[1ex]
   \midrule
\multirow{4}{*}{MedQA} 
 & Llama2-7b    & 0.181911 & 0.19254  & 0.19879   & 0.191288     & 0.228796 & 0.238798 & 0.00606  & 0.015623  & 0.007533  & 0.00791         & 0.00669 & 0.007449 \\
 & Llama3-8b    & 0.028131 & 0.08939  & 0.121274  & 0.207542     & 0.163158 & 0.178161 & 0.0166   & 0.012721  & 0.008949  & 0.03            & 0.00861 & 0.010613 \\
 & phi4         & 0.05126  & 0.09127  & 0.115648  & 0.176285     & 0.119399 & 0.116273 & 0.02853  & 0.046391  & 0.05184   & 0.058272        & 0.05787 & 0.05535  \\
 & Qwen2.5-32b  & 0.078141 & 0.06126  & 0.07314   & 0.128151     & 0.088143 & 0.075015 & 0.03067  & 0.020881  & 0.033491  & 0.047763        & 0.03295 & 0.032673 \\%[1ex]
   \midrule
\multirow{4}{*}{RACE-m} 
 & Llama2-7b  & 0.16253 & 0.26130 & 0.22254 & 0.13502     & 0.24317 & 0.24567 & 0.00741 & 0.01935 & 0.03113 & 0.01820  & 0.061396 & 0.062452 \\
 & Llama3-8b    & 0.05938 & 0.18003 & 0.09814 & 0.12752     & 0.10252 & 0.12189 & 0.05006 & 0.05534 & 0.04303 & 0.04812        & 0.01986  & 0.02156 \\
 & phi4        & 0.09689 & 0.15753  & 0.09314 & 0.09689     & 0.13127 & 0.13565 & 0.02585 & 0.04808 &0.02775  & 0.032335        & 0.01727 & 0.01938 \\
 & Qwen2.5-32b  & 0.16940 & 0.17566 & 0.17691 & 0.17691     & 0.24567 & 0.25255 &0.00695& 0.04720  & 0.05091 &  0.07564        & 0.07986 &  0.08066 \\%[1ex]
   \midrule
\multirow{4}{*}{RACE-h} 
 & Llama2-7b & 0.17566 & 0.31818 & 0.32318 & 0.33256     & 0.31818 & 0.32256 & 0.01748 & 0.02600 & 0.01719 & 0.01613         & 0.021382 & 0.021339 \\
 & Llama3-8b   & 0.05563 & 0.22316 & 0.12189 & 0.15565     & 0.163782 & 0.149404 & 0.045399 & 0.01684 & 0.031577 & 0.034098        & 0.030134  & 0.030341 \\
 & phi4       & 0.08939 & 0.19566  & 0.150030 & 0.13315     & 0.19316 & 0.19566 & 0.019294 & 0.035576 &  0.02874   & 0.03037        & 0.02238 & 0.040637 \\
 & Qwen2.5-32b    & 0.24754 & 0.22254 & 0.21754 & 0.21316     & 0.29505 & 0.30006 & 0.016826 & 0.02004  &0.02105 & 0.022801        & 0.03156 & 0.04110 \\
\bottomrule
\end{tabular}%
}
\caption{RCE and (calibrated) ECE for white-box methods, across different models and datasets}
\label{appendix:tab:wb:calib}
\end{table*}







\subsection{Additional Visualizations for ROC Curves}
\cref{appendix:fig:ROC} presents the ROC curves for \phiName.
\baselinePTrue achieves much better performance than other confidence measures on the more challenging datasets, likely because \phiName is a relatively advanced model.
On the easier datasets, where we could observe a bigger performance gap between different confidence measures, it is also interesting to see that the general shapes (and rankings at different FPR) are relatively consistent across C-QA and QASC, suggesting stability of \uqeval.



\def \FigAUROCVisHorizontalBar{
\begin{figure}[t]
  \centering
  \begin{subfigure}[b]{\columnwidth}
    \centering
    \includegraphics[width=\columnwidth]{figures/qasc_blackbox.png}
    \caption{AUROC of different black-box methods.}
    \label{fig:cqa_blackbox}
  \end{subfigure}
  \begin{subfigure}[b]{\columnwidth}
    \centering
    \includegraphics[width=\columnwidth]{figures/qasc_whitebox.png}
    \caption{AUROC of different white-box methods.}
    \label{fig:another_dataset}
  \end{subfigure}

  \caption{(a) and (b) show the performance of 4 different LLMs and 12 different confidence estimation methods on the QASC dataset. A higher AUROC indicates better performance.}
  \label{fig:llm_perspective}
\end{figure}

\begin{figure}[t]
  \centering
  \begin{subfigure}[b]{\columnwidth}
    \centering
    \includegraphics[width=\columnwidth]{figures/medqa_blackbox.png}
    \caption{AUROC of different black-box methods.}
    \label{fig:cqa_blackbox}
  \end{subfigure}
  \begin{subfigure}[b]{\columnwidth}
    \centering
    \includegraphics[width=\columnwidth]{figures/medqa_whitebox.png}
    \caption{AUROC of different white-box methods.}
    \label{fig:another_dataset}
  \end{subfigure}

  \caption{(a) and (b) show the performance of 4 different LLMs and 12 different confidence estimation methods on the MedQA dataset. A higher AUROC indicates a better performance.}
  \label{fig:llm_perspective}
\end{figure}
}
%/FigAUROCVisHorizontalBar


% \begin{figure*}
%     \centering
%     \includegraphics[width=0.99\linewidth]{figures/comparison_qwen.pdf}
%     \caption{The comparison of different evaluation metrics using our method to quantify Qwen-2.5-32b model's uncertainty on datasets: RACE-h (harder) and RACE-m (easier).}
%     \label{fig:compareQwen}
% \end{figure*}

% \begin{figure*}
%     \centering
%     \includegraphics[width=0.99\linewidth]{figures/comparison_llama7b.pdf}
%     \caption{The comparison of different evaluation metrics using our method to quantify Llama2-7b model's uncertainty on datasets: RACE-h (harder) and RACE-m (easier).}
%     \label{fig:compareLlama2}
% \end{figure*}

% \begin{figure*}
%     \centering
%     \includegraphics[width=0.99\linewidth]{figures/comparison_llama8b.pdf}
%     \caption{The comparison of different evaluation metrics using our method to quantify Llama3-8b model's uncertainty on datasets: RACE-h (harder) and RACE-m (easier).}
%     \label{fig:compareLlama3}
% \end{figure*}

\begin{figure*}[htbp]
    \centering
    % Row 1: Two subfigures side by side
    \begin{subfigure}[b]{0.45\textwidth}
        \centering
        \includegraphics[width=\textwidth]{figures/c-qa.png}
        \caption{C-QA Dataset}
        \label{fig:subfig1}
    \end{subfigure}
    \hfill
    \begin{subfigure}[b]{0.45\textwidth}
        \centering
        \includegraphics[width=\textwidth]{figures/qasc.png}
        \caption{QASC Dataset}
        \label{fig:subfig2}
    \end{subfigure}

    % Row 2: Two subfigures side by side
    \begin{subfigure}[b]{0.45\textwidth}
        \centering
        \includegraphics[width=\textwidth]{figures/extra/phi4_cqa_race_m_10_update1.png}
        \caption{RACE-m Dataset}
        \label{fig:subfig3}
    \end{subfigure}
    \hfill
    \begin{subfigure}[b]{0.45\textwidth}
        \centering
        \includegraphics[width=\textwidth]{figures/extra/phi4_cqa_race_h_10_update1.png}
        \caption{RACE-h Dataset}
        \label{fig:subfig4}
    \end{subfigure}

    % Row 3: One subfigure occupying most of the width
    \begin{subfigure}[b]{0.45\textwidth}
        \centering
        \includegraphics[width=\textwidth]{figures/medqa.png}
        \caption{MedQA Dataset}
        \label{fig:subfig5}
    \end{subfigure}
    
    \caption{Comparison of different evaluation metrics using our method to quantify the \phiName model's confidence scores across five datasets (C-QA, QASC, RACE-m, RACE-h, MedQA), with increasing difficulty.}
    \label{appendix:fig:ROC}
\end{figure*}

% \section{AI Assistant Usage}

% We used GPT for grammar checking and Copilot as an assistive tool.
\end{document}
\endinput
%%
%% End of file `sample-sigconf.tex'.
