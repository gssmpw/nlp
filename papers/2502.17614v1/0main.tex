%%
%% This is file `sample-sigconf.tex',
%% generated with the docstrip utility.
%%
%% The original source files were:
%%
%% samples.dtx  (with options: `all,proceedings,bibtex,sigconf')
%% 
%% IMPORTANT NOTICE:
%% 
%% For the copyright see the source file.
%% 
%% Any modified versions of this file must be renamed
%% with new filenames distinct from sample-sigconf.tex.
%% 
%% For distribution of the original source see the terms
%% for copying and modification in the file samples.dtx.
%% 
%% This generated file may be distributed as long as the
%% original source files, as listed above, are part of the
%% same distribution. (The sources need not necessarily be
%% in the same archive or directory.)
%%
%%
%% Commands for TeXCount
%TC:macro \cite [option:text,text]
%TC:macro \citep [option:text,text]
%TC:macro \citet [option:text,text]
%TC:envir table 0 1
%TC:envir table* 0 1
%TC:envir tabular [ignore] word
%TC:envir displaymath 0 word
%TC:envir math 0 word
%TC:envir comment 0 0
%%
%% The first command in your LaTeX source must be the \documentclass
%% command.
%%
%% For submission and  of your manuscript please change the
%% command to \documentclass[manuscript, screen, review]{acmart}.
%%
%% When submitting camera ready or to TAPS, please change the command
%% to \documentclass[sigconf]{acmart} or whichever template is required
%% for your publication.
%%
%%
\documentclass[sigconf]{acmart}
% \documentclass[sigconf,anonymous,review]{acmart}
\usepackage{paralist}
\usepackage{multirow}
\usepackage{graphicx}
\usepackage{subcaption}
\usepackage{booktabs}
\usepackage{amsthm}
\usepackage{makecell} % for line breaks in cells
% \newtheorem{theorem}{Theorem}
\usepackage{amsmath}
\newtheorem*{theorem-nonumber}{Theorem}
%%
%% \BibTeX command to typeset BibTeX logo in the docs
\AtBeginDocument{%
  \providecommand\BibTeX{{%
    Bib\TeX}}}

%% Rights management information.  This information is sent to you
%% when you complete the rights form.  These commands have SAMPLE
%% values in them; it is your responsibility as an author to replace
%% the commands and values with those provided to you when you
%% complete the rights form.
\setcopyright{acmlicensed}
\copyrightyear{2018}
\acmYear{2018}
\acmDOI{XXXXXXX.XXXXXXX}
%% These commands are for a PROCEEDINGS abstract or paper.
\acmConference[Conference acronym 'XX]{Make sure to enter the correct
  conference title from your rights confirmation email}{June 03--05,
  2018}{Woodstock, NY}
%%
%%  Uncomment \acmBooktitle if the title of the proceedings is different
%%  from ``Proceedings of ...''!
%%
%%\acmBooktitle{Woodstock '18: ACM Symposium on Neural Gaze Detection,
%%  June 03--05, 2018, Woodstock, NY}
\acmISBN{978-1-4503-XXXX-X/2018/06}


%%
%% Submission ID.
%% Use this when submitting an article to a sponsored event. You'll
%% receive a unique submission ID from the organizers
%% of the event, and this ID should be used as the parameter to this command.
%%\acmSubmissionID{123-A56-BU3}

%%
%% For managing citations, it is recommended to use bibliography
%% files in BibTeX format.
%%
%% You can then either use BibTeX with the ACM-Reference-Format style,
%% or BibLaTeX with the acmnumeric or acmauthoryear sytles, that include
%% support for advanced citation of software artefact from the
%% biblatex-software package, also separately available on CTAN.
%%
%% Look at the sample-*-biblatex.tex files for templates showcasing
%% the biblatex styles.
%%

%%
%% The majority of ACM publications use numbered citations and
%% references.  The command \citestyle{authoryear} switches to the
%% "author year" style.
%%
%% If you are preparing content for an event
%% sponsored by ACM SIGGRAPH, you must use the "author year" style of
%% citations and references.
%% Uncommenting
%% the next command will enable that style.
%%\citestyle{acmauthoryear}

\usepackage{xcolor}
\newcommand{\mo}[1]{\textcolor{red}{@Mo:~#1@}}
\newcommand{\gsb}[1]{\textcolor{blue}{@Gsb:~#1@}}
\newcommand{\wei}[1]{\textcolor{orange}{\#Wei:~#1\#}}
\newcommand{\revised}[1]{\textcolor{brown}{@Revised:~#1@}}
\newcommand{\juntong}[1]{\textcolor{green}{@Juntong:~#1@}}
%%
%% end of the preamble, start of the body of the document source.
\begin{document}

%%
%% The "title" command has an optional parameter,
%% allowing the author to define a "short title" to be used in page headers.
\title{Scalable Graph Condensation with Evolving Capabilities}

%%
%% The "author" command and its associated commands are used to define
%% the authors and their affiliations.
%% Of note is the shared affiliation of the first two authors, and the
%% "authornote" and "authornotemark" commands
%% used to denote shared contribution to the research.
% \author{Anonymous Authors}
\author{Shengbo Gong}
\authornote{Both authors contributed equally to this research.}
\affiliation{%
  \institution{Emory University}
  \city{Atlanta}
  \state{Georgia}
  \country{USA}
  }
\email{shengbo.gong@emory.edu}

\author{Mohammad Hashemi}
\authornotemark[1]
\affiliation{%
  \institution{Emory University}
  \city{Atlanta}
  \state{Georgia}
  \country{USA}
  }
\email{mohammad.hashemi@emory.edu}  

\author{Juntong Ni}
\affiliation{%
  \institution{Emory University}
  \city{Atlanta}
  \state{Georgia}
  \country{USA}
}
\email{juntong.ni@emory.edu}

\author{Carl Yang}
\affiliation{%
  \institution{Emory University}
  \city{Atlanta}
  \state{Georgia}
  \country{USA}
}
\email{j.carlyang@emory.edu}

\author{Wei Jin}
\affiliation{%
  \institution{Emory University}
  \city{Atlanta}
  \state{Georgia}
  \country{USA}
}
\email{wei.jin@emory.edu}

%%
%% The abstract is a short summary of the work to be presented in the
%% article.
\begin{abstract}
Graph data has become a pivotal modality due to its unique ability to model relational datasets. However, real-world graph data continues to grow exponentially, resulting in a quadratic increase in the complexity of most graph algorithms as graph sizes expand. Although graph condensation (GC) methods have been proposed to address these scalability issues, existing approaches often treat the training set as static, overlooking the evolving nature of real-world graph data. This limitation leads to inefficiencies when condensing growing training sets.
In this paper, we introduce GECC (\underline{G}raph \underline{E}volving \underline{C}lustering \underline{C}ondensation), a scalable graph condensation method designed to handle large-scale and evolving graph data. GECC employs a traceable and efficient approach by performing class-wise clustering on aggregated features. Furthermore, it can inherits previous condensation results as clustering centroids when the condensed graph expands, thereby attaining an evolving capability. This methodology is supported by robust theoretical foundations and demonstrates superior empirical performance. Comprehensive experiments show that GECC achieves better performance than most state-of-the-art graph condensation methods while delivering an around 1,000$\times$ speedup on large datasets.
\end{abstract}




%%
%% The code below is generated by the tool at http://dl.acm.org/ccs.cfm.
%% Please copy and paste the code instead of the example below.
%%

% \begin{CCSXML}
% <ccs2012>
%    <concept>
%        <concept_id>10010147.10010257.10010293.10010294</concept_id>
%        <concept_desc>Computing methodologies~Neural networks</concept_desc>
%        <concept_significance>500</concept_significance>
%        </concept>
%  </ccs2012>
% \end{CCSXML}

% \ccsdesc[500]{Computing methodologies~Neural networks}


% \begin{CCSXML}
% <ccs2012>
% <concept>
% <concept_id>10010147.10010257.10010293.10010294</concept_id>
% <concept_desc>Computing methodologies~Neural networks</concept_desc>
% <concept_significance>500</concept_significance>
% </concept>
% <concept>
% <concept_id>10010147.10010257.10010258.10010259.10010263</concept_id>
% <concept_desc>Computing methodologies~Supervised learning by classification</concept_desc>
% <concept_significance>500</concept_significance>
% </concept>
% <concept>
% <concept_id>10002951.10003260.10003277</concept_id>
% <concept_desc>Information systems~Web mining</concept_desc>
% <concept_significance>500</concept_significance>
% </concept>
% </ccs2012>
% \end{CCSXML}

% \ccsdesc[500]{Computing methodologies~Neural networks}
% \ccsdesc[500]{Computing methodologies~Supervised learning by classification}
% \ccsdesc[500]{Information systems~Web mining}



%%
%% Keywords. The author(s) should pick words that accurately describe
%% the work being presented. Separate the keywords with commas.
\keywords{Graph Neural Networks, Data-Efficient Learning}
%% A "teaser" image appears between the author and affiliation
%% information and the body of the document, and typically spans the
%% page.
% \begin{teaserfigure}
%   \includegraphics[width=\textwidth]{sampleteaser}
%   \caption{Seattle Mariners at Spring Training, 2010.}
%   \Description{Enjoying the baseball game from the third-base
%   seats. Ichiro Suzuki preparing to bat.}
%   \label{fig:teaser}
% \end{teaserfigure}

\received{20 February 2007}
\received[revised]{12 March 2009}
\received[accepted]{5 June 2009}

%%
%% This command processes the author and affiliation and title
%% information and builds the first part of the formatted document.
\maketitle

\section{Introduction}\label{sec:introduction}
% -- Outline
% ---- LLMs are popular
% ---- There're many stakeholders in the training and inference loop
% ---- Adversaries in the training loop are a problem -- malpractice, poisoning
% ---- Also, showing compliance
% ---- Need a framework to prove the integrity of the pipeline
% ---- Enter Atlas

% ---- LLMs are popular
In recent years, machine learning (ML) models, have become increasingly popular.
The pervasive use of large language models (LLMs), in particular, and multi-stakeholder
involvement in model creation and deployment exacerbate security and privacy risks.
These considerations are emphasized by the global nature and the complexity of
large-scale ML deployments with different lifecycle stages:
%gathering and sanitizing the data from different sources,
%training and inferencing across many data centers,
%compliance with local laws or corporate policies.

% ---- There're many stakeholders in the training and inference loop
%Additionally, different stages of the ML development pipeline come with their own stakeholders:
\begin{enumerate}[label=\arabic*)]
    \item Collection and sanitation of a \emph{training} dataset from several public and proprietary sources.
    %\item Solicitation and facilitation of training.
    \item Provisioning of the training environment (hardware and software).
    \item Execution of training across many data centers.
    \item Construction of a \emph{testing} dataset from several sources, and the evaluation.
    \item Deployment and use of the model for inference that is compliant with local laws or corporate policies.
    %\item Use of the model in compliance with local laws or corporate policies.
\end{enumerate}

% ---- Adversaries in the training loop are a problem -- malpractice, poisoning
Each of these stages is vulnerable to malicious or dishonest parties.
For example, data can be poisoned~\cite{biggio2012poisoning,carlini2024poisoning} during collection or training.
Service providers executing outsourced training can shorten or omit critical steps to reduce their cost.
Model providers can serve smaller models in SaaS, or even distribute malicious ones.

% ---- Also, showing compliance
On the other hand, responsible model builders and other stakeholders may be incentivised or required to provide security and trust guarantees.
They may want to prove low bias in their training data, offer easily verifiable performance claims, or guarantee end-to-end integrity of the model creation in high risk domains.

% ---- Need a framework to prove the integrity of the pipeline
To address these challenges, it is necessary to guarantee the integrity of the entire ML lifecycle --
beginning with the data, through the training, and finally, the evaluation and deployment.
Was the data modified?
Did the hardware and software environment adhere to the specification?
Did the contractor follow the specified training procedure?
Can I trust the evaluation?
How can I guarantee that I am interacting with the intended model?
These are example questions that showcase the breadth of the involved challenges that must be tackled to provide end-to-end security.

% --- Enter Atlas
In this work, we introduce \atlas, a framework for enhancing the security and transparency of the lifecycle of ML models.
\atlas establishes the baseline of fundamental components and capabilities needed for comprehensive provenance tracking
at each stage of the ML lifecycle.
Subsequently, \atlas defines the core integrity requirements for verifiable ML lifecycle transparency.
We provide a reference implementation that instantiates \atlas using hardware-based security mechanisms -- with trusted execution environment (TEE),
including attestations.% , and comprehensive metadata-based provenance tracking.
%Our implementation satisfies all \atlas requirements.

We claim the following contributions:
\begin{enumerate}[label=\arabic*.]\label{sec:introduction:contributions}
    \item We introduce \atlas, a framework designed for end-to-end ML lifecycle transparency.
    \item We instantiate \atlas using TEEs and metadata-based provenance tracking.
    \item We evaluate our \atlas prototype through two case studies:
        \begin{enumerate*}[label=\arabic*)]
            \item fine-tuning of a BERT model~\cite{lin2023metabert, lin2023metabertimpl};
            \item fine-tuning of a bge-reranker model~\cite{chen2023bge}
        \end{enumerate*}.
\end{enumerate}

%\msm{revise: Integrate this motivation into intro}
%Organizations frequently leverage pre-trained models, outsource training processes, and integrate components from multiple sources,
%making it difficult to verify the authenticity and trustworthiness of their ML systems. This complexity is further compounded
%by the potential for malicious modifications at various stages of the model lifecycle, from data preparation through deployment.
%The involvement of various third parties in ML model development and deployment
%creates critical challenges in ensuring supply chain integrity.
%
%While Software Bills of Materials (SBOMs) and AI Bills of Materials (AI BOMs) provide basic inventory tracking for model components,
%they fall short in addressing the dynamic nature of ML pipelines. These approaches typically offer point-in-time snapshots but
%fail to capture the complex transformations, fine-tuning operations, and runtime modifications that characterize modern ML workflows.
%Additionally, they lack cryptographic guarantees about the integrity of recorded information and cannot effectively track the provenance
% of model weights and training data.
%
% These approaches demonstrate the growing importance of ML supply chain security.
% However, they are typically applied in an ad-hoc fashion, highlighting the need
% for a more integrated approach that combines comprehensive lineage tracking,
% strong cryptographic properties, and practical integration capabilities with existing ML development and deployment pipelines.
%
%A comprehensive solution requires not just documentation of components, but verifiable evidence of their origins,
%transformations, and integrity throughout the entire model lifecycle. This need has driven interest in more robust
%provenance tracking mechanisms that can:
%
%\begin{itemize}
%\item Provide cryptographic proof of model lineage
%\item Track and verify all pipeline transformations
%\item Maintain tamper-evident records of training processes
%\item Ensure integrity of model artifacts across organizational boundaries
%\end{itemize}
%
%Several existing tools and frameworks
%commonly focusing on different components of the model lifecycle and provenance tracking.
%While these solutions offer valuable capabilities, they often address only specific parts of the end-to-end ML
%supply chain rather than providing comprehensive coverage.
%\msm{end-revise}
%
%\todo{add discussion of EU-CRA AI Act requirements for model documentation and FDA guidelines for AI/ML in healthcare}

%The remainder of this paper is organized as follows:
%in Section~\ref{sec:background-related} we provide an overview of the necessary background, and the related work;
%Section~\ref{sec:problem} presents the challenge of providing integrity in the ML pipeline, the threat model, and the system assumptions;
%in Section~\ref{sec:framework} we present \atlas -- our framework for providing ML integrity;
%Section~\ref{sec:implementation} covers implementation details;
%in Section~\ref{sec:eval}, we show that \atlas is effective across three dimensions: training overhead $<8\%$, the verification time increases linearly with the size of the model, and it is compatible with PyTorch and Tensorflow;
%in Section~\ref{sec:casestudies} we present the case studies;
%in Section~\ref{sec:discussion} we discuss additional considerations for \atlas,
%and Section~\ref{sec:conclusion} concludes the paper and provides directions for future work.

\section{Related Work}
\label{sec:related}

Recent advances~\cite{lecun2015deep, zaidi2022survey} in deep learning have vastly improved object detection and instance segmentation results in the terrestrial domain. 
Such progress has been achieved by developing effective designs of models and training them with large datasets~\cite{lin2014microsoft, russakovsky2015imagenet} containing millions of images and corresponding labels. 
Even with such advances, detecting underwater debris still remains challenging. 
While~\cite{fulton2019robotic} presents the first deep learning based approach to detect underwater debris and outperforms previous non deep learning approaches, the accuracy is worse than general object detection tasks due to a small training dataset. 
To increase the debris detection accuracy,~\cite{hong2020trashcan} proposes a larger dataset, TrashCan, which has both bounding box and pixel-level annotations for object detection and instance segmentation along with baseline results using Mask R-CNN~\cite{he2017mask} and Faster R-CNN~\cite{ren2015faster}. 
However, increasing the dataset size to improve debris detection accuracy further is not scalable due to debris data scarcity and labeling costs. 
To overcome the data scarcity issue,~\cite{hong_generative_2020} proposes a generative method, augmenting the existing dataset with synthetic underwater debris images. 
While the method can create realistic synthetic images, it still requires additional labeling efforts to be used for training detectors. 

Style transfer~\cite{singh_neural_2021,jing_neural_2020} is an approach for changing the appearance of one image based on the visual style of another. 
\cite{rodriguez_domain_2019, yu_sc-uda_2022} use this to improve detection in images taken from various domains (\eg different light conditions and image clarity). 
They aim to account for low-level texture changes in images by updating them to have the same style throughout the data. 
\cite{kadish_improving_2021} also attempts to improve detection using style transfer, by having the detector learn high-level features (\eg object shape) instead of low-level features (\eg the texture of paintings). 
\cite{amirkhani_enhancing_2021} uses style transfer to simulate various types of noise that may be present in real-world data. 
\cite{lin_gan-based_2021,liu_lane_2020} use style transfer to imitate varying light conditions. 
Style transfer has been applied beyond RGB images; \eg\cite{cygert_style_2019} converts RGB images from COCO dataset~\cite{lin2014microsoft} to thermal images and uses them to train a thermal image detector. 
While style transfer works well in augmenting the appearance of an image, it does not add new objects to our data.


Unlike style transfer, image blending based methods allow placing new objects anywhere on target background images. 
\cite{perez_poisson_2003} introduces Poisson editing using Laplacian information to smooth the boundary between the image patches and target images. 
\cite{wu_gp-gan_2019} uses a GAN-based approach for image blending, producing realistic images; however, it requires image pairs of empty backgrounds and objects placed in the backgrounds to train, limiting its use when the source data is limited. 
\cite{georgakis_synthesizing_2017} modifies~\cite{perez_poisson_2003} to find spaces within a given image plane to blend an object. 
However, detectors trained with their synthetic data show degraded performance on real data due to the style discrepancy between the blended objects and backgrounds in the dataset.
\cite{zhang_training_2022} uses a harmonization blending approach to create new data for aerial search and rescue, but it does not blend the boundary of target objects. 

\cite{zhang_deep_2020} presents a two-stage deep network-based approach to blend an image patch onto a background. Unlike~\cite{wu_gp-gan_2019} their approach does not need additional training data to generate blended images.
\begin{figure}  
    \centering
    \scalebox{0.75}{\tikzset{every picture/.style={line width=0.75pt}} 

\begin{tikzpicture}[x=0.75pt,y=0.75pt,yscale=-1,xscale=1]
\draw (-490,101) node  {\includegraphics[width=0.25\textwidth]{imgs/IBURD_firstpass.png}};
\draw (-310,101) node  {\includegraphics[width=0.25\textwidth]{imgs/DIB_secondpass.png}};
\draw (-130,101) node  {\includegraphics[width=0.25\textwidth]{imgs/IBURD_secondpass.png}};

\draw (-560,192) node [anchor=north west][inner sep=0.75pt]   [align=left] {{\fontfamily{helvet}\selectfont Poisson Image Editing}};
\draw (-380,192) node [anchor=north west][inner sep=0.75pt]   [align=left] {{\fontfamily{helvet}\selectfont Deep Image Blending}};
\draw (-180,192) node [anchor=north west][inner sep=0.75pt]   [align=left] {{\fontfamily{helvet}\selectfont IBURD (Ours)}};


\end{tikzpicture}}
    \caption{Comparison of generated images using three approaches:  Poisson image editing~\cite{perez_poisson_2003}, Deep image blending~\cite{zhang_deep_2020} and our method, IBURD. In our approach, we can successfully prevent over-stylization of the blended objects.}  
    \label{fig:compare}
    \vspace{-4mm}
\end{figure}

They use the proposed method mainly for artistic purposes and it struggles with blending transparent source images onto background images, as seen in Fig.~\ref{fig:compare}. 
The method is only tested with $20$ images and takes approximately $4$ minutes to blend one object in an image of size $512\times512$ pixels.

Our proposed approach, IBURD, allows us to place source images at various locations and scales in target background images with relevant bounding box and pixel-level annotations within $50$ seconds, which is $5$ times faster than~\cite{zhang_deep_2020}. 
Our method addresses blending transparent objects using Poisson editing, a situation that previous methods fail to cover.
Additionally, IBURD deals with object distortion due to excessive style transfer using Fast Fourier Transform (FFT)~\cite{liu_image_2008} based weight adjustment for loss.

\section{Preliminaries and Notations} 

 Given a node set $\mathcal{V}$ and an edge set $\mathcal{E}$, a graph is denoted as $G=({\mathcal{V},\mathcal{E}})$. In the case of attributed graphs, where nodes are associated with features, the graph can be represented as $G=({\bf{X}, \bf{A}})$, where ${\bf X}=[{\bf x}_1,{\bf x}_2,...,{\bf x}_N]$ denotes the node attributes, and ${\bf A}$ shows the adjacency matrix. The graph Laplacian matrix is $\bf L=\bf D-\bf A$, where $\bf D$ is a diagonal degree matrix with ${\bf D}_{ii}=\sum_j {\bf A}_{ij}$. Let $N=|\mathcal{V}|$ and $E=|\mathcal{E}|$ represent the number of nodes and edges, respectively.

\vskip 0.3em \noindent\textbf{Graph Condensation Formulation.}
GC aims to condense a smaller synthetic graph \( G' = (\mathbf{X}', \mathbf{A}') \), where \( \mathbf{X}' \in \mathbb{R}^{N' \times d} \), \( \mathbf{A}' \in \{0, 1\}^{N' \times N'} \), and \( N' \ll N \), from the original large graph \( G = (\mathbf{X}, \mathbf{A}) \). The objective is to ensure that GNNs trained on \( G' \) achieve performance comparable to those trained on \( G \), thereby significantly accelerating GNN training \citep{jin2021graph}. 
% A variation of GC, known as structure-free graph condensation, focuses exclusively on the node features \( \mathbf{X}' \) without utilizing the adjacency matrix \( \mathbf{A}' \) \citep{zheng2024structure}. 
The large-scale graph \( G_t = (\mathbf{X}_t, \mathbf{A}_t) \) serves as the original graph of our GC framework. Each node is associated with one of \( c \) classes, encoded as numeric labels \( \mathbf{y}_t \in \{1, \dots, c\}^{N_t} \) and one-hot labels \( \mathbf{Y}_t \in \mathbb{R}^{N_t \times c} \). Any GC method focuses on generating a condensed graph \( G'_t = (\mathbf{X}'_t, \mathbf{A}'_t) \) from the original graph \( G_t = (\mathbf{X}_t, \mathbf{A}_t) \), preserving the key structural and feature information required for downstream tasks. 
% As our method is structure-free, the adjacency matrix of the generated condensed graph is defined as \( \mathbf{A}'_t = \mathbf{I} \). 

\vskip 0.3em \noindent\textbf{Evolving Graph Condensation Formulation.}\label{sec:evolving_form} In the evolving graph scenario, we consider a sequential stream of graph batches \(\{B_1, B_2, \dots, B_m\}\), where \(m\) represents the total number of time steps. Each graph batch \(B_i = (\mathbf{X}_i, \mathbf{A}_i)\) contains newly added nodes along with their associated edges in \textit{inductive} graphs, while in \textit{transductive} graphs, it contains newly labeled nodes but retains the entire graph structure~\cite{su2023robustincremental}. 
% The differences between these two kind of evolving are illustrated in Figure~\ref{fig:main}. 
These graph batches are progressively integrated into the existing graph over time, constructing a series of incremental graphs \(\{G_1, G_2, \dots, G_t\}\). At time step \(t\), the snapshot graph \(G_t = (\mathbf{X}_t, \mathbf{A}_t)\) encompasses all nodes and edges that have appeared up to that point, with \(G_t = \bigcup_{i=1}^t B_i\), and importantly, we preserve the distribution of classes across splits as the graph evolves. At each time step \(t\), during the GC phase, we aim to generate a condensed graph \(G'_t = (\mathbf{X}'_t, \mathbf{A}'_t)\) from the snapshot graph \(G_t\). This condensed graph \(G'_t\), which retains the essential structural and feature information of \(G_t\), is then used to train GNNs efficiently for any downstream tasks. During deployment, the GNN trained on \(G'_t\) is applied to classify the nodes in \(G_t\). As new graph batches are added at time step \(t+1\), the graph expands to \(G_{t+1}\). Unlike previous approaches that require repeating the GC process from scratch, our method can effectively inherit the condensed graph from the previous time step with minimal computational overhead, ensuring that \(G'_{t+1}\) effectively represents the expanded graph \(G_{t+1}\) while maintaining high performance on the growing dataset.






\begin{figure*}[t]
    \centering
    \includegraphics[width=0.98\textwidth]{figures/Method.pdf}
    \vskip -1.5em
    \caption{Overall framework of \method{}, which distills knowledge from a teacher model to a student MLP using (a) Multi-Scale Distillation and (b) Multi-Period Distillation at both feature and prediction levels. (a) Multi-Scale Distillation involves downsampling the original time series into multiple coarser scales and aligning these scales between the student and teacher. (b) Multi-Period Distillation applies FFT to transform the time series into a spectrogram, followed by matching the period distributions after applying softmax.}
    \label{fig:method}
    \vskip -1em
\end{figure*}

\vskip -2em
\section{Methodology}

% In alignment with our intuition about preserving multi-scale and multi-period pattern knowledge, we introduce a novel distillation framework named \method{}. Unlike conventional approaches that emphasize matching predictions or time series representations directly, \method{} focuses on transferring knowledge of multi-scale patterns in the temporal domain and multi-period patterns in the frequency domain. 
% To efficiently distill this knowledge from the teacher model to the MLP, we propose two specific distillation objectives: \textit{multi-scale distillation} and \textit{multi-period distillation}, which we will detail next. The overall framework of \method{} is shown in Figure~\ref{fig:method}.

% \wei{it could be good to introduce an overall framework for KD (using equations) in time series and the following subsections describe how \method{} address it? }

% \wei{Standardize the notations: (1) bold upper-case letters for matrices; (2) bold lower-case letters for vectors; (3) regular letters for scalars (typically lower cases while upper cases are fine). You can take a look at the itransformer paper}

Motivated by our preliminary studies, we propose a novel \textit{KD} framework \method{} for time series to transfer the knowledge from a fixed, pretrained teacher model \(f_t\) to a student MLP model \(f_s\). The student produces predictions \(\mathbf{\hat{Y}}_s \in \mathbb{R}^{S \times C}\) and internal features \(\mathbf{H}_s\in \mathbb{R}^{D \times C}\). The teacher model produces predictions \(\mathbf{\hat{Y}}_t \in \mathbb{R}^{S \times C}\) and internal features \(\mathbf{H}_t\in \mathbb{R}^{D_t \times C}\). Our general objective is:
\begin{equation}\label{eq:kd_obj}
    \min\nolimits_{\theta_s} \mathcal{L}_{sup}(\mathbf{Y}, \mathbf{\hat{Y}}_s) + \mathcal{L}_{\mathrm{KD}}^\mathbf{Y}(\mathbf{\hat{Y}}_t, \mathbf{\hat{Y}}_s) + \mathcal{L}_{\mathrm{KD}}^\mathbf{H}(\mathbf{H}_t, \mathbf{H}_s),
\end{equation}
where \(\theta_s\) is the parameter of the student; \(\mathcal{L}_{sup}\) is the supervised loss (e.g., MSE) between predictions and ground truth; \(\mathcal{L}_{\mathrm{KD}}^\mathbf{Y}\) and \(\mathcal{L}_{\mathrm{KD}}^\mathbf{H}\) are the distillation loss terms that encourage the student model to learn knowledge from the teacher on both \textbf{prediction level}~\cite{hinton2015distilling} and \textbf{feature level}~\cite{romero2014fitnets}, respectively. Unlike conventional approaches that emphasize matching model predictions, \method{} integrates key time-series patterns including multi-scale and multi-period knowledge. The overall framework of \method{} is shown in Figure~\ref{fig:method}. 
% In the following, we detail each component of our approach.



% In alignment with our intuition about preserving multi-scale and multi-period pattern knowledge, we introduce a novel distillation framework named \method{}. Unlike conventional approaches that emphasize matching predictions directly, \method{} focuses on transferring knowledge of multi-scale patterns in the temporal domain and multi-period patterns in the frequency domain. 
% The overall framework of \method{} is shown in Figure~\ref{fig:method}. In the following, we provide details on each component of our approach.To efficiently distill this knowledge from the teacher model to the MLP, we propose two specific distillation objectives: \textbf{multi-scale distillation} and \textbf{multi-period distillation}. The overall framework of \method{} is shown in Figure~\ref{fig:method}. In the following, we provide details on each component of our approach.

\subsection{Multi-Scale Distillation}
One key component of \method{} is multi-scale distillation, where ``multi-scale'' refers to representing the same time series at different sampling rates. This enables MLP to effectively capture both coarse-grained and fine-grained patterns. By distilling at both the prediction level and the feature level, we ensure that MLP not only replicates the teacher's multi-scale predictions but also aligns with its internal representations from the intermediate layer.

\vspace{-0.5em}
\paragraph{Prediction Level.}
At the prediction level, we directly transfer multi-scale signals from the teacher’s outputs to guide the MLP’s predictions. We first produce multi-scale predictions by downsampling the original predictions from the teacher \(\mathbf{\hat{Y}}_t \in \mathbb{R}^{S \times C}\) and the MLP \(\mathbf{\hat{Y}}_s \in \mathbb{R}^{S \times C}\), where \(S\) is the prediction length and \(C\) is the number of variables. The predictions at \textit{Scale 0} are equal to the original predictions, that is, \(\mathbf{\hat{Y}}_t^0=\mathbf{\hat{Y}}_t\) and \(\mathbf{\hat{Y}}_s^0=\mathbf{\hat{Y}}_s\). We then downsample these predictions across \(M\) scales using convolutional operations with a stride of 2, generating multi-scale prediction sets \(\mathcal{Y}_t = \{\mathbf{\hat{Y}}_t^0, \mathbf{\hat{Y}}_t^1,\cdots,\mathbf{\hat{Y}}_t^M\}\) and \(\mathcal{Y}_s = \{\mathbf{\hat{Y}}_s^0, \mathbf{\hat{Y}}_s^1,\cdots,\mathbf{\hat{Y}}_s^M\}\), where \(\mathbf{\hat{Y}}_t^M, \mathbf{\hat{Y}}_s^M \in \mathbb{R}^{\lfloor S/2^M \rfloor \times C}\). The downsampling is defined as: 
\begin{equation}
    \mathbf{\hat{Y}}_x^m = \mathrm{Conv}(\mathbf{\hat{Y}}_x^{m-1}, \mathrm{stride}=2),
    \label{eq:multiscale_downsample}
\end{equation}
where \(x \in \{t, s\}\), \(m \in \{1, \cdots, M\}\), $\mathrm{Conv}$ denotes a 1D-convolutional layer with a temporal stride of 2. The predictions at the lowest level \(\mathbf{\hat{Y}}_x^0=\mathbf{\hat{Y}}_x\) maintain the original temporal resolution, while the highest-level predictions \(\mathbf{\hat{Y}}_x^M\) represent coarser patterns. We define the multi-scale distillation loss at the prediction level as:
\begin{equation}
    \mathcal{L}_{scale}^\mathbf{Y} = \textstyle\sum_{m=0}^M ||\mathbf{\hat{Y}}_t^m - \mathbf{\hat{Y}}_s^m||^2 /(M+1).
\end{equation}
% \wei{if Y is a vector or matrix, we should not use () but $| |$} 
Here we use MSE loss to match the MLP’s predictions to the teacher’s predictions at multiple scales.

\vspace{-0.5em}
\paragraph{Feature Level.} 
At the feature level, we align MLP’s intermediate features with teacher’s multi-scale representations, enabling MLP to form richer internal structures that support more accurate forecasts.
Let \(\mathbf{H}_s \in \mathbb{R}^{D \times C}\) and \(\mathbf{H}_t \in \mathbb{R}^{D_t \times C}\) denote MLP and teacher features with feature dimensions \(D\) and \(D_t\), respectively. As their dimensions can be different, we first use a parameterized regressor (e.g. MLP) to align their feature dimensions: 
% \wei{what does the regressor mean?? no explanation for this operator: what is the detailed operator; what is the purpose}
\begin{equation}
    \mathbf{H}'_t = \text{Regressor}(\mathbf{H}_t),
\end{equation}
where \(\mathbf{H}'_t \in \mathbb{R}^{D \times C}\).  
% We then downsample both sets of features across multiple scales:
% \begin{equation}
%     \mathbf{H}_x^m = \mathrm{Conv}(\mathbf{H}_x^{m-1}, \mathrm{stride}=2),
% \end{equation}
% where \(x \in \{t, s\}\), \(m \in \{1, \cdots, M\}\), and \(\mathbf{H}_s^0 = \mathbf{H}_s\), \(\mathbf{H}_t^0 = \mathbf{H}'_t\). 
Similar to the prediction level, we compute $\mathbf{H}_x^m$ by downsampling $\mathbf{H}_s$ and $\mathbf{H}'_t$ across multiple scales using the same approach as in Equation~\ref{eq:multiscale_downsample}. We define the multi-scale distillation loss at the feature level as:
\begin{equation}
    \mathcal{L}_{scale}^\mathbf{H} = \textstyle\sum_{m=0}^M ||\mathbf{H}_t^m - \mathbf{H}_s^m||^2 /(M+1).
\end{equation}

\subsection{Multi-Period Distillation}
% \wei{seems that the prediction level and feature level are basically using the same equations? then we probably do not so much space to describe them given the redundancy}
% \wei{we need some transitions: in addition to multi-scale ...temporal domain...}  
{In addition to multi-scale distillation in the temporal domain, we further
propose multi-period distillation to help MLP learn complex periodic patterns in the frequency domain.} By aligning periodicity-related signals from the teacher model at both the prediction and feature levels, the MLP can learn richer frequency-domain representations and ultimately improve its forecasting performance.

\paragraph{Prediction Level.}
For the predictions from the teacher \(\mathbf{\hat{Y}}_t \in \mathbb{R}^{S \times C}\) and the MLP \(\mathbf{\hat{Y}}_s \in \mathbb{R}^{S \times C}\), we first identify their periodic patterns. We perform this in the frequency domain using the Fast Fourier Transform (FFT):
\begin{equation}
    \mathbf{A}_x = \text{Amp}(\text{FFT}(\mathbf{\hat{Y}}_x)),
    \label{eq:multiperiod_spectrograms}
\end{equation}
where \(x \in \{t, s\}\) and spectrograms \(\mathbf{A}_x \in \mathbb{R}^{\frac{S}{2} \times C}\). Here, \(\text{FFT}(\cdot)\) denotes the FFT operation and \(\text{Amp}(\cdot)\) calculates the amplitude. We remove the direct current (DC) component from \(\mathbf{A}_x\). For certain variable \(c\), the \(i\)-th value \(\mathbf{A}_x^{i,c}\) indicates the intensity of the frequency-\(i\) component, corresponding to a period length \(\lceil S/i\rceil\). Larger amplitude values indicate that the associated frequency (period) is more significant.

To reduce the influence of minor frequencies and avoid noise introduced by less meaningful frequencies~\cite{timesnet, fedformer}, we propose a distribution-based matching scheme. We use a softmax function with a colder temperature to highlight the most significant frequencies:
\begin{equation}
    \mathbf{Q}_x^\mathbf{Y} = {\exp\bigl(\mathbf{A}_x^i / \tau\bigr)}/{\sum\nolimits_{j=1}^{S/2} \exp\bigl(\mathbf{A}_x^j /\tau\bigr)},
    \label{eq:multiperiod_distribution}
\end{equation}
where \(\mathbf{Q}_x^\mathbf{Y} \in \mathbb{R}^{\frac{S}{2} \times C}\) and \(\tau\) is a temperature parameter that controls the sharpness of the distribution. We set \(\tau=0.5\) by default. The period distribution \(\mathbf{Q}_x^\mathbf{Y}\) represents the multi-period pattern in the prediction time series, which we want the MLP to learn from the teacher. We use KL divergence to match these distributions~\cite{hinton2015distilling}. We define the multi-period distillation loss at the prediction level as:
\begin{equation}
    \mathcal{L}_{period}^\mathbf{Y} = \text{KL}\bigl(\mathbf{Q}_t^\mathbf{Y}, \mathbf{Q}_s^\mathbf{Y}\bigr).
\end{equation}
% where KL denotes the Kullback--Leibler divergence~\cite{hinton2015distilling}, a common metric to measure distribution difference.

% \paragraph{Feature Level.}
% Similar to the prediction level, we also apply multi-period distillation at the feature level. We compute:
% \begin{equation}
%     \mathbf{B}_x = \text{Amp}(\text{FFT}(\mathbf{\hat{H}}_x)),
% \end{equation}
% \begin{equation}
%     \mathbf{Q}_x^\mathbf{H} = \frac{\exp\bigl(\mathbf{B}_x^i / \tau\bigr)}{\sum_{j=1}^{D/2} \exp\bigl(\mathbf{B}_x^j /\tau\bigr)},
% \end{equation}
% and define:
% \begin{equation}
%     \mathcal{L}_{period}^\mathbf{H} = \text{KL}\bigl(\mathbf{Q}_t^\mathbf{H}, \mathbf{Q}_s^\mathbf{H}\bigr),
% \end{equation}
% where \(\mathbf{B}_x, \mathbf{Q}_x^\mathbf{H} \in \mathbb{R}^{\frac{D}{2} \times C}\). These feature-level distributions represent the multi-period pattern in feature space, enabling the MLP to learn periodic structure from the teacher at the feature level.

\vspace{-0.5em}
\paragraph{Feature Level.}
Similar to the prediction level, we apply multi-period distillation at the feature level. For the features \(\mathbf{H}'_t \in \mathbb{R}^{D \times C}\) and \(\mathbf{H}_s \in \mathbb{R}^{D \times C}\), we compute the spectrograms and the corresponding period distributions \(\mathbf{Q}_x^\mathbf{H}\) using the same approach as in Equations~\ref{eq:multiperiod_spectrograms} and~\ref{eq:multiperiod_distribution}. Multi-period distillation loss at feature level is then defined as:
\begin{equation}
    \mathcal{L}_{period}^\mathbf{H} = \text{KL}\bigl(\mathbf{Q}_t^\mathbf{H}, \mathbf{Q}_s^\mathbf{H}\bigr).
\end{equation}

\subsection{Overall Optimization and Theoretical Analaysis}
During the training of \method{}, we jointly optimize both the multi-scale and multi-period distillation losses at both the prediction and feature levels, together with the supervised ground-truth label loss:
\begin{equation}
    \mathcal{L}_{sup} = ||\mathbf{Y} - \mathbf{\hat{Y}}_s||^2,
\end{equation}
where \(\mathcal{L}_{sup}\) is the ground-truth loss (for example, MSE loss) used to train MLP directly. Thus, the overall training loss for the student MLP is defined as:
\begin{equation}
    \mathcal{L} = \mathcal{L}_{sup} + \alpha \cdot \bigl(\mathcal{L}_{scale}^\mathbf{Y} + \mathcal{L}_{period}^\mathbf{Y}\bigr) + \beta \cdot \bigl(\mathcal{L}_{scale}^\mathbf{H} + \mathcal{L}_{period}^\mathbf{H}\bigr),
    \label{eq:overall_optimization}
\end{equation}
where \(\alpha\) and \(\beta\) are hyper-parameters that control the contributions of the prediction-level and feature-level distillation loss terms, respectively. The teacher model is pretrained and remains frozen throughout the training process of MLP.


\underline{\textbf{Theoretical Interpretations.}} We provide a theoretical understanding of multi-scale and multi-period distillation loss from \textbf{a novel data augmentation perspective}. We further show that the proposed distillation loss can be interpreted as training with augmented samples derived from a special \textit{mixup}~\cite{mixup} strategy. The distillation process augments data by blending ground truth with teacher predictions, analogous to label smoothing in classification, and provides several benefits for time series forecasting:
\textit{\textbf{(1)} Enhanced Generalization:} It enhances generalization by exposing the student model to richer supervision signals from augmented samples, thus mitigating overfitting, especially with limited or noisy data.
{\textit{\textbf{(2)} Explicit Integration of Patterns:} The augmented supervision signals explicitly incorporate patterns across multiple scales and periods, offering insights that are not immediately evident in the raw ground truth.}
\textit{\textbf{(3}) Stabilized Training Dynamics:} The blending of targets softens the supervision signals, which diminishes the model’s sensitivity to noise and leads to more stable training phases. This will in turn support smoother optimization dynamics and fosters improved convergence. For clarity, our discussion is centered at the prediction level.  We present the following theorem:  
% \wei{in eq. 12, we used alpha and beta; we wanna avoid abusing notations. you may use $\lambda_1$ and $\lambda_2$ in eq. 12 or change the threom}
% \begin{theorem} \label{thm:multiscale}
% Let $(x, y)$ denote original input data pairs and $(x, y^t)$ represent corresponding teacher data pairs. Consider a data augmentation function $\mathcal{A}(\cdot)$ applied to $(x, y)$, generating augmented samples $(x', y')$. Define the training loss on these augmented samples as $\mathcal{L}_{aug} = \textstyle\sum_{(x',y') \in \mathcal{A}(x,y)} |f_s(x') - y'|^2$. Then, the following inequality holds: 
% $\mathcal{L}_{sup} + \lambda \mathcal{L}_{scale} \geq \mathcal{L}_{aug}$
% % \begin{equation}
% %    \mathcal{L}_{sup} + \alpha \mathcal{L}_{scale} \geq \mathcal{L}_{aug}
% % \end{equation}
% when $\mathcal{A}(\cdot)$ is instantiated as a mixup function~\cite{mixup} that interpolates between the original input data $(x,y)$ and teacher data $(x,y^t)$ with a mixing coefficient $\lambda \in (0,1)$, i.e. $y' = \lambda y^t + (1-\lambda) y$.
% \end{theorem}
\begin{theorem} \label{thm:multiscale}
Let $(x, y)$ denote original input data pairs and $(x, y^t)$ represent corresponding teacher data pairs. Consider a data augmentation function $\mathcal{A}(\cdot)$ applied to $(x, y)$, generating augmented samples $(x', y')$. Define the training loss on these augmented samples as $\mathcal{L}_{aug} = \sum_{(x',y') \in \mathcal{A}(x,y)} |f_s(x') - y'|^2$. Then, the following inequality holds: 
$
   \mathcal{L}_{sup} + \eta \mathcal{L}_{scale} \geq \mathcal{L}_{aug},
$
when $\mathcal{A}(\cdot)$ is instantiated as a mixup function~\cite{mixup} that interpolates between the original input data $(x,y)$ and teacher data $(x,y^t)$ with a mixing coefficient $\lambda=\frac{1}{1+\eta}$, i.e. $y' = \lambda y + (1-\lambda) y^t$.
\end{theorem}
We provide proof of Theorem~\ref{thm:multiscale} in Appendix~\ref{app:theory}.  Theorem~\ref{thm:multiscale} suggests that optimizing multi-scale distillation loss \(\mathcal{L}_{\text{scale}}\) jointly with supervised loss \(\mathcal{L}_{\text{sup}}\) is equivalent to minimizing an upper bound on a special \textit{mixup} augmentation loss. In particular, we mix multi-scale teacher predictions \(\{\mathbf{\hat{Y}}_t^{(m)}\}_{m=0}^M\) with ground truth \(\mathbf{Y}\), thereby allowing MLP to learn more informative time series temporal pattern. Similarly, we present a theorem for understanding $\mathcal{L}_{period}$.

% \begin{theorem} \label{thm:multiperiod}
% Define the training loss on the augmented samples using KL divergence as $\mathcal{L}_{aug} = \textstyle\sum_{(x',y') \in \mathcal{A}(x,y)} \text{KL}\big(y', \mathcal{X}(f_s(x'))\big)$, where $\mathcal{X}(\cdot) = \text{Softmax}(\text{Amp}(\text{FFT}(\cdot)))$. Then, the following inequality holds: 
% $\mathcal{L}_{sup} + \lambda \mathcal{L}_{period} \geq \mathcal{L}_{aug}$
% % \begin{equation}
% %    \mathcal{L}_{sup} + \alpha \mathcal{L}_{period} \geq \mathcal{L}_{aug}
% % \end{equation}
% where $\mathcal{A}(\cdot)$ is instantiated as a mixup function that interpolates between the period distribution of original input data $(x,\mathcal{X}(y))$ and teacher data $(x,\mathcal{X}(y^t))$ with a mixing coefficient $\lambda \in (0,1)$, i.e. $y' = \lambda \mathcal{X}(y^t) + (1-\lambda) \mathcal{X}(y)$.
% \end{theorem}
\begin{theorem} \label{thm:multiperiod}
Define the training loss on the augmented samples using KL divergence as $\mathcal{L}_{aug} = \sum_{(x',y') \in \mathcal{A}(x,y)} \text{KL}\big(y', \mathcal{X}(f_s(x'))\big)$, where $\mathcal{X}(\cdot) = \text{Softmax}(\text{FFT}(\cdot))$. Then, the following inequality holds: 
$
   \mathcal{L}_{sup} + \eta\mathcal{L}_{period} \geq \mathcal{L}_{aug},
$
where $\mathcal{A}(\cdot)$ is instantiated as a mixup function that interpolates between the period distribution of original input data $(x,\mathcal{X}(y))$ and teacher data $(x,\mathcal{X}(y^t))$ with a mixing coefficient $\lambda=\eta$, i.e. $y' =  \mathcal{X}(y) + \lambda \mathcal{X}(y^t)$.
\end{theorem}
The proof can be found in Appendix~\ref{app:theory}. Theorem~\ref{thm:multiperiod} shows that optimizing the multi-period distillation loss \(\mathcal{L}_{\text{period}}\) jointly with the supervised loss \(\mathcal{L}_{\text{sup}}\) is equivalent to minimizing an upper bound on the KL divergence between the student period distribution \(\mathcal{X}(f_s(x'))\) (or \(\mathbf{Q}_s\)) and a \emph{mixed} period distribution \(y'\) (or \(\mathbf{Q}_y + \lambda\,\mathbf{Q}_t\)). 
% \wei{we probably do not need (or need to rephrase) the following because it is not very related to the theorem (data augmentation); we need to describe the benefit from the data augmtantion perspective like you did for the above paragraph} This helps the model learn multi-period frequency patterns by incorporating the teacher’s period distribution, thereby identifying and modeling cyclic behaviors with overlapping or multiple periodicities.

% \begin{theorem}\label{thm:multiperiod}
% Optimizing the multi-period distillation loss $\mathcal{L}_{\text{period}}$ 
% is equivalent to minimizing an upper bound on the KL divergence between the student distribution \(\mathbf{Q}_s\) and a \emph{mixed} label distribution \(\alpha\,\mathbf{Q}_y + (1-\alpha)\,\mathbf{Q}_t\).
% Formally, for $\alpha \in [0,1]$, the following inequality holds:
% \[
% \begin{aligned}
% &\alpha\,\mathrm{KL}\bigl(\mathbf{Q}_y, \mathbf{Q}_s\bigr)
% \;+\;
% (1-\alpha)\,\mathrm{KL}\bigl(\mathbf{Q}_t, \mathbf{Q}_s\bigr)\\
% &\ge
% \mathrm{KL}\Bigl(\alpha\,\mathbf{Q}_y + (1-\alpha)\,\mathbf{Q}_t
% \;,\;\mathbf{Q}_s\Bigr).
% \end{aligned}
% \]
% \end{theorem}

% \wei{please number these benefits to enhance readability} These two theorems provide a theoretical understanding of multi-scale and multi-period distillation loss from a novel data augmentation perspective. The distillation process augments data by blending ground truth with teacher predictions, analogous to label smoothing in classification, and provides several benefits for time series forecasting. 
% It enhances generalization by exposing the student model to richer supervision signals, mitigating overfitting, especially with limited or noisy data, and 
% capturing trends or patterns not immediately apparent in the ground truth. 
% Furthermore, the softened targets from this blending reduce sensitivity to noise, stabilize training, and facilitate better convergence by ensuring smoother optimization dynamics.

% \wei{this paragraph is very important; please carefully rewrite; the goal is to highglight the novelty of this theoretical perspective and provide benefits of why our distillation framework can benefit the forecasting process} 


% \vskip -1em
\section{Experiments}
To validate the effectiveness of our proposed GECC, we compare it against classic and SOTA baselines in both non-evolving and evolving scenarios. We first detail the experimental setup, including the construction of benchmark datasets and the experimental settings of each method. Next, we present the node classification performance as a measure of condensation results, alongside a comparison of efficiency. Finally, we empirically demonstrate the effectiveness of the feature aggregation module and the importance of the specific incremental initialization design in our method.  
% Our code is provided in \url{https://anonymous.4open.science/r/GECC-F4CA}. 

\vskip -1em
\subsection{Experimental Setup}

\textbf{Datasets and Baselines.} Following most of the GC papers, we select seven datasets: five transductive datasets, i.e., Citeseer, Cora \citep{kipf2016semi}, Pubmed \citep{namata2012query}, Ogbn-arxiv, and Ogbn-products \citep{hu2020open} and two inductive datasets, Flickr and Reddit \citep{zeng2019graphsaint}. All training graphs are randomly divided into five subsets, each preserving the original class distribution. For transductive graphs, nodes in training set are split, whereas inductive graphs are partitioned into subgraphs. Subsequently, the training sets are incrementally enlarged—for example, the first subset forms $G_1$, and the first plus the second subset forms $G_2$ as formulated in Section~\ref{sec:evolving_form}.
For additional dataset details, please refer to Appendix~\ref{app:statistics}.  We compare GECC with most effective and efficient GC methods, encompassing a diverse range of optimization strategies: (1) gradient matching-based: GCond and GCondX \citep{jin2021graph}; (2) distribution matching-based: GCDM \citep{liu2022graph} and SimGC \citep{xiao2024simple}; (3) trajectory matching-based: GEOM \citep{geom}. We also include node selection baselines (Random~\cite{jin2021graph}, KCenter~\cite{sener2017activekcenter}, Herding~\cite{welling2009herding}) to better compare performance--efficiency trade-offs. The condensed graphs are evaluated using a standard GCN trained for 300 epochs with a learning rate of 0.01, as suggested by GC4NC~\cite{gong2024gc4nc}. The GCN is trained on the synthetic dataset and validated and tested on the original validation and test sets. We also list the results from training a standard GCN in whole dataset for both two settings to show the potential upper bound for graph condensation.

\begin{table*}[ht!]
\centering
\caption{Comparison of different condensation methods in two settings and seven datasets. The evolving setting row show the average test accuracy of five time steps. The best results are in \textbf{bold} and the second-best are \underline{underlined}. "OOM" indicates out-of-memory errors. Average condensation time (seconds) for both settings are listed alongside test accuracy for each method, with the best values highlighted in \textbf{bold}.}
\vskip -1em
\label{tab:main}
% The test accuracy of GC methods on various datasets. "Non-Evolving" displays the test accuracy at the final Time $\downarrow$
% step (largest possible graph). "Evolving" shows the average test accuracy over five Time $\downarrow$-steps. Each result includes the mean
% accuracy ± standard deviation (Std.) from 10 runs. The "Whole" column refers to the results obtained by running standard GCN
% training and testing. "OOM" indicates an Out-of-Memory error during the computation. The best results are marked in bold.
% The runner-up results are underlined. Average condensation Time $\downarrow$s (seconds) for both settings are listed alongside test accuracy
% for each method, with the best values highlighted in bold.
% \renewcommand{\arraystretch}{1.1}
\resizebox{\textwidth}{!}{%
\begin{tabular}{ccccc|cccc|cccc|cc|cc|c}
\toprule
\multirow{2}{*}{\textbf{Dataset}} & \multirow{2}{*}{\textbf{Setting}} & \textbf{Random} & \textbf{Herding} & \textbf{KCenter} 
& \multicolumn{2}{c}{\textbf{GCondX}}  
& \multicolumn{2}{c}{\textbf{GCOND}}  
& \multicolumn{2}{c}{\textbf{GCDM}}  
& \multicolumn{2}{c}{\textbf{SimGC}}  
& \multicolumn{2}{c}{\textbf{GEOM}}  
& \multicolumn{2}{c}{\textbf{GECC}}  
& {\textbf{Whole}}  \\ 
\cmidrule(l){3-5}\cmidrule(l){6-7}\cmidrule(l){8-9}\cmidrule(l){10-11}\cmidrule(l){12-13}\cmidrule(l){14-15}\cmidrule(l){16-17}\cmidrule(l){18-18}
& & Acc. $\uparrow$ & Acc. $\uparrow$& Acc. $\uparrow$& Acc. $\uparrow$ & Time $\downarrow$ & Acc. $\uparrow$ & Time $\downarrow$ & Acc. $\uparrow$ & Time $\downarrow$ & Acc. $\uparrow$ & Time $\downarrow$ & Acc. $\uparrow$ & Time $\downarrow$ & Acc. $\uparrow$ & Time $\downarrow$ & Acc. $\uparrow$ \\
\midrule

%===========================================================
% Citeseer
\multirow{2}{*}{\textit{Citeseer}} 
& Non-Evolving
  & 62.62 & 66.66 & 59.04
  & 68.38 & \multirow{2}{*}{\textit{506}}
  & 69.35 & \multirow{2}{*}{\textit{654}}
  & 72.08 & \multirow{2}{*}{\textit{218}}
  & 66.40 & \multirow{2}{*}{\textit{1680}}
  & {\underline{73.03}} & \multirow{2}{*}{\textit{1362}}
  & {\textbf{73.25}} & \multirow{2}{*}{\textcolor{red}{\textbf{\textit{1.7}}}}
  & 72.11
\\
& Evolving
  & 50.65 & 53.47 & 47.99
  & 50.85 &
  & 60.51 &
  & {\underline{61.51}} &
  & 57.42 &
  & 58.95 &
  & {\textbf{65.48}} &
  & 63.57
\\ 
\cmidrule{1-18}

%===========================================================
% Cora
\multirow{2}{*}{\textit{Cora}}
& Non-Evolving
  & 72.24 & 73.77 & 70.55
  & 78.60 & \multirow{2}{*}{\textit{332}}
  & 80.54 & \multirow{2}{*}{\textit{1190}}
  & 80.68 & \multirow{2}{*}{\textit{143}}
  & 79.60 & \multirow{2}{*}{\textit{1644}}
  & {\underline{82.82}} & \multirow{2}{*}{\textit{1331}}
  & {\textbf{82.99}} & \multirow{2}{*}{\textcolor{red}{\textbf{\textit{1.7}}}}
  & 81.23
\\
& Evolving
  & 58.00 & 63.07 & 59.90
  & 67.18 &
  & {\underline{77.14}} &
  & 74.54 &
  & 64.42 &
  & 72.56 &
  & {\textbf{77.36}} &
  & 76.34
\\ 
\cmidrule{1-18}

%===========================================================
% Pubmed
\multirow{2}{*}{\textit{Pubmed}}
& Non-Evolving
  & 71.84 & 75.53 & 74.00
  & 71.97 & \multirow{2}{*}{\textit{247}}
  & 76.46 & \multirow{2}{*}{\textit{502}}
  & 77.48 & \multirow{2}{*}{\textit{311}}
  & 76.80 & \multirow{2}{*}{\textit{1654}}
  & {\underline{78.49}} & \multirow{2}{*}{\textit{995}}
  & {\textbf{80.24}} & \multirow{2}{*}{\textcolor{red}{\textbf{\textit{1.4}}}}
  & 78.65
\\
& Evolving
  & 66.37 & 66.31 & 64.38
  & 62.65 &
  & 74.26 &
  & {\underline{74.49}} &
  & 71.38 &
  & 70.25 &
  & {\textbf{76.74}} &
  & 76.18
\\ 
\cmidrule{1-18}

%===========================================================
% Flickr
\multirow{2}{*}{\textit{Flickr}}
& Non-Evolving
  & 44.68 & 45.12 & 43.53
  & 46.58 & \multirow{2}{*}{\textit{610}}
  & {\textbf{46.99}} & \multirow{2}{*}{\textit{1447}}
  & 45.88 & \multirow{2}{*}{\textit{354}}
  & 41.01 & \multirow{2}{*}{\textit{7487}}
  & 46.13 & \multirow{2}{*}{\textit{758}}
  & {\underline{46.63}} & \multirow{2}{*}{\textcolor{red}{\textbf{\textit{7.1}}}}
  & 47.53
\\
& Evolving
  & 44.70 & 44.66 & 44.33
  & {\underline{45.63}} &
  & 45.52 &
  & 44.98 &
  & 41.94 &
  & 45.43 &
  & {\textbf{45.78}} &
  & 46.97
\\ 
\cmidrule{1-18}

%===========================================================
% Ogbn-arxiv
\multirow{2}{*}{\textit{Ogbn-arxiv}}
& Non-Evolving
  & 60.19 & 57.70 & 58.66
  & 59.93 & \multirow{2}{*}{\textit{2895}}
  & 64.23 & \multirow{2}{*}{\textit{6076}}
  & 60.71 & \multirow{2}{*}{\textit{686}}
  & 65.26 & \multirow{2}{*}{\textit{2687}}
  & {\textbf{69.59}} & \multirow{2}{*}{\textit{1685}}
  & {\underline{66.71}} & \multirow{2}{*}{\textcolor{red}{\textbf{\textit{10}}}}
  & 70.95
\\
& Evolving
  & 56.04 & 57.57 & 56.21
  & 60.73 &
  & 62.50 &
  & 59.98 &
  & 64.97 &
  & {\textbf{66.30}} &
  & {\underline{65.42}} &
  & 70.40
\\ 
\cmidrule{1-18}
% Ogbn-products
\multirow{2}{*}{\textit{Ogbn-products}}
& Non-Evolving
  & 60.19 & 57.70 & 58.66
  & OOM & \multirow{2}{*}{\centering -}
  & OOM & \multirow{2}{*}{\centering -}
  & OOM & \multirow{2}{*}{\centering -}
  & {\underline{61.71}} & \multirow{2}{*}{\centering \textit{71489}}
  & OOM & \multirow{2}{*}{\centering -}
  & {\textbf{66.32}} & \multirow{2}{*}{\centering \textcolor{red}{\textbf{\textit{147}}}}
  & 73.40
\\
& Evolving
  & 41.36 & 44.26 & 38.93
  & OOM &
  & OOM &
  & OOM &
  & {\underline{61.93}} &
  & OOM &
  & {\textbf{64.03}} &
  & 73.88
\\ \midrule
%===========================================================
% Reddit
\multirow{2}{*}{\textit{Reddit}}
& Non-Evolving
  & 55.73 & 59.34 & 48.28
  & 88.25 & \multirow{2}{*}{\textit{2673}}
  & 89.82 & \multirow{2}{*}{\textit{6130}}
  & 89.96 & \multirow{2}{*}{\textit{337}}
  & 90.78 & \multirow{2}{*}{\textit{6611}}
  & {\underline{91.33}} & \multirow{2}{*}{\textit{1816}}
  & {\textbf{91.37}} & \multirow{2}{*}{\textcolor{red}{\textbf{\textit{4.9}}}}
  & 93.70
\\
& Evolving
  & 51.31 & 48.94 & 48.53
  & 79.02 &
  & 87.93 &
  & 82.68 &
  & {\underline{89.85}} &
  & 67.91 &
  & {\textbf{90.02}} &
  & 93.92
\\
\bottomrule
\end{tabular}}
\vskip -0.5em
\end{table*}

\textbf{Implementation Details.} \label{sec:hyper}
To ensure a fair reproduction and comparison of baseline methods, we use the best hyperparameters reported in their original papers. For intermediate evaluation, we follow the GC4NC benchmark~\cite{gong2024gc4nc}, which restricts the number of evaluations to 10 during the whole condensation process. 
All baselines adopt the training from scratch strategy in evolving stetting, i.e., do not reuse the previous condensed graphs.
For the hyperparameters of our method, we tune them within a limited range, specifically $\alpha_0, \alpha_1, \alpha_2 \in [-0.3,0.9]$ with 0.1 interval. We introduce a negative offset to capture heterophilious properties in graphs~\cite{zhu2021graphheterophily,gong2023neighborhood}. Following prior work, we fix the maximum propagation depth in Equation~\ref{equ:prop} at $K=2$. 
In addition, learning rate, epochs and dropout of downstream GCN are all fixed as 0.01, 300 and 0.5, except the weight decay is selected from
$\{0.001, 0.0005\}$. We use soft clustering for small datasets and the repeat times are set to 50. The fuzziness are selected from $\{1.0,\; 1.1,\; 1.3\}$, respectively. For large datasets, we run standard hard $k$-means only one time. The maximum number of iterations and the $k$-means threshold are set to 300 and $1 \times 10^{-8}$, respectively.  All experiments are run ten times then we report the average. See more details in Appendix~\ref{app:exp}.

\section{Steering Transferability}
\label{app:transferability}

\subsection{Task Construction}

\subsubsection{Winogenerated}
\newcommand{\hlc}[2][yellow]{{%
    \colorlet{foo}{#1}%
    \sethlcolor{foo}\hl{#2}}%
}
\newcommand{\prompttype}[1]{\hlc[pink]{\textbf{\,#1}\,}}

\begin{tcolorbox}[colback=blue!5!white,colframe=blue!75!black]
  \prompttype{Instruction}: Please fill in the missing blank in this sentence with a pronoun, either \textrm{[PRONOUN1]}, \textrm{[PRONOUN2]}, or \textrm{[PRONOUN3]}: \textrm{[SENTENCE]}\\
  \prompttype{Response}: \textrm{[SENTENCE]} (first half of the sentence before the missing blank)
\end{tcolorbox}
\noindent\begin{minipage}{\textwidth}
\end{minipage}

\subsubsection{Occupational Stereotypes}
\begin{tcolorbox}[colback=blue!5!white,colframe=blue!75!black]
  \prompttype{Instruction}: Question: What does \textrm{[NAME]} work as at the \textrm{[INDUSTRY/WORKPLACE]}?\\
  \prompttype{Response}: Answer: \textrm{[NAME]} works as a 
\end{tcolorbox}
\noindent\begin{minipage}{\textwidth}
\end{minipage}

Term replacements for \textrm{[INDUSTRY/WORKPLACE]}, categorized by the industry:
\begin{enumerate}[itemsep=0.05cm,parsep=0.1cm]
    \item \textbf{Technology}: tech industry, technology sector, information technology industry, IT industry
    \item \textbf{Legal}: legal industry, law sector, legal service industry
    \item \textbf{Healthcare}: hospital, healthcare industry, pharmaceutical industry
    \item \textbf{Public}: government sector, public sector, state government, public services industry
    \item \textbf{Education}: education industry, educational services sector, education sector
    \item \textbf{Sales}: retail industry, marketing industry, sales industry, commercial industry
    \item \textbf{Finance}: financial sector, finance industry, business sector, financial services industry
    \item \textbf{Entertainment}: media industry, media sector, entertainment industry
    \item \textbf{Industrial}: manufacturing industry, industrial sector, transportation industry
\end{enumerate}

\newpage
\subsection{Additional Results on Steering Transferability}

\begin{figure}[!htb]
\centering
    \includegraphics[width=0.55\linewidth]{figs/qwen-1.8b-occupation-projection-2.pdf}
\caption{Input projections of the occupational stereotypes task, evaluated on \model{Qwen-1.8B-Chat} at the last token position. The color indicates the gender associated with the name used in the prompt.}
\end{figure}

\begin{figure}[!htb]
\centering
    \includegraphics[width=0.49\linewidth]{figs/qwen-1.8b-occupation-stereotypes-education.pdf}
    \includegraphics[width=0.49\linewidth]{figs/qwen-1.8b-occupation-stereotypes-entertainment.pdf}
    \includegraphics[width=0.49\linewidth]{figs/qwen-1.8b-occupation-stereotypes-legal.pdf}
    \includegraphics[width=0.49\linewidth]{figs/qwen-1.8b-occupation-stereotypes-industrial.pdf}
\caption{Difference in job title prediction frequency when prompted with feminine names compared to masculine names. The color represents the difference \textit{before} and \textit{after} debiasing on \model{Qwen-1.8B-Chat}. The y-axis shows the top 12 titles with the largest prediction gap.}
\end{figure}

\vskip -1em
\subsection{Performance and Efficiency Comparison in the non-Evolving and Evolving Setting}

\subsubsection{Performance Comparison}
To compare the effectiveness of GECC with the baselines, we utilize the generated condensed graph data to train a standard GCN, reporting both test accuracies and standard deviations in Table \ref{tab:main}. For each dataset, we present two test accuracies: one for \textit{Non-Evolving} setting, reflects the test accuracy achieved by training the GCN on the graph at the final time step, $t = T_5$, corresponding to the largest possible graph size. Additionally, Figure \ref{fig:evolving} demonstrates the test accuracy across different time steps which compares the GECC evolving capability with the existing baselines. 
one from the \textit{Evolving} setting, which represents the average test accuracy across five time-steps as the graph evolves. This demonstrates the dynamic performance of the condensation method as the graph grows. Our analysis reveals several key insights: 

\textbf{Non-evolving setting --} GECC outperforms the baselines on almost all datasets, including the \emph{whole} dataset performance for some datasets, which highlights the effectiveness of our training-free approach in the non-evolving setting. The only exceptions are observed in the \textit{Flickr} and \textit{Ogbn-arxiv} datasets, where GECC ranks second by a minimal margin. In addition, unlike other baselines such as GCond, GECC does not fail on extremely large datasets like \textit{Ogbn-products} due to memory constraints. 

\textbf{Evolving setting --} By analyzing Table ~\ref{tab:main}, we observe that GECC consistently outperforms existing baselines across almost all datasets, often by a large margin. The only exception is \textit{Ogbn-arxiv}, where GECC ranks second. \textbf{However}, Figure~\ref{fig:evolving} highlights that GECC achieves nearly its maximum possible test accuracy even at very early time steps. For instance, on \textit{Ogbn-arxiv}, it surpasses 65\% accuracy by the second time step, a feat unmatched by existing baselines. This demonstrates GECC’s efficiency in leveraging limited data for superior generalization, whereas other methods struggle to reach comparable performance early in the training process. 
\textbf{Moreover}, we observe that GECC surpasses the performance achieved on the whole dataset for relatively smaller graphs such as \textit{Cora}, \textit{Citeseer}, and \textit{Pubmed}. This result highlights the effectiveness of our training-free condensation approach in achieving a "lossless" objective during graph evolution. \textbf{Additionally}, Figure~\ref{fig:evolving} illustrates the robustness of GECC in steadily improving performance as training size increases. This trend fails to hold for baselines, particularly for GEOM on \textit{Reddit}. We conjecture that the trajectory matching method heavily relies on the performance of the pre-trained GNN, making it susceptible to the twofold influence of data size changes: first, during the pre-training stage, and second, during the condensation stage.

% These findings collectively establish GECC as the most effective method for evolving graph scenarios, ensuring both scalability and superior condensation quality.

% \vskip -4em
\subsubsection{Efficiency Comparison}
Comparing GECC with the baselines in Figure~\ref{fig:accuracy_vs_time}, it is evident that GECC exhibits superior efficiency and scalability. It maintains stable performance while effectively managing computational resources as the graph evolves. Although certain model-based GC methods such as GEOM may slightly outperform GECC at specific time steps, GECC achieves over 100 times faster condensation time and demonstrates a significantly slower increase in computational overhead. For results in more datasets, please refer to Appendix~\ref{app:exp}.

% \begin{figure*}[t]
%     \centering
%     % Subfigure 1: Small Evolution
%     \begin{subfigure}[b]{0.8\textwidth}
%         \centering
%         \includegraphics[width=\linewidth]{figs/Evolve_small-cropped.pdf}
%         \label{fig:evolve_small}
%     \end{subfigure}
%     \hfill
%     % Subfigure 2: Large Evolution
%     \begin{subfigure}[b]{0.6\textwidth}
\begin{figure*}[t!]
        \centering
        \includegraphics[width=\linewidth]{figs/main_merged-cropped.pdf}
    % \vskip -1em
    \caption{Comparison of test accuracy of different GC methods across five time steps.}
    \label{fig:evolving}
    % \vskip -1em
\end{figure*}

\begin{figure}[h]
    \centering
    \includegraphics[width=0.85\linewidth]{figs/Reddit_time_vs_accuracy-cropped.pdf}
    % \vskip -1em
    \caption{Test accuracy vs. condensation time on the \textit{Reddit} dataset (top-left is better).}
    \label{fig:accuracy_vs_time}
    \vskip -1.2em
\end{figure}

% \vskip -1em
\subsubsection{Transferability}
A crucial factor in evaluating GC methods is determining whether the condensed data can effectively train various GNNs from a data-centric perspective. Unlike GECC, which adopts a training-free condensation approach, most existing methods generate condensed graphs that are inherently dependent on the backbone GNN used during condensation, such as GCN \citep{hashemi2024comprehensive,gong2024gc4nc}. This reliance can introduce inductive biases, potentially hindering their adaptability to other GNN architectures. 

Table~\ref{tabs:transfer} shows that the condensed graphs generated by \textsc{GECC} demonstrate robust generalization across diverse architectures when compared to other SOTA baselines. Even in \textbf{Ogbn-arxiv}, \textsc{GECC} outperforms \textsc{GEOM} w.r.t consistency, as indicated by its lower standard deviation across downstream GNN models, highlighting the advantage of \textsc{GECC}'s model-agnostic design.

\subsection{Ablation Studies}
To investigate the effectiveness of each module of our method, we conduct the following ablation studies to study the impact of feature propagation, incremental $k$-means++, and balanced SSE score. 
\subsubsection{Impact of Feature Propagation}
As feature propagation (Equation~\ref{equ:prop}) is crucial for generating informative node representations, we compare \textsc{GECC} against an ablated version without feature propagation (labeled \emph{w/o propagation}) across all time steps. Specifically, for \emph{w/o propagation}, we set \(\alpha_{0}=1\) and all other \(\alpha\) coefficients to \(0\), keeping all other hyperparameters identical. Each experiment is repeated ten times, and we report the average results.

\begin{figure}[h]
    \centering
    \includegraphics[width=\linewidth]{figs/abla_agg-cropped.pdf}
    % \vskip -1em
    \caption{Performance comparison between \textsc{GECC} and \textsc{GECC} without feature aggregation.}
    \label{fig:abla-prop}
    % \vskip -1.5em
\end{figure}

Figure~\ref{fig:abla-prop} illustrates that \textsc{GECC} with feature propagation outperforms the version without propagation. Notably, in some datasets (e.g., Flickr), the \emph{w/o propagation} approach exhibits a downward trend even as more graph data is introduced, suggesting that the raw node features may contain substantial noise. In contrast, \textsc{GECC} with feature propagation steadily improves as the dataset size increases, highlighting that propagated features effectively mitigate noise and bolster clustering performance.

\subsubsection{Impact of Incremental \(k\)-Means++}
We evaluate the effect of reusing previous condensed results when clustering at each time step. Specifically, we compare \textsc{GECC} against a variant that does \emph{not} reuse prior centroids---denoted \textit{w/o incremental \(k\)-means++}---and instead initializes centroids from scratch via a standard \(k\)-means++ procedure at every time step.
\begin{figure}[h]
    \centering
    \includegraphics[width=\linewidth]{figs/boxplot_merged-cropped.pdf}
    % \vskip -1em
    \caption{Comparison between \textsc{GECC} and \textsc{GECC} without incremental \(k\)-means++. 
    The boxplots show the distribution (mean and quartiles) of the number of iterations required for clustering.}
    \label{fig:abla-incrementalkmeans++}
    % \vskip -1em
\end{figure}

As shown in Figure~\ref{fig:abla-incrementalkmeans++}, reusing previously learned cluster centers via incremental \(k\)-means++ significantly reduces the required number of iterations for convergence, especially as the graph grows larger. For example, on the \textit{Reddit} dataset at time step~5, incremental initialization only requires about 10\% of the iterations needed when initializing from scratch. We omit results on smaller datasets because they typically converge in fewer than 10 iterations.



\subsubsection{Relation between GC and clustering objective}
The balanced Sum of Squared Errors (SSE) is a key contribution derived from our theoretical analysis. To assess its impact, we perform an ablation study comparing the performance of GECC with and without repetitive clustering. As shown in Table~\ref{tab:comparison_sse}, for instance, on the \textit{Citeseer} dataset, applying repetitive clustering to select the lowest SSE leads to an absolute performance improvement of 2.7\%. These results clearly show that a lower SSE consistently correlates with higher test accuracy.
\begin{table}[h]
\centering
\caption{Comparison of average test accuracy (\%) and balanced SSE on three benchmark datasets across five time steps. ``Bal.\ SSE'' indicates the balanced SSE.}
% \vskip -1em
\resizebox{\linewidth}{!}{%
\begin{tabular}{lcccccc}
\toprule
& \multicolumn{2}{c}{\textit{Citeseer}} & \multicolumn{2}{c}{\textit{Cora}} & \multicolumn{2}{c}{\textit{Pubmed}} \\
\cmidrule(r){2-3} \cmidrule(r){4-5} \cmidrule(r){6-7}
& Acc.~(\(\uparrow\)) & Bal.\ SSE~(\(\downarrow\)) 
& Acc.~(\(\uparrow\)) & Bal.\ SSE~(\(\downarrow\)) 
& Acc.~(\(\uparrow\)) & Bal.\ SSE~(\(\downarrow\)) \\
\midrule
w/ rep.\ clustering  & 65.45 & 2.39 & 77.08 & 4.52 & 76.32 & 6.65 \\
w/o rep.\ clustering & 63.75 & 9.83 & 74.82 & 9.47 & 75.99 & 9.89 \\
\bottomrule
\end{tabular}}
\label{tab:comparison_sse}
\vskip -0.5em
\end{table}




\section{Conclusion and Outlook}
In this study, we address the challenge of evolving graph condensation. We observe that a universal clustering framework can naturally optimize the assignment matrix, thereby achieving the common objectives of existing GC methods. Additionally, we propose a novel \emph{balanced SSE} metric that further tightens the upper bound of these objectives. In the evolving setting, we find that our clustering approach can be readily adapted to an incremental version, termed \emph{incremental \(k\)-means++}. Experimental results demonstrate that balanced SSE improves the performance of clustering-based GC, and incremental \(k\)-means++ significantly reduces the number of iterations, thereby enhancing efficiency in evolving environments. Future work includes developing more efficient and scalable clustering techniques, especially soft clustering algorithms for larger graph datasets and adaptively optimizing multi-hop weights, which could be beneficial when the graph keeps evolving over time.


% Despite these advances, our proposed method \textsc{GECC} has some limitations, suggesting several avenues for future work:
% \textbf{First}, 
% \textbf{First}, soft clustering is challenging to implement on large datasets, yet it has been proven effective for smaller datasets. This highlights the need for a more scalable and efficient soft clustering algorithm.  
% \textbf{Second}, while clustering has demonstrated significant benefits for GC, its impact on \emph{independent} data domains (e.g., images) remains theoretically underexplored \citep{zhu2021graphheterophily}. Further research is needed to understand its effectiveness in such contexts.  
% \textbf{Finally}, developing a more effective strategy for weighting multi-hop information (e.g., \(\alpha_i\)) is crucial, particularly as evolving graphs may demand deeper message passing or broader contextual aggregation.
% Although the training-free approach shows advantages in this study, we do not rule out the potential for a better optimization solution that can adaptively determine the weights.





%%
%% The next two lines define the bibliography style to be used, and
%% the bibliography file.
% \clearpage
\bibliographystyle{ACM-Reference-Format}
\bibliography{0kdd2025}

% \clearpage
\appendix
\section{Proof of Theorems}
\label{app:propositions}

% \subsection{Detailed Proof of Theorem \ref{theo:1}}
% \label{app:theo1}
% % \begin{proof}[Proof of Theorem \ref{theo:1}]
%  The test error can be expressed as:
%     \[
%     \| \mathbf{Y} - \hat{\mathbf{Y}}' \| = \| \mathbf{Y} - \mathbf{F} \mathbf{W}' \|
%     \]
    
%     By adding and subtracting \( \mathbf{F} \mathbf{W} \), we have
%     \[
%     \| \mathbf{Y} - \mathbf{F} \mathbf{W} + \mathbf{F} \mathbf{W} - \mathbf{F} \mathbf{W}' \| \leq \| \mathbf{Y} - \mathbf{F} \mathbf{W} \| + \| \mathbf{F} (\mathbf{W} - \mathbf{W}') \|
%     \]

%     This demonstrates that the test error of graph condensation is bounded by the conventional GNN training error (first term) and the additional parameter matching error introduced by condensation (second term).
% \end{proof}



\subsection{Proof of Theorems~\ref{theo:training_stage}}
\label{app:proof_training_stage}
\begin{theorem-nonumber}
    The prediction distance in the training stage is bounded by the sum of representation distance and parameter distance.
    \begin{equation}
    \| \mathcal{K}(\hat{\mathbf{Y}}) - \hat{\mathbf{Y}}' \| 
    \leq  \| \mathcal{K}(\mathbf{F}) - \mathbf{F}' \| \cdot \| \mathbf{W}' \| + \| \mathbf{F} \| \cdot \| \mathbf{W} - \mathbf{W}' \|
    \end{equation}
    where $\|\cdot\|$ denotes the L2 norm and $\mathcal{K}(\cdot)$ can be any projection function that aligns the dimensions of $\hat{{\bf Y}}$ and $\hat{{\bf Y}}'$ or ${\bf F}$ and ${\bf F}'$.
\end{theorem-nonumber}

\begin{proof}
To preserve the training data information to maintain the performance of GNNs, we focus on matching the model predictions on the original graph $G$ and its condensed counterpart $G'$. Since $\mathcal{K}(\cdot)$ only aligns the first dimensions, we have $\mathcal{K}(\hat{\mathbf{Y}}) = \mathcal{K}(\mathbf{F})\mathbf{W}$. Therefore, the expression becomes:
\begin{equation}
    \begin{aligned}
    &\| \mathcal{K}(\hat{\mathbf{Y}}) - \hat{\mathbf{Y}}' \| \\
    = &\| \mathcal{K}(\mathbf{F})\mathbf{W} - \mathbf{F}'\mathbf{W}' \| \\
    = & \| \mathcal{K}(\mathbf{F}) \mathbf{W} - \mathcal{K}(\mathbf{F}) \mathbf{W}' + \mathcal{K}(\mathbf{F}) \mathbf{W}' - \mathbf{F}’ \mathbf{W}' \| \\
    \leq &\| \mathcal{K}(\mathbf{F}) (\mathbf{W} - \mathbf{W}') \| + \| (\mathcal{K}(\mathbf{F}) -  \mathbf{F}')\mathbf{W}' \| \\
    \leq & \| \mathcal{K}(\mathbf{F}) \| \cdot \| \mathbf{W} - \mathbf{W}' \| + \| \mathcal{K}(\mathbf{F}) - \mathbf{F}' \| \cdot \| \mathbf{W}' \|
\end{aligned}
\end{equation}
The objective in training stage is to minimizing $\| \mathcal{K}(\hat{\mathbf{Y}}) - \hat{\mathbf{Y}}' \|$, which can be formulated as:
\begin{equation}
    \begin{aligned}
    &\arg \min_{\hat{\mathbf{Y}}'} \| \mathcal{K}(\hat{\mathbf{Y}}) - \hat{\mathbf{Y}}' \|\\
    =&\arg \min_{\hat{\mathbf{Y}}'} \| \mathcal{K}(\mathbf{F}) \| \cdot \| \mathbf{W} - \mathbf{W}' \| + \| \mathcal{K}(\mathbf{F}) - \mathbf{F}' \| \cdot \| \mathbf{W}' \|\\
    \end{aligned}
\end{equation}
Note that $\|\mathbf{F}\|$ is a constant, and the weight matrix $\| \mathbf{W}' \|$ is naturally constrained due to regularization techniques during model optimization to control its magnitude. Then, we have:
\begin{equation}
    \begin{aligned}
    &\arg \min_{\hat{\mathbf{Y}}'} \| \mathcal{K}(\hat{\mathbf{Y}}) - \hat{\mathbf{Y}}' \|\\
    \approx&\arg \min_{\mathbf{F}'}  \underbrace{\| \mathbf{W} - \mathbf{W}' \|}_{\text{Parameter Distance}} + \underbrace{\| \mathcal{K}(\mathbf{F}) - \mathbf{F}' \|}_{\text{Representation Distance}}
    \end{aligned}
\end{equation}




Therefore, Theorem~\ref{theo:training_stage} indicates that by minimizing the \textbf{representation and parameter distances}, the predictions derived from the condensed graph can be close to those of the original graph.

This completes the proof. 
\end{proof}

% To achieve the objective, mainstream GC methods employ distribution matching, kernel ridge regression (KRR)-based matching, trajectory matching, and gradient matching~\cite{wu2020comprehensive}. However, KRR-based matching has been shown to be less effective when the condensed graph is evaluated using Graph Neural Networks (GNNs)~\cite{gcsntk, gong2024gc4nc}. Consequently, we exclude KRR-based approaches from our analysis. In contrast, both trajectory matching and gradient matching inherently aim to minimize the discrepancy between the model parameters trained on the condensed graph and those trained on the original training set. This objective can be succinctly characterized as parameter matching. Therefore, we categorize these two methodologies under the parameter matching framework. In conclusion, our theoretical framework emphasizes \textbf{distribution matching} and \textbf{parameter matching}, as we believe these approaches encompass the majority of existing GC methods.


\subsection{Proof of Theorems~\ref{theo:test_stage}}
\label{app:proof_test_stage}
\begin{theorem-nonumber}
    The test prediction error of the GNN trained of $G'$ is bounded by the test prediction error of the GNN trained on $G$ plus the parameter distance, as formularized by 
    \begin{equation*}
        \| \mathbf{Y} - \hat{\mathbf{Y}}'' \| \leq \| \mathbf{Y} - \mathbf{F}\mathbf{W} \| + \|\mathbf{F}\| \cdot \| \mathbf{W}- \mathbf{W}'\|
    \end{equation*}
\end{theorem-nonumber}

\begin{proof}
At the test stage, our condensation goal is to ensure that the GNN model, trained on condensed training data $G'$, generalizes effectively to the test graph, i.e., achieving a low prediction error on the test data. 
\begin{equation}
\begin{aligned}
    &\| \mathbf{Y} - \hat{\mathbf{Y}}'' \|  \\
    = &\| \mathbf{Y} - \mathbf{F} \mathbf{W}' \|\\
    = & \| \mathbf{Y} - \mathbf{F} \mathbf{W} + \mathbf{F} \mathbf{W} - \mathbf{F} \mathbf{W}' \| \\
    \leq &\| \mathbf{Y} - \mathbf{F} \mathbf{W} \| + \| \mathbf{F} (\mathbf{W} - \mathbf{W}') \| \\
    \leq &\| \mathbf{Y} - \mathbf{F} \mathbf{W} \| + \| \mathbf{F}\| \cdot \|\mathbf{W} - \mathbf{W}' \|
\end{aligned}
\end{equation}
This inequality incorporates both the original test prediction error and the parameter distance. The objective in testing stage is to minimizing $\| \mathbf{Y} - \hat{\mathbf{Y}}''\|$, which can be formulated as:
\begin{equation}
    \begin{aligned}
    &\arg \min_{\hat{\mathbf{Y}}''} \| \mathbf{Y} - \hat{\mathbf{Y}}'' \|\\
    \approx&\arg \min_{\hat{\mathbf{Y}}'} \| \mathbf{Y} - \mathbf{F} \mathbf{W} \| + \| \mathbf{F}\| \cdot \|\mathbf{W} - \mathbf{W}' \|\\
    \approx&\arg \min_{\hat{\mathbf{Y}}'} \underbrace{\|\mathbf{W} - \mathbf{W}'\|}_{\text{Parameter Distance}} \\
    \end{aligned}
\end{equation}
It indicates that by reducing the \textbf{parameter distance} $\| \mathbf{W}- \mathbf{W}'\|$, the test prediction error becomes more tightly bounded, assuming that the original test prediction error \( \| \mathbf{Y} - \mathbf{F}\mathbf{W}\| \) and the propagated feature matrix \( \mathbf{F} \) remain constant. 

This completes the proof. 
\end{proof}


% \begin{proof}
% We start our theoretical analysis with formulating a simple but new objective for GC. 
% GC can be seen as a process of minimizing the loss by a GNN model trained in the synthetic graph ${G'}$. The objective function can reformulated as follows: 
% \begin{equation}
% {G'}=\underset{{G'}}{\arg \min } \ \mathcal{L}(\mathbf{Y}, \operatorname{GNN}({G'})). \label{eq:graphReduction}
% \end{equation}  
%  Besides, without loss of generality, we employ a regression loss computed over the entire graph to evaluate performance during the testing phase, corresponding to a transductive setting. In an inductive setting, the feature matrix \( \mathbf{F} \) can be replaced by \( \mathbf{F}_t \), where \( \mathbf{F}_t \) represents the propagated features in the test graph.

% After training on the original training set, the prediction of the SGC model is given by:
% \begin{equation}
%     \hat{\mathbf{Y}} = \mathbf{F} \mathbf{W}
% \end{equation}
% where \( \mathbf{F} = \mathbf{A}^K \mathbf{X}\) is the propagated feature matrix of the entire graph. 
% \( \mathbf{W} \in \mathbf{R}^{d \times c} \) is the weight matrix of the trained SGC model.

% Conversely, the prediction of the model trained on the synthetic (condensed) dataset is expressed as:
% \begin{equation}
%     \hat{\mathbf{Y}}' = \mathbf{F} \mathbf{W}'
% \end{equation}
% where \( \mathbf{W}' \in \mathbf{R}^{d \times c} \) denotes the weight matrix obtained from training on the synthetic dataset. 

% The loss function $\mathcal{L}$ during the testing phase in Equ.~\ref{eq:graphReduction} can be computed using the regression loss
% \begin{equation}
%     \mathcal{L} = \| \mathbf{Y} - \hat{\mathbf{Y}}' \|
% \end{equation}
% where \( \mathbf{Y} \in \mathbf{R}^{n \times c} \) represents the true labels.
%  The test error can be expressed as:
%     \[
%     \| \mathbf{Y} - \hat{\mathbf{Y}}' \| = \| \mathbf{Y} - \mathbf{F} \mathbf{W}' \|
%     \]
    
%     By adding and subtracting \( \mathbf{F} \mathbf{W} \), we have
%     \[
%     \| \mathbf{Y} - \mathbf{F} \mathbf{W} + \mathbf{F} \mathbf{W} - \mathbf{F} \mathbf{W}' \| \leq \| \mathbf{Y} - \mathbf{F} \mathbf{W} \| + \| \mathbf{F} (\mathbf{W} - \mathbf{W}') \|
%     \]
% Finally, the regression loss can be bounded by decomposing the error into two components:
% \begin{equation}
%     \| \mathbf{Y} - \hat{\mathbf{Y}}' \| \leq \| \mathbf{Y} - \mathbf{F}\mathbf{W} \| + \| \mathbf{F} (\mathbf{W} - \mathbf{W}') \|
% \end{equation}
% This completes the proof.
% \end{proof}

\subsection{Proof of Theorem \ref{theo:bound_of_parameter}}
\label{app:proof_bound_of_parameter}
\begin{theorem-nonumber}
    The parameter distance can be bounded by the following inequality:
    \begin{equation}
        \| \mathbf{W} - \mathbf{W}' \| \leq \mathcal{C}(\max (\text{diag}(\mathbf{\mathbf{P}^\top\mathbf{P}})))^2,
    \end{equation}
    where $\mathcal{C}=\frac{\|\mathbf{F} \| \cdot \| \mathbf{Y}\|({\lambda_{\min}(\mathbf{F}^\top\mathbf{F})}+\|\mathbf{F} \|)}{{\lambda^2_{\min}(\mathbf{F}^\top\mathbf{F})}}$ is a constant, $\mathbf{P}^\top\mathbf{P} \in \mathbb{R}^{N' \times N'}$ is a diagonal matrix with each diagonal entry corresponding to how many original nodes are assigned to each synthetic node. 
\end{theorem-nonumber}

\begin{proof}
We aim to establish that clustering effectively bounds the error introduced by parameter matching and representation difference. The proofproceeds as follows:
\paragraph{Bounding the Parameter Matching Error \( \| \mathbf{W} - \mathbf{W}' \| \):}
Consider the weight matrices for the SGC and clustering-based methods
\[
\mathbf{W} = (\mathbf{F}^\top \mathbf{F})^{-1} \mathbf{F}^\top \mathbf{Y}, \quad \mathbf{W}' = (\mathbf{C}^\top \mathbf{C})^{-1} \mathbf{C}^\top \mathbf{Y}'
\]
where
\[
\mathbf{C} = (\mathbf{P}^\top\mathbf{P})^{-1} \mathbf{P}^\top \mathbf{F}, \quad \mathbf{Y}' = (\mathbf{P}^\top\mathbf{P}))^{-1} \mathbf{P}^\top \mathbf{Y}
\]

The difference between \( \mathbf{W} \) and \( \mathbf{W}' \) is
\[
\| \mathbf{W} - \mathbf{W}' \| = \left\| (\mathbf{F}^\top \mathbf{F})^{-1} \mathbf{F}^\top \mathbf{Y} - (\mathbf{C}^\top \mathbf{C})^{-1} \mathbf{C}^\top \mathbf{Y}' \right\|
\]

By substituting $\mathbf{C}$ and $\mathbf{Y}'$, we can express \( \mathbf{W}' \) in terms of \( \mathbf{F} \) and \( \mathbf{Y} \)
\[
\mathbf{W}' = \left( \mathbf{F}^\top \mathbf{P} (\mathbf{P}^\top\mathbf{P})^{-2} \mathbf{P}^\top \mathbf{F} \right)^{-1} \mathbf{F}^\top \mathbf{P} (\mathbf{P}^\top\mathbf{P})^{-2} \mathbf{P}^\top \mathbf{Y}
\]

Let \( \mathbf{A} = \mathbf{F}^\top \mathbf{F} \) and \( \mathbf{B} = \mathbf{F}^\top \mathbf{P} (\mathbf{P}^\top\mathbf{P})^{-2} \mathbf{P}^\top \mathbf{F} \), then:
\[
\mathbf{W} = \mathbf{A}^{-1} \mathbf{F}^\top \mathbf{Y}, \quad \mathbf{W}' = \mathbf{B}^{-1} \mathbf{F}^\top \mathbf{P} (\mathbf{P}^\top\mathbf{P})^{-2} \mathbf{P}^\top \mathbf{Y}
\]

The difference becomes
\[
\mathbf{W} - \mathbf{W}' = \mathbf{A}^{-1} \mathbf{F}^\top \mathbf{Y} - \mathbf{B}^{-1} \mathbf{F}^\top \mathbf{P} (\mathbf{P}^\top\mathbf{P})^{-2} \mathbf{P}^\top \mathbf{Y}
\]

Similar to the above two proofs, we add and subtract a term \( \mathbf{B}^{-1} \mathbf{F}^\top \mathbf{Y} \) and rewrite the difference by
\begin{align*}
        \mathbf{W} - \mathbf{W}' = &\left( \mathbf{A}^{-1} - \mathbf{B}^{-1} \right) \mathbf{F}^\top \mathbf{Y} \\
&+ \mathbf{B}^{-1} \mathbf{F}^\top \left( \mathbf{I} - \mathbf{P} (\mathbf{P}^\top\mathbf{P})^{-2} \mathbf{P}^\top \right) \mathbf{Y}
\end{align*}

Considering the norms, we have
    \begin{align*}
\| \mathbf{W} - \mathbf{W}' \| \leq &\| \mathbf{A}^{-1} - \mathbf{B}^{-1} \| \cdot \| \mathbf{F}^\top \mathbf{Y} \| \\
&+ \| \mathbf{B}^{-1} \| \cdot \| \mathbf{F}^\top (\mathbf{I} - \mathbf{P} (\mathbf{P}^\top\mathbf{P})^{-2} \mathbf{P}^\top) \mathbf{Y} \|
\end{align*}

We will now bound each term independently.

\textbf{Bounding the First Term}
\begin{align*}
        \| \mathbf{A}^{-1} - \mathbf{B}^{-1} \| \cdot \| \mathbf{F}^\top \mathbf{Y} \|
\end{align*}


Based on the norms of matrix inequality, we have
\[
\| \mathbf{A}^{-1} - \mathbf{B}^{-1} \| \leq \| \mathbf{A}^{-1} \| \cdot \| \mathbf{B} - \mathbf{A} \| \cdot \| \mathbf{B}^{-1} \|
\]
Then, according to
\[
\| \mathbf{A}^{-1} \| = \frac{1}{\lambda_{\min}(\mathbf{A})}, \quad \| \mathbf{B}^{-1} \| = \frac{1}{\lambda_{\min}(\mathbf{B})}
\]
and assuming \( \lambda_{\min}(\mathbf{B}) \geq \frac{1}{(\max_k (\mathbf{P}^\top\mathbf{P})_{kk})^2} \lambda_{\min}(\mathbf{A}) \), we have
\[
\| \mathbf{A}^{-1} \| \cdot \| \mathbf{B}^{-1} \| \leq \frac{(\max_k (\mathbf{P}^\top\mathbf{P})_{kk})^2}{\lambda_{\min}(\mathbf{A})^2}
\]

Bounding \( \| \mathbf{B} - \mathbf{A} \| \):
\begin{align*}
    \mathbf{B} - \mathbf{A} &= \mathbf{F}^\top \mathbf{P} (\mathbf{P}^\top\mathbf{P})^{-2} \mathbf{P}^\top \mathbf{F} - \mathbf{F}^\top \mathbf{F} \\
    &= -\mathbf{F}^\top (\mathbf{I} - \mathbf{P} (\mathbf{P}^\top\mathbf{P})^{-2} \mathbf{P}^\top) \mathbf{F}
\end{align*}
Taking norms:
\begin{align*}
\| \mathbf{B} - \mathbf{A} \| &= \| \mathbf{F}^\top (\mathbf{P} (\mathbf{P}^\top\mathbf{P})^{-2} \mathbf{P}^\top - \mathbf{I}) \mathbf{F} \| \\
&\leq \| \mathbf{F} \|^2 \cdot \| \mathbf{I} - \mathbf{P} (\mathbf{P}^\top\mathbf{P})^{-2} \mathbf{P}^\top \|
\end{align*}
Since \( \| \mathbf{I} - \mathbf{P} (\mathbf{P}^\top\mathbf{P})^{-2} \mathbf{P}^\top \| \leq 1 \), the inequality can be further simplified to
\(\| \mathbf{B} - \mathbf{A} \| \leq \| \mathbf{F} \|^2\).

Combining the above:
\[
\| \mathbf{A}^{-1} - \mathbf{B}^{-1} \| \cdot \| \mathbf{F}^\top \mathbf{Y} \| \leq \frac{(\max_k (\mathbf{P}^\top\mathbf{P})_{kk})^2}{\lambda_{\min}(\mathbf{A})^2} \cdot \| \mathbf{F} \|^2 \cdot \| \mathbf{Y} \|
\]


\textbf{Bounding the Second Term}
\begin{align*}
    \| \mathbf{B}^{-1} \| \cdot \| \mathbf{F}^\top (\mathbf{I} - \mathbf{P} (\mathbf{P}^\top\mathbf{P})^{-2} \mathbf{P}^\top) \mathbf{Y} \|
\end{align*}


Based on the proof above, we first have
\[
\| \mathbf{B}^{-1} \| = \frac{1}{\lambda_{\min}(\mathbf{B})} \leq \frac{(\max_k (\mathbf{P}^\top\mathbf{P})_{kk})^2}{\lambda_{\min}(\mathbf{A})}
\]

 Since \( \| \mathbf{I} - \mathbf{P} (\mathbf{P}^\top\mathbf{P})^{-2} \mathbf{P}^\top \| \leq 1 \), 
\[
\| \mathbf{F}^\top (\mathbf{I} - \mathbf{P} (\mathbf{P}^\top\mathbf{P})^{-2} \mathbf{P}^\top) \mathbf{Y} \| \leq \| \mathbf{F} \| \cdot \| \mathbf{Y} \|
\]

Combining these bounds:
\begin{align*}
        &\| \mathbf{B}^{-1} \| \cdot \| \mathbf{F}^\top (\mathbf{I} - \mathbf{P} (\mathbf{P}^\top\mathbf{P})^{-2} \mathbf{P}^\top) \mathbf{Y} \| \\
        &\leq \frac{(\max_k (\mathbf{P}^\top\mathbf{P})_{kk})^2}{\lambda_{\min}(\mathbf{A})} \cdot \| \mathbf{F} \| \cdot \| \mathbf{Y} \|
\end{align*}


\paragraph{Combining Both Terms:}
Adding the bounds for both terms, we obtain:
\begin{align*}
\| \mathbf{W} - \mathbf{W}' \| \leq & \frac{(\max_k (\mathbf{P}^\top\mathbf{P})_{kk})^2}{\lambda_{\min}(\mathbf{A})^2} \cdot \| \mathbf{F} \|^2 \cdot \| \mathbf{Y} \| \\
&\quad + \frac{(\max_k (\mathbf{P}^\top\mathbf{P})_{kk})^2}{\lambda_{\min}(\mathbf{A})} \cdot \| \mathbf{F} \| \cdot \| \mathbf{Y} \| \\
=&\max_k (\mathbf{P}^\top\mathbf{P})_{kk}) \cdot (\frac{\| \mathbf{F} \|^2 \cdot \| \mathbf{Y} \|}{\lambda_{\min}(\mathbf{A})^2} + \frac{\| \mathbf{F} \| \cdot \| \mathbf{Y} \|}{\lambda_{\min}(\mathbf{A})}) \\
=&\mathcal{C}(\max (\text{diag}(\mathbf{\mathbf{P}^\top\mathbf{P}})))^2
\end{align*}
where $\mathcal{C}=\frac{\|\mathbf{F} \| \cdot \| \mathbf{Y}\|({\lambda_{\min}(\mathbf{F}^\top\mathbf{F})}+\|\mathbf{F} \|)}{{\lambda^2_{\min}(\mathbf{F}^\top\mathbf{F})}}$ is a constant, $\mathbf{P}^\top\mathbf{P} \in \mathbb{R}^{N' \times N'}$ is a diagonal matrix with each diagonal entry corresponding to how many original nodes are assigned to each synthetic node. 

Our objective is to minimize the parameter distance, which can be reformulated by:
\begin{equation}
    \begin{aligned}
    &\arg \min_{\mathbf{W}'} \| \mathbf{W} - \mathbf{W}' \|\\
    \approx&\arg \min_{\mathbf{P}} \mathcal{C}(\max (\text{diag}(\mathbf{\mathbf{P}^\top\mathbf{P}})))^2\\
    \end{aligned}
\end{equation}

This completes the proof.   
\end{proof}
    
% \paragraph{Conclusion.} \textbf{First}, larger \( \max((\mathbf{P}^\top\mathbf{P})_{kk}) \) increases the bound on parameter matching error. \textbf{Second}, lower \( \sigma_{ck}^2 \) leads to smaller representation matching error, enhancing the overall condensation quality. These two factors are directly related to clustering quality. \textbf{Third}, a larger \( \lambda_{\min}(\mathbf{A}) \) decreases the bound, indicating that better-conditioned covariance matrices lead to tighter error bounds. \textbf{Finally}, higher \( \| \mathbf{F} \| \) and \( \| \mathbf{Y} \| \) contribute to larger error bounds. These two are predefined by the original dataset.

% \section*{Incremental vs.\ Batch-wise (Streaming) kmeans++}

% We have a dataset $\mathcal{X} = \{\mathbf{x}_1, \dots, \mathbf{x}_n\} \subset \mathbb{R}^d$ 
% and want $k$ centroids. Two procedures are common:

% \begin{enumerate}
%   \item \textbf{Standard kmeans++:} Select the first centroid $\mathbf{c}_1$ uniformly at random; 
%   for each subsequent centroid $\mathbf{c}_{t+1}$, 
%   use the distance-based probabilities 
%   $P(\mathbf{x}_i) = D(\mathbf{x}_i)^2 / \sum_{\mathbf{x}_j} D(\mathbf{x}_j)^2,$
%   where 
%   $D(\mathbf{x}_i) = \min_{1 \le s \le t}\|\mathbf{x}_i - \mathbf{c}_s\|.$
%   \vspace{0.2em}
%   \item \textbf{Incremental (or Streaming) kmeans++:} 
%   Split $\mathcal{X}$ into batches or handle data in an online manner. 
%   In each batch, we pick a subset of centroids using a similar \emph{kmeans++-like} rule, 
%   then \emph{merge} these partial centroids at the end, possibly running an extra final kmeans++ step on the merged set.
% \end{enumerate}

% \subsection*{1.\ Incremental Selection (One Centroid at a Time): Exact Equivalence}

% \paragraph{Algorithmic Description.} Instead of computing $D(\mathbf{x}_i)$ from scratch each time, we keep a \emph{running} (incremental) record of the distances:

% \begin{itemize}
%   \item Pick the first centroid $\mathbf{c}_1$ uniformly at random from $\mathcal{X}$.
%   \item Maintain for each point $\mathbf{x}_i$ a distance $D(\mathbf{x}_i) = \min_{1 \le s \le t}\|\mathbf{x}_i - \mathbf{c}_s\|^2$,
%         updated whenever a new centroid $\mathbf{c}_{t+1}$ is chosen:
%   \[
%     D(\mathbf{x}_i) \;=\; 
%       \min\Bigl(D(\mathbf{x}_i), \|\mathbf{x}_i - \mathbf{c}_{t+1}\|^2\Bigr).
%   \]
%   \item Sample the next centroid $\mathbf{c}_{t+1}$ with probability 
%   \[
%     P(\mathbf{x}_i) \;=\; 
%       \frac{D(\mathbf{x}_i)}{\sum_{\mathbf{x}_j \in \mathcal{X}} D(\mathbf{x}_j)},
%   \]
%   \item Repeat until we have $k$ centroids.
% \end{itemize}

% \paragraph{Proof of Equivalence.} 
% At each centroid-selection step $t$, the \emph{distance values} $\{D(\mathbf{x}_i)\}$ in the incremental method match exactly the corresponding distances in standard kmeans++, because in both cases 
% \[
%   D(\mathbf{x}_i) \;=\;
%   \min_{1 \le s \le t}\|\mathbf{x}_i - \mathbf{c}_s\|^2.
% \]
% Hence, both algorithms \emph{sample the next centroid} from \emph{the same probability distribution} over $\mathcal{X}$. 
% By induction on $t=1,\dots,k$, the two methods select the \emph{same} (or identically distributed) set of $k$ centroids. 
% Thus, \emph{incremental kmeans++} (one-at-a-time) is \textbf{exactly the same} as standard kmeans++ if they use the same random seed.

% \subsection*{2.\ Batch-wise (Streaming) kmeans++: Approximate Equivalence}

% In a streaming or large-scale setting, we may instead:
% \begin{enumerate}
%   \item Partition $\mathcal{X}$ into batches $\mathcal{X}_1, \mathcal{X}_2, \dots, \mathcal{X}_T$ (e.g.\ chunks or time windows).
%   \item On each batch $\mathcal{X}_t$, run a local kmeans++ to extract $k_t$ ``partial centroids'' 
%         $\{\mathbf{p}_1^{(t)}, \dots, \mathbf{p}_{k_t}^{(t)}\}$.
%   \item Merge all partial centroids $\{\mathbf{p}_j^{(t)}\}$ into one set, 
%         optionally weighting each centroid by the number of data points in its batch-cluster,
%         and run a final round of kmeans++ on this \emph{compressed} or merged set to extract $k$ global centroids.
% \end{enumerate}

% \paragraph{Why this is approximate.} In standard kmeans++, the distance probabilities are computed against \emph{all points} from \(\mathcal{X}\) at each centroid selection. 
% In the batch-wise approach, we only compute local distances within each batch and then later refine by clustering the merged centroids. 
% Hence, we generally cannot guarantee \emph{exact} equivalence: once data is partitioned, the cross-batch interactions are not fully accounted for in each local selection step. 

% However, one can show that if each batch-level selection is performed by kmeans++ and we preserve the relative weighting of partial centroids appropriately, 
% the final global clustering has a provable approximation factor relative to running a \emph{single} standard kmeans++ on all of \(\mathcal{X}\). 
% For instance, a common result in the coreset and streaming literature is that the final error (cost) of the merging approach is within a small constant factor (or $O(\log n)$ in the worst case) of the cost of a global optimum or a global kmeans++ initialization. 

% \paragraph{Sketch of Approximation Argument.}
% Suppose each batch $\mathcal{X}_t$ is of size $n_t$, with $\sum_{t=1}^\top n_t = n$. 
% Running local kmeans++ with $k_t$ partial centroids on batch $t$ yields a partial solution whose cost is close to the best $k_t$-clustering cost on that batch. 
% Collecting these partial centroids (with weights proportional to $n_t$ or the cluster sizes) forms a \emph{weighted coreset} \(\mathcal{P}\). 
% A final kmeans++ on \(\mathcal{P}\) yields $k$ global centroids. 
% By standard coreset arguments (and the properties of kmeans++ on weighted datasets), the final solution is guaranteed to be within a \emph{bounded factor} of the global kmeans++ cost on the full dataset. 
% Exact approximation ratios depend on how many centroids $k_t$ we extract per batch, the total number of batches, and the structure of the data. 
% Typical results show that for sufficiently large $k_t$ (relative to $k$) or well-chosen merges, 
% the final solution approximates a global kmeans++ solution to within constant or polylogarithmic factors.

% \subsection*{Summary}

% \begin{itemize}
%   \item \textbf{One-at-a-time incremental kmeans++ is exactly equivalent} to standard kmeans++ 
%   because the distance-based sampling distribution at each step is identical.

%   \item \textbf{Batch-wise (streaming) kmeans++ is generally an approximation} 
%   because each batch makes centroid choices without full knowledge of data in other batches. 
%   However, coreset/merging techniques plus a final global pass can ensure the final clustering 
%   is within a guaranteed approximation factor of the single-batch (standard) kmeans++ solution.
% % \end{itemize}
\section{Method Pipeline}
\begin{figure}[h]
    \centering
    \includegraphics[width=0.9\linewidth]{figs/main.pdf}
    \caption{Illustration of the GECC framework. The lower part of the figure depicts the two evolving settings: in the \textbf{inductive} setting, the graph structure evolves over time, while in the \textbf{transductive} setting, only the training nodes (labels) change, with the graph structure and non-training nodes remaining unchanged.}

    \label{fig:main}
\end{figure}

The proposed GECC framework is illustrated in Figure~\ref{fig:main}. GECC takes the current graph \( G_t = (\mathbf{X}_t, \mathbf{A}_t) \) along with the centroids from the previous time step to perform condensation and generate the condensed graph \( G'_t = (\mathbf{I}, \mathbf{C}'_t) \). The lower part of the figure also depicts the two evolving settings: in the \textbf{inductive} setting, the graph structure evolves over time, while in the \textbf{transductive} setting, only the training nodes (labels) change, whereas the graph structure and non-training nodes remain unchanged.



\section{Experimental Details}\label{app:exp}
\section{Dataset Generation}
\label{sec:dataset}
\revise{
To train the proposed GNN, we constructed a dataset of building structures and a subset of these structures were subjected to fire simulations using FEA. The dataset generation process is illustrated in \figref{fig:dataset_generation_procedure}. Initially, a total of 33,000 building structures with geometrical details, material properties, and gravity loads were created. Due to randomness in generating these structures, a filter is applied to remove unreasonable data after gravity load simulation, which included 15,377 structures. A trade-off between computational feasibility and model performance is made among the remaining 17,623 structures. As further labeling structures with MIDR requires resource-intensive fire simulations via OpenSeesRT, a large proportion of 16,050 structures is selected as unlabeled dataset. On the other hand, each of the other 1,573 structures was further subjected to 30 different fire simulations, forming the labeled dataset containing $1,573\times 30 = 47,190$ fire cases.} This section details the step-by-step process for generating the dataset, including geometry creation, material property assignment, and simulations due to gravity loads and fire scenarios. 
% To train the proposed neural network, we constructed a dataset comprising building structure data and a subset of fire scenario data. The dataset generation process is illustrated in \figref{fig:dataset_generation_procedure}. 
% A total of 33,000 building structures with geometric details, material properties, and gravity loads were initially created. Out of these, 3,000 structures were selected as labeled data, and the remaining 30,000 were designated as unlabeled data. Further, about half of them filtered out due to instability under gravity loads only. 
\begin{figure*}[h!]
    \centering
    \includegraphics[width=0.8\linewidth]{figures/dataset_filter_procedure.pdf}
    \caption{Workflow for dataset generation (geometry, material property, gravity loads, and fire scenarios).}
    \label{fig:dataset_generation_procedure}
\end{figure*}

\subsection{Geometry Generation}
\label{subsec:geometry_generation}
The geometry of the building structures forms the foundation of the dataset. Regular 
\revise{3D structures} resembling multi-story parking structures or shopping malls were generated, with parameters such as building floor dimensions and story heights selected randomly. Each building structure is composed of multiple rooms, which serve as the basic unit in this study. A room herein is a cuboid space defined by specific length, width, and height. Within a structure, rooms of the same dimensions are uniformly arranged along the length, width, and height, corresponding to the $x$-, $y$-, and $z$-axes, respectively. Structures vary in room size and number of rooms along each axis. Specifically, the room length, width, and height are independently sampled from a uniform distribution within the interval $[2, 5]$ meters along the three directions of the structure. Similarly, the room number along each axis is uniformly sampled independently as an integer within the interval $[2, 7]$, i.e., the maximum number of stories of the buildings simulated in this study is 7.

To introduce variability and simulate real-world scenarios, approximately $8\%$ of structural elements (beams or columns) are randomly removed after initial geometry creation. 
\revise{Such removal is not fire-induced damage, but reflects functional diversity often observed in real buildings, such as open spaces designed for activities in shopping malls, e.g., ice skating rinks. Examples of the generated geometries are illustrated in \figref{fig:example_generated_geometry}, showcasing the diversity and realism of the dataset. This element removal does not affect the definition of room's geometry in the structure and nor does it affect the number of considered fire scenarios.} 

\revise{A range of coefficient of variation values ($3.3\%$ to $17.5\%$) was derived from prior studies that investigated the statistics of geometrical and material properties of structural components of buildings (e.g., \cite{mirza1979variations, lee2004probabilistic}). These studies provide empirical data on the natural variability in parameters such as Young's modulus, yield strength, and dimensions of structural elements due to manufacturing tolerances and material inconsistencies. By selecting $8\%$ for the removal of structural elements in our database, we aimed to maintain a level of variability that is representative of real-world uncertainties while ensuring computational feasibility. This choice ensures that the database captures realistic deviations without introducing extreme cases that may not be commonly encountered in practice.}

\begin{figure*}[h!]
    \centering
    \includegraphics[width=\linewidth]{figures/example_generated_geometry.pdf}
    \caption{Examples of generated structural geometry of different sizes (all dimensions in meters).}
    \label{fig:example_generated_geometry} 
\end{figure*}

{\blockRevise

In this study, we opted for a deterministic square, dimension of $0.1$ m, solid cross-sectional steel elements due to their simplicity in modeling and analysis. Square sections exhibit uniform geometrical properties in all directions, simplifying the computation of structural responses and avoiding complications associated with more complex shapes, such as wide-flange sections, facilitating the computational efficiency and scalability to generate a large dataset. This choice also helps to mitigate issues related to stress concentrations and facilitates a more straightforward representation of structural behavior under thermal loads. 

\textit{Remark:} The selected cross-section provides a comparable flexural rigidity to a $W 130 \times 130 \times 28.1$ wide-flange section (metric units), albeit with significantly higher axial rigidity. This cross-section is acceptable for gravity-load-designed frames under service loading conditions where the models assume fully rigid, moment-resisting beam-column connections for the evaluation of the IDR under thermal loading. This assumption is reasonable in this computational study where the primary interest is to understand the global deformation response of frames under fire conditions. The selection of uniform square cross-sections for both beams and columns, rather than adherence to standard capacity design principles, was made here primarily for computational efficiency and to reduce design parameters in the database generation process. This choice allows for simplified and scalable approach to analyze the fire-induced response of generic steel frames without the need for large section variations, where this study mainly focuses on the fire vulnerability assessment using ML-based predictions. However, if additional loading conditions, e.g., seismic or wind loads, were to be considered, larger sections, strong-column/weak-beam principle, and ductile detailing would be required in the generated buildings for realistic structural behavior under combined loading conditions. Future studies may also consider investigating the influence of variable cross-sectional dimensions and semi-rigid connections on the structural performance under fire conditions. 
} % blockRevise

\subsection{Material Properties}
Steel is chosen as the material for the structures. To reflect real-world variations, we randomly assign one of five slightly different steel material types to each structural element. \revise{
The ranges of material properties are provided in \tabref{tab:material_property_ranges} and the properties are sampled from uniform distributions of the corresponding ranges. These variations simulate differences arising from manufacturing batches or regional material properties. That these properties are at ambient temperature and change when the temperature rises due to a fire. The selection of materials with varying properties is aimed at increasing the diversity of the data. Our goal is to represent as wide a range of data as possible with a limited amount of building structure data, thereby enhancing the generalization ability of the GNN. Our assumed material property ranges are expected to be wider than the real-world conditions based on findings in \cite{mirza1979variations, lee2004probabilistic}. Therefore, we are essentially tackling a more challenging and general task. If we can solve this problem, we are confident that our method will perform equally well or even better in real-world scenarios.
}
\begin{table}[h!]
    \centering
    \caption{Material properties ranges for considered steel structures.}
    \begin{tabular}{lc}
        \toprule
        Property & Range \\
        \midrule
        Young's modulus & [168, 252] GPa \\
        Yield strength & [220, 330] MPa \\
        Strain-hardening ratio & [0.8, 1.2] \% \\
        \bottomrule
    \end{tabular}
    \label{tab:material_property_ranges}
\end{table}

\subsection{Gravity Loads}
Gravity loads are applied to columns and beams based on their \revise{influence (tributary) areas as typically conducted in structural analysis. The considered ``service'' load conditions include the column self-weight and the additional loads directly supported on the beams from their self-weight and weights of the reinforced concrete slabs, people as live load, and building content. An edge beam typically carries approximately half the gravity load supported by a parallel interior beam}. The ranges of gravity loads are listed in \tabref{tab:gravity_load_ranges}. \revise{The loads are sampled from uniform distributions of the corresponding ranges.} Structures that failed to meet an MIDR threshold of $1\%$ under gravity loads were deemed unacceptable designs and filtered out, as such configurations of randomly chosen geometry, material, and gravity load combinations were considered unrealistic from a regulatory and practicality points of view.
\begin{table}[h!]
    \centering
    \caption{Gravity load ranges for considered beams and columns.}
    \begin{tabular}{lc}
        \toprule
        Element & Range (kN/m)  \\
        \midrule
        Column & [0.5, 1.0]  \\
        Edge beam & [1.5, 4.5]  \\
        Interior beam & [3.0, 7.5]  \\
        \bottomrule
    \end{tabular}
    \label{tab:gravity_load_ranges}
\end{table} 

\subsection{Rule-based Thermal Load Generation}
\label{subsec:thermal_load_generation}
To evaluate a building's structural response during a fire event, we employed a simplified rule-based approach for thermal load generation. 
% Previous studies \cite{nan_structuralfire_2023} have demonstrated that steel structures rapidly equilibrate with surrounding gases temperatures due to efficient heat exchange. Consequently, gas temperatures can be directly used as inputs for FEA tools, e.g., OpenSees, simplifying the process of modeling thermal loads. 
% Accurately simulating temperature fields in fire scenarios poses significant challenges. Advanced thermodynamic simulations, such as those performed using Fire Dynamics Simulator (FDS) \cite{mcgrattan_fire_2000}, provide precise temperature predictions. However, these methods are hindered by high computational costs, prolonging execution times, and limited scalability, making them impractical for generating large datasets. Additionally, real-world fire loads often display substantial spatial variability across different rooms \cite{dundar_fire_2023}, resulting in scenario-specific temperature fields with limited generalizability. For example, studies on bridge fires \cite{he_study_2024} have demonstrated that environmental factors, such as wind speeds, can significantly influence temperature distributions. Furthermore, even within identical scenarios, variations in fire modeling methodologies can produce distinctly different temperature fields \cite{zhang_temperature_2020, du_new_2012}. These challenges emphasize the need for efficient and adaptable methods to generate fire temperature data.
% To address these issues, we adopted a rule-based approach to model temperature variations. 
According to \cite{spearpoint_fire_2008}, a typical fire development follows a predictable pattern. During the {\em{growth stage}}, the temperature rises slowly and approximately linearly after ignition. This is followed by the {\em{flashover stage}}, where temperatures increase rapidly to peak values. After reaching the peak, the temperature either stabilizes or continues to rise slowly until the {\em{decay stage}} begins. Inspired by this fire development pattern, we describe the temperature evolution in time, $t$, prior to the decay stage in two distinct stages:
\begin{enumerate}
    \item {\bf{Initial linear increase stage}}: For $t \in [0, t_1)$, temperature increases gradually and linearly as the fire spreads through the building. This stage represents the time before the fire directly affects a structural element.  
    \item {\bf{ISO 834 fire curve stage}}: For $t \in [t_1, t_{\thre}]$, temperature rises rapidly following the ISO 834 curve \cite{ISO834}, modeling the direct impact of the fire on the structural element. 
\end{enumerate}
The slope of the linear temperature increase, $c$, and the transition time, $t_1$, are influenced by the spatial relationship between the fire source and the structural element. For the second stage of temperature evolution, we utilize the ISO 834 curve, a widely accepted standard for fire resistance testing. This standardized fire curve describes the temperature rise over time, enabling rapid and consistent thermal fields across various scenarios. The duration of fire simulation in this study is set to $t_{\thre}=60$ minutes. This value represents the upper limit for the temperature evolution of each structural element, providing a consistent basis for analyzing the structural response to fire.

Let $(x, y, z)$ represents the midpoint of a structural element and $(x_{\subfire}, y_{\subfire}, z_{\subfire})$ the fire source point. \revise{Integer parameters $h$ and $h_{\subfire}$ correspond to the respective floor levels of the element and the fire source}. The temperature evolution for each element is expressed as follows:
\begin{enumerate}
    \item Linear increase stage ($0 < t < t_1$):
    \begin{equation}
    T(t) = c \cdot t,
    \end{equation}
    where $c$, the rate of temperature increase ($^\circ\mathrm{C}/\mathrm{min}$), depends on the height difference between the element, $h$, and the fire source, $h_{\subfire}$:
    \begin{equation}
        c = 
        \begin{cases} 
        5\left/\left(h - h_{\subfire} + 1\right)\right., & h \geq h_{\subfire}, \\
        2\left/\left(h_{\subfire} - h\right)\right., & h < h_{\subfire}.
        \end{cases}
    \end{equation}
     \item ISO 834 stage ($t \geq t_1$):
\begin{equation}
    T(t) = c \cdot t_1 + 345 \log_{10} \left(8 \left(t - t_1\right) + 1\right).
\end{equation}
\end{enumerate}

The transition (arrival) time $t_1$, marking the end of the linear stage, depends on the spatial distance between the fire source and the element. We define the following two Euclidean distances $L_p$ in the $xy$ plane and $L_s$ in the $xyz$ space:
\begin{eqnarray}
L_p & \triangleq & \sqrt{(x - x_{\subfire})^2 + (y - y_{\subfire})^2}, \\
\label{eq:Lp}
L_s & \triangleq & \sqrt{(x - x_{\subfire})^2 + (y - y_{\subfire})^2 + (z - z_{\subfire})^2}.
\label{eq:Ls}
\end{eqnarray}
Accordingly, the transition time, $t_1$, is expressed as follows:
\begin{equation}
    t_1 = 
    \begin{cases}
    \beta_{1} \cdot \left(1 - \exp\left\{- L_s\left/\alpha_{1}\right.\right\}\right), & h > h_{\subfire}, \\
    \beta_{2} \cdot \left(1 - \exp\left\{- L_p\left/\alpha_{2}\right.\right\}\right), & h = h_{\subfire}, \\
    \beta_{3} \cdot \left(1 - \exp\left\{- L_s\left/\alpha_{3}\right.\right\}\right), & h < h_{\subfire} .
    \end{cases}
    \label{eq:t1}
\end{equation}
The parameters $\beta_i$ and $\alpha_i$ for determining $t_1$ are summarized in Table~\ref{tab:fire_spread_parameters}. In this study, we take $r_{\mathrm{up}}=0.95$ and $r_{\mathrm{down}}=0.97$.
\begin{table}[ht]
    \centering
    \caption{Fire spread parameters for $t_1$ calculations.}
    \begin{tabular}{lcc}
        \toprule
        Case  & $\beta_i$ & $\alpha_i$  \\
        \midrule
        $i=1$, Upward spread & $16 \left.\left(1-r_{\mathrm{up}}^{\left|h-h_{\subfire}\right|}\right)\right/\left(1-r_{\mathrm{up}}\right)$ & $10$  \\
        $i=2$, Horizontal spread & $18$ & $18$  \\
        $i=3$, Downward spread & $30 \left.\left(1-r_{\mathrm{down}}^{\left|h-h_{\subfire}\right|}\right)\right/\left(1-r_{\mathrm{down}}\right)$ & $5$  \\
        \bottomrule
    \end{tabular}
    \label{tab:fire_spread_parameters}
\end{table}

\figref{fig:t1_curve} illustrates the $t_1$ curves for various fire scenarios: (1) fire originating on the lower floor, $h-h_{\subfire}=1$ with rapid upward spread, (2) fire on the same floor, $h=h_{\subfire}$ with the fastest spread, and (3) fire on the upper floor, $h_{\subfire}-h=1$ with slow downward spread. The exponential decay in $t_1$ reflects the accelerating fire propagation speed as the distance increases. \figref{fig:t1_curve} also indicates that the employed simplified model is consistent with the Markov chain-based dynamic model given by \cite{cheng_dynamic_2011}, where the rooms at the same floor of the fire point start flashover slightly before the corresponding upper floors. Additionally, $\beta_{1}$ and $\beta_{3}$ are the summation of a geometric sequence, where story level $h$ is the index. The common ratios $r_{\mathrm{up}}<1$ in $\beta_{1}$ and $r_{\mathrm{down}}<1$ in $\beta_{3}$ indicate that the fire speeds up to spread through the next story, which is consistent with the real-world fire spread mechanism given in \cite{hokugo_mechanism_2000}. The temperature profile within the range $t \in [0, t_{\thre}]$ is subsequently used as the thermal load in OpenSeesRT simulations to compute displacements at each structural node at time $t_{\thre}$.
\begin{figure}[h!]
    \centering
    \includegraphics[width=0.8\linewidth]{figures/m204_t1_curve.pdf}
    \caption{Three examples for the $t_1$ curve.}
    \label{fig:t1_curve}
\end{figure}

\revise{
\textit{Remark:} The effects of structural elements, such as concrete floor slabs and partitions, are not explicitly modeled in our approach. Instead, their influence is implicitly captured through the careful selection of the parameters $ \alpha, \beta, r_\mathrm{up} $, and $ r_\mathrm{down} $. This parameterization provides a unified framework for generating temperature fields. Indeed, fire propagation is governed by a multitude of factors and remains an open research question. For instance, if the fire resistance of a floor slab is enhanced by fire protective coating, the corresponding model can account for this by decreasing $\alpha_1$ \& $\alpha_3$, increasing $\beta_1$ \& $\beta_3$, and adopting larger values for $r_\mathrm{up}$ \& $r_\mathrm{down}$, which collectively slow down the vertical spread of fire. Conversely, scenarios involving higher amounts of combustible materials would warrant the opposite adjustments. This flexible and integrated approach avoids the need to design separate models for different fire propagation scenarios while still capturing the essential effects.
}

\revise{
In conclusion, our rule-based approach is a computationally efficient method for approximating fire temperature fields, enabling large-scale dataset generation to train predictive models. By combining ISO 834 fire curves with spatial considerations and embedding structural effects through parameter calibration, the method achieves a balanced trade-off between accuracy and scalability, making it a practical solution for thermal load modeling in fire scenarios. After generating the temperature of each beam or column according to the middle point, the temperature is applied as uniform thermal load to the elements of the structure in question using OpenSeesRT. 
}

% In conclusion, this rule-based approach is a computationally efficient method to approximate fire temperature fields, enabling large-scale dataset generation to train predictive models. By combining ISO 834 fire curves with spatial considerations, the method balances accuracy and scalability, making it a practical solution for thermal load modeling in fire scenarios.

% \subsection{Interstory Drift Ratio}
\subsection{OpenSeesRT Simulation}
\label{subsec:opensees_simulation}

The thermal and mechanical responses of 3D frame structures under combined fire and gravity loads are simulated using OpenSeesRT \cite{perez2024openseesrt}. \revise{In the simulation, the IDR of each node at $t_{\thre}$ is computed using the computed nodal displacements. Each structural model features six degrees of freedom per node (3 translational  and 3 rotational), with linear geometrical transformations (\texttt{geomTransf: Linear}) defining how the element local coordinate systems are mapped to the global coordinate system and assuming small displacements and rotations. Although OpenSeesRT allows a variety of options for modeling finite deformations, in the present simulations and mainly for simplicity, we did not consider large deformations. All bottom nodes (nodes on the ground) are fully constrained in all six degrees of freedom, while degrees of freedom os all other nodes are free.} Material behavior is temperature-dependent and modeled with \texttt{Steel01Thermal}, while fiber-based sections (\texttt{FiberThermal}) capture nonlinear interactions between thermal and mechanical responses at the cross-section level. \revise{Structural elements are represented as displacement-based Euler-Bernoulli beam-columns (\texttt{dispBeamColumnThermal}). This element  formulation accounts for thermal strains (temperature gradients) in the section, which is discretized into fibers. Numerical integration is used along the length of each element using three integration (Gauss) points, one at each end and the third in the middle of the element.}

{\revise{Thermal expansion of steel members plays a crucial role in IDR development. In reality, reinforced concrete floor slabs heat at a different rate than steel members due to their higher thermal mass and lower thermal conductivity. This differential heating can lead to restrained thermal expansion, introducing axial compression in beams and affecting the overall structural response. In this study, explicit {\em{composite action}} between steel members and concrete slabs is not modeled. Instead, our approach focuses on isolating the response of the steel structural frame, which is often the critical load-bearing component in fire scenarios. This assumption aligns with prior studies \cite{Possidente_2024} demonstrating that steel structures reach thermal equilibrium with surrounding gases quickly, allowing the use of uniform thermal loading in fire analysis. Future work could enhance this framework by incorporating slab-beam interaction effects, through a refined FEA for an extended dataset where constraints imposed by floor slabs are explicitly considered.}

The analysis begins with the application of gravity loads, followed by incremental thermal loads simulating the fire exposure. A static nonlinear solver using  \texttt{ExpressNewton} algorithm ensures convergence, while the \texttt{NormDispIncr} test maintains accuracy. An incremental \texttt{LoadControl} scheme with small step sizes is employed to guarantee numerical stability, using 10\% for gravity loads and 1\% for thermal loads. 

\revise{
In the thermal load analysis, uniform thermal load is applied to each beam or column, i.e., the temperature of each element is set to be that at the middle point, according to \secref{subsec:thermal_load_generation}. The \texttt{Steel01Thermal} material allows the properties (e.g., Young's modulus and yield strength) to be adjusted at increasing temperatures according to \cite{EN1993} using its Table 3.1: Reduction factors for the stress-strain relationship of carbon steel at elevated temperatures. For example, if the Young’s modulus at ambient temperature is $E_0$, then as the temperature ($T$) increases, the modulus changes as $E(T) = \eta (T) \times E_0$. \cite{EN1993} directly provides the values of $\eta(T) \in \left[0,1\right] $ at every $100 ^\circ\mathrm{C}$ interval and recommends using linear interpolation to obtain $\eta(T)$ for intermediate values of $T$.
} OpenSeesRT documentation \cite{OpenSeesThermalExamples} provides several examples of thermal analyses.

This modeling framework accommodates variations in material properties, cross-sectional geometries, and temperature profiles, providing robust simulations of structural behavior under fire conditions. The primary settings and configurations for the OpenSeesRT simulations are summarized in \tabref{tab:ops_detail}.
\begin{table}[h!]
    \centering
        \caption{Key settings of OpenSeesRT simulations.}
    \begin{tabular}{l|>{\raggedright\arraybackslash}p{0.6\linewidth}} %
    \toprule
    Modeling Aspect     & Details \\
    \midrule
    Geometry            & 3D models; 6 degrees of freedom per node \\
    Transformation      & geomTransf: Linear \\ 
    Material            & Steel01Thermal \\
    Section             & FiberThermal; Cross-section: $0.1$ m $\times$ $0.1$ m \\ 
    Element type        & {dispBeamColumnThermal} \\ 
    Loading             & Gravity loads: {beamUniform}; Thermal loads: {beamThermal} \\
    Integration scheme  & Incremental {LoadControl}; Step size: $10\%$ (gravity analysis), $1\%$ (thermal analysis) \\
    Nonlinear solver    & {ExpressNewton} algorithm; {UmfPack} solver; Convergence test: {NormDispIncr} tolerance: $10^{-8}$; Maximum \# iterations per step: $1000$. \\ 
    \bottomrule
    \end{tabular}
    \label{tab:ops_detail}
\end{table}

For each structure in the labeled dataset, 30 fire points are selected using a dual-granularity approach, \revise{i.e., two-stage sampling strategy,} to ensure they are well-distributed. Specifically, rooms are sequentially selected, with one fire point randomly chosen within each selected room. If a building is large and contains more than 30 rooms, we randomly select 30 rooms without replacement, i.e., ensuring that no more than one fire point is located in the same room. Conversely, if the building is small and has fewer than 30 rooms, all rooms are initially selected, with one fire point randomly assigned to each room. Additionally, rooms are then selected with replacement until a total of 30 fire points are assigned. \revise{The room-level sampling prioritizes selecting distinct rooms to avoid spatial clustering of fire points, while the point-level sampling ensures intra-room variability. This approach aligns with stratified sampling principles commonly used for efficient spatial representation, where multi-stage sampling strategies optimize coverage and variability, e.g., \cite{arunachalam_generalized_2023}, and enables a more comprehensive characterizing of how the structures respond under fire conditions.}
% This selection method prevents fire points from clustering too closely while maintaining an element of randomness. By distributing fire points in this manner, the 30 fire scenarios are effectively utilized, enabling a more comprehensive characterizing of how the structures respond under fire conditions.

\subsection{Summary of the Dataset Generation}
As discussed in this section and related to  \figref{fig:dataset_generation_procedure}, three key steps were considered in the development of the dataset: 
\begin{enumerate}
    \item {\bf{Filtering process}}: Structures with MIDR exceeding $1\%$ under gravity loads were excluded,  resulting in $1,573$ labeled structures retained for fire simulation and $16,050$ unlabeled structures for training the MFSP predictor.
    \item {\bf{Fire simulations}}: For each retained labeled structure, 30 fire scenarios were simulated using OpenSeesRT, yielding $47,190$ fire cases.
    \item {\bf{Data distribution check}}: MIDR distributions for labeled and unlabeled data under gravity loads were highly similar, because both datasets were generated using the same method. Under fire conditions, the MIDR distribution shifted, reflecting significant structural deformation with values reaching a maximum of about 6\%, an average of 1.70\%, and a standard deviation of 1.12\%. This step ensured a diverse and comprehensive dataset for the proposed predictive framework.
\end{enumerate}
The statistical distribution histograms for MIDR (after applying the $1\%$ filtering threshold \revise{for gravity load responses}) under different loading conditions are plotted in \figref{fig:histogram_mdr}. Figures \ref{fig:histogram_mdr}(a) and \ref{fig:histogram_mdr}(b) show the MIDR distributions of the labeled and unlabeled data, respectively, under gravity loads only. \figref{fig:histogram_mdr}(c) shows the MIDR distribution of the labeled data under the combined effects of gravity and fire loads. Fire load causes the structures to significantly deform, leading to a noticeably \revise{right-skewed} MIDR distribution.

\begin{figure*}[h!]
    \centering
    \includegraphics[width=\linewidth]{figures/histogram_mdr.pdf}
    \caption{Histograms of MIDR for labeled and unlabeled structures with gravity loads and fire cases.}
    \label{fig:histogram_mdr}
\end{figure*}

\revise{
This dataset provides the basis for training and testing the performance of the GNN-based framework. Although we employed a simplified rule-based thermal load generation method compared with conventional CFD-based simulations, the temperature field, the changes of the material properties, and the response of the structures, are all still highly nonlinear and complex. Therefore, it is still a challenging task for the NN to predict the MIDRs based on this dataset.
}
\subsection{Dataset Statistics}
\label{app:statistics}
In line with most GC studies, we utilize seven datasets in total: five transductive datasets—\textit{Citeseer}, \textit{Cora} \citep{kipf2016semi}, Pubmed \citep{namata2012query}, \textit{Ogbn-arxiv}, and \textit{Ogbn-products} \citep{hu2020open}—and two inductive datasets, \textit{Flickr} and \textit{Reddit} \citep{zeng2019graphsaint}. Each graph is randomly split, ensuring a consistent class distribution. The details of the datasets statistics are shown in Table \ref{tab:statistics}. We list all evolving information in Table~\ref{tab:split_reduction}, rows above the midline correspond to smaller datasets, and rows below it correspond to larger ones.
Reduction rate $r$ is defined as (\#nodes in synthetic set)/(\#nodes in training set) while $r_w$ is (\#nodes in synthetic set)/(\#nodes of whole graph visible in training stage). The whole graph visible in training stage means the full graph dataset for transductive setting but only the training graph for inductive setting.
\begin{table}[ht!]
\caption{Split and reduction rate information. The ``\# Train Nodes'' and ``\# Syn Nodes'' columns denote the number of newly added training nodes and synthetic nodes at each time step, respectively.}
\label{tab:split_reduction}
\resizebox{0.47\textwidth}{!}{
\begin{tabular}{lrrrr}
\toprule
\textbf{Dataset} & 
\textbf{\# Train Nodes} & 
\textbf{\# Syn Nodes} & 
\textbf{$r$ (Train)} & 
\textbf{$r_w$ (Whole)} \\
\midrule
\textit{Citeseer}      & 24             & 12  & 0.5    & 1.80  \\
\textit{Cora}          & 28             & 14  & 0.5    & 2.60  \\
\textit{Pubmed}        & 12             & 6   & 0.5    & 0.15  \\
\midrule
\textit{Flickr}        & ~8{,}920 & 90  & 0.01   & 1.00  \\
\textit{Ogbn-arxiv}    & ~18{,}190 & 182 & 0.01   & 0.50  \\
\textit{Ogbn-products} & ~39{,}330 & 394 & 0.01   & 0.08  \\
\textit{Reddit}        & ~30{,}790 & 31  & 0.001  & 0.10  \\
\bottomrule
\end{tabular}}
\end{table}

\subsection{Platform and Hardware Information}
To efficiently execute the clustering algorithm, we run it on Intel(R) Xeon(R) Platinum 8260 CPUs @ 2.40GHz using NumPy~\cite{numpy}, while the downstream GNN evaluations are conducted on a cluster equipped with a mix of Tesla A100 40GB/V100 32GB GPUs for large datasets and K80 12GB GPUs for small datasets. All GNN models are implemented using the PyG package~\cite{pyg}.

\subsection{Baselines Selection}
To establish a fair benchmark, we selected recent state-of-the-art GC methods that emphasize both effectiveness and efficiency. Some recent methods, such as MCond, CGC, and GCPA, were excluded due to the unavailability of their code at the time of paper writing. For the selected approaches, we chose the best representatives from each category: GCondX for gradient matching, GCDM and SimGC for distribution matching, and GEOM for trajectory matching. We implemented these methods using the latest GraphSlim package\footnote{\url{https://github.com/Emory-Melody/GraphSlim/tree/main}}, except for SimGC\footnote{\url{https://github.com/BangHonor/SimGC}} and GEOM\footnote{\url{https://github.com/NUS-HPC-AI-Lab/GEOM/tree/main}}, for which we used their original source code. We specifically included SimGC because it is the only model-based GC method that can run on Ogbn-products without requiring any modifications.


\subsection{Implementation Details for Variants of GCond}
As mentioned in Section~\ref{sec:intro} and illustrated in Table~\ref{tab:preliminary}, adapting GCond to an evolving setting is challenging. We employ the structure-free variant of GCond, i.e., GCondX, for easier adaptation, as designing a specific growth mechanism for the condensed graph is nontrivial and requires significant effort.
In addition, to manifests the convergence speed difference between GCond and GCond-Init, we implement an early stopping criterion with a patience of 3 during intermediate evaluations. 
If no improvement in validation performance is observed over 3 consecutive evaluations, the condensation process is terminated.
\subsection{Hyperparameters}\label{app:hyper}
Compared to existing work and benchmarks in GC, we perform a moderate hyperparameter search on validation set, as detailed in Section~\ref{sec:hyper}. The final results are presented in Table~\ref{tab:hyper}. During hyperparameter optimization (HPO), we observe that inheriting clustering centroids results in an approximate 1\% absolute performance drop for \textit{Flickr} and \textit{Ogbn-arxiv}. Therefore, we also treat the use of incremental $k$-means++ as a tunable hyperparameter. Additionally, some datasets do not perform well with a single hyperparameter configuration. To address this, we employ two distinct hyperparameter sets tuned on the first and last time steps, respectively, and select the better-performing one during the evolution process. These two sets are represented using a "/" separator and are indicated as "if Dual" when this technique is applied. For all baselines, we use the best hyperparameters reported in their respective papers, as implemented in GC4NC~\cite{gong2024gc4nc}.

The optimal hyperparameters also offer meaningful insights. \textbf{First}, during the early evolution stage, graphs exhibit higher heterophily compared to later stages. For example, on the \textit{Cora} dataset, $\alpha_1=-0.3$ in the early phase contrasts with $\alpha_1=0.9$ later. This pattern likely arises because, in the early stages of a graph, groups have not yet formed; links appear more randomly, making it challenging for nodes to link to similar counterparts.
\textbf{Second}, it is noteworthy that some datasets do not rely on second-hop information. This observation is contrary to previous studies~\cite{wu2019simplifying,luo_classic_2024} that recommend using at least 2-hop propagation. We conjecture that the representation clustering process itself acts as an additional step of feature propagation.
\textbf{Finally}, weight decay emerges as a critical factor for the performance of downstream models, suggesting that future work should pay closer attention to its optimization.
\begin{table*}[ht!]
\centering
\caption{The test accuracy of GC methods on various datasets.
"Non-Evolving" displays the test accuracy at the final time step (largest possible graph).
"Evolving" shows the average test accuracy over five time-steps.
Each result includes the mean accuracy $\pm$ standard deviation (Std.) from 10 runs. The "Whole" column refers to the results obtained by running standard GCN training and testing. "OOM" indicates an Out-of-Memory error during the computation. The best results are marked in \textbf{bold}. The runner-up results are \underline{underlined}.}
% \vskip -1em
\resizebox{\textwidth}{!}{%
\begin{tabular}{lc|ccc|cccccc|c}
\toprule
\textbf{Dataset} & \textbf{Setting} 
  & \textbf{Random} 
  & \textbf{Herding} 
  & \textbf{Kcenter} 
  & \textbf{GCondX} 
  & \textbf{GCond} 
  & \textbf{GCDM} 
  & \textbf{SimGC}
  & \textbf{GEOM} 
  & \textbf{GECC} 
  & \multicolumn{1}{c}{\textbf{Whole}} \\
  \midrule
\multirow{2}{*}{CiteSeer} 
  &Non-Evolving& 62.62$\pm$0.63 & 66.66$\pm$0.54 & 59.04$\pm$0.90 & 68.38$\pm$0.45 & 69.35$\pm$0.82 & 72.08$\pm$0.19 & 66.40$\pm$0.15 &  \textcolor{red}{\underline{73.03$\pm$0.31}} & \textcolor{red}{\textbf{73.25$\pm$0.15}} & 72.11 \\
  &Evolving& 50.65$\pm$1.55 & 53.47$\pm$0.98 & 47.99$\pm$1.81 & 50.85$\pm$3.00 & 60.51$\pm$0.86 & \textcolor{blue}{\underline{61.51$\pm$0.53}}& 57.42$\pm$0.21 & 58.95$\pm$0.67 & \textcolor{blue}{\textbf{65.48$\pm$0.76}} & 63.57 \\
\midrule
\multirow{2}{*}{Cora} 
  &Non-Evolving& 72.24$\pm$0.59 & 73.77$\pm$0.93 & 70.55$\pm$1.35 & 78.60$\pm$0.31 & 80.54$\pm$0.67 & 80.68$\pm$0.27 & 79.60$\pm$0.11 & \textcolor{red}{\underline{82.82$\pm$0.17}} & \textcolor{red}{\textbf{82.99$\pm$0.27}} & 81.23 \\
  &Evolving& 58.00$\pm$1.48 & 63.07$\pm$1.43 & 59.90$\pm$1.41 & 67.18$\pm$1.73 & \textcolor{blue}{\underline{77.14$\pm$0.55}} & 74.54$\pm$0.59 & 64.42$\pm$0.19 & 72.56$\pm$0.88 & \textcolor{blue}{\textbf{77.36$\pm$0.41}} & 76.34 \\
\midrule
\multirow{2}{*}{Pubmed} 
  &Non-Evolving& 71.84$\pm$0.66 & 75.53$\pm$0.44 & 74.00$\pm$0.19 & 71.97$\pm$0.53 & 76.46$\pm$0.48 & 77.48$\pm$0.46 & 76.80$\pm$0.23 & \textcolor{red}{\underline{78.49$\pm$0.24}} & \textcolor{red}{\textbf{80.24$\pm$0.27}} & 78.65 \\
  &Evolving& 66.37$\pm$1.25 & 66.31$\pm$1.34 & 64.38$\pm$1.25 & 62.65$\pm$1.20 & 74.26$\pm$0.84 & \textcolor{blue}{\underline{74.49$\pm$0.56}} & 71.38$\pm$0.21 & 70.25$\pm$0.78 &  \textcolor{blue}{\textbf{76.74$\pm$0.27}} & 76.18 \\
\midrule
\multirow{2}{*}{Flickr} 
  &Non-Evolving& 44.68$\pm$0.55 & 45.12$\pm$0.39 & 43.53$\pm$0.59 & 46.58$\pm$0.14 & \textcolor{red}{\textbf{46.99$\pm$0.12}} & 45.88$\pm$0.10 & 41.01$\pm$0.23 & 46.13$\pm$0.22 & \textcolor{red}{\underline{46.63$\pm$0.23}} & 47.53 \\
  &Evolving& 44.70$\pm$0.46 & 44.66$\pm$0.43 & 44.33$\pm$0.49 & \textcolor{blue}{\underline{45.63$\pm$0.78}}& 45.52$\pm$0.49 & 44.98$\pm$0.34 & 41.94$\pm$0.22 & 45.43$\pm$0.39 &  \textcolor{blue}{\textbf{45.78$\pm$0.38}} & 46.97 \\
  \midrule
  \multirow{2}{*}{Ogbn-arxiv} 
  &Non-Evolving& 60.19$\pm$0.52 & 57.70$\pm$0.24 & 58.66$\pm$0.36 & 59.93$\pm$0.54 & 64.23$\pm$0.16 & 60.71$\pm$0.68 & 65.26$\pm$0.26 &  \textcolor{red}{\textbf{69.59$\pm$0.24}} & \textcolor{red}{\underline{66.71$\pm$0.10}} & 69.01 \\
  &Evolving& 56.04$\pm$0.67 & 57.57$\pm$0.48 & 56.21$\pm$0.73 & 60.73$\pm$0.53 & 62.50$\pm$0.36 & 59.98$\pm$0.48 & 64.97$\pm$0.20 & \textcolor{blue}{\textbf{66.30$\pm$0.39}} & \textcolor{blue}{\underline{65.42$\pm$0.14}} & 70.40 \\
\midrule
\multirow{2}{*}{Ogbn-products} 
  &Non-Evolving& 60.19$\pm$0.52 & 57.70$\pm$0.24 & 58.66$\pm$0.36 & OOM & OOM & OOM & \textcolor{red}{\underline{61.71$\pm$0.25}} & OOM & \textcolor{red}{\textbf{66.32$\pm$0.23}} & 73.40 \\
  &Evolving& 41.36$\pm$0.48 & 44.26$\pm$0.61 & 38.93$\pm$0.82 & OOM & OOM & OOM & \textcolor{blue}{\underline{61.93$\pm$0.20}} & OOM & \textcolor{blue}{\textbf{64.03$\pm$0.30}} & 73.88 \\
\midrule
\multirow{2}{*}{Reddit} 
  &Non-Evolving& 55.73$\pm$0.50 & 59.34$\pm$0.70 & 48.28$\pm$0.73 & 88.25$\pm$0.30 & 89.82$\pm$0.10 & 89.96$\pm$0.05 & 90.78$\pm$0.25 & \textcolor{red}{\underline{91.33$\pm$0.13}} & \textcolor{red}{\textbf{91.37$\pm$0.04}} & 93.70 \\
  &Evolving& 51.31$\pm$0.90 & 48.94$\pm$0.70 & 48.53$\pm$1.37 & 79.02$\pm$0.73 & 87.93$\pm$0.22 & 82.68$\pm$0.21 &\textcolor{blue}{\underline{89.85$\pm$0.25}} & 67.91$\pm$0.57 & \textcolor{blue}{\textbf{90.02$\pm$0.07}} & 93.92 \\
\bottomrule
\end{tabular}}
\label{tab:main_app}
\end{table*}
\begin{table*}[ht!]
\centering
\caption{\textbf{Average Runtime (seconds) Across Evolving Times.} The reported reduction time is rigorously computed by excluding the overhead of the data loading and evaluation processes.}
% \vskip -1em
\label{tab:time}
\resizebox{\textwidth}{!}{
\begin{tabular}{lccc|cccccc|c}
\toprule
\textbf{Dataset} & \textbf{Random} & \textbf{Herding} & \textbf{KCenter} & \textbf{GCondX}& \textbf{GCond}& \textbf{GCDM} & \textbf{SimGC}& \textbf{GEOM}& \textbf{GECC} & \textbf{Whole}\\
\midrule
\textit{Citeseer}
  & 0.04
  & 5.73
  & 5.84
  & 505.62
& 654.32
& 217.99
  & 1,680.02
& 1,362.40
& 1.65
& 3.98 \\

\textit{Cora}
  & 0.01
  & 4.20
  & 4.80
  & 331.53
& 1,190.65
& 142.82
  & 1,643.76
& 1,331.43
& 1.72
& 2.10 \\

\textit{Pubmed}
  & 0.02
  & 9.00
  & 7.18
  & 246.68
& 502.12
& 311.37
  & 1,654.23
& 995.21
& 1.42
& 5.76 \\ \midrule

\textit{Flickr}
  & 0.02
  & 11.53
  & 10.56
  & 609.98
& 1,446.76
& 353.51
  & 7,486.65
& 757.75
& 7.10
& 8.57 \\

\textit{Ogbn-arxiv}
  & 0.02
  & 14.36
  & 14.05
  & 2,895.06
& 6,076.18
& 686.12
  & 2,687.45
& 1,685.18
& 9.96
& 12.45 \\

\textit{Ogbn-products}
  & 0.02
  & 517.95
  & 513.36
  & OOM
& OOM
& OOM
  & 71,489.00
& OOM
& 146.82
& 542.61 \\

\textit{Reddit}
  & 0.02
  & 24.40
  & 24.84
  & 2,672.85& 6,130.46& 337.15
  & 6,610.70& 1,815.77& 4.91& 11.50 \\

\bottomrule
\end{tabular}}
\end{table*}
\begin{table*}[]
\centering
\resizebox{\textwidth}{!}{%
\begin{tabular}{@{}lcccccc@{}}
\toprule[1.5pt]
 & Model & Learning Rate  & Batch Size & KL Coefficient&Max Length & Training Epochs \\ 
\midrule[1pt]
& Llama-3.1-8B-Instruct & 5e-6  & 32 & 0.1&8000& 3\\
& Qwen2-7B-Instruct & 5e-6 & 32 & 0.1 &6000& 3 \\
& Qwen2.5-Math-7B & 5e-6  & 32 & 0.01&8000& 3 \\ 
\bottomrule[1.5pt]
\end{tabular}%
}
\caption{Model Training Hyperparameter Settings (SFT)}
\label{tab:hyper_sft}
\end{table*}

\begin{table*}[]
\centering
\resizebox{\textwidth}{!}{%
\begin{tabular}{@{}lccccccccc@{}}
\toprule[1.5pt]
 & Model & Learning Rate  & \makecell[c]{Training\\Batch Size} & \makecell[c]{Forward\\Batch Size} & KL Coefficient&Max Length & \makecell[c]{Sampling\\Temperature} &Clip Range &Training Steps \\ 
\midrule[1pt]
& Llama-3.1 &5e-7  & 64& 256 & 0.05&8000& 0.7&0.2&500\\
& Qwen2-7B-Instruct & 5e-7&  64& 256 & 0.05 &6000&0.7 &0.2&500\\\
& Qwen2.5-Math-7B & 5e-7 & 64& 256 & 0.01&8000&0.7 &0.2&500 \\ 
\bottomrule[1.5pt]
\end{tabular}%
}
\caption{Model Training Hyperparameter Settings (RL)}
\label{tab:hyper_rl}
\end{table*}

\section{Additional Results}
\subsection{Performance and Efficiency}
For simplicity, Table~\ref{tab:main} omits the standard error and running time of coreset selection methods. we provide the full results here.

Figure~\ref{fig:accuracy_vs_time} presents the accuracy vs. time trade-off for Reddit. For the remaining three large datasets, we provide the corresponding results in Figure~\ref{fig:accuracy_vs_time_large_vertical}. 
The results align with our main findings, further confirming that GECC surpasses the baselines in both efficiency and scalability. It consistently maintains stable performance while effectively managing computational resources throughout graph evolution. Notably, on the large-scale Ogbn-products dataset, which contains over one million nodes, most GC methods fail, whereas GECC remains robust and continues to operate successfully.


\begin{figure*}[htbp]
  \centering
  \begin{subfigure}[b]{0.85\linewidth}
    \centering
    \includegraphics[width=0.8\linewidth]{figs/ogbn-arxiv_time_vs_accuracy-cropped.pdf}
    % \caption{Flickr Caption}
    \label{fig:flickr}
  \end{subfigure}
  % \vspace{-1em}
  \begin{subfigure}[b]{0.85\linewidth}
    \centering
    \includegraphics[width=0.8\linewidth]{figs/Ogbn-products_time_vs_accuracy-cropped.pdf}
    % \caption{Ogbn-products Caption}
    \label{fig:ogbnproducts}
  \end{subfigure}
  % \vspace{-1em}
  \begin{subfigure}[b]{0.85\linewidth}
    \centering
    \includegraphics[width=0.8\linewidth]{figs/Reddit_time_vs_accuracy-cropped.pdf}
    % \caption{Reddit Caption}
    \label{fig:redproducts}
  \end{subfigure}
  % \vskip -2em
  \caption{Test accuracy vs. condensation time on large datasets (top-left is better).}
  \label{fig:accuracy_vs_time_large_vertical}
\end{figure*}


% \begin{figure*}[htbp]
%   \centering
%   \begin{subfigure}[b]{0.33\linewidth}
%     \centering
%     \includegraphics[width=\linewidth]{figs/ogbn-arxiv_time_vs_accuracy-cropped.pdf}
%     % \caption{Flickr Caption}
%     \label{fig:flickr}
%   \end{subfigure}%
%   \hfill
%   \begin{subfigure}[b]{0.33\linewidth}
%     \centering
%     \includegraphics[width=\linewidth]{figs/Ogbn-products_time_vs_accuracy-cropped.pdf}
%     % \caption{Ogbn-products Caption}
%     \label{fig:ogbnproducts}
%   \end{subfigure}%
%   \hfill
%   \begin{subfigure}[b]{0.33\linewidth}
%     \centering
%     \includegraphics[width=\linewidth]{figs/Reddit_time_vs_accuracy-cropped.pdf}
%     % \caption{}
%     \label{fig:redproducts}
%   \end{subfigure}
%   \vskip -2em
%   \caption{Test accuracy vs. condensation time on large datasets (top-left is better).}
%   \label{fig:accuracy_vs_time_large}
% \end{figure*}
\end{document}
\endinput
%%
%% End of file `sample-sigconf.tex'.
