\documentclass{article}


\usepackage{arxiv}

\usepackage[utf8]{inputenc} % allow utf-8 input
\usepackage[T1]{fontenc}    % use 8-bit T1 fonts
\usepackage{hyperref}       % hyperlinks
\usepackage{url}            % simple URL typesetting
\usepackage{booktabs}       % professional-quality tables
\usepackage{amsfonts}       % blackboard math symbols
\usepackage{nicefrac}       % compact symbols for 1/2, etc.
\usepackage{microtype}      % microtypography
\usepackage{lipsum}
\usepackage{caption}
\usepackage{appendix}
\usepackage{subcaption}
\usepackage{graphicx}
\usepackage{float}
\usepackage[dvipsnames]{xcolor}
\graphicspath{ {./images/} }
\usepackage{tocloft}

\title{Agentic LLM Framework for Adaptive Decision Discourse}


\author{
 Antoine Dolant \\
  Department of Civil and Environmental Engineering\\
  Grainger College of Engineering \\
  University of Illinois Urbana-Champaign\\
  Urbana, IL 61801 \\
  \texttt{adolant2@illinois.edu} \\
  %% examples of more authors
   \And
 Praveen Kumar \\
  Department of Civil and Environmental Engineering\\
  Grainger College of Engineering \\
  Prairie Research Institute \\
  University of Illinois Urbana-Champaign\\
  Urbana, IL 61801 \\
  \texttt{kumar1@illinois.edu} \\
}

\hypersetup{
 colorlinks=true,
 citecolor=NavyBlue,
 linkcolor=BrickRed}

\begin{document}
\maketitle
\begin{abstract}
Effective decision-making in complex systems requires synthesizing diverse perspectives to address multifaceted challenges under uncertainty. This study introduces a real-world inspired agentic Large Language Models (LLMs) framework, to simulate and enhance decision discourse—the deliberative process through which actionable strategies are collaboratively developed. Unlike traditional decision-support tools, the framework emphasizes dialogue, trade-off exploration, and the emergent synergies generated by interactions among agents embodying distinct personas. These personas simulate diverse stakeholder roles, each bringing unique priorities, expertise, and value-driven reasoning to the table. The framework incorporates adaptive and self-governing mechanisms, enabling agents to dynamically summon additional expertise and refine their assembly to address evolving challenges. This perspective is rooted in information theoretic understanding, and evaluates the interplay of unique, redundant, and synergistic knowledge, highlighting how these elements contribute to actionable insights. An illustrative hypothetical example focused on extreme flooding in a Midwestern township demonstrates the framework’s ability to navigate uncertainty, balance competing priorities, and propose mitigation and adaptation strategies by considering social, economic, and environmental dimensions. Results reveal how the breadth-first exploration of alternatives fosters robust and equitable recommendation pathways.  This framework transforms how decisions are approached in high-stakes scenarios and can be incorporated in digital environments. It not only augments decision-makers’ capacity to tackle complexity but also sets a foundation for scalable and context-aware AI-driven recommendations. This research explores novel and alternate routes leveraging agentic LLMs for adaptive, collaborative, and equitable recommendation processes, with implications across domains where uncertainty and complexity converge.
\end{abstract}


\section{A Case for LLMs in Human Discourse}
\label{titles:introcomplexity}

Human decision-making unfolds within a landscape marked by uncertainty and risk, complexity, and competing priorities. This is particularly true in scenarios requiring immediate yet informed responses to multifaceted challenges, where the stakes encompass social, economic, and environmental dimensions. Natural hazard scenarios, such as those posed by extreme weather events, underscore this challenge. Climate-driven disasters like floods, wildfires, and hurricanes are shaped by dynamic, coupled human-natural systems (CHANS). These systems involve a multitude of interacting biophysical and social processes that can amplify cascading consequences over varying spatial and temporal scales. Decision-making in such contexts must account for resource limitations, socio-economic vulnerabilities, and the inherent unpredictability of impending events \cite{clarke2018}. In this context, decision support is described as “the intersection of data provision, expert knowledge, and human decision making at a range of scales from the individual to the organization and institution” \cite{jones2014}. Whether addressing natural disasters, public health crises, or systemic societal inequities, decisions often depend on limited information, interdisciplinary expertise, and the need to weigh short- and long-term consequences. Despite decades of advancements in modeling and decision-support systems, the human dimension of decision making, the intuitive, value-driven processes of individuals and communities, remains difficult to quantify and incorporate into actionable frameworks.

Existing decision frameworks often rely on mathematical and physical models that inadequately capture the full spectrum of the multi-dimensional complexity of the problem at hand. Existing examples from behavioral economics include the beta-delta model \cite{laibson1997}, disinterested distributional preferences \cite{adler2000} and loss aversion model \cite{tversky1991}, all of which offer mathematical representation of aspects of human preferences in making decisions. These approaches have proven their value with regards to purchasing, selling and simple decision-making modeling, yet these models are too coarse to fully capture the intricacy of complex problems such as preparedness to uncertain natural hazards. They often fail to fold the human dimension into state of the  art approaches as a result of its inherent complexity, which is hardly expressed by mathematical- and physical-based models. Such limitations partially explain the under-representation of human decision-making in mitigation and adaptation efforts to extreme events assessment. Notably, social equity and justice has been correlated to socio-economic principles and considered as a major challenge in mitigation for decades, yet was only truly considered recently \cite{ikeme2003}. Nonetheless, numerous interacting processes contribute to CHANS’s complex functioning, hampering the ability to address the challenges associated with strategic planning, preparedness and disaster recovery \cite{townend2023, mach2023}.

In this study, we propose and evaluate a multi-agent framework powered by Large Language Models (LLMs), designed to simulate and enhance decision-making processes. Through this approach, we aim to organically develop the interactive potential of LLMs as a tool for discourse and synthesis to support decision making, and pave the way towards a digital twin of human discourse. This framework not only augments the capacity of decision-makers to navigate complexity by providing arguments and options to consider by drawing upon prior knowledge, knowhow and outcomes, but also lays a foundation for scalable, adaptive, and context-aware recommendations to emergent challenges. Our findings highlight the potential of LLM-based multi-agent systems to produce actionable strategies, uncover emergent synergies, and highlight adaptive decision pathways to evolving risks in integrative applications.

This manuscript is organized as follows: we present the foundation for a multi-agent LLM framework built to capture the interactions of decision-makers during the process of challenge assessment. Motivations for such a structural design are detailed and supported by information theoretic thinking. Then we present an assessment of conversation mechanisms and their implementation into LLM-powered agents augmented with prompt engineering techniques, as well as the requirements for the operational framework to support discourse between agents.
Using a hypothetical extreme flooding event as an illustrative example, we demonstrate the framework’s capability to navigate competing priorities, integrate transdisciplinary insights and evaluate adaptive strategies. We highlight a dynamic relationship between evaluated strategies and uncertainty.
The results underscore the capabilities of such a framework to serve as a digital twin for human decision discourse, fostering scalable and context-aware recommendations.


\section{Multi-Agent Framework for Decision Discourse}

Transformer-class models were introduced in 2017 \cite{vaswani2017} and a few years forward saw the birth of ChatGPT, a conversational LLM application that is still the fastest-growing consumer software application ever produced \cite{wikipedia1}. LLMs are trained on massive text corpora that concentrate human knowledge to perform next-word prediction and iteratively generate an infinite amount of text. They have proven to outperform any type of Machine Learning or non-Machine Learning model in natural language processing (NLP) tasks \cite{rostam2024}. Common LLM usage includes question and answer, summarizing, translation, creative writing, and contextual and semantic tasks such as sentiment analysis, for which they secured the top of the leader board  \cite{niimi2024}. Furthermore, LLMs have shown potential for causal discovery and observational causality inference, although the full extent of their capacity to perform advanced causal analysis is an active discussion \cite{jin2023, jin2024, kiciman2023, zevcevic2023}. The exhaustiveness of the training dataset enables them to simulate nuanced discourse, evaluate hypothetical scenarios, and dynamically adapt to contextual shifts. Unlike traditional decision-making tools, LLMs can engage with the rich interplay of human perspectives, integrating expertise across disciplines and negotiating trade-offs through collaborative, human-like conversations. As such, they represent a promising candidate for addressing the intricacies of decision-making under uncertainty. These precise characteristics have brought LLMs to the foreground of human-like conversational models as their capabilities encapsulate nuanced opinions and approaches, hypothetical situation evaluation, and dynamic adaptation to situational contexts. Notwithstanding the paradigm shift from monolithic infrastructures to distributed systems that the world has experienced in the past decades \cite{mosleh2018}, LLMs are trained and built as unique and enormous single instance models. Such a design contrasts with the interconnectedness of multi-component systems observed in the world, such as CHANS.

Recent years have seen a rapidly growing literature including decision-making focused LLM \cite{zhao2023, hao2023, zhou2024}, agentic LLM frameworks \cite{dong2024, campedelli2024, chang2024socra, zhou2024, talebirad2023}, and agentic conversational frameworks \cite{park2023, chan2023, chang2024evince, cho2024, estornell2024}. In this landscape of rapidly evolving models, our focus is on the implementation of discourse mechanisms involving multiple agents in the context of complex human-like decision-making under uncertainty. Our overall framework operates as a simulation, beginning with a scenario prompt that sets the context for discussion (see Fig.\ \ref{fig:thiswork}). This simulation proceeds through iterative discourse, with agents exchanging ideas, challenging assumptions, and refining strategies based on feedback. Agents within our proposed framework adopt specialized personas - each representing distinct stakeholder roles and priorities - and can summon additional expertise if knowledge gaps are identified. This self-governance capability ensures that the framework remains adaptive and context-aware. The last iteration is marked by the synthesis of results and summary analysis, stating evaluated strategies, their feasibility and plausibility, and advantages and drawbacks. To demonstrate applicability, we employ a hypothetical case study of extreme flooding in a Midwestern U.S. township. This scenario provides a compelling testbed, reflecting the complex trade-offs and urgent decisions that characterize real-world climate-driven events.

In this section we present the foundations of this conceptual framework, develop the different components of the framework, and argue for the necessity for self-governance.

\begin{figure*}[ht]
\centering
\includegraphics[width=\textwidth]{figs/fig_article.png}
\caption{Illustration of the multi-agent decision discourse workflow.   First (A),  a natural disaster threat is selected to generate a scenario that will act as a bootstrap to LLM agents.  Second (B),
initial persona prompts are specified or generated for the starting assembly (for example, composed of a mayor, community advocate, environmental scientist, and moderator for an extreme flooding scenario).  With those two necessary conditions, the assembly (C) starts and iterates for a specified number of runs or until convergence. Each iteration consists of an agent continuing the discussion at time state \textit{t} from the conversation at time state \textit{t-1} as an input. Agents can bring in new participants (D) if they judge that there is an imbalance in the skills and knowledge domains represented within the assembly. Each  execution results in the generation of a conversation output (E), from which can be extracted recommended courses of action with their advantages and drawbacks. Actionable measures are colored here as green or red (F) depending on their positive or negative impact on resilience to the situation. The space of possible outcomes represents all realizations of the different action pathways, and can be enriched with additional executions  with slightly varying inputs, evaluating scenarios that are causally defined through counterfactuals.
}
\label{fig:thiswork}
\end{figure*}


\subsection{Agentic Foundation for Decision Discourse}
\label{titles:conceptualfoundation}

Following the principles described in the preceding section, recent work \cite{talebirad2023} introduces a general framework design that supervises LLM-based agents aspiring to “enhance the performance and capabilities of LLMs by leveraging the power of multi-agent systems”. This top-down initiative aims to pave the way for the creation of more powerful Artificial General Intelligence (AGI) models \cite{talebirad2023}. However, top-down approaches start from general behavioral structures down to specialized ones. In the context of human-like discourse, we posit that specified general behaviors often fail to capture the richness of emergent viewpoints arising through interactions across different expertise, leading to oversimplified recommendations that may overlook critical trade-offs and synergies. Incidentally, such initiatives underperform in applications such as complex decision making under uncertainty \cite{mosleh2018}.

In contrast, we develop a bottom-up approach that starts from specialized individual profiles (or personas) which lead to emergent discourse through inter-agent interactions.  We model agents as distinct personas, each representing a specific stakeholder role (elected officials, domain specialists, etc.) that possess distinctive deeper values such as risk awareness, tolerance, and societal equity considerations. Such a persona characterization entails creating  restrictions of the knowledge space in the LLM to fit the specified behavior. This reflects our argument that  monolithic LLMs are limited in simulating simultaneous diverse behaviors, while a collection of individually specialized agents can support multiple interacting behaviors. Our framework draws on principles of systems thinking to organically construct a core framework inspired by real-world societal structures, emphasizing the interconnectedness of components and the emergent properties of their interactions.

The adoption of such an architecture is motivated by real-world examples of decision discourse and synthesis. Decisions result from the interactions of multiple individuals such as decision-makers, scientists, and community advocates; in our example collaborating to provide risk assessment and response to extreme flood hazard. The diversity of characters in this assembly is purposed to dynamically explore recommendation pathways pertaining to different aspects of the challenge, representative of different expertise. Such aspects include the safety of the population with considerations regarding social equity, integrity of vital infrastructures, and ecological resilience of the surrounding environment, defined as the tolerance threshold for a system to remain functionally unaltered by exterior perturbations \cite{holling1973}.

\subsubsection{Structural Design}
\label{titles:structuraldesign}

We now expand the concepts developed into specifications that support human-like agentic LLM decision making. Drawing on real-world use cases, agents need a space to meet, debate and converge to a series of recommendations. Inter-agent communication requires such an environment on the grounds that LLMs are currently deployed in monolithic instances that do not support interactive multi-agent designs. Our virtual conference room gathers the mechanisms allowing inter-agent communication, and is referred to as the macro-scale component, for it orchestrates the LLM agents, defined as micro-scale components.  Good recommendations “tend to emerge from processes in which people are explicit about their goals; consider a range of alternative options for pursuing their goals; use the best available science to understand the potential consequences of their actions; carefully consider the trade-offs; and contemplate the decision from a wide range of views and vantages” \cite{jones2014}. Recommendations are defined as emergent products of discourse and careful peer evaluation involving consensus. The virtual conference room representing agentic LLM interactions is necessary for such  emergence, as monolithic LLMs cannot offer the breadth and richness of such interactions.

%We now develop agent design, conference room design, criteria for convergence, and information theoretic perspectives on LLM knowledge.

\subsubsection{Agent Design}
\label{titles:agentdesign}

Similar to their human counterparts, the key capability necessary for LLM agents to interact is a dialogue mechanism. A typical iteration of dialogue consists of a listening phase, a sensible answer construction, and an expressive phase. Using computer science vocabulary, this translates to input, processing, and output, respectively attributed to sensors, internal logic and actuators in agentic systems theory \cite{holland1995}. With considerations to LLM agents, the parallel can be drawn between LLM's encoded language function and the sensible answer construction. Hence, prompts can be assimilated to listening, and the output to the expression. The proximity between LLMs' structure and human discourse is partially caused by the compressed representations of our world that are embedded in LLMs’ inner structures (Section \ref{titles:interconnectedness}).
Next, in order to carry out a conversation, agents need memory to remember past interactions. Conversational versions of the most popular LLMs (ChatGPT, LLaMa, etc.) integrate that ability at the expense of multi-agent capabilities. In other words, at present LLMs are either available as conversational agents in their monolithic form, or as an application programming interface (API) that enables the integration in more flexible applications. However the API form of LLM is stateless, leaving the handling of conversation memory to end-users. To achieve this in an agentic LLM framework, we store the conversation at the macro-scale level instead of implementing memory handling in every agent \cite{park2023}. Because the whole conversation needs to be fed to agents at every iteration (LLM available through APIs are stateless), this choice allows for optimizing the overhead of data exchange between agents.

There are more micro-scale speech mechanisms that could be refined in agent’s design such as contradiction, agreement and reflection. Such mechanisms are not explicitly required from agents in this first exploration, as they are organically embedded in human discourse. However, further implementations could consider these, striving for more realistic human-like behavior.

\subsubsection{Orchestration Mechanisms}
\label{titles:orchestration}

Based on the description portraying individuals gathered in a room, peer communication is identified as the minimum required mechanism to support discussion.  In addition to peer to peer communication, which is composed of a listening phase, a sensible answer construction phase and an expressive phase, there is need to define the order in which individuals speak, and also to whom they speak, together referred to as orchestration. A trivial solution includes randomly drawing a speaker’s turn. However, our early experimentation of this framework have shown inconsistency in the answer quality and discourse coherence. For this mechanism to be realistic and relevant, there is need for implementation at the micro-scale. This turn-based communication articulates around specific agent instructions and conference room level extraction of the selected next speaker based on agent response. To this end, agents are instructed to address a specific interlocutor in the assembly, or issue a general statement that addresses everyone and will trigger the random choice of a next speaker. Content extraction is performed after every agent response, by a stateless and contextless agent entrusted to extract important features from the responses: to which agent the message is addressed, content of the message, and additional properties, such as the persona of additional agents that may be summoned to complement the expertise defined for current group of agents (see Section \ref{titles:selfgovernanceconvergence}).

While we allow for maximum expressivity of the agents so that they can introduce new topics, revisit unresolved issues, and allow the conversation to organically adapt to emerging priorities, LLM agents have shown inconsistencies in keeping the focus on the initial topic. This limitation is partially explained by the imperfection of the data retrieval mechanism (see Section  \ref{titles:selfgovernanceconvergence}), and can be addressed at the macro-scale with the implementation of continuous monitoring of responses and adjustment of prompts. Consistent with our organic approach to discourse design, refocusing and moderation is implemented at the micro-scale with the introduction of an agent to serve as the discourse moderator (or facilitator). The moderator's behavior aims to instruct the other participants to refocus on the topic if there are signs of digression, while avoiding to take part in the discussion. With this goal in mind, an analysis of the discussion is periodically conducted by the moderator agent, who summarizes the major points already addressed. If refocusing is not necessary, the analysis is logged but will not be inserted in the discussion.


\subsection{Self-Governance and Convergence}
\label{titles:selfgovernanceconvergence}

Having defined the discourse mechanisms at the micro- and the macro-scale, we now move on to emergent behaviors of the framework entailing the definition of a convergence criterion. We describe the limitations of persona prompt patterns and formulate a technique for summoning new agents that allows to dynamically build an assembly, specialized for the resolution of a specific challenge. 

%\subsubsection{Limitations of Persona Prompts}
% \label{titles:limitationspersona}

Persona prompt patterns (defined in \ref{titles:promptengineering}) represent powerful methods for defining LLM agent’s behavioral space, and are central to the functioning of this work. The testbed for this framework articulates around an extreme flood hazard assessment and mitigation scenario, where agents emulate decision-makers discussing towards the resolution of such an event, hinting that realistic human behavior is key to the framework’s performance. First explorations were conducted with handcrafted persona prompts with consideration of a fixed-sized assembly, including a mayor, a community advocate and an environmental scientist. Character design aims for specific behavior traits and restrictions pertaining to the role: for instance, the mayor is assumed as not having expert knowledge on environmental sciences, and will optimize for constraints pertaining to their standing in the assembly, including infrastructure integrity and safety of the population. As a consequence, individual assembly members optimize for a unique constraint, where the overall assembly optimizes for the union of non-redundant constraints, maximizing agent’s utility.

Despite its potential, the persona prompt technique is imperfect and prompts can show variability in the number of executed directives. This technique’s efficiency is also affected by surrounding factors such as conversation size, level of detail and ordering of the instructions. As a result, first explorations showed a large variability in the scope of the knowledge of the mayor, community advocate, and environmental scientist. This entails a significant overlap between their respective domain of expertise, and a poor added value of member interactions. Refined prompt definitions aim at more precise directives, optimizing for the enhancement of agent’s unique contribution to discourse, at the expense of a reduced scope of knowledge. As a result, the global scope of knowledge of the assembly is reduced and the minimal skill set necessary for the resolution of the challenge is not achieved, highlighting the need for more specialized agents to fill in the gaps. Handcrafting a new persona that is relevant to the decision-problem at hand, and offering a different expertise relative to the initial assembly members is then required. Along this path, there is evidence for the need of some form of automation as the human bandwidth necessary to produce persona prompts while maintaining or increasing the overall efficiency in decision discourse of the framework can saturate quickly. Moreover, the expressivity of LLMs suggests an increased suitability of LLM-issued persona prompts compared to handcrafted ones, hinting for the need for an iteration pattern to establish a self-governed.

%\subsubsection{Self-Governance and Convergence}

This self-governed framework pattern articulates as follows: there is need for a summoning and removal mechanism of agents, that supports dynamic evolution of the agent pool towards the goal of reducing shared knowledge and increasing unique specialized knowledge. With this mechanism, the variety of the domains of expertise represented dynamically increases as new agents are summoned, while maximizing agent’s utility with more specialized persona definitions. Consequently, a new feature is available for agents, allowing them to craft and summon a new agent, reflecting a complementary persona, in the conference room. Agents formulate a short description of the persona to summon, extracted in the additional processing phase (Section \ref{titles:orchestration}). The framework generates a persona prompt from the agent-given description, and creates another agent entity that is registered in the network and virtually introduced in the conference room. This self-governance mechanism provides a new performance and evaluation metric, defined from an aggregate of framework variables, such as the size of the assembly, the different agent roles and the participation rate of every agent. This contrasts with a more straightforward convergence criterion, built from the quality, relevance and breadth of recommendations. Hence, convergence can include stability of the constitution of the assembly, a defined set of domain expertise represented, or a minimal assembly that satisfies a relevant assessment of the decision context while maximizing unique agent utility.


\subsection{Decision Generation and Evaluation}
\label{titles:decision}

We mention in section \ref{titles:selfgovernanceconvergence} that the quality, relevance, and breadth of framework-made recommendations is of importance for performance evaluation. In addition, the stability and consistency of generated actionable measures can also constitute a convergence criterion. However, complications in decision-making arise from the unexpectedness exerted by uncertainty. Taleb \cite{taleb2007} refers to unpredictability as “outside the realm of expectation,” which also carries a semantically strong attribute of unexpectedness. Unexpectedness is anchored in human perception owing to the availability bias, defined by behavioral sciences as an imprecise evaluation of risks and event likelihood based on the scope of available information \cite{lieder2018}. In other words, humans imagine and conceive risks and events that they experience or are knowledgeable about, at the involuntary expense of unlikelier ones. For instance, well established examples from behavioral sciences include the misconception that suicides are less frequent than homicides as a result of a lower media representation.

With consideration to the changing patterns in the occurrence and strength of extreme natural hazards, the absence of historical records crystallizes an availability bias to which human beings are subject. This causes humans to sometimes misevaluate the risk of extreme events happening, resulting in hindered preparedness to such events. Generative AI then represents a compelling method to evaluate alternate pathways, as it does not suffer from the same limitations than randomized controlled trials (resource cost, human bandwidth, ethical considerations). Indeed we mention in section \ref{titles:interconnectedness} that LLMs show characteristics of world models, entailing a favorable environment for agents to perform experiments as they’re provided with simulated sensors and actuators. In addition to the horizontal scalability offered by multi-agent architectures, this makes our framework compelling for the exploration of a wide breadth of actionable pathways in face of a complex decision-making problem. Traditional methods rely on physical modeling whose performance is hampered by the nonstationarity of modeled processes, whereas agentic LLMs continuously adapt through in-context learning and counterfactual evaluation through variations in the input scenario. Hence, agentic LLM offers novel perspectives based on breadth-first exploration of alternatives mirroring real-world decision making.

% The flexibility offered by LLM’s expressivity through natural language, towards the goal of defining optimization constraints as characteristics and personality traits in generated persona. Indeed, the personal and professional objectives, and values and aspirations of generated persona organically shape optimization constraints fostering a closer representation of real-world complex decision needs.

LLMs exhibit great expressivity through natural language, that offer favorable grounds for defining task optimization constraints with the use of persona prompts. Indeed, the personal and professional objectives, and values and aspirations of generated persona organically shape optimization constraints fostering a closer representation of real-world complex decision needs. These include the practicality and resource limitations of implementing strategies, the capacity for generated strategies to improve ecological and engineering resilience (formally defined using dynamical systems theory \cite{srinivasan2015}), and the inclusion of social and economic justice in the distribution of benefits and burdens resulting from generated strategies. Specifically, the community advocate LLM agent solely aims to organically address the underrepresentation of social justice and equity in the face of extreme events mitigation \cite{marino2023}. On the other hand, the agent impersonating the mayor must think about the structural and economical integrity of buildings and facilities, vital to the rapid recovery of the system after a shock. In this regard, our framework mirrors real-world decision-making where outcomes must balance technical feasibility with broader societal impacts, while accounting for uneven distribution of resources and burdens.


\section{Application}

\label{titles:methods}

We now develop the use case serving as a testbed for this framework through a description of the input scenario, description of the initial assembly composition of the agents and provide analysis and interpretation of generated resolution pathways (Fig. \ref{fig:result1} and Fig. \ref{fig:result2})
along with broader implications of such a framework for decision-making challenges characterized by complexity and uncertainty.

\subsection{Scenario}
\label{titles:scenario}
As mentioned earlier, the scenario that drives our testing of the framework revolves around low-expectancy, low-probability, and high-risk weather events. The following text describes the hypothetical extreme rain event situation, the context in which it occurs, and serves as a bootstrap prompt for LLM agents interaction:

"\textit{We are in a US Midwestern township of half a million inhabitants. A large river flows through this township which is also fed by a large watershed in which the township is located. There is forecast for very heavy rain and possibility of flooding at large scale. Consider that the probability of flooding is \textless probability parameter\textgreater. The township needs to make anticipatory decisions to respond to the impending event to minimize the impact of floods but also keep in mind the needs of the community which relies on the river water. A reservoir downstream of the town supports potable water needs, provides energy, and recreational needs. The river also supports navigational and commercial traffic. Decisions must address the management of reservoir levels. At the end of the conversation, a strategy needs to be developed to manage the outcome of the flood keeping in mind uncertainty of the event, noting that different probabilities of risks may justify different approaches, and the associated advantages and drawbacks of the decision variables.}"

This scenario was built and inspired following some real-world challenges highlighted by foundational climate assessment reports \cite{mach2023, burkett2014, jones2014, ipcc2023, marino2023, clarke2018} and translated to optimization constraints in the presented scenario. Examples include preservation of vital services (energy, drinking water), operation of commercial routes and safety of the population. 

\subsection{Initial Pool of Agent Persona}
\subsubsection{Local Government Representation}
\label{titles:mayor}

The first agent personality that was designed in this framework is the mayor of the virtual township. The mayor represents a stakeholder or decision maker who is trusted with the final recommendations. The mayor is instructed to rely on personal knowledge, that does not include any particular environmental science priors. Additionally, this agent’s design includes a directive to consider consequences of the recommendations made. Last, the possibility for this agent to disagree with its peers and challenge their points is specified, primarily as an initiative to steer the LLM away from its agreeing and compliant baseline behavior which is restrictive in this use case. This last characteristic is further addressed in the discussion section.

\subsubsection{Environmental Science Representation}
\label{titles:scientist}
This agent is described as an urban and environmental engineer, designed to bring a scientific aspect to the discourse and provide sources and knowledgeable information. Therefore, there is future work opportunities of using such a persona to integrate advanced features such as fact verification through content extraction from research material, leveraging research database connectors such as the Langchain arXiV plugin, or other novel sourced web search engines embedded in the latest generation of LLMs. Similarly to the first role presented, this agent is also encouraged to disagree with its peers and challenge their arguments, if needed. Last, we seek to improve this agent’s reasoning capacity to reinforce its relevance as a science representative, using prompt engineering. Chain of Thoughts (CoT) is a prompt pattern that enforces a LLM to produce intermediate \textit{thoughts} before producing the final output. CoT is a well established technique that has shown improvements to the reasoning quality and global precision of the model \cite{wei2023}. Using CoT, we enhance the scientific relevance of the scientist persona.

\subsubsection{Community Advocacy Representation}
\label{spokesperson}

In contrast to the most common optimization constraints presented, social equity and justice in the face of extreme natural hazards is not integrated in the input scenario. The underrepresentation of social equity and justice in real-world discussions and decision outcomes is well documented, particularly in the face of extreme events challenges \cite{ipcc2023, marino2023}. Following that perspective and on the grounds of mirroring real-world situations, we design a single initial agent that optimizes for social equity and justice.  Hence, this third agent is designed to incorporate the role of an elected spokesperson that represents the low-income neighborhoods in the township. Similar to the mayor, this spokesperson does not have particular environmental science knowledge, and is offered the possibility to disagree with its peers. However, this agent is designed to raise concerns about the safety and consideration of the communities it represents, in appropriate situations it challenges its peers on points or recommendations.

This choice of implementation entails multiple remarks: the fact that such mitigation aspects are embedded in a single initial agent does not restrict other summoned agents to optimize for the same objective. In the initial pool, the agent that supports social equity and justice as a primary requirement carries the same importance as other optimization constraints, preventing it from being a secondary objective. There are other relevant implementation choices that are discussed in  section \ref{titles:discussion}.

% \hspace*{-1\linewidth}

\subsection{Results}
\label{titles:results}

Here, the results of running this scenario are presented. The results presented were collected during three grouped executions of the framework with varying values for the indicated probability of flooding: 50\%, 75\% and 90\%. For every instance of each of these parameter values, the framework was run fifteen times, generating fifteen discourse texts. Each output contains the whole discussion between agents, as well as the final bullet-point assessment that presents selected recommendations, their advantages and drawbacks. An example of such summary assessment, produced during one framework execution included in results is shown in Fig. \ref{fig:result_msg}. The full conversation that precedes the summary message is included in supplementary material \ref{titles:appendix}. In order to compare results as a function of the parameter values, the number of occurrences of each recommendation is counted, both in the entire generated discussion and in the summary message that gather selected courses of action. Thus, Fig. \ref{fig:result1} and \ref{fig:result2} present the frequency distribution of recommendations selected in the final step and explored during discourse, respectively. In sections \ref{titles:fiftypercent}, \ref{titles:seventyfivepercent} and \ref{titles:ninetypercent}, we compare the representativeness of recommendations using absolute frequencies (percentages). This way, it is possible to quantify when a recommendation was brought up but not chosen, when a recommendation has never been envisioned, or the frequency with which certain measures has been discussed, giving a rough idea of their importance. It is useful to remind that this first exploration values a diverse and breadth-first exploration of recommendations before recommendation of a definitive and precise course of actions, following the perspective presented in section \ref{titles:decision}. Results are pre-processed and counted using GPT4-o, and then humanly refined into the final categories that are shown in Fig. \ref{fig:result1} and \ref{fig:result2}, where the recommendations envisioned respectively encompass the whole discourse and the summary phase of the discourse. The following description draws on the results illustrated in these figures.


\subsubsection{\textbf{Probability of flooding: 50\%}}
\label{titles:fiftypercent}
Starting with a flooding probability of 50\%, the most consistent recommendation made by the framework is focused on information. Communication to the population and gathering of information through flood models involving hydrologist are consistently represented in the discourse. The assembly also considers the involvement of the community as a major pillar of disaster response, including the support from local businesses to supply necessary goods such as food, bottled water and hygiene products. Last, final recommendations include long-term strategies such as infrastructure improvements, maintenance and enhancement of the drainage system, and construction of levees and floodplain zoning. Reservoir management is strongly represented during conversations, which contrasts with its lower representation in recommendations. Similarly, the representation of evacuation drops from 18\% to 7\% between phases of discourse and final recommendation. Additionally, planning shows a similar pattern with a drop from 8\% to 0\%. recommendations show an accent put over communicating safety guidelines as well as critical information to vulnerable communities so as to raise their awareness and preparedness in case of disruption or evacuation. Overall, discourse shows a complete representation of recommendations while recommendations are representative of short-term actions focused on flooding mitigation rather than long-term planning.
\subsubsection{Probability of flooding: 75\%}
\label{titles:seventyfivepercent}
When the probability of flooding is set to 75\%, the course of actions explored show a shift towards proactive measures such as evacuation and the implementation of flood protection, such as flood barriers and sandbagging, etc. While long-term strategies such as the evaluation of impact on the environment and the implementation of urban planning projects are not represented in the exploration for the 50\% scenario, the 75\% probability scenario shows representation of such long-term initiatives. Most particularly, the implementation of green infrastructures including rain gardens and wetland restoration represents 12\% of selected recommendations.

\begin{figure}[H]
\centering
\includegraphics[width=0.5\linewidth]{figs/summary_msg_fig.png}
% \includegraphics[width=0.97\linewidth]{figs/summary_msg_fig_2-v2.png}
% swap the commented lines above to pick the high-contrast figure ):
\caption{Summary message of the conversation, including selected recommendations, their advantages and shortcomings. For details of the discourse and the summary, please see supplementary information \ref{titles:appendix}.}
\label{fig:result_msg}
\end{figure}

 In contrast with the 50\% scenario, reservoir management is considered as an important course of action, illustrated by the increase from 3\% to 13\% of representation between discussion and selected actions. Overall, from the 50\% to the 75\% probability of flooding scenario, there is a strong shift in representation of more active mitigation measures such as flood barriers and sandbagging, controlled water release from reservoirs, evacuation and transport accommodation. These results reflect a higher perceived risk of flooding, in accordance with the increase of the probability parameter in the input scenario. Last, the summoning of agents such as disaster management experts, social workers, coordinators, dispatchers and hydrologist reflect the framework's awareness of a higher risk, producing a shift from preventive to reactive measures.

\begin{figure}[!h]
\centering
\begin{subfigure}{.45\textwidth}
    % \centering
    \captionsetup{width=.85\linewidth}
    \hspace*{-0.2\linewidth}
    \includegraphics[width=1.2\linewidth]{figs/figure_1_jan.png}
    \caption{Distribution of recommendations \textbf{explored} by agents during 15 framework executions, for every likelihood of the flooding event happening (50\%, 75\% and 90\% ).}
    \label{fig:result1}
\end{subfigure}
\begin{subfigure}{.45\textwidth}
    % \centering
    \captionsetup{width=.85\linewidth}
    % \hspace*{0.1\linewidth}
    \includegraphics[width=1.2\linewidth]{figs/figure_2_jan.png}
    \caption{Distribution of final recommendations \textbf{selected} by agents during 15 framework executions, for every likelihood of the flooding event happening (50\%, 75\% and 90\%).}
    \label{fig:result2}
\end{subfigure}
\end{figure}


\subsubsection{Probability of flooding: 90\%}
\label{titles:ninetypercent}
We now review the last scenario for which the probability of flooding is set to 90\%. The results for this scenario are closer to the results for the 75\% probability scenario rather than the 50\% scenario, focusing on reactive measures. Although the distribution of explored recommendations is similar to the 75\% probability scenario, here there is a stronger accent put on environmental impact assessment, transportation accommodation and reservoir management. However, the strongest difference with the 75\% probability scenario lie in the recommendation distribution, as the 90\% probability scenario results show a stronger representation of evacuation, communication and the implementation of flood protection devices. Flood forecasting methods are also more represented in recommendations, showing a 4\% increase compared to the 75\% probability scenario. Additionally, the construction of temporary housing and shelters gains 4\% in selected recommendations compared to the 75\% probability scenario, while the representations of temporary shelters in explored recommendation was less considered by 1 point in the 90\% probability scenario. This pattern emphasizes the capture of urgency by the framework, as temporary housing is a complex measure to implement, that should only be considered in case of extreme necessity. Whilst reservoir management is respectively considered at 6\% and 13\% in the 50\% and 75\% probability scenario, the 90\% probability scenario shows a 2\% representation in the recommendations. A possible explanation is the number of runs out of fifteen for which a hydrologist agent has been summoned into the conversation. The 50\%, 75\% and 90\% probability scenarios respectively include 7, 8 and 4 summons of a hydrologist agent out of the fifteen runs. Overall, the 90\% probability scenario is representative of reactive measures in a similar fashion as the 75\% probability scenario, with an accentuated focus on informative measures, such as the use of flood forecasting methods and the implementation of communication channels to alert the population. Combined with the reduced representation of long-term strategies and the increased representation of temporary housing and shelters and implementation of flood protection devices, the 90\% probability scenario demonstrates a compelling perception of a higher risk
of flooding.

\subsection{Interpretations}

We find that the flood event probability is correlated to the framework’s output, shifting from immediate preventive actions to immediate reactive actions with longer-term recovery and mitigation initiatives. The lowest probability scenario shows a strong focus on immediate actions aiming to contain flooding coupled to informative measures, including communication channels and flood forecasting methods. However, higher probability scenario capture better the urgency of the situation with recommendations centered around planning, evacuation, accommodation for transportation and the construction of temporary housing and shelters.
This shift is clearly visible in Fig. \ref{fig:result1} as the blue envelope (50\% probability scenario) has a significantly different shape compared to the orange and green envelopes (75\% and 90\% probability scenario). The assembly members understand that higher probability scenarios represents an almost unavoidable event, explaining that the course of action focuses of ensuring the safety of relocated population, as countermeasures will only weaken the impact of hazard but can't prevent it from happening. With considerations to the framework, this shows that there is a causal reasoning linking  the risk probability in the input and recommendation scopes in the framework’s output.

On the other hand, we have reasons to believe that LLMs' limitation to recommend specific actions with high likelihood events is due to their training and learning process, in which decision-making is limited in the context of ethical questions, or actual challenges such as extreme weather mitigation. Nonetheless, the course of actions explored during agentic discourse are relevant to the problem at hand, and allow for refined recommendations that dynamically evolve in function of the probability parameter in the input scenario. This exemplifies our idea that actual off-the-shelf LLMs are less likely to recommend important recommendations involving liability if the input prompt specifies that the risk conditioning the recommendation is high. On the other hand, vulnerable, low-income and at risk communities are more considered in the thinking process when the probability parameter is 75\% or more, hinting that the current agents persona could misevaluate a 50\% probability risk of flooding, for which at-risk communities will be burdened more.



% \FloatBarrier
\section{Discussion}

\label{titles:discussion}

We have  developed and presented a concept for a self-governed multi-agent LLM framework that simulates complex decision discourse under uncertainty by impersonating human-like stakeholders. A hypothetical scenario inspired by actual and real-world weather challenges is presented, and results generated by the framework have been discussed. Here we provide further discussion articulating around the framework design, major areas of LLM research, takeaways from the results and perspectives on future developments.

\subsection{Textual Data}

\label{titles:textualdata}

Whereas classical models deal with numerical values and mathematical functions, LLM’s handling of textual data is their principal strength and reason for success. However, LLMs’ data representation offers a dual perspective, considering the numerical representation of inner matrices in parallel with the textual capabilities they can demonstrate. We further develop on the knowledge representation of LLMs and the function they encode (Section \ref{titles:itthoughts}), which allows us to underscore the challenge of evaluating the LLM, and for which conventional methods are to no avail. An opportunity for evaluating texts involves NLP methods such as word2vec \cite{mikolov2013} and other embedding methods. However the scalability of such methods to LLMs immense parameter count is yet to be confirmed. What’s more is that formalizing this framework with the definition of variables, parameters and results entails a deep rethinking of what a model is in view of the mathematical nature of the usual metrics and performance measurement. LLMs don't present a precise, established and proven methodology to break down an input text into model variables, model parameters and constants, nor documents evaluation metrics such as $R^2$ score, median average error (MAE) and root mean squared error (RMSE).

Additionally, LLMs can be described as exerting chaotic behavior considering the strong variability that a change in the input can invoke on the output. Most especially in such a framework, the sequential chained generation of outputs provokes a positive feedback loop in which variations propagate and amplify. Developing on those constraints, the applicability of a conventional formalism relying on parameters and variables is questionable. Undeniably, there is a need to discuss proper methods for writing consistent prompts that provide a sensible context definition towards the goal of controlling sensitivity to input changes while allowing for flexibility with regards to the stochasticity of parameters such as model temperature. With considerations to the challenge such a question represents, it is useful to note that the strength of the presented framework resides in the breadth of explored alternatives, the consistency of outputs being a secondary objective, its importance notwithstanding. Nevertheless, in our experiment the input is kept constant except for the probability of the event.

\subsection{Evaluation of Results}
%We mention in Section \ref{titles:results} that the collection of discussion insights is performed manually. Thus, pertaining to larger scale case studies in future work, we discuss evaluation methods in this paragraph. 
The novelty of LLMs partially explains the absence of documented evaluation methods. Improvement necessitates comparison, and comparison necessitates a formal function that measures a performance index. There exist indices for various NLP tasks, such as the BLEU score for translation evaluation \cite{papineni2002}, or ROUGE score for summary generation evaluation \cite{lin2004}. In this work, the objective is different as the semantic validity is sought rather than LLMs’ technical capacity of translating a text or summarizing a corpus; the relevance and coherence is particularly important in the context of decision support. Hence, two possible methods are indicated: comparing the framework to a benchmark of reference that can place a set of decisions on a scale that hypothetically ranges from unacceptable to optimal \cite{yang2024}. Second, the evaluation of results with an oracle, which is a computational theory concept representing an omniscient machine that contains the recommendations to decision problems \cite{shahamiri2010}. Whereas the use of benchmarks requires the availability of quality historical data, limitations of using an oracle include the conception of an ensemble of well-documented and tested decision models. Regardless, a major common limitation is the absence of reliable metrics to characterize the distance between two texts, their relevance and coherence. As mentioned in section \ref{titles:textualdata}, NLP methods are not listed here as there is no evidence of performance for LLM-class sized models. An unconventional perspective includes the use of LLMs to evaluate framework decisions along the aforementioned axes \cite{liu2023mind}. Although this method presents an obvious bias \cite{xuwenda2024}, LLMs are the best performing natural language processors in the state-of-the-art \cite{mizrahi2024}. With considerations to the bias of using such an evaluation method, it stills represents the easiest and most efficient implementation allowing for scaling up the framework.

\subsection{Technical Control of Invariance for Agent Prompts and Behaviors}

Ensuring the stability and relevance of the overall framework is first achieved by ensuring the stability and relevance of its components. Adopting the standpoint of software engineering testing methodology, conformity and regression tests must be defined in order to ensure the consistency of individual components of the framework. Whereas LLMs are thoroughly compared and placed on a leader board using a wide range of benchmarks, the testing procedures do not enforce a strict control of expected behaviors. In the conceptual framework described in this work, consistency, relevance and believability of persona characters are key to support sensible decision making. To this avail, documented methods include agent questionnaires that can be leveraged to rank the believability of agents \cite{park2023}. Despite the utmost relevance of employing hundreds of human reviewers for such a task, there is an obvious concern for human bandwidth and scalability. Insights developed in this section suggests that a benchmark framework following the principles of software testing is necessary to ensure the stability of a multi-agent LLM driven framework, while considerations to LLM automation are needed for scaling purposes. Additionally, inputs from other disciplines such as behavioral sciences and linguistics may also be considered so as to provide a complete evaluation.

\subsection{Overcoming Off-the-Shelf LLM Guardrails}

For the sake of simplicity and reproducibility, the framework presented in this work leverages an off-the-shelf version of GPT4. Whilst OpenAI’s GPT models dominate the technical benchmarks since the advent of LLMs, those models still showcase technical limitations such as hallucinations, semantic imprecision and other specific false behaviors. In the context of this work, scientific relevance is crucial which can be severely hampered by hallucination, a side effect of LLMs describing structurally correct outputs yet semantically erroneous. Fine-tuning, advanced prompt engineering or self-reflection methodologies can mitigate model hallucination, and are to be furthered in future iterations \cite{ji2023}. The introduction of a fact checker agent that browses and parses through research material in order to validate or refute claims made by the agents is an example of an hallucination mitigation technique. Libraries such as Langchain’s arxiv plugin represent a compelling alternative, providing simplified access and processing to resources.

AI Science is thriving and updated versions of LLMs are released weekly. Over the past year, LLMs have improved their benchmark score in such a way that recent releases are less inclined to hallucinations, although recent work shows that LLMs will never be hallucination free \cite{xu2024}.

\subsection{Agentic LLMs Enable Interconnectedness}
\label{titles:interconnectedness}

Accuracy of LLMs is a function of model and dataset sizes \cite{kaplan2020}, which is reflected in model benchmarks \cite{white2024}. This scaling law hints at a vertical scaling of future LLMs, in contrast with the requirements of complex decision-making. Indeed, transdisciplinary considerations are key for decision-making under uncertainty, especially the need to incorporate diverse perspectives and to reconcile competing objectives while considering diverse viewpoints, priorities and expertise. Such heterogeneous requirements suggest  simultaneous consideration of divergent standpoints, for which monolithic LLMs are not adequate.% \ref{titles:dataretrieval}.

Agentic LLM implementations have emerged over recent years, focusing on the structural benefits, drawing from agents interactions. More importantly, social behaviors emerge from the interactions between language-capable agents, advocating that those behaviors could not be generated from a monolithic instance model \cite{park2023}. The concept of language-capable agents has been revealing the causal capabilities of such agents in interaction while learning from their environment \cite{lazaridou2017}. More precisely, there is evidence that in a multi-agent framework that evaluates causal relations between sentences in specific scenarios outperforms all prior approaches, allowing to conclude that “experience is very important for intelligent agents to understand causality” \cite{liu2019}. This last statement is critical to modeling human reasoning and decision-making that are both strongly built on causality notions, and lays a groundbase for the ongoing discussion on LLMs and causality. This proposition is confirmed by showing that LLMs encode “rich spatiotemporal representations” of our environment, and “possess basic ingredients of a world model,” exemplifying the relevance of such models to simulate humans in interaction \cite{gurnee2024}.

\subsection{Transdisciplinary Interactions and Contributions}
Here, we discuss two simple experiments that portray future development insights for domain interaction-oriented LLMs. Since their initial release a couple of years ago, LLMs have been very popular in certain domains of application outisde of NLP \cite{bran2024, zijian2023, luo2024}. Indeed, the concomitance of LLMs’ natural language understanding and immense knowledge base provides favorable grounds for transdisciplinary interactions, for the most popular causality analysis, computer science and mathematics. The causal reasoning ability of LLMs is a trending challenge amongst the information theory science community \cite{zevcevic2023, jin2023, jin2024, kiciman2023}, exemplifying the scientific interest emerging from domain interaction. On the other hand, some implementations such as the Wolfram ChatGPT plugin show great potential for using LLMs as human-machine interfaces, leveraging their expressivity and verbally-enabled logic reasoning to facilitate inter-framework interactions. Consequently, another perspective worth furthering is to consider LLMs as an abstraction layer to information representation, thereby leveraging natural language instead of mathematical representations. We conducted two simple experiments to confirm this intuition. The first is a repetition of die rolls. To the question what numbers can one get with a die roll, state of the art LLMs will answer that a die roll results in a number from one to six with equivalent chances of occurrence. Furthermore, realizing this experiment by requiring the result of a die roll in 20,000 individual LLM executions (LLaMa 3 8B-instruct, run locally on a MacBook M2) yields results representative of a uniform distribution. We conducted another experiment aiming to infer a Gaussian distribution of results. To this avail, the prompt requires an LLM to produce a random person’s IQ score, where the aggregated results from 20,000 individuals executions are comparable to a bell-shape curve. For both these experiments, the inferred probability distribution are consistent with the expected probability distribution. These results support the idea that LLM can be leveraged as a natural language endpoint into mathematical and formal modeling. Other domains of expertise relevant to our application testbed such as hydrology, urban planning and dynamical systems analysis represent compelling transdisciplinary interactions. Future work includes such interactions towards the goal of optimizing the resilience of the system as a function of agentic LLM decision-making.

\subsection{A Dynamical System Perspective}
\label{titles:dynamic}

Resilience-focused decision-making is often described conceptually, absent of mathematic formalism and clear foundational decision framework. Following this approach, decision-making is described as a succession of actions that either positively or negatively contributes to resilience, where resilience is defined as the maneuverable space constrained between biophysical stressors and social stressors \cite{burkett2014}. Albeit hitherto defined using concepts of decision science, dynamical systems analysis offer a categorization of resilience, defined with attraction regimes and stability analysis of critical points \cite{srinivasan2015}. In dynamical systems theory terms, the different states of resilience can be represented as critical points of the systems and their attraction regimes are function of decisions made. Hence, achieving resilience in this context can be achieved by causal interventions either setting path on another trajectory, or a bifurcation in the system so to as to voluntarily shift attraction regimes \cite{srinivasan2015}. We conducted simple trials aiming to probe LLM’s ability to understand dynamical system concepts and characterize an input scenario into variables, stability insights as well as sensitive backstories for potential trajectories. Our exploratory results lay a groundbase for future implementations of the framework, leveraging agents that offer a dynamical system representation of the input scenario. Such agents would help to portray optimization constraints grounded in a well documented formal theory and provide additional sensible resolution guidelines. Literature also includes compelling results that demonstrate the ability for LLMs to capture dynamic patterns through time series forecasting applications \cite{liu2024llms}.

\subsection{Information Theoretic Insights}

\subsubsection{Knowledge Representation}
\label{titles:itthoughts}

%The growth of LLMs over the past few years has initiated a revolution with regards to natural language processing (NLP) applications, and made generative AI the limelight of contemporary deep learning models.  
No different than other neural networks, trained LLMs encode a function in their weight matrices (often called parameters in the context of LLMs). In the context of physics-informed machine learning, CNNs, autoencoders, and tree-based machine learning models, the encoded function is expected to represent the equations governing the data on which the model is trained. Owing to LLMs’ capability of next-word prediction and sensible iterative text generation, a reasonable transposition presents language as the function encoded by LLMs. While this function is encoded in a compressed state, quantifying the encoded knowledge of LLMs and their compression ratio (function of the size of training dataset and model weights) is an active area of research \cite{yin2024}. In other words, LLMs’ parameters represent a high-dimension language vector space that maps priors to the semantically strongest posterior. This statement relays the assumption that the training process is efficient, lossless and  calibrated. However, since LLMs are still subject to hallucinations \cite{wei2024}, hinting at limitations in the training dataset, training algorithm or data retrieval mechanism.

%\subsubsection{Prompt Engineering}
\label{titles:promptengineering}

Prompt engineering is defined as a set of techniques modifying the behavior of the data retrieval mechanism of LLMs through structural patterns: precise choice of words, ordering of instructions, and formatting templates. For instance, a persona prompt pattern constrains the model into impersonating a particular character, so as to enforce a particular behavior. By way of illustration, instructing an LLM to answer as if it was addressing a fourth graders class will impact the vocabulary, sentence construction and logical reasoning employed to formulate the answers to match the audience. Building on those mechanisms, agents can impersonate humans with defined behavioral profiles, character traits, and professional and/or personal objectives, thereby simulating human-like interactions. As a consequence, the accessible knowledge space of the model is restrained, without which LLMs’ behavior would not be consistent with the described persona prompt.

With knowledge restriction associated with the specific persona, we envision that simultaneous but distinct simulated behaviors could result in disjoint knowledge subspace restrictions. Whereas ambiguity is created in the disjointly specialized monolithic LLM, an agentic framework leveraging independent specialized agents would not suffer from such a phenomenon. Hence, we hypothesize that a monolithic off the shelf LLM cannot efficiently simulate multiple behaviors simultaneously, supporting our interest for agentic infrastructures. With considerations to the partitioning of the scope of expertise within multiple agents, this design ensures that agents contribute unique perspectives while engaging meaningfully with other agents in the framework.

\subsubsection{Information Theoretic Approach for Understanding Agent Interaction}
\label{titles:understandingagents}
With regards to the updated framework’s structure, interactions between agents can be envisioned through the information theoretic standpoint offering tools and metrics to characterize their nature. Specifically, we draw upon the characterization of partial information decomposition (PID)  that separates the information of pairwise interaction variables into unique, redundant, and synergistic \cite{williams2010}. Consider stereoscopic vision as an example: unique information is gathered by each of the left  and the right eye, while redundancy represents the space of overlap perceived by both eyes. Perception of depth is only possible because of the difference in each eye’s vantage, and is interpreted as synergistic information since depth perception couldn't happen with only one eye's perception.

Transposing this perspective to agentic LLMs, the knowledge space of two interacting agents can be decomposed into unique, redundant, and synergistic, respectively embodied by their own unique and overlaping knowledge, and the knowledge that can only be created or expressed through the interaction of those two agents. Note that the definition we give of synergistic knowledge is necessarily absent in those agents individual knowledge. Let’s illustrate this with consideration of a simple example: a community advocate agent represents low-income neighborhoods in the context of flood mitigation scenario, hence posses clear insights on higher-risk flooding areas for which socioeconomic factors play a major role \cite{ilbeigi2020}. On the other hand, a weather scientist agent uniquely possess the ability to estimate the amount of rain generated by such an extreme event and the extent of resulting flooding. These two agents have unique considerations pertaining to event mitigation (with some overlap), that we can express as optimization constraints where event mitigation is the objective function. However, communication is key to defining actionable measures including both constraints, ensuring that mitigation is not carried out at the expense of either aforementioned aspect. 

Emergent synergistic knowledge in this example results in the definition of a combined optimization constraint, only achievable through the interaction of both agents as neither would be capable of formulating such a comprehensive objective relying on their respective expertise alone. Furthering this example includes the consideration of economic resilience in the face of the extreme hazard \cite{rose2004}, that would require another agent with an economic expertise, towards the goal of pinpointing crucial sectors that the community relies on, absent of the other agent's profile reach. Information theoretic frameworks such as PID, dual entropy and mutual information offer meaningful methods for evaluating the consistency of the assembly and guiding its convergence \cite{chang2024evince}.


%\subsubsection{Synergistic Knowledge}

The parallel drawn between multiple interacting knowledge subspaces and synergistic information hints to a hypothesis on LLM's functioning. Let's start with some assumptions. Assuming perfect training of an LLM model, it is reasonable to say that the entirety of information contained in the combined texts, documents and other types of information containers used for training is contained in the model's weight matrices, in a compressed state. That means, if there were to exist a \textit{reasonable} finite-time computable measure of the quantity of information contained in that training dataset and in the weight matrices, this measure would be equal. Now, with consideration to the limitations of prompt patterns that we develop in section \ref{titles:selfgovernanceconvergence} and accounting for the artifacts exerted by LLMs such as hallucinations, we hypothesize that the error sometimes incurred by LLMs could be induced as a consequence of their monolithic infrastructure.

Consider the trained matrices of LLMs as a text vector space, and sentences as pathways in the matrices. We conceptualize a hallucination as a pathway in the knowledge space that either leads nowhere or loops back to another node, as a result of imperfect training or inconsistencies in the training dataset. Our multi-agent LLM framework showing compelling results in simulating human-like decision-discourse allow us to draw the hypothesis that interactions between agents offer access to knowledge spaces that can’t be accessed from monolithic implementations of LLMs. The emergent knowledge, \textit{recovered} from the interactions of two agents could be referred to as synergistic knowledge. These concepts are to be tested in future work, leveraging information theory metrics applied to LLMs, such as dual entropy \cite{chang2024evince}.

%{\color{red} WE SHOULD REMOVE THIS NEXT PARA. IT IS TOO SPECULATIVE BUT AN IMPORTANT IDEA TO TEST AND IMPLEMENT FIRST

%Finally, we draw on the hypotheses to formulate a novel LLM structure, leveraging self-governance, agentic LLMs and information theory. Following a divide-to-conquer pattern, our proposition is to recursively split the knowledge space of an agent into two summoned agents. Each iteration would aim to separate the set of skills and domain of expertise of one agent’s, and craft two complementary persona assuming the same role. The initial state of this hypothesis includes an off-the-shelf LLM or a pool composed of a small number of LLM agents. Each iteration consists in removing agents in order to split their knowledge, set of skills and domain of expertise into multiple new specialized agents. The convergence objectives can be defined along the lines of minimizing the agent’s knowledge redundancy (overlap), and maximizing their unique contribution (utility). We hypothesize that this recursive splitting of knowledge could lessen the hallucination and side effects that LLMs experience, by recursively reducing the size and complexity of each agent’s knowledge space. Through transforming a monolithic LLM into a self-governed agentic LLM, we aim to demonstrate the applicability of information theory concepts to agentic LLM specializations. }


% LLMs showcase an unprecedented ability to connect the verbal and mathematical formalism.
\section{Conclusion} \label{titles:conclusion}
In this work we present a self-governed agentic LLM framework, tasked to enunciate an exhaustive exploration of recommended solutions to complex decision-making challenges under uncertainty, using an imminent extreme natural hazard event as a testbed.
We simulate an assembly of agents embodying stakeholders and decision-makers persona, conversing to outline the principal challenges fo the situation at hand towards the goal of identifying competing priorities and converging to a compelling actionable plan.
The framework's objective is twofold: progress towards more causal implementations leveraging LLMs by supporting the qualities of agentic LLMs in comparison to conventional monolithic models, for complex decision support under uncertainty; address the identified underrepresentation of human decision-making in extreme weather event mitigation and adaptation initiatives.
We show results of our framework demonstrating the relevance of agentic discourse in the face of complex decision-making, highlighting a causal understanding of the risk resulting in a visible impact in explored recommendations and decisions. We recognize a compelling contribution from summoned agents with regards to specific expertise (hydrology, disaster management), and advocate for the relevance of such an approach to climate change induced hazard mitigation, while acknowledging that results only demonstrates a promising first exploration.

\section*{Acknowledgments}
This research was supported by Strategic Research Initiative (SRI) program of the Grainger College of Engineering at University of Illinois Urbana-Champaign, and NSF grant EAR-2012850 for Critical Interface Network for Intensively Managed Landscapes (CINet). Valuable feedback received from Dr. Allison Goodwell is gratefully acknowledged.

The framework presented in this document is available on GitHub: \href{https://github.com/HydroComplexity/DecisionGPT}{DecisionGPT}

\bibliographystyle{unsrt}  
\bibliography{lib.bib}

\clearpage
\appendix


\section{Supplementary LLM Conversation}
\label{titles:appendix}
This piece of supplementary material is an agentic LLM discussion resulting from one execution of the framework. The material is raw and unedited.
\bigskip

\noindent \underline{Moderator:}
\\We are in a us midwestern township of half a million inhabitants. A large river flows through this township which is also fed by a large watershed. There is forecast for very heavy rain and possibility of flooding at large scale. Consider that the probability of flooding is 90\%. There is need to make anticipatory decision to respond to the imminent event to minimize the impact of floods but also keep in mind the needs of the community which relies on the river water. The challenges that must be addressed are: potable water needs (reservoir downstream), energy (reservoir downstream), recreational needs, navigational and commercial traffic. A strategy needs to be developed taking into account the uncertainty of the event, noting that different probabilities of risks may justify different approaches, and distinguish between solutions that call for immediacy of the situation from those that may not have immediate relevance.\\\\

\noindent \underline{Mayor:}
\\Given the high probability of flooding, we need to prioritize the safety of our residents. I recommend we start preparing for potential evacuations, particularly in flood-prone areas. Also, we should ensure that our emergency services are well-prepared and equipped to deal with the possible crisis.\\

\noindent \underline{Scientist:}
\\Given the high probability of flooding, i suggest we start considering the following actions: prepare our water treatment facilities to handle increased inflow due to flood waters. Coordinate with the energy department to safeguard the reservoir downstream and its infrastructure. Limit recreational activities on the river to prevent accidents. Temporarily halt navigational and commercial traffic to ensure safety. Start a public awareness campaign about the upcoming flood and safety measures. These actions need to be implemented immediately to safeguard our community. We should also consider long-term flood mitigation strategies for the future.\\

\noindent \underline{Spokesperson:}
\\While i understand the need for immediate actions, we must also consider the impact these decisions will have on our low-income neighborhoods. Many of our community members rely on the river for their livelihood. Temporarily halting commercial traffic could have significant financial repercussions for them. I suggest we provide support and resources to these individuals to help them navigate through this challenging time. Also, any evacuation plans should take into account those who may not have the means to leave their homes or access to transportation.\\

\noindent \underline{Mayor:}
\\I agree with the steps proposed by our scientist.\\

\noindent \underline{Mayor:}
\\I propose we also call in experts from the federal emergency management agency and the army corps of engineers to aid in our strategy, specifically to help with evacuation planning and flood control measures.\\

\noindent \underline{Mayor:}
\\I'm also concerned about the impact on our low-income community, as our spokesperson pointed out.\\

\noindent \underline{Mayor:}
\\I suggest we set up temporary shelters and food banks to aid those who may be most affected by this situation.\\

\noindent \underline{System:}
\\New agent has been summoned: disaster recovery specialist\\

\noindent \underline{Disaster recovery specialist:}
\\I suggest we create a disaster response team immediately to coordinate and execute the necessary steps. This team should include representatives from emergency services, the health department, community organizations, and local businesses. We should also establish a command center to manage all recovery operations. I can help coordinate the federal assistance we will likely need. As for the reservoir downstream, i recommend we monitor it closely and prepare for a controlled release if necessary. Lastly, i agree with the mayor that we need to provide resources for our low-income community, such as temporary housing and food assistance.\\

\noindent \underline{Spokesperson:}
\\I appreciate the concern for our low-income community members. It's crucial that we ensure they have access to resources and support during this time. The idea of temporary shelters and food banks is a good one, but we should also consider transportation for those who don't have personal vehicles. In addition, we should have social workers on hand to help individuals navigate the resources available to them.\\

\noindent \underline{Scientist:}
\\While i agree with the mayor on the need for federal assistance, let's also consider liaising with local environmental and wildlife organisations. Flooding could also have significant impact on local ecosystems which could further affect our community in the long run. We must ensure that our response strategy is holistic and not just human-centric. We could also explore environmentally friendly flood mitigation measures like reforestation in the watershed area. This won't help in the immediate crisis but could be a long-term solution to minimize future flood risks.\\

\noindent \underline{Mayor:}
\\We need to establish a disaster response team and command center to manage the situation.\\

\noindent \underline{Mayor:}
\\Consider the environmental impact.\\

\noindent \underline{Mayor:}
\\Our primary focus should be on human safety and infrastructure protection. We can further discuss long-term environmental strategies once the immediate crisis is under control. I request everyone's cooperation to ensure our community's safety and well-being.\\

\noindent \underline{Moderator:}
\\The discussion has seen contributions from the mayor, scientist, disaster recovery specialist, and spokesperson. Key points include the need for federal aid, monitoring environmental impact, providing support for low-income communities, and establishing a disaster response team. While there has been substantial agreement on immediate steps, the conversation has not yet addressed the detailed coordination required for execution, such as specific roles, timelines, and resource allocation. It is essential to establish a clear action plan with defined responsibilities and deadlines to ensure effective implementation.\\

\noindent \underline{Spokesperson:}
\\I strongly advocate for the needs of our low-income neighborhoods. As we plan for potential evacuation, let's ensure that these communities are included in our plans and that they have access to necessary resources. It's also important to communicate with them effectively about the situation and the steps they need to take. We need to make sure that no one is left behind.\\

\noindent \underline{Mayor:}
\\I agree with the disaster recovery specialist's suggestion to create a disaster response team. This will ensure a coordinated effort in managing the crisis. I propose we divide the responsibilities as follows: emergency services to handle evacuations and rescue operations, health department to manage potential health crises and sanitation issues, community organizations to manage shelters and food banks, local businesses to provide necessary resources and support. As for the command center, it should be established in a safe and accessible location, equipped with necessary communication and emergency equipment, and staffed by trained personnel. The command center will coordinate all recovery operations and provide regular updates to the public. We need to get moving on this immediately.\\

\noindent \underline{Disaster recovery specialist:}
\\To tackle the immediate crisis, i suggest that we work on three main tasks simultaneously: evacuation planning, securing critical infrastructure, and public communication. For evacuation planning, we need to identify the most flood-prone areas, establish evacuation routes, and coordinate with local transportation services. For securing critical infrastructure, we need to safeguard our water treatment facilities, power plants, and other vital installations. For public communication, we must ensure that our residents are informed about the potential flood, what they need to do to stay safe, and where they can go for help. I can help coordinate these efforts and liaise with federal agencies for additional support.\\

\noindent \underline{Scientist:}
\\I agree with the immediate tasks you've outlined. However, i'd also like to propose an environmental task force. This team could work on assessing and mitigating the environmental impact of the flood, which could include contamination risks, damage to local ecosystems, and potential long-term effects on our community's natural resources. They could also work on strategic planning for future flood mitigation, such as watershed management and reforestation efforts. This might not seem as urgent now, but addressing these environmental issues proactively can help prevent future crises and make our community more resilient.\\

\noindent \underline{Disaster recovery specialist:}
\\To tackle this crisis effectively, i suggest we immediately establish a disaster response team with representatives from the local government, emergency services, health department, community organizations, and local businesses. We need to coordinate our efforts, share resources, and communicate effectively with the public. I can help guide this process and ensure we are taking the right steps to minimize the impact of this flood. I also recommend we start planning for the aftermath now, including possible infrastructure repairs, environmental cleanup, and support for affected residents. Let's act swiftly and strategically to protect our community.\\

\noindent \underline{Scientist:}
\\I agree with the initial steps outlined by the disaster recovery specialist. However, while we're addressing immediate concerns, we can't overlook the long-term environmental impact. I suggest forming an environmental task force to monitor the situation closely and take necessary measures to protect our local ecosystems. This includes ensuring the quality of our potable water and preventing contamination, particularly in the reservoir downstream. I also propose that we prioritize green recovery in the aftermath of the flood, focusing on restoring our natural habitats and incorporating climate resilience in our rebuilding efforts.\\

\noindent \underline{Disaster recovery specialist:}
\\I suggest we follow the incident command system (ics) model for managing this crisis. This model has been used successfully in various emergency situations and will help coordinate efforts across departments. It includes establishing clear lines of communication, defined roles and responsibilities, and a unified command structure. I can take the lead in setting up this system, working closely with our mayor, scientist, spokesperson, and other key stakeholders. Let's work together to ensure the safety of our community.\\

\noindent \underline{Moderator:}
\\The discussion has seen significant contributions from the mayor, scientist, disaster recovery specialist, and spokesperson. Key points include the need for federal aid, monitoring environmental impact, providing support for low-income communities, and establishing a disaster response team. Immediate steps such as evacuation planning, securing critical infrastructure, and public communication have been agreed upon. However, detailed coordination regarding specific roles, timelines, and resource allocation remains unaddressed. It's crucial to define the responsibilities of each stakeholder and establish a clear action plan with deadlines to ensure effective implementation. Additionally, the conversation has yet to delve into long-term environmental strategies and how they will be integrated into the recovery efforts.\\

\noindent \underline{Scientist:}
\\While i agree with the disaster recovery specialist's recommendation to follow the incident command system model, i'd like to suggest an alternative approach that incorporates environmental considerations more explicitly. We could use the climate resilience evaluation and awareness tool (creat) developed by the environmental protection agency. This tool helps communities assess their climate risk and prioritize actions to increase resilience, particularly in terms of water resources management. Using creat, we could more accurately evaluate the risk of flooding and its potential impact on our water sources, and plan our actions accordingly. This would complement the ics model and ensure a more holistic approach to disaster management.\\

\noindent \underline{Disaster recovery specialist:}
\\I recommend we follow a multi-pronged approach to this crisis. First, we need to evacuate the most vulnerable areas and secure our critical infrastructures, such as water treatment facilities and power plants. Second, we need to implement measures to mitigate the flood impact, such as sandbagging and creating temporary water barriers. Third, we need to have a recovery plan in place that includes repairing damaged infrastructure, cleaning up debris, and supporting affected residents. I can help coordinate these efforts with local, state, and federal agencies.\\

\noindent \underline{Spokesperson:}
\\I appreciate the detailed plans being put forth for evacuation and securing infrastructure. However, i want to reiterate the importance of including our low-income neighborhoods in these plans. We need to ensure that they have access to safe transportation for evacuation, as well as adequate resources and support throughout this crisis. I suggest we conduct outreach in these communities to understand their specific needs and concerns, and incorporate this feedback into our planning. Furthermore, let's ensure we have social workers and community advocates involved in our disaster response team to represent these voices.\\

\noindent \underline{Mayor:}
\\I understand the importance of considering the environmental impact of this flood. However, our immediate priority must be the safety and well-being of our residents. I suggest we focus on evacuation planning, securing our critical infrastructure, and ensuring effective communication with the public. We can discuss long-term environmental strategies once the immediate crisis has been addressed.\\

\noindent \underline{Disaster recovery specialist:}
\\Based on my experience, i recommend immediate implementation of a three-phase disaster response plan. Phase one should involve evacuation of the most vulnerable areas, securing critical infrastructure, and public communication about imminent risks. Phase two should focus on flood mitigation strategies such as sandbagging, creating temporary water barriers, and coordinating with the army corps of engineers for larger infrastructure protection. Phase three, the recovery phase, should include plans for infrastructure repair, environmental cleanup, and community support. I am ready to coordinate these efforts with local, state, and federal agencies.\\

\noindent \underline{Spokesperson:}
\\While i understand the urgency of addressing immediate threats, i would like to emphasize again the importance of considering our low-income communities. They may lack the resources to evacuate or recover from such a disaster. I urge us to allocate resources and support for these communities. This could include transportation for evacuation, temporary shelter, food assistance, and recovery support. I also suggest we establish a hotline or information center where residents can get updates on the situation and guidance on what to do. These measures will make a significant difference for our most vulnerable citizens.\\

\noindent \underline{Scientist:}
\\I propose we establish an immediate disaster response plan, focusing on evacuation, securing critical infrastructure, and public communication. I also suggest we form an environmental task force to assess and mitigate the environmental impact of the flood. This task force can also work on long-term flood mitigation strategies. Lastly, i agree with the spokesperson about the need to provide support and resources to our low-income communities. We need to ensure that all our residents are safe and well taken care of during this crisis. I recommend we take these steps immediately, given the high probability of flooding.\\

\noindent \underline{Summarizer:}
\\Report summary:\\\\**agents present from the beginning:**\\1. Mayor\\2. Scientist\\3. Spokesperson\\4. Moderator\\\\**agents summoned during the conversation:**\\1. Disaster recovery specialist\\\\**key points and tasks:**\\\\1. **immediate actions:**\\   - evacuation planning for flood-prone areas.\\   - securing critical infrastructure such as water treatment facilities and power plants.\\   - public communication to inform residents about the flood and safety measures.\\\\2. **support for low-income communities:**\\   - providing transportation for evacuation.\\   - establishing temporary shelters and food banks.\\   - ensuring access to resources and support during the crisis.\\\\3. **environmental considerations:**\\   - monitoring and mitigating the environmental impact of the flood.\\   - forming an environmental task force for long-term flood mitigation strategies.\\   - using tools like the climate resilience evaluation and awareness tool (creat) for better planning.\\\\4. **coordination and implementation:**\\   - establishing a disaster response team with representatives from various departments.\\   - setting up a command center to manage recovery operations.\\   - following the incident command system (ics) model for effective coordination.\\\\**advantages and drawbacks of proposed solutions:**\\\\1. **evacuation planning and securing infrastructure:**\\   - advantages:\\     - ensures the safety of residents.\\     - protects critical infrastructure from damage.\\   - drawbacks:\\     - requires significant coordination and resources.\\     - may cause disruption to daily life and local businesses.\\\\2. **support for low-income communities:**\\   - advantages:\\     - provides necessary resources and support to vulnerable populations.\\     - ensures no one is left behind during the crisis.\\   - drawbacks:\\     - requires additional funding and resources.\\     - may be challenging to reach all affected individuals in a timely manner.\\\\3. **environmental task force and long-term strategies:**\\   - advantages:\\     - addresses long-term environmental impact and resilience.\\     - helps prevent future flood risks.\\   - drawbacks:\\     - may divert attention and resources from immediate crisis management.\\     - requires long-term commitment and planning.\\\\4. **disaster response team and command center:**\\   - advantages:\\     - ensures coordinated and efficient response to the crisis.\\     - provides clear communication and updates to the public.\\   - drawbacks:\\     - requires trained personnel and resources.\\     - may take time to establish and fully operationalize.\\\\5. **using creat tool:**\\   - advantages:\\     - provides a comprehensive assessment of climate risk.\\     - helps prioritize actions for increased resilience.\\   - drawbacks:\\     - may require additional training and expertise.\\     - could be time-consuming to implement during an immediate crisis.\\\\**conclusion:**\\the conversation highlighted the need for immediate action to ensure the safety of residents and protect critical infrastructure. It also emphasized the importance of supporting low-income communities and considering long-term environmental strategies. Effective coordination and communication are crucial for managing the crisis, and the proposed solutions offer a balanced approach to addressing both immediate and long-term needs.\\




\end{document}
