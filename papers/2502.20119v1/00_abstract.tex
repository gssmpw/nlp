\begin{abstract}
Understanding the stroke-based evolution of visual artworks is useful for advancing artwork learning, appreciation, and interactive display. While the stroke sequence of renowned artworks remains largely unknown, formulating this sequence for near-natural image drawing processes can significantly enhance our understanding of artistic techniques. This paper introduces a novel method for approximating artwork stroke evolution through a proximity-based clustering mechanism. We first convert pixel images into vector images via parametric curves and then explore the clustering approach to determine the sequence order of extracted strokes. Our proposed algorithm demonstrates the potential to infer stroke sequences in unknown artworks. We evaluate the performance of our method using WikiArt data and qualitatively demonstrate the plausible stroke sequences. Additionally, we demonstrate the robustness of our approach to handle a wide variety of input image types such as line art, face sketches, paintings, and photographic images. By exploring stroke extraction and sequence construction, we aim to improve our understanding of the intricacies of the art development techniques and the step-by-step reconstruction process behind visual artworks, thereby enriching our understanding of the creative journey from the initial sketch to the final artwork. 
\end{abstract}