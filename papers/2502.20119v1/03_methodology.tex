\newpage
\section{Methodology}
In this work, we formulate the problem of stroke sequence generation in visual art, where our goal is to build a model that can produce a pragmatic drawing process for input art. Given an input image $I$, our objective is to construct a stroke sequence $Stroke\_seq$ = $(seq_0, seq_1, seq_2, \ldots seq_N$) such that each stroke ($seq_n$), when put in a sequence, aligns and constructs the scene semantically, i.e., $Stroke\_seq (I) \approx I$. 
Here, the semantics of the stroke sequence implies the semblance of a pragmatic drawing process. Towards this,  we present an unsupervised stroke sequence generation method in an open-world setting. Fig \ref{fig:Outline} presents the outline of the proposed method, where we decompose input art and sketch into a set of strokes and compose stroke sequencing to give the semblance of a pragmatic drawing process. 

\begin{figure}[!t]
    \centering
    \includegraphics[width=\textwidth]{Samples/Fig2_Outline.pdf}
    \caption{Outline of the proposed method for sketch to paint}
    \label{fig:Outline}
\end{figure}

Given an input image, we first process it through a sketch generator that distills the input painting into a line drawing that effectively extracts the underlying sketch. Specifically, our sketch generator leverages the state-of-the-art line drawing generator \cite{chan2022learning} that builds on depth information and CLIP features. 

\begin{equation}
     S = SketchGenerator (I) 
\end{equation}

Here, $SketchGenerator$ extracts line drawings by probing the geometrics and semantics of an input image using a pre-trained model \cite{chan2022learning}. 
Once we have sketch and original art, we feed these inputs into individual streams of sketch stroke construction and sequencing (Section \ref{sscs}) and paint stroke construction and sequencing (Section \ref{pscs}) to make a unified global sequencing. 


\subsection{Sketch Stroke Construction and Sequencing} \label{sscs}
Upon receiving a sketch ($S$), the Sketch Stroke Construction and Sequencing (SSCS) module orchestrates the sequence of strokes via sketch stroke construction and sketch stroke sequence generator, as illustrated in Figure \ref{fig:Sketch_seq}. 
\begin{figure}[ht!]
    \centering
    \includegraphics[width=\textwidth]{Samples/Fig3_Sketch_Strokes.pdf}
    \caption{Framework for sketch stroke construction and sequencing}
    \label{fig:Sketch_seq}
\end{figure}


\subsubsection{Sketch Stroke Construction}
Given an input sketch $S$ of a scene, a raster image with a set of pixels, the stroke construction module produces a set of strokes ($S\_Strokes_N$) based on vector curves. In the past, various vector curves have been investigated for stroke design \cite {zhang2023towards,liu2009beyond,carlier2020deepsvg,chan2022learning}. 
In this work, we mainly employ the stroke representation in terms of Line, Quadratic B\'{e}zier Curve (QBC), Cubic B\'{e}zier Curve (CBC), Circular Arc (CA), and Elliptical Arc (EA). Here, the position of control points determines the shape of the curves. The number of control points that define the shape of line, QBC, CBC, CA, AND EA 
%quadratic B\'{e}zier, cubic B\'{e}zier, circular arc, and elliptical arc 
are $2$, $3$, $4$, $3$, and $3$, respectively. 
Formally, the control points along with color parameter are defined as:  
% \begin{equation}
% $[x_0, y_0, x_1, y_1, {\#color}]$ (line), \\
% $[x_0, y_0, x_1, y_1, x_2, y_2, {\#color}]$ (QBC), \\
% $[x_0, y_0, x_1, y_1, x_2, y_2, x_3, y_3, {\#color}]$ (CBC), \\
% $[x_0, y_0, x_1, y_1, x_2, y_2, {\#color}]$ (CA), and \\ $[x_0, y_0, x_1, y_1, x_2, y_2, {\#color}]$ (EA). \\ 
\begin{align*}
&[x_0, y_0, x_1, y_1, {\#color}] \longrightarrow (line), \\
&[x_0, y_0, x_1, y_1, x_2, y_2, {\#color}] \longrightarrow (QBC), \\
&[x_0, y_0, x_1, y_1, x_2, y_2, x_3, y_3, {\#color}] \longrightarrow (CBC), \\
&[x_0, y_0, x_1, y_1, x_2, y_2, {\#color}] \longrightarrow (CA), and \\ &[x_0, y_0, x_1, y_1, x_2, y_2, {\#color}]\longrightarrow  (EA). 
\end{align*}

Here, $x_n, y_n$ denote the control points of curves and $\#color$ indicates the associated grey scale curve intensity. The stroke construction module can be stated as: 
\[ S\_Strokes_N = StrokeConstruct(S), \hspace{4mm} \text{where} \hspace{1mm} S\_Strokes_N=\{s_0, s_1, s_2, \ldots , s_N\} \]

Here, $S$ denotes sketch and the $StrokeConstruct$ converts the pixel input into a set of geometric curves in terms of line, QBC, CBC, CA, and EA. This can be achieved through any vectorizing method, such as \cite{vectwebsite}.

\subsubsection{Stroke Sequence Generator}
Upon obtaining the vector stroke set of the input sketch $(S\_Strokes_N)$, we first organize the strokes into coherent groups based on proximity. In the context of perceptual grouping, in Gestalt theory\cite{li2018universal}, \textit{similarity} and \textit{proximity} are the most influential human factors. Various works have explored these factors in the context of image segmentation \cite{chen2017deeplab,wang2017unsupervised,ren2003learning} and sketch segmentation \cite{sun2012free,qi2013sketching,schneider2016example}. 

Since we are only interested in stroke sequencing, we drop the similarity measure and probe only the proximity measure for vector curves to build clusters. Specifically, we explore the Hierarchical Clustering method \cite{zhao2005hierarchical} by calculating pairwise distance on coordinates of the starting control points to generate stroke clusters. 
This notion of proximity-grouping helps us to segment the input sketch into semantically smaller groups. 
The stroke clusters of the input image can be obtained as
\[S\_Stroke\_clusters = Hierarchical\_clustering (S\_Strokes_N, dist_{prox})\] 
Here, $dist_{prox} = 0$ makes each vector as an individual group, and $dist_{prox} = \infty$ forms all vectors as a single group. 


\subsubsection{Intra-Cluster Sequencing}With the proximity based hierarchical clustering, each stroke cluster $S\_Stroke\_clusters_i$ within $S\_Stroke\_clusters$ establishes an ordered sequence for the strokes $(s_0, s_1,$ $\ldots, s_n)$ contained in that cluster and thus supports the inherent \textit{intra-cluster} sequencing. Specifically, we exploit Ward’s-based linkage matrix from hierarchical clustering to construct an ordered sequence of strokes within each cluster by minimizing variance and implicitly suggesting proximity-based sequencing.

\subsubsection{Inter-Cluster Sequencing}To further regulate the sequencing order of retrieved clusters, we employ the optimization method on generated clusters. Particularly, we select the center coordinates of each cluster as reference points and adopt the formulation of the classical ``\textit{Travelling Salesman Problem''} (TSP). Formally, given a set of $M$ stroke clusters $S\_Stroke\_Clusters$ = $\{C_1, C_2,$ $\ldots, C_M\}$, 
where each cluster \(C_i\) is represented by its centroid coordinates (\(P_{ij}=(p_i, p_j)\)), 
the objective is to come up with the shortest possible trip that visits each stroke cluster exactly once and returns to the starting cluster. This notion helps to achieve \textit{inter-cluster} sequencing over ordered strokes.  

\[
S\_Stroke\_seq = TSP(S\_Stroke\_Clusters_M (P_{ij}))
\]

Here, the decision variables and objective function remain unaltered, i.e., it uses the binary decision variable to determine the tour between the clusters and the objective function is ${Minimize} \sum_{i=1}^{M} \sum_{j=1, j \neq i}^{M} c_{ij} \cdot b_{ij}$
where \(c_{ij}\) represents the distance between the centroids of stroke clusters \(i\) and \(j\). 
And, \(b_{ij}\) is a binary decision variable indicating whether there is a direct tour between stroke clusters \(C_i\) and \(C_j\). 

\subsection{Paint Stroke Construction and Sequencing} \label{pscs}
Similar to Sketch Stroke Construction and Sequencing (SSCS) as presented above, we formulate RGB stroke sequencing for input art towards colouring the constructed sketch, referred as Paint Stroke Construction and Sequencing (PSCS). 
The layout of the PSCS is depicted in Figure \ref{fig:Paint_seq}. Given an RGB input, our paint stroke construction module harvests the set of coloured vector strokes ($P\_Strokes_T$). 

\begin{figure}[!t]
    \centering
    \includegraphics[width=\textwidth]{Samples/Fig4_Paint_Strokes.pdf}
    \caption{Framework for paint stroke construction and sequencing}
    \label{fig:Paint_seq}
\end{figure} 

\[ P\_Strokes = StrokeConstruct(X) \hspace{4mm} \text{where} \hspace{1mm} P\_Strokes_T={s_0, s_1, s_2, \ldots , s_T.} \]

Here, $X$ denotes input art and the $StrokeConstruct$ produces the RGB geometric curves in terms of line, QBC, CBC, CA, and EA. 
The fundamental difference between sketch strokes and paint strokes lies in their intensity values. Sketch strokes are usually gray-scale strokes, consisting varying shades of gray. 
Whereas, paint strokes utilize RGB color information to capture color appearance. 

As in sketch stroke sequence generator, we employ Hierarchical Clustering method on coordinates of the starting control points of RGB vector curves to articulate semantic segments. 
Further, TSP based optimization method is incorporated on extracted clusters to determine sequencing order. 
\[P\_Stroke\_clusters = Hierarchical\_clustering (P\_Strokes_T, dist_{prox}).\] 
\[
P\_Stroke\_seq = TSP(P\_Stroke\_Clusters_V (Q_{ij}))
\]
Considering that users commonly sketch before applying color, we stream SSCS before integrating it to PSCS, after computing their respective sequence order. 

\begin{equation}
    Stroke\_seq_{\text{Global}} = (S\_Stroke\_seq + P\_Stroke\_seq)
\end{equation}

Here, $Stroke\_seq_{Global}$ portrays the stroke-by-stroke evolution of artworks from initial sketch to final execution.

We summarize our method for \textit{Sketch \& Paint} as Algorithm \ref{alg:cap} below.

\begin{algorithm}
\caption{Sketch \& Paint Stroke Construction and Sequence Generation}\label{alg:cap}
\begin{algorithmic}
\Require Input image $I$
\Ensure Final stroke sequence $Stroke\_seq_{Global}$
\State Read input image $I$
\State Generate Sketch ($S$) from $I$ by generating line drawing of an input image 
    \State \hspace{\algorithmicindent} $Sketch \leftarrow SketchGenerator(I)$
    \State Extract set of strokes via converting pixel sketch into geometric curves
    \State \hspace{\algorithmicindent}$Strokes_{Sketch}$: $S\_Strokes_N \leftarrow StrokeConstruct(S)$
    %\STATE $Strokes_N \leftarrow Strokes$
    
    \State Employ proximity based hierarchical clustering on $S\_Strokes_N$
    \State \hspace{\algorithmicindent} $S\_Stroke\_clusters_M \leftarrow \text{Hierarchical\_clustering}(S\_Strokes_N, dist_{\text{proximity}})$
    %$Stroke\_clusters$:
    \State Devise TSP on centroids of stroke clusters
    \State \hspace{\algorithmicindent} $S\_Stroke\_seq = TSP(S\_Stroke\_Clusters_M (P_{ij}))$
\If{painting == TRUE (for an RGB image ($X$))}
        \State Extract set of strokes via converting pixel image into geometric curves
        \State \hspace{\algorithmicindent}$Strokes_{RGB}$: $P\_Strokes_T \leftarrow StrokeConstruct(X)$
        \State Employ proximity based hierarchical clustering on $P\_Strokes_T$
    \State \hspace{\algorithmicindent}$P\_Stroke\_clusters_V \leftarrow \text{Hierarchical\_clustering}(P\_Strokes_T, dist_{\text{proximity}})$
    %$Stroke\_clusters$: 
    \State Devise TSP on centroids of stroke clusters
    \State \hspace{\algorithmicindent} $P\_Stroke\_seq = TSP(P\_Stroke\_Clusters_V (Q_{ij}))$
    \EndIf
    \State  $Stroke\_seq_{Global} \leftarrow$ $S\_Stroke\_seq$ + $P\_Stroke\_seq$
    
    \Return $Stroke\_seq_{Global}$
\end{algorithmic}
\end{algorithm}
