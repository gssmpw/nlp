\section{Discussion and Conclusion}
In this work, we addressed the challenge of constructing stroke sequences for unlabeled artworks - from initial sketch strokes to final painting strokes. Specifically, we formulate a method for ordering stroke sequences through clustering and optimizing vector curves to facilitate a pragmatic drawing process. Our approach significantly advances the understanding of artwork evolution by generating stroke sequences for complex artworks. We validated the effectiveness of our algorithm across various data forms, demonstrating its capability to manage diverse inputs and produce coherent drawing processes. 

Our approach, at the moment, is agnostic to the semantic content of the artwork. The alignment performance of the existing algorithm might further improve if we introduce region-based grouping (for example, by leveraging models such as  Segment Anything Models (SAM)) as artists tend to complete one semantic region before moving on to another. Additionally, we might benefit from the datasets that capture the artist's stroke sequences on the drawings. Alternatively, one can explore using art annotators to establish stroke sequences for a limited number of artworks. Implementing human feedback mechanisms or preference-based modeling could align the algorithm more closely with human drawing styles. We believe that these enhancements would not only deepen our comprehension of the drawing process but also open up possibilities for practical applications, including next-stroke generation, sketch completion, and artwork retrieval. Moreover, this type of work provides a foundation for enriching art education and appreciation by offering insights into the sequential evolution of artworks.



