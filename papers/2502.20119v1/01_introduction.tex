\section{Introduction}
Visual art is an essential part of humanity. It allows us to explore, express, and communicate ideas, emotions, perspectives, and experiences. Visual arts encompass various mediums such as drawing, painting, sculpture, and photography. In the realm of drawings and paintings, we encounter a wide range of styles, themes, and artistic movements that reflect the creativity and cultural diversity of human existence. Understanding the complex landscape of art development is important for art education and appreciation.

Digital technology can help bridge the gap between traditional art forms and modern accessibility \cite{brown2020routledge, carrozzino2010beyond}. For example, high-resolution digital reconstructions for artworks allow museums to create virtual exhibits with interactive features such as zooming into specific painting parts or viewing various stroke forms of artwork, thereby enhancing visitor understanding and engagement \cite{carvajal2020virtual, carrozzino2010beyond}.  Similarly, understanding the \textit{process of constructing a painting} is extremely educative.  Investigating the extraction of strokes from famous artworks and constructing a stroke order can provide significant insights into the creative process and foster a deeper understanding of artistic craftsmanship and innovation.  Moreover, detailed dynamic digital reconstruction of artworks also serves as a valuable educational resource \cite{sylaiou2017exploring, lockee2014visual}, enabling students and researchers to closely study historical paintings, understand the used techniques, and explore the artist’s process. It can be surmised that by studying and emulating the techniques of master artists through digital reconstructions, students can develop their skills in painting and drawing, gaining insights into fundamental aspects such as brush control, color mixing, and composition. 

Interactive models that help understand the drawing process, therefore, can prove to be extremely useful. However, building methods that mimic the pragmatic drawing process is extremely challenging. It is hard to define the semantics of each piece of art, the geometries of sketch/painting, the number and the order of strokes, the individual stroke attributes like length, color, shape, \& texture, and the overall evolution of an artwork.  The fundamental challenges in building the interactive drawing model are: (i) How do we represent and extract stroke-level information for complex art (ii) What is the plausible way to construct the sequence order of strokes that mimic a drawing process?

Some works have explored the explication of the drawing process in the recent past. Fu \textit{et al} \cite{fu2011animated} introduce an algorithm designed to animate pre-drawn line drawings by determining stroke order, though the method struggles with accuracy over complex sketches. Subsequently, Sketch-RNN \cite{ha2017neural} learns the construction of stroke sequences of hand-drawn sketches by training an ML model on thousands of human-drawn images. However, its learning is based on the available labeled stroked data and it does not learn directly from the sketches. Furthermore, the method does not scale for complex sketches involving shading and textures. Recently, Tong \textit{et al.} \cite{tong2021sketch} show advancement with an image-to-pencil translation method that produces %high-quality 
sketches and demonstrates the drawing process. Additionally, several works \cite{vinker2023clipascene, vinker2022clipasso} have investigated new levels of abstraction in object sketching through geometric and semantic simplifications. Despite these advancements, accurately recreating the intricate details of complex artworks remains a significant challenge. Furthermore, neural painting techniques \cite{liu2021paint, huang2019learning, song2024processpainter} employing reinforcement learning have attempted to generate stroke sequences for non-photo-realistic image recreation. Nevertheless, these methods struggle with the high computational demands of deep reinforcement learning and lack an inherent sequence order while generating strokes.

With these observations, we explore the extraction of stroke sequences from artworks, aiming to construct a step-by-step process, that an artist might use for the evolution of an artwork from an initial sketch to the final complete painting (E.g., as shown in Figure \ref{fig:Constructed}). We leverage Scalable Vector Graphics (SVG) 
\cite {zhang2023towards,liu2009beyond,carlier2020deepsvg,mateja2023animatesvg} to extract stroke-level data through inverse graphics as in \cite{das2020beziersketch, romaszko2017vision,kulkarni2015deep,rodriguez2023starvector}. We choose to represent the input image in terms of parameterized curves such as B\'{e}zier curves due to their simple structure and encoding ability of complex and finer details of the input image.

As discussed, the fundamental requirements of the interactive generation model are stroke-by-stroke data of input and the sequence order in which the evolution takes place, as if the art is drawn by a human. To achieve this, in our work,  we explore perceptual grouping inspired by the theory of Gestalt laws \cite{wagemans2012century} to reason out the sequence order for unlabeled stroke data and mimic the pragmatic drawing process for sketching and painting.
 
Our work differentiates from many of the present methods such as animated drawing \cite{fu2011animated}, Sketch-RNN \cite{ha2017neural}, CLIPasso \cite{vinker2022clipasso}, or paint transformer \cite{liu2021paint} in that we suggest an integrated mechanism for the evolution of artwork from both \textit{sketch and paint} perspective and that we aim at recreating the original artwork. The constructed stroke sequence not only enables building interactive sketching and painting models but can also be useful for downstream tasks like art completion, manipulation, generation, and retrieval. 

The contributions of our work can be summarized as: 

\begin{itemize}
    \item We present a simple yet effective procedural algorithm to construct the stroke sequencing order for stroke evolution. The robustness of the proposed method enables us to handle large numbers of strokes, thereby facilitating the examination of complex images.
    \item To the best of our knowledge, we are the first to address the stroke-by-stroke evolution of complex artworks and natural images. 
    \item In comparison to other methods that limit stroke order to sketches, our work, termed \textit {sketch \& paint}, presents the stroke-by-stroke sequencing for both sketching and color painting.
    \item Further, we demonstrate the generalizability of the proposed method on various forms of visual data like sketches, clip art, and natural images. We validate the efficacy of the proposed method using samples extracted from publicly available datasets like WikiArt, VectroFlow, and FS2K-SDE.
\end{itemize}

