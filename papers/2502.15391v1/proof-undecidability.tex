\newcommand{\Instrs}{\mathit{Instrs}}
\newcommand{\Count}{\mathit{Count}}
\newcommand{\Minsk}{\mathit{Minsk}}

To show that the two mentioned problems are undecidable,
we propose a reduction from the halting problem for a (two-counter) Minsky machine \cite{minsky-computation-67}.
A Minsky machine is a finite sequence of instructions $\Instrs$ manipulating two natural variables $c1$ and $c2$,
called counters, and each instruction has one of the following form:
\begin{enumerate}
\item $\ell:ci=ci+1;\mathtt{goto}~\ell';$
\item $\ell:\mathtt{if}~ci==0?~\mathtt{goto}~ \ell'; \mathtt{else}~ci=ci-1;\mathtt{goto}~\ell'';$
\end{enumerate}
where $\ell$.$\ell'$ and $\ell''$ are labels.
All labels are followed by an instruction except $\ell_f$, the final label.
The halting problem asks if the Minsky machine starting from an initial label $\ell_0$
with the two counters set to $0$ halts, \ie reaches the label $\ell_f$.

\begin{figure}[htbp]
\begin{center}
\scalebox{0.9}{
\begin{tikzpicture}[node distance=1.5cm]
  \tikzstyle{every state}=[inner sep=3pt,minimum size=20pt]

  \node(Q)[petri-p,draw=black,label=180:{$\ell$}]{};
  \node(ReqI1)[petri-t2,draw=black,fill=black,right of=Q,label=90:{$\ell\_\mathit{reqi1}$},xshift=-2em]{};
  \node(aux1)[petri-p,draw=black,right of=ReqI1,xshift=-2em]{};
  \node(AckI1)[petri-t2,draw=black,fill=black,right of=aux1,label=90:{$\ell\_\mathit{acki1}$},xshift=-2em]{};
  \node(R)[petri-p,draw=black,right of=AckI1,xshift=-2em,label=00:{$\ell'$}]{};

  \path (Q) edge [->,thick,line width=1pt](ReqI1);
  \path (ReqI1) edge [->,thick,line width=1pt](aux1);
  \path (aux1) edge [->,thick,line width=1pt](AckI1);
  \path (AckI1) edge [->,thick,line width=1pt](R);

 \node(Q2)[petri-p,draw=black,label=180:{$\ell$},right of=R,xshift=2em]{};
  \node(ReqD1)[petri-t2,draw=black,fill=black,above right of=Q2,label=90:{$\ell\_\mathit{reqd1}$},xshift=-2em]{};
  \node(aux2)[petri-p,draw=black,right of=ReqD1,xshift=-2em]{};
  \node(AckD1)[petri-t2,draw=black,fill=black,right of=aux2,label=90:{$\ell\_\mathit{ackd1}$},xshift=-2em]{};
  \node(R2)[petri-p,draw=black,right of=AckD1,xshift=-2em,label=00:{$\ell''$}]{};
  \node(Zero1)[petri-t2,draw=black,fill=black,below right of=Q2,label=-90:{$\ell\_\mathit{zero1}$},xshift=-2em]{};
   \node(S)[petri-p,draw=black,right of=Zero1,xshift=-2em,label=00:{$\ell'$}]{};
  

  \path (Q2) edge [->,thick,line width=1pt,out=90,in=180](ReqD1);
  \path (ReqD1) edge [->,thick,line width=1pt](aux2);
  \path (aux2) edge [->,thick,line width=1pt](AckD1);
  \path (AckD1) edge [->,thick,line width=1pt](R2);
  \path (Q2) edge [->,thick,line width=1pt,out=-90,in=180](Zero1);
  \path (Zero1) edge [->,thick,line width=1pt](S);

  \node(lab1)[below of=ReqI1,xshift=1em]{(a) Encoding $\ell:c1=c1+1;\mathtt{goto}~\ell';$};
   \node(lab1)[below of=Zero1,xshift=2em]{(b) Encoding $\ell: \begin{array}{l}\mathtt{if}~c1==0?~\mathtt{goto}~ \ell';\\ \mathtt{else}~c1=c1-1;\mathtt{goto}~\ell'';\end{array}$};

\end{tikzpicture}
}
\end{center}
\vspace*{-\baselineskip}
\caption{Simulating a Minsky machine with process type $\Minsk$}
\label{fig:proc-minsk}
\end{figure}


\begin{figure}[htbp]
\begin{center}
\scalebox{0.7}{
\begin{tikzpicture}[node distance=1.5cm]
  \tikzstyle{every state}=[inner sep=3pt,minimum size=20pt]
  
\node(0C1)[petri-p,draw=black,label=00:{$\mathit{0C1}$}]{};
\node(Z1)[petri-t2,draw=black,fill=black,above of=0C1,label=90:{$\mathit{z1}$}]{};
\node(I1)[petri-t,draw=black,fill=black,below left of=0C1,label=180:{$\mathit{i1}$}]{};
\node(D1)[petri-t,draw=black,fill=black,below right of=0C1,label=0:{$\mathit{ad1}$},yshift=2em]{};
\node(0to1)[petri-p,draw=black,below of=I1,yshift=2em]{};
\node(1to0)[petri-p,draw=black,below of=D1,yshift=2.5em]{};
\node(AI1)[petri-t,draw=black,fill=black,below of=0to1,label=180:{$\mathit{ai1}$},yshift=2em]{};
\node(NZ1)[petri-t,draw=black,fill=black,below of=1to0,label=0:{$\mathit{nextz1}$},yshift=2.5em]{};
\node(1to0b)[petri-p,draw=black,below of=NZ1,yshift=2.5em]{};
\node(AD1)[petri-t,draw=black,fill=black,below of=1to0b,label=0:{$\mathit{d1}$},yshift=2.5em]{};
\node(1C1)[petri-p,draw=black,below right of=AI1,label=90:{$\mathit{1C1}$}]{};
\node(PI1)[petri-t2,draw=black,fill=black,below left of=1C1,label=90:{$\mathit{pi1}$},yshift=2em]{};
\node(pi1toqi1)[petri-p,draw=black,left of=PI1,xshift=2em]{};
\node(QI1)[petri-t2,draw=black,fill=black,left of=pi1toqi1,label=90:{$\mathit{qi1}$},xshift=2em]{};
\node(qi1tori1)[petri-p,draw=black,below left of=QI1,yshift=1em]{};
\node(RI1)[petri-t2,draw=black,fill=black,below right of=qi1tori1,label=-90:{$\mathit{ri1}$},yshift=1em]{};
\node(ri1tosi1)[petri-p,draw=black,right of=RI1,xshift=-2em]{};
\node(SI1)[petri-t2,draw=black,fill=black,right of=ri1tosi1,label=-90:{$\mathit{si1}$},xshift=-2em]{};

\node(PD1)[petri-t2,draw=black,fill=black,below right of=1C1,label=90:{$\mathit{pd1}$},yshift=2em]{};
\node(pd1toqd1)[petri-p,draw=black,right of=PD1,xshift=-2em]{};
\node(QD1)[petri-t2,draw=black,fill=black,right of=pd1toqd1,label=90:{$\mathit{qd1}$},xshift=-2em]{};
\node(qd1tord1)[petri-p,draw=black,below right of=QD1,yshift=1em]{};
\node(RD1)[petri-t2,draw=black,fill=black,below left of=qd1tord1,label=-90:{$\mathit{rd1}$},yshift=1em]{};
\node(rd1tosd1)[petri-p,draw=black,left of=RD1,xshift=2em]{};
\node(SD1)[petri-t2,draw=black,fill=black,left of=rd1tosd1,label=-90:{$\mathit{sd1}$},xshift=2em]{};


\path (0C1) edge [->,thick,line width=1pt,out=30,in=0](Z1);
\path (Z1) edge [->,thick,line width=1pt,out=180,in=150](0C1);
\path (0C1) edge [->,thick,line width=1pt,out=-150,in=90](I1);
\path (D1) edge [->,thick,line width=1pt,out=90,in=-30](0C1);
\path (I1) edge [->,thick,line width=1pt](0to1);
\path (1to0b) edge [->,thick,line width=1pt](NZ1);
\path (NZ1) edge [->,thick,line width=1pt](1to0);
\path (1to0) edge [->,thick,line width=1pt](D1);
\path (0to1) edge [->,thick,line width=1pt](AI1);
\path (AD1) edge [->,thick,line width=1pt](1to0b);
\path (AI1) edge [->,thick,line width=1pt,out=-90,in=150](1C1);
\path (1C1) edge [->,thick,line width=1pt,out=30,in=-90](AD1);

\path (1C1) edge [->,thick,line width=1pt](PI1);
\path (PI1) edge [->,thick,line width=1pt](pi1toqi1);
\path (pi1toqi1) edge [->,thick,line width=1pt](QI1);
\path (QI1) edge [->,thick,line width=1pt](qi1tori1);
\path (qi1tori1) edge [->,thick,line width=1pt](RI1);
\path (RI1) edge [->,thick,line width=1pt](ri1tosi1);
\path (ri1tosi1) edge [->,thick,line width=1pt](SI1);
\path (SI1) edge [->,thick,line width=1pt](1C1);

\path (1C1) edge [->,thick,line width=1pt](PD1);
\path (PD1) edge [->,thick,line width=1pt](pd1toqd1);
\path (pd1toqd1) edge [->,thick,line width=1pt](QD1);
\path (QD1) edge [->,thick,line width=1pt](qd1tord1);
\path (qd1tord1) edge [->,thick,line width=1pt](RD1);
\path (RD1) edge [->,thick,line width=1pt](rd1tosd1);
\path (rd1tosd1) edge [->,thick,line width=1pt](SD1);
\path (SD1) edge [->,thick,line width=1pt](1C1);

 \node[petri-tok] at (0C1) {};

\end{tikzpicture}
}
\end{center}
\vspace*{-\baselineskip}
\caption{Process type $\Count_1$ to simulate the first counter}
\label{fig:proc-count1}
\end{figure}


We fix a Minsky machine given by its set of instructions $\Instrs$.
In order to simulate this model, we consider a family of systems which all have the following shape:
there is a central vertex which simulates the instructions of the machine,
on the right side of this vertex there is a sequence of vertices to encode the value of the counter $c1$
and on its left side a sequence of vertices to encode the value of the counter $c2$.
The central node has process type $\Minsk$, it is represented on \figref{fig:proc-minsk},
where we notice that there is a place for each instruction label
and furthermore the initial marking only puts a token in the place $\ell_0$.
The vertices to encode the value of counter $c1$ are labeled by the process type $\Count_1$
depicted on \figref{fig:proc-count1} and the vertices to encode the value of counter $c2$
are labeled the process type $\Count_2$ which is similar to $\Count_1$.
We remark that in the process types $\Count_1$, there are two specific places $0C1$ and $1C1$,
and that initially the place $0C1$ is marked.
During a simulation of the machine with a behavior,
the number of vertices of process type $\Count_1$ for which a token is in place $1C1$
corresponds to the current value of $c1$.
To ease the explanation, we shall call the central vertex,
the controller and the vertices labeled by $\Count_i$ the counting processes for $c_i$, for $i=1,2$.
The grammar $\grammar_\Minsk$ producing the desired family of systems is given on
\figref{fig:grammmar-minsk} and an example of a system belonging to
$\alangof{}{\algof{S}}{\grammar_{\Minsk}}$ is depicted on \figref{fig:undec4}
(we omit the label on the edges to ease presentation).

We now provide the main ingredients of the simulation.
At the beginning the controller is in state $\ell_0$
and all the counting processes are in states $0C1$ or $0C2$ according to their process type,
$\Count_1$ or $\Count_2$.
At any `big step' (\ie when forgetting intermediate steps needed for the correctness) of the simulation,
when the controller is in a state corresponding to a label of the Minsky machine,
the sequence of the counting processes of type $\Count_1$ will always have the following form:
it begins with a sequence of processes in state $1C1$ and ends with a sequence of processes in state $0C1$
(the same property holds for the counting processes of type $\Count_2$ considering the state $1C2$ and $0C2$).
Moreove, the number of processes in state $1C1$ (\resp $1C2$) represents the value of the first (\resp second)
counter at this stage of the simulation. We have then the following behavior:
\begin{itemize}
    \item when the controller simulates an instruction of the form $\ell:c1=c1+1;\mathtt{goto}~\ell';$
        it begins to request for an increment with the transition $\ell\_reqi1$ and it waits for its acknowledgment with $\ell\_acki1$.
        The request is then transmitted by all the processes in state $1C1$
        to the first counting process in state $0C1$ which changes its state and acknowledges it moving to $1C1$,
        the acknowledgment being transmitted back to the controller.
        If there is no process of type $\Count_1$ in state $0C1$, the simulation is stuck,
        it means that the system is not `big' enough to simulate correctly the Minsky machine.
    \item when the controller simulates an instruction of the form
        $\ell:\mathtt{if}~c1==0?~\mathtt{goto}~ \ell';\linebreak[0] \mathtt{else}~c1=c1-1;\mathtt{goto}~\ell'';$
        and the counter $c1$ is equal to $0$,
        this means that there is no counting process of type $\Count_1$ in state $1C1$,
        then the controller takes the transition $\ell\_zero1$ together with the first counting process of type
        $\Count_1$ which takes the transition $z1$.
    \item when the controller simulates an instruction of the form
        $\ell:\mathtt{if}~c1==0?~\mathtt{goto}~ \ell';\linebreak[0] \mathtt{else}~c1=c1-1;\mathtt{goto}~\ell'';$
        and the counter $c1$ has a strictly positive value, then, as for the increment,
        the controller requests a decrement with the transition $\ell\_decqi1$,
        this request is transmitted to the last counting process of type $\Count_1$ in state $1C1$
        (there is necessarily one) and if this last process has a right neighbor whose state is $0C1$,
        then it moves to $0C1$ acknowledging the decrement,
        this acknowledgment being transmitted back to the controller which goes in state $\ell''$.
        Note that each counting process in state $1C1$ can choose nondeterministically
        whether it is or not the last one in state $1C1$ by taking the transition $pd1$
        (it transmits the decrement request) or $d1$ (it chooses it is the last one),
        but if it makes a bad choice, the simulation will be stuck.
    \item Finally, we have that the machine halts if and only if there is a system and an execution
        in its associated behavior where the controller ends in $\ell_f$.
\end{itemize}


We now provide an example on how the transmission of the information is performed in a system.
We consider a Minsky machine with two instructions
$\ell_0:c1=c1+1;\mathtt{goto}~\ell';$ and $\ell':c1=c1+1;\mathtt{goto}~\ell'';$.
\figref{fig:behav} presents a partial representation of a behavior simulating these instructions
with two counter processes of type $\Count_1$
(to make the figure readable, we put a dotted line between two transitions when they are synchronized).
First, the  controller simulates the instruction $\ell_0:c1=c1+1;\mathtt{goto}~\ell';$,
it takes the transition $\ell_0\_reqi1$ which says that it requests an increment,
this transitions is `synchronized' with two transitions $i1$ and $pi1$ of the first counting process
on its right and since the state of this process is initially $0C1$
then this first process takes the transition $i1$.
Afterwards, the controller waits for an acknowledgment of its request for an increment
which will arrive when it takes the transition $\ell_0\_acki1$.
This latter transition is synchronized with the two transitions $ai_1$ and $si_1$
of the first counting process.
In our case, the pair $(\ell_0\_acki1,ai_1)$ takes place and the controller ends in state $\ell'$
and the first counting process in $1C1$.
The controller can now simulate $\ell':c1=c1+1;\mathtt{goto}~\ell'';$.
It takes the transition $\ell'\_reqi1$, but this time the first counting process (which is in state $1C1$)
takes the transition $pi1$ signifying it transmits the request to its neighbor (if there is one).
Afterwards, if the first counting process has a neighbor, it  fires the transition $qi1$
(which synchronizes with the transitions $i1$ and $pi1$ of its right neighbor),
and its neighbor which is in state $0C1$ takes the transition $i1$,
they then both move with the pair $(ri1,ai1)$ and the second counting process arrives in state $1C1$
whereas the first counting process is able to acknowledge the fact that the increment has been performed
to the controller process with the synchronization $(\ell'\_acki1,si_1)$.
We end in a configuration where the controller is in state $\ell''$
and the two first counting processes of type $\Count_1$ are in state $1C1$.

\begin{figure}
\begin{minipage}{1\textwidth}
\begin{center}
  $$
  \begin{array}{ccl}
     & \rightarrow &X_1\\
    X_1 & \rightarrow & X_2\\
    X_1 & \rightarrow & \restrict{\slabs}{\algof{S}}(\rename{\alpha_1}{\algof{S}}( X_1 \pop{\algof{S}}Y_1)) \\
    X_2 & \rightarrow & Init\\
    X_2 & \rightarrow & \restrict{\slabs}{\algof{S}}(\rename{\alpha_2}{\algof{S}}( Y_2 \pop{\algof{S}}X_2))\\
    
    Init & \rightarrow & \bigoplus^{\algof{S}}_{\ell \in L_{inc1}}\big (\sgraph{(\ell\_reqi1,i1)}{\asrc_0}{\asrc_1}{\algof{S}} \pop{\algof{S}} \sgraph{(\ell\_reqi1,pi1)}{\asrc_0}{\asrc_1}{\algof{S}} \pop{\algof{S}}\\
    & & \sgraph{(\ell\_acki1,ai1)}{\asrc_0}{\asrc_1}{\algof{S}} \pop{\algof{S}} \sgraph{(\ell\_acki1,si1)}{\asrc_0}{\asrc_1}{\algof{S}}\big) \pop{\algof{S}}\\
    & & \bigoplus^{\algof{S}}_{\ell \in L_{dec1}} \big( \sgraph{(\ell\_zero1,z1)}{\asrc_0}{\asrc_1}{\algof{S}} \pop{\algof{S}} \\
    & & \sgraph{(\ell\_reqd1,d1)}{\asrc_0}{\asrc_1}{\algof{S}} \pop{\algof{S}} \sgraph{(\ell\_reqd1,pd1)}{\asrc_0}{\asrc_1}{\algof{S}} \pop{\algof{S}}\\
    & & \sgraph{(\ell\_ackd1,ad1)}{\asrc_0}{\asrc_1}{\algof{S}} \pop{\algof{S}} \sgraph{(\ell\_ackd1,sd1)}{\asrc_0}{\asrc_1}{\algof{S}}\big) \pop{\algof{S}}\\
    & & \bigoplus^{\algof{S}}_{\ell \in L_{inc2}}\big( \sgraph{(i2,\ell\_reqi2)}{\asrc_2}{\asrc_0}{\algof{S}} \pop{\algof{S}} \sgraph{(pi2,\ell\_reqi2)}{\asrc_2}{\asrc_0}{\algof{S}} \pop{\algof{S}}\\
    & & \sgraph{(ai2,\ell\_acki2)}{\asrc_2}{\asrc_0}{\algof{S}} \pop{\algof{S}} \sgraph{(si2,\ell\_acki2)}{\asrc_2}{\asrc_0}{\algof{S}}\big) \pop{\algof{S}}\\
    & & \bigoplus^{\algof{S}}_{\ell \in L_{dec2}}\big( \sgraph{(z2,\ell\_zero2)}{\asrc_2}{\asrc_0}{\algof{S}} \pop{\algof{S}} \\
    & & \sgraph{(d2,\ell\_reqd2)}{\asrc_2}{\asrc_0}{\algof{S}} \pop{\algof{S}} \sgraph{(pd2,\ell\_reqd2)}{\asrc_2}{\asrc_0}{\algof{S}} \pop{\algof{S}}\\
    & & \sgraph{(ad2,\ell\_ackd2)}{\asrc_2}{\asrc_0}{\algof{S}} \pop{\algof{S}} \sgraph{(sd2,\ell\_ackd2)}{\asrc_2}{\asrc_0}{\algof{S}}\big)\\
    
    Y_1 & \rightarrow & \sgraph{(qi1,i1)}{\asrc_1}{\asrc_3}{\algof{S}} \pop{\algof{S}} \sgraph{(qi1,pi1)}{\asrc_1}{\asrc_3}{\algof{S}} \pop{\algof{S}}\\
    & & \sgraph{(ri1,ai1)}{\asrc_1}{\asrc_3}{\algof{S}} \pop{\algof{S}} \sgraph{(ri1,si1)}{\asrc_1}{\asrc_3}{\algof{S}} \pop{\algof{S}}\\
    & & \sgraph{(qd1,d1)}{\asrc_1}{\asrc_3}{\algof{S}} \pop{\algof{S}} \sgraph{(qd1,pd1)}{\asrc_1}{\asrc_3}{\algof{S}} \pop{\algof{S}}\\
    & & \sgraph{(rd1,ad1)}{\asrc_1}{\asrc_3}{\algof{S}} \pop{\algof{S}} \sgraph{(rd1,sd1)}{\asrc_1}{\asrc_3}{\algof{S}}\pop{\algof{S}}\\
 & & \sgraph{(nextz1,z1)}{\asrc_1}{\asrc_3}{\algof{S}}\\
    
    Y_2 & \rightarrow & \sgraph{(i2,qi2)}{\asrc_4}{\asrc_2}{\algof{S}} \pop{\algof{S}} \sgraph{(pi2,qi2)}{\asrc_4}{\asrc_2}{\algof{S}} \pop{\algof{S}}\\
    & & \sgraph{(ai2,ri2)}{\asrc_4}{\asrc_2}{\algof{S}} \pop{\algof{S}} \sgraph{(si2,ri2)}{\asrc_4}{\asrc_2}{\algof{S}} \pop{\algof{S}}\\
    & & \sgraph{(d2,qd2)}{\asrc_4}{\asrc_2}{\algof{S}} \pop{\algof{S}} \sgraph{(pd2,qd2)}{\asrc_4}{\asrc_2}{\algof{S}} \pop{\algof{S}}\\
    & & \sgraph{(ad2,rd2)}{\asrc_4}{\asrc_2}{\algof{S}} \pop{\algof{S}} \sgraph{(sd2,rd2)}{\asrc_4}{\asrc_2}{\algof{S}}\pop{\algof{S}}\\
 & & \sgraph{(z2,nextz2)}{\asrc_4}{\asrc_2}{\algof{S}}\\
  \end{array}
  $$
\end{center}
\end{minipage}
\vspace{3em}

\begin{minipage}{\textwidth}
  \begin{center}
    with $\ptypeof{\asrc_0}=\Minsk$, $\ptypeof{\asrc_1}=\ptypeof{\asrc_3}=\Count_1$, $\ptypeof{\asrc_2}=\ptypeof{\asrc_4}=\Count_2$,\\$\alpha_1=\{\asrc1 \leftrightarrow \asrc3\}$, \\$\alpha_2=\{\asrc2 \leftrightarrow \asrc4\}$,\\$\slabs=\{\asrc1,\asrc2\}$,\\
    $L_{inci}=\{\ell \mid \ell:ci=ci+1;\mathtt{goto}~\ell'; \in Instrs\}$ for $i \in \{1,2\}$\\ $L_{deci}=\{\ell \mid \ell: \mathtt{if}~ci==0?~\mathtt{goto}~ \ell'; \mathtt{else}~ci=ci-1;\mathtt{goto}~\ell''; \in Instrs\}$ for $i \in \{1,2\}$
  \end{center}
  \end{minipage}
\caption{Grammar $\grammar_\Minsk$ to produce systems for the simulation of a Minsky machine}
\label{fig:grammmar-minsk}
\end{figure}



We now present more formal arguments to explain why the reduction holds.
We first describe the systems belonging to the language of the  grammar $\grammar_{\Minsk}$
(presented on \figref{fig:grammmar-minsk}) in the algebras of systems.
Given two naturals $m,n \geq 1$, we define the system
$\asys^{m,n}_{\mathit{\Minsk}}=(\verts,\edges,\vlab)$ where:
\begin{itemize}
    \item $\verts=\{v^2_m,\ldots,v^2_1,v^c,v^1_1,\ldots,v^1_n\}$;
    \item $\edges=$\\$\{(v^2_i,a,v^2_{i-1}) \mid i \in [2,m] \mbox{ and } a \in \{(i2,qi2),(pi2,qi2),(ai2,ri2),(si2,ri2),\linebreak[0](d2,qd2),\linebreak[0](pd2,qd2),(ad2,rd2),(sd2,rd2),,(z2,nextz2)\}\} \cup $\\
    $\{(v^1_i,a,v^1_{i+1}) \mid i \in [1,n-1] \mbox{ and } a \in \{(qi1,i1),(qi1,pi1),(ri1,ai1),(ri1,si1),\linebreak[0](qd1,d1),\linebreak[0](qd1,pd1),(rd1,ad1),(rd1,sd1),(nextz1,z1)\}\} \cup$\\
    $\{(v^2_1,a,v^c) \mid a \in \{(i2,\ell\_reqi2), (pi2,\ell\_reqi2), (ai2,\ell\_acki2),(si2,\ell\_acki2)\mid \ell \in L_{inc2} \}\} \cup$\\
    $\{(v^2_1,a,v^c) \mid a \in \{(z2,\ell\_zero2),(d2,\ell\_reqd2),(pd2,\ell\_reqd2),(ad2,\ell\_ackd2),(sd2,\ell\_ackd2)\mid \ell \in L_{dec2} \}\} \cup$
    $\{(v^c,a,v^1_1) \mid a \in \{(\ell\_reqi1,i1), (\ell\_reqi1,pi1), (\ell\_acki1,ai1),(\ell\_acki1,si1)\mid \ell \in L_{inc1} \}\} \cup$\\
    $\{(v^c,a,v^1_1) \mid a \in \{(\ell\_zero1,z1),(\ell\_reqd1,d1),(\ell\_reqd1,pd1),(\ell\_ackd1,ad1),(\ell\_ackd1,sd1)\mid \ell \in L_{dec1} \}\}$ where $L_{inc1}$, $L_{dec1}$, $L_{inc2}$ and $L_{dec2}$ are defined on \figref{fig:grammmar-minsk};
    \item $\vlab(v^2_i)=\Count_2$ for all $i \in [1,m]$, $\vlab(v^1_i)=\Count_1$ for all $i \in [1,n]$ and $\vlab(v^c)=\Minsk$.

\end{itemize}

We can easily see that
$\alangof{}{\algof{S}}{\grammar_{\Minsk}}=\set{\asys^{m,n}_{\Minsk} \mid m \geq 1 \mbox{ and } n \geq 1}$.

We have given on Figures \ref{fig:proc-count1} and \ref{fig:proc-minsk},
the way process types $\Count_1$, $\Count_2$ and $\Minsk$ are built.
We notice that in the process type $\Minsk$,
there is a single vertex labelled by the halting label of the Minsky machine $\ell_f$. We let  $\amark_f:\mathcal{Q}
  \rightarrow\nat$ (where $\mathcal{Q}$ is the set of states of the different processes) be the marking such that
$\amark_f(\ell_f)=1$ and $\amark_f(q)=0$ for all other places $q$.
Our reduction claims that the Minsky machine halts iff the answer to
$\paramreach{\grammar_{\Minsk}}{\{\ell_f\}}{\amark_f}$
(\resp $\paramcover{\grammar_{\Minsk}}{\{\ell_f\}}{\amark_f}$) is positive.

Let us explain why the reduction holds for the two verification problems.
First note that if the answer to $\paramreach{\grammar_{\Minsk}}{\{\ell_f\}}{\amark_f}$
is positive then so is the answer to $\paramcover{\grammar_{\Minsk}}{\{\ell_f\}}{\amark_f}$
by definition of the problem.
Now assume the answer to $\paramcover{\grammar_{\Minsk}}{\{\ell_f\}}{\amark_f}$ is positive,
it means that there exists a system $\asys^{m,n}_{\Minsk}=(\verts,\edges,\vlab)$
with $\verts=\{v^2_m,\ldots,v^2_1,v^c,v^1_1,\ldots,v^1_n\}$
where $v^c$ is the only vertex such that $\vlab(v^c)=\Minsk$
and a marking $\overline{\amark}\in\cover{\behof{\asys^{m,n}_{\Minsk}}}$ verifying
$\overline{\amark}(\ell_f,v^c)=\amark(\ell_f)=1$.
By definition, there exists hence a marking $\overline{\amark}' \in \reach{\behof{\asys^{m,n}_\Minsk}}$
such that $\overline{\amark} \leq \overline{\amark}'$.
But since since $\behof{\asys^{m,n}_\Minsk}$ is an automata-like PN,
we have $\overline{\amark}'(\ell_f,v^c)=1$ and since $v^c$ is the only vertex such that $\vlab(v^c)=\Minsk$,
we conclude that the answer to $\paramreach{\grammar_{\Minsk}}{\{\ell_f\}}{\amark_f}$ is positive.

\begin{figure}[htbp]
\begin{center}
  \scalebox{1}{
    \begin{tikzpicture}[node distance=1.5cm]
      \tikzstyle{every state}=[inner sep=3pt,minimum size=20pt]
      \node(v0)[gnode,draw=black,label=-90:{$Minsk$}]{};
      \node(v1)[gnode,draw=black,label=-90:{$\Count_1$},right of=v0,xshift=-1em]{};
      \node(v2)[gnode,draw=black,label=-90:{$\Count_1$},right of=v1,xshift=-1em]{};
      \node(v3)[gnode,draw=black,label=-90:{$\Count_1$},right of=v2,xshift=0em]{};
      \node(v4)[gnode,draw=black,label=-90:{$\Count_1$},right of=v3,xshift=-1em]{};
      \node(v1b)[gnode,draw=black,label=-90:{$\Count_2$},left of=v0,xshift=1em]{};
      \node(v2b)[gnode,draw=black,label=-90:{$\Count_2$},left of=v1b,xshift=1em]{};
      \node(v3b)[gnode,draw=black,label=-90:{$\Count_2$},left of=v2b,xshift=0em]{};
      \node(v4b)[gnode,draw=black,label=-90:{$\Count_2$},left of=v3b,xshift=1em]{};

      \path (v0) edge [-,thick,line width=1pt] (v1);
      \path (v1) edge [-,thick,line width=1pt] (v2);
      \path (v2) edge [-,thick,line width=1pt,dotted] (v3);
      \path (v3) edge [-,thick,line width=1pt] (v4);
      \path (v0) edge [-,thick,line width=1pt] (v1b);
      \path (v1b) edge [-,thick,line width=1pt] (v2b);
      \path (v2b) edge [-,thick,line width=1pt,dotted] (v3b);
      \path (v3b) edge [-,thick,line width=1pt] (v4b);
\end{tikzpicture}
  }
\end{center}
\vspace*{-\baselineskip}
\caption{Shape of a sytem for the simulation of a Minsky machine}
\label{fig:undec4}
\end{figure}

\begin{figure}[htbp]
\begin{center}
\scalebox{0.7}{
\begin{tikzpicture}[node distance=1.5cm]
  \tikzstyle{every state}=[inner sep=3pt,minimum size=20pt]

\node(0C1)[petri-p,draw=black,label=00:{$\mathit{0C1}$}]{};
\node(Z1)[petri-t2,draw=black,fill=black,above of=0C1,label=90:{$\mathit{z1}$}]{};
\node(I1)[petri-t,draw=black,fill=black,below left of=0C1,label=180:{$\mathit{i1}$}]{};
\node(D1)[petri-t,draw=black,fill=black,below right of=0C1,label=0:{$\mathit{ad1}$},yshift=2em]{};
\node(0to1)[petri-p,draw=black,below of=I1,yshift=2em]{};
\node(1to0)[petri-p,draw=black,below of=D1,yshift=2.5em]{};
\node(AI1)[petri-t,draw=black,fill=black,below of=0to1,label=180:{$\mathit{ai1}$},yshift=2em]{};
\node(NZ1)[petri-t,draw=black,fill=black,below of=1to0,label=0:{$\mathit{nextz1}$},yshift=2.5em]{};
\node(1to0b)[petri-p,draw=black,below of=NZ1,yshift=2.5em]{};
\node(AD1)[petri-t,draw=black,fill=black,below of=1to0b,label=0:{$\mathit{d1}$},yshift=2.5em]{};
\node(1C1)[petri-p,draw=black,below right of=AI1,label=90:{$\mathit{1C1}$}]{};

\node(PI1)[petri-t,draw=black,fill=black,below left of=1C1,label=180:{$\mathit{pi1}$}]{};
\node(pi1toqi1)[petri-p,draw=black,below of=PI1,yshift=2em]{};
\node(QI1)[petri-t,draw=black,fill=black,below of=pi1toqi1,label=-110:{$\mathit{qi1}$},yshift=2em]{};

\node(qi1tori1)[petri-p,draw=black,below right of=QI1,xshift=0em]{};


\node(RI1)[petri-t,draw=black,fill=black,above right of=qi1tori1,label=-70:{$\mathit{ri1}$}]{};
\node(ri1tosi1)[petri-p,draw=black,above of=RI1,yshift=-2em]{};
\node(SI1)[petri-t,draw=black,fill=black,above of=ri1tosi1,label=0:{$\mathit{si1}$},yshift=-2em]{};

\path (0C1) edge [->,thick,line width=1pt,out=30,in=0](Z1);
\path (Z1) edge [->,thick,line width=1pt,out=180,in=150](0C1);
\path (0C1) edge [->,thick,line width=1pt,out=-150,in=90](I1);
\path (D1) edge [->,thick,line width=1pt,out=90,in=-30](0C1);
\path (I1) edge [->,thick,line width=1pt](0to1);
\path (1to0) edge [->,thick,line width=1pt](D1);
\path (1to0b) edge [->,thick,line width=1pt](NZ1);
\path (NZ1) edge [->,thick,line width=1pt](1to0);
\path (0to1) edge [->,thick,line width=1pt](AI1);
\path (AD1) edge [->,thick,line width=1pt](1to0b);
\path (AI1) edge [->,thick,line width=1pt,out=-90,in=150](1C1);
\path (1C1) edge [->,thick,line width=1pt,out=30,in=-90](AD1);

\path (1C1) edge [->,thick,line width=1pt](PI1);
\path (PI1) edge [->,thick,line width=1pt](pi1toqi1);
\path (pi1toqi1) edge [->,thick,line width=1pt](QI1);
\path (QI1) edge [->,thick,line width=1pt](qi1tori1);


\path (qi1tori1) edge [->,thick,line width=1pt](RI1);
\path (RI1) edge [->,thick,line width=1pt](ri1tosi1);

\path (ri1tosi1) edge [->,thick,line width=1pt](SI1);
\path (SI1) edge [->,thick,line width=1pt](1C1);

\node[petri-tok] at (0C1) {};

\node(0C1bis)[petri-p,draw=black,label=00:{$\mathit{0C1}$},right of=0C1,xshift=10em]{};
\node(Z1bis)[petri-t2,draw=black,fill=black,above of=0C1bis,label=90:{$\mathit{z1}$}]{};
\node(I1bis)[petri-t,draw=black,fill=black,below left of=0C1bis,label=180:{$\mathit{i1}$}]{};
\node(D1bis)[petri-t,draw=black,fill=black,below right of=0C1bis,label=0:{$\mathit{ad1}$},yshift=2em]{};
\node(0to1bis)[petri-p,draw=black,below of=I1bis,yshift=2em]{};
\node(1to0bis)[petri-p,draw=black,below of=D1bis,yshift=2.5em]{};
\node(AI1bis)[petri-t,draw=black,fill=black,below of=0to1bis,label=180:{$\mathit{ai1}$},yshift=2em]{};
\node(NZ1bis)[petri-t,draw=black,fill=black,below of=1to0bis,label=0:{$\mathit{nextz1}$},yshift=2.5em]{};
\node(1to0bbis)[petri-p,draw=black,below of=NZ1bis,yshift=2.5em]{};
\node(AD1bis)[petri-t,draw=black,fill=black,below of=1to0bbis,label=0:{$\mathit{d1}$},yshift=2.5em]{};
\node(1C1bis)[petri-p,draw=black,below right of=AI1bis,label=90:{$\mathit{1C1}$}]{};

\node(PI1bis)[petri-t,draw=black,fill=black,below left of=1C1bis,label=180:{$\mathit{pi1}$}]{};
\node(pi1toqi1bis)[petri-p,draw=black,below of=PI1bis,yshift=2em]{};
\node(QI1bis)[petri-t,draw=black,fill=black,below of=pi1toqi1bis,label=180:{$\mathit{qi1}$},yshift=2em]{};
\node(qi1tori1bis)[petri-p,draw=black,below right of=QI1bis]{};
\node(RI1bis)[petri-t,draw=black,fill=black,above right of=qi1tori1bis,label=0:{$\mathit{ri1}$}]{};
\node(ri1tosi1bis)[petri-p,draw=black,above of=RI1bis,yshift=-2em]{};
\node(SI1bis)[petri-t,draw=black,fill=black,above of=ri1tosi1bis,label=0:{$\mathit{si1}$},yshift=-2em]{};

\path (0C1bis) edge [->,thick,line width=1pt,out=30,in=0](Z1bis);
\path (Z1bis) edge [->,thick,line width=1pt,out=180,in=150](0C1bis);
\path (0C1bis) edge [->,thick,line width=1pt,out=-150,in=90](I1bis);
\path (D1bis) edge [->,thick,line width=1pt,out=90,in=-30](0C1bis);
\path (I1bis) edge [->,thick,line width=1pt](0to1bis);
\path (1to0bbis) edge [->,thick,line width=1pt](NZ1bis);
\path (NZ1bis) edge [->,thick,line width=1pt](1to0bis);
\path (1to0bis) edge [->,thick,line width=1pt](D1bis);
\path (0to1bis) edge [->,thick,line width=1pt](AI1bis);
\path (AD1bis) edge [->,thick,line width=1pt](1to0bbis);
\path (AI1bis) edge [->,thick,line width=1pt,out=-90,in=150](1C1bis);
\path (1C1bis) edge [->,thick,line width=1pt,out=30,in=-90](AD1bis);

\path (1C1bis) edge [->,thick,line width=1pt](PI1bis);
\path (PI1bis) edge [->,thick,line width=1pt](pi1toqi1bis);
\path (pi1toqi1bis) edge [->,thick,line width=1pt](QI1bis);
\path (QI1bis) edge [->,thick,line width=1pt](qi1tori1bis);
\path (qi1tori1bis) edge [->,thick,line width=1pt](RI1bis);
\path (RI1bis) edge [->,thick,line width=1pt](ri1tosi1bis);
\path (ri1tosi1bis) edge [->,thick,line width=1pt](SI1bis);
\path (SI1bis) edge [->,thick,line width=1pt](1C1bis);

\node[petri-tok] at (0C1bis) {};

\node(L)[petri-p,draw=black,label=180:{$\ell$},left of=1C1,xshift=-25em]{};
\node(ReqI1)[petri-t2,draw=black,fill=black,right of=L,label=90:{$\ell\_\mathit{reqi1}$},xshift=-2em]{};

\node(aux1)[petri-p,draw=black,right of=ReqI1,xshift=-2em]{};
\node(AckI1)[petri-t2,draw=black,fill=black,right of=aux1,label=90:{$\ell\_\mathit{acki1}$},xshift=-2em]{};

\node(L2)[petri-p,draw=black,right of=AckI1,label=90:{$\ell'$},xshift=-1.5em]{};
\node(ReqI1bis)[petri-t2,draw=black,fill=black,right of=L2,label=90:{$\ell'\_\mathit{reqi1}$},xshift=-2em]{};

\node(aux1bis)[petri-p,draw=black,right of=ReqI1bis,xshift=-2em]{};
\node(AckI1bis)[petri-t2,draw=black,fill=black,right of=aux1bis,label=90:{$\ell'\_\mathit{acki1}$},xshift=-2em]{};

\node(L3)[petri-p,draw=black,right of=AckI1bis,label=0:{$\ell''$},xshift=-2em]{};

\path (L) edge [->,thick,line width=1pt](ReqI1);
\path (ReqI1) edge [->,thick,line width=1pt](aux1);
 \path (aux1) edge [->,thick,line width=1pt](AckI1);
  \path (AckI1) edge [->,thick,line width=1pt](L2);
  \path (L2) edge [->,thick,line width=1pt](ReqI1bis);
  \path (ReqI1bis) edge [->,thick,line width=1pt](aux1bis);
  
  \path (aux1bis) edge [->,thick,line width=1pt](AckI1bis);
  \path (AckI1bis) edge [->,thick,line width=1pt](L3);
  

\node[petri-tok] at (L) {};


\path (ReqI1) edge [loosely dotted,line width=1pt, bend left] (I1);
\path (AckI1) edge [loosely dotted,line width=1pt, bend left] (AI1);
\path (ReqI1) edge [loosely dotted,line width=1pt, bend right] (PI1);
\path (AckI1) edge [loosely dotted,line width=1pt, bend right] (SI1);
\path (ReqI1bis) edge [loosely dotted,line width=1pt, bend left] (I1);
\path (AckI1bis) edge [loosely dotted,line width=1pt, bend left] (AI1);
\path (ReqI1bis) edge [loosely dotted,line width=1pt, bend right] (PI1);
\path (AckI1bis) edge [loosely dotted,line width=1pt, bend right] (SI1);

\path (QI1) edge [loosely dotted,line width=1pt] (I1bis);
\path (RI1) edge [loosely dotted,line width=1pt] (AI1bis);
\path (QI1) edge [loosely dotted,line width=1pt] (PI1bis);
\path (RI1) edge [loosely dotted,line width=1pt] (SI1bis);

\end{tikzpicture}
}
\end{center}
\vspace*{-\baselineskip}
\caption{Partial representation of a behavior for the simulation of a Minsky machine}
\label{fig:behav}
\end{figure}


Assume now that the Minsky machine halts.
Let $m$ and $n$ be the maximum counter value taken by the counter $c2$ (\resp the counter $c1$)
during the execution of the machine starting at $\ell_0$ with $c1$ and $c2$ set at $0$ and ending in $\ell_f$.
We can simulate this execution, as we have explained before, in the behavior $\behof{\asys^{m+1,n+1}_\Minsk}$
to reach a marking with a token in the place $(\ell_f,v^c)$.
As a matter of fact, the answer to $\paramcover{\grammar_{\Minsk}}{\{\ell_f\}}{\amark_f}$ is positive.

On the other side if the answer to $\paramcover{\grammar_\Minsk}{\{\ell_f\}}{\amark_f}$ is positive,
there exists a system $\asys^{m,n}_\Minsk \in \alangof{}{\algof{S}}{\grammar_\Minsk}$
and a marking $\overline{\amark} \in \reach{\behof{\asys^{m,n}_\Minsk}}$
such that $\overline{\amark}(\ell_f,v^c)=1$.
The only way to reach such a marking from the initial marking of $\behof{\asys^{m,n}_\Minsk}$
is to simulate faithfully at each step an instruction of the Minsky machine
(if the behavior performs a wrong non-deterministic choice or if the vertices encoding the counters are not enough,
the simulation get stuck and there is no way to put a token in a place of the process type $\Minsk$ corresponding to a label of the machine).
We hence can rebuild an execution of the Minsky machine which ends in the label $\ell_f$. \qed

