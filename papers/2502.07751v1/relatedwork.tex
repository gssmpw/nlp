\section{Related Works}
Recent approaches for integrating ST and scRNA-seq data can be broadly categorized into several groups based on their methodological foundations:

\textbf{Alignment and Mapping-based Methods:} Tangram~\cite{Tangram} employs deep learning to align scRNA-seq data with spatial data by learning a mapping between the two modalities. This approach enables high-resolution spatial mapping of cell types and states. Similarly, SpaGE~\cite{SpaGE} enhances spatial gene expression by learning gene-gene relationships from scRNA-seq data and transferring this information to spatial data.

\textbf{Deep Generative Models:} Several methods leverage advanced generative modeling approaches. scVI~\cite{scVI} provides a deep generative framework for single-cell transcriptomics that can be extended to spatial data integration. Building upon this foundation, SpatialScope~\cite{SpatialScope} employs deep generative models specifically designed to integrate spatial and single-cell transcriptomics data.

\textbf{Transport-based Methods:} Methods such as SpaOTsc~\cite{SpaOTsc} and novoSpaRc~\cite{novoSpaRc} utilize optimal transport theory. SpaOTsc infers spatial and signaling relationships between cells from transcriptomic data, while novoSpaRc offers flexible spatial reconstruction of gene expression using optimal transport principles.

\textbf{Reference-based Enhancement:} stPlus~\cite{stPlus} takes a reference-based approach to enhance ST data accuracy. This method focuses on improving the resolution and quality of spatial data by leveraging information from scRNA-seq references.

\textbf{Diffusion-based Models:} Recent approaches have explored diffusion models for this integration task. stDiff~\cite{stDiff} employs a diffusion model for imputing ST through single-cell data. SpaDiT~\cite{li2024spadit} builds upon this framework by incorporating transformer architectures for spatial gene expression prediction, demonstrating the potential of combining diffusion models with attention mechanisms.

In summary, while existing methods for integrating ST and scRNA-seq data have made significant progress, they generally lack explicit modeling of causal relationships between genes, a critical factor in biological systems. Alignment and mapping-based methods like Tangram and SpaGE excel in spatial alignment but rely heavily on predefined relationships. Deep generative models, including scVI and SpatialScope, provide flexible frameworks but often overlook biological dependencies in the data. Transport-based approaches such as SpaOTsc and novoSpaRc emphasize spatial reconstruction but fail to capture gene regulatory dynamics. Even recent diffusion-based models like stDiff and SpaDiT, which integrate advanced generative techniques, focus primarily on spatial continuity rather than causal dependencies. In contrast, our approach, CausalGeD, uniquely incorporates gene-gene causal relationships through a novel combination of diffusion processes and autoregressive modeling. This enables more accurate and biologically meaningful integration, addressing the limitations of prior methods and advancing the field both technically and scientifically.