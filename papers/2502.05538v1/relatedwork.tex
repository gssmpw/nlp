\section{Related Works}
\label{relatedworks}
Existing studies, as summarized in TABLE \ref{Table_comp}, have explored channel estimation in RIS-assisted wireless systems, including least square (LS), minimum mean square error (MMSE) \cite{MMSE}, matrix factorization \cite{MF}, and compressed sensing \cite{CS} methods. Despite the great potential of these techniques, the high demands on the antenna array response (e.g., sparsity) and the high complexity of the required computations greatly restrict their practical applications. To address these issues, data-driven methodologies, notably deep learning (DL) algorithms \cite{DL_1}, have emerged as viable alternatives for dealing with the complex non-linear correlations inherent in receiving signal samples. In particular, \cite{DNN-1} employed pilot design and non-local attention modules in neural networks by gradually pruning less significant neurons, reducing the pilot transmission overhead, and improving channel estimation accuracy. 


The performance of DL  generally scales with both the model size and the volume of training data. However, large models and high-volume datasets impose heavy computational demands on a single node, often exceeding  its  computation capability.  Distributed machine learning (DML) is a powerful solution to address this challenge  by leveraging parallel processing\cite{DML1}. It divides data into batches or partitions the model across multiple compute nodes for concurrent training of DL models, followed by inter-node communication to update model parameters. 
The research in \cite{DML} applied DML to downlink channel estimation in RIS-based networks and proposed a hierarchical neural network-based DML approach. This method improves estimation accuracy by pre-classifying user groups under different scenarios and then training scenario-specific models using the corresponding data. However,  in DML method, the transmission processes involved in data distribution and model updates  lead to increased   communication overhead and   a risk of data leakage, thereby threatening user privacy. % Furthermore, training accuracy on non-independent and identically distributed (non-IID) datasets remains to be improved. 

\emph{To address the challenges of high communication overhead and privacy leakage, federated learning (FL) schemes, as special kinds of DML, have been recently proposed. In FL, rather than transmitting the entire dataset, only model updates, such as the gradients of model parameters, are exchanged.} There have been some important works on how to utilize FL for RIS channel estimation \cite{FL_Elbir, FL_Shen_Qin}. Elbir \textit{et. al.} applied FL to channel estimation for RIS-assisted MIMO systems\cite{FL_Elbir}. In comparison with  the centralized learning method, the ChannelNet learning method in \cite{FL_Elbir} significantly lowers the  amount of communication overhead by utilizing the FL method. However, the performance, while approaching that of MMSE, remains suboptimal due to the aggregation of parameter updates from all participating users. Compared to \cite{FL_Elbir}, \cite{FL_Shen_Qin} achieves high-precision channel estimation while reducing the number of model parameters by integrating a deep residual network into the FL framework, thereby further decreasing the communication overhead. 
In addition, \cite{FL_Bin} employs hierarchical residual networks and federated learning to classify user distribution scenarios, addressing the decline in channel estimation accuracy caused by uneven user location distribution.
Moreover, the large size of the local learning models places a significant computational burden on end-user devices, requiring powerful computing capabilities that exceed the resources available on many existing devices. \emph{To adapt to the different computing and storage capabilities of the end-user, the network model needs to be flexibly resizable, and heterogeneous federated learning (HFL) is ideally suited to address the requirement.} The authors of \cite{HFL1} articulated customized model aggregation strategies to cope with different types of heterogeneity and  increase the feasibility. Data-free knowledge distillation techniques were introduced to HFL, enabling the efficient transfer of model knowledge without requiring access to raw data. This approach is particularly beneficial in resource-constrained environments  \cite{HFL2}.
%这里写出异构联邦的需求%% 为什么用联盟解决FL组合,和相关研究

In multi-user RIS-MIMO networks, channel attenuation varies significantly among geographically dispersed users. This variability necessitates partitioning users into multiple federated learning (FL) groups, with each group aggregating its own channel model. 
However, \emph{most existing FL applications in channel estimation do not address the optimization of user grouping; instead, they use predefined FL groups. This approach  may deteriorate channel estimation accuracy in the aggregated models.}
A coalition formation game provides a framework for multiple agents (players) to cooperate and share resources (e.g., local model parameters of neural networks) to form coalitions for mutual benefits or shared goals (e.g., improved channel estimation accuracy). This framework supports user group partitioning,   allowing players to form coalitions with others to improve their outcomes  and determining  how the overall benefits, such as rewards  or utility, are distributed among members. \emph{Therefore, the problem of federated learning user grouping can be readily converted into a coalition formation game.} However, few studies have addressed the FL user grouping problem by using coalition formation games.
In \cite{Coalition_1} and \cite{Coalition_2}, coalition formation was employed to quantify user data contribution \cite{Coalition_1}, and prevent user selfishness \cite{Coalition_2}. However, all the above works focus on a single FL aggregated model. Specifically,  the work in \cite{Coalition_1} was the first to incorporate coalition games into personalized federated learning, utilizing Shapley values to assess the contribution of other users' models to an individual model, subsequently selecting parameters that enhance performance and mitigate degradation from malicious users. Literature The authors of \cite{Coalition_2} introduced a coalition member trust mechanism to prevent selfish devices from harming their coalition, mitigating the problems of non-independent and identically distributed (I.I.D) data and communication bottlenecks. 
Moreover, the authors of \cite{CF_CE_1} investigated the trade-off between channel estimation accuracy and communication overhead in centralized computation using clustering algorithms and proved the algorithm's stable convergence. This   reveals the feasibility of combining coalition formation games with federated learning for channel estimation.

%如何寻找联盟稳定结构 问题没闭式的目标表达式
To address the FL user grouping problem through coalition formation games, two subproblems need to be tackled: 1) Establishing   stable coalitions  is essential: This involves  designing an overall payoff function, a payoff distribution policy, and a coalition formation order that align to improve channel estimation accuracy, while ensuring the existence of a stable coalition partition. 2) The identification of a stable coalition partition. 
In particular, for subproblem 2),  the coalition formation of the FL user 
is mathematically intractable  under an unknown channel model, especially when coupled with the iterative framework of federated learning. This makes it infeasible to formulate the overall coalition payoff  and utility   distribution among coalition members, both of which are essential for guiding coalition formation. 
Conventional   approaches, such as mixed-integer programming and heuristic algorithms (e.g., ant colony optimization, genetic algorithms), are no longer applicable.  In order to tackle those problems, a data-driven deep reinforcement learning (DRL) approach can be considered to address the coalition formation problem. 
The authors of \cite{RL_CL_1} introduced the DRL-based coalition formation solution for peer-to-peer energy trading problem, aiming to minimize the cost of energy trading. In addition, coalition formation games are also applied in multi-agent task allocation to model task preferences of different nodes, enhancing task completion efficiency through collaboration among multiple coalitions \cite{RL_CL_2,RL_CL_3}. 

% \cite{FL+RL_1} proposed an asynchronous federated learning (AFL) framework for multi unmanned aerial vehicle (UAV) networks to address the challenges posed by dynamic channel conditions and varying computational capabilities of devices. It introduces the asynchronous advantage actor-critic (A3C) method for device selection and resource allocation, mitigating the impact of low-quality devices on learning efficiency and accuracy. Furthermore, \cite{FL+RL_3} presents a federated learning algorithm in this paper utilizing DRL to dynamically optimize Internet of Things (IoT) device clustering and resource allocation. Despite lowering privacy leaks and increasing modeling efficiency, it poses a significant challenge to the end user's processing and storage capabilities.

%HFL
% \emph{To adapt to the different computing and storage capabilities of the endpoints, the network model needs to be flexibly resizable, and heterogeneous federated learning is ideally suited to address the requirement.} Some researches \cite{HFL1,HFL2,HFL3}focus on heterogeneous federated learning (HFL), where \cite{HFL1} proposes customized model aggregation strategies to cope with different types of heterogeneity and to increase the flexibility of the system. Data-free knowledge distillation techniques are introduced to HFL, enabling efficiently transfer model knowledge without access to raw data, which is particularly useful in resource-constrained environments\cite{HFL2}. Moreover, transfer learning (TL) is also based on similar features and knowledge to initialize the model parameters and accelerate the training process of federated learning.

\vspace{-1.2em}