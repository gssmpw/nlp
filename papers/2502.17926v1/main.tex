

%% This is file `sample-sigconf.tex',
%% generated with the docstrip utility.
%%
%% The original source files were:
%%
%% samples.dtx  (with options: `all,proceedings,bibtex,sigconf')
%% 
%% IMPORTANT NOTICE:
%% 
%% For the copyright see the source file.
%% 
%% Any modified versions of this file must be renamed
%% with new filenames distinct from sample-sigconf.tex.
%% 
%% For distribution of the original source see the terms
%% for copying and modification in the file samples.dtx.
%% 
%% This generated file may be distributed as long as the
%% original source files, as listed above, are part of the
%% same distribution. (The sources need not necessarily be
%% in the same archive or directory.)
%%
%%
%% Commands for TeXCount
%TC:macro \cite [option:text,text]
%TC:macro \citep [option:text,text]
%TC:macro \citet [option:text,text]
%TC:envir table 0 1
%TC:envir table* 0 1
%TC:envir tabular [ignore] word
%TC:envir displaymath 0 word
%TC:envir math 0 word
%TC:envir reply 0 0
%%
%% The first command in your LaTeX source must be the \documentclass
%% command.
%%
%% For submission and review of your manuscript please change the
%% command to \documentclass[manuscript, screen, review]{acmart}.
%%
%% When submitting camera ready or to TAPS, please change the command
%% to \documentclass[sigconf]{acmart} or whichever template is required
%% for your publication.
%%
%%
\documentclass[sigconf]{acmart}
\usepackage{caption}
\usepackage{graphicx}

% \settopmatter{printacmref=false}
\usepackage{url}
\usepackage{hyperref}
\usepackage[most]{tcolorbox}
\usepackage{multirow}
\newtcolorbox{promptbox}{
    colback=gray!5,     % width=0.95\linewidth,
    colframe=black!75,  
    left=1em,          
    right=0.5em,       
    top=0.5em,         
    bottom=0.5em,      
    sharp corners,     
    boxrule=1pt         
}
% %%
% %% \BibTeX command to typeset BibTeX logo in the docs
% \AtBeginDocument{%
%   \providecommand\BibTeX{{%
%     Bib\TeX}}}

% %% Rights management information.  This information is sent to you
% %% when you complete the rights form.  These commands have SAMPLE
% %% values in them; it is your responsibility as an author to replace
% %% the commands and values with those provided to you when you
% %% complete the rights form.
% \setcopyright{acmlicensed}
% \copyrightyear{2018}
% \acmYear{2018}
% \acmDOI{XXXXXXX.XXXXXXX}
% %% These commands are for a PROCEEDINGS abstract or paper.
% \acmConference[Conference acronym 'XX]{Make sure to enter the correct
%   conference title from your rights confirmation email}{June 03--05,
%   2018}{Woodstock, NY}
%%
%%  Unreply \acmBooktitle if the title of the proceedings is different
%%  from ``Proceedings of ...''!
%%
%%\acmBooktitle{Woodstock '18: ACM Symposium on Neural Gaze Detection,
%%  June 03--05, 2018, Woodstock, NY}
% \acmISBN{978-1-4503-XXXX-X/2018/06}


%%
%% Submission ID.
%% Use this when submitting an article to a sponsored event. You'll
%% receive a unique submission ID from the organizers
%% of the event, and this ID should be used as the parameter to this command.
%%\acmSubmissionID{123-A56-BU3}

%%
%% For managing citations, it is recommended to use bibliography
%% files in BibTeX format.
%%
%% You can then either use BibTeX with the ACM-Reference-Format style,
%% or BibLaTeX with the acmnumeric or acmauthoryear sytles, that include
%% support for advanced citation of software artefact from the
%% biblatex-software package, also separately available on CTAN.
%%
%% Look at the sample-*-biblatex.tex files for templates showcasing
%% the biblatex styles.
%%

%%
%% The majority of ACM publications use numbered citations and
%% references.  The command \citestyle{authoryear} switches to the
%% "author year" style.
%%
%% If you are preparing content for an event
%% sponsored by ACM SIGGRAPH, you must use the "author year" style of
%% citations and references.
%% UnReplying
%% the next command will enable that style.
%%\citestyle{acmauthoryear}


%%
%% end of the preamble, start of the body of the document source.
\begin{document}

%%
%% The "title" command has an optional parameter,
%% allowing the author to define a "short title" to be used in page headers.
\title{FediverseSharing: A Novel Dataset on Cross-Platform Interaction Dynamics between Threads and Mastodon Users}

%%
%% The "author" command and its associated commands are used to define
%% the authors and their affiliations.
%% Of note is the shared affiliation of the first two authors, and the
%% "authornote" and "authornotemark" commands
%% used to denote shared contribution to the research.
\author{Ujun Jeong}
\affiliation{%
  \institution{Arizona State University}
  \city{Tempe}
  \country{USA}}
\email{ujeong1@asu.edu}

\author{Alimohammad Beigi}
\affiliation{%
  \institution{Arizona State University}
  \city{Tempe}
  \country{USA}}
\email{abeigi@asu.edu}

\author{Anique Tahir}
\affiliation{%
  \institution{Arizona State University}
  \city{Tempe}
  \country{USA}}
\email{artahir@asu.edu}

\author{Susan Xu Tang}
\affiliation{%
  \institution{Arizona State University}
  \city{Tempe}
  \country{USA}}
\email{Susan.Tang@asu.edu}
\author{H. Russell Bernard}
\affiliation{%
  \institution{Arizona State University}
  \city{Tempe}
  \country{USA}}
\email{asuruss@asu.edu}


\author{Huan Liu}
\affiliation{%
  \institution{Arizona State University}
  \city{Tempe}
  \country{USA}}
\email{huanliu@asu.edu}

%%
%% By default, the full list of authors will be used in the page
%% headers. Often, this list is too long, and will overlap
%% other information printed in the page headers. This command allows
%% the author to define a more concise list
%% of authors' names for this purpose.
\renewcommand{\shortauthors}{Jeong et al.}

%%
%% The abstract is a short summary of the work to be presented in the
%% article.


%%
%% The code below is generated by the tool at http://dl.acm.org/ccs.cfm.
%% Please copy and paste the code instead of the example below.
%%
% \begin{CCSXML}
% <ccs2012>
%  <concept>
%   <concept_id>00000000.0000000.0000000</concept_id>
%   <concept_desc>Do Not Use This Code, Generate the Correct Terms for Your Paper</concept_desc>
%   <concept_significance>500</concept_significance>
%  </concept>
%  <concept>
%   <concept_id>00000000.00000000.00000000</concept_id>
%   <concept_desc>Do Not Use This Code, Generate the Correct Terms for Your Paper</concept_desc>
%   <concept_significance>300</concept_significance>
%  </concept>
%  <concept>
%   <concept_id>00000000.00000000.00000000</concept_id>
%   <concept_desc>Do Not Use This Code, Generate the Correct Terms for Your Paper</concept_desc>
%   <concept_significance>100</concept_significance>
%  </concept>
%  <concept>
%   <concept_id>00000000.00000000.00000000</concept_id>
%   <concept_desc>Do Not Use This Code, Generate the Correct Terms for Your Paper</concept_desc>
%   <concept_significance>100</concept_significance>
%  </concept>
% </ccs2012>
% \end{CCSXML}

% \ccsdesc[500]{Do Not Use This Code~Generate the Correct Terms for Your Paper}
% \ccsdesc[300]{Do Not Use This Code~Generate the Correct Terms for Your Paper}
% \ccsdesc{Do Not Use This Code~Generate the Correct Terms for Your Paper}
% \ccsdesc[100]{Do Not Use This Code~Generate the Correct Terms for Your Paper}

%%
%% Keywords. The author(s) should pick words that accurately describe
%% the work being presented. Separate the keywords with commas.
\keywords{Platform Integration, Community Interaction, Mastodon, Threads}
%% A "teaser" image appears between the author and affiliation
%% information and the body of the document, and typically spans the
%% page.

% \begin{teaserfigure}
%   \includegraphics[width=\textwidth]{sampleteaser}
%   \caption{Seattle Mariners at Spring Training, 2010.}
%   \Description{Enjoying the baseball game from the third-base
%   seats. Ichiro Suzuki preparing to bat.}
%   \label{fig:teaser}
% \end{teaserfigure}

% \received{20 February 2007}
% \received[revised]{12 March 2009}
% \received[accepted]{5 June 2009}

%%
%% This command processes the author and affiliation and title
%% information and builds the first part of the formatted document.
\begin{abstract}
Traditional social media platforms, once envisioned as digital town squares, face growing criticism over corporate control, content moderation, and privacy concerns. Events such as Twitter’s acquisition (now X) and major policy changes have driven users toward alternative platforms like Mastodon and Threads. However, this diversification has led to user dispersion and fragmented discussions across isolated social media platforms.

To address these issues, federation protocols like ActivityPub have been adopted, with Mastodon leading efforts to build decentralized yet interconnected networks. In March 2024, Threads joined this federation by introducing its Fediverse Sharing service, which enables interactions such as posts, replies, and likes between Threads and Mastodon users as if on a unified platform. 
% This platform-level integration offers a unique opportunity to study user behavior driven by the federation protocol.

Building on this development, we introduce \texttt{FediverseSharing}, the first dataset capturing interactions between 20,000+ Threads users and 20,000+ Mastodon users over a ten-month period. This dataset serves as a foundation for studying cross-platform interactions and the impact of federation as two separate platforms integrate.
The data and code will be available upon acceptance.

\end{abstract}




\maketitle


\begin{figure}[H]
  \centering
  \includegraphics[width=0.35\textwidth]{detailed_overview_anime.drawio-cropped.pdf}
  \captionsetup{singlelinecheck=off} % Ensures proper formatting
\caption{Federation for cross-platform interactions enabled by the ActivityPub protocol\protect\footnotemark. On the left, Mastodon users from two servers; on the right, Threads users, can post, reply, and like. These actions are synchronized across platforms by replicating data across their databases. In other cases, when multiple users from each platform like the same post, their actions are also shared across Threads and Mastodon.}
  \label{fig:overview_anime}
\end{figure}

\footnotetext{Technical details are available at \url{https://www.w3.org/TR/activitypub/}, with the basic mechanism of federation covered in the preliminaries section of this paper.}


\section{Introduction}

Growing frustrations with traditional social media platforms—rooted in concerns over content moderation, privacy measures, and corporate influence—have led many users to seek alternative spaces for online interaction~\cite{fiesler2020moving}. This discontent became particularly evident following Twitter’s acquisition on October 27, 2022, which triggered a mass exodus of users exploring alternatives~\cite{jeong2024exploring, cava2023drivers, he2023flocking}. As a result, the social media landscape has become increasingly divided between centralized and decentralized platforms, each offering distinct features and user bases.

Several prominent alternatives emerged in the wake of this shift~\cite{jeong2024user}. \textit{Threads}, a centralized platform created by Meta, operates under corporate control and algorithmic content recommendation, amassing over 130 million users. In contrast, \textit{Mastodon} is a non-profit, open-source project with a decentralized network of independent instances, allowing users greater control over moderation and data, and has over 10 million users.
% Other platforms, such as Nostr and Bluesky, have contributed to the diversification.

However, this growing diversification has also introduced challenges, particularly in isolating discussions, trends, and content within separate platforms~\cite{di2024characterizing, oxford2024social}. The limited communication between platforms restricts information flow, while user migration splits communities, making it harder to stay engaged and connected to their original networks~\cite{la2022network, la2021understanding, raman2019challenges}. As a result, users had to either maintain multiple accounts across different platforms or risk missing out on relevant discussions outside their chosen platform.

Fortunately, federation-based protocols like \textit{ActivityPub} have emerged as a solution to the growing fragmentation of social media platforms~\cite{zignani2018follow}. By connecting independent networks through standardized mechanisms, ActivityPub allows users to interact across different instances. Mastodon is an early adopter of ActivityPub, using it to synchronize posts, replies, and likes within its federated network. A major milestone was later reached on March 17, 2024, when Threads adopted ActivityPub through its Fediverse Sharing feature, enabling direct interaction between Threads and Mastodon users. Figure~\ref{fig:overview_anime} illustrates how ActivityPub facilitates these cross-platform interactions by synchronizing actions and replicating data across respective platform databases~\cite{schmidt2025concept}.


In this paper, we present \texttt{FediverseSharing}, the first large-scale dataset capturing cross-platform interactions between Threads and Mastodon users. Collected through Mastodon's public API, the dataset includes profiles and behaviors of 20,000+ Threads users who opted into FediverseSharing and 20,000+ Mastodon users who interacted with them. Spanning ten months post-launch, it documents posts, replies, and likes shared across the platforms. By examining these interactions, we provide insights into how federation-driven platform integration impacts user engagement, content creation, and intergroup dynamics within a federated network.

% As large-scale online interactions become harder to study due to fragmented, siloed platforms with varying access and governance~\cite{ng2024smi}, this dataset provides a valuable resource for social media researchers.

Our key contributions are as follows:

\begin{itemize}
\item \textbf{First Cross-platform Interaction Dataset}: We propose a unique way to collect data for user interactions between Threads and Mastodon, including posts, replies, and likes.

\item \textbf{Impact of Federation-driven Platform Integration}: Our longitudinal data provides an analysis of users' behavioral shifts before and after the introduction of Fediverse Sharing.

\item \textbf{Insights into Federated Social Network}: We analyze different instances within a federated network, highlighting key factors associated with federation and user interactions.
\end{itemize}



% \section{Introduction}


% Growing frustrations with centralized social media platforms—rooted in concerns over content moderation, privacy measures, and corporate influence—have led many users to seek alternative spaces for online interaction~\cite{fiesler2020moving}. This discontent became particularly evident following Twitter’s acquisition on October 27, 2022, which triggered a mass exodus of users exploring alternatives within a decentralized ecosystem~\cite{jeong2024exploring, cava2023drivers, he2023flocking}. In response, emerging platforms such as Mastodon and Threads have positioned themselves as viable options, offering different models of social media engagement.

% However, this diversification has also introduced challenges, particularly in the form of social fragmentation~\cite{di2024characterizing, oxford2024social}. The lack of communication across platforms has led to a dichotomy in information flow, where discussions, trends, and content remain isolated within specific ecosystems~\cite{la2022network, la2021understanding, raman2019challenges}. Migration across platforms has also divided user bases, making it difficult to recreate communities from their previous platforms~\cite{jeong2024user}.

% Previously, users had to either maintain multiple accounts across different platforms or risk missing out on relevant discussions outside their chosen network. Fortunately, the development of interoperable protocols, particularly ActivityPub~\cite{zignani2018follow}, has begun bridging these gaps between platforms. A significant milestone in this effort was reached on March 17, 2024, when Threads introduced Fediverse Sharing. This feature allows users to engage with Mastodon—one of the largest alternative platforms that grew following the Twitter exodus—by enabling seamless interactions through posts, replies, and likes. Figure~\ref{fig:overview_anime} illustrates how ActivityPub facilitates cross-platform interactions between Threads and Mastodon by synchronizing actions and replicating data across their respective databases~\cite{schmidt2025concept}.

% In this paper, we introduce \texttt{FediverseSharing}, the first large-scale dataset capturing cross-platform interactions between Threads and Mastodon users—two platforms supporting ActivityPub. Collected via Mastodon’s public API, our dataset includes profiles and activity from 20,000+ Threads users who opted into FediverseSharing and 20,000+ Mastodon users who interacted with them. Spanning ten months post-launch, it documents posts, replies, and likes, enabling analysis of user interactions, engagement trends, and content dynamics across platforms. By examining cross-platform interactions, we provide insights into how federation affects user behavior, content creation, and engagement patterns. As large-scale online interactions become harder to study due to fragmented, siloed platforms with varying access and governance~\cite{ng2024smi}, this dataset provides a valuable resource for social media researchers.


% Our key contributions are as follows:

% \begin{itemize}
% \item \textbf{First Cross-platform Interaction Dataset}: We propose a unique way to collect data for user interactions between Threads and Mastodon, including posts, replies, and likes.

% \item \textbf{Impact of Federation-driven Platform Integration}: Our longitudinal data provides an analysis of users' behavioral shifts before and after the introduction of Fediverse Sharing.

% \item \textbf{Insights into Federated Social Networks}: We analyze different instances within federated networks, highlighting key factors related to federation and user interactions.

% \end{itemize}

% Platform integration is traditionally a corporate decision studied in business and e-commerce, aiming to connect users across multiple platforms through a gateway module~\cite{schreieck2024typology}. Digital platform integrations typically follow a two-sided market structure~\cite{evans2016matchmakers, rochet2003platform}, where platforms link distinct user groups with complementary needs. This approach not only facilitates intermediary connections between services but also expands the overall market by increasing cross-platform accessibility. For instance, Google integrated YouTube with Google Search, enhancing content recommendations and visibility, ultimately driving higher user engagement and advertisement revenue~\cite{parker2017platform, huang2017social}.

% In the context of social media, integrations have primarily focused on increasing content visibility across platforms rather than enabling direct user-to-user interaction. For example, Facebook’s integration with TripAdvisor and Yelp in early 2010's allowed users to share reviews and ratings within their Facebook but did not support direct communication between users of different platforms~\cite{cao2024consequences, huang2017social}. Similarly, Facebook and Instagram enable content sharing, yet their integration remains limited to content distribution rather than fostering cross-platform social interactions. Prior research has demonstrated that platform integration increases content volume through network effects, where corporate strategies leverage cross-platform visibility to drive market expansion~\cite{abdelkafi2019multi}. This underscores how integration decisions are often motivated by business objectives rather than user-driven needs for deeper connectivity.

% A contrasting approach has emerged with the rise of open-source and decentralized social media platforms, such as Mastodon and Bluesky, which emphasize greater user autonomy and interoperability. Protocols like ActivityPub (Mastodon) and ATProtocol (Bluesky) enable richer user communication across distinct social media networks, allowing users to engage with multiple communities while maintaining control over their data and interactions~\cite{zhang2024emergence}. As social media becomes more decentralized and interconnected, these federated integration models raise important questions about their impact on user behavior, platform governance, and the broader evolution of the social media ecosystem.
\section{Related Work}
We review two key aspects of this dataset project in the context of social media: (1) platform interactions and (2) community interactions, discussing their implications for cross-platform interaction.

\subsection{Platform Integration}
Platform integration is a strategic business decision, often structured as a two-sided market, where platforms act as intermediaries connecting user groups with complementary needs~\cite{schreieck2024typology, evans2016matchmakers, rochet2003platform}. This model facilitates interactions between distinct user segments, driving engagement and market expansion. By linking services, integrations enhance accessibility and create new revenue streams. For example, Google’s integration of YouTube with Search refines content recommendations, reinforcing creator-audience relationships while increasing ad revenue~\cite{parker2017platform, huang2017social}.


Social media platforms have historically emphasized content visibility over direct cross-platform engagement. Facebook’s early integrations with TripAdvisor and Yelp enabled content sharing through Facebook but did not support interaction between users across different services~\cite{cao2024consequences, huang2017social}. Similarly, Facebook and Instagram facilitate content distribution while maintaining distinct user bases, leveraging network effects to increase audience reach~\cite{abdelkafi2019multi}.

Recently, open-source platforms like Mastodon and Bluesky have promoted federation-driven social media, shifting control away from corporate-owned platforms~\cite{la2021understanding, jeong2024bluetempnet}. Protocols like ActivityPub and AT Protocol foster interoperability, allowing different platforms to communicate. However, these integrations raise new challenges regarding governance, content moderation, and the long-term sustainability of federated social networks~\cite{zhang2024emergence}.


\subsection{Community Interaction}
Community dynamics play a fundamental role in shaping human behavior, as individuals' actions are influenced by their group identity, perceptions of outsiders, and the stability of their communities~\cite{schmidt2025concept}. Offline research, such as studies on ethnogenesis, examines how intergroup interactions contribute to the formation of new collective identities~\cite{okamura1981situational}. When cultures clash between communities, individuals actively share and signal aspects of their identity to build trust within their group, reinforcing social cohesion while delineating boundaries with outsiders~\cite{banerjee2022situational}.

Similarly, online communities function as digital tribes, where trust is built through the exchange and reinforcement of community norms~\cite{pettigrew2013does, demoulin2013intergroup}. Platforms like Reddit illustrate how engagement in toxic or partisan subreddits can spread hostility, influencing the behavior of adjacent communities~\cite{russo2024stranger, kumar2018community}. Research on political polarization suggests that exposure to diverse perspectives, including out-group perspectives, can reduce animosity and affective polarization, fostering better understanding of opposing groups~\cite{cohen2010competitive}. While such exposure may lower outgroup animosity, it is often shown to be insufficient to alter polarization.

Federated social networks offer a new model of community interaction, distinguished by decentralized governance and inconsistent moderation. Unlike corporate platforms with standardized policies, federated networks rely on community-driven oversight, leading to variations in content regulation and trust-building~\cite{schreieck2024typology}. While this grants users more autonomy, it also raises concerns about stability, harmful content, and ideological fragmentation. Without centralized enforcement, it remains uncertain whether these platforms will foster healthier discourse or deepen divisions.

\subsection{Implications of Our Study}
We distinguish our study as the first to analyze cross-platform interactions at the platform level within a federated social media ecosystem. Our FediverseSharing dataset offers key insights into the relationship between platform integration and user behavior. We examine the first integration between these structurally distinct platforms, highlighting how federation reshapes online interactions. As large-scale social media research faces growing challenges from fragmentation, restricted access, and diverse governance structures ~\cite{ng2024smi}, our dataset serves as a crucial resource for understanding the new social media landscape.









% \subsection{User Migration on Social Media}
% The emergence of decentralized social networks has influenced user migration patterns and reshaped the social media landscape. Mastodon, an open-source decentralized platform, offers a federated alternative to traditional social media by enabling users to join independent servers, or ``instances,'' each governed by unique policies and moderation rules~\cite{zignani2018follow}. This decentralized architecture grants users greater autonomy over their data and interactions, fostering a diverse and user-driven ecosystem. Studies on platform migration have shown that users seek decentralized alternatives due to concerns over content moderation, privacy, and corporate influence in mainstream social media~\cite{jeong2024exploring, jeong2024user}.

% In contrast, Meta’s Threads, launched in July 5th, 2023, has taken a novel approach by integrating with the decentralized ecosystem through its adoption of the ActivityPub protocol. This move allows Threads users to interact with individuals on Mastodon and other federated platforms without requiring separate accounts, effectively bridging the gap between centralized and decentralized social networks~\cite{cava2023drivers}. Research on cross-platform migration suggests that major shifts, such as the Twitter (now X) exodus, highlight user responses to changes in platform governance~\cite{jeong2024user}. However, integrating federated and corporate platforms introduces new challenges related to governance, interoperability, and user retention, which require further examination~\cite{cohen2010competitive}.

% As social media platforms continue to evolve, understanding the dynamics of user migration, platform integration, and community interaction becomes increasingly important. Future research should focus on the long-term effects of these developments on user behavior and platform sustainability.


\section{Preliminaries}

Understanding the Fediverse Sharing feature requires familiarity with key concepts, including ActivityPub, Federation, Fediverse, and specific social media platforms such as Mastodon and Threads.
% We provide an overview of these topics, which will be crucial for comprehending how content is shared across decentralized networks.

\subsection{ActivityPub}

ActivityPub is a decentralized social networking protocol that enables seamless interaction across compatible services. It establishes a structured framework for sharing posts, replies, likes, and follows between interconnected systems. Each user on an ActivityPub-compliant service is represented as an actor, capable of sending and receiving interactions within the network. The protocol standardizes the distribution of status—any user-generated message, either in the form of a post or a reply. By ensuring a uniform way to handle these interactions, ActivityPub enables seamless interoperability between diverse services within the Fediverse.

\subsection{Federated Social Network (Fediverse)}

The Fediverse is a decentralized network of interconnected social media instances that operate using open protocols such as ActivityPub. Unlike centralized social networks, where a single entity controls user data and interactions, the Fediverse is composed of independently hosted instances (i.e., servers) that form a decentralized ecosystem. These instances interconnect while ensuring user autonomy and data ownership, fostering a more private, user-controlled, and community-driven online environment~\cite{jeong2024exploring}.

Figure~\ref{fig:Mastodon_Federation} illustrates how federation works, using the example of Mastodon instances connecting with Threads and other Mastodon servers. When platforms implement ActivityPub, they automatically establish connections between instances. This structure allows users to interact with both local members and those on other recognized federated instances. Additionally, users or instance administrators can manually block communication with specific instances or users if they choose to de-federate.

\begin{figure}
  \includegraphics[width=0.46\textwidth]{ActivitiPub_details.drawio.pdf}
\caption{Three key scenarios illustrating how federation is established between ActivityPub-based instances. Federation occurs when a user in a source instance becomes aware of a target instance through one of the following actions: (1) direct search, (2) following a user, or (3) interacting with a post (e.g., liking or replying). Each of these scenarios independently triggers message exchanges between instances.}
\label{fig:Mastodon_Federation}
\end{figure}
\subsection{Mastodon and Threads}
The Fediverse comprises a diverse range of services, including microblogging, photo sharing, and more. Among them, Mastodon remains the largest, representing 58.5\% of ActivityPub-based instances\footnote{https://fedidb.org/}. It functions as a decentralized microblogging platform akin to Twitter but operates across independently managed instances. Each instance has its own administrator, policies, and community guidelines while still enabling users to follow and engage with others across different instances through ActivityPub.
Threads, a text-based social media platform launched by Meta on July 5, 2023, as a competitor to Twitter/X, has recently integrated with the Fediverse. On March 17, 2024, Meta introduced Fediverse Sharing, enabling Threads to federate with Mastodon. This platform integration comes with certain options for cross-platform interactions, which will be discussed in the following section.

\subsection{Interactions Between Threads and Mastodon}

Currently, Fediverse Sharing supports cross-platform interactions in three different ways:

\begin{itemize}
    % \item \textbf{Posting and Replying:} Mastodon users can reply to public posts from Threads users, and Threads users can respond to Mastodon content. However, some limitations exist in how threaded replies are displayed.
\item \textbf{Status (Post and Reply):} Users on both platforms are allowed to create and share posts and replies. Mastodon users can also reply to posts generated by Threads users.

\item \textbf{Like:} Users on both platforms can like each other's posts and replies. Users are notified when their content is liked and who likes it across the platforms.
\item \textbf{Follow:} Threads users can follow Mastodon accounts, and Mastodon users can follow Threads accounts, expanding content visibility across platforms.

\end{itemize}

Since the launch of Fediverse Sharing on March 17, 2024, Threads has enhanced its interoperability with Mastodon. However, certain limitations persist. For instance, Threads users cannot reply to replies from Mastodon users, restricting the flow of sequential interactions between the two platforms. Additionally, due to privacy concerns, Threads' social graph—which displays followers and followees—remains inaccessible to Mastodon users.
\section{Data Collection}  

The official Threads API\footnote{\url{https://developers.facebook.com/docs/threads/}} does not currently support profile searches\footnote{Profile search differs from a keyword search in the Threads API, which only allows retrieving a user's posts containing a specific keyword and requires user approval via the Meta Developer App.} or any read permissions. This limitation restricts the collection of any users directly on Threads. To address this, we leveraged Mastodon's official API\footnote{\url{https://docs.joinmastodon.org/api/}}, which provides access to Threads users who voluntarily enabled \textit{Fediverse Sharin}g, which makes their profiles and status publicly accessible on Mastodon.

\begin{figure}
  \includegraphics[width=0.5\textwidth]{Visual_FediSharingCollection.drawio.pdf}
\caption{Data collection process with Interaction Driven Snowball Sampling (IDSS), where new Threads users are discovered through their replies to the posts of initial seed users.}
  \label{fig:fediversesharing_data_collection}
\end{figure}

However, Mastodon's API imposes a 10,000-account limit on profile searches, limiting a scalable data collection strategy. To overcome this, we introduce \textbf{Interaction-Driven Snowball Sampling (IDSS)}, as shown in Figure~\ref{fig:fediversesharing_data_collection}, which consists of three key stages:

\begin{enumerate}  
    \item \textbf{Seed User Collection:} Identify a set of Threads accounts with Fediverse Sharing through our Mastodon instance.  
    \item \textbf{User Interaction Collection:} Retrieve posts of Threads users and repliers on those posts by Mastodon’s API.  
    \item \textbf{Snowball Sampling through User Interaction:} Expand the seed user set by collecting additional Threads users who have replied on posts, then repeat the process from Stage 1.
\end{enumerate}  

\subsection{Seed User Collection}  

Seed user collection is flexible and up to the study’s design choice, even possible through a manual sampling from Threads. In our case, we automated this process by maximizing the Mastodon API’s capabilities within the platform’s constraints. Specifically, we searched for profiles containing \texttt{@threads.net} on \texttt{mastodon.social}, one of the largest and most widely federated Mastodon instances, founded by Eugen Rochko. This method yielded 9,635 publicly accessible Threads user profiles after removing false positives where the keyword appeared outside the account handle.

\subsection{User Interaction Collection}

We selected \texttt{mastodon.social} as the primary query endpoint of the Mastodon API to collect user statuses, including posts and replies, due to its extensive federation with other Mastodon instances. This broad connectivity provides access to a diverse range of users across the Fediverse, including not only Threads users but also those from various Mastodon instances federated with \texttt{mastodon.social}. Using this endpoint, we first retrieved the complete set of posts from Threads users available on Mastodon, along with all replies associated with those posts. Note that Threads users' posts are accessible on Mastodon only after they enable Fediverse Sharing, and only posts made afterward will be available.

\subsection{Snowball Sampling through User Interaction}

Since Mastodon's API does not provide direct access to the social graph of Threads users, we expanded our dataset using an interaction-driven approach. Specifically, we identified Threads users by filtering accounts with handles ending in \texttt{@threads.net} that had replied to seed users' posts. These newly discovered Threads users were then saved as additional seed users through an iterative process inspired by snowball sampling.

\subsection{Data Validation and Privacy Measures}

As of January 17, 2025, we identified 79.1\% (20,456 out of 25,873) of all Threads accounts on \texttt{mastodon.social}, verified with the instance administrator. We confirmed these users’ activity in the Fediverse using WebFinger\footnote{\url{https://github.com/heliomass/Threads-Federation-Tracker}}, a protocol for decentralized account discovery that retrieves user metadata from their domain.

The data collection also left us with data on 20,231 Mastodon users across 1,417 instances who interacted with these Threads users. To protect user privacy, user-specific attributes, status content, and metadata will not be shared directly; instead, Mastodon API scripts with status IDs will be provided.

\section{Novel Study on Cross-platform Interaction}
This is the first study on cross-platform interaction. For simplicity of our study, we denote $\mathcal{U}_T$ as the set of collected Threads users who enabled Fediverse Sharing and $\mathcal{U}_M$ as those Mastodon users who interacted with $\mathcal{U}_T$ by replying on their  posts.

\subsection{Characteristics of Users on Each Platform}

\subsubsection{Joining Fediverse}
We analyzed peak trends in joining dates to understand user motivations in adopting federated networks. Figure~\ref{fig:fediverse_join_dates_by_threads} shows that Threads users had two peaks: the launch of FediverseSharing Beta (March 17, 2024) and cross-platform interactions (July 5, 2024). In contrast, Mastodon users spiked on October 27, 2022, aligning with Twitter's leadership change. This suggests Threads users joined due to platform features, while Mastodon users migrated in response to external disruptions~\cite{jeong2024exploring, cava2023drivers, he2023flocking}.

\subsubsection{Numerical Profile Attributes}  

We analyzed numerical attributes in user profiles to understand key differences in users' characteristics between Threads and Mastodon. Figure~\ref{fig:user_profile_metrics_compare} visualizes these variations, revealing the distinct characteristics of each platform. Key observations include:


\begin{itemize}  
\item \textbf{\#Followers:} Both platforms exhibit similar follower distributions, though Threads has a higher median and a wider spread, with more extreme values.  
\item \textbf{\#Following:} The distribution of followings is similar across both platforms, although Mastodon shows a marginally higher median and a broader range of extreme values. 
\item \textbf{\#Status:} The number of statuses follows a comparable distribution on both platforms. However, Mastodon has a higher median and a broader spread of extreme values.  
\end{itemize}  

Given the absence of algorithmic recommendations, Mastodon users must actively expand their social connections and post more frequently to maintain engagement. This differs from Threads, where follower counting plays a more central role, aligning with the findings on common social media's focus on audience reach ~\cite{jeong2024exploring, jeong2024user}.

\begin{figure}
  \includegraphics[width=0.45\textwidth]{violin_plot_comparison.pdf}
  \caption{Violin plots comparing three numerical attributes of user profiles in log-scaled values for Mastodon users ($\mathcal{U}_M$) and Threads users ($\mathcal{U}_T$). The width of each plot indicates the density of values, central tendency, and potential outliers. Each pair of distributions are statistically different based on the KS-test with a significance of $p<0.05$.}
  \label{fig:user_profile_metrics_compare}
\end{figure}

\begin{figure}
  \includegraphics[width=0.45\textwidth]{user_join_counts.pdf}   
  \caption{Joining dates of users on Mastodon since 2022. Threads users ($\mathcal{U}_T$) joined Mastodon by activating Fediverse Sharing, while Mastodon users ($\mathcal{U}_M$) joined by registration.}
  \label{fig:fediverse_join_dates_by_threads}
\end{figure}

\subsubsection{Language Diversity}
The primary audience and country of each platform may vary, and language serves as a key indicator of the diversity of the user base. To assess this, we employed \texttt{LangID}, a language detection model by Meta with 99.6\% accuracy on standard datasets~\cite{joulin2016bag}. We detected users' primary language based on their biographies. The result revealed that over 90\% of $\mathcal{U}_{M}$ prefer English, while only 75\% of $\mathcal{U}_{T}$ do. This suggests that Threads users exhibit greater linguistic diversity within the federated network.
% These insights reveal how Mastodon and Threads users use federation in varied ways.

% across 176 languages
% Figure~\ref{fig:user_bio_lang_dist} presents the distribution of


% (i.e. the various user groups interested in the Fediverse Sharing feature.)

% \subsubsection{Summary} Add some summary box over here

\begin{figure}
\includegraphics[width=0.46\textwidth]{language_distribution.pdf}
\caption{Distribution of languages in user biographies, sorted from most to least frequent. The remaining languages are grouped under the category ``Others'' for simplicity.}
\label{fig:user_bio_lang_dist}
\end{figure}

\begin{tcolorbox}[colback=black!5!white,colframe=black!75!black,title=Summary 1 (User Profile Analysis),rounded corners]
% \small The varied characteristics of migrant groups indicate that each platform had its unique audience. Even with a substantial presence on Twitter compared to non-migrants, migrants  explored and established accounts on new platforms.
% Users joined platforms for different reasons; for instance, Mastodon users primarily migrated from Twitter, whereas Threads users joined in response to new platform features. Furthermore, Threads users tend to focus more on follower count, while Mastodon users engage in more manual curation. Additionally, Mastodon is more English-centric, whereas Threads exhibits broader language diversity.
Users joined platforms for different reasons—Mastodon users, many migrating from Twitter, focused on status count and network curation, while Threads users prioritized new features and follower count. Mastodon is more English-centric, while Threads supports wider languages.
\end{tcolorbox}


\subsection{How Do Users Interact Across Platforms?}
\subsubsection{Daily Status Trend}
Figure~\ref{fig:metnion_trend} illustrates the daily volume of statuses from Threads and Mastodon users, along with statuses mentioning platform-related keywords. After the beta release of Fediverse Sharing, we observed a gradual upward trend in status activity, which peaked following the second update on July 5th. This update introduced enhanced features such as users in Threads can like and reply to statuses in Mastodon.

To assess how much attention each platform received, we analyzed how users mentioned them in their statuses. Mentions of the other platform were minimal, while users predominantly referenced their own. This self-referencing pattern reflects identity-marking behaviors, where individuals emphasize their group affiliation. Ethnogenesis studies suggest this is a common phenomenon, as people tend to assert their identity explicitly to build trust during initial interactions with other groups~\cite{banerjee2022situational}.

\begin{figure*}[htbp]
  \includegraphics[width=0.95\textwidth]{metnion_trend.drawio.pdf}
% \caption{Daily post volume trends for Threads users ($\mathcal{U}_T$, left) and Mastodon users ($\mathcal{U}_M$, right). The blue line represents the total number of posts, the orange line highlights posts mentioning \texttt{threads.net}, and the green line shows posts mentioning any Mastodon instance, which comprises a total of 25,733 instance names (e.g., \texttt{mastodon.social}, \texttt{mstdn.social}, etc.)}
\caption{Daily post volume trends for Threads users ($\mathcal{U}_T$) and Mastodon users ($\mathcal{U}_M$). The blue line represents the total number of posts, the orange line highlights posts mentioning \texttt{threads.net}, and the green line shows posts mentioning one of Mastodon instances, which comprises a total of 1,417 instance names (e.g., \texttt{mastodon.social}, \texttt{mstdn.social}, etc.)}
  \label{fig:metnion_trend}
\end{figure*}

\subsubsection{Average Status Volume Shift}
\begin{table*}[ht]
\centering
\begin{tabular}{ccccccc}
\hline
\textbf{Time Point}& \textbf{Status Type} & \textbf{Avg. Before} & \textbf{Avg. After} & \textbf{Change Rate} & \textbf{$t$-statistic} & \textbf{$p$-value} \\
\hline
\multirow{2}{*}{\textbf{$t_1$} (Mar 13th, 2024)} & Post  & 3.03   & 3.96   & 30.63\% & -11.11 & 0.000*** \\
                                 & Reply & 1.57   & 1.80   & 14.45\% & -8.16  & 0.000*** \\
\midrule
\multirow{2}{*}{\textbf{$t_2$} (July 5th, 2024)} & Post  & 3.17   & 4.35   & 37.22\% & -24.80 & 0.000*** \\
                                 & Reply & 1.60   & 1.89   & 18.12\% & -15.33 & 0.000*** \\
\hline
\end{tabular}
\caption{Average status levels in $\mathcal{U}_M$ before and after the introduction of Fediverse Sharing, measured at two distinct time points: $t_1$ (beta launch) and $t_2$ (post-update). The $p$-value is rounded to three digits, with significance indicated as $p^{***} < 0.001$.}
\label{table:before_after_activity_level}
\end{table*}


To account for potential shifts in the average number of statuses, we analyzed the status volume of $\mathcal{U}_M$ following the Fediverse Sharing updates. To enable meaningful comparisons, we first normalized the status volume by considering the number of users eligible to create a status. Here, the average activity levels before and after the updates were adjusted based on the duration of each period. The normalization was defined:

\[
\text{Normalized Status} = \frac{\text{\#Status}}{\text{\#Joined\:Users} \times \text{Days in Period}}
\]

Subsequently, we conducted a T-test on two distribution pairs. As shown in Table~\ref{table:before_after_activity_level}, the increase in status was more pronounced than in replies, likely due to the limited implementation of sequential replying. Nevertheless, both the average of posts and replies demonstrated statistically significant increases ($p < 0.001$).  This analysis focused on the status of $\mathcal{U}_M$, as historical data for $\mathcal{U}_T$ is unavailable. Due to API limit in Mastodon, the status for Threads users is only accessible from the point they joined federated networks, preventing retrospective analysis of their prior activity.


\subsubsection{Cross-platform Interaction Asymmetry}

Figure~\ref{fig:cross_platform_interaction_scatter} shows that cross-platform interactions between Threads and Mastodon follow a power-law distribution, dominated by a few interactions. The estimated power-law exponent is 1.24 for Mastodon-to-Threads and 1.39 for Threads-to-Mastodon, highlighting distinct engagement patterns. Mastodon-to-Threads interactions are more frequent, while Threads-to-Mastodon interactions are significantly lower, indicating asymmetry. The smaller $\alpha$ for Mastodon-to-Threads suggests concentrated engagement driven by a few highly active users.

\begin{figure}
  \includegraphics[width=0.435\textwidth]{cross_platform_interaction_power_law.pdf}
\caption{Scatter plot of cross-platform interactions between Threads users ($\mathcal{U}_T$) and Mastodon users ($\mathcal{U}_M$) through reply or like. Both interactions follow a power-law distribution.}
\label{fig:cross_platform_interaction_scatter}  
\end{figure}
% \subsubsection{Interacted Users} those users who replyed to Threads at least one time.

% \subsubsection{Non-Interacted Users}extracted from non-federated servers with similar profile statistics. Compare replies diversity and other profile-level features. Reply diversity statistical analysis as well.


% \begin{figure}
%   \includegraphics[width=0.5\textwidth]{post_reply_by_mastodon.jpg}
%   \caption{Trend of the daily number of posts and replies generated by Mastodon users ($\mathcal{U}_M$).}
%   \label{fig:post_reply_trend_mastodon}
% \end{figure}

% \subsubsection{Interaction Diversity}
% Higher diversity relates to more interaction with Threads.
\begin{tcolorbox}[colback=black!5!white,colframe=black!75!black,title=Summary 2 (Cross-platform Interaction Analysis),rounded corners]
% \small The varied characteristics of migrant groups indicate that each platform had its unique audience. Even with a substantial presence on Twitter compared to non-migrants, migrants  explored and established accounts on new platforms.
The monthly average of status updates increased after the Fediverse Sharing updates, with a large number of statuses referencing the platform itself rather than the other. Cross-platform interaction is asymmetrical, with Mastodon users engaging more frequently than Threads users.
\end{tcolorbox}
\subsection{What Do Users Express through Contents?}
\subsubsection{Toxicity across User Generated Contents}
Understanding how toxicity varies among users is important because exposure to different levels of toxicity can reveal group identify, platform moderation levels, and the overall quality of discourse. Encountering a broad spectrum of toxicity can present challenges for maintaining civil discussions, while also offering valuable insights into how various online communities will respond to different groups with hostility.

Figure~\ref{fig:toxicitycomparison_user_posts} shows the distribution of user-level toxicity scores, measured using the Perspective API\footnote{https://perspectiveapi.com}. This multilingual model assigns a toxicity score between 0 and 1, and we averaged the toxicity scores across users' status to determine their average toxicity level. Our analysis revealed differences in user-level toxicity across platforms. Mastodon exhibits higher toxicity levels than Threads, though both platforms have averages below 0.1 (very low toxicity). Mastodon has a wider range of toxicity scores, likely due to its decentralized structure, lack of central moderation, and fewer content restrictions. In contrast, Threads’ centralized platform and active corporate moderation result in a more friendly user experience.


% finetuned RoBERTa model~\cite{logacheva2022paradetox}.
% + entire politeness  of posts measured based on finetuned BERT model~\cite{danescu2013computational} and 
\begin{figure}
  \includegraphics[width=0.465\textwidth]{toxicitycomparison_user_posts_roberta.pdf}
\caption{Average user toxicity scores between Threads users ($\mathcal{U}_T$) and Mastodon users ($\mathcal{U}_M$). A user's toxicity score is calculated as the average toxicity score across all their posts.}  \label{fig:toxicitycomparison_user_posts}  
\end{figure}


\subsubsection{Topic Diversity Trend}
\begin{figure}
  \includegraphics[width=0.48\textwidth]{gini_coefficient_topics.pdf}
\caption{Gini coefficients of topic distribution over time for status by Threads users ($\mathcal{U}_T$) and Mastodon users ($\mathcal{U}_M$).}
% A higher Gini coefficient indicates a more imbalanced distribution of topics across posts.
\label{fig:gini_topic_trend}
\end{figure}

Figure~\ref{fig:gini_topic_trend} illustrates the evolution of topic diversity in historical status from $\mathcal{U}_M$ and $\mathcal{U}_T$, quantified using the Gini coefficient:

\[
Gini\:Coefficient = 1 - \sum_{i=1}^{n} p_i^2
\]
% \[
% H(X) = - \sum_{i=1}^{n} p_i \log_2 p_i
% \]
where \( p_i \) represents the proportion of status in cluster \( i \), and \( n \) is the total number of clusters.

We began by sampling 10K statuses per month to ensure balanced coverage across both platforms and reduce bias from unequal sample sizes. To identify topic clusters, we employed a transformer-based embedding \texttt{paraphrase-multilingual-MiniLM-L12-v2}. We then applied BERTopic~\cite{grootendorst2022bertopic}, a DBSCAN-based algorithm that does not require a predefined number of clusters. After filtering out unassigned dummy clusters and those with 10 or fewer statuses, we calculated the Gini coefficient for each month.


Figure~\ref{fig:gini_topic_trend} shows that Mastodon’s topic distribution imbalance decreases over time, while Threads exhibits an increase. However, since all values remain below 0.43, both platforms still maintain relatively balanced topic distributions. This divergence in trends is worth further examination to better understand how federation relates to conversational dynamics across platforms.

\subsubsection{Out-group Animosity across Platforms}

We analyzed the distribution of status referencing toward Mastodon and Threads to assess user sentiment toward the opposing platform. Animosity toward an out-group—has been shown to reinforce polarization, potentially hindering the successful integration of these platforms.

To quantify sentiment toward each platform, we analyzed the text content of statuses \( a \in \mathcal{A} \) using a fine-tuned Aspect-Based Sentiment Analysis model built on DeBERTa~\cite{he2020deberta, yang2021back}. This model classifies sentiment into three categories—\textit{positive}, \textit{neutral}, or \textit{negative}—for two aspects: \textit{Threads} and \textit{Mastodon}. 

We denote sentiment classification from the source to the target platform as \( \text{cls}(a, \mathcal{S}, \mathcal{T}) \), where \( \mathcal{S} \) is the source platform of the status and user, and \( \mathcal{T} \) is the target platform toward which the sentiment is expressed. Our analysis was performed at two levels:

\begin{enumerate}
    \item \textbf{Status-level}: For each status \( a \in \mathcal{A} \), we examined the distribution of sentiment labels \( \text{cls}(a, \mathcal{S}, \mathcal{T}) \) to determine how users express sentiment toward the target platform.
    
    \item \textbf{User-level}: For each user \( u \in \mathcal{U} \) with their corresponding set of statuses \( \mathcal{A}_u \), we computed the user’s overall sentiment toward the target platform based on the most frequent status labels.
    % \[
    % \text{cls}(u, \mathcal{T}, \mathcal{S}) = \arg\max_{s \in \{\text{positive, neutral, negative}\}} \sum_{a \in \mathcal{A}_u} \mathbb{1}[\text{cls}(a, \mathcal{T}, \mathcal{S}) = s]
    % \]
    where the user's sentiment is determined by the most frequent sentiment label across their statuses.
\end{enumerate}

Table~\ref{table:aspect_based_sentiment} presents sentiment distribution across source and target platforms. Threads users express more positivity toward Mastodon, while Mastodon users exhibit slightly greater negativity toward Threads. However, neutrality remains dominant on both platforms. Despite Mastodon's slight negativity, the prevalence of neutral and positive sentiment in Threads suggests potential for constructive cross-platform engagement in the future.


\begin{table}[]
\centering
\resizebox{0.4825\textwidth}{!}{%
\begin{tabular}{c c c c c c}
\hline
\textbf{Source} & \textbf{Target} & \textbf{Level} & \textbf{Positive} & \textbf{Neutral} & \textbf{Negative} \\
\hline
\multirow{2}{*}{Mastodon} & \multirow{2}{*}{Threads} & Status & 7.66\%  & 82.26\%  & 10.07\%  \\
                          &                         & User & 8.93\%   & 78.46\%  & 12.61\% \\
\midrule
\multirow{2}{*}{Threads}   & \multirow{2}{*}{Mastodon} & Status & 22.12\%  & 68.41\%  & 9.47\%  \\
                          &                         & User & 27.36\%  & 58.27\%  & 14.38\% \\
\hline
\end{tabular}%
}
\caption{Aspect-based sentiment distribution in two directions: (1) from Mastodon users to Threads and (2) from Threads users to Mastodon. Sentiment is analyzed at two levels: ``Status'' indicates the sentiment in each post or reply, while ``User'' represents the overall sentiment of each user.}
\label{table:aspect_based_sentiment}
\end{table}

% \begin{table}[]
% \centering
% \begin{tabular}{c c c c c}
% \hline
% \textbf{Direction} & \textbf{Level} & \textbf{Positive} & \textbf{Neutral} & \textbf{Negative} \\
% \hline
% \multirow{2}{*}{\parbox{2cm}{\centering Mastodon \\ $\rightarrow$ \\ Threads}} & Status & 7.66\%  & 82.26\%  & 10.07\%  \\
%                        & User & 8.93\%   & 78.46\%  & 12.61\% \\
% \midrule
% \multirow{2}{*}{\parbox{2cm}{\centering Threads \\ $\rightarrow$ \\ Mastodon}} & Status & 22.12\%  & 68.41\%  & 9.47\%  \\
%                        & User & 27.36\%  & 58.27\%  & 14.38\% \\
% \hline
% \end{tabular}
% \caption{Aspect-based sentiment distribution between Mastodon and Threads. The sentiment labels are shown at two levels: ``Status'' refers to the combination of posts and replies, while ``User'' reflects the overall sentiment of a user.}
% \end{table}


\begin{tcolorbox}[colback=black!5!white,colframe=black!75!black,title=Summary 3 (User Generated Content Analysis),rounded corners]
% \small The varied characteristics of migrant groups indicate that each platform had its unique audience. Even with a substantial presence on Twitter compared to non-migrants, migrants  explored and established accounts on new platforms.
Users from Threads and Mastodon show distinct content preferences. Toxicity is low on both, but Mastodon has a higher variance. Topic diversity evolved in opposite directions between platforms, though both remain balanced. While sentiment towards the other platform is mostly neutral, Threads users express more positivity toward Mastodon, indicating further openness to Mastodon.
\end{tcolorbox}

\subsection{What Instances Federate and Interact?}

Each Mastodon user \( u \in \mathcal{U}_M \) is affiliated with an instance \( i \) denoted as \( \mathcal{I}_M \), formally defined as:

\[
\mathcal{I}_M = \{ i \mid i = \text{home instance of } u, \; u \in \mathcal{U}_M \}
\]

\subsubsection{Instances Interacting with Threads}

Figure~\ref{fig:mastodon_users_interacted_servers} shows the distribution of home instances among \( \mathcal{U}_M \). Among the 1,417 Mastodon instances, 63.1\% (12,907 out of 20,456) of users in \( \mathcal{U}_M \) belong to \texttt{mastodon.social}, one of the largest instances in Mastodon.

\begin{figure}
  \centering
  \includegraphics[width=0.5\textwidth]{instance_distribution.pdf}
  \caption{Distribution of those interacted Mastodon users' home instances (\(\mathcal{I}_M\)). The top 10 instances are labeled with their respective percentages.}
  \label{fig:mastodon_users_interacted_servers}
\end{figure}
\subsubsection{Instance Size, Federation, and User Interaction}
We examined how instance size, defined by the number of registered users, impacts federation and interaction with Threads. Table~\ref{tab:coefficients} presents statistical relationships involving three key independent variables:

\begin{itemize}
    \item \textbf{\#Unique Interacted Users}: The proportion of users within an instance who interacted with Threads users.
    \item \textbf{\#Cross-Platform Interactions}: The frequency of interactions between an instance’s users and Threads users.
    \item \textbf{\#Federated Instances}: The number of Mastodon instances an instance has federated with.
\end{itemize}

The linear regression analysis shows that the overall cross-platform interaction count ($\beta = -0.667$) has a negative relationship with instance size, suggesting a possible bottleneck limiting the volume of interactions. However, given the small magnitude of this coefficient, this constraint appears to have only a minor impact, indicating that even smaller instances can sustain a high interaction volume with Threads. Conversely, the number of users experiencing cross-platform interaction ($\beta = 2.426$) is positively correlated with instance size, suggesting that larger instances offer greater opportunities for users to engage in cross-platform interactions at least once. Additionally, the number of federated instances ($\beta = 1.162$) positively relates to instance size, indicating that federation is mainly driven by users communicating across multiple instances.

Thus, although larger instances generally federate more extensively, increased federation alone does not necessarily lead to greater interaction specifically with Threads. Instead, it provides more chance for users to experience cross-platform interactions.
\begin{table}
\centering
\begin{tabular}{ccccc}
\hline
\textbf{Features}                   & \textbf{Coef.} & \textbf{SE}  & \textbf{$p$-value} & \textbf{OR}  \\
\hline
 \#Unique Interacted Users & 2.426   & 0.177 & 0.000   & 11.314\\
\#Cross-Platform Interaction & -0.667   &  0.144 &  0.000***   & 0.513 \\
\#Federated Instances        & 1.162   &  0.086 &  0.000***       & 3.196 \\
\hline
\end{tabular}
\caption{Results from OLS regression analyses exploring the relationship between instance size and three instance-level features. The table reports coefficient estimates (Coef.), standard errors (SE), and odds ratios (OR). Statistical significance at $p^{***} < 0.001$. All values are rounded to three decimal places.}
\label{tab:coefficients}
\end{table}


% \begin{figure}
%   \centering
%   \includegraphics[width=0.5\textwidth]{interaction_ratio_servers.jpg}
%   \caption{Scatter plot depicting the number of registered users in each Mastodon instance and their interaction ratio with Threads posts. Linear regression line is clipped at zero.}
%   \label{fig:interaction_ratio_servers}
% \end{figure}

\begin{tcolorbox}[colback=black!5!white,colframe=black!75!black,title=Summary 4 (Federation of Instances Analysis),rounded corners]
% \small The varied characteristics of migrant groups indicate that each platform had its unique audience. Even with a substantial presence on Twitter compared to non-migrants, migrants  explored and established accounts on new platforms.
Most Mastodon users who interacted with Threads are from mastodon.social, the largest Mastodon instance. However, a bottleneck in cross-platform interaction exists, as total interactions do not scale with instance size growth. Given that federation is driven primarily at the user level, maintaining user motivation and engagement becomes even more critical to sustainable platform integration.
\end{tcolorbox}

% \section{Downstream Tasks (TBD)}
% I am listing some potential applications we can try
% \begin{itemize}
%     \item \textbf{Cross-platform Virality}: Analyze if viral posts on Threads also gain traction on Mastodon.
%     \item \textbf{Temporal Analysis of Reposts}: Measure time delays between posts across platforms and identify leading vs. lagging platforms.
%     \item \textbf{Homophily vs. Heterophily}: Study if users engage more with similar interests (homophily) or diverse viewpoints (heterophily).
%     \item \textbf{Content Migration}: Track when controversial or banned content moves from one platform to another.
%     \item \textbf{Sentiment Shifts}: Investigate if sentiment towards a topic changes across Threads and Mastodon.
%     \item \textbf{Polarization Detection}: Measure signs of ideological polarization in user interactions.
% \end{itemize}

 % or they perform actions that keep their presence synced across instances
\section{Limitations}
Threads users are not discoverable through the Mastodon API unless a Mastodon user attempts to access their account or status. Since inactive Threads users leave no persistent records on any instance, their visibility remains limited. Additionally, replies from Threads users who have not activated Fediverse sharing are invisible to Mastodon users, which can hinder the understanding of complete conversations across platforms. Our findings are based on a subset of Threads users who manually opted into Fediverse sharing and should not be generalized to the entire Threads user base. Furthermore, we could not analyze the following feature between Threads and Mastodon, introduced on December 2, 2024, as Threads restricts access to user networks, rendering this aspect of the study infeasible.


\section{Future Work}
We will establish a controlled methodology to derive generalized causal effects in cross-platform interactions using the difference-in-differences (DID) analysis~\cite{cao2024consequences}. To construct a control group, we will survey Mastodon users who were unaware of Threads. This will help determine whether increased status update activity results from platform integration, providing insights into how interconnected ecosystems shape user engagement~\cite{huang2017social}. Furthermore, we will examine whether trust-building behavior influences self-referencing in status updates, assessing trust as a causal factor in how users present their platform affiliation. Ethnogenesis studies suggest that trust drives identity disclosure in initial interactions, making it essential to explore similar patterns in cross-platform dynamics~\cite{banerjee2022situational, okamura1981situational}. Last, we will investigate whether behavioral differences between Threads and Mastodon users stem from their structural differences, with Threads being centralized and Mastodon decentralized. These differences likely reflect variations in users’ goals and ways of engaging with each platform, which may also impact the extent of cross-platform interactions~\cite{jeong2024exploring, la2022network}.


\section{Conclusion}
This study introduces the \texttt{FediverseSharing} dataset, a novel resource capturing cross-platform interactions between users on Threads and Mastodon. Through a detailed analysis of user behavior, we uncover key insights into Fediverse joining patterns, engagement trends, and shifting preferences resulting from platform integration. Our findings highlight how integration unfolds across different interaction levels and topic diversities, revealing nuanced changes in users' content preferences between platforms. Additionally, we examine emerging trends in user sentiment, topic diversity, and interaction disparities across the platforms. This research provides a foundational dataset and analytical framework for future studies on cross-platform interaction dynamics within federated environments. By making this dataset publicly available, we encourage further exploration into the effects of platform integration, federated social networks, and broader shifts within the evolving online platform ecosystem.

\newpage
% \section{Acknowledgement}
% The authors would like to acknowledge the support of the Army Research Laboratory (ARL), Department of Defense (DoD), U.S. Army (AMC), under grant number W911NF-24-2-0175. Any interpretations or conclusions presented in this work are those of the authors and do not necessarily reflect the views of the funding agency.

\bibliographystyle{plain}
\bibliography{reference}

\end{document}
\endinput
%%
%% End of file `sample-sigconf.tex'.
