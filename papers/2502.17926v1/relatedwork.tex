\section{Related Work}
We review two key aspects of this dataset project in the context of social media: (1) platform interactions and (2) community interactions, discussing their implications for cross-platform interaction.

\subsection{Platform Integration}
Platform integration is a strategic business decision, often structured as a two-sided market, where platforms act as intermediaries connecting user groups with complementary needs~\cite{schreieck2024typology, evans2016matchmakers, rochet2003platform}. This model facilitates interactions between distinct user segments, driving engagement and market expansion. By linking services, integrations enhance accessibility and create new revenue streams. For example, Google’s integration of YouTube with Search refines content recommendations, reinforcing creator-audience relationships while increasing ad revenue~\cite{parker2017platform, huang2017social}.


Social media platforms have historically emphasized content visibility over direct cross-platform engagement. Facebook’s early integrations with TripAdvisor and Yelp enabled content sharing through Facebook but did not support interaction between users across different services~\cite{cao2024consequences, huang2017social}. Similarly, Facebook and Instagram facilitate content distribution while maintaining distinct user bases, leveraging network effects to increase audience reach~\cite{abdelkafi2019multi}.

Recently, open-source platforms like Mastodon and Bluesky have promoted federation-driven social media, shifting control away from corporate-owned platforms~\cite{la2021understanding, jeong2024bluetempnet}. Protocols like ActivityPub and AT Protocol foster interoperability, allowing different platforms to communicate. However, these integrations raise new challenges regarding governance, content moderation, and the long-term sustainability of federated social networks~\cite{zhang2024emergence}.


\subsection{Community Interaction}
Community dynamics play a fundamental role in shaping human behavior, as individuals' actions are influenced by their group identity, perceptions of outsiders, and the stability of their communities~\cite{schmidt2025concept}. Offline research, such as studies on ethnogenesis, examines how intergroup interactions contribute to the formation of new collective identities~\cite{okamura1981situational}. When cultures clash between communities, individuals actively share and signal aspects of their identity to build trust within their group, reinforcing social cohesion while delineating boundaries with outsiders~\cite{banerjee2022situational}.

Similarly, online communities function as digital tribes, where trust is built through the exchange and reinforcement of community norms~\cite{pettigrew2013does, demoulin2013intergroup}. Platforms like Reddit illustrate how engagement in toxic or partisan subreddits can spread hostility, influencing the behavior of adjacent communities~\cite{russo2024stranger, kumar2018community}. Research on political polarization suggests that exposure to diverse perspectives, including out-group perspectives, can reduce animosity and affective polarization, fostering better understanding of opposing groups~\cite{cohen2010competitive}. While such exposure may lower outgroup animosity, it is often shown to be insufficient to alter polarization.

Federated social networks offer a new model of community interaction, distinguished by decentralized governance and inconsistent moderation. Unlike corporate platforms with standardized policies, federated networks rely on community-driven oversight, leading to variations in content regulation and trust-building~\cite{schreieck2024typology}. While this grants users more autonomy, it also raises concerns about stability, harmful content, and ideological fragmentation. Without centralized enforcement, it remains uncertain whether these platforms will foster healthier discourse or deepen divisions.

\subsection{Implications of Our Study}
We distinguish our study as the first to analyze cross-platform interactions at the platform level within a federated social media ecosystem. Our FediverseSharing dataset offers key insights into the relationship between platform integration and user behavior. We examine the first integration between these structurally distinct platforms, highlighting how federation reshapes online interactions. As large-scale social media research faces growing challenges from fragmentation, restricted access, and diverse governance structures ~\cite{ng2024smi}, our dataset serves as a crucial resource for understanding the new social media landscape.









% \subsection{User Migration on Social Media}
% The emergence of decentralized social networks has influenced user migration patterns and reshaped the social media landscape. Mastodon, an open-source decentralized platform, offers a federated alternative to traditional social media by enabling users to join independent servers, or ``instances,'' each governed by unique policies and moderation rules~\cite{zignani2018follow}. This decentralized architecture grants users greater autonomy over their data and interactions, fostering a diverse and user-driven ecosystem. Studies on platform migration have shown that users seek decentralized alternatives due to concerns over content moderation, privacy, and corporate influence in mainstream social media~\cite{jeong2024exploring, jeong2024user}.

% In contrast, Meta’s Threads, launched in July 5th, 2023, has taken a novel approach by integrating with the decentralized ecosystem through its adoption of the ActivityPub protocol. This move allows Threads users to interact with individuals on Mastodon and other federated platforms without requiring separate accounts, effectively bridging the gap between centralized and decentralized social networks~\cite{cava2023drivers}. Research on cross-platform migration suggests that major shifts, such as the Twitter (now X) exodus, highlight user responses to changes in platform governance~\cite{jeong2024user}. However, integrating federated and corporate platforms introduces new challenges related to governance, interoperability, and user retention, which require further examination~\cite{cohen2010competitive}.

% As social media platforms continue to evolve, understanding the dynamics of user migration, platform integration, and community interaction becomes increasingly important. Future research should focus on the long-term effects of these developments on user behavior and platform sustainability.