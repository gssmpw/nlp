% This must be in the first 5 lines to tell arXiv to use pdfLaTeX, which is strongly recommended.
\pdfoutput=1
% In particular, the hyperref package requires pdfLaTeX in order to break URLs across lines.

\documentclass[11pt]{article}

% Change "review" to "final" to generate the final (sometimes called camera-ready) version.
% Change to "preprint" to generate a non-anonymous version with page numbers.
\usepackage[preprint]{acl}

% Standard package includes
\usepackage{times}
\usepackage{latexsym}

% For proper rendering and hyphenation of words containing Latin characters (including in bib files)
\usepackage[T1]{fontenc}
% For Vietnamese characters
% \usepackage[T5]{fontenc}
% See https://www.latex-project.org/help/documentation/encguide.pdf for other character sets

% This assumes your files are encoded as UTF8
\usepackage[utf8]{inputenc}

% This is not strictly necessary, and may be commented out,
% but it will improve the layout of the manuscript,
% and will typically save some space.
\usepackage{microtype}

% This is also not strictly necessary, and may be commented out.
% However, it will improve the aesthetics of text in
% the typewriter font.
\usepackage{inconsolata}

%Including images in your LaTeX document requires adding
%additional package(s)
\usepackage{graphicx}

% Custom lib for tables
\usepackage{booktabs}
\usepackage{multirow}
\usepackage{caption}
\usepackage{adjustbox} % for resizing if needed
\usepackage{array}
\usepackage{xspace}
\usepackage{cleveref}
%\usepackage{minted}
\usepackage{tcolorbox}

\newcommand{\methodlong}{Agentic Long-Context Understanding\xspace}
\newcommand{\method}{AgenticLU\xspace}
\newcommand{\coc}{CoC\xspace}
\newcommand{\coclong}{Chain-of-Clarifications\xspace}
\newcommand{\jingbo}[1]{\textcolor{blue}{\textbf{Jingbo}: #1}}
\newcommand{\yufan}[1]{\textcolor{green}{\textbf{Yufan}: #1}}

% If the title and author information does not fit in the area allocated, uncomment the following
%
%\setlength\titlebox{<dim>}
%
% and set <dim> to something 5cm or larger.

\title{Self-Taught Agentic Long-Context Understanding}

% Author information can be set in various styles:
% For several authors from the same institution:
% \author{Author 1 \and ... \and Author n \\
%         Address line \\ ... \\ Address line}
% if the names do not fit well on one line use
%         Author 1 \\ {\bf Author 2} \\ ... \\ {\bf Author n} \\
% For authors from different institutions:
% \author{Author 1 \\ Address line \\  ... \\ Address line
%         \And  ... \And
%         Author n \\ Address line \\ ... \\ Address line}
% To start a separate ``row'' of authors use \AND, as in
% \author{Author 1 \\ Address line \\  ... \\ Address line
%         \AND
%         Author 2 \\ Address line \\ ... \\ Address line \And
%         Author 3 \\ Address line \\ ... \\ Address line}

% \author{First Author \\
%   Affiliation / Address line 1 \\
%   Affiliation / Address line 2 \\
%   Affiliation / Address line 3 \\
%   \texttt{email@domain} \\\And
%   Second Author \\
%   Affiliation / Address line 1 \\
%   Affiliation / Address line 2 \\
%   Affiliation / Address line 3 \\
%   \texttt{email@domain} \\}

\author{
 \textbf{Yufan Zhuang\textsuperscript{1,2}},
 \textbf{Xiaodong Yu\textsuperscript{1}},
 \textbf{Jialian Wu\textsuperscript{1}},
 \textbf{Ximeng Sun\textsuperscript{1}},
 \textbf{Ze Wang\textsuperscript{1}},\\
 \textbf{Jiang Liu\textsuperscript{1}},
 \textbf{Yusheng Su\textsuperscript{1}},
 \textbf{Jingbo Shang\textsuperscript{2}},
 \textbf{Zicheng Liu\textsuperscript{1}},
 \textbf{Emad Barsoum\textsuperscript{1}}
\\
 \textsuperscript{1}AMD,
 \textsuperscript{2}UC San Diego
% \\
% \{\href{mailto:y5zhuang@ucsd.edu}{y5zhuang}, \href{mailto:jshang@ucsd.edu}{jshang}\}@ucsd.edu \\
% \{\href{mailto:xiaodong.yu@amd.com}{xiaodong.yu}, \href{mailto:zicheng.liu@amd.com}{zicheng.liu}\}@amd.com
}

\begin{document}
\maketitle
\begin{abstract}

% \jingbo{How about this flow:
% (1) The most challenging questions for LLMs are arguably the long-context questions that require multi-round of clarifications and retrieval. 
% (2) We propose to learn to decide when and what to clarify.
% (3) Specifically, following a pre-defined tree search workflow, we apply a LLM to long-context QA datasets to generate a variety of reasoning traces including the clarification questions and the contextual ground answers.
% (4) We then utilize these traces to fine-tune the LLM via SFT and DPO so the model can decide when to ask a specific clarification question, automating the agentic workflow to answer complex questions.
% (5) Extensive experiments ... 
% }

Answering complex, long-context questions remains a major challenge for large language models (LLMs) as it requires effective question clarifications and context retrieval. 
We propose \methodlong (\method), a framework designed to enhance an LLM’s understanding of such queries by integrating targeted self-clarification with contextual grounding within an agentic workflow.
At the core of \method is Chain-of-Clarifications (\coc), where models refine their understanding through self-generated clarification questions and corresponding contextual groundings. 
By scaling inference as a tree search where each node represents a \coc step, we achieve 97.8\% answer recall on NarrativeQA with a search depth of up to three and a branching factor of eight.
To amortize the high cost of this search process to training, we leverage the preference pairs for each step obtained by the \coc workflow and perform two-stage model finetuning: (1) supervised finetuning to learn effective decomposition strategies, and (2) direct preference optimization to enhance reasoning quality. This enables \method models to generate clarifications and retrieve relevant context effectively and efficiently in a single inference pass.
Extensive experiments across seven long-context tasks demonstrate that \method significantly outperforms state-of-the-art prompting methods and specialized long-context LLMs, achieving robust multi-hop reasoning while sustaining consistent performance as context length grows. \renewcommand\thefootnote{}\footnotetext{Code and data is available at: \url{https://github.com/EvanZhuang/AgenticLU}.}\renewcommand\thefootnote{\arabic{footnote}}


% One of the most challenging tasks for large language models (LLMs) is answering complex, long-context questions that require multiple rounds of clarification and retrieval. 
% We propose \methodlong (\method), a framework that learns to refine their understanding of complex, long-context questions by combining self-clarification and contextual grounding within an agentic workflow. 


% One of the most challenging tasks for large language models (LLMs) is answering complex, long-context questions that require effective question clarification and context retrieval. 
% We propose \methodlong (\method), a framework that refines an LLM’s understanding of complex, long-context queries by integrating targeted self-clarification with contextual grounding within an agentic workflow.
% Specifically, we apply LLM to long-context QA datasets (with ground-truth answers) to generate reasoning traces through tree search over candidate reasoning steps. By exploring multiple paths, the search process automatically discovers both effective and ineffective traces --- successful branches that reach correct answers serve as positive examples, while failed branches are collected as negative examples.
% The tree search is effective as we were able to solve 97.8\% of the problems with up to three rounds of self-clarification and contextual grounding.
% We then form preference pairs for each step in the agentic workflow and train the model in two stages: (1) supervised finetuning to learn decomposition strategies, and (2) direct preference optimization to refine reasoning quality. 
% This enables models to synthesize appropriate clarifications, locate relevant information, and generate accurate responses, even when context lengths stretch up to 128K tokens. 
% Extensive experiments across seven long-context tasks show that \method significantly outperforms state-of-the-art prompting methods and specialized long-context models, achieving robust multi-hop reasoning while maintaining consistent performance as context lengths grow. 


%We introduce \methodlong (\method), a framework for training large language models (LLMs) to refine their understanding of complex, long-context questions by combining self-clarification and contextual grounding.
%Our approach employs a tree search process \jingbo{one thing to clarify here is ``what is this tree?'' What are the branches? It somehow implies both positive and negative samples, but it remains mysterious to the reviewers.} that efficiently generates high-quality training data without requiring manual annotations\jingbo{we need to start with a dataset where we know the groundtruth answer. is my understanding correct? if so, no annotation only applies to the process?}, automatically discovering effective reasoning paths and context groundings, and creating preference pairs for each step of the agentic workflow.
%The framework uses a two-stage training recipe: first applying supervised fine-tuning (SFT) to learn decomposition strategies, followed by direct preference optimization (DPO) to refine reasoning quality.
%\jingbo{are we assuming the LLM is kind of good in the first place, so in the tree search it can find some possible paths? What will happen if we didn't find a path to the correct answer? We will discard that tree or only use the paths there as negative cases?}
%This enables models to systematically break down complex queries, precisely locate relevant information in extended contexts, and generate accurate responses.
%Extensive evaluations across seven long-context tasks (8K-128K tokens) demonstrate that \method significantly outperforms both state-of-the-art prompting methods and specialized long-context models, showing particular strength in multi-hop reasoning tasks while maintaining consistent performance across increasing context lengths.
% Our results show that explicit training for self-clarification and contextual grounding can substantially improve LLMs' long-context understanding capabilities.
\end{abstract}


\section{Introduction}
Surveys provide an essential tool for probing public opinions on societal issues, especially as opinions vary over time and across subpopulations.
However, surveys are also costly, time-consuming, and require careful calibration to mitigate non-response and sampling biases \cite{choi2004catalog, bethlehem2010selection}. 
Recent work suggests that large language models (LLMs) can assist public opinion studies by predicting survey responses across different subpopulations, explored in both social science ~\cite{argyle2023out,bail2024can,ashokkumar2024predicting,manning2024automated} and NLP~\cite{santurkar2023whose,chu2023language,moon-etal-2024-virtual,hamalainen2023evaluating,chiang2023can}.
Such capabilities could substantially enhance the survey development process, not as a replacement for human participants but as a 
tool for researchers to conduct pilot testing, identify subpopulations to over-sample, and test analysis pipelines prior to conducting the full survey  \cite{rothschild2024survey}.

\begin{figure}
    \centering
    \includegraphics[width=1.0\linewidth]{figures/teaser.pdf}
    \caption{Illustration of our method and \OURDATA. We collect survey data from two survey families—ATP from Pew Research~\cite{atp} (forming \OURDATA-Train) and GSS from NORC~\cite{davern2024gss} (forming \OURDATA-Eval). 
    LLMs are fine-tuned on \OURDATA-Train and evaluated on both OpinionQA~\cite{santurkar2023whose} and \OURDATA-Eval to assess generalization of distributional opinion prediction across unseen survey topics, survey families, and subpopulations.
    }
    \label{fig:teaser}
\end{figure}

Prior work in steering language models, \textit{i.e.} conditioning models to reflect the opinions of a specific subpopulation, has primarily investigated different prompt engineering techniques~\cite{santurkar2023whose, moon-etal-2024-virtual, park2024generative}. However, prompting alone has shown limited success in generating completions that accurately reflect the distributions of survey responses collected from human subjects. Off-the-shelf LLMs~\cite{achiam2023gpt, dubey2024llama, jiang2023mistral} have shown to mirror the opinions of certain US subpopulations such as the wealthy and educated \cite{santurkar2023whose,gallegos2024bias,deshpande2023toxicity,kim2023ai}, while generating stereotypical or biased predictions of underrepresented groups ~\cite{cheng2023compost,cheng2023marked,wang2024large}. Furthermore, these models often fail to capture the variation of human opinions within a subpopulation \cite{kapania2024simulacrum, park2024diminished}.
While fine-tuning presents opportunities to address these limitations ~\cite{chu2023language, he2024community}, existing methods fail to train models that accurately predict opinion distributions across diverse survey question topics and subpopulations.

\vspace{-5pt}
\paragraph{The present work.}
Here, we propose directly fine-tuning LLMs on large-scale, high-quality survey data,
consisting of questions about diverse topics and responses from each subpopulation, defined by demographic, socioeconomic, and ideological traits.
By casting pairs of (subpopulation, survey question) as input prompts, we train the LLM to align its response distribution against that of human subjects in a supervised manner.
We posit that survey data is particularly well-suited for fine-tuning LLMs since: (1) We can train the model with clear \textbf{subpopulation-response pairs} that explicitly link group identities and expressed opinions,
which is rare in LLMs' pre-training corpora,
(2) Large-scale opinion polls are carefully designed and calibrated (\textit{e.g.} using post-stratification) to estimate \textbf{representative} human responses, in contrast with LLMs' pre-training data where certain populations are over- or underrepresented, 
(3) Our training objective explicitly aligns model predictions with response \textbf{distributions} from each subpopulation, enabling LLMs to capture variance within human subpopulations.

Training on public opinion survey data has remained under-explored due to the limited availability of structured survey datasets. 
To this end, we curate and release \textbf{\OURDATA} (\textbf{Sub}population-level \textbf{P}ublic \textbf{O}pinion \textbf{P}rediction), a dataset of 70K subpopulation-response distribution pairs ($6.5\times$ larger compared to previous datasets).
We show that fine-tuning LLMs on \OURDATA significantly improves the distributional match between LLM generated and human responses, and improvements are consistent across subpopulations of varying sizes.
Additionally, the improvement generalizes to \textit{unseen} subpopulations, survey waves (topics), and survey families, \textit{i.e.} surveys administered by different institutions.
Such broad generalization is particularly critical for real-world public opinions research, where practitioners are most in need of synthetic data for survey questions or subpopulations (or both) that they have not tested before.

Our contributions are summarized as follows:
\vspace{-3mm}
\begin{itemize}[leftmargin=3.3mm]
\setlength\itemsep{2pt}

\item We show that training LLMs on response distributions from survey data significantly improves their ability to predict the opinions of subpopulations, reducing the Wasserstein distance between LLM and human distributions by 32-46\% compared to top-performing baselines. (\Cref{section_experiments_prediction_of_opinion_distributions})
\vspace{-1mm}
\item We show that the performance of the fine-tuned LLMs strongly generalizes to out-of-distribution data, including unseen subpopulations, new survey waves, and different survey families. 
(\Cref{section_experiments_prediction_of_opinion_distributions} and \Cref{section_experiments_per_group})
\vspace{-1mm}
\item We release \OURDATA, a curated and pre-processed dataset of public opinion survey results that is $6.5\times$ larger than existing datasets, enabling fine-tuning at scale.
\end{itemize}
\section{Related Work}
\label{sec:related_work}
\paragraph{Challenges in Long Context Understanding}
LLMs struggle with long contexts despite supporting up to 2M tokens~\cite{dubey2024llama3,reid2024gemini}. 
The ``lost-in-the-middle'' effect~\cite{liu2024lost} and degraded performance on long-range tasks~\cite{li2023loogle} highlight these issues. To address this, ProLong~\cite{prolong} finetunes base models on a large, carefully curated long-context corpus. While this approach improves performance on long-range tasks, it comes at a significant cost, requiring training with an additional 40B tokens and long-input sequences.


%Recent studies have highlighted significant challenges in LLMs' processing of extended contexts. While models like Llama-3~\cite{dubey2024llama3} and Gemini~\cite{reid2024gemini} support context windows up to 128K or even 2M tokens, they struggle with effective utilization of this capacity. 
%The ``lost-in-the-middle'' phenomenon~\cite{liu2024lost} shows that models often fail to leverage information from the middle of long contexts, while \citet{li2023loogle} demonstrated that performance degrades significantly on tasks requiring long-range dependencies. 
%To address this issue, ProLong~\cite{prolong} finetunes base models on a large, carefully curated long-context corpus. While this approach improves performance on long-range tasks, it comes at a significant cost, requiring training with an additional 40B tokens and long-input sequences.

% ProLong~\cite{prolong} provides a solution through extensive continued pretraining (40B tokens) on long-context data, but this approach requires significant computational resources and may not work well for tasks require more than raw long-context abilities.
% These studies suggest that merely increasing the context window size is insufficient; enhancing true long-context understanding remains a significant challenge.

\paragraph{Inference-time Scaling for Long-Context}
The Self-Taught Reasoner (STaR) framework \citep{zelikman2022star} iteratively generates rationales to refine reasoning, with models evaluating answers and finetuning on correct reasoning paths. \citet{wang2024multi} introduced Model-induced Process Supervision (MiPS), automating verifier training by generating multiple completions and assessing accuracy, boosting PaLM 2's performance on math and coding tasks. \citet{li2024large} proposed an inference scaling pipeline for long-context tasks using Bayes Risk-based sampling and fine-tuning, though their evaluation is limited to shorter contexts (10K tokens) compared to ours (128K tokens).

%The Self-Taught Reasoner (STaR) framework, proposed by \citet{zelikman2022star}, presents a method where language models iteratively generate step-by-step rationales to improve reasoning capabilities. This approach involves the model generating rationales for questions, evaluating the correctness of the answers, and fine-tuning based on successful reasoning paths. %Building upon this, \citet{zelikman2024quiet} introduced \textsc{Quiet-STaR}, which enables models to generate internal rationales at each token to enhance predictions. These methods aim to improve inference-time reasoning without extensive human supervision. 
%Furthermore, \citet{wang2024multi} introduced Model-induced Process Supervision (MiPS), an automated data curation method that eliminates the need for human annotation in training verifiers. MiPS involves the model generating multiple completions of an intermediate solution step and calculating the accuracy based on the proportion of correct completions. Their approach significantly improved the performance of PaLM 2 on math and coding tasks. 
%Building on these ideas, \citet{li2024large} proposed an inference scaling pipeline for long-context tasks where LLM outputs are sampled and weighted using Bayes Risk, followed by fine-tuning on preferred outputs. Although their approach shares similarities with ours, their evaluation focuses on much shorter context lengths (around 10K tokens) compared to ours (up to 128K tokens).

% On this line of research, \citet{li2024large} proposed inference scaling pipeline to sample outputs from LLMs and weight them with Bayes Risk. They then finetune the model on preferred outputs. While sharing similarity with our approach, the context length of the problems considered in the paper is significantly shorter (around 10K) than ours (up to 128K).

\paragraph{Agentic Workflow for Long-Context} 
Agentic workflows~\cite{yao2022react} enable LLMs to autonomously manage tasks by generating internal plans and refining outputs iteratively. 
The LongRAG framework~\cite{zhao-etal-2024-dual} enables an LLM and an RAG module to collaborate on long-context tasks by breaking down the input into smaller segments, processing them individually, and integrating the results to form a coherent output.
Chain-of-Agents (CoA)~\cite{zhang2024chain} tackles long-context tasks through decomposition and multi-agent collaboration. In CoA, the input text is divided into segments, each handled by a worker agent that processes its assigned portion and communicates its findings to the next agent in the sequence.
Unlike these, our approach employs a single LLM that orchestrates its own reasoning and retrieval without relying on multiple components. By dynamically structuring its process and iteratively refining long-context information, our model reduces complexity while maintaining efficiency.



% \paragraph{Agentic Workflow for Long-Context}
% The concept of agentic workflows~\cite{yao2022react} in LLMs involves structuring models to autonomously manage tasks by generating and following internal plans or chains of thought. This approach allows models to handle complex tasks by decomposing them into manageable steps and iteratively refining their outputs. For instance, the LongRAG framework~\cite{zhao-etal-2024-dual} enables an LLM and an RAG module to collaborate on long-context tasks by breaking down the input into smaller segments, processing them individually, and integrating the results to form a coherent output. This method enhances the model's ability to manage and reason over extended contexts by leveraging internal planning and iterative refinement. Chain-of-Agents (CoA)~\cite{zhang2024chain} addresses the challenges of processing long-context tasks by leveraging multi-agent collaboration among LLMs. In CoA, the input text is divided into segments, each handled by a worker agent that processes its assigned portion and communicates its findings to the next agent in the sequence.
% Different from these approaches, we focus on an agentic system with a single primary LLM that autonomously orchestrates its reasoning and retrieval processes without relying on multiple interacting components. 
% Instead of distributing tasks across separate entities, our model dynamically structures its own reasoning process, iteratively retrieving, attending to, and refining long-context information within a unified framework. This enables efficient handling of extended contexts while reducing the complexity introduced by multi-agent coordination.


 
\section{The Context Size Gap}
% Many contemporary LLMs claim to have extremely long context sizes.
% Proprietary models like GPT4o~\cite{openai2023gpt4}, Claude3.5-Sonnet~\cite{claude3} and Google Gemini-2.0-Pro~\cite{team2023gemini} claims to have 128K, 200K, and 2M tokens. 
% Open sourced models such as Llama3 series of models claims 128K context length~\cite{dubey2024llama3}, Qwen2.5 just released long context versions of 7B and 14B models supporting up to 1M tokens~\cite{yang2025qwen2}. Seems with scaling and efficient long-context attention mechanisms, the evolving context size could be able to solve most of the current long-context problems.


% Recent large language models have made strong claims about their context lengths. For example, proprietary models such as GPT4o~\cite{openai2023gpt4}, Claude3.5-Sonnet~\cite{claude3}, and Google Gemini-2.0-Pro~\cite{team2023gemini} state that they can handle up to 128K, 200K, and 2M tokens, respectively. Meanwhile, open-source models like the Llama3 series~\cite{dubey2024llama3} advertise 128K-token contexts, and Qwen2.5’s latest long-context versions (7B/14B) reportedly support contexts of up to 1M tokens~\cite{yang2025qwen2}. At first glance, these rapid developments—enabled by scaling and more efficient attention mechanisms—appear poised to solve many of the current long-context challenges.

% However, the situation is more nuanced than the numbers suggest. Recent studies~\cite{prolong,yen2024helmet,shang2024ai} have shown that the \emph{effective} context size of an LLM (the length over which it can reliably perform tasks such as information retrieval and complex reasoning) often diverges from its claimed, or \emph{nominal}, context length. To illustrate this gap, we conducted an experiment demonstrating that models’ reasoning abilities diminish as context lengths grow, reinforcing the discrepancy between nominal and effective context sizes.

State-of-the-art LLMs have made strong claims about their context lengths, supporting hundreds of thousands of input tokens. However, recent studies~\cite{prolong,yen2024helmet,shang2024ai} have shown that the \emph{effective} context size of an LLM (the length over which it can reliably perform tasks such as information retrieval and complex reasoning) often diverges from its claimed, or \emph{nominal}, context length. 

% To illustrate this gap, we conducted an experiment demonstrating that models’ reasoning abilities diminish as context lengths grow, reinforcing the discrepancy between nominal and effective context sizes.

To illustrate this gap, we evaluate Llama3.1-8B-Instruct, which supports a 128K-token context, on the HotPotQA dataset to test multi-hop QA performance at various input lengths (8K, 16K, 32K, 64K, and 128K). We artificially expand the input by adding irrelevant context and measure the accuracy of its answers  using GPT-4o as a judge. As shown in \cref{fig:hotpotQA}, The model’s performance degrades substantially as increasing context length, demonstrating the discrepancy between nominal and effective context sizes.

\begin{figure}[t!]
    \begin{center}
    \includegraphics[width=0.8\columnwidth , keepaspectratio]{img/hotpotqa.pdf}
    \end{center}
    \caption{\textbf{Effective context size is smaller than nominal context size.} 
    Performance of Llama3.1-8B-Instruct (advertised 128K-token context) on the HotPotQA dataset 
    drops sharply as input length increases (8K, 16K, 32K, 64K, 128K), illustrating the 
    gap between nominal and effective context capacities.}
    \label{fig:hotpotQA}
\end{figure}


While expanding nominal context capacity is undoubtedly important, we argue that it is not sufficient for solving all long-context problems. By analogy with computer memory, simply having more capacity does not guarantee efficient or accurate computation; one must also manage the ``loading'' of relevant information in and out of this memory. Therefore, we propose an agentic workflow aimed at helping LLMs process and interpret extended contexts more intelligently. 

% To further streamline this approach, we apply inference-time scaling on the base model during the data construction period to automatically generate agentic reasoning traces and paragraph groundings, reducing the need for labor-intensive human annotation. With our newly generated data, we further perform SFT and DPO on the base model to distill Chain of Clarification (CoC) ability from the sampled agentic reasoning traces and paragraph groundings. In inference of our finetuned model (\method), we do not require inference-time scaling which saves the inference cost. (See details in Sec.~\ref{sec:methodology})

% However, the reality is a little bit more complex that that.
% It has been pointed out in recent research~\cite{prolong,yen2024helmet, shang2024ai} that LLM's claimed context size is often mismatched with its effective context size, i.e., the context size that it can perform non-trivial information processing tasks such as information retrieval and reasoning.

% We perform an experiment to illustrate the gap, illustrating that model's reasoning ability decays as the context grow, solidifying the existence of the gap between nominal context size and effective context size.
% experiment result will be added as a figure here

% To solve the long context problems, we believe extending the nominal context size is a key step but not the most important step.
% We foresight that the context size would behave like computer memories, to process large-scale computation challenges it is true that one needs a big enough memory, but more importantly, the algorithm to smartly load in and load out things into and out of the memory.
% Therefore we propose the agentic workflow to encourage LLMs to integetly process and comprehend long-context queries. 
% And to efficiently obtain training data without costly human labeling effort, we utilize techniques of inference-time scaling to generate traces of agentic reasoning and paragraph grounding.





\section{\coclong Workflow}
\label{sec:methodology}
Our approach centers on enhancing long-context comprehension through an iterative, self-refining process that blends inference-time scaling with agentic reasoning. 
We coin this agentic workflow Chain-of-Clarifications (\coc).
In this section, we detail its key components, including the self-clarification process and the pointback mechanism, as illustrated in~\cref{fig:pipeline}.

Our proposed \coc framework is designed to mitigate the gap between nominal and effective context sizes in large language models. 
Rather than processing the entire long context and potentially multi-hop questions in a single pass, our methodology decomposes the task into a sequence of targeted sub-tasks. At each \coc step, the model autonomously:

\begin{itemize}
    \item \textbf{Generates clarifying questions} by identifying areas of the long input that require further elaboration or are prone to misinterpretation.
    \item \textbf{Pointbacks to relevant context} by using a pointback mechanism that highlights critical segments of the context by naming the index of relevant paragraphs. 
    In the data collection phase, this is done by iteratively querying the LLM about the relevance of each paragraph with respect to the question.
    After training, the model is finetuned to generate the related paragraph indexes directly in a single pass.  
    \item \textbf{Answers clarifying questions} by integrating highlighted context into consideration to build a more accurate and contextually grounded understanding of the long document.
    \item \textbf{Answers the original question} by combining all newly gathered clarifications, the model attempts to generate a valid answer to the original question.
\end{itemize}

It is important to note a key distinction between \coc path generation during data collection and the actual task deployment of the agentic workflow. In the data generation phase, we prompt the LLM to iteratively process each chunk of input text along with its self-generated clarifying questions, ensuring accurate retrieval of relevant context. 
% However, this approach is computationally expensive, as each contextual grounding step involves hundreds of inference calls. To mitigate this cost, we limit prompting to the data collection stage, where we prioritize obtaining high-quality training data.
During training, rather than relying on repeated inference calls, we finetune the model to directly generate the indexes of relevant paragraphs using pointback examples, effectively amortizing the computational cost into training. This enables the model to internalize the retrieval process, allowing it to dynamically synthesize relevant clarifications and contextual references at inference time without requiring extensive additional prompting.


% We generate long-context reasoning traces as learning targets using inference-time scaling. For each question, multiple traces are constructed in a tree structure, where the most effective trace is retained for later SFT and DPO. 
% Meanwhile, failed traces serve as negative examples in DPO to refine the model’s reasoning quality.

% Our framework leverages an inference-time scaling process to generate detailed reasoning traces that serve as learning targets for subsequent SFT and DPO. During inference, the model generates multiple candidate traces for a given question, structured in a tree-like formation where each branch represents a distinct sequence of clarifications, pointbacks, and reasoning refinements. This multiplicity of traces allows the model to explore diverse paths of inference, capturing a rich variety of reasoning patterns over extended contexts.

% From this tree of candidate traces, a selection mechanism identifies the most coherent and contextually grounded trace based on the correctness of the final answer. 
% Only the best-performing trace is retained for later SFT, ensuring that the fine-tuning process is guided by high-quality examples that exemplify effective long-context comprehension.

% In the data collection phase, this process is executed over a wide array of long documents and question prompts, thereby accumulating a robust dataset of refined reasoning traces. 
% These traces are subsequently used to fine-tune the model in an SFT framework, where the objective is to align the model's output with the high-quality reasoning patterns observed during inference. The training objective encourages the model to internalize the iterative process of generating clarifying questions, executing targeted pointbacks, and refining its internal state—thereby enhancing its ability to handle extended contexts in a single forward pass during deployment.

% This integration of inference-time scaling with SFT not only augments the effective context utilization but also creates a feedback loop that continuously improves the model’s performance on long-context tasks. By leveraging dynamically generated traces as learning targets, our approach offers a scalable pathway to bridge the gap between nominal context size and the effective reasoning capabilities required for complex, long-form documents.


\section{Data Generation \& Model Training}

\paragraph{Dataset} 
We use the NarrativeQA~\cite{kocisky-etal-2018-narrativeqa} dataset to facilitate long-context QA and generate agentic workflow traces with 14.7K QA pairs in the training set. NarrativeQA is designed for reading comprehension over narrative texts, such as books and movie scripts, where each example includes a full story and a set of corresponding QA pairs. This dataset emphasizes deeper reasoning and long-context understanding, as many questions require synthesizing information from multiple parts of the narrative rather than focusing solely on particular local context. Its relatively long passages make NarrativeQA particularly suitable for testing and refining agentic reasoning in large language models, as the answers often depend on weaving together details spanning the entire text.

\paragraph{Base Model} Our base model is \textit{Llama3.1-8B-Instruct}~\cite{dubey2024llama3}, an 8-billion-parameter instruction-tuned Llama model. This model is built on the same transformer architecture as Llama3, but with additional fine-tuning data to improve its performance on multi-turn dialogue and instruction-following tasks.

\subsection{\coc Path Construction}
We employ a test-time scaling approach to generate \coc paths. For each question, we construct a tree of search paths where each node represents a distinct clarification question posed by the LLM.

In our experiments, we use a branching factor of 8 at each depth and select the most promising trace based on an evaluation score that combines:
\begin{itemize}
    \item \textbf{Semantic similarity}, measured by the RougeL~\cite{lin-2004-rouge} score relative to the ground truth.
    \item \textbf{Discrete correctness}, evaluated by a binary verification using GPT4o-mini.
\end{itemize}

In the data construction process, the relevant context is found by iteratively querying the LLM about the relevance of all chunked passages. Here we use 512 as the chunk size. This process is compute-intensive but only happens in data collection. 
After the training, the LLM will directly generate the paragraph numbers of the relevant context as shown in the lower right of~\cref{fig:pipeline}.

% \begin{figure}[t!]
    \begin{center}
    \includegraphics[width=0.9\columnwidth , keepaspectratio]{img/datastat.pdf}
    \end{center}
    \caption{\textbf{Most questions can be answered with up to three rounds of clarifications.} 
    A single round of clarification suffices for the majority of questions (92\%), with a smaller fraction requiring two or three additional clarifications. 
    We cap inference-time clarifications at three rounds to strike a balance between coverage (solving 97.4\% of questions) and computational cost.}
    \label{fig:dataset_stat}
\end{figure}

For most long-context tasks, a single clarification question suffices because the required reasoning is not highly complex. 92\% of the questions in our experiments are resolved correctly with just one round of clarification. More challenging tasks may require multiple rounds of clarification: two rounds resolve 53\% of the remaining 8\%, and three rounds resolve 35\% of the remaining 4\%. Because of the exponentially increasing cost—and given that 97.4\% of the training questions are already solved—we limit the maximum depth of our inference scaling to 3.

The statistics of the collected dataset are shown in~\cref{tab:narrativeqa_stats}. The total number of conditional generation tokens that the LLM trained on is 17M tokens, with input that has an average length of 67K and a max length of 128K tokens.

\begin{table*}[t]
    \centering
    \footnotesize
    \renewcommand{\arraystretch}{1.2} % Adjust row spacing
    \resizebox{\textwidth}{!}{%
    \begin{tabular}{l l c l c}
        \toprule
        \textbf{Dataset} & \textbf{Stimuli Type} & \textbf{Channels/Electrodes} & \textbf{Stimuli Details} & \textbf{Subjects} \\
        \midrule
        Zuco 1.0 \cite{hollenstein2018zuco} & Text & 128 & Sentences from Stanford Sentiment Treebank, Wikipedia corpus & 12 \\
        Zuco 2.0 \cite{hollenstein2019zuco} & Text & 128 & Expanded subjects with similar content as Zuco 1.0 & 18 \\
        Alice \cite{bhattasali2020alice} & Text & 61 + 1 ground & 2,129 words, 84 sentences from \textit{Alice in Wonderland} & 52 \\
        Envisioned Speech \cite{kumar2018envisioned} & Imagined Speech & 14 & 20 text stimuli (digits, characters), 10 objects & 23 \\
        Alljoined \cite{xu2024alljoined} & Image & 64 & 10,000 images per participant from 80 MS-COCO categories & 8 \\
        ImageNet EEG \cite{spampinato2017deep} & Image & 128 & 40 ImageNet classes, 50 images/class, 2000 total & 6 \\
        DM-RE2I \cite{zeng2023dcae} & Image & 32 & 200 ImageNet images across 26 subjects & 26 \\
        Texture Perception \cite{orima2021analysis} & Image & 19 & 166 grayscale natural texture images & 15 \\
        THINGS-EEG \cite{grootswagers2022human} & Image & 64 & 22,248 images across 1854 object concepts & 50 \\
        DCAE \cite{zeng2023dcae} & Image & 32 & 200 ImageNet images (cats, dogs, flowers, pandas) & 26 \\
        ThoughtViz \cite{tirupattur2018thoughtviz} & Imagined Objects & 14 & EEG recorded while participants imagined digits, characters, and objects & 23 \\
        OCED \cite{kaneshiro2015representational} & Image & 128 & 12 images per 6 object categories & 10 \\
        NMED-T \cite{losorelli2017nmed} & Music & 128 & 10 songs (4:30-5:00 mins) with tempos 56-150 BPM & 20 \\
        NMED-H \cite{kaneshiro2016naturalistic} & Music & 125 & 4 versions of 4 songs, total 16 stimuli & 48 \\
        KARA ONE \cite{zhao2015classifying} & Text, Audio, Speech & 64 & Rest state, stimulus, imagined speech, speaking task & 12 \\
        Japanese Speech EEG \cite{mizuno2024investigation} & Audio & 64 & 503 spoken sentences (male/female speaker) & 1 \\
        Phrase/Word Speech EEG \cite{park2024towards} & Audio & 64 & Audio of 13 words/phrases, followed by speech replication & 10 \\
        \bottomrule
    \end{tabular}%
    }
    \caption{EEG-Based Datasets from Surveyed Studies with Text, Image and Audio/Speech/Music Stimuli}
    \label{tab:datasets}
\end{table*}


\subsection{\coc Path Distillation}
We employ a two-stage finetuning recipe: Supervised Fine-Tuning (SFT) followed by Direct Preference Optimization (DPO)~\cite{rafailov2024direct}, to convert our base model into a long-context understanding agent.
The dataset statistics is described in~\cref{tab:narrativeqa_stats}, with input length up to 128K tokens. 

\paragraph{Supervised Fine-Tuning} In the first phase, we finetune \textit{Llama3.1-8B-Instruct} using the generated \coc paths. Each training example includes (1) the full context from NarrativeQA, (2) the question, and (3) the step-by-step reasoning trace leading to the final answer. 
By exposing the model to these traces, we encourage it to internalize multi-step reasoning strategies and context grounding for the long-context inputs. 
The SFT stage uses a standard cross-entropy loss on the next-token prediction task, ensuring the model learns how to produce consistent and complete reasoning sequences.

\paragraph{Direct Preference Optimization} 
In the second phase, we apply Direct Preference Optimization to further refine the model’s output quality. 
To create preference pairs, we sample incorrect workflow traces as negative examples with using GPT4o-mini as the judge for answer correctness from the test-time scaling. 
DPO explicitly optimizes the model to generate higher-ranked responses more frequently, thus aligning the agent’s outputs with desirable characteristics, such as clarity, correctness, and coherence. This stage ensures that even among valid reasoning paths, the model learns to prioritize the most instructive reasoning.

The details for the two-phase training are listed in~\cref{asec:hyperparameters}.

\section{Evaluation}
\label{sec:evaluation}

\begin{table*}[ht]
  \centering
  \caption{Performance difference of \method and its base, Llama3.1-8B-Instruct ($\delta=$\method-8B minus Llama3.1-8B), on long context (the 128K tasks) and short-context benchmarks (6 regular tasks including ARC, GSM8K, and MMLU), the details of the short-context performance can be found in~\cref{asec:short_context}. Scores represent accuracy, with \method demonstrating significantly improved performance across long-context tasks with minimal effect on regular task performance.}
  \label{tab:performance-long-benchmarks}
  \resizebox{\textwidth}{!}{
\begin{tabular}{l|r|rrrrrrrr}
  \toprule
  \textbf{Model} & \textbf{Short Avg} & HotpotQA & Natural Questions & TriviaQA & PopQA & NarrativeQA & InfiniQA & InfiniChoice & \textbf{Long Avg}  \\
  \midrule
  Llama3.1-8B  & \textbf{62.3}  & 40.0   & 56.1    & 80.6    & 56.1   & 38.0    & 48.0   & 55.0     & \textbf{53.4}  \\
 AgenticLU ($\delta$)    & \textbf{-0.6} & +31.1 & +21.7 & +7.7 & +9.4 & +18.0  & +2.0 & +13.0  & \color{red}{\textbf{+14.7}} \\
  \bottomrule
\end{tabular}
}
\end{table*}

\begin{figure*}[t!]
    \begin{center}
    \includegraphics[width=\linewidth , keepaspectratio]{img/rag_result.pdf}
    \end{center}
    % \caption{\textbf{Long-context performance across context lengths and tasks.} Results showing QA accuracy across seven tasks (four RAG tasks: HotpotQA, Natural Questions, TriviaQA, PopQA; three LongQA tasks: NarrativeQA, InfbenchQA, InfbenchChoice) for varying input lengths from 8K to 128K tokens. 
    % We compare the finetuned \method-8B against the base Llama3.1-8B model and baselines including prompting methods (Step-by-Step, Plan-and-Solve, Fact-and-Reflect, LongRAG) and ProLong-8B, a long-context model continued pretrained from Llama3-8B with additional 40B tokens. 
    % \method-8B consistently maintains strong performance across most tasks and context lengths.}
    \caption{\textbf{Main results on 7 long-context tasks across context lengths from 8K to 128K.} Our \method-8B (dotted orange) achieves significant improvements on \emph{all} tasks over our base model Llama3.1-8B (solid orange). We also compare with the prompting methods (Step-by-Step, Plan-and-Solve, Fact-and-Reflect, LongRAG) and the state-of-the-art ProLong-8B model. \method-8B consistently maintains strong performance across most tasks and context lengths.}
    \label{fig:rag_result}
\end{figure*}


% \begin{table*}[ht]
  \centering
  \caption{Performance on infbench\_qa, infbench\_choice, and narrativeqa Across Context Lengths}
  \label{tab:prompting_table}
  \resizebox{\textwidth}{!}{
  \begin{tabular}{l *{15}{c}}
    \toprule
    & \multicolumn{5}{c}{infbench\_qa} & \multicolumn{5}{c}{infbench\_choice} & \multicolumn{5}{c}{narrativeqa} \\
    \cmidrule(lr){2-6} \cmidrule(lr){7-11} \cmidrule(lr){12-16}
    Model            & 8K   & 16K  & 32K  & 64K  & 128K & 8K   & 16K  & 32K  & 64K  & 128K & 8K   & 16K  & 32K  & 64K  & 128K \\
    \midrule
    Llama3.1 8B      & 17 & 31 & 36 & 40 & 48 & 9  & 12 & 24 & 39 & 55 & 15 & 19 & 27 & 35 & 38 \\
    -StepbyStep      & 21 & 36 & 36 & 45 & 43 & 15 & 13 & 41 & 41 & 44 & 23 & 30 & 36 & 51 & 43 \\
    -Plan\&Solve     & 17 & 26 & 32 & 41 & 40 & 27 & 15 & 48 & 55 & 58 & 22 & 25 & 38 & 41 & 39 \\
    -Fact\&Reflect   & 19 & 30 & 40 & 42 & 37 & 20 & 14 & 38 & 51 & 56 & 21 & 35 & 37 & 42 & 46 \\
    LongRAG          & --   & --   & --   & --   & --   & --   & --   & --   & --   & --   & --   & --   & --   & --   & --   \\
    AgenticLU        & --   & --   & --   & --   & 62 & --   & --   & --   & --   & 67 & --   & --   & --   & --   & 57 \\
    \bottomrule
  \end{tabular}
  }
\end{table*}
% \input{tab/rag_table}
% \begin{table*}[!tbh]
\centering
\resizebox{\textwidth}{!}{%
\begin{tabular}{clcccccccccc}
\toprule
\multicolumn{1}{l}{} &  & \multicolumn{3}{c}{\textbf{CLINC}} & \multicolumn{3}{c}{\textbf{BANKING}} & \multicolumn{3}{c}{\textbf{StackOverflow}} & \multicolumn{1}{l}{} \\ \midrule
\multicolumn{1}{c|}{\textbf{KCR}} & \multicolumn{1}{l|}{\textbf{Methods}} & \textbf{ACC} & \textbf{ARI} & \multicolumn{1}{c|}{\textbf{NMI}} & \textbf{ACC} & \textbf{ARI} & \multicolumn{1}{c|}{\textbf{NMI}} & \textbf{ACC} & \textbf{ARI} & \multicolumn{1}{c|}{\textbf{NMI}} & \textbf{Average} \\ \midrule
\multicolumn{1}{c|}{} & \multicolumn{1}{l|}{GCD (CVPR 2022)} & 83.29 & 76.77 & \multicolumn{1}{c|}{93.22} & 21.17 & 9.35 & \multicolumn{1}{c|}{43.41} & 17.00 & 3.42 & \multicolumn{1}{c|}{14.57} & 40.24 \\
\multicolumn{1}{c|}{} & \multicolumn{1}{l|}{SimGCD (ICCV 2023)} & 83.24 & 75.89 & \multicolumn{1}{c|}{92.79} & 25.62 & 12.67 & \multicolumn{1}{c|}{47.46} & 18.50 & 6.49 & \multicolumn{1}{c|}{17.91} & 42.29 \\
\multicolumn{1}{c|}{} & \multicolumn{1}{l|}{Loop (ACL 2024)} & 84.89 & 77.43 & \multicolumn{1}{c|}{93.26} & 21.56 & 10.24 & \multicolumn{1}{c|}{44.77} & 18.80 & 5.76 & \multicolumn{1}{c|}{17.54} & 41.58 \\
\multicolumn{1}{c|}{\multirow{-4}{*}{5\%}} & \multicolumn{1}{l|}{\cellcolor{blue!18}\textbf{\MethodName (Ours)}} & \cellcolor{blue!18}\textbf{88.18} & \cellcolor{blue!18}\textbf{82.40} & \multicolumn{1}{c|}{\cellcolor{blue!18}\textbf{94.94}} & \cellcolor{blue!18}\textbf{30.94} & \cellcolor{blue!18}\textbf{18.32} & \multicolumn{1}{c|}{\cellcolor{blue!18}\textbf{54.05}} & \cellcolor{blue!18}\textbf{22.30} & \cellcolor{blue!18}\textbf{8.32} & \multicolumn{1}{c|}{\cellcolor{blue!18}\textbf{21.25}} & \cellcolor{blue!18}\textbf{46.74} \\ \midrule
\multicolumn{1}{c|}{} & \multicolumn{1}{l|}{GCD (CVPR 2022)} & 82.04 & 75.95 & \multicolumn{1}{c|}{93.33} & 59.09 & 46.34 & \multicolumn{1}{c|}{76.22} & 75.40 & 56.01 & \multicolumn{1}{c|}{72.66} & 70.78 \\
\multicolumn{1}{c|}{} & \multicolumn{1}{l|}{SimGCD (ICCV 2023)} & 84.71 & 77.08 & \multicolumn{1}{c|}{93.27} & 60.03 & 47.80 & \multicolumn{1}{c|}{76.53} & 77.10 & 57.70 & \multicolumn{1}{c|}{72.30} & 71.84 \\
\multicolumn{1}{c|}{} & \multicolumn{1}{l|}{Loop (ACL 2024)} & 84.89 & 78.12 & \multicolumn{1}{c|}{93.52} & 64.97 & 53.05 & \multicolumn{1}{c|}{79.14} & 80.50 & \textbf{62.97} & \multicolumn{1}{c|}{75.98} & 74.79 \\
\multicolumn{1}{c|}{\multirow{-4}{*}{10\%}} & \multicolumn{1}{l|}{\cellcolor{blue!18}\textbf{\MethodName (Ours)}} & \cellcolor{blue!18}\textbf{88.71} & \cellcolor{blue!18}\textbf{83.29} & \multicolumn{1}{c|}{\cellcolor{blue!18}\textbf{95.21}} & \cellcolor{blue!18}\textbf{67.99} & \cellcolor{blue!18}\textbf{57.30} & \multicolumn{1}{c|}{\cellcolor{blue!18}\textbf{82.23}} & \cellcolor{blue!18}\textbf{82.40} & \cellcolor{blue!18}62.81 & \multicolumn{1}{c|}{\cellcolor{blue!18}\textbf{79.67}} & \cellcolor{blue!18}\textbf{77.73} \\ 
\midrule
\multicolumn{1}{c|}{} & \multicolumn{1}{l|}{DeepAligned (AAAI 2021)} & 74.07 & 64.63 & \multicolumn{1}{c|}{88.97} & 49.08 & 37.62 & \multicolumn{1}{c|}{70.50} & 54.50 & 37.96 & \multicolumn{1}{c|}{50.86} & 58.69 \\
\multicolumn{1}{c|}{} & \multicolumn{1}{l|}{MTP-CLNN (ACL 2022)} & 83.26 & 76.20 & \multicolumn{1}{c|}{93.17} & 65.06 & 52.91 & \multicolumn{1}{c|}{80.04} & 74.70 & 54.80 & \multicolumn{1}{c|}{73.35} & 72.61 \\
\multicolumn{1}{c|}{} & \multicolumn{1}{l|}{GCD (CVPR 2022)} & 82.31 & 75.45 & \multicolumn{1}{c|}{92.94} & 69.64 & 58.30 & \multicolumn{1}{c|}{82.17} & 81.60 & 65.90 & \multicolumn{1}{c|}{78.76} & 76.34 \\
\multicolumn{1}{c|}{} & \multicolumn{1}{l|}{ProbNID (ACL 2023)} & 71.56 & 63.25 & \multicolumn{1}{c|}{89.21} & 55.75 & 44.25 & \multicolumn{1}{c|}{74.37} & 54.10 & 38.10 & \multicolumn{1}{c|}{53.70} & 60.48 \\
\multicolumn{1}{c|}{} & \multicolumn{1}{l|}{USNID (TKDE 2023)} & 83.12 & 77.95 & \multicolumn{1}{c|}{94.17} & 65.85 & 56.53 & \multicolumn{1}{c|}{81.94} & 75.76 & 65.45 & \multicolumn{1}{c|}{74.91} & 75.08 \\
\multicolumn{1}{c|}{} & \multicolumn{1}{l|}{SimGCD (ICCV 2023)} & 84.44 & 77.53 & \multicolumn{1}{c|}{93.44} & 69.55 & 57.86 & \multicolumn{1}{c|}{81.71} & 79.80 & 65.19 & \multicolumn{1}{c|}{79.09} & 76.51 \\
\multicolumn{1}{c|}{} & \multicolumn{1}{l|}{CsePL (EMNLP 2023)} & 86.16 & 79.65 & \multicolumn{1}{c|}{94.07} & 71.06 & 60.36 & \multicolumn{1}{c|}{83.22} & 79.47 & 64.92 & \multicolumn{1}{c|}{74.88} & 77.09 \\
\multicolumn{1}{c|}{} & \multicolumn{1}{l|}{ALUP (NAACL 2024)} & 88.40 & 82.44 & \multicolumn{1}{c|}{94.84} & 74.61 & 62.64 & \multicolumn{1}{c|}{84.06} & 82.20 & 64.54 & \multicolumn{1}{c|}{76.58} & 78.92 \\
\multicolumn{1}{c|}{} & \multicolumn{1}{l|}{Loop (ACL 2024)} & 86.58 & 80.67 & \multicolumn{1}{c|}{94.38} & 71.40 & 60.95 & \multicolumn{1}{c|}{83.37} & 82.20 & 66.29 & \multicolumn{1}{c|}{79.10} & 78.33 \\
\multicolumn{1}{c|}{\multirow{-10}{*}{25\%}} & \multicolumn{1}{l|}{\cellcolor{blue!18}\textbf{\MethodName (Ours)}} & \cellcolor{blue!18}\textbf{91.51} & \cellcolor{blue!18}\textbf{87.07} & \multicolumn{1}{c|}{\cellcolor{blue!18}\textbf{96.27}} & \cellcolor{blue!18}\textbf{76.98} & \cellcolor{blue!18}\textbf{66.00} & \multicolumn{1}{c|}{\cellcolor{blue!18}\textbf{85.62}} & \cellcolor{blue!18}\textbf{84.10} & \cellcolor{blue!18}\textbf{71.01} & \multicolumn{1}{c|}{\cellcolor{blue!18}\textbf{80.90}} & \cellcolor{blue!18}\textbf{82.16} \\ 

\midrule

\multicolumn{1}{c|}{} & \multicolumn{1}{l|}{DeepAligned (AAAI 2021)} & 80.70 & 72.56 & \multicolumn{1}{c|}{91.59} & 59.38 & 47.95 & \multicolumn{1}{c|}{76.67} & 74.52 & 57.62 & \multicolumn{1}{c|}{68.28} & 69.92 \\
\multicolumn{1}{c|}{} & \multicolumn{1}{l|}{MTP-CLNN (ACL 2022)} & 86.18 & 80.17 & \multicolumn{1}{c|}{94.30} & 70.97 & 60.17 & \multicolumn{1}{c|}{83.42} & 80.36 & 62.24 & \multicolumn{1}{c|}{76.66} & 77.16 \\
\multicolumn{1}{c|}{} & \multicolumn{1}{l|}{GCD (CVPR 2022)} & 86.53 & 81.06 & \multicolumn{1}{c|}{94.60} & 74.42 & 63.83 & \multicolumn{1}{c|}{84.84} & 85.60 & 72.20 & \multicolumn{1}{c|}{80.12} & 80.36 \\
\multicolumn{1}{c|}{} & \multicolumn{1}{l|}{ProbNID (ACL 2023)} & 82.62 & 75.27 & \multicolumn{1}{c|}{92.72} & 63.02 & 50.42 & \multicolumn{1}{c|}{77.95} & 73.20 & 62.46 & \multicolumn{1}{c|}{74.54} & 72.47 \\
\multicolumn{1}{c|}{} & \multicolumn{1}{l|}{USNID (TKDE 2023)} & 87.22 & 82.87 & \multicolumn{1}{c|}{95.45} & 73.27 & 63.77 & \multicolumn{1}{c|}{85.05} & 82.06 & 71.63 & \multicolumn{1}{c|}{78.77} & 80.01 \\
\multicolumn{1}{c|}{} & \multicolumn{1}{l|}{SimGCD (ICCV 2023)} & 87.24 & 81.65 & \multicolumn{1}{c|}{94.83} & 74.42 & 64.17 & \multicolumn{1}{c|}{85.08} & 82.00 & 70.67 & \multicolumn{1}{c|}{80.44} & 80.06 \\
\multicolumn{1}{c|}{} & \multicolumn{1}{l|}{CsePL (EMNLP 2023)} & 88.66 & 83.14 & \multicolumn{1}{c|}{95.09} & 76.94 & 66.66 & \multicolumn{1}{c|}{85.65} & 85.68 & 71.99 & \multicolumn{1}{c|}{80.28} & 81.57 \\
\multicolumn{1}{c|}{} & \multicolumn{1}{l|}{ALUP (NAACL 2024)} & 90.53 & 84.84 & \multicolumn{1}{c|}{95.97} & 79.45 & 68.78 & \multicolumn{1}{c|}{86.79} & 86.70 & 73.85 & \multicolumn{1}{c|}{81.45} & 83.15 \\
\multicolumn{1}{c|}{} & \multicolumn{1}{l|}{Loop (ACL 2024)} & 90.98 & 85.15 & \multicolumn{1}{c|}{95.59} & 75.06 & 65.70 & \multicolumn{1}{c|}{85.43} & 85.90 & 72.45 & \multicolumn{1}{c|}{80.56} & 81.87 \\
\multicolumn{1}{c|}{\multirow{-10}{*}{50\%}} & \multicolumn{1}{l|}{\cellcolor{blue!18}\textbf{\MethodName (Ours)}} & \cellcolor{blue!18}\textbf{94.53} & \cellcolor{blue!18}\textbf{90.79} & \multicolumn{1}{c|}{\cellcolor{blue!18}\textbf{97.12}} & \cellcolor{blue!18}\textbf{80.26} & \cellcolor{blue!18}\textbf{70.40} & \multicolumn{1}{c|}{\cellcolor{blue!18}\textbf{87.65}} & \cellcolor{blue!18}\textbf{89.40} & \cellcolor{blue!18}\textbf{78.92} & \multicolumn{1}{c|}{\cellcolor{blue!18}\textbf{85.04}} & \cellcolor{blue!18}\textbf{86.01} \\ 

\bottomrule
\end{tabular}%
}
% \caption{Main results.}
\caption{Main results of \MethodName compared to baseline methods across different datasets and known category ratios (KCR). \MethodName outperforms both standard GCD approaches and the latest LLM-based work Loop \cite{an-etal-2024-generalized}, showing significant improvements especially on the challenging BANKING dataset and with limited known categories. Performance gains are observed across most KCRs, metrics, and datasets.}
\label{tab:main_result}
\end{table*}



In this section, we assess our method~\method using a suite of evaluation tasks drawn from the HELMET long-context benchmark~\cite{yen2024helmet}. Our experiments focus on testing models’ ability to retain, process, and reason over extended contexts ranging from 8K to 128K tokens.

\subsection{Tasks and Metrics}
We evaluate our models and baselines on the Helmet~\cite{yen2024helmet} long-context evaluation benchmark's retrieval-augmented generation (RAG) and long-range QA (LongQA) tasks ranging from 8K, 16K, 32K, 64K, to 128K.

We use GPT-4o as the judge for answer correctness, with the prompt template shown in~\cref{asec:prompt_template}. 
We report accuracies for all datasets.


The RAG test suite includes: 
(1) \textbf{HotpotQA}~\cite{yang-etal-2018-hotpotqa}, a multi-hop reasoning dataset over Wikipedia; 
(2) \textbf{Natural Questions}~\cite{kwiatkowski2019natural}, real user queries with Wikipedia-based short and long answers; 
(3) \textbf{TriviaQA}~\cite{JoshiTriviaQA2017}, a large-scale trivia dataset with question-answer pairs linked to evidence documents; 
(4) \textbf{PopQA}~\cite{mallen2023llm_memorization}, a dataset testing model memorization with fact-based questions from popular culture.

The LongQA test suite includes: 
(1) \textbf{NarrativeQA}~\cite{kocisky-etal-2018-narrativeqa}, a reading comprehension dataset with Wikipedia summaries and story-based Q\&A; 
(2) \textbf{InfiniteBench QA}~\cite{zhang-etal-2024-bench}, a long-range QA benchmark requiring reasoning over extended contexts; 
(3) \textbf{InfiniteBench Multiple-Choice}~\cite{zhang-etal-2024-bench}, a multiple-choice variant of the previous evaluating reading comprehension over long documents.

For the four RAG tasks, each question is put alongside a set of relevant contexts, and the overall input length is increased by appending irrelevant context. Consequently, these tasks become strictly more difficult as the context window expands. 
In contrast, for the three LongQA tasks, the relevant context may not appear in the truncated input (the first 8K, 16K, or 128K tokens). Hence, performance might improve at longer input lengths simply because the necessary information becomes available only after including more tokens.

%Given these differences in data construction, we use the four RAG datasets as our primary evaluation benchmark and report results on the three LongQA benchmarks only at the largest input size.
% \subsection{Baselines}
% We compare~\method against a diverse set of strong baselines that represent different approaches to handling long-context tasks.

\subsection{Baselines} 

We compare~\method against a diverse set of strong baselines representing different approaches for handling long-context tasks. Our comparisons include two main categories. 

Under prompting methods we consider techniques that require no additional model training. In particular, we evaluate (a) the chain-of-thought approach~\cite{kojima2022large}, which encourages models to decompose complex questions into intermediate reasoning steps; (b) fact-and-reflection prompting~\cite{zhao-etal-2024-fact}, which iteratively verifies and refines factual claims to enhance consistency; (c) plan-and-solve prompting~\cite{wang2023plan}, where the model first outlines a high-level plan before sequentially executing it to address structured reasoning tasks; and (d) LongRAG~\cite{zhao-etal-2024-longrag} where a hybrid RAG system is used to retrieve relevant context to generate global summaries and local details~\footnote{Note that LongRAG provided finetuned models as well. But the SFT-ed Llama3-8B only supports 8K context length. Thus we did not include it in our comparison.}. 

In the fine-tuning category, we focus on models that have been specifically adapted for extended context data. For a substantial comparison, we employ Prolong-8B-512K~\cite{prolong}—a model based on the Llama3 8B architecture that has been further trained on an additional 40B tokens of long-context data.
%and we also include recent long-range models such as Qwen2.5-14B-1M~\cite{yang2025qwen2}, which support context lengths up to 1M tokens with a larger and stronger base model.


\subsection{Main Results}

The performance of \method and baseline models is shown in~\cref{fig:rag_result}. 

%Our analysis highlights the strengths of \method-8B, our proposed method, which leverages a \textit{self-clarification} mechanism to enhance long-context reasoning.

\paragraph{Self-clarification significantly improves multi-hop reasoning.} \method-8B consistently surpasses other methods in HotpotQA. By iteratively refining its understanding, resolving ambiguities, and verifying intermediate steps, the model achieves higher accuracy, particularly as context length increases.

\paragraph{Robust performance across diverse datasets.} Unlike baseline models, \method-8B maintains consistently strong performance across RAG and LongQA benchmarks, demonstrating its ability to adapt effectively to different long-context tasks.

\paragraph{Reduced performance degradation with longer contexts.} While most models experience significant accuracy drops as context length increases, \method-8B remains stable. Its self-clarification and pointback mechanisms effectively filter noise from irrelevant information, allowing the model to extract and prioritize essential evidence.


\paragraph{Fine-tuning vs. prompting trade-offs.} While structured prompting techniques like \textit{plan-and-solve} improve short-context reasoning, they struggle with extreme context lengths (e.g., 128K tokens). In contrast, \method-8B, through targeted finetuning with self-clarification and pointback, maintains robust long-context reasoning without relying on complex prompting strategies. Although ProLong-8B, another finetuned model, achieves strong results, it comes with significantly higher training costs. \method-8B, by contrast, is more data-efficient and generalizes better to novel tasks, making it a more practical and effective solution for long-context reasoning.

Overall, these results underscore the effectiveness of \method-8B in tackling long-context understanding challenges. The integration of self-clarification plays a crucial role in improving grounding, reasoning, and comprehension in long-context settings.


\subsection{Performance on Short-Context Tasks}
To demonstrate that our fine-tuning process preserves the model's general capabilities while enhancing long-context understanding, we evaluated the finetuned model on a diverse set of standard benchmarks. These include elementary and advanced reasoning tasks ARC Easy and ARC Challenge~\cite{Clark2018ThinkYH}, mathematical problem-solving GSM8K~\cite{cobbe2021training}, MathQA~\cite{amini2019mathqa}, and broad knowledge assessment MMLU~\cite{hendryckstest2021, hendrycks2021ethics}, MMLU-Pro~\cite{wang2024mmlupro}.

We report the average performance across short-context tasks in~\cref{tab:performance-long-benchmarks}, and each individual task result can be found in~\cref{asec:short_context}. We find that the short-context performance is well preserved, demonstrating that \method's core reasoning and problem-solving abilities remain strong and are not compromised by the significant improvements to its long-context understanding powers.

% The results are shown in~\cref{tab:performance-benchmarks}. % Briefly explain the benchmarks here
% By performing well on these short-context tasks, the finetuned model demonstrates that its core reasoning and problem-solving abilities remain strong and are not compromised by improvements to its long-context understanding powers.



\section{Analyses \& Ablation Studies}
\label{sec:analysis}
In this section, we take a closer look at how each part of our approach affects long-context understanding and retrieval. Specifically, we study three main questions: (1) Can the finetuned system benefit from multi-round \coc? (2) Does adding clarifications and pointing back to the original document help the model understand and utilize the context more accurately? (3) How much additional compute overhead does \method add to the process?

\begin{table}
\centering
\caption{We evaluate the performance of adding additional self-clarification and contextual grounding rounds at inference time. The gain from self-clarification is close to optimal at the initial round.}
\resizebox{\columnwidth}{!}{
\begin{tabular}{lccccc}
\toprule
\textbf{Model} & \textbf{HotpotQA} & \textbf{NaturalQ} & \textbf{PopQA} & \textbf{TriviaQA} & \textbf{Avg} \\
\midrule
Llama-3.1-8B         & 40.0  & 56.1  & 56.1  & 80.6  & 58.2 \\
\method-8B           & 71.1  & 77.8  & 65.5  & 88.3  & 75.7 \\
\quad (w/ 2 rounds)  & 71.1  & 76.7  & 67.2  & 91.7  & 76.7 \\
\quad (w/ 3 rounds)  & 75.5  & 78.8  & 68.3  & 91.1  & 78.4 \\
\bottomrule
\end{tabular}
}
\label{tab:multiround}
\end{table}

\subsection{How many rounds of \coc are needed?}
\paragraph{Setup.}
We add additional rounds of reasoning in the evaluation and see if the LLM can benefit from multi-rounds of reasoning at test-time.

\paragraph{Analysis.}
The results, presented in Table~\ref{tab:multiround}, indicate that additional rounds of agentic reasoning do provide performance improvements. 

This suggests that while significant benefits of self-clarification are achieved in the first round, additional rounds still contribute to further improvements. 
One possible explanation is the nature of our dataset: approximately 92\% of the questions are resolved within a single round of clarification. However, for the remaining cases, extended reasoning allows the model to refine its understanding, leading to measurable gains in performance with more clarification and reasoning.

% This suggests that most of the benefits of self-clarification are already realized in the first round.
% One potential explanation is the nature of our dataset: approximately 92\% of the questions are resolved within a single round of clarification. 
% As a result, the model may have adapted to favor shorter reasoning paths, making additional rounds of clarification redundant in most cases. 
% This highlights an important consideration for long-context modeling—while iterative refinement can be useful, its effectiveness is contingent on the complexity and structure of the dataset.


% Please add the following required packages to your document preamble:
% \usepackage{multirow}
% \begin{table*}[]
% \begin{tabular}{cccccccc}
% \hline
% \multirow{2}{*}{}
% & \multirow{2}{*}{NObj} 
% & \multirow{2}{*}{OBB over} 
% & \multirow{2}{*}{Out-of-Boundary} 
% & \multicolumn{2}{c}{Rel. Pos. Distribution}  \\
% & 
% &
% &                                 
% & desk-chair        
% & bed-nightstand        
% &                         
% \\ \hline
% Baseline
% &                       
% &                                   
% &                             
% &                                 
% &                    
% &                       
% &                         
% \\ \hline
% Hybrid Pos.                                                                         
% &                        
% &     
% &                              
% &                                  
% &                   
% &                       
% &                         
% \\ \hline
% Relative Pos. + Optim.                                                              &                        &                                    &                              &                                  &                    &                       &                         \\ \hline
% \begin{tabular}[c]{@{}c@{}}Relative Pos. + Retrieval + Optim.\\ (Full)\end{tabular} &                        &                                    &                              &                                  &                    &                       &                \\ \hline        
% \end{tabular}
% \caption{Ablation table.}
% \end{table*}



% Please add the following required packages to your document preamble:
% \usepackage{multirow}
\begin{table*}
\setlength{\tabcolsep}{10pt}
\centering

% \vspace{-0.2cm}
\begin{tabular}{cc|ccc|ccc}
\hline
\multicolumn{2}{c}{Baselines} & \multicolumn{3}{c}{Bedrooms} & \multicolumn{3}{c}{Living Rooms} \\

\hline
Hierarchy-Aware Net. & D$\&$C Optim. & {OOB↓} & {Overlap↓} & {KL Div.↓}  & {OOB↓} & {Overlap↓} & {KL Div.↓} \\
\hline

$\times$ & $\times$ &  0.80 & 0.09 & 0.23 & 0.98 & 0.10 & 0.24  \\
\hline

\checkmark & $\times$ &  0.73&  0.18&  0.09&  0.93&  0.21& 0.16  \\\hline

$\times$ & \checkmark & 0.00 & 0.00 & 0.23 & 0.00 & 0.00 & 0.23  \\\hline

\checkmark & \checkmark &  \textbf{0.00}& \textbf{0.00}& \textbf{0.09}&  \textbf{0.00}&  \textbf{0.00}& \textbf{0.13} \\ 
\hline

\end{tabular}
\caption{Quantitative evaluation of ablation study to validate our key designs: the hierarchy-aware network (Hierarchy-Aware Net.) to infer fine-grained relative placements and divide-and-conquer optimization (D$\&$C optim.) to solve the final layouts.}
\label{tab:ablation_study}
% \vspace{-0.2cm}
\end{table*}

\subsection{Do Self-Clarifications and Pointback Help in Long-Context Understanding?}
\paragraph{Setup.}
To evaluate the impact of each component in our agentic workflow, we compare the full \method-8B model against two variants: one without the self-clarification step and another without the contextual grounding (\emph{pointback}) step. 
We use the four RAG datasets with 128K context length as the evaluation benchmark, and compare the performance alongside the original model.

\paragraph{Analysis.}
Table~\ref{tab:ablation} shows the results on four QA benchmarks with a 128K context length. Removing self-clarification leads to an absolute performance drop of at least 10 points across most tasks (e.g., from 71.1\% to 57.8\% on HotpotQA), confirming that the model benefits from clarifying its own uncertainties when the context is long. Meanwhile, omitting pointback yields degenerate results, indicating that pinpointing relevant information at each stage is crucial for long-context QA. Overall, these findings highlight the importance of both clarifications and context-grounding to maximize retrieval accuracy and robustness in lengthy documents.

\subsection{How much additional compute cost does \method impose in generation?}
Since additional generation steps are introduced in the QA process, we assess the overhead in inference time.
Naïvely, long-context inference and multi-round conversations could significantly amplify compute costs. However, by leveraging prefix caching to store computed KV caches, the additional cost scales linearly with the number of newly generated tokens rather than exponentially.

To quantify this overhead, we conduct a runtime evaluation on 100 queries with a 128K context size. The results, summarized in~\cref{tab:runtime_overhead}, demonstrate that the additional computational overhead remains minimal when using prefix caching.
\begin{table}
  \centering
  \caption{Performance Overhead Comparison between direct answering baseline and \method.}
  \label{tab:runtime_overhead}
  \resizebox{\columnwidth}{!}{
  \begin{tabular}{lcc}
    \toprule
    \textbf{Metric} & \textbf{Baseline} & \textbf{\method} \\
    \midrule
    Runtime Overhead & 100\%   & 101.93\% \\
    Avg Tokens Generated in One Round  & 76.28   & 1205.38    \\
    \bottomrule
  \end{tabular}
}
% \vspace{-3ex}
\end{table}

\section{Conclusion}
\label{sec:conclusion}

In this work, we introduce Agentic Long-Context Understanding (\method), a framework designed to enhance large language models' ability to process and reason over long-context inputs with self-generated data. 
By incorporating an agentic workflow (\coc) that dynamically refines model reasoning through self-clarifications and contextual grounding, \method significantly improves LLM's long context understanding capabilities.

Through a combination of trace data collection and two-stage post-training, our approach enables models to autonomously explore multiple reasoning paths, distill the most effective clarification strategies, and improve their understanding of lengthy documents. 
Extensive evaluations on long-context benchmarks demonstrate that AgenticLU outperforms existing prompting techniques and finetuned baselines, maintaining strong performance across context lengths up to 128K tokens. 
Additionally, ablation studies confirm that self-clarification and pointback mechanisms play a crucial role in improving retrieval and reasoning over long-contexts.



\section*{Limitations}
Despite its effectiveness in long-context reasoning, \method has notable limitations. One key drawback is its inability to autonomously determine when to stop multi-round reasoning. While additional rounds of self-clarification can improve performance, the model follows a fixed number of reasoning steps rather than dynamically assessing when further refinement is necessary. This can lead to inefficiencies, where the model either stops too early, missing potential improvements, or continues reasoning unnecessarily, expending computational resources without significant gains.

Developing a fully agentic mechanism remains an open challenge. Ideally, the model should assess its confidence in an intermediate response and decide whether further clarification is needed. Future work should explore approaches that enable AgenticLU to regulate its reasoning depth dynamically, optimizing both efficiency and performance.

% Bibliography entries for the entire Anthology, followed by custom entries

\bibliography{anthology,custom}
% Custom bibliography entries only
%\bibliography{custom}


\appendix
\subsection{Lloyd-Max Algorithm}
\label{subsec:Lloyd-Max}
For a given quantization bitwidth $B$ and an operand $\bm{X}$, the Lloyd-Max algorithm finds $2^B$ quantization levels $\{\hat{x}_i\}_{i=1}^{2^B}$ such that quantizing $\bm{X}$ by rounding each scalar in $\bm{X}$ to the nearest quantization level minimizes the quantization MSE. 

The algorithm starts with an initial guess of quantization levels and then iteratively computes quantization thresholds $\{\tau_i\}_{i=1}^{2^B-1}$ and updates quantization levels $\{\hat{x}_i\}_{i=1}^{2^B}$. Specifically, at iteration $n$, thresholds are set to the midpoints of the previous iteration's levels:
\begin{align*}
    \tau_i^{(n)}=\frac{\hat{x}_i^{(n-1)}+\hat{x}_{i+1}^{(n-1)}}2 \text{ for } i=1\ldots 2^B-1
\end{align*}
Subsequently, the quantization levels are re-computed as conditional means of the data regions defined by the new thresholds:
\begin{align*}
    \hat{x}_i^{(n)}=\mathbb{E}\left[ \bm{X} \big| \bm{X}\in [\tau_{i-1}^{(n)},\tau_i^{(n)}] \right] \text{ for } i=1\ldots 2^B
\end{align*}
where to satisfy boundary conditions we have $\tau_0=-\infty$ and $\tau_{2^B}=\infty$. The algorithm iterates the above steps until convergence.

Figure \ref{fig:lm_quant} compares the quantization levels of a $7$-bit floating point (E3M3) quantizer (left) to a $7$-bit Lloyd-Max quantizer (right) when quantizing a layer of weights from the GPT3-126M model at a per-tensor granularity. As shown, the Lloyd-Max quantizer achieves substantially lower quantization MSE. Further, Table \ref{tab:FP7_vs_LM7} shows the superior perplexity achieved by Lloyd-Max quantizers for bitwidths of $7$, $6$ and $5$. The difference between the quantizers is clear at 5 bits, where per-tensor FP quantization incurs a drastic and unacceptable increase in perplexity, while Lloyd-Max quantization incurs a much smaller increase. Nevertheless, we note that even the optimal Lloyd-Max quantizer incurs a notable ($\sim 1.5$) increase in perplexity due to the coarse granularity of quantization. 

\begin{figure}[h]
  \centering
  \includegraphics[width=0.7\linewidth]{sections/figures/LM7_FP7.pdf}
  \caption{\small Quantization levels and the corresponding quantization MSE of Floating Point (left) vs Lloyd-Max (right) Quantizers for a layer of weights in the GPT3-126M model.}
  \label{fig:lm_quant}
\end{figure}

\begin{table}[h]\scriptsize
\begin{center}
\caption{\label{tab:FP7_vs_LM7} \small Comparing perplexity (lower is better) achieved by floating point quantizers and Lloyd-Max quantizers on a GPT3-126M model for the Wikitext-103 dataset.}
\begin{tabular}{c|cc|c}
\hline
 \multirow{2}{*}{\textbf{Bitwidth}} & \multicolumn{2}{|c|}{\textbf{Floating-Point Quantizer}} & \textbf{Lloyd-Max Quantizer} \\
 & Best Format & Wikitext-103 Perplexity & Wikitext-103 Perplexity \\
\hline
7 & E3M3 & 18.32 & 18.27 \\
6 & E3M2 & 19.07 & 18.51 \\
5 & E4M0 & 43.89 & 19.71 \\
\hline
\end{tabular}
\end{center}
\end{table}

\subsection{Proof of Local Optimality of LO-BCQ}
\label{subsec:lobcq_opt_proof}
For a given block $\bm{b}_j$, the quantization MSE during LO-BCQ can be empirically evaluated as $\frac{1}{L_b}\lVert \bm{b}_j- \bm{\hat{b}}_j\rVert^2_2$ where $\bm{\hat{b}}_j$ is computed from equation (\ref{eq:clustered_quantization_definition}) as $C_{f(\bm{b}_j)}(\bm{b}_j)$. Further, for a given block cluster $\mathcal{B}_i$, we compute the quantization MSE as $\frac{1}{|\mathcal{B}_{i}|}\sum_{\bm{b} \in \mathcal{B}_{i}} \frac{1}{L_b}\lVert \bm{b}- C_i^{(n)}(\bm{b})\rVert^2_2$. Therefore, at the end of iteration $n$, we evaluate the overall quantization MSE $J^{(n)}$ for a given operand $\bm{X}$ composed of $N_c$ block clusters as:
\begin{align*}
    \label{eq:mse_iter_n}
    J^{(n)} = \frac{1}{N_c} \sum_{i=1}^{N_c} \frac{1}{|\mathcal{B}_{i}^{(n)}|}\sum_{\bm{v} \in \mathcal{B}_{i}^{(n)}} \frac{1}{L_b}\lVert \bm{b}- B_i^{(n)}(\bm{b})\rVert^2_2
\end{align*}

At the end of iteration $n$, the codebooks are updated from $\mathcal{C}^{(n-1)}$ to $\mathcal{C}^{(n)}$. However, the mapping of a given vector $\bm{b}_j$ to quantizers $\mathcal{C}^{(n)}$ remains as  $f^{(n)}(\bm{b}_j)$. At the next iteration, during the vector clustering step, $f^{(n+1)}(\bm{b}_j)$ finds new mapping of $\bm{b}_j$ to updated codebooks $\mathcal{C}^{(n)}$ such that the quantization MSE over the candidate codebooks is minimized. Therefore, we obtain the following result for $\bm{b}_j$:
\begin{align*}
\frac{1}{L_b}\lVert \bm{b}_j - C_{f^{(n+1)}(\bm{b}_j)}^{(n)}(\bm{b}_j)\rVert^2_2 \le \frac{1}{L_b}\lVert \bm{b}_j - C_{f^{(n)}(\bm{b}_j)}^{(n)}(\bm{b}_j)\rVert^2_2
\end{align*}

That is, quantizing $\bm{b}_j$ at the end of the block clustering step of iteration $n+1$ results in lower quantization MSE compared to quantizing at the end of iteration $n$. Since this is true for all $\bm{b} \in \bm{X}$, we assert the following:
\begin{equation}
\begin{split}
\label{eq:mse_ineq_1}
    \tilde{J}^{(n+1)} &= \frac{1}{N_c} \sum_{i=1}^{N_c} \frac{1}{|\mathcal{B}_{i}^{(n+1)}|}\sum_{\bm{b} \in \mathcal{B}_{i}^{(n+1)}} \frac{1}{L_b}\lVert \bm{b} - C_i^{(n)}(b)\rVert^2_2 \le J^{(n)}
\end{split}
\end{equation}
where $\tilde{J}^{(n+1)}$ is the the quantization MSE after the vector clustering step at iteration $n+1$.

Next, during the codebook update step (\ref{eq:quantizers_update}) at iteration $n+1$, the per-cluster codebooks $\mathcal{C}^{(n)}$ are updated to $\mathcal{C}^{(n+1)}$ by invoking the Lloyd-Max algorithm \citep{Lloyd}. We know that for any given value distribution, the Lloyd-Max algorithm minimizes the quantization MSE. Therefore, for a given vector cluster $\mathcal{B}_i$ we obtain the following result:

\begin{equation}
    \frac{1}{|\mathcal{B}_{i}^{(n+1)}|}\sum_{\bm{b} \in \mathcal{B}_{i}^{(n+1)}} \frac{1}{L_b}\lVert \bm{b}- C_i^{(n+1)}(\bm{b})\rVert^2_2 \le \frac{1}{|\mathcal{B}_{i}^{(n+1)}|}\sum_{\bm{b} \in \mathcal{B}_{i}^{(n+1)}} \frac{1}{L_b}\lVert \bm{b}- C_i^{(n)}(\bm{b})\rVert^2_2
\end{equation}

The above equation states that quantizing the given block cluster $\mathcal{B}_i$ after updating the associated codebook from $C_i^{(n)}$ to $C_i^{(n+1)}$ results in lower quantization MSE. Since this is true for all the block clusters, we derive the following result: 
\begin{equation}
\begin{split}
\label{eq:mse_ineq_2}
     J^{(n+1)} &= \frac{1}{N_c} \sum_{i=1}^{N_c} \frac{1}{|\mathcal{B}_{i}^{(n+1)}|}\sum_{\bm{b} \in \mathcal{B}_{i}^{(n+1)}} \frac{1}{L_b}\lVert \bm{b}- C_i^{(n+1)}(\bm{b})\rVert^2_2  \le \tilde{J}^{(n+1)}   
\end{split}
\end{equation}

Following (\ref{eq:mse_ineq_1}) and (\ref{eq:mse_ineq_2}), we find that the quantization MSE is non-increasing for each iteration, that is, $J^{(1)} \ge J^{(2)} \ge J^{(3)} \ge \ldots \ge J^{(M)}$ where $M$ is the maximum number of iterations. 
%Therefore, we can say that if the algorithm converges, then it must be that it has converged to a local minimum. 
\hfill $\blacksquare$


\begin{figure}
    \begin{center}
    \includegraphics[width=0.5\textwidth]{sections//figures/mse_vs_iter.pdf}
    \end{center}
    \caption{\small NMSE vs iterations during LO-BCQ compared to other block quantization proposals}
    \label{fig:nmse_vs_iter}
\end{figure}

Figure \ref{fig:nmse_vs_iter} shows the empirical convergence of LO-BCQ across several block lengths and number of codebooks. Also, the MSE achieved by LO-BCQ is compared to baselines such as MXFP and VSQ. As shown, LO-BCQ converges to a lower MSE than the baselines. Further, we achieve better convergence for larger number of codebooks ($N_c$) and for a smaller block length ($L_b$), both of which increase the bitwidth of BCQ (see Eq \ref{eq:bitwidth_bcq}).


\subsection{Additional Accuracy Results}
%Table \ref{tab:lobcq_config} lists the various LOBCQ configurations and their corresponding bitwidths.
\begin{table}
\setlength{\tabcolsep}{4.75pt}
\begin{center}
\caption{\label{tab:lobcq_config} Various LO-BCQ configurations and their bitwidths.}
\begin{tabular}{|c||c|c|c|c||c|c||c|} 
\hline
 & \multicolumn{4}{|c||}{$L_b=8$} & \multicolumn{2}{|c||}{$L_b=4$} & $L_b=2$ \\
 \hline
 \backslashbox{$L_A$\kern-1em}{\kern-1em$N_c$} & 2 & 4 & 8 & 16 & 2 & 4 & 2 \\
 \hline
 64 & 4.25 & 4.375 & 4.5 & 4.625 & 4.375 & 4.625 & 4.625\\
 \hline
 32 & 4.375 & 4.5 & 4.625& 4.75 & 4.5 & 4.75 & 4.75 \\
 \hline
 16 & 4.625 & 4.75& 4.875 & 5 & 4.75 & 5 & 5 \\
 \hline
\end{tabular}
\end{center}
\end{table}

%\subsection{Perplexity achieved by various LO-BCQ configurations on Wikitext-103 dataset}

\begin{table} \centering
\begin{tabular}{|c||c|c|c|c||c|c||c|} 
\hline
 $L_b \rightarrow$& \multicolumn{4}{c||}{8} & \multicolumn{2}{c||}{4} & 2\\
 \hline
 \backslashbox{$L_A$\kern-1em}{\kern-1em$N_c$} & 2 & 4 & 8 & 16 & 2 & 4 & 2  \\
 %$N_c \rightarrow$ & 2 & 4 & 8 & 16 & 2 & 4 & 2 \\
 \hline
 \hline
 \multicolumn{8}{c}{GPT3-1.3B (FP32 PPL = 9.98)} \\ 
 \hline
 \hline
 64 & 10.40 & 10.23 & 10.17 & 10.15 &  10.28 & 10.18 & 10.19 \\
 \hline
 32 & 10.25 & 10.20 & 10.15 & 10.12 &  10.23 & 10.17 & 10.17 \\
 \hline
 16 & 10.22 & 10.16 & 10.10 & 10.09 &  10.21 & 10.14 & 10.16 \\
 \hline
  \hline
 \multicolumn{8}{c}{GPT3-8B (FP32 PPL = 7.38)} \\ 
 \hline
 \hline
 64 & 7.61 & 7.52 & 7.48 &  7.47 &  7.55 &  7.49 & 7.50 \\
 \hline
 32 & 7.52 & 7.50 & 7.46 &  7.45 &  7.52 &  7.48 & 7.48  \\
 \hline
 16 & 7.51 & 7.48 & 7.44 &  7.44 &  7.51 &  7.49 & 7.47  \\
 \hline
\end{tabular}
\caption{\label{tab:ppl_gpt3_abalation} Wikitext-103 perplexity across GPT3-1.3B and 8B models.}
\end{table}

\begin{table} \centering
\begin{tabular}{|c||c|c|c|c||} 
\hline
 $L_b \rightarrow$& \multicolumn{4}{c||}{8}\\
 \hline
 \backslashbox{$L_A$\kern-1em}{\kern-1em$N_c$} & 2 & 4 & 8 & 16 \\
 %$N_c \rightarrow$ & 2 & 4 & 8 & 16 & 2 & 4 & 2 \\
 \hline
 \hline
 \multicolumn{5}{|c|}{Llama2-7B (FP32 PPL = 5.06)} \\ 
 \hline
 \hline
 64 & 5.31 & 5.26 & 5.19 & 5.18  \\
 \hline
 32 & 5.23 & 5.25 & 5.18 & 5.15  \\
 \hline
 16 & 5.23 & 5.19 & 5.16 & 5.14  \\
 \hline
 \multicolumn{5}{|c|}{Nemotron4-15B (FP32 PPL = 5.87)} \\ 
 \hline
 \hline
 64  & 6.3 & 6.20 & 6.13 & 6.08  \\
 \hline
 32  & 6.24 & 6.12 & 6.07 & 6.03  \\
 \hline
 16  & 6.12 & 6.14 & 6.04 & 6.02  \\
 \hline
 \multicolumn{5}{|c|}{Nemotron4-340B (FP32 PPL = 3.48)} \\ 
 \hline
 \hline
 64 & 3.67 & 3.62 & 3.60 & 3.59 \\
 \hline
 32 & 3.63 & 3.61 & 3.59 & 3.56 \\
 \hline
 16 & 3.61 & 3.58 & 3.57 & 3.55 \\
 \hline
\end{tabular}
\caption{\label{tab:ppl_llama7B_nemo15B} Wikitext-103 perplexity compared to FP32 baseline in Llama2-7B and Nemotron4-15B, 340B models}
\end{table}

%\subsection{Perplexity achieved by various LO-BCQ configurations on MMLU dataset}


\begin{table} \centering
\begin{tabular}{|c||c|c|c|c||c|c|c|c|} 
\hline
 $L_b \rightarrow$& \multicolumn{4}{c||}{8} & \multicolumn{4}{c||}{8}\\
 \hline
 \backslashbox{$L_A$\kern-1em}{\kern-1em$N_c$} & 2 & 4 & 8 & 16 & 2 & 4 & 8 & 16  \\
 %$N_c \rightarrow$ & 2 & 4 & 8 & 16 & 2 & 4 & 2 \\
 \hline
 \hline
 \multicolumn{5}{|c|}{Llama2-7B (FP32 Accuracy = 45.8\%)} & \multicolumn{4}{|c|}{Llama2-70B (FP32 Accuracy = 69.12\%)} \\ 
 \hline
 \hline
 64 & 43.9 & 43.4 & 43.9 & 44.9 & 68.07 & 68.27 & 68.17 & 68.75 \\
 \hline
 32 & 44.5 & 43.8 & 44.9 & 44.5 & 68.37 & 68.51 & 68.35 & 68.27  \\
 \hline
 16 & 43.9 & 42.7 & 44.9 & 45 & 68.12 & 68.77 & 68.31 & 68.59  \\
 \hline
 \hline
 \multicolumn{5}{|c|}{GPT3-22B (FP32 Accuracy = 38.75\%)} & \multicolumn{4}{|c|}{Nemotron4-15B (FP32 Accuracy = 64.3\%)} \\ 
 \hline
 \hline
 64 & 36.71 & 38.85 & 38.13 & 38.92 & 63.17 & 62.36 & 63.72 & 64.09 \\
 \hline
 32 & 37.95 & 38.69 & 39.45 & 38.34 & 64.05 & 62.30 & 63.8 & 64.33  \\
 \hline
 16 & 38.88 & 38.80 & 38.31 & 38.92 & 63.22 & 63.51 & 63.93 & 64.43  \\
 \hline
\end{tabular}
\caption{\label{tab:mmlu_abalation} Accuracy on MMLU dataset across GPT3-22B, Llama2-7B, 70B and Nemotron4-15B models.}
\end{table}


%\subsection{Perplexity achieved by various LO-BCQ configurations on LM evaluation harness}

\begin{table} \centering
\begin{tabular}{|c||c|c|c|c||c|c|c|c|} 
\hline
 $L_b \rightarrow$& \multicolumn{4}{c||}{8} & \multicolumn{4}{c||}{8}\\
 \hline
 \backslashbox{$L_A$\kern-1em}{\kern-1em$N_c$} & 2 & 4 & 8 & 16 & 2 & 4 & 8 & 16  \\
 %$N_c \rightarrow$ & 2 & 4 & 8 & 16 & 2 & 4 & 2 \\
 \hline
 \hline
 \multicolumn{5}{|c|}{Race (FP32 Accuracy = 37.51\%)} & \multicolumn{4}{|c|}{Boolq (FP32 Accuracy = 64.62\%)} \\ 
 \hline
 \hline
 64 & 36.94 & 37.13 & 36.27 & 37.13 & 63.73 & 62.26 & 63.49 & 63.36 \\
 \hline
 32 & 37.03 & 36.36 & 36.08 & 37.03 & 62.54 & 63.51 & 63.49 & 63.55  \\
 \hline
 16 & 37.03 & 37.03 & 36.46 & 37.03 & 61.1 & 63.79 & 63.58 & 63.33  \\
 \hline
 \hline
 \multicolumn{5}{|c|}{Winogrande (FP32 Accuracy = 58.01\%)} & \multicolumn{4}{|c|}{Piqa (FP32 Accuracy = 74.21\%)} \\ 
 \hline
 \hline
 64 & 58.17 & 57.22 & 57.85 & 58.33 & 73.01 & 73.07 & 73.07 & 72.80 \\
 \hline
 32 & 59.12 & 58.09 & 57.85 & 58.41 & 73.01 & 73.94 & 72.74 & 73.18  \\
 \hline
 16 & 57.93 & 58.88 & 57.93 & 58.56 & 73.94 & 72.80 & 73.01 & 73.94  \\
 \hline
\end{tabular}
\caption{\label{tab:mmlu_abalation} Accuracy on LM evaluation harness tasks on GPT3-1.3B model.}
\end{table}

\begin{table} \centering
\begin{tabular}{|c||c|c|c|c||c|c|c|c|} 
\hline
 $L_b \rightarrow$& \multicolumn{4}{c||}{8} & \multicolumn{4}{c||}{8}\\
 \hline
 \backslashbox{$L_A$\kern-1em}{\kern-1em$N_c$} & 2 & 4 & 8 & 16 & 2 & 4 & 8 & 16  \\
 %$N_c \rightarrow$ & 2 & 4 & 8 & 16 & 2 & 4 & 2 \\
 \hline
 \hline
 \multicolumn{5}{|c|}{Race (FP32 Accuracy = 41.34\%)} & \multicolumn{4}{|c|}{Boolq (FP32 Accuracy = 68.32\%)} \\ 
 \hline
 \hline
 64 & 40.48 & 40.10 & 39.43 & 39.90 & 69.20 & 68.41 & 69.45 & 68.56 \\
 \hline
 32 & 39.52 & 39.52 & 40.77 & 39.62 & 68.32 & 67.43 & 68.17 & 69.30  \\
 \hline
 16 & 39.81 & 39.71 & 39.90 & 40.38 & 68.10 & 66.33 & 69.51 & 69.42  \\
 \hline
 \hline
 \multicolumn{5}{|c|}{Winogrande (FP32 Accuracy = 67.88\%)} & \multicolumn{4}{|c|}{Piqa (FP32 Accuracy = 78.78\%)} \\ 
 \hline
 \hline
 64 & 66.85 & 66.61 & 67.72 & 67.88 & 77.31 & 77.42 & 77.75 & 77.64 \\
 \hline
 32 & 67.25 & 67.72 & 67.72 & 67.00 & 77.31 & 77.04 & 77.80 & 77.37  \\
 \hline
 16 & 68.11 & 68.90 & 67.88 & 67.48 & 77.37 & 78.13 & 78.13 & 77.69  \\
 \hline
\end{tabular}
\caption{\label{tab:mmlu_abalation} Accuracy on LM evaluation harness tasks on GPT3-8B model.}
\end{table}

\begin{table} \centering
\begin{tabular}{|c||c|c|c|c||c|c|c|c|} 
\hline
 $L_b \rightarrow$& \multicolumn{4}{c||}{8} & \multicolumn{4}{c||}{8}\\
 \hline
 \backslashbox{$L_A$\kern-1em}{\kern-1em$N_c$} & 2 & 4 & 8 & 16 & 2 & 4 & 8 & 16  \\
 %$N_c \rightarrow$ & 2 & 4 & 8 & 16 & 2 & 4 & 2 \\
 \hline
 \hline
 \multicolumn{5}{|c|}{Race (FP32 Accuracy = 40.67\%)} & \multicolumn{4}{|c|}{Boolq (FP32 Accuracy = 76.54\%)} \\ 
 \hline
 \hline
 64 & 40.48 & 40.10 & 39.43 & 39.90 & 75.41 & 75.11 & 77.09 & 75.66 \\
 \hline
 32 & 39.52 & 39.52 & 40.77 & 39.62 & 76.02 & 76.02 & 75.96 & 75.35  \\
 \hline
 16 & 39.81 & 39.71 & 39.90 & 40.38 & 75.05 & 73.82 & 75.72 & 76.09  \\
 \hline
 \hline
 \multicolumn{5}{|c|}{Winogrande (FP32 Accuracy = 70.64\%)} & \multicolumn{4}{|c|}{Piqa (FP32 Accuracy = 79.16\%)} \\ 
 \hline
 \hline
 64 & 69.14 & 70.17 & 70.17 & 70.56 & 78.24 & 79.00 & 78.62 & 78.73 \\
 \hline
 32 & 70.96 & 69.69 & 71.27 & 69.30 & 78.56 & 79.49 & 79.16 & 78.89  \\
 \hline
 16 & 71.03 & 69.53 & 69.69 & 70.40 & 78.13 & 79.16 & 79.00 & 79.00  \\
 \hline
\end{tabular}
\caption{\label{tab:mmlu_abalation} Accuracy on LM evaluation harness tasks on GPT3-22B model.}
\end{table}

\begin{table} \centering
\begin{tabular}{|c||c|c|c|c||c|c|c|c|} 
\hline
 $L_b \rightarrow$& \multicolumn{4}{c||}{8} & \multicolumn{4}{c||}{8}\\
 \hline
 \backslashbox{$L_A$\kern-1em}{\kern-1em$N_c$} & 2 & 4 & 8 & 16 & 2 & 4 & 8 & 16  \\
 %$N_c \rightarrow$ & 2 & 4 & 8 & 16 & 2 & 4 & 2 \\
 \hline
 \hline
 \multicolumn{5}{|c|}{Race (FP32 Accuracy = 44.4\%)} & \multicolumn{4}{|c|}{Boolq (FP32 Accuracy = 79.29\%)} \\ 
 \hline
 \hline
 64 & 42.49 & 42.51 & 42.58 & 43.45 & 77.58 & 77.37 & 77.43 & 78.1 \\
 \hline
 32 & 43.35 & 42.49 & 43.64 & 43.73 & 77.86 & 75.32 & 77.28 & 77.86  \\
 \hline
 16 & 44.21 & 44.21 & 43.64 & 42.97 & 78.65 & 77 & 76.94 & 77.98  \\
 \hline
 \hline
 \multicolumn{5}{|c|}{Winogrande (FP32 Accuracy = 69.38\%)} & \multicolumn{4}{|c|}{Piqa (FP32 Accuracy = 78.07\%)} \\ 
 \hline
 \hline
 64 & 68.9 & 68.43 & 69.77 & 68.19 & 77.09 & 76.82 & 77.09 & 77.86 \\
 \hline
 32 & 69.38 & 68.51 & 68.82 & 68.90 & 78.07 & 76.71 & 78.07 & 77.86  \\
 \hline
 16 & 69.53 & 67.09 & 69.38 & 68.90 & 77.37 & 77.8 & 77.91 & 77.69  \\
 \hline
\end{tabular}
\caption{\label{tab:mmlu_abalation} Accuracy on LM evaluation harness tasks on Llama2-7B model.}
\end{table}

\begin{table} \centering
\begin{tabular}{|c||c|c|c|c||c|c|c|c|} 
\hline
 $L_b \rightarrow$& \multicolumn{4}{c||}{8} & \multicolumn{4}{c||}{8}\\
 \hline
 \backslashbox{$L_A$\kern-1em}{\kern-1em$N_c$} & 2 & 4 & 8 & 16 & 2 & 4 & 8 & 16  \\
 %$N_c \rightarrow$ & 2 & 4 & 8 & 16 & 2 & 4 & 2 \\
 \hline
 \hline
 \multicolumn{5}{|c|}{Race (FP32 Accuracy = 48.8\%)} & \multicolumn{4}{|c|}{Boolq (FP32 Accuracy = 85.23\%)} \\ 
 \hline
 \hline
 64 & 49.00 & 49.00 & 49.28 & 48.71 & 82.82 & 84.28 & 84.03 & 84.25 \\
 \hline
 32 & 49.57 & 48.52 & 48.33 & 49.28 & 83.85 & 84.46 & 84.31 & 84.93  \\
 \hline
 16 & 49.85 & 49.09 & 49.28 & 48.99 & 85.11 & 84.46 & 84.61 & 83.94  \\
 \hline
 \hline
 \multicolumn{5}{|c|}{Winogrande (FP32 Accuracy = 79.95\%)} & \multicolumn{4}{|c|}{Piqa (FP32 Accuracy = 81.56\%)} \\ 
 \hline
 \hline
 64 & 78.77 & 78.45 & 78.37 & 79.16 & 81.45 & 80.69 & 81.45 & 81.5 \\
 \hline
 32 & 78.45 & 79.01 & 78.69 & 80.66 & 81.56 & 80.58 & 81.18 & 81.34  \\
 \hline
 16 & 79.95 & 79.56 & 79.79 & 79.72 & 81.28 & 81.66 & 81.28 & 80.96  \\
 \hline
\end{tabular}
\caption{\label{tab:mmlu_abalation} Accuracy on LM evaluation harness tasks on Llama2-70B model.}
\end{table}

%\section{MSE Studies}
%\textcolor{red}{TODO}


\subsection{Number Formats and Quantization Method}
\label{subsec:numFormats_quantMethod}
\subsubsection{Integer Format}
An $n$-bit signed integer (INT) is typically represented with a 2s-complement format \citep{yao2022zeroquant,xiao2023smoothquant,dai2021vsq}, where the most significant bit denotes the sign.

\subsubsection{Floating Point Format}
An $n$-bit signed floating point (FP) number $x$ comprises of a 1-bit sign ($x_{\mathrm{sign}}$), $B_m$-bit mantissa ($x_{\mathrm{mant}}$) and $B_e$-bit exponent ($x_{\mathrm{exp}}$) such that $B_m+B_e=n-1$. The associated constant exponent bias ($E_{\mathrm{bias}}$) is computed as $(2^{{B_e}-1}-1)$. We denote this format as $E_{B_e}M_{B_m}$.  

\subsubsection{Quantization Scheme}
\label{subsec:quant_method}
A quantization scheme dictates how a given unquantized tensor is converted to its quantized representation. We consider FP formats for the purpose of illustration. Given an unquantized tensor $\bm{X}$ and an FP format $E_{B_e}M_{B_m}$, we first, we compute the quantization scale factor $s_X$ that maps the maximum absolute value of $\bm{X}$ to the maximum quantization level of the $E_{B_e}M_{B_m}$ format as follows:
\begin{align}
\label{eq:sf}
    s_X = \frac{\mathrm{max}(|\bm{X}|)}{\mathrm{max}(E_{B_e}M_{B_m})}
\end{align}
In the above equation, $|\cdot|$ denotes the absolute value function.

Next, we scale $\bm{X}$ by $s_X$ and quantize it to $\hat{\bm{X}}$ by rounding it to the nearest quantization level of $E_{B_e}M_{B_m}$ as:

\begin{align}
\label{eq:tensor_quant}
    \hat{\bm{X}} = \text{round-to-nearest}\left(\frac{\bm{X}}{s_X}, E_{B_e}M_{B_m}\right)
\end{align}

We perform dynamic max-scaled quantization \citep{wu2020integer}, where the scale factor $s$ for activations is dynamically computed during runtime.

\subsection{Vector Scaled Quantization}
\begin{wrapfigure}{r}{0.35\linewidth}
  \centering
  \includegraphics[width=\linewidth]{sections/figures/vsquant.jpg}
  \caption{\small Vectorwise decomposition for per-vector scaled quantization (VSQ \citep{dai2021vsq}).}
  \label{fig:vsquant}
\end{wrapfigure}
During VSQ \citep{dai2021vsq}, the operand tensors are decomposed into 1D vectors in a hardware friendly manner as shown in Figure \ref{fig:vsquant}. Since the decomposed tensors are used as operands in matrix multiplications during inference, it is beneficial to perform this decomposition along the reduction dimension of the multiplication. The vectorwise quantization is performed similar to tensorwise quantization described in Equations \ref{eq:sf} and \ref{eq:tensor_quant}, where a scale factor $s_v$ is required for each vector $\bm{v}$ that maps the maximum absolute value of that vector to the maximum quantization level. While smaller vector lengths can lead to larger accuracy gains, the associated memory and computational overheads due to the per-vector scale factors increases. To alleviate these overheads, VSQ \citep{dai2021vsq} proposed a second level quantization of the per-vector scale factors to unsigned integers, while MX \citep{rouhani2023shared} quantizes them to integer powers of 2 (denoted as $2^{INT}$).

\subsubsection{MX Format}
The MX format proposed in \citep{rouhani2023microscaling} introduces the concept of sub-block shifting. For every two scalar elements of $b$-bits each, there is a shared exponent bit. The value of this exponent bit is determined through an empirical analysis that targets minimizing quantization MSE. We note that the FP format $E_{1}M_{b}$ is strictly better than MX from an accuracy perspective since it allocates a dedicated exponent bit to each scalar as opposed to sharing it across two scalars. Therefore, we conservatively bound the accuracy of a $b+2$-bit signed MX format with that of a $E_{1}M_{b}$ format in our comparisons. For instance, we use E1M2 format as a proxy for MX4.

\begin{figure}
    \centering
    \includegraphics[width=1\linewidth]{sections//figures/BlockFormats.pdf}
    \caption{\small Comparing LO-BCQ to MX format.}
    \label{fig:block_formats}
\end{figure}

Figure \ref{fig:block_formats} compares our $4$-bit LO-BCQ block format to MX \citep{rouhani2023microscaling}. As shown, both LO-BCQ and MX decompose a given operand tensor into block arrays and each block array into blocks. Similar to MX, we find that per-block quantization ($L_b < L_A$) leads to better accuracy due to increased flexibility. While MX achieves this through per-block $1$-bit micro-scales, we associate a dedicated codebook to each block through a per-block codebook selector. Further, MX quantizes the per-block array scale-factor to E8M0 format without per-tensor scaling. In contrast during LO-BCQ, we find that per-tensor scaling combined with quantization of per-block array scale-factor to E4M3 format results in superior inference accuracy across models. 


\end{document}
