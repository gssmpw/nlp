\section{Related Research}
We built our sentiment index based on POMS____ to visualize collective emotions in society. 
POMS is a questionnaire-based method that categorizes individuals' psychological states into six emotional categories: ``Anger," ``Confusion," ``Depression," ``Fatigue," ``Tension," and ``Vigor." 
POMS can measure emotions such as ``Fatigue'' and ``Tension'' which are not covered by other indicators. 
Furthermore, its validity has already been demonstrated in prior studies conducted in both English____ and Japanese____, making it suitable for the purposes of this study.
Therefore we developed a new sentiment index for understanding and visualizing the overall emotional state of society based on POMS.

In the previous studies____, emotional words were selected for each emotional category, and an emotional word dictionary was constructed. The six emotional indicators were then calculated by normalizing the ratio between the daily posting time series of each emotional word $w_{k}(t)$ from the set of emotion $e$'s emotional words $E_{e}$, and the total number of posts $w_{all}(t)$, resulting in $I_{e}(t)$:

\begin{eqnarray} r_{e}(t) &=& \sum_{k \in E_{e}}\left(\frac{w_{k}(t)}{w_{all}(t)} \right) \\
I_{e}(t) &=& \frac{r_{e}(t)- \langle r_{e}(t) \rangle }{\sigma_{e}}
\end{eqnarray}

Here, $\langle \cdot \rangle$ denotes the average, and $\sigma_{e}$ represents the standard deviation of the ratio $r_{e}(t)$. This indicator has been used successfully to visualize the collective emotions of Japanese society.

\begin{figure}[tb] \centering \includegraphics[width=\linewidth]{poms.png} \caption{POMS-based sentiment index proposed by Sano, Fig.1(A) in ____} \label{fig1} \end{figure}

It demonstrated the use of this emotional time series $I_e(t)$ during the Great East Japan Earthquake on March 11, 2011. Immediately after the earthquake, the ``Tension" indicator significantly increased as a collective emotion on blog platforms, followed by increases in the ``Depression" and ``Confusion" indicators (Fig.~\ref{fig1}). Similar emotional responses have been observed during other incidents and accidents in society.

In addition, the periodicity of the emotional indicator time series was statistically demonstrated, showing a distinct weekly cycle with significant differences between weekends and weekdays. Seasonal cycles were also observed in the ``Tension" and ``Fatigue" indicators (Fig.~\ref{fig2}).

\begin{figure}[htbp] \centering \includegraphics[width=\linewidth]{sano_date.png} \caption{Periodical properties of the sentiment index, Fig.2(A) in ____} \label{fig2} \end{figure}

Studies that apply POMS to large-scale blog data have laid the foundation for technologies that visualize societal emotional states. However, these studies face limitations due to the nature of blog platforms, such as weaker connections between users. Additionally, since POMS was originally developed for mental health diagnostics, it tends to emphasize negative emotions. An updated version, POMS2, introduces the ``Friendliness" category, providing a more nuanced classification of emotional states.

Consequently, research on societal emotional dynamics has increasingly shifted to platforms like X (formerly Twitter), which offer higher frequency, real-time data, and greater accessibility to user information. In addition, a positive-negative index is often used for simplicity.