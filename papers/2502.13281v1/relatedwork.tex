\section{Related Work}
\subsection{Informal Learning and Social Learning}
Informal learning is learning that happens naturally, outside of formal training, and is spontaneous, unstructured, and often self-directed \cite{marsick1990informal, Eraut2004}. It can consist of solitary learning experiences such as trial and error and social learning experiences like peer influence. Social learning specifically refers to learning that occurs through the interaction, observation, and imitation of others \cite{Bandura1977}. Social learning, especially through peer-to-peer exchanges, discussions with colleagues, and shared problem solving have been observed as integral parts of informal learning in professional contexts \cite{Livingstone2001, Eraut2004}.

Additional studies further emphasize the social nature of workplace learning. For example, Lave and Wenger describe learning as something that happens within communities of practice, while Boud, Keogh, and Walker point out that much professional development occurs through everyday interactions and reflective practice \cite{Lave_Wenger_1991, boud1985reflection}. More recently, Billett has shown that workplace participatory practices play a key role in ongoing professional growth \cite{Billett2001}. Together, these studies suggest that informal learning at work is not just about individual effort, but also relies heavily on social interaction and collaboration. Building upon these insights, our study examines the roles of social and experiential learning—the two key components of informal learning—in the adoption of AI tools in industry. 


% ReelFramer, 2024; Schreiber \& Siege, 2022). Tao Long et al. (2024) further emphasize that AI users refine their interactions over time through hands-on experimentation, gradually adjusting prompts and workflows to better fit their needs.

% While this decentralized learning process is common, it raises concerns about whether informal learning fosters deep AI proficiency or reinforces surface-level use. Employees might plateau at a given skill level, remain unaware of advanced features, or mistakenly believe they have mastered an AI tool (Ackerman et al., 2020). Prior studies lack empirical evidence on whether users fully explore AI capabilities or if their learning stops at basic interactions (Kang et al., 2021).

% Our study examines these gaps by analyzing how industry professionals engage with M365 Copilot. We assess whether informal learning leads to true mastery or functional engagement, providing insights into improving AI education in the workplace.

\subsection{AI Adoption At Work}
Recent studies demonstrate that generative AI can substantially enhance workplace productivity. Recent research reveals that tools like ChatGPT help professionals complete tasks faster and with improved quality \cite{brynjolfsson2024generativeaiwork}. In one experiment, 453 college-educated professionals using ChatGPT produced higher-quality work, and Dell et al. found that consultants with access to ChatGPT generated more innovative ideas \cite{dell2023navigating}. Similar benefits are observed in technical domains, where studies by Cui and Peng report a 26\% increase in weekly coding tasks \cite{cui2024effects, peng2023impact}. In another creative context, studies demonstrate how a human-AI co-creative system can help journalists translate print articles into engaging, narrative-driven video reels, highlighting AI’s potential to support content retargeting through multiple narrative framings \cite{wang2024reelframer, schmid2021narrative,opal,kantosalo2016modes}.

At the same time, several studies highlight important limitations of AI support. Research on creativity support tools and investigations into tasks requiring core human skills indicate that while AI can assist with specific tasks, AI does not fully replicate human judgment or creativity \cite{tao2023ai, long2024novelty}. Additional research further cautions that for complex tasks, reliance on AI may lead to incorrect recommendations and documents challenges users face when integrating AI outputs into their work processes \cite{dell2023navigating, kim2024unlocking}. Moreover, Randazzo suggests that AI tends to impact specific work processes rather than transforming entire industries \cite{randazzo2024cyborgs}.

While a number of studies have examined AI adoption at the organizational level—such as Neumann, Guirguis, and Steiner’s comparative case study of Swiss public organizations that explored structural approaches to managing AI innovation—there remains a notable gap in understanding AI adoption from the individual perspective \cite{Neumann}. Recent calls in the literature urge research that considers AI adoption from the viewpoint of individual citizens and workers \cite{pencheva,Wirtz03102021}. In response, our research focuses on how employees in the United States adopt AI tools like M365 Copilot in their daily work by examining informal and social learning behaviors, like trial and error, peer discussions, and self-directed exploration. This approach bridges the gap between top–down organizational strategies and grassroots-level experiences, providing a more complete understanding of AI adoption and offering practical guidance for designing user-centered training and support systems.

% Discoverability is a major challenge in AI adoption (Durilong et al., 2021). Many users find AI features by accident or through peer recommendations, rather than intentional exploration (Training Industry, 2024; Liao et al., 2022). This contrasts with traditional software adoption, where users engage with structured documentation or guided onboarding.

% If employees rely only on features they initially discover, they may never reach full proficiency or uncover advanced functionalities (Wang \& Johnson, 2023). Studies show that AI tools often support more sophisticated interactions than what most users engage with (ReelFramer, 2024), but there is little research on whether users actively seek out new capabilities or assume they already know everything the tool can do. Tao Long et al. (2024) suggest that users may fine-tune AI interactions over time, but this does not necessarily translate to comprehensive mastery of AI tools.

% Unlike prior research that primarily relies on large-scale surveys, our study provides qualitative insights from in-depth interviews with 10 industry professionals, offering a bottom-up, grassroots perspective of AI adoption in industry. By capturing real-world adoption behaviors, we examine how users discover and integrate new features into their workflows. We identify key barriers to discoverability, including lack of awareness, reliance on peer learning, and usability friction, shedding light on why certain features remain underutilized despite their potential benefits (Chen \& Horvitz, 2023). This focused approach complements broader survey-based studies by offering a deeper understanding of user experiences in real work environments.

\subsection{Microsoft 365 Copilot Tools}
M365 Copilot is a large language model (LLM)-backed, chat-based AI productivity tool integrated with widely used applications such as Microsoft Word, PowerPoint, Power BI, and Teams. As one of the leading AI productivity tools in the industry, M365 Copilot has millions of paid users and is endorsed by numerous global enterprises for its significant contributions to efficiency, productivity, and cost savings \cite{forrester-m365-copilot, althoff2025value}. Its widespread adoption provides a unique opportunity to observe trends in AI tool adoption across a heterogeneous workforce. Consequently, M365 Copilot serves as an ideal proxy for understanding the broader learning and adoption behaviors associated with AI productivity tools across industries. For this reason, our study examines how individuals across various industries adopt M365 Copilot to reveal overall insights of grassroots-level workplace AI adoption.