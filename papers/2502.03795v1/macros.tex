\DeclareMathOperator{\sym}{sym}

\definecolor{jz}{rgb}{0.1,0.45,0.1}
% provide three commenting commands for each author
\newcommand{\jz}[1]{{\color{jz}{#1}}}
\newcommand{\jzs}[1]{{\sout{\color{jz}{#1}}}}
\newcommand{\jztd}[2][]{\todo[color=jz!40,#1]{#2}}

\definecolor{rr}{rgb}{0,0,1}
% provide three commenting commands for each author
\newcommand{\rr}[1]{{\color{rr}{#1}}}
\newcommand{\rrs}[1]{{\sout{\color{rr}{#1}}}}
\newcommand{\rrtd}[2][]{\todo[color=rr!40,#1]{#2}}

\definecolor{ymm}{rgb}{1,0.53,0.0}
% provide three commenting commands for each author
\newcommand{\ymm}[1]{{\color{ymm}{#1}}}
\newcommand{\ymms}[1]{{\sout{\color{ymm}{#1}}}}
\newcommand{\ymmtd}[2][]{\todo[color=ymm!40,#1]{#2}}

% JZ
\renewcommand{\norm}[2][]{\| #2 \|_{#1}}

% RRs macros
\newcommand{\dkl}{\mathcal{D}_{\rm KL}}




% ---------------
\newcommand{\R}{\mathbb{R}}
\newcommand{\N}{\mathbb{N}}
\newcommand{\bsv}{{\boldsymbol v}}
\newcommand{\bsw}{{\boldsymbol w}}
\newcommand{\bsk}{{\boldsymbol k}}
\newcommand{\bsl}{{\boldsymbol l}}
\newcommand{\bsm}{{\boldsymbol m}}
\newcommand{\bse}{{\boldsymbol e}}



\newcommand{\bsalpha}{{\boldsymbol \alpha}}
\newcommand{\bslambda}{{\boldsymbol \lambda}}
\newcommand{\bsmu}{{\boldsymbol \mu}}



\renewcommand{\set}[2]{\{#1\,:\,#2\}}
\newcommand{\setl}[2]{ \{#1\, :\,#2 \}}
\newcommand{\setc}[2]{\left\{#1\, :\,#2\right\}}
% trace
\DeclareMathOperator{\tr}{tr}
% complex number i
\newcommand{\ii}{\mathrm{i}}
\newcommand{\dd}{\;\mathrm{d}}
\newcommand{\e}{\mathrm{e}}
% ---------------

\numberwithin{equation}{section}

\definecolor{plum}{rgb}{.4,0,.4}
\definecolor{BrickRed}{rgb}{0.6,0,0}
\newcommand{\td}[1]{{\color{red}{[\textbf{TODO} #1]}}}
% \newcommand{\td}[1]{{\color{red} [\marginnote{\textbf{TODO}} #1]}}

\newcommand{\mjt}[1]{{\color{blue} \marginnote{\textbf{MJT}} #1}} %rather than \textcolor, which breaks on certain inputs

%from djhsu --->

\def\ddefloop#1{\ifx\ddefloop#1\else\ddef{#1}\expandafter\ddefloop\fi}

% \cA, \cB, ...
\def\ddef#1{\expandafter\def\csname c#1\endcsname{\ensuremath{\mathcal{#1}}}}
\ddefloop ABCDEFGHIJKLMNOPQRSTUVWXYZ\ddefloop

% \sA, \sB, ...
\def\ddef#1{\expandafter\def\csname s#1\endcsname{\ensuremath{\mathsf{#1}}}}
\ddefloop ABCDEFGHIJKLMNOPQRSTUVWXYZ\ddefloop

%<--- from djhsu

\def\E{\mathbf{E}}
\def\PP{\mathbf{P}}

\def\Reals{\mathbb{R}}
\def\Naturals{\mathbb{N}}

\def\argmin{\operatornamewithlimits{arg\,min}}

\def\Wass{\cW}

\def\deq{:=}

\def\wh#1{\widehat{#1}}
\def\bd#1{\boldsymbol{#1}}
\def\bz{\bd{z}}
\def\bZ{\bd{Z}}
\def\bP{\bd{P}}
	\def\tr{{\mathrm{tr}}}
\def\Osc{{\mathrm{Osc}}}
\def\tO{{\tilde{\cO}}}
\def\tOm{\tilde{\Omega}}
\def\barw{\overline{w}}

\def\1{{\mathbf 1}}

\def\ave#1{\langle #1 \rangle}
\def\Ave#1{\left\langle #1 \right\rangle}
\def\trn{{\hbox{\it\tiny T}}} %transpose
 
\def\eps{\varepsilon}


\newtheorem{theorem}{Theorem}[section]
\newtheorem{lemma}[theorem]{Lemma}
\newtheorem{proposition}[theorem]{Proposition}
\newtheorem{corollary}[theorem]{Corollary}
\newtheorem{assumption}[theorem]{Assumption}
\newtheorem{remark}[theorem]{Remark}
\newtheorem{example}[theorem]{Example}
\newtheorem{definition}[theorem]{Definition}
