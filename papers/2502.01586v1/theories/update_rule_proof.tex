\section{Grassmann Exponential}
\label{appendix:grassmann-proof}

\updaterule*

{\bf proof.} Using Grassmannina mathematics, we know that every \( \Delta \in T_P Gr(n,p) \) is of the form  
\begin{equation}
     \Delta = Q \begin{pmatrix} 0 & B^T \\ B & 0 \end{pmatrix} Q^T = \left[ Q \begin{pmatrix} 0 & -B^T \\ B & 0 \end{pmatrix} Q^T, P \right]
\end{equation}
Then the lift of \( \Delta \in T_P Gr(n,p) \) to \( Q = \begin{pmatrix} U & U_{\perp} \end{pmatrix} \) can also be calculated explicitly as follows:
\begin{equation}
    \Delta^{\text{hor}}_Q = [\Delta, P] Q = Q \begin{pmatrix} 0 & -B^T \\ B & 0 \end{pmatrix}
\end{equation}

To resume our proof, we need to define the orthogonal group and specifying its tangent space.
\begin{definition}[ \textbf{Orthogonal Group}]\label{def:orthogroup}
 The orthogonal group \( O(n) \) is defined as the set of all \( n \times n \) matrices \( Q \) over \( \mathbb{R} \) such that \( Q^T Q = Q Q^T = I_n \), where \( Q^T \) is the transpose of \( Q \) and \( I_n \) is the \( n \times n \) identity matrix:
 \begin{equation*}
O(n) = \{ Q \in \mathbb{R}^{n \times n} \mid Q^T Q = I_n = Q Q^T \}.
 \end{equation*}
 \end{definition}
 
Then the tangent space of the orthogonal group \( O(n) \) at a point \( Q \), denoted \( T_Q O(n) \), is defined as the set of matrices of the form \( Q\Omega \), where \( \Omega \in \mathbb{R}^{n \times n} \) is a skew-symmetric matrix, i.e., \( \Omega^T = -\Omega \):
\begin{equation*}
    T_Q O(n) = \{ Q\Omega \mid \Omega \in \mathbb{R}^{n \times n}, \Omega^T = -\Omega \}.
\end{equation*}
The geodesic from \( Q \in O(n) \) in direction \( Q\Omega \in T_Q O(n) \) is calculated via 
\begin{equation}
    \text{Exp}_Q^O(tQ\Omega) = Q \exp_m(t\Omega),
\end{equation}
If \( P \in Gr(n,p) \) and \( \Delta \in T_P Gr(n,p) \) with \( \Delta^{\text{hor}}_Q = Q \begin{pmatrix} 0 & -B^T \\ B & 0 \end{pmatrix} \), the geodesic in the Grassmannian is therefore
\begin{equation}
\label{eq:32}
\text{Exp}^{Gr}_P(t\Delta) = \pi^{OG} \left( Q \exp_m \left( t \begin{pmatrix} 0 & -B^T \\ B & 0 \end{pmatrix} \right) \right).
\end{equation}
where \(\pi^{OG}\) is the projection from \(O(n)\) to \(Gr(n, p)\).
If the thin SVD of \( B \) is given by  
\begin{equation*}
B = U_\perp^T \Delta^{\text{hor}}_U = U_\perp^T \hat{Q} \Sigma V^T
\end{equation*}
with \( W := U_\perp^T \hat{Q} \in St(n-p, r), \Sigma \in \mathbb{R}^{r \times r}, V \in St(p, r) \). Let \( W_\perp, V_\perp \) be suitable orthogonal completions. Then, 
\begin{equation*}
\exp_m \begin{pmatrix} 0 & -B^T \\ B & 0 \end{pmatrix} = \begin{pmatrix} V & V_\perp & 0 & 0 \\ 0 & 0 & W & W_\perp \end{pmatrix} \begin{pmatrix} \cos(\Sigma) & 0 & -\sin(\Sigma) & 0 \\ 0 & I_{p-r} & 0 & 0 \\ \sin(\Sigma) & 0 & \cos(\Sigma) & 0 \\ 0 & 0 & 0 & I_{n-p-r} \end{pmatrix} \begin{pmatrix} V^T & 0 \\ V_\perp^T & 0 \\ 0 & W^T \\ 0 & W_\perp^T \end{pmatrix},
\end{equation*}
which leads to the desired result when inserted into \eqref{eq:32}. For more mathematical details, you can refer to \citet{edelman1998geometryalgorithmsorthogonalityconstraints}, \citet{Bendokat_2024}, or other useful resources on Grassmann geometry.


