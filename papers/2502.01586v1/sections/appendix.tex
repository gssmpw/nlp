\section{Convergence of SubTrack-Grad}
\label{appendix:C}

\convergence*

{\bf proof.} To demonstrate that SubTrack-Grad converges to the global minimum during training, we begin by deriving the recursive form of the gradients.

Let \(\otimes\) denote the Kronecker product. Then, \( vec(AXB) = (B^\top \otimes A) vec(X)\). 

By applying \(vec\) to the gradient form given in the theorem, we obtain:
\begin{equation}
\label{eq:11}
    g_t = vec(G_t) = vec(\sum_i A_i + \sum_i B_iWC_i) = a_t - D_t w_t 
\end{equation}
where \(g_t := vec(G_t)\), \(w_t := vec(W_t)\), \(a_t := \frac1N\sum_i vec(A_{i,t})\), and \(D_t = \frac1N\sum_i C_{i,t} \otimes B_{i,t}\).

As defined in the theorem, let \(P_t = {S_t^l}^\top G_t S_t^r\). Its vectorized form can be expressed using the Kronecker product as follows:

\begin{equation}
\begin{gathered}
    p_t = vec(P_t) = vec({S_t^l}^\top G_t S_t^r) = ({S_t^r}^\top \otimes {S_t^l}^\top)vec(G_t) \\ \quad
    = {(S_t^r \otimes S_t^l)}^\top vec(G_t) = {(S_t^r \otimes S_t^l)}^\top g_t 
    \label{eq:p_t}
\end{gathered}
\end{equation}
Now recalling \(\widehat{G}_t\) from \eqref{eq:full_project_back}, it can be written as:
\begin{equation*}
    \widehat{G}_{t} = S_{t}^l {S_{t}^l}^\top G_{t} S_{t}^r {S_{t}^r}^\top
\end{equation*}
Thus, its vectorized form will be:
\begin{equation}
\begin{gathered}
    vec(\widehat{G}_t) = \widehat{g}_t = vec(S_{t}^l {S_{t}^l}^\top G_{t} S_{t}^r {S_{t}^r}^\top) = vec(S_{t}^l P_t {S_{t}^r}^\top) \\ \quad
    = (S_{t}^r \otimes S_{t}^l)vec(P_t) = (S_{t}^r \otimes S_{t}^l)p_t
    \label{eq:g_hat_t}
\end{gathered}
\end{equation}
This is where the constant subspace assumption becomes necessary. To derive the recursive form of \(g_t\), we assume that the projection matrices remain fixed throughout training, i.e., \(S_t^r = S^r\) and \(S_t^l = S^l\). Consequently, we can restate equations \eqref{eq:p_t} and \eqref{eq:g_hat_t} as follows:
\begin{eqnarray}
p_t = {(S^r \otimes S^l)}^\top g_t \label{eq:14} \\
\widehat{g}_t = (S^r \otimes S^l) p_t \label{eq:15}
\end{eqnarray}
Then we can write the recursive form of \(g_t\):
\begin{equation}
\begin{gathered}
\label{eq:16}
    g_t = a_t - D_t w_t = (a_t - a_{t-1}) + (D_{t-1} - D_t) w_t + a_{t-1} - D_{t-1}w_t \\ \quad 
    = e_t + a_{t-1} - D_{t-1}(w_{t-1} + \mu \widehat{g}_{t-1}) = e_t + g_{t-1} - \mu D_{t-1} \widehat{g}_{t-1}  
\end{gathered}
\end{equation}
where \(e_t := (a_t - a_{t-1}) + (D_{t-1} - D_t) w_t\). 

Note that in deriving \eqref{eq:16}, we utilized the general form of the weight update rule, \( w_{t+1} = w_t - \mu g_t \), which can be rewritten as \( w_t = w_{t+1} + \mu g_t \). By applying this rule along with \eqref{eq:11}, we arrive at the second equality in \eqref{eq:16} as follows:
\begin{equation*}
\begin{gathered}
    g_t = a_t - D_t w_t = a_t - D_t w_t - g_{t-1} + g_{t-1}  \\ \quad
    = a_t - D_t w_t - a_{t-1} + D_{t-1}w_{t-1} + a_{t-1} - D_{t-1}w_{t-1} \\ \quad
    = a_t - D_t w_t - a_{t-1} + D_{t-1}(w_t + \mu g_{t-1}) + a_{t-1} - D_{t-1}(w_t + \mu g_{t-1}) \\ \quad
    = a_t - D_t w_t - a_{t-1} + D_{t-1}w_t + \mu D_{t-1} g_{t-1} + a_{t-1} - D_{t-1}w_t - \mu D_{t-1} g_{t-1} \\ \quad
    = a_t - a_{t-1} + (D_{t-1} - D_t)w_t + a_{t-1} - D_{t-1}
\end{gathered}
\end{equation*}

To obtain \(p_t\) from this recursive formulation, we can left-multiply by \({(S^r \otimes S^l)}^\top\), as shown in \eqref{eq:15}:
\begin{eqnarray}
\begin{split}
    p_t = {(S^r \otimes S^l)}^\top e_t + {(S^r \otimes S^l)}^\top g_{t-1} - 
    \mu {(S^r \otimes S^l)}^\top D_{t-1} \widehat{g}_{t-1}
\end{split}
\end{eqnarray}
Now, based on \eqref{eq:14} and \eqref{eq:15}, \(p_t\) can be written as:
\begin{eqnarray}
    p_t = {(S^r \otimes S^l)}^\top e_t + p_{t-1} - \mu {(S^r \otimes S^l)}^\top D_{t-1} {(S^r \otimes S^l)}p_{t-1} \label{eq:recurstive_pt}
\end{eqnarray}
Let define:
\begin{equation}
\begin{gathered}
 \widehat{D}_t := {(S^r \otimes S^l)}^\top D_t (S^r \otimes S^l) = \frac1N \sum_i {(S^r \otimes S^l)}^\top (C_{i,t} \otimes B_{i,t}) (S^r \otimes S^l) \\ \quad
 =\frac1N \sum_i ({S^r}^\top C_{i,t}S^r) \otimes ({S^l}^\top B_{i,t} S^l) 
 \end{gathered}
\end{equation}
Then we can expand \eqref{eq:recurstive_pt} and show that:
\begin{equation}
    p_t = (I - \mu \hat D_{t-1})p_{t-1} + (S^r \otimes S^l)^\top e_t
    \label{eq:20}
\end{equation}
Note that \(S^l\) and \(S^r\) are orthonormal matrices. This is ensured because the subspace is initialized using the SVD of \(G_0\), and the Grassmannian update rule provided in \eqref{eq:update-role} preserves the orthonormality of the subspace matrices throughout training. Since \(S^l\) and \(S^r\) are orthonormal, we have \({S^l}^\top S^l = I\) and \({S^r}^\top S^r = I\). Consequently, we can bound the norm of the second term in \eqref{eq:20} as follows:
\begin{equation}
\|(S^r \otimes S^l)^\top e_t\|_2 = \|vec({S^l}^\top E_t S^r)\|_2 = \|{S^l}^\top E_t S^r\|_F \le \|E_t\|_F
\end{equation}
Here \(E_t\) is the matrix form of \(e_t\), and as declared before, \(e_t := (a_t - a_{t-1}) + (D_{t-1} - D_t) w_t\), thus:
\begin{equation}
E_t := \frac1N\sum_i (A_{i,t} - A_{i,t-1}) + \frac1N\sum_i (B_{i,t-1} W_t C_{i,t-1} - B_{i,t} W_t C_{i,t}) \label{eq:22}
\end{equation}
Next, we need to find an upper bound for the norm of each term in \eqref{eq:22} to establish an upper bound for \(\|E_t\|_F\). Based on the assumptions of the theorem, \(A_i\), \(B_i\), and \(C_i\) exhibit L-Lipschitz continuity with constants \(L_A\), \(L_B\), and \(L_C\), respectively. Additionally, \(\|W_t\|_F\) is bounded by a scalar \(M\). We have:
\begin{eqnarray}
    \|A_t - A_{t-1}\|_F &\le& L_A \|W_t - W_{t-1}\|_F = \mu L_A \|\tilde G_{t-1}\|_F \le \mu L_A \|P_{t-1}\|_F 
\end{eqnarray}
In the first equality, we apply \eqref{eq:update_rule}, while the last equality holds due to \eqref{eq:15} and the orthonormality of the projection matrices. The subsequent two inequalities can be derived similarly using these equations.
\begin{equation}
\begin{gathered}
    \|(B_t - B_{t-1})W_t C_{t-1}\|_F \le L_B \|W_t - W_{t-1}\|_F \|W_t\|_F \|C_{t-1}\|_F 
    \\ \quad
    = \mu L_B L_C M^2 \|P_{t-1}\|_F  
\end{gathered}
\end{equation}
\\
\begin{equation}
\begin{gathered}
    \|B_t W_t (C_{t-1} - C_t)\|_F \le L_C  \|B_t\|_F \|W_t\|_F\|W_{t-1} - W_t\|_F 
    \\ \quad
    = \mu L_B L_C M^2 \|P_{t-1}\|_F
\end{gathered}
\end{equation}
We can now derive the bound for \(\|E_t\|_F\) as follows:
\begin{equation}
\begin{gathered}
    \|E_t\|_F \le \mu L_A \|\tilde G_{t-1}\|_F \le \mu L_A \|P_{t-1}\|_F + \mu L_B L_C M^2 \|P_{t-1}\|_F + \mu L_B L_C M^2 \|P_{t-1}\|_F \\ \quad 
    = \mu(L_A + 2L_BL_CM^2)\|P_{t-1}\|_F
\end{gathered}
\end{equation}
To calculate the norm bound for the first term in \eqref{eq:20}, we first need to establish the bounds for \(\widehat{D}_t\). This involves estimating the minimum eigenvalue of \(\widehat{D}_t\). 

If we define \(\gamma_{min, i,t} = \lambda_{min}({S^l}^\top B_{i,t} S^l)\lambda_{min}({S^r}^\top C_{i,t}S^r)\), then it follows that \(\lambda_{min}(({S^l}^\top B_{i,t} S^l)\otimes({S^r}^\top C_{i,t}S^r)) = \gamma_{min, i,t}\). Consequently, \(\widehat{D}_t\) will satisfy the following inequality for every unit vector \(\vv v\):

\begin{equation}
    \vv v^\top \widehat{D}_t \vv v = \frac1N \sum_i \vv v^\top \left[({S^l}^\top B_{i,t} S^l)\otimes({S^r}^\top C_{i,t}S^r)\right]\vv v \ge \frac1N \sum_i \gamma_{min, i,t} 
\end{equation}
this actually provides a lower bound for eigenvalues of \(\widehat{D}_t\), thus:
\begin{eqnarray}
    \lambda_{\max}(I - \mu \widehat{D}_{t-1}) \le 1 - \frac{\mu}{N} \sum_i \gamma_{min, i,t-1}
\end{eqnarray}
considering the definition of \(\kappa_t\) in the theorem, we can now easily show that:
\[
    \|P_t\|_F \leq [1-\mu(\kappa_{t-1} - L_A - 2L_B L_C M^2)]\|P_{t-1}\|_F.
\]
and completing the proof.

While SubTrack-Grad utilizes right/left projections to reduce memory consumption, the proof is presented using both projection matrices to ensure generality. Here, we demonstrate how the proof proceeds under the assumption \( m \leq n \) (without loss of generality), which allows the use of the left projection matrix.

Using the left projection matrix, the current formulation of \( P_t \), defined as \( P_t = {S_t^l}^\top G_t S_t^r \), simplifies to \( P_t = {S_t^l}^\top G_t \). Similarly, \( \widehat{G}_t = S_t^l {S_t^l}^\top G_t S_t^r {S_t^r}^\top \) reduces to \( \widehat{G}_t = S_t^l {S_t^l}^\top G_t \). From this point, the proof continues by substituting \( S_t^r \) with the identity matrix, allowing the derivation of the vectorized forms of \( g_t \), \( \widehat{g}_t \), \( p_t \), and related terms.

The remainder of the proof remains largely unaffected. It can be readily verified that the recursive formulation of \( g_t \) is unchanged. Although the definition of \( P_t \) is modified, it continues to satisfy the bounds required for convergence, ensuring that \( P_t \) converges to 0 when the left projection matrix is used.
\section{Grassmann Exponential}
\label{appendix:grassmann-proof}

\updaterule*

{\bf proof.} Using Grassmannina mathematics, we know that every \( \Delta \in T_P Gr(n,p) \) is of the form  
\begin{equation}
     \Delta = Q \begin{pmatrix} 0 & B^T \\ B & 0 \end{pmatrix} Q^T = \left[ Q \begin{pmatrix} 0 & -B^T \\ B & 0 \end{pmatrix} Q^T, P \right]
\end{equation}
Then the lift of \( \Delta \in T_P Gr(n,p) \) to \( Q = \begin{pmatrix} U & U_{\perp} \end{pmatrix} \) can also be calculated explicitly as follows:
\begin{equation}
    \Delta^{\text{hor}}_Q = [\Delta, P] Q = Q \begin{pmatrix} 0 & -B^T \\ B & 0 \end{pmatrix}
\end{equation}

To resume our proof, we need to define the orthogonal group and specifying its tangent space.
\begin{definition}[ \textbf{Orthogonal Group}]\label{def:orthogroup}
 The orthogonal group \( O(n) \) is defined as the set of all \( n \times n \) matrices \( Q \) over \( \mathbb{R} \) such that \( Q^T Q = Q Q^T = I_n \), where \( Q^T \) is the transpose of \( Q \) and \( I_n \) is the \( n \times n \) identity matrix:
 \begin{equation*}
O(n) = \{ Q \in \mathbb{R}^{n \times n} \mid Q^T Q = I_n = Q Q^T \}.
 \end{equation*}
 \end{definition}
 
Then the tangent space of the orthogonal group \( O(n) \) at a point \( Q \), denoted \( T_Q O(n) \), is defined as the set of matrices of the form \( Q\Omega \), where \( \Omega \in \mathbb{R}^{n \times n} \) is a skew-symmetric matrix, i.e., \( \Omega^T = -\Omega \):
\begin{equation*}
    T_Q O(n) = \{ Q\Omega \mid \Omega \in \mathbb{R}^{n \times n}, \Omega^T = -\Omega \}.
\end{equation*}
The geodesic from \( Q \in O(n) \) in direction \( Q\Omega \in T_Q O(n) \) is calculated via 
\begin{equation}
    \text{Exp}_Q^O(tQ\Omega) = Q \exp_m(t\Omega),
\end{equation}
If \( P \in Gr(n,p) \) and \( \Delta \in T_P Gr(n,p) \) with \( \Delta^{\text{hor}}_Q = Q \begin{pmatrix} 0 & -B^T \\ B & 0 \end{pmatrix} \), the geodesic in the Grassmannian is therefore
\begin{equation}
\label{eq:32}
\text{Exp}^{Gr}_P(t\Delta) = \pi^{OG} \left( Q \exp_m \left( t \begin{pmatrix} 0 & -B^T \\ B & 0 \end{pmatrix} \right) \right).
\end{equation}
where \(\pi^{OG}\) is the projection from \(O(n)\) to \(Gr(n, p)\).
If the thin SVD of \( B \) is given by  
\begin{equation*}
B = U_\perp^T \Delta^{\text{hor}}_U = U_\perp^T \hat{Q} \Sigma V^T
\end{equation*}
with \( W := U_\perp^T \hat{Q} \in St(n-p, r), \Sigma \in \mathbb{R}^{r \times r}, V \in St(p, r) \). Let \( W_\perp, V_\perp \) be suitable orthogonal completions. Then, 
\begin{equation*}
\exp_m \begin{pmatrix} 0 & -B^T \\ B & 0 \end{pmatrix} = \begin{pmatrix} V & V_\perp & 0 & 0 \\ 0 & 0 & W & W_\perp \end{pmatrix} \begin{pmatrix} \cos(\Sigma) & 0 & -\sin(\Sigma) & 0 \\ 0 & I_{p-r} & 0 & 0 \\ \sin(\Sigma) & 0 & \cos(\Sigma) & 0 \\ 0 & 0 & 0 & I_{n-p-r} \end{pmatrix} \begin{pmatrix} V^T & 0 \\ V_\perp^T & 0 \\ 0 & W^T \\ 0 & W_\perp^T \end{pmatrix},
\end{equation*}
which leads to the desired result when inserted into \eqref{eq:32}. For more mathematical details, you can refer to \citet{edelman1998geometryalgorithmsorthogonalityconstraints}, \citet{Bendokat_2024}, or other useful resources on Grassmann geometry.




\section{Time Complexity Analysis}
\label{appendix:time-complexity}

Table \ref{tab:time-subtrack-update-step} presents the time complexity breakdown for the subspace update step in the SubTrack-Grad algorithm assuming a $m \times n$ gradient matrix and rank $r$ projection matrix, where $r \ll m \leq n$. As outlined in Algorithm \ref{alg:SubTrack}, the subspace update step begins by solving the least squares problem (\ref{eq:loss-function}) to estimate the optimal update for $S_t$, the $m \times r$ orthonormal matrix. This operation has a time complexity of $O(mr^2)$. Computing the residual and the partial derivative with respect to $S_t$ requires $O(mrn)$ and $O(mnr)$ time respectively. This is because the solution to the least squares problem, $A$, has shape $ r \times n$ which is multiplied by $S_t$ in the residual $R=G_t - S_tA$, resulting in time complexity $O(mrn)$. The following operation for the partial derivative is $-2RA^T$, where the matrix multiplication has $O(mnr)$ complexity. The tangent vector computation (\ref{eq:tangent-vector}) which involves an identity transformation and matrix multiplication has time complexity of $O(m^2r)$.  The rank-1 approximation step uses largest singular value from the SVD of the $ m \times r$ tangent vector, and has time complexity of $O(mr^2)$. Finally, the update rule as shown in  (\ref{eq:update_rule}) which has a time complexity of $O(mr^2)$. The overall complexity of the algorithm is dominated by the matrix multiplication calculations of time complexity $O(mnr)$. However, unlike GaLore, since we avoid computing SVD operation on the $m \times n$ gradient matrix, which has complexity of $O(nm^2)$, the overall update step in SubTrack-Grad is still more efficient with respect to time complexity. 

\begin{table}[H]
\vskip 0.15in
\caption{\small Time Complexity for SubTrack-Grad Subspace Update}
\vskip 0.15in
\label{tab:time-subtrack-update-step}
\begin{center}

\begin{tabular}{l|c}
\toprule
\bf Computation Step  & \bf Time \\
\midrule
\midrule
Cost function  & $O(mr^2)$\\
\midrule
Residual & $O(mrn)$\\
\midrule
Partial derivative & $O(mnr)$\\
\midrule
Tangent vector $\Delta F$  & $O(m^2r)$\\
\midrule
Rank-$1$ approximation of $\Delta F$  & $O(mr^2)$\\
\midrule
Update rule   & $O(mr^2)$\\ 
\midrule
\midrule
\bf Overall  &  $O(mnr)$\\
\bottomrule 
\end{tabular}
\end{center}
\vskip -0.1in
\end{table}

\section{Memory and Time Comparison}
\label{appendix:mem-time}
Table \ref{tab:lama-time} presents the wall-times measured to compare the computational efficiency of SubTrack-Grad and GaLore. Additionally, Table \ref{tab:lama-mem} shows the peak memory consumption for each architecture during training, using the hyperparameters detailed in \ref{tab:pt_hyperparameters}, on an NVIDIA A100 GPU with 40GB of memory. These values were used to create the bar plots shown in \ref{fig2:a}.
For wall-time comparisons, each architecture was trained for 2,000 steps, ensuring exactly 10 subspace updates while excluding evaluation steps. The number of subspace updates was doubled compared to the fine-tuning experiments because larger models and more complex tasks naturally require more frequent updates. This adjustment allowed for more accurate measurements considering this demand. 
\begin{table*}[h]
\caption{\small Wall-time comparison for pre-training Llama-based architectures with varying model sizes on the C4 dataset. The experiments consist of 2000 iterations, corresponding to exactly 10 subspace update steps. The last two columns present the average percentage change in runtime relative to Full-Rank and GaLore, respectively.}
\label{tab:lama-time}
\begin{center}
\resizebox{\textwidth}{!}{%
\begin{tabular}{l|ccccc|c|c|c}
\toprule
 & \bf 60M & \bf 130M & \bf 350M & \bf 1B & \bf 3B  & \bf {\small Avg}& \bf {\small w.r.t FR}& \bf {\small w.r.t GaLore}  \\
  & r=128 & r=256 & r=256 & r=512 & r=512  &  &  & \\
\midrule
\midrule
{\bf Full-Rank}   
                    &524.0      &1035.1      &1396.4      
                    &974.9      &1055.9      &997.3 &- & -\\
\midrule
\midrule
{\bf BAdam}
                    &511.3      &779.2      &961.6      
                    &798.6      &1004.1     &811.0  &-18.68\%  &- \\
\midrule
{\bf GaLore}
                    &547.8      &1094.2      &1589.0      
                    &1729.5     &2715.5      &1535.2 & +53.94\% & - \\
\midrule
{\bf SubTrack-Grad}
                    &534.9      &1061.0      &1465.9       
                    &1191.1     &1385.7      &1127.7 & +13.08\% & -26.54\% \\
\bottomrule
\end{tabular}
}
\end{center}
\end{table*}

\begin{table*}[h]
\caption{\small Peak memory consumption comparison when pre-training Llama-based architectures with different sizes on C4 dataset. The last two columns present the average percentage change in memory consumption relative to Full-Rank and GaLore, respectively.}
\label{tab:lama-mem}
\begin{center}
\resizebox{\textwidth}{!}{%
\begin{tabular}{l|ccccc|c|c|c}
\toprule
 & \bf 60M & \bf 130M & \bf 350M & \bf 1B & \bf 3B  & \bf {\small Avg}& \bf {\small w.r.t FR} & \bf {\small w.r.t GaLore} \\
  & r=128 & r=256 & r=256 & r=512 & r=512  & &  &   \\
\midrule
\midrule
{\bf Full-Rank}   
                    &16.86 &25.32 &28.67 &18.83 &34.92 &24.92    & - & - \\  
\midrule
\midrule
{\bf BAdam}
                    &13.34 &20.01 &21.11 &9.70 &15.25 &15.88    &-36.28\%  & - \\   
\midrule
{\bf GaLore}
                    &16.89 &25.51 &27.85 &15.24 &26.03 &22.30    &-10.51\% & - \\  
\midrule
{\bf SubTrack-Grad}
                    &16.89 &25.52 &27.85 &15.24 &26.16 &22.33    &-10.39\% &+0.13\% \\ 
\bottomrule
\end{tabular}
}
\end{center}
\end{table*}


Table \ref{tab:glue-time} and Table \ref{tab:glue-mem} present the wall-time and peak memory consumption, respectively, for fine-tuning RoBERTa-Base on GLUE tasks. These values were used to generate the bar plots shown in \ref{fig2:c}. For wall-time comparisons, the same settings were used, but evaluation steps were excluded, and fine-tuning was limited to number of iterations to ensure exactly 5 subspace update steps during the process.
\begin{table*}[h]
\caption{\small Wall-time comparison when fine-tuning RoBERTa-Base on GLUE tasks for different ranks $r$ up to a number of iterations to exactly include 5 subspace update steps. The last two columns present the average percentage change in runtime relative to Full-Rank and GaLore, respectively.}
\label{tab:glue-time}
\begin{center}
\resizebox{\textwidth}{!}{%
\begin{tabular}{l|cccccccc|c|c|c}
\toprule
& \bf COLA & \bf STS-B & \bf MRPC & \bf RTE & \bf SST-2 & \bf MNLI & \bf QNLI & \bf QQP & \bf {\small Avg} & \bf {\small w.r.t FR} & \bf {\small w.r.t GaLore} \\ 
\midrule
\midrule
{\bf Full-Rank}         
                    &105.0      &124.0      &142.2      &260.1
                    &95.0       &146.3      &161.3      &119.0     &144.1 &- &-   \\
\midrule
\midrule
{\bf BAdam}         
                    &92.7       &103.3      &111.4      &177.9
                    &85.9       &116.0      &122.1      &99.1       &113.5  &-21.24\% &-   \\
\midrule
\midrule
{\bf GaLore} (r=4)    
                    &189.2      &194.7      &198.0      &310.7
                    &185.0      &205.7      &215.1      &187.3     &210.7 &+46.22\% &-   \\

\midrule
{\bf  SubTrack-Grad} (r=4)     
                    &145.2      &155.7      &168.3      &286.2
                    &141.2      &172.6      &189.9      &151.3     &176.3 &+22.48\% &-16.23   \\
\midrule
\midrule
{\bf GaLore} (r=8)     
                    &189.9      &195.2      &195.9      &312.3
                    &183.9      &205.1      &215.7      &187.6     &210.7 &+46.22\% &-   \\

\midrule
{\bf  SubTrack-Grad} (r=8)   
                    &145.2      &156.1      &168.4      &286.2
                    &143.2      &172.0      &188.9      &152.2     &176.6 &+22.35 &-16.33\%   \\
\bottomrule
\end{tabular}
}
\end{center}
\end{table*}

\begin{table*}[h]
\caption{\small Peak memory consumption when fine-tuning RoBERTa-Base on GLUE tasks for different ranks $r$. The last two columns present the average percentage change in memory consumption relative to Full-Rank and GaLore, respectively.}
\label{tab:glue-mem}
\begin{center}
\resizebox{\textwidth}{!}{%
\begin{tabular}{l|cccccccc|c|c|c}
\toprule
& \bf COLA & \bf STS-B & \bf MRPC & \bf RTE & \bf SST-2 & \bf MNLI & \bf QNLI & \bf QQP & \bf {\small Avg} & \bf {\small w.r.t FR} & \bf {\small w.r.t GaLore} \\ 
\midrule
\midrule
{\bf Full-Rank}         
                    &4.13  &4.24  &4.29  &8.16  
                    &3.88  &7.66  &9.50  &7.80 &6.21  & - & - \\    
\midrule
\midrule
{\bf BAdam}         
                    &2.12  &2.15  &2.30  &3.69  
                    &1.99  &2.13  &2.24  &2.84 &2.43  & -60.87\% & - \\    
\midrule
\midrule
{\bf GaLore} (r=4)    
                    &3.64  &4.12  &3.85  &7.27  
                    &3.52  &6.84  &8.75  &6.58 &5.57  & -10.31\% & - \\     

\midrule
{\bf SubTrack-Grad} (r=4)     
                    &3.64  &4.12  &3.86  &7.27  
                    &3.37  &6.73  &8.75  &6.58 &5.54  &-10.79\% & -0.54\% \\     
\midrule
\midrule
{\bf GaLore} (r=8)     
                    &3.64  &4.12  &3.86  &7.27  
                    &3.53  &6.84  &8.75  &6.58 &5.57  & -10.31\% & - \\      
\midrule
{\bf SubTrack-Grad} (r=8)   
                    &3.65  &4.12  &3.86  &7.27  
                    &3.38  &6.74  &8.75  &6.58 &5.54  & -10.79\% & -0.54\% \\     
\bottomrule
\end{tabular}
}
\end{center}
\end{table*}


Table \ref{tab:superglue-time} and Table \ref{tab:superglue-mem} present the wall-time and peak memory consumption for each method when fine-tuning RoBERTa-Large on SuperGLUE tasks. These values were used to generate the bar plots shown in \ref{fig2:b}. For wall-time comparisons, as with GLUE tasks, evaluation steps were omitted to ensure accurate measurements of fine-tuning time, and the number of iterations was adjusted to include exactly 5 subspace updates during fine-tuning.
\begin{table*}[h]
\caption{\small Wall-time comparison when fine-tuning RoBERTa-Large on SuperGLUEup to a number of iterations to exactly include 5 subspace update steps. The last two columns present the average percentage change in runtime relative to Full-Rank and GaLore, respectively.}
\label{tab:superglue-time}
\begin{center}
\resizebox{\textwidth}{!}{%
\begin{tabular}{l|cccccc|c|c|c}
\toprule
 & \bf BoolQ & \bf CB & \bf COPA & \bf WIC & \bf WSC  & \bf AX$_g$ & \bf {\small Avg} & \bf {\small w.r.t FR} & \bf {\small w.r.t GaLore} \\ 
\midrule
\midrule
{\bf Full-Rank}   
                    &1406.3     &171.5      &35.1       
                    &250.4      &216.4      &52.1 &355.3   & - & - \\    
\midrule
\midrule
{\bf BAdam}
                    &794.9     &93.5      &45.9      
                    &163.9      &137.2      &38.1 &212.25  &-40.26\% & - \\  
\midrule
{\bf GaLore} (r=8)
                    &1536.8     &272.2      &235.3      
                    &483.0      &393.2      &219.9 &523.4  &+47.31\%  & - \\  
\midrule
{\bf SubTrack-Grad} (r=8)
                    &1468.0     &198.1      &82.2      
                    &326.1      &266.7      &86.6 &404.62   &+13.88\% &-22.69\% \\   
\bottomrule
\end{tabular}
}
\end{center}
\end{table*}

\begin{table*}[h]
\caption{\small Peak memory consumption comparison when fine-tuning RoBERTa-Large on SuperGLUE. The last two columns present the average percentage change in memory consumption relative to Full-Rank and GaLore, respectively.}
\label{tab:superglue-mem}
\begin{center}
\resizebox{\textwidth}{!}{%
\begin{tabular}{l|cccccc|c|c|c}
\toprule
 & \bf BoolQ & \bf CB & \bf COPA & \bf WIC & \bf WSC &  \bf AX$_g$ & \bf {\small Avg} & \bf {\small w.r.t FR} & \bf {\small w.r.t GaLore} \\ 
\midrule
\midrule
{\bf Full-Rank}   
                    &20.16  &15.15  &7.86    
                    &7.96   &9.11    &7.81 &11.37  & - & - \\    
                    % &11.37 \\
\midrule
\midrule
{\bf BAdam}
                    &10.80  &5.78  &3.36    
                    &3.26  &4.57  &3.26  &5.17  &-54.52\%  & - \\      
                    % &9.06 \\
\midrule
{\bf GaLore} (r=8)
                    &19.08 &12.53  &5.32    
                    &5.94  &7.26   &5.34  &9.24 & -18.73\% & - \\      
                    % &9.06 \\
\midrule
{\bf SubTrack-Grad} (r=8)
                    &19.09 &12.40  &5.12    
                    &5.95  &7.06     &5.35  &9.16 & -19.44\% & -0.87\% \\      
                    % &9.00 \\
\bottomrule
\end{tabular}
}
\end{center}
\end{table*}


\section{Fine-Tuning RoBERTa}
\label{appendix:A}
RoBERTa-Base was fine-tuned on GLUE tasks using the hyperparameters detailed in Table \ref{tab:ft_hyperparameters}, matching those reported in the GaLore \citep{zhao2024galorememoryefficientllmtraining} for rank-\(4\) and rank-\(8\) subspaces, with a subspace update interval set at 500 iterations.
\begin{table*}[h]
    \caption{\small Hyperparameters of fine-tuning RoBERTa-Base on GLUE tasks.}
    \centering
    \label{tab:ft_hyperparameters}
    \resizebox{\textwidth}{!}{%
    \begin{tabular}{l|c|cccccccc}
    \midrule
    \midrule
    & & MNLI   & SST-2 & MRPC    & CoLA    & QNLI    & QQP     & RTE     & STS-B   \\
    \midrule
    \midrule
    % \midrule
    Shared Parameters & Batch Size    & 16     & 16    & 16      & 32      & 16      & 16      & 16      & 16      \\
    & Learning Rate & 1E-05  & 1E-05     & 3E-05   & 3E-05   & 1E-05   & 1E-05   & 1E-05   & 1E-05   \\
    & \# Epochs   &  \multicolumn{8}{c}{30} \\
    & Max Seq. Len. &  \multicolumn{8}{c}{512} \\

    \midrule
    SubTrack-Grad & SubTrack-Grad Step-Size &0.1 &0.001 &5.0 &5.0 &0.01 &1.0 &8.0 &10.0   \\
    \& GaLore, r = 4 & Subspace Update Interval &  \multicolumn{8}{c}{500} \\
    Parameters & \(\alpha\) &  \multicolumn{8}{c}{4} \\
    \midrule
    SubTrack-Grad & SubTrack-Grad Step-Size &0.1 &0.1 &5.0 &13.0 &0.1 &1.0 &5.0 &10.0   \\
    \& GaLore, r = 8 & Subspace Update Interval &  \multicolumn{8}{c}{500} \\
    Parameters & \(\alpha\) &  \multicolumn{8}{c}{2} \\
    \midrule
    BAdam Parameters & Block Switch Interval &  \multicolumn{8}{c}{100} \\
    & Switch Mode &  \multicolumn{8}{c}{Random} \\
    \bottomrule
    \end{tabular}
    }
\end{table*}


We also fine-tuned RoBERTa-Large on SuperGLUE tasks using the hyperparameters from \citet{luo2024badammemoryefficientparameter}, as detailed in Table \ref{tab:sg_hyperparameters}, with the exception that we fine-tuned each task for 30 epochs.
\begin{table*}[h]
    \caption{Hyperparameters of fine-tuning RoBERTa-Large on SuperGLUE tasks.}
    
    \centering
    \label{tab:sg_hyperparameters}
    \resizebox{\textwidth}{!}{%
    \begin{tabular}{l|c|ccccccc}
    \midrule
    \midrule
    & & BoolQ   & CB & COPA    & WIC    & WSC     & AX$_g$   \\
    \midrule
    \midrule
    Shared Parameters & Batch Size    & \multicolumn{6}{c}{16}  \\
    & \# Epochs     & \multicolumn{6}{c}{30}  \\
    & Learning Rate & \multicolumn{6}{c}{1e-5}  \\
    & Max Seq. Len. & \multicolumn{6}{c}{512}  \\
    \midrule
    SubTrack-Grad \& & SubTrack-Grad Step-Size &0.1      &1.0     &50.0       &50.0       &1       &1.0   \\
    GaLore Parameters & Subspace Update Interval &500      &100     &100       &500       &250       &100   \\
    & Rank Config. & \multicolumn{6}{c}{8}  \\
    & \(\alpha\) & \multicolumn{6}{c}{4}  \\
    \midrule
    BAdam Parameters & Block Switch Interval &100     &50       &50       &100       &50       &50  \\
    & Switch Mode & \multicolumn{6}{c}{Random}  \\
    \bottomrule
    \end{tabular}
    }
\end{table*}


\section{Pre-Training LLama-Based Architectures}
\label{appendix:B}
To manage resources, we pre-trained all five Llama-based architectures for 10,000 iterations using hyperparameters reported in Table \ref{tab:pt_hyperparameters}. While larger models typically require more iterations, this setup was sufficient for comparing methods rather than creating perfectly optimized models. 
\begin{table*}[h]
    \caption{Hyperparameters of pre-training Llama-based architectures.}
    \centering
    \label{tab:pt_hyperparameters}
    \begin{tabular}{l|c|ccccc}
    \midrule
    \midrule
    & & 60M   & 130M & 350M & 1B & 3B   \\
    \midrule
    \midrule
    Architectural & Hidden        &512  &768  &1024 &2048 &4096 \\
    Parameters &Intermediate  &1376 &2048 &2736 &5461 &11008 \\
    &Heads         &8    &12   &16 &24 &32 \\
    &Layers        &8    &12   &24 &32 &32 \\
    \midrule
    Shared Parameters & Learning Rate &1e-3  &1e-3  &1e-3 &1e-5 &1e-5   \\
    & Batch Size    &128  &128  &64 &8 &8   \\
    & Iterations & \multicolumn{5}{c}{10K} \\
    & Gradient Accumulation & \multicolumn{5}{c}{2} \\
    & Gradient Clipping & \multicolumn{5}{c}{1.0} \\
    & Warmup Steps & \multicolumn{5}{c}{1000} \\
    & scale & \multicolumn{5}{c}{0.25} \\
    & dtype & \multicolumn{5}{c}{bfloat16} \\
    \midrule
    SubTrack \& &Rank          &128  &256  &256 &512 &512   \\
    GaLore Parameters & Subspace Update Interval & \multicolumn{5}{c}{200} \\
    &SubTrack-Grad Step-Size & \multicolumn{5}{c}{10} \\
    \midrule
    BAdam Parameters & Block Swithch Interval & \multicolumn{5}{c}{100} \\
    & Switch Mode & \multicolumn{5}{c}{Random} \\
    \bottomrule
    \end{tabular}
\end{table*}


\section{Time and Performance Consistency}
\label{appendix:time-consistency}
Table \ref{tab:time-perf-consistency} demonstrates that increasing the update frequency significantly increases GaLore's runtime, while the performance of both methods remains comparable. This underscores SubTrack-Grad's efficiency in reducing runtime without sacrificing performance. Notably, this advantage becomes even more pronounced for tasks that demand more frequent updates.
\begin{table*}[h]
\caption{\small Comparing changes in performance while decreasing the subspace update interval (increasing the frequency of subspace updates) on fine-tuning RoBERTa-Base on COLA task.}
\label{tab:time-perf-consistency}
\begin{center}
\resizebox{\textwidth}{!}{%
\begin{tabular}{l|cccccccccc}
\toprule
 & \bf 500 & \bf 450 & \bf 400 & \bf 350 & \bf 300 & \bf 250 & \bf 200 & \bf 150 & \bf 100 & \bf 50 \\ 
\midrule
\midrule
{\bf GaLore's Wall-Time}  & 
                    458.79& 460.19& 471.36& 483.04& 495.83& 520.28& 556.14& 605.1& 715.87& 1038.79 \\
{\bf GaLore's Performance}  & 
                    0.5358& 0.5285& 0.5385& 0.5356& 0.5309& 0.54821& 0.5332& 0.5387& 0.5463& 0.5463 \\ 
\midrule
{\bf SubTrack-Grad's Wall-Time}&   
                    408.22& 408.44& 408.47& 409.25& 409.17& 409.76& 410.29& 411.77& 413.35& 417.07 \\ 
{\bf SubTrack-Grad's Performance}  & 
                    0.5332& 0.5382& 0.5364& 0.5385& 0.5354& 0.5364& 0.5416& 0.5385& 0.5364& 0.5673 \\ 
\bottomrule
\end{tabular}
}
\end{center}
\end{table*}


\section{Convergence vs. Wall-Time}
\label{appendix:perf-wt}
Figure \ref{fig:pert-wt} depicts the changes in the training loss function relative to wall-time. The results demonstrate that SubTrack-Grad's wall-time reduction has minimal impact on the learning process, showcasing a convergence pattern comparable to other methods without introducing significant computational overhead.
\begin{figure}[h]
    \centering
    \includegraphics[width=\textwidth]{images/perf_wt.pdf}
    \caption{\small{The figures present changes in training loss relative to wall-time for pre-training the Llama-3B architecture on the C4 dataset and fine-tuning RoBERTa-Large on the BoolQ dataset. They demonstrate that SubTrack maintains the overall learning process for both pre-training and fine-tuning while avoiding significant computational overhead.}}
    \label{fig:pert-wt}
\vspace{-3mm}
\end{figure}


