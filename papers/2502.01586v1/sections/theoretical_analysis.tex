\section{Theoretical Analysis}
\label{sec:theory}
In this section, we analyze the convergence of SubTrack-Grad using theoretical analysis.  To begin, the weights update rule of SubTrack-Grad is as follows:
\begin{equation}
\small
    W_t = W_0 + \sum_{t'=0}^{t'=t-1} \widehat{G}_{t'}
    \label{eq:update_rule}
\end{equation}
As previously mentioned, we use left projection if \(m \leq n\), where \(m\) and \(n\) are the dimensions of the gradient matrix, and vice versa. Thus, \(\widehat{G}_{t'}\) can be computed as shown in \eqref{eq:project_back}.
\begin{equation}
\small
    \widehat{G}_{t'} = 
    \begin{cases}
        S_{t'} \rho_{t'} (S_{t'}^\top G_{t'}), & \text{if \(m \leq n\)} \\
        \rho_{t'} (G_{t'} S_{t'})S_{t'}^\top, & \text{otherwise}
    \end{cases}
    \label{eq:project_back}
\end{equation}
Here, \(S_{t'}\) is the projection matrix that projects the gradient onto the subspace, and \(\rho_{t'}\) is representing the entry-wise regularizer used in the optimizer. If we use the full projection, then \(\widehat{G}_{t'}\) will be computed as shown in \eqref{eq:full_project_back}; where \(S_{t'}^l\) and \(S_{t'}^r\) are the rank-\(r\) left and right projection matrices.
\begin{equation}
\small
    \widehat{G}_{t'} = S_{t'}^l \rho_{t'} ({S_{t'}^l}^\top G_{t'} S_{t'}^r) {S_{t'}^r}^\top
    \label{eq:full_project_back}
\end{equation}

\subsection{{Local boundedness and Lipschitz  properties of $\ell_{in}$, $\ell_{out}$, and their derivatives }}

\begin{proposition}[{Local} boundedness]\label{prop:uniform_boundedness}
Under \cref{assump:compact,assump:convexity_lin,assump:K_bounded,assump:reg_lin_lout}, the functions $(\omega,x,y)\mapsto \ell_{out}(\omega,h_{\omega}^{\star}(x),y)$, $(\omega,x,y)\mapsto \partial_{\omega}\ell_{out}(\omega,h_{\omega}^{\star}(x),y)$, and $(\omega,x,y)\mapsto\partial_v \ell_{out}(\omega, h^\star_\omega(x), y)$ are bounded over $\textnormal{hull}(\Omega)\times\mathcal{X}\times\mathcal{Y}$ by some positive constant $M_{out}$. Similarly, the functions $(\omega,x,y)\mapsto\partial_v \ell_{in}(\omega, h^\star_\omega(x), y)$, $(\omega,x,y)\mapsto\partial_v^2 \ell_{in}(\omega, h^\star_\omega(x), y)$, and $(\omega,x,y)\mapsto\partial_{\omega, v}^2 \ell_{in}(\omega, h^\star_\omega(x), y)$  are bounded over $\textnormal{hull}(\Omega)\times\mathcal{X}\times\mathcal{Y}$ by some positive constant $M_{in}$. 
The constants $M_{out}$ and $M_{in}$ are defined as:
\begin{align*}
M_{out}&\coloneqq\sup_{\omega\in\textnormal{hull}(\Omega),v\in\mathcal{V},y\in\mathcal{Y}}\max\left(\verts{\ell_{out}(\omega,v,y)},\Verts{\partial_{\omega}\ell_{out}(\omega,v,y)},\verts{\partial_v\ell_{out}(\omega, v, y)}\right)>0,\\
M_{in}&\coloneqq\sup_{\omega\in\textnormal{hull}(\Omega),v\in\mathcal{V},y\in\mathcal{Y}}\max\left(\verts{\partial_v\ell_{in}(\omega, v, y)}, \verts{\partial_v^2\ell_{in}(\omega, v, y)},\Verts{\partial_{\omega, v}^2\ell_{in}(\omega, v, y)}\right)>0,
\end{align*}
where $\mathcal{V}\subset\mathbb{R}$ is the compact interval defined in \cref{prop:bound_hstaromega}.
\end{proposition}

\begin{proof}
By \cref{prop:bound_hstaromega}, we have that $h^\star_\omega(x)\in\mathcal{V}\coloneqq\left[-\frac{B\kappa}{\lambda},\frac{B\kappa}{\lambda}\right]\subset\mathbb{R}$, for any $x\in\mathcal{X}$. From \cref{assump:reg_lin_lout}, we know that $\ell_{in}$, $\ell_{out}$, and their partial derivatives are all continuous on $\text{hull}(\Omega)\times\mathcal{V}\times\mathcal{Y}$. Also, $\mathcal{Y}$ is compact by \cref{assump:compact}. Thus, $\text{hull}(\Omega)\times\mathcal{V}\times\mathcal{Y}$ is compact. As every continuous function defined over a compact space is bounded, we obtain that:
\begin{align*}
    \sup_{\omega\in\text{hull}(\Omega),x\in\mathcal{X},y\in\mathcal{Y}}\verts{\ell_{out}(\omega,h_{\omega}^{\star}(x),y)}&\leq\sup_{\omega\in\text{hull}(\Omega),v\in\mathcal{V},y\in\mathcal{Y}}\verts{\ell_{out}(\omega,v,y)}<+\infty,\\
    \sup_{\omega\in\text{hull}(\Omega),x\in\mathcal{X},y\in\mathcal{Y}}\Verts{\partial_{\bullet} \ell_{\circ}(\omega,h_{\omega}^{\star}(x),y)}&\leq\sup_{\omega\in\text{hull}(\Omega),v\in\mathcal{V},y\in\mathcal{Y}}\Verts{\partial_{\bullet} \ell_{\circ}(\omega,v,y)}<+\infty,
\end{align*}
where $\bullet \in \{\{v\}, \{w\}, \{w,v\}\}$ and $\circ \in \{in,out\}$. This implies the desired result.
\end{proof}

\begin{proposition}[Local Lipschitz continuity]\label{prop:uniform_Lipschitzness}
	Under \cref{assump:compact,assump:convexity_lin,assump:K_bounded,assump:reg_lin_lout}, there exists a positive constant $\lipout$ so that for any $(x,y)$ in $\mathcal{X}\times \mathcal{Y}$, 
	the functions $\omega\mapsto \ell_{out}(\omega,h_{\omega}^{\star}(x),y)$, $\omega\mapsto \partial_{\omega}\ell_{out}(\omega,h_{\omega}^{\star}(x),y)$, and $\omega\mapsto\partial_v \ell_{out}(\omega, h^\star_\omega(x), y)$ are locally $\frac{\lipout}{\lambda}$-Lipschitz continuous over $\textnormal{hull}(\Omega)$. Similarly, there exists a positive constant $\lipin$ so that for any $(x,y)$ in $\mathcal{X}\times \mathcal{Y}$, the functions $\omega\mapsto\partial_v \ell_{in}(\omega, h^\star_\omega(x), y)$, $\omega\mapsto\partial_v^2 \ell_{in}(\omega, h^\star_\omega(x), y)$, and $\omega\mapsto\partial_{\omega, v}^2 \ell_{in}(\omega, h^\star_\omega(x), y)$ are locally $\frac{\lipin}{\lambda}$-Lipschitz continuous $\textnormal{hull}(\Omega)$.  The constants $\lipout$ and $\lipin$ are defined, for any $0<\lambda\leq \Lambda$, as:
	\begin{align*}
	    \lipout&\coloneqq\left(\Lambda+M_{in}\kappa\right)\max\left(M_{out},\bar{M}_{out}\right)>0\\
	    \lipin&\coloneqq\left(\Lambda+M_{in}\kappa\right)\max\left(M_{in},\bar{M}_{in}\right)>0,
    \end{align*}
    where:
    \begin{align*}
      \bar{M}_{out}&\coloneqq\sup_{\omega\in\textnormal{hull}(\Omega), v\in\mathcal{V}, y\in\mathcal{Y}}\max\left(\Verts{\partial_\omega^2\ell_{out}(\omega, v, y)}_{\op},\Verts{\partial_{\omega, v}^2\ell_{out}(\omega, v, y)},\verts{\partial_v^2\ell_{out}(\omega, v, y)}\right)>0,\\
      \bar{M}_{in}&\coloneqq\sup_{\omega\in\textnormal{hull}(\Omega), v\in\mathcal{V}, y\in\mathcal{Y}}\max\left(\Verts{\partial_\omega\partial_v^2\ell_{in}(\omega, v, y)},\verts{\partial_v^3\ell_{in}(\omega, v, y)},\Verts{\partial_\omega\partial_{\omega, v}^2\ell_{in}(\omega, v, y)}\right)>0,
    \end{align*}
    with $M_{in}$ and $M_{out}$ being the positive constants defined in \cref{prop:uniform_boundedness}, and $\mathcal{V}\subset\mathbb{R}$ is the compact interval defined in \cref{prop:bound_hstaromega}.
\end{proposition}


\begin{proof}
For any $(\omega,x,y)\in\text{hull}(\Omega)\times \mathcal{X}\times \mathcal{Y}$, we have:
\begin{align*}
    \Verts{\nabla_\omega\ell_{out}(\omega, h^\star_\omega(x), y)}&=\Verts{\partial_\omega\ell_{out}(\omega, h^\star_\omega(x), y)+\partial_v\ell_{out}(\omega, h^\star_\omega(x), y)\partial_\omega h^\star_\omega(x)}\\
    &\leq\Verts{\partial_\omega\ell_{out}(\omega, h^\star_\omega(x), y)}+\verts{\partial_v\ell_{out}(\omega, h^\star_\omega(x), y)}\Verts{\partial_\omega h^\star_\omega}_{\op}\Verts{K(x,\cdot)}_\mathcal{H}\\
    &\leq M_{out}\left(1+\frac{M_{in}\kappa}{\lambda}\right)\\
    &\leq\frac{M_{out}\left(\Lambda+M_{in}\kappa\right)}{\lambda},
\end{align*}
where the first line uses the chain rule, the second line applies the triangle inequality and the reproducing property of the RKHS $\mathcal{H}$, the third line follows from \cref{prop:uniform_boundedness} to bound the derivatives of $\ell_{out}$, from \cref{prop:lip_hstaromega}, which states that the function $\omega\mapsto h^\star_\omega$ is $\frac{L\sqrt{\kappa}}{\lambda}$-Lipschitz continuous with $L\coloneqq\sup_{\omega\in\text{hull}(\Omega),v\in\mathcal{V},y\in\mathcal{Y}}\left\|\partial_{\omega, v}^2 \ell_{in}(\omega, v, y)\right\|<M_{in}$, to bound $\Verts{\partial_\omega h^\star_\omega}_{\op}$, and from \cref{assump:K_bounded} to bound $\Verts{K(x,\cdot)}_\mathcal{H}$, and the last line is a direct consequence of $0<\lambda\leq \Lambda$. In a similar way, we obtain:
\begin{gather*}
    \Verts{\nabla_\omega\partial_\omega\ell_{out}(\omega, h^\star_\omega(x), y)}_{\op}\leq\frac{\bar{M}_{out}\left(\Lambda+M_{in}\kappa\right)}{\lambda},\quad\Verts{\nabla_\omega\partial_v \ell_{out}(\omega, h^\star_\omega(x), y)}\leq\frac{\bar{M}_{out}\left(\Lambda+M_{in}\kappa\right)}{\lambda},\\
    \Verts{\nabla_\omega\partial_v \ell_{in}(\omega, h^\star_\omega(x), y)}\leq\frac{M_{in}\left(\Lambda+M_{in}\kappa\right)}{\lambda},\quad\Verts{\nabla_\omega\partial_v^2 \ell_{in}(\omega, h^\star_\omega(x), y)}\leq\frac{\bar{M}_{in}\left(\Lambda+M_{in}\kappa\right)}{\lambda},\\
    \Verts{\nabla_\omega\partial_{\omega, v}^2 \ell_{in}(\omega, h^\star_\omega(x), y)}_{\op}\leq\frac{\bar{M}_{in}\left(\Lambda+M_{in}\kappa\right)}{\lambda}.
\end{gather*}
Combining all these bounds concludes the proof.
\end{proof}

\section{Generalization Properties}\label{app:sec_conv}

{As before, let $\Omega$ be an arbitrary compact subset of $\mathbb{R}^d$.}

\subsection{Point-wise estimates}\label{app:subsec_point_est}
We present a point-wise upper-bound on the value error $\verts{\mathcal{F}(\omega)-\widehat{\mathcal{F}}(\omega) }$ and gradient error $\Verts{\nabla\mathcal{F}(\omega)-\widehat{\nabla\mathcal{F}}(\omega) }$. To this end, we introduce the following notation for the error between the inner and outer objectives and their empirical approximations evaluated at the optimal inner solution $h_{\omega}^{\star}$: 
\begin{align*}
    \Dout\coloneqq \verts{L_{out}(\omega, h^\star_\omega)-\widehat{L}_{out}(\omega, h^\star_\omega)}, 
    \qquad 
    \Din \coloneqq \verts{L_{in}(\omega, h^\star_\omega)-\widehat{L}_{in}(\omega, h^\star_\omega)}.
\end{align*}
By abuse of notation, we introduce the following error between  partial derivatives of $L_{in}$ and $\widehat{L}_{in}$ (resp. $L_{out}$ and $\widehat{L}_{out}$), evaluated at $(\omega, h_{\omega}^{\star})$, \textit{i.e.},
\begin{align*}
  \Douth &\coloneqq \Verts{\partial_h L_{out}(\omega, h^\star_\omega)-\partial_h\widehat{L}_{out}(\omega, h^\star_\omega)}_{\mathcal{H}},\quad 
  &\Doutw &\coloneqq \Verts{\partial_\omega L_{out}(\omega, h^\star_\omega)-\partial_\omega\widehat{L}_{out}(\omega, h^\star_\omega)},\\
  \Dinh &\coloneqq \Verts{\partial_h L_{in}(\omega, h^\star_\omega)-\partial_h\widehat{L}_{in}(\omega, h^\star_\omega)}_{\mathcal{H}},\quad 
  &\Dinw &\coloneqq \Verts{\partial_\omega L_{in}(\omega, h^\star_\omega)-\partial_\omega\widehat{L}_{in}(\omega, h^\star_\omega)},\\
  \Dinhh &\coloneqq \Verts{\partial_{h}^2 L_{in}(\omega, h^\star_\omega)-\partial_h^2\widehat{L}_{in}(\omega, h^\star_\omega)}_{\op},\quad 
  &\Dinwh &\coloneqq \Verts{\partial_{\omega,h}^2 L_{in}(\omega, h^\star_\omega)-\partial_{\omega,h}^2\widehat{L}_{in}(\omega, h^\star_\omega)}_{\op}.
\end{align*}

\begin{proposition}\label{prop:diff_hstar_hhat}
    Under \cref{assump:compact,assump:convexity_lin,assump:K_bounded,assump:reg_lin_lout}, the following holds for any $\omega\in\Omega$:
    \begin{equation*}
        \left\|h^\star_\omega-\hat{h}_\omega\right\|_\mathcal{H}\leq\frac{1}{\lambda}\left\| \partial_h\widehat{L}_{in}(\omega, h^\star_\omega)\right\|_\mathcal{H} = \frac{1}{\lambda}\Dinh.
    \end{equation*}
\end{proposition}
\begin{proof}
Let $\omega\in\Omega$. 
The function $h\mapsto\widehat{L}_{in}(\omega, h)$ is $\lambda$-strongly convex and Fr\'echet differentiable by  \cref{prop:strong_convexity_Lin,prop:fre_diff_L}. Moreover,  $\hat{h}_{\omega}$ is the minimizer of $h\mapsto\widehat{L}_{in}(\omega, h)$ by definition. 
Therefore, using \cref{lem:h_min_hstar}, we obtain a control on the distance in $\mathcal{H}$ to the optimum $\hat{h}_{\omega}$ of $h\mapsto\widehat{L}_{in}(\omega,h)$ in terms of the gradient $\partial_{h}\widehat{L}_{in}(\omega,h)$:
\begin{align*}
	\left\|h-\hat{h}_\omega\right\|_\mathcal{H}\leq\frac{1}{\lambda}\Verts{\partial_h\widehat{L}_{in}(\omega, h)}_\mathcal{H}, \qquad \forall h\in \mathcal{H}.
\end{align*}
In particular, choosing $h= h^\star_\omega$ yields the first inequality. The fact that $\Verts{\partial_h\widehat{L}_{in}(\omega, h_{\omega}^{\star})}_\mathcal{H}=\Dinh$ follows from the optimality of $h_{\omega}^{\star}$ which implies that  $\partial_h L_{in}(\omega, h_{\omega}^{\star})=0$. 
\end{proof}

\begin{proposition}\label{prop:lip_continuity_out}
Under \cref{assump:compact,assump:convexity_lin,assump:K_bounded,assump:reg_lin_lout}, the following inequalities hold for any $\omega\in\Omega$:
\begin{align*}
	\Eout &\coloneqq\verts{\widehat{L}_{out}(\omega, h^\star_\omega)-\widehat{L}_{out}(\omega, \hat{h}_\omega)}\leq C_{out}\Verts{h^\star_\omega-\hat{h}_\omega}_\mathcal{H},\\
	\Eouth&\coloneqq\Verts{\partial_h\widehat{L}_{out}(\omega, h^\star_\omega)-\partial_h\widehat{L}_{out}(\omega, \hat{h}_\omega)}_\mathcal{H} \leq C_{out}\Verts{h^\star_\omega-\hat{h}_\omega}_\mathcal{H},\\
	\Eoutw&\coloneqq\Verts{\partial_\omega\widehat{L}_{out}(\omega, h^\star_\omega)-\partial_\omega\widehat{L}_{out}(\omega, \hat{h}_\omega)} \leq C_{out}\Verts{h^\star_\omega-\hat{h}_\omega}_\mathcal{H},\\
	\Einhh&\coloneqq\Verts{\partial_h^2\widehat{L}_{in}(\omega, h^\star_\omega)-\partial_h^2\widehat{L}_{in}(\omega, \hat{h}_\omega)}_{\op}\leq C_{in}\Verts{h^\star_\omega-\hat{h}_\omega}_\mathcal{H},\\
	\Einwh&\coloneqq\Verts{\partial_{\omega, h}^2\widehat{L}_{in}(\omega, h^\star_\omega)-\partial_{\omega, h}^2\widehat{L}_{in}(\omega, \hat{h}_\omega)}_{\op} \leq C_{in}\Verts{h^\star_\omega-\hat{h}_\omega}_\mathcal{H}.
\end{align*}
The positive constants $C_{out}$ and $C_{in}$ are defined as:
\begin{align*}
    C_{out}&\coloneqq\max\left(M_{out}\sqrt{\kappa},\bar{M}_{out}\kappa,\bar{M}_{out}\sqrt{\kappa}\right)>0,\\
    C_{in}&\coloneqq\max\left(\bar{M}_{in}\kappa\sqrt{\kappa},\bar{M}_{in}\kappa,M_{in}\sqrt{d\kappa}\right)>0,
\end{align*}
where $M_{out}$, $\bar{M}_{out}$, and $\bar{M}_{in}$ are the positive constants defined in \cref{prop:uniform_boundedness,prop:uniform_Lipschitzness}.
\end{proposition}

\begin{proof}
\textbf{Lipschitz continuity of some functions of interest. }Let $\omega\in\Omega$. According to \cref{prop:bound_hstaromega}, both $h^\star_\omega(x)$ and $\hat{h}_\omega(x)$ lie in the compact interval $\mathcal{V}\coloneqq\left[-\frac{B\kappa}{\lambda},\frac{B\kappa}{\lambda}\right]\subset\mathbb{R}$, for any $x\in\mathcal{X}$, where $B\coloneqq\sup_{\omega\in\text{hull}(\Omega),y\in\mathcal{Y}}\left|\partial_v \ell_{in}(\omega, 0, y)\right|>0$. By \cref{assump:compact}, $\mathcal{Y}$ is a compact set. Hence $\Omega\times\mathcal{V}\times\mathcal{Y}$ is a compact set as well. Furthermore, by \cref{assump:reg_lin_lout}, $(\omega, v, y)\mapsto\ell_{in}(\omega, v, y)$, $(\omega, v, y)\mapsto\ell_{out}(\omega, v, y)$, and their derivatives are all continuous over the compact domain $\Omega\times\mathcal{V}\times\mathcal{Y}$. Therefore, these functions and their derivatives are bounded on this domain. In particular, this also holds when $v$ takes the specific values $h^\star_\omega(x)$ or $\hat{h}_\omega(x)$. Let $\bar{v}$ be either $h^\star_\omega(x)$ or $\hat{h}_\omega(x)$, for any $x\in\mathcal{X}$. For any $\omega\in\Omega$, and $y\in\mathcal{Y}$, we have:
\begin{align*}
    \verts{\partial_v\ell_{out}(\omega, \bar{v}, y)}&\leq\sup_{\omega\in\Omega,v\in\mathcal{V},y\in\mathcal{Y}}\verts{\partial_v\ell_{out}(\omega, v, y)}\leq M_{out}<+\infty,\\
   \verts{\partial_v^2\ell_{out}(\omega, \bar{v}, y)}&\leq\sup_{\omega\in\Omega,v\in\mathcal{V},y\in\mathcal{Y}}\verts{\partial_v^2\ell_{out}(\omega, v, y)}\leq\bar{M}_{out}<+\infty,\\
   \Verts{\partial_{\omega, v}^2\ell_{out}(\omega, \bar{v}, y)}&\leq\sup_{\omega\in\Omega,v\in\mathcal{V},y\in\mathcal{Y}}\Verts{\partial_{\omega, v}^2\ell_{out}(\omega, v, y)}\leq\bar{M}_{out}<+\infty,\\
   \verts{\partial_v^3\ell_{in}(\omega, \bar{v}, y)}&\leq\sup_{\omega\in\Omega,v\in\mathcal{V},y\in\mathcal{Y}}\verts{\partial_v^3\ell_{in}(\omega, v, y)}\leq\bar{M}_{in}<+\infty,\\
   \Verts{\partial_v\partial_{\omega, v}^2\ell_{in}(\omega, \bar{v}, y)}_{\op}&\leq\sup_{\omega\in\Omega,v\in\mathcal{V},y\in\mathcal{Y}}\Verts{\partial_\omega\partial_v^2\ell_{in}(\omega, v, y)}\leq\bar{M}_{in}<+\infty
\end{align*}
This means that $v\in\mathcal{V}\mapsto\ell_{out}(\omega, v, y)$, $v\in\mathcal{V}\mapsto\partial_v\ell_{out}(\omega, v, y)$, $v\in\mathcal{V}\mapsto\partial_\omega\ell_{out}(\omega, v, y)$, $v\in\mathcal{V}\mapsto\partial_v^2\ell_{in}(\omega, v, y)$, and $v\in\mathcal{V}\mapsto\partial_{\omega, v}^2\ell_{in}(\omega, v, y)$ are Lipschitz continuous, with Lipschitz constants $M_{out}$, $\bar{M}_{out}$, $\bar{M}_{out}$, $\bar{M}_{in}$, and $\bar{M}_{in}$, respectively, for any $\omega\in\Omega$ and $y\in\mathcal{Y}$.

\textbf{Upper-bounds. }We have:
\begin{align*}
    \Eout\coloneqq\verts{\widehat{L}_{out}(\omega, h^\star_\omega)-\widehat{L}_{out}(\omega, \hat{h}_\omega)}&=\verts{\frac{1}{m}\sum_{j=1}^m\ell_{out}(\omega, h^\star_\omega(\tilde{x}_j), \tilde{y}_j) - \frac{1}{m}\sum_{j=1}^m\ell_{out}(\omega, \hat{h}_\omega(\tilde{x}_j), \tilde{y}_j)}\\
    &\leq\frac{1}{m}\sum_{j=1}^m\verts{\ell_{out}(\omega, h^\star_\omega(\tilde{x}_j), \tilde{y}_j)-\ell_{out}(\omega, \hat{h}_\omega(\tilde{x}_j), \tilde{y}_j)}\\
    &\leq\frac{M_{out}}{m}\sum_{j=1}^m\verts{h^\star_\omega(\tilde{x}_j)-\hat{h}_\omega(\tilde{x}_j)}\\
    &\leq M_{out}\sqrt{\kappa}\Verts{h^\star_\omega-\hat{h}_\omega}_\mathcal{H},
\end{align*}
where the first line uses the definition of $(\omega, h)\mapsto\widehat{L}_{out}(\omega, h)$, the second line applies the triangle inequality, the third line leverages the fact that $v\mapsto\ell_{out}(\omega, v, y)$ is $M_{out}$-Lipschitz continuous, for any $\omega\in\Omega$ and $y\in\mathcal{Y}$, and the last line follows from the reproducing property of the RKHS $\mathcal{H}$, Cauchy-Schwarz's inequality, and \cref{assump:K_bounded} to bound $\Verts{K(x,\cdot)}_\mathcal{H}$ by $\sqrt{\kappa}$. Similarly, we obtain:
\begin{gather*}
    \partial_h\Eout\leq\bar{M}_{out}\kappa\Verts{h^\star_\omega-\hat{h}_\omega}_\mathcal{H},\quad\partial_\omega\Eout\leq\bar{M}_{out} \sqrt{\kappa}\Verts{h^\star_\omega-\hat{h}_\omega}_\mathcal{H},\quad\partial_h^2\Ein\leq\bar{M}_{in} \kappa\sqrt{\kappa}\Verts{h^\star_\omega-\hat{h}_\omega}_\mathcal{H},\\
    \partial_{\omega, h}^2\Ein\leq\bar{M}_{in} \kappa\Verts{h^\star_\omega-\hat{h}_\omega}_\mathcal{H}.
\end{gather*}
Combining all the bounds finishes the proof.
\end{proof}

\begin{proposition}\label{prop:bounded_derivatives}
Under \cref{assump:compact,assump:convexity_lin,assump:K_bounded,assump:reg_lin_lout}, the following inequalities hold for any $\omega\in\Omega$:
\begin{align*}
	\Verts{\partial_{h}L_{out}(\omega,h_{\omega}^{\star})}_{\mathcal{H}}\leq C_{out}, \quad \Verts{\partial^2_{\omega,h}L_{in}(\omega,h_{\omega}^{\star}) }_{\op}\leq C_{in},\quad 
	\Verts{\partial^2_{\omega,h}\widehat{L}_{in}(\omega,\hat{h}_{\omega}) }_{\op}\leq C_{in},
\end{align*}
where $C_{out}$ and $C_{in}$ are the positive constants defined in \cref{prop:lip_continuity_out}.
\end{proposition}

\begin{proof}Let $\omega\in\Omega$.

\textbf{Upper-bound on $\Verts{\partial_{h}L_{out}(\omega,h_{\omega}^{\star})}_{\mathcal{H}}$. }We have:
\begin{align*}
    \left\|\partial_h L_{out}(\omega, h^\star_\omega)\right\|_\mathcal{H}&=\Big\|\mathbb{E}_\mathbb{Q}\left[\partial_v \ell_{out}(\omega, h^\star_\omega(x), y)K(x,\cdot)\right]\Big\|_\mathcal{H}\\
    &\leq\mathbb{E}_\mathbb{Q}\Big[\left|\partial_v \ell_{out}(\omega, h^\star_\omega(x), y)\right|\left\|K(x,\cdot)\right\|_\mathcal{H}\Big]\\
    &\leq\sqrt{\kappa}\mathbb{E}_\mathbb{Q}\Big[\left|\partial_v \ell_{out}(\omega, h^\star_\omega(x), y)\right|\Big],
\end{align*}
where the first line follows from \cref{prop:fre_diff_L}, the second line results from the triangle inequality, and the last line uses \cref{assump:K_bounded} to bound $\Verts{K(x,\cdot)}_\mathcal{H}$ by $\sqrt{\kappa}$.  Furthermore, we know by \cref{prop:bound_hstaromega} that $(\omega,h_{\omega}^{\star}(x),y)$ belongs to the compact subset $\Omega\times \mathcal{V}\times \mathcal{Y}$ and by \cref{prop:uniform_boundedness} that $\partial_v \ell_{out}(\omega, h^\star_\omega(x), y)$ is bounded by a  constant $M_{out}$ on $\text{hull}(\Omega)\times \mathcal{V}\times \mathcal{Y}$. Hence, it follows that:
\begin{align*}
    \left\|\partial_h L_{out}(\omega, h^\star_\omega)\right\|_\mathcal{H}
    \leq \sqrt{\kappa} M_{out} \le C_{out},
\end{align*}
where $C_{out}$ is defined in \cref{prop:lip_continuity_out}. 

\textbf{Upper-boud on $\quad \Verts{\partial^2_{\omega,h}L_{in}(\omega,h_{\omega}^{\star}) }_{\op}$. }According to \cref{prop:fre_diff_L_v}, $\partial_{\omega, h}^2 L_{in}(\omega, h^\star_\omega)$ is a Hilbert-Schmidt operator, which points to:
\begin{equation}\label{eq:norm_op_partial_omega_h_Lin}
    \left\|\partial_{\omega, h}^2L_{in}(\omega, h^\star_\omega)\right\|_{\op}\leq\left\|\partial_{\omega, h}^2L_{in}(\omega, h^\star_\omega)\right\|_{\hs}=\sqrt{\sum_{l=1}^d\left\|\partial_{\omega_k, h}^2L_{in}(\omega, h^\star_\omega)\right\|_\mathcal{H}^2}.
\end{equation}
This means that to find an upper-bound on $\left\|\partial_{\omega, h}^2L_{in}(\omega, h^\star_\omega)\right\|_{\op}$, it suffices to establish an upper-bound on $\left\|\partial_{\omega_l, h}^2L_{in}(\omega, h^\star_\omega)\right\|_\mathcal{H}^2$ for any $l\in\{1,\ldots,d\}$. For a fixed $l\in\{1,\ldots,d\}$, we have:
\begin{align*}
    \left\|\partial_{\omega_l, h}^2L_{in}(\omega, h^\star_\omega)\right\|_\mathcal{H}^2&=\Big\|\mathbb{E}_\mathbb{P}\left[\partial_{\omega_l,v}^2 \ell_{in}(\omega, h^\star_\omega(x), y)K(x,\cdot)\right]\Big\|_\mathcal{H}^2\\
    &\leq\mathbb{E}_\mathbb{P}\left[\left|\partial_{\omega_l, v}^2 \ell_{in}(\omega, h^\star_\omega(x), y)\right|^2\left\|K(x,\cdot)\right\|_\mathcal{H}^2\right]\\
    &\leq\mathbb{E}_\mathbb{P}\left[\left\|\partial_{\omega, v}^2 \ell_{in}(\omega, h^\star_\omega(x), y)\right\|^2\right]\kappa,
\end{align*}
where the first line follows from \cref{prop:fre_diff_L_v}, the second line is a consequence of Jensen's inequality applied on the convex function $\|\cdot\|^2$, and the last line applies \cref{assump:K_bounded} to bound $\Verts{K(x,\cdot)}_\mathcal{H}^2$ by $\kappa$. Incorporating this upper-bound into \cref{eq:norm_op_partial_omega_h_Lin} yields:
\begin{equation*}
    \left\|\partial_{\omega, h}^2L_{in}(\omega, h^\star_\omega)\right\|_{\op}\leq\sqrt{\mathbb{E}_\mathbb{P}\left[\left\|\partial_{\omega, v}^2 \ell_{in}(\omega, h^\star_\omega(x), y)\right\|^2\right] d\kappa}\leq M_{in}\sqrt{d\kappa}\leq C_{out}, 
\end{equation*}
where we used \cref{prop:uniform_boundedness} to bound $\partial_{\omega, v}^2 \ell_{in}(\omega, h^\star_\omega(x), y)$ by the constant $M_{in}$.

\textbf{Upper-bound on $\Verts{\partial^2_{\omega,h}\widehat{L}_{in}(\omega,\hat{h}_{\omega}) }_{\op}$. }The derivation of this upper bound follows the same steps as the previous one, with the only differences being the use of $\widehat{L}_{in}$ instead of $L_{in}$, and $\hat{h}_\omega$ instead of $h^\star_\omega$.

Note that in the last step of each of the three upper-bounds, we used the fact that the functions we are dealing with are continuous (by \cref{assump:reg_lin_lout}) on $\Omega\times\mathcal{V}\times\mathcal{Y}$, which is compact because $\Omega$ and $\mathcal{Y}$ are compact by \cref{assump:compact} and $\mathcal{V}$ is a compact interval of $\mathbb{R}$ defined in \cref{prop:bound_hstaromega}. Hence, those functions are bounded.
\end{proof}


\begin{proposition}[{Approximation bounds}]\label{prop:grad_app_bound}
    Under \cref{assump:compact,assump:convexity_lin,assump:K_bounded,assump:reg_lin_lout}, the following holds for any $\omega\in\Omega$:
    \begin{gather*}
    \verts{\mathcal{F}(\omega)-\widehat{\mathcal{F}}(\omega)}\leq 
    \Dout
    +\frac{C_{out}}{\lambda}\Dinh,\\
 	\Verts{\nabla\mathcal{F}(\omega)-\widehat{\nabla\mathcal{F}}(\omega)}
 	\leq  
 	\Doutw  + \frac{C_{in}}{\lambda}\Douth + \frac{C_{out}C_{in}}{\lambda^2}\Dinhh + \frac{C_{out}}{\lambda}\Dinwh + \frac{C_{out}}{\lambda}\parens{1 + 2\frac{C_{in}}{\lambda}  + \frac{C_{in}^2}{\lambda^2} }\Dinh,
    \end{gather*}
 where the constants $C_{in}$ and $C_{out}$ are given in {\cref{prop:lip_continuity_out}}.
\end{proposition}
\begin{proof}
In all what follows, we fix a value for $\omega$ in $\Omega$. We start by controlling the value function, then its gradient.

{\bf Control on the value function.}
By the triangle inequality, we have:
\begin{equation}\label{eq:diff_f_int}
    \verts{\mathcal{F}(\omega)-\hat{\mathcal{F}}(\omega)}\leq\underbrace{\verts{L_{out}(\omega, h^\star_\omega)-\widehat{L}_{out}(\omega, h^\star_\omega)}}_{\delta_{\omega}^{out}} +\underbrace{\verts{\widehat{L}_{out}(\omega, h^\star_\omega)-\widehat{L}_{out}(\omega, \hat{h}_\omega)}}_{\Eout},
\end{equation}
According to \cref{prop:lip_continuity_out}, the error term $\Eout$ is controlled by the norm of the difference $h^\star_\omega-\hat{h}_\omega$, \textit{i.e.}, $\Eout\leq C_{out}\Verts{h^\star_\omega-\hat{h}_\omega}_\mathcal{H}$. 
Moreover, by \cref{prop:diff_hstar_hhat}, we know that $\Verts{h^\star_\omega-\hat{h}_\omega}_\mathcal{H}\leq \frac{1}{\lambda}\Dinh$. Therefore, combining both bounds yields: 
	$\Eout\leq \frac{C_{out}}{\lambda}\Dinh$. 
The upper-bound on value function follows by substituting the previous inequality into \cref{eq:diff_f_int}. 

{\bf Control on the gradient.}
By \cref{prop:tot_grad_int}, we have the following expression for the total gradient $\nabla\mathcal{F}$:
\begin{align*}
    \nabla\mathcal{F}(\omega)=\partial_\omega L_{out}(\omega, h^\star_\omega)-\partial_{\omega, h}^2 L_{in}(\omega, h^\star_\omega)\left(\partial_h^2 L_{in}(\omega, h^\star_\omega)\right)^{-1}\partial_h L_{out}(\omega, h^\star_\omega).
\end{align*}
{Similarly, the gradient estimator $\widehat{\nabla\mathcal{F}}$ is defined by replacing $L_{out}$ and $L_{in}$ by their empirical versions $\widehat{L}_{out}$ and $\widehat{L}_{in}$, and $h_{\omega}^{\star}$ by $\hat{h}_{\omega}\coloneqq\arg\min_{h\in \mathcal{H}} \widehat{L}_{in}(\omega,h)$ in the above expression, \textit{i.e.},}
\begin{align*}
    \widehat{\nabla\mathcal{F}}(\omega)=\partial_\omega\widehat{L}_{out}(\omega, \hat{h}_\omega)-\partial_{\omega, h}^2 \widehat{L}_{in}(\omega, \hat{h}_\omega)\left(\partial_h^2 \widehat{L}_{in}(\omega, \hat{h}_\omega)\right)^{-1}\partial_h \widehat{L}_{out}(\omega, \hat{h}_\omega).
\end{align*}
To simplify notations, for any $h\in\mathcal{H}$, we introduce the following operators $R(h), \hat{R}(h):\mathcal{H}\to\Omega$:    \begin{align*}
        R(h)=\partial_{\omega, h}^2L_{in}(\omega, h)\left(\partial_h^2 L_{in}(\omega, h)\right)^{-1}\quad\text{and}\quad 
        \hat{R}(h)=\partial_{\omega, h}^2\widehat{L}_{in}(\omega, h)\left(\partial_h^2 \widehat{L}_{in}(\omega, h)\right)^{-1}.
    \end{align*}
    The difference $\nabla\mathcal{F}(\omega)-\widehat{\nabla\mathcal{F}}(\omega)$ can be decomposed as:
    \begin{align*}
        \nabla\mathcal{F}(\omega)-\widehat{\nabla\mathcal{F}}(\omega)=&\left(\partial_\omega L_{out}(\omega,h^\star_\omega)-\partial_\omega \widehat{L}_{out}(\omega,h^\star_\omega)\right)+\left(\partial_\omega \widehat{L}_{out}(\omega,h^\star_\omega)-\partial_\omega \widehat{L}_{out}(\omega,\hat{h}_\omega)\right)\\
        &-\hat{R}(\hat{h}_\omega)\parens{\left(\partial_h L_{out}(\omega,h^\star_\omega)-\partial_h\widehat{L}_{out}(\omega,h^\star_\omega)\right)+ \left(\partial_h\widehat{L}_{out}(\omega,h^\star_\omega)-\partial_h\widehat{L}_{out}(\omega,\hat{h}_\omega)\right)}\\
                &-\left(R(h^\star_\omega)-\hat{R}(\hat{h}_\omega)\right)\partial_h L_{out}(\omega,h^\star_\omega).
    \end{align*}
By taking the norm of the above equality and using triangle inequality, we obtain the following upper-bound: 
    \begin{align}
    \begin{split}
        \Verts{\nabla\mathcal{F}(\omega)-\widehat{\nabla\mathcal{F}}(\omega)}
        & \leq
        \underbrace{\Verts{\partial_\omega L_{out}(\omega,h^\star_\omega)-\partial_\omega \widehat{L}_{out}(\omega,h^\star_\omega)}}_{\Doutw}+\underbrace{\Verts{\partial_\omega \widehat{L}_{out}(\omega,h^\star_\omega)-\partial_\omega \widehat{L}_{out}(\omega,\hat{h}_\omega)}}_{\Eoutw}\\
        +&\Verts{\hat{R}(\hat{h}_{\omega})}_{\op}\parens{\underbrace{\Verts{\partial_h L_{out}(\omega,h^\star_\omega)-\partial_h\widehat{L}_{out}(\omega,h^\star_\omega)}_{\mathcal{H}}}_{\Douth} + \underbrace{\Verts{\partial_h\widehat{L}_{out}(\omega,h^\star_\omega)-\partial_h\widehat{L}_{out}(\omega,\hat{h}_\omega)}_{\mathcal{H}}}_{\Eouth}}\\
                +& \Verts{R(h^\star_\omega)-\hat{R}(\hat{h}_\omega)}_{\op}\Verts{\partial_h L_{out}(\omega,h^\star_\omega)}_{\mathcal{H}}.
                \end{split}\label{eq:diff_grad_inter}
    \end{align}
    Next, we provide upper-bounds on $\Verts{R(h_{\omega}^{\star})- \hat{R}(\hat{h}_{\omega})}_{\op}$ and $\Verts{\hat{R}(\hat{h}_{\omega})}_{\op}$ in terms of derivatives of $L_{in}$ and $\widehat{L}_{in}$. 

    \textbf{Upper-bounds on $\Verts{R(h_{\omega}^{\star})- \hat{R}(\hat{h}_{\omega})}_{\op}$ and $\Verts{\hat{R}(\hat{h}_{\omega})}_{\op}$. } 
    %
    By application of {\cref{prop:fre_diff_L_v,prop:strong_convexity_Lin}}, we deduce that $\partial_{\omega,h}^2 L_{in}(\omega,h_{\omega}^{\star})$, $\partial_{h}^2 L_{in}(\omega,h_{\omega}^{\star})$, $\partial_{\omega,h}^2 \widehat{L}_{in}(\omega,\hat{h}_{\omega})$ and $\partial_{h}^2 \widehat{L}_{in}(\omega,\hat{h}_{\omega})$ are all bounded operators. Moreover, since $L_{in}$ and $\widehat{L}_{in}$ are $\lambda$-strongly  convex in their second argument by {\cref{prop:strong_convexity_Lin}}, it follows that $\partial_{h}^2 L_{in}(\omega,h_{\omega}^{\star})\geq \lambda\Id_{\mathcal{H}}$ and $\partial_{h}^2 \widehat{L}_{in}(\omega,\hat{h}_{\omega})\geq \lambda \Id_{\mathcal{H}}$. We can therefore apply \cref{lem:tech_res_1} which yields the following inequalities:  
    \begin{align*}
        \Verts{R(h_{\omega}^{\star})- \hat{R}(\hat{h}_{\omega})}_{\op}
        \leq & \frac{1}{\lambda^2}\Verts{\partial_{\omega,h}^2 L_{in}(\omega,h_{\omega}^{\star})}_{\op}\Verts{\partial_{h}^2 L_{in}(\omega,h_{\omega}^{\star}) - \partial_{h}^2 \widehat{L}_{in}(\omega,\hat{h}_{\omega})}_{\op}\\ 
        &+ \frac{1}{\lambda}\Verts{\partial_{\omega,h}^2 L_{in}(\omega,h_{\omega}^{\star}) - \partial_{\omega,h}^2 \widehat{L}_{in}(\omega,\hat{h}_{\omega})}_{\op},\\
        \Verts{\hat{R}(\hat{h}_{\omega})}_{\op}\leq&\frac{1}{\lambda} \Verts{\partial_{\omega,h}^2\widehat{L}_{in}(\omega,\hat{h}_{\omega})}_{\op}. 
    \end{align*}
By applying the triangle inequality to both terms of the first inequality above, we obtain:
\begin{align*}
        \Verts{R(h_{\omega}^{\star})- \hat{R}(\hat{h}_{\omega})}_{\op}
        \leq & \frac{1}{\lambda^2}\Verts{\partial_{\omega,h}^2 L_{in}(\omega,h_{\omega}^{\star})}_{\op}\parens{\underbrace{\Verts{\partial_{h}^2 L_{in}(\omega,h_{\omega}^{\star}) - \partial_{h}^2 \widehat{L}_{in}(\omega,h^\star_{\omega})}_{\op}}_{\Dinhh} + \underbrace{\Verts{\partial_{h}^2 \widehat{L}_{in}(\omega,h_{\omega}^{\star}) - \partial_{h}^2 \widehat{L}_{in}(\omega,\hat{h}_{\omega})}_{\op}}_{\Einhh}}\\ 
        &+ \frac{1}{\lambda}\parens{\underbrace{\Verts{\partial_{\omega,h}^2 L_{in}(\omega,h_{\omega}^{\star}) - \partial_{\omega,h}^2 \widehat{L}_{in}(\omega,h_{\omega}^{\star})}_{\op}}_{\Dinwh} + \underbrace{\Verts{\partial_{\omega,h}^2 \widehat{L}_{in}(\omega,h_{\omega}^{\star}) - \partial_{\omega,h}^2 \widehat{L}_{in}(\omega,\hat{h}_{\omega})}_{\op}}_{\Einwh}}.
    \end{align*}
{\bf Final bound.} We can now substitute the above bounds on $\Verts{R(h_{\omega}^{\star})- \hat{R}(\hat{h}_{\omega})}_{\op}$ and $\Verts{\hat{R}(\hat{h}_{\omega})}_{\op}$ into \cref{eq:diff_grad_inter} to obtain the following upper-bound on the gradient error:
 \begin{equation} \label{eq:diff_grad_inter_2}
\begin{aligned}
 	\Verts{\nabla\mathcal{F}(\omega)-\widehat{\nabla\mathcal{F}}(\omega)}
 	&\leq  
 	\Doutw + \Eoutw + \frac{1}{\lambda}\Verts{\partial_{\omega,h}^2 \widehat{L}_{in}(\omega,\hat{h}_{\omega})}_{\op}\parens{\Douth + \Eouth}\\
 	+& \Verts{\partial_{h}L_{out}(\omega,h_{\omega}^{\star})}_{\mathcal{H}}\parens{ \frac{1}{\lambda^2}\Verts{\partial^2_{\omega,h}L_{in}(\omega,h_{\omega}^{\star}) }_{\op} \parens{\Dinhh + \Einhh} + \frac{1}{\lambda}\parens{\Dinwh + \Einwh} }.
 \end{aligned}
 \end{equation}
Furthermore, by \cref{prop:bounded_derivatives}, we have the following upper-bounds on the derivatives of $L_{in}$ and $L_{out}$:
\begin{align*}
	\Verts{\partial_{h}L_{out}(\omega,h_{\omega}^{\star})}_{\mathcal{H}}\leq C_{out}, \quad \Verts{\partial^2_{\omega,h}L_{in}(\omega,h_{\omega}^{\star}) }_{\op}\leq C_{in},\quad 
	\Verts{\partial^2_{\omega,h}\widehat{L}_{in}(\omega,\hat{h}_{\omega}) }_{\op}\leq C_{in}. 
\end{align*}
Incorporating the above bounds into \cref{eq:diff_grad_inter_2}, we further get:
 \begin{align*}
 	\Verts{\nabla\mathcal{F}(\omega)-\widehat{\nabla\mathcal{F}}(\omega)}
 	\leq & 
 	\Doutw + \Eoutw + \frac{C_{in}}{\lambda}\parens{\Douth + \Eouth}\\
 	&+ C_{out}\parens{ \frac{C_{in}}{\lambda^2}\parens{\Dinhh + \Einhh} + \frac{1}{\lambda}\parens{\Dinwh + \Einwh} }.
 \end{align*}
By \cref{prop:lip_continuity_out}, we can upper-bound the error terms $\Eoutw, \Eouth$ by $C_{out}\Verts{h_{\omega}^{\star}-\hat{h}_{\omega}}_{\mathcal{H}}$ and $ \Einhh,\Einwh$ by the difference $C_{in}\Verts{h_{\omega}^{\star}-\hat{h}_{\omega}}_{\mathcal{H}}$. Furthermore, since  $\Verts{h_{\omega}^{\star}-\hat{h}_{\omega}}_{\mathcal{H}}\leq \frac{1}{\lambda}\Dinh$ by \cref{prop:diff_hstar_hhat}, we can further show that gradient error satisfies the desired bound:
\begin{align*}
 	\Verts{\nabla\mathcal{F}(\omega)-\widehat{\nabla\mathcal{F}}(\omega)}
 	\leq 
 	\Doutw  + \frac{C_{in}}{\lambda}\Douth + C_{out}\parens{ \frac{C_{in}}{\lambda^2}\Dinhh + \frac{1}{\lambda}\Dinwh}+ \frac{C_{out}}{\lambda}\parens{1 + 2\frac{C_{in}}{\lambda}  + \frac{C_{in}^2}{\lambda^2} }\Dinh.
\end{align*}
\end{proof}

\subsection{Maximal inequalities}\label{app_subsec:max_in}
\begin{proposition}[Maximal inequalities for  empirical processes]\label{prop:exp_uni_bound}
Let $\Lambda$ be a positive constant. 
Under \cref{assump:compact,assump:K_bounded,assump:reg_lin_lout,assump:convexity_lin}, the following maximal inequalities hold for any $0<\lambda \leq \Lambda$:
\begin{align*}
	\mathbb{E}_{\mathbb{Q}}\brackets{\sup_{\omega\in \Omega} \Dout }
	 &\leq {\sqrt{\frac{1}{\lambda^2 m}}c(\Omega)\max(M_{out}{\lipout}\diam(\Omega),\Lambda {M_{out}^2})}\\
	 \mathbb{E}_{\mathbb{Q}}\brackets{\sup_{\omega\in \Omega} \Doutw }
	&\leq {\sqrt{\frac{d}{\lambda^2m}}c(\Omega)\max(M_{out}\lipout\diam(\Omega),\Lambda M_{out}^2)}, 
\end{align*}
    where $c(\Omega)$ is a positive constant {greater than 1} that depends only on $\Omega$ and $d$, while {$\lipout$ and $M_{out}$ are positive constants defined in \cref{prop:uniform_boundedness,prop:uniform_Lipschitzness}}. 
\end{proposition}
\begin{proof}
We will apply the result of \cref{prop:empirical_process} which provides maximal inequalities for real-valued empirical processes that are uniformly bounded {and} Lipschitz in their parameter. To this end, consider the parametric families:
\begin{align*}
	\mathcal{T}_{l}^{out} &\coloneqq\braces{\mathcal{X}\times \mathcal{Y}\ni (x,y)\mapsto \partial_{w_l} \ell_{out}(\omega,h_{\omega}^{\star}(x),y)\mid\omega\in \Omega}, \qquad 1\leq l \leq d\\
	\mathcal{T}_{0}^{out} &\coloneqq\braces{\mathcal{X}\times \mathcal{Y}\ni (x,y)\mapsto  \ell_{out}(\omega,h_{\omega}^{\star}(x),y)\mid\omega\in \Omega}.
\end{align*} 
For any $0\leq l\leq d$, these real-valued functions are uniformly bounded by a positive constant $M_{out}$, thanks to {\cref{prop:uniform_boundedness}}. 
Moreover, by {\cref{prop:uniform_Lipschitzness}}, the functions $\omega\mapsto \partial_{\omega_l}\ell_{out}(\omega,h_{\omega}^{\star}(x),y)$, and $\omega\mapsto \ell_{out}(\omega,h_{\omega}^{\star}(x),y)$ are all $\lambda^{-1}\lipout$-Lipschitz for any $(x,y)\in \mathcal{X}\times \mathcal{Y}$. Hence, \cref{prop:empirical_process} is applicable to each of these families, with $\PP$ set to $\mathbb{Q}$ and $\mathcal{Z}$ set to $\mathcal{X}\times \mathcal{Y}$.  We treat both $\Dout$ and $\Doutw$ separately.

{\bf A maximal inequality for $\Dout$.}
For $l=0$, we readily apply \cref{prop:empirical_process} with $p=1$ to get the following maximal inequality for $\Dout$:

\begin{align*}
	\mathbb{E}_{\mathbb{Q}}\brackets{\sup_{\omega\in \Omega} \Dout } \coloneqq &
	\mathbb{E}_{\mathbb{Q}}\brackets{ \sup_{\omega\in \Omega} \verts{ \mathbb{E}_{(x,y)\sim \mathbb{Q}}\brackets{\ell_{out}(\omega,h_{\omega}^{\star}(x),y) } - \frac{1}{m}\sum_{j=1}^m \ell_{out}(\omega,h_{\omega}^{\star}(\tilde{x}_j),\tilde{y}_j) } }\\
	 \leq& \sqrt{\frac{1}{\lambda^2 m}}c(\Omega)\max({\lipout}\diam(\Omega),\Lambda {M_{out}^2}). 
\end{align*}

{\bf A maximal inequality for $\Doutw$.}
{We now turn to  $\Doutw$, which involves vector-valued processes (as an error between the gradient and its estimate).} While the maximal inequalities in \cref{prop:empirical_process} hold for real-valued processes, we will first obtain maximal inequalities for each component appearing in $\Doutw$ and then sum these to control $\Doutw$.
To this end, we first use the Cauchy-Schwarz inequality which implies that $\mathbb{E}_{\mathbb{Q}}\brackets{\sup_{\omega\in \Omega} (\Doutw) }\leq \mathbb{E}_{\mathbb{Q}}\brackets{\sup_{\omega\in \Omega} (\Doutw)^2 }^{\frac{1}{2}}$. Thus we only need to control $\mathbb{E}_{\mathbb{Q}}\brackets{\sup_{\omega\in \Omega} (\Doutw)^2 }$. Simple calculations show that:
\begin{align*}
	\mathbb{E}_{\mathbb{Q}}\brackets{\sup_{\omega\in \Omega} \Doutw }^2&\leq \mathbb{E}_{\mathbb{Q}}\brackets{\sup_{\omega\in \Omega} (\Doutw)^2 }\\
	&\leq  \sum_{l=1}^d \mathbb{E}_{\mathbb{Q}}\brackets{\sup_{\omega\in \Omega}\verts{\mathbb{E}_{(x,y)\sim\mathbb{Q}}\brackets{\partial_{w_l}\ell_{out}(\omega,h_{\omega}^{\star}(x),y)} - \frac{1}{m}\sum_{j=1}^m \partial_{\omega_l}\ell_{out}\parens{\omega,h_{\omega}^{\star}(\tilde{x}_j),\tilde{y}_j}  }^2}\\
	&\leq \parens{\sqrt{\frac{d}{\lambda^2m}}c(\Omega)\max(\lipout\diam(\Omega),\Lambda M_{out}^{{2}})}^2,
\end{align*}
where the last inequality follows by application of \cref{prop:empirical_process} with $p=2$ to each term in the right-hand side of the first inequality for $1\leq l\leq d$. We get the desired bound on  $\mathbb{E}_{\mathbb{Q}}\brackets{\sup_{\omega\in \Omega} \Doutw }$ by taking the square root of the above inequality.
\end{proof}

\begin{proposition}[Maximal inequalities for RKHS-valued empirical processes]\label{prop:exp_uni_bound_2}
Let $\Lambda$ be a positive constant. 
Under \cref{assump:compact,assump:K_bounded,assump:reg_lin_lout,assump:convexity_lin}, the following maximal inequalities hold for any $0<\lambda\leq \Lambda$:
 \begin{align*}
 	\mathbb{E}_{\mathbb{Q}}\brackets{\sup_{\omega\in \Omega} \Douth }&\leq 
 	{\lambda^{-\frac{1}{4}}m^{-\frac{1}{2}}  \parens{c(\Omega)\max\parens{\widetilde{M}_{out,1}\widetilde{L}_{out,1}\diam(\Omega),\Lambda\widetilde{M}_{out,1}^2}}^{\frac{1}{4}}},\\
 	\mathbb{E}_{\mathbb{P}}\brackets{\sup_{\omega\in \Omega} \Dinh }&\leq 
 	{\lambda^{-\frac{1}{4}}n^{-\frac{1}{2}} \parens{c(\Omega)\max\parens{\widetilde{M}_{in,1}\widetilde{L}_{in,1}\diam(\Omega),\Lambda\widetilde{M}_{in,1}^2}}^{\frac{1}{4}}},\\
 	 	\mathbb{E}_{\mathbb{P}}\brackets{\sup_{\omega\in \Omega} \Dinwh }&\leq 
 	{\lambda^{-\frac{1}{4}}n^{-\frac{1}{2}}d^{\frac{1}{2}} \parens{c(\Omega)\max\parens{\widetilde{M}_{in,1}\widetilde{L}_{in,1}\diam(\Omega),\Lambda\widetilde{M}_{in,1}^2}}^{\frac{1}{4}}},\\
 	\mathbb{E}_{\mathbb{P}}\brackets{\sup_{\omega\in \Omega} \Dinhh }&\leq 
 	{\lambda^{-\frac{1}{4}}n^{-\frac{1}{2}} \parens{c(\Omega)\max\parens{\widetilde{M}^2_{in,2}\widetilde{L}_{in,2}\diam(\Omega),\Lambda\widetilde{M}^2_{in,2}}}^{\frac{1}{4}}},
 \end{align*}
 where $c(\Omega)$ is a positive constant {greater than $1$} that depends only on $\Omega$ and $d$, {$\widetilde{L}_{out,1},\widetilde{L}_{in,1},\widetilde{L}_{in,2},\widetilde{M}_{out,1},\widetilde{M}_{in,1},\widetilde{M}_{in,2}$ are positive constants defined as:
 \begin{gather*}
     \widetilde{L}_{out,1}\coloneqq 2\lipout M_{out}\kappa,\quad\widetilde{L}_{in,1}\coloneqq 2\lipin M_{in}\kappa,\quad\widetilde{L}_{in,2}\coloneqq 2\lipin M_{in}\kappa^2,\\
     \widetilde{M}_{out,1}\coloneqq M_{out}^2\kappa,\quad\widetilde{M}_{in,1}\coloneqq M_{in}^2\kappa,\quad\widetilde{M}_{in,2}\coloneqq M_{in}^2\kappa^2,
 \end{gather*}
 and $\lipout,\lipin,M_{out},M_{in}$ are positive constants given in \cref{prop:uniform_boundedness,prop:uniform_Lipschitzness}.}
 
\end{proposition}

\begin{proof}
Consider parametric families of real-valued functions indexed by $\Omega$ of the form:
\begin{align*}
	\mathcal{T}_{s,a}\coloneqq\braces{ t_{\omega}: ((x,y),(x',y'))\mapsto  f_s(\omega,x,y)f_s(\omega,x',y')K^{a}(x,x')\mid\omega\in \Omega },
\end{align*}
where $a\in \{1,2\}$,  $s$ is an integer satisfying {$0\leq s\leq d+2$}, and $f_s(\omega,x,y)$ are real-valued functions given by:
\begin{gather*}
	f_0: (\omega,x,y)\mapsto\partial_v \ell_{out}(\omega, h^\star_\omega(x), y),\qquad 
	f_1:(\omega,x,y)\mapsto\partial_v \ell_{in}(\omega, h^\star_\omega(x), y),\qquad
	f_2:(\omega,x,y)\mapsto\partial_v^2 \ell_{in}(\omega, h^\star_\omega(x), y),\\
	f_{2+l}: (\omega,x,y)\mapsto\partial_{\omega_l, v}^2 \ell_{in}(\omega, h^\star_\omega(x), y), \quad 1\leq l\leq d.
\end{gather*}
For any $1\leq s\leq d+2$, the real-valued functions $f_s$ are uniformly bounded by a positive constant $M_{in}$ thanks to {\cref{prop:uniform_boundedness}}. Moreover, since the kernel $K$ is bounded by {$\kappa$ due to \cref{assump:K_bounded}}, it follows that all elements $t_{\omega}$ of $\mathcal{T}_{s,a}$ are uniformly bounded by {$\widetilde{M}_{in,a}\coloneqq M_{in}^2\kappa^a$}. Moreover, for $1\leq s\leq d+2$, the functions $\omega\mapsto f_{s}(\omega,x,y)$, are $\lambda^{-1}\lipin$-Lipschitz for any $(x,y)\in\mathcal{X}\times\mathcal{Y}$, by {\cref{prop:uniform_Lipschitzness}}. Hence, it follows that the maps $\omega\mapsto t_{\omega}((x,y),(x',y'))$ are $\lambda^{-1}\widetilde{L}_{in,a}$-Lipschitz with {$\widetilde{L}_{in,a}\coloneqq 2\lipin M_{in}\kappa^a$} for any $(x,y)$ and $(x',y')$ in $\mathcal{X}\times \mathcal{Y}$. Similarly, for $s=0$, we get the same properties, albeit, with different constants, \textit{i.e.}, the family {$\mathcal{T}_{0,a}$} is uniformly bounded by a constant {$\widetilde{M}_{out,a}\coloneqq M_{out}^2\kappa^a$} with {$M_{out}$ introduced in \cref{prop:uniform_boundedness}}, and is $\lambda^{-1}\widetilde{L}_{out,a}$-Lipschitz in its parameter with {$\widetilde{L}_{out,a}\coloneqq 2\lipout M_{out}\kappa^a$} where $\lipout$ is given in {\cref{prop:uniform_Lipschitzness}}.
Hence, the maximal inequality in \cref{lem:u_stat_sup_p_omega} is applicable to each of these families with $\mathcal{Z}$ set to $\mathcal{X}\times \mathcal{Y}$, and {$\PP$ set either to $\mathbb{P}$ for $1\leq s\leq d+2$, or to $\mathbb{Q}$ for $s=0$}.
For conciseness, in all what follows, we will write $z = (x,y)$ and $z_i = (x_i,y_i)$ and $\tilde{z}_j = (\tilde{x}_j,\tilde{y}_j)$ for $1\leq i\leq n$ and $1\leq j\leq m$. 

{\bf Maximal inequalities for $\Douth$ and $\Dinh$.}
We control $\Douth$ first as $\Dinh$ will be dealt with similarly. Using Cauchy-Schwarz inequality and standard calculus, we have that:
\begin{align*}
	\mathbb{E}_{\mathbb{Q}}\brackets{\sup_{\omega\in \Omega} \Douth }^2
	&\leq \mathbb{E}_{\mathbb{Q}}\brackets{\sup_{\omega\in \Omega} (\Douth)^2 }\\
	&\coloneqq \mathbb{E}_{\mathbb{Q}}\brackets{ \sup_{\omega\in \Omega} \Verts{ \mathbb{E}_{(x,y)\sim \mathbb{Q}}\brackets{\partial_v\ell_{out}(\omega,h_{\omega}^{\star}(x),y)K(x,\cdot)} - \frac{1}{m}\sum_{j=1}^m \partial_v\ell_{out}(\omega,h_{\omega}^{\star}(\tilde{x}_j),\tilde{y}_j)K(\tilde{x}_j,\cdot) }_{\mathcal{H}}^2 }\\
	&= \mathbb{E}_{\mathbb{Q}}\brackets{ \sup_{\omega\in \Omega} \mathbb{E}_{z,z'\sim\mathbb{Q}\otimes\mathbb{Q}}\left[t_\omega(z,z')\right]+\frac{1}{m^2}\sum_{i,j=1}^m t_\omega(z_i,z_j)-\frac{2}{m}\sum_{j=1}^m\mathbb{E}_{z\sim\mathbb{Q}}\left[t_\omega(z,\tilde{z}_j)\right]},
\end{align*}
 where $t_\omega(z,z') \coloneqq \partial_v\ell_{out}(\omega,h_{\omega}^{\star}(x),y)\partial_v\ell_{out}(\omega,h_{\omega}^{\star}(x'),y')K(x,x')\in \mathcal{T}_{0,1}$. The last term is precisely what \cref{lem:u_stat_sup_p_omega} controls when applying it to the family $\mathcal{T}_{0,1}$ and choosing $\PP$ to be $\mathbb{Q}$. Therefore, the following maximal inequality holds by application of \cref{lem:u_stat_sup_p_omega}:
 \begin{align*}
 	\mathbb{E}_{\mathbb{Q}}\brackets{\sup_{\omega\in \Omega} \Douth }\leq 
 	\lambda^{-\frac{1}{4}}m^{-\frac{1}{2}}  \parens{c(\Omega)\max\parens{\widetilde{M}_{out,1}\widetilde{L}_{out,1}\diam(\Omega),\Lambda\widetilde{M}_{out,1}^2}}^{\frac{1}{4}},
 \end{align*}
 where $c(\Omega)$ is a positive constant {greater than $1$} the depends only on $\Omega$ and $d$. We obtain a similar inequality for $\Dinh$ by carrying out similar calculations, then applying \cref{lem:u_stat_sup_p_omega} to the family $\mathcal{T}_{1,1}$ and choosing $\mathbb{P}$ for the probability distribution $\PP$. The resulting bound is then of the form:
\begin{align*}
 	\mathbb{E}_{\mathbb{P}}\brackets{\sup_{\omega\in \Omega} \Dinh }\leq 
 	\lambda^{-\frac{1}{4}}n^{-\frac{1}{2}} \parens{c(\Omega)\max\parens{\widetilde{M}_{in,1}\widetilde{L}_{in,1}\diam(\Omega),\Lambda\widetilde{M}_{in,1}^2}}^{\frac{1}{4}}.
 \end{align*}

{\bf A maximal inequality for $\Dinwh$.} We have:
\begin{align*}
	\mathbb{E}_{\mathbb{P}}\brackets{\sup_{\omega\in \Omega} \Dinwh }^2
	&\stackrel{(a)}{\leq} \mathbb{E}_{\mathbb{P}}\brackets{\sup_{\omega\in \Omega} (\Dinwh)^2 }\\
	&\stackrel{(b)}{\coloneqq} \mathbb{E}_{\mathbb{P}}\brackets{ \sup_{\omega\in \Omega} \Verts{ \mathbb{E}_{(x,y)\sim \mathbb{P}}\brackets{\partial_{\omega,v}^2\ell_{in}(\omega,h_{\omega}^{\star}(x),y)K(x,\cdot)} - \frac{1}{n}\sum_{i=1}^n \partial_{\omega,v}^2\ell_{in}(\omega,h_{\omega}^{\star}(x_i),y_i)K(x_i,\cdot) }_{\op}^2 }\\
	&\stackrel{(c)}{\leq}\mathbb{E}_{\mathbb{P}}\brackets{ \sup_{\omega\in \Omega} \Verts{ \mathbb{E}_{(x,y)\sim \mathbb{P}}\brackets{\partial_{\omega,v}^2\ell_{in}(\omega,h_{\omega}^{\star}(x),y)K(x,\cdot)} - \frac{1}{n}\sum_{i=1}^n \partial_{\omega,v}^2\ell_{in}(\omega,h_{\omega}^{\star}(x_i),y_i)K(x_i,\cdot) }_{\hs}^2 }\\
	&\stackrel{(d)}{=}\sum_{l=1}^d \mathbb{E}_{\mathbb{P}}\brackets{ \sup_{\omega\in \Omega} \Verts{ \mathbb{E}_{(x,y)\sim \mathbb{P}}\brackets{\partial_{\omega_l,v}^2\ell_{in}(\omega,h_{\omega}^{\star}(x),y)K(x,\cdot)} - \frac{1}{n}\sum_{i=1}^n \partial_{\omega_l,v}^2\ell_{in}(\omega,h_{\omega}^{\star}(x_i),y_i)K(x_i,\cdot) }_{\mathcal{H}}^2 }\\
	&\stackrel{(e)}{=} \sum_{l=1}^d \mathbb{E}_{\mathbb{P}}\brackets{ \sup_{\omega\in \Omega} \mathbb{E}_{z,z'\sim\mathbb{P}\otimes\mathbb{P}}\left[t_{\omega,l}(z,z')\right]+\frac{1}{n^2}\sum_{i,j=1}^n t_{\omega,l}(z_i,z_j)-\frac{2}{n}\sum_{i=1}^n\mathbb{E}_{z\sim\mathbb{P}}\left[t_{\omega,l}(z,z_i)\right]},
\end{align*}
where we introduced $t_{\omega,l}(z,z') \coloneqq \partial_{\omega_l,v}^2\ell_{in}(\omega,h_{\omega}^{\star}(x),y)\partial_{\omega_l,v}^2\ell_{in}(\omega,h_{\omega}^{\star}(x'),y')K(x,x')\in \mathcal{T}_{2+l,1}$. Here, (a) follows by Cauchy-Schwarz inequality, (b) is obtained by definition of $\Dinwh$, while (c) uses the general fact that the operator norm of an operator is upper-bounded by its Hilbert-Schmidt norm which is finite in our case by application of \cref{prop:fre_diff_L_v}. Moreover, (d) further uses the Hilbert-Schmidt norm of an operator  in terms of the norm of its rows, while (e) simply expands the squared RKHS norm and uses the reproducing property in the RKHS $\mathcal{H}$. Each term in the last item (e) is precisely what \cref{lem:u_stat_sup_p_omega} controls when applying it to the families $\mathcal{T}_{2+l,1}$ for $1\leq l\leq d$ and choosing $\PP$ to be $\mathbb{P}$.  Therefore, the following maximal inequality holds by a direct application of \cref{lem:u_stat_sup_p_omega}:
 \begin{align*}
 	\mathbb{E}_{\mathbb{P}}\brackets{\sup_{\omega\in \Omega} \Dinwh }\leq 
 	\lambda^{-\frac{1}{4}}n^{-\frac{1}{2}}d^{\frac{1}{2}} \parens{c(\Omega)\max\parens{\widetilde{M}_{in,1}\widetilde{L}_{in,1}\diam(\Omega),\Lambda\widetilde{M}_{in,1}^2}}^{\frac{1}{4}},
 \end{align*}
 where $c(\Omega)$ is a positive constant greater than $1$ that depends only on $\Omega$ and $d$.

{\bf A maximal inequality for $\Dinhh$.} We will use a similar approach as for $\Dinwh$. We have:
\begin{align*}
	&\mathbb{E}_{\mathbb{P}}\brackets{\sup_{\omega\in \Omega} \Dinhh }^2\\
	&\stackrel{(a)}{\leq} \mathbb{E}_{\mathbb{P}}\brackets{\sup_{\omega\in \Omega} (\Dinhh)^2 }\\
	&\stackrel{(b)}{\coloneqq} \mathbb{E}_{\mathbb{P}}\brackets{ \sup_{\omega\in \Omega} \Verts{ \mathbb{E}_{(x,y)\sim \mathbb{P}}\brackets{\partial_{v}^2\ell_{in}(\omega,h_{\omega}^{\star}(x),y)K(x,\cdot)\otimes K(x,\cdot)} - \frac{1}{n}\sum_{i=1}^n \partial_{v}^2\ell_{in}(\omega,h_{\omega}^{\star}(x_i),y_i)K(x_i,\cdot)\otimes K(x_i,\cdot) }_{\op}^2 }\\
	&\stackrel{(c)}{\leq}\mathbb{E}_{\mathbb{P}}\brackets{ \sup_{\omega\in \Omega} \Verts{ \mathbb{E}_{(x,y)\sim \mathbb{P}}\brackets{\partial_{v}^2\ell_{in}(\omega,h_{\omega}^{\star}(x),y)K(x,\cdot)\otimes K(x,\cdot)} - \frac{1}{n}\sum_{i=1}^n \partial_{v}^2\ell_{in}(\omega,h_{\omega}^{\star}(x_i),y_i)K(x_i,\cdot)\otimes K(x_i,\cdot) }_{\hs}^2 }\\
	&\stackrel{(d)}{=} \mathbb{E}_{\mathbb{P}}\brackets{\sup_{\omega\in \Omega} \mathbb{E}_{z,z'\sim\mathbb{P}\otimes\mathbb{P}}\left[t_\omega(z,z')\right]+\frac{1}{n^2}\sum_{i,j=1}^n t_\omega(z_i,z_j)-\frac{2}{n}\sum_{i=1}^n\mathbb{E}_{z\sim\mathbb{P}}\left[t_\omega(z,z_i)\right]},
\end{align*}
where we introduced $t_{\omega}(z,z')\coloneqq\partial_{v}^2\ell_{in}(\omega,x,y)\partial_{v}^2\ell_{in}(\omega,x',y')K^2(x,x')\in \mathcal{T}_{2,2}$. Here, (a) follows by Cauchy-Schwarz inequality, (b) is obtained by definition of $\Dinhh$, while (c) uses the general fact that the operator norm of an operator is upper-bounded by its Hilbert-Schmidt norm which is finite in our case by application of \cref{prop:fre_diff_L_v}. Moreover, (d) further uses the identity in \cref{lem:hs_identity} for computing the Hilbert-Schmidt norm of sum/expectation of tensor-product operators. 
The last item (d) is precisely what \cref{lem:u_stat_sup_p_omega} controls when applying it to the family $\mathcal{T}_{2,2}$  and choosing $\PP$ to be $\mathbb{P}$.  Therefore, the following maximal inequality holds by direct application of \cref{lem:u_stat_sup_p_omega}:
 \begin{align*}
 	\mathbb{E}_{\mathbb{P}}\brackets{\sup_{\omega\in \Omega} \Dinhh }\leq 
 	\lambda^{-\frac{1}{4}}n^{-\frac{1}{2}} \parens{c(\Omega)\max\parens{\widetilde{M}_{in,2}\widetilde{L}_{in,2}\diam(\Omega),\Lambda\widetilde{M}^2_{in,2}}}^{\frac{1}{4}},
 \end{align*}
 where $c(\Omega)$ is a positive constant greater than $1$ that depends only on $\Omega$ and $d$.
\end{proof}
\subsection{Proof of \cref{th:generalizationBounds}}\label{app_sub:main_proof}
\begin{theorem}[Generalization bounds]\label{th:gen_bound}
The following holds under \cref{assump:compact,assump:convexity_lin,assump:K_bounded,assump:reg_lin_lout}:
\begin{align*}
    \mathbb{E}\brackets{\sup_{\omega\in\Omega}\verts{\mathcal{F}(\omega)-\hat{\mathcal{F}}(\omega)}}
    &\lesssim
    \frac{1}{\lambda m^{\frac{1}{2}}}+\frac{C_{out}}{\lambda^{\frac{5}{4}} n^{\frac{1}{2}}}\\
    \mathbb{E}\left[\sup_{\omega\in\Omega}\left\|\nabla\mathcal{F}(\omega)-\widehat{\nabla\mathcal{F}}(\omega)\right\|\right]
    &\lesssim
    \frac{1}{\lambda}\left(d^{\frac{1}{2}} + \frac{C_{in}}{\lambda^{\frac{1}{4}}}\right)\frac{1}{m^{\frac{1}{2}}} + \frac{C_{out}}{\lambda^{\frac{5}{4}}}\parens{2 + 3\frac{C_{in}}{\lambda}+\frac{C_{in}^2}{\lambda^2}}\frac{1}{n^{\frac{1}{2}}},
\end{align*}
where the constants $C_{in}$ and $C_{out}$ are given in \cref{prop:lip_continuity_out}.
\end{theorem}

\begin{proof}
Using the point-wise estimates in \cref{prop:grad_app_bound} and taking their supremum over $\Omega$ followed by the expectations over data, the following error bounds hold:
\begin{align*}
   \mathbb{E}\brackets{\sup_{\omega\in\Omega} \verts{\mathcal{F}(\omega)-\hat{\mathcal{F}}(\omega)}}
   \leq & 
    \mathbb{E}_{\mathbb{Q}}\brackets{\sup_{\omega\in\Omega}\Dout}
    +\frac{C_{out}}{\lambda}\mathbb{E}_{\mathbb{P}}\brackets{\sup_{\omega\in\Omega}\Dinh},\\
    \mathbb{E}\brackets{\sup_{\omega\in\Omega}\Verts{\nabla\mathcal{F}(\omega)-\widehat{\nabla\mathcal{F}}(\omega)}}
 	\leq &  
    \mathbb{E}_{\mathbb{Q}}\brackets{\sup_{\omega\in\Omega}\Doutw}  + \frac{C_{in}}{\lambda}\mathbb{E}_{\mathbb{Q}}\brackets{\sup_{\omega\in\Omega}\Douth}\\ 
    &+\frac{C_{out}}{\lambda}\parens{1 + 2\frac{C_{in}}{\lambda}+\frac{C_{in}^2}{\lambda^2}} \mathbb{E}_{\mathbb{P}}\brackets{\sup_{\omega\in\Omega}\Dinh}\\
    &+ \frac{C_{out}C_{in}}{\lambda^2}\mathbb{E}_{\mathbb{P}}\brackets{\sup_{\omega\in\Omega}\Dinhh} + \frac{C_{out}}{\lambda}\mathbb{E}_{\mathbb{P}}\brackets{\sup_{\omega\in\Omega}\Dinwh}.
\end{align*}
Furthermore, we can use the maximal inequalities in  \cref{prop:exp_uni_bound,prop:exp_uni_bound_2} to control each term appearing in the right-hand side of the above inequalities:
\begin{align*}
    \mathbb{E}\brackets{\sup_{\omega\in\Omega} \verts{\mathcal{F}(\omega)-\hat{\mathcal{F}}(\omega)}}
   \leq & 
   R\parens{ m^{-\frac{1}{2}}\lambda^{-1}  + C_{out}n^{-\frac{1}{2}}\lambda^{-(1+\frac{1}{4})}}\\
    \mathbb{E}\brackets{\sup_{\omega\in\Omega}\Verts{\nabla\mathcal{F}(\omega)-\widehat{\mathcal{F}}(\omega)}}
 	\leq &  R\Big( m^{-\frac{1}{2}}\lambda^{-1}d^{\frac{1}{2}} + C_{in}m^{-\frac{1}{2}}\lambda^{-(1+\frac{1}{4})} + C_{out}n^{-\frac{1}{2}}\lambda^{-(1+\frac{1}{4})}\parens{1 + 2\frac{C_{in}}{\lambda}+\frac{C_{in}^2}{\lambda^2}}\\
 	&\qquad+ C_{out}C_{in}n^{-\frac{1}{2}}\lambda^{-(2+\frac{1}{4})}  +  C_{out}n^{-\frac{1}{2}}\lambda^{-(1+\frac{1}{4})}\Big),
\end{align*}
where the constant $R$ depends only on the Lipschitz constants $\lipin$, $\lipout$, the upper-bounds $M_{in}$, $M_{out}$, the bound $\kappa$ on the kernel, the set $\Omega$ and the dimension $d$. Rearranging the obtained upper-bounds concludes the proof.
\end{proof}

The proof of Theorem \ref{th:convergence} is provided in Appendix \ref{appendix:C}, based on \citet{zhao2024galorememoryefficientllmtraining}. Note that while both GaLore and SubTrack-Grad assume that the subspace remains unchanged for the proof of convergence, GaLore must limit the number of updates to ensure convergence, as each update can potentially change the entire subspace. In contrast, SubTrack-Grad leverages rank-\(1\) updates to the subspace, preventing drastic changes with each update. While a deeper analysis of slowly changing subspaces and their impact on convergence remains an open problem, in practice, this allows SubTrack-Grad to perform more frequent updates.

Here we investigate the Grassmannian update rule presented in \eqref{eq:update-role}, which is
a direct application of Grassmann geometry \citep{edelman1998geometryalgorithmsorthogonalityconstraints, Bendokat_2024}.

\begin{definition}[\textbf{Exponential Map}]
\label{def:exp-map}
    The exponential map \( \exp_p : T_pM \to M \) on a Riemannian manifold \( M \) is a mapping that assigns the point \( \gamma(1) \in M \) to each tangent vector \( \Delta \in T_pM \), where \( T_pM \) is the tangent space of \( M \) at \( p \), and \( \gamma \) is the unique geodesic originating at \( p \) with initial velocity \( \Delta \). This map establishes a relationship between geodesics and the Riemannian exponential, such that \( \gamma(t) = \exp_p(t\Delta) \) for \( t \in \mathbb{R} \).
 \end{definition}
 \begin{definition}[\textbf{Stiefel Manifold}]\label{def:st}
    The Stiefel manifold \(St(n, p)\), parametrizes the set of all \(n \times p\) orthonormal matrices \(U\), each representing a rank-\(p\) subspace of \(\mathbb{R}^n\).
 \end{definition}
  \begin{definition}[\textbf{Grassmann Manifold}]\label{def:gr}
    The Grassmannian manifold \(Gr(n, p)\) parametrizes the set of all \(p\)-dimensional subspaces of \(\mathbb{R}^n\). Each point can be represented by a projection matrix \(P = UU^T\), where \(U \in St(n, p)\).
 \end{definition}
\begin{restatable}[\bf Grassmann Exponential]{theorem}{updaterule}
\label{theorem:gr_exp} 
    Let \( P = UU^T \in \text{Gr}(n,p) \) be a point on the Grassmannian, where \( U \in \text{St}(n,p) \) is the orthonormal basis of the corresponding subspace. Consider a tangent vector \( \Delta \in T_P\text{Gr}(n,p) \), and let \( \Delta_U^{\text{hor}} \) denote the horizontal lift of \( \Delta \) to the horizontal space at \( U \) in the Stiefel manifold \( \text{St}(n,p) \). Suppose the thin SVD of \( \Delta_U^{\text{hor}} \) is given by
\( \Delta_U^{\text{hor}} = \hat{Q} \Sigma V^T, \)
    where \( \hat{Q} \in \text{St}(n,r) \), \( \Sigma = \text{diag}(\sigma_1, \ldots, \sigma_r) \) contains the nonzero singular values of \( \Delta_U^{\text{hor}} \) with \( r = \min(p, n-p) \), and \( V \in \text{St}(p,r) \). The Grassmann exponential map, representing the geodesic emanating from \( P \) in the direction \( \Delta \), is given by:
    \begin{equation*}
    \small
    \text{Exp}_P^{\text{Gr}}(t\Delta) = \big[ UV\cos(t\Sigma)V^T + \hat{Q}\sin(t\Sigma)V^T + UV_\perp V_\perp^T \big],
    \end{equation*}
    where \( V_\perp \in \mathbb{R}^{p \times (p-r)} \) is any orthogonal complement of \( V \).
\end{restatable}


The proof of Theorem \ref{theorem:gr_exp} can be found in Appendix \ref{appendix:grassmann-proof}. Leveraging this theorem and our notation in section \ref{sec:method}, one can easily verify that the subspace update rule is as follows:
\begin{equation*}
\small
    S_{t+1}(\eta) = (S_t\widehat{V}_F \quad \widehat{U}_F) \begin{pmatrix} \cos{\widehat{\Sigma}_F \eta} \\ \sin{\widehat{\Sigma}_F \eta} \end{pmatrix} \widehat{V}^\top_F + S_t(I - \widehat{V}_F\widehat{V}^\top_F)
\end{equation*}
This update rule generally converges to a stable subspace if the step size \(\eta\) decreases over time \citep{balzano2011onlineidentificationtrackingsubspaces}. However, a decreasing step size can impair the ability to accurately track and adapt to subspace changes. Consequently, SubTrack-Grad uses a constant step size during training and fine-tuning to effectively adjust subspaces. This approach does not hinder convergence, as proved in Theorem \ref{th:convergence}, which guarantees convergence as long as changes are controlled to maintain the stable subspace assumption. 
