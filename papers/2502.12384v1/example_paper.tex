%%%%%%%% ICML 2025 EXAMPLE LATEX SUBMISSION FILE %%%%%%%%%%%%%%%%%

\documentclass{article}

% Recommended, but optional, packages for figures and better typesetting:
\usepackage{microtype}
\usepackage{graphicx}
\usepackage{subfigure}
\usepackage{booktabs} % for professional tables

% hyperref makes hyperlinks in the resulting PDF.
% If your build breaks (sometimes temporarily if a hyperlink spans a page)
% please comment out the following usepackage line and replace
% \usepackage{icml2025} with \usepackage[nohyperref]{icml2025} above.
\usepackage{hyperref}


% Attempt to make hyperref and algorithmic work together better:
\newcommand{\theHalgorithm}{\arabic{algorithm}}

% Use the following line for the initial blind version submitted for review:
% \usepackage{icml2025}

% If accepted, instead use the following line for the camera-ready submission:
\usepackage[accepted]{icml2025}

% For theorems and such
\usepackage{amsmath}
\usepackage{amssymb}
\usepackage{mathtools}
\usepackage{amsthm}
\usepackage{url}
\def\UrlBreaks{\do\/\do-}

% if you use cleveref..
\usepackage[capitalize,noabbrev]{cleveref}

%%%%%%%%%%%%%%%%%%%%%%%%%%%%%%%%
% THEOREMS
%%%%%%%%%%%%%%%%%%%%%%%%%%%%%%%%
\theoremstyle{plain}
\newtheorem{theorem}{Theorem}[section]
\newtheorem{proposition}[theorem]{Proposition}
\newtheorem{lemma}[theorem]{Lemma}
\newtheorem{corollary}[theorem]{Corollary}
\theoremstyle{definition}
\newtheorem{definition}[theorem]{Definition}
\newtheorem{assumption}[theorem]{Assumption}
\theoremstyle{remark}
\newtheorem{remark}[theorem]{Remark}

% Todonotes is useful during development; simply uncomment the next line
%    and comment out the line below the next line to turn off comments
%\usepackage[disable,textsize=tiny]{todonotes}
\usepackage[textsize=tiny]{todonotes}

% user defined
\usepackage{booktabs}       % professional-quality tables
\usepackage{amsfonts}       % blackboard math symbols
\usepackage{bm}
% \usepackage{algpseudocode}
\usepackage{enumitem}
\usepackage{xcolor}

\usepackage{subfigure}
\usepackage{wrapfig}


\newcommand{\zz}[1]{\textcolor{red}{#1}}
\newcommand{\yequan}[1]{\textcolor{blue}{#1}}
\newcommand{\xinling}[1]{\textcolor{green}{#1}}

\usepackage{amsmath}
\usepackage{amssymb}
\usepackage{mathtools}
\usepackage{amsthm}
\DeclareMathAlphabet \mathbfcal{OMS}{cmsy}{b}{n}

\newcommand{\ten}[1]{\mathbfcal{#1}} %mathcal
\newcommand{\mat}[1]{\mathbf{#1}}
\newcommand{\tmat}[1]{\mathbf{\tilde{#1}}}

\newcommand{\SL}[1]{\textcolor{purple}{SL: #1}}

\usepackage{subfiles}
\usepackage{float}

% The \icmltitle you define below is probably too long as a header.
% Therefore, a short form for the running title is supplied here:
\icmltitlerunning{Scalable Back-Propagation-Free Training of Optical Physics-Informed Neural Networks}

\begin{document}

\twocolumn[
\icmltitle{Scalable Back-Propagation-Free Training of Optical Physics-Informed Neural Networks}

% It is OKAY to include author information, even for blind
% submissions: the style file will automatically remove it for you
% unless you've provided the [accepted] option to the icml2025
% package.

% List of affiliations: The first argument should be a (short)
% identifier you will use later to specify author affiliations
% Academic affiliations should list Department, University, City, Region, Country
% Industry affiliations should list Company, City, Region, Country

% You can specify symbols, otherwise they are numbered in order.
% Ideally, you should not use this facility. Affiliations will be numbered
% in order of appearance and this is the preferred way.
\icmlsetsymbol{equal}{*}

\begin{icmlauthorlist}
\icmlauthor{Yequan Zhao}{equal,ucsb}
\icmlauthor{Xinling Yu}{equal,ucsb}
\icmlauthor{Xian Xiao}{hpe}
\icmlauthor{Zhixiong Chen}{ucsb}
\icmlauthor{Ziyue Liu}{ucsb}
\icmlauthor{Geza Kurczveil}{hpe}
\icmlauthor{Raymond G. Beausoleil}{hpe}
%\icmlauthor{}{sch}
\icmlauthor{Sijia Liu}{msu}
\icmlauthor{Zheng Zhang}{ucsb}
%\icmlauthor{}{sch}
%\icmlauthor{}{sch}
\end{icmlauthorlist}

\icmlaffiliation{ucsb}{Department of Electrical and Computer Engineering, University of California, Santa Barbara}
\icmlaffiliation{hpe}{Hewlett Packard Labs, Hewlett Packard Enterprise}
\icmlaffiliation{msu}{Department of Computer Science and Engineering, Michigan State University}

\icmlcorrespondingauthor{Yequan Zhao}{yequan\_zhao@ucsb.edu}
% \icmlcorrespondingauthor{Zheng Zhang}{zhengzhang@ece.ucsb.edu}

% You may provide any keywords that you
% find helpful for describing your paper; these are used to populate
% the "keywords" metadata in the PDF but will not be shown in the document
\icmlkeywords{Machine Learning, ICML}

\vskip 0.3in
]

% this must go after the closing bracket ] following \twocolumn[ ...

% This command actually creates the footnote in the first column
% listing the affiliations and the copyright notice.
% The command takes one argument, which is text to display at the start of the footnote.
% The \icmlEqualContribution command is standard text for equal contribution.
% Remove it (just {}) if you do not need this facility.

% \printAffiliationsAndNotice{}  % leave blank if no need to mention equal contribution
\printAffiliationsAndNotice{\icmlEqualContribution} % otherwise use the standard text.

\begin{abstract}
Physics-informed neural networks (PINNs) have shown promise in solving partial differential equations (PDEs), with growing interest in their energy-efficient, real-time training on edge devices. Photonic computing offers a potential solution to achieve this goal because of its ultra-high operation speed. However, the lack of photonic memory and the large device sizes prevent training real-size PINNs on photonic chips. This paper proposes a completely back-propagation-free (BP-free) and highly salable framework for training real-size PINNs on silicon photonic platforms. Our approach involves three key innovations: (1) a sparse-grid Stein derivative estimator to avoid the BP in the loss evaluation of a PINN, (2) a dimension-reduced zeroth-order optimization via tensor-train decomposition to achieve better scalability and convergence in BP-free training, and (3) a scalable on-chip photonic PINN training accelerator design using photonic tensor cores. We validate our numerical methods on both low- and high-dimensional PDE benchmarks. Through circuit simulation based on real device parameters, we further demonstrate the significant performance benefit (e.g., real-time training, huge chip area reduction) of our photonic accelerator.
%Our framework addresses the fundamental challenges of photonic AI and will enable real-time training of real-size PINNs on photonic chips.   
\end{abstract}

\vspace{-5pt}
\section{Introduction} 
Partial differential equations (PDE) are used to describe numerous engineering problems, such as electromagnetic and thermal analysis of IC chips~\citep{kamon1993fasthenry,li2004efficient}, medical imaging~\citep{villena2015marie}, and safety verification of autonomous systems~\citep{bansal2021deepreach}. Traditional numerical solvers (\textit{e.g.,} finite-difference, finite-element methods) have been well studied, but they are prohibitively expensive for high-dimensional PDEs due to the exponential increase of the unknown variables with respect to spatial/temporal/parameter dimensions. This bottleneck becomes more significant in PDE-constrained inverse and control problems, since the forward problem needs to be solved many times in an outer iteration loop. 

Physics-informed neural networks (PINNs)~\citep{lagaris1998artificial,dissanayake1994neural,raissi2019physics} have emerged as a promising approach to solve both forward and inverse problems. Due to the discretization-free nature, PINN is more suitable for solving high-dimensional or parametric PDEs, but current PINN training is still expensive. For example, training a PINN for robotic safety analysis~\citep{bansal2021deepreach} can easily take $>10$ hours on a powerful GPU. 
Despite the development of operator learning~\citep{lu2021learning}, a PINN often needs to be trained \textit{from scratch} again to obtain a high-quality solution once the PDE initial/boundary conditions or measurement data change. In applications such as safety verification and control~\citep{bansal2021deepreach,onken2021neural} of autonomous systems, dynamic modeling of high-speed vehicles~\citep{chrosniak2023deep}, state monitoring of electricity grids~\cite{huang2022applications}, and electrical property tomography~\citep{yu2023pifon}, the underlying PINNs must be trained {\it ultra-fast} with constrained resources [e.g. limited hardware size, weight and power (SWaP)] to enable real-time decision making. 

It is difficult to achieve (almost) real-time PINN training on a resource-constrained edge platform via the conventional electronic computing paradigm. As a result, emerging computing platforms, such as integrated photonics, are considered to achieve this ambitious goal~\citep{NaPSAC}. Photonic computing provides a promising low-energy and high-speed solution for various AI tasks because of the ultra-high operation speed of light. 
Many optical neural network (ONN) inference accelerators have been proposed~\citep{shen2017deep,tait2016microring,zhu2022space}. 
However, designing a photonic training accelerator for real-size PINNs (e.g., a network with hundreds of neurons per layer) remains an open question due to two fundamental challenges.
\begin{itemize}[leftmargin=*]
\vspace{-10pt}
    \item {\bf Large device footprints and low integration density.} Photonic multiply-accumulate (MAC) units such as Mach-Zehnder interferometers (MZIs) are much larger ($\sim$10s of microns) than CMOS transistors. A real-size PINN with $>10^5$ parameters can easily exceed the available chip size with the square scaling rule, where a $N \times N$ weight matrix requires $O(N^2)$ MZIs ~\citep{triMZI,rectangleMZI}. In fact, even the state-of-the-art photonic AI {\it inference} accelerator~\citep{Ramey2020} can only handle $64\times 64$ weight matrices. Training a PINN on an photonic chip will face more significant scalability issue. 
\vspace{-5pt}
    \item {\bf Difficulty of on-chip back propagation (BP).} It is hard to realize BP on photonic chips due to the lack of memory to store the computational graphs and intermediate results. Several BP-free and \textit{in-situ} BP methods~\citep{GuDAC, GuAAAI, filipovich2022silicon, buckley2022general, oguz2023forward,hughes2018training, pai2023experimentally} are proposed, but their scalability remains a major bottleneck. This becomes more severe in PINN, since its loss function also includes (high-order) derivative terms. Subspace learning~\citep{gu2021l2ight} may scale up BP-based training, but still needs to save intermediate states. %Due to lack of photonic memory, additional optical-electronic-optical conversion is needed, leading to dramatic energy and latency overhead.
\end{itemize}  
\vspace{-5pt}

% but with an unrealistic hardware complexity to implement the sparse back-propagation computation.
BP-free training methods, especially stochastic zeroth-order optimization (ZO)~\citep{nesterov2017random, liu2020primer} or forward-forward method~\citep{hinton2022forward}, are easier to implement on edge hardware, since they do not need to detect or save any intermediate states. However, the scalability issue remains in {\it end-to-end} training, since the dimension-dependent gradient estimation error causes slow or even no convergence on PINNs with hundreds of neurons per layer. ZO training shows great success in fine-tuning large language models (LLMs)~\citep{malladi2023fine,yang2024adazeta,zhangrevisiting,gautamvariance}, since the gradient of a well-pretrained LLM has a low intrinsic dimension on fine-tuning tasks. Unfortunately, such a low-dimensional structure does not exist in end-to-end training, preventing the convergence of ZO optimization in training realistic PINNs. \citep{GuDAC,GuAAAI} utilized ZO training on a photonic chip, but it only fine-tuned a small portion of model parameters based on an offline pre-trained model.  


% We intend to approximate the derivative terms in PINN's loss and update model parameters using only a few forward evaluations.
% We intend to perform BP-free training via stochastic zeroth-order (ZO) optimization~\cite{duchi2015optimal,nesterov2017random,shamir2017optimal,balasubramanian2022zeroth,liu2020primer}, 
% which uses a few forward evaluations to  as well as the gradients of model parameters. 
% In the realm of deep learning, ZO optimization was primarily used for crafting black-box adversarial examples to assess neural network robustness~\cite{chen2017zoo, liu2020min}, and for parameter and memory-efficient model fine-tuning~\cite{malladi2023fine,zhang2022how}. Notably, ZO optimization has been rarely used in neural network training from scratch, because the variance of the ZO gradient estimation is large when the number of training variables increases. 
% We then present a scalable BP-free approach for training PINNs and propose the first scalable optical training framework that can handle realistic large-size PINNs on the integrated photonic platform.

Different from the recent work of fine-tuning~\citep{GuDAC,GuAAAI,malladi2023fine,yang2024adazeta,zhangrevisiting,gautamvariance}, we investigate  {\bf end-to-end BP-free training of real-size PINNs on photonic chips from scratch}. This is a more challenging task because of (1) the differential operators in the PINN loss evaluation, and (2) the large number of optimization variables that cause divergence in end-to-end ZO training, (3) the scalability issue and lack of photonic memory on photonic chips.
This paper presents, for the first time, a {\bf real-size} and {\bf real-time} photonic accelerator to train PINNs with hundreds of neurons per layer on an integrated photonic platform. \textbf{Our novel contributions are summarized as follows:}
\begin{itemize}[leftmargin=*]
\vspace{-5pt}
    \item{\bf Two-Level BP-free PINN Training.} We present novel BP-free approaches in two implementation levels of PINN training. Firstly, we propose a sparse-grid Stein estimator to calculate the (high-order) derivative terms in the PINN loss. Secondly, we propose a tensor-compressed variance reduction approach to improve the convergence of ZO-SGD. These innovations can completely bypass the need for photonic memory, and greatly improve the convergence of on-chip BP-free training. 
    \vspace{-5pt}
    \item{\bf A Scalable Photonic Design.} We present a scalable and easy-to-implement photonic accelerator design. 
    We reuse a tensorized ONN inference accelerator, and just add a digital controller to implement on-chip BP-free training. We present two designs: one implements the whole model on a single chip, and another uses a single photonic tensor core with time multiplexing. Our design can scale up to train real-size PINNs with hundreds of neurons per layer.
    % We propose a low-rank tensor-compressed BP-free training that effectively reduces the training dimensionality. This approach greatly improves the convergence of BP-free training and reduces the hardware overhead of photonics devices, thus scaling up BP-free training to train realistic large-size PINNs on integrated photonics chips.
    \vspace{-5pt}
    \item {\bf Numerical Experiments and Hardware Emulation.} We validate our method in solving a variaty of PDEs. Our two-level BP-free PINN training achieves a competitive error compared to standard PINN training with BP, and achieves the lowest error compared with previous photonic on-chip training methods.
    We further evaluate the performance of our photonic training accelerator on solving a Black-Scholes PDE. The simulation results show that our design can reduce the number of MZIs by $42.7\times$, with only 1.64 seconds to solve this equation. %\zz{make sure this is the correct data for the Black-Scholes example.}
\end{itemize} 
\vspace{-5pt}
   
To the best of our knowledge, this is the first optical PINN training framework to solve real-size PDEs. Our approach shows the great promise of photonic computing in solving AI-based scientific computing problems. Our results can also be extended to lightweight neural networks other than PINNs and other edge platforms, as shown in Appendix \ref{apdx:Broader Impacts}.

% Xinling
% We need to mention the motivation of PINNs, why not FDM/FEM, and why training PINNs on device
% Various methods, such as finite difference \cite{strikwerda2004finite} and finite element \cite{rao2017finite}, have been developed to analyze physical systems and approximate their behavior by numerically solving PDEs. However, solving high-dimensional PDEs with numerical methods presents challenges due to the curse of dimensionality in problem discretization. In response to this challenge, Physics-Informed Neural Networks (PINNs) \cite{raissi2019physics} have emerged as an effective mesh-free approach for automatically solving PDEs. 
% In many real-world scientific and engineering applications, physical systems can often be described by the following general form of partial differential equation (PDE):
% , which can either be fixed during the training or dynamically sampled every iteration

% \vspace{-10pt}
\section{Background}
% \vspace{-5pt}
%This section introduces the necessary background of Physics-Informed Neural Networks (PINN) as well as Optical Neural Networks (ONN). 

{\bf Physics-Informed Neural Networks (PINNs).}
Consider a generic PDE:
\begin{equation}
\begin{aligned}
\mathcal{N}[\bm{u}(\bm{x},t)]&=l(\bm{x}, t), ~\quad \bm{x} \in \Omega,~~t \in[0, T],\\
\mathcal{I}[\bm{u}(\bm{x},0)] &= g(\bm{x}), ~~~\quad \bm{x} \in \Omega,\\
\mathcal{B}[\bm{u}(\bm{x},t)] &= h(\bm{x}, t),  \quad \boldsymbol{x} \in \partial \Omega, ~~t \in[0, T],  
\end{aligned}
\label{general PDE}
\end{equation}
where $\bm{x}$ and $t$ are the spatial and temporal coordinates; $\Omega \subset \mathbb{R}^{D}$, $\partial \Omega$ and $T$ denote the spatial domain, domain boundary and time horizon, respectively; $\mathcal{N}$ is a nonlinear differential operator; $\mathcal{I}$ and $\mathcal{B}$ represent the initial and boundary condition; $\bm{u} \in \mathbb{R}^{n}$ is the solution for the PDE described above. In the contexts of PINNs~\citep{raissi2019physics}, a solution network $\bm{u}_{\bm{\theta}}(\bm{x},t)$, parameterized by $\bm{\theta}$, is substituted into PDE \eqref{general PDE}, resulting in a residual defined as:
\begin{equation}
r_{\bm{\theta}}(\bm{x},t):=\mathcal{N}[\bm{u}_{\bm{\theta}}(\bm{x},t)]-l(\bm{x}, t).
\label{PDE residual}
\end{equation}
The parameters $\bm{\theta}$ can be trained by minimizing the loss:
\begin{equation}
\mathcal{L}(\bm{\theta})=\mathcal{L}_r(\bm{\theta})+\lambda_{0}\mathcal{L}_0(\bm{\theta})+\lambda_{b}\mathcal{L}_b (\bm{\theta}).
\label{PINNs loss}
\end{equation}
Here $\mathcal{L}_r (\bm{\theta})$, $\mathcal{L}_0 (\bm{\theta})$ and $\mathcal{L}_b (\bm{\theta})$ are the residuals associated with the PDE operator, the initial condition and boundary condition, respectively. The residual of PDE operator
\begin{equation}
\begin{aligned}
\mathcal{L}_r (\bm{\theta}) &= \frac{1}{N_{r}}\sum_{i=1}^{N_{r}}  \left\|r_{\bm{\theta}}(\bm{x_{r}}^{i},t_{r}^{i})\right\|_{2}^{2}, 
\end{aligned}
\label{loss terms}
\end{equation}
involves (high-order) derivative terms. 

% \begin{equation}
% \begin{aligned}
% \mathcal{L}_r (\bm{\theta}) &= \frac{1}{N_{r}}\sum_{i=1}^{N_{r}}  \left\|r_{\bm{\theta}}(\bm{x_{r}}^{i},t_{r}^{i})\right\|_{2}^{2}, \\
% \mathcal{L}_0 (\bm{\theta}) &= \frac{1}{N_{0}}\sum_{i=1}^{N_{0}}\left\|\mathcal{I}[\bm{u}_{\bm{\theta}}(\bm{x_{0}}^{i},0)] - g(\bm{x_{0}}^{i})\right\|_{2}^{2},\\
% \mathcal{L}_b (\bm{\theta}) &= \frac{1}{N_{b}}\sum_{i=1}^{N_{b}}\left\|\mathcal{B}[\bm{u}_{\bm{\theta}}(\bm{x_{b}}^{i},t_{b}^{i})] - h(\bm{x_{b}}^{i}, t_{b}^{i})\right\|_{2}^{2}
% \end{aligned}
% \label{loss terms}
% \end{equation}
%are the residuals of the PDE, the initial condition and boundary condition, respectively. %The data points $\left\{(\bm{x_{r}}^{i},t_{r}^{i})\right\}_{i=1}^{N_r}$, $\left\{(\bm{x_{0}}^{i})\right\}_{i=1}^{N_0}$ and $\left\{(\bm{x_{b}}^{i},t_{b}^{i})\right\}_{i=1}^{N_b}$ are randomly drawn from $\Omega \times [0,T]$, $\Omega$ and $\partial\Omega$, respectively; The weight coefficients $\lambda_0$ and $\lambda_b$ balance the three loss terms in the composite loss function \eqref{PINNs loss}, which can either be user-specified or automatically tuned \cite{wang2021understanding,wang2022and}. 
% \paragraph{PINNs with Hard Constraints.} The preliminary work of PINNs \cite{raissi2019physics} employs a fully connected neural network (FCNN) as an approximation to the PDE solution. This approach has proven to be remarkably successful in a plethora of science and engineering applications that rely on PDEs \cite{raissi2020hidden,jagtap2022physics,bansal2021deepreach,baldan2023physics,yu2023pifon,chen2020physics}. However, in certain situations, FCNN might not be the most effective choice. For instance, in inverse design and safety verification problems, the boundary and terminal conditions in \eqref{general PDE} have to be perfectly met, yet the loss terms in \eqref{loss terms} merely act as soft constraints. To ensure the satisfaction of these conditions, \cite{lu2021physics} impose exactly conditions into the solution network architecture. 
% \cite{liu2022tt} utilizes tensor-compressed neural network \cite{novikov2015tensorizing} to enable training PINNs with limited memory and computing resources.

%Some research~\cite{he2023learning} computed the derivations of $\bm{u}_{\bm{\theta}}(\bm{x},t)$ with respect to $\bm{x}$ and $t$ via a Stein estimator, but computing the gradient with respect to network parameters $\bm{\theta}$ remains a major challenges on edge devices.

{\bf Zeroth-Order (ZO) Optimization.}
We consider minimizing a loss function $\mathcal{L}(\bm{\theta})$ by updating  parameters $\bm{\theta} \in \mathbb{R}^d$ iteratively using a (stochastic) gradient descent method: 
\begin{equation}
    \bm{\theta}_t \leftarrow \bm{\theta}_{t-1}-\alpha \bm{g}
    \label{SGD update}
\end{equation}
where $\bm{g}$ denotes the (stochastic) gradient of the loss $\mathcal{L}$ w.r.t. model parameters $\bm{\theta}$. ZO optimization uses a few forward function queries to approximate the gradient $\bm{g}$:
\begin{equation}
   \bm{g} \approx \hat{\nabla}_{\bm{\theta}}\mathcal{L}(\bm{\theta})=
    \sum_{i=1}^N \frac{1}{N\mu} \left[\mathcal{L}\left(\bm{\theta}+\mu \bm{\xi}_i\right)-\mathcal{L}(\bm{\theta}-\mu \bm{\xi}_i)\right] \bm{\xi}_i.
\label{ZO gradient estimation}
\end{equation}
Here $\{\bm{\xi}_i\}_{i=1}^N$ are some perturbation vectors and $\mu$ is the sampling radius, which is typically small. We consider the random gradient estimator (RGE), in which $\{\bm{\xi}_i\}_{i=1}^N$ are $N$ i.i.d. samples drawn from a distribution $\rho(\bm{\xi})$ with zero mean and unit variance. 
%The expectation of $\hat{\nabla}_{\bm{\theta}}\mathcal{L}$ is unbiased w.r.t. the gradient of the smoothed function $f_\mu(\bm{x}):=\mathbb{E}_{\bm{\xi} \sim \rho(\bm{\xi})}[f(\bm{x}+\mu \bm{\xi})]$, however biased w.r.t. the true gradient $\nabla_{\bm{\theta}}\mathcal{L}$~\citep{berahas2022theoretical}. 
The variance of RGE involves a dimension-dependent factor $O(d/N)$ given $\mu = O(1/\sqrt{N})$~\citep{liu2020primer}. ZO optimization has been used extensively in signal processing and adversarial machine learning \citep{ZOSGD, duchi2015optimal, ZOSCD, chen2019zo,shamir2017optimal,cai2021zeroth}. Recently, ZO optimization has achieved great success in fine-tuning LLMs~\citep{malladi2023fine,yang2024adazeta,zhangrevisiting,gautamvariance}, due to the low intrinsic dimensionality of the gradient information. Without low-dimensional structures, ZO optimization scales poorly in end-to-end training of real-size neural networks due to the large dimension-dependent gradient variance. Recently, \citep{chen2023deepzero} improved the scalability of end-to-end ZO training by exploiting model sparsity, but its coordinate-wise gradient estimation is prohibitively expensive for edge devices or real-time applications. 

% Two perturbation methods are commonly used:
% \begin{itemize}[leftmargin=*]
%     \item {\bf Random gradient estimators (RGE)}:  $\{\bm{\xi}_i\}_{i=1}^N$ are $N$ i.i.d. samples drawn from a distribution $\rho(\bm{\xi})$ with zero mean and unit variance (e.g., a multivariate Gaussian distribution or Rademacher distribution). 
%     \item {\bf Coordinate-wise gradient estimator (CGE)} [or finite difference (FD)]: $N=d$ and $\bm{\xi}_i$ denotes the \textit{i}-th elementary basis vector, with one at the \textit{i}-th coordinate and zeros elsewhere.
% \end{itemize}

% In both cases, the expectation of $\hat{\nabla}_{\bm{\theta}}\mathcal{L}$ is unbiased w.r.t. the gradient of the smoothed function $f_\mu(\bm{x}):=\mathbb{E}_{\bm{\xi} \sim \rho(\bm{\xi})}[f(\bm{x}+\mu \bm{\xi})]$, however biased w.r.t. the true gradient $\nabla_{\bm{\theta}}\mathcal{L}$~\cite{berahas2022theoretical}. 
% In comparison to CGE, RGE is typically more query-efficient, requiring a less number of function evaluations $N$ compared to the number of optimization variables $d$. 
% The variances of both RGE and CGE involve a dimension-dependent factor $O(d/N)$ given $\mu = O(1/\sqrt{N})$~\cite{liu2020primer}.

% The mean squared estimation (MSE) error of RGE is given by \cite{liu2020primer}:
% \yequan{accepted}
% \SL{[If you want to say both RGE and CGE at the same time, then I would suggest just saying the variances of both RGE and CGE involve a dimension-dependent  factor $O(d/N)$ given $\sigma = O(1/\sqrt{N})$ \cite{liu2020primer}.]}
% \begin{equation}
% \begin{aligned}
%     \mathbb{E}\left[\|\hat{\nabla}_{\bm{\theta}}\mathcal{L}(\bm{\theta}) &-\nabla_{\bm{\theta}}\mathcal{L}(\bm{\theta})\|_2^2\right] = O\left(\frac{d}{N}\right)\|\nabla_{\bm{\theta}}\mathcal{L}(\bm{\theta})\|_2^2 \\
%     &+ O\left(\frac{\sigma^2 d^3}{\phi(d)N}\right)
%     + O\left(\frac{\sigma^2 d^3+\sigma^2 d}{\phi(d)}\right).
% \end{aligned}
% \label{ZO estimation error}
% \end{equation}

{\bf Optical Neural Networks (ONN) and On-chip ONN Training.} Photonic AI accelerators are expected to outperform their electronic counterparts due to low latency, ultra-high throughput, high energy efficiency, and high parallelism~\citep{mcmahon2023physics}. 
% Many optical inference accelerators have been reported, such as the MZI meshes~\citep{shen2017deep,rectangleMZI}, microring resonator (MRR) weight banks~\citep{tait2016microring}, MRR crossbar~\citep{ohno2022si}, directional coupler crossbar~\citep{Feldmann2021}, balanced homodyne detection~\citep{hamerly2019large}, and integrated chip diffractive neural network~\citep{zhu2022space}. 
Due to limited scalability, state-of-the-art photonic AI accelerators can only handle weight matrices of size 64$\times$64~\citep{Ramey2020}. As a result, large-scale optical matrices are computed by tiles or blocks with time multiplexing, requiring intensive memory access to store the intermediate data.
% That means that E/O and O/E conversions and DAC/ADCs are involved during memory access. 
\cite{demirkiran2023electro} shows that only $\sim$10\text{\%} of the overall power is consumed in optical devices. Applying a pre-trained model on non-ideal photonic chips usually faces significant performance degradation. On-chip training is essential to mitigate this degradation. 
% Existing on-chip training algorithms include brute-force phase tuning~\citep{shen2017deep}, neuroevolution~\citep{zhang2019efficient}, and an adjoint variable method which requires optical power monitoring inside each device. 
However, there is no access to intermediate states or full gradients on the photonic chip. The existing BP-based method \cite{hughes2018training, wright2022deep, pai2023experimentally} requires external hardware to perform gradient computation, which is bulky and not scalable. Several BP-free methods are proposed to address this issue \citep{GuDAC, GuAAAI, filipovich2022silicon, buckley2022general, oguz2023forward}. However, these methods can only handle a small number of training parameters.

% \vspace{-10pt}
\section{Two-Level BP-free Training for PINNs}
% \vspace{-5pt}
Current PINN training methodologies rely on BP for both loss evaluations [Eq. (\ref{PINNs loss})] and model parameter updates [Eq. (\ref{SGD update})]. These BP computations are hard to implement on photonic chips. This section proposes a two-level BP-free PINN training framework to avoid such a challenge. %We first propose a sparse-grid Stein estimator for BP-free loss evaluation. Then we propose a tensor-compressed ZO optimization for gradient-descent PINN model parameter update. 
This approach improves the convergence of the training framework and the scalability on photonic chips, enabling {\it end-to-end} training of real-size PINNs with hundreds of neurons per layer.

%we propose a dimension-reduced zeroth-order (ZO) optimization via tensor-train decomposition to update the model parameters. With both methods applied, we enable training real-size PINNs (\textit{i.e.,} over 100 neurons per layers) using forward propagations only.

\vspace{-5pt}
\subsection{Level 1: BP-Free PINN Loss Evaluation}

\vspace{-5pt}
\subsubsection{Stein Derivative Estimation}
For an input $\bm{x}\in \mathbb{R}^{D}$ and an approximated PDE solution $\bm{u}_{\bm{\theta}}(\bm{x})\in \mathbb{R}^{n}$ parameterized by $\bm{\theta}$, we consider the first-order derivative $\nabla_{\bm{x}} \bm{u}_{\bm{\theta}}$ and Laplacian $\Delta \bm{u}_{\bm{\theta}}$ involved in the loss function of a PINN training. Our implementation leverages the Stein  estimator~\citep{stein1981estimation}. Specifically, we represent the PDE solution $\bm{u}_{\bm{\theta}}(\bm{x})$ via a Gaussian smoothed model:
\begin{equation}
\bm{u}_{\bm{\theta}}(\bm{x})=\mathbb{E}_{\bm{\delta} \sim \mathcal{N}\left(\bm{0}, \sigma^2 \bm{I}\right)} f_{\bm{\theta}}(\bm{x}+\bm{\delta}),
\label{gaussian smoothed model}
\end{equation}
where $f_{\bm{\theta}}$ is a neural network with parameters $\bm{\theta}$; $\bm{\delta} \in \mathbb{R}^{D}$ is the random noise sampled from a multivariate Gaussian distribution ${\cal N}(\bm{0}, \sigma^2 \bm{I})$. The first-order derivative and Laplacian of $\bm{u}_{\bm{\theta}}(\bm{x})$ can be written as:
\begin{equation}
\begin{aligned}
 \nabla_{\bm{x}} \bm{u}_{\bm{\theta}}=\mathbb{E}_{\bm{\delta} \sim \mathcal{N}\left(\bm{0}, \sigma^2 \bm{I}\right)} & \left[\frac{\bm{\delta}}{2 \sigma^2}(f_{\bm{\theta}}(\bm{x}+\bm{\delta}) - f_{\bm{\theta}}(\bm{x}-\bm{\delta}))\right],\\
\Delta \bm{u}_{\bm{\theta}}=\mathbb{E}_{\bm{\delta} \sim \mathcal{N}\left[\bm{0}, \sigma^2 \bm{I}\right)}  & \left [f_{\bm{\theta}}(\bm{x}+\bm{\delta}) +
f_{\bm{\theta}}(x-\bm{\delta})-2 f_{\bm{\theta}}(\bm{x})\right]\\
& \times \frac{\|\bm{\delta}\|^2-\sigma^2 D}{2 \sigma^4}.  %\nonumber
\end{aligned}
\label{stein derivative estimator}
\end{equation}
In~\cite{he2023learning}, the above expectation is computed by evaluating $f_{\bm{\theta}}(\bm{x}+\bm{\delta})$ and $f_{\bm{\theta}}(\bm{x}-\bm{\delta})$ at a set of i.i.d. Monte Carlo samples of $\bm{\delta}$. However, Monte Carlo needs massive (e.g., $>10^3$) function queries. Therefore, it is highly desirable to develop a more efficient BP-free method for evaluating derivative terms in the loss function. 

\vspace{-5pt}
\subsubsection{Sparse-Grid Stein Derivative Estimator}
Now we use the sparse grid techniques~\citep{garcke2006sparse,gerstner1998numerical} to significantly reduce the number of function queries in the Stein derivative estimator, while maintaining high accuracy in numerical integration. This approach has been widely used in uncertainty quantification, but has not been utilized for training PINNs.

We begin with a sequence of univariate quadrature rules $V=\left\{V_l: l \in \mathbb{N}\right\}$. Here $l$ denotes an accuracy level so that any polynomial function of order $\leq l$ can be exactly integrated with $V_l$. Each rule $V_l$ specifies $n_{l}$ nodes $N_{l}=\left\{\delta_{1},\dots,\delta_{n_{l}}\right\}$ and weight function $w_l: N_{l} \rightarrow \mathbb{R}$. With $V_{k}$, the integration a $f$ over a random variable $\delta$ is written as:
\begin{equation}
\int_{\mathbb{R}} f(\delta) p(\delta) \, d\delta \approx 
V_{k}[f]=\sum_{\delta_j \in N_{k}} w_{k}(\delta_j) f(\delta_j).
\end{equation}
Here $p(\delta)$ is the probability density function (PDF) of $\delta$. 

Next, we consider the multivariate integration of a function $f$ over a random vector $\bm{\delta}=(\delta^{1},\dots,\delta^{D})$, where $p(\bm{\delta})=\prod_{m=1}^{D}p(\delta^{m})$ is the joint PDF. Let the multi-index $\bm{l} = (l_1, l_2, ..., l_D) \in \mathbb{N}^{D}$ specify the desired integration accuracy for each dimension. We use the Smolyak algorithm~\citep{gerstner1998numerical} to construct sparse grids. For any non-negative integer $q$, define $\mathbb{N}_{q}^{D} = \left\{\bm{l}\in \mathbb{N}^{D}: \sum_{m=1}^{D}l_{m} = D+q\right\}$ and $\mathbb{N}_{q}^{D} = \emptyset$ for $q<0$. The level-$k$ Smolyak rule $A_{D,k}$ for $D$-dim integration can be written as~\citep{wasilkowski1995explicit}:
\small
\begin{equation}
\begin{aligned}
A_{D,k}[f] =    \sum_{q=k-D}^{k-1} & (-1)^{k-1-q}\left(\begin{array}{c}
D-1 \\ k-1-q
\end{array}\right)  \times \\
& \sum_{\bm{l} \in \mathbb{N}_q^D}       \left(V_{l_{1}} \otimes \cdots \otimes V_{l_{D}} \right)[f] .
\end{aligned}
\end{equation}\normalsize
It follows that:
\begin{equation}
\begin{aligned}
A_{D,k}[f] = \sum_{q=k-D}^{k-1} \sum_{\bm{l} \in \mathbb{N}_q^D} \sum_{\delta^{1} \in N_{l_{1}}} \cdots \sum_{\delta^{D} \in N_{l_{D}}} (-1)^{k-1-q} \times\\
\left(\begin{array}{c}
D-1 \\ k-1-q
\end{array}\right)  \prod_{m=1}^{D} w_{l_{m}}(\delta^{m})f(\delta^{1},\dots,\delta^{D}), \nonumber
\end{aligned}
\end{equation}
which is a weighted sum of function evaluations $f(\bm{\delta})$ for $\bm{\delta} \in \bigcup_{q=k-D}^{k-1} \bigcup_{\mathbf{l} \in \mathbb{N}_q^D} \left(N_{l_{1}} \times \cdots \times N_{l_{D}}\right)$. %The corresponding weight is $(-1)^{k-1-q}\left(\begin{array}{c} D-1 \\ k-1-q \end{array}\right) \prod_{m=1}^{D} w_{l_{m}}(\delta^{m})$. 
For the same $\bm{\delta}$ that appears multiple times for different combinations of values of $\bm{l}$, we only need to evaluate $f$ once and sum up the respective weights beforehand. The resulting level-$k$ sparse quadrature rule defines a set of $n_{L}$ nodes $S_{L}=\left\{\bm{\delta}_{1},\dots,\bm{\delta}_{n_{L}}\right\}$ and the corresponding weights $\left\{w_{1},\dots,w_{n_{L}}\right\}$. The $D$-dim integration can then be efficiently computed with the sparse grids as:
\begin{equation}
\int_{\mathbb{R}^{D}}f(\bm{\delta})p(\bm{\delta}) d \bm{\delta} \approx A_{D,k}[f] = \sum_{j=1}^{n_{L}} w_{j}f(\bm{\delta}_{j}).
\end{equation}
In practice, since the sparse grids and the weights do not depend on $f$, they can be pre-computed for the specific quadrature rule, dimension $D$, and accuracy level $k$. 

Finally, we implement the Stein derivative estimator in Eq. (\ref{stein derivative estimator}) via the sparse-grid integration. Noting that $\bm{\delta} \sim \mathcal{N}\left(\bm{0}, \sigma^2 \bm{I}\right)$, we can use univariate Gaussian quadrature rules as basis to construct a level-$k$ sparse Gaussian quadrature rule $A_{D,k}^{*}$ for $D$-variate integration. Then the first-order derivative and Laplacian in Eq. (\ref{stein derivative estimator})  is approximated as:
\small
\begin{equation}
\begin{aligned}
\nabla_{\bm{x}} \bm{u}_{\bm{\theta}}& \approx  \sum_{j=1}^{n_{L}^{*}}w_{j}^{*}  \left[\frac{\bm{\delta}_{j}^{*}}{2 \sigma^2}(f_{\bm{\theta}}(\bm{x}+\bm{\delta}_{j}^{*}) - f_{\bm{\theta}}(\bm{x}-\bm{\delta}_{j}^{*}))\right],  \\
\Delta \bm{u}_{\bm{\theta}}  & \approx \sum_{j=1}^{n_{L}^{*}}w_{j}^{*} \left(\frac{\|\bm{\delta}_{j}^{*}\|^2-\sigma^2 D}{2 \sigma^4}\right) \times \\
& \left( f_{\bm{\theta}}(\bm{x}+\bm{\delta}_{j}^{*}) +
f_{\bm{\theta}}(x-\bm{\delta}_{j}^{*})-2 f_{\bm{\theta}}(\bm{x}) \right),
\end{aligned}
\label{sparse-grid stein derivative estimator}
\end{equation} \normalsize
where $\bm{\delta}_{j}^{*}$ and $w_{j}^{*}$ are defined by the sparse grid $A_{D,k}^{*}$. 

\begin{figure*}[!t]
    \centering
    % \vspace{-20pt}
    \includegraphics[width=0.8\linewidth]{fig/tensor_train.pdf}
    \caption{Tensor-train decomposition: matrix $\mat{W}$ is folded to a multi-way tensor $\ten{W}$ and decomposed into $L$ small TT cores $\{\ten{G}_k\}_{k=1}^L$.}
    \label{fig:tensor_train}
    \vspace{-10pt}
\end{figure*}

{\bf Remark:} With the sparse-grid Stein estimator in Eq. (\ref{sparse-grid stein derivative estimator}), we can compute the derivatives in Eq. (\ref{PDE residual}) and the loss of PINN in Eq. (\ref{PINNs loss}) without using any BP computation. Recent work explored efficient computation of differential operators \cite{cho2024separable,shi2024stochastic}, however, they still need the automatic differentiation, which is not available for optical neural networks. Our BP-free sparse-grid loss evaluation offers two benefits:
\begin{itemize}[leftmargin=*]
\vspace{-10pt}
    \item The number of forward evaluations (i.e. $n_{L}^{*}$) is usually significantly smaller than the number of Monte Carlo samples required to evaluate Eq. (\ref{stein derivative estimator}). For example, a level-3 sparse-grid Gaussian quadrature for a 3-dim PDE requires only 25 function evaluations, compared to thousands in Monte Carlo-based estimation~\cite{he2023learning}.
    \vspace{-5pt}
    \item The sparse-grid Stein estimator transforms differential operators into the weighted sums of forward evaluations. This often leads to a smoother loss function and thus a better generalization accuracy~\cite{wen2018smoothout}. This will be empirically shown in Section~\ref{subsec:numerical}.
\end{itemize}


\vspace{-10pt}
\subsection{Level 2: Tensor-Compressed ZO Training}\label{subsec:TT}

To avoid BP in the update of PINN model parameters, we use the ZO gradient estimator in Eq (\ref{ZO gradient estimation}) to perform gradient-descent iteration. Considering the inquiry complexity, we use randomized gradient estimation to implement (\ref{ZO gradient estimation}). The gradient mean squared approximation error scales with the perturbation dimension $d$~\citep{berahas2022theoretical}:
$
    \mathbb{E}\left[\|\hat{\nabla}_{\bm{\theta}}\mathcal{L}(\bm{\theta}) - \nabla_{\bm{\theta}}\mathcal{L}(\bm{\theta}) \|_2^2 \right] = O\left(\frac{d}{N}\right)\|\nabla_{\bm{\theta}}\mathcal{L}(\bm{\theta})\|_2^2+O\left(\frac{\mu^2 d^3}{N}\right)+O\left(\mu^2 d\right)
$.
% \begin{equation}
%     \mathbb{E}\left[\|\hat{\nabla}_{\bm{\theta}}\mathcal{L}(\bm{\theta}) - \nabla_{\bm{\theta}}\mathcal{L}(\bm{\theta}) \|_2^2 \right] = O\left(\frac{d}{N}\right)\|\nabla_{\bm{\theta}}\mathcal{L}(\bm{\theta})\|_2^2+O\left(\frac{\mu^2 d^3}{N}\right)+O\left(\mu^2 d\right)
% \end{equation}
The convergence rate also scales with $d$ as $O(\sqrt{d}/\sqrt{T})$ in non-convex unconstrained optimization~\citep{berahas2022theoretical}.
Real-size PINNs typically have hundreds of neurons per hidden layer, and the total number of model parameters can easily exceed $10^5$ or $10^6$.
As a result, ZO optimization converges slowly or even fails to converge in {\it end-to-end} PINN training. 
      
\subsubsection{Tensor-Compressed ZO Optimization.} We propose to significantly reduce the gradient variance via a \textit{low-rank} tensor-compressed training as shown in Fig.~\ref{fig:tensor_train}. Tensor compression has been well studied for functional approximation and data/model compression~\cite{lubich2013dynamical,zhang2016big,novikov2015tensorizing}, but it has not been studied for variance reduction in ZO training.

Let $\bm{W} \in \mathbb{R}^{M\times N}$ be a weight matrix in a PINN. We factorize its dimension sizes as $M = \prod^{L}_{i=1}m_i$ and $N = \prod^{L}_{j=1}n_j$, fold $\bm{W}$ into a $2L$-way tensor $\mathbfcal{W} \in \mathbb{R}^{m_1\times m_2 \times \dots \times m_L \times n_1 \times n_2 \times \dots \times n_L}$, and write $\ten{W}$ with the tensor-train (TT) decomposition \citep{oseledets2011tensor}:
\begin{equation}
{\mathbfcal{W}}(i_1, i_2, \dots, i_L, j_1, j_2, \dots, j_L)
\approx \prod^{L} \limits_{k=1} \mat{G}_k(i_k, j_k)
\end{equation}
Here $\mat{G}_k(i_k, j_k) \in \mathbb{R}^{r_{k-1} \times r_{k}}$ is the $(i_k, j_k)$-th slice of the TT-core $\mathbfcal{G}_k \in \mathbb{R}^{r_{k-1}\times m_k \times n_k \times r_k}$ by fixing its $2$nd and $3$rd indices as $i_k$ and $j_k$, respectively. The vector $(r_0, r_1, \dots, r_{L})$ is called the TT-ranks with $r_0=r_{L}=1$. TT representation reduces the number of variables from $\prod_{k=1}^{L} m_k n_k$ to $\sum_{k=1}^{L}r_{k-1}m_k n_k r_k$. The compression ratio is controlled by the TT-ranks, which can be learnt automatically~\citep{hawkins2021bayesian,hawkins2022towards}. 

In ZO training, we directly train the TT factors $\{\mathbfcal{G}_k \}_{k=1}^L$. 
Take a weight matrix with size $512\times 512$ for example, the original dimension $d=2.62\times 10^5$, while the reduced number of variables in TT factors is $d'=256$ (with tensor size $8\times 4\times 4\times 4 \times 4 \times 4\times 4\times 8$, and TT-rank (1,2,2,2,1)).
This reduces the problem dimensionality $d$ by $1023\times$, leading to dramatic {\it variance reduction} of the ZO gradient in Eq. (\ref{ZO gradient estimation}). As will be shown in Table \ref{tab:weight training}, such dimension reduction does little harm to the model learning capacity, but greatly improves the ZO training convergence.
%In addition, the original matrix-vector product is replaced with low-cost and memory-efficient tensor-network contraction in the forward evaluations~\citep{yang2024comera}. %This offers both memory and computing cost reduction in the ZO training process. 

% \begin{figure*}[t]
%     \centering
%     % \vspace{-15pt}
%     \includegraphics[width=0.74\textwidth]{fig/TONN_training.pdf}
%     \caption{\label{fig:TONN_training} The overall architecture of the BP-free optical training accelerator.}
%     % \vspace{-10pt}
% \end{figure*}

{\bf Comparison with other ZO Training.} Other techniques have also been reported to improve the convergence of ZO training, such as sparse ZO optimization~\citep{chen2023deepzero,liu2024sparse} and ZO variance-reduced gradient descent (SVRG)~\citep{liu2018zeroth}. Although these techniques can improve the convergence of ZO training, they cannot reduce the hardware complexity (i.e., the number of photonic devices needed for hardware implementation). The ZO SVRG method needs storing previous gradient information, thus can cause huge memory overhead and is not suitable for photonic implementation. Our method, as will be shown in Section~\ref{label: photonic design}, can improve both the convergence and scalability of photonic training. Our method may also be combined with these existing approaches to achieve further better performance. %For instance, model sparsity may be exploited at the TT factor level to get further convergence improvement in ZO training. 
We leave this to our future work.  

% In practice, such stochasticity of \textit{ZO gradient estimation based training} actually helps to jump out of sub-optimal at the beginning of training, quickly explores till a roughly converged solution. However, we may still find non-trivial performance degradation due to the stochastic error. 

% \paragraph{A Hybrid ZO Optimizer}\label{subsec:hybrid}
% With the above dimension-reduced zeroth-order optimization, we can employ either RGE or CGE for ZO gradient estimation and perform BP-free training. In practice, the RGE method converges very slowly in the late stage of training due to the large gradient error. CGE needs fewer training epochs due to more accurate gradient estimation, but it needs many more forward evaluations per gradient estimation since it only perturbs one model parameter in each forward evaluation. 

% To enhance both the accuracy and efficiency of the whole ZO training process, we employ a hybrid ZO training scheme that involves two stages: The ZO-RGE coarse training rapidly explores a roughly converged solution with a small number of loss evaluations. When the coarse training fails to learn (e.g., the training loss exhibits trivial updates for several epochs), the optimizer switches to ZO-CGE to fine-tune the model. 


\vspace{-5pt}
\section{Design with Integrated Photonics}\label{label: photonic design}
\vspace{-5pt}

\begin{figure*}[t]
    \centering
    \vspace{-5pt}
    \includegraphics[width=0.9\textwidth]{fig/TONN.pdf}
    \vspace{-10pt}
    \caption{(a) The overall architecture of the BP-free optical training accelerator. (b) TONN space multiplexing (TONN-SM) architecture. (c) TONN time multiplexing (TONN-TM) architecture.}
    \label{fig:TONN}
     \vspace{-10pt}
\end{figure*}

This section presents the design of our photonic PINN training accelerator. Due to the BP-free nature, we can reuse a photonic inference accelerator to easily finish the training hardware design. The tensor-compressed ZO training can greatly reduce the number of required photonic devices, providing much better scalability than existing work. 

    {\bf Overall Architecture.}
    Figure \ref{fig:TONN} (a) shows the architecture of our optical PINN training accelerator. The accelerator consists of an ONN inference engine and a digital control system to implement BP-free PINN training. As explained in Appendix~\ref{appendix:ONN Basics}, standard ONN~\citep{shen2017deep} architecture uses singular value decomposition (SVD) to implement matrix-vector multiplication (MVM), and unitary matrices are implemented with MZI meshes~\citep{rectangleMZI}. For a $N\times N$ weight matrix, this requires $O(N^2)$ MZIs, which is {\it infeasible} for practical PINNs. In contrast, our method utilizes tensor-compressed ZO training, therefore we utilize the tensorized ONN (TONN) accelerator~\citep{xiao2021large} as our inference engine. A TONN inference accelerator only implements the photonic TT cores $\{\ten{G}_k\}_{k=1}^L$ instead of the matrix $\mat{W}$ on an integrated photonic chip, greatly reducing the number of MZIs required for large-scale implementation. 
    % Consequently, the PINN $\bm{u}_{\bm{\Phi}}(\bm{x},t)$ is parameterized by all programmable MZI phases $\bm{\Phi}_k$ in each photonic TT-core $\ten{G}_k(\bm{\Phi}_k)$. 
    The target of on-chip ONN training is to find the optimal MZI phases $\bm{\Phi}$ under various variations. 
    We implement BP-free PINN training by updating the MZI phases $\bm{\Phi}_k$ in each photonic TT core $\ten{G}_k(\bm{\Phi}_k)$. 
    %In the following section, we will give the details of our TONN design and BP-free training implementation.
    
    % The TT-based optical neural network design can greatly reduce the number of photonic devices, latency and energy cost. Furthermore, it can reduce the number of on-chip training variables and improve the convergence of the on-chip training framework.  
    % With our proposed BP-free training approach, we can easily convert the TONN inference accelerator to a training accelerator by adding an additional digital control system. 
    
    {\bf Two Tensorized ONN (TONN) Inference Accelerators.}
    Here we present two designs for the TONN inference. 
    The space multiplexing design (TONN-SM in Fig.~\ref{fig:TONN} (b)) integrates the whole tensor-compressed model on a single chip. Each TT core is implemented by several identical photonic tensor cores.
    Tensor multiplications between input data and all TT-cores are realized in a single clock cycle by cascading the photonic TT-cores in the space domain and adding parallelism in the wavelength domain \citep{xiao2021large}. 
    The time multiplexing design (TONN-TM in Fig.~\ref{fig:TONN}(c)) uses a single wavelength-parallel photonic tensor core~\citep{Xiao2023}. In each clock cycle, the photonic tensor core with parallel processing in the wavelength domain is updated to multiply with the input tensor. The intermediate output data are then stored in the buffer for the next cycle. 
    TONN-SM is fast and ``memory-free". TONN-TM exhibits a smaller footprint at the cost of higher latency and additional memory requirements.
    We provide more details in Appendix \ref{appendix:TONN Implementation Details}.

    % % \begin{wrapfigure}{r}{0.42\textwidth}
    % \begin{figure}[t]
    % % \vspace{-10pt}
    % \centering
    % \includegraphics[width=0.42\textwidth]{fig/TONN_TM.pdf}
    % % \vspace{-20pt}
    % \caption{\label{fig:TONN_2} TONN-TM architecture. 
    % % The designed inference accelerator using a single wavelength-parallel photonic tensor core with time multiplexing.
    % }
    % \vspace{-10pt}
    % \end{figure}
    % % \end{wrapfigure}

    {\bf BP-free On-chip PINN Training.} 
    BP-free training repeatedly calls a TONN inference engine to evaluate the loss and estimate the gradients, then update the MZI phases. 
    To get the ZO gradient $\hat{\nabla}_{\boldsymbol{\Phi}}\mathcal{L}(\boldsymbol{\Phi})$ given by Eq. (\ref{ZO gradient estimation}), the digital control system generates Rademacher random perturbations (entries are integers +1 or -1 with equal probability) and re-program the MZIs with the perturbed phase values $\boldsymbol{\Phi}+\mu \boldsymbol{\xi}$. Here we set $\mu$ as the minimum control resolution of MZI phase tuning. Loss evaluation $\mathcal{L}(\bm{\Phi}+\mu \boldsymbol{\xi})$ requires a few inferences with perturbed input data to estimate first- and second-order derivatives by sparse-grid Stein estimator.
    The digital controller gathers the gradient estimation of $N$ i.i.d. perturbations, and update the MZI phases with $\hat{\nabla}_{\boldsymbol{\Phi}}\mathcal{L}(\boldsymbol{\Phi})$.

    % \begin{figure*}[t]
    % \centering
    % % \vspace{-25pt}
    % \includegraphics[width=0.9\textwidth]{fig/TONN_SM.pdf}
    % \caption{\label{fig:TONN_1} TONN-SM architecture. PTC: photonic tensor core, DAC: digital-analog converter, ADC: analog-digital converter.}
    % % \vspace{-10pt}
    % \end{figure*}
    
 % \vspace{-10pt}
\section{Experimental Results}
% \vspace{-5pt}
% \zz{Per our dicussion, we need to fix the following issue: (1) add comparison with DeepZero (in terms of forward evaluations) in Fig. 5, (2) change Table 4 to the Black-Scholes PDE, emphasize MZI, area and runt-time reduction and de-emphasize energy cost. }
To validate our method, we consider 4 PDE benchmarks: (1) a 1-dim Black-Scholes equation modeling call option price dynamics in financial markets, (2) a 20-dim Hamilton-Jacobi-Bellman (HJB) equation arising from optimal control of robotics and autonomous systems, (3) a 1-dim Burgers' equation \cite{hao2023pinnacle}, (4) a 2-dim Darcy Flow problem \cite{li2020fourier}. Detailed PDE formulations are given in Appendix \ref{appendix:PDE details}. The baseline neural networks are: 3-layer MLPs with 128 neurons per layer and \texttt{tanh} activation for Black-Scholes, 3-layer MLP with 512 neurons per layer and \texttt{sine} activation for 20-dim HJB, 5-layer MLPs with 100 neurons per layer and \texttt{tanh} activation for Burgers' equation and the Darcy flow problem. We remark that MLPs are widely accepted architectures in PINN. The challenges of PINN training arise mainly from the complex optimization landscape caused by differential operators in the loss rather than the chosen neural network architectures or sizes. 

The PINN models are trained using Adam optimizer (learning rate 1e-3), implementing first-order (FO) and ZO training approaches. FO training uses true gradients computed by BP, whereas ZO training uses RGE gradient estimation. For ZO training, we set the query number $N=1$, smoothing factor $\mu=0.01$, and use tensor-wise gradient estimation.
%We train the model for 10000 iterations using Adam as the optimizer using true gradients computed by back-propagation in first-order (\textbf{FO}) training or gradient estimation in zeroth-order (\textbf{ZO}) training. The learning rate is set to 1e-3. In ZO training, the query number $N$ is set as 1, the smoothing factor $\mu$ is set as 0.01, and we apply a tensor-wise gradient estimation scheme 
%(\textit{i.e.,} perturb one tensor and estimate the gradients of that tensor at a time, repeat it sequentially for all tensors, and finally gather all gradients to perform one parameter update step).
We evaluate model accuracy on a hold-out set using the relative $\ell_2$ error $\|\hat{u}-u\|^2 / \|u \|^2$ in domain $\Omega$, where $\hat{u}$ is the model prediction and $u$ is the reference solution. We repeat all experiments three times and record the mean values and standard deviations. Detailed settings are provided in Appendix \ref{appendix:Training Set-ups}.
Anonymous code is available at \href{https://anonymous.4open.science/r/optical_pinn_training-00D6}{\texttt{repo}}.

% \paragraph{Black-Scholes Equation} 
% Considering the following Black-Scholes equation used to describe option price dynamics:
% \begin{equation}
% \begin{aligned}
% &\partial_t u + \frac{1}{2}\sigma^2 x^2 \partial_{xx} u + rx\partial_x u - ru = 0, \quad x \in [0,200],~~t \in [0,T],\\
% &u(x,T) = \max(x-K,0), \quad x \in [0,200],  \\
% &u(0,t) = 0, \quad u(200,t) = 200 - Ke^{-r(T-t)},\quad t \in [0,T],
% \end{aligned}
% \end{equation}
% where $u(x,t)$ is the option price, $x$ is the stock price, $\sigma=0.2$ is volatility, $r=0.05$ is risk-free rate, $K=100$ is strike price, and $T=1$ is expiration time. The true solution is
% \begin{equation}
% u(x,t) = xN(d_1) - Ke^{-r(T-t)}N(d_2),
% \end{equation} where $d_1$ and $d_2$ are calculated as:
% \begin{equation}
% \begin{aligned}
% &d_1 = \frac{\ln(x/K) + (r + \sigma^2/2)(T-t)}{\sigma\sqrt{T-t}}, \\
% &d_2 = d_1 - \sigma\sqrt{T-t},
% \end{aligned}
% \end{equation}
% and $N(\cdot)$ is the cumulative distribution function of the standard normal distribution. The base neural network is a 3-layer MLP with 128 neurons and \texttt{tanh} activation in each hidden layer. In tensor-train (TT) compressed training, the input layer ($2\times 128$) and the output layer ($128\times 1$) are left as-is, while we fold the hidden layer as size $4\times 4\times 8\times 8\times 4\times 4$. We preset the TT-ranks as [1,$r$,$r$,1]. By varying $r$ we can control the compression ratio.

% \paragraph{20-dim HJB Equation}
% We consider the following 20-dim HJB PDE arising from high-dim optimal control of robots and autonomous systems:
% \begin{equation}
% \begin{aligned}
% &\partial_t u(\bm{x}, t)+\Delta u(\bm{x}, t)-0.05 \left\|\nabla_{\bm{x}}u(\bm{x}, t)\right\|_{2}^{2}=-2, \\
% &u(\bm{x}, 1)=\left\|\bm{x}\right\|_{1}, \quad \bm{x} \in [0,1]^{20}, ~~t \in[0, 1].
% \end{aligned}
% \end{equation}
% Here $\left\|\cdot\right\|_{p}$ denotes an $\ell_p$ norm. The exact solution is $u(\bm{x},t)=\left\|\bm{x}\right\|_{1}+1-t$. The base neural network is a 3-layer MLP with 512 neurons and \texttt{sine} activation in each hidden layer. In tensor-train (TT) compressed training, we fold the input layer and the hidden layer as size $1\times 1\times 3\times 7\times 8\times 4\times 4\times 4$ and $4\times 4\times 4\times 8\times 8\times 4\times 4\times 4$, respectively. We preset the TT-ranks as [1,$r$,$r$,$r$,1]. The output layer ($512\times 1$) is left as-is.

\vspace{-5pt}
\subsection{Numerical Results of Solving Various PDEs}
\label{subsec:numerical}
We first evaluate the numerical performance of our BP-free PINNs training algorithm. We conduct training in the \textit{weight domain}, where the trainable parameters are the weight matrices $\mat{W}$ (tensor cores $\ten{G}$ in tensor-compressed training) with tractable gradients to enable FO training as baselines.
% , but not MZI phases $\mat{\Phi}$ due to the intractable gradients of MZI phases $\bm{U}(\bm{\Phi})$ and $\bm{V}(\bm{\Phi})$ that 

%\vspace{-10pt}


% \begin{wraptable}{r}{0.5\textwidth}
\begin{table}[t]
    \vspace{-10pt}
    \begin{minipage}{\linewidth}
        \centering
        \caption{Relative $\ell_2$ error of FO training using different loss computation methods.}
        % \resizebox{\linewidth}{!}{
        % Table generated by Excel2LaTeX from sheet 'ICML'
        \begin{tabular}{c|ccc}
        \toprule
        \toprule
        Problem & AD    & SE    & SG (ours) \\
        \midrule
        Black-Scholes & 5.35E-02 & 5.41E-02 & \textbf{5.28E-02} \\
        20-dim HJB & 1.99E-03 & 1.52E-03 & \textbf{8.16E-04} \\
        Burgers & 1.37E-02 & 1.98E-02 & \textbf{1.31E-02} \\
        Darcy Flow & 7.57E-02 & 7.85E-02 & \textbf{7.47E-02} \\
        \bottomrule
        \bottomrule
        \end{tabular}%
             
        % }
        \label{tab:loss computation}%
    \end{minipage}

    \begin{minipage}{\linewidth}
        \centering
        % \captionsetup{type=table}
        \caption{Relative $\ell_2$ error achieved using different training methods. }
        \resizebox{\linewidth}{!}{
        % Table generated by Excel2LaTeX from sheet 'ICML'
        \begin{tabular}{c|cc|cc}
        \toprule
        \toprule
        Problem & \multicolumn{2}{c|}{FO Training} & \multicolumn{2}{c}{ZO Training} \\
        \cmidrule{2-5}      & Standard & TT    & Standard & TT \\
        \midrule
        Black-Scholes & \textbf{5.28E-02} & 5.97E-02 & 3.91E-01 & \textbf{8.30E-02} \\
        20-dim HJB & 8.16E-04 & \textbf{2.05E-04} & 6.86E-03 & \textbf{1.54E-03} \\
        Burgers & \textbf{1.31E-02} & 4.49E-02 & 4.41E-01 & \textbf{1.63E-01} \\
        Darcy Flow & \textbf{7.47E-02} & 8.77E-02 & 1.34E-01 & \textbf{9.05E-02} \\
        \bottomrule
        \bottomrule
        \end{tabular}%
        
        
        }
        \label{tab:weight training}%
    \end{minipage}
    \vspace{-10pt}
\end{table}
% \end{wraptable}


{\bf Effectiveness of BP-free Loss Computation:} We consider three methods for computing derivatives in the loss of PINN: 1) BP via automatic differentiation (\textbf{AD}) as a gold reference, 2) BP-free Monte Carlo Stein Estimator (\textbf{SE})~\citep{he2023learning} using 2048 random samples, and 3) our proposed BP-free method via sparse-grid (\textbf{SG}). Details are provided in Appendix \ref{appendix:Loss Evaluation Set-ups}.
% We compare our sparse-grid (\textbf{SG}) loss computation with those using automatic-differentiation (\textbf{AD}) and Monte Carlo-based Stein estimator (\textbf{SE}), respectively. 
We perform FO training on standard PINNs for fair comparison. As shown in Table \ref{tab:loss computation}, 
% The BP-free loss computation does not hurt the training performance, and our \textbf{SG} method is competitive compared to the original PINN loss evaluation using \textbf{AD} while requiring much less forward evaluations than {\bf SE}.
our proposed \textbf{SG} outperforms {\bf SE} while requiring much less forward evaluations. In most cases \textbf{SG} outperforms \textbf{AD}. We hypothesize that this is attributed to the smoothed loss that {\it improves generalization}~\cite{wen2018smoothout}. 

%\vspace{-10pt}
 

% % \begin{wrapfigure}{r}{0.5\textwidth}
% \begin{figure}[t]
% % \vspace{-15pt}
%     \begin{minipage}{\linewidth}
%         \centering
%         \includegraphics[width=\linewidth]{fig/weight_domain.pdf}
%         \vspace{-25pt}
%         \caption{Relative $\ell_2$ error curves of weight domain training for Black Scholes equation (left) and 20dim-HJB (right) equation, respectiely.}
%         \label{fig:weight domain}
%     \end{minipage}

%     \begin{minipage}{\linewidth}
%         \centering
%         \includegraphics[width=\linewidth]{fig/deep_zero.pdf}
%         \vspace{-25pt}
%         \caption{Comparison of ZO training methods.}
%         \label{fig:deep zero}
%     \end{minipage}
% % \vspace{-10pt}
% \end{figure}
% % \end{wrapfigure}



{\bf Evaluation of BP-free PINN Training.} We compare the \textbf{FO} training (BP) and \textbf{ZO} training (BP-free) in the standard (\textbf{Std.}) uncompressed and our tensor-train (\textbf{TT}) compressed formats. We employ {\bf SG} loss computation for all experiments. Table \ref{tab:weight training} summarizes the results. 
\textbf{TT dimension-reduction does little harm to the accuracy of the PINN model:}
Tensor-compressed training reduces the dimension by $20.44\times, 142.27\times, 24.74\times, 24.74\times$ for Black-Scholes, 20-dim HJB, Burgers', and Darcy Flow, respectively.
% For the Black-Scholes equation, the dimension of standard training is 17025. The dimension of tensor-compressed training is reduced to 833 (20.44$\times$ fewer) by folding the hidden layer as size $4\times 4\times 8\times 8\times 4\times 4$ and decomposing with a TT-rank $(1,2,2,1)$. 
% For the 20-dim HJB equation, the dimension of standard training is 274433. The dimension of tensor-compressed training is reduced to 1929 (142.27$\times$ fewer) by folding the input layer and the hidden layer as size $1\times 1\times 3\times 7\times 8\times 4\times 4\times 4$ and $4\times 4\times 4\times 8\times 8\times 4\times 4\times 4$, respectively. Both the input layer and hidden layer are decomposed with a TT-rank $(1,2,2,2,1)$. 
The first two columns list the relative $\ell_2$ error achieved after FO training. 
TT compressed training achieves an error similar to standard training with FC hidden layers. We provided ablation studies on different TT-ranks and hidden layer sizes in Appendix \ref{appendix:Ablation Studies} to validate our design. The ablation study shows that our chosen neural networks are {\it not} over-parameterized. 
\textbf{TT dimension reduction greatly improves the convergence of ZO training:}
The last two columns list the relative $\ell_2$ error achieved after ZO training. Standard ZO training fails to converge well due to the high gradient variance which stems from the high dimensionality.
Using TT to reduce the variance of the gradient, our ZO training method achieves much better convergence and final accuracy. 
This showcases that our proposed TT compressed ZO optimization is the key to the success of BP-free training on real-size PINNs. 
The observations above clearly demonstrate that our method can bypass BP in both loss evaluation and model parameter updates, and is still capable of learning a good solution.
More results are available in Appendix \ref{appendix:Detailed Results of Weight-domain Training}.

{\bf Remark.} 
There is an avoidable performance gap between FO training and all ZO training, due to the additional variance term of ZO gradient estimation. While this gap cannot be completely eliminated, it may be narrowed by using more forward passes per iteration in the late training stage to achieve a low-variance ZO gradient [\textit{e.g.,} ZO-RGE with a large N or coordinate-wise gradient estimator used in \texttt{DeepZero}~\citep{chen2023deepzero}]. Overall, our method is the most computation-efficient to train \textit{from scratch}. As shown in Fig. \ref{fig:deep zero}, standard ZO training fails to converge well; \texttt{DeepZero} may eventually converge to a good solution, however, at the cost of $200\times$ more forward passes.

\begin{figure}[t]
    \centering
    \includegraphics[width=\linewidth]{fig/deep_zero.pdf}
    \vspace{-20pt}
    \caption{Training efficiency comparison of ZO training methods.}
    \label{fig:deep zero}
    \vspace{-10pt}
\end{figure}

\begin{figure*}[!t]
    \centering
    \includegraphics[width=\linewidth]{fig/phase_domain.pdf}
     \vspace{-20pt}
    \caption{The first two subfigures show the relative $\ell_2$ error of Black-Scholes and 20-dim HJB equations learned by different ONN training methods. The last two subfigures show the ground truth $u(x)$, and the learned solution $\hat{u}(x)$ using our proposed method.}
    \label{fig:phase domain}
     \vspace{-10pt}
\end{figure*}

\vspace{-5pt}
\subsection{Hardware Performance Simulation} 
We further simulate on-chip \textit{phase-domain} training where the training parameters are MZI phases $\boldsymbol{\Phi}$ that parameterize the weight matrix $\mat{W}(\boldsymbol{\Phi})$ (TT-cores $\ten{G}(\boldsymbol{\Phi})$ in our proposed method). Simulation codes are implemented with an open-source PyTorch-centric ONN library \texttt{TorchONN}. ONN Simulation settings are provided in Appendix \ref{appendix:ONN Simulation Settings}.
We follow~\cite{gu2021l2ight} to consider a hardware-restricted objective that considers various ONN non-idealities. Detailed set-up is provided in Appendix \ref{appendix:ONN Non-ideality}.

%\vspace{-10pt}
{\bf Training Performance.} 
Table \ref{tab:photonics training} compares our method with existing on-chip BP-free ONN training methods, including \texttt{FLOPS}~\citep{GuDAC} and subspace training \texttt{L\textsuperscript{2}ight}~\citep{gu2021l2ight}. Note that existing methods do NOT support PINN training. We apply the same sparse-grid loss computation in all methods. 
We use the same number of ONN forward evaluations per step in different BP-free training methods for fair comparisons.
The first two subfigures in Fig. \ref{fig:phase domain} shows the relative $\ell_2$ error curves of different training protocols.
Real-size PINNs training are very high-dimensional optimization problems (18k MZIs for Black-Scholes, 280k MZIs for 20dim-HJB, and 73k MZIs for Burgers' and Darcy Flow) for ONN on-chip training.
\texttt{FLOPS} can only handle toy-size neural networks ($20\sim 30$ neurons per layer, $\sim 1k$ MZIs) and fail to converge well on real-size PINNs, thus is not capable of solving realistic PDEs due to the limited scalability. 
Subspace BP training method \texttt{L\textsuperscript{2}ight} enables on-chip FO training of ONN, however the trainable parameters are restricted to the diagonal matrix $\bm{\Sigma}(\bm{\Phi})$ while orthogonal matrices $\bm{U}(\bm{\Phi})$ and $\bm{V}(\bm{\Phi})$ are frozen at random initialization due to the intractable gradients. Such restricted learnable space hinders the degree of freedom for training PINNs from scratch. As a result, \texttt{L\textsuperscript{2}ight} only finds a roughly converged solution with a large relative $\ell_2$ error.
% It also comes with great hardware overhead, as 1) \texttt{L\textsuperscript{2}ight} reduces the trainable parameters, but does not reduce the photonics hardware, and 2) to implement the BP training, a reciprocal ONN hardware, extra memory to save all intermediate results, and extra controlling units for sparse sampling are needed. As a result, to integrate \texttt{L\textsuperscript{2}ight} on a single photonic chip is not practical.
Our tensor-compressed BP-free training achieves the lowest relative $\ell_2$ error after on-chip training. 
We also visualize the learned solution $\hat{u}$ to examine the quality (the last two subfigures in Fig. \ref{fig:phase domain}).
% Table~\ref{tab:photonics training} shows that our method requires much fewer MZIs.
The above results show that our method is the most scalable solution to enable real-size PINNs training, capable of solving realistic PDEs on photonic computing hardware. The on-chip phase-domain training results normally show some performance degradation compared with the numerical results of weight-domain training, due to the limited control resolution, device uncertainties, \textit{etc.}. More experiment results are available in Appendix \ref{appendix:Detailed Results of Phase-domain Training}.


\begin{table}[t]
  \centering
  \caption{Relative $\ell_2$ error in different photonic training methods. %We count the number of MZIs to implement the whole network, the number of trainable parameters, and the relative $\ell_2$ error of the learned solution.
  }
    % Table generated by Excel2LaTeX from sheet 'Tables'
  \resizebox{\linewidth}{!}{
    % % Table generated by Excel2LaTeX from sheet 'Tables'
    % \begin{tabular}{c|ccc|ccc}
    % \toprule
    % \toprule
    % \multicolumn{1}{r}{} &       & \multicolumn{1}{l}{Black Scholes} &       &       & \multicolumn{1}{l}{20dim-HJB} &  \\
    % \cmidrule{2-7}\multicolumn{1}{c}{} & \# MZIs & \# Trainable MZIs & rel. $\ell_2$ error & \# MZIs & \# Trainable MZIs & rel. $\ell_2$ error \\
    % \midrule
    % \texttt{FLOPS}~\cite{GuDAC} & 18,065 & 18,065 & 0.667 & 279,232 & 279,232 & 1.40E-02 \\
    % \texttt{L\textsuperscript{2}ight}~\cite{gu2021l2ight} & 18,065 & {2,561}   & 0.203 & 279,232 & {35,841} & 4.09E-03 \\
    % Ours  & \textbf{1,685} & 1,685 & \textbf{0.103} & \textbf{2,057} & 2,057 & \textbf{1.57E-03} \\
    % \bottomrule
    % \bottomrule
    % \end{tabular}%

    % Table generated by Excel2LaTeX from sheet 'ICML'
    \begin{tabular}{c|ccc}
    \toprule
    \toprule
    Problem & {\texttt{FLOPS}} & {\texttt{L\textsuperscript{2}ight}} & Ours \\
          & \cite{GuDAC} & \cite{gu2021l2ight} &  \\
    \midrule
    Black-Scholes & 6.67E-01 & 2.03E-01 & \textbf{1.03E-01} \\
    20-dim HJB & 1.40E-02 & 4.09E-03 & \textbf{1.57E-03} \\
    Burgers & 4.47E-01 & 5.69E-01 & \textbf{2.68E-01} \\
    Darcy Flow & 4.76E-01 & 1.54E-01 & \textbf{9.10E-02} \\
    \bottomrule
    \bottomrule
    \end{tabular}%
    
    
  } 
  \label{tab:photonics training}%
  \vspace{-5pt}
\end{table}%



% \begin{wraptable}{r}{0.5\textwidth}
\begin{table}[t]
\vspace{-10pt}
\centering
\caption{Implementation results a $128\times 128$ hidden layer in solving Black-Scholes equation. The latency means total on-chip training time. SM: space multiplexing. TM: time multiplexing.}
\resizebox{\linewidth}{!}{
    % Table generated by Excel2LaTeX from sheet 'ICML'
    \begin{tabular}{cccc}
    \toprule
    \toprule
          & \# of MZIs & Footprint ($mm^2$) & Training time (s) \\
    \midrule
    ONN-SM & 16384 & 3975.68 (infeasible) & 1.74 \\
    Ours (w/ TONN-SM) & \textbf{384} & \textbf{102.72} & \textbf{1.64} \\
    \midrule
    ONN-TM & 64    & 18.72 & 52.27 \\
    Ours (w/ TONN-TM) & 64    & 18.72 & \textbf{9.80} \\
    \bottomrule
    \bottomrule
    \end{tabular}%
    
}
\label{tab:hardware-cost}%
\vspace{-10pt}
\end{table}
% \end{wraptable}

{\bf System Performance.} Table \ref{tab:hardware-cost} compares the on-chip training system performance to implement a $128\times 128$ hidden layer for solving the Black-Scholes equation. The model size is much larger than existing photonic training accelerators that only support $20\sim 30$ neurons per layer \cite{bandyopadhyay2022single, pai2023experimentally}. We compare our accelerator design with the conventional ONN design in both space multiplexing and time multiplexing designs. 
It is not practical for a single photonic chip to integrate a $128\times 128$ matrix due to the huge device sizes and the insurmountable optical loss.
In comparison, our method reduces the number of MZIs by $42.7\times$, which is the key to enabling whole-model integration (TONN-SM) with a reasonable footprint. 
The simulation results show that our photonic accelerator achieve ultra-high-speed PINN training (1.64-second training time) to solve the Black-Scholes equation.
A detailed breakdown of system performance analysis is provided in Appendix \ref{appendix:system performance}. 

{\bf Remark. } The model size that a photonic AI accelerator can handle is much smaller than its electronic counterparts, due to the larger sizes of photonic devices. It is worthnoting that our {\it training} accelerator can handle much larger neural network models than the state-of-the-art photonic {\it inference} accelerator~\citep{Ramey2020}.

% \subsection{Ablation}

% In this section, we evaluate the influence of different design choices.

% {\bf Comparison of dimensionality/variance reduction approaches.} We compare our TT-compressed model with standard MLPs with dense fully connected (FC) layers and sparse-pruned (SP) MLPs in Table~\ref{tab:HJB dim reduction}. For a fair comparison, all networks have similar model parameters. TT-compressed and SP model can reach better results than FC models as TT and SP can preserve the wide hidden layer. The SP model shows superior efficiency in FO training and achieves similar result with the TT model in ZO training. Note that the SP model only reduces the trainable parameters but not the model parameters, so TT models can achieve better memory and computation saving.
    
        
% \section{Relevant Work}
% \paragraph{Optical Nerual Networks Training}
% Implementing on-chip BP training for ONN is challenging, and not practical or scalable due to the following constraints:
% 1) BP requires full observability of intermediate computing results, however to track and detect the whole optical field is not scalable or practical, 2) the derivatives of photonics devices' controlling units (\textit{e.g.,} phase shifters of MZIs, voltages of MRRs) are either inaccessible or computational prohibitive, 3) The unknown fabrication imperfections and environmental noises also make the photonics device a black box.
% Back-propagation-free (BP-free) training methods including stochastic zeroth-order (ZO) optimization~\cite{GuDAC, GuAAAI}, forward-forward method~\cite{hinton2022forward}, feedback-alignment method~\cite{filipovich2022silicon} are considered better alternatives since they only need to implement a forward (inference) computation hardware. However, prior methods are hard to scale to larger ONNs due to dimension-dependent convergence rates or unrealistic hardware complexity.

% \paragraph{Back-propagation-free training}
%     Due to the complexity of performing BP on edge devices, several BP-free training algorithms have been proposed. These methods have gained more attentions in recent years as BP is also considered “biologically implausible”. ZO optimization \cite{ZOSGD, duchi2015optimal, ZOSCD, chen2019zo,shamir2017optimal,balasubramanian2022zeroth,cai2021zeroth} plays an important role in tackling signal processing and machine learning problems, where actual gradient information is infeasible.
%     For specific use cases, please refer to the  survey by \citet{liu2020primer}.
%     But most ZO optimizations scale poorly when training real-size neural networks from scratch, due to their dimension-dependent gradient errors. Forward-forward algorithm was proposed for biologically plausible learning~\cite{hinton2022forward}. Other BP-free training frameworks include the forward gradient method~\cite{baydin2022gradients}, which updates the weights based on the directional gradient computed by forward-mode AD along a random direction. This method is further scaled up by leveraging activity perturbation and local loss for variance reduction\cite{ren2022scaling}. 
%     However, existing BP-free methods focus on training on HPC, lacking special concerns for the special challenges of on-device training. Furthermore, BP-free training of PINN remains an empty field. 
\vspace{-10pt}
\section{Conclusion}
\vspace{-5pt}
This paper has proposed a two-level BP-free training approach to train real-size physics-informed neural networks (PINNs) on optical computing hardware. 
Specifically, our method integrates a sparse-grid Stein derivative estimator to avoid BP in loss evaluation and a tensor-compressed ZO optimization to avoid BP in model parameter update. The tensor-compressed ZO optimization can simultaneously reduce the ZO gradient variance and model parameters, thus scaling up optical training to real-size PINNs with hundreds of neurons per layer. We have further designed the BP-free training on an integrated photonic platform. Our approach has successfully solved various PDE benchmarks with the smallest relative error compared with existing photonic on-chip training protocols. Future studies of variance reduction can help narrow the performance gap between ZO training and FO training. Our tensor-compressed BP-free training method is not restricted to PINNs. It can be easily extended to other applications on photonic and other types of edge platform where the hardware cost to implement BP is not feasible. We refer the reader to Appendix \ref{apdx:Broader Impacts} for a discussion and additional results on the broader impacts.


% In the unusual situation where you want a paper to appear in the
% references without citing it in the main text, use \nocite
% \newpage
\section*{Impact Statement}
This paper presents work whose goal is to advance the field of Machine Learning. There are many potential societal consequences of our work, none which we feel must be specifically highlighted here

%\Urlmuskip=0mu plus 1mu\relax
\bibliography{example_paper}
\bibliographystyle{icml2025}


%%%%%%%%%%%%%%%%%%%%%%%%%%%%%%%%%%%%%%%%%%%%%%%%%%%%%%%%%%%%%%%%%%%%%%%%%%%%%%%
%%%%%%%%%%%%%%%%%%%%%%%%%%%%%%%%%%%%%%%%%%%%%%%%%%%%%%%%%%%%%%%%%%%%%%%%%%%%%%%
% APPENDIX
%%%%%%%%%%%%%%%%%%%%%%%%%%%%%%%%%%%%%%%%%%%%%%%%%%%%%%%%%%%%%%%%%%%%%%%%%%%%%%%
%%%%%%%%%%%%%%%%%%%%%%%%%%%%%%%%%%%%%%%%%%%%%%%%%%%%%%%%%%%%%%%%%%%%%%%%%%%%%%%
\newpage
\appendix
\onecolumn
% \section{You \emph{can} have an appendix here.}

\section{ONN Basics}\label{appendix:ONN Basics}
\subsection{MZI-based ONN Architecture.}
We focus on the ONN~\citep{shen2017deep} architecture with singular value decomposition (SVD) to implement matrix-vector multiplication (MVM), i.e., $y=\bm{W}x=\bm{U\Sigma V^*}x$. The unitary matrices $\bm{U}$ and $\bm{V^*}$ are implemented by MZIs in Clements mesh~\citep{rectangleMZI}. 
The parametrization of $\bm{U}$ and $\bm{V^*}$ is given by $\bm{U}(\bm{\Phi}^U)=\bm{D}^U \prod_{i=k}^2 \prod_{j=1}^{i-1} \bm{R}_{i j}\left(\phi_{i j}^U\right), \bm{V}^*(\bm{\Phi}^V)=\bm{D}^V \prod_{i=k}^2 \prod_{j=1}^{i-1} \bm{R}_{i j}\left(\phi_{i j}^V\right)$, where $\bm{D}$ is a diagonal matrix, and each 2-dimensional rotator $\bm{R}_{ij}(\phi_{ij})$ can be implemented by a reconfigurable $2\times 2$ MZI containing one phase shifter ($\phi$) and two 50/50 splitters, which can produce interference of input light signals as follows:
\begin{equation}
\binom{y_1}{y_2}=\left(\begin{array}{cc}
\cos \phi & \sin \phi \\
-\sin \phi & \cos \phi
\end{array}\right)\binom{x_1}{x_2}
\end{equation}

The diagonal matrix $\boldsymbol{\Sigma}$ is implemented by on-chip attenuators, e.g., single-port MZIs, to perform signal scaling. The parameterization is given by $\boldsymbol{\Sigma}\left(\boldsymbol{\Phi}^S\right)=\max \left(\left|\boldsymbol{\Sigma}\right|\right) \operatorname{diag}\left(\cdots, \cos \phi_{i}^S, \cdots\right)$.
We denoted all programmable phases as $\bm{\Phi}$ and $\bm{W}$ is parameterized by $\bm{W}(\bm{\Phi})=\bm{U}(\bm{\Phi}^U) \bm{\Sigma}(\bm{\Phi}^S) \bm{V}^*(\bm{\Phi}^V)$. 

To implement a $N\times N$ matrix on ONN, $O(N^2)$ MZIs are required no matter the large ONN is implemented with a single large MZI mesh or multiple smaller MZI meshes. For a single MZI mesh implementation, the number of MZIs is $(\frac{N(N-1)}{2} + N + \frac{N(N-1)}{2})=N^2$. To implement with multiple smaller MZI meshes, saying implementing a $N\times N$ matrix by $\frac{N}{k}\times \frac{N}{k}$ blocks with size of with $k\times k$, the number of MZIs is $\frac{N}{k}\times \frac{N}{k}\times (\frac{k(k-1)}{2} + k + \frac{k(k-1)}{2})=N^2$.
Due to the high MZI cost and large MZI footprint, the space-multiplexing implementation of ONN is not realistic for large weight matrices. 

\subsection{Intractable Gradients of MZI Phases}
The analytical gradient w.r.t each MZI phases is given by:
\begin{equation}
\frac{\partial \mathcal{L}}{\partial \boldsymbol{R}_{i j}}=\left(\boldsymbol{D} \boldsymbol{R}_{n 1} \boldsymbol{R}_{n 2} \boldsymbol{R}_{n 3}\right)^T \nabla_y \mathcal{L} x^T\left(\cdots \boldsymbol{R}_{32} \boldsymbol{R}_{21} \boldsymbol{\Sigma} \boldsymbol{V}^*\right)^T
\end{equation}

\begin{equation}
\frac{\partial \mathcal{L}}{\partial \phi_{i j}}=\operatorname{Tr}\left(\left(\frac{\partial \mathcal{L}}{\partial \boldsymbol{R}_{i j}} \odot \frac{\partial \boldsymbol{R}_{i j}}{\partial \phi_{i j}}\right)\left(e_i+e_j\right)\left(e_i+e_j\right)^T\right)
\end{equation}

This analytical gradient is computationally-prohibitive, and requires detecting the whole optical field to read out all intermediate states $x$, which is not practical or scalable on integrated photonics chip.

\section{TONN Implementation Details}\label{appendix:TONN Implementation Details}
\subsection{ Additional details on TONN-SM Architecture}\label{apdx: Additional details on TONN-SM Architecture}

\begin{figure}[t]
\centering

\includegraphics[width=0.9\textwidth]{fig/TONN_SM.pdf}
\caption{\label{fig:apdx_TONN_1} (The same as Figure \ref{fig:TONN} (c)) TONN-SM architecture. PTC: photonic tensor core, DAC: digital-analog converter, ADC: analog-digital converter.}

\end{figure}
    
{
 In TONN-SM, the input data $\textbf{x}\in \mathbb{R}^{N}$, is folded to a d-way tensor $\ten{X} \in \mathbb{R}^{N_d\times \cdots \times N_1}$. The indices of the input tensor is then represented by $g$ wavelength division multiplexing (WDM) channels at $N/g$ inputs of the tensor cores, where $g = N_{d/2}\times \ldots \times N_1$. The light source is provided by a $g$-wavelength comb laser and power splitters. The splitted WDM light is modulated by $g$-wavelength optical modulator arrays, then multiplied by each of the photonic tensor core layers, and finally detected by $g$-wavelength WDM microring add-drop filter and detector arrays. The photonic tensor core layer $k$ ($k=d, \ldots, 1, k\neq d/2+1$) consists of $h_k$ number of $R_{k-1}M_k\times N_kR_k$ MZI meshes (tensor cores) and an optical passive cross-connect to switch indices of $M_k$ and $N_{k-1}$. Here, $h_k = M_d\ldots M_{k+1}N_{k-1}\ldots N_{d/2+1}$ for $d/2 \textless k \textless d$ or $M_{d/2}\ldots M_{k+1}N_{k-1}\ldots N_1$ for $k\leq d/2$. For TT-core $d/2+1$, the optical passive cross-connect is replaced by a passive wavelength-space cross-connect to switch the indices between the wavelength domain ($N_{d/2}, \ldots, N_1$) and the space domain ($M_d, \ldots, M_{d/2+1}$).
}

\subsection{Bit Accuracy}
In this paper, we assume a weight stationary (WS) scheme, where the weight matrices are programmed into the phase shifters in the MZI mesh, and the input vectors are encoded in the high-speed (10 GHz) optical signals. In each training iteration, the same weights (phases) are multiplied with batched (e.g., 1000) input data. As a result, the update rate of the phase shifters is 10 GHz/1000 = 10 MHz. In a system-level study of MZI-mesh-based photonic AI accelerators, a 12-bit DAC is enough to support the 8-bit accuracy of the weights \cite{demirkiran2023electro}. Considering that a 12-bit DAC with a 10 MHz sampling rate is very mature \cite{TIADC}, assuming 8-bit weights (phases) in our setting is reasonable.

The minimum optical SNR at the output of the MZI mesh is $SNR = 2^{b_{out}}$, where $b_{out}$ is the required bit accuracy of the output of the matrix multiplication. The optical SNR can be improved by increasing the input laser power, reducing the optical insertion loss, increasing the optical gain, and increasing the sensitivity of the photodiodes. For instance, the platform in \cite{liang2022energy} can provide lasers with high wall-plug efficiency, optical modulators, and MZIs with low insertion loss, on-chip optical gain, and quantum dot avalanche photodiodes with low sensitivity. Furthermore, the tensor decomposition in our work reduces the number of cascaded stages of MZIs, significantly reducing the insertion loss induced by cascaded MZIs.

\subsection{Interconnection between Digital Control System ad TONN}
The digital control system is implemented via electronic-photonic co-integration that contains an FPGA or ASIC for controlling and digital calculations required by BP-free training, digital electronic memory (e.g., DRAM) for weight and data storage and buffering, and ADC/DACs for converting the digital data to the tuning voltages of the modulators and phase shifters. As a result, no additional optical devices are required other than the TONN inference accelerator we introduced in the paper. The noise induced by the digital control system is decided by the bit accuracy of the ADCs and DACs. Regarding synchronization between the digital system and TONN. In the WS scheme, weight buffers are used, which means that the weights for the next set of matrix multiplication are loaded into the weight buffer, while the MZI mesh performs matrix multiplication with the current weight values. The latency is limited by the tuning mechanism of the phase shifters. In our case, the tuning mechanism is the III-V-on-silicon metal-oxide-semiconductor capacitor (MOSCAP) \citep{liang2022energy}, which has a modulation speed of tens of GHz.

\section{Experiment Settings}\label{appendix:Training Set-ups}
% This section provides more details of our experimental settings for both the Black Scholes equation and the 20-dim HJB PDE. 
\subsection{PDE details} \label{appendix:PDE details}
\paragraph{1-dim Black-Scholes Equation.} 
We examine the Black-Scholes equation for option price dynamics:
\begin{equation}
\begin{aligned}
&\partial_t u + \frac{1}{2}\sigma^2 x^2 \partial_{xx} u + rx\partial_x u - ru = 0, \quad x \in [0,200],~~t \in [0,T],\\
&u(x,T) = \max(x-K,0), \quad x \in [0,200],  \\
&u(0,t) = 0, \quad u(200,t) = 200 - Ke^{-r(T-t)}, \quad t \in [0,T],
\end{aligned}
\end{equation}
where $u(x,t)$ is the option price, $x$ is the stock price, $\sigma=0.2$ is volatility, $r=0.05$ is risk-free rate, $K=100$ is strike price, and $T=1$ is expiration time. The analytical solution is:
\begin{equation}
u(x,t) = xN(d_1) - Ke^{-r(T-t)}N(d_2),
\end{equation} with $d_1$ and $d_2$ defined as:
\begin{equation}
\begin{aligned}
&d_1 = \frac{\ln(x/K) + (r + \sigma^2/2)(T-t)}{\sigma\sqrt{T-t}}, \\
&d_2 = d_1 - \sigma\sqrt{T-t},
\end{aligned}
\end{equation}
where $N(\cdot)$ is the cumulative distribution function of the standard normal distribution. The base neural network is a 3-layer MLP with 128 neurons and \texttt{tanh} activation in each hidden layer. In tensor-train (TT) compressed training, the input layer ($2\times 128$) and the output layer ($128\times 1$) are left as-is, while we fold the hidden layer as size $4\times 4\times 8\times 8\times 4\times 4$. We preset the TT-ranks as [1,$r$,$r$,1], where $r$ controls the compression ratio.

\paragraph{20-dim HJB Equation.}
We consider the following 20-dim HJB PDE for high-dimensional optimal control:
\begin{equation}
\begin{aligned}
&\partial_t u(\bm{x}, t)+\Delta u(\bm{x}, t)-0.05 \left\|\nabla_{\bm{x}}u(\bm{x}, t)\right\|_{2}^{2}=-2, \\
&u(\bm{x}, 1)=\left\|\bm{x}\right\|_{1}, \quad \bm{x} \in [0,1]^{20}, ~~t \in[0, 1].
\end{aligned}
\end{equation}
Here $\left\|\cdot\right\|_{p}$ denotes an $\ell_p$ norm. The exact solution is $u(\bm{x},t)=\left\|\bm{x}\right\|_{1}+1-t$. The base network is a 3-layer MLP with 512 neurons and \texttt{sine} activation in each hidden layer. For TT compression, we fold the input layer and hidden layers as size $1\times 1\times 3\times 7\times 8\times 4\times 4\times 4$ and $4\times 4\times 4\times 8\times 8\times 4\times 4\times 4$, respectively,with TT-ranks [1,$r$,$r$,$r$,1]. The output layer ($512\times 1$) is left as-is.

{

\textbf{1-dim Burgers’ Equation \cite{hao2023pinnacle}:}
\begin{equation}
    \partial_t u + u \partial_x u = \nu  \partial_{xx} u, \quad (x, t) \in [-1, 1] \times [0, 1],
\end{equation}
where the viscosity $\nu = \frac{0.01}{\pi}$. The initial and boundary conditions are:
\begin{align}
    u(x, 0) &= -\sin(\pi x), \quad x \in [-1, 1], \\
    u(-1, t) &= u(1, t) = 0, \quad t \in [0, 1].
\end{align}

\textbf{2-dim Darcy Flow \cite{li2020fourier}:}
\begin{align}
\nabla \cdot(k(\mathbf{x}) \nabla u(\mathbf{x}))=f(\mathbf{x}), \quad \mathbf{x} \in \Omega,\\
u(\mathbf{x}) = 0, \quad \mathbf{x} \in \partial \Omega,
\end{align}
where $k(\mathbf{x})$ is the permeability field, $u(\mathbf{x})$ is the pressure, and $f(\mathbf{x})$ is the forcing function. We define $\Omega = [0, 1]^2$, set $f(\mathbf{x}) = 1$, and use a piecewise constant function for $k(\mathbf{x})$ as shown in Fig.~\ref{fig:darcy kx}.

\begin{figure}[H]
    \centering
    \includegraphics[width=0.5\linewidth]{fig/appendix/apdx_darcy_kx.pdf}
    \caption{ Permeability field in the Darcy flow problem.}
    \label{fig:darcy kx}
\end{figure}
}

For both 1-dim Burgers' equation and 2-dim Darcy Flow, our baseline model aligns with the state-of-the-art PINN benchmark from \cite{hao2023pinnacle}. It comprises a fully connected neural network with five hidden layers containing 100 neurons, totaling 30,701 trainable parameters. The dimension of our tensor-compressed training is reduced to 1,241 by folding the weight matrices in hidden layers as size $4\times 5\times 5\times 5\times 5\times 4$ and decomposing it with a TT-rank $(1, 2, 2, 1)$. We trained the models for 40,000 iterations on the 1-dim Burgers' equation and for 20,000 iterations on the 2-dim Darcy flow. All other training configurations were kept consistent with our main experimental setups.

\subsection{Loss Evaluation Set-ups.}\label{appendix:Loss Evaluation Set-ups} We compare three methods for computing derivatives in the loss function \eqref{PINNs loss}: 1) automatic differentiation (\textbf{AD}) as a golden reference, 2) Monte Carlo-based Stein Estimator (\textbf{SE})~\cite{he2023learning}, and 3) our sparse-grid (\textbf{SG}) method. 
For Black-Scholes, we approximate the solution $u_{\bm{\theta}}$ using a neural network $f_{\bm{\theta}}(\bm{x},t)$, which can be either the base network or its TT-compressed version. In the \textbf{AD} approach, $u_{\bm{\theta}}(\bm{x},t)=f_{\bm{\theta}}(\bm{x},t)$, while for \textbf{SE} and \textbf{SG}, $u_{\bm{\theta}}(\bm{x},t)=\mathbb{E}_{(\bm{\delta_{\bm{x}}},\delta_{t}) \sim \mathcal{N}\left(\bm{0}, \sigma^{2} \bm{I}\right)}f_{\bm{\theta}}(\bm{x}+\bm{\delta_{\bm{x}}},t+\delta_{t})$. We set the noise level $\sigma$ to 1e-3 in \textbf{SE} and \textbf{SG}, using 2048 samples in \textbf{SE} and 13 samples in \textbf{SG} with a level-3 sparse Gaussian quadrature rule to approximate the expectations \eqref{gaussian smoothed model} and \eqref{stein derivative estimator}. For HJB, we employ a transformed neural network $f_{\bm{\theta}}^{'}(\bm{x},t) = (1-t)f_{\bm{\theta}}(\bm{x},t) + \left\|\bm{x}\right\|_{1}$, where $f_{\bm{\theta}}(\bm{x},t)$ is the base or TT-compressed network. The solution approximation follows the same pattern as in the Black-Scholes case. Here the transformed network is designed to ensure that our approximated solution either exactly satisfies (\textbf{AD}) or closely adheres to the terminal condition (\textbf{SE}, \textbf{SG}), allowing us to focus solely on minimizing the HJB residual during training. We set the noise level $\sigma$ to 0.1 in \textbf{SE} and \textbf{SG}, using 1024 samples in \textbf{SE} and 925 samples in \textbf{SG} with a level-3 sparse Gaussian quadrature rule.

\subsection{Training Set-ups.} We implemented all methods in PyTorch, utilizing an NVIDIA GTX 2080Ti GPU and an Intel(R) Xeon(R) Gold 5218 CPU @ 2.30GHz.

\subsection{Data Sampling.} For Black-Scholes, we uniformly sample 100 random residual points, 10 initial points, and 10 boundary points on each boundary per epoch to evaluate the PDE loss \eqref{PINNs loss}. For HJB, we select 100 random residual points per epoch. The model architecture for the HJB equation incorporates the terminal condition, eliminating the need for an additional terminal loss term. For Burgers', we uniformly sample 1200 random residual points, 100 initial points, and 100 boundary points on each boundary per epoch to evaluate the PDE loss \eqref{PINNs loss}. For Darcy flow, we sample the residual points on a fixed 241 x 241 uniform grid and encode hard boundary constraints in the model architecture.

\section{Weight-domain Training}\label{appendix:Detailed Results of Weight-domain Training}

{
 In this section, we provide the training curves of weight-domain training in Fig. \ref{fig:apdx_weight}. The curves denote averaged relative $\ell_2$ error over three independent experiments and shades denote the corresponding standard deviations. 
 We also provide the extended results of Table \ref{tab:loss computation} and \ref{tab:weight training}. Each relative $\ell_2$ error takes the form $\text{mean} \pm \text{std}$, where $\text{mean}$ denotes the averaged result over three independent experiments, and $\text{std}$ denotes the corresponding standard deviation. 
}

\begin{figure}[H]
    \vspace{-10pt}
    \centering
    \includegraphics[width=0.8\linewidth]{fig/appendix/apdx_weight.pdf}
        \caption{Relative $\ell_2$ error curves of weight domain training for Black-Scholes equation (left) and 20-dim HJB equation (right), respectively. The value at each step is averaged across three runs, and the shade indicates the standard deviation. 
    }
    \vspace{-20pt}
\label{fig:apdx_weight}
\end{figure}

% Table generated by Excel2LaTeX from sheet 'ICLR'
\begin{table}[H]
  \centering
  \caption{Relative $\ell_2$ error of FO training using
different loss computation methods. We report the averaged results and standard deviations across three runs.}
    % Table generated by Excel2LaTeX from sheet 'ICML'
    \begin{tabular}{c|ccc}
    \toprule
    \toprule
    Problem & AD    & SE    & SG (ours) \\
    \midrule
    Black-Scholes & (5.35$\pm$0.13)E-02 & (5.41$\pm$0.09)E-02 & \boldmath{}\textbf{(5.28$\pm$0.05)E-02}\unboldmath{} \\
    20-dim HJB & (1.99$\pm$0.15)E-03 & (1.52$\pm$0.14)E-03 & \boldmath{}\textbf{(8.16$\pm$1.24)E-04}\unboldmath{} \\
    Burgers & (1.37$\pm$0.04)E-02 & (1.98$\pm$0.15)E-02 & \boldmath{}\textbf{(1.31$\pm$0.05)E-02}\unboldmath{} \\
    Darcy Flow & (7.57$\pm$0.28)E-02 & (7.85$\pm$0.40)E-01 & \boldmath{}\textbf{(7.47$\pm$0.41)E-02}\unboldmath{} \\
    \bottomrule
    \bottomrule
    \end{tabular}%
    
    
    
  \label{tab:addlabel}%
\end{table}%

% Table generated by Excel2LaTeX from sheet 'ICLR'
\begin{table}[H]
  \centering
  \caption{Relative $\ell_2$ error achieved using different training methods. We report the averaged results and standard deviations across three runs.}
    % Table generated by Excel2LaTeX from sheet 'ICML'
    \begin{tabular}{c|cc|cc}
    \toprule
    \toprule
    Problem & \multicolumn{2}{c|}{FO Training} & \multicolumn{2}{c}{ZO Training} \\
    \cmidrule{2-5}      & Standard & TT    & Standard & TT \\
    \midrule
    Black-Scholes & (5.28$\pm$0.05)E-02 & \boldmath{}\textbf{(5.97$\pm$0.01)E-02}\unboldmath{} & (3.91$\pm$0.05)E-01 & \boldmath{}\textbf{(8.30$\pm$0.08)E-02}\unboldmath{} \\
    20-dim HJB & (8.16$\pm$1.24)E-04 & \boldmath{}\textbf{(2.05$\pm$0.39)E-04}\unboldmath{} & (6.86$\pm$0.27)E-03 & \boldmath{}\textbf{(1.54$\pm$0.35)E-03}\unboldmath{} \\
    Burgers & (1.31$\pm$0.05)E-02 & \boldmath{}\textbf{(4.49$\pm$0.58)E-02}\unboldmath{} & (4.41$\pm$0.09)E-01 & \boldmath{}\textbf{(1.63$\pm$0.25)E-02}\unboldmath{} \\
    Darcy Flow & \boldmath{}\textbf{(7.47$\pm$0.41)E-02}\unboldmath{} & \boldmath{}\textbf{(8.77$\pm$0.11)E-02}\unboldmath{} & (1.34$\pm$0.18)E-01 & \boldmath{}\textbf{(9.05$\pm$0.29)E-02}\unboldmath{} \\
    \bottomrule
    \bottomrule
    \end{tabular}%
    
  \label{tab:addlabel}%
\end{table}%

\section{Ablation Studies}\label{appendix:Ablation Studies}
\subsection{Tensor-train (TT) Ranks}
{
TT-rank determination is a trade-off between model compression ratio and model expressivity. The TT-ranks can be empirically determined, or adaptively determined by automatic rank determination algorithms \cite{hawkins2021bayesian, yang2024comera}. To validate our tensor-train (TT) rank choice, we add an ablation study on different TT ranks. The results are provided in Table \ref{tab:abla_ttrank} below. We tested tensor-train compressed training with different TT-ranks on solving 20-dim HJB equations. The model setups are the same as illustrated in Appendix A.2. We fold the input layer and hidden layers as size $1\times 1\times 3\times 7\times 8\times 4\times 4\times 4$ and $4\times 4\times 4\times 8\times 8\times 4\times 4\times 4$, respectively, with TT-ranks [1,$r$,$r$,$r$,1]. We use automatic differentiation for loss evaluation and first-order (FO) gradient descent to update model parameters. Other training setups are the same as illustrated in Appendix A.3.  The results reveal that models with larger TT-ranks have better model expressivity and achieve smaller relative $\ell_2$ error. However, increasing TT-ranks increases the hardware complexity (e.g., number of MZIs) of photonics implementation as it increases the number of parameters. Therefore, we chose a small TT-rank as 2, which provides enough expressivity to solve the PDE equations, while maintaining a small model size.
}

% Table generated by Excel2LaTeX from sheet 'MZI TTM HJB'
\begin{table}[H]
  \centering
  \caption{Ablation study on tensor-train (TT) ranks when training the TT compressed model on solving 20-dim HJB equations. We report the average error and the standard deviation across three runs.}
    \begin{tabular}{ccccc}
    \toprule
    \toprule
    TT-rank & 2     & 4     & 6     & 8 \\
    \midrule
    Params & 1,929 & 2,705 & 3,865 & 5,409 \\
    rel. $\ell_2$ error & (3.17$\pm$1.16)E-04 & (2.45$\pm$0.82)E-04 & (4.00$\pm$3.69)E-05 & (3.02$\pm$3.16)E-05 \\
    \bottomrule
    \bottomrule
    \end{tabular}%
  \label{tab:abla_ttrank}%
\end{table}%



\subsection{Hidden layer width of baseline MLP model}
{
We also performed an ablation study on the hidden layer width of the baseline MLP model. We trained 3-layer MLPs with different hidden layer widths to solve the 20-dim HJB equation. We use automatic differentiation for loss evaluation and first-order (FO) gradient descent to update model parameters. Other training setups are the same as illustrated in Appendix A.3. The results are shown in Table \ref{tab:abla_hidden}. The MLP model with a smaller hidden layer width leads to larger testing errors. This indicates that a large hidden layer is favored to ensure enough model expressivity. The MLP model used in our submission does not have an overfitting problem. 
}

% Table generated by Excel2LaTeX from sheet 'MZI TTM HJB'
\begin{table}[H]
  \centering
  \caption{Ablation study on hidden layer size of baseline 3-layer MLP model when learning 20-dim HJB equation. We report the average error and the standard deviation across three runs.}

    \resizebox{\linewidth}{!}{
    \begin{tabular}{cccccc}
    \toprule
    \toprule
    Hidden layer size & 512   & 256   & 128   & 64    & 32 \\
    \midrule
    Params & 274,433 & 71,681 & 19,457 & 5,633 & 1,793 \\
    rel. $\ell_2$ error & (2.72$\pm$0.23)E-03 & (4.31$\pm$0.19)E-03 & (7.51$\pm$0.36)E-03 & (8.15$\pm$0.67)E-03 & (9.25$\pm$0.27)E-03 \\
    \bottomrule
    \bottomrule
    \end{tabular}%
    }
  \label{tab:abla_hidden}%
\end{table}%

\section{Phase-domain Training}

\subsection{ONN Simulation Settings} \label{appendix:ONN Simulation Settings}
{
We apply the same setups as that in \texttt{L\textsuperscript{2}ight} \cite{gu2021l2ight} to implement uncompressed ONNs in baseline methods \texttt{FLOPS} \cite{GuDAC} and \texttt{L\textsuperscript{2}ight} \cite{gu2021l2ight}. The linear projection in an ONN adopts blocking matrix multiplication, where the $M\times N$ weight matrix is partitioned into $P\times P$ blocks of size $k\times k$. Here $P=\lceil M/k \rceil, Q=\lceil N/k \rceil$. Implementing ONNs with smaller MZI blocks is more practical and robust, and provides enough trainable parameters ($N^2/k$ singular values) for first-order based method \texttt{L\textsuperscript{2}ight}. Following the analysis provided in \cite{gu2021l2ight}, we select $k=8$ for practical consideration. 

The weight matrix $\boldsymbol{W}$ is parameterized by MZI phases $\boldsymbol{\Phi}$ as $\boldsymbol{W}(\boldsymbol{\Phi})=\left\{\boldsymbol{W}_{p q}\left(\boldsymbol{\Phi}_{p q}\right)\right\}_{p=0, q=0}^{p=P-1, q=Q-1}$. Each block $\boldsymbol{W}_{pq}$ is parameterized as $\boldsymbol{W}_{p q}\left(\boldsymbol{\Phi}_{p q}\right)=\boldsymbol{U}_{p q}\left(\boldsymbol{\Phi}_{p q}^U\right) \boldsymbol{\Sigma}_{p q}\left(\boldsymbol{\Phi}_{p q}^S\right) \boldsymbol{V}_{p q}^*\left(\boldsymbol{\Phi}_{p q}^V\right)$. 

\texttt{FLOPs} \cite{GuDAC} is a ZO based method. We use zeroth-order gradient estimation to estimate the gradients of all MZI phases (i.e., $\boldsymbol{\Phi}_{pq}^U, \boldsymbol{\Phi}_{pq}^S, \boldsymbol{\Phi}_{pq}^V)$

\texttt{L\textsuperscript{2}ight} \cite{gu2021l2ight} is a subspace FO based method. Due to the intractable gradients for $\boldsymbol{\Phi}_{pq}^U$ and $\boldsymbol{\Phi}_{pq}^V$, only the MZI phase shifters in the diagonal matrix $\boldsymbol{\Phi}_{pq}^S$ are trainable. This restricts the training space (i.e., subspace training).
}

\subsection{ONN Non-ideality}\label{appendix:ONN Non-ideality}
We follow~\cite{gu2021l2ight} to consider the following hardware-restricted objective $\bm{\Phi}^* = {\arg \min}_{\bm{\Phi}} \mathcal{L}(\bm{W}(\bm{\Omega} \bm{\Gamma} \mathcal{Q}(\bm{\Phi})+\bm{\Phi}_b))$, which jointly considers various ONN non-ideality including control resolution limit $\mathcal{Q}(\cdot)$, phase-shifter $\gamma$ coefficient drift $\bm{\Gamma}\sim \mathcal{N}(\gamma, \sigma^2_\gamma)$ caused by fabrication variations, thermal cross-talk between adjacent devices $\bm{\Omega}$, and phase bias due to manufacturing error $\bm{\Phi}_b \sim \mathcal{U}(0,2\pi)$.

\paragraph{Limited Phase-tuning Control Resolution.} Given the control resolution limits, we can only achieve discretized MZI phase tuning. We assume the phases $\phi$ is uniformly quantized into 8-bit within $[0, 2\pi]$ for phases in $\bm{U}(\bm{\Phi}^U)$, $\bm{\Sigma}(\bm{\Phi}^S)$, $\bm{V}^*(\bm{\Phi}^V)$. 

\paragraph{Phase-shifter Variation.}
We assume the real phase shift $\tilde{\phi}=\frac{\gamma + \Delta \gamma}{\gamma}\phi$, which is proportional to the device-related parameter. We assume $\Delta \gamma \sim \mathcal{N}(0,0.002^2)$. We formulate this error as a diagonal matrix $\boldsymbol{\Gamma}$ multiplied on the phase shift $\boldsymbol{\Phi}'=\boldsymbol{\Gamma} \boldsymbol{\Phi}$.

\paragraph{MZI Crosstalk.}
The crosstalk effect can be modeled as coupling matrix $\boldsymbol{\Omega}$,

\begin{equation}
\begin{aligned}
&\left(\begin{array}{c}
\phi_0^c \\
\phi_1^c \\
\vdots \\
\phi_{N-1}^c
\end{array}\right)=\left(\begin{array}{cccc}
\omega_{0,0} & \omega_{0,1} & \cdots & \omega_{0, N-1} \\
\omega_{1,0} & \omega_{1,1} & \cdots & \omega_{1, N-1} \\
\vdots & \vdots & \ddots & \vdots \\
\omega_{N-1,0} & \omega_{N-1,1} & \cdots & \omega_{N-1, N-1}
\end{array}\right)\left(\begin{array}{c}
\phi_0^v \\
\phi_1^v \\
\vdots \\
\phi_{N-1}^v
\end{array}\right) \\
& \text { s.t. } \omega_{i, j}=1, \quad \forall i=j \\
& \omega_{i, j}=0, \quad \forall i \neq j \text { and } \phi_j \in \mathcal{P} \\
& 0 \leq \omega_{i, j}<1, \quad \forall i \neq j \text { and } \phi_j \in \mathcal{A} .
\end{aligned}
\end{equation}

The diagonal factor $\omega_{i,j}$, $i=j$ is the self-coupling coefficient, $\omega_{i,j}$, $i\neq j$ is the mutual coupling coefficient. We follow~\cite{gu2021l2ight} to assume the self-coupling coefficient to be 1, and the mutual coupling coefficient is 0.005 for adjacent MZIs.

\subsection{Extended Experiment Results} \label{appendix:Detailed Results of Phase-domain Training}
{
 In this section, we provide the extended results of Table \ref{tab:photonics training} and Figure \ref{fig:phase domain}. Each relative $\ell_2$ error takes the form $\text{mean} \pm \text{std}$, where $\text{mean}$ denotes the averaged result over three independent experiments, and $\text{std}$ denotes the corresponding standard deviation. The curves denote averaged relative $\ell_2$ error over three independent experiments and shades denote the corresponding standard deviations.
}

% Table generated by Excel2LaTeX from sheet 'ICLR'
\begin{table}[H]
  \centering
  \caption{Comparison between different photonic training methods. We report the averaged relative $\ell_2$ error and standard deviations across three runs.}
  \resizebox{\linewidth}{!}{
    % Table generated by Excel2LaTeX from sheet 'ICML'
    \begin{tabular}{c|ccc|ccc}
    \toprule
    \toprule
    \multicolumn{1}{r}{} &       & \multicolumn{1}{l}{Black Scholes} &       &       & \multicolumn{1}{l}{20-dim HJB} &  \\
    \cmidrule{2-7}\multicolumn{1}{c}{} & \# MZIs & \# Trainable MZIs & rel. $\ell_2$ error & \# MZIs & \# Trainable MZIs & rel. $\ell_2$ error \\
    \midrule
    \texttt{FLOPS}~\cite{GuDAC} & 18,065 & 18,065 & 0.663$\pm$0.045 & 279,232 & 279,232 & (1.38$\pm$0.07)E-02 \\
    \texttt{L\textsuperscript{2}ight}~\cite{gu2021l2ight} & 18,065 & 2,561 & 0.192$\pm$0.381 & 279,232 & 35,841 & (2.95$\pm$0.99)E-03 \\
    Ours  & \textbf{1,685} & 1,685 & \boldmath{}\textbf{0.114$\pm$0.095}\unboldmath{} & \textbf{2,057} & 2,057 & \boldmath{}\textbf{(2.10$\pm$0.55)E-03}\unboldmath{} \\
    \bottomrule
    \bottomrule
    \end{tabular}%
    
  \label{tab:addlabel}%
  }
\end{table}%

% Table generated by Excel2LaTeX from sheet 'ICML'
\begin{table}[htbp]
  \centering
  \caption{Comparison between different photonic training methods. We report the averaged relative $\ell_2$ error and standard deviations across three runs.}
    \begin{tabular}{c|ccc|ccc}
    \toprule
    \toprule
    \multicolumn{1}{r}{} &       & \multicolumn{1}{l}{Burgers} &       &       & \multicolumn{1}{l}{Darcy Flow} &  \\
\cmidrule{2-7}    \multicolumn{1}{c}{} & \# MZIs & \# Trainable MZIs & rel. $\ell_2$ error & \# MZIs & \# Trainable MZIs & rel. $\ell_2$ error \\
    \midrule
    \texttt{FLOPS}~\cite{GuDAC} & 72,889 & 72,889 & 0.447$\pm$0.003 & 72,889 & 72,889 & 0.476$\pm$0.002 \\
    \texttt{L\textsuperscript{2}ight}~\cite{gu2021l2ight} & 4,665 & 4,665 & 0.569$\pm$0.003 & 4,665 & 4,665 & 0.154$\pm$0.017 \\
    Ours  & \textbf{2,516} & 2,516 & \boldmath{}\textbf{0.268$\pm$0.010}\unboldmath{} & \textbf{2,516} & 2,516 & \boldmath{}\textbf{0.091$\pm$0.005}\unboldmath{} \\
    \bottomrule
    \bottomrule
    \end{tabular}%
  \label{tab:addlabel}%
\end{table}%


\begin{figure}[H]
\centering
\begin{minipage}[t]{0.45\textwidth}
    \centering
    \subfigure{
        \includegraphics[width=\textwidth]{fig/appendix/apdx_phase_BS.pdf}
        \label{fig:apdx_phase_BS}
    }
\end{minipage}
\begin{minipage}[t]{0.45\textwidth}
    \centering
    \subfigure{
        \includegraphics[width=\textwidth]{fig/appendix/apdx_phase_HJB.pdf}
        \label{fig:apdx_phase_HJB}
    }
\end{minipage}
\quad
\caption{Relative $\ell_2$ error curves of phase domain training for Black-Scholes equation (left) and 20-dim HJB equation (right), respectively. The value at each step is averaged across three runs, and the shade indicates the standard deviation. 
}
\label{fig:abla_loss}
\end{figure}

\begin{figure}[H]
    \centering
    \includegraphics[width=0.8\linewidth]{fig/appendix/apdx_BS_Solution.pdf}
    \caption{Visualization of Black-Scholes equation in photonic on-chip learning simulation. The left subfigure shows the ground truth $u(x)$, and the right subfigure shows the learned solution $\hat{u}(x)$ using our proposed BP-free PINNs training method.}
    \label{fig:enter-label}
\end{figure}

\section{System Performance Evaluation}\label{appendix:system performance}
We evaluate the system performance of learning the Black-Scholes equation.
The system performance for the accelerators based on ONNs and TONNs are evaluated and compared assuming the III-V-on-Si device platform~\cite{liang2022energy}. 
The total number of wavelengths used is 8~\cite{xiao2021large}. The SVD implementation of the arbitrary matrices is considered in the calculation.

\subsection{Footprint:} Only the footprint of the photonic devices, which occupy the major area of the accelerator, is used for comparison. The photonic footprint includes the areas of hybrid silicon comb laser, microring resonator (MRR) modulator arrays, photonic tensor cores, MRR add-drop filters, photodiodes, and electrical cross-connects. 

% Table generated by Excel2LaTeX from sheet 'Tables'
\begin{table}[htbp]
  \centering
  \caption{Footprint breakdown. All units are $mm^2$.}
  % \resizebox{\linewidth}{!}{
    % Table generated by Excel2LaTeX from sheet 'ICML'
    \begin{tabular}{ccccccc}
    \toprule
    \toprule
          & Laser & MRR Mod. & Tensor core & Photodetector & Cross-connect & Total \\
    \midrule
    ONN-SM & 25.6  & 1.28  & 3947.52 & 1.28  & /     & 3975.68 \\
    TONN-SM & 1.6   & 0.8   & 97.92 & 0.8   & 1.6   & 102.72 \\
    \midrule
    ONN-TM & 1.6   & 0.4   & 16.32 & 0.4   & /     & 18.72 \\
    TONN-TM & 1.6   & 0.4   & 16.32 & 0.4   & /     & 18.72 \\
    \bottomrule
    \bottomrule
    \end{tabular}%
    
% }
  \label{tab:footprint breakdown}%
\end{table}%


% \textbf{Energy Consumption:} The total energy of the accelerators is mainly consumed in the ADC, DAC, and digital control systems. Here, we focus on the photonic energy consumption per forward, which consists of five parts: laser wall-plug power, microring modulator power, MZI mesh power, microring add-drop filter power, and PD receiver power. The conventional ONN has insurmountable optical loss due to the square scaling rule, so the energy cannot be calculated.
        
% \textbf{Latency:} The latency per inference in on-chip training is calculated by: $t_{\rm{inference}}=n_{\rm{cycle}}*(t_{\rm{DAC}}+t_{\rm{tuning}}+t_{\rm{opt}}+t_{\rm{ADC}})+t_{\rm{DIG}}$, where $t_{\rm{DAC}}$ is the DAC conversion delay ($\sim$24 ns), $t_{\rm{tuning}}$ is the metal-oxide-semiconductor capacitor (MOSCAP) phase shifter tuning delay ($\sim$0.1 ns), $t_{\rm{opt}}$ is the propagation latency of optical signal ($\sim$51.2 ns for ONN, $\sim$1.6 ns for TONN-SM), $t_{\rm{ADC}}$ is the ADC delay($\sim$24 ns), and $t_{\rm{DIG}}$ is the digital computation overhead ($\sim$500 ns) for gradient calculation and phase updates. 
        
% \textbf{Training Efficiency:} In our 20D-HJB example, we need 42 inferences for each loss evaluation and 10 loss evaluations for gradient estimation. Suppose a mini-batch size of 100, 4.20E4 inferences are required for one epoch. The energy consumption per epoch is estimated as 2.71E-04 J and the latency per epoch is estimated as 0.23 ms for TONN-SM. On average training reaches a good solution after 5000 epochs, which corresponds to 1.36 J and 1.15 s for solving a 20D-HJB equation.

\subsection{Latency:} 
% Since our BP-free training only repeats multiple inferences (forward passes) to train the model, we first calculate the on-chip latency per forward pass, and multiply it by the number of forwards to convergence as the total training latency. 

\paragraph{Latency per Inference.} 
The latency per inference is calculated by: 
\begin{equation}
    t_{\rm{inference}}=n_{\rm{cycle}}*(t_{\rm{DAC}}+t_{tuning}+t_{\rm{opt}}+t_{\rm{ADC}})
\end{equation}
where $t_{\rm{DAC}}$ is the DAC conversion delay ($\sim$24 ns), $t_{\rm{tuning}}$ is the metal-oxide-semiconductor capacitor (MOSCAP) phase shifter tuning delay ($\sim$0.1 ns), $t_{\rm{opt}}$ is the propagation latency of optical signal ($\sim$3.20 ns for ONN, $\sim$0.64 ns for TONN-SM, and $\sim$0.21 ns for TONN-TM), $t_{\rm{ADC}}$ is the ADC delay($\sim$24 ns). The TONN-TM uses 6 cycles for one inference, while ONN and TONN-SM only needs 1 cycle.
The latency per inference is estimated at 51.30 ns for ONN, 48.74 ns for TONN-SM, and 289.86 ns for TONN-TM.

\paragraph{Latency per Epoch.}
The latency per epoch is calculated by: 
\begin{equation}
    t_{epoch} = (t_{inference} \times N_{point} \times N_{loss} + t_{tuning}) \times N_{grads} +  t_{DIG}
\end{equation}
$t_{\rm{DIG}}$ is the digital computation overhead ($\sim$500 ns) for gradient accumulation and phase updates at the end of each epoch. 
New random perturbation samples could be sampled from environment in parallel with optical inference, so we didn't include this overhead.
We use $N_{point}=130, N_{loss}=13, N_{grads}=2$.
The latency per epoch is estimated at 0.174 ms for ONN, 0.164 ms for TONN-SM, and 0.980 ms for TONN-TM.

\paragraph{Total Training Latency.} 
On average our BP-free training finds a good solution after 10000 epochs of update. The total training latency is estimated as 1.74 s for ONN, 1.64 s for TONN-SM, and 9.80 s for TONN-TM. Table \ref{tab:latency} summarizes the breakdown of training latency.
% Table generated by Excel2LaTeX from sheet 'ICLR'
\begin{table}[htbp]
  \centering
  \caption{Latency breakdown. The results are based on simulation. ONN-1 and TONN-SM denote space-multiplexing implementation. ONN-2 and TONN-TM denote time-multiplexing implementation.}
  \resizebox{\linewidth}{!}{
    % Table generated by Excel2LaTeX from sheet 'ICML'
        \begin{tabular}{cccccc}
        \toprule
        \toprule
              & Cycles & Latency per Inference (ns) & Time per epoch (ms) & Time to converge (s) & rel. $\ell_2$ error \\
        \midrule
        ONN-SM & 1     & 51.30 & 0.17  & 1.74  & 0.667 \\
        TONN-SM & 1     & 48.74 & 0.16  & 1.64  & \textbf{0.103} \\
        \midrule
        ONN-TM & 32    & 1545.92 & 5.23  & 52.27 & 0.667 \\
        TONN-TM & 6     & 289.86 & 0.98  & 9.80  & \textbf{0.103} \\
        \bottomrule
        \bottomrule
        \end{tabular}%
        
    }
  \label{tab:latency}%
\end{table}%

\section{Broader Impacts} \label{apdx:Broader Impacts}

Real-time PDE solvers on edge devices are desired by many civil and defense applications. However, electronic computing devices fail to meet the requirements.
Our main motivation is to propose a completely back-propagation-free training for PINNs and realize real-time PINNs training (i.e., real-time PDE solver) on photonics computing chips. However, our tensor-train compressed zeroth-order training method can be generally applied to other applications, and our BP-free training framework can be applied to other resource-constrained edge platforms.

\subsection{Extension to Image Classification}
{
 Our tensor-compressed zeroth-order training is a general back-propagation-free training method that applies to lightweight neural networks other than PINNs. 
In this section, we extended our tensor-compressed zeroth-order training to the image classification task on the MNIST dataset. Note that our proposed sparse-grid loss evaluation is designed for PINN training only, so sparse-grid is not used here.

Our baseline model is a two-layer MLP (784$\times$1024, 1024$\times$10) with 814,090 parameters. The dimension of our tensor-compressed training is reduced to 3,962 by folding the input and output layer as size $7\times 4\times 4\times 7\times 8\times 4\times 4\times 8$ and $8\times 4\times 4\times 8\times 1\times 5\times 2\times 1$, respectively. Both the input layer and the output layer are decomposed with a TT-rank $(1, 6, 6, 6, 1)$. Models are trained for 15,000 iterations with a batch size 2,000, using Adam optimizer with an initial learning rate 1e-3 and decayed by 0.8 every 3,000 iterations. In ZO training, we set query number $N=10$ and smoothing factor $\mu=0.01$. 


Table \ref{tab:mnist weight} compares results of weight domain training.
\begin{itemize}
    \item Our tensor-train (TT) compressed training does not harm the model expressivity, as TT training achieved a similar test accuracy as standard training in first-order (FO) training.
    \item Our TT compressed training greatly improves the convergence of ZO training and reduces the performance gap between ZO and FO.
\end{itemize}

Table \ref{tab:mnist phase} compares results of phase domain training. Our method outperforms the baseline ZO training method \texttt{FLOPS} \cite{GuDAC}. This is attributed to the tensor-train (TT) dimension reduction that reduced gradient variance. Note that the performance gap between phase domain training and weight domain training could be attributed to the low-precision quantization, hardware imperfections, etc., as illustrated in Section 5.2. 
Our ZO training method did not surpass the FO subspace training method \texttt{L\textsuperscript{2}ight} \cite{gu2021l2ight}.
The performance of \texttt{L\textsuperscript{2}ight} versus our method should be considered case by case. \texttt{L\textsuperscript{2}ight} does not have additional gradient errors due to its FO optimization. Meanwhile, its sub-space training can prevent the solver from achieving a good optimal solution. The real performance depends on the trade-off of these two facts. In our PINN experiments, \texttt{L\textsuperscript{2}ight} underperforms our method because the limitation of its sub-space training plays a dominant role. \texttt{L\textsuperscript{2}ight} performs better on the MNIST dataset, probably because the model is more over-parameterized that even subspace training can achieve a good optimal solution.

The results on the MNIST dataset are consistent with our claims in the submission and support our claim that our method can be extended to image problems with higher dimensions.
}

% Table generated by Excel2LaTeX from sheet 'MNIST'
\begin{table}[H]
  \centering
  \caption{Validation accuracy of weight domain training on MNIST dataset. We report the averaged accuracy and the standard deviation across three runs.}
    \begin{tabular}{ccccc}
    \toprule
    \toprule
    Method & Standard, FO & TT, FO & Standard, ZO & TT, ZO (ours) \\
    \midrule
    Val. Accuracy (\%) & 97.83$\pm$1.02 & 97.26$\pm$0.15 & 83.83$\pm$0.44 & 93.21$\pm$0.46 \\
    \bottomrule
    \bottomrule
    \end{tabular}%
  \label{tab:mnist weight}%
\end{table}%

% Table generated by Excel2LaTeX from sheet 'MNIST'
\begin{table}[H]
  \centering
  \caption{Validation accuracy of phase domain training on MNIST dataset. We report the averaged accuracy and the standard deviation across three runs.}
    % Table generated by Excel2LaTeX from sheet 'MNIST'
    \begin{tabular}{cccc}
    \toprule
    \toprule
    \multicolumn{1}{l}{Method} & \texttt{FLOPS} \cite{GuDAC} & \texttt{L\textsuperscript{2}ight} \cite{gu2021l2ight} & ours \\
    \midrule
    Val. Accuracy (\%) & 41.72$\pm$5.50 & 95.80$\pm$0.48 & 87.91$\pm$0.59 \\
    \bottomrule
    \bottomrule
    \end{tabular}%
    
  \label{tab:mnist phase}%
\end{table}%

\subsection{Implementation on other Edge Devices}
Our framework can be easily extended to various resource-constrained hardware platforms. As depicted in Fig. \ref{fig:TONN_training}, the TONN inference accelerator can be replaced by any existing inference accelerator on the edge device. Implementing back-propagation-free training requires only minimal additional modules. These modules can be efficiently implemented either through an external software-based control system (MCU/edge CPU) or directly integrated into the hardware architecture, significantly simpler to design compared to implementing back-propagation computation graphs. BP-free training has been extended for memory-efficient on-device training on microcontrollers \cite{zhao2024poor}, low-cost edge processors (\textit{e.g.,} Raspberry Pi Zero 2) \cite{sugiura2025elasticzo}, smartphones \cite{peng2024pocketllm}, edge GPUs \cite{gao2024enabling}, etc.
We believe our method can also be extended to other resource-constrained hardware platforms including field-programmable gate arrays (FPGAs), application-specific integrated circuits (ASICs), and emerging computing paradigms such as probabilistic circuits. 

\end{document}


% This document was modified from the file originally made available by
% Pat Langley and Andrea Danyluk for ICML-2K. This version was created
% by Iain Murray in 2018, and modified by Alexandre Bouchard in
% 2019 and 2021 and by Csaba Szepesvari, Gang Niu and Sivan Sabato in 2022.
% Modified again in 2023 and 2024 by Sivan Sabato and Jonathan Scarlett.
% Previous contributors include Dan Roy, Lise Getoor and Tobias
% Scheffer, which was slightly modified from the 2010 version by
% Thorsten Joachims & Johannes Fuernkranz, slightly modified from the
% 2009 version by Kiri Wagstaff and Sam Roweis's 2008 version, which is
% slightly modified from Prasad Tadepalli's 2007 version which is a
% lightly changed version of the previous year's version by Andrew
% Moore, which was in turn edited from those of Kristian Kersting and
% Codrina Lauth. Alex Smola contributed to the algorithmic style files.
