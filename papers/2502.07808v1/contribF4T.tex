\catcode`\@=11
\let\repdecnomb=\repdecn@mb
\def\figvectU#1[#2]{{\n@rmeuc\v@leur{#2}\invers@\v@leur\v@leur%
    \delt@=\repdecn@mb{\v@leur}\unit@\edef\v@ldelt@{\repdecn@mb{\delt@}}%
    \figg@tXYTD{#2}\v@lX=\v@ldelt@\v@lX\v@lY=\v@ldelt@\v@lY%
    \figv@ctCreg#1(\v@lX,\v@lY)}\ignorespaces}

% Appel : \figptnormBezier Newpt:=ratio,param[Pt1,Pt2,Pt3,Pt4]
\def\figptnormBezier #1:=#2,#3[#4,#5,#6,#7]{
\setc@ntr@l{2}
\figvectDBezier -30 : 1, #3 [#4,#5,#6,#7]
\figvectU -30[-30]
\figptBezier -31::#3[#4,#5,#6,#7]
\figpttra -32:=-31/#2,-30/
\figptrot #1:=-32/-31,90/
}

\newdimen\di@m
\newdimen\di@mA
\newdimen\di@mB
\newdimen\di@men
\newcount\cpt
\newcount\cptt
\newcount\umcpt
% Appel : \psEpsBezier eps[Pt1,Pt2,Pt3,Pt4]
\def\psEpsBezier #1[#2,#3,#4,#5]{
\setc@ntr@l{2}
\cpt=-26\di@m=0.06666666666pt\umcpt=0
\loop\ifnum\cpt<-10
\di@men=\umcpt\di@m
\figptnormBezier\cpt:=#1,\repdecn@mb{\di@men}[#2,#3,#4,#5]
\advance\cpt by1\advance\umcpt by 1
\repeat

\figptscontrol -25 [-26,-25,-24,-23]
\figptscontrol -22 [-23,-22,-21,-20]
\figptscontrol -19 [-20,-19,-18,-17]
\figptscontrol -16 [-17,-16,-15,-14]
\figptscontrol -13 [-14,-13,-12,-11]
\psBezier 5[-26,-25,-24,-23,-22,-21,-20,-19,-18,-17,-16,-15,-14,-13,-12,-11]
}
% Appel : \psEpsLayer eps,nbpts[Pt1,...]
\def\psEpsLayer #1,#2[#3]{
\cptt=0
\def\list@num{#3,0}
\def\nb@pts{#2}
\extrairelepremi@r\p@intd\de\list@num%
\loop\ifnum\cptt<\nb@pts%
\advance\cptt by 1
\edef\p@inta{\p@intd}
\extrairelepremi@r\p@intb\de\list@num%
\extrairelepremi@r\p@intc\de\list@num%
\extrairelepremi@r\p@intd\de\list@num%
{\psEpsBezier #1[\p@inta,\p@intb,\p@intc,\p@intd]}%
\repeat
} 


\catcode`\@=12