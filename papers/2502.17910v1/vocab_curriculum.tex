\documentclass{article}


% if you need to pass options to natbib, use, e.g.:
%     \PassOptionsToPackage{numbers, compress}{natbib}
% before loading neurips_2024


% ready for submission
% \usepackage[preprint]{neurips_2024}
\usepackage[preprint,nonatbib]{neurips_2024}



% to compile a preprint version, e.g., for submission to arXiv, add add the
% [preprint] option:
%     \usepackage[preprint]{neurips_2024}


% to compile a camera-ready version, add the [final] option, e.g.:
%     \usepackage[final]{neurips_2024}


% to avoid loading the natbib package, add option nonatbib:
%    \usepackage[nonatbib]{neurips_2024}


\usepackage[utf8]{inputenc} % allow utf-8 input
\usepackage[T1]{fontenc}    % use 8-bit T1 fonts
\usepackage{hyperref}       % hyperlinks
% \usepackage{natbib}         % for author-year citations
\usepackage{url}            % simple URL typesetting
\usepackage{booktabs}       % professional-quality tables
\usepackage{amsfonts}       % blackboard math symbols
\usepackage{nicefrac}       % compact symbols for 1/2, etc.
\usepackage{microtype}      % microtypography
\usepackage{xcolor}         % colors
% \usepackage{natbib}
\usepackage{algorithm}
\usepackage{algorithmic}
\usepackage{amsmath}

\usepackage{graphicx}
% \usepackage{afterpage}
\usepackage{float}
\usepackage[numbers,sort&compress]{natbib}



\title{Scaling LLM Pre-training with \\ Vocabulary Curriculum}
% \title{Scaling LLM Pre-training with Vocabulary Curriculum}


% The \author macro works with any number of authors. There are two commands
% used to separate the names and addresses of multiple authors: \And and \AND.
%
% Using \And between authors leaves it to LaTeX to determine where to break the
% lines. Using \AND forces a line break at that point. So, if LaTeX puts 3 of 4
% authors names on the first line, and the last on the second line, try using
% \AND instead of \And before the third author name.


\author{%
  Fangyuan Yu \\
  Temus\\
  % \texttt{fangyuan.yu18@gmail.com} \\
}


\begin{document}


\maketitle


\begin{abstract}

Modern language models rely on static vocabularies, fixed before pretraining, in contrast to the adaptive vocabulary acquisition observed in human language learning. To bridge this gap, we introduce vocabulary curriculum learning, an approach that improves pretraining efficiency with log-linear scaling gains relative to vocabulary size. Our method alternates between entropy-guided vocabulary expansion and model optimization, enabling models to learn transferable representations across diverse tokenization granularities. This approach naturally gives rise to an optimal computation allocation pattern: longer tokens capture predictable content, while shorter tokens focus on more complex, harder-to-predict contexts. Experiments on small-scale GPT models demonstrate improved scaling efficiency, reinforcing the effectiveness of dynamic tokenization. We release our code to support further research and plan to extend our experiments to larger models and diverse domains.

\end{abstract}


\section{Introduction}

\begin{figure}[H]
    \centering
    \includegraphics[width=0.9\textwidth]{scale-better-incre-vocab.png}
    \caption{Scaling better with vocabulary curriculum}
    \label{fig:scaling_vocab}
\end{figure}
Modern language model pre-training relies on static vocabularies, fixed before training and detached from the model's learning dynamics—unlike human language acquisition. This fixed approach limits models' ability to adapt to different levels of linguistic granularity, potentially hindering efficiency and performance. While humans acquire language hierarchically, starting with basic units before building more complex representations, language models typically operate with predetermined tokenization schemes.

Our approach dynamically merges predictable tokens, enabling the model to allocate computational resources more efficiently and shift focus toward harder-to-predict patterns. This results in an adaptive curriculum that evolves alongside the model's capabilities. The vocabulary curriculum learning strategy begins with basic units (characters) and progressively expands to more complex representations, allocating more capacity to regions of high modeling entropy and refining the model's understanding of difficult linguistic structures. A digram of our approach is provided in \ref{fig:vocab_curriculum_main}

Empirical results from pre-training GPT models \cite{radford2019gpt2} on the enwiki8 dataset \cite{hutter2006enwiki} highlight two key advantages of our vocabulary curriculum learning approach:

\begin{enumerate}
    \item It improves model performance across various vocabulary sizes, consistently achieving lower bits-per-character (BPC) compared to traditional fixed-vocabulary training.
    \item It enhances scaling efficiency—models trained with a vocabulary curriculum exhibit a shallower slope (0.109 vs. 0.147) in log-scale vocabulary size vs. BPC plots, indicating more effective utilization of larger vocabularies.
\end{enumerate}

As shown in Figure \ref{fig:scaling_vocab}, models trained with incremental vocabulary curriculum learning (red) exhibit a steeper improvement curve compared to compute-matching baselines (blue). The log-scale vocabulary size vs. bits-per-character (BPC) plot reveals that vocabulary curriculum learning achieves a slope of -0.147, meaning it leverages larger vocabularies more effectively than compute-matching learning, which only reaches -0.109.

Additionally, we observe that the curated vocabulary naturally forms a hierarchical structure, where longer tokens become increasingly predictable (lower BPC), while shorter tokens remain harder to predict (higher BPC). This structural organization emerges organically from our training process, reinforcing the effectiveness of our dynamic tokenization strategy.

Our key contributions are:

\begin{itemize}
    \item A dynamic vocabulary creation system that adapts based on model entropy
    \item A curriculum learning approach for tokenization that improves scaling efficiency
    \item Evidence that hierarchical token organization emerges naturally from our approach
\end{itemize}
While our focus is on language modeling, we believe this scaling effect can generalize to other modalities and domains, as byte sequences serve as the fundamental building blocks of digital data.

\begin{figure}[t]
    \centering
    \includegraphics[width=1.0\textwidth]{better-scale-vocab-curriculum-1.png}
    \caption{Scaling better with vocabulary curriculum}
    \label{fig:vocab_curriculum_main}
\end{figure}

\section{Relevant Work}

\subsection{Tokenization Methods and Limitations}

Standard tokenization approaches like Byte Pair Encoding (BPE) \cite{gage1994bpe}, \cite{sennrich2016bpe} rely on static co-occurrence statistics detached from model learning. This creates representational limitations, particularly evident in early language models' struggles with mathematical operations \cite{radford2019gpt2}. Naive BPE tokenization produces inconsistent representations of numbers—for example, "711" might be encoded as a single token while "703" requires multiple tokens. This inconsistency makes it harder for models to learn arithmetic operations compared to specialized approaches that assign unique tokens to all 1-3 digit integers \cite{singh2024tokenizationcountsimpacttokenization}.

Even with a fixed vocabulary, different encoding strategies can produce varying segmentations of the same text. BPE-dropout \cite{provilkov2020bpedropoutsimpleeffectivesubword} leverages this property by introducing stochasticity during training, showing improvements in neural machine translation. More recent work exploits segmentation equivariance during inference to enhance reasoning through self-consistency \cite{sathe2024improvingselfconsistencyllms}. Additionally, research \cite{tao2024scalinglawsvocabularylarger} has established the existence of optimal vocabulary sizes for BPE-style tokenization, which correlate with model size, a log-linear relationship is observed between perplexity and vocabulary size. 

\subsection{Curriculum Learning}

Curriculum learning \cite{Bengio2009CurriculumL} progressively increases task difficulty during training to improve model performance. While successful in LLM post-training \cite{yu2024iterativegraphalignment, lee2024llm2llmboostingllmsnovel}, effective curriculum strategies for pre-training remain challenging \cite{campos2021curriculumlearninglanguagemodeling}. Previous attempts at vocabulary-based curricula for decoder-only models found no improvements \cite{feng2024childdirectedspeecheffectivetraining}, highlighting the difficulty of designing effective curricula for language model pre-training. Our work addresses these limitations with a novel adaptive approach to vocabulary curriculum.

\subsection{Entropy Aware Tokenization}

Recent work has begun exploring entropy-aware tokenization. The Byte Latent Transformer \cite{pagnoni2024bytelatenttransformerpatches} builds tokenization vocabularies using separately trained small language models. However, this approach creates vocabularies that are detached from the actual model's entropy patterns and cannot be dynamically updated during training.

Our work differs by integrating vocabulary building directly into the training process, allowing the tokenization scheme to evolve with the model's developing understanding of the text. This creates a true curriculum that adapts to the specific learning trajectory of each model, rather than imposing a static or pre-computed vocabulary structure.

\section{Approach}

Given a text corpus $\mathcal{D}$ consisting of numerous character sequences, where each sequence $x_{1:m} \in \mathcal{D}$ consists of characters $x_i$ (or bytes). A vocabulary $\mathcal{V}$ and an encoding function $e(x_{1:m}|\mathcal{V})$ together define a tokenization scheme that converts character sequences to token sequences $s_{1:n}$. Language modeling then focuses on predicting the next token: $p(s_n|s_{1:n-1})$ through minimizing entropy $H(s_{t}|s_{1:t-1})$.

We propose a dynamic tokenization framework that jointly learns the vocabulary and encoding strategy alongside the language model. Our approach consists of two key components: (1) entropy-guided vocabulary update and (2) vocabulary curriculum learning.

\subsection{Entropy-Guided Vocabulary Update}

Given a trained language model $f$, we identify mergeable token sequences based on their predictability. For a sequence $(s_1, s_2, \dots, s_n)$, we compute the entropy $H(s_t | s_{1:t-1})$ for each token. A sequence is considered mergeable if all tokens after the first position exhibit monotonically decreasing entropy below threshold $\epsilon$:

$$\text{mergeable}(s_{1:n}) \iff \forall t > 1: H(s_t | s_{1:t-1}) < H(s_{t-1} | s_{1:t-2}) \land H(s_t | s_{1:t-1}) < \epsilon$$

The vocabulary update process can either increase or decrease the vocabulary size:
$$\mathcal{V}_{k+1} = \begin{cases} 
\text{add}(\mathcal{V}_k, f, \mathcal{D}) & \text{for vocabulary expansion} \\
\text{reduce}(\mathcal{V}_k, n_{target}) & \text{for vocabulary reduction}
\end{cases}$$

where $n_{target}$ is the target vocabulary size.

For each new token added to $\mathcal{V}_{k+1}$, we expand the model's embedding layer $W_E \in \mathbb{R}^{|\mathcal{V}| \times d}$ and language modeling head $W_L \in \mathbb{R}^{|\mathcal{V}| \times d}$:
\begin{equation}
    W_E[v_{new}] = h_t^{(L)}, \quad W_L[v_{new}] = W_L[v_t]
\end{equation}
where $h_t^{(L)}$ represents the final hidden state for the merged sequence.

\begin{figure}[H]
    \centering
    \includegraphics[width=0.8\textwidth]{vocab-incre.png}
    \caption{Token grouping process based on entropy patterns from a trained character-level language model}
    \label{fig:group_token}
\end{figure}

Unlike BPE which prohibits merges across space characters, our encoding function $e(x_{1:m}|\mathcal{V})$ allows unrestricted merging. The encoding process identifies longest valid token sequences using a sliding window approach, optimized through a trie structure for efficient prefix matching. To speed up encoding speed for long sequences, we employ batch encoding with additional tokenization at batch boundaries, considering connection sequences of length up to $\max_{v \in \mathcal{V}} |v|$.

The vocabulary management preserves several invariants: (1) non-leaf tokens are preserved during removal to maintain dependencies, (2) token indices reflect merge dependencies where tokens with smaller indices cannot be merges of tokens with larger indices, and (3) token indices align with rows in $W_E$ and $W_L$, enabling vocabulary reduction through prefix slicing.


\subsection{Vocabulary Curriculum Learning}

The curriculum learning process starts with the base vocabulary $\mathcal{V}_0 = \mathcal{A}$ (the character alphabet) and alternates between model optimization and vocabulary updates. At each stage $k$:

1. \textbf{Model Training}: Train the language model $f$ with current vocabulary $\mathcal{V}_k$ using cross-entropy loss:
   $$\mathcal{L}_k = -\sum_{x_{1:m} \in \mathcal{D}} \sum_{t} \log p(s_t | s_{1:t-1}; \mathcal{V}_k)$$
   where $s_{1:n} = e(x_{1:m}|\mathcal{V}_k)$ is the encoded sequence.

2. \textbf{Vocabulary Update}: Based on the trained model's entropy patterns, either expand the vocabulary through entropy-guided merging or reduce it through prefix slicing as defined in the previous section:
   $$\mathcal{V}_{k+1} = \begin{cases} 
   \text{add}(\mathcal{V}_k, f, \mathcal{D}) & \text{for expansion phase} \\
   \text{reduce}(\mathcal{V}_k, n_{target}) & \text{for reduction phase}
   \end{cases}$$

This iterative process continues until reaching the desired model performance or vocabulary size constraints.


\section{Experiments}

We investigate two key questions: (1) Is learning transferable across different vocabularies? and (2) Does vocabulary curriculum improve model performance? 

\subsection{Experimental Setup}

We evaluate our approach on a cleaned version of enwiki8 dataset using a small GPT architecture (context length 512, 6 layers, 6 attention heads, embedding dimension 384, ~10M parameters). The initial vocabulary $\mathcal{V}_0$ consists of 92 characters, and the model is trained with dropout 0.2 without bias terms. For vocabulary updates, we set the entropy threshold $\epsilon=0.3$ and limit per-iteration vocabulary growth to 3K tokens. 

\subsection{Incremental Vocabulary Curriculum}

Our primary experiment consists of 5 iterations of vocabulary expansion, starting from a base model with minimal vocabulary (92) and progressively training models with larger vocabularies (4359, 7941, 11382, 14819, 18276). Each iteration uses the previous model's checkpoint for vocabulary addition. We cap the vocabulary at 18K based on compute-matching experiments showing performance deterioration beyond this size, aligning with observations in \cite{tao2024scalinglawsvocabularylarger} that optimal vocabulary size correlates with model size.

\begin{figure}[H]
    \centering
    \includegraphics[width=1.0\textwidth]{scale-better-incre-vocab.png}
    \caption{Incremental vocabulary learning shows noticeable improvement which scales with vocabulary size in log-linear fashion}
    \label{fig:scaling_vocab1}
\end{figure}

To isolate curriculum effects from training duration, we compare against compute-matching baselines where models are trained from scratch with equivalent total iterations. As shown in Figure \ref{fig:scaling_vocab1}, training with progressively increasing vocabulary reveals a log-linear relationship between vocabulary size and Bits Per Character (BPC), with curriculum learning demonstrating a steeper improvement curve compared to baseline training, detailed BPC at each iteration is documented in \ref{tab:bpc_comparison}


\begin{table}[htbp]
  \centering
  \caption{Comparison of BPC Values Across Different Vocabulary Sizes}
  \label{tab:bpc_comparison}
  \begin{tabular}{lrrrrrr}
    \toprule
    \textbf{Method} & \multicolumn{6}{c}{\textbf{Vocabulary Size}} \\
    \cmidrule(lr){2-7}
    & \textbf{92} & \textbf{4,359} & \textbf{7,941} & \textbf{11,382} & \textbf{14,819} & \textbf{18,276} \\
    \midrule
    incre\_vocab\_curriculum & 1.7141 & 1.5131 & 1.4385 & 1.4032 & 1.3853 & 1.3764 \\
    compute\_matching & 1.7141 & 1.5303 & 1.5103 & 1.5035 & 1.4780 & 1.4637 \\
    \midrule
    \% Improvement & 0.00\% & 1.12\% & 4.75\% & 6.67\% & 6.27\% & 5.96\% \\
    \bottomrule
  \end{tabular}
\end{table}



\subsection{Analysis of Improvement Mechanisms}

To understand the source of these improvements, we analyze per-token BPC distributions across different checkpoints. Figure \ref{fig:iter1_bpc_token_len} shows that at vocabulary size 4359, longer tokens consistently achieve better compression rates, validating our entropy-aware token addition approach.

\begin{figure}
    \centering
    \includegraphics[width=1.0\textwidth]{token-perp-iter1.png}
    \caption{Longer tokens has smaller BPC, contributing to smaller global BPC}
    \label{fig:iter1_bpc_token_len}
\end{figure}

Further analysis across iterations (Figure \ref{fig:bpc-per-len-across-iter} and Table \ref{tab:bpc_per_iter}) reveals two key patterns:
1. Newly created tokens are progressively longer and achieve lower BPC
2. Original shorter tokens become more challenging to model, showing slight BPC increases

\begin{figure}
    \centering
    \includegraphics[width=1.0\textwidth]{bpc-token-len-iter.png}
    \caption{Longer tokens has smaller BPC, contributing to smaller global BPC}
    \label{fig:bpc-per-len-across-iter}
\end{figure}

\begin{table}[h]
\centering
\begin{tabular}{cccccccc}
\hline
\textbf{Token Group} & \textbf{Iter 0} & \textbf{Iter 1} & \textbf{Iter 2} & \textbf{Iter 3} & \textbf{Iter 4} & \textbf{Iter 5} \\
\hline
Iter 0 tokens & 2.26 & 4.20 & 4.24 & 4.47 & 4.60 & 4.82 \\
Iter 1 tokens & - & 1.11 & 1.22 & 1.30 & 1.41 & 1.47 \\
Iter 2 tokens & - & - & 0.78 & 0.83 & 0.88 & 0.91 \\
Iter 3 tokens & - & - & - & 0.77 & 0.80 & 0.84 \\
Iter 4 tokens & - & - & - & - & 0.76 & 0.80 \\
Iter 5 tokens & - & - & - & - & - & 0.76 \\
\hline
\end{tabular}
\caption{BPC values across training iterations. Each row represents tokens introduced at a specific iteration, while columns show how their BPC values change in subsequent iterations. Note the increasing BPC trend for early tokens (top rows) and lower initial BPC for tokens introduced later (bottom rows).}
\label{tab:bpc_per_iter}
\end{table}


This suggests that our curriculum enables the model to effectively learn hierarchical patterns, with longer tokens capturing predictable sequences while shorter tokens specialize in harder-to-predict contexts.

Notably, when testing a decremental vocabulary curriculum (reducing vocabulary size over time), we observe performance comparable to direct training but without additional improvements, suggesting that the benefits of curriculum learning are specifically tied to the incremental approach. 

\subsection{Implication and Future work}

Optimal vocabulary size is correlated with model size \cite{tao2024scalinglawsvocabularylarger}, following this insight, we suspect the scaling improvement might be better for bigger model size. We'll work on extending our experiments therein.  The effectiveness of incremental vocabulary learning suggests its potential application in other modality than text, for instance, in bGPT \cite{wu2024languagemodelsbytemodels}  all digital files can be converted into byte sequences, where the scaling power of vocabulary curriculum could be leveraged to compress the context, as well as improve modeling accuracy. 






% Papers to be submitted to NeurIPS 2024 must be prepared according to the
% instructions presented here. Papers may only be up to {\bf nine} pages long,
% including figures. Additional pages \emph{containing only acknowledgments and
% references} are allowed. Papers that exceed the page limit will not be
% reviewed, or in any other way considered for presentation at the conference.


% The margins in 2024 are the same as those in previous years.


% Authors are required to use the NeurIPS \LaTeX{} style files obtainable at the
% NeurIPS website as indicated below. Please make sure you use the current files
% and not previous versions. Tweaking the style files may be grounds for
% rejection.


% \subsection{Retrieval of style files}


% The style files for NeurIPS and other conference information are available on
% the website at
% \begin{center}
%   \url{http://www.neurips.cc/}
% \end{center}
% The file \verb+neurips_2024.pdf+ contains these instructions and illustrates the
% various formatting requirements your NeurIPS paper must satisfy.


% The only supported style file for NeurIPS 2024 is \verb+neurips_2024.sty+,
% rewritten for \LaTeXe{}.  \textbf{Previous style files for \LaTeX{} 2.09,
%   Microsoft Word, and RTF are no longer supported!}


% The \LaTeX{} style file contains three optional arguments: \verb+final+, which
% creates a camera-ready copy, \verb+preprint+, which creates a preprint for
% submission to, e.g., arXiv, and \verb+nonatbib+, which will not load the
% \verb+natbib+ package for you in case of package clash.


% \paragraph{Preprint option}
% If you wish to post a preprint of your work online, e.g., on arXiv, using the
% NeurIPS style, please use the \verb+preprint+ option. This will create a
% nonanonymized version of your work with the text ``Preprint. Work in progress.''
% in the footer. This version may be distributed as you see fit, as long as you do not say which conference it was submitted to. Please \textbf{do
%   not} use the \verb+final+ option, which should \textbf{only} be used for
% papers accepted to NeurIPS.


% At submission time, please omit the \verb+final+ and \verb+preprint+
% options. This will anonymize your submission and add line numbers to aid
% review. Please do \emph{not} refer to these line numbers in your paper as they
% will be removed during generation of camera-ready copies.


% The file \verb+neurips_2024.tex+ may be used as a ``shell'' for writing your
% paper. All you have to do is replace the author, title, abstract, and text of
% the paper with your own.


% The formatting instructions contained in these style files are summarized in
% Sections \ref{gen_inst}, \ref{headings}, and \ref{others} below.


% \section{General formatting instructions}
% \label{gen_inst}


% The text must be confined within a rectangle 5.5~inches (33~picas) wide and
% 9~inches (54~picas) long. The left margin is 1.5~inch (9~picas).  Use 10~point
% type with a vertical spacing (leading) of 11~points.  Times New Roman is the
% preferred typeface throughout, and will be selected for you by default.
% Paragraphs are separated by \nicefrac{1}{2}~line space (5.5 points), with no
% indentation.


% The paper title should be 17~point, initial caps/lower case, bold, centered
% between two horizontal rules. The top rule should be 4~points thick and the
% bottom rule should be 1~point thick. Allow \nicefrac{1}{4}~inch space above and
% below the title to rules. All pages should start at 1~inch (6~picas) from the
% top of the page.


% For the final version, authors' names are set in boldface, and each name is
% centered above the corresponding address. The lead author's name is to be listed
% first (left-most), and the co-authors' names (if different address) are set to
% follow. If there is only one co-author, list both author and co-author side by
% side.


% Please pay special attention to the instructions in Section \ref{others}
% regarding figures, tables, acknowledgments, and references.


% \section{Headings: first level}
% \label{headings}


% All headings should be lower case (except for first word and proper nouns),
% flush left, and bold.


% First-level headings should be in 12-point type.


% \subsection{Headings: second level}


% Second-level headings should be in 10-point type.


% \subsubsection{Headings: third level}


% Third-level headings should be in 10-point type.


% \paragraph{Paragraphs}


% There is also a \verb+\paragraph+ command available, which sets the heading in
% bold, flush left, and inline with the text, with the heading followed by 1\,em
% of space.


% \section{Citations, figures, tables, references}
% \label{others}


% These instructions apply to everyone.


% \subsection{Citations within the text}


% The \verb+natbib+ package will be loaded for you by default.  Citations may be
% author/year or numeric, as long as you maintain internal consistency.  As to the
% format of the references themselves, any style is acceptable as long as it is
% used consistently.


% The documentation for \verb+natbib+ may be found at
% \begin{center}
%   \url{http://mirrors.ctan.org/macros/latex/contrib/natbib/natnotes.pdf}
% \end{center}
% Of note is the command \verb+\citet+, which produces citations appropriate for
% use in inline text.  For example,
% \begin{verbatim}
%    \citet{hasselmo} investigated\dots
% \end{verbatim}
% produces
% \begin{quote}
%   Hasselmo, et al.\ (1995) investigated\dots
% \end{quote}


% If you wish to load the \verb+natbib+ package with options, you may add the
% following before loading the \verb+neurips_2024+ package:
% \begin{verbatim}
%    \PassOptionsToPackage{options}{natbib}
% \end{verbatim}


% If \verb+natbib+ clashes with another package you load, you can add the optional
% argument \verb+nonatbib+ when loading the style file:
% \begin{verbatim}
%    \usepackage[nonatbib]{neurips_2024}
% \end{verbatim}


% As submission is double blind, refer to your own published work in the third
% person. That is, use ``In the previous work of Jones et al.\ [4],'' not ``In our
% previous work [4].'' If you cite your other papers that are not widely available
% (e.g., a journal paper under review), use anonymous author names in the
% citation, e.g., an author of the form ``A.\ Anonymous'' and include a copy of the anonymized paper in the supplementary material.


% \subsection{Footnotes}


% Footnotes should be used sparingly.  If you do require a footnote, indicate
% footnotes with a number\footnote{Sample of the first footnote.} in the
% text. Place the footnotes at the bottom of the page on which they appear.
% Precede the footnote with a horizontal rule of 2~inches (12~picas).


% Note that footnotes are properly typeset \emph{after} punctuation
% marks.\footnote{As in this example.}


% \subsection{Figures}


% \begin{figure}
%   \centering
%   \fbox{\rule[-.5cm]{0cm}{4cm} \rule[-.5cm]{4cm}{0cm}}
%   \caption{Sample figure caption.}
% \end{figure}


% All artwork must be neat, clean, and legible. Lines should be dark enough for
% purposes of reproduction. The figure number and caption always appear after the
% figure. Place one line space before the figure caption and one line space after
% the figure. The figure caption should be lower case (except for first word and
% proper nouns); figures are numbered consecutively.


% You may use color figures.  However, it is best for the figure captions and the
% paper body to be legible if the paper is printed in either black/white or in
% color.


% \subsection{Tables}


% All tables must be centered, neat, clean and legible.  The table number and
% title always appear before the table.  See Table~\ref{sample-table}.


% Place one line space before the table title, one line space after the
% table title, and one line space after the table. The table title must
% be lower case (except for first word and proper nouns); tables are
% numbered consecutively.


% Note that publication-quality tables \emph{do not contain vertical rules.} We
% strongly suggest the use of the \verb+booktabs+ package, which allows for
% typesetting high-quality, professional tables:
% \begin{center}
%   \url{https://www.ctan.org/pkg/booktabs}
% \end{center}
% This package was used to typeset Table~\ref{sample-table}.


% \begin{table}
%   \caption{Sample table title}
%   \label{sample-table}
%   \centering
%   \begin{tabular}{lll}
%     \toprule
%     \multicolumn{2}{c}{Part}                   \\
%     \cmidrule(r){1-2}
%     Name     & Description     & Size ($\mu$m) \\
%     \midrule
%     Dendrite & Input terminal  & $\sim$100     \\
%     Axon     & Output terminal & $\sim$10      \\
%     Soma     & Cell body       & up to $10^6$  \\
%     \bottomrule
%   \end{tabular}
% \end{table}

% \subsection{Math}
% Note that display math in bare TeX commands will not create correct line numbers for submission. Please use LaTeX (or AMSTeX) commands for unnumbered display math. (You really shouldn't be using \$\$ anyway; see \url{https://tex.stackexchange.com/questions/503/why-is-preferable-to} and \url{https://tex.stackexchange.com/questions/40492/what-are-the-differences-between-align-equation-and-displaymath} for more information.)

% \subsection{Final instructions}

% Do not change any aspects of the formatting parameters in the style files.  In
% particular, do not modify the width or length of the rectangle the text should
% fit into, and do not change font sizes (except perhaps in the
% \textbf{References} section; see below). Please note that pages should be
% numbered.


% \section{Preparing PDF files}


% Please prepare submission files with paper size ``US Letter,'' and not, for
% example, ``A4.''


% Fonts were the main cause of problems in the past years. Your PDF file must only
% contain Type 1 or Embedded TrueType fonts. Here are a few instructions to
% achieve this.


% \begin{itemize}


% \item You should directly generate PDF files using \verb+pdflatex+.


% \item You can check which fonts a PDF files uses.  In Acrobat Reader, select the
%   menu Files$>$Document Properties$>$Fonts and select Show All Fonts. You can
%   also use the program \verb+pdffonts+ which comes with \verb+xpdf+ and is
%   available out-of-the-box on most Linux machines.


% \item \verb+xfig+ "patterned" shapes are implemented with bitmap fonts.  Use
%   "solid" shapes instead.


% \item The \verb+\bbold+ package almost always uses bitmap fonts.  You should use
%   the equivalent AMS Fonts:
% \begin{verbatim}
%    \usepackage{amsfonts}
% \end{verbatim}
% followed by, e.g., \verb+\mathbb{R}+, \verb+\mathbb{N}+, or \verb+\mathbb{C}+
% for $\mathbb{R}$, $\mathbb{N}$ or $\mathbb{C}$.  You can also use the following
% workaround for reals, natural and complex:
% \begin{verbatim}
%    \newcommand{\RR}{I\!\!R} %real numbers
%    \newcommand{\Nat}{I\!\!N} %natural numbers
%    \newcommand{\CC}{I\!\!\!\!C} %complex numbers
% \end{verbatim}
% Note that \verb+amsfonts+ is automatically loaded by the \verb+amssymb+ package.


% \end{itemize}


% If your file contains type 3 fonts or non embedded TrueType fonts, we will ask
% you to fix it.


% \subsection{Margins in \LaTeX{}}


% Most of the margin problems come from figures positioned by hand using
% \verb+\special+ or other commands. We suggest using the command
% \verb+\includegraphics+ from the \verb+graphicx+ package. Always specify the
% figure width as a multiple of the line width as in the example below:
% \begin{verbatim}
%    \usepackage[pdftex]{graphicx} ...
%    \includegraphics[width=0.8\linewidth]{myfile.pdf}
% \end{verbatim}
% See Section 4.4 in the graphics bundle documentation
% (\url{http://mirrors.ctan.org/macros/latex/required/graphics/grfguide.pdf})


% A number of width problems arise when \LaTeX{} cannot properly hyphenate a
% line. Please give LaTeX hyphenation hints using the \verb+\-+ command when
% necessary.

% \begin{ack}
% Use unnumbered first level headings for the acknowledgments. All acknowledgments
% go at the end of the paper before the list of references. Moreover, you are required to declare
% funding (financial activities supporting the submitted work) and competing interests (related financial activities outside the submitted work).
% More information about this disclosure can be found at: \url{https://neurips.cc/Conferences/2024/PaperInformation/FundingDisclosure}.


% Do {\bf not} include this section in the anonymized submission, only in the final paper. You can use the \texttt{ack} environment provided in the style file to automatically hide this section in the anonymized submission.
% \end{ack}

% \begin{bibliography}
% \bibliography{reference}
% \end{bibliography}

\begin{thebibliography}{99}

\bibitem{martinez2023climbcurriculumlearninginfantinspired}
R.~D. Martinez, Z.~Goriely, H.~McGovern, C.~Davis, A.~Caines, P.~Buttery, and L.~Beinborn, ``CLIMB: Curriculum Learning for Infant-inspired Model Building,'' \textit{arXiv preprint arXiv:2311.08886}, 2023.

\bibitem{provilkov2020bpedropoutsimpleeffectivesubword}
I.~Provilkov, D.~Emelianenko, and E.~Voita, ``BPE-Dropout: Simple and Effective Subword Regularization,'' \textit{Association for Computational Linguistics}, pp. 1882--1892, 2020.

\bibitem{sathe2024improvingselfconsistencyllms}
A.~Sathe, D.~Aggarwal, and S.~Sitaram, ``Improving Self Consistency in LLMs through Probabilistic Tokenization,'' \textit{ICML2024}, 2024.

\bibitem{pagnoni2024bytelatenttransformerpatches}
A.~Pagnoni, R.~Pasunuru, P.~Rodriguez, J.~Nguyen, B.~Muller, M.~Li, C.~Zhou, L.~Yu, J.~Weston, L.~Zettlemoyer, G.~Ghosh, M.~Lewis, A.~Holtzman, and S.~Iyer, ``Byte Latent Transformer: Patches Scale Better Than Tokens,'' \textit{arXiv: 2412.09871}, 2024.

\bibitem{tao2024scalinglawsvocabularylarger}
C.~Tao, Q.~Liu, L.~Dou, N.~Muennighoff, Z.~Wan, P.~Luo, M.~Lin, and N.~Wong, ``Scaling Laws with Vocabulary: Larger Models Deserve Larger Vocabularies,'' \textit{NeurIPS 2024}, 2024.

\bibitem{tian2024tokenizeworldobjectlevelknowledge}
R.~Tian, B.~Li, X.~Weng, Y.~Chen, E.~Schmerling, Y.~Wang, B.~Ivanovic, and M.~Pavone, ``Tokenize the World into Object-level Knowledge to Address Long-tail Events in Autonomous Driving,'' \textit{arXiv preprint arXiv:2407.00959}, 2024.

\bibitem{singh2024tokenizationcountsimpacttokenization}
A.~K. Singh and D.~J. Strouse, ``Tokenization counts: the impact of tokenization on arithmetic in frontier LLMs,'' \textit{arXiv: 2402.14903}, 2024.

\bibitem{gage1994bpe}
P.~Gage, ``A New Algorithm for Data Compression,'' \textit{The C Users Journal}, 1994.

\bibitem{sennrich2016bpe}
R.~Sennrich, B.~Haddow, and A.~Birch, ``Neural Machine Translation of Rare Words with Subword Units,'' in \textit{Proceedings of the 54th Annual Meeting of the Association for Computational Linguistics (ACL)}, pp. 1715--1725, 2016.

\bibitem{radford2019gpt2}
A.~Radford, J.~Wu, R.~Child, D.~Luan, D.~Amodei, and I.~Sutskever, ``Language Models are Unsupervised Multitask Learners,'' \textit{OpenAI Blog}, 2019.

\bibitem{hutter2006enwiki}
M.~Hutter, ``The Human Knowledge Compression Prize,'' 2006.

\bibitem{wu2024languagemodelsbytemodels}
S.~Wu, X.~Tan, Z.~Wang, R.~Wang, X.~Li, and M.~Sun, ``Beyond Language Models: Byte Models are Digital World Simulators,'' \textit{arXiv: 2402.19155}, 2024.

\bibitem{Bengio2009CurriculumL}
Y.~Bengio, J.~Louradour, R.~Collobert, and J.~Weston, ``Curriculum learning,'' in \textit{International Conference on Machine Learning}, 2009.

\bibitem{lin2025rho1tokensneed}
Z.~Lin, Z.~Gou, Y.~Gong, X.~Liu, Y.~Shen, R.~Xu, C.~Lin, Y.~Yang, J.~Jiao, N.~Duan, and W.~Chen, ``Rho-1: Not All Tokens Are What You Need,'' \textit{arXiv:2404.07965}, 2025.

\bibitem{yu2024iterativegraphalignment}
F.~Yu, H.~S. Arora, and M.~Johnson, ``Iterative Graph Alignment,'' \textit{arXiv:2408.16667}, 2024.

\bibitem{campos2021curriculumlearninglanguagemodeling}
D.~Campos, ``Curriculum learning for language modeling,'' \textit{arXiv:2108.02170}, 2021.

\bibitem{warstadt2023papersbabylmchallenge}
A.~Warstadt, L.~Choshen, A.~Mueller, A.~Williams, E.~Wilcox, and C.~Zhuang, ``Call for Papers -- The BabyLM Challenge: Sample-efficient pretraining on a developmentally plausible corpus,'' \textit{arXiv: 2301.11796}, 2023.

\bibitem{chollet2019measureintelligence}
F.~Chollet, ``On the Measure of Intelligence,'' \textit{arXiv preprint arXiv:1911.01547}, 2019.

\bibitem{feng2024childdirectedspeecheffectivetraining}
S.~Y. Feng, N.~D. Goodman, and M.~C. Frank, ``Is Child-Directed Speech Effective Training Data for Language Models?,'' \textit{arXiv preprint arXiv:2408.03617}, 2024.

\bibitem{chen2024squidlongcontextnew}
W.~Chen, Z.~Li, S.~Xin, and Y.~Wang, ``Squid: Long Context as a New Modality for Energy-Efficient On-Device Language Models,'' \textit{arXiv preprint arXiv:2408.15518}, 2024.

\bibitem{lee2024llm2llmboostingllmsnovel}
N.~Lee, T.~Wattanawong, S.~Kim, K.~Mangalam, S.~Shen, G.~Anumanchipalli, M.~W. Mahoney, K.~Keutzer, and A.~Gholami, ``LLM2LLM: Boosting LLMs with Novel Iterative Data Enhancement,'' \textit{arXiv preprint arXiv:2403.15042}, 2024.

\end{thebibliography}

% \section*{References}


% References follow the acknowledgments in the camera-ready paper. Use unnumbered first-level heading for
% the references. Any choice of citation style is acceptable as long as you are
% consistent. It is permissible to reduce the font size to \verb+small+ (9 point)
% when listing the references.
% Note that the Reference section does not count towards the page limit.
% \medskip


% {
% \small


% [1] Alexander, J.A.\ \& Mozer, M.C.\ (1995) Template-based algorithms for
% connectionist rule extraction. In G.\ Tesauro, D.S.\ Touretzky and T.K.\ Leen
% (eds.), {\it Advances in Neural Information Processing Systems 7},
% pp.\ 609--616. Cambridge, MA: MIT Press.


% [2] Bower, J.M.\ \& Beeman, D.\ (1995) {\it The Book of GENESIS: Exploring
%   Realistic Neural Models with the GEneral NEural SImulation System.}  New York:
% TELOS/Springer--Verlag.


% [3] Hasselmo, M.E., Schnell, E.\ \& Barkai, E.\ (1995) Dynamics of learning and
% recall at excitatory recurrent synapses and cholinergic modulation in rat
% hippocampal region CA3. {\it Journal of Neuroscience} {\bf 15}(7):5249-5262.
% }


%%%%%%%%%%%%%%%%%%%%%%%%%%%%%%%%%%%%%%%%%%%%%%%%%%%%%%%%%%%%

% \appendix

% \section{Appendix / supplemental material}


% Optionally include supplemental material (complete proofs, additional experiments and plots) in appendix.
% All such materials \textbf{SHOULD be included in the main submission.}

%%%%%%%%%%%%%%%%%%%%%%%%%%%%%%%%%%%%%%%%%%%%%%%%%%%%%%%%%%%%

% \newpage
% \section*{NeurIPS Paper Checklist}

% %%% BEGIN INSTRUCTIONS %%%
% The checklist is designed to encourage best practices for responsible machine learning research, addressing issues of reproducibility, transparency, research ethics, and societal impact. Do not remove the checklist: {\bf The papers not including the checklist will be desk rejected.} The checklist should follow the references and follow the (optional) supplemental material.  The checklist does NOT count towards the page
% limit. 

% Please read the checklist guidelines carefully for information on how to answer these questions. For each question in the checklist:
% \begin{itemize}
%     \item You should answer \answerYes{}, \answerNo{}, or \answerNA{}.
%     \item \answerNA{} means either that the question is Not Applicable for that particular paper or the relevant information is Not Available.
%     \item Please provide a short (1–2 sentence) justification right after your answer (even for NA). 
%    % \item {\bf The papers not including the checklist will be desk rejected.}
% \end{itemize}

% {\bf The checklist answers are an integral part of your paper submission.} They are visible to the reviewers, area chairs, senior area chairs, and ethics reviewers. You will be asked to also include it (after eventual revisions) with the final version of your paper, and its final version will be published with the paper.

% The reviewers of your paper will be asked to use the checklist as one of the factors in their evaluation. While "\answerYes{}" is generally preferable to "\answerNo{}", it is perfectly acceptable to answer "\answerNo{}" provided a proper justification is given (e.g., "error bars are not reported because it would be too computationally expensive" or "we were unable to find the license for the dataset we used"). In general, answering "\answerNo{}" or "\answerNA{}" is not grounds for rejection. While the questions are phrased in a binary way, we acknowledge that the true answer is often more nuanced, so please just use your best judgment and write a justification to elaborate. All supporting evidence can appear either in the main paper or the supplemental material, provided in appendix. If you answer \answerYes{} to a question, in the justification please point to the section(s) where related material for the question can be found.

% IMPORTANT, please:
% \begin{itemize}
%     \item {\bf Delete this instruction block, but keep the section heading ``NeurIPS paper checklist"},
%     \item  {\bf Keep the checklist subsection headings, questions/answers and guidelines below.}
%     \item {\bf Do not modify the questions and only use the provided macros for your answers}.
% \end{itemize} 
 

% %%% END INSTRUCTIONS %%%


% \begin{enumerate}

% \item {\bf Claims}
%     \item[] Question: Do the main claims made in the abstract and introduction accurately reflect the paper's contributions and scope?
%     \item[] Answer: \answerTODO{} % Replace by \answerYes{}, \answerNo{}, or \answerNA{}.
%     \item[] Justification: \justificationTODO{}
%     \item[] Guidelines:
%     \begin{itemize}
%         \item The answer NA means that the abstract and introduction do not include the claims made in the paper.
%         \item The abstract and/or introduction should clearly state the claims made, including the contributions made in the paper and important assumptions and limitations. A No or NA answer to this question will not be perceived well by the reviewers. 
%         \item The claims made should match theoretical and experimental results, and reflect how much the results can be expected to generalize to other settings. 
%         \item It is fine to include aspirational goals as motivation as long as it is clear that these goals are not attained by the paper. 
%     \end{itemize}

% \item {\bf Limitations}
%     \item[] Question: Does the paper discuss the limitations of the work performed by the authors?
%     \item[] Answer: \answerTODO{} % Replace by \answerYes{}, \answerNo{}, or \answerNA{}.
%     \item[] Justification: \justificationTODO{}
%     \item[] Guidelines:
%     \begin{itemize}
%         \item The answer NA means that the paper has no limitation while the answer No means that the paper has limitations, but those are not discussed in the paper. 
%         \item The authors are encouraged to create a separate "Limitations" section in their paper.
%         \item The paper should point out any strong assumptions and how robust the results are to violations of these assumptions (e.g., independence assumptions, noiseless settings, model well-specification, asymptotic approximations only holding locally). The authors should reflect on how these assumptions might be violated in practice and what the implications would be.
%         \item The authors should reflect on the scope of the claims made, e.g., if the approach was only tested on a few datasets or with a few runs. In general, empirical results often depend on implicit assumptions, which should be articulated.
%         \item The authors should reflect on the factors that influence the performance of the approach. For example, a facial recognition algorithm may perform poorly when image resolution is low or images are taken in low lighting. Or a speech-to-text system might not be used reliably to provide closed captions for online lectures because it fails to handle technical jargon.
%         \item The authors should discuss the computational efficiency of the proposed algorithms and how they scale with dataset size.
%         \item If applicable, the authors should discuss possible limitations of their approach to address problems of privacy and fairness.
%         \item While the authors might fear that complete honesty about limitations might be used by reviewers as grounds for rejection, a worse outcome might be that reviewers discover limitations that aren't acknowledged in the paper. The authors should use their best judgment and recognize that individual actions in favor of transparency play an important role in developing norms that preserve the integrity of the community. Reviewers will be specifically instructed to not penalize honesty concerning limitations.
%     \end{itemize}

% \item {\bf Theory Assumptions and Proofs}
%     \item[] Question: For each theoretical result, does the paper provide the full set of assumptions and a complete (and correct) proof?
%     \item[] Answer: \answerTODO{} % Replace by \answerYes{}, \answerNo{}, or \answerNA{}.
%     \item[] Justification: \justificationTODO{}
%     \item[] Guidelines:
%     \begin{itemize}
%         \item The answer NA means that the paper does not include theoretical results. 
%         \item All the theorems, formulas, and proofs in the paper should be numbered and cross-referenced.
%         \item All assumptions should be clearly stated or referenced in the statement of any theorems.
%         \item The proofs can either appear in the main paper or the supplemental material, but if they appear in the supplemental material, the authors are encouraged to provide a short proof sketch to provide intuition. 
%         \item Inversely, any informal proof provided in the core of the paper should be complemented by formal proofs provided in appendix or supplemental material.
%         \item Theorems and Lemmas that the proof relies upon should be properly referenced. 
%     \end{itemize}

%     \item {\bf Experimental Result Reproducibility}
%     \item[] Question: Does the paper fully disclose all the information needed to reproduce the main experimental results of the paper to the extent that it affects the main claims and/or conclusions of the paper (regardless of whether the code and data are provided or not)?
%     \item[] Answer: \answerTODO{} % Replace by \answerYes{}, \answerNo{}, or \answerNA{}.
%     \item[] Justification: \justificationTODO{}
%     \item[] Guidelines:
%     \begin{itemize}
%         \item The answer NA means that the paper does not include experiments.
%         \item If the paper includes experiments, a No answer to this question will not be perceived well by the reviewers: Making the paper reproducible is important, regardless of whether the code and data are provided or not.
%         \item If the contribution is a dataset and/or model, the authors should describe the steps taken to make their results reproducible or verifiable. 
%         \item Depending on the contribution, reproducibility can be accomplished in various ways. For example, if the contribution is a novel architecture, describing the architecture fully might suffice, or if the contribution is a specific model and empirical evaluation, it may be necessary to either make it possible for others to replicate the model with the same dataset, or provide access to the model. In general. releasing code and data is often one good way to accomplish this, but reproducibility can also be provided via detailed instructions for how to replicate the results, access to a hosted model (e.g., in the case of a large language model), releasing of a model checkpoint, or other means that are appropriate to the research performed.
%         \item While NeurIPS does not require releasing code, the conference does require all submissions to provide some reasonable avenue for reproducibility, which may depend on the nature of the contribution. For example
%         \begin{enumerate}
%             \item If the contribution is primarily a new algorithm, the paper should make it clear how to reproduce that algorithm.
%             \item If the contribution is primarily a new model architecture, the paper should describe the architecture clearly and fully.
%             \item If the contribution is a new model (e.g., a large language model), then there should either be a way to access this model for reproducing the results or a way to reproduce the model (e.g., with an open-source dataset or instructions for how to construct the dataset).
%             \item We recognize that reproducibility may be tricky in some cases, in which case authors are welcome to describe the particular way they provide for reproducibility. In the case of closed-source models, it may be that access to the model is limited in some way (e.g., to registered users), but it should be possible for other researchers to have some path to reproducing or verifying the results.
%         \end{enumerate}
%     \end{itemize}


% \item {\bf Open access to data and code}
%     \item[] Question: Does the paper provide open access to the data and code, with sufficient instructions to faithfully reproduce the main experimental results, as described in supplemental material?
%     \item[] Answer: \answerTODO{} % Replace by \answerYes{}, \answerNo{}, or \answerNA{}.
%     \item[] Justification: \justificationTODO{}
%     \item[] Guidelines:
%     \begin{itemize}
%         \item The answer NA means that paper does not include experiments requiring code.
%         \item Please see the NeurIPS code and data submission guidelines (\url{https://nips.cc/public/guides/CodeSubmissionPolicy}) for more details.
%         \item While we encourage the release of code and data, we understand that this might not be possible, so “No” is an acceptable answer. Papers cannot be rejected simply for not including code, unless this is central to the contribution (e.g., for a new open-source benchmark).
%         \item The instructions should contain the exact command and environment needed to run to reproduce the results. See the NeurIPS code and data submission guidelines (\url{https://nips.cc/public/guides/CodeSubmissionPolicy}) for more details.
%         \item The authors should provide instructions on data access and preparation, including how to access the raw data, preprocessed data, intermediate data, and generated data, etc.
%         \item The authors should provide scripts to reproduce all experimental results for the new proposed method and baselines. If only a subset of experiments are reproducible, they should state which ones are omitted from the script and why.
%         \item At submission time, to preserve anonymity, the authors should release anonymized versions (if applicable).
%         \item Providing as much information as possible in supplemental material (appended to the paper) is recommended, but including URLs to data and code is permitted.
%     \end{itemize}


% \item {\bf Experimental Setting/Details}
%     \item[] Question: Does the paper specify all the training and test details (e.g., data splits, hyperparameters, how they were chosen, type of optimizer, etc.) necessary to understand the results?
%     \item[] Answer: \answerTODO{} % Replace by \answerYes{}, \answerNo{}, or \answerNA{}.
%     \item[] Justification: \justificationTODO{}
%     \item[] Guidelines:
%     \begin{itemize}
%         \item The answer NA means that the paper does not include experiments.
%         \item The experimental setting should be presented in the core of the paper to a level of detail that is necessary to appreciate the results and make sense of them.
%         \item The full details can be provided either with the code, in appendix, or as supplemental material.
%     \end{itemize}

% \item {\bf Experiment Statistical Significance}
%     \item[] Question: Does the paper report error bars suitably and correctly defined or other appropriate information about the statistical significance of the experiments?
%     \item[] Answer: \answerTODO{} % Replace by \answerYes{}, \answerNo{}, or \answerNA{}.
%     \item[] Justification: \justificationTODO{}
%     \item[] Guidelines:
%     \begin{itemize}
%         \item The answer NA means that the paper does not include experiments.
%         \item The authors should answer "Yes" if the results are accompanied by error bars, confidence intervals, or statistical significance tests, at least for the experiments that support the main claims of the paper.
%         \item The factors of variability that the error bars are capturing should be clearly stated (for example, train/test split, initialization, random drawing of some parameter, or overall run with given experimental conditions).
%         \item The method for calculating the error bars should be explained (closed form formula, call to a library function, bootstrap, etc.)
%         \item The assumptions made should be given (e.g., Normally distributed errors).
%         \item It should be clear whether the error bar is the standard deviation or the standard error of the mean.
%         \item It is OK to report 1-sigma error bars, but one should state it. The authors should preferably report a 2-sigma error bar than state that they have a 96\% CI, if the hypothesis of Normality of errors is not verified.
%         \item For asymmetric distributions, the authors should be careful not to show in tables or figures symmetric error bars that would yield results that are out of range (e.g. negative error rates).
%         \item If error bars are reported in tables or plots, The authors should explain in the text how they were calculated and reference the corresponding figures or tables in the text.
%     \end{itemize}

% \item {\bf Experiments Compute Resources}
%     \item[] Question: For each experiment, does the paper provide sufficient information on the computer resources (type of compute workers, memory, time of execution) needed to reproduce the experiments?
%     \item[] Answer: \answerTODO{} % Replace by \answerYes{}, \answerNo{}, or \answerNA{}.
%     \item[] Justification: \justificationTODO{}
%     \item[] Guidelines:
%     \begin{itemize}
%         \item The answer NA means that the paper does not include experiments.
%         \item The paper should indicate the type of compute workers CPU or GPU, internal cluster, or cloud provider, including relevant memory and storage.
%         \item The paper should provide the amount of compute required for each of the individual experimental runs as well as estimate the total compute. 
%         \item The paper should disclose whether the full research project required more compute than the experiments reported in the paper (e.g., preliminary or failed experiments that didn't make it into the paper). 
%     \end{itemize}
    
% \item {\bf Code Of Ethics}
%     \item[] Question: Does the research conducted in the paper conform, in every respect, with the NeurIPS Code of Ethics \url{https://neurips.cc/public/EthicsGuidelines}?
%     \item[] Answer: \answerTODO{} % Replace by \answerYes{}, \answerNo{}, or \answerNA{}.
%     \item[] Justification: \justificationTODO{}
%     \item[] Guidelines:
%     \begin{itemize}
%         \item The answer NA means that the authors have not reviewed the NeurIPS Code of Ethics.
%         \item If the authors answer No, they should explain the special circumstances that require a deviation from the Code of Ethics.
%         \item The authors should make sure to preserve anonymity (e.g., if there is a special consideration due to laws or regulations in their jurisdiction).
%     \end{itemize}


% \item {\bf Broader Impacts}
%     \item[] Question: Does the paper discuss both potential positive societal impacts and negative societal impacts of the work performed?
%     \item[] Answer: \answerTODO{} % Replace by \answerYes{}, \answerNo{}, or \answerNA{}.
%     \item[] Justification: \justificationTODO{}
%     \item[] Guidelines:
%     \begin{itemize}
%         \item The answer NA means that there is no societal impact of the work performed.
%         \item If the authors answer NA or No, they should explain why their work has no societal impact or why the paper does not address societal impact.
%         \item Examples of negative societal impacts include potential malicious or unintended uses (e.g., disinformation, generating fake profiles, surveillance), fairness considerations (e.g., deployment of technologies that could make decisions that unfairly impact specific groups), privacy considerations, and security considerations.
%         \item The conference expects that many papers will be foundational research and not tied to particular applications, let alone deployments. However, if there is a direct path to any negative applications, the authors should point it out. For example, it is legitimate to point out that an improvement in the quality of generative models could be used to generate deepfakes for disinformation. On the other hand, it is not needed to point out that a generic algorithm for optimizing neural networks could enable people to train models that generate Deepfakes faster.
%         \item The authors should consider possible harms that could arise when the technology is being used as intended and functioning correctly, harms that could arise when the technology is being used as intended but gives incorrect results, and harms following from (intentional or unintentional) misuse of the technology.
%         \item If there are negative societal impacts, the authors could also discuss possible mitigation strategies (e.g., gated release of models, providing defenses in addition to attacks, mechanisms for monitoring misuse, mechanisms to monitor how a system learns from feedback over time, improving the efficiency and accessibility of ML).
%     \end{itemize}
    
% \item {\bf Safeguards}
%     \item[] Question: Does the paper describe safeguards that have been put in place for responsible release of data or models that have a high risk for misuse (e.g., pretrained language models, image generators, or scraped datasets)?
%     \item[] Answer: \answerTODO{} % Replace by \answerYes{}, \answerNo{}, or \answerNA{}.
%     \item[] Justification: \justificationTODO{}
%     \item[] Guidelines:
%     \begin{itemize}
%         \item The answer NA means that the paper poses no such risks.
%         \item Released models that have a high risk for misuse or dual-use should be released with necessary safeguards to allow for controlled use of the model, for example by requiring that users adhere to usage guidelines or restrictions to access the model or implementing safety filters. 
%         \item Datasets that have been scraped from the Internet could pose safety risks. The authors should describe how they avoided releasing unsafe images.
%         \item We recognize that providing effective safeguards is challenging, and many papers do not require this, but we encourage authors to take this into account and make a best faith effort.
%     \end{itemize}

% \item {\bf Licenses for existing assets}
%     \item[] Question: Are the creators or original owners of assets (e.g., code, data, models), used in the paper, properly credited and are the license and terms of use explicitly mentioned and properly respected?
%     \item[] Answer: \answerTODO{} % Replace by \answerYes{}, \answerNo{}, or \answerNA{}.
%     \item[] Justification: \justificationTODO{}
%     \item[] Guidelines:
%     \begin{itemize}
%         \item The answer NA means that the paper does not use existing assets.
%         \item The authors should cite the original paper that produced the code package or dataset.
%         \item The authors should state which version of the asset is used and, if possible, include a URL.
%         \item The name of the license (e.g., CC-BY 4.0) should be included for each asset.
%         \item For scraped data from a particular source (e.g., website), the copyright and terms of service of that source should be provided.
%         \item If assets are released, the license, copyright information, and terms of use in the package should be provided. For popular datasets, \url{paperswithcode.com/datasets} has curated licenses for some datasets. Their licensing guide can help determine the license of a dataset.
%         \item For existing datasets that are re-packaged, both the original license and the license of the derived asset (if it has changed) should be provided.
%         \item If this information is not available online, the authors are encouraged to reach out to the asset's creators.
%     \end{itemize}

% \item {\bf New Assets}
%     \item[] Question: Are new assets introduced in the paper well documented and is the documentation provided alongside the assets?
%     \item[] Answer: \answerTODO{} % Replace by \answerYes{}, \answerNo{}, or \answerNA{}.
%     \item[] Justification: \justificationTODO{}
%     \item[] Guidelines:
%     \begin{itemize}
%         \item The answer NA means that the paper does not release new assets.
%         \item Researchers should communicate the details of the dataset/code/model as part of their submissions via structured templates. This includes details about training, license, limitations, etc. 
%         \item The paper should discuss whether and how consent was obtained from people whose asset is used.
%         \item At submission time, remember to anonymize your assets (if applicable). You can either create an anonymized URL or include an anonymized zip file.
%     \end{itemize}

% \item {\bf Crowdsourcing and Research with Human Subjects}
%     \item[] Question: For crowdsourcing experiments and research with human subjects, does the paper include the full text of instructions given to participants and screenshots, if applicable, as well as details about compensation (if any)? 
%     \item[] Answer: \answerTODO{} % Replace by \answerYes{}, \answerNo{}, or \answerNA{}.
%     \item[] Justification: \justificationTODO{}
%     \item[] Guidelines:
%     \begin{itemize}
%         \item The answer NA means that the paper does not involve crowdsourcing nor research with human subjects.
%         \item Including this information in the supplemental material is fine, but if the main contribution of the paper involves human subjects, then as much detail as possible should be included in the main paper. 
%         \item According to the NeurIPS Code of Ethics, workers involved in data collection, curation, or other labor should be paid at least the minimum wage in the country of the data collector. 
%     \end{itemize}

% \item {\bf Institutional Review Board (IRB) Approvals or Equivalent for Research with Human Subjects}
%     \item[] Question: Does the paper describe potential risks incurred by study participants, whether such risks were disclosed to the subjects, and whether Institutional Review Board (IRB) approvals (or an equivalent approval/review based on the requirements of your country or institution) were obtained?
%     \item[] Answer: \answerTODO{} % Replace by \answerYes{}, \answerNo{}, or \answerNA{}.
%     \item[] Justification: \justificationTODO{}
%     \item[] Guidelines:
%     \begin{itemize}
%         \item The answer NA means that the paper does not involve crowdsourcing nor research with human subjects.
%         \item Depending on the country in which research is conducted, IRB approval (or equivalent) may be required for any human subjects research. If you obtained IRB approval, you should clearly state this in the paper. 
%         \item We recognize that the procedures for this may vary significantly between institutions and locations, and we expect authors to adhere to the NeurIPS Code of Ethics and the guidelines for their institution. 
%         \item For initial submissions, do not include any information that would break anonymity (if applicable), such as the institution conducting the review.
%     \end{itemize}

% \end{enumerate}


\end{document}