\documentclass[11pt]{article}
\usepackage{graphicx} % Required for inserting images


\usepackage{times}
\usepackage{latexsym}

\usepackage[T1]{fontenc}

\usepackage[utf8]{inputenc}

\usepackage{microtype}

\usepackage{inconsolata}

\usepackage{graphicx}


\usepackage{amsmath}
\usepackage{amssymb}
\usepackage{multirow}
\usepackage{booktabs}
\usepackage{catchfile}

\usepackage[boxed]{algorithm}
\usepackage{varwidth}
\usepackage[noEnd=true,indLines=false]{algpseudocodex}
\usepackage{cleveref}
\makeatletter
\@addtoreset{ALG@line}{algorithm}
\renewcommand{\ALG@beginalgorithmic}{\small}
\algrenewcommand\alglinenumber[1]{\small #1:}
\makeatother

\usepackage[normalem]{ulem}
\usepackage{todonotes}

\usepackage{lipsum}    %
\usepackage{comment}   %
\usepackage{graphicx}  %
\usepackage{pifont}    %

\usepackage[font=small,labelfont=bf]{caption}
\usepackage{float}     %
\usepackage{booktabs}  %
\usepackage{subcaption}  %

\usepackage{listings}

\usepackage{amsthm}  %


\title{Verifying Classification with Limited Disclosure}
\author{Siddharth Bhandari, Liren Shan}
\date{}



\begin{document}

\maketitle

\begin{abstract}%
  We consider the multi-party classification problem introduced by Dong, Hartline, and Vijayaraghavan (2022) motivated by electronic discovery. In this problem, our goal is to design a protocol that guarantees the requesting party receives nearly all responsive documents while minimizing the disclosure of nonresponsive documents. We develop verification protocols that certify the correctness of a classifier by disclosing a few nonresponsive documents. 
  
  We introduce a combinatorial notion called the \emph{Leave-One-Out} dimension of a family of classifiers and show that the number of nonresponsive documents disclosed by our protocol is at most this dimension in the realizable setting, where a perfect classifier exists in this family. 
  For linear classifiers with a margin, we characterize the trade-off between the margin and the number of nonresponsive documents that must be disclosed for verification. Specifically, we establish a trichotomy in this requirement:
  for $d$ dimensional instances, when the margin exceeds $1/3$, verification can be achieved by revealing only $O(1)$ nonresponsive documents; when the margin is exactly $1/3$, in the worst case, at least $\Omega(d)$ nonresponsive documents must be disclosed; when the margin is smaller than $1/3$, verification requires $\Omega(e^d)$ nonresponsive documents.
  We believe this result is of {independent interest} with applications to {coding theory} and {combinatorial geometry}.
  We further extend our protocols to the nonrealizable setting defining an analogous combinatorial quantity robust Leave-One-Out dimension, and to scenarios where the protocol is tolerant to misclassification errors by Alice.
\end{abstract}

\section{Introduction}


\begin{figure}[t]
\centering
\includegraphics[width=0.6\columnwidth]{figures/evaluation_desiderata_V5.pdf}
\vspace{-0.5cm}
\caption{\systemName is a platform for conducting realistic evaluations of code LLMs, collecting human preferences of coding models with real users, real tasks, and in realistic environments, aimed at addressing the limitations of existing evaluations.
}
\label{fig:motivation}
\end{figure}

\begin{figure*}[t]
\centering
\includegraphics[width=\textwidth]{figures/system_design_v2.png}
\caption{We introduce \systemName, a VSCode extension to collect human preferences of code directly in a developer's IDE. \systemName enables developers to use code completions from various models. The system comprises a) the interface in the user's IDE which presents paired completions to users (left), b) a sampling strategy that picks model pairs to reduce latency (right, top), and c) a prompting scheme that allows diverse LLMs to perform code completions with high fidelity.
Users can select between the top completion (green box) using \texttt{tab} or the bottom completion (blue box) using \texttt{shift+tab}.}
\label{fig:overview}
\end{figure*}

As model capabilities improve, large language models (LLMs) are increasingly integrated into user environments and workflows.
For example, software developers code with AI in integrated developer environments (IDEs)~\citep{peng2023impact}, doctors rely on notes generated through ambient listening~\citep{oberst2024science}, and lawyers consider case evidence identified by electronic discovery systems~\citep{yang2024beyond}.
Increasing deployment of models in productivity tools demands evaluation that more closely reflects real-world circumstances~\citep{hutchinson2022evaluation, saxon2024benchmarks, kapoor2024ai}.
While newer benchmarks and live platforms incorporate human feedback to capture real-world usage, they almost exclusively focus on evaluating LLMs in chat conversations~\citep{zheng2023judging,dubois2023alpacafarm,chiang2024chatbot, kirk2024the}.
Model evaluation must move beyond chat-based interactions and into specialized user environments.



 

In this work, we focus on evaluating LLM-based coding assistants. 
Despite the popularity of these tools---millions of developers use Github Copilot~\citep{Copilot}---existing
evaluations of the coding capabilities of new models exhibit multiple limitations (Figure~\ref{fig:motivation}, bottom).
Traditional ML benchmarks evaluate LLM capabilities by measuring how well a model can complete static, interview-style coding tasks~\citep{chen2021evaluating,austin2021program,jain2024livecodebench, white2024livebench} and lack \emph{real users}. 
User studies recruit real users to evaluate the effectiveness of LLMs as coding assistants, but are often limited to simple programming tasks as opposed to \emph{real tasks}~\citep{vaithilingam2022expectation,ross2023programmer, mozannar2024realhumaneval}.
Recent efforts to collect human feedback such as Chatbot Arena~\citep{chiang2024chatbot} are still removed from a \emph{realistic environment}, resulting in users and data that deviate from typical software development processes.
We introduce \systemName to address these limitations (Figure~\ref{fig:motivation}, top), and we describe our three main contributions below.


\textbf{We deploy \systemName in-the-wild to collect human preferences on code.} 
\systemName is a Visual Studio Code extension, collecting preferences directly in a developer's IDE within their actual workflow (Figure~\ref{fig:overview}).
\systemName provides developers with code completions, akin to the type of support provided by Github Copilot~\citep{Copilot}. 
Over the past 3 months, \systemName has served over~\completions suggestions from 10 state-of-the-art LLMs, 
gathering \sampleCount~votes from \userCount~users.
To collect user preferences,
\systemName presents a novel interface that shows users paired code completions from two different LLMs, which are determined based on a sampling strategy that aims to 
mitigate latency while preserving coverage across model comparisons.
Additionally, we devise a prompting scheme that allows a diverse set of models to perform code completions with high fidelity.
See Section~\ref{sec:system} and Section~\ref{sec:deployment} for details about system design and deployment respectively.



\textbf{We construct a leaderboard of user preferences and find notable differences from existing static benchmarks and human preference leaderboards.}
In general, we observe that smaller models seem to overperform in static benchmarks compared to our leaderboard, while performance among larger models is mixed (Section~\ref{sec:leaderboard_calculation}).
We attribute these differences to the fact that \systemName is exposed to users and tasks that differ drastically from code evaluations in the past. 
Our data spans 103 programming languages and 24 natural languages as well as a variety of real-world applications and code structures, while static benchmarks tend to focus on a specific programming and natural language and task (e.g. coding competition problems).
Additionally, while all of \systemName interactions contain code contexts and the majority involve infilling tasks, a much smaller fraction of Chatbot Arena's coding tasks contain code context, with infilling tasks appearing even more rarely. 
We analyze our data in depth in Section~\ref{subsec:comparison}.



\textbf{We derive new insights into user preferences of code by analyzing \systemName's diverse and distinct data distribution.}
We compare user preferences across different stratifications of input data (e.g., common versus rare languages) and observe which affect observed preferences most (Section~\ref{sec:analysis}).
For example, while user preferences stay relatively consistent across various programming languages, they differ drastically between different task categories (e.g. frontend/backend versus algorithm design).
We also observe variations in user preference due to different features related to code structure 
(e.g., context length and completion patterns).
We open-source \systemName and release a curated subset of code contexts.
Altogether, our results highlight the necessity of model evaluation in realistic and domain-specific settings.





\section{Realizable Setting}

In this section, we consider the instance $(X,f)$ that is realizable by some hypothesis class $\calH$. 
In Section~\ref{sec:realizable_general}, we provide a multi-party verification  protocol for the general hypothesis class that achieves perfect recall and nonresponsive disclosure at most the Leave-One-Out dimension (Definition~\ref{defn:leave_one_out}) of the hypothesis class.
In Section~\ref{sec:realizable_linear_margin}, we characterize the Leave-One-Out dimension for the family of linear classifiers with margin $\gamma$.

% \begin{theorem}\label{thm:protocol_realizable}
%     Consider a hypothesis class $\calH$ of binary classifiers on the set $X$ with Leave-One-Out dimension $k$.  
%     Let $f:X\to \{+1,-1\}$ be the true label for responsive/non-responsive documents and $(X,f)$ be a realizable instance with respect to $\calH$. Then, \Cref{alg:abstract_critical_points_protocol} is a multi-party verification protocol with the follwoing properties.
%     \begin{enumerate}
%         \item (Recall) The recall is $1$.
%         \item (Non-responsive disclosure) If Alice reports the labels of all documents correctly, the non-responsive disclosure is at most $k$.
%         \item (Truthful) Alice's best strategy is to report all labels correctly. 
%         \item (Efficiency) Given an oracle $\calO$ for checking membership in $\calH$, the protocol run in time $O(|X|)$.
%     \end{enumerate}
% \end{theorem}

\begin{theorem}\label{thm:protocol_realizable}
    Let $\calH$ be a hypothesis class of binary classifiers on a set $X$ with Leave-One-Out dimension $k$.  
    Suppose $f: X \to \{+1,-1\}$ represents the true labels for responsive and nonresponsive documents, and let $(X,f)$ be a realizable instance with respect to $\calH$. Then, \Cref{alg:abstract_critical_points_protocol} defines a multi-party verification protocol with the following properties:
    \begin{enumerate}
        \item \textbf{(Recall)} The recall is $1$.
        \item \textbf{(Nonresponsive Disclosure)} If Alice reports all labels correctly, the number of disclosed nonresponsive documents is at most $k$.
        \item \textbf{(Truthfulness)} Alice's best strategy is to report all labels truthfully.
        \item \textbf{(Efficiency)} Given an oracle $\calO$ for membership testing in $\calH$, the protocol runs in time $O(|X|)$.
    \end{enumerate}
    % Furthermore, there is a subset $X'\subseteq X$ and a instance $(X',f)$, where $f(x) = -1$ for all $x \in X'$, such that $(X',f)$ is realizable wrt to $\calH$ (by restricting the domain to $X'$) and any protocol with recall $1$ will have nonresponsive disclosure at least $k$.
    % Furthermore, there exists a subset $X' \subseteq X$ such that the instance $(X', f)$, where $f(x) = -1$ for all $x \in X'$, is realizable with respect to $\calH$ (when restricted to $X'$) and has the following property.
    % Any protocol for $(X',f)$ with recall $1$ must incur a nonresponsive disclosure of at least $k$.
    Furthermore, there exists a subset $X' \subseteq X$ such that $(X', f)$, with $f(x) = -1$ for all $x \in X'$, is realizable with respect to $\calH$ (when restricted to $X'$), and any protocol achieving recall $1$ on $(X', f)$ must incur a nonresponsive disclosure of $k-1$.

\end{theorem}


\subsection{Protocol for General Hypothesis Class}\label{sec:realizable_general}

Consider the  multi-party verification protocol \Cref{alg:abstract_critical_points_protocol} when the input is an instance $(X,f)$, realizable by a general hypothesis class $\calH$. The protocol contains a subroutine for computing critical points as shown in Algorithm~\ref{alg:abstract_critical_points_computation}. 
This protocol makes $O(|X|)$ queries to an oracle $\calO$ that checks whether a set of labeled points is realizable by the class $\calH$.
This protocol works as follows.
First, Alice provides Trent with the entire set of documents $X$ along with her labels. Next, Trent applies Algorithm~\ref{alg:abstract_critical_points_computation} to identify critical points based on Alice’s labeling. 
In this algorithm, Trent iterates over each document labeled by Alice as negative, temporarily flips its label to positive, and checks—using an oracle—whether any classifier in $\calH$ can perfectly classify the resulting dataset. If no such classifier exists, that document is removed from consideration. All remaining negatives after this procedure are deemed critical points. Finally, the protocol sends all documents labeled as positive and all critical points to Bob for verification.

There is a key difference between our protocol and the critical points protocol from~\cite{dong2022classification}. In our protocol, if no classifier in $\calH$ perfectly classifies $T_1$ and $T_2$, we remove a point in Step 5. 
By contrast, the protocol in~\cite{dong2022classification} does not remove any points. 
This distinction allows our protocol to handle a general hypothesis class $\calH$. 
To illustrate, consider a classifier $h \in \calH$ that labels $X_A^+$ and two additional points $x_1,x_2 \in X_A^-$ as positive and labels all remaining points $X \setminus (X_A^+\cup \{x_1,x_2\})$ as negatives. 
If there is no classifier in $\calH$ that labels exactly $X_A^+ \cup \{x_1\}$ as positive or exactly $X_A^+ \cup \{x_2\}$ as positive, then the protocol in~\cite{dong2022classification} will not mark $x_1$ or $x_2$ as critical points. Consequently, it can not distinguish $h$ from the classifier that aligns perfectly with Alice’s report.

We now prove that our protocol satisfies Theorem~\ref{thm:protocol_realizable}. 

\begin{algorithm2e}[htb]
\caption{Critical Points Protocol for Hypothesis Class $\calH$}
\label{alg:abstract_critical_points_protocol}
\DontPrintSemicolon
\LinesNumbered


\KwIn{Subroutine for computing critical points (\Cref{alg:abstract_critical_points_computation}), Labeled points from Alice}
\KwOut{A subset of points sent to Bob}

Alice sends all points $X$ to Trent\;
Alice reports to Trent a set $X_A^+ \subseteq X$ as positive ($X_A^- = X \setminus X_A^+$ as negative)\;

% \If{Alice's report $(X_A^+, X_A^-)$ is not separable by a $\gamma$-margin linear classifier}{
%     Trent sends all points $X$ to Bob and the protocol ends\;
% }

Trent computes the critical points $C(X_A^+)$ using Algorithm~\ref{alg:abstract_critical_points_computation} with input $X_A^+$\;
Trent sends points $X_A^+ \cup C(X_A^+)$ to Bob\;
Bob labels the points and sends labels to Trent\;
Trent checks the agreement of reports from Alice and Bob and sends any disputed points to the court to settle\;

\If{the court disagrees with Alice's label on any disputed points}{
    Trent sends all points $X$ to Bob\;
}

\end{algorithm2e}


\begin{algorithm2e}[htb]
\caption{Computing Critical Points for Hypothesis Class $\calH$}
\label{alg:abstract_critical_points_computation}
\DontPrintSemicolon
\LinesNumbered

\KwIn{A set of $n$ points $X = \{x_1,x_2,\dots, x_n\} \subset \mathbb{R}^d$, a set $X_A^+ \subseteq X$ of positive points reported by Alice, and an oracle $\calO$ for checking realizability within hypothesis class $H$.}
\KwOut{A set of critical points $C(X_A^+) \subseteq X_A^-$, where $X_A^- = X \setminus X_A^+$.}

Set $M = X_A^-$\;
\For{each $x_i \in X_A^-$}{
    Set $T_1 = X_A^+ \cup \{x_i\}$ and $T_2 = M \setminus \{x_i\}$\;
    \If{labeling $T_1$ as $+$ and $T_2$ as $-$ is an {invalid} labeling under $\calH$ according to $\calO$}{
        Remove $x_i$ from $M$, ie, $M = M \setminus \{x_i\}$\;
    }
}
Set the critical points $C(X_A^+) = M$\;

\end{algorithm2e}


\begin{proof}[Proof of Theorem~\ref{thm:protocol_realizable}]
    \textbf{$\mathrm{Recall} = 1$:} Let $X_A^+$ and $X_A^-$ be the set of all positives and the set of all negatives reported by Alice, respectively. 
    Without loss of generality, we assume that all points in $X_A^+$ are true positive. 
    (If this is false, Bob will identify any misclassified negative points within $X_A^+$, as these points are always sent to him for verification.)
    Then, let $\calH(X_A^+)$ be the set of all classifiers in $\calH$ that satisfy the following conditions: (1) all points in $X_A^+$ are still classified as positive; and (2) at least one point in $X_A^-$ is classified as positive. 
    
    We first show that using $C(X_A^+)$ we can distinguish two cases: (1) the labels reported by Alice are correct; and (2) Alice labels some true positive points as negative, i.e. the true classifier is in $\calH(X_A^+)$. 
    Specifically, we show that for any classifier $h \in \calH(X_A^+)$, there exists a critical point in $C(X_A^+)$ classified as positive by $h$. 
    Note that critical points $C(X_A^+) \subseteq X_A^-$. 
    If all critical points $C(X_A^+)$ are true negative, then we are in case (1); otherwise, we are in case (2).
    
    We show this by contradiction. Suppose there is no point in $C(X_A^+)$ classified as positive by $h$. 
    Consider any classifier $h \in \calH(X_A^+)$ that classifies at least a point in $X_A^-$ as positive. We use $X_h^+ = \{x \in X: h(x) = 1\}$ to denote the points classified as positive by $h$. Then, we have $X_h^+ \cap X_A^- \neq \varnothing$. 
    Since no point in $C(X_A^+)$ is classified as positive by $h$, all points in $X_h^+ \cap X_A^-$ are removed in Algorithm~\ref{alg:abstract_critical_points_computation}. 
    Now, consider the last point $x_i$ in $X_h^+ \cap X_A^-$ that is removed in Algorithm~\ref{alg:abstract_critical_points_computation}. Let $M_i$ be the set $M$ at the beginning of the iteration at point $x_i$. Since $x_i$ is the last point in $X_h^+ \cap X_A^-$, we have $M_i \setminus \{x_i\} \cap X_h^+ = \varnothing$. Thus, $T_1 = X_A^+ \cup \{x_i\}$ and $T_2 = M_i \setminus \{x_i\}$ are perfectly classified by the classifier $h \in \calH$, which implies that $x_i$ is not removed. 

    Hence, the protocol always guarantees perfect recall, $\mathrm{Recall} = 1$ since if Alice hides any true responsive documents, then Bob will detect such a document and then the court or Trent will reveal all documents to Bob. 

    % \textbf{$\mathrm{Nonresponsive~Disclosure} \leq k$:} Next, we show that the number of critical points is at most the Leave-One-Out dimension $k$.
    % Suppose $C(X_A^+)$ contains $m$ points $v_1,v_2, \dots, v_m \in X$. 
    % We claim that for each $v_i$, there exists a classifier $h_i \in \calH$ that satisfies $h_i(v_i) = 1$ and $h_i(v_j) = -1$ for all $j \neq i$; (In fact $h_i\in \calH(X_A^+)$.) Consider the iteration corresponding to the point $v_i$ in Algorithm~\ref{alg:abstract_critical_points_computation}. Let $M_i$ be the set $M$ at the beginning of this iteration. Since $v_i$ is not removed, there is a classifier $h \in \calH$ that separates $\{v_i\}$ from $M_i \setminus \{v_i\}$. Note that the critical points $C(X_A^+)$ is a subset of $M_i$. This classifier $h$ satisfies the required property.
    % Thus, for each point $v_i$ in $C(X_A^+)$, there exists a classifier in $\calH$ that classifies $v_i$ as positive and other points in $C(X_A^+)$ as negative. By Definition~\ref{defn:leave_one_out}, the number of critical points is at most the Leave-One-Out dimension of $\calH$, i.e. $|C(X_A^+)| \leq k$.

    \textbf{$\mathrm{Nonresponsive~Disclosure} \leq k$:} We now show that the number of critical points is at most the Leave-One-Out dimension $k$.
    Suppose $C(X_A^+)$ contains $m$ points, denoted as $v_1, v_2, \dots, v_m \in X$. We claim that for each $v_i$, there exists a classifier $h_i \in \calH$ such that $h_i(v_i) = 1$ and $h_i(v_j) = -1$ for all $j \neq i$ (in fact, $h_i \in \calH(X_A^+)$). 
    Consider the iteration corresponding to the point $v_i$ in Algorithm~\ref{alg:abstract_critical_points_computation}. Let $M_i$ be the set $M$ at the beginning of this iteration. Since $v_i$ is not removed, there exists a classifier $h \in \calH$ that separates $\{v_i\}$ from $M_i \setminus \{v_i\}$. Noting that the critical points $C(X_A^+)$ form a subset of $M_i$, this classifier $h$ satisfies the desired property.
    Thus, for each point $v_i$ in $C(X_A^+)$, there exists a classifier in $\calH$ that classifies $v_i$ as positive while classifying all other points in $C(X_A^+)$ as negative. By Definition~\ref{defn:leave_one_out}, it follows that the number of critical points is at most the Leave-One-Out dimension of $\calH$, i.e., $|C(X_A^+)| \leq k$.

    Overall, if Alice labels all documents of $X$ correctly, then the protocol only discloses nonresponsive documents in $C(X_A^+)$ to Bob, which is at most $k$ in size. 


    % Finally, we bound the recall and nonresponsive disclosure. This protocol always guarantees perfect recall, $\mathrm{Recall} = 1$ since if Alice hides any true responsive documents, then Bob will detect such a document and then the court or Trent will reveal all documents to Bob. 
    % If Alice labels all documents $X$ correctly, then the protocol only discloses nonresponsive documents in $C(X_A^+)$ to Bob, which is at most $k$. 
    \emph{Truthfulness and Runtime:} Since reporting false labels is guaranteed to reveal all the documents to Bob, it is in Alice's best interest to be truthful. 
    Further, as we only call the membership oracle $\calO$ one per document in $X$ the runtime is bounded by $O(|X|)$.
    
    \emph{Lower Bound:} Let $C \subseteq X$ be the subset witnessing the Leave-One-Out dimension of $X$, so that $|C| = k$ and for each $c \in C$, there exists a classifier $h_c \in \calH$ satisfying $h_c(c) = 1$ and $h_c(c') \neq 1$ for all $c' \in C \setminus \{c\}$. 
    Choose an arbitrary $c \in C$ with its corresponding hypothesis $h_c$, and define $X' = C \setminus \{c\}$. Further, set $f(x) = -1$ for all $x \in X'$. Clearly, $(X', f)$ is realized by $h_c$ when restricted to $X'$. 
    Now, consider any protocol with recall $1$. If the protocol discloses fewer than $k-1$ nonresponsive documents, then there exists some $c' \in X'$ that is not revealed. However, the ground truth could instead be given by $h_{c'}$, which labels $c'$ as $+1$ and all other points in $X'$ as $-1$. In this case, the protocol fails to reveal all relevant documents to Bob, leading to a contradiction.
    
    % \emph{Lower Bound:} Let $C\subseteq X$ be the subset that witnesses the Leave-One-Out dimension of $X$: so $|C|=k$ and for all $c\in C$ there exists a classifier $h_c\in \calH$ such that $h(c) = 1$ and $h(c') \neq 1$ for all $c' \in C\setminus \{c\}$. 
    % Now choose an arbitrary $c\in C$ with the corresponding hypothesis $h_c$, and let $X' = C\setminus\set{c}$. 
    % Further, let $f(x) = -1$ for all $x \in X'$. Then, clearly $(X',f)$ is realized by $h_c$ when restricted to $X'$. 
    % Now, consider any protocol with recall being $1$.
    % If this protocol has nonresponsive disclore lesser than $k-1$, then there is a certain document in $X'$, say $c'$ which isn't revelead by the protcol on input $(X',f)$. However, in this case it is possible that the ground truth was actually given by $h_{c'}$, which lables $c'$ as $+1$ and everthing else in $X'$ as $-1$. Thus, we will not be able to reveal all the relevant documents to Bob, and hence we get a contradiction.
    
    % Let $S \subseteq X_A^-$ be the smallest set that satisfies for any classifier $h \in \calH_\gamma$, if $h$ classifies any point in $X_A^-$ as positive, then there exists a point in $S$ classified as positive by $h$. Suppose $S$ contains $k$ points $v_1,v_2, \dots, v_k \in \bbR^d$. For each $v_i$, there exists a classifier $h_i \in \calH_\gamma$ such that $h_i(v_i) = 1$ and $h_i(v_j) = -1$ for all $j \neq i$. If there is no such $h_i$, then we can remove $v_i$ from $S$ and get a smaller set satisfying the condition. 
    
    % Let $w_i \in \bbR^d$ be the unit-length weight vector of the classifier $h_i(x) = \sgn(\inner{w_i}{x})$. Let $\bar{v}_i = v_i/\|v_i\|_2$. Since $h_i$ has margin at least $\gamma$, we have $\inner{\bar{v}_i}{w_i} \geq \gamma$ and $\inner{\bar{v}_j}{w_i} \leq -\gamma$ for any $j \neq i$. 

    
    % Therefore by Lemma~\ref{lemma:skew-obtuse_lemma}, we have the number of points in $C_\gamma(X_A^+)$ is at most 
    % $$
    % k \leq \frac{2+2\gamma}{3\gamma-1}.
    % $$
\end{proof}





%\subsection{Linear Classification with Margin}

In this section, we provide an E-Discovery protocol for realizable instances with large margins. 

\begin{theorem}\label{thm:protocol_realizable}
    For any instance linearly separable by margin $\gamma > 1/3$, there is a protocol that takes $\gamma$ as input and satisfies
    \begin{enumerate}
        \item (Recall) The recall is $1$.
        \item (Non-responsive disclosure) If Alice reports all labels truthfully, the non-responsive disclosure is at most 
        $$
        \frac{2+2\gamma}{3\gamma -1}.
        $$
        \item (Truthful) Alice's best strategy is to truthfully report all labels. 
    \end{enumerate}
\end{theorem}

\input{Algorithm/Algo_Protocol}
\input{Algorithm/Algo_Critical}

\begin{proof}[Proof of Theorem~\ref{thm:protocol_realizable}]
    The protocol first asks Alice to report labels of all documents in $X$ to Trent. Then, Trent checks whether the labels reported by Alice are linear separable by a $\gamma$-margin classifier. If the labels reported by Alice are not linear separable by margin $\gamma$, then reveal all documents to Bob. 
    
    Now suppose the labels reported by Alice are linear separable by margin $\gamma$. Let $X_A^+$ and $X_A^-$ be the set of all positives and the set of all negatives reported by Alice, respectively. Then, let $\calH_\gamma$ be the set of all linear classifiers that satisfy the following conditions: (1) all points in $X_A^+$ are still classified as positive; and (2) the margin of the classifier is at least $\gamma$; and (3) at least one point in $X_A^-$ is classified as positive. 
    
    We first show that $C_\gamma(X_A^+)$ can distinguish two cases: (1) the labels reported by Alice are correct; and (2) Alice labels some true positive points as negative, i.e. the true classifier is in $\calH_\gamma$. 
    Specifically, we show that for any classifier $h \in \calH_\gamma$, there exists a critical point in $C_\gamma(X_A^+)$ classified as positive by $h$. 
    Note that critical points $C_\gamma(X_A^+) \subseteq X_A^-$. 
    If all critical points $C_\gamma(X_A^+)$ are true negative, then we have case (1); otherwise, we are in case (2).
    We show this by contradiction. Suppose there is no point in $C_\gamma(X_A^+)$ classified as positive by $h$. 
    Consider any classifier $h \in \calH_\gamma$ that classifies at least a point in $X_A^-$ as positive. We use $X_h^+ = \{x \in X: h(x) = 1\}$ to denote the points classified as positive by $h$. Then, we have $X_h^+ \cap X_A^- \neq \varnothing$. Since no point in $C_\gamma(X_A^+)$ is classified as positive by $h$, all points in $X_h^+ \cap X_A^-$ are removed in Algorithm~\ref{alg:large_margin_critical}. Now, consider the last point $x_i$ in $X_h^+ \cap X_A^-$ that are removed in Algorithm~\ref{alg:large_margin_critical}. Let $M_i$ be the set $M$ at the beginning of the iteration at point $x_i$. Since $x_i$ is the last point in $X_h^+ \cap X_A^-$, we have $M_i \setminus \{x_i\} \cap X_h^+ = \varnothing$. Thus, $T_1 = X_A^+ \cup \{x_i\}$ and $T_2 = M_i \setminus \{x_i\}$ are separated by a $\gamma$-margin linear classifier $h$, which implies that $x_i$ is not removed. 

    Suppose $C_\gamma(X_A^+)$ contains $k$ points $v_1,v_2, \dots, v_k \in X$. We now show that for each $v_i$, there exists a linear classifier $h_i$ satisfies two properties: (1) $h_i(v_i) = 1$ and $h_i(v_j) = -1$ for all $j \neq i$; and (2) $h_i$ has a margin $\gamma$ on $C_\gamma(X_A^+)$. ($h_i$ is not required to be in $\calH_\gamma$.) Consider the iteration corresponding to the point $v_i$ in Algorithm~\ref{alg:large_margin_critical}. Let $M_i$ be the set $M$ at the beginning of this iteration. Since $v_i$ is not removed, there is a $\gamma$-margin linear classifier $h$ that separates $\{v_i\}$ from $M_i \setminus \{v_i\}$. Note that the critical points $C_\gamma(X_A^+)$ is a subset of $M_i$. This classifier $h$ satisfies two properties for the point $v_i$.
    
    % Let $S \subseteq X_A^-$ be the smallest set that satisfies for any classifier $h \in \calH_\gamma$, if $h$ classifies any point in $X_A^-$ as positive, then there exists a point in $S$ classified as positive by $h$. Suppose $S$ contains $k$ points $v_1,v_2, \dots, v_k \in \bbR^d$. For each $v_i$, there exists a classifier $h_i \in \calH_\gamma$ such that $h_i(v_i) = 1$ and $h_i(v_j) = -1$ for all $j \neq i$. If there is no such $h_i$, then we can remove $v_i$ from $S$ and get a smaller set satisfying the condition. 
    
    Let $w_i \in \bbR^d$ be the unit-length weight vector of the classifier $h_i(x) = \sgn(\inner{w_i}{x})$. Let $\bar{v}_i = v_i/\|v_i\|_2$. Since $h_i$ has margin at least $\gamma$, we have $\inner{\bar{v}_i}{w_i} \geq \gamma$ and $\inner{\bar{v}_j}{w_i} \leq -\gamma$ for any $j \neq i$. 

    
    Therefore by Lemma~\ref{lemma:skew-obtuse_lemma}, we have the number of points in $C_\gamma(X_A^+)$ is at most 
    $$
    k \leq \frac{2+2\gamma}{3\gamma-1}.
    $$
\end{proof}



\subsection{Leave-One-Out Dimension of Linear Classifiers with Margin}\label{sec:realizable_linear_margin}

In this section, we characterize the Leave-One-Out dimension of the linear classifiers with a margin.

Without loss of generality, we assume that the linear classifier passes through the origin in $\R^d$, as one can add an extra dimension for the bias if needed. Let $w \in \R^d$ be the unit-length weight vector of a linear classifier $h$, so that for any $x \in X$, we have $h(x) = \operatorname{sign}(\langle w, x \rangle)$.

We define the margin of a linear classifier as follows. 
\begin{definition}
\label{defn:margin}
    Given a set of points $X \subset \R^d$, the \emph{margin} of a linear classifier $h$ with unit-length weight vector $w$ on $X$ is defined as 
    $$
    \gamma(w,X) = \min_{x \in X} \frac{|\inner{w}{x}|}{\|x\|_2}.
    $$
\end{definition}
Then, for any $\gamma \in [0,1]$ and $X\subseteq \R^d$, we use $\calH_\gamma(X)$ to denote the hypothesis class of all linear classifiers with a margin at least $\gamma$ on $X$.

We show the following trichotomy of the Leave-One-Out dimension of the hypothesis class $\calH_\gamma(X)$.

\begin{theorem}\label{thm:Leave-One-Out-linear-realizable}
    For any set of points $X \subset \R^d$, any $\gamma \in [0,1]$, the Leave-One-Out dimension $k$, of the class $\calH_\gamma(X)$ is 
    \begin{enumerate}
        \item $k \leq \frac{2+2\gamma}{3\gamma -1}$,  if $\gamma > \frac{1}{3}$.
    \end{enumerate}
    Further, there are sets $X$ such that 
    % the the Leave-One-Out dimension of the class $\calH_\gamma(X)$ is 
    \begin{enumerate}
        \item $k \geq \Omega(d)$, if $\gamma = \frac{1}{3}$;
        \item $k \geq \exp(\Omega((1/3 - \gamma)^2d))$, if $\gamma < \frac{1}{3}$.
    \end{enumerate}
    
\end{theorem}
% \todosid{merge with corollary 5 and add efficiency of oracle claim...and statment changed for accuracy}

\begin{corollary}
    Armed with Theorems~\ref{thm:protocol_realizable} and \ref{thm:Leave-One-Out-linear-realizable}, we analyze instances that are linearly separable with a large margin $\gamma > 1/3$, i.e., there exists $w \in \mathbb{R}^d$ such that for all $x \in X$:
\[
\frac{\langle x, w \rangle \cdot f(x)}{\|x\|_2} \geq \gamma.
\]
For such instances, \Cref{alg:abstract_critical_points_protocol} achieves a nonresponsive disclosure of at most $O(1)$, independent of the instance size $|X|$ and ambient dimension $d$.  
% Furthermore, there is an efficient oracle $\calO$ that verifies the label membership in this class using the hard support vector machine (SVM) (See Theorem 15.8 in~\citet*{shalev2014understanding}). Hence,  \Cref{alg:abstract_critical_points_protocol} is overall efficient.
Furthermore, an efficient oracle $\calO$ verifies label membership in this class using the hard support vector machine (SVM) (see Theorem 15.8 in~\citet*{shalev2014understanding}). Thus, \Cref{alg:abstract_critical_points_protocol} is overall efficient.

For instances with smaller margins, however, any protocol ensuring $\mathrm{Recall} = 1$ must incur a nonresponsive disclosure that scales either linearly with $d$ when $\gamma = 1/3$, or exponentially with $d$ when $\gamma < 1/3$.

\end{corollary}
% \begin{corollary}
%     Armed with Theorem~\ref{thm:protocol_realizable} and Theorem~\ref{thm:Leave-One-Out-linear-realizable}, for the instances that are linearly separable by a large margin $\gamma > 1/3$, ie, there exists  $w \in \bbR^d$ such that for any $x \in X$:
%     $
%     \frac{\inner{x}{w} \cdot f(x)}{\|x\|_2}  \geq \gamma
%     $
%     \Cref{alg:abstract_critical_points_protocol} achieves a nonresponsive disclosure at most $O(1)$, which does not depend on the size of instance $|x|$ and the ambient dimension $d$. 
%     While there are instances with smaller margins, for any protocol with $\mathrm{Recall}=1$ the nonresponsive disclosure could be dependent linearly ($\gamma = 1/3$) or exponentially ($\gamma < 1/3$) on the ambient dimension.
% \end{corollary}

% We prove this theorem using Lemma~\ref{lemma:skew-obtuse_lemma} .

% We now consider the instance $(X,f)$ that is realizable by the hypothesis class $\calH_{\gamma}(X)$. 
% This means the instance $(X,f)$ is linearly separable by margin $\gamma$ since there exists a linear classifier with unit-length weight vector $w \in \bbR^d$ such that for any $x \in X$
%     $$
%     \frac{\inner{x}{w} \cdot f(x)}{\|x\|_2}  \geq \gamma.
%     $$
% By Theorem~\ref{thm:protocol_realizable} and Theorem~\ref{thm:Leave-One-Out-linear-realizable}, for the instances that are linearly separable by a large margin $\gamma > 1/3$, our protocol achieves the nonresponsive disclosure at most $O(1)$, which does not depend on the size of instance $n$ and the dimension $d$. While, in the worst-case instance, the minimal nonresponsive disclosure should contain all nonresponsive documents (See Lemma~\ref{lemma:constructing_skob_family}).

% \begin{corollary}\label{corollay:protocol_realizable}
%     For any instance linearly separable by margin $\gamma > 1/3$, there is a protocol that takes $\gamma$ as input and satisfies
%     \begin{enumerate}
%         \item (Recall) The recall is $1$.
%         \item (Non-responsive disclosure) If Alice reports all labels truthfully, the non-responsive disclosure is at most 
%         $$
%         \frac{2+2\gamma}{3\gamma -1}.
%         $$
%         \item (Truthful) Alice's best strategy is to truthfully report all labels. 
%     \end{enumerate}
% \end{corollary}



%\subsection{Skew-Obtuse Family of Vectors}\label{sec:skew-obtuse family}

% We now bound the Leave-One-Out dimension of linear classifiers with a large margin $\gamma > 1/3$. Our analysis will rely on \Cref{lemma:skew-obtuse_lemma} which is about the geometry of a skew-obtuse family of vectors which might be of independent interest with applications to coding theory/combinatorial geoemtry, as similar statment about usual obtuse vector families are used to prove distance bounds.
% Additionally, we belive that the proof is more nuances that the usual idea of looking at the Gram Matrix of the vector family.

% We now bound the {Leave-One-Out dimension} of linear classifiers with a large margin $\gamma > 1/3$. 
% The proof of \Cref{thm:Leave-One-Out-linear-realizable} follow directly from \Cref{lemma:skew-obtuse_lemma} by thinking of $V,W$ appearing in the lemma as $C$ (the set witnessing the Leave-One-Our dimension) and the corresponding hypothesis $h_c$ respectively.
The proof of \Cref{thm:Leave-One-Out-linear-realizable} follows directly from \Cref{lemma:skew-obtuse_lemma} by interpreting $V$ and $W$ in the lemma as $C$ (the set witnessing the \emph{Leave-One-Out dimension}) and the corresponding hypothesis $h_c$, respectively.
The lemma concerns the geometry of a \emph{skew-obtuse family of vectors}, a result that may be of \emph{independent interest} with applications to \emph{coding theory} and \emph{combinatorial geometry}. Similar statements about standard obtuse vector families are used to establish distance bounds.  
Additionally, we believe that the proof is more {nuanced} than the usual approach of analyzing the \emph{Gram matrix} of the vector family.
The proof is detailed in \Cref{apx:constructing_skob_family}


\begin{lemma}[Skew-Obtuse Family of Vectors]
\label{lemma:skew-obtuse_lemma}
    Let $V= \bar{v}_1,\bar{v}_2,\cdots, \bar{v}_k$ be a $k$ unit vectors in $\R^d$. Further, let  $W = w_1,w_2,\cdots, w_k$ be $k$ other unit vectors in $\R^d$. Suppose there exists a $\gamma\in [0,1]$ such that the following holds:
    \begin{enumerate}
        \item $\inner{\bar{v}_i}{w_i} \geq \gamma$ and
        \item $\inner{\bar{v}_i}{w_j} \leq -\gamma$ if $i\neq j$.
    \end{enumerate}
     Then, if $\gamma>1/3$ we have $k \leq \frac{2+2\gamma}{3\gamma-1}.$ Further if $\gamma=1/3$, there exists a skew-obtuse family of vectors with $k\geq \Omega(d)$ and if $\gamma<1/3$ then there exists such a family with $k\geq \exp{(\Omega(({1/3-\gamma})^2d))}$.
\end{lemma}

% \begin{proof}
%     Let $V= (\bar{v}_1,\bar{v}_2,\cdots, \bar{v}_k)^T$ be a $k\times d$ matrix. 
%     Similarly, we define $W = (w_1,w_2,\cdots, w_k)^T$ be a $k\times d$ matrix. Then, we have $VW^T$ is a $k\times k$ matrix, where each entry is $\inner{\bar{v}_i}{w_j}$. 
%     Let $\{a_1, a_2, \dots, a_d\}$ be the $d$ column vectors of matrix $V$. Let $\{b_1,b_2,\dots, b_d\}$ be the $d$ column vectors of $W$.

%     We first consider the trace of matrix $VW^T$. Since each diagonal entry of $VW^T$ satisfies $\inner{\bar{v}_i}{w_j} \geq \gamma$, we have 
%     $$
%     \Tr(VW^T) = \sum_{i=1}^k \inner{\bar{v}_i}{w_i} \geq k \gamma.
%     $$
%     Since $\bar{v}_i$ and $w_i$ are all unit vectors, we have $\Tr(VW^T) \leq k$.
%     We also have 
%     $$
%     \Tr(VW^T) = \Tr(WV^T) = \sum_{i=1}^d \inner{a_i}{b_i} = \sum_{i=1}^d \|a_i\|\|b_i\|\cos(\theta_i),
%     $$
%     where $\theta_i$ is the angle between vectors $a_i$ and $b_i$. Hence, we get
%     \begin{equation}\label{eq:trace}
%     \sum_{i=1}^d \|a_i\|\|b_i\|\cos(\theta_i) = \Tr(VW^T) \geq k\gamma.
%     \end{equation}

%     We now consider the sum of all entries in matrix $VW^T$. Since all off-diagonal entries of matrix $VW^T$ satisfies $\inner{\bar{v}_i}{w_j} \leq -\gamma$. We use $\one$ to denote all one vector. Then, we have the sum of all entries is
%     $$
%     \one^T VW^T \one = \Tr(VW^T) + \sum_{i\neq j} \inner{\bar{v}_i}{w_j} \leq \Tr(VW^T) - k(k-1)\gamma.
%     $$
%     Note that $VW^T = \sum_{i=1}^d a_i b_i^T$. Thus, we have $\one^T VW^T \one = \sum_{i=1}^d \inner{\one}{a_i} \inner{b_i}{\one}$. Let $\alpha_i$ be the angle between $\one$ and $-a_i$ and $\beta_i$ be the angle between $\one$ and $b_i$. Then, we have 
%     $$
%     - \one^T VW^T \one = \sum_{i=1}^d \inner{\one}{-a_i} \inner{b_i}{\one} = \sum_{i=1}^d \|\one\|^2 \|a_i\|\|b_i\| \cos(\alpha_i) \cos(\beta_i). 
%     $$
%     Since the angle between $-a_i$ and $b_i$ is $\pi - \theta_i$, we have $\alpha_i + \beta_i \geq \pi - \theta_i$. Since the cosine function is decreasing and log-concave on $[0,\pi/2]$, and negative in $[\pi/2, \pi]$, we have 
%     $$
%     \cos(\alpha_i) \cos(\beta_i) \leq \cos^2\left(\frac{\pi-\theta_i}{2}\right).
%     $$
%     By combining the equations above, we have
%     \begin{equation}\label{eq:all_sum}
%     k \sum_{i=1}^d \|a_i\|\|b_i\| \cos^2\left(\frac{\pi-\theta_i}{2}\right) \geq - \one^T VW^T \one \geq - \Tr(VW^T) + k(k-1)\gamma \geq - k + k(k-1)\gamma,
%     \end{equation}
%     where the last inequality is due to $\Tr(VW^T) \leq k$.

%     By combining Equations~(\ref{eq:trace}) and~(\ref{eq:all_sum}), we have 
%     $$
%     \sum_{i=1}^d \|a_i\|\|b_i\| = \sum_{i=1}^d \|a_i\|\|b_i\| \left(2\cos^2\left(\frac{\pi-\theta_i}{2}\right) + \cos(\theta_i)\right) \geq 3k\gamma -2 - 2\gamma.
%     $$
%     By Cauchy-Schwarz inequality, we have 
%     $$
%     \left(\sum_{i=1}^d \|a_i\|\|b_i\| \right)^2 \leq \sum_{i=1}^d \|a_i\|^2 \cdot \sum_{i=1}^d \|b_i\|^2 = k^2,
%     $$
%     where the last equality is from $\sum_{i=1}^d \|a_i\|^2 = \sum_{i=1}^d \|b_i\|^2 = k$ since matrices $V$ and $W$ are both consists of $k$ unit row vectors. Therefore, we have 
%     $$
%     k \leq \frac{2+2\gamma}{3\gamma-1}.
%     $$

%     For the lower bounds see Appendix~\ref{apx:constructing_skob_family}.
% \end{proof}

% %\subsection{Hard Small Margin Instances}\label{sec:hard}

\begin{lemma}[Constructing Skew-Obtuse Family of Vectors when margin $\gamma\leq 1/3$]
\label{lemma:constructing_skob_family}
    For any $\gamma \leq 1/3$, there exists an instance that can be separated by a linear classifier with margin $\gamma$ such that all negatives must be revealed to achieve recall $1$. 
\end{lemma}

The proof of Lemma~\ref{lemma:constructing_skob_family} is in Appendix~\ref{apx:constructing_skob_family}.

% \begin{proof}
%      Let $\varepsilon = 1/\sqrt{3}$.
%     We first consider the case when $\gamma =1/3$.
%    The dataset $X$ contains $n$ data points $x_1,\dots,x_n$ in $\bbR^{n+1}$ such that 
%     $$
%     x_i = \varepsilon \cdot e_1 + \sqrt{1-\varepsilon^2} \cdot e_{i+1},
%     $$
%     where $e_i$ is the $i$-th standard basis vector. Each data point $x_i$ has a unit length. 

%     Consider the following $n$ different label functions $f_1,\dots, f_n$ on this dataset $X$. The function $f_i$ labels the point $x_i$ as positive and all other points as negative. 
%     Let $h_i$ be the linear classifier with the weight vector $w_i = -\varepsilon \cdot e_1 + \sqrt{1-\varepsilon^2} \cdot e_{i+1}$. Then, we have $\inner{h_i}{x_i} = 1-2\varepsilon^2  \geq \gamma$ and $\inner{h_i}{x_j} = -\varepsilon^2  \leq -\gamma$ for all $j \neq i$. 
%     Thus, each instance $(X,f_i)$ is linear separable by margin $\gamma$ and $h_i$ is a $\gamma$-margin classifier for this instance $(X,f_i)$. Therefore, to distinguish these instances and achieve recall $1$, any protocol must reveal all data points for verification.

%     Next we consider the case when $\gamma = 1/3 - \eta$ for some $\eta >0$.
%     Let $\beta = (3/2)\eta$.
%     It is well-known that by picking unit vectors randomly in $\R^d$ we can construct a collection of vectors $v_1, v_2, \ldots v_k$ such that $\inner{v_i}{v_j} \leq \beta$ if $i\neq j$ and $k\geq \exp{(\Omega(\beta^2 d))}$ (see \Cref{lem:exp_many_neearly_orthogoanl_vectors} in the appendix for a proof). 
%     WLOG we can assume that $v_1,\ldots,v_k$ are in the span of $e_2,\ldots,e_{d+1}$. Now, for all $i\in [k]$ we set 
%     $$
%     x_i = \varepsilon \cdot e_1 + \sqrt{1-\varepsilon^2} \cdot v_{i}.
%     $$
%     Further, by setting $w_i = -\varepsilon \cdot e_1 + \sqrt{1-\varepsilon^2} \cdot v_i$ we have that 
%     \[
%     \inner{v_i}{w_i} = -\varepsilon^2 + 1 - \varepsilon^2 = 1/3 \geq \gamma
%     \]
%     and for $i\neq j$
%     \[
%     \inner{v_i}{w_j} \leq -\varepsilon^2 + (1 - \varepsilon^2)\eta = -1/3 + \eta = -\gamma.
%     \]
% \end{proof}
\section{Nonrealizable Setting}
\label{sec:non_realizable}

% In this section, we consider the nonrealizable setting in which there is no classifier in the hypothesis class that can perfectly classify the instance. 
% In this setting, we consider instances that can be classified by a classifier $h$ in a hypothesis class $\calH$ not necessarily perefectly but with a small number of errors. 
In this setting, we consider instances that can be classified by a classifier $h$ in a hypothesis class $\calH$, not necessarily \emph{perfectly}, but with a small number of errors.
We first introduce the \emph{robust Leave-One-Out dimension} and provide a robust protocol for general hypothesis classes that achieves high recall and nonresponsive disclosure at most the robust Leave-One-Out dimension of $\calH$.
Then, we characterize the robust Leave-One-Out dimension of linear classifiers with a margin. 

\begin{definition}[robust Leave-One-Out dimension]
\label{defn:robust_leave_one_out}
     Given a set system $(X,\calS)$ and a slack $L \in \N$, the \emph{robust Leave-One-Out} dimension is the cardinality of the largest set $C\subseteq X$ such that for each element $c \in C$, there exists $S \in \calS$ with $c \in S \cap C$ and $|S\cap C| \leq L+1$.
\end{definition}

\begin{theorem}\label{thm:robust_protocol_general}
    Let $\calH$ be a hypothesis class of binary classifiers on a set $X$ with robust Leave-One-Out dimension $k$ when the slack is $L$.  
    Suppose $f: X \to \{+1,-1\}$ represents the true labels for responsive and nonresponsive documents, and let $(X,f)$ be an instance classified by a classifier in $\calH$ with at most $L$ errors. Then, \Cref{alg:robust_abstract_critical_points_protocol} (in Appendix~\ref{apx:robust_protocol}) defines a multi-party verification protocol with the following properties:
    \begin{enumerate}
        \item \textbf{(Recall)} The recall is at least $1-L/n^+$, where $n^+$ is the cardinality of the true responsive documents.
        \item \textbf{(Nonresponsive Disclosure)} If Alice reports all labels correctly, the number of disclosed nonresponsive documents is at most $k$.
        \item \textbf{(Efficiency)} Given an empirical risk minimization (ERM) oracle $\calO$ for $\calH$ (see \Cref{asp:oracle}), the protocol runs in time $O(|X|)$.
    \end{enumerate}
\end{theorem}

%\subsection{Robust Critical Points Protocol}\label{sec:robust_protocol_general}

% We now describe the robust critical points protocol for the nonrealizable setting. 
The proof of \Cref{thm:robust_protocol_general} appears in Appendix~\ref{apx:robust_protocol} where we also describe \Cref{alg:robust_abstract_critical_points_protocol} in detail. 
The protocol follows the same framework as Algorithm~\ref{alg:abstract_critical_points_protocol}, but instead of using Algorithm~\ref{alg:abstract_critical_points_computation}, it uses Algorithm~\ref{alg:robust_critical} as a subroutine to compute robust critical points.
We also assume access to the following empirical risk minimization (ERM) oracle.

\begin{assumption}\label{asp:oracle}
    Suppose there is an oracle $\calO$ that given any instance $(X,f)$, a hypothesis class $\calH$, a point $x \in X$, finds a classifier in $\calH$ that classifies $x$ as positive and minimizes the total error among all classifiers in $\calH$ that label $x$ as positive.
\end{assumption}

%\begin{algorithm2e}[htb]
\caption{Robust Critical Points Protocol}
\label{alg:robust_protocol}
\DontPrintSemicolon
\LinesNumbered

\KwIn{Subroutine for computing robust critical points (\Cref{alg:robust_critical}), Labeled points from Alice}
\KwOut{A subset of points sent to Bob}

Alice sends all points $X$ to Trent\;
Alice reports to Trent a set $X_A^+ \subseteq X$ as positive ($X_A^- = X \setminus X_A^+$ as negative)\;

% \If{Alice's report $(X_A^+, X_A^-)$ is not separable by a $\gamma$-margin linear classifier}{
%     Trent sends all points $X$ to Bob and the protocol ends\;
% }

Trent computes the critical points $C(X_A^+)$ using Algorithm~\ref{alg:robust_critical} with input $X_A^+$\;
Trent sends points $X_A^+ \cup C(X_A^+)$ to Bob\;
Bob labels the points and sends labels to Trent\;
Trent checks the agreement of reports from Alice and Bob and sends any disputed points to the court to settle\;

\If{the court disagrees with Alice's label on any disputed points}{
    Trent sends all points $X$ to Bob\;
}

\end{algorithm2e}

\begin{algorithm2e}[htb]
\caption{Computing Robust Critical Points}
\label{alg:robust_critical}
\DontPrintSemicolon
\LinesNumbered

\KwIn{A set of $n$ points $X = \{x_1,x_2,\dots, x_n\} \subset \mathbb{R}^d$, a set $X_A^+ \subseteq X$ of positive points reported by Alice, an error tolerance $L \in \mathbb{N}$, and an oracle $\calO$ as in Assumption~\ref{asp:oracle}.}
\KwOut{A set of critical points $C_{L}(X_A^+) \subseteq X_A^-$, where $X_A^- = X \setminus X_A^+$.}

Set $M = X_A^-$\;
\For {each $x_i \in X_A^-$}{
    Set $T_1 = X_A^+ \cup \{x_i\}$ and $T_2 = M \setminus \{x_i\}$\;
    Create an instance $(X',f')$ with $X'=T_1 \cup T_2$ labeling $T_1$ as positive and $T_2$ as negative\;
    Find a classifier in $\calH$ with error at most $L$ on $(X',f')$ that classifies $x_i$ as positive using the oracle $\calO$\;
    \If{no such classifier in $\calH$}{
        Remove $x_i$ from $M$, $M = M \setminus \{x_i\}$\;
    }
}
Set the critical points $C_{L}(X_A^+) = M$\;

\end{algorithm2e}


% \begin{algorithm2e}[htb]
% \caption{Computing Robust Large Margin Critical Points}
% \label{alg:robust_critical}
% \DontPrintSemicolon
% \LinesNumbered

% \KwIn{A set of $n$ points $X = \{x_1,x_2,\dots, x_n\} \subset \mathbb{R}^d$, a set $X_A^+ \subseteq X$ of positive points reported by Alice, a margin $\gamma \in (1/3,1]$, a error tolerance $L \in \mathbb{N}$, and an oracle $\calO$ in Assumption~\ref{asp:oracle}.}
% \KwOut{A set of critical points $C_{\gamma,L}(X_A^+) \subseteq X_A^-$, where $X_A^- = X \setminus X_A^+$.}

% Set $M = X_A^-$\;
% \For {each $x_i \in X_A^-$}{
%     Set $T_1 = X_A^+ \cup \{x_i\}$ and $T_2 = M \setminus \{x_i\}$\;
%     Create an instance $(X',f')$ with points $T_1 \cup T_2$ labeling $T_1$ as positive and $T_2$ as negative\;
%     Find a linear classifiers on $(X',f')$ with margin $\gamma$ and total error $L$ that classifies $x_i$ as positive using the oracle $\calO$\;
%     \If{no linear classifier with margin $\gamma$ and total error $L$ classifies $x_i$ as positive}{
%         Remove $x_i$ from $M$, $M = M \setminus \{x_i\}$\;
%     }
% }
% Set the critical points $C_{\gamma,L}(X_A^+) = M$\;

% \end{algorithm2e}


%\subsection{Robust Leave-One-Out Dimension of Linear Classifiers with Margin}\label{sec:robust_leave_one_out}

We now characterize the robust Leave-One-Out dimension of linear classifiers with a fixed margin. 
In the nonrealizable setting, a large-margin classifier may incur two types of errors: (1) margin error; and (2) classification error. The margin errors occur when points are classified correctly but do not satisfy the margin condition. 
The classification errors occur when points are misclassified by the classifier. 
We consider instances that can be classified by a large margin classifier with a small total error, where the total error contains both the margin errors and the classification errors. 

\begin{definition}
    An instance $(X,f)$ is classified by a linear classifier with margin $\gamma \in [0,1]$ and total error $L \in \N$ if the number of total error is at most
    $$
    |\{x: \inner{\bar{x}}{w} \cdot f(x) < \gamma\}|  \leq L,
    $$
    where $w$ is the weight vector of this classifier and $\bar{x} = x / \|x\|$ be the normalized vector for point $x$.
\end{definition}

For any instance $(X,f)$, any margin $\gamma \in [0,1]$, and any error slack $L \in \N$, let $\calH_{\gamma,L}(X,f)$ be the class of linear classifiers that classifies $(X,f)$ with margin $\gamma$ and total error at most $L$. We then characterize the robust Leave-One-Out dimension of this class $\calH_{\gamma,L}(X,f)$ for a large margin $\gamma > 1/3$. 
% The proof is deferred to Appendix~\ref{apx:robust_protocol}.

\begin{theorem}\label{thm:robust_leave-one-out}
    For any instance $(X,f)$, any $1/3 < \gamma \leq 1$, and any error slack $L \in \N$, the robust Leave-One-Out dimension of the class $\calH_{\gamma, L}(X,f)$ is 
    $$
        \frac{(2+2L)(\gamma+1)}{3\gamma -1}.
    $$
\end{theorem}

% \begin{theorem}\label{thm:robust_protocol}
%     Consider any instance with $n$ points that can be classified by a classifier with margin $\gamma > 1/3$ and total error $L \in \N$.
%     there is a protocol that takes $\gamma$ and $L$ as input and satisfies
%     \begin{enumerate}
%         \item \textbf{(Recall)} The recall is at least $1 - L/n$.
%         \item \textbf{(Non-responsive disclosure)} If Alice reports all labels truthfully, the non-responsive disclosure is at most 
%         $$
%         \frac{(2+2L)(\gamma+1)}{3\gamma -1}.
%         $$
%         %\item (Truthful) Alice's best strategy is to truthfully report all labels. 
%     \end{enumerate}    
% \end{theorem}









% \begin{lemma}[Robust Skew-Obtuse Family of Vectors]
% \label{lemma:robust-skew-obtuse_lemma}
%     Let $V= \bar{v}_1,\bar{v}_2,\cdots, \bar{v}_k$ be a $k$ unit vectors in $\R^d$. Further, let  $W = w_1,w_2,\cdots, w_k$ be $k$ other unit vectors in $\R^d$. Suppose there exists a $\gamma\in [0,1]$ and $L < k$ such that the following holds: 
%     For each $i \in [k]$
%     \begin{enumerate}
%         \item $\inner{\bar{v}_i}{w_i} \geq \gamma$ and
%         \item $\inner{\bar{v}_j}{w_i} \leq -\gamma$ for at least $k - 1 - L$ vectors $\bar v_j$.
%     \end{enumerate}
%      Then, if $\gamma>1/3$ we have $k \leq \frac{(2+2L)(\gamma+1)}{3\gamma-1}.$ Further if $\gamma=1/3$, there exists a skew-obtuse family of vectors with $k\geq \Omega(d)$ and if $\gamma<1/3$ then with $k\geq \exp{(\Omega_{1/3-\gamma}(d))}$.
% \end{lemma}

% \begin{proof}
%     Let $V= (\bar{v}_1,\bar{v}_2,\cdots, \bar{v}_k)^T$ be a $k\times d$ matrix. 
%     Similarly, we define $W = (w_1,w_2,\cdots, w_k)^T$ be a $k\times d$ matrix. Then, we have $VW^T$ is a $k\times k$ matrix, where each entry is $\inner{\bar{v}_i}{w_j}$. 
%     Let $\{a_1, a_2, \dots, a_d\}$ be the $d$ column vectors of matrix $V$. Let $\{b_1,b_2,\dots, b_d\}$ be the $d$ column vectors of $W$.

%     We first consider the trace of matrix $VW^T$. Since each diagonal entry of $VW^T$ satisfies $\inner{\bar{v}_i}{w_j} \geq \gamma$, we have 
%     $$
%     \Tr(VW^T) = \sum_{i=1}^k \inner{\bar{v}_i}{w_i} \geq k \gamma.
%     $$
%     Since $\bar{v}_i$ and $w_i$ are all unit vectors, we have $\Tr(VW^T) \leq k$.
%     We also have 
%     $$
%     \Tr(VW^T) = \Tr(WV^T) = \sum_{i=1}^d \inner{a_i}{b_i} = \sum_{i=1}^d \|a_i\|\|b_i\|\cos(\theta_i),
%     $$
%     where $\theta_i$ is the angle between vectors $a_i$ and $b_i$. Hence, we get
%     \begin{equation}\label{eq:robust-trace}
%     \sum_{i=1}^d \|a_i\|\|b_i\|\cos(\theta_i) = \Tr(VW^T) \geq k\gamma.
%     \end{equation}

%     We now consider the sum of all entries in matrix $VW^T$. 
%     For each column $i \in [k]$ of the matrix $VW^T$, we know that there are at least $k-1-L$ off-diagonal entries with value at most $\inner{\bar{v}_j}{w_i} \leq -\gamma$.
%     For other off-diagonal entries in that column, we upper bound them by one since $\bar{v}_j$ and $w_i$ are unit vectors.
%     We use $\one$ to denote all one vector. Then, we have the sum of all entries is
%     $$
%     \one^T VW^T \one = \Tr(VW^T) + \sum_{i\neq j} \inner{\bar{v}_j}{w_i} \leq \Tr(VW^T) - k(k-1 -L)\gamma + kL.
%     $$
%     Note that $VW^T = \sum_{i=1}^d a_i b_i^T$. Thus, we have $\one^T VW^T \one = \sum_{i=1}^d \inner{\one}{a_i} \inner{b_i}{\one}$. Let $\alpha_i$ be the angle between $\one$ and $-a_i$ and $\beta_i$ be the angle between $\one$ and $b_i$. Then, we have 
%     $$
%     - \one^T VW^T \one = \sum_{i=1}^d \inner{\one}{-a_i} \inner{b_i}{\one} = \sum_{i=1}^d \|\one\|^2 \|a_i\|\|b_i\| \cos(\alpha_i) \cos(\beta_i). 
%     $$
%     Since the angle between $-a_i$ and $b_i$ is $\pi - \theta_i$, we have $\alpha_i + \beta_i \geq \pi - \theta_i$. Since the cosine function is log-concave on $[0,\pi]$, we have 
%     $$
%     \cos(\alpha_i) \cos(\beta_i) \leq \cos^2\left(\frac{\pi-\theta_i}{2}\right).
%     $$
%     By combining the equations above, we have
%     \begin{align*}
%     k \sum_{i=1}^d \|a_i\|\|b_i\| \cos^2\left(\frac{\pi-\theta_i}{2}\right) \geq - \one^T VW^T \one \geq - \Tr(VW^T) + k(k-1-L)\gamma - kL.
%     \end{align*}
%     Since $\Tr(VW^T) \leq k$, we have
%     \begin{equation}\label{eq:robust-all-sum}
%         k \sum_{i=1}^d \|a_i\|\|b_i\| \cos^2\left(\frac{\pi-\theta_i}{2}\right) \geq -k+k(k-1-L)\gamma - kL.
%     \end{equation}

%     By combining Equations~(\ref{eq:robust-trace}) and~(\ref{eq:robust-all-sum}), we have 
%     $$
%     \sum_{i=1}^d \|a_i\|\|b_i\| = \sum_{i=1}^d \|a_i\|\|b_i\| \left(2\cos^2\left(\frac{\pi-\theta_i}{2}\right) + \cos(\theta_i)\right) \geq (3k-2-2L)\gamma -2 - 2L.
%     $$
%     By Cauchy-Schwarz inequality, we have 
%     $$
%     \left(\sum_{i=1}^d \|a_i\|\|b_i\| \right)^2 \leq \sum_{i=1}^d \|a_i\|^2 \cdot \sum_{i=1}^d \|b_i\|^2 = k^2,
%     $$
%     where the last equality is from $\sum_{i=1}^d \|a_i\|^2 = \sum_{i=1}^d \|b_i\|^2 = k$ since matrices $V$ and $W$ are both consists of $k$ unit row vectors. Therefore, we have 
%     $$
%     k \leq \frac{(2+2L)(\gamma+1)}{3\gamma-1}.
%     $$

%     For the lower bounds see \Cref{theorem:constructing_skob_family}.
% \end{proof}

%\subsection{Generalized Skew-Obtuse Family of Vectors}\label{sec:general_skew-obtuse}

% The proof of Theorem~\ref{thm:robust_leave-one-out} follows from the following generalized version of Lemma~\ref{lemma:skew-obtuse_lemma} for the geometric properties of the skew-obtuse family of vectors, in the same way \Cref{thm:robust_leave-one-out} followed from \Cref{lemma:skew-obtuse_lemma}. 
% The proof of the lemma is deferred to Appendix~\ref{apx:proof_general_skew_obtuse}.
% The proof of Theorem~\ref{thm:robust_leave-one-out} follows from a generalized version of Lemma~\ref{lemma:skew-obtuse_lemma}, which extends the geometric properties of a \emph{skew-obtuse family of vectors}. 
% The argument mirrors the way \Cref{thm:Leave-One-Out-linear-realizable} follows from \Cref{lemma:skew-obtuse_lemma}.  
% We will use \Cref{lemma:generalized-skew-obtuse_lemma} with $\alpha = \beta = \gamma$.
% Consider any set $C\subseteq X$ witnessing the robust Leave-One-Out dimension for $\calH_{\gamma,L}(X,f)$. Now, for any point $c\in C$ there is a hypothesis $h_c \in \calH_{\gamma,L}(X,f)$ such that it classifies $c$ as positive and at most $L$ other points in $C$ as positive. Besides these points there can also be other points $C$ classified as negative by $h_c$ but with low margin.
% However, by definition of $\calH_{\gamma,L}(X,f)$ there are at most $L$ total points which are either misclassified or have low margin. Hence, for every $c\in C$ there is unit vector $w_c$ (the weight vector of $h_c$) which satisfies the two properties mentioned in \Cref{lemma:generalized-skew-obtuse_lemma}.  
% The proof of this lemma is deferred to Appendix~\ref{apx:proof_general_skew_obtuse}.
The proof of Theorem~\ref{thm:robust_leave-one-out} follows from a generalized version of Lemma~\ref{lemma:skew-obtuse_lemma}, which extends the geometric properties of a \emph{skew-obtuse family of vectors}. The argument mirrors the way \Cref{thm:Leave-One-Out-linear-realizable} follows from \Cref{lemma:skew-obtuse_lemma}.  
We apply \Cref{lemma:generalized-skew-obtuse_lemma} with $\alpha = \beta = \gamma$. Let $C \subseteq X$ be a set witnessing the robust Leave-One-Out dimension for $\calH_{\gamma,L}(X,f)$. For any $c \in C$, there exists a hypothesis $h_c \in \calH_{\gamma,L}(X,f)$ that classifies $c$ as positive and at most $L$ other points in $C$ as positive. Additionally, $h_c$ may classify some points in $C$ as negative but with low margin.  
By definition of $\calH_{\gamma,L}(X,f)$, at most $L$ points are either misclassified or have low margin. Thus, for each $c \in C$, there exists a unit vector $w_c$ (the weight vector of $h_c$) satisfying the two properties in \Cref{lemma:generalized-skew-obtuse_lemma}.  
The proof of this lemma is deferred to Appendix~\ref{apx:proof_general_skew_obtuse}.

\begin{lemma}[Robust Skew-Obtuse Family of Vectors]
\label{lemma:generalized-skew-obtuse_lemma}
    Let $V= \bar{v}_1,\bar{v}_2,\cdots, \bar{v}_k$ be a $k$ unit vectors in $\R^d$. Further, let  $W = w_1,w_2,\cdots, w_k$ be $k$ other unit vectors in $\R^d$. Suppose there exists $\alpha,\beta\in [0,1]$ and an error parameter $L$ such that the following holds:
    \begin{enumerate}
        \item $\inner{\bar{v}_i}{w_i} \geq \alpha$ and
        \item for all $j\in [k]$ we have $\left|i\in[k]\colon  \inner{\bar{v}_i}{w_j} \leq -\beta\right|\geq k-1-L$. 
    \end{enumerate}
     Then, if $\alpha+2\beta>1$ we have $ k \leq \frac{(2+2L)(1+\beta)}{\alpha+2\beta-1}.$
     % Further if $\gamma=1/3$, there exists a skew-obtuse family of vectors with $k\geq \Omega(d)$ and if $\gamma<1/3$ then with $k\geq \exp{(\Omega_{1/3-\gamma}(d))}$.
\end{lemma}

% \begin{proof}
%     Let $V= (\bar{v}_1,\bar{v}_2,\cdots, \bar{v}_k)^T$ be a $k\times d$ matrix. 
%     Similarly, we define $W = (w_1,w_2,\cdots, w_k)^T$ be a $k\times d$ matrix. Then, we have $VW^T$ is a $k\times k$ matrix, where each entry is $\inner{\bar{v}_i}{w_j}$. 
%     Let $\{a_1, a_2, \dots, a_d\}$ be the $d$ column vectors of matrix $V$. Let $\{b_1,b_2,\dots, b_d\}$ be the $d$ column vectors of $W$.

%     We first consider the trace of matrix $VW^T$. Since each diagonal entry of $VW^T$ satisfies $\inner{\bar{v}_i}{w_j} \geq \alpha$, we have 
%     $$
%     \Tr(VW^T) = \sum_{i=1}^k \inner{\bar{v}_i}{w_i} \geq k \alpha.
%     $$
%     Since $\bar{v}_i$ and $w_i$ are all unit vectors, we have $\Tr(VW^T) \leq k$.
%     We also have 
%     $$
%     \Tr(VW^T) = \Tr(WV^T) = \sum_{i=1}^d \inner{a_i}{b_i} = \sum_{i=1}^d \|a_i\|\|b_i\|\cos(\theta_i),
%     $$
%     where $\theta_i$ is the angle between vectors $a_i$ and $b_i$. Hence, we get
%     \begin{equation}\label{eq:trace2}
%     \sum_{i=1}^d \|a_i\|\|b_i\|\cos(\theta_i) = \Tr(VW^T) \geq k\alpha.
%     \end{equation}

%     We now consider the sum of all entries in matrix $VW^T$. Since all off-diagonal entries of matrix $VW^T$ satisfies $\inner{\bar{v}_i}{w_j} \leq -\beta$. We use $\one$ to denote all one vector. Then, we have the sum of all entries is
%     $$
%     \one^T VW^T \one = \Tr(VW^T) + \sum_{i\neq j} \inner{\bar{v}_i}{w_j} \leq \Tr(VW^T) - k(k-1)\beta.
%     $$
%     Note that $VW^T = \sum_{i=1}^d a_i b_i^T$. Thus, we have $\one^T VW^T \one = \sum_{i=1}^d \inner{\one}{a_i} \inner{b_i}{\one}$. Let $\alpha_i$ be the angle between $\one$ and $-a_i$ and $\beta_i$ be the angle between $\one$ and $b_i$. Then, we have 
%     $$
%     - \one^T VW^T \one = \sum_{i=1}^d \inner{\one}{-a_i} \inner{b_i}{\one} = \sum_{i=1}^d \|\one\|^2 \|a_i\|\|b_i\| \cos(\alpha_i) \cos(\beta_i). 
%     $$
%     Since the angle between $-a_i$ and $b_i$ is $\pi - \theta_i$, we have $\alpha_i + \beta_i \geq \pi - \theta_i$. Since the cosine function is log-concave on $[0,\pi]$, we have 
%     $$
%     \cos(\alpha_i) \cos(\beta_i) \leq \cos^2\left(\frac{\pi-\theta_i}{2}\right).
%     $$
%     By combining the equations above, we have
%     \begin{equation}\label{eq:all_sum2}
%     k \sum_{i=1}^d \|a_i\|\|b_i\| \cos^2\left(\frac{\pi-\theta_i}{2}\right) \geq - \one^T VW^T \one \geq - \Tr(VW^T) + k(k-1)\beta \geq - k + k(k-1)\beta,
%     \end{equation}
%     where the last inequality is due to $\Tr(VW^T) \leq k$.

%     By combining Equations~(\ref{eq:trace2}) and~(\ref{eq:all_sum2}), we have 
%     $$
%     \sum_{i=1}^d \|a_i\|\|b_i\| = \sum_{i=1}^d \|a_i\|\|b_i\| \left(2\cos^2\left(\frac{\pi-\theta_i}{2}\right) + \cos(\theta_i)\right) \geq k(\alpha+2\beta) -2 - 2\beta.
%     $$
%     By Cauchy-Schwarz inequality, we have 
%     $$
%     \left(\sum_{i=1}^d \|a_i\|\|b_i\| \right)^2 \leq \sum_{i=1}^d \|a_i\|^2 \cdot \sum_{i=1}^d \|b_i\|^2 = k^2,
%     $$
%     where the last equality is from $\sum_{i=1}^d \|a_i\|^2 = \sum_{i=1}^d \|b_i\|^2 = k$ since matrices $V$ and $W$ are both consists of $k$ unit row vectors. Therefore, we have 
%     $$
%     k \leq \frac{2+2\beta}{\alpha+2\beta-1}.
%     $$

% \end{proof}

% \begin{proof}
%     Let $V= (\bar{v}_1,\bar{v}_2,\cdots, \bar{v}_k)^T$ be a $k\times d$ matrix. 
%     Similarly, we define $W = (w_1,w_2,\cdots, w_k)^T$ be a $k\times d$ matrix. Then, we have $VW^T$ is a $k\times k$ matrix, where each entry is $\inner{\bar{v}_i}{w_j}$. 
%     Let $\{a_1, a_2, \dots, a_d\}$ be the $d$ column vectors of matrix $V$. Let $\{b_1,b_2,\dots, b_d\}$ be the $d$ column vectors of $W$.

%     We first consider the trace of matrix $VW^T$. Since each diagonal entry of $VW^T$ satisfies $\inner{\bar{v}_i}{w_j} \geq \alpha$, we have 
%     $$
%     \Tr(VW^T) = \sum_{i=1}^k \inner{\bar{v}_i}{w_i} \geq k \alpha.
%     $$
%     Since $\bar{v}_i$ and $w_i$ are all unit vectors, we have $\Tr(VW^T) \leq k$.
%     We also have 
%     $$
%     \Tr(VW^T) = \Tr(WV^T) = \sum_{i=1}^d \inner{a_i}{b_i} = \sum_{i=1}^d \|a_i\|\|b_i\|\cos(\theta_i),
%     $$
%     where $\theta_i$ is the angle between vectors $a_i$ and $b_i$. Hence, we get
%     \begin{equation}\label{eq:robust-trace}
%     \sum_{i=1}^d \|a_i\|\|b_i\|\cos(\theta_i) = \Tr(VW^T) \geq k\alpha.
%     \end{equation}

%     We now consider the sum of all entries in matrix $VW^T$. 
%     For each column $i \in [k]$ of the matrix $VW^T$, we know that there are at least $k-1-L$ off-diagonal entries with value at most $\inner{\bar{v}_j}{w_i} \leq -\beta$.
%     For other off-diagonal entries in that column, we upper bound them by one since $\bar{v}_j$ and $w_i$ are unit vectors.
%     We use $\one$ to denote all one vector. Then, we have the sum of all entries is
%     $$
%     \one^T VW^T \one = \Tr(VW^T) + \sum_{i\neq j} \inner{\bar{v}_j}{w_i} \leq \Tr(VW^T) - k(k-1 -L)\beta + kL.
%     $$
%     Note that $VW^T = \sum_{i=1}^d a_i b_i^T$. Thus, we have $\one^T VW^T \one = \sum_{i=1}^d \inner{\one}{a_i} \inner{b_i}{\one}$. Let $\phi_i$ be the angle between $\one$ and $-a_i$ and $\psi_i$ be the angle between $\one$ and $b_i$. Then, we have 
%     $$
%     - \one^T VW^T \one = \sum_{i=1}^d \inner{\one}{-a_i} \inner{b_i}{\one} = \sum_{i=1}^d \|\one\|^2 \|a_i\|\|b_i\| \cos(\phi_i) \cos(\psi_i). 
%     $$
%     Since the angle between $-a_i$ and $b_i$ is $\pi - \theta_i$, we have $\phi_i + \psi_i \geq \pi - \theta_i$. Since the cosine function is log-concave on $[0,\pi]$, we have 
%     $$
%     \cos(\phi_i) \cos(\psi_i) \leq \cos^2\left(\frac{\pi-\theta_i}{2}\right).
%     $$
%     By combining the equations above, we have
%     \begin{align*}
%     k \sum_{i=1}^d \|a_i\|\|b_i\| \cos^2\left(\frac{\pi-\theta_i}{2}\right) \geq - \one^T VW^T \one \geq - \Tr(VW^T) + k(k-1-L)\beta - kL.
%     \end{align*}
%     Since $\Tr(VW^T) \leq k$, we have
%     \begin{equation}\label{eq:robust-all-sum}
%         k \sum_{i=1}^d \|a_i\|\|b_i\| \cos^2\left(\frac{\pi-\theta_i}{2}\right) \geq -k+k(k-1-L)\beta - kL.
%     \end{equation}

%     By combining Equations~(\ref{eq:robust-trace}) and~(\ref{eq:robust-all-sum}), we have 
%     $$
%     \sum_{i=1}^d \|a_i\|\|b_i\| = \sum_{i=1}^d \|a_i\|\|b_i\| \left(2\cos^2\left(\frac{\pi-\theta_i}{2}\right) + \cos(\theta_i)\right) \geq k\alpha+(2k-2-2L)\beta -2 - 2L.
%     $$
%     By Cauchy-Schwarz inequality, we have 
%     $$
%     \left(\sum_{i=1}^d \|a_i\|\|b_i\| \right)^2 \leq \sum_{i=1}^d \|a_i\|^2 \cdot \sum_{i=1}^d \|b_i\|^2 = k^2,
%     $$
%     where the last equality is from $\sum_{i=1}^d \|a_i\|^2 = \sum_{i=1}^d \|b_i\|^2 = k$ since matrices $V$ and $W$ are both consists of $k$ unit row vectors. Therefore, we have 
%     $$
%     k \leq \frac{(2+2L)(\beta+1)}{\alpha + 2\beta-1}.
%     $$

%     For the lower bounds see Lemma~\ref{lemma:constructing_skob_family}.
% \end{proof}
\section{Error-Tolerant Protocol}

In this section, we provide protocols that are tolerant of classification errors made by Alice. For protocols proposed in previous sections, if Bob detects one document misclassified by Alice, the court or Trent reveals all documents to Bob. 
However, \citet*{grossman2010technology,grossman2012inconsistent} show that, in practice, even the most skilled human reviewers can unintentionally make classification errors.
Therefore, we aim to design protocols that are less strict on Alice's classification, imposing a gradual rather than harsh penalty for Alice's misclassification errors.
We show that our previous protocols can be converted to the following error-tolerant protocols.

\begin{theorem}
\label{thm:error_tolerant_protocol}
    Suppose there exists a multi-party classification protocol that satisfies if Bob does not detect any document misclassified by Alice, then the recall is at least $\alpha$ and the nonresponsive disclosure is at most $k$.
    Then, there exists an error-tolerant protocol that guarantees:
    \begin{enumerate}
        \item \textbf{(Recall)} The recall is at least $\alpha$;
        \item \textbf{(Nonresponsive Disclosure)} The nonresponsive disclosure is at most $k\cdot E$ if the protocol detects $E$ documents misclassified by Alice.
    \end{enumerate}
\end{theorem}

\begin{proof}
    The error-tolerant protocol is constructed as follows. Let $P$ be a multi-party classification protocol that ensures a recall of at least $\alpha$ and discloses at most $k$ nonresponsive documents when Bob does not detect any document misclassified by Alice. (This is not the number of errors made by the optimal classifier.)
    The error-tolerant protocol iteratively calls protocol $P$ until an iteration of $P$ completes without any detected misclassification by Alice.
    
    If Bob identifies a document misclassified by Alice in any call of $P$, Alice is required to relabel the document correctly. 
    Since each detected misclassification triggers at most $k$ additional nonresponsive disclosures per iteration, the total nonresponsive disclosure remains bounded by $k \cdot E$, where $E$ is the number of errors detected.
\end{proof}

\begin{algorithm2e}[htb]
\caption{Computing Critical Points with Verified Points}
\label{alg:critical_negative}
\DontPrintSemicolon
\LinesNumbered

\KwIn{A set of $n$ points $X = \{x_1,x_2,\dots, x_n\} \subset \mathbb{R}^d$, a set $A^- \subseteq X$ of verified negative points, a set $X_A^+ \subseteq X$ of positive points reported by Alice, and an oracle $\calO$ for checking realizability within hypothesis class $H$.}
\KwOut{A set of critical points $C(X_A^+, A^-) \subseteq X_A^-$, where $X_A^- = X \setminus X_A^+$.}

Set $M = X_A^-$\;
\For{each $x_i \in X_A^- \setminus A^-$}{
    Set $T_1 = X_A^+ \cup \{x_i\}$ and $T_2 = M \setminus \{x_i\}$\;
    \If{labeling $T_1$ as $+$ and $T_2$ as $-$ is an {invalid} labeling under $\calH$ according to $\calO$}{
        Remove $x_i$ from $M$, ie, $M = M \setminus \{x_i\}$\;
    }
}
Set the critical points $C(X_A^+,A^-) = M$\;

\end{algorithm2e}


Note that the error-tolerant protocol framework shown in Theorem~\ref{thm:error_tolerant_protocol} may call the one-round multi-party protocol multiple times. 
If Bob detects misclassified documents, then in later calls, instead of certified positive documents reported by Alice, there may also be some negative documents verified by Bob in previous calls. 
Thus, we can modify the algorithms for computing the critical points (Algorithm~\ref{alg:abstract_critical_points_computation}) and robust critical points (Algorithm~\ref{alg:robust_critical}) to achieve fewer nonresponsive disclosure by taking those certified negative documents into account. 
We show the modified algorithm for computing the critical points in the realizable setting in Algorithm~\ref{alg:critical_negative}.
Algorithm~\ref{alg:robust_critical} for the robust critical points in the nonrealizable setting can be extended similarly.
In practice, using these modified algorithms to compute the critical points in the error-tolerant protocol can further reduce the nonresponsive disclosure while preserving the recall guarantee.
\section{Conclusion}
In this work, we propose a simple yet effective approach, called SMILE, for graph few-shot learning with fewer tasks. Specifically, we introduce a novel dual-level mixup strategy, including within-task and across-task mixup, for enriching the diversity of nodes within each task and the diversity of tasks. Also, we incorporate the degree-based prior information to learn expressive node embeddings. Theoretically, we prove that SMILE effectively enhances the model's generalization performance. Empirically, we conduct extensive experiments on multiple benchmarks and the results suggest that SMILE significantly outperforms other baselines, including both in-domain and cross-domain few-shot settings.

% Acknowledgments---Will not appear in anonymized version
%\acks{We thank a bunch of people and funding agency.}

\bibliographystyle{plainnat}
\bibliography{ref}

\newpage

\appendix

\section{Proof of Lemma~\ref{lemma:skew-obtuse_lemma}}\label{apx:constructing_skob_family}

% In this section, we prove the Lemma~\ref{lemma:constructing_skob_family}

\begin{proof}
    Let $V= (\bar{v}_1,\bar{v}_2,\cdots, \bar{v}_k)^T$ be a $k\times d$ matrix. 
    Similarly, we define $W = (w_1,w_2,\cdots, w_k)^T$ be a $k\times d$ matrix. Then, we have $VW^T$ is a $k\times k$ matrix, where each entry is $\inner{\bar{v}_i}{w_j}$. 
    Let $\{a_1, a_2, \dots, a_d\}$ be the $d$ column vectors of matrix $V$. Let $\{b_1,b_2,\dots, b_d\}$ be the $d$ column vectors of $W$.

    We first consider the trace of matrix $VW^T$. Since each diagonal entry of $VW^T$ satisfies $\inner{\bar{v}_i}{w_j} \geq \gamma$, we have 
    $$
    \Tr(VW^T) = \sum_{i=1}^k \inner{\bar{v}_i}{w_i} \geq k \gamma.
    $$
    Since $\bar{v}_i$ and $w_i$ are all unit vectors, we have $\Tr(VW^T) \leq k$.
    We also have 
    $$
    \Tr(VW^T) = \Tr(WV^T) = \sum_{i=1}^d \inner{a_i}{b_i} = \sum_{i=1}^d \|a_i\|\|b_i\|\cos(\theta_i),
    $$
    where $\theta_i$ is the angle between vectors $a_i$ and $b_i$. Hence, we get
    \begin{equation}\label{eq:trace}
    \sum_{i=1}^d \|a_i\|\|b_i\|\cos(\theta_i) = \Tr(VW^T) \geq k\gamma.
    \end{equation}

    We now consider the sum of all entries in matrix $VW^T$. Since all off-diagonal entries of matrix $VW^T$ satisfies $\inner{\bar{v}_i}{w_j} \leq -\gamma$. We use $\one$ to denote all one vector. Then, we have the sum of all entries is
    $$
    \one^T VW^T \one = \Tr(VW^T) + \sum_{i\neq j} \inner{\bar{v}_i}{w_j} \leq \Tr(VW^T) - k(k-1)\gamma.
    $$
    Note that $VW^T = \sum_{i=1}^d a_i b_i^T$. Thus, we have $\one^T VW^T \one = \sum_{i=1}^d \inner{\one}{a_i} \inner{b_i}{\one}$. Let $\alpha_i$ be the angle between $\one$ and $-a_i$ and $\beta_i$ be the angle between $\one$ and $b_i$. Then, we have 
    $$
    - \one^T VW^T \one = \sum_{i=1}^d \inner{\one}{-a_i} \inner{b_i}{\one} = \sum_{i=1}^d \|\one\|^2 \|a_i\|\|b_i\| \cos(\alpha_i) \cos(\beta_i). 
    $$
    Since the angle between $-a_i$ and $b_i$ is $\pi - \theta_i$, we have $\alpha_i + \beta_i \geq \pi - \theta_i$. Since the cosine function is decreasing and log-concave on $[0,\pi/2]$, and negative in $[\pi/2, \pi]$, we have 
    $$
    \cos(\alpha_i) \cos(\beta_i) \leq \cos^2\left(\frac{\pi-\theta_i}{2}\right).
    $$
    By combining the equations above, we have
    \begin{equation}\label{eq:all_sum}
    k \sum_{i=1}^d \|a_i\|\|b_i\| \cos^2\left(\frac{\pi-\theta_i}{2}\right) \geq - \one^T VW^T \one \geq - \Tr(VW^T) + k(k-1)\gamma \geq - k + k(k-1)\gamma,
    \end{equation}
    where the last inequality is due to $\Tr(VW^T) \leq k$.

    By combining Equations~(\ref{eq:trace}) and~(\ref{eq:all_sum}), we have 
    $$
    \sum_{i=1}^d \|a_i\|\|b_i\| = \sum_{i=1}^d \|a_i\|\|b_i\| \left(2\cos^2\left(\frac{\pi-\theta_i}{2}\right) + \cos(\theta_i)\right) \geq 3k\gamma -2 - 2\gamma.
    $$
    By Cauchy-Schwarz inequality, we have 
    $$
    \left(\sum_{i=1}^d \|a_i\|\|b_i\| \right)^2 \leq \sum_{i=1}^d \|a_i\|^2 \cdot \sum_{i=1}^d \|b_i\|^2 = k^2,
    $$
    where the last equality is from $\sum_{i=1}^d \|a_i\|^2 = \sum_{i=1}^d \|b_i\|^2 = k$ since matrices $V$ and $W$ are both consists of $k$ unit row vectors. Therefore, we have 
    $$
    k \leq \frac{2+2\gamma}{3\gamma-1}.
    $$

    For the lower bounds see below.
\end{proof}

\begin{proof}[Construction of skew-obtuse family of vectors]
     Let $\varepsilon = 1/\sqrt{3}$.
    We first consider the case when $\gamma =1/3$.
   The dataset $X$ contains $n$ data points $x_1,\dots,x_n$ in $\bbR^{n+1}$ such that 
    $$
    x_i = \varepsilon \cdot e_1 + \sqrt{1-\varepsilon^2} \cdot e_{i+1},
    $$
    where $e_i$ is the $i$-th standard basis vector. Each data point $x_i$ has a unit length. 

    Consider the following $n$ different label functions $f_1,\dots, f_n$ on this dataset $X$. The function $f_i$ labels the point $x_i$ as positive and all other points as negative. 
    Let $h_i$ be the linear classifier with the weight vector $w_i = -\varepsilon \cdot e_1 + \sqrt{1-\varepsilon^2} \cdot e_{i+1}$. Then, we have $\inner{h_i}{x_i} = 1-2\varepsilon^2  \geq \gamma$ and $\inner{h_i}{x_j} = -\varepsilon^2  \leq -\gamma$ for all $j \neq i$. 
    Thus, each instance $(X,f_i)$ is linear separable by margin $\gamma$ and $h_i$ is a $\gamma$-margin classifier for this instance $(X,f_i)$. Therefore, to distinguish these instances and achieve recall $1$, any protocol must reveal all data points for verification.

    Next we consider the case when $\gamma = 1/3 - \eta$ for some $\eta >0$.
    Let $\beta = (3/2)\eta$.
    It is well-known that by picking unit vectors randomly in $\R^d$ we can construct a collection of vectors $v_1, v_2, \ldots v_k$ such that $\inner{v_i}{v_j} \leq \beta$ if $i\neq j$ and $k\geq \exp{(\Omega(\beta^2 d))}$ (see \Cref{lem:exp_many_neearly_orthogoanl_vectors} below for a proof). 
    WLOG we can assume that $v_1,\ldots,v_k$ are in the span of $e_2,\ldots,e_{d+1}$. Now, for all $i\in [k]$ we set 
    $$
    x_i = \varepsilon \cdot e_1 + \sqrt{1-\varepsilon^2} \cdot v_{i}.
    $$
    Further, by setting $w_i = -\varepsilon \cdot e_1 + \sqrt{1-\varepsilon^2} \cdot v_i$ we have that 
    \[
    \inner{v_i}{w_i} = -\varepsilon^2 + 1 - \varepsilon^2 = 1/3 \geq \gamma
    \]
    and for $i\neq j$
    \[
    \inner{v_i}{w_j} \leq -\varepsilon^2 + (1 - \varepsilon^2)\eta = -1/3 + \eta = -\gamma.
    \]
\end{proof}

\begin{lemma}[Exponentially many nearly orthogonal vectors]
\label{lem:exp_many_neearly_orthogoanl_vectors}
For any $\epsilon > 0$, there exists a collection of $N = \exp(c\epsilon^2d)$ unit vectors in $\mathbb{R}^d$ such that the absolute value of the inner product between any pair is at most $\epsilon$, where $c > 0$ is an absolute constant.
\end{lemma}

\begin{proof}
Let us construct the vectors by selecting $N = \exp(c\epsilon^2d)$ vectors independently and uniformly at random from $\{\pm \frac{1}{\sqrt{d}}\}^d$. We will show that with positive probability, all pairwise inner products are bounded by $\epsilon$ in absolute value.

For any fixed pair of vectors $v_1, v_2$ chosen according to this distribution we an analyze the inner product $\langle v_1, v_2 \rangle$ as the sum of $d$ independent random variables where each term in this sum has mean zero and magnitude $\frac{1}{d}$.
% \begin{itemize}
%     \item Each coordinate of each vector is independently $\pm \frac{1}{\sqrt{d}}$
%     \item The inner product $\langle v_1, v_2 \rangle$ is the sum of $d$ independent random variables
%     \item Each term in this sum has mean zero and magnitude $\frac{1}{d}$
% \end{itemize}

By Hoeffding's inequality, for any fixed pair of vectors:
\[
    \mathbb{P}(|\langle v_1, v_2 \rangle| > \epsilon) \leq 2\exp(-2\epsilon^2d)
\]

By the union bound:
\[
    \mathbb{P}(\exists i,j: |\langle v_i, v_j \rangle| > \epsilon) \leq \binom{N}{2} \cdot 2\exp(-2\epsilon^2d) < N^2\exp(-2\epsilon^2d)
\]

Substituting $N = \exp(c\epsilon^2d)$ where $c < 1$:
\[
    N^2\exp(-2\epsilon^2d) = \exp(2c\epsilon^2d - 2\epsilon^2d) < 1
\]

Therefore, with positive probability, all pairwise inner products are at most $\epsilon$ in absolute value, proving the existence of such a collection.
\end{proof}

\section{Proof of Theorem~\ref{thm:robust_protocol_general}}\label{apx:robust_protocol}

In this section, we prove the guarantees of our robust critical points protocol given below.

\begin{algorithm2e}[htb]
\caption{Robust Critical Points Protocol for Hypothesis Class $\calH$}
\label{alg:robust_abstract_critical_points_protocol}
\DontPrintSemicolon
\LinesNumbered


\KwIn{Subroutine for computing robust critical points (\Cref{alg:robust_critical}), Labeled points from Alice}
\KwOut{A subset of points sent to Bob}

Alice sends all points $X$ to Trent\;
Alice reports to Trent a set $X_A^+ \subseteq X$ as positive ($X_A^- = X \setminus X_A^+$ as negative)\;

% \If{Alice's report $(X_A^+, X_A^-)$ is not separable by a $\gamma$-margin linear classifier}{
%     Trent sends all points $X$ to Bob and the protocol ends\;
% }

Trent computes the critical points $C(X_A^+)$ using Algorithm~\ref{alg:robust_critical} with input $X_A^+$\;
Trent sends points $X_A^+ \cup C(X_A^+)$ to Bob\;
Bob labels the points and sends labels to Trent\;
Trent checks the agreement of reports from Alice and Bob and sends any disputed points to the court to settle\;

\If{the court disagrees with Alice's label on any disputed points}{
    Trent sends all points $X$ to Bob\;
}

\end{algorithm2e}

% \begin{algorithm2e}[htb]
\caption{Robust Critical Points Protocol}
\label{alg:robust_protocol}
\DontPrintSemicolon
\LinesNumbered

\KwIn{Subroutine for computing robust critical points (\Cref{alg:robust_critical}), Labeled points from Alice}
\KwOut{A subset of points sent to Bob}

Alice sends all points $X$ to Trent\;
Alice reports to Trent a set $X_A^+ \subseteq X$ as positive ($X_A^- = X \setminus X_A^+$ as negative)\;

% \If{Alice's report $(X_A^+, X_A^-)$ is not separable by a $\gamma$-margin linear classifier}{
%     Trent sends all points $X$ to Bob and the protocol ends\;
% }

Trent computes the critical points $C(X_A^+)$ using Algorithm~\ref{alg:robust_critical} with input $X_A^+$\;
Trent sends points $X_A^+ \cup C(X_A^+)$ to Bob\;
Bob labels the points and sends labels to Trent\;
Trent checks the agreement of reports from Alice and Bob and sends any disputed points to the court to settle\;

\If{the court disagrees with Alice's label on any disputed points}{
    Trent sends all points $X$ to Bob\;
}

\end{algorithm2e}

\begin{proof}[Proof of \Cref{thm:robust_protocol_general}]
    Let $h^*$ be the best classifier in the hypothesis class $\calH$ on the true labels $(X,f)$. We know that $h^*$ classifies $(X,f)$ with at most $L$ errors.
    Let $X_A^+$ and $X_A^-$ be the set of all positives and the set of all negatives reported by Alice, respectively. Let $f_A$ be this labeling function reported by Alice.
    Without loss of generality, we assume that all points in $X_A^+$ are true positive. 
    (If this assumption is false, Bob will identify any misclassified negative points within $X_A^+$, as these points are always sent to him for verification.)
    Next, define $\calH_{L}(X_A^+)$ be the set of all classifiers in $\calH$ that satisfies the following conditions: (a) the instance $(X,f_A)$ are not classified by this classifier with error at most $L$; and (b)
    there is a labeling $f'$ of $X$ such that all points in $X_A^+$ are labeled as positive, i.e.,  $f'(x) = +1$ for all $x\in X_A^+$ and the labeled dataset $(X,f')$ is classified by this classifier with error at most $L$. 
    Without ambiguity, we use $\calH_L$ to denote this hypothesis set. 
    
    We first show that the critical points $C_{L}(X_A^+)$ computed by Algorithm~\ref{alg:robust_critical} can distinguish two cases: (1) Alice's report $(X,f_A)$ is classified by the best classifier $h^*$ with error at most $L$; and (2) Alice labels some true positive points as negative. 
    Specifically, we show that there exists a true positive point in  $C_{L}(X_A^+)$ if the best classifier $h^*$ is in $\calH_L$. If there is a true positive in $C_{L}(X_A^+)$, then we are in case (2); otherwise, the best classifier $h^*$ is not in $\calH_L$, which means we are in case (1) since the condition (b) of $\calH_L$ is always satisfied by the best classifier $h^*$.
    
    Suppose the best classifier $h^*$ is $h \in \calH_L$. Let $E^- = \{x \in X_A^+: h(x) = -1 \}$ be the points in $X_A^+$ that are classified by $h$ as negative. Since $h \in \calH_L$, by the definition of $\calH_L$, we have $|E^-| \leq L$ since errors in $E^-$ can not be avoided by relabeling $X_A^-$.
    Let $E^+ = \{x \in X_A^-: h(x) = +1\}$ be the points in $X_A^-$ that are classified by $h$ as positive. We must have $|E^+| > L-|E^-| \geq 0$ since if $|E^+| \leq L-|E^-|$, then $(X,f_A)$ is classified by $h$ with error at most $L$, which contradicts $h \in \calH_L$. 
    Consider the last $L-|E^-|+1$ points in $E^+$ visited in Algorithm~\ref{alg:robust_critical}, denoted by $C_h$.
    When Algorithm~\ref{alg:robust_critical} visits these points, by flipping the label of the visited point, the classifier $h$ can classify the new instance with error at most $L$. Thus, all these $L-|E^-|+1$ points $C_h$ are not removed and are contained in critical points $C_{L}(X_A^+)$. 
    Since $h$ is the best classifier, there exists at least a true positive in these $L-|E^-|+1$ points $C_h$, otherwise, $h$ makes more than $L$ errors on the true labels. Thus, this implies there exists a true positive point in $C_{L}(X_A^+)$.

    We now show that the number of robust critical points in $C_{L}(X_A^+)$ is at most the robust Leave-One-Out dimension of the hypothesis class $\calH$. 
    Suppose $C_{L}(X_A^+)$ contains $k$ points $v_1,v_2, \dots, v_k \in X$. 
    We now show that for each $v_i$, there exists a classifier $h_i \in \calH$ satisfies two properties: (1) $v_i$ is classified as positive, i.e. $h(v_i) = +1$; and (2) there are at least $k-1-L$ points in $\{v_i\}$ are classified as negative, $\left|\{j\in[k]\colon  h(v_j) = -1\}\right|\geq k-1-L$.
    Consider the iteration corresponding to the point $v_i$ in Algorithm~\ref{alg:robust_critical}. Let $M_i$ be the set $M$ at the beginning of this iteration. Since $v_i$ is not removed, there is a classifier $h \in \calH$ that classifies $v_i$ as positive and has at most $L$ errors. Note that the critical points $C_L(X_A^+)$ is a subset of $M_i$. This classifier $h$ satisfies two properties for the point $v_i$.
    Therefore, the number of robust critical points is at most the robust Leave-One-Out dimension of the hypothesis class $\calH$.

    Finally, we bound the recall and nonresponsive disclosure. 
    We first show that the recall is at least $1-L/n^+$. 
    If Bob detects any documents misclassified by Alice, then all documents are disclosed to him, which implies a perfect recall, $\mathrm{Recall} = 1$. 
    If Bob does not detect any misclassified document, then by the above analysis, we are in case (1) Alice's report $(X,f_A)$ is classified by the best classifier with error at most $L$.
    Since points in $X_A^+$ are always sent to Bob, Alice can only hide true positive points as negative.
    We now show that Alice can hide at most $L$ true positives.
    Since the true labels are classified by the best classifier with at most $L$ errors, there are at most $L$ true positive points classified as negative by the best classifier.
    If Alice hides more than $L$ true positives as negative, then there exists a true positive point $x$ that is labeled as negative by Alice and is classified as positive by the best classifier.
    When Algorithm~\ref{alg:robust_critical} visits this point $x$, this point is considered a critical point in $C_L(X_A^+)$ since the best classifier satisfies the condition.
    Thus, Alice can hide at most $L$ true positives as negative.
    If Alice labels all documents correctly, then Bob will not detect any misclassified documents. Thus, the nonresponsive disclosure is the number of robust critical points, which is at most the robust Leave-One-Out dimension of the hypothesis class $\calH$. 
    % Let $S \subseteq X_A^-$ be the smallest set that satisfies for any classifier $h \in \calH_\gamma$, if $h$ classifies any point in $X_A^-$ as positive, then there exists a point in $S$ classified as positive by $h$. Suppose $S$ contains $k$ points $v_1,v_2, \dots, v_k \in \bbR^d$. For each $v_i$, there exists a classifier $h_i \in \calH_\gamma$ such that $h_i(v_i) = 1$ and $h_i(v_j) = -1$ for all $j \neq i$. If there is no such $h_i$, then we can remove $v_i$ from $S$ and get a smaller set satisfying the condition. 
    
    %Let $w_i \in \bbR^d$ be the unit-length weight vector of the classifier $h_i(x) = \sgn(\inner{w_i}{x})$. Let $\bar{v}_i = v_i/\|v_i\|_2$. Since $h_i$ has margin at least $\gamma$, we have $\inner{\bar{v}_i}{w_i} \geq \gamma$ and $\inner{\bar{v}_j}{w_i} \leq -\gamma$ for any $j \neq i$. 
    % Therefore by Lemma~\ref{lemma:generalized-skew-obtuse_lemma}, we have the number of points in $C_\gamma(X_A^+)$ is at most 
    % $$
    % k \leq \frac{(2+2L)(\gamma+1)}{3\gamma-1}.
    % $$
\end{proof}


% \begin{proof}[Proof of Theorem~\ref{thm:robust_protocol}]
%     The protocol first asks Alice to report labels of all documents in $X$ to Trent. Then, Trent checks whether the labels reported by Alice can be classified by a linear classifier with margin $\gamma$ and total error $L$. 
%     If there is no such linear classifier, then Trent reveals all documents to Bob. 
    
%     Now suppose the labels reported by Alice can be classified by a linear classifier with margin $\gamma$ and total error $L$. 
%     Let $X_A^+$ and $X_A^-$ be the set of all positives and the set of all negatives reported by Alice, respectively. Let $f_A$ be this labeling function reported by Alice.
%     Without loss of generality, we assume that all points in $X_A^+$ are true positive. 
%     (If this assumption is false, Bob will identify any misclassified negative points within $X_A^+$, as these points are always sent to him for verification.)
%     Next, define $\calH_{\gamma,L}$ be the set of all linear classifiers that satisfies the following conditions: (a) the instance $(X,f_A)$ are not classified by this linear classifier with margin $\gamma$ and total error $L$; and (b)
%     there is a labeling $f'$ of $X$ such that all points in $X_A^+$ are labeled as positive, i.e.,  $f'(x) = +1$ for all $x\in X_A^+$ and the labeled dataset $(X,f')$ is classified with margin $\gamma$ and total error $L$. 
%     Without ambiguity, we use $\calH$ to denote this hypothesis set. 
    
%     We first show that the critical points $C_{\gamma,L}(X_A^+)$ computed by Algorithm~\ref{alg:robust_critical} can distinguish two cases: (1) Alice's report $(X,f_A)$ is classified by the true classifier with margin $\gamma$ and total error $L$; and (2) Alice labels some true positive points as negative. 
%     Specifically, we show that there exists a true positive point in  $C_{\gamma,L}(X_A^+)$ if the true classifier is in $\calH$. If there is a true positive in $C_{\gamma,L}(X_A^+)$, then we are in case (2); otherwise, the true classifier is not in $\calH$, which means we are in case (1) since the condition (b) of $\calH$ is always satisfied by the true classifier.
    
%     Suppose the true classifier is $h \in \calH$. Let $w$ be the weight vector of this classifier $h$. We denote the margin error points of this classifier $h$ by $E_{\gamma} = \{x \in X: |\inner{\bar{x}}{w}| < \gamma \}$, where $\bar{x} = x/\|x\|$ is the normalized vector for $x$. Note that all margin errors do not depend on the labels of points. Let $E_{\gamma}^- = \{x \in X_A^+: \inner{\bar{x}}{w} \leq -\gamma \}$ be the points in $X_A^+$ that are classified by $h$ as negative and satisfy the margin condition. Since $h \in \calH$, by the definition of $\calH$, we have $|E_{\gamma} \cup E_{\gamma}^-| \leq L$ since errors in $E_{\gamma} \cup E_{\gamma}^-$ can not be avoided by relabeling $X_A^-$.
%     Let $E_{\gamma}^+ = \{x \in X_A^-: \inner{\bar{x}}{w} \geq \gamma\}$ be the points in $X_A^-$ that are classified by $h$ as positive and satisfy the margin condition. We must have $|E_{\gamma}^+| > L-|E_{\gamma} \cup E_{\gamma}^-| \geq 0$ since if $|E_{\gamma}^+| \leq L-|E_{\gamma} \cup E_{\gamma}^-|$, then $(X,f_A)$ is classified by $h$ with margin $\gamma$ and error $L$, which contradicts $h \in \calH$. 
%     Consider the last $L-|E_{\gamma} \cup E_{\gamma}^-|+1$ points in $E_{\gamma}^+$ visited in Algorithm~\ref{alg:robust_critical}, denoted by $C_h$.
%     When Algorithm~\ref{alg:robust_critical} visits these points, by flipping the label of the visited point, the classifier $h$ can classify the new instance with margin $\gamma$ and error $L$. Thus, all these $L-|E_{\gamma} \cup E_{\gamma}^-|+1$ points $C_h$ are not removed and are contained in critical points $C_{\gamma,L}(X_A^+)$. 
%     Since $h$ is the true classifier, there exists at least a true positive in these $L-|E_{\gamma} \cup E_{\gamma}^-|+1$ points $C_h$, which implies there exists a true positive point in $C_{\gamma,L}(X_A^+)$.

%     We now show that critical points in $C_{\gamma,L}(X_A^+)$ have a special structure, which forms a robust generalized skew-obtuse family of vectors. Suppose $C_{\gamma,L}(X_A^+)$ contains $k$ points $v_1,v_2, \dots, v_k \in X$. 
%     Let $\bar{v}_i = v_i / \|v_i\|$ be the normalized vector of $v_i$.
%     We now show that for each $v_i$, there exists a linear classifier $h_i$ with unit-length weight vector 
%     $w_i$ satisfies two properties: (1) $v_i$ is classified as positive and satisfies the margin condition, i.e. $\inner{\bar{v}_i}{w_i} \geq \gamma$; and (2) there are at least $k-1-L$ points in $\{v_i\}$ are classified as negative and satisfy the margin condition, $\left|j\in[k]\colon  \inner{\bar{v}_j}{w_i} \leq -\gamma\right|\geq k-1-L$.
%     Consider the iteration corresponding to the point $v_i$ in Algorithm~\ref{alg:abstract_critical_points_computation}. Let $M_i$ be the set $M$ at the beginning of this iteration. Since $v_i$ is not removed, there is a linear classifier $h$ that classifies $v_i$ as positive and has margin $\gamma$ and error $L$. Note that the critical points $C_\gamma(X_A^+)$ is a subset of $M_i$. This classifier $h$ satisfies two properties for the point $v_i$.
    
%     % Let $S \subseteq X_A^-$ be the smallest set that satisfies for any classifier $h \in \calH_\gamma$, if $h$ classifies any point in $X_A^-$ as positive, then there exists a point in $S$ classified as positive by $h$. Suppose $S$ contains $k$ points $v_1,v_2, \dots, v_k \in \bbR^d$. For each $v_i$, there exists a classifier $h_i \in \calH_\gamma$ such that $h_i(v_i) = 1$ and $h_i(v_j) = -1$ for all $j \neq i$. If there is no such $h_i$, then we can remove $v_i$ from $S$ and get a smaller set satisfying the condition. 
    
%     %Let $w_i \in \bbR^d$ be the unit-length weight vector of the classifier $h_i(x) = \sgn(\inner{w_i}{x})$. Let $\bar{v}_i = v_i/\|v_i\|_2$. Since $h_i$ has margin at least $\gamma$, we have $\inner{\bar{v}_i}{w_i} \geq \gamma$ and $\inner{\bar{v}_j}{w_i} \leq -\gamma$ for any $j \neq i$. 

    
%     Therefore by Lemma~\ref{lemma:generalized-skew-obtuse_lemma}, we have the number of points in $C_\gamma(X_A^+)$ is at most 
%     $$
%     k \leq \frac{(2+2L)(\gamma+1)}{3\gamma-1}.
%     $$
% \end{proof}
\section{Proof of Lemma~\ref{lemma:generalized-skew-obtuse_lemma}}\label{apx:proof_general_skew_obtuse}

In this section, we prove the size of the robust generalized skew-obtuse family of vectors. 

% \begin{lemma}[Robust Generalized Skew-Obtuse Family of Vectors]
% \label{lemma:generalized-skew-obtuse_appendix}
%     Let $V= \bar{v}_1,\bar{v}_2,\cdots, \bar{v}_k$ be a $k$ unit vectors in $\R^d$. Further, let  $W = w_1,w_2,\cdots, w_k$ be $k$ other unit vectors in $\R^d$. Suppose there exists $\alpha,\beta\in [0,1]$ and an error parameter $L$ such that the following holds:
%     \begin{enumerate}
%         \item $\inner{\bar{v}_i}{w_i} \geq \alpha$ and
%         \item for all $j\in [k]$ we have $\left|i\in[k]\colon  \inner{\bar{v}_i}{w_j} \leq -\beta\right|\geq k-1-L$. 
%     \end{enumerate}
%      Then, if $\alpha+2\beta>1$ we have $ k \leq \frac{(2+2L)(1+\beta)}{\alpha+2\beta-1}.$
%      % Further if $\gamma=1/3$, there exists a skew-obtuse family of vectors with $k\geq \Omega(d)$ and if $\gamma<1/3$ then with $k\geq \exp{(\Omega_{1/3-\gamma}(d))}$.
% \end{lemma}

\begin{proof}[Proof of Lemma~\ref{lemma:generalized-skew-obtuse_lemma}]
    Let $V= (\bar{v}_1,\bar{v}_2,\cdots, \bar{v}_k)^T$ be a $k\times d$ matrix. 
    Similarly, we define $W = (w_1,w_2,\cdots, w_k)^T$ be a $k\times d$ matrix. Then, we have $VW^T$ is a $k\times k$ matrix, where each entry is $\inner{\bar{v}_i}{w_j}$. 
    Let $\{a_1, a_2, \dots, a_d\}$ be the $d$ column vectors of matrix $V$. Let $\{b_1,b_2,\dots, b_d\}$ be the $d$ column vectors of $W$.

    We first consider the trace of matrix $VW^T$. Since each diagonal entry of $VW^T$ satisfies $\inner{\bar{v}_i}{w_j} \geq \alpha$, we have 
    $$
    \Tr(VW^T) = \sum_{i=1}^k \inner{\bar{v}_i}{w_i} \geq k \alpha.
    $$
    Since $\bar{v}_i$ and $w_i$ are all unit vectors, we have $\Tr(VW^T) \leq k$.
    We also have 
    $$
    \Tr(VW^T) = \Tr(WV^T) = \sum_{i=1}^d \inner{a_i}{b_i} = \sum_{i=1}^d \|a_i\|\|b_i\|\cos(\theta_i),
    $$
    where $\theta_i$ is the angle between vectors $a_i$ and $b_i$. Hence, we get
    \begin{equation}\label{eq:robust-trace}
    \sum_{i=1}^d \|a_i\|\|b_i\|\cos(\theta_i) = \Tr(VW^T) \geq k\alpha.
    \end{equation}

    We now consider the sum of all entries in matrix $VW^T$. 
    For each column $i \in [k]$ of the matrix $VW^T$, we know that there are at least $k-1-L$ off-diagonal entries with value at most $\inner{\bar{v}_j}{w_i} \leq -\beta$.
    For other off-diagonal entries in that column, we upper bound them by one since $\bar{v}_j$ and $w_i$ are unit vectors.
    We use $\one$ to denote all one vector. Then, we have the sum of all entries is
    $$
    \one^T VW^T \one = \Tr(VW^T) + \sum_{i\neq j} \inner{\bar{v}_j}{w_i} \leq \Tr(VW^T) - k(k-1 -L)\beta + kL.
    $$
    Note that $VW^T = \sum_{i=1}^d a_i b_i^T$. Thus, we have $\one^T VW^T \one = \sum_{i=1}^d \inner{\one}{a_i} \inner{b_i}{\one}$. Let $\phi_i$ be the angle between $\one$ and $-a_i$ and $\psi_i$ be the angle between $\one$ and $b_i$. Then, we have 
    $$
    - \one^T VW^T \one = \sum_{i=1}^d \inner{\one}{-a_i} \inner{b_i}{\one} = \sum_{i=1}^d \|\one\|^2 \|a_i\|\|b_i\| \cos(\phi_i) \cos(\psi_i). 
    $$
    Since the angle between $-a_i$ and $b_i$ is $\pi - \theta_i$, we have $\phi_i + \psi_i \geq \pi - \theta_i$. Since the cosine function is log-concave on $[0,\pi]$, we have 
    $$
    \cos(\phi_i) \cos(\psi_i) \leq \cos^2\left(\frac{\pi-\theta_i}{2}\right).
    $$
    By combining the equations above, we have
    \begin{align*}
    k \sum_{i=1}^d \|a_i\|\|b_i\| \cos^2\left(\frac{\pi-\theta_i}{2}\right) \geq - \one^T VW^T \one \geq - \Tr(VW^T) + k(k-1-L)\beta - kL.
    \end{align*}
    Since $\Tr(VW^T) \leq k$, we have
    \begin{equation}\label{eq:robust-all-sum}
        k \sum_{i=1}^d \|a_i\|\|b_i\| \cos^2\left(\frac{\pi-\theta_i}{2}\right) \geq -k+k(k-1-L)\beta - kL.
    \end{equation}

    By combining Equations~(\ref{eq:robust-trace}) and~(\ref{eq:robust-all-sum}), we have 
    $$
    \sum_{i=1}^d \|a_i\|\|b_i\| = \sum_{i=1}^d \|a_i\|\|b_i\| \left(2\cos^2\left(\frac{\pi-\theta_i}{2}\right) + \cos(\theta_i)\right) \geq k\alpha+(2k-2-2L)\beta -2 - 2L.
    $$
    By Cauchy-Schwarz inequality, we have 
    $$
    \left(\sum_{i=1}^d \|a_i\|\|b_i\| \right)^2 \leq \sum_{i=1}^d \|a_i\|^2 \cdot \sum_{i=1}^d \|b_i\|^2 = k^2,
    $$
    where the last equality is from $\sum_{i=1}^d \|a_i\|^2 = \sum_{i=1}^d \|b_i\|^2 = k$ since matrices $V$ and $W$ are both consists of $k$ unit row vectors. Therefore, we have 
    $$
    k \leq \frac{(2+2L)(\beta+1)}{\alpha + 2\beta-1}.
    $$

    % For the lower bounds see Lemma~\ref{lemma:constructing_skob_family}.
\end{proof}



%
\subsection{Possible New Lemma}

\begin{lemma}[New Robust Generalized Skew-Obtuse Family of Vectors]
\label{lemma:new_generalized-skew-obtuse_lemma}
    Let $V= \bar{v}_1,\bar{v}_2,\cdots, \bar{v}_k$ be a $k$ unit vectors in $\R^d$. Further, let  $W = w_1,w_2,\cdots, w_k$ be $k$ other unit vectors in $\R^d$. Suppose there exists $\alpha,\beta\in [0,1]$ and an error parameter $L$ such that the following holds:
    \begin{enumerate}
        \item for an $i$ chosen uniformly from $[k]$ we have $\E[\inner{\bar{v}_i}{w_i}] \geq \alpha$ and
        % \item for all $j\in [k]$ we have $\left|i\in[k]\colon  \inner{\bar{v}_i}{w_j} \leq -\beta\right|\geq k-1-L$. 
        \item there is a set $G$ of pairs $(i,j)$ with $i\neq j$ of size at least $k(k-1-L)$ such that the when picking $i,j$ uniformly from $G$ we have $\E[\inner{\bar{v}_i}{w_j}] \leq -\beta$
    \end{enumerate}
     Then, if $\alpha+2\beta>1$ we have $ k \leq \frac{(2+2L)(1+\beta)}{\alpha+2\beta-1}.$
     % Further if $\gamma=1/3$, there exists a skew-obtuse family of vectors with $k\geq \Omega(d)$ and if $\gamma<1/3$ then with $k\geq \exp{(\Omega_{1/3-\gamma}(d))}$.
\end{lemma}


\end{document}
