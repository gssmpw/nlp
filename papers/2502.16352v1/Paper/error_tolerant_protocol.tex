\section{Error-Tolerant Protocol}

In this section, we provide protocols that are tolerant of classification errors made by Alice. For protocols proposed in previous sections, if Bob detects one document misclassified by Alice, the court or Trent reveals all documents to Bob. 
However, \citet*{grossman2010technology,grossman2012inconsistent} show that, in practice, even the most skilled human reviewers can unintentionally make classification errors.
Therefore, we aim to design protocols that are less strict on Alice's classification, imposing a gradual rather than harsh penalty for Alice's misclassification errors.
We show that our previous protocols can be converted to the following error-tolerant protocols.

\begin{theorem}
\label{thm:error_tolerant_protocol}
    Suppose there exists a multi-party classification protocol that satisfies if Bob does not detect any document misclassified by Alice, then the recall is at least $\alpha$ and the nonresponsive disclosure is at most $k$.
    Then, there exists an error-tolerant protocol that guarantees:
    \begin{enumerate}
        \item \textbf{(Recall)} The recall is at least $\alpha$;
        \item \textbf{(Nonresponsive Disclosure)} The nonresponsive disclosure is at most $k\cdot E$ if the protocol detects $E$ documents misclassified by Alice.
    \end{enumerate}
\end{theorem}

\begin{proof}
    The error-tolerant protocol is constructed as follows. Let $P$ be a multi-party classification protocol that ensures a recall of at least $\alpha$ and discloses at most $k$ nonresponsive documents when Bob does not detect any document misclassified by Alice. (This is not the number of errors made by the optimal classifier.)
    The error-tolerant protocol iteratively calls protocol $P$ until an iteration of $P$ completes without any detected misclassification by Alice.
    
    If Bob identifies a document misclassified by Alice in any call of $P$, Alice is required to relabel the document correctly. 
    Since each detected misclassification triggers at most $k$ additional nonresponsive disclosures per iteration, the total nonresponsive disclosure remains bounded by $k \cdot E$, where $E$ is the number of errors detected.
\end{proof}

\begin{algorithm2e}[htb]
\caption{Computing Critical Points with Verified Points}
\label{alg:critical_negative}
\DontPrintSemicolon
\LinesNumbered

\KwIn{A set of $n$ points $X = \{x_1,x_2,\dots, x_n\} \subset \mathbb{R}^d$, a set $A^- \subseteq X$ of verified negative points, a set $X_A^+ \subseteq X$ of positive points reported by Alice, and an oracle $\calO$ for checking realizability within hypothesis class $H$.}
\KwOut{A set of critical points $C(X_A^+, A^-) \subseteq X_A^-$, where $X_A^- = X \setminus X_A^+$.}

Set $M = X_A^-$\;
\For{each $x_i \in X_A^- \setminus A^-$}{
    Set $T_1 = X_A^+ \cup \{x_i\}$ and $T_2 = M \setminus \{x_i\}$\;
    \If{labeling $T_1$ as $+$ and $T_2$ as $-$ is an {invalid} labeling under $\calH$ according to $\calO$}{
        Remove $x_i$ from $M$, ie, $M = M \setminus \{x_i\}$\;
    }
}
Set the critical points $C(X_A^+,A^-) = M$\;

\end{algorithm2e}


Note that the error-tolerant protocol framework shown in Theorem~\ref{thm:error_tolerant_protocol} may call the one-round multi-party protocol multiple times. 
If Bob detects misclassified documents, then in later calls, instead of certified positive documents reported by Alice, there may also be some negative documents verified by Bob in previous calls. 
Thus, we can modify the algorithms for computing the critical points (Algorithm~\ref{alg:abstract_critical_points_computation}) and robust critical points (Algorithm~\ref{alg:robust_critical}) to achieve fewer nonresponsive disclosure by taking those certified negative documents into account. 
We show the modified algorithm for computing the critical points in the realizable setting in Algorithm~\ref{alg:critical_negative}.
Algorithm~\ref{alg:robust_critical} for the robust critical points in the nonrealizable setting can be extended similarly.
In practice, using these modified algorithms to compute the critical points in the error-tolerant protocol can further reduce the nonresponsive disclosure while preserving the recall guarantee.