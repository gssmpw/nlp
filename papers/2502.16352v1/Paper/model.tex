\subsection{Model}

% We now describe the set-up for verifying a classification in a multi-party setting, introduced by~\cite{dong2022classification}.
% We now outline the framework for verifying classification in a multi-party setting, as introduced by~\cite{dong2022classification}.
% In this problem, the defendant (Alice) possesses a set of documents that are embedded in a $d$-dimensional space, $X \subset \R^d$. The plaintiff (Bob) has issued a request for production to Alice to require documents responsive to this request. Let $f: X \to \{-1,+1\}$ be the true labeling function such that each document $x\in X$ is responsive if its label is $f(x) = 1$; otherwise, it is nonresponsive. 
% Then, the set of documents with true labels $(X,f)$ is called an instance. 
We now outline the framework for verifying classification in a multi-party setting, as introduced by~\cite{dong2022classification}.
In this setting, the defendant (Alice) possesses a set of documents embedded in a $d$-dimensional space, $X \subset \mathbb{R}^d$. The plaintiff (Bob) issues a request for production, seeking documents responsive to his request. Let $f: X \to \{-1,+1\}$ denote the true labeling function, where a document $x \in X$ is considered responsive if $f(x) = 1$ and nonresponsive otherwise.
The pair $(X, f)$, consisting of the document set and its true labels, is referred to as an instance.

% A multi-party protocol for \sid{verifying classification} consists of three parties, Alice, Bob, and a trusted third party, Trent. Trent can be served by the court system, or any cloud computing service provider. The protocol involves multiple rounds of interaction between these three parties. Trent has access to all documents and can implement computation on them, but can not label these documents. 
% Instead, Trent will require Alice and Bob to review and label specific documents as the protocol unfolds.
% \todosid{Whenever necessary the court can be asked to adjudicate on the true label of a document. However, this is treated as an expensive operation due to the time and cost required in legal proceedings: hence, this should be used sparingly.}
% Eventually, a subset of documents $B \subseteq X$ is disclosed to Bob at the end of the protocol.
% \todosid{Added a  line about using the court to adjudicate on the true label of the document should be mentioned.}
A multi-party protocol for verifying classification involves three parties: Alice, Bob, and a trusted third party, Trent. Trent may be represented by the court system or a cloud computing service provider.\footnote{For further details about this assumption please see \cite*{dong2022classification}[Introduction].}  
Alice and Bob have the ability to label documents but only Alice holds $X$.
The protocol unfolds through multiple rounds of interaction among these parties.
While Trent has access to all documents and can perform computations on them, they cannot assign labels.
Instead, Trent facilitates the process by requiring Alice and Bob to review and label specific documents.
When necessary, the court may be called upon to adjudicate the true label of a document. However, this is considered an expensive operation due to the time and cost associated with legal proceedings and should therefore be used sparingly.
Furthermore, if the true label is nonresponsive this would also count as a nonresponsive disclosure, which hurts Alice's privacy.
At the conclusion of the protocol, a subset of documents, $B \subseteq X$, is disclosed to Bob.
% Through the execution of the protocol Alice and Bob may deviate maliciouly in theri best interest.
Throughout the execution of the protocol, Alice and Bob may deviate maliciously in their best interest.



We evaluate the performance of a protocol using the following metrics as in \citet*{dong2024error}. 
% Let $n = |X|$, $n^+ = |\{x\in X: f(x) = 1\}|$ and $n^- = |\{x\in X: f(x) = -1\}|$ denote the number of documents, responsive documents, and nonresponsive documents, respectively. The efficacy of a protocol is measured by \emph{Recall}, defined as the fraction of responsive documents retrieved and disclosed to Bob, i.e. 
% $$\mathrm{Recall} = \frac{|\{x\in B : f(x) = 1\}|}{n^+}.$$
% The privacy loss of the defendant in a protocol is measured by \emph{Nonresponsive Disclosure}, which is the number of nonresponsive documents revealed to Bob, which includes those documents on which the court adjudicates, when Alice reports her labels to the protocol correctly, i.e. 
% $$\mathrm{Nonresponsive\ Disclosure} = |\{x\in B: f(x) = -1\}|.$$
% \todosid{More precise defintion?}
%We consider the number of nonresponsive documents disclosed rather than their fraction, as each nonresponsive document contains sensitive private information regardless of the total number of such documents. 
% \begin{definition}[Recall and Nonresponsive Disclosure]
% Let $(X,f)$ be an instance and let $n^+ = |\{x \in X : f(x) = 1\}|$ denote the number of responsive documents. 

% The \emph{Recall} of a protocol is the fraction of responsive documents retrieved and disclosed to Bob in the worst-case (over responses by Alice), when Bob acts honestly:
% \[
% \mathrm{Recall} = \frac{|\{x \in B : f(x) = 1\}|}{n^+}.
% \]

% The \emph{Nonresponsive Disclosure} measures the privacy loss of Alice in the worst-case (over responses by Bob) and is defined as the number of nonresponsive documents revealed to Bob, including those adjudicated by the court, when Alice reports her labels to the protocol correctly:
% \[
% \mathrm{Nonresponsive\ Disclosure} = |\{x \in B : f(x) = -1\}|.
% \]
% \end{definition}
\begin{definition}[Recall and Nonresponsive Disclosure]
Let $(X,f)$ be an instance, and let $n^+ = |\{x \in X : f(x) = 1\}|$ denote the number of responsive documents.

The \emph{Recall} of a protocol is the worst-case fraction of responsive documents retrieved and disclosed to Bob (over Alice's responses), assuming Bob acts faithfully:
\[
\mathrm{Recall} = \frac{|\{x \in B : f(x) = 1\}|}{n^+}.
\]

The \emph{Nonresponsive Disclosure} quantifies Alice’s privacy loss in the worst case (over Bob’s responses). It is defined as the number of nonresponsive documents revealed to Bob, including those adjudicated by the court, assuming Alice reports her labels correctly:
\[
\mathrm{Nonresponsive\ Disclosure} = |\{x \in B : f(x) = -1\}|.
\]
\end{definition}

% \todosid{More precise defintion?}
Then, our goal is to design a protocol that achieves a Recall as close to $1$ as possible while minimizing the Nonresponsive Disclosure.
% \todosid{Here, there should be some measure of how many documents the court is asked to label? Because if the court lables everything then recall is $1$ and NRD is $0$.}

% We now introduce the \emph{Leave-One-Out} dimension of a family of classifiers.
% A set system consists of a set $X$ and a collection $\calS$ of subsets of $X$. 
% We first define the following Leave-One-Out dimension of a set system. 

% \begin{definition}[Leave-One-Out dimension]
% \label{defn:leave_one_out}
%      Given a set system $(X,\calS)$, the \emph{Leave-One-Out} dimension of $S$ is the cardinality of the largest set $C\subseteq X$ such that for each element $c \in C$, there exists $S \in \calS$ with $S \cap C = \{c\}$.
% \end{definition}

% Consider a class of binary classifiers on $X$, denoted by $\calF = \{f:X \to \{-1,+1\}\}$. We define the associated set family as $\calS = \{f^+ : f \in \calF\}$ where $f^+ = \{x: f(x) = 1\}$ is the set of elements classified as positive by $f$. The \emph{Leave-One-Out} dimension of this classifier class $\calF$ is defined as the Leave-One-Out dimension of the set family $\calS$. This means there exists a set $C$ with the size of the Leave-One-Out dimension of $\calF$ such that for each element $c \in C$, a classifier $f \in \calF$ classifies this element as positive and all other elements $C \setminus \{c\}$ as negative.   

% Given a class $\calF$ of binary classifiers on $X$, an instance $(X,f)$ is realizable by this class if there exists a classifier $h \in \calF$ that perfectly classifies this instance, i.e. $h(x) = f(x)$ for any $x\in X$. Otherwise, this instance is called nonrealizable.

%An instance $(X,f)$ is linearly separable if there exists a linear classifier that perfectly classifies these documents. 
% For a linear classifier, we define the margin of a 
% We consider the realizable instance $(X,f)$ which is linear separable by a large margin classifier. 
% \begin{definition}
%     An instance $(X,f)$ where $X \subset \R^d$ is linearly separable by margin $\gamma$ if there exists a unit vector $w \in \bbR^d$ such that for any $x \in X$
%     $$
%     \frac{\inner{x}{w} \cdot f(x)}{\|x\|_2}  \geq \gamma.
%     $$
% \end{definition}



