%\subsection{Skew-Obtuse Family of Vectors}\label{sec:skew-obtuse family}

% We now bound the Leave-One-Out dimension of linear classifiers with a large margin $\gamma > 1/3$. Our analysis will rely on \Cref{lemma:skew-obtuse_lemma} which is about the geometry of a skew-obtuse family of vectors which might be of independent interest with applications to coding theory/combinatorial geoemtry, as similar statment about usual obtuse vector families are used to prove distance bounds.
% Additionally, we belive that the proof is more nuances that the usual idea of looking at the Gram Matrix of the vector family.

% We now bound the {Leave-One-Out dimension} of linear classifiers with a large margin $\gamma > 1/3$. 
% The proof of \Cref{thm:Leave-One-Out-linear-realizable} follow directly from \Cref{lemma:skew-obtuse_lemma} by thinking of $V,W$ appearing in the lemma as $C$ (the set witnessing the Leave-One-Our dimension) and the corresponding hypothesis $h_c$ respectively.
The proof of \Cref{thm:Leave-One-Out-linear-realizable} follows directly from \Cref{lemma:skew-obtuse_lemma} by interpreting $V$ and $W$ in the lemma as $C$ (the set witnessing the \emph{Leave-One-Out dimension}) and the corresponding hypothesis $h_c$, respectively.
The lemma concerns the geometry of a \emph{skew-obtuse family of vectors}, a result that may be of \emph{independent interest} with applications to \emph{coding theory} and \emph{combinatorial geometry}. Similar statements about standard obtuse vector families are used to establish distance bounds.  
Additionally, we believe that the proof is more {nuanced} than the usual approach of analyzing the \emph{Gram matrix} of the vector family.
The proof is detailed in \Cref{apx:constructing_skob_family}


\begin{lemma}[Skew-Obtuse Family of Vectors]
\label{lemma:skew-obtuse_lemma}
    Let $V= \bar{v}_1,\bar{v}_2,\cdots, \bar{v}_k$ be a $k$ unit vectors in $\R^d$. Further, let  $W = w_1,w_2,\cdots, w_k$ be $k$ other unit vectors in $\R^d$. Suppose there exists a $\gamma\in [0,1]$ such that the following holds:
    \begin{enumerate}
        \item $\inner{\bar{v}_i}{w_i} \geq \gamma$ and
        \item $\inner{\bar{v}_i}{w_j} \leq -\gamma$ if $i\neq j$.
    \end{enumerate}
     Then, if $\gamma>1/3$ we have $k \leq \frac{2+2\gamma}{3\gamma-1}.$ Further if $\gamma=1/3$, there exists a skew-obtuse family of vectors with $k\geq \Omega(d)$ and if $\gamma<1/3$ then there exists such a family with $k\geq \exp{(\Omega(({1/3-\gamma})^2d))}$.
\end{lemma}

% \begin{proof}
%     Let $V= (\bar{v}_1,\bar{v}_2,\cdots, \bar{v}_k)^T$ be a $k\times d$ matrix. 
%     Similarly, we define $W = (w_1,w_2,\cdots, w_k)^T$ be a $k\times d$ matrix. Then, we have $VW^T$ is a $k\times k$ matrix, where each entry is $\inner{\bar{v}_i}{w_j}$. 
%     Let $\{a_1, a_2, \dots, a_d\}$ be the $d$ column vectors of matrix $V$. Let $\{b_1,b_2,\dots, b_d\}$ be the $d$ column vectors of $W$.

%     We first consider the trace of matrix $VW^T$. Since each diagonal entry of $VW^T$ satisfies $\inner{\bar{v}_i}{w_j} \geq \gamma$, we have 
%     $$
%     \Tr(VW^T) = \sum_{i=1}^k \inner{\bar{v}_i}{w_i} \geq k \gamma.
%     $$
%     Since $\bar{v}_i$ and $w_i$ are all unit vectors, we have $\Tr(VW^T) \leq k$.
%     We also have 
%     $$
%     \Tr(VW^T) = \Tr(WV^T) = \sum_{i=1}^d \inner{a_i}{b_i} = \sum_{i=1}^d \|a_i\|\|b_i\|\cos(\theta_i),
%     $$
%     where $\theta_i$ is the angle between vectors $a_i$ and $b_i$. Hence, we get
%     \begin{equation}\label{eq:trace}
%     \sum_{i=1}^d \|a_i\|\|b_i\|\cos(\theta_i) = \Tr(VW^T) \geq k\gamma.
%     \end{equation}

%     We now consider the sum of all entries in matrix $VW^T$. Since all off-diagonal entries of matrix $VW^T$ satisfies $\inner{\bar{v}_i}{w_j} \leq -\gamma$. We use $\one$ to denote all one vector. Then, we have the sum of all entries is
%     $$
%     \one^T VW^T \one = \Tr(VW^T) + \sum_{i\neq j} \inner{\bar{v}_i}{w_j} \leq \Tr(VW^T) - k(k-1)\gamma.
%     $$
%     Note that $VW^T = \sum_{i=1}^d a_i b_i^T$. Thus, we have $\one^T VW^T \one = \sum_{i=1}^d \inner{\one}{a_i} \inner{b_i}{\one}$. Let $\alpha_i$ be the angle between $\one$ and $-a_i$ and $\beta_i$ be the angle between $\one$ and $b_i$. Then, we have 
%     $$
%     - \one^T VW^T \one = \sum_{i=1}^d \inner{\one}{-a_i} \inner{b_i}{\one} = \sum_{i=1}^d \|\one\|^2 \|a_i\|\|b_i\| \cos(\alpha_i) \cos(\beta_i). 
%     $$
%     Since the angle between $-a_i$ and $b_i$ is $\pi - \theta_i$, we have $\alpha_i + \beta_i \geq \pi - \theta_i$. Since the cosine function is decreasing and log-concave on $[0,\pi/2]$, and negative in $[\pi/2, \pi]$, we have 
%     $$
%     \cos(\alpha_i) \cos(\beta_i) \leq \cos^2\left(\frac{\pi-\theta_i}{2}\right).
%     $$
%     By combining the equations above, we have
%     \begin{equation}\label{eq:all_sum}
%     k \sum_{i=1}^d \|a_i\|\|b_i\| \cos^2\left(\frac{\pi-\theta_i}{2}\right) \geq - \one^T VW^T \one \geq - \Tr(VW^T) + k(k-1)\gamma \geq - k + k(k-1)\gamma,
%     \end{equation}
%     where the last inequality is due to $\Tr(VW^T) \leq k$.

%     By combining Equations~(\ref{eq:trace}) and~(\ref{eq:all_sum}), we have 
%     $$
%     \sum_{i=1}^d \|a_i\|\|b_i\| = \sum_{i=1}^d \|a_i\|\|b_i\| \left(2\cos^2\left(\frac{\pi-\theta_i}{2}\right) + \cos(\theta_i)\right) \geq 3k\gamma -2 - 2\gamma.
%     $$
%     By Cauchy-Schwarz inequality, we have 
%     $$
%     \left(\sum_{i=1}^d \|a_i\|\|b_i\| \right)^2 \leq \sum_{i=1}^d \|a_i\|^2 \cdot \sum_{i=1}^d \|b_i\|^2 = k^2,
%     $$
%     where the last equality is from $\sum_{i=1}^d \|a_i\|^2 = \sum_{i=1}^d \|b_i\|^2 = k$ since matrices $V$ and $W$ are both consists of $k$ unit row vectors. Therefore, we have 
%     $$
%     k \leq \frac{2+2\gamma}{3\gamma-1}.
%     $$

%     For the lower bounds see Appendix~\ref{apx:constructing_skob_family}.
% \end{proof}