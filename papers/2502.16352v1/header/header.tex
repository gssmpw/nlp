\usepackage[utf8]{inputenc}

%% full size page
\usepackage{fullpage}
\usepackage[margin = 2.5cm]{geometry}

%% AMS packages
\usepackage{amsmath, amsthm, amssymb}
\usepackage{thmtools}
\usepackage{thm-restate}
\usepackage{mathtools}  
\usepackage{xfrac} 
%\usepackage[normalem]{ulem}

\usepackage{placeins} 
\usepackage{times}

%%bibliography
% \usepackage{amsthm}
\usepackage{url}
\usepackage{array}


%% algorithm
%\usepackage{algorithm}
%\usepackage{algpseudocode}
\usepackage[ruled,vlined]{algorithm2e}
% Create a new environment named "algorithm2e" that behaves like "algorithm"
\newenvironment{algorithm2e}[1][]{%
    \begin{algorithm}[#1]%
}{%
    \end{algorithm}
}

%% graphics and colors
\usepackage{graphicx}
\usepackage{color}
%\usepackage[dvipsnames]{xcolor}

%%bibliography
\usepackage[round]{natbib}
\usepackage[hyperindex,breaklinks]{hyperref}
\usepackage{url}

%%color box figure
\usepackage{tcolorbox}

%%draw figure
\usepackage{tikz}

%%misc
\usepackage{nicefrac}

%\citestyle{acmauthoryear}

% \declaretheorem[name=Theorem]{theorem}

%% define theorems, lemmas, claims
\newtheorem{theorem}{Theorem}[section]
\newtheorem{claim}[theorem]{Claim}
\newtheorem{corollary}[theorem]{Corollary}
\newtheorem{proposition}[theorem]{Proposition}
\newtheorem{lemma}[theorem]{Lemma}
\newtheorem{definition}[theorem]{Definition}
\newtheorem{observation}[theorem]{Observation}
\newtheorem{question}[theorem]{Question}
\newtheorem{assumption}[theorem]{Assumption}
\newtheorem*{remark*}{Remark}


% \newenvironment{numberedtheorem}[1]{%
% \renewcommand{\thetheorem}{#1}%
% \begin{theorem}}{\end{theorem}\addtocounter{theorem}{-1}}

% \newenvironment{numberedlemma}[1]{%
% \renewcommand{\thetheorem}{#1}%
% \begin{lemma}}{\end{lemma}\addtocounter{theorem}{-1}}

% \newenvironment{oneshot}[1]{\@begintheorem{#1}{\unskip}}{\@endtheorem}

% \makeatletter
% \newtheorem*{rep@theorem}{\rep@title}
% \newcommand{\newreptheorem}[2]{%
% \newenvironment{rep#1}[1]{%
%  \def\rep@title{#2 \ref{##1}}%
%  \begin{rep@theorem}}%
%  {\end{rep@theorem}}}
% \makeatother


%cleveref package loaded at the end
\usepackage{cleveref}
\crefname{theorem}{theorem}{theorems}
\Crefname{theorem}{Theorem}{Theorems}

\crefname{lemma}{lemma}{lemmas}
\Crefname{lemma}{Lemma}{Lemmas}

\crefname{claim}{claim}{claims}
\Crefname{claim}{Claim}{Claims}

\crefname{corollary}{corollary}{corollaries}
\Crefname{corollary}{Corollary}{Corollaries}

\crefname{proposition}{proposition}{propositions}
\Crefname{proposition}{Proposition}{Propositions}

\crefname{definition}{definition}{definitions}  % Explicitly set "Definition"
\Crefname{definition}{Definition}{Definitions}

\crefname{observation}{observation}{observations}
\Crefname{observation}{Observation}{Observations}

\crefname{question}{question}{questions}
\Crefname{question}{Question}{Questions}

\crefname{assumption}{assumption}{assumptions}
\Crefname{assumption}{Assumption}{Assumptions}

\crefname{algorithm}{algorithm}{algorithms}
\Crefname{algorithm}{Algorithm}{Algorithms}

\crefname{AlgoLine}{line}{lines}  % If using `algorithm2e` with line numbers
\Crefname{AlgoLine}{Line}{Lines}


%% probability notation
\DeclareMathOperator{\cov}{cov}
\DeclareMathOperator{\sgn}{\mathbf{sgn}}
\DeclareMathOperator{\E}{\mathbf{E}}
\DeclareMathOperator{\Var}{\mathbf{Var}}
\DeclareMathOperator{\one}{\mathbf{1}}
\newcommand{\given}{\mid}
\DeclareMathOperator{\Ball}{Ball}
\DeclareMathOperator{\tr}{tr}

% rounding up and down
\newcommand {\roundup}   [1] {{\lceil {#1} \rceil}}
\newcommand {\rounddown} [1] {{\lfloor {#1} \rfloor}}

%% black board letters
\newcommand{\bbB}{\mathbb{B}}
\newcommand{\bbC}{\mathbb{C}}
\newcommand{\bbR}{\mathbb{R}}
\newcommand{\bbZ}{\mathbb{Z}}

%% calligraphic letters 
\newcommand{\calA}{\mathcal{A}}
\newcommand{\calB}{\mathcal{B}}
\newcommand{\calC}{\mathcal{C}}
\newcommand{\calD}{\mathcal{D}}
\newcommand{\calE}{\mathcal{E}}
\newcommand{\calF}{\mathcal{F}}
\newcommand{\calG}{\mathcal{G}}
\newcommand{\calH}{\mathcal{H}}
\newcommand{\calI}{\mathcal{I}}
\newcommand{\calJ}{\mathcal{J}}
\newcommand{\calK}{\mathcal{K}}
\newcommand{\calL}{\mathcal{L}}
\newcommand{\calM}{\mathcal{M}}
\newcommand{\calN}{\mathcal{N}}
\newcommand{\calO}{\mathcal{O}}
\newcommand{\calP}{\mathcal{P}}
\newcommand{\calQ}{\mathcal{Q}}
\newcommand{\calR}{\mathcal{R}}
\newcommand{\calS}{\mathcal{S}}
\newcommand{\calT}{\mathcal{T}}
\newcommand{\calU}{\mathcal{U}}
\newcommand{\calV}{\mathcal{V}}
\newcommand{\calW}{\mathcal{W}}
\newcommand{\calX}{\mathcal{X}}
\newcommand{\calY}{\mathcal{Y}}
\newcommand{\calZ}{\mathcal{Z}}

\newcommand{\N}{\mathbb{N}}
\newcommand{\R}{\mathbb{R}}
\DeclareMathOperator*{\argmin}{arg\,min}
\DeclareMathOperator*{\argmax}{arg\,max}




\newcommand{\err}{\mathrm{err}}
\newcommand{\errstar}{\mathrm{err}^*}

\DeclareMathOperator{\REC}{REC}
\DeclareMathOperator{\NRD}{NRD}
\DeclareMathOperator{\FN}{FN}
\DeclareMathOperator{\FP}{FP}

\DeclareMathOperator{\Tr}{Tr}

\newcommand{\conv}{\mathrm{conv}}
\newcommand{\cone}{\mathrm{cone}}
\newcommand{\inner}[2]{\langle #1, #2\rangle}