\clearpage
\setcounter{page}{1}
\maketitlesupplementary
\section{Experimental Setup Details}
\subsection{Hyperparameters for DT1: Action Classification}
\label{appx:dt1}
As discussed in Section~\ref{sec:dt1}, we evaluate MoFM on NTU-RGB+D~\cite{ntu60Paper} (60 classes) and NTU-RGB+D-120~\cite{ntu120paper} (120 classes) for action recognition. The detailed hyperparameters for training on both datasets are presented in Table~\ref{tab:training_setup_ntu}.
%We used the NTU-RGB+D \cite{ntu60Paper} and NTU-RGB+D-120 \cite{ntu120paper} datasets for action classification experiments, following both Cross-Subject (X-Sub) and Cross-View (X-View) splits. The NTU-120 dataset was also used for one-shot action classification experiments. Tables  \ref{tab:dt1_training_setup_nt60} and  \ref{tab:dt1_training_setup_ntu120} summarize the hyperparameters for reproducibility.
\iffalse
\begin{table}[h]
    \centering
    \caption{Fine-uning setup details for NTU RGB+D \cite{ntu60Paper}.}
    \begin{tabular}{l c}
        \toprule
        \toprule
        \textbf{Parameter}            & \textbf{Value}                \\ \hline
        Epoch                         & 20                            \\ 
        Batch Size                    & 24                             \\ 
        Optimizer                     & AdamW                         \\ 
        Adam \(\beta_1\), \(\beta_2\) & 0.9, 0.999                    \\ 
        Weight Decay                  & 3e-4                          \\ 
        Base Learning Rate             & 1e-8                        \\ 
        Peak Learning Rate             & 1e-4                        \\ 
        Warmup Period                 & 5 epochs                      \\ 
        Learning Rate Decay           & Cosine                        \\ 
        Dropout                       & 0.4                           \\ \bottomrule
    \end{tabular}
    \label{tab:dt1_training_setup_nt60}
\end{table}

\begin{table}[h]
    \centering
    \caption{Fine-tuning setup details for NTU-120.}
    \begin{tabular}{l c}
        \toprule
        \toprule
        \textbf{Parameter}            & \textbf{Value}                \\ \hline
        Epoch                         & 16                            \\ 
        Batch Size                    & 24                             \\ 
        Optimizer                     & AdamW                         \\ 
        Adam \(\beta_1\), \(\beta_2\) & 0.9, 0.999                    \\ 
        Weight Decay                  & 3e-4                          \\ 
        Base Learning Rate             & 1e-8                        \\ 
        Peak Learning Rate             & 1e-4                        \\ 
        Warmup Period                 & 5 epochs                      \\ 
        Learning Rate Decay           & Cosine                        \\ 
        Dropout                       & 0.4                           \\ \bottomrule
    \end{tabular}
    \label{tab:dt1_training_setup_ntu120}
\end{table}
\fi

\begin{table}[h]
    \centering
    \caption{Fine-tuning setup details for action classification on NTU RGB+D \cite{ntu60Paper} and  NTU-RGB+D-120 \cite{ntu120paper} datasets.}
    \begin{tabular}{l c c}
        \toprule
        \toprule
        \textbf{Parameter}            & \textbf{NTU-60} & \textbf{NTU-120}  \\ \hline
        Epoch                         & 16              & 20                 \\ 
        Batch Size                    & \multicolumn{2}{c}{24}              \\ 
        Optimizer                     & \multicolumn{2}{c}{AdamW}           \\ 
        Adam \(\beta_1\), \(\beta_2\) & \multicolumn{2}{c}{0.9, 0.999}      \\ 
        Weight Decay                  & \multicolumn{2}{c}{3e-4}            \\ 
        Base Learning Rate            & \multicolumn{2}{c}{1e-8}            \\ 
        Peak Learning Rate            & \multicolumn{2}{c}{1e-4}            \\ 
        Warmup Period                 & \multicolumn{2}{c}{5 epochs}        \\ 
        Learning Rate Decay           & \multicolumn{2}{c}{Cosine}          \\ 
        Dropout                       & \multicolumn{2}{c}{0.4}             \\ \bottomrule
    \end{tabular}
    \label{tab:training_setup_ntu}
\end{table}

\subsection{Hyperparameters for DT2: One-shot Action Calssification}
\label{appx:dt2}
We evaluate MoFM for one-shot action recognition on NTU-RGB+D-120~\cite{ntu120paper}. The model is trained on the auxiliary set (100 samples per class) using supervised contrastive learning~\cite{khosla2020supervised}, where \(m=2\) samples per class are randomly selected in each batch. Training hyperparameters are summarized in Table~\ref{tab:dt2_training_setup}.
\begin{table}[h]
    \centering
    \caption{Fine-tuning setup details for one-shot action classification on NTU-RGB+D-120 \cite{ntu120paper} dataset.} 
    \begin{tabular}{l c}
        \toprule
        \toprule
        \textbf{Parameter}            & \textbf{Value}                \\ \hline
        Epoch                         & 5                            \\ 
        Batch Size                    & 16                             \\ 
        Optimizer                     & AdamW                         \\ 
        Adam \(\beta_1\), \(\beta_2\) & 0.9, 0.999                    \\ 
        Weight Decay                  & 3e-4                          \\ 
        Base Learning Rate             & 1.e-8                        \\ 
        Peak Learning Rate             & 1.e-5                        \\ 
        Warmup Period (Epochs)        & 0.3                       \\ 
        Learning Rate Decay           & Cosine                        \\ 
        Dropout                       & 0.4                           \\ 
        Contrastive Temp              & 0.1                            \\ \bottomrule
    \end{tabular}
    \label{tab:dt2_training_setup}
\end{table}

\subsection{Hyperparameters for DT3: Self-supervised Human Anomaly Detection}
\label{appx:dt3}
%As detailed in \cref{sec:sht}, we utilized the widely used SHT \cite{liu2018future} dataset to evaluate and demonstrate the self-supervised anomaly detection capabilities of our model. In alignment with the standard convention for self-supervised anomaly detection, the model was exclusively trained on normal data to effectively learn and understand normal behavior patterns. The detailed hyperparameters for fine-tuning can be found in \cref{appx:dt3}. Please note that the hyperparameters remain the same for HR-SHT, as it is a subset of the SHT dataset. HR-SHT \cite{morais2019learning} uses the same training set but features a different test set, which focuses exclusively on human-related anomalies.
As discussed in Section~\ref{sec:sht}, we evaluate MoFM on the SHT~\cite{liu2018future} dataset and its human-centric subset HR-SHT~\cite{morais2019learning} for anomaly detection. The training hyperparameters for both datasets are presented in Table~\ref{tab:fine_tuning_setup_dt3}. Please note that the hyperparameters remain the same for HR-SHT, as it is a subset of the SHT dataset. HR-SHT \cite{morais2019learning} uses the same training set but features a different test set, which focuses exclusively on human-related anomalies.
\begin{table}[h]
    \centering
    \caption{Fine-tuning setup details for self-supervised anomaly detection on SHT \cite{liu2018future} dataset.}
    \begin{tabular}{l c}
        \toprule
        \toprule
        Epoch                         & 1                            \\ 
        Batch Size                    & 16                             \\ 
        Optimizer                     & Adam                        \\ 
        Adam \(\beta_1\), \(\beta_2\) & 0.9, 0.999                    \\ 
        Weight Decay                  & 6e-4                          \\ 
        Base Learning Rate             & 1e-8                        \\ 
        Peak Learning Rate             & 6e-4                        \\ 
        % Warmup Period (Epochs)        & \hl{?}                      \\ 
        Learning Rate Decay           & OneCycleLR                        \\ 
        Dropout                       & 0.4                           \\ \bottomrule
    \end{tabular}
    \label{tab:fine_tuning_setup_dt3}
\end{table}



\subsection{Hyperparameters for DT4: Supervised Human Anomaly Detection}
\label{appx:dt4}
As discussed in Section~\ref{sec:sup_anomaly}, for supervised anomaly detection we utilized the UBnormal dataset \cite{Acsintoae_CVPR_2022}. Detailed hyperparameters for the fine-tuning process are provided in Table \ref{tab:extended_fine_tuning_setup_dt4} to ensure reproducibility.
\begin{table}[h]
    \centering
    \caption{Fine-tuning setup details for supervised anomaly detection on UBnormal \cite{Acsintoae_CVPR_2022} dataset.}
    \begin{tabular}{l c}
        \toprule
        \toprule
        \textbf{Parameter}            & \textbf{Value}                \\ \hline
        Epoch                         & 8                            \\ 
        Batch Size                    & 16                            \\ 
        Optimizer                     & Adam                         \\ 
        Adam \(\beta_1\), \(\beta_2\) & 0.9, 0.999                    \\ 
        Weight Decay                  & 3e-4                          \\ 
        Base Learning Rate             & 1e-8                        \\ 
        Peak Learning Rate             & 1e-4                        \\ 
        % Warmup Period (Epochs)        & \hl{?}                      \\ 
        Learning Rate Decay           & OneCycleLR \\ 
        Dropout                       & 0.4                           \\ \bottomrule
    \end{tabular}
    \label{tab:extended_fine_tuning_setup_dt4}
\end{table}
