\begin{figure*}[t]
    \begin{tabular}{c}
        % \hspace{-5.0em}
        \begin{minipage}[b]{\linewidth}
            \centering
            \includegraphics[width=0.9\linewidth]{model/ver5.3.1.pdf}
            \vspace{-0.5em}
            \subcaption{Self-dynamics factor set (i.e., $\ith{\selfdynamics} = \{ \imode, \ieig \}$)}
            \label{fig:model:uni}
            \vspace{-0.4em}
        \end{minipage} \\
        \begin{minipage}[b]{\linewidth}
            \centering
            \includegraphics[width=0.9\linewidth]{model/ver5.4.2.pdf}
            \vspace{-0.8em}
            \subcaption{Single regime parameter set (i.e., $\regime = \{ \demixing, \first{\selfdynamics}, ...,  \dth{\selfdynamics} \}$)}
            \label{fig:model:multi}
        \end{minipage}
    \end{tabular}
    \vspace{-1.4em}
    \caption{Illustration of \method:
    (a) we extract the latent temporal dynamics from the $i$-th univariate inherent signal $\ith{\ind}$, which behaves as a dynamical system.
    % In addition, the eigenvalues $\ieig$, which imply the temporal dynamics of the mode $\imode$, are interpretable.
    (b) The multivariate time series is described by mixing matrix $\demixing^{-1}$ and a collection of $d$ self-dynamics factor sets $\{ \first{\selfdynamics}, ..., \dth{\selfdynamics} \}$.
    % Each factor set corresponds to the evolution of $i$-th inherent signal $\ind_i$ as a dynamical system.
    The mixing matrix $\demixing^{-1}$ is not the same matrix as the causal adjacency matrix $\mB$,
    it is instrumental in identifying the \relation.}
    \label{fig:model}
    \vspace{-1.0em}
\end{figure*}