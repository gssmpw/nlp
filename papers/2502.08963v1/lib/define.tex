% \setlength{\textfloatsep}{0.0020pt}% tight fig,tab...
\setlength{\textfloatsep}{1.0pt}% tight fig,tab...
% \setlength{\textfloatsep}{0.8pt}% tight fig,tab...
% \setlength{\textfloatsep}{0.4pt}% tight fig,tab...

\newcommand{\mkclean}{
    \renewcommand{\reminder}[1]{}
    \renewcommand{\duty}[1]{}
    % \renewcommand{\temphide}{\noop}
    % \renewcommand{\bigreminder}[1]{}
}
\newcommand{\TSK}[1]{#1}
\newcommand{\tensakuClean}{
    \renewcommand{\TSK}[1]{}
}

% \setlength\intextsep{0pt}

\newcommand{\unclear}[1]{{\color{red}{#1}}\noindent}

% the modified reference of "subcaption".
\captionsetup[subfigure]{labelformat=simple}
\renewcommand{\thesubfigure}{ (\alph{subfigure})}

% root path of figure
\graphicspath{{./fig/}}

\renewcommand{\dblfloatpagefraction}{1.0}

\newcommand{\vswd}{\vspace{0.0em}}
%%%%%%%%%%%%%%%%%%%%%%%%%%%%%%%%%%%%%%%%%
% from cf: shorthands - also they make tighter lists
%\newcommand{\bit}{\vspace{-0.5em}\begin{itemize*}}
\newcommand{\bit}{\vspace{-0em}\begin{itemize}}
\newcommand{\eit}{\end{itemize}\vspace{-0.2em}}
\newcommand{\ben}{\vspace{-0em}\begin{enumerate}}
\newcommand{\een}{\end{enumerate}\vspace{-0.2em}}
\newcommand{\bea}{\vspace{-0em}\begin{eqnarray}}
\newcommand{\eea}{\end{eqnarray}\vspace{-0.0em}}
\newcommand{\beq}{\vspace{-0.0em}\begin{equation}}
\newcommand{\eeq}{\end{equation}\vspace{-0.0em}}

% Mathematics
\newcommand{\argmax}{\mathop{\rm arg~max}}
% \newcommand{\argmax}{\mathop{\rm arg\,max}\mathchoice{\nolimits}{\limits}{\limits}{\limits}}
\newcommand{\argmin}{\mathop{\rm arg~min}}
% \newcommand{\argmin}{\operatorname{arg\,min}}
\newcommand{\emp}{\bf \underline}
\newcommand{\mrk}{$\surd$}  % check mark
\newcommand{\R}{\mathbb{R}}
\newcommand{\C}{\mathbb{C}}

% Windows
\newcommand{\tmst}{t_m}
\newcommand{\tmed}{t_c}
\newcommand{\tfst}{t_s}
\newcommand{\tfed}{t_f}
\newcommand{\lenc}{l_c}
\newcommand{\lena}{l_s}
\newcommand{\lene}{l_e}

\newcommand{\goal}[1]{ {\noindent {$\Rightarrow$} \em {#1} } }
\newcommand{\hide}[1]{}
% \newcommand{\comment}[1]{ [COMMENT: {\tiny {#1} } END-COMMENT] }
\newcommand{\reminder}[1]{\textsf{\textcolor{red}{#1}}}
\newcommand{\add}[1]{\textsf{\textcolor{blue}{#1}}}
% \newcommand{\reminder}[1]{#1}
% \newcommand{\add}[1]{#1}
\newcommand{\duty}[1]{{\textsf{\textcolor{blue}{[#1's job]}}}}
\newcommand{\vectornorm}[1]{\left|\left|#1\right|\right|}
%\newcommand{\mypara}[1]{\vspace{-1.0em}\paragraph*{\bf #1}}s
%\newcommand{\mypara}[1]{\vspace{-0.80em}\paragraph*{\bf #1}}
%\newcommand{\mypara}[1]{\vspace{0.40em}\noindent{\bf #1.}}
\newcommand{\mypara}[1]{\vspace{1.00em}\noindent\textbf{#1.}}
\newcommand{\myparabf}[1]{\vspace{0.00em}\noindent\textbf{#1}}
% \newcommand{\myparaitemize}[1]{\underline{\textbf{#1:}}}
\newcommand{\myparaitemize}[1]{\noindent{\textbf{#1.}}}

\newcommand{\bi}{\bfseries\itshape}
\newcommand{\ac}[1]{\acute{#1}}

\newcommand{\prop}[1]{({\bf{#1}})\xspace}


% ============
% Environments
% ============
% \newtheorem{algo}{Algorithm}
% \newtheorem{algo}{\textsc{Algorithm}}
\newtheorem{observation}{\textsc{Observation}}
\newtheorem{problem}{Problem}
% \newtheorem{problem}{\textsc{Problem}}
\newtheorem{lemma}{\textsc{Lemma}}
% \newtheorem{proof}{\textsc{Proof}}
\newtheorem{sublemma}{Lemma}[lemma]
\newtheorem{definition}{Definition}
\newtheorem{model}{Model}
% \newtheorem{definition}{\textsc{Definition}}
% \renewcommand{\thelemma}{\Alph{section}.\arabic{lemma}}


%=================
% concept words
%=================
\newcommand{\targetdata}{time-evolving data streams\xspace} 
\newcommand{\targetdatashort}{data streams\xspace}
\newcommand{\relation}{time-evolving causality\xspace}
\newcommand{\relations}{time-evolving causalities\xspace}
\newcommand{\Relation}{Time-evolving causality\xspace}
\newcommand{\RELATION}{Time-evolving Causality\xspace}
\newcommand{\self}{latent temporal self-dynamics\xspace}


% ===================
% mathematical symbols
% ===================
% matrix
\newcommand{\mat}[1]{\bm{#1}}
\newcommand{\mA}{\mat{A}}
\newcommand{\mB}{\mat{B}}
\newcommand{\mE}{\mat{E}}
\newcommand{\mH}{\mat{H}}
\newcommand{\mX}{\mat{X}}
\newcommand{\mW}{\mat{W}}

% vector
\newcommand{\vect}[1]{\bm{#1}}
\newcommand{\rowvect}[1]{\vect{#1}}
\newcommand{\vh}{\vect{h}}
\newcommand{\vw}{\vect{w}}
\newcommand{\vx}{\vect{x}}
\newcommand{\vxt}{\vx_t}
\newcommand{\vs}{\vect{s}}
\newcommand{\vvec}{\vect{v}}

% model
\newcommand{\latent}{\vect{s}}
\newcommand{\ind}{\vect{e}}
\newcommand{\iind}{\vect{e}_i}
\newcommand{\est}{\vect{v}}
\newcommand{\demixing}{\mat{W}}

% alg
\newcommand{\regime}{\boldsymbol{\theta}}
\newcommand{\regimeset}{\boldsymbol{\Theta}}
\newcommand{\update}{\boldsymbol{\omega}}
\newcommand{\updateset}{\boldsymbol{\Omega}}
\newcommand{\modelparam}{\mathcal{F}}
\newcommand{\candparam}{\mathcal{C}}
\newcommand{\selfdynamics}{\mathcal{D}}

% time-delay embedding
\newcommand{\hankel}{\mat{H}}
\newcommand{\embed}[1]{{\textit{g}({#1})}}
\newcommand{\invembed}[1]{{\textit{g}^{-1}({#1})}}

% DMD
\newcommand{\nmodes}{k}
\newcommand{\modes}{\mat{\Phi}}
\newcommand{\eigs}{\mat{\Lambda}}
\newcommand{\leftsig}{\mat{U}}
\newcommand{\imode}{\ith{\mat{\Phi}}}
\newcommand{\ieig}{\ith{\mat{\Lambda}}}
\newcommand{\ileftsig}{\ith{\mat{U}}}
\newcommand{\trans}{\mat{A}}

% misc
% \newcommand{\1th}[1]{{{#1}_{(i)}}}
% \newcommand{\ith}[1]{{{#1}_{(i)}}}
% \newcommand{\dth}[1]{{{#1}_{(i)}}}
\newcommand{\forgetting}{\mu}
\newcommand{\Forgetting}{\mat{M}}
\newcommand{\ith}[2][]{%
    \ifthenelse{\isempty{#1}}%
    {% 第二引数が省略された場合
        #2_{(i)}
    }%
    {% 第二引数が与えられた場合
        {#2_{(i)}^{#1}}
    }%
}
\newcommand{\first}[2][]{%
    \ifthenelse{\isempty{#1}}%
    {% 第二引数が省略された場合
        #2_{(1)}
    }%
    {% 第二引数が与えられた場合
        {#2_{(1)}^{#1}}
    }%
}
\newcommand{\dth}[2][]{%
    \ifthenelse{\isempty{#1}}%
    {% 第二引数が省略された場合
        #2_{(d)}
    }%
    {% 第二引数が与えられた場合
        {#2_{(d)}^{#1}}
    }%
}
\newcommand{\ntotal}{N}	% # of total counts
\newcommand{\mathN}{\mathbb{N}}
\newcommand{\w}{\tau}


% ===================
% Proposed Algorithms
% ===================
\newcommand{\method}{\textsc{ModePlait}\xspace}
\newcommand{\modelestimator}{\textsc{ModeEstimator}\xspace}
\newcommand{\modelgenerator}{\textsc{ModeGenerator}\xspace}
\newcommand{\modelidentifier}{\textsc{NameIdentifier}\xspace}
\newcommand{\regimeupdate}{\textsc{RegimeUpdater}\xspace}


% ===========
% Experiments
% ===========
% Dataset Names
\newcommand{\synthetic}{\textit{\#0 synthetic}\xspace}
\newcommand{\psynthetic}{\textit{(\#0) synthetic}\xspace}
\newcommand{\covid}{\textit{\#1 covid19}\xspace}
\newcommand{\pcovid}{\textit{(\#1) covid19}\xspace}
\newcommand{\googletrend}{\textit{\#2 web-search}\xspace}
\newcommand{\pgoogletrend}{\textit{(\#2) web-search}\xspace}
\newcommand{\chickendance}{\textit{\#3 chicken-dance}\xspace}
\newcommand{\pchickendance}{\textit{(\#3) chicken-dance}\xspace}
\newcommand{\exercise}{\textit{\#4 exercise}\xspace}
\newcommand{\pexercise}{\textit{(\#4) exercise}\xspace}

\AtBeginDocument{%
  \providecommand\BibTeX{{%
    \normalfont B\kern-0.5em{\scshape i\kern-0.25em b}\kern-0.8em\TeX}}}
