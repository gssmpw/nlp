%% The "author" command and its associated commands are used to define
\author{Naoki Chihara}
% \orcid{1234-5678-9012}
\affiliation{%
  \institution{SANKEN, Osaka University}
  \state{Osaka}
  \country{Japan}
}
\email{naoki88@sanken.osaka-u.ac.jp}

\author{Yasuko Matsubara}
% \orcid{1234-5678-9012}
\affiliation{%
  \institution{SANKEN, Osaka University}
  \state{Osaka}
  \country{Japan}
}
\email{yasuko@sanken.osaka-u.ac.jp}

\author{Ren Fujiwara}
% \orcid{1234-5678-9012}
\affiliation{%
  \institution{SANKEN, Osaka University}
  \state{Osaka}
  \country{Japan}
}
\email{r-fujiwr88@sanken.osaka-u.ac.jp}

\author{Yasushi Sakurai}
% \orcid{1234-5678-9012}
\affiliation{%
  \institution{SANKEN, Osaka University}
  \state{Osaka}
  \country{Japan}
}
\email{yasushi@sanken.osaka-u.ac.jp}

%% By default, the full list of authors will be used in the page
% \renewcommand{\shortauthors}{Chihara, et al.}

%% The abstract is a short summary of the work to be presented in the article.
\begin{abstract}
\begin{abstract}


The choice of representation for geographic location significantly impacts the accuracy of models for a broad range of geospatial tasks, including fine-grained species classification, population density estimation, and biome classification. Recent works like SatCLIP and GeoCLIP learn such representations by contrastively aligning geolocation with co-located images. While these methods work exceptionally well, in this paper, we posit that the current training strategies fail to fully capture the important visual features. We provide an information theoretic perspective on why the resulting embeddings from these methods discard crucial visual information that is important for many downstream tasks. To solve this problem, we propose a novel retrieval-augmented strategy called RANGE. We build our method on the intuition that the visual features of a location can be estimated by combining the visual features from multiple similar-looking locations. We evaluate our method across a wide variety of tasks. Our results show that RANGE outperforms the existing state-of-the-art models with significant margins in most tasks. We show gains of up to 13.1\% on classification tasks and 0.145 $R^2$ on regression tasks. All our code and models will be made available at: \href{https://github.com/mvrl/RANGE}{https://github.com/mvrl/RANGE}.

\end{abstract}


\end{abstract}

%%
%% The code below is generated by the tool at http://dl.acm.org/ccs.cfm.
%% Please copy and paste the code instead of the example below.
%%
\begin{CCSXML}
<ccs2012>
   <concept>
       <concept_id>10002951.10003227.10003351.10003446</concept_id>
       <concept_desc>Information systems~Data stream mining</concept_desc>
       <concept_significance>500</concept_significance>
       </concept>
 </ccs2012>
\end{CCSXML}

\ccsdesc[500]{Information systems~Data stream mining}

%%
%% Keywords. The author(s) should pick words that accurately describe
%% the work being presented. Separate the keywords with commas.
\keywords{Time series,
Stream processing,
Dynamical system
}

%% A "teaser" image appears between the author and affiliation
%% information and the body of the document, and typically spans the
%% page.
% \received{8 August 2024}
% \received[revised]{12 March 2009}
% \received[accepted]{5 June 2009}
