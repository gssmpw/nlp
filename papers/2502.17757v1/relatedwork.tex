\section{Related Work}
\label{related}
The integration of machine learning into financial risk management, with a focus on hedging, has seen growing interest for its capacity to address practical complexities.  Traditional approaches, such as the Black-Scholes framework~\citep{black1973pricing}, rely on closed-form solutions for derivative pricing and hedging. While theoretically sound, these methods struggle to accommodate market frictions, transaction costs, and path-dependent dynamics~\citep{shreve2005stochastic}. 
Deep hedging has revolutionized financial risk management in derivative markets by leveraging deep learning to navigate complex market dynamics and address challenges such as transaction costs and liquidity constraints~\citep{buehler2019deep}. Practical insights and applications of deep hedging, emphasizing its model-independent nature and ability to optimize hedging strategies, have also been highlighted in industry reports~\citep{gao2023deeper}\citep{neagu2024deep}. Collectively, these works demonstrate the transformative potential of deep learning in improving risk management and hedging efficiency in financial markets.

Deep hedging frameworks often employ reinforcement learning~\citep{cao2021deep} and neural network architectures~\citep{ruf2019neural} to optimize hedging strategies in dynamic environments. These models extend classical paradigms by incorporating market features and frictions, enabling adaptive risk management. However, challenges such as action dependence, non-convex optimization landscapes, and sensitivity to noisy data~\citep{karakida2019universal} remain significant barriers to their practical deployment. Moreover, existing methods often require extensive computational resources, limiting their scalability.


The No-Transaction Band (NTB) Network~\citep{imaki2021no}, which is a specialized method within the deep hedging framework, addresses the challenge of transaction costs by introducing a constrained trading strategy that limits activity to a no-transaction band. While this approach effectively reduces transaction costs, it also exhibits significant limitations. NTB's rigid constraints can restrict adaptability to rapidly changing market conditions and complex trading environments. Additionally, its design may limit the model's capacity to capture complex dependencies in market data, potentially reducing its efficacy in high-dimensional and volatile scenarios. These weaknesses highlight the need for more flexible and robust solutions to manage the multifaceted challenges of deep hedging in diverse market environments.


Recent advancements in the integration of financial theory with machine learning have also demonstrated promising directions for managing complex financial risks.  \citet{pesenti2024risk} propose a dynamic risk budgeting framework that employs dynamic risk measures to allocate risks across time and assets, providing a structured approach to risk management. Their use of an actor-critic method highlights the potential for machine learning to operationalize advanced financial concepts. Similarly, \citet{moresco2024uncertainty} develop a framework to model uncertainty in stochastic processes through robust risk measures and recursive representations, offering valuable tools for dynamic decision-making under uncertainty. Furthermore, \citet{pesenti2023portfolio} extend these ideas to active portfolio management, employing a Wasserstein ball approach to balance benchmark tracking with distributionally robust optimization. While these approaches effectively incorporate financial theory into machine learning frameworks, they often  require extensive computational resources to handle complex dynamics and adapt to evolving market conditions. Additionally, their focus on specific financial applications limits their generalizability to broader scenarios, such as hedging under high-dimensional, noisy, and volatile environments. These challenges underscore the need for more flexible, computationally efficient, and interpretable solutions that can seamlessly adapt to diverse market dynamics while maintaining robust performance. 


In this work, we introduce DHLNN to address the persistent challenges of computational inefficiency, sensitivity to noisy data, and optimization complexity in deep learning-based financial applications. Unlike traditional approaches, DHLNN innovates by enhancing the training procedure itself. Through the integration of a periodic fixed-gradient optimization method and linearized training dynamics, our framework stabilizes training, accelerates convergence, and significantly improves robustness in volatile and dynamic market environments.
The generalizability of this innovation sets DHLNN apart from other deep learning methods. By providing a streamlined, scalable solution that can be seamlessly integrated into diverse neural network models, our framework offers a universally applicable strategy for improving training efficiency and robustness. This positions DHLNN as a pivotal advancement not only for hedging and risk management but also for broader applications in financial machine learning. These contributions underline the potential of our approach to redefine how deep learning models are trained and applied in financial contexts.