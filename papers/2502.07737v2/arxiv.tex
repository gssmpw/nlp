%%%%%%%% ICML 2025 EXAMPLE LATEX SUBMISSION FILE %%%%%%%%%%%%%%%%%

\documentclass{article}

% Recommended, but optional, packages for figures and better typesetting:
\usepackage{microtype}
\usepackage{graphicx}
\usepackage{subfigure}
\usepackage{booktabs} % for professional tables

% hyperref makes hyperlinks in the resulting PDF.
% If your build breaks (sometimes temporarily if a hyperlink spans a page)
% please comment out the following usepackage line and replace
% \usepackage{icml2025} with \usepackage[nohyperref]{icml2025} above.
\usepackage{hyperref}


% Attempt to make hyperref and algorithmic work together better:
% \newcommand{\theHalgorithm}{\arabic{algorithm}}

% Use the following line for the initial blind version submitted for review:
% \usepackage{icml2025}

% If accepted, instead use the following line for the camera-ready submission:
\usepackage[accepted]{icml2025}


% For theorems and such
\usepackage{amsmath}
\usepackage{amssymb}
\usepackage{mathtools}
\usepackage{amsthm}

% if you use cleveref..
\usepackage[capitalize,noabbrev]{cleveref}

\usepackage{hyperref}
\usepackage{url}
\usepackage{hyperref}
\usepackage{graphics}
\usepackage[utf8]{inputenc} % allow utf-8 input
\usepackage[T1]{fontenc}    % use 8-bit T1 fonts
\usepackage{hyperref}       % hyperlinks
\usepackage{url}            % simple URL typesetting
\usepackage{booktabs}       % professional-quality tables
\usepackage{amsfonts}       % blackboard math symbols
\usepackage{nicefrac}       % compact symbols for 1/2, etc.
\usepackage{microtype}      % microtypography
\usepackage{xcolor}         % colors
% \usepackage[square,sort,comma,numbers]{natbib}
\usepackage[utf8]{inputenc}
\usepackage{mathtools}
\usepackage{amsthm}
\usepackage{arydshln}
\usepackage{multirow}
\usepackage{wrapfig, lipsum, booktabs}
\usepackage{paralist, tabularx}
\usepackage{balance}
\usepackage{pgfplots}
\usetikzlibrary{pgfplots.groupplots}
\pgfplotsset{compat=1.3}
\usepackage{tikz}
\usetikzlibrary{patterns}
\usepackage{pgf-pie}
\usepackage{adjustbox}
\usepackage{colortbl}
\usepackage{subcaption}
\usepackage{xspace}
\usepackage{enumitem}

%%%%%%%%%%%%%%%%%%%%%%%%%%%%%%%%
% THEOREMS
%%%%%%%%%%%%%%%%%%%%%%%%%%%%%%%%
\theoremstyle{plain}
\newtheorem{theorem}{Theorem}[section]
\newtheorem{proposition}[theorem]{Proposition}
\newtheorem{lemma}[theorem]{Lemma}
\newtheorem{corollary}[theorem]{Corollary}
\theoremstyle{definition}
\newtheorem{definition}[theorem]{Definition}
\newtheorem{assumption}[theorem]{Assumption}
\theoremstyle{remark}
\newtheorem{remark}[theorem]{Remark}

% Todonotes is useful during development; simply uncomment the next line
%    and comment out the line below the next line to turn off comments
%\usepackage[disable,textsize=tiny]{todonotes}
\usepackage[textsize=tiny]{todonotes}

\makeatletter
\newcommand{\rmnum}[1]{\romannumeral #1}
\newcommand{\Rmnum}[1]{\expandafter\@slowromancap\romannumeral #1@}
\makeatother

\definecolor{my_yellow}{RGB}{248,236,198}
\definecolor{my_green}{RGB}{197,224,180}

\definecolor{demphcolor}{RGB}{144, 144, 144}
\newcommand{\demph}[1]{\textcolor{demphcolor}{#1}}

\newcommand{\acronym}[1]{\underline{\textbf{#1}}}
\newcommand{\modelname}{NBP\xspace}

\newcommand{\rsh}[1]{\textcolor{orange}{[#1 - rsh]}}

% The \icmltitle you define below is probably too long as a header.
% Therefore, a short form for the running title is supplied here:
\icmltitlerunning{Next Block Prediction: Video Generation via Semi-Autoregressive Modeling}

\begin{document}

\twocolumn[
\icmltitle{Next Block Prediction: Video Generation via Semi-Autoregressive Modeling}

% It is OKAY to include author information, even for blind
% submissions: the style file will automatically remove it for you
% unless you've provided the [accepted] option to the icml2025
% package.

% List of affiliations: The first argument should be a (short)
% identifier you will use later to specify author affiliations
% Academic affiliations should list Department, University, City, Region, Country
% Industry affiliations should list Company, City, Region, Country

% You can specify symbols, otherwise they are numbered in order.
% Ideally, you should not use this facility. Affiliations will be numbered
% in order of appearance and this is the preferred way.
\icmlsetsymbol{equal}{*}

\begin{icmlauthorlist}
\icmlauthor{Shuhuai Ren}{1}
\icmlauthor{Shuming Ma}{2}
\icmlauthor{Xu Sun}{1}
\icmlauthor{Furu Wei}{2}
\end{icmlauthorlist}

\icmlaffiliation{1}{National Key Laboratory for Multimedia Information Processing, School of Computer Science, Peking University}
\icmlaffiliation{2}{Microsoft Research}

\icmlcorrespondingauthor{Xu Sun}{xusun@pku.edu.cn}
\icmlcorrespondingauthor{Furu Wei}{fuwei@microsoft.com}

\hypersetup{urlcolor=black}
\begin{center}
\url{https://renshuhuai-andy.github.io/NBP-project/}
% \vspace{-2em} 
\end{center}

% You may provide any keywords that you
% find helpful for describing your paper; these are used to populate
% the "keywords" metadata in the PDF but will not be shown in the document
\icmlkeywords{Machine Learning, ICML}

\vskip 0.3in
]

% this must go after the closing bracket ] following \twocolumn[ ...

% This command actually creates the footnote in the first column
% listing the affiliations and the copyright notice.
% The command takes one argument, which is text to display at the start of the footnote.
% The \icmlEqualContribution command is standard text for equal contribution.
% Remove it (just {}) if you do not need this facility.

\printAffiliationsAndNotice{}  % leave blank if no need to mention equal contribution
% \printAffiliationsAndNotice{\icmlEqualContribution} % otherwise use the standard text.

\begin{abstract}
Next-Token Prediction (NTP) is a de facto approach for autoregressive (AR) video generation, but it suffers from suboptimal unidirectional dependencies and slow inference speed.  
In this work, we propose a semi-autoregressive (semi-AR) framework, called \acronym{N}ext-\acronym{B}lock \acronym{P}rediction (\modelname), for video generation. 
By uniformly decomposing video content into equal-sized blocks (e.g., rows or frames), we shift the generation unit from individual tokens to blocks, allowing each token in the current block to simultaneously predict the corresponding token in the next block. 
Unlike traditional AR modeling, our framework employs bidirectional attention within each block, enabling tokens to capture more robust spatial dependencies. By predicting multiple tokens in parallel, \modelname models significantly reduce the number of generation steps, leading to faster and more efficient inference. 
Our model achieves FVD scores of 103.3 on UCF101 and 25.5 on K600, outperforming the vanilla NTP model by an average of 4.4. 
Furthermore, thanks to the reduced number of inference steps, the \modelname model generates 8.89 frames (128$\times$128 resolution) per second, achieving an 11$\times$ speedup. We also explored model scales ranging from 700M to 3B parameters, observing significant improvements in generation quality, with FVD scores dropping from 103.3 to 55.3 on UCF101 and from 25.5 to 19.5 on K600, demonstrating the scalability of our approach.
\end{abstract}

\section{Introduction}
% ar建模和decoder-only架构在文本领域很强,将其它模态(特别是视频)的建模也引入,构造一套统一的结构一直是大家的梦想
% 之前有些工作将ar直接应用于视频生成,例如光栅顺序扫描,但这种建模方式不太理想
% 我们提出了semi-ar的构想,基于之前的帧序列直接预测下一个帧,而非单个token
% 效果
The advance of Large Language Models (LLMs) such as ChatGPT~\citep{chatgpt}, GPT-4~\citep{Achiam2023GPT4TR} and LLaMA~\citep{Touvron2023LLaMAOA} has cemented the preeminence of Autoregressive (AR) modeling in the realm of natural language processing (NLP). This AR modeling approach, combined with the decoder-only Transformer architecture~\citep{Vaswani2017AttentionIA}, has been pivotal in achieving advanced levels of linguistic understanding, generation, and reasoning~\citep{wei2022emergent, gpt-o1, chen-etal-2024-pca}. 
Recently, there is a growing interest in extending AR modeling from language to other modalities, such as images and videos, to develop a unified multimodal framework~\citep{gpt4o, Team2024ChameleonME, Lu2023UnifiedIO2S, Wu2023NExTGPTAM, chen2024next}. Such an AR-based framework brings numerous benefits: (1) It allows for the utilization of the well-established infrastructure and learning recipes from the LLM community~\citep{Dao2022FlashAttentionFA, kwon2023efficient}; (2) The scalability and generalizability of AR modeling, empirically validated in LLMs~\citep{Kaplan2020ScalingLF, Yu2023ScalingAM}, can be extended to the multimodal domains to strengthen models~\citep{henighan2020scaling}; (3) Cognitive abilities observed in LLMs can be transferred and potentially amplified with multimodal data, moving closer to the goal of artificial general intelligence~\citep{bubeck2023sparks}.

Given the inherently autoregressive nature of video data in temporal dimensions, video generation is a natural area for extending AR modeling. 
Vanilla AR methods for video generation typically follow the Next-Token Prediction (NTP) approach, i.e., tokenize video into discrete tokens, then predict each subsequent token based on the previous ones. 
However, this approach has notable limitations. First, the generation order of NTP often follows a unidirectional raster-scan pattern~\citep{hong2022cogvideo, Wang2024OmniTokenizerAJ, Yan2021VideoGPTVG}, which fails to capture strong 2D correlations within video frames, limiting the modeling of spatial dependencies~\citep{Tian2024VisualAM}. Second, NTP necessitates a significant number of forward passes during inference (e.g., 1024 steps to generate a 16-frame clip), which reduces efficiency and increases the risk of error propagation~\citep{bengio2015scheduled}.

In this work, we propose a semi-autoregressive (semi-AR) framework, called \acronym{N}ext-\acronym{B}lock \acronym{P}rediction (\modelname), for video generation. 
To better model local spatial dependencies and improve inference efficiency, our framework shifts the generation unit from individual tokens to blocks (e.g., rows or frames). The objective is also redefined from next-token to next-block prediction, where each token in the current block simultaneously predicts the corresponding token in the next block. 
In contrast to the vanilla AR framework, which attends solely to prefix tokens, our \modelname approach allows tokens to attend to all tokens within the same block via bidirectional attention, thus capturing more robust spatial relationships. By predicting multiple tokens in parallel, \modelname models significantly reduce the number of generation steps, resulting in faster and more computationally efficient inference. 

Experimental results on the UCF-101~\citep{Soomro2012UCF101AD} and Kinetics-600 (K600)~\citep{Carreira2018ASN} datasets demonstrate the superiority of our semi-AR framework. With the same model size (700M parameters), \modelname achieves a 103.3 FVD on UCF101 and a 25.5 FVD on K600, surpassing the vanilla NTP model by 4.4. Additionally, due to the reduced number of inference steps, \modelname models can generate 8.89 frames (128$\times$128 resolution) per second, achieving an 11$\times$ speedup in inference. Compared to previous state-of-the-art token-based models, our approach proves to be the most effective. Scaling experiments with models ranging from 700M to 3B parameters show a significant improvement in generation quality, with FVD scores dropping from 103.3 to 55.3 on UCF101 and from 25.5 to 19.5 on K600, highlighting the scalability of the framework. We hope this work inspires further advancements in the field.

\begin{figure*}[htbp]
    \centering
    \begin{minipage}[b]{0.39\textwidth}
        \centering
        \includegraphics[width=.9\textwidth]{figs/tokenizer.pdf}
        \caption{3D discrete token map produced by our video tokenizer. The input video consists of \colorbox{my_yellow}{one initial frame}, followed by $n$ \colorbox{my_green}{clips}, with each clip containing $F_T$ frames. $x^{(i)}_{j}$ indicates the $j^{th}$ video token in the $i^{th}$ clip.}
        \label{fig:tokenizer}
    \end{minipage}
    \hfill
    \begin{minipage}[b]{0.59\textwidth}
        \centering
        \includegraphics[width=\textwidth]{figs/block.pdf}
        \caption{Examples of block include token-wise, row-wise, and frame-wise representations. When the block size is set to 1$\times$1$\times$1, it degenerates into a token, as used in vanilla AR modeling. Note that the actual token corresponds to a 3D cube, we omit the time dimension here for clarity.}
        \label{fig:block_example}
    \end{minipage}
\end{figure*}

\section{Related Work}
\paragraph{Video Generation.}
Prevalent video generation frameworks in recent years include Generative Adversarial Networks (GANs)~\citep{Yu2022GeneratingVW, Skorokhodov2021StyleGANVAC}, diffusion models~\citep{Ho2022ImagenVH, Ge2023PreserveYO, Gupta2023PhotorealisticVG, Yang2024CogVideoXTD}, autoregressive models~\citep{hong2022cogvideo, Yan2021VideoGPTVG, Kondratyuk2023VideoPoetAL}, etc. 
GANs can generate videos with rich details and high visual realism, but their training is often unstable and prone to mode collapse. In contrast, diffusion models exhibit more stable training processes and typically produce results with greater consistency and diversity~\citep{Yang2022DiffusionMA}. 
Nevertheless, AR models demonstrate significant potential for processing multi-modal data (e.g., text, images, audio, and video) within a unified framework, offering strong scalability and generalizability. To align with the trend of natively multimodal development~\citep{gpt4o}, this paper focuses on exploring video generation using AR modeling.

\paragraph{Autoregressive Models for Video Generation.}
With the success of the GPT series models~\citep{Brown2020LanguageMA}, a range of studies has applied AR modeling to both image~\citep{Chen2020GenerativePF, Lee2022AutoregressiveIG, wang2024loong, pang2024randar} and video generation~\citep{hong2022cogvideo, Wang2024OmniTokenizerAJ, Yan2021VideoGPTVG}. 
For image generation, traditional methods divide an image into a sequence of tokens following a raster-scan order and then predict each subsequent token based on the preceding ones. In video generation, this process is extended frame by frame to produce temporally-coherence content. 
However, conventional AR models predict only one token at a time, resulting in a large number of forward steps during inference. This significantly impairs the generation speed, especially for high-resolution images or videos containing numerous tokens~\citep{liu2024lumina-mgpt}.

\paragraph{Semi-Autoregressive Models.}
To improve the efficiency of AR models, early NLP researchers has explored semi-autoregressive modeling by generating spans of tokens instead of individual tokens per step~\citep{wang2018semi}. However, due to the variable length of text generation targets, it is challenging to predefine span sizes. Furthermore, fixed-length spans can disrupt semantic coherence and completeness, leading to significant degradation in generation quality; for instance, using a span length of 6 results in a 12\% drop in performance for English-German translation tasks~\citep{wang2018semi}.
More advanced semi-AR approaches, such as parallel decoding~\citep{Stern2018BlockwisePD} and speculative decoding~\citep{Xia2022SpeculativeDE}, typically use multiple output heads or additional modules (e.g., draft models) to predict several future tokens based on the last generated token~\citep{Gu2017NonAutoregressiveNM, Gloeckle2024BetterF}. 
In the context of video, where content can be uniformly decomposed into equal-sized blocks (e.g., row by row or frame by frame), we propose a framework where each token in the last block predicts the corresponding token in the next block, without requiring additional heads or modules. 

% To improve the efficiency of AR models, researchers in the NLP field have explored semi-autoregressive modeling~\citep{wang2018semi}, parallel decoding~\citep{Stern2018BlockwisePD} and speculative decoding~\citep{Xia2022SpeculativeDE} algorithms. These methods typically use multiple output heads or additional modules (e.g., draft models) to predict several future tokens based on the last generated token~\citep{Gu2017NonAutoregressiveNM, Gloeckle2024BetterF}. Given that video content can be uniformly decomposed into blocks of equal size (e.g., row by row or frame by frame), we propose a framework where each token in the last block predicts the corresponding token in the next block, without requiring additional heads or modules. 

\paragraph{Multi-token Prediction in Image Generation.}
Recent work in the image generation field has also shown a pattern of multi-token prediction, albeit with different motivations and approaches. 
For example, VAR~\citep{Tian2024VisualAM} employs a coarse-to-fine strategy across resolution scales, whereas our method processes spatiotemporal blocks at original resolution, achieving over 2$\times$ token efficiency (256 vs. 680 tokens for a 256$\times$256 frame). 
Unlike MAR~\citep{Li2024AutoregressiveIG}, which relies on randomized masking (70\% mask rate) and suffers from partial supervision (30\% of unmasked tokens do not receive supervision), our approach eliminates mask token modeling entirely, ensuring full supervision and improved training efficiency. 
While other works explore specialized token combinations~\citep{li2023lformer,wang2024parallelized}, our method minimizes architectural priors, enabling seamless adaptation from pre-trained NTP models and superior performance, especially for video generation.


\section{Method}
In this section, we first introduce our video tokenizer $\S$~\ref{subsec:video-tokenization}, highlighting its two key features: joint image-video tokenization and temporal causality, both of which facilitate our semi-AR modeling approach. 
Next, we provide a detailed comparison between vanilla Next-Token Prediction (NTP) ($\S$~\ref{subsec:ar}) and our \acronym{N}ext-\acronym{B}lock \acronym{P}rediction (\modelname) modeling ($\S$~\ref{subsec:semi-ar}). 
Our \modelname framework employs a block-wise objective function and attention masking, enabling more efficient capture of spatial dependencies and significantly improving inference speed.

\subsection{Preliminary I: Video Tokenization}
\label{subsec:video-tokenization}
We reproduce closed-source MAGVITv2~\cite{yu2023language} as our video tokenizer, which is based on a causal 3D CNN architecture. 
Given a video $\mathbf{X} \in \mathbb{R}^{T \times H \times W \times 3}$ in RGB space,\footnote{Images can be considered as ``static'' videos with $T=1$.} MAGVITv2 encodes it into a feature map $\mathbf{Z} \in \mathbb{R}^{T' \times H' \times W' \times d}$, where $(T', H', W')$ is the latent size of $\mathbf{Z}$, and $d$ is the hidden dimension of its feature vectors.
After that, we apply a quantizer to convert this feature map $\mathbf{Z}$ into a discrete tokens map $\mathbf{Q} \in \mathbb{V}^{T' \times H' \times W'}$ (illustrated in Fig.~\ref{fig:tokenizer}), where $\mathbb{V}$ represents a visual vocabulary of size $|\mathbb{V}|=K$. 
After tokenization, these discrete tokens $\mathbf{Q}$ can be passed through a causal 3D CNN decoder to reconstruct the video $\hat{\mathbf{X}}$. 
We note that MAGVITv2 has two major advantages:

\paragraph{(1) Joint Image-Video Tokenization.} MAGVITv2 allows tokenizing images and videos with a shared vocabulary. 
To achieve this, the number of frames in an input video, $T$, must satisfy $T=1+n \times F_T$, meaning the video comprises an initial frame followed by $n$ clips, each containing $F_T$ frames. 
When $n=0$, the video contains only the initial frame, thus simplifying the video to an image. 
Both the initial frame and each subsequent clip are discretized into a $(1, H', W')$ token map. Therefore, the latent temporal dimension $T'$ of the token map $\mathbf{Q}$ equals to $1+n$, which achieves $F_T$ times downsampling ratio on the temporal dimension (except for the first frame). 
Additionally, $H' = \frac{H}{F_H}$ and $W' = \frac{W}{F_W}$, where $F_H, F_W$ are spatial downsampling factors.

\paragraph{(2) Temporal Causality.} The causal 3D CNN architecture ensures that the tokenization and detokenization of each clip depend only on the preceding clips, facilitating autoregressive modeling along the temporal dimension, which will be discussed further in $\S$~\ref{subsec:semi-ar}.
% \rsh{todo add finish sentence}

\begin{figure*}[tbp]
\centering
\includegraphics[width=.9\textwidth]{figs/framework.pdf}
\caption{Comparison between a vanilla autoregressive (AR) framework based on next-token prediction (left) and our semi-AR framework based on next-block prediction (right). $x^{(i)}_{j}$ indicates the $j^{th}$ video token in the $i^{th}$ block, with each block containing $L$ tokens. 
The dashed line in the right panel presents that the $L$ tokens generated in the current step are duplicated and concatenated with prefix tokens, forming the input for the next step's prediction during inference.}
\label{fig:framework}
\end{figure*}

\subsection{Preliminary II: Autoregressive Modeling for Video Generation}
\label{subsec:ar}
Inspired by the success of AR models in the field of NLP, previous work~\citep{Yan2021VideoGPTVG, Wu2021GODIVAGO, Wu2021NWAVS} has extended AR models to video generation. Typically, these methods flatten the 3D video token input $\mathbf{Q} \in \mathbb{V}^{T' \times H' \times W'}$ into a 1D token sequence. 
Let \colorbox{my_green}{$C^{(t)}=\{x^{(t)}_1, x^{(t)}_{2}, \dots, x^{(t)}_{L}\}$} be the set of tokens in the $t^{th}$ clip, where $L = H' \times W' = |C^{(t)}|$ is the total number of tokens in each clip, and every clip contains an equal number of tokens. 
Specially, when $t=0$, \colorbox{my_yellow}{$C^{(0)}$} denotes the first frame's tokens. 
Therefore, the 1D token sequence can be represented as 
$($
\colorbox{my_yellow}{$C^{(0)}$}
$, \dots,$
\colorbox{my_green}{$C^{(T')}$}
$)=($
\colorbox{my_yellow}{$x^{(0)}_1, x^{(0)}_2, \dots,x^{(0)}_L$}
$, \dots,$
\colorbox{my_green}{$x^{(T')}_1, x^{(T')}_2, \dots, x^{(T')}_L$}
$)$. 
In the AR framework, the next-token probability is conditioned on the preceding tokens, where each token $x^{(t)}_l$ depends only on its prefix $(x^{(<t)}_l, x^{(t)}_{<l})$. This unidirectional dependency allows the likelihood of the 1D sequence to be factorized as: 
\begin{equation}
\label{eq:ar}
p\left(x^{(0)}_1, \dots, x^{(T')}_L \right)
=\prod_{t=1}^{T'} \prod_{l=1}^{L} p\left(x^{(t)}_l \mid x^{(<t)}_l, x^{(t)}_{<l} \right)
\end{equation}
Since only one token is predicted per step, the inference process can become computationally expensive and time-consuming~\citep{liu2024lumina-mgpt}, motivating the exploration of more efficient methods, such as semi-AR models~\citep{wang2018semi}, to improve both speed and scalability.


\subsection{Semi-AR Modeling via Next Block Modeling}
\label{subsec:semi-ar}
In contrast to text, which consists of variable-length words and phrases, video content can be uniformly decomposed into equal-sized blocks (e.g., rows or frames). Fig.~\ref{fig:block_example} shows examples of token-wise, row-wise, and frame-wise block representations. 
Based on this, we propose a semi-autoregressive (semi-AR) framework named \acronym{N}ext-\acronym{B}lock \acronym{P}rediction (\modelname), where each token in the current block predicts the corresponding token in the next block. 
Fig.~\ref{fig:framework} illustrates an example of next-clip prediction, where each clip is treated as a block, and the next clip is predicted based on the preceding clips. 
This approach introduces two key differences compared to vanilla NTP modeling: 
\textbf{(1) Change in the generation target.} In \modelname, the $l^{th}$ token $x_l^{(t)}$ in the $t^{th}$ clip predicts $x_l^{(t+1)}$ in the next clip, rather than $x_{l+1}^{(t)}$ as in NTP. 
\textbf{(2) Increase in the number of generation targets.} Instead of predicting one token at a time, all $L$ tokens $x_{1:L}^{(t)}$ simultaneously predict the corresponding $L$ tokens $x_{1:L}^{(t+1)}$ in the next clip.
Accordingly, the \modelname objective function can be expressed as: 
\begin{equation}
\label{eq:semi-ar}
p\left(x^{(0)}_1, \ldots, x^{(T')}_L \right) 
= \prod_{t=1}^{T'} p\left( \colorbox{my_green}{$x_{1:L}^{(t)}$} \mid \colorbox{my_yellow}{$x_{1:L}^{(0)}$}, \ldots, \colorbox{my_green}{$x_{1:L}^{(t-1)}$} \right)
% = p\left(C^{(0)}, \ldots, C^{(T')}\right) 
% = \prod_{t=1}^{T'} p\left(C^{(t)} \mid C^{(0)}, \ldots, C^{(t-1)}\right)
\end{equation}
By adjusting the block size, the framework can generate videos using different generation units. To ensure the effectiveness of this approach, four key components are designed:

\paragraph{(1) Initial Condition.}
In NTP models, a special token (e.g., \texttt{[begin\_of\_video]}) is typically used as the initial condition. In the \modelname setting, we can introduce a block of special tokens to serve as the initial condition for generating the first block. 
However, our preliminary experiments revealed that learning the parallel generation from the special token block to the first block is quite challenging. To address this issue, we propose two methods:
\textbf{(i) Taking the first frame $C^{(0)}$ as the initial condition.} In practice, following~\citet{girdhar2023emu}, users can upload an image as the first frame, or call an off-the-shelf text-to-image model (e.g., SDXL~\citep{podell2023sdxl}) to generate it. 
\textbf{(ii) Adopting a hybrid generation process}~\citep{wang2024parallelized}. Specifically, we can use per-token AR generation for the tokens in the first block. After the first block is generated, we then shift to per-block semi-AR generation. 
In order to make a fair comparison with other baselines, we used method (ii) in our experiments rather than relying on an extra first frame. 
Lastly, we note that both NTP and \modelname models can accept various inputs (e.g., text) as additional conditions (see Fig.~\ref{fig:framework}).

\begin{figure}
    \centering
    \includegraphics[width=.9\linewidth]{figs/attn.pdf}
    \caption{Causal attention mask in NTP modeling v.s. block-wise attention mask in \modelname modeling. The x-axis and y-axis represent keys and queries, respectively.}
\label{fig:attn-mask}
\end{figure}


\paragraph{(2) Block-wise Attention.}
To better capture spatial dependency, we allow tokens to attend to all tokens within the same block via bidirectional attention. Fig.~\ref{fig:attn-mask} compares traditional causal attention in NTP modeling with block-wise attention in \modelname modeling. 

\paragraph{(3) Block Size and Block Shape.}
The size and shape of blocks significantly influence generation quality, prompting us to conduct a comprehensive ablation study in 
$\S$~\ref{subsec:ablation} to identify the optimal configuration. 
Generally, we exclude blocks that span multiple frames (block shape with $T>1$) for several reasons:
\textbf{(i) Temporal Compression Constraints}: Input videos are sampled at 8 FPS or 16 FPS and undergo 4$\times$ temporal downsampling during tokenization, resulting in substantial information compression along the temporal dimension. Modeling rapidly changing content simultaneously across frames presents considerable challenges. 
\textbf{(ii) Causal Temporal Dynamics}: Our goal for the \modelname framework is not only to excel in video generation but also to serve as a potential world model~\citep{bruce2024genie, ha2018world}. Since videos represent the world in spatiotemporal dimensions and temporal changes are inherently causal, we aim to preserve complete causality in the temporal dimension during generation. Using a block shape with $T=1$ avoids introducing bidirectional temporal attention, aligning with our philosophy of employing an autoregressive generator (a decoder-only transformer) and a tokenizer like MagVITv2 with $T=1$ as the temporal unit. Results in Table~\ref{tab:block_shape} confirm that the block shape with $T=1$ achieve superior model performance.


\paragraph{(4) Inference Process.}
To illustrate the inference process of next-block prediction, we consider a scenario where each block corresponds to a clip. As shown in the right panel of Fig.~\ref{fig:framework}, during inference, the last $L$ tokens of the current output represent the predicted tokens for the next block.
These tokens are retained and concatenated with clip prefix, forming the input for the next step.  
By transitioning from token-by-token to block-by-block prediction, the \modelname framework leverages parallelization, reducing the number of generation steps by a factor of $L$, thereby decreasing computational cost and accelerating inference. 


% Based on the above designs, we summary the training and inference features of our semi-AR framework: 
% \paragraph{Training Dynamics.}
% % follow rvq
% % compared to ar
% % compared to mar, dense supervised signal
% Though there exists other work prediction multiple tokens per step, e.g., MAR~\citep{Li2024AutoregressiveIG}
% With block-level prediction, each training step provides a "denser" supervision signal, meaning the model receives feedback on multiple tokens simultaneously. This leads to more efficient learning, as the model updates based on richer information at each step, improving convergence during training.


\section{Experiments}
\subsection{Experimental Setups}


\paragraph{Video Tokenizer.}
As MAGVITv2 is not open-sourced, we implemented it based on the original paper. In contrast to the official implementation, which utilizes LFQ~\citep{yu2023language} as its quantizer, we adopt FSQ~\citep{Mentzer2023FiniteSQ} due to its simplicity and reduced number of loss functions and hyper-parameters. Following the original paper's recommendations, we set the FSQ levels to $[8, 8, 8, 5, 5, 5]$, and the size of the visual vocabulary is 64K. 
Moreover, we employ PatchGAN~\citep{Isola2016ImagetoImageTW} instead of StyleGAN~\citep{Karras2018ASG} to enhance training stability.  
The reconstruction performance of our tokenizer is presented in Table~\ref{tab:video_reconstruction}, and additional training details are available in Appendix~\ref{app:model}. We note that MAGVITv2 is not open-sourced, we have made every effort to replicate its results. 
Our tokenizer surpasses OmniTokenizer~\cite{Wang2024OmniTokenizerAJ}, MAGVITv1~\cite{yu2023magvit}, and other models in performance. However, due to limited computational resources, we did not pre-train on ImageNet~\citep{Russakovsky2014ImageNetLS} or employ a larger visual vocabulary (e.g., 262K as in the original MAGVITv2), which slightly impacts our results compared to the official MAGVITv2. 
Nevertheless, we note that the primary objective of this paper is to validate the semi-AR framework, rather than to achieve state-of-the-art tokenizer performance.


\paragraph{Generator Training Details.}
We train decoder-only transformers on 17-frame videos with a resolution of 128$\times$128, using the UCF-101~\citep{Soomro2012UCF101AD} and K600~\citep{Carreira2018ASN} datasets. 
With spatial downsampling factors of $F_H=F_W=8$ and temporal downsampling of $F_T=4$, the resulting 3D token map for each video sample has dimensions $(T', H', W')=(5, 16, 16)$, yielding a total of 1280 tokens. 
We train our model for 100K steps with a total batch size of 256 and 64 respectively. 
Model sizes range from 700M to 3B parameters, with training spanning approximately two weeks on 32 NVIDIA A100 GPUs. The full model configuration and training hyper-parameters are provided in Appendix~\ref{app:model}. 
We train the models from scratch, rather than initializing from a pre-trained LLM checkpoint, as these text-based checkpoints provide minimal benefit for video generation~\citep{zhang2023pre}. 
We use LLaMA~\citep{Touvron2023LLaMAOA} vocabulary (32K tokens) as the text vocabulary and merge it with the video vocabulary (64K tokens) to form the final vocabulary. Since our primary focus is video generation, we compute the loss only on video tokens, which leads to improved performance. 

\begin{table*}[]
\centering
\caption{Comparison of next-token prediction (NTP) and next-block prediction (\modelname) models in terms of performance and speed, evaluated on the K600 dataset (5-frame condition, 12 frames (768 tokens) to predict). Inference time was measured on a single A100 Nvidia GPU. All models are implemented by us under the same setting and trained for 20 epochs. FPS denotes ``frame per second''. The measurement of inference speed includes tokenization and de-tokenization processes. KV-cache is used for both models.}
\label{table:semiar-ar-scale}
% \setlength{\tabcolsep}{2.0pt}
% \vspace{0.04in}
\begin{adjustbox}{max width=\linewidth}

\begin{tabular}{@{}c|l|c|c|cc@{}}
\toprule
Model Size & Modeling Method & \# Block size & FVD $\downarrow$ & \# Forward steps & Inference speed (FPS) $\uparrow$ \\ \midrule
\multirow{2}{*}{700M}          & NTP     & 1 (1$\times$1$\times$1)   & 37.4 & 768           & 0.80                  \\
                               & \modelname (Ours)  & 16 (1$\times$1$\times$16) & \textbf{33.6} & \textbf{48}            & \textbf{8.89}                  \\ \midrule
\multirow{2}{*}{1.2B}          & NTP  & 1 (1$\times$1$\times$1)    & 31.4 & 768           & 0.75                  \\
                               & \modelname (Ours)  & 16 (1$\times$1$\times$16) & \textbf{28.6} & \textbf{48}            & \textbf{6.70}                  \\ \midrule
\multirow{2}{*}{3B}            & NTP  & 1 (1$\times$1$\times$1)    & 29.0 & 768           & 0.60                  \\
                               & \modelname (Ours)  & 16 (1$\times$1$\times$16)  & \textbf{26.5} & \textbf{48}            & \textbf{4.29}                  \\ \bottomrule
\end{tabular}


\end{adjustbox}
\end{table*}



\paragraph{Evaluation Protocol.}
We evaluate our models on the UCF-101 dataset for class-conditional generation task and the K600 dataset for frame prediction task. To assess video quality, we use the standard metric of Fréchet Video Distance (FVD)\cite{unterthiner2018towards}. Additional evaluation details can be found in Appendix\ref{app:eval}.


% \begin{figure}
%     \centering
%     \includegraphics[width=.8\linewidth]{figs/k600_700M_1B_3B.pdf}
% \caption{Validation loss of various sizes of semi-AR models from 700M to 3B. \rsh{change loss curve}}
% \label{fig:model_para}
% \end{figure}

% \begin{figure}[htbp]
% \includegraphics[width=\linewidth]{figs/k600_block_size.pdf}
% \caption{Training loss of various block sizes from 1 to 256. }
% \label{fig:block_size}
% \end{figure}


\begin{figure}[htbp]
    \centering
    \begin{minipage}[b]{0.49\linewidth}
        \centering
        \includegraphics[width=\linewidth]{figs/k600_700M_1B_3B_500k.pdf}
        \caption{Validation loss of various sizes of semi-AR models from 700M to 3B. %\rsh{change loss curve}
        }
\label{fig:model_para}
    \end{minipage}
    \hfill
    \begin{minipage}[b]{0.49\linewidth}
        \centering
        \includegraphics[width=\linewidth]{figs/k600_block_size_500k.pdf}
        \caption{Validation loss of various block sizes from 1 to 256. }
        \label{fig:block_size}
    \end{minipage}
\end{figure}

\subsection{Comparison of Next-Token Prediction and Next-Block Prediction}

We first conduct a fair comparison between next-token prediction (NTP) and our next-block prediction (\modelname) under the same experimental setting. 
All experiments are performed on the K600 dataset, which has a much larger data volume compared to UCF-101 (413K vs. 9.5K) and features a strict training-test split, thereby ensuring more generalizable results. 
Table~\ref{table:semiar-ar-scale} highlights the superiority of our approach in three key aspects: generation quality, inference efficiency, and scalability.

\paragraph{Generation Quality.}
Across all model sizes, \modelname with a 1$\times$1$\times$16 block size consistently outperforms NTP models in terms of generation quality (measured by FVD). For instance, the 700M \modelname model achieves an FVD of 33.6, outperforming the NTP model by 3.8 points. Furthermore, a \modelname model with only 1.2B parameters achieves a comparable performance to a 3B NTP model (28.6 vs. 29.0 FVD). This suggests that the block size of 1$\times$1$\times$16 is a more effective generation unit for autoregressive modeling in the video domain. 

% This must be in the first 5 lines to tell arXiv to use pdfLaTeX, which is strongly recommended.
\pdfoutput=1
% In particular, the hyperref package requires pdfLaTeX in order to break URLs across lines.

\documentclass[11pt]{article}

% Change "review" to "final" to generate the final (sometimes called camera-ready) version.
% Change to "preprint" to generate a non-anonymous version with page numbers.
\usepackage{acl}

% Standard package includes
\usepackage{times}
\usepackage{latexsym}

% Draw tables
\usepackage{booktabs}
\usepackage{multirow}
\usepackage{xcolor}
\usepackage{colortbl}
\usepackage{array} 
\usepackage{amsmath}

\newcolumntype{C}{>{\centering\arraybackslash}p{0.07\textwidth}}
% For proper rendering and hyphenation of words containing Latin characters (including in bib files)
\usepackage[T1]{fontenc}
% For Vietnamese characters
% \usepackage[T5]{fontenc}
% See https://www.latex-project.org/help/documentation/encguide.pdf for other character sets
% This assumes your files are encoded as UTF8
\usepackage[utf8]{inputenc}

% This is not strictly necessary, and may be commented out,
% but it will improve the layout of the manuscript,
% and will typically save some space.
\usepackage{microtype}
\DeclareMathOperator*{\argmax}{arg\,max}
% This is also not strictly necessary, and may be commented out.
% However, it will improve the aesthetics of text in
% the typewriter font.
\usepackage{inconsolata}

%Including images in your LaTeX document requires adding
%additional package(s)
\usepackage{graphicx}
% If the title and author information does not fit in the area allocated, uncomment the following
%
%\setlength\titlebox{<dim>}
%
% and set <dim> to something 5cm or larger.

\title{Wi-Chat: Large Language Model Powered Wi-Fi Sensing}

% Author information can be set in various styles:
% For several authors from the same institution:
% \author{Author 1 \and ... \and Author n \\
%         Address line \\ ... \\ Address line}
% if the names do not fit well on one line use
%         Author 1 \\ {\bf Author 2} \\ ... \\ {\bf Author n} \\
% For authors from different institutions:
% \author{Author 1 \\ Address line \\  ... \\ Address line
%         \And  ... \And
%         Author n \\ Address line \\ ... \\ Address line}
% To start a separate ``row'' of authors use \AND, as in
% \author{Author 1 \\ Address line \\  ... \\ Address line
%         \AND
%         Author 2 \\ Address line \\ ... \\ Address line \And
%         Author 3 \\ Address line \\ ... \\ Address line}

% \author{First Author \\
%   Affiliation / Address line 1 \\
%   Affiliation / Address line 2 \\
%   Affiliation / Address line 3 \\
%   \texttt{email@domain} \\\And
%   Second Author \\
%   Affiliation / Address line 1 \\
%   Affiliation / Address line 2 \\
%   Affiliation / Address line 3 \\
%   \texttt{email@domain} \\}
% \author{Haohan Yuan \qquad Haopeng Zhang\thanks{corresponding author} \\ 
%   ALOHA Lab, University of Hawaii at Manoa \\
%   % Affiliation / Address line 2 \\
%   % Affiliation / Address line 3 \\
%   \texttt{\{haohany,haopengz\}@hawaii.edu}}
  
\author{
{Haopeng Zhang$\dag$\thanks{These authors contributed equally to this work.}, Yili Ren$\ddagger$\footnotemark[1], Haohan Yuan$\dag$, Jingzhe Zhang$\ddagger$, Yitong Shen$\ddagger$} \\
ALOHA Lab, University of Hawaii at Manoa$\dag$, University of South Florida$\ddagger$ \\
\{haopengz, haohany\}@hawaii.edu\\
\{yiliren, jingzhe, shen202\}@usf.edu\\}



  
%\author{
%  \textbf{First Author\textsuperscript{1}},
%  \textbf{Second Author\textsuperscript{1,2}},
%  \textbf{Third T. Author\textsuperscript{1}},
%  \textbf{Fourth Author\textsuperscript{1}},
%\\
%  \textbf{Fifth Author\textsuperscript{1,2}},
%  \textbf{Sixth Author\textsuperscript{1}},
%  \textbf{Seventh Author\textsuperscript{1}},
%  \textbf{Eighth Author \textsuperscript{1,2,3,4}},
%\\
%  \textbf{Ninth Author\textsuperscript{1}},
%  \textbf{Tenth Author\textsuperscript{1}},
%  \textbf{Eleventh E. Author\textsuperscript{1,2,3,4,5}},
%  \textbf{Twelfth Author\textsuperscript{1}},
%\\
%  \textbf{Thirteenth Author\textsuperscript{3}},
%  \textbf{Fourteenth F. Author\textsuperscript{2,4}},
%  \textbf{Fifteenth Author\textsuperscript{1}},
%  \textbf{Sixteenth Author\textsuperscript{1}},
%\\
%  \textbf{Seventeenth S. Author\textsuperscript{4,5}},
%  \textbf{Eighteenth Author\textsuperscript{3,4}},
%  \textbf{Nineteenth N. Author\textsuperscript{2,5}},
%  \textbf{Twentieth Author\textsuperscript{1}}
%\\
%\\
%  \textsuperscript{1}Affiliation 1,
%  \textsuperscript{2}Affiliation 2,
%  \textsuperscript{3}Affiliation 3,
%  \textsuperscript{4}Affiliation 4,
%  \textsuperscript{5}Affiliation 5
%\\
%  \small{
%    \textbf{Correspondence:} \href{mailto:email@domain}{email@domain}
%  }
%}

\begin{document}
\maketitle
\begin{abstract}
Recent advancements in Large Language Models (LLMs) have demonstrated remarkable capabilities across diverse tasks. However, their potential to integrate physical model knowledge for real-world signal interpretation remains largely unexplored. In this work, we introduce Wi-Chat, the first LLM-powered Wi-Fi-based human activity recognition system. We demonstrate that LLMs can process raw Wi-Fi signals and infer human activities by incorporating Wi-Fi sensing principles into prompts. Our approach leverages physical model insights to guide LLMs in interpreting Channel State Information (CSI) data without traditional signal processing techniques. Through experiments on real-world Wi-Fi datasets, we show that LLMs exhibit strong reasoning capabilities, achieving zero-shot activity recognition. These findings highlight a new paradigm for Wi-Fi sensing, expanding LLM applications beyond conventional language tasks and enhancing the accessibility of wireless sensing for real-world deployments.
\end{abstract}

\section{Introduction}

In today’s rapidly evolving digital landscape, the transformative power of web technologies has redefined not only how services are delivered but also how complex tasks are approached. Web-based systems have become increasingly prevalent in risk control across various domains. This widespread adoption is due their accessibility, scalability, and ability to remotely connect various types of users. For example, these systems are used for process safety management in industry~\cite{kannan2016web}, safety risk early warning in urban construction~\cite{ding2013development}, and safe monitoring of infrastructural systems~\cite{repetto2018web}. Within these web-based risk management systems, the source search problem presents a huge challenge. Source search refers to the task of identifying the origin of a risky event, such as a gas leak and the emission point of toxic substances. This source search capability is crucial for effective risk management and decision-making.

Traditional approaches to implementing source search capabilities into the web systems often rely on solely algorithmic solutions~\cite{ristic2016study}. These methods, while relatively straightforward to implement, often struggle to achieve acceptable performances due to algorithmic local optima and complex unknown environments~\cite{zhao2020searching}. More recently, web crowdsourcing has emerged as a promising alternative for tackling the source search problem by incorporating human efforts in these web systems on-the-fly~\cite{zhao2024user}. This approach outsources the task of addressing issues encountered during the source search process to human workers, leveraging their capabilities to enhance system performance.

These solutions often employ a human-AI collaborative way~\cite{zhao2023leveraging} where algorithms handle exploration-exploitation and report the encountered problems while human workers resolve complex decision-making bottlenecks to help the algorithms getting rid of local deadlocks~\cite{zhao2022crowd}. Although effective, this paradigm suffers from two inherent limitations: increased operational costs from continuous human intervention, and slow response times of human workers due to sequential decision-making. These challenges motivate our investigation into developing autonomous systems that preserve human-like reasoning capabilities while reducing dependency on massive crowdsourced labor.

Furthermore, recent advancements in large language models (LLMs)~\cite{chang2024survey} and multi-modal LLMs (MLLMs)~\cite{huang2023chatgpt} have unveiled promising avenues for addressing these challenges. One clear opportunity involves the seamless integration of visual understanding and linguistic reasoning for robust decision-making in search tasks. However, whether large models-assisted source search is really effective and efficient for improving the current source search algorithms~\cite{ji2022source} remains unknown. \textit{To address the research gap, we are particularly interested in answering the following two research questions in this work:}

\textbf{\textit{RQ1: }}How can source search capabilities be integrated into web-based systems to support decision-making in time-sensitive risk management scenarios? 
% \sq{I mention ``time-sensitive'' here because I feel like we shall say something about the response time -- LLM has to be faster than humans}

\textbf{\textit{RQ2: }}How can MLLMs and LLMs enhance the effectiveness and efficiency of existing source search algorithms? 

% \textit{\textbf{RQ2:}} To what extent does the performance of large models-assisted search align with or approach the effectiveness of human-AI collaborative search? 

To answer the research questions, we propose a novel framework called Auto-\
S$^2$earch (\textbf{Auto}nomous \textbf{S}ource \textbf{Search}) and implement a prototype system that leverages advanced web technologies to simulate real-world conditions for zero-shot source search. Unlike traditional methods that rely on pre-defined heuristics or extensive human intervention, AutoS$^2$earch employs a carefully designed prompt that encapsulates human rationales, thereby guiding the MLLM to generate coherent and accurate scene descriptions from visual inputs about four directional choices. Based on these language-based descriptions, the LLM is enabled to determine the optimal directional choice through chain-of-thought (CoT) reasoning. Comprehensive empirical validation demonstrates that AutoS$^2$-\ 
earch achieves a success rate of 95–98\%, closely approaching the performance of human-AI collaborative search across 20 benchmark scenarios~\cite{zhao2023leveraging}. 

Our work indicates that the role of humans in future web crowdsourcing tasks may evolve from executors to validators or supervisors. Furthermore, incorporating explanations of LLM decisions into web-based system interfaces has the potential to help humans enhance task performance in risk control.






\section{Related Work}
\label{sec:relatedworks}

% \begin{table*}[t]
% \centering 
% \renewcommand\arraystretch{0.98}
% \fontsize{8}{10}\selectfont \setlength{\tabcolsep}{0.4em}
% \begin{tabular}{@{}lc|cc|cc|cc@{}}
% \toprule
% \textbf{Methods}           & \begin{tabular}[c]{@{}c@{}}\textbf{Training}\\ \textbf{Paradigm}\end{tabular} & \begin{tabular}[c]{@{}c@{}}\textbf{$\#$ PT Data}\\ \textbf{(Tokens)}\end{tabular} & \begin{tabular}[c]{@{}c@{}}\textbf{$\#$ IFT Data}\\ \textbf{(Samples)}\end{tabular} & \textbf{Code}  & \begin{tabular}[c]{@{}c@{}}\textbf{Natural}\\ \textbf{Language}\end{tabular} & \begin{tabular}[c]{@{}c@{}}\textbf{Action}\\ \textbf{Trajectories}\end{tabular} & \begin{tabular}[c]{@{}c@{}}\textbf{API}\\ \textbf{Documentation}\end{tabular}\\ \midrule 
% NexusRaven~\citep{srinivasan2023nexusraven} & IFT & - & - & \textcolor{green}{\CheckmarkBold} & \textcolor{green}{\CheckmarkBold} &\textcolor{red}{\XSolidBrush}&\textcolor{red}{\XSolidBrush}\\
% AgentInstruct~\citep{zeng2023agenttuning} & IFT & - & 2k & \textcolor{green}{\CheckmarkBold} & \textcolor{green}{\CheckmarkBold} &\textcolor{red}{\XSolidBrush}&\textcolor{red}{\XSolidBrush} \\
% AgentEvol~\citep{xi2024agentgym} & IFT & - & 14.5k & \textcolor{green}{\CheckmarkBold} & \textcolor{green}{\CheckmarkBold} &\textcolor{green}{\CheckmarkBold}&\textcolor{red}{\XSolidBrush} \\
% Gorilla~\citep{patil2023gorilla}& IFT & - & 16k & \textcolor{green}{\CheckmarkBold} & \textcolor{green}{\CheckmarkBold} &\textcolor{red}{\XSolidBrush}&\textcolor{green}{\CheckmarkBold}\\
% OpenFunctions-v2~\citep{patil2023gorilla} & IFT & - & 65k & \textcolor{green}{\CheckmarkBold} & \textcolor{green}{\CheckmarkBold} &\textcolor{red}{\XSolidBrush}&\textcolor{green}{\CheckmarkBold}\\
% LAM~\citep{zhang2024agentohana} & IFT & - & 42.6k & \textcolor{green}{\CheckmarkBold} & \textcolor{green}{\CheckmarkBold} &\textcolor{green}{\CheckmarkBold}&\textcolor{red}{\XSolidBrush} \\
% xLAM~\citep{liu2024apigen} & IFT & - & 60k & \textcolor{green}{\CheckmarkBold} & \textcolor{green}{\CheckmarkBold} &\textcolor{green}{\CheckmarkBold}&\textcolor{red}{\XSolidBrush} \\\midrule
% LEMUR~\citep{xu2024lemur} & PT & 90B & 300k & \textcolor{green}{\CheckmarkBold} & \textcolor{green}{\CheckmarkBold} &\textcolor{green}{\CheckmarkBold}&\textcolor{red}{\XSolidBrush}\\
% \rowcolor{teal!12} \method & PT & 103B & 95k & \textcolor{green}{\CheckmarkBold} & \textcolor{green}{\CheckmarkBold} & \textcolor{green}{\CheckmarkBold} & \textcolor{green}{\CheckmarkBold} \\
% \bottomrule
% \end{tabular}
% \caption{Summary of existing tuning- and pretraining-based LLM agents with their training sample sizes. "PT" and "IFT" denote "Pre-Training" and "Instruction Fine-Tuning", respectively. }
% \label{tab:related}
% \end{table*}

\begin{table*}[ht]
\begin{threeparttable}
\centering 
\renewcommand\arraystretch{0.98}
\fontsize{7}{9}\selectfont \setlength{\tabcolsep}{0.2em}
\begin{tabular}{@{}l|c|c|ccc|cc|cc|cccc@{}}
\toprule
\textbf{Methods} & \textbf{Datasets}           & \begin{tabular}[c]{@{}c@{}}\textbf{Training}\\ \textbf{Paradigm}\end{tabular} & \begin{tabular}[c]{@{}c@{}}\textbf{\# PT Data}\\ \textbf{(Tokens)}\end{tabular} & \begin{tabular}[c]{@{}c@{}}\textbf{\# IFT Data}\\ \textbf{(Samples)}\end{tabular} & \textbf{\# APIs} & \textbf{Code}  & \begin{tabular}[c]{@{}c@{}}\textbf{Nat.}\\ \textbf{Lang.}\end{tabular} & \begin{tabular}[c]{@{}c@{}}\textbf{Action}\\ \textbf{Traj.}\end{tabular} & \begin{tabular}[c]{@{}c@{}}\textbf{API}\\ \textbf{Doc.}\end{tabular} & \begin{tabular}[c]{@{}c@{}}\textbf{Func.}\\ \textbf{Call}\end{tabular} & \begin{tabular}[c]{@{}c@{}}\textbf{Multi.}\\ \textbf{Step}\end{tabular}  & \begin{tabular}[c]{@{}c@{}}\textbf{Plan}\\ \textbf{Refine}\end{tabular}  & \begin{tabular}[c]{@{}c@{}}\textbf{Multi.}\\ \textbf{Turn}\end{tabular}\\ \midrule 
\multicolumn{13}{l}{\emph{Instruction Finetuning-based LLM Agents for Intrinsic Reasoning}}  \\ \midrule
FireAct~\cite{chen2023fireact} & FireAct & IFT & - & 2.1K & 10 & \textcolor{red}{\XSolidBrush} &\textcolor{green}{\CheckmarkBold} &\textcolor{green}{\CheckmarkBold}  & \textcolor{red}{\XSolidBrush} &\textcolor{green}{\CheckmarkBold} & \textcolor{red}{\XSolidBrush} &\textcolor{green}{\CheckmarkBold} & \textcolor{red}{\XSolidBrush} \\
ToolAlpaca~\cite{tang2023toolalpaca} & ToolAlpaca & IFT & - & 4.0K & 400 & \textcolor{red}{\XSolidBrush} &\textcolor{green}{\CheckmarkBold} &\textcolor{green}{\CheckmarkBold} & \textcolor{red}{\XSolidBrush} &\textcolor{green}{\CheckmarkBold} & \textcolor{red}{\XSolidBrush}  &\textcolor{green}{\CheckmarkBold} & \textcolor{red}{\XSolidBrush}  \\
ToolLLaMA~\cite{qin2023toolllm} & ToolBench & IFT & - & 12.7K & 16,464 & \textcolor{red}{\XSolidBrush} &\textcolor{green}{\CheckmarkBold} &\textcolor{green}{\CheckmarkBold} &\textcolor{red}{\XSolidBrush} &\textcolor{green}{\CheckmarkBold}&\textcolor{green}{\CheckmarkBold}&\textcolor{green}{\CheckmarkBold} &\textcolor{green}{\CheckmarkBold}\\
AgentEvol~\citep{xi2024agentgym} & AgentTraj-L & IFT & - & 14.5K & 24 &\textcolor{red}{\XSolidBrush} & \textcolor{green}{\CheckmarkBold} &\textcolor{green}{\CheckmarkBold}&\textcolor{red}{\XSolidBrush} &\textcolor{green}{\CheckmarkBold}&\textcolor{red}{\XSolidBrush} &\textcolor{red}{\XSolidBrush} &\textcolor{green}{\CheckmarkBold}\\
Lumos~\cite{yin2024agent} & Lumos & IFT  & - & 20.0K & 16 &\textcolor{red}{\XSolidBrush} & \textcolor{green}{\CheckmarkBold} & \textcolor{green}{\CheckmarkBold} &\textcolor{red}{\XSolidBrush} & \textcolor{green}{\CheckmarkBold} & \textcolor{green}{\CheckmarkBold} &\textcolor{red}{\XSolidBrush} & \textcolor{green}{\CheckmarkBold}\\
Agent-FLAN~\cite{chen2024agent} & Agent-FLAN & IFT & - & 24.7K & 20 &\textcolor{red}{\XSolidBrush} & \textcolor{green}{\CheckmarkBold} & \textcolor{green}{\CheckmarkBold} &\textcolor{red}{\XSolidBrush} & \textcolor{green}{\CheckmarkBold}& \textcolor{green}{\CheckmarkBold}&\textcolor{red}{\XSolidBrush} & \textcolor{green}{\CheckmarkBold}\\
AgentTuning~\citep{zeng2023agenttuning} & AgentInstruct & IFT & - & 35.0K & - &\textcolor{red}{\XSolidBrush} & \textcolor{green}{\CheckmarkBold} & \textcolor{green}{\CheckmarkBold} &\textcolor{red}{\XSolidBrush} & \textcolor{green}{\CheckmarkBold} &\textcolor{red}{\XSolidBrush} &\textcolor{red}{\XSolidBrush} & \textcolor{green}{\CheckmarkBold}\\\midrule
\multicolumn{13}{l}{\emph{Instruction Finetuning-based LLM Agents for Function Calling}} \\\midrule
NexusRaven~\citep{srinivasan2023nexusraven} & NexusRaven & IFT & - & - & 116 & \textcolor{green}{\CheckmarkBold} & \textcolor{green}{\CheckmarkBold}  & \textcolor{green}{\CheckmarkBold} &\textcolor{red}{\XSolidBrush} & \textcolor{green}{\CheckmarkBold} &\textcolor{red}{\XSolidBrush} &\textcolor{red}{\XSolidBrush}&\textcolor{red}{\XSolidBrush}\\
Gorilla~\citep{patil2023gorilla} & Gorilla & IFT & - & 16.0K & 1,645 & \textcolor{green}{\CheckmarkBold} &\textcolor{red}{\XSolidBrush} &\textcolor{red}{\XSolidBrush}&\textcolor{green}{\CheckmarkBold} &\textcolor{green}{\CheckmarkBold} &\textcolor{red}{\XSolidBrush} &\textcolor{red}{\XSolidBrush} &\textcolor{red}{\XSolidBrush}\\
OpenFunctions-v2~\citep{patil2023gorilla} & OpenFunctions-v2 & IFT & - & 65.0K & - & \textcolor{green}{\CheckmarkBold} & \textcolor{green}{\CheckmarkBold} &\textcolor{red}{\XSolidBrush} &\textcolor{green}{\CheckmarkBold} &\textcolor{green}{\CheckmarkBold} &\textcolor{red}{\XSolidBrush} &\textcolor{red}{\XSolidBrush} &\textcolor{red}{\XSolidBrush}\\
API Pack~\cite{guo2024api} & API Pack & IFT & - & 1.1M & 11,213 &\textcolor{green}{\CheckmarkBold} &\textcolor{red}{\XSolidBrush} &\textcolor{green}{\CheckmarkBold} &\textcolor{red}{\XSolidBrush} &\textcolor{green}{\CheckmarkBold} &\textcolor{red}{\XSolidBrush}&\textcolor{red}{\XSolidBrush}&\textcolor{red}{\XSolidBrush}\\ 
LAM~\citep{zhang2024agentohana} & AgentOhana & IFT & - & 42.6K & - & \textcolor{green}{\CheckmarkBold} & \textcolor{green}{\CheckmarkBold} &\textcolor{green}{\CheckmarkBold}&\textcolor{red}{\XSolidBrush} &\textcolor{green}{\CheckmarkBold}&\textcolor{red}{\XSolidBrush}&\textcolor{green}{\CheckmarkBold}&\textcolor{green}{\CheckmarkBold}\\
xLAM~\citep{liu2024apigen} & APIGen & IFT & - & 60.0K & 3,673 & \textcolor{green}{\CheckmarkBold} & \textcolor{green}{\CheckmarkBold} &\textcolor{green}{\CheckmarkBold}&\textcolor{red}{\XSolidBrush} &\textcolor{green}{\CheckmarkBold}&\textcolor{red}{\XSolidBrush}&\textcolor{green}{\CheckmarkBold}&\textcolor{green}{\CheckmarkBold}\\\midrule
\multicolumn{13}{l}{\emph{Pretraining-based LLM Agents}}  \\\midrule
% LEMUR~\citep{xu2024lemur} & PT & 90B & 300.0K & - & \textcolor{green}{\CheckmarkBold} & \textcolor{green}{\CheckmarkBold} &\textcolor{green}{\CheckmarkBold}&\textcolor{red}{\XSolidBrush} & \textcolor{red}{\XSolidBrush} &\textcolor{green}{\CheckmarkBold} &\textcolor{red}{\XSolidBrush}&\textcolor{red}{\XSolidBrush}\\
\rowcolor{teal!12} \method & \dataset & PT & 103B & 95.0K  & 76,537  & \textcolor{green}{\CheckmarkBold} & \textcolor{green}{\CheckmarkBold} & \textcolor{green}{\CheckmarkBold} & \textcolor{green}{\CheckmarkBold} & \textcolor{green}{\CheckmarkBold} & \textcolor{green}{\CheckmarkBold} & \textcolor{green}{\CheckmarkBold} & \textcolor{green}{\CheckmarkBold}\\
\bottomrule
\end{tabular}
% \begin{tablenotes}
%     \item $^*$ In addition, the StarCoder-API can offer 4.77M more APIs.
% \end{tablenotes}
\caption{Summary of existing instruction finetuning-based LLM agents for intrinsic reasoning and function calling, along with their training resources and sample sizes. "PT" and "IFT" denote "Pre-Training" and "Instruction Fine-Tuning", respectively.}
\vspace{-2ex}
\label{tab:related}
\end{threeparttable}
\end{table*}

\noindent \textbf{Prompting-based LLM Agents.} Due to the lack of agent-specific pre-training corpus, existing LLM agents rely on either prompt engineering~\cite{hsieh2023tool,lu2024chameleon,yao2022react,wang2023voyager} or instruction fine-tuning~\cite{chen2023fireact,zeng2023agenttuning} to understand human instructions, decompose high-level tasks, generate grounded plans, and execute multi-step actions. 
However, prompting-based methods mainly depend on the capabilities of backbone LLMs (usually commercial LLMs), failing to introduce new knowledge and struggling to generalize to unseen tasks~\cite{sun2024adaplanner,zhuang2023toolchain}. 

\noindent \textbf{Instruction Finetuning-based LLM Agents.} Considering the extensive diversity of APIs and the complexity of multi-tool instructions, tool learning inherently presents greater challenges than natural language tasks, such as text generation~\cite{qin2023toolllm}.
Post-training techniques focus more on instruction following and aligning output with specific formats~\cite{patil2023gorilla,hao2024toolkengpt,qin2023toolllm,schick2024toolformer}, rather than fundamentally improving model knowledge or capabilities. 
Moreover, heavy fine-tuning can hinder generalization or even degrade performance in non-agent use cases, potentially suppressing the original base model capabilities~\cite{ghosh2024a}.

\noindent \textbf{Pretraining-based LLM Agents.} While pre-training serves as an essential alternative, prior works~\cite{nijkamp2023codegen,roziere2023code,xu2024lemur,patil2023gorilla} have primarily focused on improving task-specific capabilities (\eg, code generation) instead of general-domain LLM agents, due to single-source, uni-type, small-scale, and poor-quality pre-training data. 
Existing tool documentation data for agent training either lacks diverse real-world APIs~\cite{patil2023gorilla, tang2023toolalpaca} or is constrained to single-tool or single-round tool execution. 
Furthermore, trajectory data mostly imitate expert behavior or follow function-calling rules with inferior planning and reasoning, failing to fully elicit LLMs' capabilities and handle complex instructions~\cite{qin2023toolllm}. 
Given a wide range of candidate API functions, each comprising various function names and parameters available at every planning step, identifying globally optimal solutions and generalizing across tasks remains highly challenging.



\section{Preliminaries}
\label{Preliminaries}
\begin{figure*}[t]
    \centering
    \includegraphics[width=0.95\linewidth]{fig/HealthGPT_Framework.png}
    \caption{The \ourmethod{} architecture integrates hierarchical visual perception and H-LoRA, employing a task-specific hard router to select visual features and H-LoRA plugins, ultimately generating outputs with an autoregressive manner.}
    \label{fig:architecture}
\end{figure*}
\noindent\textbf{Large Vision-Language Models.} 
The input to a LVLM typically consists of an image $x^{\text{img}}$ and a discrete text sequence $x^{\text{txt}}$. The visual encoder $\mathcal{E}^{\text{img}}$ converts the input image $x^{\text{img}}$ into a sequence of visual tokens $\mathcal{V} = [v_i]_{i=1}^{N_v}$, while the text sequence $x^{\text{txt}}$ is mapped into a sequence of text tokens $\mathcal{T} = [t_i]_{i=1}^{N_t}$ using an embedding function $\mathcal{E}^{\text{txt}}$. The LLM $\mathcal{M_\text{LLM}}(\cdot|\theta)$ models the joint probability of the token sequence $\mathcal{U} = \{\mathcal{V},\mathcal{T}\}$, which is expressed as:
\begin{equation}
    P_\theta(R | \mathcal{U}) = \prod_{i=1}^{N_r} P_\theta(r_i | \{\mathcal{U}, r_{<i}\}),
\end{equation}
where $R = [r_i]_{i=1}^{N_r}$ is the text response sequence. The LVLM iteratively generates the next token $r_i$ based on $r_{<i}$. The optimization objective is to minimize the cross-entropy loss of the response $\mathcal{R}$.
% \begin{equation}
%     \mathcal{L}_{\text{VLM}} = \mathbb{E}_{R|\mathcal{U}}\left[-\log P_\theta(R | \mathcal{U})\right]
% \end{equation}
It is worth noting that most LVLMs adopt a design paradigm based on ViT, alignment adapters, and pre-trained LLMs\cite{liu2023llava,liu2024improved}, enabling quick adaptation to downstream tasks.


\noindent\textbf{VQGAN.}
VQGAN~\cite{esser2021taming} employs latent space compression and indexing mechanisms to effectively learn a complete discrete representation of images. VQGAN first maps the input image $x^{\text{img}}$ to a latent representation $z = \mathcal{E}(x)$ through a encoder $\mathcal{E}$. Then, the latent representation is quantized using a codebook $\mathcal{Z} = \{z_k\}_{k=1}^K$, generating a discrete index sequence $\mathcal{I} = [i_m]_{m=1}^N$, where $i_m \in \mathcal{Z}$ represents the quantized code index:
\begin{equation}
    \mathcal{I} = \text{Quantize}(z|\mathcal{Z}) = \arg\min_{z_k \in \mathcal{Z}} \| z - z_k \|_2.
\end{equation}
In our approach, the discrete index sequence $\mathcal{I}$ serves as a supervisory signal for the generation task, enabling the model to predict the index sequence $\hat{\mathcal{I}}$ from input conditions such as text or other modality signals.  
Finally, the predicted index sequence $\hat{\mathcal{I}}$ is upsampled by the VQGAN decoder $G$, generating the high-quality image $\hat{x}^\text{img} = G(\hat{\mathcal{I}})$.



\noindent\textbf{Low Rank Adaptation.} 
LoRA\cite{hu2021lora} effectively captures the characteristics of downstream tasks by introducing low-rank adapters. The core idea is to decompose the bypass weight matrix $\Delta W\in\mathbb{R}^{d^{\text{in}} \times d^{\text{out}}}$ into two low-rank matrices $ \{A \in \mathbb{R}^{d^{\text{in}} \times r}, B \in \mathbb{R}^{r \times d^{\text{out}}} \}$, where $ r \ll \min\{d^{\text{in}}, d^{\text{out}}\} $, significantly reducing learnable parameters. The output with the LoRA adapter for the input $x$ is then given by:
\begin{equation}
    h = x W_0 + \alpha x \Delta W/r = x W_0 + \alpha xAB/r,
\end{equation}
where matrix $ A $ is initialized with a Gaussian distribution, while the matrix $ B $ is initialized as a zero matrix. The scaling factor $ \alpha/r $ controls the impact of $ \Delta W $ on the model.

\section{HealthGPT}
\label{Method}


\subsection{Unified Autoregressive Generation.}  
% As shown in Figure~\ref{fig:architecture}, 
\ourmethod{} (Figure~\ref{fig:architecture}) utilizes a discrete token representation that covers both text and visual outputs, unifying visual comprehension and generation as an autoregressive task. 
For comprehension, $\mathcal{M}_\text{llm}$ receives the input joint sequence $\mathcal{U}$ and outputs a series of text token $\mathcal{R} = [r_1, r_2, \dots, r_{N_r}]$, where $r_i \in \mathcal{V}_{\text{txt}}$, and $\mathcal{V}_{\text{txt}}$ represents the LLM's vocabulary:
\begin{equation}
    P_\theta(\mathcal{R} \mid \mathcal{U}) = \prod_{i=1}^{N_r} P_\theta(r_i \mid \mathcal{U}, r_{<i}).
\end{equation}
For generation, $\mathcal{M}_\text{llm}$ first receives a special start token $\langle \text{START\_IMG} \rangle$, then generates a series of tokens corresponding to the VQGAN indices $\mathcal{I} = [i_1, i_2, \dots, i_{N_i}]$, where $i_j \in \mathcal{V}_{\text{vq}}$, and $\mathcal{V}_{\text{vq}}$ represents the index range of VQGAN. Upon completion of generation, the LLM outputs an end token $\langle \text{END\_IMG} \rangle$:
\begin{equation}
    P_\theta(\mathcal{I} \mid \mathcal{U}) = \prod_{j=1}^{N_i} P_\theta(i_j \mid \mathcal{U}, i_{<j}).
\end{equation}
Finally, the generated index sequence $\mathcal{I}$ is fed into the decoder $G$, which reconstructs the target image $\hat{x}^{\text{img}} = G(\mathcal{I})$.

\subsection{Hierarchical Visual Perception}  
Given the differences in visual perception between comprehension and generation tasks—where the former focuses on abstract semantics and the latter emphasizes complete semantics—we employ ViT to compress the image into discrete visual tokens at multiple hierarchical levels.
Specifically, the image is converted into a series of features $\{f_1, f_2, \dots, f_L\}$ as it passes through $L$ ViT blocks.

To address the needs of various tasks, the hidden states are divided into two types: (i) \textit{Concrete-grained features} $\mathcal{F}^{\text{Con}} = \{f_1, f_2, \dots, f_k\}, k < L$, derived from the shallower layers of ViT, containing sufficient global features, suitable for generation tasks; 
(ii) \textit{Abstract-grained features} $\mathcal{F}^{\text{Abs}} = \{f_{k+1}, f_{k+2}, \dots, f_L\}$, derived from the deeper layers of ViT, which contain abstract semantic information closer to the text space, suitable for comprehension tasks.

The task type $T$ (comprehension or generation) determines which set of features is selected as the input for the downstream large language model:
\begin{equation}
    \mathcal{F}^{\text{img}}_T =
    \begin{cases}
        \mathcal{F}^{\text{Con}}, & \text{if } T = \text{generation task} \\
        \mathcal{F}^{\text{Abs}}, & \text{if } T = \text{comprehension task}
    \end{cases}
\end{equation}
We integrate the image features $\mathcal{F}^{\text{img}}_T$ and text features $\mathcal{T}$ into a joint sequence through simple concatenation, which is then fed into the LLM $\mathcal{M}_{\text{llm}}$ for autoregressive generation.
% :
% \begin{equation}
%     \mathcal{R} = \mathcal{M}_{\text{llm}}(\mathcal{U}|\theta), \quad \mathcal{U} = [\mathcal{F}^{\text{img}}_T; \mathcal{T}]
% \end{equation}
\subsection{Heterogeneous Knowledge Adaptation}
We devise H-LoRA, which stores heterogeneous knowledge from comprehension and generation tasks in separate modules and dynamically routes to extract task-relevant knowledge from these modules. 
At the task level, for each task type $ T $, we dynamically assign a dedicated H-LoRA submodule $ \theta^T $, which is expressed as:
\begin{equation}
    \mathcal{R} = \mathcal{M}_\text{LLM}(\mathcal{U}|\theta, \theta^T), \quad \theta^T = \{A^T, B^T, \mathcal{R}^T_\text{outer}\}.
\end{equation}
At the feature level for a single task, H-LoRA integrates the idea of Mixture of Experts (MoE)~\cite{masoudnia2014mixture} and designs an efficient matrix merging and routing weight allocation mechanism, thus avoiding the significant computational delay introduced by matrix splitting in existing MoELoRA~\cite{luo2024moelora}. Specifically, we first merge the low-rank matrices (rank = r) of $ k $ LoRA experts into a unified matrix:
\begin{equation}
    \mathbf{A}^{\text{merged}}, \mathbf{B}^{\text{merged}} = \text{Concat}(\{A_i\}_1^k), \text{Concat}(\{B_i\}_1^k),
\end{equation}
where $ \mathbf{A}^{\text{merged}} \in \mathbb{R}^{d^\text{in} \times rk} $ and $ \mathbf{B}^{\text{merged}} \in \mathbb{R}^{rk \times d^\text{out}} $. The $k$-dimension routing layer generates expert weights $ \mathcal{W} \in \mathbb{R}^{\text{token\_num} \times k} $ based on the input hidden state $ x $, and these are expanded to $ \mathbb{R}^{\text{token\_num} \times rk} $ as follows:
\begin{equation}
    \mathcal{W}^\text{expanded} = \alpha k \mathcal{W} / r \otimes \mathbf{1}_r,
\end{equation}
where $ \otimes $ denotes the replication operation.
The overall output of H-LoRA is computed as:
\begin{equation}
    \mathcal{O}^\text{H-LoRA} = (x \mathbf{A}^{\text{merged}} \odot \mathcal{W}^\text{expanded}) \mathbf{B}^{\text{merged}},
\end{equation}
where $ \odot $ represents element-wise multiplication. Finally, the output of H-LoRA is added to the frozen pre-trained weights to produce the final output:
\begin{equation}
    \mathcal{O} = x W_0 + \mathcal{O}^\text{H-LoRA}.
\end{equation}
% In summary, H-LoRA is a task-based dynamic PEFT method that achieves high efficiency in single-task fine-tuning.

\subsection{Training Pipeline}

\begin{figure}[t]
    \centering
    \hspace{-4mm}
    \includegraphics[width=0.94\linewidth]{fig/data.pdf}
    \caption{Data statistics of \texttt{VL-Health}. }
    \label{fig:data}
\end{figure}
\noindent \textbf{1st Stage: Multi-modal Alignment.} 
In the first stage, we design separate visual adapters and H-LoRA submodules for medical unified tasks. For the medical comprehension task, we train abstract-grained visual adapters using high-quality image-text pairs to align visual embeddings with textual embeddings, thereby enabling the model to accurately describe medical visual content. During this process, the pre-trained LLM and its corresponding H-LoRA submodules remain frozen. In contrast, the medical generation task requires training concrete-grained adapters and H-LoRA submodules while keeping the LLM frozen. Meanwhile, we extend the textual vocabulary to include multimodal tokens, enabling the support of additional VQGAN vector quantization indices. The model trains on image-VQ pairs, endowing the pre-trained LLM with the capability for image reconstruction. This design ensures pixel-level consistency of pre- and post-LVLM. The processes establish the initial alignment between the LLM’s outputs and the visual inputs.

\noindent \textbf{2nd Stage: Heterogeneous H-LoRA Plugin Adaptation.}  
The submodules of H-LoRA share the word embedding layer and output head but may encounter issues such as bias and scale inconsistencies during training across different tasks. To ensure that the multiple H-LoRA plugins seamlessly interface with the LLMs and form a unified base, we fine-tune the word embedding layer and output head using a small amount of mixed data to maintain consistency in the model weights. Specifically, during this stage, all H-LoRA submodules for different tasks are kept frozen, with only the word embedding layer and output head being optimized. Through this stage, the model accumulates foundational knowledge for unified tasks by adapting H-LoRA plugins.

\begin{table*}[!t]
\centering
\caption{Comparison of \ourmethod{} with other LVLMs and unified multi-modal models on medical visual comprehension tasks. \textbf{Bold} and \underline{underlined} text indicates the best performance and second-best performance, respectively.}
\resizebox{\textwidth}{!}{
\begin{tabular}{c|lcc|cccccccc|c}
\toprule
\rowcolor[HTML]{E9F3FE} &  &  &  & \multicolumn{2}{c}{\textbf{VQA-RAD \textuparrow}} & \multicolumn{2}{c}{\textbf{SLAKE \textuparrow}} & \multicolumn{2}{c}{\textbf{PathVQA \textuparrow}} &  &  &  \\ 
\cline{5-10}
\rowcolor[HTML]{E9F3FE}\multirow{-2}{*}{\textbf{Type}} & \multirow{-2}{*}{\textbf{Model}} & \multirow{-2}{*}{\textbf{\# Params}} & \multirow{-2}{*}{\makecell{\textbf{Medical} \\ \textbf{LVLM}}} & \textbf{close} & \textbf{all} & \textbf{close} & \textbf{all} & \textbf{close} & \textbf{all} & \multirow{-2}{*}{\makecell{\textbf{MMMU} \\ \textbf{-Med}}\textuparrow} & \multirow{-2}{*}{\textbf{OMVQA}\textuparrow} & \multirow{-2}{*}{\textbf{Avg. \textuparrow}} \\ 
\midrule \midrule
\multirow{9}{*}{\textbf{Comp. Only}} 
& Med-Flamingo & 8.3B & \Large \ding{51} & 58.6 & 43.0 & 47.0 & 25.5 & 61.9 & 31.3 & 28.7 & 34.9 & 41.4 \\
& LLaVA-Med & 7B & \Large \ding{51} & 60.2 & 48.1 & 58.4 & 44.8 & 62.3 & 35.7 & 30.0 & 41.3 & 47.6 \\
& HuatuoGPT-Vision & 7B & \Large \ding{51} & 66.9 & 53.0 & 59.8 & 49.1 & 52.9 & 32.0 & 42.0 & 50.0 & 50.7 \\
& BLIP-2 & 6.7B & \Large \ding{55} & 43.4 & 36.8 & 41.6 & 35.3 & 48.5 & 28.8 & 27.3 & 26.9 & 36.1 \\
& LLaVA-v1.5 & 7B & \Large \ding{55} & 51.8 & 42.8 & 37.1 & 37.7 & 53.5 & 31.4 & 32.7 & 44.7 & 41.5 \\
& InstructBLIP & 7B & \Large \ding{55} & 61.0 & 44.8 & 66.8 & 43.3 & 56.0 & 32.3 & 25.3 & 29.0 & 44.8 \\
& Yi-VL & 6B & \Large \ding{55} & 52.6 & 42.1 & 52.4 & 38.4 & 54.9 & 30.9 & 38.0 & 50.2 & 44.9 \\
& InternVL2 & 8B & \Large \ding{55} & 64.9 & 49.0 & 66.6 & 50.1 & 60.0 & 31.9 & \underline{43.3} & 54.5 & 52.5\\
& Llama-3.2 & 11B & \Large \ding{55} & 68.9 & 45.5 & 72.4 & 52.1 & 62.8 & 33.6 & 39.3 & 63.2 & 54.7 \\
\midrule
\multirow{5}{*}{\textbf{Comp. \& Gen.}} 
& Show-o & 1.3B & \Large \ding{55} & 50.6 & 33.9 & 31.5 & 17.9 & 52.9 & 28.2 & 22.7 & 45.7 & 42.6 \\
& Unified-IO 2 & 7B & \Large \ding{55} & 46.2 & 32.6 & 35.9 & 21.9 & 52.5 & 27.0 & 25.3 & 33.0 & 33.8 \\
& Janus & 1.3B & \Large \ding{55} & 70.9 & 52.8 & 34.7 & 26.9 & 51.9 & 27.9 & 30.0 & 26.8 & 33.5 \\
& \cellcolor[HTML]{DAE0FB}HealthGPT-M3 & \cellcolor[HTML]{DAE0FB}3.8B & \cellcolor[HTML]{DAE0FB}\Large \ding{51} & \cellcolor[HTML]{DAE0FB}\underline{73.7} & \cellcolor[HTML]{DAE0FB}\underline{55.9} & \cellcolor[HTML]{DAE0FB}\underline{74.6} & \cellcolor[HTML]{DAE0FB}\underline{56.4} & \cellcolor[HTML]{DAE0FB}\underline{78.7} & \cellcolor[HTML]{DAE0FB}\underline{39.7} & \cellcolor[HTML]{DAE0FB}\underline{43.3} & \cellcolor[HTML]{DAE0FB}\underline{68.5} & \cellcolor[HTML]{DAE0FB}\underline{61.3} \\
& \cellcolor[HTML]{DAE0FB}HealthGPT-L14 & \cellcolor[HTML]{DAE0FB}14B & \cellcolor[HTML]{DAE0FB}\Large \ding{51} & \cellcolor[HTML]{DAE0FB}\textbf{77.7} & \cellcolor[HTML]{DAE0FB}\textbf{58.3} & \cellcolor[HTML]{DAE0FB}\textbf{76.4} & \cellcolor[HTML]{DAE0FB}\textbf{64.5} & \cellcolor[HTML]{DAE0FB}\textbf{85.9} & \cellcolor[HTML]{DAE0FB}\textbf{44.4} & \cellcolor[HTML]{DAE0FB}\textbf{49.2} & \cellcolor[HTML]{DAE0FB}\textbf{74.4} & \cellcolor[HTML]{DAE0FB}\textbf{66.4} \\
\bottomrule
\end{tabular}
}
\label{tab:results}
\end{table*}
\begin{table*}[ht]
    \centering
    \caption{The experimental results for the four modality conversion tasks.}
    \resizebox{\textwidth}{!}{
    \begin{tabular}{l|ccc|ccc|ccc|ccc}
        \toprule
        \rowcolor[HTML]{E9F3FE} & \multicolumn{3}{c}{\textbf{CT to MRI (Brain)}} & \multicolumn{3}{c}{\textbf{CT to MRI (Pelvis)}} & \multicolumn{3}{c}{\textbf{MRI to CT (Brain)}} & \multicolumn{3}{c}{\textbf{MRI to CT (Pelvis)}} \\
        \cline{2-13}
        \rowcolor[HTML]{E9F3FE}\multirow{-2}{*}{\textbf{Model}}& \textbf{SSIM $\uparrow$} & \textbf{PSNR $\uparrow$} & \textbf{MSE $\downarrow$} & \textbf{SSIM $\uparrow$} & \textbf{PSNR $\uparrow$} & \textbf{MSE $\downarrow$} & \textbf{SSIM $\uparrow$} & \textbf{PSNR $\uparrow$} & \textbf{MSE $\downarrow$} & \textbf{SSIM $\uparrow$} & \textbf{PSNR $\uparrow$} & \textbf{MSE $\downarrow$} \\
        \midrule \midrule
        pix2pix & 71.09 & 32.65 & 36.85 & 59.17 & 31.02 & 51.91 & 78.79 & 33.85 & 28.33 & 72.31 & 32.98 & 36.19 \\
        CycleGAN & 54.76 & 32.23 & 40.56 & 54.54 & 30.77 & 55.00 & 63.75 & 31.02 & 52.78 & 50.54 & 29.89 & 67.78 \\
        BBDM & {71.69} & {32.91} & {34.44} & 57.37 & 31.37 & 48.06 & \textbf{86.40} & 34.12 & 26.61 & {79.26} & 33.15 & 33.60 \\
        Vmanba & 69.54 & 32.67 & 36.42 & {63.01} & {31.47} & {46.99} & 79.63 & 34.12 & 26.49 & 77.45 & 33.53 & 31.85 \\
        DiffMa & 71.47 & 32.74 & 35.77 & 62.56 & 31.43 & 47.38 & 79.00 & {34.13} & {26.45} & 78.53 & {33.68} & {30.51} \\
        \rowcolor[HTML]{DAE0FB}HealthGPT-M3 & \underline{79.38} & \underline{33.03} & \underline{33.48} & \underline{71.81} & \underline{31.83} & \underline{43.45} & {85.06} & \textbf{34.40} & \textbf{25.49} & \underline{84.23} & \textbf{34.29} & \textbf{27.99} \\
        \rowcolor[HTML]{DAE0FB}HealthGPT-L14 & \textbf{79.73} & \textbf{33.10} & \textbf{32.96} & \textbf{71.92} & \textbf{31.87} & \textbf{43.09} & \underline{85.31} & \underline{34.29} & \underline{26.20} & \textbf{84.96} & \underline{34.14} & \underline{28.13} \\
        \bottomrule
    \end{tabular}
    }
    \label{tab:conversion}
\end{table*}

\noindent \textbf{3rd Stage: Visual Instruction Fine-Tuning.}  
In the third stage, we introduce additional task-specific data to further optimize the model and enhance its adaptability to downstream tasks such as medical visual comprehension (e.g., medical QA, medical dialogues, and report generation) or generation tasks (e.g., super-resolution, denoising, and modality conversion). Notably, by this stage, the word embedding layer and output head have been fine-tuned, only the H-LoRA modules and adapter modules need to be trained. This strategy significantly improves the model's adaptability and flexibility across different tasks.


\section{Experiment}
\label{s:experiment}

\subsection{Data Description}
We evaluate our method on FI~\cite{you2016building}, Twitter\_LDL~\cite{yang2017learning} and Artphoto~\cite{machajdik2010affective}.
FI is a public dataset built from Flickr and Instagram, with 23,308 images and eight emotion categories, namely \textit{amusement}, \textit{anger}, \textit{awe},  \textit{contentment}, \textit{disgust}, \textit{excitement},  \textit{fear}, and \textit{sadness}. 
% Since images in FI are all copyrighted by law, some images are corrupted now, so we remove these samples and retain 21,828 images.
% T4SA contains images from Twitter, which are classified into three categories: \textit{positive}, \textit{neutral}, and \textit{negative}. In this paper, we adopt the base version of B-T4SA, which contains 470,586 images and provides text descriptions of the corresponding tweets.
Twitter\_LDL contains 10,045 images from Twitter, with the same eight categories as the FI dataset.
% 。
For these two datasets, they are randomly split into 80\%
training and 20\% testing set.
Artphoto contains 806 artistic photos from the DeviantArt website, which we use to further evaluate the zero-shot capability of our model.
% on the small-scale dataset.
% We construct and publicly release the first image sentiment analysis dataset containing metadata.
% 。

% Based on these datasets, we are the first to construct and publicly release metadata-enhanced image sentiment analysis datasets. These datasets include scenes, tags, descriptions, and corresponding confidence scores, and are available at this link for future research purposes.


% 
\begin{table}[t]
\centering
% \begin{center}
\caption{Overall performance of different models on FI and Twitter\_LDL datasets.}
\label{tab:cap1}
% \resizebox{\linewidth}{!}
{
\begin{tabular}{l|c|c|c|c}
\hline
\multirow{2}{*}{\textbf{Model}} & \multicolumn{2}{c|}{\textbf{FI}}  & \multicolumn{2}{c}{\textbf{Twitter\_LDL}} \\ \cline{2-5} 
  & \textbf{Accuracy} & \textbf{F1} & \textbf{Accuracy} & \textbf{F1}  \\ \hline
% (\rownumber)~AlexNet~\cite{krizhevsky2017imagenet}  & 58.13\% & 56.35\%  & 56.24\%& 55.02\%  \\ 
% (\rownumber)~VGG16~\cite{simonyan2014very}  & 63.75\%& 63.08\%  & 59.34\%& 59.02\%  \\ 
(\rownumber)~ResNet101~\cite{he2016deep} & 66.16\%& 65.56\%  & 62.02\% & 61.34\%  \\ 
(\rownumber)~CDA~\cite{han2023boosting} & 66.71\%& 65.37\%  & 64.14\% & 62.85\%  \\ 
(\rownumber)~CECCN~\cite{ruan2024color} & 67.96\%& 66.74\%  & 64.59\%& 64.72\% \\ 
(\rownumber)~EmoVIT~\cite{xie2024emovit} & 68.09\%& 67.45\%  & 63.12\% & 61.97\%  \\ 
(\rownumber)~ComLDL~\cite{zhang2022compound} & 68.83\%& 67.28\%  & 65.29\% & 63.12\%  \\ 
(\rownumber)~WSDEN~\cite{li2023weakly} & 69.78\%& 69.61\%  & 67.04\% & 65.49\% \\ 
(\rownumber)~ECWA~\cite{deng2021emotion} & 70.87\%& 69.08\%  & 67.81\% & 66.87\%  \\ 
(\rownumber)~EECon~\cite{yang2023exploiting} & 71.13\%& 68.34\%  & 64.27\%& 63.16\%  \\ 
(\rownumber)~MAM~\cite{zhang2024affective} & 71.44\%  & 70.83\% & 67.18\%  & 65.01\%\\ 
(\rownumber)~TGCA-PVT~\cite{chen2024tgca}   & 73.05\%  & 71.46\% & 69.87\%  & 68.32\% \\ 
(\rownumber)~OEAN~\cite{zhang2024object}   & 73.40\%  & 72.63\% & 70.52\%  & 69.47\% \\ \hline
(\rownumber)~\shortname  & \textbf{79.48\%} & \textbf{79.22\%} & \textbf{74.12\%} & \textbf{73.09\%} \\ \hline
\end{tabular}
}
\vspace{-6mm}
% \end{center}
\end{table}
% 

\subsection{Experiment Setting}
% \subsubsection{Model Setting.}
% 
\textbf{Model Setting:}
For feature representation, we set $k=10$ to select object tags, and adopt clip-vit-base-patch32 as the pre-trained model for unified feature representation.
Moreover, we empirically set $(d_e, d_h, d_k, d_s) = (512, 128, 16, 64)$, and set the classification class $L$ to 8.

% 

\textbf{Training Setting:}
To initialize the model, we set all weights such as $\boldsymbol{W}$ following the truncated normal distribution, and use AdamW optimizer with the learning rate of $1 \times 10^{-4}$.
% warmup scheduler of cosine, warmup steps of 2000.
Furthermore, we set the batch size to 32 and the epoch of the training process to 200.
During the implementation, we utilize \textit{PyTorch} to build our entire model.
% , and our project codes are publicly available at https://github.com/zzmyrep/MESN.
% Our project codes as well as data are all publicly available on GitHub\footnote{https://github.com/zzmyrep/KBCEN}.
% Code is available at \href{https://github.com/zzmyrep/KBCEN}{https://github.com/zzmyrep/KBCEN}.

\textbf{Evaluation Metrics:}
Following~\cite{zhang2024affective, chen2024tgca, zhang2024object}, we adopt \textit{accuracy} and \textit{F1} as our evaluation metrics to measure the performance of different methods for image sentiment analysis. 



\subsection{Experiment Result}
% We compare our model against the following baselines: AlexNet~\cite{krizhevsky2017imagenet}, VGG16~\cite{simonyan2014very}, ResNet101~\cite{he2016deep}, CECCN~\cite{ruan2024color}, EmoVIT~\cite{xie2024emovit}, WSCNet~\cite{yang2018weakly}, ECWA~\cite{deng2021emotion}, EECon~\cite{yang2023exploiting}, MAM~\cite{zhang2024affective} and TGCA-PVT~\cite{chen2024tgca}, and the overall results are summarized in Table~\ref{tab:cap1}.
We compare our model against several baselines, and the overall results are summarized in Table~\ref{tab:cap1}.
We observe that our model achieves the best performance in both accuracy and F1 metrics, significantly outperforming the previous models. 
This superior performance is mainly attributed to our effective utilization of metadata to enhance image sentiment analysis, as well as the exceptional capability of the unified sentiment transformer framework we developed. These results strongly demonstrate that our proposed method can bring encouraging performance for image sentiment analysis.

\setcounter{magicrownumbers}{0} 
\begin{table}[t]
\begin{center}
\caption{Ablation study of~\shortname~on FI dataset.} 
% \vspace{1mm}
\label{tab:cap2}
\resizebox{.9\linewidth}{!}
{
\begin{tabular}{lcc}
  \hline
  \textbf{Model} & \textbf{Accuracy} & \textbf{F1} \\
  \hline
  (\rownumber)~Ours (w/o vision) & 65.72\% & 64.54\% \\
  (\rownumber)~Ours (w/o text description) & 74.05\% & 72.58\% \\
  (\rownumber)~Ours (w/o object tag) & 77.45\% & 76.84\% \\
  (\rownumber)~Ours (w/o scene tag) & 78.47\% & 78.21\% \\
  \hline
  (\rownumber)~Ours (w/o unified embedding) & 76.41\% & 76.23\% \\
  (\rownumber)~Ours (w/o adaptive learning) & 76.83\% & 76.56\% \\
  (\rownumber)~Ours (w/o cross-modal fusion) & 76.85\% & 76.49\% \\
  \hline
  (\rownumber)~Ours  & \textbf{79.48\%} & \textbf{79.22\%} \\
  \hline
\end{tabular}
}
\end{center}
\vspace{-5mm}
\end{table}


\begin{figure}[t]
\centering
% \vspace{-2mm}
\includegraphics[width=0.42\textwidth]{fig/2dvisual-linux4-paper2.pdf}
\caption{Visualization of feature distribution on eight categories before (left) and after (right) model processing.}
% 
\label{fig:visualization}
\vspace{-5mm}
\end{figure}

\subsection{Ablation Performance}
In this subsection, we conduct an ablation study to examine which component is really important for performance improvement. The results are reported in Table~\ref{tab:cap2}.

For information utilization, we observe a significant decline in model performance when visual features are removed. Additionally, the performance of \shortname~decreases when different metadata are removed separately, which means that text description, object tag, and scene tag are all critical for image sentiment analysis.
Recalling the model architecture, we separately remove transformer layers of the unified representation module, the adaptive learning module, and the cross-modal fusion module, replacing them with MLPs of the same parameter scale.
In this way, we can observe varying degrees of decline in model performance, indicating that these modules are indispensable for our model to achieve better performance.

\subsection{Visualization}
% 


% % 开始使用minipage进行左右排列
% \begin{minipage}[t]{0.45\textwidth}  % 子图1宽度为45%
%     \centering
%     \includegraphics[width=\textwidth]{2dvisual.pdf}  % 插入图片
%     \captionof{figure}{Visualization of feature distribution.}  % 使用captionof添加图片标题
%     \label{fig:visualization}
% \end{minipage}


% \begin{figure}[t]
% \centering
% \vspace{-2mm}
% \includegraphics[width=0.45\textwidth]{fig/2dvisual.pdf}
% \caption{Visualization of feature distribution.}
% \label{fig:visualization}
% % \vspace{-4mm}
% \end{figure}

% \begin{figure}[t]
% \centering
% \vspace{-2mm}
% \includegraphics[width=0.45\textwidth]{fig/2dvisual-linux3-paper.pdf}
% \caption{Visualization of feature distribution.}
% \label{fig:visualization}
% % \vspace{-4mm}
% \end{figure}



\begin{figure}[tbp]   
\vspace{-4mm}
  \centering            
  \subfloat[Depth of adaptive learning layers]   
  {
    \label{fig:subfig1}\includegraphics[width=0.22\textwidth]{fig/fig_sensitivity-a5}
  }
  \subfloat[Depth of fusion layers]
  {
    % \label{fig:subfig2}\includegraphics[width=0.22\textwidth]{fig/fig_sensitivity-b2}
    \label{fig:subfig2}\includegraphics[width=0.22\textwidth]{fig/fig_sensitivity-b2-num.pdf}
  }
  \caption{Sensitivity study of \shortname~on different depth. }   
  \label{fig:fig_sensitivity}  
\vspace{-2mm}
\end{figure}

% \begin{figure}[htbp]
% \centerline{\includegraphics{2dvisual.pdf}}
% \caption{Visualization of feature distribution.}
% \label{fig:visualization}
% \end{figure}

% In Fig.~\ref{fig:visualization}, we use t-SNE~\cite{van2008visualizing} to reduce the dimension of data features for visualization, Figure in left represents the metadata features before model processing, the features are obtained by embedding through the CLIP model, and figure in right shows the features of the data after model processing, it can be observed that after the model processing, the data with different label categories fall in different regions in the space, therefore, we can conclude that the Therefore, we can conclude that the model can effectively utilize the information contained in the metadata and use it to guide the model for classification.

In Fig.~\ref{fig:visualization}, we use t-SNE~\cite{van2008visualizing} to reduce the dimension of data features for visualization.
The left figure shows metadata features before being processed by our model (\textit{i.e.}, embedded by CLIP), while the right shows the distribution of features after being processed by our model.
We can observe that after the model processing, data with the same label are closer to each other, while others are farther away.
Therefore, it shows that the model can effectively utilize the information contained in the metadata and use it to guide the classification process.

\subsection{Sensitivity Analysis}
% 
In this subsection, we conduct a sensitivity analysis to figure out the effect of different depth settings of adaptive learning layers and fusion layers. 
% In this subsection, we conduct a sensitivity analysis to figure out the effect of different depth settings on the model. 
% Fig.~\ref{fig:fig_sensitivity} presents the effect of different depth settings of adaptive learning layers and fusion layers. 
Taking Fig.~\ref{fig:fig_sensitivity} (a) as an example, the model performance improves with increasing depth, reaching the best performance at a depth of 4.
% Taking Fig.~\ref{fig:fig_sensitivity} (a) as an example, the performance of \shortname~improves with the increase of depth at first, reaching the best performance at a depth of 4.
When the depth continues to increase, the accuracy decreases to varying degrees.
Similar results can be observed in Fig.~\ref{fig:fig_sensitivity} (b).
Therefore, we set their depths to 4 and 6 respectively to achieve the best results.

% Through our experiments, we can observe that the effect of modifying these hyperparameters on the results of the experiments is very weak, and the surface model is not sensitive to the hyperparameters.


\subsection{Zero-shot Capability}
% 

% (1)~GCH~\cite{2010Analyzing} & 21.78\% & (5)~RA-DLNet~\cite{2020A} & 34.01\% \\ \hline
% (2)~WSCNet~\cite{2019WSCNet}  & 30.25\% & (6)~CECCN~\cite{ruan2024color} & 43.83\% \\ \hline
% (3)~PCNN~\cite{2015Robust} & 31.68\%  & (7)~EmoVIT~\cite{xie2024emovit} & 44.90\% \\ \hline
% (4)~AR~\cite{2018Visual} & 32.67\% & (8)~Ours (Zero-shot) & 47.83\% \\ \hline


\begin{table}[t]
\centering
\caption{Zero-shot capability of \shortname.}
\label{tab:cap3}
\resizebox{1\linewidth}{!}
{
\begin{tabular}{lc|lc}
\hline
\textbf{Model} & \textbf{Accuracy} & \textbf{Model} & \textbf{Accuracy} \\ \hline
(1)~WSCNet~\cite{2019WSCNet}  & 30.25\% & (5)~MAM~\cite{zhang2024affective} & 39.56\%  \\ \hline
(2)~AR~\cite{2018Visual} & 32.67\% & (6)~CECCN~\cite{ruan2024color} & 43.83\% \\ \hline
(3)~RA-DLNet~\cite{2020A} & 34.01\%  & (7)~EmoVIT~\cite{xie2024emovit} & 44.90\% \\ \hline
(4)~CDA~\cite{han2023boosting} & 38.64\% & (8)~Ours (Zero-shot) & 47.83\% \\ \hline
\end{tabular}
}
\vspace{-5mm}
\end{table}

% We use the model trained on the FI dataset to test on the artphoto dataset to verify the model's generalization ability as well as robustness to other distributed datasets.
% We can observe that the MESN model shows strong competitiveness in terms of accuracy when compared to other trained models, which suggests that the model has a good generalization ability in the OOD task.

To validate the model's generalization ability and robustness to other distributed datasets, we directly test the model trained on the FI dataset, without training on Artphoto. 
% As observed in Table 3, compared to other models trained on Artphoto, we achieve highly competitive zero-shot performance, indicating that the model has good generalization ability in out-of-distribution tasks.
From Table~\ref{tab:cap3}, we can observe that compared with other models trained on Artphoto, we achieve competitive zero-shot performance, which shows that the model has good generalization ability in out-of-distribution tasks.


%%%%%%%%%%%%
%  E2E     %
%%%%%%%%%%%%


\section{Conclusion}
In this paper, we introduced Wi-Chat, the first LLM-powered Wi-Fi-based human activity recognition system that integrates the reasoning capabilities of large language models with the sensing potential of wireless signals. Our experimental results on a self-collected Wi-Fi CSI dataset demonstrate the promising potential of LLMs in enabling zero-shot Wi-Fi sensing. These findings suggest a new paradigm for human activity recognition that does not rely on extensive labeled data. We hope future research will build upon this direction, further exploring the applications of LLMs in signal processing domains such as IoT, mobile sensing, and radar-based systems.

\section*{Limitations}
While our work represents the first attempt to leverage LLMs for processing Wi-Fi signals, it is a preliminary study focused on a relatively simple task: Wi-Fi-based human activity recognition. This choice allows us to explore the feasibility of LLMs in wireless sensing but also comes with certain limitations.

Our approach primarily evaluates zero-shot performance, which, while promising, may still lag behind traditional supervised learning methods in highly complex or fine-grained recognition tasks. Besides, our study is limited to a controlled environment with a self-collected dataset, and the generalizability of LLMs to diverse real-world scenarios with varying Wi-Fi conditions, environmental interference, and device heterogeneity remains an open question.

Additionally, we have yet to explore the full potential of LLMs in more advanced Wi-Fi sensing applications, such as fine-grained gesture recognition, occupancy detection, and passive health monitoring. Future work should investigate the scalability of LLM-based approaches, their robustness to domain shifts, and their integration with multimodal sensing techniques in broader IoT applications.


% Bibliography entries for the entire Anthology, followed by custom entries
%\bibliography{anthology,custom}
% Custom bibliography entries only
\bibliography{main}
\newpage
\appendix

\section{Experiment prompts}
\label{sec:prompt}
The prompts used in the LLM experiments are shown in the following Table~\ref{tab:prompts}.

\definecolor{titlecolor}{rgb}{0.9, 0.5, 0.1}
\definecolor{anscolor}{rgb}{0.2, 0.5, 0.8}
\definecolor{labelcolor}{HTML}{48a07e}
\begin{table*}[h]
	\centering
	
 % \vspace{-0.2cm}
	
	\begin{center}
		\begin{tikzpicture}[
				chatbox_inner/.style={rectangle, rounded corners, opacity=0, text opacity=1, font=\sffamily\scriptsize, text width=5in, text height=9pt, inner xsep=6pt, inner ysep=6pt},
				chatbox_prompt_inner/.style={chatbox_inner, align=flush left, xshift=0pt, text height=11pt},
				chatbox_user_inner/.style={chatbox_inner, align=flush left, xshift=0pt},
				chatbox_gpt_inner/.style={chatbox_inner, align=flush left, xshift=0pt},
				chatbox/.style={chatbox_inner, draw=black!25, fill=gray!7, opacity=1, text opacity=0},
				chatbox_prompt/.style={chatbox, align=flush left, fill=gray!1.5, draw=black!30, text height=10pt},
				chatbox_user/.style={chatbox, align=flush left},
				chatbox_gpt/.style={chatbox, align=flush left},
				chatbox2/.style={chatbox_gpt, fill=green!25},
				chatbox3/.style={chatbox_gpt, fill=red!20, draw=black!20},
				chatbox4/.style={chatbox_gpt, fill=yellow!30},
				labelbox/.style={rectangle, rounded corners, draw=black!50, font=\sffamily\scriptsize\bfseries, fill=gray!5, inner sep=3pt},
			]
											
			\node[chatbox_user] (q1) {
				\textbf{System prompt}
				\newline
				\newline
				You are a helpful and precise assistant for segmenting and labeling sentences. We would like to request your help on curating a dataset for entity-level hallucination detection.
				\newline \newline
                We will give you a machine generated biography and a list of checked facts about the biography. Each fact consists of a sentence and a label (True/False). Please do the following process. First, breaking down the biography into words. Second, by referring to the provided list of facts, merging some broken down words in the previous step to form meaningful entities. For example, ``strategic thinking'' should be one entity instead of two. Third, according to the labels in the list of facts, labeling each entity as True or False. Specifically, for facts that share a similar sentence structure (\eg, \textit{``He was born on Mach 9, 1941.''} (\texttt{True}) and \textit{``He was born in Ramos Mejia.''} (\texttt{False})), please first assign labels to entities that differ across atomic facts. For example, first labeling ``Mach 9, 1941'' (\texttt{True}) and ``Ramos Mejia'' (\texttt{False}) in the above case. For those entities that are the same across atomic facts (\eg, ``was born'') or are neutral (\eg, ``he,'' ``in,'' and ``on''), please label them as \texttt{True}. For the cases that there is no atomic fact that shares the same sentence structure, please identify the most informative entities in the sentence and label them with the same label as the atomic fact while treating the rest of the entities as \texttt{True}. In the end, output the entities and labels in the following format:
                \begin{itemize}[nosep]
                    \item Entity 1 (Label 1)
                    \item Entity 2 (Label 2)
                    \item ...
                    \item Entity N (Label N)
                \end{itemize}
                % \newline \newline
                Here are two examples:
                \newline\newline
                \textbf{[Example 1]}
                \newline
                [The start of the biography]
                \newline
                \textcolor{titlecolor}{Marianne McAndrew is an American actress and singer, born on November 21, 1942, in Cleveland, Ohio. She began her acting career in the late 1960s, appearing in various television shows and films.}
                \newline
                [The end of the biography]
                \newline \newline
                [The start of the list of checked facts]
                \newline
                \textcolor{anscolor}{[Marianne McAndrew is an American. (False); Marianne McAndrew is an actress. (True); Marianne McAndrew is a singer. (False); Marianne McAndrew was born on November 21, 1942. (False); Marianne McAndrew was born in Cleveland, Ohio. (False); She began her acting career in the late 1960s. (True); She has appeared in various television shows. (True); She has appeared in various films. (True)]}
                \newline
                [The end of the list of checked facts]
                \newline \newline
                [The start of the ideal output]
                \newline
                \textcolor{labelcolor}{[Marianne McAndrew (True); is (True); an (True); American (False); actress (True); and (True); singer (False); , (True); born (True); on (True); November 21, 1942 (False); , (True); in (True); Cleveland, Ohio (False); . (True); She (True); began (True); her (True); acting career (True); in (True); the late 1960s (True); , (True); appearing (True); in (True); various (True); television shows (True); and (True); films (True); . (True)]}
                \newline
                [The end of the ideal output]
				\newline \newline
                \textbf{[Example 2]}
                \newline
                [The start of the biography]
                \newline
                \textcolor{titlecolor}{Doug Sheehan is an American actor who was born on April 27, 1949, in Santa Monica, California. He is best known for his roles in soap operas, including his portrayal of Joe Kelly on ``General Hospital'' and Ben Gibson on ``Knots Landing.''}
                \newline
                [The end of the biography]
                \newline \newline
                [The start of the list of checked facts]
                \newline
                \textcolor{anscolor}{[Doug Sheehan is an American. (True); Doug Sheehan is an actor. (True); Doug Sheehan was born on April 27, 1949. (True); Doug Sheehan was born in Santa Monica, California. (False); He is best known for his roles in soap operas. (True); He portrayed Joe Kelly. (True); Joe Kelly was in General Hospital. (True); General Hospital is a soap opera. (True); He portrayed Ben Gibson. (True); Ben Gibson was in Knots Landing. (True); Knots Landing is a soap opera. (True)]}
                \newline
                [The end of the list of checked facts]
                \newline \newline
                [The start of the ideal output]
                \newline
                \textcolor{labelcolor}{[Doug Sheehan (True); is (True); an (True); American (True); actor (True); who (True); was born (True); on (True); April 27, 1949 (True); in (True); Santa Monica, California (False); . (True); He (True); is (True); best known (True); for (True); his roles in soap operas (True); , (True); including (True); in (True); his portrayal (True); of (True); Joe Kelly (True); on (True); ``General Hospital'' (True); and (True); Ben Gibson (True); on (True); ``Knots Landing.'' (True)]}
                \newline
                [The end of the ideal output]
				\newline \newline
				\textbf{User prompt}
				\newline
				\newline
				[The start of the biography]
				\newline
				\textcolor{magenta}{\texttt{\{BIOGRAPHY\}}}
				\newline
				[The ebd of the biography]
				\newline \newline
				[The start of the list of checked facts]
				\newline
				\textcolor{magenta}{\texttt{\{LIST OF CHECKED FACTS\}}}
				\newline
				[The end of the list of checked facts]
			};
			\node[chatbox_user_inner] (q1_text) at (q1) {
				\textbf{System prompt}
				\newline
				\newline
				You are a helpful and precise assistant for segmenting and labeling sentences. We would like to request your help on curating a dataset for entity-level hallucination detection.
				\newline \newline
                We will give you a machine generated biography and a list of checked facts about the biography. Each fact consists of a sentence and a label (True/False). Please do the following process. First, breaking down the biography into words. Second, by referring to the provided list of facts, merging some broken down words in the previous step to form meaningful entities. For example, ``strategic thinking'' should be one entity instead of two. Third, according to the labels in the list of facts, labeling each entity as True or False. Specifically, for facts that share a similar sentence structure (\eg, \textit{``He was born on Mach 9, 1941.''} (\texttt{True}) and \textit{``He was born in Ramos Mejia.''} (\texttt{False})), please first assign labels to entities that differ across atomic facts. For example, first labeling ``Mach 9, 1941'' (\texttt{True}) and ``Ramos Mejia'' (\texttt{False}) in the above case. For those entities that are the same across atomic facts (\eg, ``was born'') or are neutral (\eg, ``he,'' ``in,'' and ``on''), please label them as \texttt{True}. For the cases that there is no atomic fact that shares the same sentence structure, please identify the most informative entities in the sentence and label them with the same label as the atomic fact while treating the rest of the entities as \texttt{True}. In the end, output the entities and labels in the following format:
                \begin{itemize}[nosep]
                    \item Entity 1 (Label 1)
                    \item Entity 2 (Label 2)
                    \item ...
                    \item Entity N (Label N)
                \end{itemize}
                % \newline \newline
                Here are two examples:
                \newline\newline
                \textbf{[Example 1]}
                \newline
                [The start of the biography]
                \newline
                \textcolor{titlecolor}{Marianne McAndrew is an American actress and singer, born on November 21, 1942, in Cleveland, Ohio. She began her acting career in the late 1960s, appearing in various television shows and films.}
                \newline
                [The end of the biography]
                \newline \newline
                [The start of the list of checked facts]
                \newline
                \textcolor{anscolor}{[Marianne McAndrew is an American. (False); Marianne McAndrew is an actress. (True); Marianne McAndrew is a singer. (False); Marianne McAndrew was born on November 21, 1942. (False); Marianne McAndrew was born in Cleveland, Ohio. (False); She began her acting career in the late 1960s. (True); She has appeared in various television shows. (True); She has appeared in various films. (True)]}
                \newline
                [The end of the list of checked facts]
                \newline \newline
                [The start of the ideal output]
                \newline
                \textcolor{labelcolor}{[Marianne McAndrew (True); is (True); an (True); American (False); actress (True); and (True); singer (False); , (True); born (True); on (True); November 21, 1942 (False); , (True); in (True); Cleveland, Ohio (False); . (True); She (True); began (True); her (True); acting career (True); in (True); the late 1960s (True); , (True); appearing (True); in (True); various (True); television shows (True); and (True); films (True); . (True)]}
                \newline
                [The end of the ideal output]
				\newline \newline
                \textbf{[Example 2]}
                \newline
                [The start of the biography]
                \newline
                \textcolor{titlecolor}{Doug Sheehan is an American actor who was born on April 27, 1949, in Santa Monica, California. He is best known for his roles in soap operas, including his portrayal of Joe Kelly on ``General Hospital'' and Ben Gibson on ``Knots Landing.''}
                \newline
                [The end of the biography]
                \newline \newline
                [The start of the list of checked facts]
                \newline
                \textcolor{anscolor}{[Doug Sheehan is an American. (True); Doug Sheehan is an actor. (True); Doug Sheehan was born on April 27, 1949. (True); Doug Sheehan was born in Santa Monica, California. (False); He is best known for his roles in soap operas. (True); He portrayed Joe Kelly. (True); Joe Kelly was in General Hospital. (True); General Hospital is a soap opera. (True); He portrayed Ben Gibson. (True); Ben Gibson was in Knots Landing. (True); Knots Landing is a soap opera. (True)]}
                \newline
                [The end of the list of checked facts]
                \newline \newline
                [The start of the ideal output]
                \newline
                \textcolor{labelcolor}{[Doug Sheehan (True); is (True); an (True); American (True); actor (True); who (True); was born (True); on (True); April 27, 1949 (True); in (True); Santa Monica, California (False); . (True); He (True); is (True); best known (True); for (True); his roles in soap operas (True); , (True); including (True); in (True); his portrayal (True); of (True); Joe Kelly (True); on (True); ``General Hospital'' (True); and (True); Ben Gibson (True); on (True); ``Knots Landing.'' (True)]}
                \newline
                [The end of the ideal output]
				\newline \newline
				\textbf{User prompt}
				\newline
				\newline
				[The start of the biography]
				\newline
				\textcolor{magenta}{\texttt{\{BIOGRAPHY\}}}
				\newline
				[The ebd of the biography]
				\newline \newline
				[The start of the list of checked facts]
				\newline
				\textcolor{magenta}{\texttt{\{LIST OF CHECKED FACTS\}}}
				\newline
				[The end of the list of checked facts]
			};
		\end{tikzpicture}
        \caption{GPT-4o prompt for labeling hallucinated entities.}\label{tb:gpt-4-prompt}
	\end{center}
\vspace{-0cm}
\end{table*}
% \section{Full Experiment Results}
% \begin{table*}[th]
    \centering
    \small
    \caption{Classification Results}
    \begin{tabular}{lcccc}
        \toprule
        \textbf{Method} & \textbf{Accuracy} & \textbf{Precision} & \textbf{Recall} & \textbf{F1-score} \\
        \midrule
        \multicolumn{5}{c}{\textbf{Zero Shot}} \\
                Zero-shot E-eyes & 0.26 & 0.26 & 0.27 & 0.26 \\
        Zero-shot CARM & 0.24 & 0.24 & 0.24 & 0.24 \\
                Zero-shot SVM & 0.27 & 0.28 & 0.28 & 0.27 \\
        Zero-shot CNN & 0.23 & 0.24 & 0.23 & 0.23 \\
        Zero-shot RNN & 0.26 & 0.26 & 0.26 & 0.26 \\
DeepSeek-0shot & 0.54 & 0.61 & 0.54 & 0.52 \\
DeepSeek-0shot-COT & 0.33 & 0.24 & 0.33 & 0.23 \\
DeepSeek-0shot-Knowledge & 0.45 & 0.46 & 0.45 & 0.44 \\
Gemma2-0shot & 0.35 & 0.22 & 0.38 & 0.27 \\
Gemma2-0shot-COT & 0.36 & 0.22 & 0.36 & 0.27 \\
Gemma2-0shot-Knowledge & 0.32 & 0.18 & 0.34 & 0.20 \\
GPT-4o-mini-0shot & 0.48 & 0.53 & 0.48 & 0.41 \\
GPT-4o-mini-0shot-COT & 0.33 & 0.50 & 0.33 & 0.38 \\
GPT-4o-mini-0shot-Knowledge & 0.49 & 0.31 & 0.49 & 0.36 \\
GPT-4o-0shot & 0.62 & 0.62 & 0.47 & 0.42 \\
GPT-4o-0shot-COT & 0.29 & 0.45 & 0.29 & 0.21 \\
GPT-4o-0shot-Knowledge & 0.44 & 0.52 & 0.44 & 0.39 \\
LLaMA-0shot & 0.32 & 0.25 & 0.32 & 0.24 \\
LLaMA-0shot-COT & 0.12 & 0.25 & 0.12 & 0.09 \\
LLaMA-0shot-Knowledge & 0.32 & 0.25 & 0.32 & 0.28 \\
Mistral-0shot & 0.19 & 0.23 & 0.19 & 0.10 \\
Mistral-0shot-Knowledge & 0.21 & 0.40 & 0.21 & 0.11 \\
        \midrule
        \multicolumn{5}{c}{\textbf{4 Shot}} \\
GPT-4o-mini-4shot & 0.58 & 0.59 & 0.58 & 0.53 \\
GPT-4o-mini-4shot-COT & 0.57 & 0.53 & 0.57 & 0.50 \\
GPT-4o-mini-4shot-Knowledge & 0.56 & 0.51 & 0.56 & 0.47 \\
GPT-4o-4shot & 0.77 & 0.84 & 0.77 & 0.73 \\
GPT-4o-4shot-COT & 0.63 & 0.76 & 0.63 & 0.53 \\
GPT-4o-4shot-Knowledge & 0.72 & 0.82 & 0.71 & 0.66 \\
LLaMA-4shot & 0.29 & 0.24 & 0.29 & 0.21 \\
LLaMA-4shot-COT & 0.20 & 0.30 & 0.20 & 0.13 \\
LLaMA-4shot-Knowledge & 0.15 & 0.23 & 0.13 & 0.13 \\
Mistral-4shot & 0.02 & 0.02 & 0.02 & 0.02 \\
Mistral-4shot-Knowledge & 0.21 & 0.27 & 0.21 & 0.20 \\
        \midrule
        
        \multicolumn{5}{c}{\textbf{Suprevised}} \\
        SVM & 0.94 & 0.92 & 0.91 & 0.91 \\
        CNN & 0.98 & 0.98 & 0.97 & 0.97 \\
        RNN & 0.99 & 0.99 & 0.99 & 0.99 \\
        % \midrule
        % \multicolumn{5}{c}{\textbf{Conventional Wi-Fi-based Human Activity Recognition Systems}} \\
        E-eyes & 1.00 & 1.00 & 1.00 & 1.00 \\
        CARM & 0.98 & 0.98 & 0.98 & 0.98 \\
\midrule
 \multicolumn{5}{c}{\textbf{Vision Models}} \\
           Zero-shot SVM & 0.26 & 0.25 & 0.25 & 0.25 \\
        Zero-shot CNN & 0.26 & 0.25 & 0.26 & 0.26 \\
        Zero-shot RNN & 0.28 & 0.28 & 0.29 & 0.28 \\
        SVM & 0.99 & 0.99 & 0.99 & 0.99 \\
        CNN & 0.98 & 0.99 & 0.98 & 0.98 \\
        RNN & 0.98 & 0.99 & 0.98 & 0.98 \\
GPT-4o-mini-Vision & 0.84 & 0.85 & 0.84 & 0.84 \\
GPT-4o-mini-Vision-COT & 0.90 & 0.91 & 0.90 & 0.90 \\
GPT-4o-Vision & 0.74 & 0.82 & 0.74 & 0.73 \\
GPT-4o-Vision-COT & 0.70 & 0.83 & 0.70 & 0.68 \\
LLaMA-Vision & 0.20 & 0.23 & 0.20 & 0.09 \\
LLaMA-Vision-Knowledge & 0.22 & 0.05 & 0.22 & 0.08 \\

        \bottomrule
    \end{tabular}
    \label{full}
\end{table*}




\end{document}


\paragraph{Inference Efficiency.}
To generate a 12-frame video (128$\times$128 resolution, 768 tokens), a 700M NTP model requires 768 forward steps during inference, taking 15.04 seconds (FPS=0.80). 
In contrast, our \modelname model with a 1$\times$1$\times$16 block size predicts all tokens in a row simultaneously, requiring only 48 steps and 1.35 seconds to generate the video (FPS=8.89)—over 11 times faster than the NTP model. 
Since \modelname modifies only the target output and attention mask, it is compatible with the most efficient AR inference frameworks, such as memory-efficient attention~\citep{xFormers2022}, offering the potential for further speed improvements. 
We now briefly discuss the sources of these efficiency gains. 
In scenarios utilizing KV-Cache, the overall computation cost during each inference step for NTP involves multiplying vectors (current token) with matrices (model weights), which is primarily \textbf{IO-bound} due to the movement of matrices. Conversely, in the NBP model, the computation involves multiplying matrices (current block) with matrices (model weights), making it \textbf{compute-bound}, with reduced IO overhead due to larger block sizes. Given this distinction and assuming adequate GPU parallelism, the NBP framework can achieve significantly faster speeds compared to NTP. This efficiency gain is due to the reduced frequency of IO operations and the more effective utilization of computational resources in processing larger data blocks simultaneously.

\paragraph{Scalability.}
As model size increases from 700M to 1.2B and 3B parameters, the FVD of \modelname models improves from 33.6 to 28.6 and 26.5, respectively. This demonstrates that \modelname exhibits similar scalability to NTP models, with the potential for even greater performance as model size and computational resources increase. Fig.~\ref{fig:model_para} and Fig.~\ref{fig:vary_size_gen} present the validation loss curves and generation examples for different model sizes, respectively. 
As the models grow larger, the generated content exhibits greater stability and enhanced visual detail. 

\subsection{Benchmarking with Previous Systems}
\label{subsec:benchmark}
Table~\ref{tab:video_syn} presents our model's performance compared to strong baselines using various modeling approaches, including GAN, diffusion, mask token modeling (MTM), and vanilla AR methods. 
For UCF-101, the evaluation task is class-conditional video generation, where models generate videos based on a given class name. 
% Since our method utilizes an image as the initial visual condition, alongside the classname, we take the first frame from the training videos into condition additionally. This ensures no information leakage from the test set. 
Our Semi-AR model, with 3B parameters, achieves an FVD of 55.3, surpassing HPDM-L~\citep{skorokhodov2024hierarchical} and MAGVITv2~\cite{yu2023language} by 11 and 2.7 FVD points, respectively.

For K600, the evaluation task is frame prediction, where all models predict future frames based on the same 5-frame condition from the validation set. Our 700M model achieves an FVD of 25.5, outperforming the strongest AR baseline, OmniTokenizer, by 7.4 FVD points.
While our model exhibits a performance gap compared to MAGVITv2, it achieves this result with significantly lower training computation (e.g., 77 epochs vs. MAGVITv2's 360 epochs). Scaling up the model size narrows this gap, with a 6-point improvement in FVD observed. Given the strong scalability of our semi-AR framework, we believe that with larger model sizes and increased training volumes, our approach could surpass MAGVITv2, akin to how large language models (LLMs)~\citep{Brown2020LanguageMA} have outperformed BERT~\citep{devlin2018bert} in NLP.



\subsection{Visualizations}
\paragraph{Video Reconstruction.}
Fig.~\ref{fig:vis_recons} compares the video reconstruction results of OmniTokenizer~\citep{Wang2024OmniTokenizerAJ} and our tokenizer. Our method significantly outperforms the baseline in both image clarity and motion stability. 




\paragraph{Video Generation.}
The class-conditional generation results for UCF-101 are shown in Fig.\ref{fig:ucf_gen}, while the frame prediction results for K600 are shown in Figs.\ref{fig:our_gen}-\ref{fig:vis_gen}. The visualizations demonstrate that our model accurately predicts subsequent frames with high clarity and temporal coherence, even in scenarios involving large motion dynamics. 
% Fig.~\ref{fig:our_gen_app} shows more generation results of our 3B model. 
% Moreover, we exhibit the potential of our method for generating videos of arbitrary lengths by employing a cyclical process, where each newly generated frame is recursively used as a condition for the subsequent frame generation.


% \paragraph{Class-conditional Video Generation.}
% % copy from omni
% The class-conditional generation results are shown in Figure 5-8. Our model could synthesize visually coherent and contextually accurate images and videos, showcasing the strengths of OmniTokenizer in facilitating generative tasks.


\subsection{Ablation Study and Analysis}
\label{subsec:ablation}
In this subsection, we conduct an ablation study on block size and block shape, then analyze the attention patterns in our \modelname models.


\paragraph{Ablation Study on Block Size.}
We experiment with different block sizes, ranging from $[1, 8, 16, 32, 64, 256]$\footnote{The full 3D size of the blocks are 1$\times$1$\times$1, 1$\times$1$\times$8, 1$\times$1$\times$16, 1$\times$2$\times$16, 1$\times$4$\times$16, 1$\times$16$\times 16$, respectively.}, to evaluate their impact on model performance. A block size of 1, 16, and 256 corresponds to token-by-token (NTP), row-by-row, and clip-by-clip generation, respectively. 
Fig.~\ref{fig:block_size} shows the validation loss curves for various block sizes. As block size decreases, learning becomes easier due to the increased prefix conditioning, which simplifies the prediction task and results in lower validation loss. 
However, due to the exposure bias associated with (semi-)AR modeling~\citep{ranzato2015sequence}, validation loss under the teacher-forcing setting does not completely correlate with final performance during inference~\citep{deng2024causal}. Notably, the smallest block size (i.e., a single token) does not yield optimal performance. As shown in Fig.~\ref{fig:block_size_fvd_fps}, a block size of 16 achieves the best generation quality, with an FVD improvement of 3.5 points, reaching 25.5. 
Block size is critical for balancing generation quality (FVD) and efficiency (FPS). While larger blocks (e.g., 1$\times$16$\times$16) lead to faster inference speeds (up to 17.14 FPS), performance degrades, indicating that generating an entire clip in one step is overly challenging. 
Additionally, inference decoding methods significantly influence results. As demonstrated in Fig.~\ref{fig:vary_block_gen}, traditional Top-P Top-K decoding can lead to screen fluctuations~\citep{lezama2022improved}, as it struggles to model spatial dependencies within large blocks, highlighting the need for improved decoding strategies in \modelname scenarios. 

\paragraph{Ablation Study on Block Shape.}
We explore the performance of various block shapes on K600, using the 700M model, the results are shown in Table~\ref{tab:block_shape}. 
Our findings indicate that the official block shape of T$\times$H$\times$W=1$\times$1$\times$16 (generating row by row) outperforms other tested shapes such as 1$\times$4$\times$4 and 2$\times$1$\times$8. We attribute this to two main factors: 
\textbf{(1) Token Relationships within a Single Block}: The shape of the 1$\times$1$\times$16 block allows tokens within the block to represent a complete, continuous row, maintaining integrity without cross-row interruptions. In contrast, block shapes like 1$\times$4$\times$4 and 2$\times$1$\times$8 involve generating complex relationships across multiple rows and columns—or even frames—on a smaller spatial scale, posing greater challenges~\citep{ren2023testa}. 
\textbf{(2) Relationships between Blocks}: The 1$\times$1$\times$16 block shape simplifies the modeling process to primarily vertical relationships between rows, which enhances continuity and consistency during generation, thereby reducing breaks and error accumulation.


\begin{figure}[htbp]
\includegraphics[width=\linewidth]{figs/fvd_fps_for_block.pdf}
\caption{Generation quality (FVD, lower is better) and inference speed (FPS, higher is better) of various block sizes from 1 to 256.}
\label{fig:block_size_fvd_fps}
\end{figure}

\begin{table}[htbp]
\centering
\caption{Generation quality (FVD) of various block shape.}
\label{tab:block_shape}
\begin{tabular}{ccc}
\toprule
Block Size & Block Shape (T$\times$H$\times$W) & FVD$\downarrow$ \\
\midrule
16 & 1$\times$4$\times$4 & 33.4 \\
16 & 2$\times$1$\times$8 & 29.2 \\ 
16 & 1$\times$1$\times$16 & \textbf{25.5} \\\midrule
8  & 2$\times$2$\times$2  & 32.7 \\
8  & 1$\times$1$\times$8  & \textbf{25.7} \\
\bottomrule
\end{tabular}
\end{table}

\begin{figure*}[tbp]
\centering
\includegraphics[width=.9\textwidth]{figs/ucf.pdf}
\caption{Visualization of class-conditional generation (UCF-101) results of our method. The text below each video clip is the class name.}
\label{fig:ucf_gen}
\end{figure*}

\begin{figure*}[tbp]
\centering
\includegraphics[width=.9\textwidth]{figs/our_gen.pdf}
\caption{Visualization of frame prediction (K600) results of our method.}
\label{fig:our_gen}
\end{figure*}

\begin{figure*}[tbp]
\centering
\includegraphics[width=.9\textwidth]{figs/vis_gen.pdf}
\caption{Frame prediction results of OmniTokenizer and our method. The left part is the condition, and the right part is the predicted subsequent sequence.}
\label{fig:vis_gen}
\end{figure*}

\paragraph{Analysis of Attention Pattern.} 
We analyze the attention pattern in our \modelname framework using an example of next-clip prediction, where each block corresponds to a clip. 
Fig.~\ref{fig:txt_2clips_attn} shows the attention weights on UCF-101. Unlike the lower triangular distribution observed in AR models, our attention is characterized by a staircase pattern across blocks. In addition to high attention scores along the diagonal, the map reveals vertical stripe-like highlighted patterns, indicating that tokens at certain positions receive attention from all tokens. 
Fig.~\ref{fig:spatial-attn} illustrates the spatial attention distribution for a specific query (marked by \textcolor{red}{red $\times$}). This query can attend to all tokens within the clip, rather than being restricted to only the preceding tokens in a raster-scan order, enabling more effective spatial dependency modeling.

\section{Conclusion}
In this paper, we introduced a novel approach to video generation called Next Block Prediction using a semi-autoregressive modeling framework. This framework offers a more efficient and scalable solution for video generation, combining the advantages of parallelization with improved spatial-temporal dependency modeling. This method not only accelerates inference but also maintains or improves the quality of generated content, demonstrating strong potential for future applications in multimodal AI.


% Acknowledgements should only appear in the accepted version.
% \section*{Acknowledgements}

% \textbf{Do not} include acknowledgements in the initial version of
% the paper submitted for blind review.

% If a paper is accepted, the final camera-ready version can (and
% usually should) include acknowledgements.  Such acknowledgements
% should be placed at the end of the section, in an unnumbered section
% that does not count towards the paper page limit. Typically, this will 
% include thanks to reviewers who gave useful comments, to colleagues 
% who contributed to the ideas, and to funding agencies and corporate 
% sponsors that provided financial support.

\section*{Impact Statement}
This work advances the field of video generation through the development of \modelname. While recognizing the potential of this technology, we carefully consider its societal implications, particularly regarding potential misuse and ethical challenges. The model's capabilities could be exploited to create harmful content, including deepfakes for misinformation campaigns or other malicious purposes. Furthermore, we acknowledge the critical importance of ensuring that generated content adheres to ethical standards by avoiding the perpetuation of harmful stereotypes and respecting cultural diversity. 
To mitigate these risks, we will explore a comprehensive framework of safeguards, including (1) robust digital watermarking to ensure traceability and accountability of generated content; (2) reinforcement learning with human feedback to align model outputs with ethical guidelines and reduce potential harm; and (3) clear usage policies and restrictions. These measures collectively aim to promote responsible development and deployment of video generation technology while maximizing its positive societal impact.

% In the unusual situation where you want a paper to appear in the
% references without citing it in the main text, use \nocite
\nocite{langley00}

\bibliography{example_paper}
\bibliographystyle{icml2025}


%%%%%%%%%%%%%%%%%%%%%%%%%%%%%%%%%%%%%%%%%%%%%%%%%%%%%%%%%%%%%%%%%%%%%%%%%%%%%%%
%%%%%%%%%%%%%%%%%%%%%%%%%%%%%%%%%%%%%%%%%%%%%%%%%%%%%%%%%%%%%%%%%%%%%%%%%%%%%%%
% APPENDIX
%%%%%%%%%%%%%%%%%%%%%%%%%%%%%%%%%%%%%%%%%%%%%%%%%%%%%%%%%%%%%%%%%%%%%%%%%%%%%%%
%%%%%%%%%%%%%%%%%%%%%%%%%%%%%%%%%%%%%%%%%%%%%%%%%%%%%%%%%%%%%%%%%%%%%%%%%%%%%%%
\newpage
\appendix
\onecolumn
\section{Implementation Details}

\subsection{Task Definitions}
\label{app:task}
We introduce the tasks used in our training and evaluation. Each task is characterized by a few adjustable settings such as interior condition shape and optionally prefix condition. 
Given a video of shape $T\times H \times W$, we define the tasks as following:
\begin{itemize}
    \item Class-conditional Generation (CG)
    
    \begin{itemize}
        \item Prefix condition: class label.
    \end{itemize}
    
    % \item Class-conditional Frame Prediction (CFP)
    
    % \begin{itemize}
    %     \item Prefix condition: class label.
    %     \item Interior condition:  $t$ frames at the beginning; $t=1$.
    %     % \item Padding: replicate the last given frame.
    % \end{itemize}

    \item Frame Prediction (FP)
    
    \begin{itemize}
        \item Interior condition: $t$ frames at the beginning; $t=5$ for K600 dataset.
        % \item Padding: replicate the last given frame.
    \end{itemize}
    
\end{itemize}
% As we stated in $\S$~\ref{subsec:benchmark}, for UCF-101, other baselines perform the CG task, while our models perform the CFP task, as our method utilizes an image as an initial visual condition, alongside the classname. We take the first frame from the training videos into condition additionally. This ensures no information leakage from the test set. 
As we stated in $\S$~\ref{subsec:benchmark}, for UCF-101, all methods perform the CG task, while for K600, all methods perform the FP task.

\subsection{Model Configuration}
\label{app:model}

\paragraph{Video Tokenizer.}
Our video tokenizer shares the same model architecture with MAGVITv2~\cite{yu2023language}.

\paragraph{Decoder-only Generator.}
Table~\ref{tab:model_config} shows the configuration for the decoder-only generator. We use separate position encoding for text and video.  
We do not use advanced techniques in large language models, such as rotary position embedding (RoPE)~\citep{su2024roformer}, SwiGLU MLP, or RMS Norm~\citep{Touvron2023LLaMAOA}, which we believe could bring better performance.
% The configurations are mainly following LLaMA~\citep{Touvron2023LLaMAOA}

\subsection{Training}
\paragraph{Video Tokenizer.}
Table~\ref{tab:tok_train_config} shows the training configurations of our video tokenizer. 

\paragraph{Decoder-only Generator.}
Table~\ref{tab:gen_train_config} shows the training configurations of our video generator.

For both tokenizer and generator training, the video samples are all 17 frames, frame stride 1, 128$\times$128 resolution.

\subsection{Evaluation}
\label{app:eval}

\paragraph{Evaluation metrics.}
The FVD~\cite{unterthiner2018towards} is used as the primary evaluation metric. 
We follow the official implementation\footnote{\url{https://github.com/google-research/google-research/tree/master/frechet_video_distance}} in extracting video features with an I3D model trained on Kinetics-400~\cite{carreira2017quo}.
% We report Inception Score (IS)~\cite{saito2020train}\footnote{\url{https://github.com/pfnet-research/tgan2}} on the UCF-101 dataset which is calculated with a C3D~\cite{tran2015learning} model trained on UCF-101. 
% We further include image quality metrics: PSNR, SSIM~\cite{wang2004image} and LPIPS~\cite{zhang2018unreasonable} (computed by the VGG features).

\paragraph{Sampling protocols.}
We follow the sampling protocols from previous works~\cite{yu2023language, ge2022long,clark2019adversarial} when eveluating on the standard benchmarks, i.e. UCF-101, and Kinetics-600.
We sample 17-frame clips from each dataset without replacement to form the real distribution in FVD and extract condition inputs from them to feed to the model.
We continuously run through all the samples required (e.g., 40,000 for UCF-101) with a single data loader and compute the mean and standard deviation for 4 folds. 
We use top-$p$ and top-$k$ sampling with $k=16,000$ and $p=0.9$. 

Below are detailed setups for each dataset:
% Note that the evaluation resolution may be different from the generation resolution, where the generated samples are bilinear resized to the target resolution.

\begin{itemize}
    \item UCF-101: 
    \begin{itemize}
        \item Dataset: 9.5K videos for training, 101 classes.
        \item Number of samples: 10,000$\times$4.
        \item Resolution: 128$\times$128.
        \item Real distribution: random clips from the training videos.
        \item Video FPS: 8.
    \end{itemize}
    \item Kinetics-600:
    \begin{itemize}
        \item Dataset: 384K videos for training and 29K videos for evaluation.
        \item Number of samples: 50,000$\times$4.
        \item Generation resolution: 128$\times$128.
        \item Evaluation resolution: 64$\times$64, via central crop and bilinear resize.
        \item Video FPS: 25.
        % \item Real distribution: 6 sampled clips (2 temporal windows and 3 spatial crops) from each evaluation video.
        % \item COMMIT decoding: uniform schedule, temperature 7.5.
    \end{itemize}
\end{itemize}

\begin{table*}
\caption{Model sizes and architecture configurations of our generation model. The configurations are following LLaMA~\citep{Touvron2023LLaMAOA}.
}
\centering
\begin{tabular}{@{}lcccc@{}}
\toprule
Model & Parameters & Layers & Hidden Size & Heads \\
\midrule
\modelname-XL & 700M & 24 & 1536 &  16 \\
\modelname-XXL & 1.2B & 24 & 2048 &  32 \\
\modelname-3B & 3B & 32 & 3072 &  32 \\
\bottomrule
\end{tabular}
\label{tab:model_config}
% \vspace{-3mm}
\end{table*}
% Please add the following required packages to your document preamble:
% \usepackage{booktabs}
\begin{table}[]
\caption{Training configurations of video tokenizer.}
\label{tab:tok_train_config}
\centering
\begin{tabular}{l|cc}
\toprule
Hyper-parameters                  & UCF101                          & K600                            \\ \midrule
Video  FPS              & 8 & 8 \\
Latent shape                      & 5$\times$16$\times$16           & 5$\times$16$\times$16           \\
Vocabulary size                   & 64K                           & 64K                           \\
Embedding dimension               & 6                               & 6                               \\
Initialization                    & Random                          & Random                          \\
Peak learning rate                & 5e-5                            & 1e-4                            \\
Learning rate schedule            & linear & linear              \\
Warmup ratio                      & 0.01                            & 0.01                            \\
Perceptual loss weight            & 0.1                             & 0.1                             \\
Generator adversarial loss weight & 0.1                             & 0.1                             \\
Optimizer                         & Adam                            & Adam                            \\
Batch size                        & 256                             & 256                             \\
Epoch                             & 2000                            & 100                             \\ \bottomrule
\end{tabular}
\end{table}
% Please add the following required packages to your document preamble:
% \usepackage{booktabs}
\begin{table}[]
\caption{Training configurations of video generator (base model).}
\label{tab:gen_train_config}
\centering
\begin{tabular}{l|cc}
\toprule
Hyper-parameters                  & UCF101                          & K600                            \\ \midrule
Video  FPS              &  8  &  16 \\
Latent shape                      & 5$\times$16$\times$16           & 5$\times$16$\times$16           \\
Vocabulary size                   & 96K (including 32K text tokens)                          & 64K                           \\
Initialization                    & Random                          & Random                          \\
Peak learning rate                & 6e-4                            & 1e-3                            \\
Learning rate schedule            & linear & linear              \\
Warmup steps                      & 5,000                             & 10,000                             \\
Weight decay & 0.01 & 0.01 \\
Optimizer                         & Adam (0.9, 0.98)                           & Adam (0.9, 0.98)                           \\
Dropout & 0.1 & 0.1 \\
Batch size                        & 256                             & 64                             \\
Epoch                             & 2560                            & 77                             \\ \bottomrule
\end{tabular}
\end{table}

\section{Performance of Video Tokenizer}
\label{sec:vid-tok}
\begin{table}[h]
\centering
\caption{Compression Rates for Different Models.}
\label{tokenizer_compression_rate}
\begin{tabular}{lcccccc}
\toprule
 & Baichuan2 & ChatGLM2 & Llama2 & MiniCPM & Megrez \\
\midrule
Vocab Size & 125,696 & 64,794 & 32,000 & 122,753 & 120,000 \\
\midrule
\multicolumn{6}{c}{\textbf{Compression Rate (Bytes/Tokens)}} \\
\midrule
Chinese & 3.64 & 3.54 & 1.87 & 3.73 & 5.02 \\
English & 4.12 & 4.02 & 3.78 & 4.14 & 4.28 \\
Code & 2.71 & 2.71 & 2.74 & 2.81 & 2.69 \\
Paper & 2.74 & 2.88 & 2.97 & 2.93 & 3.48 \\
\midrule
Average & 3.30 & 3.29 & 2.84 & 3.40 & 3.86 \\
\bottomrule
\end{tabular}
\end{table}

We present the reconstruction performance of our tokenizer in Table~\ref{tab:video_reconstruction}. Our tokenizer achieves 15.50 rFVD on UCF-101 and 6.73 rFVD on K600, surpassing OmniTokenizer~\cite{Wang2024OmniTokenizerAJ}, MAGVITv1~\cite{yu2023magvit}, and other models. Fig.~\ref{fig:vis_recons} compares the video reconstruction results of OmniTokenizer~\citep{Wang2024OmniTokenizerAJ} and our tokenizer. Our method significantly outperforms the baseline in both image clarity and motion stability. 

\section{Visualization}
We provide additional visualization of video generation results. 
Fig.~\ref{fig:vary_size_gen} shows results of various model sizes (700M, 1.2B and 3B).
Fig.~\ref{fig:vary_block_gen} shows results of various block sizes (1$\times$1$\times$1, 1$\times$1$\times$16 and 1$\times$16$\times$16).


\begin{figure*}[tbp]
\centering
\includegraphics[width=\textwidth]{figs/vis_recons.pdf}
\caption{Video reconstruction results (17 frames 128$\times$128 resolution at 25 fps and shown at 6.25 fps) of OmniTokenizer and our method. }
\label{fig:vis_recons}
\end{figure*}



% \begin{figure*}[tbp]
% \centering
% \includegraphics[width=\textwidth]{figs/our_gen_app.pdf}
% \caption{Visualization of video generation results of our 3B model.}
% \label{fig:our_gen_app}
% \end{figure*}

\begin{figure*}[tbp]
\centering
\includegraphics[width=\textwidth]{figs/vary_size_gen.pdf}
\caption{Visualization of video generation results of various model sizes (700M, 1.2B, and 3B).}
\label{fig:vary_size_gen}
\end{figure*}

\begin{figure*}[tbp]
\centering
\includegraphics[width=\textwidth]{figs/vary_block_gen.pdf}
\caption{Visualization of video generation results of various block sizes (1$\times$1$\times$1, 1$\times$1$\times$16 and 1$\times$16$\times$16).}
\label{fig:vary_block_gen}
\end{figure*}


% \begin{figure}[htbp]
% \includegraphics[width=\linewidth]{figs/k600_block_size.pdf}
% \caption{Training loss of various block sizes from 1 to 256. }
% \label{fig:block_size}
% \end{figure}



\begin{figure*}[tbp]
\centering
\includegraphics[width=.9\textwidth]{figs/txt_2clips_attn.pdf}
\caption{Attention weights of next-clip prediction on UCF-101. The horizontal and vertical axis represent the keys and queries, respectively. Two red lines on each axis divide the axis into three segments, corresponding to the text (classname), the first clip, and the second clip. The brightness of each pixel reflects the attention score. We downweight the attention to text tokens by $5\times$ to provide a more clear visualization.}
\label{fig:txt_2clips_attn}
\end{figure*}

\begin{figure*}[tbp]
\centering
\includegraphics[width=.9\textwidth]{figs/crop_img_attn.pdf}
\caption{Spatial attention distribution for a specific query (represented by \textcolor{red}{red $\times$}) on UCF-101.}
\label{fig:spatial-attn}
\end{figure*}
%%%%%%%%%%%%%%%%%%%%%%%%%%%%%%%%%%%%%%%%%%%%%%%%%%%%%%%%%%%%%%%%%%%%%%%%%%%%%%%
%%%%%%%%%%%%%%%%%%%%%%%%%%%%%%%%%%%%%%%%%%%%%%%%%%%%%%%%%%%%%%%%%%%%%%%%%%%%%%%


\end{document}


% This document was modified from the file originally made available by
% Pat Langley and Andrea Danyluk for ICML-2K. This version was created
% by Iain Murray in 2018, and modified by Alexandre Bouchard in
% 2019 and 2021 and by Csaba Szepesvari, Gang Niu and Sivan Sabato in 2022.
% Modified again in 2023 and 2024 by Sivan Sabato and Jonathan Scarlett.
% Previous contributors include Dan Roy, Lise Getoor and Tobias
% Scheffer, which was slightly modified from the 2010 version by
% Thorsten Joachims & Johannes Fuernkranz, slightly modified from the
% 2009 version by Kiri Wagstaff and Sam Roweis's 2008 version, which is
% slightly modified from Prasad Tadepalli's 2007 version which is a
% lightly changed version of the previous year's version by Andrew
% Moore, which was in turn edited from those of Kristian Kersting and
% Codrina Lauth. Alex Smola contributed to the algorithmic style files.
