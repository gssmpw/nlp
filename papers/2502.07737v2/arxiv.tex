%%%%%%%% ICML 2025 EXAMPLE LATEX SUBMISSION FILE %%%%%%%%%%%%%%%%%

\documentclass{article}

% Recommended, but optional, packages for figures and better typesetting:
\usepackage{microtype}
\usepackage{graphicx}
\usepackage{subfigure}
\usepackage{booktabs} % for professional tables

% hyperref makes hyperlinks in the resulting PDF.
% If your build breaks (sometimes temporarily if a hyperlink spans a page)
% please comment out the following usepackage line and replace
% \usepackage{icml2025} with \usepackage[nohyperref]{icml2025} above.
\usepackage{hyperref}


% Attempt to make hyperref and algorithmic work together better:
% \newcommand{\theHalgorithm}{\arabic{algorithm}}

% Use the following line for the initial blind version submitted for review:
% \usepackage{icml2025}

% If accepted, instead use the following line for the camera-ready submission:
\usepackage[accepted]{icml2025}


% For theorems and such
\usepackage{amsmath}
\usepackage{amssymb}
\usepackage{mathtools}
\usepackage{amsthm}

% if you use cleveref..
\usepackage[capitalize,noabbrev]{cleveref}

\usepackage{hyperref}
\usepackage{url}
\usepackage{hyperref}
\usepackage{graphics}
\usepackage[utf8]{inputenc} % allow utf-8 input
\usepackage[T1]{fontenc}    % use 8-bit T1 fonts
\usepackage{hyperref}       % hyperlinks
\usepackage{url}            % simple URL typesetting
\usepackage{booktabs}       % professional-quality tables
\usepackage{amsfonts}       % blackboard math symbols
\usepackage{nicefrac}       % compact symbols for 1/2, etc.
\usepackage{microtype}      % microtypography
\usepackage{xcolor}         % colors
% \usepackage[square,sort,comma,numbers]{natbib}
\usepackage[utf8]{inputenc}
\usepackage{mathtools}
\usepackage{amsthm}
\usepackage{arydshln}
\usepackage{multirow}
\usepackage{wrapfig, lipsum, booktabs}
\usepackage{paralist, tabularx}
\usepackage{balance}
\usepackage{pgfplots}
\usetikzlibrary{pgfplots.groupplots}
\pgfplotsset{compat=1.3}
\usepackage{tikz}
\usetikzlibrary{patterns}
\usepackage{pgf-pie}
\usepackage{adjustbox}
\usepackage{colortbl}
\usepackage{subcaption}
\usepackage{xspace}
\usepackage{enumitem}

%%%%%%%%%%%%%%%%%%%%%%%%%%%%%%%%
% THEOREMS
%%%%%%%%%%%%%%%%%%%%%%%%%%%%%%%%
\theoremstyle{plain}
\newtheorem{theorem}{Theorem}[section]
\newtheorem{proposition}[theorem]{Proposition}
\newtheorem{lemma}[theorem]{Lemma}
\newtheorem{corollary}[theorem]{Corollary}
\theoremstyle{definition}
\newtheorem{definition}[theorem]{Definition}
\newtheorem{assumption}[theorem]{Assumption}
\theoremstyle{remark}
\newtheorem{remark}[theorem]{Remark}

% Todonotes is useful during development; simply uncomment the next line
%    and comment out the line below the next line to turn off comments
%\usepackage[disable,textsize=tiny]{todonotes}
\usepackage[textsize=tiny]{todonotes}

\makeatletter
\newcommand{\rmnum}[1]{\romannumeral #1}
\newcommand{\Rmnum}[1]{\expandafter\@slowromancap\romannumeral #1@}
\makeatother

\definecolor{my_yellow}{RGB}{248,236,198}
\definecolor{my_green}{RGB}{197,224,180}

\definecolor{demphcolor}{RGB}{144, 144, 144}
\newcommand{\demph}[1]{\textcolor{demphcolor}{#1}}

\newcommand{\acronym}[1]{\underline{\textbf{#1}}}
\newcommand{\modelname}{NBP\xspace}

\newcommand{\rsh}[1]{\textcolor{orange}{[#1 - rsh]}}

% The \icmltitle you define below is probably too long as a header.
% Therefore, a short form for the running title is supplied here:
\icmltitlerunning{Next Block Prediction: Video Generation via Semi-Autoregressive Modeling}

\begin{document}

\twocolumn[
\icmltitle{Next Block Prediction: Video Generation via Semi-Autoregressive Modeling}

% It is OKAY to include author information, even for blind
% submissions: the style file will automatically remove it for you
% unless you've provided the [accepted] option to the icml2025
% package.

% List of affiliations: The first argument should be a (short)
% identifier you will use later to specify author affiliations
% Academic affiliations should list Department, University, City, Region, Country
% Industry affiliations should list Company, City, Region, Country

% You can specify symbols, otherwise they are numbered in order.
% Ideally, you should not use this facility. Affiliations will be numbered
% in order of appearance and this is the preferred way.
\icmlsetsymbol{equal}{*}

\begin{icmlauthorlist}
\icmlauthor{Shuhuai Ren}{1}
\icmlauthor{Shuming Ma}{2}
\icmlauthor{Xu Sun}{1}
\icmlauthor{Furu Wei}{2}
\end{icmlauthorlist}

\icmlaffiliation{1}{National Key Laboratory for Multimedia Information Processing, School of Computer Science, Peking University}
\icmlaffiliation{2}{Microsoft Research}

\icmlcorrespondingauthor{Xu Sun}{xusun@pku.edu.cn}
\icmlcorrespondingauthor{Furu Wei}{fuwei@microsoft.com}

\hypersetup{urlcolor=black}
\begin{center}
\url{https://renshuhuai-andy.github.io/NBP-project/}
% \vspace{-2em} 
\end{center}

% You may provide any keywords that you
% find helpful for describing your paper; these are used to populate
% the "keywords" metadata in the PDF but will not be shown in the document
\icmlkeywords{Machine Learning, ICML}

\vskip 0.3in
]

% this must go after the closing bracket ] following \twocolumn[ ...

% This command actually creates the footnote in the first column
% listing the affiliations and the copyright notice.
% The command takes one argument, which is text to display at the start of the footnote.
% The \icmlEqualContribution command is standard text for equal contribution.
% Remove it (just {}) if you do not need this facility.

\printAffiliationsAndNotice{}  % leave blank if no need to mention equal contribution
% \printAffiliationsAndNotice{\icmlEqualContribution} % otherwise use the standard text.

\begin{abstract}
Next-Token Prediction (NTP) is a de facto approach for autoregressive (AR) video generation, but it suffers from suboptimal unidirectional dependencies and slow inference speed.  
In this work, we propose a semi-autoregressive (semi-AR) framework, called \acronym{N}ext-\acronym{B}lock \acronym{P}rediction (\modelname), for video generation. 
By uniformly decomposing video content into equal-sized blocks (e.g., rows or frames), we shift the generation unit from individual tokens to blocks, allowing each token in the current block to simultaneously predict the corresponding token in the next block. 
Unlike traditional AR modeling, our framework employs bidirectional attention within each block, enabling tokens to capture more robust spatial dependencies. By predicting multiple tokens in parallel, \modelname models significantly reduce the number of generation steps, leading to faster and more efficient inference. 
Our model achieves FVD scores of 103.3 on UCF101 and 25.5 on K600, outperforming the vanilla NTP model by an average of 4.4. 
Furthermore, thanks to the reduced number of inference steps, the \modelname model generates 8.89 frames (128$\times$128 resolution) per second, achieving an 11$\times$ speedup. We also explored model scales ranging from 700M to 3B parameters, observing significant improvements in generation quality, with FVD scores dropping from 103.3 to 55.3 on UCF101 and from 25.5 to 19.5 on K600, demonstrating the scalability of our approach.
\end{abstract}

\section{Introduction}
% ar建模和decoder-only架构在文本领域很强,将其它模态(特别是视频)的建模也引入,构造一套统一的结构一直是大家的梦想
% 之前有些工作将ar直接应用于视频生成,例如光栅顺序扫描,但这种建模方式不太理想
% 我们提出了semi-ar的构想,基于之前的帧序列直接预测下一个帧,而非单个token
% 效果
The advance of Large Language Models (LLMs) such as ChatGPT~\citep{chatgpt}, GPT-4~\citep{Achiam2023GPT4TR} and LLaMA~\citep{Touvron2023LLaMAOA} has cemented the preeminence of Autoregressive (AR) modeling in the realm of natural language processing (NLP). This AR modeling approach, combined with the decoder-only Transformer architecture~\citep{Vaswani2017AttentionIA}, has been pivotal in achieving advanced levels of linguistic understanding, generation, and reasoning~\citep{wei2022emergent, gpt-o1, chen-etal-2024-pca}. 
Recently, there is a growing interest in extending AR modeling from language to other modalities, such as images and videos, to develop a unified multimodal framework~\citep{gpt4o, Team2024ChameleonME, Lu2023UnifiedIO2S, Wu2023NExTGPTAM, chen2024next}. Such an AR-based framework brings numerous benefits: (1) It allows for the utilization of the well-established infrastructure and learning recipes from the LLM community~\citep{Dao2022FlashAttentionFA, kwon2023efficient}; (2) The scalability and generalizability of AR modeling, empirically validated in LLMs~\citep{Kaplan2020ScalingLF, Yu2023ScalingAM}, can be extended to the multimodal domains to strengthen models~\citep{henighan2020scaling}; (3) Cognitive abilities observed in LLMs can be transferred and potentially amplified with multimodal data, moving closer to the goal of artificial general intelligence~\citep{bubeck2023sparks}.

Given the inherently autoregressive nature of video data in temporal dimensions, video generation is a natural area for extending AR modeling. 
Vanilla AR methods for video generation typically follow the Next-Token Prediction (NTP) approach, i.e., tokenize video into discrete tokens, then predict each subsequent token based on the previous ones. 
However, this approach has notable limitations. First, the generation order of NTP often follows a unidirectional raster-scan pattern~\citep{hong2022cogvideo, Wang2024OmniTokenizerAJ, Yan2021VideoGPTVG}, which fails to capture strong 2D correlations within video frames, limiting the modeling of spatial dependencies~\citep{Tian2024VisualAM}. Second, NTP necessitates a significant number of forward passes during inference (e.g., 1024 steps to generate a 16-frame clip), which reduces efficiency and increases the risk of error propagation~\citep{bengio2015scheduled}.

In this work, we propose a semi-autoregressive (semi-AR) framework, called \acronym{N}ext-\acronym{B}lock \acronym{P}rediction (\modelname), for video generation. 
To better model local spatial dependencies and improve inference efficiency, our framework shifts the generation unit from individual tokens to blocks (e.g., rows or frames). The objective is also redefined from next-token to next-block prediction, where each token in the current block simultaneously predicts the corresponding token in the next block. 
In contrast to the vanilla AR framework, which attends solely to prefix tokens, our \modelname approach allows tokens to attend to all tokens within the same block via bidirectional attention, thus capturing more robust spatial relationships. By predicting multiple tokens in parallel, \modelname models significantly reduce the number of generation steps, resulting in faster and more computationally efficient inference. 

Experimental results on the UCF-101~\citep{Soomro2012UCF101AD} and Kinetics-600 (K600)~\citep{Carreira2018ASN} datasets demonstrate the superiority of our semi-AR framework. With the same model size (700M parameters), \modelname achieves a 103.3 FVD on UCF101 and a 25.5 FVD on K600, surpassing the vanilla NTP model by 4.4. Additionally, due to the reduced number of inference steps, \modelname models can generate 8.89 frames (128$\times$128 resolution) per second, achieving an 11$\times$ speedup in inference. Compared to previous state-of-the-art token-based models, our approach proves to be the most effective. Scaling experiments with models ranging from 700M to 3B parameters show a significant improvement in generation quality, with FVD scores dropping from 103.3 to 55.3 on UCF101 and from 25.5 to 19.5 on K600, highlighting the scalability of the framework. We hope this work inspires further advancements in the field.

\begin{figure*}[htbp]
    \centering
    \begin{minipage}[b]{0.39\textwidth}
        \centering
        \includegraphics[width=.9\textwidth]{figs/tokenizer.pdf}
        \caption{3D discrete token map produced by our video tokenizer. The input video consists of \colorbox{my_yellow}{one initial frame}, followed by $n$ \colorbox{my_green}{clips}, with each clip containing $F_T$ frames. $x^{(i)}_{j}$ indicates the $j^{th}$ video token in the $i^{th}$ clip.}
        \label{fig:tokenizer}
    \end{minipage}
    \hfill
    \begin{minipage}[b]{0.59\textwidth}
        \centering
        \includegraphics[width=\textwidth]{figs/block.pdf}
        \caption{Examples of block include token-wise, row-wise, and frame-wise representations. When the block size is set to 1$\times$1$\times$1, it degenerates into a token, as used in vanilla AR modeling. Note that the actual token corresponds to a 3D cube, we omit the time dimension here for clarity.}
        \label{fig:block_example}
    \end{minipage}
\end{figure*}

\section{Related Work}
\paragraph{Video Generation.}
Prevalent video generation frameworks in recent years include Generative Adversarial Networks (GANs)~\citep{Yu2022GeneratingVW, Skorokhodov2021StyleGANVAC}, diffusion models~\citep{Ho2022ImagenVH, Ge2023PreserveYO, Gupta2023PhotorealisticVG, Yang2024CogVideoXTD}, autoregressive models~\citep{hong2022cogvideo, Yan2021VideoGPTVG, Kondratyuk2023VideoPoetAL}, etc. 
GANs can generate videos with rich details and high visual realism, but their training is often unstable and prone to mode collapse. In contrast, diffusion models exhibit more stable training processes and typically produce results with greater consistency and diversity~\citep{Yang2022DiffusionMA}. 
Nevertheless, AR models demonstrate significant potential for processing multi-modal data (e.g., text, images, audio, and video) within a unified framework, offering strong scalability and generalizability. To align with the trend of natively multimodal development~\citep{gpt4o}, this paper focuses on exploring video generation using AR modeling.

\paragraph{Autoregressive Models for Video Generation.}
With the success of the GPT series models~\citep{Brown2020LanguageMA}, a range of studies has applied AR modeling to both image~\citep{Chen2020GenerativePF, Lee2022AutoregressiveIG, wang2024loong, pang2024randar} and video generation~\citep{hong2022cogvideo, Wang2024OmniTokenizerAJ, Yan2021VideoGPTVG}. 
For image generation, traditional methods divide an image into a sequence of tokens following a raster-scan order and then predict each subsequent token based on the preceding ones. In video generation, this process is extended frame by frame to produce temporally-coherence content. 
However, conventional AR models predict only one token at a time, resulting in a large number of forward steps during inference. This significantly impairs the generation speed, especially for high-resolution images or videos containing numerous tokens~\citep{liu2024lumina-mgpt}.

\paragraph{Semi-Autoregressive Models.}
To improve the efficiency of AR models, early NLP researchers has explored semi-autoregressive modeling by generating spans of tokens instead of individual tokens per step~\citep{wang2018semi}. However, due to the variable length of text generation targets, it is challenging to predefine span sizes. Furthermore, fixed-length spans can disrupt semantic coherence and completeness, leading to significant degradation in generation quality; for instance, using a span length of 6 results in a 12\% drop in performance for English-German translation tasks~\citep{wang2018semi}.
More advanced semi-AR approaches, such as parallel decoding~\citep{Stern2018BlockwisePD} and speculative decoding~\citep{Xia2022SpeculativeDE}, typically use multiple output heads or additional modules (e.g., draft models) to predict several future tokens based on the last generated token~\citep{Gu2017NonAutoregressiveNM, Gloeckle2024BetterF}. 
In the context of video, where content can be uniformly decomposed into equal-sized blocks (e.g., row by row or frame by frame), we propose a framework where each token in the last block predicts the corresponding token in the next block, without requiring additional heads or modules. 

% To improve the efficiency of AR models, researchers in the NLP field have explored semi-autoregressive modeling~\citep{wang2018semi}, parallel decoding~\citep{Stern2018BlockwisePD} and speculative decoding~\citep{Xia2022SpeculativeDE} algorithms. These methods typically use multiple output heads or additional modules (e.g., draft models) to predict several future tokens based on the last generated token~\citep{Gu2017NonAutoregressiveNM, Gloeckle2024BetterF}. Given that video content can be uniformly decomposed into blocks of equal size (e.g., row by row or frame by frame), we propose a framework where each token in the last block predicts the corresponding token in the next block, without requiring additional heads or modules. 

\paragraph{Multi-token Prediction in Image Generation.}
Recent work in the image generation field has also shown a pattern of multi-token prediction, albeit with different motivations and approaches. 
For example, VAR~\citep{Tian2024VisualAM} employs a coarse-to-fine strategy across resolution scales, whereas our method processes spatiotemporal blocks at original resolution, achieving over 2$\times$ token efficiency (256 vs. 680 tokens for a 256$\times$256 frame). 
Unlike MAR~\citep{Li2024AutoregressiveIG}, which relies on randomized masking (70\% mask rate) and suffers from partial supervision (30\% of unmasked tokens do not receive supervision), our approach eliminates mask token modeling entirely, ensuring full supervision and improved training efficiency. 
While other works explore specialized token combinations~\citep{li2023lformer,wang2024parallelized}, our method minimizes architectural priors, enabling seamless adaptation from pre-trained NTP models and superior performance, especially for video generation.


\section{Method}
In this section, we first introduce our video tokenizer $\S$~\ref{subsec:video-tokenization}, highlighting its two key features: joint image-video tokenization and temporal causality, both of which facilitate our semi-AR modeling approach. 
Next, we provide a detailed comparison between vanilla Next-Token Prediction (NTP) ($\S$~\ref{subsec:ar}) and our \acronym{N}ext-\acronym{B}lock \acronym{P}rediction (\modelname) modeling ($\S$~\ref{subsec:semi-ar}). 
Our \modelname framework employs a block-wise objective function and attention masking, enabling more efficient capture of spatial dependencies and significantly improving inference speed.

\subsection{Preliminary I: Video Tokenization}
\label{subsec:video-tokenization}
We reproduce closed-source MAGVITv2~\cite{yu2023language} as our video tokenizer, which is based on a causal 3D CNN architecture. 
Given a video $\mathbf{X} \in \mathbb{R}^{T \times H \times W \times 3}$ in RGB space,\footnote{Images can be considered as ``static'' videos with $T=1$.} MAGVITv2 encodes it into a feature map $\mathbf{Z} \in \mathbb{R}^{T' \times H' \times W' \times d}$, where $(T', H', W')$ is the latent size of $\mathbf{Z}$, and $d$ is the hidden dimension of its feature vectors.
After that, we apply a quantizer to convert this feature map $\mathbf{Z}$ into a discrete tokens map $\mathbf{Q} \in \mathbb{V}^{T' \times H' \times W'}$ (illustrated in Fig.~\ref{fig:tokenizer}), where $\mathbb{V}$ represents a visual vocabulary of size $|\mathbb{V}|=K$. 
After tokenization, these discrete tokens $\mathbf{Q}$ can be passed through a causal 3D CNN decoder to reconstruct the video $\hat{\mathbf{X}}$. 
We note that MAGVITv2 has two major advantages:

\paragraph{(1) Joint Image-Video Tokenization.} MAGVITv2 allows tokenizing images and videos with a shared vocabulary. 
To achieve this, the number of frames in an input video, $T$, must satisfy $T=1+n \times F_T$, meaning the video comprises an initial frame followed by $n$ clips, each containing $F_T$ frames. 
When $n=0$, the video contains only the initial frame, thus simplifying the video to an image. 
Both the initial frame and each subsequent clip are discretized into a $(1, H', W')$ token map. Therefore, the latent temporal dimension $T'$ of the token map $\mathbf{Q}$ equals to $1+n$, which achieves $F_T$ times downsampling ratio on the temporal dimension (except for the first frame). 
Additionally, $H' = \frac{H}{F_H}$ and $W' = \frac{W}{F_W}$, where $F_H, F_W$ are spatial downsampling factors.

\paragraph{(2) Temporal Causality.} The causal 3D CNN architecture ensures that the tokenization and detokenization of each clip depend only on the preceding clips, facilitating autoregressive modeling along the temporal dimension, which will be discussed further in $\S$~\ref{subsec:semi-ar}.
% \rsh{todo add finish sentence}

\begin{figure*}[tbp]
\centering
\includegraphics[width=.9\textwidth]{figs/framework.pdf}
\caption{Comparison between a vanilla autoregressive (AR) framework based on next-token prediction (left) and our semi-AR framework based on next-block prediction (right). $x^{(i)}_{j}$ indicates the $j^{th}$ video token in the $i^{th}$ block, with each block containing $L$ tokens. 
The dashed line in the right panel presents that the $L$ tokens generated in the current step are duplicated and concatenated with prefix tokens, forming the input for the next step's prediction during inference.}
\label{fig:framework}
\end{figure*}

\subsection{Preliminary II: Autoregressive Modeling for Video Generation}
\label{subsec:ar}
Inspired by the success of AR models in the field of NLP, previous work~\citep{Yan2021VideoGPTVG, Wu2021GODIVAGO, Wu2021NWAVS} has extended AR models to video generation. Typically, these methods flatten the 3D video token input $\mathbf{Q} \in \mathbb{V}^{T' \times H' \times W'}$ into a 1D token sequence. 
Let \colorbox{my_green}{$C^{(t)}=\{x^{(t)}_1, x^{(t)}_{2}, \dots, x^{(t)}_{L}\}$} be the set of tokens in the $t^{th}$ clip, where $L = H' \times W' = |C^{(t)}|$ is the total number of tokens in each clip, and every clip contains an equal number of tokens. 
Specially, when $t=0$, \colorbox{my_yellow}{$C^{(0)}$} denotes the first frame's tokens. 
Therefore, the 1D token sequence can be represented as 
$($
\colorbox{my_yellow}{$C^{(0)}$}
$, \dots,$
\colorbox{my_green}{$C^{(T')}$}
$)=($
\colorbox{my_yellow}{$x^{(0)}_1, x^{(0)}_2, \dots,x^{(0)}_L$}
$, \dots,$
\colorbox{my_green}{$x^{(T')}_1, x^{(T')}_2, \dots, x^{(T')}_L$}
$)$. 
In the AR framework, the next-token probability is conditioned on the preceding tokens, where each token $x^{(t)}_l$ depends only on its prefix $(x^{(<t)}_l, x^{(t)}_{<l})$. This unidirectional dependency allows the likelihood of the 1D sequence to be factorized as: 
\begin{equation}
\label{eq:ar}
p\left(x^{(0)}_1, \dots, x^{(T')}_L \right)
=\prod_{t=1}^{T'} \prod_{l=1}^{L} p\left(x^{(t)}_l \mid x^{(<t)}_l, x^{(t)}_{<l} \right)
\end{equation}
Since only one token is predicted per step, the inference process can become computationally expensive and time-consuming~\citep{liu2024lumina-mgpt}, motivating the exploration of more efficient methods, such as semi-AR models~\citep{wang2018semi}, to improve both speed and scalability.


\subsection{Semi-AR Modeling via Next Block Modeling}
\label{subsec:semi-ar}
In contrast to text, which consists of variable-length words and phrases, video content can be uniformly decomposed into equal-sized blocks (e.g., rows or frames). Fig.~\ref{fig:block_example} shows examples of token-wise, row-wise, and frame-wise block representations. 
Based on this, we propose a semi-autoregressive (semi-AR) framework named \acronym{N}ext-\acronym{B}lock \acronym{P}rediction (\modelname), where each token in the current block predicts the corresponding token in the next block. 
Fig.~\ref{fig:framework} illustrates an example of next-clip prediction, where each clip is treated as a block, and the next clip is predicted based on the preceding clips. 
This approach introduces two key differences compared to vanilla NTP modeling: 
\textbf{(1) Change in the generation target.} In \modelname, the $l^{th}$ token $x_l^{(t)}$ in the $t^{th}$ clip predicts $x_l^{(t+1)}$ in the next clip, rather than $x_{l+1}^{(t)}$ as in NTP. 
\textbf{(2) Increase in the number of generation targets.} Instead of predicting one token at a time, all $L$ tokens $x_{1:L}^{(t)}$ simultaneously predict the corresponding $L$ tokens $x_{1:L}^{(t+1)}$ in the next clip.
Accordingly, the \modelname objective function can be expressed as: 
\begin{equation}
\label{eq:semi-ar}
p\left(x^{(0)}_1, \ldots, x^{(T')}_L \right) 
= \prod_{t=1}^{T'} p\left( \colorbox{my_green}{$x_{1:L}^{(t)}$} \mid \colorbox{my_yellow}{$x_{1:L}^{(0)}$}, \ldots, \colorbox{my_green}{$x_{1:L}^{(t-1)}$} \right)
% = p\left(C^{(0)}, \ldots, C^{(T')}\right) 
% = \prod_{t=1}^{T'} p\left(C^{(t)} \mid C^{(0)}, \ldots, C^{(t-1)}\right)
\end{equation}
By adjusting the block size, the framework can generate videos using different generation units. To ensure the effectiveness of this approach, four key components are designed:

\paragraph{(1) Initial Condition.}
In NTP models, a special token (e.g., \texttt{[begin\_of\_video]}) is typically used as the initial condition. In the \modelname setting, we can introduce a block of special tokens to serve as the initial condition for generating the first block. 
However, our preliminary experiments revealed that learning the parallel generation from the special token block to the first block is quite challenging. To address this issue, we propose two methods:
\textbf{(i) Taking the first frame $C^{(0)}$ as the initial condition.} In practice, following~\citet{girdhar2023emu}, users can upload an image as the first frame, or call an off-the-shelf text-to-image model (e.g., SDXL~\citep{podell2023sdxl}) to generate it. 
\textbf{(ii) Adopting a hybrid generation process}~\citep{wang2024parallelized}. Specifically, we can use per-token AR generation for the tokens in the first block. After the first block is generated, we then shift to per-block semi-AR generation. 
In order to make a fair comparison with other baselines, we used method (ii) in our experiments rather than relying on an extra first frame. 
Lastly, we note that both NTP and \modelname models can accept various inputs (e.g., text) as additional conditions (see Fig.~\ref{fig:framework}).

\begin{figure}
    \centering
    \includegraphics[width=.9\linewidth]{figs/attn.pdf}
    \caption{Causal attention mask in NTP modeling v.s. block-wise attention mask in \modelname modeling. The x-axis and y-axis represent keys and queries, respectively.}
\label{fig:attn-mask}
\end{figure}


\paragraph{(2) Block-wise Attention.}
To better capture spatial dependency, we allow tokens to attend to all tokens within the same block via bidirectional attention. Fig.~\ref{fig:attn-mask} compares traditional causal attention in NTP modeling with block-wise attention in \modelname modeling. 

\paragraph{(3) Block Size and Block Shape.}
The size and shape of blocks significantly influence generation quality, prompting us to conduct a comprehensive ablation study in 
$\S$~\ref{subsec:ablation} to identify the optimal configuration. 
Generally, we exclude blocks that span multiple frames (block shape with $T>1$) for several reasons:
\textbf{(i) Temporal Compression Constraints}: Input videos are sampled at 8 FPS or 16 FPS and undergo 4$\times$ temporal downsampling during tokenization, resulting in substantial information compression along the temporal dimension. Modeling rapidly changing content simultaneously across frames presents considerable challenges. 
\textbf{(ii) Causal Temporal Dynamics}: Our goal for the \modelname framework is not only to excel in video generation but also to serve as a potential world model~\citep{bruce2024genie, ha2018world}. Since videos represent the world in spatiotemporal dimensions and temporal changes are inherently causal, we aim to preserve complete causality in the temporal dimension during generation. Using a block shape with $T=1$ avoids introducing bidirectional temporal attention, aligning with our philosophy of employing an autoregressive generator (a decoder-only transformer) and a tokenizer like MagVITv2 with $T=1$ as the temporal unit. Results in Table~\ref{tab:block_shape} confirm that the block shape with $T=1$ achieve superior model performance.


\paragraph{(4) Inference Process.}
To illustrate the inference process of next-block prediction, we consider a scenario where each block corresponds to a clip. As shown in the right panel of Fig.~\ref{fig:framework}, during inference, the last $L$ tokens of the current output represent the predicted tokens for the next block.
These tokens are retained and concatenated with clip prefix, forming the input for the next step.  
By transitioning from token-by-token to block-by-block prediction, the \modelname framework leverages parallelization, reducing the number of generation steps by a factor of $L$, thereby decreasing computational cost and accelerating inference. 


% Based on the above designs, we summary the training and inference features of our semi-AR framework: 
% \paragraph{Training Dynamics.}
% % follow rvq
% % compared to ar
% % compared to mar, dense supervised signal
% Though there exists other work prediction multiple tokens per step, e.g., MAR~\citep{Li2024AutoregressiveIG}
% With block-level prediction, each training step provides a "denser" supervision signal, meaning the model receives feedback on multiple tokens simultaneously. This leads to more efficient learning, as the model updates based on richer information at each step, improving convergence during training.


\section{Experiments}
\subsection{Experimental Setups}


\paragraph{Video Tokenizer.}
As MAGVITv2 is not open-sourced, we implemented it based on the original paper. In contrast to the official implementation, which utilizes LFQ~\citep{yu2023language} as its quantizer, we adopt FSQ~\citep{Mentzer2023FiniteSQ} due to its simplicity and reduced number of loss functions and hyper-parameters. Following the original paper's recommendations, we set the FSQ levels to $[8, 8, 8, 5, 5, 5]$, and the size of the visual vocabulary is 64K. 
Moreover, we employ PatchGAN~\citep{Isola2016ImagetoImageTW} instead of StyleGAN~\citep{Karras2018ASG} to enhance training stability.  
The reconstruction performance of our tokenizer is presented in Table~\ref{tab:video_reconstruction}, and additional training details are available in Appendix~\ref{app:model}. We note that MAGVITv2 is not open-sourced, we have made every effort to replicate its results. 
Our tokenizer surpasses OmniTokenizer~\cite{Wang2024OmniTokenizerAJ}, MAGVITv1~\cite{yu2023magvit}, and other models in performance. However, due to limited computational resources, we did not pre-train on ImageNet~\citep{Russakovsky2014ImageNetLS} or employ a larger visual vocabulary (e.g., 262K as in the original MAGVITv2), which slightly impacts our results compared to the official MAGVITv2. 
Nevertheless, we note that the primary objective of this paper is to validate the semi-AR framework, rather than to achieve state-of-the-art tokenizer performance.


\paragraph{Generator Training Details.}
We train decoder-only transformers on 17-frame videos with a resolution of 128$\times$128, using the UCF-101~\citep{Soomro2012UCF101AD} and K600~\citep{Carreira2018ASN} datasets. 
With spatial downsampling factors of $F_H=F_W=8$ and temporal downsampling of $F_T=4$, the resulting 3D token map for each video sample has dimensions $(T', H', W')=(5, 16, 16)$, yielding a total of 1280 tokens. 
We train our model for 100K steps with a total batch size of 256 and 64 respectively. 
Model sizes range from 700M to 3B parameters, with training spanning approximately two weeks on 32 NVIDIA A100 GPUs. The full model configuration and training hyper-parameters are provided in Appendix~\ref{app:model}. 
We train the models from scratch, rather than initializing from a pre-trained LLM checkpoint, as these text-based checkpoints provide minimal benefit for video generation~\citep{zhang2023pre}. 
We use LLaMA~\citep{Touvron2023LLaMAOA} vocabulary (32K tokens) as the text vocabulary and merge it with the video vocabulary (64K tokens) to form the final vocabulary. Since our primary focus is video generation, we compute the loss only on video tokens, which leads to improved performance. 

\begin{table*}[]
\centering
\caption{Comparison of next-token prediction (NTP) and next-block prediction (\modelname) models in terms of performance and speed, evaluated on the K600 dataset (5-frame condition, 12 frames (768 tokens) to predict). Inference time was measured on a single A100 Nvidia GPU. All models are implemented by us under the same setting and trained for 20 epochs. FPS denotes ``frame per second''. The measurement of inference speed includes tokenization and de-tokenization processes. KV-cache is used for both models.}
\label{table:semiar-ar-scale}
% \setlength{\tabcolsep}{2.0pt}
% \vspace{0.04in}
\begin{adjustbox}{max width=\linewidth}

\begin{tabular}{@{}c|l|c|c|cc@{}}
\toprule
Model Size & Modeling Method & \# Block size & FVD $\downarrow$ & \# Forward steps & Inference speed (FPS) $\uparrow$ \\ \midrule
\multirow{2}{*}{700M}          & NTP     & 1 (1$\times$1$\times$1)   & 37.4 & 768           & 0.80                  \\
                               & \modelname (Ours)  & 16 (1$\times$1$\times$16) & \textbf{33.6} & \textbf{48}            & \textbf{8.89}                  \\ \midrule
\multirow{2}{*}{1.2B}          & NTP  & 1 (1$\times$1$\times$1)    & 31.4 & 768           & 0.75                  \\
                               & \modelname (Ours)  & 16 (1$\times$1$\times$16) & \textbf{28.6} & \textbf{48}            & \textbf{6.70}                  \\ \midrule
\multirow{2}{*}{3B}            & NTP  & 1 (1$\times$1$\times$1)    & 29.0 & 768           & 0.60                  \\
                               & \modelname (Ours)  & 16 (1$\times$1$\times$16)  & \textbf{26.5} & \textbf{48}            & \textbf{4.29}                  \\ \bottomrule
\end{tabular}


\end{adjustbox}
\end{table*}



\paragraph{Evaluation Protocol.}
We evaluate our models on the UCF-101 dataset for class-conditional generation task and the K600 dataset for frame prediction task. To assess video quality, we use the standard metric of Fréchet Video Distance (FVD)\cite{unterthiner2018towards}. Additional evaluation details can be found in Appendix\ref{app:eval}.


% \begin{figure}
%     \centering
%     \includegraphics[width=.8\linewidth]{figs/k600_700M_1B_3B.pdf}
% \caption{Validation loss of various sizes of semi-AR models from 700M to 3B. \rsh{change loss curve}}
% \label{fig:model_para}
% \end{figure}

% \begin{figure}[htbp]
% \includegraphics[width=\linewidth]{figs/k600_block_size.pdf}
% \caption{Training loss of various block sizes from 1 to 256. }
% \label{fig:block_size}
% \end{figure}


\begin{figure}[htbp]
    \centering
    \begin{minipage}[b]{0.49\linewidth}
        \centering
        \includegraphics[width=\linewidth]{figs/k600_700M_1B_3B_500k.pdf}
        \caption{Validation loss of various sizes of semi-AR models from 700M to 3B. %\rsh{change loss curve}
        }
\label{fig:model_para}
    \end{minipage}
    \hfill
    \begin{minipage}[b]{0.49\linewidth}
        \centering
        \includegraphics[width=\linewidth]{figs/k600_block_size_500k.pdf}
        \caption{Validation loss of various block sizes from 1 to 256. }
        \label{fig:block_size}
    \end{minipage}
\end{figure}

\subsection{Comparison of Next-Token Prediction and Next-Block Prediction}

We first conduct a fair comparison between next-token prediction (NTP) and our next-block prediction (\modelname) under the same experimental setting. 
All experiments are performed on the K600 dataset, which has a much larger data volume compared to UCF-101 (413K vs. 9.5K) and features a strict training-test split, thereby ensuring more generalizable results. 
Table~\ref{table:semiar-ar-scale} highlights the superiority of our approach in three key aspects: generation quality, inference efficiency, and scalability.

\paragraph{Generation Quality.}
Across all model sizes, \modelname with a 1$\times$1$\times$16 block size consistently outperforms NTP models in terms of generation quality (measured by FVD). For instance, the 700M \modelname model achieves an FVD of 33.6, outperforming the NTP model by 3.8 points. Furthermore, a \modelname model with only 1.2B parameters achieves a comparable performance to a 3B NTP model (28.6 vs. 29.0 FVD). This suggests that the block size of 1$\times$1$\times$16 is a more effective generation unit for autoregressive modeling in the video domain. 

\documentclass{MITstyle}

%\usepackage[table]{xcolor}
\usepackage{chngcntr}
\usepackage{hyperref}
\usepackage{microtype}

\title{A Lightweight and Extensible Cell Segmentation and Classification Model for Whole Slide Images}

\author{Nikita Shvetsov~$^{1, }$\footnote{Correspondence e-mail: nikita.shvetsov@uit.no}, Thomas K. Kilvaer~$^{2, 3}$, Masoud Tafavvoghi~$^{4}$, Anders Sildnes~$^{1}$, \\ Kajsa Møllersen~$^{4}$, Lill-Tove Rasmussen Busund~$^{5, 6}$, Lars Ailo Bongo~$^{1}$ \\
%
\vspace{1em} % Space between authors and afilliations
%
\normalfont{\small $^{1}$Department of Computer Science, UiT The Arctic University of Norway}\\
\normalfont{\small $^{2}$Department of Oncology, University Hospital of North Norway}\\
\normalfont{\small $^{3}$Department of Clinical Medicine, UiT The Arctic University of Norway}\\
\normalfont{\small $^{4}$Department of Community Medicine, UiT The Arctic University of Norway}\\
\normalfont{\small $^{5}$Department of Medical Biology, UiT The Arctic University of Norway} \\
\normalfont{\small $^{6}$Department of Clinical Pathology, University Hospital of North Norway} %\vspace{2em}
}

\begin{document}
\maketitle

\section*{Abstract}

% \begin{abstract}
% Developing clinically useful cell-level analysis tools in digital pathology remains challenging due to limitations in dataset granularity, inconsistent annotations, computational demands of advanced models, and difficulties in integrating new technologies into clinical workflows. To address these challenges, we propose a multi-faceted solution that enhances data quality, model performance, and usability to create a lightweight and extensible cell segmentation and classification model.

% First, we update data labels by employing a cross-relabeling process that refines the labels of two existing datasets, PanNuke and MoNuSAC, to create a new unified dataset with enhanced granularity, encompassing seven distinct cell types. Second, we leverage the H-Optimus foundation model as a fixed encoder to improve feature representation for simultaneous cell segmentation and classification tasks. Third, to address the computational demands of foundation models, we employ knowledge distillation to reduce model size and complexity while maintaining comparable performance. Finally, to facilitate integration into clinical workflows, we integrate the distilled model into the QuPath software, a widely used open-source platform in digital pathology.

% Our results demonstrate improvements in cell segmentation and classification performance using the H‑Optimus-based model compared to a CNN-based model. Specifically, the average $R^2$ improved from 0.575 to 0.871, and the average $PQ$ score improved from 0.450 to 0.492, indicating better alignment with actual cell counts and enhanced segmentation and classification quality. Furthermore, the distilled student model maintains performance comparable to the larger foundation model while reducing the parameter count by a factor of 48.
% Overall, by reducing computational complexity and integrating it into existing workflows, the proposed approach may significantly impact diagnostic processes, reduce the workload of pathologists, and contribute to improved patient outcomes. Though our approach shows potential enhancements in efficiency and usability of cell segmentation and classification models in digital pathology, extensive validation is needed to deploy these models in clinical practice.
% \end{abstract}

%%% shortened abstract
\begin{abstract}
Developing clinically useful cell-level analysis tools in digital pathology remains challenging due to limitations in dataset granularity, inconsistent annotations, high computational demands, and difficulties integrating new technologies into workflows. To address these issues, we propose a solution that enhances data quality, model performance, and usability by creating a lightweight, extensible cell segmentation and classification model. 

First, we update data labels through cross-relabeling to refine annotations of PanNuke and MoNuSAC, producing a unified dataset with seven distinct cell types. Second, we leverage the H-Optimus foundation model as a fixed encoder to improve feature representation for simultaneous segmentation and classification tasks. Third, to address foundation models' computational demands, we distill knowledge to reduce model size and complexity while maintaining comparable performance. Finally, we integrate the distilled model into QuPath, a widely used open-source digital pathology platform. 

Results demonstrate improved segmentation and classification performance using the H-Optimus-based model compared to a CNN-based model. Specifically, average $R^2$ improved from 0.575 to 0.871, and average $PQ$ score improved from 0.450 to 0.492, indicating better alignment with actual cell counts and enhanced segmentation quality. The distilled model maintains comparable performance while reducing parameter count by a factor of 48. By reducing computational complexity and integrating into workflows, this approach may significantly impact diagnostics, reduce pathologist workload, and improve outcomes. Although the method shows promise, extensive validation is necessary prior to clinical deployment.
\end{abstract}
\clearpage

\section{Introduction}
In digital pathology, accurate segmentation and classification of cells are crucial for many diagnostic, prognostic, and predictive analyses \cite{Jaber_Beziaeva_etal._2019,Lin_Pan_etal._2022,Park_Ock_etal._2022,Shen_Choi_etal._2024}. Nowadays, developments in computational pathology offer multiple solutions \cite{H._Qu_P._Wu_etal._2020,Javed_Mahmood_etal._2020} to utilize cell-level datasets to train machine learning models that solve these problems. The quality and specificity of training datasets are critical for robust and accurate models. Adhering to the principle of "garbage in, garbage out", it is essential to ensure that these datasets are extensively and accurately labeled with distinct classes that reflect the diverse biological characteristics of different cell types. Unfortunately, the number of open-source datasets comprising such high-quality annotations is limited. Existing cell segmentation datasets \cite{Gamper_Koohbanani_etal._2019,Graham_Vu_etal._2019,Verma_Kumar_etal._2021} may offer extensive annotations for certain cell types while providing more general labels for others. For example, in PanNuke, which is one of the largest open-source datasets comprising labeled cells, various types of morphologically and functionally different inflammatory cells like macrophages and lymphocytes are clustered in a broad "inflammatory" class. Consequently, these classes are frequently omitted from analyses or aggregated into broader meta-classes \cite{Gamper_Koohbanani_etal._2020} and likely interfere with other cell classes included in the dataset. This and similar inconsistencies in annotation granularity limit the ability of machine learning models to learn the comprehensive and nuanced features necessary for accurate cell segmentation and classification. To address these challenges, methods for refining and standardizing dataset annotations are essential to enhance the quality of training data.

A complementary approach to mitigate the absence of high-quality training data is the use of foundation models. Foundation models as encoders are defined as large-scale, versatile networks pre-trained on vast, diverse datasets using self-supervised learning, contrasting with convolutional neural network (CNN) pre-trained encoders that rely on supervised learning with labeled data. In practice, foundation models leverage enormous amounts of weakly or unlabeled data from millions of whole slide images (WSIs) and employ self-attention mechanisms to capture long-range dependencies and global context \cite{Chen_Ding_etal._2024,Saillard_Jenatton_etal._2024,Vorontsov_Bozkurt_etal._2024,Xu_Usuyama_etal._2024}. As a consequence, foundation models are able to produce transferable feature representations across different cell types and tissue environments. The feature representations can be leveraged by decoder networks to produce segmentation masks and pixel-level classifications. Because foundation models have comprehensive feature representations, they can be effectively fine-tuned using much smaller amounts of cell-level data compared to the large datasets needed to train models from scratch. Furthermore, foundation models incorporate adversarial training elements or contrastive learning \cite{Chen_Ding_etal._2024,Xu_Usuyama_etal._2024}, enhancing their resilience and adaptability by exposing them to challenging and varied scenarios during training. This may result in more generalizable models, often making them well-suited for diverse and complex tasks in digital pathology.

Despite the inherent advantages of foundation models, their deployment for practical use faces its own obstacles. In particular, they require substantial computational power, financial investments and rigorous testing to ensure reliability and efficacy for a given task \cite{Akkus_Dangott_etal._2022,Dragomir_Cocuz_etal._2022,Go_2022,Jafri_Farooqui_etal._2024}. Moreover, while foundation models enhance feature representation and performance, they depend on the quality of available annotations for decoder fine-tuning and, like any other model, cannot resolve existing inconsistencies or ambiguities in data labels. Therefore, there remains a critical need for solutions that address both data quality and practical deployment considerations.
Further, integrating new technologies into existing clinical workflows often encounters resistance, as it necessitates adjustments to established diagnostic processes. So, there is a need to develop solutions that could be integrated into current practices, minimizing the burden on medical professionals to adopt new tools \cite{King_Williams_etal._2023}.

Existing solutions \cite{Goldsborough_Philps_etal._2024,Hörst_Rempe_etal._2024}, while addressing some aspects of these challenges, fall short in providing a comprehensive approach. To address the data quality and clinical deployment issues, we propose a multi-faceted solution that encompasses data refinement, model optimization, and integration with existing pathology tools (\hyperref[fig:fig1]{Figure 1}). The outcome is a lightweight cell segmentation and classification model that can be integrated into digital pathology workflows for practical clinical use.

\begin{figure}[h!]
    \centering
    \includegraphics[width=\textwidth, height=0.82\textheight, keepaspectratio]{images/Figure_1.pdf}
    \caption{Overview of the proposed solution, including 1) Data refinement using cross-relabeling, 2) Teacher model development and fine tuning, 3) Student model optimization with knowledge distillation and 4) Student model and QuPath integration}
    \label{fig:fig1}
\end{figure}
\clearpage

Our approach begins with preparing the data for the fine-tuning and training of the machine learning models. We create a refined dataset, acquired via cross-relabeling two cell-level datasets, enhancing annotation specificity and consistency of the labeled data. Subsequently, we create a cell segmentation and classification model based on the foundation model. We leverage the foundation model as a fixed encoder and fine-tune a decoder using the refined dataset to improve generalization across diverse tissue- and cell types.
To ensure that the model remains lightweight and deployable in a possibly resource-constrained environment, we employ knowledge distillation to approximate the functionality of the foundation model. Finally, to facilitate the practical application of our model in digital pathology workflows, we integrate it with the QuPath \cite{Bankhead_Loughrey_etal._2017} application. Each methodological component contributes to the overarching goal of enhancing model performance, generalizability, and usability in clinical settings.

The primary contributions of this paper are:
\begin{enumerate}
    \item \textit{Data labels refinement through cross-relabeling:}
    
    We propose a new method for refining labels of cell-level datasets through cross-relabeling. This method employs classification models to re-label broad and ambiguous instances, resulting in a more diverse dataset. Our evaluation demonstrates that these classification models achieve high accuracy on test subsets, indicating the reliability of the method for label refinement.

    \item \textit{Enhanced model performance via foundation models:}
    
    We employ a foundation model as a feature extractor for the cell segmentation and classification task. In comparison with training a CNN model from scratch, the foundation model backbone only needs fine-tuning, which significantly reduces training time, computational resources and data requirements. We show that using a foundation model encoder leads to better performance in cell segmentation and classification networks than using a CNN-based encoder. This improvement may enable the model to generalize more effectively across various tissue types and imaging methods.
    
    \item \textit{Model optimization through knowledge distillation:}
    
    We show that a smaller student model trained using knowledge distillation on the refined dataset obtained via our cross-relabeling approach from a foundation model achieves comparable performance in cell segmentation and quantification tasks. As a result, this model is more suitable for deployment in environments without high-performance computing resources.
    
    \item \textit{Integration with QuPath:}
    
    We integrate the distilled cell segmentation and classification model into QuPath, a widely used open-source digital pathology platform, to accelerate clinical adaptation by enabling pathologists to more easily incorporate advanced computational tools into their existing workflows.
\end{enumerate}

Through these methodological steps, we aim to bridge the gap between advanced machine learning techniques and practical clinical applications, making accurate and efficient digital pathology accessible in a broader range of healthcare settings.

\section{Refining Existing Datasets Using Cross-Relabeling}
To address the limitations of sparse and ambiguous labeling of cell-level datasets, we propose a generalizable cross-relabeling strategy that can be applied to any dataset containing broadly categorized or imprecisely labeled cell types. This approach involves training and subsequently leveraging classification models to refine broad categories into more specific or biologically relevant classes.
When applied to cell-level data, the methodology includes extracting individual cell images from the dataset patches, preprocessing these images to standardize the size and accommodate partial cells, and then training deep learning classifiers capable of distinguishing between the finer cell subtypes within the coarser categories. 
To illustrate our approach, we focus on the PanNuke \cite{Gamper_Koohbanani_etal._2020, Gamper_Koohbanani_etal._2019} and MoNuSAC \cite{Verma_Kumar_etal._2021} datasets that we have used to train models for cell quantification in our previous works \cite{Shvetsov_Grønnesby_etal._2022,Shvetsov_Sildnes_etal._2024}. We find that for better cell differentiation we have to introduce more granular labels. PanNuke includes a broad classification of "inflammatory" cells, encompassing lymphocytes, macrophages, and neutrophils. Each cell type differs significantly in structure, function, and clinical relevance. Conversely, MoNuSAC uses the label "epithelial" for a class that comprises both benign epithelial cells and malignant neoplastic cells. This practice makes it challenging to differentiate between benign and malignant epithelial cells in the dataset, which is a critical distinction when identifying tumor areas within tissue samples. To address these issues, we implement a cross-relabeling strategy as shown in \hyperref[fig:fig2]{Figure 2}. The key components are two classification models: one is trained on singular cell images from PanNuke data to classify the epithelial meta-class into epithelial and neoplastic classes. The other is trained on MoNuSAC to refine the inflammatory class into lymphocytes, neutrophils, and macrophages.

\begin{figure}[h!]
    \centering
    \includegraphics[width=\textwidth]{images/Figure_2.pdf}
    \caption{Refined dataset generation via cross relabeling}
    \label{fig:fig2}
\end{figure}

The refining approach consists of three consecutive steps. The first is the preprocessing step, in which we extract individual cells from both datasets (\hyperref[fig:fig3]{Figure 3}). The specifics of PanNuke and MoNuSAC patch preparation before cell preprocessing are provided in \hyperref[chap:S1]{Appendix S1}.

\begin{figure}[h!]
    \centering
    \includegraphics[width=\textwidth]{images/Figure_3.pdf}
    \caption{Cell instances preprocessing including (1) cell map extraction, (2) bounding box delineation, (3) adjusting cell boxes and (4) cropping and resizing of cell images}
    \label{fig:fig3}
\end{figure}

During preprocessing, we extract cell type maps from the ground truth label mask and calculate bounding boxes around each cell instance. To accommodate partial cells at patch borders, a common issue in cropped patch images, we employ mirror padding and extend the field of view of the cell label by 15 pixels to capture adjacent cells. We then crop and resize the identified regions to $64 \times 64$ pixels using bicubic interpolation.

The preprocessed PanNuke dataset comprises 68,031 neoplastic and 23,207 epithelial cell images, while MoNuSAC comprises  33,104 lymphocytes, 1,252 neutrophils, and 1,695 macrophages, which we subsequently use in training cell classification models and classifying the cell image data \hyperref[fig:S2]{Appendix Figure S2 (1)}. 

The next step is to train two distinct ResNet50-based classifiers tailored to address the specific labeling challenges inherent in each dataset. We use ResNet50 for classification models due to its proven effectiveness for image classification tasks in histopathology \cite{pan2022reviewmachinelearningapproaches}, and its compatibility with small images. For the PanNuke dataset, we design the classifier, trained on MoNuSAC data, to disaggregate the heterogeneous "inflammatory" cell category into distinct subtypes: lymphocytes, macrophages, and neutrophils. Similarly, for the MoNuSAC dataset, the classifier is trained on PanNuke data and distinguishes between benign and malignant epithelial cells within the overarching "epithelial" label. By applying these targeted classifiers to their respective datasets, we assign more specific labels to individual cell instances, thus enabling us to create a unified dataset.
To ensure a balanced representation of classes, we train both models on datasets that had been equalized to match the size of the least represented class. Thus, we obtain datasets comprising 23,207 samples per class for PanNuke and 1,252 samples per class for MoNuSAC data. Next, we partition both of them into training (70\%), validation (20\%), and testing (10\%) subsets. To mitigate the risk of overfitting, we use a single dropout layer with a rate of p=0.5 in both models and data augmentation using randomized color perturbations, rotation, and horizontal and vertical flipping. We employ AdamW optimizer and the cross-entropy loss function for the training criterion.

To evaluate the two trained models, we measure the classification accuracy on the respective test subsets. The accuracies on the test subset for both classifiers are presented in \hyperref[tab:1]{Table 1}. The PanNuke model achieves an average accuracy of 93.57\%, with higher accuracy for neoplastic cells (96.06\%) compared to epithelial cells (86.26\%). The confusion matrix in Figure A3.1 shows that the model predominantly distinguishes accurately between epithelial and neoplastic tissues, with a substantial number of correct classifications and relatively few misclassifications. The MoNuSAC model demonstrates an average accuracy of 98.92\%, excelling in classifying lymphocytes (99.67\%) and macrophages (94.12\%), with lower performance for neutrophils (85.71\%). The confusion matrix in Figure A3.2 shows that the model identifies lymphocytes and performs reasonably well with macrophages and neutrophils.

\begin{table}[h!]
\renewcommand{\arraystretch}{1.5}
  \centering
  \caption{Cell classification results for PanNuke and MoNuSAC trained models (CI 95\%).}
  \label{tab:1}
  \begin{tabular}{|l|c|c|}
   \hline
   %\rowcolor{gray!30}
    Accuracy               & PanNuke model              & MoNuSAC model              \\
    \hline
    Average      & 0.936 (0.931--0.941)         & 0.989 (0.986--0.993)        \\
    \hline
    Neoplastic   & 0.961 (0.956--0.965)         & -                          \\
    \hline
    Epithelial   & 0.863 (0.849--0.877)         & -                          \\
    \hline
    Lymphocytes  & -                          & 0.997 (0.995--0.999)        \\
    \hline
    Neutrophils  & -                          & 0.857 (0.796--0.918)        \\
    \hline
    Macrophages  & -                          & 0.941 (0.906--0.976)        \\
    \hline
  \end{tabular}
\end{table}

Finally, during the last step, we use the model trained on PanNuke data for epithelial cells in MoNuSAC and the model trained on MoNuSAC for the inflammatory cells class in PanNuke. Specifically, we use classifier models to relabel epithelial cells in MoNuSAC and inflammatory cells in PanNuke data. Then we combine cells with refined labels and the rest of the cells in both datasets to create a refined dataset (\hyperref[fig:S2]{Appendix Figure S2 (2)}). The process of relabeling cells and visualizing them on a patch is shown in \hyperref[fig:fig4]{Figure 4}. The cell counts in the refined dataset are provided in \hyperref[tab:S4]{Appendix Table S4}.

\begin{figure}[h!]
    \centering
    \includegraphics[width=\textwidth, height=0.42\textheight, keepaspectratio]{images/Figure_4.pdf}
    \caption{Cell relabeling procedure for epithelial and inflammatory cell classes}
    \label{fig:fig4}
\end{figure}

%\hfill

Relabeling and combining datasets have been explored in a prior study \cite{Parulekar_Kanwat_etal._2023}, where consecutive fine-tuning on multiple datasets was employed to account for hierarchical class label structures. While the method presented in \cite{Parulekar_Kanwat_etal._2023} is intuitive, it often lacks consistency and requires multiple fine-tuning runs, which can be cumbersome and time-consuming. 
In contrast, cross-relabeling simplifies this process by using specialized classification models tailored to each dataset's specific labeling challenges. This approach provides better transparency and produces a unified dataset encompassing seven distinct cell types across multiple tissue samples, enhancing data diversity for further model training or fine-tuning.

Despite these improvements, cross-relabeling does not entirely resolve issues related to poor labeling quality or the amount of labeled data. Specifically, our results show lower accuracies persist for underrepresented classes, such as macrophages, which may stem from a limited sample availability and intrinsic challenges in distinguishing these cells based solely on H\&E staining. Furthermore, while our method enhances label specificity, it relies on the initial quality of the broad labels; thus, any fundamental inaccuracies in the original annotations can propagate through the relabeling process. Addressing the overall problem of limited data labels may require integrating additional data sources or utilizing complementary immunohistochemical staining methods.
Although the reported performance metrics are obtained from evaluations on the native test sets of each dataset, it is important to note that the primary application of these classifiers is to perform cross-relabeling, where a model trained on one dataset (e.g., PanNuke) is applied to another (e.g., MoNuSAC) and vice versa. We acknowledge that a more systematic evaluation of cross-dataset generalization is needed and could be performed in future work.

Overall, the refined dataset produced by our approach can enhance the supervised training or fine-tuning of cell segmentation and classification models, especially those that utilize pre-trained foundation models to improve feature extraction robustness. In addition, these models can detect nuanced classes that enable researchers to conduct more detailed analyses of biological processes in computational pathology.

\section{Foundation models for robust cell segmentation and classification}

Accurate cell segmentation and classification in digital pathology are hindered by limited labeled data and the fact that conventional CNNs are unable to capture global contextual information due to their local receptive field constraints \cite{Gheflati_Rivaz_2022,Yang_Marcus_etal.}. Traditional approaches in cell quantification have predominantly relied on CNN encoders, such as ResNet50, given their proven effectiveness in semantic segmentation tasks \cite{Deshmane_2023,Graham_Vu_etal._2019,Mukasheva_Koishiyeva_etal._2024,Stringer_Wang_etal._2021}. However, approaches that include fine-tuning of pretrained CNNs, data augmentation, and stain normalization to partially increase data variability and address staining differences often fail to achieve the necessary generalization and robustness across diverse tissue types and staining conditions \cite{G._Wang_W._Li_etal._2018,Gao_Bagci_etal._2018,Karim_El_Khoury_Martin_Fockedey_etal._2021}.

To overcome these challenges, we leverage an encoder-decoder network that uses a foundation model as the encoder and a CNN upsampling decoder (\hyperref[fig:fig5]{Figure 5}) for simultaneous cell segmentation and classification in 2D patches extracted from WSIs. Foundation models with transformer-based architectures are viable alternatives to CNN-based encoders \cite{Shamshad_Khan_etal._2023,Sourget_2023}. They enable the creation of more advanced architectures that can decode or transform learned features more effectively \cite{Chen_Duan_etal._2023,Cheng_Misra_etal._2022,Xie_Wang_etal._2021}.

\begin{figure}[h!]
    \centering
    \includegraphics[width=\textwidth]{images/Figure_5.pdf}
    \caption{UNETR-like model with foundational model as backbone}
    \label{fig:fig5}
\end{figure}

By utilizing a transformer-based encoder, we incorporate global contextual information into the feature extraction process, which is a key advantage of such architectures \cite{Chen_Lu_etal._2021}. This foundation model integration facilitates accurate pixel-wise segmentation and classification without the need for extensive encoder training, thereby potentially improving generalization across varied cellular structures and tissue types.
In our implementation, we employ a modified UNETR \cite{Hatamizadeh_Tang_etal._2021} architecture that combines a vision transformer (ViT) \cite{Dosovitskiy_Beyer_etal._2021} encoder with a CNN-based decoder. The encoder utilizes the pretrained H-Optimus foundation model, which contains 1.1 billion parameters and is trained on over 500,000 H\&E stained WSIs \cite{Saillard_Jenatton_etal._2024}. We extract outputs from four evenly spaced transformer blocks $Z_i$, where $i \in [1, 14, 26, 38]$, to serve as residual connections for the CNN decoder. We select these blocks based on our observation that features from non-adjacent levels of the encoder lead to better overall performance on the test subset.

The CNN decoder upsamples the feature representations, acquired from the transformer blocks, to generate an intermediate vector that is handled by two task-specific layers that generate cell segmentation and classification masks. The first task-specific layer is the ‘Cellpose head’,  which is used to delineate cell instances. The layer generates horizontal and vertical gradient maps to form vector fields that are refined through gradient tracking in a post-processing step using the Cellpose algorithm \cite{Stringer_Wang_etal._2021}, known for its efficacy in cell segmentation tasks and generalizability across multiple domains \cite{Pachitariu_Stringer_2022,Stringer_Pachitariu_2024}. The second task-specific layer is the "Cell type head", which assigns labels to individual pixels. In the post-processing step, we determine the output classification label of each segmented cell instance by majority voting over the labeled pixels that comprise the cell in the segmentation map.

To evaluate model performance and measure the impact of adding a foundation model as backbone, we compare it to a ResNet50-based model. ResNet50 is a widely used solution for encoders in segmentation architectures in the medical domain \cite{Deshmane_2023,Graham_Vu_etal._2019,Mukasheva_Koishiyeva_etal._2024,Stringer_Wang_etal._2021}. For the H-Optimus-based model, we utilize frozen weights for the encoder and only fine-tune the decoder to take advantage of the extensive pre-training of the foundation model. For the ResNet50-based model we start with ImageNet \cite{Deng_Dong_etal.} weights and train both encoder and decoder parts. Hyperparameters for the training step are set to be identical, where possible, for comparable evaluation. 
For this evaluation, we deliberately use the PanNuke dataset to provide a standardized and controlled comparison between the H‑Optimus and ResNet50-based models (\hyperref[fig:S2]{Appendix Figure S2 (3)}). Specifically, we use two of the default PanNuke dataset splits (66\%) for training and validation, and reserve the third split (33\%) for testing.

To address the challenge of cell class imbalance in the PanNuke dataset, which is a common characteristic in most cell-level H\&E patch datasets, both models’ training processes employ a weighted loss function comprising cross-entropy and focal loss \cite{Lin_Goyal_etal._2018}. The focal loss component is adjusted with coefficients derived from each cell class' instance frequency, emphasizing learning from underrepresented classes and enhancing the model's sensitivity to rare but significant cellular patterns. The cross-entropy loss is augmented with spectral decoupling regularization \cite{Pezeshki_Kaba_etal._2021,Pohjonen_Stürenberg_etal._2022} and spatially varying label smoothing \cite{Islam_Glocker_2021}, which potentially stabilizes training and improves generalization in case of complex tissue morphologies. For optimization, we employ the \textit{AdamW} \cite{Loshchilov_Hutter_2019} to counter unbalanced class scenarios, with cosine annealing learning rate scheduler.

We utilize the scikit-learn library \cite{Van_der_Walt_Schönberger_etal._2014} and HoVer-Net \cite{Graham_Vu_etal._2019} implementations of $R^2$ (the coefficient of determination) and $PQ$ (panoptic quality) to evaluate our experiments. Complete mathematical formulations and detailed explanations of these metrics are provided in \hyperref[chap:S5]{Appendix S5}. To compute confidence intervals, we use nonparametric bootstrapping, where after calculating the metric on the full sample, we generated 1000 bootstrap replicates by resampling with replacement and then determined the 95\% confidence intervals as the 2.5th and 97.5th percentiles of the resulting empirical distribution.

%\hfill

The model comparisons are summarized in \hyperref[tab:2]{Table 2}. The H‑Optimus-based model achieves higher $R^2$ across all cell classes compared to the ResNet50-based model, which means that its predictions are more closely aligned with the PanNuke cell counts, indicating a stronger correlation with the observed data. Notably, the improvement of $R^2_{dead}$ may be an indicator of better global contextual representations provided by the foundation model backbone. In terms of segmentation and classification quality combined, measured by the PQ score, the H‑Optimus-based model demonstrates notable improvements across most cell classes. Overall, the average $R^2$ improved from 0.575 to 0.871, while the average $PQ$ score improved from 0.450 to 0.492, demonstrating better performance of the H-Optimus-based model.

\begin{table}[h!]
\renewcommand{\arraystretch}{1.5}
  \centering
  \caption{Cell quantification metrics for baseline and proposed models (CI 95\%).}
  \label{tab:2}
  \begin{tabular}{|l|c|c|}
    \hline
    %\rowcolor{gray!30}
    Metric             & Resnet50-based            & H-optimus-based              \\
    \hline
    $R^2_{neoplastic}$    & 0.681 (0.576--0.769)       & \textbf{0.941 (0.917--0.960)} \\
    \hline
    $R^2_{inflammatory}$  & 0.863 (0.778--0.903)       & \textbf{0.949 (0.918--0.966)} \\
    \hline
    $R^2_{connective}$    & 0.600 (0.488--0.698)       & 0.609 (0.436--0.772)          \\
    \hline
    $R^2_{dead}$          & 0.097 (-11.389--0.669)     & 0.925 (0.404--0.982)          \\
    \hline
    $R^2_{epithelial}$    & 0.635 (0.490--0.747)       & \textbf{0.930 (0.886--0.964)} \\
    \hline
    $PQ_{neoplastic}$       & 0.517 (0.499--0.535)       & \textbf{0.589 (0.575--0.604)} \\
    \hline
    $PQ_{inflammatory}$     & 0.455 (0.429--0.482)       & \textbf{0.528 (0.507--0.549)} \\
    \hline
    $PQ_{connective}$       & 0.416 (0.400--0.431)       & \textbf{0.451 (0.436--0.465)} \\
    \hline
    $PQ_{dead}$             & 0.374 (0.342--0.408)       & 0.292 (0.209--0.365)          \\
    \hline
    $PQ_{epithelial}$       & 0.488 (0.460--0.519)       & \textbf{0.599 (0.579--0.618)} \\
    \hline
  \end{tabular}
\end{table}

Our results  show that integrating the H‑Optimus foundation model within the UNETR architecture enhances the model's ability to segment and classify cells across diverse tissues from PanNuke data. The pretrained transformer encoder provides robust feature representations, resulting in higher average $R^2$ and $PQ$ scores compared to the CNN-based model. This leads to more reliable cell quantification and more accurate downstream analysis. Additionally, the streamlined fine-tuning process reduces computational overhead and training time, making the model more adaptable for new data.

Despite these advancements, the foundation model-based approach does not fully resolve all challenges related to cell segmentation and classification. We observe lower metric scores for underrepresented classes in the training data. Furthermore, foundation models typically encompass billions of parameters, resulting in substantial computational and memory requirements. It therefore poses challenges for deployment in resource-constrained environments, limiting their practical applicability in certain clinical settings.

\section{Model optimization via Knowledge Distillation}

To address the limitations posed by the extensive size of foundation models, we implement knowledge distillation — a model compression technique that leverages the teacher-student paradigm \cite{Hinton_Vinyals_etal._2015}. By training a smaller, more efficient student model to replicate the output of a larger, pre-trained teacher model, we retain performance while significantly reducing the model's complexity and resource requirements (\hyperref[fig:fig6]{Figure 6}).

\begin{figure}[h!]
    \centering
    \includegraphics[width=\textwidth, height=0.45\textheight, keepaspectratio]{images/Figure_6.pdf}
    \caption{Knowledge distillation framework for training a student model using a pre-trained teacher}
    \label{fig:fig6}
\end{figure}

We employ knowledge distillation to compress the H‑Optimus-based teacher model into a more efficient student model. The teacher model is the modified UNETR architecture with the H‑Optimus foundation model described in the previous chapter. The student model is based on a UNet architecture augmented with residual connections and incorporates a smaller ViT encoder with 9 million parameters \cite{Steiner_Kolesnikov_etal._2022,Wightman_2019}. 

First, we fine-tune the teacher model using the refined dataset from the cross-relabeling procedure (Section 2). Initially we train the decoder of the teacher model while keeping the encoder weights frozen. We split the refined dataset into train (70\%), validation (20\%) and test (10\%) subsets (\hyperref[fig:S2]{Appendix Figure S2 (4)}). During fine-tuning, we use the train and validation subsets, while leaving the test subset for model evaluation. We set the training procedure and model hyperparameters to be identical to those that were used to demonstrate the utility of foundation models for the simultaneous cell segmentation and classification task.

Next, we perform knowledge distillation from teacher to student using the refined dataset used to fine-tune the teacher model. The student model is trained to replicate the teacher model's outputs. We utilize a specialized loss function that aligns the student's predicted probability distribution with the teacher's, incorporating the teacher's class probability distribution derived from the output. Following the methodology of Hinton et al. \cite{Hinton_Vinyals_etal._2015}, we experiment with various hyperparameter settings for the temperature ($T$) and the balancing coefficients ($\alpha$ and $\beta$) in the loss function. We vary $T$ from 1 to 20 and adjust $\alpha$ and $\beta$ to balance the distillation and student losses. Through iterative tuning and evaluation, we identify that setting $T=14$, $\alpha=0.3$, and $\beta=0.7$ yields a configuration that converges and closely approximates the teacher model's performance during training.

Finally, we assess the performance of both models using the $R^2$ and $PQ$ (defined in \hyperref[chap:S5]{Appendix S5}) on the test set of the refined dataset (\hyperref[tab:3]{Table 3}). We observe that the 95\% confidence intervals overlap for most cell types, so we cannot claim statistically significant performance differences between the teacher and student models. One exception appears in the neoplastic class. The teacher model produces an $R^2$ of 0.919, while the student model shows an $R^2$ of 0.852. In addition, the student model achieves higher $PQ$ values for the neoplastic and connective classes, though the confidence intervals show overlap.

\begin{table}[h!]
\renewcommand{\arraystretch}{1.5}
  \centering
  \caption{Cell quantification metrics for teacher and distilled student models (CI 95\%).}
  \label{tab:3}
  \begin{tabular}{|l|c|c|}
    \hline
    %\rowcolor{gray!30}
    Metric & Teacher & Student \\
    \hline
    $R^2_{neoplastic}$    & \textbf{0.919} (0.898--0.939) & 0.852 (0.800--0.891) \\
    \hline
    $R^2_{lymphocyte}$    & 0.969 (0.956--0.977)         & 0.969 (0.956--0.978) \\
    \hline
    $R^2_{connective}$    & 0.694 (0.548--0.809)         & 0.618 (0.469--0.741) \\
    \hline
    $R^2_{dead}$          & 0.755 (0.400--0.908)         & 0.424 (0.100--0.731) \\
    \hline
    $R^2_{epithelial}$    & 0.922 (0.870--0.958)         & 0.843 (0.738--0.917) \\
    \hline
    $R^2_{macrophage}$    & 0.384 (-0.369--0.724)        & 0.704 (0.352--0.859) \\
    \hline
    $R^2_{neutrofil}$     & 0.854 (0.578--0.929)         & 0.833 (0.502--0.925) \\
    \hline
    $PQ_{neoplastic}$       & 0.581 (0.569--0.593)         & 0.601 (0.588--0.613) \\
    \hline
    $PQ_{lymphocyte}$       & 0.536 (0.520--0.553)         & 0.563 (0.544--0.579) \\
    \hline
    $PQ_{connective}$       & 0.436 (0.421--0.451)         & 0.457 (0.441--0.474) \\
    \hline
    $PQ_{dead}$             & 0.272 (0.235--0.315)         & 0.279 (0.201--0.369) \\
    \hline
    $PQ_{epithelial}$       & 0.522 (0.500--0.545)         & 0.530 (0.506--0.555) \\
    \hline
    $PQ_{macrophage}$       & 0.524 (0.459--0.588)         & 0.474 (0.405--0.543) \\
    \hline
    $PQ_{neutrofil}$        & 0.541 (0.490--0.592)         & 0.565 (0.522--0.607) \\
    \hline
  \end{tabular}
\end{table}


We further decompose the $PQ$ metric into its $SQ$ and $DQ$ components (\hyperref[tab:S6]{Appendix Table S6}). Both models produce nearly identical $SQ$ values, which indicates that they predict instance boundaries with similar precision. Although the student model shows some improvement in $DQ$ scores for certain classes, the confidence intervals overlap and do not confirm a statistically significant difference.

We observe that the student and teacher models yield comparable detection performance despite the student model using a much smaller and simpler architecture. A model with fewer parameters reduces the risk of overfitting when training data are scarce relative to the model’s complexity \cite{Farias_Ludermir_etal._2022}. The knowledge distillation process also encourages the student model to focus on the most generalizable detection features learned from the teacher. These factors enable the student model to achieve similar detection performance across different cell types.

Additionally, considering the model sizes reported in \hyperref[tab:4]{Table 4}, the distilled model achieves a significant reduction compared to the teacher model, with a 48-fold decrease in parameter count and a 5.5-fold reduction in on-disk size. In inference mode, the teacher model requires 16 GB of VRAM for a batch size of 32, while the distilled model only needs 3 GB of VRAM for the same batch size. These reductions make the distilled model significantly more practical for fine-tuning and deployment in resource-constrained environments.

\begin{table}[h!]
\renewcommand{\arraystretch}{1.5}
  \centering
  \caption{Parameter counts and size of teacher and distilled model}
  \label{tab:4}
  \adjustbox{max width=\textwidth}{%
  \begin{tabular}{|l|c|c|c|}
    \hline
    %\rowcolor{gray!30}
    Metric & H-optimus-based (Teacher) & mobileViT-based (Student) & Magnitude of difference \\
    \hline
    Parameters count       & 1,158,917,906   & \textbf{24,093,393}   & \textbf{48x}  \\
    \hline
    Estimated Total Size (MB) & 87,912       & \textbf{15,935}    & \textbf{5.5x} \\
    \hline
  \end{tabular}%
}
\end{table}

%\hfill

With recent advancements in complex network architectures and the use of pretrained encoders to achieve state-of-the-art performance \cite{Baumann_Dislich_etal._2024,Hörst_Rempe_etal._2024} in cell segmentation and classification tasks, model size, computational complexity, and processing times have increased. This limits the scalability and accessibility of these models. As we demonstrate, this may be mitigated using knowledge distillation. Studies in the field of natural language processing have demonstrated the efficacy of knowledge distillation in retaining the capabilities of the teacher model while achieving significant reductions in size and complexity \cite{Huangpu_Gao_2024,Sun_Yu_etal.}. 

We demonstrate the feasibility of knowledge distillation in digital pathology, specifically for cell segmentation and classification tasks. Moreover, we achieve this performance while also significantly reducing the parameter count. In addressing the challenge of knowledge transfer, we found that distillation from a transformer-based model to a smaller transformer is more straightforward than attempting to map transformer features to CNN blocks. In our experiments, using a CNN-based network as a student results in worse cell quantification performance due to the structural constraints of CNN feature space dimensions. 

Although our primary approach relies on a transformer-based student model that performs well, it can be further optimized to incorporate advantages from CNN architectures. For example, employing alternative techniques such as using ViT adapters \cite{Chen_Duan_etal._2023} or $1 \times 1$ convolutions to adjust feature map sizes may be beneficial for harnessing CNN advantages like enhanced local feature extraction. Moreover, if additional performance improvements are desired, the process can be further enhanced by applying supplementary knowledge distillation techniques, such as self-distillation \cite{Zhang_Song_etal._2019} or online distillation \cite{Houyon_Cioppa_etal._2023}.

Despite these promising results, further validation on independent datasets is necessary to fully understand the model's limitations. Underrepresented classes may pose challenges when addressing complex cases. Pathologists need to validate these models to adopt them in clinical settings. While the distilled models are smaller and more deployable, a technological gap persists because pathologists traditionally rely on established methods for inspecting WSIs and diagnosing diseases. Addressing the complexities involved in deploying models for inference and supporting pathologists in adopting new tools is essential for integrating these models into clinical workflows.

\section{Model integration with QuPath}
Digital pathology tools with graphical user interfaces are essential for visualizing and analyzing WSIs. To make our student model useful in clinical pathology workflows, it needs to be integrated into a tool that enables inspecting regions, creating annotations, and providing quantitative analyses of biomarkers. Therefore, we integrate the trained student model from the previous chapter into the QuPath open‑source platform \cite{Bankhead_Loughrey_etal._2017}. QuPath provides the required annotation, visualization, and analysis tools to interpret complex histological data, including workflows for cell segmentation, classification, and quantification (\hyperref[fig:fig7]{Figure 7}). 

\begin{figure}[h!]
    \centering
    \includegraphics[width=\textwidth]{images/Figure_7.pdf}
    \caption{Visualization of model-generated cell quantification annotations (left) and the corresponding unannotated slide (right) in QuPath}
    \label{fig:fig7}
\end{figure}

To identify the regions in a WSI critical for prognosticating tumor development, such as specific tumor areas or border regions without overlapping healthy tissue, the pathologist uses QuPath to outline these regions. Then, the pathologist initiates a cell segmentation and classification script through the QuPath interface for the selected regions. The resulting annotations and quantified cell information are then directly overlaid onto the WSI in the QuPath interface. Additional design and implementation details are in \hyperref[chap:S7]{Appendix S7}. 

Two common approaches for integrating deep learning models into QuPath are Java‑based native QuPath extensions \cite{Goldsborough_Philps_etal._2024} and the execution of RESTful API requests to a model server coupled with handling the response via an extension, as demonstrated in the application of cell segmentation models applied to immunofluorescence images \cite{Sugawara_2023}. While the community is actively working on these integration strategies, there is currently no universal solution that fully addresses all integration and performance requirements.

Extensions may offer better integration with QuPath, allowing slightly improved performance and more widespread usage of the built-in QuPath models, but they lack the flexibility to customize models and modify their behavior. For example, the newest version of QuPath includes models such as StarDist \cite{Weigert_Schmidt} and InstanSeg \cite{Goldsborough_Philps_etal._2024} that can perform cell segmentation. Both models pose limitations when applied to simultaneous cell segmentation and classification. StarDist performs well only on convex, round shapes by design, whereas some neoplastic, inflammatory, and connective cells exhibit complex and non-convex shapes. InstanSeg provides only semantic segmentation without assigning classes to the segmented cells.

%\hfill

In contrast, our approach offers an alternative integration strategy. It utilizes the paquo library to directly interact with QuPath’s internal application programming interface from within Python. This enables data exchange and processing without the need for intermediate conversion steps and provides greater control over model customization, retraining, and the incorporation of custom processing steps.

The integration of our custom model with QuPath underscores its potential to significantly enhance the diagnostic process by reducing the time burden on pathologists and enabling them to focus on more complex interpretative tasks using familiar software. Leveraging a tool that is already well-established among pathologists increases the likelihood of its adoption into daily clinical workflows. The quantitative data generated through the automated workflow is critical for both clinical decision-making and research, facilitating more accurate biomarker analysis, enabling robust statistical evaluations, and supporting hypothesis generation and testing. Additionally, by streamlining cell segmentation and classification, the tool enhances the scalability and reproducibility of pathological assessments, ultimately contributing to improved diagnostic accuracy and patient outcomes.

\section{Conclusion and future work}

In this study, we address critical challenges in digital pathology and tackle the usability and deployment issues of the developed models in standard computing environments without the need for high-performance computing systems. Our multi-faceted approach encompasses data refinement through cross-relabeling, leveraging foundation models for robust cell segmentation and classification, optimizing model performance via knowledge distillation, and integrating the optimized model into the QuPath software for practical application. This approach is used to construct a capable, versatile, and adjustable model for cell segmentation and classification, with enhanced performance and usability.

\begin{sloppypar}
While our approach shows potential in the field of computational pathology, certain limitations persist. 
For example, our implementation currently exhibits lower performance in detecting macrophages. 
This serves as an instance of the broader challenge of accurately identifying complex cell types. In order to address this issue, extending our approach to incorporate additional data sources, exploring alternative modeling approaches, and integrating other imaging modalities such as immunohistochemical staining may help improve detection accuracy. Moreover, although the distilled model reduces computational demands, integrating advanced deep learning models into clinical practice requires addressing technological gaps and potential resistance to adopting new tools within established diagnostic processes.
\end{sloppypar}

Future work could focus on several key areas to refine the proposed approach and facilitate its adoption in clinical environments. Enhancing the cell-relabeling process with additional datasets \cite{Graham_Jahanifar_etal._2021} could improve the representation of underrepresented cell types and enhance overall model performance. Also, incorporating additional data sources, such as multi-modal imaging or complementary staining methods, may address limitations related to cell type differentiation and class imbalance. Exploring other foundation models \cite{Vorontsov_Bozkurt_etal._2024,Zimmermann_Vorontsov_etal._2024} or introducing additional modalities \cite{Ding_Wagner_etal._2024,Vaidya_Zhang_etal._2025} may provide alternative architectures better suited to specific tasks or offer improved efficiency. Implementing more complex knowledge distillation techniques \cite{Houyon_Cioppa_etal._2023,Zhang_Song_etal._2019} could further optimize the model's performance and adaptability. Additionally, deeper integration with QuPath or other digital pathology software could provide pathologists more control over cell quantification analysis directly within the QuPath interface, thereby increasing accessibility and usability. Such enhancements would not only refine model performance but also ensure greater adaptability and scalability within various clinical environments. Finally, extensive validation of the model by pathologists and benchmarking against independent datasets are essential steps toward establishing the model's reliability and fostering confidence in its clinical utility.

\section*{Acknowledgments} 
This work was funded in part by the Research Council of Norway grant no. 309439 SFI Visual Intelligence, and the North Norwegian Health Authority grant no. HNF1521-20.

\bibliographystyle{IEEEtran}
\begin{sloppypar}
\begin{thebibliography}{99}

\bibitem{chaplot2020neural} Chaplot, Devendra Singh, et al. "Neural topological slam for visual navigation." Proceedings of the IEEE/CVF conference on computer vision and pattern recognition. 2020.

\bibitem{maksymets2021thda} Maksymets, Oleksandr, et al. "Thda: Treasure hunt data augmentation for semantic navigation." Proceedings of the IEEE/CVF International Conference on Computer Vision. 2021.

\bibitem{mezghan2022memory} Mezghan, Lina, et al. "Memory-augmented reinforcement learning for image-goal navigation." 2022 IEEE/RSJ International Conference on Intelligent Robots and Systems (IROS). IEEE, 2022.

\bibitem{al2022zero} Al-Halah, Ziad, Santhosh Kumar Ramakrishnan, and Kristen Grauman. "Zero experience required: Plug \& play modular transfer learning for semantic visual navigation." Proceedings of the IEEE/CVF Conference on Computer Vision and Pattern Recognition. 2022.

\bibitem{ye2021auxiliary} Ye, Joel, et al. "Auxiliary tasks and exploration enable objectgoal navigation." Proceedings of the IEEE/CVF international conference on computer vision. 2021.

\bibitem{chaplot2020object} Chaplot, Devendra Singh, et al. "Object goal navigation using goal-oriented semantic exploration." Advances in Neural Information Processing Systems 33 (2020)

\bibitem{ramakrishnan2022poni} Ramakrishnan, Santhosh Kumar, et al. "Poni: Potential functions for objectgoal navigation with interaction-free learning." Proceedings of the IEEE/CVF Conference on Computer Vision and Pattern Recognition. 2022.

\bibitem{ramrakhya2022habitat} Ramrakhya, Ram, et al. "Habitat-web: Learning embodied object-search strategies from human demonstrations at scale." Proceedings of the IEEE/CVF Conference on Computer Vision and Pattern Recognition. 2022.

\bibitem{mousavian2019visual} Mousavian, Arsalan, et al. "Visual representations for semantic target driven navigation." 2019 International Conference on Robotics and Automation (ICRA). IEEE, 2019.

\bibitem{dhariwal2021diffusion} Dhariwal, Prafulla, and Alexander Nichol. "Diffusion models beat gans on image synthesis." Advances in neural information processing systems 34 (2021)

\bibitem{ho2022classifier} Ho, Jonathan, and Tim Salimans. "Classifier-free diffusion guidance." arXiv preprint arXiv:2207.12598 (2022).

\bibitem{nichol2021glide} Nichol, Alex, et al. "Glide: Towards photorealistic image generation and editing with text-guided diffusion models." arXiv preprint arXiv:2112.10741 (2021)

\bibitem{brooks2023instructpix2pix} Brooks, Tim, Aleksander Holynski, and Alexei A. Efros. "Instructpix2pix: Learning to follow image editing instructions." Proceedings of the IEEE/CVF Conference on Computer Vision and Pattern Recognition. 2023.

\bibitem{fu2023guiding} Fu, Tsu-Jui, et al. "Guiding instruction-based image editing via multimodal large language models." arXiv preprint arXiv:2309.17102 (2023).

\bibitem{geng2024instructdiffusion} Geng, Zigang, et al. "Instructdiffusion: A generalist modeling interface for vision tasks." Proceedings of the IEEE/CVF Conference on Computer Vision and Pattern Recognition. 2024.

\bibitem{zhou2024minedreamer} Zhou, Enshen, et al. "Minedreamer: Learning to follow instructions via chain-of-imagination for simulated-world control." arXiv preprint arXiv:2403.12037 (2024).

\bibitem{zhou2023esc} Zhou, Kaiwen, et al. "Esc: Exploration with soft commonsense constraints for zero-shot object navigation." International Conference on Machine Learning. PMLR, 2023.

\bibitem{yu2023l3mvn} Yu, Bangguo, Hamidreza Kasaei, and Ming Cao. "L3mvn: Leveraging large language models for visual target navigation." 2023 IEEE/RSJ International Conference on Intelligent Robots and Systems (IROS). IEEE, 2023.

\bibitem{gadre2023cows} Gadre, Samir Yitzhak, et al. "Cows on pasture: Baselines and benchmarks for language-driven zero-shot object navigation." Proceedings of the IEEE/CVF Conference on Computer Vision and Pattern Recognition. 2023.

\bibitem{shah2023navigation} Shah, Dhruv, et al. "Navigation with large language models: Semantic guesswork as a heuristic for planning." Conference on Robot Learning. PMLR, 2023.

\bibitem{cai2024bridging} Cai, Wenzhe, et al. "Bridging zero-shot object navigation and foundation models through pixel-guided navigation skill." 2024 IEEE International Conference on Robotics and Automation (ICRA). IEEE, 2024.

\bibitem{yu2023co} Yu, Bangguo, Hamidreza Kasaei, and Ming Cao. "Co-NavGPT: Multi-robot cooperative visual semantic navigation using large language models." arXiv preprint arXiv:2310.07937 (2023).

\bibitem{wu2024voronav} Wu, Pengying, et al. "Voronav: Voronoi-based zero-shot object navigation with large language model." arXiv preprint arXiv:2401.02695 (2024).

\bibitem{qin2023mp5} Qin, Yiran, et al. "Mp5: A multi-modal open-ended embodied system in minecraft via active perception." arXiv preprint arXiv:2312.07472 (2023).

\bibitem{du2024learning} Du, Yilun, et al. "Learning universal policies via text-guided video generation." Advances in Neural Information Processing Systems 36 (2024).

\bibitem{ajay2024compositional} Ajay, Anurag, et al. "Compositional foundation models for hierarchical planning." Advances in Neural Information Processing Systems 36 (2024).

\bibitem{liang2024skilldiffuser} Liang, Zhixuan, et al. "Skilldiffuser: Interpretable hierarchical planning via skill abstractions in diffusion-based task execution." Proceedings of the IEEE/CVF Conference on Computer Vision and Pattern Recognition. 2024.

\bibitem{heusel2017gans} Heusel, Martin, et al. "Gans trained by a two time-scale update rule converge to a local nash equilibrium." Advances in neural information processing systems 30 (2017).

\bibitem{zhang2018unreasonable} Zhang, Richard, et al. "The unreasonable effectiveness of deep features as a perceptual metric." Proceedings of the IEEE conference on computer vision and pattern recognition. 2018.

\bibitem{brown2020language} Brown, Tom B. "Language models are few-shot learners." arXiv preprint arXiv:2005.14165 (2020).

\bibitem{podell2023sdxl} Podell, Dustin, et al. "Sdxl: Improving latent diffusion models for high-resolution image synthesis." arXiv preprint arXiv:2307.01952 (2023).

\bibitem{brohan2022rt} Brohan, Anthony, et al. "Rt-1: Robotics transformer for real-world control at scale." arXiv preprint arXiv:2212.06817 (2022).

\bibitem{brohan2023rt} Brohan, Anthony, et al. "Rt-2: Vision-language-action models transfer web knowledge to robotic control." arXiv preprint arXiv:2307.15818 (2023).

\bibitem{li2024manipllm} Li, Xiaoqi, et al. "Manipllm: Embodied multimodal large language model for object-centric robotic manipulation." Proceedings of the IEEE/CVF Conference on Computer Vision and Pattern Recognition. 2024.

\bibitem{shah2023vint} Shah, Dhruv, et al. "ViNT: A foundation model for visual navigation." arXiv preprint arXiv:2306.14846 (2023).

\bibitem{liu2024visual} Liu, Haotian, et al. "Visual instruction tuning." Advances in neural information processing systems 36 (2024).

\bibitem{hu2021lora} Hu, Edward J., et al. "Lora: Low-rank adaptation of large language models." arXiv preprint arXiv:2106.09685 (2021).

\bibitem{qin2023supfusion} Qin, Yiran, et al. "SupFusion: Supervised LiDAR-camera fusion for 3D object detection." Proceedings of the IEEE/CVF International Conference on Computer Vision. 2023.

\bibitem{qin2024worldsimbench} Qin, Yiran, et al. "Worldsimbench: Towards video generation models as world simulators." arXiv preprint arXiv:2410.18072 (2024).

\bibitem{yu2025gamefactory} Yu, Jiwen, et al. "GameFactory: Creating New Games with Generative Interactive Videos." arXiv preprint arXiv:2501.08325 (2025).

\bibitem{zhou2024code} Zhou, Enshen, et al. "Code-as-Monitor: Constraint-aware Visual Programming for Reactive and Proactive Robotic Failure Detection." arXiv preprint arXiv:2412.04455 (2024).

\bibitem{zhang2024ad} Zhang, Zaibin, et al. "AD-H: Autonomous Driving with Hierarchical Agents." arXiv preprint arXiv:2406.03474 (2024).

\bibitem{wang2024toward} Wang, Chaoqun, et al. "Toward Accurate Camera-based 3D Object Detection via Cascade Depth Estimation and Calibration." arXiv preprint arXiv:2402.04883 (2024).

\bibitem{huang2024story3d} Huang, Yuzhou, et al. "Story3d-agent: Exploring 3d storytelling visualization with large language models." arXiv preprint arXiv:2408.11801 (2024).

\bibitem{savinov2018semi} Savinov, Nikolay, Alexey Dosovitskiy, and Vladlen Koltun. "Semi-parametric topological memory for navigation." arXiv preprint arXiv:1803.00653 (2018).

\bibitem{majumdar2022zson} Majumdar, Arjun, et al. "Zson: Zero-shot object-goal navigation using multimodal goal embeddings." Advances in Neural Information Processing Systems 35 (2022): 32340-32352.

\bibitem{yadav2023offline} Yadav, Karmesh, et al. "Offline visual representation learning for embodied navigation." Workshop on Reincarnating Reinforcement Learning at ICLR 2023. 2023.

\bibitem{yadav2023ovrl} Yadav, Karmesh, et al. "Ovrl-v2: A simple state-of-art baseline for imagenav and objectnav." arXiv preprint arXiv:2303.07798 (2023).

\bibitem{sun2024fgprompt} Sun, Xinyu, et al. "FGPrompt: fine-grained goal prompting for image-goal navigation." Advances in Neural Information Processing Systems 36 (2024).

\bibitem{zhu2017target} Zhu, Yuke, et al. "Target-driven visual navigation in indoor scenes using deep reinforcement learning." 2017 IEEE international conference on robotics and automation (ICRA). IEEE, 2017.

\bibitem{koh2024generating} Koh, Jing Yu, Daniel Fried, and Russ R. Salakhutdinov. "Generating images with multimodal language models." Advances in Neural Information Processing Systems 36 (2024).

\bibitem{krantz2022instance} Krantz, Jacob, et al. "Instance-specific image goal navigation: Training embodied agents to find object instances." arXiv preprint arXiv:2211.15876 (2022).

\bibitem{schulman2017proximal} Schulman, John, et al. "Proximal policy optimization algorithms." arXiv preprint arXiv:1707.06347 (2017).

\bibitem{anderson2018evaluation} Anderson, Peter, et al. "On evaluation of embodied navigation agents." arXiv preprint arXiv:1807.06757 (2018).

\bibitem{lin2024navcot} Lin, Bingqian, et al. "NavCoT: Boosting LLM-Based Vision-and-Language Navigation via Learning Disentangled Reasoning." arXiv preprint arXiv:2403.07376 (2024).

\bibitem{NavGPT} Zhou, Gengze, Yicong Hong, and Qi Wu. "Navgpt: Explicit reasoning in vision-and-language navigation with large language models." Proceedings of the AAAI Conference on Artificial Intelligence.

\bibitem{hahn2021no} Hahn, Meera, et al. "No rl, no simulation: Learning to navigate without navigating." Advances in Neural Information Processing Systems 34 (2021): 26661-26673.

\bibitem{li2025t2isafety} Li, Lijun, et al. "T2ISafety: Benchmark for Assessing Fairness, Toxicity, and Privacy in Image Generation." arXiv preprint arXiv:2501.12612 (2025).

\bibitem{an2024agfsync} An, Jingkun, et al. "AGFSync: Leveraging AI-Generated Feedback for Preference Optimization in Text-to-Image Generation." arXiv preprint arXiv:2403.13352 (2024).


\end{thebibliography}
\end{sloppypar}

\clearpage
\beginsupplement
\section*{Appendix}
\renewcommand{\thesubsection}{S\arabic{subsection}}

\subsection{\label{chap:S1}PanNuke and MoNuSAC preprocessing}
The PanNuke dataset comprises a set of 7,901 RGB patches, each with dimensions of $256 \times 256$ pixels, which we set as the standard patch size for our analysis. In contrast, the MoNuSAC dataset encompasses 294 images of heterogeneous dimensions. To standardize the MoNuSAC images with our experiments, we implement a standardization protocol. Specifically, for images exceeding the dimensions of $256 \times 256$ pixels, we segment them into equal-sized patches and apply mirror padding to the remaining portions to avoid information loss at the peripherals. Patches with dimensions less than $128 \times 128$ pixels are excluded from the dataset due to the insufficient resolution to capture relevant cellular details. For patches where either dimension falls between 128 and 256 pixels, we employ upsampling to achieve the standard patch size. As a result, we obtain a total of 2,823 RGB patches derived from the MoNuSAC dataset for subsequent analysis. For additional details on the MoNuSAC data preparation process, refer to the source code \cite{Shvetsov_2025a}.
\clearpage

\subsection{\label{chap:S2}Data usage for the methodology}

\counterwithin{figure}{subsection}
\renewcommand{\thefigure}{S\arabic{subsection}}

\begin{figure}[h!]
    \centering
    \includegraphics[width=\textwidth, height=0.85\textheight, keepaspectratio]{images/A2.pdf}
    \caption{Overview of the methodology for cross-labeling, dataset refinement, and model comparison. (1) Cross-relabeling - training and testing cell classification models, (2) Cross-relabeling - using cell classification models to create refined dataset, (3) Fine-tuning and training models for comparison, (4) Student knowledge distillation with refined dataset}
    \label{fig:S2}
\end{figure}
\clearpage

\subsection{\label{chap:S3}Confusion matrices for classification models}
\counterwithin{figure}{subsection}
\renewcommand{\thefigure}{S\arabic{subsection}.\arabic{figure}}

\begin{figure}[h!]
    \centering
    \includegraphics[width=\textwidth, height=0.4\textheight, keepaspectratio]{images/A3_1.pdf}
    \caption{Confusion matrix for PanNuke trained model}
    \label{fig:S3.1}
\end{figure}

\begin{figure}[h!]
    \centering
    \includegraphics[width=\textwidth, height=0.4\textheight, keepaspectratio]{images/A3_2.pdf}
    \caption{Confusion matrix for MoNuSAC trained model}
    \label{fig:S3.2}
\end{figure}

\clearpage

\subsection{\label{chap:S4}Datasets cell counts}

\counterwithin{table}{subsection}
\renewcommand{\thetable}{S\arabic{subsection}}

\begin{table}[h!]
\renewcommand{\arraystretch}{2.0}
\centering
\caption{\label{tab:S4}Cell counts for PanNuke, MoNuSAC and refined datasets. Numbers in parentheses indicate preprocessed cell counts for cell classifier models training and testing.}
%\adjustbox{max width=\textwidth}{%
\begin{tabular}{|l|c|c|c|}
\hline
%\rowcolor{gray!30}
Cell type & PanNuke & MoNuSAC & Refined \\
\hline
Neoplastic & 77,403 (68,031) & - & 105,451 \\
\hline
Epithelial & 26,572 (23,207) & - & 29,926 \\
\hline
Epithelial (benign and malignant) & - & 31,402 & - \\
\hline
Inflammatory & 32,276 & - & - \\
\hline
Lymphocytes & - & 37,045 (33,104) & 65,275 \\
\hline
Neutrophils & - & 1,355 (1,252) & 3,833 \\
\hline
Macrophage & - & 1,842 (1,695) & 3,410 \\
\hline
Dead & 2,908 & - & 2,908 \\
\hline
Connective & 50,585 & - & 50,585 \\
\hline
\end{tabular}
%
%}
\end{table}



\clearpage

\subsection{\label{chap:S5}Definition of validation metrics}
\counterwithin{equation}{subsection}
\renewcommand{\theequation}{\arabic{equation}}

\subsubsection{\label{chap:S5.1}R\textsuperscript{2}}
The coefficient of determination, denoted as $R^2$, is a statistical measure that represents the proportion of variance in the dependent variable that is predictable from the independent variables. In the context of cell quantification in pathology, $R^2$ is used to assess how well the predicted quantities of different cell types in a patch align with the actual quantities observed in the ground truth data, with higher values representing more accurate quantification. $R^2$ is defined as
\begin{equation*}
R^2 = 1 - \frac{\sum_{i=1}^n (y_i - \hat{y}_i)^2}{\sum_{i=1}^n (y_i - \bar{y})^2},
\end{equation*}
where $y_i$ represents the actual number of cells of a specific type in the $i$-th image, $\hat{y}_i$ represents the predicted number of cells of that type in the $i$-th image, $\bar{y}$ is the mean of the actual numbers across all images, and $n$ is the total number of images in the dataset.

The $R^2$ metric has a range of $(-\infty, 1]$. An $R^2$ of 1 indicates perfect prediction, where all predicted values exactly match the actual values. An $R^2$ of 0 suggests that the model explains none of the variability of the response data around its mean. If $R^2$ is negative, it indicates that the model performs worse than a model that simply predicts the mean of the actual values for all observations.

\subsubsection{\label{chap:S5.2}PQ}
Panoptic Quality ($PQ$) is a comprehensive metric used to evaluate the performance of segmentation models in tasks that require both instance segmentation and classification. $PQ$ provides a single score that encapsulates both the detection accuracy (i.e., how many objects were correctly identified) and the segmentation quality (i.e., how accurately the objects' boundaries were delineated). This metric is particularly useful in multiclass scenarios where each pixel is classified into distinct categories, such as different cell types in pathology images.

$PQ$ is calculated as the product of two terms: Detection Quality ($DQ$) and Segmentation Quality ($SQ$). It can be expressed as
\begin{equation*}
PQ = DQ \cdot SQ,
\end{equation*}
where
\begin{equation*}
DQ = \frac{TP}{TP + 0.5\, FP + 0.5\, FN},
\end{equation*}
\begin{equation*}
SQ = \frac{\sum_{(p, g) \in \mathcal{M}} IoU(p, g)}{TP}.
\end{equation*}
In these formulas, $TP$ denotes the number of correctly matched instances between ground truth and prediction, $FP$ denotes the predicted instances that have no corresponding ground truth, $FN$ denotes the ground truth instances that were not detected, $IoU(p, g)$ is the Intersection over Union for a pair of matched instances $p$ (prediction) and $g$ (ground truth), and $\mathcal{M}$ is the set of matched pairs.

The $PQ$ metric is calculated for each class and is averaged across classes to provide a global performance measure.

The $PQ$ score has a range of $[0, 1.0]$, where a higher score indicates better performance in both detecting and segmenting the instances correctly. A $PQ$ of 1 signifies perfect identification and segmentation of all instances, whereas a $PQ$ of 0 indicates that no instances were correctly identified and segmented.

\clearpage

\subsection{\label{chap:S6}Segmentation and Detection quality metrics for teacher and student models}

\begin{table}[h!]
\renewcommand{\arraystretch}{2.0}
\centering
\caption{Segmentation and detection quality for student and teacher models (CI 95\%)}
\label{tab:S6}
%\adjustbox{max width=\textwidth}{%
\begin{tabular}{|l|c|c|}
\hline
%\rowcolor{gray!30}
Metric & Teacher & Student \\
\hline
$SQ_{neoplastic}$ & 0.819 (0.815--0.823) & 0.824 (0.819--0.828) \\
\hline
$SQ_{lymphocyte}$ & 0.795 (0.788--0.802) & 0.790 (0.783--0.796) \\
\hline
$SQ_{connective}$ & 0.770 (0.762--0.776) & 0.780 (0.772--0.786) \\
\hline
$SQ_{dead}$ & 0.659 (0.623--0.688) & 0.657 (0.624--0.695) \\
\hline
$SQ_{epithelial}$ & 0.780 (0.770--0.790) & 0.788 (0.779--0.797) \\
\hline
$SQ_{macrophage}$ & 0.788 (0.760--0.810) & 0.757 (0.730--0.783) \\
\hline
$SQ_{neutrofil}$ & 0.782 (0.761--0.801) & 0.775 (0.759--0.792) \\
\hline
$DQ_{neoplastic}$ & 0.706 (0.692--0.719) & 0.727 (0.712--0.741) \\
\hline
$DQ_{lymphocyte}$ & 0.675 (0.656--0.698) & 0.713 (0.691--0.734) \\
\hline
$DQ_{connective}$ & 0.566 (0.546--0.584) & 0.583 (0.565--0.602) \\
\hline
$DQ_{dead}$ & 0.410 (0.361--0.465) & 0.435 (0.306--0.561) \\
\hline
$DQ_{epithelial}$ & 0.668 (0.639--0.694) & 0.673 (0.644--0.702) \\
\hline
$DQ_{macrophage}$ & 0.657 (0.583--0.727) & 0.615 (0.531--0.703) \\
\hline
$DQ_{neutrofil}$ & 0.691 (0.625--0.753) & 0.729 (0.679--0.778) \\
\hline
\end{tabular}
%
%}
\end{table}

\clearpage

\subsection{\label{chap:S7}QuPath integration method}
We adopt an integration strategy leveraging the paquo \cite{Bayer_AG} library, a Python package that enables direct interaction with QuPath’s internal API, thereby facilitating seamless data exchange without intermediate conversion steps. The data processing pipeline (\hyperref[fig:S7]{Appendix Figure S7}) begins with the acquisition of WSIs and their associated annotations from QuPath, which are represented as Shapely \cite{Gillies_Wel_etal._2024} polygons. Utilizing paquo, we directly read, create, and modify these annotations and detections within a QuPath project in the Python environment. Images are then cropped using these polygons and processed by cell segmentation and classification models employing standard vision processing toolkits such as OpenCV, pyvips, and PyTorch. Additionally, QuPath employs Groovy scripts to initiate a Python process that starts the entire pipeline from QuPath graphical interface: fetching polygons, extracting images from them, and running deep learning model inference on the cropped images. 
The results are returned to QuPath, leveraging paquo's Python bindings to manipulate QuPath data while minimizing the computational overhead typically associated with cross-environment communication.

\counterwithin{figure}{subsection}
\renewcommand{\thefigure}{S\arabic{subsection}}

\begin{figure}[h!]
    \centering
    \includegraphics[width=\textwidth]{images/A7.pdf}
    \caption{QuPath integration workflow using Python environment}
    \label{fig:S7}
\end{figure}

Compared to traditional workflows that involve exporting annotations as GeoJSON, classifying them in Python, and reimporting them into QuPath, our approach offers several advantages. We eliminate the need to switch between programming languages, providing a cohesive and streamlined development process entirely within QuPath software and removing the necessity to use other tools. Meanwhile, we avoid storing annotations as intermediate JSON files unless required for external use or archiving. By conducting the entire inference and post-processing workflow within the Python environment, we leverage the power and flexibility of Python libraries for image processing and machine learning. This approach also enables adjustments to any set of labels and models, thereby improving its applicability.

%\hfill

The distilled model and QuPath integration code are packaged into a Docker container, enabling streamlined execution with the Docker engine. Detailed integration code and deployment instructions can be found in the GitHub repository \cite{Shvetsov_2025b}.

Despite these benefits, we acknowledge that the paquo library is a proof‑of‑concept project in its early development stage and has not been tested across all versions of QuPath.

\clearpage

\subsection{\label{chap:S8}Data and code availability statement}
All datasets, models, and code used in this study are publicly available and can be obtained from the repositories listed below. 
The PanNuke \cite{Gamper_Koohbanani_etal._2019} and MoNuSAC \cite{Verma_Kumar_etal._2021} datasets are publicly accessible, and download information along with detailed descriptions can be found in their respective articles. Preprocessing scripts for PanNuke and MoNuSAC data, as well as individual cell extraction scripts, are available on GitHub \cite{Shvetsov_2025a}. The H-Optimus foundation model used in our experiments can be downloaded from the HuggingFace repository \cite{hoptimus2024}, and model information is available on GitHub \cite{Saillard_Jenatton_etal._2024}. In addition, the integration code for QuPath and the distilled model packaged in a Docker container are provided in the repository \cite{Shvetsov_2025b}, and paquo Python library is available from the authors GitHub repository \cite{Bayer_AG}.
\clearpage

\end{document}


\paragraph{Inference Efficiency.}
To generate a 12-frame video (128$\times$128 resolution, 768 tokens), a 700M NTP model requires 768 forward steps during inference, taking 15.04 seconds (FPS=0.80). 
In contrast, our \modelname model with a 1$\times$1$\times$16 block size predicts all tokens in a row simultaneously, requiring only 48 steps and 1.35 seconds to generate the video (FPS=8.89)—over 11 times faster than the NTP model. 
Since \modelname modifies only the target output and attention mask, it is compatible with the most efficient AR inference frameworks, such as memory-efficient attention~\citep{xFormers2022}, offering the potential for further speed improvements. 
We now briefly discuss the sources of these efficiency gains. 
In scenarios utilizing KV-Cache, the overall computation cost during each inference step for NTP involves multiplying vectors (current token) with matrices (model weights), which is primarily \textbf{IO-bound} due to the movement of matrices. Conversely, in the NBP model, the computation involves multiplying matrices (current block) with matrices (model weights), making it \textbf{compute-bound}, with reduced IO overhead due to larger block sizes. Given this distinction and assuming adequate GPU parallelism, the NBP framework can achieve significantly faster speeds compared to NTP. This efficiency gain is due to the reduced frequency of IO operations and the more effective utilization of computational resources in processing larger data blocks simultaneously.

\paragraph{Scalability.}
As model size increases from 700M to 1.2B and 3B parameters, the FVD of \modelname models improves from 33.6 to 28.6 and 26.5, respectively. This demonstrates that \modelname exhibits similar scalability to NTP models, with the potential for even greater performance as model size and computational resources increase. Fig.~\ref{fig:model_para} and Fig.~\ref{fig:vary_size_gen} present the validation loss curves and generation examples for different model sizes, respectively. 
As the models grow larger, the generated content exhibits greater stability and enhanced visual detail. 

\subsection{Benchmarking with Previous Systems}
\label{subsec:benchmark}
Table~\ref{tab:video_syn} presents our model's performance compared to strong baselines using various modeling approaches, including GAN, diffusion, mask token modeling (MTM), and vanilla AR methods. 
For UCF-101, the evaluation task is class-conditional video generation, where models generate videos based on a given class name. 
% Since our method utilizes an image as the initial visual condition, alongside the classname, we take the first frame from the training videos into condition additionally. This ensures no information leakage from the test set. 
Our Semi-AR model, with 3B parameters, achieves an FVD of 55.3, surpassing HPDM-L~\citep{skorokhodov2024hierarchical} and MAGVITv2~\cite{yu2023language} by 11 and 2.7 FVD points, respectively.

For K600, the evaluation task is frame prediction, where all models predict future frames based on the same 5-frame condition from the validation set. Our 700M model achieves an FVD of 25.5, outperforming the strongest AR baseline, OmniTokenizer, by 7.4 FVD points.
While our model exhibits a performance gap compared to MAGVITv2, it achieves this result with significantly lower training computation (e.g., 77 epochs vs. MAGVITv2's 360 epochs). Scaling up the model size narrows this gap, with a 6-point improvement in FVD observed. Given the strong scalability of our semi-AR framework, we believe that with larger model sizes and increased training volumes, our approach could surpass MAGVITv2, akin to how large language models (LLMs)~\citep{Brown2020LanguageMA} have outperformed BERT~\citep{devlin2018bert} in NLP.



\subsection{Visualizations}
\paragraph{Video Reconstruction.}
Fig.~\ref{fig:vis_recons} compares the video reconstruction results of OmniTokenizer~\citep{Wang2024OmniTokenizerAJ} and our tokenizer. Our method significantly outperforms the baseline in both image clarity and motion stability. 




\paragraph{Video Generation.}
The class-conditional generation results for UCF-101 are shown in Fig.\ref{fig:ucf_gen}, while the frame prediction results for K600 are shown in Figs.\ref{fig:our_gen}-\ref{fig:vis_gen}. The visualizations demonstrate that our model accurately predicts subsequent frames with high clarity and temporal coherence, even in scenarios involving large motion dynamics. 
% Fig.~\ref{fig:our_gen_app} shows more generation results of our 3B model. 
% Moreover, we exhibit the potential of our method for generating videos of arbitrary lengths by employing a cyclical process, where each newly generated frame is recursively used as a condition for the subsequent frame generation.


% \paragraph{Class-conditional Video Generation.}
% % copy from omni
% The class-conditional generation results are shown in Figure 5-8. Our model could synthesize visually coherent and contextually accurate images and videos, showcasing the strengths of OmniTokenizer in facilitating generative tasks.


\subsection{Ablation Study and Analysis}
\label{subsec:ablation}
In this subsection, we conduct an ablation study on block size and block shape, then analyze the attention patterns in our \modelname models.


\paragraph{Ablation Study on Block Size.}
We experiment with different block sizes, ranging from $[1, 8, 16, 32, 64, 256]$\footnote{The full 3D size of the blocks are 1$\times$1$\times$1, 1$\times$1$\times$8, 1$\times$1$\times$16, 1$\times$2$\times$16, 1$\times$4$\times$16, 1$\times$16$\times 16$, respectively.}, to evaluate their impact on model performance. A block size of 1, 16, and 256 corresponds to token-by-token (NTP), row-by-row, and clip-by-clip generation, respectively. 
Fig.~\ref{fig:block_size} shows the validation loss curves for various block sizes. As block size decreases, learning becomes easier due to the increased prefix conditioning, which simplifies the prediction task and results in lower validation loss. 
However, due to the exposure bias associated with (semi-)AR modeling~\citep{ranzato2015sequence}, validation loss under the teacher-forcing setting does not completely correlate with final performance during inference~\citep{deng2024causal}. Notably, the smallest block size (i.e., a single token) does not yield optimal performance. As shown in Fig.~\ref{fig:block_size_fvd_fps}, a block size of 16 achieves the best generation quality, with an FVD improvement of 3.5 points, reaching 25.5. 
Block size is critical for balancing generation quality (FVD) and efficiency (FPS). While larger blocks (e.g., 1$\times$16$\times$16) lead to faster inference speeds (up to 17.14 FPS), performance degrades, indicating that generating an entire clip in one step is overly challenging. 
Additionally, inference decoding methods significantly influence results. As demonstrated in Fig.~\ref{fig:vary_block_gen}, traditional Top-P Top-K decoding can lead to screen fluctuations~\citep{lezama2022improved}, as it struggles to model spatial dependencies within large blocks, highlighting the need for improved decoding strategies in \modelname scenarios. 

\paragraph{Ablation Study on Block Shape.}
We explore the performance of various block shapes on K600, using the 700M model, the results are shown in Table~\ref{tab:block_shape}. 
Our findings indicate that the official block shape of T$\times$H$\times$W=1$\times$1$\times$16 (generating row by row) outperforms other tested shapes such as 1$\times$4$\times$4 and 2$\times$1$\times$8. We attribute this to two main factors: 
\textbf{(1) Token Relationships within a Single Block}: The shape of the 1$\times$1$\times$16 block allows tokens within the block to represent a complete, continuous row, maintaining integrity without cross-row interruptions. In contrast, block shapes like 1$\times$4$\times$4 and 2$\times$1$\times$8 involve generating complex relationships across multiple rows and columns—or even frames—on a smaller spatial scale, posing greater challenges~\citep{ren2023testa}. 
\textbf{(2) Relationships between Blocks}: The 1$\times$1$\times$16 block shape simplifies the modeling process to primarily vertical relationships between rows, which enhances continuity and consistency during generation, thereby reducing breaks and error accumulation.


\begin{figure}[htbp]
\includegraphics[width=\linewidth]{figs/fvd_fps_for_block.pdf}
\caption{Generation quality (FVD, lower is better) and inference speed (FPS, higher is better) of various block sizes from 1 to 256.}
\label{fig:block_size_fvd_fps}
\end{figure}

\begin{table}[htbp]
\centering
\caption{Generation quality (FVD) of various block shape.}
\label{tab:block_shape}
\begin{tabular}{ccc}
\toprule
Block Size & Block Shape (T$\times$H$\times$W) & FVD$\downarrow$ \\
\midrule
16 & 1$\times$4$\times$4 & 33.4 \\
16 & 2$\times$1$\times$8 & 29.2 \\ 
16 & 1$\times$1$\times$16 & \textbf{25.5} \\\midrule
8  & 2$\times$2$\times$2  & 32.7 \\
8  & 1$\times$1$\times$8  & \textbf{25.7} \\
\bottomrule
\end{tabular}
\end{table}

\begin{figure*}[tbp]
\centering
\includegraphics[width=.9\textwidth]{figs/ucf.pdf}
\caption{Visualization of class-conditional generation (UCF-101) results of our method. The text below each video clip is the class name.}
\label{fig:ucf_gen}
\end{figure*}

\begin{figure*}[tbp]
\centering
\includegraphics[width=.9\textwidth]{figs/our_gen.pdf}
\caption{Visualization of frame prediction (K600) results of our method.}
\label{fig:our_gen}
\end{figure*}

\begin{figure*}[tbp]
\centering
\includegraphics[width=.9\textwidth]{figs/vis_gen.pdf}
\caption{Frame prediction results of OmniTokenizer and our method. The left part is the condition, and the right part is the predicted subsequent sequence.}
\label{fig:vis_gen}
\end{figure*}

\paragraph{Analysis of Attention Pattern.} 
We analyze the attention pattern in our \modelname framework using an example of next-clip prediction, where each block corresponds to a clip. 
Fig.~\ref{fig:txt_2clips_attn} shows the attention weights on UCF-101. Unlike the lower triangular distribution observed in AR models, our attention is characterized by a staircase pattern across blocks. In addition to high attention scores along the diagonal, the map reveals vertical stripe-like highlighted patterns, indicating that tokens at certain positions receive attention from all tokens. 
Fig.~\ref{fig:spatial-attn} illustrates the spatial attention distribution for a specific query (marked by \textcolor{red}{red $\times$}). This query can attend to all tokens within the clip, rather than being restricted to only the preceding tokens in a raster-scan order, enabling more effective spatial dependency modeling.

\section{Conclusion}
In this paper, we introduced a novel approach to video generation called Next Block Prediction using a semi-autoregressive modeling framework. This framework offers a more efficient and scalable solution for video generation, combining the advantages of parallelization with improved spatial-temporal dependency modeling. This method not only accelerates inference but also maintains or improves the quality of generated content, demonstrating strong potential for future applications in multimodal AI.


% Acknowledgements should only appear in the accepted version.
% \section*{Acknowledgements}

% \textbf{Do not} include acknowledgements in the initial version of
% the paper submitted for blind review.

% If a paper is accepted, the final camera-ready version can (and
% usually should) include acknowledgements.  Such acknowledgements
% should be placed at the end of the section, in an unnumbered section
% that does not count towards the paper page limit. Typically, this will 
% include thanks to reviewers who gave useful comments, to colleagues 
% who contributed to the ideas, and to funding agencies and corporate 
% sponsors that provided financial support.

\section*{Impact Statement}
This work advances the field of video generation through the development of \modelname. While recognizing the potential of this technology, we carefully consider its societal implications, particularly regarding potential misuse and ethical challenges. The model's capabilities could be exploited to create harmful content, including deepfakes for misinformation campaigns or other malicious purposes. Furthermore, we acknowledge the critical importance of ensuring that generated content adheres to ethical standards by avoiding the perpetuation of harmful stereotypes and respecting cultural diversity. 
To mitigate these risks, we will explore a comprehensive framework of safeguards, including (1) robust digital watermarking to ensure traceability and accountability of generated content; (2) reinforcement learning with human feedback to align model outputs with ethical guidelines and reduce potential harm; and (3) clear usage policies and restrictions. These measures collectively aim to promote responsible development and deployment of video generation technology while maximizing its positive societal impact.

% In the unusual situation where you want a paper to appear in the
% references without citing it in the main text, use \nocite
\nocite{langley00}

\bibliography{example_paper}
\bibliographystyle{icml2025}


%%%%%%%%%%%%%%%%%%%%%%%%%%%%%%%%%%%%%%%%%%%%%%%%%%%%%%%%%%%%%%%%%%%%%%%%%%%%%%%
%%%%%%%%%%%%%%%%%%%%%%%%%%%%%%%%%%%%%%%%%%%%%%%%%%%%%%%%%%%%%%%%%%%%%%%%%%%%%%%
% APPENDIX
%%%%%%%%%%%%%%%%%%%%%%%%%%%%%%%%%%%%%%%%%%%%%%%%%%%%%%%%%%%%%%%%%%%%%%%%%%%%%%%
%%%%%%%%%%%%%%%%%%%%%%%%%%%%%%%%%%%%%%%%%%%%%%%%%%%%%%%%%%%%%%%%%%%%%%%%%%%%%%%
\newpage
\appendix
\onecolumn
\section{Implementation Details}

\subsection{Task Definitions}
\label{app:task}
We introduce the tasks used in our training and evaluation. Each task is characterized by a few adjustable settings such as interior condition shape and optionally prefix condition. 
Given a video of shape $T\times H \times W$, we define the tasks as following:
\begin{itemize}
    \item Class-conditional Generation (CG)
    
    \begin{itemize}
        \item Prefix condition: class label.
    \end{itemize}
    
    % \item Class-conditional Frame Prediction (CFP)
    
    % \begin{itemize}
    %     \item Prefix condition: class label.
    %     \item Interior condition:  $t$ frames at the beginning; $t=1$.
    %     % \item Padding: replicate the last given frame.
    % \end{itemize}

    \item Frame Prediction (FP)
    
    \begin{itemize}
        \item Interior condition: $t$ frames at the beginning; $t=5$ for K600 dataset.
        % \item Padding: replicate the last given frame.
    \end{itemize}
    
\end{itemize}
% As we stated in $\S$~\ref{subsec:benchmark}, for UCF-101, other baselines perform the CG task, while our models perform the CFP task, as our method utilizes an image as an initial visual condition, alongside the classname. We take the first frame from the training videos into condition additionally. This ensures no information leakage from the test set. 
As we stated in $\S$~\ref{subsec:benchmark}, for UCF-101, all methods perform the CG task, while for K600, all methods perform the FP task.

\subsection{Model Configuration}
\label{app:model}

\paragraph{Video Tokenizer.}
Our video tokenizer shares the same model architecture with MAGVITv2~\cite{yu2023language}.

\paragraph{Decoder-only Generator.}
Table~\ref{tab:model_config} shows the configuration for the decoder-only generator. We use separate position encoding for text and video.  
We do not use advanced techniques in large language models, such as rotary position embedding (RoPE)~\citep{su2024roformer}, SwiGLU MLP, or RMS Norm~\citep{Touvron2023LLaMAOA}, which we believe could bring better performance.
% The configurations are mainly following LLaMA~\citep{Touvron2023LLaMAOA}

\subsection{Training}
\paragraph{Video Tokenizer.}
Table~\ref{tab:tok_train_config} shows the training configurations of our video tokenizer. 

\paragraph{Decoder-only Generator.}
Table~\ref{tab:gen_train_config} shows the training configurations of our video generator.

For both tokenizer and generator training, the video samples are all 17 frames, frame stride 1, 128$\times$128 resolution.

\subsection{Evaluation}
\label{app:eval}

\paragraph{Evaluation metrics.}
The FVD~\cite{unterthiner2018towards} is used as the primary evaluation metric. 
We follow the official implementation\footnote{\url{https://github.com/google-research/google-research/tree/master/frechet_video_distance}} in extracting video features with an I3D model trained on Kinetics-400~\cite{carreira2017quo}.
% We report Inception Score (IS)~\cite{saito2020train}\footnote{\url{https://github.com/pfnet-research/tgan2}} on the UCF-101 dataset which is calculated with a C3D~\cite{tran2015learning} model trained on UCF-101. 
% We further include image quality metrics: PSNR, SSIM~\cite{wang2004image} and LPIPS~\cite{zhang2018unreasonable} (computed by the VGG features).

\paragraph{Sampling protocols.}
We follow the sampling protocols from previous works~\cite{yu2023language, ge2022long,clark2019adversarial} when eveluating on the standard benchmarks, i.e. UCF-101, and Kinetics-600.
We sample 17-frame clips from each dataset without replacement to form the real distribution in FVD and extract condition inputs from them to feed to the model.
We continuously run through all the samples required (e.g., 40,000 for UCF-101) with a single data loader and compute the mean and standard deviation for 4 folds. 
We use top-$p$ and top-$k$ sampling with $k=16,000$ and $p=0.9$. 

Below are detailed setups for each dataset:
% Note that the evaluation resolution may be different from the generation resolution, where the generated samples are bilinear resized to the target resolution.

\begin{itemize}
    \item UCF-101: 
    \begin{itemize}
        \item Dataset: 9.5K videos for training, 101 classes.
        \item Number of samples: 10,000$\times$4.
        \item Resolution: 128$\times$128.
        \item Real distribution: random clips from the training videos.
        \item Video FPS: 8.
    \end{itemize}
    \item Kinetics-600:
    \begin{itemize}
        \item Dataset: 384K videos for training and 29K videos for evaluation.
        \item Number of samples: 50,000$\times$4.
        \item Generation resolution: 128$\times$128.
        \item Evaluation resolution: 64$\times$64, via central crop and bilinear resize.
        \item Video FPS: 25.
        % \item Real distribution: 6 sampled clips (2 temporal windows and 3 spatial crops) from each evaluation video.
        % \item COMMIT decoding: uniform schedule, temperature 7.5.
    \end{itemize}
\end{itemize}

\begin{table*}
\caption{Model sizes and architecture configurations of our generation model. The configurations are following LLaMA~\citep{Touvron2023LLaMAOA}.
}
\centering
\begin{tabular}{@{}lcccc@{}}
\toprule
Model & Parameters & Layers & Hidden Size & Heads \\
\midrule
\modelname-XL & 700M & 24 & 1536 &  16 \\
\modelname-XXL & 1.2B & 24 & 2048 &  32 \\
\modelname-3B & 3B & 32 & 3072 &  32 \\
\bottomrule
\end{tabular}
\label{tab:model_config}
% \vspace{-3mm}
\end{table*}
% Please add the following required packages to your document preamble:
% \usepackage{booktabs}
\begin{table}[]
\caption{Training configurations of video tokenizer.}
\label{tab:tok_train_config}
\centering
\begin{tabular}{l|cc}
\toprule
Hyper-parameters                  & UCF101                          & K600                            \\ \midrule
Video  FPS              & 8 & 8 \\
Latent shape                      & 5$\times$16$\times$16           & 5$\times$16$\times$16           \\
Vocabulary size                   & 64K                           & 64K                           \\
Embedding dimension               & 6                               & 6                               \\
Initialization                    & Random                          & Random                          \\
Peak learning rate                & 5e-5                            & 1e-4                            \\
Learning rate schedule            & linear & linear              \\
Warmup ratio                      & 0.01                            & 0.01                            \\
Perceptual loss weight            & 0.1                             & 0.1                             \\
Generator adversarial loss weight & 0.1                             & 0.1                             \\
Optimizer                         & Adam                            & Adam                            \\
Batch size                        & 256                             & 256                             \\
Epoch                             & 2000                            & 100                             \\ \bottomrule
\end{tabular}
\end{table}
% Please add the following required packages to your document preamble:
% \usepackage{booktabs}
\begin{table}[]
\caption{Training configurations of video generator (base model).}
\label{tab:gen_train_config}
\centering
\begin{tabular}{l|cc}
\toprule
Hyper-parameters                  & UCF101                          & K600                            \\ \midrule
Video  FPS              &  8  &  16 \\
Latent shape                      & 5$\times$16$\times$16           & 5$\times$16$\times$16           \\
Vocabulary size                   & 96K (including 32K text tokens)                          & 64K                           \\
Initialization                    & Random                          & Random                          \\
Peak learning rate                & 6e-4                            & 1e-3                            \\
Learning rate schedule            & linear & linear              \\
Warmup steps                      & 5,000                             & 10,000                             \\
Weight decay & 0.01 & 0.01 \\
Optimizer                         & Adam (0.9, 0.98)                           & Adam (0.9, 0.98)                           \\
Dropout & 0.1 & 0.1 \\
Batch size                        & 256                             & 64                             \\
Epoch                             & 2560                            & 77                             \\ \bottomrule
\end{tabular}
\end{table}

\section{Performance of Video Tokenizer}
\label{sec:vid-tok}
\begin{table*}[]
  \centering
  \caption{Video reconstruction results on UCF-101 and K600.}
  \label{tab:video_reconstruction}
  \setlength{\tabcolsep}{2.0pt}
    \vspace{0.04in}
  \resizebox{\linewidth}{!}{
  \begin{tabular}{lccccrrrrrrrr}
    \toprule
    \multicolumn{5}{c}{} & \multicolumn{4}{c}{UCF-101} & \multicolumn{4}{c}{K600} \\
    \cmidrule(r){6-9} \cmidrule(r){10-13}
    Method & Backbone & Quantizer & Param. & \# bits& rFVD$\downarrow$  & PSNR$\uparrow$ & SSIM$\uparrow$ & LPIPS$\downarrow$ & rFVD$\downarrow$ & PSNR$\uparrow$ & SSIM$\uparrow$ & LPIPS$\downarrow$ \\
    \midrule
    MaskGIT~\cite{chang2022maskgit} & 2D CNN & VQ & 53M & 10 & 216 & 21.5 & .685 & .1140 & - & - & - & - \\
    TATS~\cite{ge2022long} & 3D CNN & VQ & 32M & 14 & 162 & - & - & - & - & - & - & - \\
    OmniTokenizer~\cite{Wang2024OmniTokenizerAJ} & ViT & VQ & 78M & 13 & 42 & 30.3 & .910 & .0733 & 27 & 28.5 & .883 & .0945 \\
    MAGVIT-v1~\cite{yu2023magvit} & 3D CNN & VQ & 158M & 10 & 25 & 22.0 & .701 & .0990 & - & - & - & - \\
    MAGVIT-v2~\cite{yu2023language} & C.-3D CNN & LFQ & 158M & 18 & 16.12 & - & - & .0694 & - & - & - & - \\
    MAGVIT-v2~\cite{yu2023language} & C.-3D CNN & LFQ & 370M & 18  & 8.62 & - & - & .0537 & - & - & - & - \\
    % BSQ~\cite{Zhao2024ImageAV} & non-BC ViT & VQ & 174M & 14 & {9.16} & {33.06} & {.9518}& {.0223} &  &  &  &  \\
    % BSQ & BC ViT & VQ & 174M & 14 & 10.76 & 32.81 & .9496 & .0236 &  &  &  &  \\
    % BSQ~\cite{Zhao2024ImageAV} & BC ViT & BSQ & 174M & 18 & {8.1} & 32.1 & .942 & .0244 & - & - & -  & - \\
    % BSQ~\cite{Zhao2024ImageAV} & BC ViT & BSQ & 174M & 36 & 4.10  & 33.8 & .961 & .0159  & - & - & -  & - \\
    \midrule
    % Ours & C.-3D CNN & FSQ & 370M & 16 & 16.03 & 28.1 & .865 & .0633 & 8.82 & 29.3 & .915 & .0828 \\
    \modelname-Tokenizer (Ours) & C.-3D CNN & FSQ & 370M & 16 & 15.50 & 29.3 & .893 & .0648 & 6.73 & 31.3 & .944 & .0828 \\
    \bottomrule
  \end{tabular}
  }
  \vspace{-10pt}
\end{table*}

We present the reconstruction performance of our tokenizer in Table~\ref{tab:video_reconstruction}. Our tokenizer achieves 15.50 rFVD on UCF-101 and 6.73 rFVD on K600, surpassing OmniTokenizer~\cite{Wang2024OmniTokenizerAJ}, MAGVITv1~\cite{yu2023magvit}, and other models. Fig.~\ref{fig:vis_recons} compares the video reconstruction results of OmniTokenizer~\citep{Wang2024OmniTokenizerAJ} and our tokenizer. Our method significantly outperforms the baseline in both image clarity and motion stability. 

\section{Visualization}
We provide additional visualization of video generation results. 
Fig.~\ref{fig:vary_size_gen} shows results of various model sizes (700M, 1.2B and 3B).
Fig.~\ref{fig:vary_block_gen} shows results of various block sizes (1$\times$1$\times$1, 1$\times$1$\times$16 and 1$\times$16$\times$16).


\begin{figure*}[tbp]
\centering
\includegraphics[width=\textwidth]{figs/vis_recons.pdf}
\caption{Video reconstruction results (17 frames 128$\times$128 resolution at 25 fps and shown at 6.25 fps) of OmniTokenizer and our method. }
\label{fig:vis_recons}
\end{figure*}



% \begin{figure*}[tbp]
% \centering
% \includegraphics[width=\textwidth]{figs/our_gen_app.pdf}
% \caption{Visualization of video generation results of our 3B model.}
% \label{fig:our_gen_app}
% \end{figure*}

\begin{figure*}[tbp]
\centering
\includegraphics[width=\textwidth]{figs/vary_size_gen.pdf}
\caption{Visualization of video generation results of various model sizes (700M, 1.2B, and 3B).}
\label{fig:vary_size_gen}
\end{figure*}

\begin{figure*}[tbp]
\centering
\includegraphics[width=\textwidth]{figs/vary_block_gen.pdf}
\caption{Visualization of video generation results of various block sizes (1$\times$1$\times$1, 1$\times$1$\times$16 and 1$\times$16$\times$16).}
\label{fig:vary_block_gen}
\end{figure*}


% \begin{figure}[htbp]
% \includegraphics[width=\linewidth]{figs/k600_block_size.pdf}
% \caption{Training loss of various block sizes from 1 to 256. }
% \label{fig:block_size}
% \end{figure}



\begin{figure*}[tbp]
\centering
\includegraphics[width=.9\textwidth]{figs/txt_2clips_attn.pdf}
\caption{Attention weights of next-clip prediction on UCF-101. The horizontal and vertical axis represent the keys and queries, respectively. Two red lines on each axis divide the axis into three segments, corresponding to the text (classname), the first clip, and the second clip. The brightness of each pixel reflects the attention score. We downweight the attention to text tokens by $5\times$ to provide a more clear visualization.}
\label{fig:txt_2clips_attn}
\end{figure*}

\begin{figure*}[tbp]
\centering
\includegraphics[width=.9\textwidth]{figs/crop_img_attn.pdf}
\caption{Spatial attention distribution for a specific query (represented by \textcolor{red}{red $\times$}) on UCF-101.}
\label{fig:spatial-attn}
\end{figure*}
%%%%%%%%%%%%%%%%%%%%%%%%%%%%%%%%%%%%%%%%%%%%%%%%%%%%%%%%%%%%%%%%%%%%%%%%%%%%%%%
%%%%%%%%%%%%%%%%%%%%%%%%%%%%%%%%%%%%%%%%%%%%%%%%%%%%%%%%%%%%%%%%%%%%%%%%%%%%%%%


\end{document}


% This document was modified from the file originally made available by
% Pat Langley and Andrea Danyluk for ICML-2K. This version was created
% by Iain Murray in 2018, and modified by Alexandre Bouchard in
% 2019 and 2021 and by Csaba Szepesvari, Gang Niu and Sivan Sabato in 2022.
% Modified again in 2023 and 2024 by Sivan Sabato and Jonathan Scarlett.
% Previous contributors include Dan Roy, Lise Getoor and Tobias
% Scheffer, which was slightly modified from the 2010 version by
% Thorsten Joachims & Johannes Fuernkranz, slightly modified from the
% 2009 version by Kiri Wagstaff and Sam Roweis's 2008 version, which is
% slightly modified from Prasad Tadepalli's 2007 version which is a
% lightly changed version of the previous year's version by Andrew
% Moore, which was in turn edited from those of Kristian Kersting and
% Codrina Lauth. Alex Smola contributed to the algorithmic style files.
