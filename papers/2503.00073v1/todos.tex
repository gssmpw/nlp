\setlength{\marginparwidth}{2cm}
\usepackage[textsize=tiny]{todonotes}
% \usepackage[textsize=tiny,disable]{todonotes}
\makeatletter
\newcommand*\iftodonotes{\if@todonotes@disabled\expandafter\@secondoftwo\else\expandafter\@firstoftwo\fi}  % defines \iftodonotes{<true>}{<false>}, thanks to https://tex.stackexchange.com/questions/126559/conditional-based-on-packageoption
\makeatother
\newcommand{\noindentaftertodo}{\iftodonotes{\noindent}{}}
% Note that these macros accept optional arguments such as size=\small, bordercolor=red, and so on.  Capitalized versions are inline paragraphs instead of margin notes.
\newcommand{\fixme}[2][]{\todo[color=yellow,size=\tiny,fancyline,caption={},#1]{#2}} % to mark stuff that you know is missing or wrong when you write the text
\newcommand{\note}[4][]{\todo[author=#2,color=#3,size=\scriptsize,fancyline,caption={},#1]{#4}} % default note settings, used by macros below.

\newcommand{\notewho}[3][]{\note[#1]{#2}{blue!40}{#3}}
\newcommand{\Fixme}[2][]{\fixme[inline,#1]{#2}\noindentaftertodo}
\newcommand{\Notewho}[3][]{\notewho[inline,#1]{#2}{#3}\noindentaftertodo}
\newcommand{\alex}[2][]{} %\note[#1]{alex}{green!10}{#2}}
\newcommand{\Alex}[2][]{} %\alex[inline,#1]{#2}\noindentaftertodo}

\newcommand{\redact}[2]{#2}
