\section{Introduction}
\label{sec:intro}

3D Gaussian Splatting (3DGS)~\citep{kerbl20233d} has been widely applied to the field of 3D reconstruction and rendering, including novel view synthesis of static scenes~\citep{kerbl20233d, yu2024mip}, mesh surface reconstruction~\citep{guedon2024sugar, yu2024gaussian}, inverse rendering~\citep{liang2024gs, gao2023relightable}, and dynamic 3D reconstruction~\citep{wu20244d, lin2024gaussian},
%
However, in scenarios with less-observed areas, such as indoor scenes and unbounded scenes, radiance field optimization often suffers from overfitting to these limited input views~\citep{li2024dngaussian}, 
resulting in unreliable and corrupted 
geometry reconstruction.
%

To mitigate the issue, recent research efforts
~\citep{li2024dngaussian, paliwal2024coherentgs, turkulainen2024dnsplatter} 
have focused on incorporating geometric priors 
from input views into the training process, thereby 
regulating the optimization of radiance field 
represented by 3D Gaussian points. 
%
For instance, DN-Splatter~\citep{turkulainen2024dnsplatter} 
integrates sensor depth and normal cues 
into the reconstruction process. 
%
However, sensor depth acquisition is costly,
and the depth prior information from pre-trained 
monocular deep models inevitably suffer from the scale ambiguity~\citep{liu2023robust}.
While the normal prior provides even better geometric details, the scale ambiguity still exists due to its monocular nature.

% %
% In contrast to geometry priors from monocular estimations, the absolute scale information can be recovered from the matching flow prior information combined with pose inputs~\citep{zhu2023lighteddepth},
% %
% An intuitive approach is to incorporate the matching prior 
% information between training views during the training process. 
% %
% However, this method can produce unreliable results when the baseline between training views is large, as their overlap may be minimal. 
% %
% Incorporating matching priors from algorithmically controllable unobserved views remains an under explored area.
% %


% In contrast to monocular priors, pairwise matching priors can provide absolute scale information of the scene.
% %
% In this paper, we introduce \textbf{F}low \textbf{D}istillation \textbf{S}ampling (FDS), an online method for distilling matching prior from a pre-trained optical flow model into the 3DGS training process.
% %
% FDS aims to enhance the geometry quality of 
% gaussian radiance field by leveraging the 
% matching prior into unobserved novel views.
% %
% Specifically, we observe that the flow between the input view and the unobserved view, generated by the match prior model, can guide the flow rasterized directly from the 3DGS geometry. This remains effective even when the radiance field is poorly optimized and the image rendered from this unobserved viewpoint is blurry during training.
% %
% Specifically, we refer to the flow generated by the match prior model as \textbf{Prior Flow},
% and the flow analytically calculated from the 3DGS geometry as \textbf{Radiance Flow}.

%
In contrast to monocular priors, pairwise matching priors can provide absolute scale information of the scene. 
%
In this paper, we introduce \textbf{F}low \textbf{D}istillation \textbf{S}ampling (FDS), an online method for distilling matching prior from a pre-trained optical flow model into the 3DGS training process. FDS aims to enhance the geometry quality of Gaussian radiance field by leveraging the matching prior into the unobserved novel view. 
%
Specifically, we observe that the flow between the input view and the unobserved view generated by the match prior model (i.e., \textbf{Prior Flow}), can guide and refine the flow analytically calculated from the 3DGS geometry (i.e., \textbf{Radiance Flow}), improving the 3DGS reconstruction quality. 
%
Moreover, better 3DGS scene will lead to more accurate Prior Flow, creating a mutually reinforcing effect between two computed flow maps. This remains effective even when the radiance field is poorly optimized and the image rendered from the unobserved viewpoint is blurry during training. 
%
In addition, a camera sampling scheme is proposed to adaptively control the overlap between input view and sampled view for better Prior Flow calculation, which allows to leverage prior geometric knowledge more profoundly and thereby better enhance the 3DGS reconstruction quality.

% %
% Thus, during training, FDS utilizes the supervision 
% of Prior Flow to optimize Radiance Flow, improving 
% the geometry and render quality of radiance field 
% from unobserved views.
% %
% As the Radiance Flow is refined 
% and image rendered from sampled viewpoint is clearer, 
% the accuracy of the Prior Flow is also improved, 
% creating a mutually reinforcing effect 
% between two computed flow maps. 
% %
% Additionally, a camera 
% sampling scheme is proposed to control the overlap 
% between training viewpoint and sampled viewpoint adaptively.
% %
% This approach allows radiance field to leverage prior 
% geometric knowledge more profoundly, 
% thereby enhancing its accuracy of reconstruction.
% %

The proposed FDS has been extensively evaluated on MushRoom~\citep{ren2024mushroom}, ScanNet (V2)~\citep{dai2017scannet}, and Replica~\citep{replica19arxiv} datasets for the task of geometry reconstruction.
%
We apply FDS to two commonly used baseline approaches, namely 3DGS~\citep{kerbl20233d} and 2DGS ~\citep{Huang2DGS2024}.
%
The results demonstrate a significant improvement in geometry reconstruction accuracy.
% 
Additionally, we provide interpretive experiments and comprehensive analysis on the effectiveness of FDS, shedding light on its influence on the quality of geometric reconstruction and novel-view rendering.
%
Our contributions are summarized as follows:
\begin{itemize}
    \item FDS leverages matching prior information to recover absolute scale, 
    significantly enhancing the geometric quality of the Gaussian radiance field.
    \item An adaptive camera sampling scheme is proposed to selectively choose unobserved views with controllable overlap with the input view, further improving the geometric quality of the Gaussian radiance field even in less-observed areas.
    
    % \item By employing FDS and its adaptive camera sampling scheme, 
    % we achieve state-of-the-art results on 3D reconstruction tasks.
    \item By employing FDS and the adaptive camera sampling scheme, 
    we bring significant improvements to state-of-the-art 3D geometry reconstruction approaches.
\end{itemize}

\begin{figure} 
  \centering
  \begin{overpic}[width=1.\columnwidth,trim=40 140 20 90,clip]{figure/physical.pdf}
	\end{overpic}
  \caption{\textbf{Pipeline of the proposed FDS.} For each
  input view, we apply the FDS camera sampling scheme to generate corresponding
  unobserved sampled view. We then compute Radiance flow base on rendered depth and
  the Prior flow from matching prior model. Finally the Prior Flow is used to supervise Radiance flow, which enhances the geometric quality of Gaussian Radiance Field.}
  \label{fig:fig1}
\end{figure}


% \begin{figure} 
%   \centering
%   \begin{overpic}[width=1.0\columnwidth,trim=40 125 80 115,clip]{figure/teasor.pdf}
%   \put(10,0){A: Previous Methods.}
%   \put(63,0){B: Our FDS.}
% 	\end{overpic}
%   \caption{\textbf{Overview of our Flow Distillation Sampling (FDS) method.} By leveraging the matching prior of unobserved views in the Gaussian radiance field shown in (a), the geometry quality of the Gaussian radiance field is significantly enhanced, as demonstrated in (b).}
% \end{figure}