\section{Related Work}
\subsection{Neural Radiance Fields (NeRF)}
NeRF~\cite{nerf2020} is a revolutionary method that has emerged to provide high-quality scene representation by fitting a neural radiance field to a set of RGB images with corresponding poses. Vanilla NeRF involves querying a deep MLP model millions of times~\cite{reiser2021kilonerf}, leading to slow training and rendering speeds. Some research efforts have tried to speed up this process by using more efficient sampling schemes~\cite{piala2021terminerf, 10328666, verbin2024ref}, while some have attempted to apply improved data structures to represent the objects or scenes~\cite{10478788, muller2022instant, sun2022direct, fridovich2022plenoxels, chen2022tensorf}. Besides, to improve the NeRF training on low-quality images, enhancements have been made to handle degradation, such as blurring~\cite{ma2022deblur, wang2023bad, qi2023e2nerf, zhou2023nerflix}, lowlight~\cite{wang2023lighting}, and reflection~\cite{guo2022nerfren, zhu2022neural}. NeRF has been applied to a broader range of scenarios, including indoor scene reconstruction~\cite{10496207, wei2023depth, chen2023structnerf, yang2023nerfvs}, human body modeling~\cite{gao2022mps, peng2024animatable}, and 3D segmentation~\cite{cen2023segment, cen2023segment_pami, liu2024sanerf}. Recently, 3D Gaussian Splatting~\cite{kerbl3Dgaussians} has made significant progress in 3D scene representation, demonstrating its effectiveness in various domains including object reconstruction~\cite{charatan2024pixelsplat, chen2024liftimage3d}, medical applications~\cite{wang2024endogslam, li2024endosparse}, and avatar creation~\cite{qian20243dgs, zhao2024chase, zhao2024sg}. 
As NeRF-based 3D assets gain popularity among creators, protecting the copyright of these assets has become crucial.


\subsection{Digital watermarking for 2D images} 
2D digital watermarking is used for image verification, authenticity, and traceability. Initial 2D watermarking methods hide data in the least significant parts of image pixels \cite{413536}. Alternatively, some techniques embed data in transformed domains \cite{lai2010digital, kang2010efficient, wang2022dtcwt}. Recently, deep learning techniques have shown significant advancements in information hiding in images~\cite{zhu2018hidden, baluja2019hiding, jing2021hinet, ying2022rwn, zeng2023watermarks}. HiDDeN~\cite{zhu2018hidden} is one of the first deep image watermarking methods that employ deep encoder-decoder networks to achieve superior performance compared to traditional approaches. UDH~\cite{zhang2020udh} proposes a universal deep hiding architecture to achieve cover-agnostic watermark embedding. From then on, many methods have focused on more robust watermark embedding and extraction under distortion conditions, such as JPEG compression~\cite{jia2021mbrs}, screen recapture~\cite{fang2018screen, fang2022pimog, fang2023denol, liu2023wrap}, and combinations of several distortions~\cite{luo2020distortion}. Besides the encoder-decoder paradigm, some invertible networks have also been used for digital watermarking~\cite{guan2022deepmih, fang2023flow}. Recently, methods have been proposed to watermark the generative content~\cite{fernandez2023stable}. However, those 2D digital watermarking methods cannot be directly applied to protect the copyright of 3D models~\cite{luo2023copyrnerf}.

\subsection{Digital watermarking for 3D models}
Early 3D watermarking approaches~\cite{ohbuchi2002frequency, son2017perceptual, al2019graph} rely on Fourier or wavelet analysis on triangular or polygonal meshes to encode messages into model frequencies. However, these techniques take much time to work as the number of points in the model increases. Later approaches~\cite{zhou2018distortion, jiang2017reversible, tsai2022integrating, hou2023separable} suggest putting watermarks into the least significant bits and the most significant bits of vertex coordinates. Recently, some studies~\cite{wang2022neural, wang2022deep, zhu2024rethinking} have explored the feasibility of deep neural networks for 3D watermarking. Yoo \etal~\cite{Yoo_2022_CVPR} propose to embed messages in 3D meshes and extract them from 2D renderings. Although neural networks are commonly present in NeRF, watermarking methods designed for neural networks~\cite{zhang2024robust, tan2023deep, zeng2023watermarks, wu2020watermarking} cannot be directly applied to NeRF watermarking.

CopyRNeRF~\cite{luo2023copyrnerf} is the first method to watermark a NeRF. However, its long training time hinders its practicality in real-world scenarios. StegaNeRF~\cite{li2023steganerf} proposes to embed steganographic information within NeRF. In this method, the detector holds the most hidden information, which can make the information vulnerable to leaks, reducing its overall robustness. WateRF~\cite{jang2024waterf} leverages a fine-tuning technique for watermarking NeRFs, but it is limited to embedding just one signature following each fine-tuning process.