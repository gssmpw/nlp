\begin{abstract} 
    \laks{Given two input graphs, finding the largest subgraph that occurs in both, i.e., finding the maximum common subgraph,} is a fundamental operator for evaluating the similarity between two graphs in graph data analysis. Existing works for solving the problem are of either theoretical or practical interest, but not both. {\chengB Specifically}, the algorithms with a theoretical guarantee on the running time are known to be not practically efficient; algorithms following the recently proposed backtracking framework called \texttt{McSplit}, run fast in practice but do not have any theoretical guarantees. In this paper, we propose a new backtracking algorithm called \texttt{RRSplit}, which at once achieves better practical efficiency and provides a non-trivial theoretical guarantee on the worst-case running time.  {\chengB To achieve} the former, we develop a series of reductions and upper bounds for reducing redundant computations, i.e., the time  for exploring some unpromising branches \laks{of exploration} that hold no maximum common subgraph. {\chengB To achieve} the latter, we formally prove that \texttt{RRSplit} incurs a worst-case time complexity which matches {\chengB the best-known complexity for the problem.} Finally, we conduct extensive experiments on {\chengE several} benchmark graph collections, and the results demonstrate that our algorithm outperforms the practical state-of-the-art by several orders of {\chengB magnitude}.
\end{abstract}