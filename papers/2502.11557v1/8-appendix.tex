\appendix
\section{Additional Experimental Results}
{\Yui
In this section, we provide additional experimental results, including \emph{comparison by varying similarities} and \emph{comparison among various reductions}.

\smallskip
\noindent\textbf{Varying the similarities of two input graphs (additional results)}. We test different problem instances as the similarity varies on \textsf{CV} and \textsf{PR}. We remark that all tested problem instances in \textsf{CV} (resp. \textsf{PR}) have their similarities vary from 0.9 to 1 (resp. equal to 1). We report the average running time in Figure~\ref{fig:appendix_all_vary_S}(a) and (b) and the average number of formed branches in Figure~\ref{fig:appendix_all_vary_S}(c) and (d). The results show the similar clues to those on \textsf{BI} and \textsf{LV}. In specific, our \texttt{RRSplit} runs around 5$\times$-10$\times$ faster and forms fewer branches than \texttt{McSplitDAL}.

\smallskip
\noindent\textbf{Varying different reductions (additional results)}. We compare \texttt{RRSplit} with three variants, namely \texttt{RRSplit-VE}, \texttt{RRSplit-MB} and \texttt{RRSplit-UB}, on \textsf{CV} and \textsf{PR}. We report the number of solved problem instances in Figure~\ref{fig:appendix_all_vary_R} (a) and (b) for varying the time limit and in Figure~\ref{fig:appendix_all_vary_R} (c) and (d) for varying the limit of number of formed branches. The results on \textsf{CV} and \textsf{PR} show similar trends to those on \textsf{BI} and \textsf{LV}. First, we can see that all four algorithms performs better than \texttt{McSplitDAL}, among which \texttt{RRSplit} performs the best. This indicates the effectiveness of the proposed vertex-equivalence based reductions and maximality based reductions. Second, we note that \texttt{RRSplit-MB} and \texttt{RRSplit-VE} achieve the comparable performance. 
}
\begin{figure}[]
		\subfigure[\textsf{Running time (CV)}]{
			\includegraphics[width=4.0cm]{figure/appendix/ICVIU11_SIM.pdf}
		}	
		\subfigure[\textsf{Running time (PR)}]{
			\includegraphics[width=4.0cm]{figure/appendix/PR15_SIM.pdf}
		}
        \subfigure[\textsf{\# of branches (CV)}]{
			\includegraphics[width=4.0cm]{figure/appendix/ICVIU11_SIMB.pdf}
		}	
		\subfigure[\textsf{\# of branches (PR)}]{
			\includegraphics[width=4.0cm]{figure/appendix/PR15_SIMB.pdf}
		}
	\caption{Comparison by varying similarities (additional results)}
	\label{fig:appendix_all_vary_S}
\end{figure}

\begin{figure}[]
		\subfigure[\textsf{Varying time limits (CV)}]{
			\includegraphics[width=4.0cm]{figure/appendix/ICVIU11_RT.pdf}
		}	
		\subfigure[\textsf{Varying time limits (PR)}]{
			\includegraphics[width=4.0cm]{figure/appendix/PR15_RT.pdf}
		}
        \subfigure[\textsf{Varying limit of \#branches (CV)}]{
			\includegraphics[width=4.0cm]{figure/appendix/ICVIU11_BT.pdf}
		}	
		\subfigure[\textsf{Varying limit of \#branches (PR)}]{
			\includegraphics[width=4.0cm]{figure/appendix/PR15_BT.pdf}
		}
	\caption{Comparison among various reductions (additional results)}
	\label{fig:appendix_all_vary_R}
\end{figure}

\section{Additional Proofs}



\smallskip
\noindent\textbf{Lemma~\ref{lemma:reduction_for_VE1}}. \emph{
    Let $(S,C,D)$ be a branch. Common subgraph $S_{iso}$ at Equation~(\ref{eq:iso1}) has been found before the formation of $(S,C,D)$.}

\begin{proof}
    We note that the recursive branching process forms a recursion tree where each tree node corresponding to a branch. Consider the path from the initial branch $(\emptyset, V_Q\times V_G,\emptyset)$ to $(S,C,D)$, there exists an ascendant branch of $(S,C,D)$, denoted by $B_{asc}=(S_{asc},C_{asc},D_{asc})$, where $u_{equ}$ is selected as the branching vertex, since $\langle u_{equ},\phi(u_{equ}) \rangle$ is in $S$.
    %
    We can see that there exists one sub-branch $B'_{asc}=(S'_{asc},C'_{asc},D'_{asc})$ of $B_{asc}$ formed by including $\langle u_{equ},v \rangle$, and all common subgraphs within $B'_{asc}$ has been found before the formation of $(S,C,D)$, since $\langle u_{equ},v \rangle$ is in $D$. We then show that common subgraph $S_{iso}$ can be found within $B_{asc}'$, i.e., $S'_{asc}\subseteq S_{iso}\subseteq S'_{asc}\cup C'_{asc}$. 
    %
    \underline{First}, we have $S'_{asc}\subseteq S_{iso}$ since (1) $S'_{asc}=S_{asc}\cup\{\langle u_{equ},v \rangle\}$, (2) $S_{sub}$ is a common subgraph in $B_{asc}$ and thus $S_{asc}\subseteq S_{sub}$, (3) $S_{asc}$ does not include $\langle u_{equ},\phi(u_{equ})\rangle$ or $\langle u,v\rangle$ since they are in $C_{asc}$ and will be included to the partial solution at $B_{asc}'$ and $(S\cup \{\langle u,v \rangle\}, C\backslash u\backslash v)$, and thus (4) by combining all the above, we have  $S'_{asc}=S_{asc}\cup\{\langle u_{equ},v \rangle\}\subseteq S_{sub}\cup\{\langle u_{equ},v \rangle\} \backslash \{\langle u_{equ},\phi(u_{equ})\rangle,\langle u,v\rangle\}\subseteq S_{iso}$. 
    %
    \underline{Second}, we have $S_{iso}\subseteq S'_{asc}\cup C'_{asc}$ based on the following two facts. 

    \begin{itemize}
        \item \textbf{Fact 1.} $S_{sub}\backslash\{\langle u_{equ},\phi(u_{equ})\rangle,\langle u,v\rangle\}\subseteq S'_{asc}\cup C'_{asc}$.
        \item \textbf{Fact 2.} $\langle u_{equ},v \rangle\in S'_{asc}$ and $\langle u,\phi(u_{equ}) \rangle\in C'_{asc}$.
    \end{itemize}
    
    Fact 1 holds since (1) $S_{sub}\subseteq S_{asc}\cup C_{asc}$ (note that $S_{sub}$ is a common subgraph in $B_{asc}$), (2) vertices $u_{equ}$ and $v$ do not appear in $S_{sub}\backslash\{\langle u_{equ},\phi(u_{equ})\rangle,\langle u,v\rangle\}$ and thus we can derive that $S_{sub}\backslash\{\langle u_{equ},\phi(u_{equ})\rangle,\langle u,v\rangle\}\subseteq (S_{asc}\cup C_{asc})\backslash u_{equ}\backslash v$ (note that $\langle u_{equ},\phi(u_{equ})\rangle$ and $\langle u,v\rangle$ are the unique vertex pairs that consist of $u_{equ}$ and $v$  in $S_{sub}$, respectively), and (3) $S'_{asc}=S_{asc}\cup \{\langle u_{equ},v\rangle\}$ and $C'_{asc}=C_{asc}\backslash u_{equ}\backslash v$ based on the branching rule.

    Fact 2 can be verified as follows. Vertex pair $\langle u_{equ},v \rangle$ is in $S'_{asc}$ since $S'_{asc}=S_{asc}\cup \{\langle u_{equ},v\rangle\}$. We note that vertices $u_{equ}$, $\phi(u_{equ})$, $u$ and $v$ appear in $C_{asc}$ since $\langle u_{equ},\phi(u_{equ})\rangle$ and $\langle u,v\rangle$ are in $C_{asc}$ discussed before. Let $C_{asc}=X_1\times Y_1 \cup  X_2\times Y_2\cup \cdots \cup  X_c\times Y_c$ where $c$ is a positive integer. 
    %
    It is no hard to verify that, for two vertices $u$ and $u'$ (resp. $v$ and $v'$) appearing in $C_{asc}$, $u$ and $u'$ are in the same subset $X_i$ (resp. $Y_i$) of $C_{asc}$ with $1\leq i\leq c$ if and only if $u$ and $u'$ (resp. $v$ and $v'$) have the same set of neighbours and non-neighbours in $q$ (resp. $g$) according to Equation~(\ref{eq:update_candidate_set}).
    %for a subset $\langle X_i,Y_i \rangle$ with $1\leq i\leq c$, all vertices in $X_i$ (resp. $Y_i$) have the same set of neighbours and non-neighbours in $q$ (resp. $g$) since otherwise they will be split into different subsets according to Equation~(\ref{eq:update_candidate_set}); for any two different subsets
    %
    Besides, we have $X_i\cap X_j=\emptyset$ and $Y_i\cap Y_j=\emptyset$ for $1\leq i\neq j \leq c$ as discussed before. %since each vertex appearing in the candidate set is split into exactly one subset according to Equation~(\ref{eq:update_candidate_set}). 
    %
    Based on the above, we assume that $u_{equ}$ appears in a subset $X_i,Y_i$ of $C_{asc}$ where $u_{equ}$ is in $X_i$ and $1\leq i\leq c$. We can deduce that $u$ is in $X_i$ since $u_{equ}$ and $u$ are structurally equivalent and thus they have the same set of neighbours and non-neighbours in $q$. Besides, we can deduce that vertices $\phi(u_{equ})$ and $v$ are in $Y_i$ since (1) $\langle u_{equ},\phi(u_{equ})\rangle$ and $\langle u,v\rangle$ are in $C_{asc}$, (2) $u$ and $v$ are in exactly one subset $X_i$, and thus (3) they must appear in $X_i\times Y_i$.   
\end{proof}

\smallskip
\noindent\textbf{Lemma~\ref{lemma:reduction_for_VE2}}. \emph{Let $(S,C,D)$ be a branch where $u$ is selected as the branching vertex. Common subgraph $S_{iso}$ {\chengC defined in} Equation~(\ref{eq:iso2}) has been found before the formation of $(S,C\backslash u,D)$ at the second group.  }
\begin{proof}
    \underline{First}, we note that $\langle u_{equ},\phi_{iso}(u_{equ}) \rangle$ is in $C\backslash u$ and also in $C$ since otherwise $S_{sub}$ cannot include $\langle u_{equ},\phi_{iso}(u_{equ}) \rangle$. This is because (1) $S\subseteq S_{sub}\subseteq S\cup C\backslash u$ since $S_{sub}$ is a common subgraph in the sub-branch $(S,C\backslash u,D)$ and (2) $S$ does not include $\langle u_{equ},\phi_{iso}(u_{equ}) \rangle$ since $u_{equ}$ appears in $C\backslash u$.
    %
    \underline{Second}, we note that $\phi_{iso}(u_{equ})$ is in $Y$. Recall that $X\times Y$ is the branching set at $(S,C,D)$. This is because (1) $u_{equ}$ is in the same subset $X$ as $u$ since $u_{equ}$ and $u$ are structurally equivalent and thus have the same set of neighbours and non-neighbours in $q$, and (2) $\langle u_{equ},\phi_{iso}(u_{equ}) \rangle$ is in $C$ as discussed before.
    %
    \underline{Third}, we can derive that there exists a sub-branch $(S\cup \{\langle u,\phi_{iso}(u_{equ}) \rangle\},C\backslash u\backslash \phi_{iso}(u_{equ}),D')$, which is formed at branch $(S,C,D)$  by including $\langle u,\phi_{iso}(u_{equ}) \rangle$ before the formation of $(S,C\backslash u,D)$, since $\phi_{iso}(u_{equ})\in Y$.
    %
    \underline{Forth}, we show that $S_{iso}$ is in $(S\cup \{\langle u,\phi_{iso}(u_{equ}) \rangle\},C\backslash u\backslash \phi_{iso}(u_{equ}),D')$, formally, $S\cup \{\langle u,\phi_{iso}(u_{equ}) \rangle\} \subseteq S_{iso}\subseteq S\cup \{\langle u,\phi_{iso}(u_{equ}) \rangle\}\cup (C\backslash u\backslash \phi_{iso}(u_{equ}))$.
    We have $S\subseteq S_{sub}\subseteq S\cup (C\backslash u)$ since $S_{sub}$ is a common subgraph in $(S,C\backslash u,D)$. Let $S'=S\cup \{\langle u,\phi_{iso}(u_{equ}) \rangle\}$, it can be proved as below.
    \begin{eqnarray}
        && S\subseteq S_{sub}\subseteq S\cup (C\backslash u)\\
        && \Rightarrow S\subseteq S_{sub}\backslash\{\langle u_{equ}, \phi_{iso}(u_{equ})\rangle\}\subseteq S\cup (C\backslash u\backslash \phi_{iso}(u_{equ})) \label{eq:llemma_eq_1}\\
        && \Rightarrow S' \subseteq S_{iso}\subseteq S'\cup (C\backslash u\backslash \phi_{iso}(u_{equ})) \label{eq:llemma_eq_2}
    \end{eqnarray}
    Note that Equation~(\ref{eq:llemma_eq_1}) holds since $\langle u_{equ},\phi_{iso}(u_{equ}) \rangle$ is in $S$; Equation~(\ref{eq:llemma_eq_2}) is derived by including the vertex pair $\langle u, \phi_{iso}(u_{equ})\rangle$.
    %We first prove the fact that $\phi_{iso}(u_{equ})\in Y^*$ since (1) $S\cup\langle u^*,\phi_{iso}(u_{equ}) \rangle$, as a subgraph of common subgraph $S_{iso}$ (i.e., $S\cup\langle u^*,\phi_{iso}(u_{equ})\subseteq S_{iso}$), is also a common subgraph based on Definition~\ref{def:CIS}, (2) $\langle u_{equ},\phi(u_{equ}) \rangle$, which contains $\phi_{iso}(u_{equ})$, is in $C\backslash u^*$ since otherwise $S_{sub}$ cannot include $\langle u_{equ},\phi(u_{equ}) \rangle$ (note that $S\subseteq S_{sub}\subseteq S\cup C\backslash u^*$ since $S_{sub}$ is a common subgraph in the sub-branch $(S,C\backslash u^*,D)$ and $S$ does not include $\langle u_{equ},\phi(u_{equ}) \rangle$ since $u_{equ}$ appears in $C\backslash u^*$), (2) $\langle u^*,\phi_{iso}(u_{equ}) \rangle$ is in 
\end{proof}

\smallskip
\noindent\textbf{Lemma~\ref{lemma:maximality}} \emph{Let $B=(S,C,D)$ be a branch {\chengC and $\langle u,v \rangle$ be a candidate vertex pair that satisfies the condition in Equation~(\ref{eq:condition}).} There exists one largest common subgraph $S_{opt}$ in the branch $B$ such that $S_{opt}$ contains 
    {\chengC $\langle u,v \rangle$.}}
\begin{proof}
    This can be proved by construction. Let $S^*=(q^*,g^*,\phi^*)$ be one largest common subgraph to be found in $B$. Note that if $S^*$ contains the candidate vertex pair $\langle u,v \rangle$, we can finish the proof by constructing $S_{opt}$ as $S^*$. Otherwise, if $\langle u,v \rangle$ is not in $S^*$, we prove the correctness by constructing one largest common subgraph $S_{opt}$ to be found in $B$ that contains candidate vertex pair $\langle u,v \rangle$, i.e., $S\subseteq S_{opt} \subseteq S\cup C$,  $|S_{opt}|=|S^*|$ and $\langle u,v \rangle\in S_{opt}$.
    %
    In general, there are four different cases.

    \smallskip
    \noindent\underline{\textbf{Case 1:}} $u\notin V_{q^*}$ and $v\in V_{g^*}$. In this case, there exists a vertex pair $\langle\phi^{*-1}(v),v \rangle$ in $S^*$ where $\phi^{*-1}$ is the inverse of $\phi^*$. We construct $S_{opt}$ by replacing the vertex pair $\langle\phi^{*-1}(v),v \rangle$ with $\langle u,v \rangle$, i.e.,
    \begin{equation}
        S_{opt}=S^*\backslash\{ \langle\phi^{*-1}(v),v \rangle\} \cup \{\langle u,v \rangle\}.
    \end{equation}
    Clearly, we have $S\subseteq S_{opt}\subseteq S\cup C$ (i.e., $S_{opt}$ is in $B$) since $S^*$ is in $B$ and $\langle u,v\rangle$ is in the candidate set $C$. Besides, we have $|S_{opt}|=|S^*|$ and $\langle u,v \rangle\in S_{opt}$ based on the above construction. Finally, we deduce that $S_{opt}$ is a common subgraph by showing that any two vertex pairs in $S_{opt}$ satisfy Equation~(\ref{eq:isomorphic}), i.e., $g_{opt}$ is isomorphic to $q_{opt}$ under the bijection $\phi_{opt}$. \underline{First}, $S^*\backslash\{\langle \phi^{*-1}(v),v\rangle\}$, as a subset of $S^*$, is a common subgraph and thus has any two vertex pairs inside satisfying Equation~(\ref{eq:isomorphic}) (note that any subset of a common subgraph is still a common subgraph); \underline{Second}, for each pair $\langle u',v' \rangle$ in $S$, $u$ is adjacent to $u'$ if and only if $v$ is adjacent to $v'$ (since $\langle u,v \rangle$ is a candidate pair which can form a common subgraph with $S$); \underline{Third}, for each pair $\langle u',v' \rangle$ in $S_{opt}\backslash S\backslash\{\langle \phi^{*-1}(v),v\rangle\}$, it is clear that $\langle u',v' \rangle$ is in one subset $ X\times Y$ of $\mathcal{P}(C)$ and thus $u$ is adjacent to $u'$ if and only if $v$ is adjacent to $v'$ based on Equation~(\ref{eq:condition}). Therefore, any two vertex pairs in $S_{opt}$ will satisfy the Equation~(\ref{eq:isomorphic}).

    \smallskip
    \noindent\underline{\textbf{Case 2:}} $u\in V_{q^*}$ and $v\notin V_{g^*}$. There exists a vertex pair $\langle u,\phi^*(u) \rangle$ in $S^*$. We construct $S_{opt}$ by replacing $\langle u,\phi^*(u) \rangle$ with $\langle u,v \rangle$, i.e., $S_{opt}=S^*\backslash \{\langle u,\phi^*(u) \rangle\}\cup\{\langle u,v \rangle\}$. Similar to Case 1, we can prove that $S_{opt}$ includes $\langle u,v \rangle$ and is one largest common subgraph to be found in $B$. 

    \smallskip
    \noindent\underline{\textbf{Case 3:}} $u\in V_{q^*}$ and $v\in V_{g^*}$. There exists two distinct vertex pairs $\langle u,\phi^*(u) \rangle$ and $\langle \phi^{*-1}(v),v \rangle$ in $S^*$. We construct $S_{opt}$ by replacing these two vertex pairs with $\langle \phi^{*-1}(v),\phi(u) \rangle$ and $\langle u,v \rangle$, formally,
    \begin{equation}
        S_{opt}\!\!=\!\!S^*\backslash\{\langle u,\phi^*(u) \rangle,\!\langle\phi^{*-1}(v),v \rangle\}\!\cup\!\{\langle \phi^{*-1}(v),\phi^*(u) \rangle,\!\langle u,v \rangle\}.
    \end{equation}
    Clearly, we have $S\subseteq S_{opt}\subseteq S\cup C$ (i.e., $S_{opt}$ is in $B$), $|S_{opt}|=|S^*|$ and $\langle u,v \rangle\in S_{opt}$ based on the above construction. We then deduce that $S_{opt}$ is a common subgraph  by showing that any two vertex pairs in $S_{opt}$ satisfy Equation~(\ref{eq:isomorphic}).
    %
    \underline{First}, $S^*\backslash\{\langle u,\phi^*(u) \rangle,\langle \phi^{*-1}(v),v\rangle\}$, as a subset of $S^*$, is a common subgraph and thus has any two vertex pairs inside satisfying Equation~(\ref{eq:isomorphic});
    %
    \underline{Second}, consider a vertex pair $\langle u',v' \rangle$ in $S^*\backslash\{\langle u,\phi^*(u) \rangle,\langle \phi^{*-1}(v),v\rangle\}$. Similar to Case 1, we can prove that $u$ is adjacent to $u'$ if and only if $v$ is adjacent to $v'$. Besides, we show that $\phi^{*-1}(v)$ is adjacent to $u'$ if and only if $\phi(u)$ is adjacent to $v'$ since (1) $(\phi^{*-1}(v),u')\in E_Q\Leftrightarrow (v,v')\in E_G$ and $(u,u')\in E_Q\Leftrightarrow (\phi^*(u),v')\in E_G$ (since the common subgraph $S^*$ contains $\{\langle u,\phi^*(u) \rangle,\langle \phi^{*-1}(v),v\rangle\}$), (2) $ (v,v')\in E_G \Leftrightarrow (u,u')\in E_Q$ (as we shown above), and thus (3) they can be combined as $(\phi^{*-1}(v),u')\in E_Q\Leftrightarrow (v,v')\in E_G \Leftrightarrow (u,u')\in E_Q \Leftrightarrow (\phi^*(u),v')\in E_G$.
    %\begin{equation}
        %(\phi^{*-1}(v),u')\in E_Q\Leftrightarrow (v,v')\in E_G \Leftrightarrow (u,u')\in E_Q \Leftrightarrow (\phi^*(u),v')\in E_G  \nonumber
    %\end{equation}
    %
    \underline{Third}, we have $(u,\phi^{*-1}(v))\in E_Q\Leftrightarrow (v,\phi^*(u))\in E_G$ since the common subgraph $S^*$ contains $\{\langle u,\phi^*(u) \rangle,\langle \phi^{*-1}(v),v\rangle\}$ and thus $(u,\phi^{*-1}(v))\in E_Q\Leftrightarrow (\phi^*(u),v)\in E_G$ (note that $(\phi^*(u),v)$ refers to the same edge as $(v,\phi^*(u))$ since the graphs $Q$ and $G$ are undirected). Therefore, any two vertex pairs in $R_{opt}$ will satisfy Equation~(\ref{eq:isomorphic}).

    \smallskip
    \noindent\underline{\textbf{Case 4:}} $u\notin V_{q^*}$ and $v\notin V_{g^*}$. We note that this case will not occur since otherwise the contradiction is derived by showing that $S^*\cup \{\langle u,v\rangle\}$ is a larger common subgraph (note that the proof is similar to Case 1 and thus be omitted).
\end{proof}


\smallskip
\noindent\textbf{Fact}. \emph{Let $(S,C,D)$ be a branch where $D=\{u_1\}\times A_1 \cup \{u_2\} \times A_2 \cup ... \cup \{u_d\}\times A_d$ and $d$ is a positive integer. For $1\leq i\neq j \leq d$, if $u_i$ and $u_j$ are structurally equivalent, we have $A_i\cap A_j = \emptyset$.}
\begin{proof}
    This can be proved by contradiction. Assume that there exists a vertex $v$ in $A_i\cap A_j$. Clearly, $u_i$ and $u_j$ are in $S$. Consider the path from the initial branch $(\emptyset,V_Q\times V_G,\emptyset)$ to the branch $(S,C,D)$ in the recursion tree. Without loss of the generality, suppose that $u_i$ is selected as the branching vertex before $u_j$ in the path. Hence, consider the ascendant branch of $(S,C,D)$, denoted by $B_{asc}=(S_{asc},C_{asc},D_{asc})$, where $u_j$ is selected as the branching vertex. We can easily deduce that $\langle u_i,v \rangle$ is in $D_{asc}$, vertex $u_j$ does not appear in $D_{asc}$ and $\langle u_j,v \rangle$ is in $C_{asc}$. For a sub-branch of $B_{asc}$ which is formed by including $\langle u_j,v \rangle$, i.e., $(S_{asc}\cup\{\langle u_j,v \rangle\}, C\backslash u_j\backslash v)$, it can be pruned by the proposed reduction at the first group since there exists a vertex pair $\langle u_i,v \rangle$ in $D_{asc}$ such that $u_i\in \Psi (u_j)$. As a result, $\langle u_j,v \rangle$ will not be included to $D$ according to the maintenance of the exclusion set, which leads to the contradiction.
\end{proof}