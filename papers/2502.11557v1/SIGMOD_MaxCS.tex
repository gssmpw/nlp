%%
%% This is file `sample-sigconf.tex',
%% generated with the docstrip utility.
%%
%% The original source files were:
%%
%% samples.dtx  (with options: `sigconf')
%% 
%% IMPORTANT NOTICE:
%% 
%% For the copyright see the source file.
%% 
%% Any modified versions of this file must be renamed
%% with new filenames distinct from sample-sigconf.tex.
%% 
%% For distribution of the original source see the terms
%% for copying and modification in the file samples.dtx.
%% 
%% This generated file may be distributed as long as the
%% original source files, as listed above, are part of the
%% same distribution. (The sources need not necessarily be
%% in the same archive or directory.)
%%
%%
%% Commands for TeXCount
%TC:macro \cite [option:text,text]
%TC:macro \citep [option:text,text]
%TC:macro \citet [option:text,text]
%TC:envir table 0 1
%TC:envir table* 0 1
%TC:envir tabular [ignore] word
%TC:envir displaymath 0 word
%TC:envir math 0 word
%TC:envir comment 0 0
%%
%%
%% The first command in your LaTeX source must be the \documentclass command.
\documentclass[sigconf]{acmart}


\usepackage[linesnumbered,ruled,vlined]{algorithm2e}
\usepackage{amsthm}
\usepackage{multirow}
\usepackage{enumitem}
\usepackage{subfigure} 
\usepackage{soul}



\newtheorem{problem}{Problem}
\newtheorem{pruning}{Pruning}
\newtheorem{theorem}{Theorem}
%\newtheorem{definition}{Definition}
\newtheorem{lemma}{Lemma}
\newtheorem{example}{Example}
\newtheorem{observation}{Observation}
\newtheorem{property}{Property}
\newtheorem{corollary}{Corollary}

%\newcommand{\cheng}{\color{blue}}
%\newcommand{\chengB}{\color{red}}
%\newcommand{\Yui}{\color{cyan}}
%\newcommand{\kaixin}{\color{orange}}

\newcommand{\cheng}{}
\newcommand{\chengB}{}
\newcommand{\Yui}{}

\if 0
\newcommand{\chengC}{}
\newcommand{\chengD}{\color{red}}
\newcommand{\YuiR}{\color{cyan}}
\newcommand{\YuiRR}{\color{cyan}}
\newcommand{\kaixin}{\color{brown}}
\newcommand{\laks}[1]{\textcolor{magenta}{#1}}
\newcommand{\LL}[1]{\textcolor{red}{[[Laks: #1]]}} 
\newcommand{\YuiRRR}{}
\fi
\newcommand{\eat}[1]{}
%%
\newcommand{\revision}{}
%% For submission

%\def\CR{1}

\newcommand{\chengC}{}
\newcommand{\chengD}{}
\newcommand{\chengE}{}
\newcommand{\YuiR}{}
\newcommand{\kaixin}{}
\newcommand{\laks}{}
\newcommand{\LL}{}
\newcommand{\YuiRR}{}
\newcommand{\YuiRRR}{}




%% \BibTeX command to typeset BibTeX logo in the docs
\AtBeginDocument{%
  \providecommand\BibTeX{{%
    \normalfont B\kern-0.5em{\scshape i\kern-0.25em b}\kern-0.8em\TeX}}}


%% Rights management information.  This information is sent to you
%% when you complete the rights form.  These commands have SAMPLE
%% values in them; it is your responsibility as an author to replace
%% the commands and values with those provided to you when you
%% complete the rights form.

\setcopyright{acmcopyright}
\copyrightyear{2018}
\acmYear{2018}
\acmDOI{XXXXXXX.XXXXXXX}

%% These commands are for a PROCEEDINGS abstract or paper.
%\acmConference[ ]{ }{ }{ }
\acmPrice{15.00}
\acmISBN{978-1-4503-XXXX-X/18/06}

%\settopmatter{printacmref=false}
%%
%% Submission ID.
%% Use this when submitting an article to a sponsored event. You'll
%% receive a unique submission ID from the organizers
%% of the event, and this ID should be used as the parameter to this command.
%%\acmSubmissionID{123-A56-BU3}

%%
%% The majority of ACM publications use numbered citations and
%% references.  The command \citestyle{authoryear} switches to the
%% "author year" style.
%%
%% If you are preparing content for an event
%% sponsored by ACM SIGGRAPH, you must use the "author year" style of
%% citations and references.
%% Uncommenting
%% the next command will enable that style.
%%\citestyle{acmauthoryear}

%%
%% end of the preamble, start of the body of the document source.
\begin{document}
\sloppy
%\include{Revision/X.response}
%%
%% The "title" command has an optional parameter,
%% allowing the author to define a "short title" to be used in page headers.
\title{Fast Maximum Common Subgraph Search: A Redundancy-Reduced Backtracking Approach}
%\author{Anonymous Authors}
%\affiliation{\institution{ }}
%\email{ }

%%
%% The "author" command and its associated commands are used to define
%% the authors and their affiliations.
%% Of note is the shared affiliation of the first two authors, and the
%% "authornote" and "authornotemark" commands
%% used to denote shared contribution to the research.
\author{Kaiqiang Yu}
\orcid{0000-0003-1153-2902}
\affiliation{%
  \institution{Nanyang Technological University}
  \country{Singapore}
}
\email{kaiqiang002@e.ntu.edu.sg}

\author{Kaixin Wang}
\orcid{0000-0002-6650-2850}
\affiliation{%
  \institution{Beijing University of Technology}
  \city{Beijing}
  \country{China}
}
\email{kaixin001@e.ntu.edu.sg}

\author{Cheng Long}
\orcid{0000-0001-6806-8405}
\authornote{Corresponding author (c.long@ntu.edu.sg)}
\affiliation{%
  \institution{Nanyang Technological University}
  \country{Singapore}
}
\email{c.long@ntu.edu.sg}

\author{Laks Lakshmanan}
\orcid{0000-0002-9775-4241}
\affiliation{%
  \institution{The University of British Columbia}
  \city{Vancouver}
  \country{Canada}
}
\email{laks@cs.ubc.ca}

\author{Reynold Cheng}
\orcid{0000-0002-9480-9809}
\affiliation{%
  \institution{The University of Hong Kong}
  \city{Hong Kong}
  \country{China}
}
\email{ckcheng@cs.hku.hk}





%%
%% By default, the full list of authors will be used in the page
%% headers. Often, this list is too long, and will overlap
%% other information printed in the page headers. This command allows
%% the author to define a more concise list
%% of authors' names for this purpose.

%%
%% The abstract is a short summary of the work to be presented in the
%% article.
\begin{abstract}
Out-of-distribution (OOD) detection and OOD generalization are widely studied in Deep Neural Networks (DNNs), yet their relationship remains poorly understood. We empirically show that the degree of Neural Collapse (NC) in a network layer is inversely related with these objectives: stronger NC improves OOD detection but degrades generalization, while weaker NC enhances generalization at the cost of detection. This trade-off suggests that a single feature space cannot simultaneously achieve both tasks. To address this, we develop a theoretical framework linking NC to OOD detection and generalization. We show that entropy regularization mitigates NC to improve generalization, while a fixed Simplex Equiangular Tight Frame (ETF) projector enforces NC for better detection. Based on these insights, we propose a method to control NC at different DNN layers. In experiments, our method excels at both tasks across OOD datasets and DNN architectures. 

\begin{comment}   

Out-of-distribution (OOD) detection and OOD generalization are critical for deploying machine learning models in real-world scenarios. While substantial progress has been made in addressing these problems independently, few works have attempted to tackle them jointly. However, existing methods often rely on auxiliary OOD training data and primarily focus on covariate-shifted OOD data that share labels with in-distribution (ID) data. In contrast, we tackle the more realistic and challenging task of jointly detecting and generalizing to semantic OOD data with disjoint labels from the ID data, without auxiliary OOD training data.
Achieving both objectives simultaneously is inherently difficult due to a fundamental conflict — OOD generalization requires enhanced transferability, while OOD detection necessitates the inhibition of transfer.
To address this, we leverage insights from neural collapse (NC) — a phenomenon in deep networks where top-layer representations suppress feature variability and adopt a Simplex Equiangular Tight Frame (ETF) structure, impairing transferability. By controlling NC, we unify OOD detection and generalization: preventing NC enhances OOD transfer while inducing NC improves OOD detection.
Our proposed method excels at both tasks across various OOD datasets and architectures. 

\end{comment}


\end{abstract}

%%
%% The code below is generated by the tool at http://dl.acm.org/ccs.cfm.
%% Please copy and paste the code instead of the example below.
%%

\begin{CCSXML}
<ccs2012>
   <concept>
       <concept_id>10002950.10003624.10003633.10010917</concept_id>
       <concept_desc>Mathematics of computing~Graph algorithms</concept_desc>
       <concept_significance>500</concept_significance>
       </concept>
   <concept>
       <concept_id>10002951.10003317.10003325</concept_id>
       <concept_desc>Information systems~Information retrieval query processing</concept_desc>
       <concept_significance>300</concept_significance>
       </concept>
 </ccs2012>
\end{CCSXML}

\ccsdesc[500]{Mathematics of computing~Graph algorithms}
\ccsdesc[500]{Information systems~Information retrieval query processing}


\keywords{Graph similarity search; maximum common subgraph; subgraph matching; subgraph isomorphism}



%%
%% This command processes the author and affiliation and title
%% information and builds the first part of the formatted document.
\maketitle

\section{Introduction}
\label{sec:intro}

Foundational models (FMs)~\cite{zhang2024data, zhou2023comprehensive} have shown remarkable progress in the healthcare domain, enabling professional-like assessment of disease diagnosis, treatment decision-making, and monitoring~\cite{zhang2023text, wang2022medclip, lu2023mi-zero}. 
Examples include LLaVA-Med~\cite{li2023llava}, Med-PaLM Multimodal~\cite{tu2024towards}, and Med-Flamingo~\cite{moor2023med}, have demonstrated their capacity on question answering, medical image analysis, and report generation.
These studies follow a predominant top-down model development strategy that requires upstream developers to collect data and train models for downstream tasks. 
Consequently, the developed model capabilities are heavily dependent on the training data, limiting their generalization performance in diverse clinical scenarios. 
For instance, Med-Gemini~\cite{yang2024advancing} reveals promising general capabilities in report generation while it lags behind state-of-the-art (SoTA) models on classification tasks, especially for out-of-domain applications. 
This indicates that while the generalizability of the foundation model is promising, more solutions are expected to meet the various specialized clinical needs.

To address these challenges, multi-center data centralization becomes essential to enhance model capacity and robustness across varied clinical scenarios~\cite{rajpurkar2022ai}. 
Centralizing distributed data can significantly improve model training and inference performance.
However, the process of medical data storage, transfer, and aggregation among centers requires extra efforts to ensure data security and system interoperability~\cite{bradford2020international}.
Moreover, a growing concern for patient privacy makes large-scale multi-center data sharing particularly challenging. 
While efforts like federated learning~\cite{wen2023survey, li2020review} can achieve good model performance on local data, the need for synchronized system coordination presents significant challenges, as clients are unable to update asynchronously. This limitation greatly restricts the practical capability of such approaches.
As a result, without a flexible collaboration, medical community still struggles to fully utilize the isolated data and local computation resources for comprehensive medical AI model development. 
To address this dilemma, open-source platforms encourage public data sharing and knowledge integration~\cite{markiewicz2021openneuro, zenodo}.
However, these platforms focus solely on raw data sharing while seldom providing collaborative model training or cooperation between different institutions.
Recently, collaborative learning has emerged as a viable approach for enhancing multi-model robustness~\cite{boulemtafes2020review}. 
For instance, software-like model development~\cite{raffel2023building} mimics software engineering practices by introducing structured workflows, enabling merging, version control, and continuous model integration.
Under this design, model ability can be strengthened with incremental knowledge updates similar to the version updating in software development. 

Although collaborative learning provides a multi-model collaboration, two key challenges remain in the leakage of raw data during collaboration~\cite{huang2023lorahub} and the synchronization of multiple collaborators~\cite{mcmahan2017communication} in the medical AI community. It is still challenging to integrate decentralized, privacy-sensitive data across institutions, leading to under-utilized insights and fragmented knowledge sharing~\cite{kaissis2020secure, rajpurkar2022ai, abdullah2021ethics}.
 To address these challenges, inspired by the collaborative software development, we propose \textbf{Med}ical \textbf{Fo}undation Models Me\textbf{rg}ing (\textbf{MedForge}), a cooperative workflow enabling continuously community-driven foundation model (FM) development.
MedForge enables a lightweight manner for individual centers to share their knowledge among multiple centers, minimizing the burden of data transmission and integration while enhancing model robustness.
Meanwhile, MedForge facilitates asynchronous and flexible collaboration, allowing individual centers to continuously update and improve medical FMs without the need for real-time synchronization.
Similar to open-source software development, MedForge incrementally updates medical knowledge and follows a sustainable model development scheme. 
This key design emphasizes a bottom-up construction of a multi-task medical FM, allowing downstream users to collaboratively build, refine, and update the upstream model according to their local resources. Our major contributions of MedForge are as below: 
\begin{enumerate}
    \item[$\bullet$] We introduce a collaborative workflow to promote the merging scheme of open-source software development. Our proposed MedForge allows distributed clinical centers to asynchronously contribute to comprehensive medical model construction while reducing transmitting costs among centers and avoiding the leakage of raw data, thus enhancing the utilization of private resources in the healthcare system. 
    \item[$\bullet$] We propose two effective knowledge-merging strategies for the asynchronous branch contribution. The MedForge-Fusion strategy updates the plugin module parameters of the main model during the merging phase, whereas the MedForge-Mixture strategy integrates the output of the plugin module by memorizing each contributor's coefficient. These strategies make MedForge more flexible and versatile. MedForge-Fusion is friendly to implement, while the MedForge-Mixture offers better performance and robustness.
    \item[$\bullet$]  We comprehensively evaluate model merging strategies to accumulate medical knowledge among multiple branch plugin modules. MedForge yields superior performance on medical classification tasks compared to other collaborative baselines across multiple datasets. We demonstrate the robustness of MedForge by shuffling the task order and evaluating various configurations of plugin modules and dataset distillation methods.
\end{enumerate}



This section formalizes the carbon-aware scheduling problem and motivates insights to contextualize 
our desiderata.

\vspace{-0.5em}
\subsection{Carbon-aware DAG scheduling problem}

Each job %
is represented as a directed acyclic graph (DAG) $\mathcal{J} = \{ \mathcal{V}, \mathcal{E} \}$, where each node in $\mathcal{V}$ is one of $n$ tasks, and each edge in $\mathcal{E}$ encodes precedence constraints between tasks -- e.g., for tasks $j, j' \in \mathcal{V}$, an edge $j \to j'$ indicates that $j'$ cannot start until after $j$ has completed.
A typical data processing cluster includes $K \geq 1$ machines (or executors).  More than one job can simultaneously run on a cluster -- e.g., given a set of current jobs $\{ \mathcal{J} \}$, the scheduler assigns tasks to machines over time while respecting precedence and capacity constraints.  We index continuous time by $t \geq 0$.

The goal of a typical scheduler is \textit{performance}, e.g., in terms of throughput, utilization, and average job completion time.  In this work, we additionally consider the goal of \textit{carbon-awareness} -- with respect to a time varying carbon signal given by a function $c(t) : t \geq 1$, a carbon-aware scheduler's objective is to minimize a combination of typical metrics (i.e., job completion time) and the overall carbon footprint (on both a per-job and a global, cluster basis).

Although future values of this carbon signal are unknown to the scheduler, in the rest of the paper, we follow prior work~\cite{Lechowicz:23, Bostandoost:24} and assume that it is bounded by constants $L$ and $U$ that are known to the scheduler, where $L \leq c(t) \leq U$.  In practice, the values of $L$ and $U$ can capture e.g., short-term forecasts of the best and worst carbon conditions on a given electric grid over the next 24 or 48 hours.




\vspace{-0.5em}
\subsection{Prior work and motivation}\label{sec:motiv}

Scheduling directed acyclic graphs (DAGs), or more broadly, precedence-constrained tasks, has been extensively studied.\\ 
Classic results establish the difficulty of this problem: even in its simplest forms, DAG scheduling is NP-hard~\cite{Lenstra:78}. 
To address this, prior work has developed heuristic methods and approximation algorithms~\cite{Su:24:Tompecs, Chudak:99, Lassota:23, Li:17, Davies:20, Davies:21, Maiti:20, Su:23}, ranging from the well-known list scheduling algorithm~\cite{Graham:66}, priority-based algorithms~\cite{Sels:12:Priority}, to more complex approaches such as genetic programming~\cite{Cheng:96:Genetic, Pezzella:08:Genetic, Davis:14:Genetic}. These methods often rely on simplifying assumptions, such as fixed task durations or centralized knowledge of the task graph.  %


In recent years, DAG scheduling has become %
a key problem in \textit{data processing frameworks} such as Apache Airflow, Beam, and Spark, which use DAGs to represent workflows. 
In Spark, each node of a job's DAG corresponds to a \textit{stage}, which encapsulates operations (\textit{tasks}) that can be executed in parallel over partitions of input data. Inter-stage dependencies impose precedence constraints: a stage can only begin once all ``parent'' stages have completed. 
Frameworks such as Spark typically implement simple scheduling strategies such as first-in, first-out (FIFO) and fair-share scheduling~\cite{SparkScheduling} -- these are explainable and efficient in terms of overhead, but suboptimal in terms of job completion time.

Recent works that revisit scheduling %
for data processing have explored %
learning-based techniques, such as reinforcement learning (RL) methods that dynamically learn scheduling policies~\cite{Hongzi:2019:Decima, Wu:18, Li:23, Grinsztajn:20, Zhou:22, Islam:21}.  Although these methods outperform default policies and hand-tuned heuristics in terms of job completion time, theoretical results for these techniques have proven difficult to obtain.



Carbon awareness adds a new dimension to the DAG scheduling problem -- an online scheduler must consider the time-varying carbon intensity while choosing to assign resources to specific nodes in the job DAG(s), with an overarching goal of reducing carbon footprint, combined with traditional metrics such as job completion time -- see \autoref{fig:motivation} for an illustration of this desired behavior for \DANISH, FIFO, and optimal schedules.
As discussed above, the state-of-the-art for carbon-agnostic DAG scheduling falls into two categories: theoretical models that focus on provably near-optimal schedules under idealized assumptions, and heuristic or learning-based methods that do not provide theoretical bounds but perform well in practice.
In adding carbon-awareness to the problem, we consider a \textit{middle ground} that balances between design goals of simplicity, interpretability, configurability, and performance. 
In particular, we seek a carbon-aware scheduler that is tractable for theoretical insight, offering provable bounds on, e.g., the trade-off between carbon and job completion time while not sacrificing the efficiency gains that come from, e.g., learning DAG structure. 













\section{The Basic Framework: \texttt{McSplit}}
\label{sec:sota}

% We first build necessary background of a basic framework called \texttt{McSplit}~\cite{mccreesh2017partitioning} for finding the maximum common subgraph. 
{\cheng \noindent\textbf{Overview.} \texttt{McSplit},  a \emph{backtracking} (aka \textit{branch-and-bound}) algorithm, \eat{also known as \emph{branch-and-bound} algorithm,} has been widely adopted for the MCS problem in recent years and has achieved the state-of-the-art performance in practice~\cite{bai2021glsearch,liu2020learning,zhoustrengthened,liu2023hybrid}.}
%
The \laks{key idea} of \texttt{McSplit} is to recursively expand a partial solution $S$ (which is {\cheng the largest} common subgraph {\cheng seen so far}) via a process of \emph{branching}. Specifically, the branching process partitions the current problem instance of finding the maximum common subgraph into several subproblem instances. Each problem instance {\cheng corresponds} to a \emph{branch} {\cheng and} is denoted by $(S,C)$. Here, $S$ is a \emph{partial solution} {\Yui (i.e., set of vertex pairs)} and $C$ is the \emph{candidate set} consisting of \emph{candidate pairs} (i.e., $\langle u, v\rangle$) used to expand the partial solution $S$. Solving the instance or branch $(S,C)$ means finding the largest common subgraph $S^*$ in the branch; a common subgraph is said to be in a branch $(S,C)$ if and only if \emph{it contains $S$ and is within the set $S\cup C$, {\Yui i.e., $S\subseteq S^*\subseteq S\cup C$}}.
%{\cheng which contains} $S$ and {\cheng is} within the set $S\cup C$.
{\Yui Given two vertex subsets $V_q\subseteq V_Q$ and $V_g\subseteq V_G$, we consider all pairs of   vertices from $V_q$ and $V_g$, i.e.,
$V_q\times V_g = \{\langle u,v \rangle\mid u\in V_q, v\in V_g\}$.
%
%Let $\langle V_Q\times V_G \rangle$ denote the set of all possible vertex pairs (i.e., $\langle V_Q\times V_G \rangle=\{\langle u,v \rangle\mid u\in V_Q, v\in V_G\}$). 
%
Note that solving the initial branch $(\emptyset,V_Q\times V_G)$ finds the largest common subgraph of $Q$ and $G$.}
%, where $\langle V_Q\times V_G \rangle$ denotes the set of all possible vertex pairs (i.e., $\langle V_Q\times V_G \rangle=\{\langle u,v \rangle\mid u\in V_Q, v\in V_G\}$).

To solve a branch $(S,C)$, the branching process selects a vertex $u$ {\cheng appearing} in $C$ as a \emph{branching vertex},  and then creates two groups of new sub-branches by either including $u$ into the solution \eat{(this corresponds to the first one)} or discarding $u$ from the candidate set {\Yui and thus also from the solution}. \eat{(this corresponds to the second one).} Specifically, \textbf{in the first group}, each formed branch includes into $S$ one candidate pair containing $u$ and excludes from $C$ all \laks{pairs} containing $u$ (note that a common subgraph has each vertex {\cheng appearing} in at most one pair); consequently, for each candidate pair that contains $u$, i.e., $\langle u,v \rangle$, \laks{we form a new branch corresponding to} $(S\cup\{\langle u,v \rangle\}, C\backslash u\backslash v)$,  
%
%{\kaixin $(S\cup\{\langle u,v \rangle\}, C\setminus (\{\langle u, \cdot \rangle\}\cup \{\langle \cdot, v \rangle\}))$}
%
where $C\backslash u\backslash v$ denotes the set obtained by removing from $C$ all  candidate pairs  containing $u$ or $v$, formally,
\begin{equation}
    C\backslash u\backslash v:= C\backslash ((\{u\}\times V_g) \cup ( V_q\times\{v\})).
\end{equation}
%
\textbf{In the second group}, we form only one branch by excluding from $C$ all candidate pairs containing $u$, {\cheng i.e.,} $(S,C\backslash u)$. Clearly, \eat{solving all the formed branches solves} the solution to  $(S,C)$ \eat{(since its solution} \laks{is the largest one among those found {\cheng from} the branches above.} We illustrate this next. 
%
\laks{
\begin{example}
\label{ex:split} 
Consider \eat{an example of {\cheng the} {\chengB branching} process on} the given pair of  input graphs in  Figure~\ref{fig:Input_graph}.  The splitting process is partially depicted in Figure~\ref{fig:example_branching} (\laks{ignore the ``$D$'' terms in the figure for now)}). For the initial branch $B_0=(\emptyset,\{u_1,u_2,...,u_7\}\times \{v_1,v_2,...,v_7\} )$, McSplit  selects the branching vertex $u_1$, and then creates the first group of branches  $B_i=(\{\langle u_1,v_i \rangle\},\{u_2,...,u_7\}\times (\{v_1,v_2,...,v_7\}\backslash \{v_i\}))$ for $1\leq i\leq 7$, each of which includes one candidate pair $\langle u_1,v_i \rangle$ into the solution, and the second group of a single branch, namely $B_8=(\emptyset,\{u_2,...,u_7\}\times \{v_1,v_2,...,v_7\})$, which excludes $u_1$ from the solution.
\end{example} 
} 


To improve the efficiency, \laks{McSplit}  further applies a \emph{reduction rule} and an \emph{upper-bound-based pruning} rule for a newly formed branch $(S,C)$. Specifically, the reduction rule {\cheng removes} from the candidate set $C$ those candidate pairs $\langle u,v\rangle$ that cannot form a common subgraph with $S$, i.e., $S\cup \{\langle u,v \rangle$\} cannot be a common subgraph, thus narrowing the search space: note that any supergraph of a non-common subgraph cannot be a common subgraph and thus we can remove them safely. The upper-bound-based pruning rule {\cheng computes} an upper bound on the \laks{size of the} largest common subgraph in the branch and {\cheng prunes} the branch if the upper bound is no larger than the \laks{size of the largest common subgraph found so far} {\YuiR (details will be discussed in Section~\ref{subsec:upper-bound})}. 
{\chengB Below, \laks{we formally state the}  reduction rule.}
%
\begin{figure}[]
		\includegraphics[width=0.45\textwidth]{figure/example_branching_v2.pdf}
  \vspace{-0.15in}
	\caption{Illustrating the backtracking process (``+'' means to {\chengB move} vertex pairs from $C$ to $S$ and ``-'' means to remove vertex pairs from $C$)}
 \vspace{-0.2in}
	\label{fig:example_branching}
\end{figure}


{\Yui
\noindent\textbf{Reduction {\chengC rule}.} \emph{Consider a branch $(S,C)$. A candidate pair $\langle u,v \rangle$ in $C$ cannot form any common subgraph with $S$ if there exists a vertex pair $\langle u',v' \rangle$ in $S$ such that $u$ and $v$ are not simultaneously adjacent or non-adjacent to $u'$ and $v'$, respectively.} 

\laks{The soundness of this rule} \eat{can be verified based on} \laks{immediately follows from}  Definition~\ref{def:CIS}. 
%
%
{
%Based on the partial set $S$, the refined candidate set $C$ can be split as several subsets, i.e.,
%\begin{equation}
%    C=X_1\times Y_1 \cup ... \cup X_c \times Y_c,
%\end{equation}
%where $c$ is a positive integer and every vertex pair $\langle u,v \rangle$ in $X_i\times Y_i$ $(1\leq i\leq c)$ 
%
We note that the above reduction rule can be applied in a recursive way and \emph{the refined candidate set $C$ can be split as several subsets, i.e., $C=X_1\times Y_1 \cup ... \cup X_c \times Y_c$ where $c$ is a positive integer and $X_i$ and $X_j$ (resp. $Y_i$ and $Y_j$) are disjoint}. We show this as follows.
%
More precisely, starting from the basis case of the initial branch $B_0$ with $S_0=\emptyset$ and $C_0= V_Q\times V_G $, none of the candidate pairs in $C_0$ can be pruned by the reduction rule since $S_0$ is empty. Then, consider the recursive case of a branch $B=(S,C)$ with $C=X_1\times Y_1 \cup ... \cup X_c \cup Y_c $ where $c$ is a positive integer and $X_i$ and $X_j$ (resp. $Y_i$ and $Y_j$) are disjoint for $1\leq i\leq c$. For one sub-branch $(S\cup\{\langle u,v\rangle\},C\backslash u\backslash v)$ formed in the first group by including $\langle u,v\rangle$ to $S$, the candidate set $C\backslash u \backslash v$ can be refined as
\begin{eqnarray}
    \label{eq:update_candidate_set}
    \bigcup_{i=1}^c  N(u,X_i)\times N(v,Y_i)  \cup  \overline{N}(u,X_i\backslash \{u\})\times \overline{N}(v, Y_i\backslash\{v\}). 
   % \{N(u,X_i)\times N(v,Y_i) \mid 1\leq i \leq c\} \cup \nonumber\\ \{\overline{N}(u,X_i\backslash \{u\})\times \overline{N}(v, Y_i\backslash\{v\}) \mid 1\leq i \leq c\}
\end{eqnarray}
since those vertex pairs in $N(u,X_i)\times \overline{N}(v, Y_i\backslash\{v\})  \cup  \overline{N}(u,X_i\backslash \{u\})\times N(v,Y_i)$ can be pruned by the above reduction. Clearly, $N(u,X_i)$ and $\overline{N}(u,X_i\backslash \{u\})$ (resp. $N(v,Y_i)$ and $\overline{N}(v, Y_i\backslash\{v\}$) are disjoint for $1\leq i\leq c$. For one sub-branch $(S,C\backslash u)$ formed in the second group, none of the candidate pairs in $C\backslash u$ can be pruned by the reduction rule since $S$ remains unchanged. Suppose $u\in X_i$, then the refined candidate set $C\backslash u$ can be represented by $X_1\times Y_1\cup ... \cup (X_i\backslash\{u\})\times Y_i \cup ...\cup X_c\times Y_c$ where any two subsets are disjoint as well.

We remark that all candidate sets $C$ mentioned in the following sections refer to the ones refined by the reduction rule and thus can be represented by $C=X_1\times Y_1\cup ...\cup X_c\times Y_c$, \laks{for some $c$.} Given this, we define
\begin{equation}
    \mathcal{P}(C)=\{X_i\times Y_i \mid 1\leq i \leq c\}.
\end{equation}
%We assume that $X_i$ and $X_j$ (resp. $Y_i$ and $Y_j$) are disjoint, which holds for the base case and   

%Consider an immediate sub-branch of $B_0$ which is formed by including candidate pair $\langle u,v \rangle$ into the partial solution. Those candidate pairs in $ N(u,V_Q)\times \overline{N}(v,V_G)$ and those in $\overline{N}(u,V_Q)\times N(v,V_G)$ can be pruned by the reduction rule. As a result, the refined candidate set $C\backslash u\backslash v$ is $ N(u,V_Q)\times N(v,V_G) \cup  \overline{N}(u,V_Q\backslash\{u\})\times \overline{N}(v,V_G\backslash\{v\}) $, which is \emph{split} as two subsets. Consider an immediate sub-branch of $B_0$ which is formed by excluding $u$, 
}

%
\if 0
We note that the above reduction rule can be applied in a \emph{recursive} way. More precisely, for the initial branch $B_0$ with $S_0=\emptyset$ and $C_0= V_Q\times V_G $, \laks{none of the candidate pairs in $C_0$ can  be pruned} by the reduction rule since $S_0$ is empty. Consider an immediate sub-branch of $B_0$ which is formed by including candidate pair $\langle u,v \rangle$ into the partial solution. Those candidate pairs in $ N(u,V_Q)\times \overline{N}(v,V_G)$ and \laks{those in} $\overline{N}(u,V_Q)\times N(v,V_G)$ can be pruned by the reduction rule. As a result, the refined candidate set is $ N(u,V_Q)\times N(v,V_G) \cup  \overline{N}(u,V_Q\backslash\{u\})\times \overline{N}(v,V_G\backslash\{v\}) $, which is \emph{split} as two subsets. 
%
In general, for a branch $(S,C)$, the refined candidate set $C$ consists of at most $2^{|S|}$ disjoint subsets, i.e., $C=X_1\times X_2\cup\cdots\cup X_c\times Y_c$ where $1\leq c\leq 2^{|S|}$. For a sub-branch of $(S,C)$ which is formed by {\chengB moving} a candidate pair $\langle u,v \rangle$ from $C$ to $S$, i.e., $(S\cup\{\langle u,v \rangle\},C\backslash u \backslash v,D)$, the candidate set $C\backslash u \backslash v$ can be refined as
%i.e., $C=X_1\times Y_1 \cup\cdots\cup X_c\times Y_c$ where $1\leq c\leq 2^{|S|}$. For a sub-branch of $(S,C)$ which is formed by {\chengB moving} a candidate pair $\langle u,v \rangle$ from $C$ to $S$, the refined candidate set is
\begin{eqnarray}
    \label{eq:update_candidate_set}
    \bigcup_{i=1}^c  N(u,X_i)\times N(v,Y_i)  \cup  \overline{N}(u,X_i\backslash \{u\})\times \overline{N}(v, Y_i\backslash\{v\}). 
   % \{N(u,X_i)\times N(v,Y_i) \mid 1\leq i \leq c\} \cup \nonumber\\ \{\overline{N}(u,X_i\backslash \{u\})\times \overline{N}(v, Y_i\backslash\{v\}) \mid 1\leq i \leq c\}
\end{eqnarray}
%Besides, for a subgraph of $(S,C)$ which is formed by excluding vertex $u$ in the second group, i.e., $(S,C\backslash u)$, we have
%\begin{equation}
%    C\backslash u:= 
%\end{equation}
\fi
\begin{example}
    \label{exp:branching}
   Consider branch $B_6$ in the first group in Figure~\ref{fig:example_branching}. We have $X_1=N(u_1,V_Q)=\{u_2,u_3\}$, $X_2=\overline{N}(u_1,V_Q\backslash\{u_1\})=\{u_4,u_5,u_6,u_7\}$, $Y_1=\{v_4,v_5\}$ and $Y_2=\{v_1,v_2,v_3,v_7\}$. Therefore, the candidate set becomes $\{u_2,u_3\}\times \{v_4,v_5\}\cup\{u_4,u_5,u_6,u_7\}\times \{v_1,v_2,v_3,v_7\}$. Then, consider a sub-branch of $B_6$ formed by further including $\langle u_7,v_7 \rangle$. We can deduce the candidate set by splitting $X_1\times Y_1 $ to $\{u_3\}\times \{v_4\} \cup \{u_2\}\times \{v_5\} $ and splitting $ X_2\times Y_2$ to $\{u_5\}\times \{v_1\}\cup  \{u_4,u_6\}\times \{v_2,v_3\}$.
\end{example}
}





\noindent\textbf{\laks{Algorithm Outline.}} We summarize the details of \texttt{McSplit} in Algorithm~\ref{alg:mcsplit}. It maintains the currently found largest common subgraph $S^*$ (Line 4) and terminates the branch by upper-bound-based pruning (Line 5). Besides, it branches by selecting (from $\mathcal{P}(C)$) a vertex $u$ and the corresponding subset $X\times Y$, called \emph{branching subset}, that $u$ belongs to (Line 6, note that all candidate pairs {\cheng containing} $u$ are within $X\times Y$), and creates two groups of branches as discussed before (Lines 8-12). 
%
For the first group, the ordering of formed branches depends on that of the candidate pairs to be included into $S$, which is specified by a policy (Line 9).
%
We note that \texttt{McSplit} \textit{adopts heuristic policies for selecting $X\times Y$, $u$, and $v$}. %Specifically, it {\Yui first selects from $C$ one subset} $\langle X\times Y \rangle$ with the smallest value of $|X|\times |Y|$, {\Yui then selects from $X$ one vertex} $u$ with the smallest vertex ID, {\Yui and finally iteratively selects from $Y$ each vertex $v$ with the smallest vertex ID for forming a sub-branch $(S\cup\{\langle u,v \rangle\}, C\backslash u\backslash v)$} (note that each vertex in $Q$ or $G$ is assigned an unique integral ID, i.e., $\{0,1,...,|V_Q|-1\}$ for $Q$ and $\{0,1,...,|V_G|-1\}$ for $G$). Therefore, $u$ and $v$ {\cheng are} selected based on the given orderings of vertices in $Q$ and $G$, which are fixed during the recursion.

\begin{algorithm}[t]
\small
\caption{An existing framework: \texttt{McSplit}~\cite{mccreesh2017partitioning}}
\label{alg:mcsplit}
\KwIn{Two graphs $Q=(V_Q,E_Q)$ and $G=(V_G,E_G)$}
\KwOut{The maximum common subgraph}
$S^*\leftarrow \emptyset$; \tcp{Global variable}
\texttt{McSplit-Rec}$(\emptyset,V_Q\times V_G)$; \textbf{Return} $S^*$;\\
%\setcounter{AlgoLine}{0}
\SetKwBlock{Enum}{Procedure \texttt{McSplit-Rec}$(S,C)$}{}
%\SetKwBlock{update}{Procedure \texttt{Update}$(S,C)$}{}
\Enum{
    \lIf{$|S|>|S^*|$}{$S^*\leftarrow S$}
    \tcc{Termination}
    \lIf{$C=\emptyset$ or the upper bound is no larger than $|S^*|$}{\textbf{return}}
    \tcc{Branching}
    {\Yui Select a branching vertex $u$ and a branching subset $X\times Y$ from $\mathcal{P}(C)$  based on a policy\;}
    $Y_{temp}\leftarrow Y$\;
    \tcc{First group: branches formed by including $u$}
    \For{$i=1,2,...,|Y|$}{
        Select and {\chengB move} a vertex $v$ from $Y_{temp}$ based on a policy\;
        Create a candidate set $C_i$ based on $\langle u,v\rangle$ and Equation~(\ref{eq:update_candidate_set})\;
        \texttt{McSplit-Rec}($S\cup\{\langle u,v\rangle\},C_i$)\;
    }
    \tcc{Second group: one branch formed by excluding $u$}
    \texttt{McSplit-Rec}($S,C\backslash u$)\;
}
\end{algorithm}

% \smallskip
% \noindent\textbf{Variants of \texttt{McSplit}.} Quite a few studies adopt \texttt{McSplit} for solving the MCS problem, and 

{\cheng Existing algorithms that are based on \texttt{McSplit} differ in the strategies of optimizing} the policies of selecting vertices in  line 6 and line 9 (e.g., via some learning-based techniques) to find the largest  \laks{common}  subgraph as soon as possible during the recursion~\cite{zhoustrengthened,liu2020learning,liu2023hybrid}. 
%
{\Yui However, these algorithms (1) provide no theoretical guarantee on the worst-case time complexity and (2) still suffer from efficiency issues in practice due to  significant redundant computations.}
%
%We note that the learned policies dynamically select the next vertex at line 6 and line 9 according to the running-time contexts.
%
%The rationale is that the earlier the largest common subgraph is found, the more branches will be pruned by the upper-bound-based pruning (Line 5).



\section{Redundancy-Reduced Splitting: \texttt{RRSplit}}
\label{sec:RRSplit}


In this part, we present our backtracking algorithm called \texttt{RRSplit}. 
%
First, we propose a vertex-equivalence-based reduction for pruning those redundant branches that hold all common subgraphs cs-isomorphic to {\chengC one that has been already found} (Section~\ref{subsec:VE-reduction}).
%
Second, we introduce a newly-designed maximality-based reduction for pruning those redundant branches that hold only non-maximal common subgraphs (Section~\ref{subsec:maximality-reduction}). 
%
{\YuiR Third, we develop a new vertex-equivalence-based upper bound on the size of common subgraphs that can be found in a branch for further pruning those branches that hold only small common subgraphs (Section~\ref{subsec:upper-bound}).}
%
\laks{Finally, we summarize the \texttt{RRSplit} algorithm, which is based on the above \eat{carefully-designed} reductions, and analyze its worst-case time complexity (Section~\ref{subsec:summary}). In particular, we show \texttt{RRSplit} has a worst-case time complexity of $O^*((|V_G|+1)^{|V_Q|})$, matching the best-known worst-case time complexity of the state of the art. We will later show (Section~\ref{sec:exp}) that unlike the state of the art,  \texttt{RRSplit} is very efficient in practice.} 



\subsection{Vertex-Equivalence-based Reduction}
\label{subsec:VE-reduction}
We first introduce the concept of \emph{common subgraph isomorphism (cs-isomorphism)}.
\begin{definition}[Common subgraph isomorphism] \laks{Consider two common subgraphs $\langle q,g,\phi \rangle$ and $\langle q',g',\phi' \rangle$ of graphs $G$ and $Q$. They are} 
    \eat{$\langle q,g,\phi \rangle$ is } said to be common subgraph isomorphic (cs-isomorphic) \eat{to $\langle q',g',\phi' \rangle$} if and only if $q$ is {\chengC isomorphic} to $q'$ (or equiv., $g$ is {\chengC isomorphic} to $g'$).
\end{definition}

\laks{All cs-isomorphic common subgraphs evidently  share the same structural information, and exploring all of them is clearly redundant.}  
% so is redundant for the purpose of 
\eat{{\chengC thus it would introduce redundancy if we explore both of them for}
finding the largest common subgraph.} \laks{We reduce this redundancy as follows.}  
For a common subgraph $\langle q,g,\phi \rangle$ to be found in a branch, we can safely {\chengC ignore the common subgraph $\langle q,g,\phi \rangle$}, {\revision if there exists another one $\langle q',g',\phi' \rangle$ that satisfies the following two conditions:
\begin{itemize}
    \item \textbf{Condition 1}: $\langle q',g',\phi' \rangle$ is cs-isomorphic to $\langle q,g,\phi \rangle$;
    \item  \textbf{Condition 2}: $\langle q',g',\phi' \rangle$ has been found before.
\end{itemize}}
%if there exists another one that is cs-isomorphic to $\langle q,g,\phi \rangle$ {\chengC (Condition 1)} and has been found before {\chengC (Condition 2)},  

\laks{To facilitate the reduction, we make use of Condition 1 and Condition 2, for which we will leverage the \emph{vertex equivalence} {\chengC property} and an \emph{auxiliary data structure} {\chengC respectively}.} 

\smallskip
\noindent\textbf{Vertex equivalence}. 
% We start with an important graph property, namely 
{\chengC The \emph{structural equivalence} property}
% , which 
has been widely used to speed up subgraph matching tasks~\cite{nguyen2019applications,yang2023structural,kim2021versatile}. Conceptually, two vertices are  \emph{structurally equivalent}  if and only if they have the \emph{same} set of neighbours. 
% Formally, we have: 
{\chengC Formally, }

\begin{definition}[Structural equivalence~\cite{nguyen2019applications}]
    %\label{def:structural_eqv}
    Two vertices $u$ and $v$ in $Q$ are structurally equivalent, denoted  $u\sim v$, if and only if  
    \begin{equation}
        \forall u'\in V_Q, (u,u')\in E_Q \Leftrightarrow (v,u')\in E_Q. 
    \end{equation}
\end{definition}

Clearly, structural equivalence is an equivalence relation. Therefore, we can partition the vertices of graph $Q$ into equivalence classes, with the equivalence class of vertex $u\in V_Q$ defined as 
\begin{equation}
    \Psi(u)  := \{u'\in V_Q \mid u'\sim u\},
\end{equation}
where $u\in V_Q$ is a representative of class $\Psi(u)$.
%
We remark that this process can be done in $O(|V_Q|\delta_Q d_Q)$ {\chengB time} where $\delta_Q$ and $d_Q$ are the degeneracy and the maximum degree of the graph $Q$, respectively~\cite{nguyen2019applications,yang2023structural,kim2021versatile}. 


\begin{figure}[]
		\includegraphics[width=0.45\textwidth]{figure/example_concepts.pdf}
	\caption{Illustrating cs-isomorphism and vertex equivalence (vertices, denoted by colored bullet circles, induce subgraphs $q$, $q_1$, $q_2$ and $g$; vertices in $\{u_1,u_2\}$ and $\{u_4,u_6,u_7\}$ are structurally equivalent, respectively; $\langle q,g,\phi \rangle$ is cs-isomorphic to $\langle q_1,g,\phi_1 \rangle$ (Case 1, say, exchange the mapped vertices of $u_1$ and $u_2$) and $\langle q_2,g,\phi_2 \rangle$  (Case 2, say, replace $u_4$ with $u_7$) where vertices with the same color indicate the bijection)}
 \label{fig:example_self_iso}
\end{figure}

Based on vertex equivalence, we can \laks{identify} several common subgraphs \laks{that are} cs-isomorphic to a given one $\langle q,g,\phi \rangle$ by swapping one vertex in $V_q$ with its structurally equivalent counterpart, which falls into  two cases.
%
{\YuiR In specific, consider a vertex $u$ in $V_q$ and one of its structurally equivalent counterparts  $u_{equ}$ in $\Psi(u)$. {\cheng We can obtain a cs-isomorphic common subgraph in two cases.} If $u_{equ}$ is also in $V_q$, {\cheng we can exchange} the mapped vertices of $u_{equ}$ and $u$, i.e., {\cheng we replace} $\langle u,\phi(u) \rangle$ and $\langle u_{equ},\phi(u_{equ}) \rangle$ with $\langle u,\phi(u_{equ}) \rangle$ and $\langle u_{equ},\phi(u) \rangle$; Otherwise, we replace $u$ with $u_{equ}$, i.e., replacing $\langle u,\phi(u)\rangle$ with $\langle u_{equ},\phi(u) \rangle$.}
%
Formally, we have the following lemma, {\cheng which can be easily verified} (see the examples in Figure~\ref{fig:example_self_iso} for a visual illustration of the lemma).
\begin{lemma}
    \label{lemma:cs}
    Let $S=\langle q,g\eat{p},\phi \rangle$ be a common subgraph of given graphs $Q$ and $G$, $u$ be a vertex in $V_q$ and $u'$ be a vertex in $\Psi(u)$. Then {\kaixin one of the following cases holds}. 
    %
    \begin{itemize}[leftmargin=*]
        \item[]\textbf{Case 1: $u'\in V_q$}. $S'=S\backslash \{\langle u,\phi(u) \rangle,\langle u',\phi(u') \rangle \}\cup \{\langle u,\phi(u') \rangle,\langle u',\\\phi(u) \rangle\}$ is a common subgraph cs-isomorphic to $S$. \;\; {\kaixin\eat{OR}} 
        \item[]\textbf{Case 2: $u'\notin V_q$}. $S'=S\backslash \{\langle u,\phi(u) \rangle\}\cup \{\langle u',\phi(u) \rangle\}$ is a common subgraph cs-isomorphic to $S$.
    \end{itemize}
\end{lemma}

\smallskip
\noindent\textbf{Auxiliary data structure}. 
% Though a few cs-isomorphic common subgraphs can be constructed based on Lemma~\ref{lemma:cs} for verifying Condition 1, none of them may have been found before and thus Condition 2 fails to satisfy. 
{\chengC To facilitate the verification of Condition 2, i.e., whether a common subgraph that is cs-isomorphic to a current one has been found before}, we introduce a new data structure, namely exclusion set (denoted by $D$).
{\chengC $D$} is recursively maintained for each branch, and thus each branch is denoted by $(S,C,D)$. Specifically, $D$ is a set of vertex pairs that have been considered for expanding the partial solution and must not be included in any common subgraphs within the branch. Formally, the exclusion set is maintained as follows (\laks{illustrated in Figure~\ref{fig:example_branching} -- see the ``$D$'' terms now!}). 
\begin{itemize}[leftmargin=*]
    \item \textbf{Initialization}. The exclusion set is initialized to {\kaixin be} empty at the initial branch, i.e., $(\emptyset, V_Q\times V_G, \emptyset)$.
    \item \textbf{Recursive update}. Consider {\chengC the} branching at a branch $(S,C,D)$. For the first group where the $i^{th}$ sub-branch $(S_i,C_i,D_i)$ is formed by including $\langle u,v_i \rangle$ into $S$, we update the exclusion set to $D_i=D\cup \{\langle u,v_1 \rangle,\langle u,v_2 \rangle,\cdots ,\langle u,v_{i-1} \rangle\}$. For the second group where one sub-branch $(S',C',D')$ is formed, we set $D'=D$.
\end{itemize}
%We note that, for a vertex pair $\langle u,v \rangle$ in exclusion set $D$, there exists some common subgraphs containing $\langle u,v \rangle$ that has been found before. This will help us to verify Condition 2.
Consider a branch $(S,C,D)$ and a vertex pair $\langle u',v' \rangle$ in the exclusion set $D$, as shown in Figure~\ref{fig:exclusion_set}. There exists an {\YuiR ancestor}\eat{ascendant branch} of $(S,C,D)$, denoted by $(S_{anc},C_{anc},D_{anc})$, where $u'$ is selected as the branching vertex. Clearly, $\langle u',v' \rangle$ is not in $D_{anc}$ and will be \laks{added} to $D_{anc}$ after $B'_{anc}$ {\chengC is formed}, \laks{i.e., more precisely $D'_{anc} = D_{anc}\cup\{(u',v')\}$}. Therefore, all common subgraphs within the sub-branch $B'_{anc}$, which must contain $\langle u',v' \rangle$, have been found before $(S,C,D)$. This will help us  verify Condition 2.
\begin{figure}[]
		\includegraphics[width=0.3\textwidth]{figure/Exclusion_set_v3.pdf}
  \vspace{-0.15in}
	\caption{Illustrating the exclusion set $D$ {\YuiR ($\langle u',v'\rangle$ is a vertex pair in $D$; $B_{anc}$ is an ancestor of $B$, where $u'$ is selected as the branching vertex)}}
 \vspace{-0.2in}
	\label{fig:exclusion_set}
\end{figure}


Based on  vertex equivalence and exclusion set, we are now ready to develop the reductions. Consider the branching process at a branch $(S=\langle q,g,\phi\rangle,C,D)$ where $X\times Y$ in $\mathcal{P}(C)$ and vertex $u$ in $X$ are selected as the branching subset and branching vertex, respectively. 
% In general, the reductions have two cases.
% {\chengC We consider }

\smallskip
\noindent\textbf{Reduction at the first group}. Consider a sub-branch formed at the first group by including one vertex pair $\langle u,v\rangle$ where $v\in Y$. We note that {\chengC \emph{each}} common subgraph $S_{sub}$ to be found in this sub-branch must include $\langle u,v \rangle$. 
%
We observe that if \emph{there exists a vertex pair $\langle u_{equ},v \rangle$ in $D$ such that $u_{equ}$ is structurally equivalent to $u$, i.e., $u_{equ}\in\Psi(u)$}, Conditions 1 \& 2 hold for $S_{sub}$ and thus the branch can be pruned. Below, we elaborate on the details. 
%

We first show that Condition 1 holds:  Clearly, $S$ contains a vertex pair $\langle u_{equ},\phi(u_{equ}) \rangle$ since otherwise $D$ will not include $\langle u_{equ},v \rangle$ according to the maintenance of $D$.
%
Therefore, we can construct the following common subgraph {\chengC $S_{iso}$, which is} cs-isomorphic to $S_{sub}$ based on Case 1 of Lemma~\ref{lemma:cs} {\chengC (essentially, we exchange the mapped vertices of $u_{equ}$ and $u$)}.
\begin{equation}    
\label{eq:iso1}
S_{iso}\!=\!S_{sub}\backslash\{{\chengC \langle u,v\rangle,\langle u_{equ},\phi(u_{equ})\rangle}\} \!\cup\! \{\langle u_{equ},v\rangle,\langle u,\phi(u_{equ}) \rangle\}
\end{equation}
{\chengC Clearly, $S_{iso}$ is cs-isomorphic to $S_{sub}$ given that $u$ and $u_{equ}$ are structurally equivalent.}

We next show that $S_{iso}$ has been found before and thus Condition 2 holds: In specific, consider an ancestor \eat{ascendant branch} of $(S,C,D)$, denoted by $(S_{anc},C_{anc},D_{anc})$, where $u_{equ}$ is selected as the branching vertex, as visually illustrated in Figure~\ref{fig:exclusion_set}. {\chengC Since} $S_{iso}$ contains $\langle u_{equ},v\rangle$, {\chengC we check} whether it has been found in the sub-branch of $(S_{anc},C_{anc},D_{anc})$, which is formed by including $\langle u_{equ},v\rangle$. The answer is interestingly positive. The rationale behind is that (1) $S_{iso}\backslash\{\langle u,\phi(u_{equ})\rangle\}$ is a subset of $S_{anc}\cup C_{anc}$ (this can be easily verified by the facts that $S_{sub} \subseteq S_{anc}\cup C_{anc}$ and $\langle u_{equ},v \rangle\in C_{anc}$) and (2) $\langle u,\phi(u_{equ}) \rangle$ constructed by ours is in $C_{anc}$ (this holds due to the vertex equivalence between $u$ and $u_{equ}$
% , which will be discussed later in Lemma~\ref{lemma:reduction_for_VE1}
). Therefore, $S_{iso}$ is a common subgraph in $(S_{anc},C_{anc},D_{anc})$, i.e., $S_{anc}\subseteq S_{iso} \subseteq S_{anc}\cup C_{anc}$, and has been found before $(S,C,D)$ since it contains $\langle u_{equ},v\rangle$. 

In summary, we have the following lemma and reduction rule.
\begin{lemma}
    \label{lemma:reduction_for_VE1}
    Let $(S,C,D)$ be a branch. Common subgraph $S_{iso}$ {\chengC defined in} Equation~(\ref{eq:iso1}) has been found before the formation of $(S,C,D)$.
\end{lemma}
\begin{proof}
    %Omitted for lack of space. See the anonymous technical report~\cite{TR} for details. 
    {\revision
    We note that the recursive branching process forms a recursion tree where each tree node corresponds to a branch. Consider the path from the initial branch $(\emptyset, V_Q\times V_G,\emptyset)$ to $(S,C,D)$, there exists an ascendant branch of $(S,C,D)$, denoted by $B_{asc}=(S_{asc},C_{asc},D_{asc})$, where $u_{equ}$ is selected as the branching vertex, since $\langle u_{equ},\phi(u_{equ}) \rangle$ is in $S$.
    %
    We can see that there exists one sub-branch $B'_{asc}=(S'_{asc},C'_{asc},D'_{asc})$ of $B_{asc}$ formed by including $\langle u_{equ},v \rangle$, and all common subgraphs within $B'_{asc}$ have been found before the formation of $(S,C,D)$, since $\langle u_{equ},v \rangle$ is in $D$. We then show that common subgraph $S_{iso}$ can be found within $B_{asc}'$, i.e., $S'_{asc}\subseteq S_{iso}\subseteq S'_{asc}\cup C'_{asc}$. 
    %
    \underline{First}, we have $S'_{asc}\subseteq S_{iso}$ since (1) $S'_{asc}=S_{asc}\cup\{\langle u_{equ},v \rangle\}$, (2) $S_{sub}$ is a common subgraph in $B_{asc}$ and thus $S_{asc}\subseteq S_{sub}$, (3) $S_{asc}$ does not include $\langle u_{equ},\phi(u_{equ})\rangle$ or $\langle u,v\rangle$ since they are in $C_{asc}$ and will be included to the partial solution at $B_{asc}'$ and $(S\cup \{\langle u,v \rangle\}, C\backslash u\backslash v)$, and thus (4) by combining all the above, we have  $S'_{asc}=S_{asc}\cup\{\langle u_{equ},v \rangle\}\subseteq S_{sub}\cup\{\langle u_{equ},v \rangle\} \backslash \{\langle u_{equ},\phi(u_{equ})\rangle,\langle u,v\rangle\}\subseteq S_{iso}$. 
    %
    \underline{Second}, we have $S_{iso}\subseteq S'_{asc}\cup C'_{asc}$ based on the following two facts. 

    \begin{itemize}
        \item \textbf{Fact 1.} $S_{sub}\backslash\{\langle u_{equ},\phi(u_{equ})\rangle,\langle u,v\rangle\}\subseteq S'_{asc}\cup C'_{asc}$.
        \item \textbf{Fact 2.} $\langle u_{equ},v \rangle\in S'_{asc}$ and $\langle u,\phi(u_{equ}) \rangle\in C'_{asc}$.
    \end{itemize}
    
    The correctness of the above facts can be verified accordingly, for which we put the details in the 
    \ifx \CR\undefined
Appendix. 
\else
technical report~\cite{TR}. 
\fi
    }
\end{proof}

%\medskip
\noindent\fbox{%
    \parbox{0.47\textwidth}{%
       \textbf{Vertex-Equivalence-based reduction at the first group}. Let $B=(S,C,D)$ be a branch. For a sub-branch of $B$ formed by including a candidate pair $\langle u,v \rangle$ in the first group, it can be pruned if there exists a vertex pair $\langle u',v \rangle$ in $D$ such that $u'\in \Psi(u)$. 
    }%
}
%\medskip

\begin{example}
        Consider again the branching process at branch $B_6=(S_6,C_6,D_6)$ in Figure~\ref{fig:example_branching}. \laks{Suppose $u_2$ is selected as the branching vertex}. Then, we can see that $u_1$ is in $\Psi(u_2)$ and $D_{6}= \{u_1\} \times \{v_1,v_2,v_3,v_4,v_5\}$. {\YuiRR Recall that $C_6=\{u_2,u_3\}\times \{v_4,v_5\}\cup\{u_4,u_5,u_6,u_7\}\times\{v_1,v_2,v_3,v_7\}$. Thus, $B_6$ has two sub-branches which are formed by including $\langle u_2,v_4 \rangle$ or $\langle u_2,v_5 \rangle$, and they can be pruned based on the above reduction. }
\end{example}


\smallskip
\noindent\textbf{Reduction at the second group}. {\YuiR Recall that vertex $u$ is selected as the branching vertex.} Consider the sub-branch formed  {\YuiR by excluding $u$} in the second group. We note that {\chengC each} common subgraph $S_{sub}=\langle q_{sub},g_{sub},\phi_{sub} \rangle$ to be found in this sub-branch must exclude vertex $u$. We observe that if \emph{$S_{sub}$ contains a vertex $u_{equ}$, {\chengC which} is in $C\backslash u$ and is structurally equivalent to $u$}, Condition 1\&2 holds for $S_{sub}$. 

In specific, Condition 1 holds {\chengC since we can construct the} following common subgraph {\chengC which is} cs-isomorphic to $S_{sub}$ {\chengC based on} Case 2 of Lemma~\ref{lemma:cs} {\chengC (essentially, we replace $u_{equ}$ with $u$)}.
\begin{equation}
    \label{eq:iso2}
    S_{iso}= S_{sub}\backslash\{\langle u_{equ}, \phi_{sub}(u_{equ})\rangle\}\cup\{ \langle u,\phi_{sub}(u_{equ}) \rangle\}
\end{equation}

Besides, we note that $S_{iso}$ contains $\langle u,\phi_{sub}(u_{equ}) \rangle$ and common subgraphs found in the first group must include $u$. We thus {\chengC check} whether $S_{iso}$ has been found in one sub-branch formed in the first group. The answer is also positive. The rationale behind is that (1) $\langle u,\phi_{sub}(u_{equ}) \rangle$ constructed by ours exists in $C$ (this holds due to the vertex equivalence between $u$ and $u_{equ}$
% , which will be discussed in Lemma~\ref{lemma:reduction_for_VE2}
) and (2) thus there exists a sub-branch formed by including $\langle u,\phi_{sub}(u_{equ}) \rangle$ in the first group, where $S_{iso}$ has been found since it includes $\langle u,\phi_{sub}(u_{equ}) \rangle$. 

In summary, we have the following lemma and reduction rule.

\begin{lemma}
\label{lemma:reduction_for_VE2}
Let $(S,C,D)$ be a branch where $u$ is selected as the branching vertex. Common subgraph $S_{iso}$ {\chengC defined in} Equation~(\ref{eq:iso2}) has been found before the formation of $(S,C\backslash u,D)$ at the second group.   
\end{lemma}
\begin{proof}
{\revision
    \underline{First}, we note that $\langle u_{equ},\phi_{iso}(u_{equ}) \rangle$ is in $C\backslash u$ and also in $C$ since otherwise $S_{sub}$ cannot include $\langle u_{equ},\phi_{iso}(u_{equ}) \rangle$. This is because (1) $S\subseteq S_{sub}\subseteq S\cup C\backslash u$ since $S_{sub}$ is a common subgraph in the sub-branch $(S,C\backslash u,D)$ and (2) $S$ does not include $\langle u_{equ},\phi_{iso}(u_{equ}) \rangle$ since $u_{equ}$ appears in $C\backslash u$.
    %
    \underline{Second}, we note that $\phi_{iso}(u_{equ})$ is in $Y$. Recall that $X\times Y$ is the branching set at $(S,C,D)$. This is because (1) $u_{equ}$ is in the same subset $X$ as $u$ since $u_{equ}$ and $u$ are structurally equivalent and thus have the same set of neighbours and non-neighbours in $q$, and (2) $\langle u_{equ},\phi_{iso}(u_{equ}) \rangle$ is in $C$ as discussed before.
    %
    \underline{Third}, we can derive that there exists a sub-branch $(S\cup \{\langle u,\phi_{iso}(u_{equ}) \rangle\},C\backslash u\backslash \phi_{iso}(u_{equ}),D')$, which is formed at branch $(S,C,D)$  by including $\langle u,\phi_{iso}(u_{equ}) \rangle$ before the formation of $(S,C\backslash u,D)$, since $\phi_{iso}(u_{equ})\in Y$.
    %
    \underline{Forth}, we show that $S_{iso}$ is in $(S\cup \{\langle u,\phi_{iso}(u_{equ}) \rangle\},C\backslash u\backslash \phi_{iso}(u_{equ}),D')$, formally, $S\cup \{\langle u,\phi_{iso}(u_{equ}) \rangle\} \subseteq S_{iso}\subseteq S\cup \{\langle u,\phi_{iso}(u_{equ}) \rangle\}\cup (C\backslash u\backslash \phi_{iso}(u_{equ}))$.
    We have $S\subseteq S_{sub}\subseteq S\cup (C\backslash u)$ since $S_{sub}$ is a common subgraph in $(S,C\backslash u,D)$. Let $S'=S\cup \{\langle u,\phi_{iso}(u_{equ}) \rangle\}$, it can be proved as below.
    \begin{eqnarray}
        && S\subseteq S_{sub}\subseteq S\cup (C\backslash u)\\
        && \Rightarrow S\subseteq S_{sub}\backslash\{\langle u_{equ}, \phi_{iso}(u_{equ})\rangle\}\subseteq S\cup (C\backslash u\backslash \phi_{iso}(u_{equ})) \label{eq:lemma_eq_1}\\
        && \Rightarrow S' \subseteq S_{iso}\subseteq S'\cup (C\backslash u\backslash \phi_{iso}(u_{equ})) \label{eq:lemma_eq_2}
    \end{eqnarray}
    Note that Equation~(\ref{eq:lemma_eq_1}) holds since $\langle u_{equ},\phi_{iso}(u_{equ}) \rangle$ is in $S$; Equation~(\ref{eq:lemma_eq_2}) is derived by including the vertex pair $\langle u, \phi_{iso}(u_{equ})\rangle$.}
\end{proof}

\noindent\fbox{%
    \parbox{0.47\textwidth}{%
       \textbf{Vertex-Equivalence-based reduction at the second group}. Let $B=(S,C,D)$ be a branch and $(S,C\backslash u,D)$ be  the sub-branch formed in the second group by excluding all candidate pairs that consist of $u$. For a vertex $u'$ appearing in $C\backslash u$, if $u'$ is structurally equivalent to $u$, i.e., $u'\in \Psi(u)$, all candidate pairs that consist of $u'$ can be pruned from  $C\backslash u$. 
    }%
}
%\medskip

\begin{example}
    Consider the branching process at $B_0$ where $u_1$ is the branching vertex in Figure~\ref{fig:example_branching}. For sub-branch $B_8$ (which is the sub-branch at the second group), we can see that $\Psi(u_1)=\{u_1,u_2\}$ and thus the candidate set of $B_8$ can be reduced to $ (V_Q\backslash\{u_1,u_2\})\times V_G$, {\YuiR i.e., to $\{u_3,u_4,u_5,u_6,u_7\}\times \{v_1,v_2,\cdots,v_7\}$}.
\end{example}


\subsection{Maximality-based Reduction}
\label{subsec:maximality-reduction}

We introduce the redundancies induced by \emph{non-maximality}. Clearly, a maximum common subgraph must be a maximal common subgraph. Therefore, exploring those branches that hold non-maximal common subgraphs only will incur redundant computations. 
%
Consider a current branch $B=(S,C,D)$. {\chengC Note that there might exist multiple common subgraphs with the largest number of vertices in the branch.}
%
We observe that \emph{there exists one largest common subgraph in $B$ that must contain one specific candidate vertex pair $\langle u,v \rangle$}. 
%
{\YuiR Given this, we can remove this candidate vertex pair $\langle u,v\rangle$ from $C$ to $S$, thereby producing one immediate sub-branch i.e., $(S\cup\{\langle u,v \rangle\},C\backslash u\backslash v)$. Clearly, solving the resulting sub-branch is enough to find the largest common subgraph (since it holds all those common subgraphs that contain $\langle u,v\rangle$ in $B$). As a result,  we can safely prune all other sub-branches for which the partial set and the candidate set do not include the candidate pair $\langle u,v\rangle$.}
%
%As a result, we can safely prune all other sub-branches \laks{for which} all common subgraphs \laks{in the sub-branch} \emph{exclude} these candidate vertex pairs. 
%
Below, we elaborate on the details.


To be specific, we observe that there exists one largest common subgraph, denoted by $S_{opt}$, in $B$ such that $S_{opt}$ must contain a candidate vertex pair $\langle u,v \rangle$ if for any subset $X\times Y$ in $\mathcal{P}(C)$, $u$ and $v$ are simultaneously adjacent or non-adjacent to all other vertices in $X$ and $Y$, respectively, i.e.,
\begin{eqnarray}
    \label{eq:condition}
    \forall X\times Y\in \mathcal{P}(C): {\revision\big(}N(u,X)=X\backslash\{u\} {\revision \text{ and }} N(v,Y)=Y\backslash\{v\}{\revision\big)} \text{ or }\nonumber\\  
        {\revision \big(}N(u,X)=\emptyset {\revision\text{ and }} N(v,Y)=\emptyset{\revision\big)},
\end{eqnarray}
Formally, we have the following lemma.

\begin{lemma}
\label{lemma:maximality}
    Let $B=(S,C,D)$ be a branch {\chengC and $\langle u,v \rangle$ be a candidate vertex pair that satisfies the condition in Equation~(\ref{eq:condition}).} There exists one largest common subgraph $S_{opt}$ in the branch $B$ such that $S_{opt}$ contains 
    {\chengC $\langle u,v \rangle$.}
    % a candidate vertex pair $\langle u,v \rangle$, if $\langle u,v \rangle$ satisfies the condition in Equation~(\ref{eq:condition}).
\end{lemma}

\begin{proof}
 {\revision
 This can be proved by construction. Let $S^*=(q^*,g^*,\phi^*)$ be one largest common subgraph to be found in $B$. Note that if $S^*$ contains the candidate vertex pair $\langle u,v \rangle$, we can finish the proof by constructing $S_{opt}$ as $S^*$. Otherwise, if $\langle u,v \rangle$ is not in $S^*$, we prove the correctness by constructing one largest common subgraph $S_{opt}$ to be found in $B$ that contains candidate vertex pair $\langle u,v \rangle$, i.e., $S\subseteq S_{opt} \subseteq S\cup C$,  $|S_{opt}|=|S^*|$ and $\langle u,v \rangle\in S_{opt}$. In general, there are four different cases, and the details can be found in the 
 \ifx \CR\undefined
Appendix. 
\else
technical report~\cite{TR}. 
\fi
 }
\if 0
    This can be proved by construction. Let $S^*=(q^*,g^*,\phi^*)$ be one largest common subgraph to be found in $B$. Note that if $S^*$ contains the candidate vertex pair $\langle u,v \rangle$, we can finish the proof by constructing $S_{opt}$ as $S^*$. Otherwise, if $\langle u,v \rangle$ is not in $S^*$, we prove the correctness by constructing one largest common subgraph $S_{opt}$ to be found in $B$ that contains candidate vertex pair $\langle u,v \rangle$, i.e., $S\subseteq S_{opt} \subseteq S\cup C$,  $|S_{opt}|=|S^*|$ and $\langle u,v \rangle\in S_{opt}$.
    %
    In general, there are four different cases.

    \smallskip
    \noindent\underline{\textbf{Case 1:}} $u\notin V_{q^*}$ and $v\in V_{g^*}$. In this case, there exists a vertex pair $\langle\phi^{*-1}(v),v \rangle$ in $S^*$ where $\phi^{*-1}$ is the inverse of $\phi^*$. We construct $S_{opt}$ by replacing the vertex pair $\langle\phi^{*-1}(v),v \rangle$ with $\langle u,v \rangle$, i.e.,
    \begin{equation}
        S_{opt}=S^*\backslash\{ \langle\phi^{*-1}(v),v \rangle\} \cup \{\langle u,v \rangle\}.
    \end{equation}
    Clearly, we have $S\subseteq S_{opt}\subseteq S\cup C$ (i.e., $S_{opt}$ is in $B$) since $S^*$ is in $B$ and $\langle u,v\rangle$ is in the candidate set $C$. Besides, we have $|S_{opt}|=|S^*|$ and $\langle u,v \rangle\in S_{opt}$ based on the above construction. Finally, we deduce that $S_{opt}$ is a common subgraph by showing that any two vertex pairs in $S_{opt}$ satisfy Equation~(\ref{eq:isomorphic}), i.e., $g_{opt}$ is isomorphic to $q_{opt}$ under the bijection $\phi_{opt}$. \underline{First}, $S^*\backslash\{\langle \phi^{*-1}(v),v\rangle\}$, as a subset of $S^*$, is a common subgraph and thus has any two vertex pairs inside satisfying Equation~(\ref{eq:isomorphic}) (note that any subset of a common subgraph is still a common subgraph); \underline{Second}, for each pair $\langle u',v' \rangle$ in $S$, $u$ is adjacent to $u'$ if and only if $v$ is adjacent to $v'$ (since $\langle u,v \rangle$ is a candidate pair which can form a common subgraph with $S$); \underline{Third}, for each pair $\langle u',v' \rangle$ in $S_{opt}\backslash S\backslash\{\langle \phi^{*-1}(v),v\rangle\}$, it is clear that $\langle u',v' \rangle$ is in one subset $ X\times Y$ of $\mathcal{P}(C)$ and thus $u$ is adjacent to $u'$ if and only if $v$ is adjacent to $v'$ based on Equation~(\ref{eq:condition}). Therefore, any two vertex pairs in $S_{opt}$ will satisfy the Equation~(\ref{eq:isomorphic}).

    \smallskip
    \noindent\underline{\textbf{Case 2:}} $u\in V_{q^*}$ and $v\notin V_{g^*}$. There exists a vertex pair $\langle u,\phi^*(u) \rangle$ in $S^*$. We construct $S_{opt}$ by replacing $\langle u,\phi^*(u) \rangle$ with $\langle u,v \rangle$, i.e., $S_{opt}=S^*\backslash \{\langle u,\phi^*(u) \rangle\}\cup\{\langle u,v \rangle\}$. Similar to Case 1, we can prove that $S_{opt}$ includes $\langle u,v \rangle$ and is one largest common subgraph to be found in $B$. 

    \smallskip
    \noindent\underline{\textbf{Case 3:}} $u\in V_{q^*}$ and $v\in V_{g^*}$. There exists two distinct vertex pairs $\langle u,\phi^*(u) \rangle$ and $\langle \phi^{*-1}(v),v \rangle$ in $S^*$. We construct $S_{opt}$ by replacing these two vertex pairs with $\langle \phi^{*-1}(v),\phi(u) \rangle$ and $\langle u,v \rangle$, formally,
    \begin{equation}
        S_{opt}\!\!=\!\!S^*\backslash\{\langle u,\phi^*(u) \rangle,\!\langle\phi^{*-1}(v),v \rangle\}\!\cup\!\{\langle \phi^{*-1}(v),\phi^*(u) \rangle,\!\langle u,v \rangle\}.
    \end{equation}
    Clearly, we have $S\subseteq S_{opt}\subseteq S\cup C$ (i.e., $S_{opt}$ is in $B$), $|S_{opt}|=|S^*|$ and $\langle u,v \rangle\in S_{opt}$ based on the above construction. We then deduce that $S_{opt}$ is a common subgraph  by showing that any two vertex pairs in $S_{opt}$ satisfy Equation~(\ref{eq:isomorphic}).
    %
    \underline{First}, $S^*\backslash\{\langle u,\phi^*(u) \rangle,\langle \phi^{*-1}(v),v\rangle\}$, as a subset of $S^*$, is a common subgraph and thus has any two vertex pairs inside satisfying Equation~(\ref{eq:isomorphic});
    %
    \underline{Second}, consider a vertex pair $\langle u',v' \rangle$ in $S^*\backslash\{\langle u,\phi^*(u) \rangle,\langle \phi^{*-1}(v),v\rangle\}$. Similar to Case 1, we can prove that $u$ is adjacent to $u'$ if and only if $v$ is adjacent to $v'$. Besides, we show that $\phi^{*-1}(v)$ is adjacent to $u'$ if and only if $\phi(u)$ is adjacent to $v'$ since (1) $(\phi^{*-1}(v),u')\in E_Q\Leftrightarrow (v,v')\in E_G$ and $(u,u')\in E_Q\Leftrightarrow (\phi^*(u),v')\in E_G$ (since the common subgraph $S^*$ contains $\{\langle u,\phi^*(u) \rangle,\langle \phi^{*-1}(v),v\rangle\}$), (2) $ (v,v')\in E_G \Leftrightarrow (u,u')\in E_Q$ (as we shown above), and thus (3) they can be combined as $(\phi^{*-1}(v),u')\in E_Q\Leftrightarrow (v,v')\in E_G \Leftrightarrow (u,u')\in E_Q \Leftrightarrow (\phi^*(u),v')\in E_G$.
    %\begin{equation}
        %(\phi^{*-1}(v),u')\in E_Q\Leftrightarrow (v,v')\in E_G \Leftrightarrow (u,u')\in E_Q \Leftrightarrow (\phi^*(u),v')\in E_G  \nonumber
    %\end{equation}
    %
    \underline{Third}, we have $(u,\phi^{*-1}(v))\in E_Q\Leftrightarrow (v,\phi^*(u))\in E_G$ since the common subgraph $S^*$ contains $\{\langle u,\phi^*(u) \rangle,\langle \phi^{*-1}(v),v\rangle\}$ and thus $(u,\phi^{*-1}(v))\in E_Q\Leftrightarrow (\phi^*(u),v)\in E_G$ (note that $(\phi^*(u),v)$ refers to the same edge as $(v,\phi^*(u))$ since the graphs $Q$ and $G$ are undirected). Therefore, any two vertex pairs in $R_{opt}$ will satisfy Equation~(\ref{eq:isomorphic}).

    \smallskip
    \noindent\underline{\textbf{Case 4:}} $u\notin V_{q^*}$ and $v\notin V_{g^*}$. We note that this case will not occur since otherwise the contradiction is derived by showing that $S^*\cup \{\langle u,v\rangle\}$ is a larger common subgraph (note that the proof is similar to Case 1 and thus be omitted).
    \fi
\end{proof}

Consider a branch $B=(S,C,D)$ where $ X^*\times Y^*$ in $\mathcal{P}(C)$ and $u^*$ in $X^*$ are selected as the branching subset and the branching vertex, {\chengC respectively}. 
Assume that there exists a vertex $v$ in $Y^*$ such that $\langle u^*,v \rangle$ satisfies the condition in Equation~(\ref{eq:condition}).
Based on the above lemma, there exists one largest common subgraph in the branch $B$ that contains candidate vertex pair $\langle u^*,v \rangle$. Therefore, we only need to form one sub-branch $(S\cup\{\langle u^*,v \rangle\},C\backslash u^*\backslash v,D\cup  \{u^*\}\times (Y^*\backslash\{v\}) )$ since other formed sub-branches will exclude the candidate vertex $\langle u^*,v \rangle$ from the found common subgraphs. We note that the exclusion set of the formed sub-branch can be updated by $D\cup  \{u^*\}\times (Y^*\backslash\{v\}) $ to enhance the pruning power of the proposed reduction at the first group. In summary, we obtain the following reduction.

%\begin{lemma}[Maximality-based reduction]
%    \label{lemma:maximality-reduction}
%    Let $B=(S,C,D)$ be a branch where $\langle X\times Y \rangle$ in $C$ and $u$ in $X$ are selected as the branching subset and the branching vertex. If there exists a candidate vertex pair $\langle u,v \rangle$ in the candidate set such that $\langle u,v \rangle$ satisfies Equation~(\ref{eq:condition}), only one sub-branch $(S\cup\{\langle u,v \rangle\},C\backslash u\backslash v,D\cup \langle \{u\}\times (Y\backslash\{v\}) \rangle)$ needs to be formed at $B$.
%\end{lemma}

\medskip
\noindent\fbox{%
    \parbox{0.47\textwidth}{%
       \textbf{Maximality-based reduction}. Let $B=(S,C,D)$ be a branch where $ X\times Y $ in $\mathcal{P}(C)$ and $u$ in $X$ are selected as the branching subset and the branching vertex. If there exists a candidate vertex pair $\langle u,v \rangle$ in the candidate set such that $\langle u,v \rangle$ satisfies Equation~(\ref{eq:condition}), only one sub-branch $(S\cup\{\langle u,v \rangle\},C\backslash u\backslash v,D\cup \{u\}\times (Y\backslash\{v\}))$ needs to be formed at $B$.
    }%
}


\begin{example}
Consider the branching at branch $B_6=(S_6,C_6,D_6)$ in Figure~\ref{fig:example_branching} where \laks{suppose} $u_2$ is selected as the branching vertex. Recall that $C_6=X_1\times Y_1 \cup X_2\times Y_2 =\{u_2,u_3\}\times \{v_4,v_5\} \cup \{u_4,u_5,u_6,u_7\}\times \{v_1,v_2,v_3,v_7\}$.
%
We note that $\langle u_2,v_5 \rangle$ satisfies Equation~(\ref{eq:condition}) since (1) $N(u_2,X_1)=X_1\backslash\{u_2\}$ and $N(v_5,Y_1)=Y_1\backslash\{v_5\}$ and (2) $N(u_2,X_2)=\emptyset$ and $N(v_5,Y_2)=\emptyset$. Therefore, we only need to explore one sub-branch $(S_6\cup\{\langle u_2,v_5\rangle\},C_6\backslash u_2\backslash v_5,D_6\cup\{\langle u_2,v_4 \rangle\})$, and other two sub-branches formed at $B_6$ can be pruned.
\end{example}




\subsection{Vertex-Equivalence-based {\chengB Upper Bound}}
\label{subsec:upper-bound}
Consider a current branch $(S,C,D)$ and the largest common subgraph $S^*$ seen so far. Clearly, we can terminate the branch $(S,C,D)$, if the upper bound on the size of common subgraphs to be found in the branch $(S,C,D)$ (or simply, the upper bound of $(S,C,D)$) is no larger than the size of $S^*$. The tighter the upper bound, the more branches we can prune. 
% To facilitate this pruning technique, we introduce the upper bound as below. 

\smallskip
\noindent\textbf{Existing upper bound.} Consider a common subgraph $S_{sub}$ to be found in the branch $(S,C,D)$. For a subset $X\times Y$ in $\mathcal{P}(C)$, we can derive
\begin{equation}
    |S_{sub}|\cap X\times Y \leq ub_{X,Y}:= \min\{|X|,|Y|\}
\end{equation}
since otherwise a common subgraph will contain two distinct vertex pairs $\langle u,v \rangle$ and $\langle u',v' \rangle$ such that $u=u'$ or $v=v'$ (which violates the definition of the bijection). Here, $ub_{X,Y}$ is the upper bound of the number of candidate pairs that are within $X\times Y$ and are in a common subgraph to be found in the branch $(S,C,D)$. Furthermore, since all subsets in $\mathcal{P}(C)$ are disjoint, 
% we can derive 
the {\chengC following} existing upper bound of branch $(S,C,D)$, denoted by $ub_{S,C}$~\cite{mccreesh2017partitioning}, 
{\chengC can be derived}.
% as below
\begin{equation}
    |S_{sub}| \leq ub_{S,C} := |S|+\sum_{ X\times Y \in \mathcal{P}(C)} ub_{X,Y}
\end{equation}

\smallskip
\noindent\textbf{Motivation.} We observe that the existing upper bound $ub_{X,Y}$ is not tight since some candidate vertex pairs in $X\times Y$ can be pruned from the candidate set $C$ {\chengC based on} the proposed vertex-equivalence-based reductions. In specific, for a candidate vertex pair $\langle u,v \rangle$, if there exists a vertex pair $\langle u',v \rangle$ in $D$ such that $u'\in \Psi(u)$, any common subgraph to be found within $(S,C,D)$ cannot include $\langle u,v \rangle$ and thus $\langle u,v \rangle$ can be pruned from the candidate set $C$. Note that this can be easily verified based on the proposed reduction at the first group. Below, we introduce our upper bound derived with the aid of the structural equivalence on vertices.

\smallskip
\noindent\textbf{New upper bound}. Consider a subset $X\times Y $ in $\mathcal{P}(C)$. Let $u$ be an arbitrary vertex in $X$. We partition $X$ and $Y$ as follows.
\begin{eqnarray}
 X_L=X\cap \Psi(u), X_R=X\backslash X_L\\
 Y_L=\{v\mid \langle u',v \rangle\in D, u'\in \Psi(u)\}, Y_R=Y\backslash Y_L,   
\end{eqnarray}
where $X_L$ consists of those vertices in $X$ that are structurally equivalent to $u$ and $Y_L$ consists of those vertices $v$ in $Y$ which appear  in a vertex pair $\langle u',v \rangle$ in $D$ where $u'\in \Psi(u)$. We then can partition $X\times Y$ as $X_L\times Y_L$, $X_L\times Y_R$, $ X_R\times Y_L$ and $X_R\times Y_R$. Clearly, all vertex pairs in $X_L\times Y_L$ can be pruned as discussed before. 
%
We note that (1) $S_{sub}$ contains at most $\min\{|X_R|,|Y|\}$ vertex pairs from $X_R\times Y_L$ and $X_R\times Y_R$ since otherwise there exists one vertex in $X_R\cup Y$ that appears in at least two distinct vertex pairs in $S_{sub}$ and thus $S_{sub}$ cannot be a common subgraph; and similarly (2) $S_{sub}$ contains at most $\min\{|X_L|,|Y_R|,\max\{|Y|-|X_R|,0\}\}$ vertex pairs from $X_L\times Y_R$ (note that the additional term $\max\{|Y|-|X_R|,0\}$ is used to ensure that the sum of $\min\{|X_R|,|Y|\}$ and $\min\{|X_L|,|Y_R|,\max\{|Y|-|X_R|,0\}\}$ is no larger than the existing upper bound $ub_{S,C}$). Therefore, $S_{sub}$ contains at most $ub_{X,Y,D}$ vertex pairs from $X\times Y$, where
\begin{equation}
    ub_{X,Y,D}:=\min\{|X_R|,|Y|\}+\min\{|X_L|,|Y_R|,\max\{|Y|-|X_R|,0\}\}.
\end{equation}
Then, we can derive our upper bound of a branch $(S,C,D)$, denoted by $ub_{S,C,D}$, i.e.,
\begin{equation}
    |S_{sub}|\leq ub_{S,C,D}:=|S|+\sum_{X\times Y\in \mathcal{P}(C)} ub_{X,Y,D}.
\end{equation}

In summary, we obtain our new upper bound $ub_{S,C,D}$ as {\chengC above}. It is {\chengC not difficult} to verify that our upper bound is tighter than the existing one, i.e., $ub_{S,C,D}\leq ub_{S,C}$: see Example~\ref{example:upper_bound} for an example where $ub_{S,C,D} < ub_{S,C}$.  
\begin{lemma}[Upper bound]
    \label{lemma:upper_bound}
    Let $(S,C,D)$ be a branch. All common subgraphs to be found in $(S,C,D)$ have the size at most $ub_{S,C,D}$.
\end{lemma}

\begin{example}
\label{example:upper_bound}
    Consider again the branching process at branch $B_6=(S_6,C_6,D_6)$ in Figure~\ref{fig:example_branching}. Recall that $C_6=X_1\times Y_1\cup X_2\times Y_2=\{u_2,u_3\}\times \{v_4,v_5\}\cup \{u_4,u_5,u_6,u_7\}\times \{v_1,v_2,v_3,v_7\}$ and $D_6=\{u_1\}\times\{v_1,v_2,...,v_5\}$. For $X_1\times Y_1$, based on $u_2$, 
    we have $X_{1L}=\{u_2\}$, $X_{1R}=\{u_3\}$, $Y_{1L}=\{v_4,v_5\}$ and $Y_{1R}=\emptyset$. Thus, we have $ub_{X_1,Y_1,D_6}=\min\{1,4\}+\min\{1,0,\max\{1,0\}\}=1$. For $X_2\times Y_2$, based on $u_4$, we have $X_{2L}=\{u_4\}$, $X_{2R}=\{u_5,u_6,u_7\}$, $Y_{2L}=\emptyset$ and $Y_{2R}=\{v_1,v_2,v_3,v_7\}$. Thus, we have $ub_{X_2,Y_2,D_6}=\min\{3,4\}+\min\{1,4,\max\{1,0\}\}=4$. Therefore, we have $ub_{S_6,C_6,D_6}=1+1+4=6$, which is smaller than the existing bound $ub_{S,C}=7$.
\end{example}

{\revision
\noindent\textbf{Remark.} Our new upper bound $ub_{S,C,D}$ varies {\chengE with} different choices of $u$ due to the partition of $X$ and $Y$. We can potentially obtain a tighter upper bound by exploring all possible choices of $u$. However, it {\chengE would} introduce a large amount of time costs, thus degrading the performance of \texttt{RRSplit}. Therefore, as a trade-off,  we randomly select $u$ when computing the upper bound.
}

\subsection{Summary and Analysis}
\label{subsec:summary}
\noindent\textbf{Summary.}
We summarize our algorithm, namely \texttt{RRSplit}, in Algorithm~\ref{alg:rrsplit}, which incorporates the newly proposed vertex-equivalence-based reductions, the maximality-based reduction and the vertex-equivalence-based upper bound. Specifically, \texttt{RRSplit} differs with \texttt{McSplit} in the following aspects. (1) It maintains one additional auxiliary data structure, namely exclusion set $D$, for each formed {\chengB branch}, which is initialized as 
the empty set and recursively updated as discussed. (2)  It {\chengB prunes} a branch $(S,C,D)$ if the newly proposed vertex-equivalence-based upper bound $ub_{S,C,D}$ is no larger than the \laks{largest common subgraph size}   seen so far, i.e., $|S^*|$ (Line 7). We remark that $ub_{S,C,D}$ is tighter than the existing one $ub_{S,C}$, i.e., $ub_{S,C,D}\leq ub_{S,C}$ and thus more branches can be pruned. (3) It creates only one sub-branch and prunes all others if the maximality-based reduction is triggered (Lines 9-11). (4) Based on the vertex-equivalence-based reduction, it prunes those sub-branches at the first group that hold all common subgraphs inside cs-isomorphic to the one found before (Lines 15-16), and refines the formed sub-branch at the second group by removing from the candidate set all those candidate vertex pairs consisting of a vertex in $\Psi(u)$ (Line 19).
%
We remark that our implementation of \texttt{RRSplit} in the experiments adopts the same heuristic policies for selecting branching subset $X\times Y$, branching vertex $u$ (Line 8) and vertex $v$ (Line 14) as \texttt{McSplit} {\chengB does}.
%
Besides, we can easily prove that \texttt{RRSplit} finds the maximum common subgraph based on our discussion above. Finally, we analyze the {space complexity and time complexity} of \texttt{RRSplit} as below. 


\begin{algorithm}{}
\small
\caption{Our proposed algorithm: \texttt{RRSplit}}
\label{alg:rrsplit}
\KwIn{Two graphs $Q=(V_Q,E_Q)$ and $G=(V_G,E_G)$}
\KwOut{The maximum common subgraph}
$S^*\leftarrow \emptyset$; \tcp{Global variable}
%$S\leftarrow\emptyset$, $C\leftarrow\langle V_Q\times V_G \rangle$, $D\leftarrow\emptyset$ \tcp{Global data structure}
\texttt{RRSplit-Rec}$(\emptyset,V_Q\times V_G,\emptyset)$\; \textbf{Return} $S^*$;\\
%\setcounter{AlgoLine}{0}
\SetKwBlock{Enum}{Procedure \texttt{RRSplit-Rec}$(S,C,D)$}{}
%\SetKwBlock{update}{Procedure \texttt{Update}$(S,C)$}{}
\Enum{
    \lIf{$|S|>|S^*|$}{$S^*\leftarrow S$}
    \tcc{Termination (Lemma~\ref{lemma:upper_bound})}
    \lIf{$C=\emptyset$}{\textbf{return}}
    \lIf{ $ub_{S,C,D}\leq |S^*|$}
        {\textbf{return}} 
    \tcc{Branching}
    Select a branching vertex $u$ and a branching subset $X\times Y$ from $\mathcal{P}(C)$  based on a policy\;
    \tcc{Maximality-based reduction}
    \If{there exists a vertex $v$ in $Y$ such that $\langle u,v \rangle$ satisfies Equation~(\ref{eq:condition})}{
        \texttt{RRSplit-Rec}($S\cup\{\langle u,v \rangle\},C\backslash u\backslash v,D\cup \{u\}\times (Y\backslash\{v\})$)\;
        \textbf{return}\;
    }
    \tcc{Branching at the first group}
    $Y_{temp}\leftarrow Y$\;
    \For{$i=1,2,...,|Y|$}{
        Select and remove a vertex $v$ from $Y_{temp}$ based on a policy\;
        \If{there exists a vertex pair $\langle u',v \rangle$ in $D$ such that $u'\in \Psi(u)$}{\textbf{continue;}}
        Refine candidate set $C\backslash u\backslash v$ as $C_i$ based on Equation~(\ref{eq:update_candidate_set})\;
        \texttt{RRSplit-Rec}($S\cup\{\langle u,v\rangle\},C_i,D\cup\{u\}\times (Y\backslash Y_{temp})$);
    }
    \tcc{Branching at the second group}
    \texttt{RRSplit-Rec}($S,C\backslash \Psi(u),D$)\;
}
\end{algorithm}

\smallskip
\noindent\textbf{Space complexity}. We note that \texttt{RRSplit} recursively maintains three global data structures, namely $S$, $C$ and $D$, for each branch, which dominate the space complexity of \texttt{RRSplit}. Let $S^*$ be the largest common subgraph between  graphs $Q$ and $G$. \underline{First}, partial solution $S$ is a set of vertex pairs and its size is bounded by $O(|S^*|)$. \underline{Second}, candidate set $C$ is also a set of vertex pairs and can be partitioned as several subsets, i.e., $C= X_1\times Y_1 \cup  X_2\times Y_2\cup\cdots\cup X_c\times Y_c$ where $c$ is a positive integer, based on Equation~(\ref{eq:update_candidate_set}). We note that subsets in $X_1,X_2,...,X_c$ (resp. $Y_1,Y_2,...,Y_c$) are mutually disjoint and $X_1\cup X_2\cup .... \cup X_c=X$ (resp. $Y_1\cup Y_2\cup .... \cup Y_c=Y$), as discussed in the proof of Lemma~\ref{lemma:reduction_for_VE1}. Therefore, $C$ can be stored {\chengC as} $c$ subsets, each of which $\langle X_i,Y_i\rangle$ ($1\leq i\leq c$) consists of two sets $X_i$ and $Y_i$. Thus, the size of $C$ is bounded by $O(|V_Q|+|V_G|)$. \underline{Third}, $D$ is a set of vertex pairs and consists of at most $|S^*|\cdot |V_G|$ different vertex pairs since for a vertex pair $\langle u,v \rangle$ in $D$, (1) $u$ must {\chengB appear} in $S$ based on our maintenance of $D$ and thus has at most $|S^*|$ different values and (2) $v$ has at most $|V_G|$ different values clearly. In summary, the space complexity of \texttt{RRSplit} is $O(|V_Q|+|S^*|\times|V_G|)$. %We remark that \texttt{McSplit} has the space complexity of $O(|V_Q|+|S^*|\times|V_G|)$, which is the same as that of \texttt{RRSplit}, since \texttt{McSplit} needs to maintain the set $Y_{temp}$ for each branch.

\smallskip
\noindent\textbf{Time complexity of the proposed reductions}. \underline{First}, the reduction at the first group takes $O(|V_Q|+|V_G|)$ {\chengB time} (Lines 15-16). In specific, $D$ is organized as several disjoint subsets, i.e., $D=\{u_1\} \times A_1 \cup  \{u_2\} \times A_2\cup \cdots \cup \{u_d\} \times A_d $ where $d$ is a positive integer. 
%
Thus, it can be conducted in two steps: (1) for each vertex $u_i$ {\chengC appearing} in $D$, it takes $O(1)$ to check whether $u_i\in \Psi(u)$ and (2) if $u_i\in \Psi(u)$, it takes $O(|A_i|)$ to check whether $\langle u_i,v \rangle\in  \{u_i\} \times A_i$.
%
We note that for any two distinct vertices $u_i$ and $u_j$ appearing in $D$ such that $u_i\in \Psi(u_j)$, it is no hard to verify that $A_i\cap A_j=\emptyset$ due to the reduction at the first group (for which we put the details of the proof in the 
\ifx \CR\undefined
Appendix\else technical report~\cite{TR}\fi). 
As a result, we have $\sum_{u_i\in \Psi(u)} (|A_i|)\leq |V_G|$.  
%
\underline{Second}, the reduction at the second group runs in $O(|X|)$ for updating $C\backslash\Psi(u)$ at Line 19, which is bounded by $O(|V_Q|)$. In specific, it can be done by removing from $X$ all vertices in $\Psi(u)$ (note that, given all structurally equivalent classes, determining whether a vertex belongs to $\Psi(u)$ can be done in $O(1)$).
%
\underline{Third}, the maximality reduction runs in $O(\sum_{\langle X',Y' \rangle\in C}|X'|+|Y'|\cdot |Y|)$, which is bounded by $O(|V_Q|+|V_G|^2)$. In specific, for each vertex in $|Y|$, it needs to check the condition in Equation~(\ref{eq:condition}).
%
{\revision \underline{Fourth}}, the new upper bound can be obtained in $O(|V_Q|+|V_G|^2)$. In specific, the time cost is dominated by the computation of $ub_{X',Y',D}$ for each subset $\langle X',Y'\rangle$ in $C$. $ub_{X',Y',D}$ can be obtained in $O(|X'|+\sum_{u_i\in\Psi(u')} |A_i|+|Y'|)$, where $u'$ is a random vertex selected from $X'$ and $\{u_i\}\times A_i$ is a subset in $D$, which is bounded by $O(|X'|+|V_G|)$. Therefore, the new upper bound can be obtained in $O(\sum_{X'\times Y'\in C} (|X'|+|V_G|))$, which is bounded by $O(|V_Q|+|V_G|^2)$.

\smallskip
\noindent\textbf{Worst-case time complexity of \texttt{RRSplit}.} We note that the worst-case time complexity of \texttt{RRSplit} is dominated by the number of recursive calls of \texttt{RRSplit-Rec} (i.e., the number of formed branches) since \texttt{RRSplit-Rec} runs in polynomials of $|V_Q|$ and $|V_G|$. Formally, we have the following theorem.
\begin{theorem}
    Assume that $|V_Q|\leq |V_G|$. The worst-case time complexity of our proposed \texttt{RRSplit}  is $O^*((|V_G|+1)^{|V_Q|})$, where $O^*(\cdot)$ suppresses the polynomials.
\end{theorem}
\begin{proof}
    It is easy to verify that the worst-case time complexity of \texttt{RRSplit} is bounded by the number of branches. Consider a branch $B=(S,C,D)$. For all sub-branches formed at $B$ by selecting a branching vertex $u^*$, we observe that only the sub-branch in the second group has the same partial solution $S$ with $B$. Based on this, we can easily deduce that there are at most $|V_Q|$ branches which share the same partial solution. Besides,  we observe that each vertex in $V_p\cup V_q$ only appears in one pair of $S$, i.e., for any two distinct pairs $\langle u,v \rangle$ and $\langle u',v' \rangle$ in $S$, we have $u\neq u'$ and $v\neq v'$. Based on this, let $|S|=k$ where $0\leq k\leq |V_Q|$, and we can deduce that there are at most $k!\binom{|V_Q|}{k}\binom{|V_G|}{k}$ different partial solutions with the size of $k$ by applying the multiplication principle (note that $V_p$ has $\binom{|V_Q|}{k}$ different choices, $V_q$ has $\binom{|V_G|}{k}$ different choices, and the bijection $\phi$ between $V_p$ and $V_q$ has $k!$ different choices). Therefore, the number of branches is at most
    \begin{equation}
       T=|V_Q| \sum_{k=0}^{|V_Q|} k! \binom{|V_Q|}{k}\binom{|V_G|}{k}.
    \end{equation}
    We then show that $T$ is bounded by $O^*((|V_G|+1)^{|V_Q|})$ as below.
    \begin{eqnarray}
       T&=&|V_Q| \sum_{k=0}^{|V_Q|} (|V_Q|-k)! \binom{|V_Q|}{k}\binom{|V_G|}{|V_Q|-k}\\
       &=&|V_Q| \sum_{k=0}^{|V_Q|} \frac{(|V_{G}|)!}{(|V_G|-|V_Q|+k)!}\binom{|V_Q|}{k}\\
       &\leq& |V_Q|\sum_{k=0}^{|V_Q|} (|V_G|)^{|V_Q|-k} \binom{|V_Q|}{k}=|V_Q|(|V_G|+1)^{|V_Q|},
    \end{eqnarray}
    where $(|V_{G}|)!/(|V_G|-|V_Q|+k)!$ is  much smaller than $(|V_G|)^{|V_Q|-k}$ clearly and $(|V_G|+1)^{|V_Q|}$ in the last equation is derived by the binomial theorem.
\end{proof}



\smallskip
\noindent\textbf{Remark.} \laks{Note that the assumption that $|V_Q|\leq |V_G|$ is not a restrictive assumption: it is realistic in practice.} We remark that to our best knowledge, the achieved worst-case time complexity $O^*((|V_G|+1)^{|V_Q|})$ of \texttt{RRSplit} \laks{matches} 
% the state-of-the-art
{\chengB the best-known worst-case time complexity for the problem}
~\cite{suters2005new}. However, the algorithm proposed in~\cite{suters2005new} is of theoretical {\chengB interest} only and is not {\chengB practically} efficient. Besides, we note that \texttt{McSplit} and its variants~\cite{zhoustrengthened,liu2020learning,liu2023hybrid,mccreesh2017partitioning} do not have any theoretical guarantees on the worst-case time complexity. 
% Therefore, \texttt{RRSplit} is efficient in both theory and practice.



\section{Experimental Results}


In this section, we systematically evaluate current LLMs' performance on our \name.



\begin{table*}[t]
    \centering
    \small
    \setlength{\tabcolsep}{3pt}
    \begin{tabular}{l|ccc|ccc|ccc | cc} 
        \toprule
        \multirow{2}{*}{} 
        & \multicolumn{3}{c|}{\textbf{EU AI Act}} & \multicolumn{3}{c|}{\textbf{GDPR}} & \multicolumn{3}{c|}{\textbf{HIPAA}} & \multicolumn{2}{c}{\textbf{ACLU}}\\ 
        %\cmidrule(lr){2-4} \cmidrule(lr){5-7} \cmidrule(lr){8-10}
        \textbf{Model}  & DP & CoT & RAG & DP & CoT & RAG & DP & CoT & RAG & DP & CoT \\ 
        \midrule
        Mistral-7B-Instruct & 49.83 & 43.50 & 45.56 & 72.29 & 68.02 & 43.38 & 45.79 & 60.74 & 64.95 & 44.92 & \textbf{72.46}\\
        Qwen-2.5-7B-Instruct & 49.90 & 65.30 & \textbf{55.83} & 89.00 & 88.81 & 82.43 & 68.69 & 72.43 & 71.49 & 50.72 & 52.17 \\
        Llama-3.1-8B-Instruct & 61.30 & 59.40 & 53.50 & 85.30 & \textbf{90.27} & \textbf{76.60} & 77.57 & 85.51 & \textbf{88.31}  & 66.17 & 66.67\\
        GPT-4o-mini & 73.76 & 66.60 & - & \textbf{92.03} & 65.69 & - & 80.84 & 67.75 & - & \textbf{69.56} & 31.88\\
        QwQ-32B & \textbf{78.22} & \textbf{75.30} & - & 80.45 & 90.08 & - & 70.09 & \textbf{88.31} & - &  55.07& 55.07 \\

        \multirow{1}{*}{Deepseek R1 (671B)} & 72.90 & 60.67 & - & 90.66 & 47.88 & - & \textbf{89.25} & 81.77 & -& 65.21 & 59.42 \\
         
    %     \midrule
    % \multirow{1}{*}{Average} & 64.31 & 61.79 & xx.xx & 84.95 & 75.12 & xx.xx & 72.03 & 76.08 & xx.xx & 59.33 & 56.76 \\
        \bottomrule
    \end{tabular}
    \vspace{-0.1in}
    \caption{Accuracy Evaluation results of the legal compliance task. All results are reported in \%.}
    \label{tab:privacy_result}
    \vspace{-0.1in}
\end{table*}








\begin{table*}[t]
    \centering
    \small
    \setlength{\tabcolsep}{5pt}
    \begin{tabular}{l|ccc|ccc|ccc}
        \toprule
        \multirow{2}{*}{} 
         & \multicolumn{3}{c|}{\textbf{Permit}} & \multicolumn{3}{c|}{\textbf{Prohibit}} & \multicolumn{3}{c}{\textbf{Not Applicable}} \\
        %\cmidrule(lr){2-4} \cmidrule(lr){5-7} \cmidrule(lr){8-10}
        \textbf{Model\&Method} & Precision & Recall & F1 & Precision & Recall & F1 & Precision & Recall & F1 \\
        \midrule
        Qwen2.5-7B-Instruct-DP & 36.17  &  55.30  &  43.74 & 68.83  &  87.54  &  77.06 & 40.62   & 7.80  &  13.09 \\
        Qwen2.5-7B-Instruct-CoT & 52.93    &51.80    &52.36 &68.06  &  85.58   & 75.82  & 77.37   & 59.50    &67.27  \\
        Qwen2.5-7B-Instruct-RAG & 49.63  &  51.99  &  50.78  &  70.45  &  54.99  &  61.77 & 73.69  &  60.50  &  66.45  \\
        Mistral-7B-Instruct-DP & 83.33  &  0.49  &  0.97 & 73.50  & 50.57  &  59.91 & 42.97  &  99.90  &  60.09  \\
        Mistral-7B-Instruct-CoT & 52.83  &  2.72  &  5.18  & 80.23   &  28.84   & 42.42  &  40.74   &  99.70   &  57.85  \\
        Mistral-7B-Instruct-RAG & 46.55  &  7.87  &  13.47 &  81.95  &  29.45  &  43.33  &  42.86   & 100.00  &  60.01  \\
        
        % \midrule
        % Average & 53.74 & 28.03 & 27.08 & 73.50 & 56.33 & 59.72 & 53.38 & 71.23 & 54.46 \\
        \bottomrule
    \end{tabular}
    \vspace{-0.1in}
    \caption{The detailed investigation of Qwen2.5-7B-Instruct and Mistral-7B-Instruct models performance over 3 classes on the AI Act cases. All results are reported in \%.}
    \label{tab:compliance_detail}
    \vspace{-0.2in}
\end{table*}


\subsection{Evaluation on Legal Compliance}
To study whether LLMs can comply with existing privacy regulations, we prompt these LLMs with our collected cases.
Table~\ref{tab:privacy_result} evaluates LLMs' legal compliance accuracies over the four domains.
The compliance results suggest the following findings.

%%% TO DO LIST
%%% 1. EU AI Act ANALYSIS, why it is so bad for LLMs
%%% 2. CoT and RAG not working on AI Act, GDPR?
%%% 3. Parsing Errors analysis or Not Relevant Analysis?
%%% 
1) \textit{The collected EU AI Act and ACLU subsets are the most challenging subsets for legal compliance. }
As outlined in Section~\ref{sec: ai act}, cases from the EU AI Act are synthesized according to its official compliance checker.
Therefore, these cases are not likely to be accessed by LLMs and LLMs can only use their reasoning abilities to determine compliance.
We further investigate the precision, recall and F1 scores for LLMs' predictions over each class on Table~\ref{tab:compliance_detail}.
Both LLMs underperform in the permitted cases.
%For example, Mistral-7B-Instruct has recall scores of no more than 8\% on permitted cases while nearly 100\% on the not-applicable cases, which implies that it classify most the permitted cases as not applicable cases.
For instance, Mistral-7B-Instruct has recall scores of no more than 8\% on permitted cases, while getting nearly 100\% on not-applicable cases.
The results suggest that LLMs cannot distinguish between permitted and not applicable cases.
Regarding the ACLU cases, they always connect with a wide range of legal regulations, including the Fourth Amendment to the United States Constitution and the Freedom of Information Act.
The ACLU data demand a more comprehensive understanding of their applicable regulations, and compliance is harder to determine.
Consequently, even the best-performing reasoner models (QwQ-32B and Deepseek R1) fail to attain satisfactory results on the two subsets.
%These results suggest that current LLMs


2) \textit{Chain-of-Thought reasoning and naive RAG implementation may not always help improve LLMs' safety and privacy compliance.}
For CoT prompting, its effectiveness is model-specific.
Our evaluation of instruction-tuned LLMs, including Mistral-7B, Qwen-2.5-7B and Llama-3.1-8B, reveals general accuracy improvements compared to direct prompting (DP).
However, this trend does not hold for all models.
Specifically, GPT-4o-mini and Deepseek R1 reasoner exhibit degraded performance when using CoT prompting.
On the other hand, the performance of our implemented naive retrieval augmented generation (RAG) method is domain-specific.
For the HIPAA domain, RAG generally leads to the best performance, which aligns with findings from prior research ~\cite{li-2024-privacychecklist}.
However, this improvement fails to extend to the EU AI Act and GDPR domains, where RAG results in notable drops in accuracy.
%\tbc{We further give a detailed analysis in xxx.}




% \begin{table*}[htbp]
%     \centering
%     \small
%     \setlength{\tabcolsep}{3pt} 
%     \begin{tabular}{llccc|ccc|ccc|ccc|ccc}
%         \toprule
%         \textbf{} & \textbf{} & \multicolumn{3}{c|}{\textbf{Mistral-7B-Instruct}} & \multicolumn{3}{c|}{\textbf{Qwen-2.5-7B-Instruct}} & \multicolumn{3}{c|}{\textbf{Llama-3.1-8B-Instruct}} & \multicolumn{3}{c|}{\textbf{GPT-4o-mini}} & \multicolumn{3}{c}{\textbf{QwQ-32B}} \\ 
%         \cmidrule(lr){3-5} \cmidrule(lr){6-8} \cmidrule(lr){9-11} \cmidrule(lr){12-14} \cmidrule(lr){15-17}
%         \textbf{Dataset} & \textbf{} & Easy & Medium & Hard & Easy & Medium & Hard & Easy & Medium & Hard & Easy & Medium & Hard & Easy & Medium & Hard \\ 
%         \midrule
%         EU AI Act & & 81.01 & 69.86 & 50.13 & 91.84 & 83.50 & 57.01 & 80.56 & 66.61 & 50.20 & 96.59 & 87.07 & 59.21 & 91.26 & 82.80 & 57.17 \\
%         GDPR & & 85.54 & 75.92 & 55.99 & 93.61 & 87.78 & 63.86 & 85.22 & 75.17 & 57.81 & 97.11 & 94.34 & 75.84 & 96.07 & 93.01 & 75.52 \\
%         HIPAA & & 85.81 & 76.26 & 56.35 & 93.72 & 87.95 & 64.22 & 85.53 & 75.59 & 58.27 & 97.17 & 94.46 & 76.11 & 98.28 & 94.68 & 78.80 \\
%         \bottomrule
%     \end{tabular}
%     \caption{Evaluation results for Easy, Medium, and Hard categories.}
%     \label{tab:mcq_results_split}
% \end{table*}


% \begin{table*}[htbp]
%     \centering
%     \small
%     \setlength{\tabcolsep}{3pt}
%     \begin{tabular}{l|cccc|cccc|cccc} 
%         \toprule
%         \multirow{2}{*}{} 
%         & \multicolumn{4}{c|}{\textbf{EU AI Act}} & \multicolumn{4}{c|}{\textbf{GDPR}} & \multicolumn{4}{c}{\textbf{HIPAA}} \\ 
%         %\cmidrule(lr){2-4} \cmidrule(lr){5-7} \cmidrule(lr){8-10}
%         \textbf{Model}  & Easy & Medium & Hard & Avg & Easy & Medium & Hard & Avg & Easy & Medium & Hard & Avg \\ 
%         \midrule
%         Mistral-7B-Instruct & 81.01 & 69.86 & 50.13 & 85.54 & 75.92 & 55.99 & 85.81 & 76.26 & 56.35 \\
%         Qwen-2.5-7B-Instruct & 91.84 & 83.50 & 57.01 & 93.61 & 87.78 & 63.86 & 93.72 & 87.95 & 64.22 \\
%         Llama-3.1-8B-Instruct & 80.56 & 66.61 & 50.20 & 85.22 & 75.17 & 57.81 & 85.53 & 75.59 & 58.27 \\
%         GPT-4o-mini & 96.59 & 87.07 & 59.21 & 97.11 & 94.34 & 75.84 & 97.17 & 94.46 & 76.11 \\
%         QwQ-32B & 91.26 & 82.80 & 57.17 & 96.07 & 93.01 & 75.52 & 98.28 & 94.68 & 78.80 \\
%         \midrule
%     \multirow{1}{*}{Average} & 88.25 & 77.97 & 54.74 & 91.51 & 85.24 & 65.80 & 92.10 & 85.79 & 66.75 \\
%         \bottomrule
%     \end{tabular}
%     \caption{Evaluation results for Easy, Medium, and Hard categories.}
%     \label{tab:mcq_results_split}
% \end{table*}


\begin{table*}[t]
    \centering
    \small
    \setlength{\tabcolsep}{3pt}
    \begin{tabular}{l|cccc|cccc|cccc} 
        \toprule
        \multirow{2}{*}{} 
        & \multicolumn{4}{c|}{\textbf{EU AI Act}} & \multicolumn{4}{c|}{\textbf{GDPR}} & \multicolumn{4}{c}{\textbf{HIPAA}} \\ 
        %\cmidrule(lr){2-4} \cmidrule(lr){5-7} \cmidrule(lr){8-10}
        \textbf{Model}  & Easy & Medium & Hard & Avg & Easy & Medium & Hard & Avg & Easy & Medium & Hard & Avg \\ 
        \midrule
        Mistral-7B-Instruct & 81.01 & 69.86 & 50.13 & 67.00 & 85.54 & 75.92 & 55.99 & 72.48 & 85.81 & 76.26 & 56.35 & 72.81 \\
        Qwen-2.5-7B-Instruct & 91.84 & 83.50 & 57.01 & 77.45 & 93.61 & 87.78 & 63.86 & 81.75 & 93.72 & 87.95 & 64.22 & 81.96 \\
        Llama-3.1-8B-Instruct & 80.56 & 66.61 & 50.20 & 65.79 & 85.22 & 75.17 & 57.81 & 72.73 & 85.53 & 75.59 & 58.27 & 73.13 \\
        GPT-4o-mini & 96.59 & 87.07 & 59.21 & 80.96 & 97.11 & 94.34 & 75.84 & 89.09 & 97.17 & 94.46 & 76.11 & 89.25 \\
        QwQ-32B & 91.26 & 82.80 & 57.17 & 77.08 & 96.07 & 93.01 & 75.52 & 88.20 & 98.28 & 94.68 & 78.80 & 90.59 \\
        \midrule
    \multirow{1}{*}{Average} & 88.25 & 77.97 & 54.74 & 73.65 & 91.51 & 85.24 & 65.80 & 80.85 & 92.10 & 85.79 & 66.75 & 81.55 \\
        \bottomrule
    \end{tabular}
    \vspace{-0.1in}
    \caption{Accuracy Evaluation results of the context understanding task. All results are reported in \%.}
    \label{tab:mcq_results_split}
    \vspace{-0.1in}
\end{table*}
\subsection{Evaluation on Context Understanding}

Besides evaluating the overall performance on the compliance task, we also convert the parsed structured cases into multiple-choice questions as stated in Section~\ref{sec: data processing} with 3 difficulty levels for the EU AI Act, GDPR, and HIPAA domain.
These questions enable us to probe how well LLMs are able to understand the context and identify the key CI parameters inside its information flows.
Table~\ref{tab:mcq_results_split} shows LLMs' performance over these multiple-choice questions.
The results of the context understanding task imply the following findings.


3) \textit{Existing LLMs can explicitly identify the CI parameters of the information flow inside the given context.}
For prompted multiple-choice questions, LLMs, on average, can reach accuracies of approximately \textasciitilde 90\% on the Easy subset, \textasciitilde 80\% on the Medium subset, and \textasciitilde 60\% on the Hard subset.
The high accuracy suggests that LLMs are well aware of the context and its key characteristics inside the context's information flow.


%% qwq vs qwen
4) \textit{LLMs' reasoning enhanced by reinforcement learning further improves the context understanding abilities.}
When comparing Qwen-2.5-7B-Instruct with Qwen's latest QwQ-32B reasoner model, Qwen's QwQ-32B has higher accuracy over most subsets, especially on the hard questions.
The result indicates that reinforcement learning helps LLMs to better understand and analyze the context.
Consequently, better context-understanding abilities further improve legal compliance, as indicated by the results of Table~\ref{tab:privacy_result}.

5) \textit{The context of EU AI Act subset is challenging for LLMs to understand.}
On average, all LLMs have comparable performance across the Easy, Medium, and Hard subsets of the GDPR and HIPAA domains.
However, their accuracies on the EU AI Act subset fall significantly behind the other two domains.
We manually examine samples within the EU AI Act and observe that their parsed roles of CI parameters are mostly abstract legal terms such as ``Law Enforcement Agencies,'' ``Importer,'' ``Operator'' and ``provider.'' 
These terms make it hard to correctly identify the stakeholders for LLMs.
In addition, compared with real cases, the AI Act's synthetic vignettes also lack narrative coherence for describing the information flows.
Hence, LLMs struggle to perform well on the multiple-choice questions of the AI Act domain.
As a result, LLMs' compliance also degrades.
%The context understanding results on the EU AI Act data suggest that it is the most challenging subset and partially explains why LLMs underperform on the EU AI Act's legal compliance task.
%This manual inspection partially explains why LLMs underperform on the legal compliance tasks associated with the EU AI Act.


%% \begin{table*}[t]
%     \centering
%     \caption{}
%     % 子表1:Location Data
%     \begin{subtable}[h]{0.46\linewidth}
%     \centering
%     \resizebox{\linewidth}{!}{
%         \begin{tabular}{c|c|cccc}
%         \toprule
%         \textbf{Location Method} & \textbf{Model} & \textbf{harm} & \textbf{sorry} & \textbf{gsm8k} & \textbf{math} \\ \midrule
%         random                &                &              &                &               &              \\ 
%         sparsegpt             & LM+MATH       &              &                &               &              \\ 
%         Importance score      &                &              &                &               &              \\ 
%         wandg                 &                & 16.00        & 24.22          & 50.34         & 14.20        \\ \bottomrule
%         \end{tabular}
%     }
%     \label{tab:location}
%     \caption{}
%     \end{subtable}
%     \hfill
%     % 子表2:Election Data
%     \begin{subtable}[h]{0.46\linewidth}
%     \centering
%     \resizebox{\linewidth}{!}{
%         \begin{tabular}{c|c|cccc}
%         \toprule
%         \textbf{Election type} & \textbf{Model} & \textbf{harm} & \textbf{sorry} & \textbf{gsm8k} & \textbf{math} \\ \midrule
%         00                    &                &              &                &               &              \\ 
%         01                    & LM+MATH       &              &                &               &              \\ 
%         10                    &                &              &                &               &              \\ 
%         11                    &                & 16.00        & 24.22          & 50.34         & 14.20        \\ \bottomrule
%         \end{tabular}
%     }
%     \label{tab:election}
%     \caption{}
%     \end{subtable}
% \end{table*}


% % % 子表3:Ablation of Disjoint Data
% \begin{table}[h]
% \centering
% \resizebox{\linewidth}{!}{
% \begin{tabular}{c|c|cccc}
% \toprule
% \textbf{Disjoint} & \textbf{Model} & \textbf{harm} & \textbf{sorry} & \textbf{gsm8k} & \textbf{math} \\ \midrule
% w/o Disjoint          & LM+MATH       &              &                &               &              \\
% w/ Disjoint           &                & 16.00        & 24.22          & 50.34         & 14.20        \\ \bottomrule
% \end{tabular}
% }
% \caption{Ablation of Disjoint Data}
% \label{tab:ablation_disjoint_data}
% \end{table}



\begin{table}[!ht]
\centering
\caption{Ablation Study. Experiments are conducted on Mistral-7B series models. $\ast$ represents LLM's instruction following ability is impaired.}
\resizebox{\linewidth}{!}{
\begin{tabular}{c|c|cc|cc}
\toprule
\multirow{2}{*}{\begin{tabular}[c]{@{}c@{}}\textbf{Ablation} \\ \textbf{Part}\end{tabular}} & \multirow{2}{*}{\begin{tabular}[c]{@{}c@{}}\textbf{Alternative} \\ \textbf{Methods}\end{tabular}} & \multicolumn{2}{c|}{\begin{tabular}[c]{@{}c@{}}\textbf{Safety}\end{tabular}} & 
\multicolumn{2}{c}{\begin{tabular}[c]{@{}c@{}}\textbf{Mathematical} \\ \textbf{Reasoning}\end{tabular}} \\ \cmidrule{3-6}
&                         &                             \textbf{HarmBench}$\downarrow$                                                     & \textbf{SORRY-Bench}$\downarrow$                                                       & \textbf{GSM8K}$\uparrow$                                          & \textbf{MATH}$\uparrow$                                   \\ \midrule

\multirow{3}{*}{Location} & Random                   & $\ast$             & $\ast$              & 25.58              & 8.66             \\ 
                         & Wanda       & $\ast$             & $\ast$               & 39.58              & 11.37             \\  
                        & SNIP               & 16.00        & 24.22          & 50.34         & 14.20        \\ \midrule

\multirow{3}{*}{Election} 
                   & 01       &  58.00            & 83.77               & 54.13              & 13.12             \\ 
                   & 10                & 35.25             & 47.33               & 50.64              & 13.30             \\ 
                    & 11               & 16.00        & 24.22          & 50.34         & 14.20        \\ \midrule

\multirow{2}{*}{Disjoint}          & \textcolor{gray}{\usym{2717}}      & 63.00             & 85.33               & 72.93              & 23.18             \\
           & \textcolor{gray}{\usym{2714}}               & 16.00        & 24.22          & 50.34         & 14.20        \\ \bottomrule
\end{tabular}
}
\label{tab:ablation}
%\vspace{-10pt}
\end{table}
\begin{figure*}[t]
\centering
\includegraphics[width=0.999\textwidth]{figs/ablations.pdf}
\vspace{-0.3in}
\caption{
Ablation studies for the legal compliance task. All results are evaluated in \%.
}
\label{fig:ablations}
\vspace{-0.15in}
\end{figure*}
%%% case study?
\subsection{Ablation Studies}

To study the effectiveness of our annotated CI parameters and applicable regulation content, we further perform ablation studies by feeding LLMs with ground truth CI parameters and regulations as stated in Section~\ref{sec: judge}.

Figure~\ref{fig:ablations} presents the accuracies of DP+CI and DP+CI+LAW across various LLMs for the legal compliance task.
By comparing DP+CI with CI, we observe that appending the contextual integrity parameters significantly improves LLMs' accuracies, particularly in the HIPAA and ACLU domains. 
Such results suggest that CI parameters indeed help LLMs better understand the context and improve legal compliance performance.
Furthermore,  for DP+CI+LAW, we augment the applicable regulations to DP+CI and obtain consistent performance gains.
Consequently, DP+CI+LAW has the best performance compared with our implemented DP, CoT, and RAG methods.
The results of DP+CI+LAW highlight the effectiveness of retrieval augmented generation methods, provided that the retrieved documents are both relevant and applicable.
%These results further suggest that naive RAG implementation may not help improve LLMs' compliance due to wrong retrieval results, implying that there are gaps between common context and legal terminologies.
Moreover, our ablation studies also imply that naive RAG implementations may degrade LLMs' compliance when the retrieval step yields irrelevant results. 
Such retrieval failures disclose a discrepancy between general context and domain-specific legal terminologies, which suggests that our \name requires a tailored retrieval module for improvement.

%suggesting that careful curation and relevance of retrieved documents are essential for improving model performance in legal applications.

%\subsection{Case Studies}


\subsection{Human Evaluations}
\label{subsec:human_eval}
%% CI para inspection
To assess whether our parsed CI parameters and judgments are reliable, three authors manually inspect the data quality.
This inspection calculates annotators' agreement with the parsed roles and associated attributes (Role), the transmission principle (TP), and the parsed judgment results (Label).
For Role agreement, we assign an integer from 0 to 3 by considering the sender, receiver and subject.
For TP and Label, we assign a binary agreement score (0 or 1).
To ensure a representative assessment, we randomly sample 30 parsed regulations and cases for each domain.
We then average and re-scale the results under 100\% for consistency, as shown in Table~\ref{tab:human_eval}.

\begin{table}[h]
\small
    \centering
    \begin{tabular}{l l|ccc}
        \toprule
        \textbf{Domain} & 
        \textbf{Type} &
        \textbf{Role} & \textbf{TP} & \textbf{Label} \\
        \midrule

        \multirow{2}{*}{HIPAA} &
        Case & 97.78 & 96.67
 & 100.00 \\
        & Law & 98.89 & 93.33
 & 96.67
 \\
        \midrule
        \multirow{2}{*}{GDPR} &
        Case & 96.67 & 96.67
 & 96.67\\
        & Law & 94.44 & 96.67
 & 93.33
 \\
        \midrule
        \multirow{2}{*}{AI Act} &
        Case & 90.00 & 93.33 & 96.67 \\
        & Law & 98.89 & 96.67
 & 96.67 \\
        
        \bottomrule
    \end{tabular}
    \vspace{-0.1in}
    \caption{Averaged Human agreement with our parsed data. Results are averaged and rescaled under \%.}  
    \vspace{-0.15in}
    \label{tab:human_eval}
\end{table}
%% Case agreement

The manual inspection results indicate that the HIPPA domain achieves the highest agreement scores among parsed cases and regulations.
This can be attributed to the fact that HIPAA is related to the medical domain, where roles and transmitted attributes are more clear and consistent.
For instance, it is frequent to observe a covered entity sharing the patient's protected health information (PHI).
Hence, it is easier to parse CI parameters.
For the EU AI Act, its cases' role has the worst performance, with an agreement score of 0.9.
We further inspect the EU AI Act synthetic cases and find that even though these cases strictly follow the question-answering chains of the compliance checker, they still suffer from narrative incoherence.
We leave the detailed case analyses in Appendix~\ref{app: case}.
\section{Related Works}\label{sec:related}
In this section, we discuss previous work relevant to BaKlaVa in five main areas: KV-cache eviction policy, profiling for determining memory budget, KV-cache quantization, cache merge, and system-level optimizations. 

\subsection{KV Cache Eviction Policy}
StreamingLLM~\cite{streamingllm} discovered the 'attention sink' effect, where early sequence tokens play a crucial role in maintaining model performance through asymmetric attention weight accumulation. H2O~\cite{h2o} introduces an eviction strategy based on cumulative attention, retaining 'heavy-hitter' key-value pairs while allowing token positions to vary. Similarly, Scissorhands~\cite{scissorhands} develops an approach that evicts based on a 'pivotal' metric, adjusting eviction rates across layers using a persistence ratio. Keyformer~\cite{keyformer} addresses the issue of token removal distorting softmax probability distributions by implementing regularization techniques to mitigate these perturbations. 

\subsection{Profiling for Determining Memory Budget}
Squeezeattention~\cite{squeezeattention} employs a dynamic approach, measuring layer importance through cosine similarity of input prompt differences pre- and post-self-attention, subsequently categorizing layers and adjusting their KV budgets. PyramidInfer~\cite{pyramidinfer} introduces a pyramid-shaped allocation strategy, prioritizing tokens with high attention values and maintaining a set of significant tokens through attention-driven updates during decoding. In comparison, Ada-KV~\cite{adakv} offers an adaptive budget allocation method that improves utilization across individual attention heads, resulting in more effective cache eviction strategies.



\subsection{KV-Cache Quantization}
GEAR~\cite{gear} takes a different approach by compressing less important entries to ultra-low precision, using a low-rank matrix for residual error approximation, and utilizing a sparse matrix for outlier correction. MiKV~\cite{notokenleftbehind} introduces a mixed-precision KV-cache quantization method, allocating precision based on token importance. QAQ~\cite{qaq} proposes a dynamic, quality-adaptive quantization approach that determines bit allocation based on token importance and sensitivity. KVQuant~\cite{hooper2024kvquant} offers strategies for smooth quantization of keys and values, including pre-RoPE quantization for keys, per-token quantization for values, and isolation of outliers in a sparse format. These diverse techniques collectively contribute to significant improvements in model compression and efficiency while maintaining performance.

\subsection{Cache Merge}
MiniCache~\cite{minicache} leverages the high angular similarity observed in middle-to-deep layer KV caches, merging key and value pairs from adjacent similar layers into shared representations. KVSharer~\cite{yang2024kvsharer}, on the other hand, exploits the counterintuitive finding that sharing KV caches between significantly different layers during inference does not substantially impact performance, prioritizing dissimilar layers for sharing based on Euclidean distance calculations. In comparison,  CaM~\cite{cam} focuses on merging keys or values of multiple evicted tokens with retained tokens using attention scores, while KVMerger~\cite{wang2024model} employs a two-step process: first clustering consecutive tokens with high cosine similarity, then merging tokens within each set into a pivotal token chosen by the attention score, using Gaussian kernel weights to emphasize contextual relevance. 

\subsection{System-Level Optimizations}
FlexGen~\cite{flexgen} proposes an SSD-based method for managing key-value (KV) caches, effectively expanding the memory hierarchy across GPU, CPU, and disk storage. This approach utilizes linear programming to optimize tensor storage and access patterns, enabling high-throughput LLM inference on hardware with limited resources. Complementing this, ALISA~\cite{alisa} introduces a dual-level KV cache scheduling framework that combines algorithmic sparsity with system-level optimization. At the algorithmic level, ALISA employs a Sparse Window Attention mechanism to identify and prioritize crucial tokens for attention computation, while at the system level, it implements a three-phase token-level dynamic scheduler to manage KV tensor allocation and balance caching and recomputation.


\section{Conclusion}\label{sec:conclusion}

In this paper, we proposed a prototype ASL generation system aimed at improving the naturalness, comprehensiveness, and overall quality of generated signs, addressing key limitations in existing approaches. Our technical evaluations indicate that our proposed approaches improve these aspects, enhancing the quality of generated ASL content. Feedback from DHH participants was mixed; while there was general interest in the system, concerns regarding visual quality and naturalness were noted. Reflecting on our design process and study findings, we discuss key insights and identify key areas for future improvement. While further work is needed, our study takes an initial step toward developing sign language generation systems that better meet the needs of the DHH and signing communities, offering real-world value.
\section{acknowledgement}
The research of Kaiqiang Yu, Kaixin Wang and Cheng Long is supported by the Ministry of Education, Singapore, under its Academic Research Fund (Tier 2 Award MOE-T2EP20221-0013 and Tier 1 Awards (RG20/24 and RG77/21)). Any opinions, findings and conclusions or recommendations expressed in this material are those of the author(s) and do not reflect the views of the Ministry of Education, Singapore.
%
Lakshmanan’s research was supported in part by a grant from
the Natural Sciences and Engineering Research Council of Canada (Grant Number RGPIN-2020-05408).
%
Reynold Cheng was supported by the Hong Kong Jockey Club Charities Trust (Project 260920140), the University of Hong Kong (Project 2409100399), the HKU Outstanding Research Student Supervisor Award 2022-23, and the HKU Faculty Exchange Award 2024 (Faculty of Engineering).

\balance
\clearpage
\bibliographystyle{ACM-Reference-Format}
\bibliography{SIGMOD_MaxCS}
\clearpage
\newpage
\appendix
\onecolumn
\section{Full Results on Longbench}
\label{appendix}
% \renewcommand{\arraystretch}{1.2} % 设置行高
\begin{table*}[ht]
\setlength{\tabcolsep}{2.5pt} % 设置列间距
\caption{\textbf{Result on Longbench.} The highest score in each task is marked in bold (except for "Full"). We also note the relative error of Twilight when integrated with the corresponding base algorithm. Green indicates an increase in score, while red indicates a decrease.}
\label{table:longbench}
    \centering
    \scalebox{0.69}{
    \begin{tabular}{lcccccccccccccc}
        \toprule
        \multirow{2}*{\textbf{Methods}} &
        \multirow{2}*{\textbf{Budget }} &
        \multicolumn{2}{c}{\textbf{Single-Doc. QA}} & \multicolumn{3}{c}{\textbf{Multi-Doc. QA}} & \multicolumn{3}{c}{\textbf{Summarization}} & \multicolumn{1}{c}{\textbf{Few-shot}} & \multicolumn{2}{c}{\textbf{Code}} & \multicolumn{1}{c}{\textbf{Synthetic}} & \multirow{2}*{\textbf{Avg. Score}}  \\
        \cmidrule(lr){3-4}\cmidrule(lr){5-7}\cmidrule(lr){8-10} \cmidrule(lr){11-11} \cmidrule(lr){12-13} \cmidrule(lr){14-14} 
        & & \textit{Qasper} & \textit{MF-en} & \textit{HotpotQA} & \textit{2WikiMQA} &  \textit{Musique} & \textit{GovReport} & \textit{QMSum} & \textit{MultiNews} & \textit{TriviaQA} &  \textit{LCC} & \textit{Repobench-P} & \textit{PR-en} \\
        \midrule
        \multicolumn{15}{c}{\textsc{Longchat-7B-32k}} \\
        \midrule
        \multirow{2}*{Full} & 32k & 29.48 & 42.11 & 30.97 & 23.74 & 13.11 & 31.03 & 22.77 & 26.09 & 83.25 & 30.50 & 52.70 & 55.62 & 36.78 \\
         & \textbf{Twilight (Avg. 146)} & 31.74 & \textbf{43.91} & 33.59 & \textbf{25.65} & \textbf{13.93} & 32.19 & \textbf{23.15} & 26.30 & 85.14 & 34.50 & 54.98 & 57.12 & 38.52\textcolor{teal}{(+4.7\%)}\\
        \midrule
        \multirow{5}*{Quest}
         & 256 & 26.00 & 32.83 & 23.23 & 22.14 & 7.45 & 22.64 & 20.98 & 25.05 & 67.40 & 33.60 & 48.70 & 45.07 & 31.26 \\
      & 1024 & 31.63 & 42.36 & 30.47 & 24.42 & 10.11 & 29.94 & 22.70 & 26.39 & 84.21 & 34.5 & 51.52 & 53.95 & 36.85 \\
       & 4096 & 29.77 & 42.71 & 32.94 & 23.94 & 13.24 & 31.54 & 22.86 & 26.45 & 84.37 & 31.50 & 53.17 & 55.52 & 37.33 \\
        & 8192 & 29.34 & 41.70 & 33.27 & 23.46 & 13.51 & 31.18 & 23.02 & 26.48 & 84.70 & 30.00 & 53.02 & 55.57 & 37.10 \\
             & \textbf{Twilight (Avg. 131)} & 31.95 & 43.28 & 31.62 & 24.87 & 13.48 & \textbf{32.21} & 22.79 & 26.33 & 84.93 & 33.50 & 54.86 & 56.70 & 38.04\textcolor{teal}{(+2.5\%)} \\
        \midrule
    \multirow{5}*{DS}
         & 256 & 28.28 & 39.78 & 27.10 & 20.75 & 9.34 & 29.68 & 21.79 & 25.69 & 83.97 & 32.00 & 52.01 & 53.44 & 35.32 \\
      & 1024 & 30.55 & 41.27 & 30.85 & 21.87 & 7.27 & 26.82 & 22.95 & 26.51 & 83.22 & 31.50 & 53.23 & 55.50 & 35.96 \\
       & 4096 & 28.95 & 41.90 & 32.52 & 23.65 & 8.07 & 29.68 & 22.75 & \textbf{26.55} & 83.34 & 30.00 & 52.77 & 55.48 & 36.31 \\
        & 8192 & 29.05 & 41.42 & 31.79 & 22.95 & 12.50 & 30.44 & 22.50 & 26.43 & 83.63 & 30.50 & 52.87 & 55.33 & 36.62 \\
             & \textbf{Twilight (Avg. 126)} & \textbf{32.34} & 43.89 & \textbf{34.67} & 25.43 & 13.84 & 31.88 & 23.01 & 26.32 & \textbf{85.29} & \textbf{35.50} & \textbf{55.03} & \textbf{57.27} & \textbf{38.71}\textcolor{teal}{(+5.7\%)} \\
        \midrule
        \multicolumn{15}{c}{\textsc{Llama-3.1-8B-Instruct}} \\
        \midrule
        \multirow{2}*{Full} & 128k & 46.17 & 53.33 & 55.36 & 43.95 & 27.08 & 35.01 & 25.24 & 27.37 & 91.18 & 99.50 & 62.17 & 57.76 & 52.01 \\
         & \textbf{Twilight (Avg. 478)} & 43.08 & 52.99 & 52.22 & 44.83 & 25.79 & 34.21 & \textbf{25.47} & 26.98 & 91.85 & \textbf{100.00} & \textbf{64.06} & 58.22 & 51.64\textcolor{red}{(-0.7\%)} \\
        \midrule
        \multirow{5}*{Quest}
         & 256 & 24.65 & 37.50 & 30.12 & 23.60 & 12.93 & 27.53 & 20.11 & 26.59 & 65.34 & 95.00 & 49.70 & 45.27 & 38.20 \\
      & 1024 & 38.47 & 49.32 & 47.43 & 38.48 & 20.59 & 33.71 & 23.67 & 26.60 & 81.94 & 99.50 & 60.78 & 52.96 & 47.79 \\
       & 4096 & 43.97 & 53.64 & 51.94 & 42.54 & 24.00 & 34.34 & 24.36 & 26.75 & 90.96 & 99.50 & 62.03 & 55.49 & 50.79 \\
        & 8192 &\textbf{44.34} & 53.25 & 54.72 & 44.84 & \textbf{25.98} & 34.62 & 24.98 & 26.70 & 91.61 & \textbf{100.00} & 62.02 & 54.20 & 51.44 \\
         & \textbf{Twilight (Avg. 427)} & 43.44 & 53.2 & 53.77 & 43.56 & 25.42 & 34.39 & 25.23 & 26.99 & 91.25 & 100.0 & 63.55 & 58.06 & 51.57\textcolor{teal}{(+0.3\%)} \\
        \midrule
    \multirow{5}*{DS}
         & 256 & 38.24 & 49.58 & 43.38 & 31.98 & 15.52 & 33.40 & 24.06 & 26.86 & 84.41 & 99.50 & 53.28 & 48.64 & 45.74 \\
      & 1024 & 42.97 & \textbf{54.65} & 51.75 & 33.92 & 20.39 & 34.50 & 24.92 & 26.71 & \textbf{92.81} & 99.50 & 62.66 & 48.37 & 49.43 \\
       & 4096 & 43.50 & 53.17 & 54.21 & 44.70 & 23.14 & \textbf{34.73} & 25.40 & 26.71 & 92.78 & 99.50 & 62.59 & 51.31 & 50.98 \\
        & 8192 & 43.82 & 53.71 & 54.19 & \textbf{45.13} & 23.72 & 34.27 & 24.98 & 26.69 & 91.61 & \textbf{100.00} & 62.40 & 52.87 & 51.14 \\
             & \textbf{Twilight (Avg. 446)} & 43.08 & 52.89 & \textbf{54.68} & 44.86 & 24.88 & 34.09 & 25.20 & \textbf{27.00} & 91.20 & \textbf{100.00} & 63.95 & \textbf{58.93} & \textbf{51.73}\textcolor{teal}{(+1.2\%)} \\
\bottomrule
\end{tabular}
}
\end{table*}

\end{document}
\endinput

%%
%% End of file `sample-sigconf.tex'.
