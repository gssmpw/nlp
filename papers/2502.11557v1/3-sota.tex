\section{The Basic Framework: \texttt{McSplit}}
\label{sec:sota}

% We first build necessary background of a basic framework called \texttt{McSplit}~\cite{mccreesh2017partitioning} for finding the maximum common subgraph. 
{\cheng \noindent\textbf{Overview.} \texttt{McSplit},  a \emph{backtracking} (aka \textit{branch-and-bound}) algorithm, \eat{also known as \emph{branch-and-bound} algorithm,} has been widely adopted for the MCS problem in recent years and has achieved the state-of-the-art performance in practice~\cite{bai2021glsearch,liu2020learning,zhoustrengthened,liu2023hybrid}.}
%
The \laks{key idea} of \texttt{McSplit} is to recursively expand a partial solution $S$ (which is {\cheng the largest} common subgraph {\cheng seen so far}) via a process of \emph{branching}. Specifically, the branching process partitions the current problem instance of finding the maximum common subgraph into several subproblem instances. Each problem instance {\cheng corresponds} to a \emph{branch} {\cheng and} is denoted by $(S,C)$. Here, $S$ is a \emph{partial solution} {\Yui (i.e., set of vertex pairs)} and $C$ is the \emph{candidate set} consisting of \emph{candidate pairs} (i.e., $\langle u, v\rangle$) used to expand the partial solution $S$. Solving the instance or branch $(S,C)$ means finding the largest common subgraph $S^*$ in the branch; a common subgraph is said to be in a branch $(S,C)$ if and only if \emph{it contains $S$ and is within the set $S\cup C$, {\Yui i.e., $S\subseteq S^*\subseteq S\cup C$}}.
%{\cheng which contains} $S$ and {\cheng is} within the set $S\cup C$.
{\Yui Given two vertex subsets $V_q\subseteq V_Q$ and $V_g\subseteq V_G$, we consider all pairs of   vertices from $V_q$ and $V_g$, i.e.,
$V_q\times V_g = \{\langle u,v \rangle\mid u\in V_q, v\in V_g\}$.
%
%Let $\langle V_Q\times V_G \rangle$ denote the set of all possible vertex pairs (i.e., $\langle V_Q\times V_G \rangle=\{\langle u,v \rangle\mid u\in V_Q, v\in V_G\}$). 
%
Note that solving the initial branch $(\emptyset,V_Q\times V_G)$ finds the largest common subgraph of $Q$ and $G$.}
%, where $\langle V_Q\times V_G \rangle$ denotes the set of all possible vertex pairs (i.e., $\langle V_Q\times V_G \rangle=\{\langle u,v \rangle\mid u\in V_Q, v\in V_G\}$).

To solve a branch $(S,C)$, the branching process selects a vertex $u$ {\cheng appearing} in $C$ as a \emph{branching vertex},  and then creates two groups of new sub-branches by either including $u$ into the solution \eat{(this corresponds to the first one)} or discarding $u$ from the candidate set {\Yui and thus also from the solution}. \eat{(this corresponds to the second one).} Specifically, \textbf{in the first group}, each formed branch includes into $S$ one candidate pair containing $u$ and excludes from $C$ all \laks{pairs} containing $u$ (note that a common subgraph has each vertex {\cheng appearing} in at most one pair); consequently, for each candidate pair that contains $u$, i.e., $\langle u,v \rangle$, \laks{we form a new branch corresponding to} $(S\cup\{\langle u,v \rangle\}, C\backslash u\backslash v)$,  
%
%{\kaixin $(S\cup\{\langle u,v \rangle\}, C\setminus (\{\langle u, \cdot \rangle\}\cup \{\langle \cdot, v \rangle\}))$}
%
where $C\backslash u\backslash v$ denotes the set obtained by removing from $C$ all  candidate pairs  containing $u$ or $v$, formally,
\begin{equation}
    C\backslash u\backslash v:= C\backslash ((\{u\}\times V_g) \cup ( V_q\times\{v\})).
\end{equation}
%
\textbf{In the second group}, we form only one branch by excluding from $C$ all candidate pairs containing $u$, {\cheng i.e.,} $(S,C\backslash u)$. Clearly, \eat{solving all the formed branches solves} the solution to  $(S,C)$ \eat{(since its solution} \laks{is the largest one among those found {\cheng from} the branches above.} We illustrate this next. 
%
\laks{
\begin{example}
\label{ex:split} 
Consider \eat{an example of {\cheng the} {\chengB branching} process on} the given pair of  input graphs in  Figure~\ref{fig:Input_graph}.  The splitting process is partially depicted in Figure~\ref{fig:example_branching} (\laks{ignore the ``$D$'' terms in the figure for now)}). For the initial branch $B_0=(\emptyset,\{u_1,u_2,...,u_7\}\times \{v_1,v_2,...,v_7\} )$, McSplit  selects the branching vertex $u_1$, and then creates the first group of branches  $B_i=(\{\langle u_1,v_i \rangle\},\{u_2,...,u_7\}\times (\{v_1,v_2,...,v_7\}\backslash \{v_i\}))$ for $1\leq i\leq 7$, each of which includes one candidate pair $\langle u_1,v_i \rangle$ into the solution, and the second group of a single branch, namely $B_8=(\emptyset,\{u_2,...,u_7\}\times \{v_1,v_2,...,v_7\})$, which excludes $u_1$ from the solution.
\end{example} 
} 


To improve the efficiency, \laks{McSplit}  further applies a \emph{reduction rule} and an \emph{upper-bound-based pruning} rule for a newly formed branch $(S,C)$. Specifically, the reduction rule {\cheng removes} from the candidate set $C$ those candidate pairs $\langle u,v\rangle$ that cannot form a common subgraph with $S$, i.e., $S\cup \{\langle u,v \rangle$\} cannot be a common subgraph, thus narrowing the search space: note that any supergraph of a non-common subgraph cannot be a common subgraph and thus we can remove them safely. The upper-bound-based pruning rule {\cheng computes} an upper bound on the \laks{size of the} largest common subgraph in the branch and {\cheng prunes} the branch if the upper bound is no larger than the \laks{size of the largest common subgraph found so far} {\YuiR (details will be discussed in Section~\ref{subsec:upper-bound})}. 
{\chengB Below, \laks{we formally state the}  reduction rule.}
%
\begin{figure}[]
		\includegraphics[width=0.45\textwidth]{figure/example_branching_v2.pdf}
  \vspace{-0.15in}
	\caption{Illustrating the backtracking process (``+'' means to {\chengB move} vertex pairs from $C$ to $S$ and ``-'' means to remove vertex pairs from $C$)}
 \vspace{-0.2in}
	\label{fig:example_branching}
\end{figure}


{\Yui
\noindent\textbf{Reduction {\chengC rule}.} \emph{Consider a branch $(S,C)$. A candidate pair $\langle u,v \rangle$ in $C$ cannot form any common subgraph with $S$ if there exists a vertex pair $\langle u',v' \rangle$ in $S$ such that $u$ and $v$ are not simultaneously adjacent or non-adjacent to $u'$ and $v'$, respectively.} 

\laks{The soundness of this rule} \eat{can be verified based on} \laks{immediately follows from}  Definition~\ref{def:CIS}. 
%
%
{
%Based on the partial set $S$, the refined candidate set $C$ can be split as several subsets, i.e.,
%\begin{equation}
%    C=X_1\times Y_1 \cup ... \cup X_c \times Y_c,
%\end{equation}
%where $c$ is a positive integer and every vertex pair $\langle u,v \rangle$ in $X_i\times Y_i$ $(1\leq i\leq c)$ 
%
We note that the above reduction rule can be applied in a recursive way and \emph{the refined candidate set $C$ can be split as several subsets, i.e., $C=X_1\times Y_1 \cup ... \cup X_c \times Y_c$ where $c$ is a positive integer and $X_i$ and $X_j$ (resp. $Y_i$ and $Y_j$) are disjoint}. We show this as follows.
%
More precisely, starting from the basis case of the initial branch $B_0$ with $S_0=\emptyset$ and $C_0= V_Q\times V_G $, none of the candidate pairs in $C_0$ can be pruned by the reduction rule since $S_0$ is empty. Then, consider the recursive case of a branch $B=(S,C)$ with $C=X_1\times Y_1 \cup ... \cup X_c \cup Y_c $ where $c$ is a positive integer and $X_i$ and $X_j$ (resp. $Y_i$ and $Y_j$) are disjoint for $1\leq i\leq c$. For one sub-branch $(S\cup\{\langle u,v\rangle\},C\backslash u\backslash v)$ formed in the first group by including $\langle u,v\rangle$ to $S$, the candidate set $C\backslash u \backslash v$ can be refined as
\begin{eqnarray}
    \label{eq:update_candidate_set}
    \bigcup_{i=1}^c  N(u,X_i)\times N(v,Y_i)  \cup  \overline{N}(u,X_i\backslash \{u\})\times \overline{N}(v, Y_i\backslash\{v\}). 
   % \{N(u,X_i)\times N(v,Y_i) \mid 1\leq i \leq c\} \cup \nonumber\\ \{\overline{N}(u,X_i\backslash \{u\})\times \overline{N}(v, Y_i\backslash\{v\}) \mid 1\leq i \leq c\}
\end{eqnarray}
since those vertex pairs in $N(u,X_i)\times \overline{N}(v, Y_i\backslash\{v\})  \cup  \overline{N}(u,X_i\backslash \{u\})\times N(v,Y_i)$ can be pruned by the above reduction. Clearly, $N(u,X_i)$ and $\overline{N}(u,X_i\backslash \{u\})$ (resp. $N(v,Y_i)$ and $\overline{N}(v, Y_i\backslash\{v\}$) are disjoint for $1\leq i\leq c$. For one sub-branch $(S,C\backslash u)$ formed in the second group, none of the candidate pairs in $C\backslash u$ can be pruned by the reduction rule since $S$ remains unchanged. Suppose $u\in X_i$, then the refined candidate set $C\backslash u$ can be represented by $X_1\times Y_1\cup ... \cup (X_i\backslash\{u\})\times Y_i \cup ...\cup X_c\times Y_c$ where any two subsets are disjoint as well.

We remark that all candidate sets $C$ mentioned in the following sections refer to the ones refined by the reduction rule and thus can be represented by $C=X_1\times Y_1\cup ...\cup X_c\times Y_c$, \laks{for some $c$.} Given this, we define
\begin{equation}
    \mathcal{P}(C)=\{X_i\times Y_i \mid 1\leq i \leq c\}.
\end{equation}
%We assume that $X_i$ and $X_j$ (resp. $Y_i$ and $Y_j$) are disjoint, which holds for the base case and   

%Consider an immediate sub-branch of $B_0$ which is formed by including candidate pair $\langle u,v \rangle$ into the partial solution. Those candidate pairs in $ N(u,V_Q)\times \overline{N}(v,V_G)$ and those in $\overline{N}(u,V_Q)\times N(v,V_G)$ can be pruned by the reduction rule. As a result, the refined candidate set $C\backslash u\backslash v$ is $ N(u,V_Q)\times N(v,V_G) \cup  \overline{N}(u,V_Q\backslash\{u\})\times \overline{N}(v,V_G\backslash\{v\}) $, which is \emph{split} as two subsets. Consider an immediate sub-branch of $B_0$ which is formed by excluding $u$, 
}

%
\if 0
We note that the above reduction rule can be applied in a \emph{recursive} way. More precisely, for the initial branch $B_0$ with $S_0=\emptyset$ and $C_0= V_Q\times V_G $, \laks{none of the candidate pairs in $C_0$ can  be pruned} by the reduction rule since $S_0$ is empty. Consider an immediate sub-branch of $B_0$ which is formed by including candidate pair $\langle u,v \rangle$ into the partial solution. Those candidate pairs in $ N(u,V_Q)\times \overline{N}(v,V_G)$ and \laks{those in} $\overline{N}(u,V_Q)\times N(v,V_G)$ can be pruned by the reduction rule. As a result, the refined candidate set is $ N(u,V_Q)\times N(v,V_G) \cup  \overline{N}(u,V_Q\backslash\{u\})\times \overline{N}(v,V_G\backslash\{v\}) $, which is \emph{split} as two subsets. 
%
In general, for a branch $(S,C)$, the refined candidate set $C$ consists of at most $2^{|S|}$ disjoint subsets, i.e., $C=X_1\times X_2\cup\cdots\cup X_c\times Y_c$ where $1\leq c\leq 2^{|S|}$. For a sub-branch of $(S,C)$ which is formed by {\chengB moving} a candidate pair $\langle u,v \rangle$ from $C$ to $S$, i.e., $(S\cup\{\langle u,v \rangle\},C\backslash u \backslash v,D)$, the candidate set $C\backslash u \backslash v$ can be refined as
%i.e., $C=X_1\times Y_1 \cup\cdots\cup X_c\times Y_c$ where $1\leq c\leq 2^{|S|}$. For a sub-branch of $(S,C)$ which is formed by {\chengB moving} a candidate pair $\langle u,v \rangle$ from $C$ to $S$, the refined candidate set is
\begin{eqnarray}
    \label{eq:update_candidate_set}
    \bigcup_{i=1}^c  N(u,X_i)\times N(v,Y_i)  \cup  \overline{N}(u,X_i\backslash \{u\})\times \overline{N}(v, Y_i\backslash\{v\}). 
   % \{N(u,X_i)\times N(v,Y_i) \mid 1\leq i \leq c\} \cup \nonumber\\ \{\overline{N}(u,X_i\backslash \{u\})\times \overline{N}(v, Y_i\backslash\{v\}) \mid 1\leq i \leq c\}
\end{eqnarray}
%Besides, for a subgraph of $(S,C)$ which is formed by excluding vertex $u$ in the second group, i.e., $(S,C\backslash u)$, we have
%\begin{equation}
%    C\backslash u:= 
%\end{equation}
\fi
\begin{example}
    \label{exp:branching}
   Consider branch $B_6$ in the first group in Figure~\ref{fig:example_branching}. We have $X_1=N(u_1,V_Q)=\{u_2,u_3\}$, $X_2=\overline{N}(u_1,V_Q\backslash\{u_1\})=\{u_4,u_5,u_6,u_7\}$, $Y_1=\{v_4,v_5\}$ and $Y_2=\{v_1,v_2,v_3,v_7\}$. Therefore, the candidate set becomes $\{u_2,u_3\}\times \{v_4,v_5\}\cup\{u_4,u_5,u_6,u_7\}\times \{v_1,v_2,v_3,v_7\}$. Then, consider a sub-branch of $B_6$ formed by further including $\langle u_7,v_7 \rangle$. We can deduce the candidate set by splitting $X_1\times Y_1 $ to $\{u_3\}\times \{v_4\} \cup \{u_2\}\times \{v_5\} $ and splitting $ X_2\times Y_2$ to $\{u_5\}\times \{v_1\}\cup  \{u_4,u_6\}\times \{v_2,v_3\}$.
\end{example}
}





\noindent\textbf{\laks{Algorithm Outline.}} We summarize the details of \texttt{McSplit} in Algorithm~\ref{alg:mcsplit}. It maintains the currently found largest common subgraph $S^*$ (Line 4) and terminates the branch by upper-bound-based pruning (Line 5). Besides, it branches by selecting (from $\mathcal{P}(C)$) a vertex $u$ and the corresponding subset $X\times Y$, called \emph{branching subset}, that $u$ belongs to (Line 6, note that all candidate pairs {\cheng containing} $u$ are within $X\times Y$), and creates two groups of branches as discussed before (Lines 8-12). 
%
For the first group, the ordering of formed branches depends on that of the candidate pairs to be included into $S$, which is specified by a policy (Line 9).
%
We note that \texttt{McSplit} \textit{adopts heuristic policies for selecting $X\times Y$, $u$, and $v$}. %Specifically, it {\Yui first selects from $C$ one subset} $\langle X\times Y \rangle$ with the smallest value of $|X|\times |Y|$, {\Yui then selects from $X$ one vertex} $u$ with the smallest vertex ID, {\Yui and finally iteratively selects from $Y$ each vertex $v$ with the smallest vertex ID for forming a sub-branch $(S\cup\{\langle u,v \rangle\}, C\backslash u\backslash v)$} (note that each vertex in $Q$ or $G$ is assigned an unique integral ID, i.e., $\{0,1,...,|V_Q|-1\}$ for $Q$ and $\{0,1,...,|V_G|-1\}$ for $G$). Therefore, $u$ and $v$ {\cheng are} selected based on the given orderings of vertices in $Q$ and $G$, which are fixed during the recursion.

\begin{algorithm}[t]
\small
\caption{An existing framework: \texttt{McSplit}~\cite{mccreesh2017partitioning}}
\label{alg:mcsplit}
\KwIn{Two graphs $Q=(V_Q,E_Q)$ and $G=(V_G,E_G)$}
\KwOut{The maximum common subgraph}
$S^*\leftarrow \emptyset$; \tcp{Global variable}
\texttt{McSplit-Rec}$(\emptyset,V_Q\times V_G)$; \textbf{Return} $S^*$;\\
%\setcounter{AlgoLine}{0}
\SetKwBlock{Enum}{Procedure \texttt{McSplit-Rec}$(S,C)$}{}
%\SetKwBlock{update}{Procedure \texttt{Update}$(S,C)$}{}
\Enum{
    \lIf{$|S|>|S^*|$}{$S^*\leftarrow S$}
    \tcc{Termination}
    \lIf{$C=\emptyset$ or the upper bound is no larger than $|S^*|$}{\textbf{return}}
    \tcc{Branching}
    {\Yui Select a branching vertex $u$ and a branching subset $X\times Y$ from $\mathcal{P}(C)$  based on a policy\;}
    $Y_{temp}\leftarrow Y$\;
    \tcc{First group: branches formed by including $u$}
    \For{$i=1,2,...,|Y|$}{
        Select and {\chengB move} a vertex $v$ from $Y_{temp}$ based on a policy\;
        Create a candidate set $C_i$ based on $\langle u,v\rangle$ and Equation~(\ref{eq:update_candidate_set})\;
        \texttt{McSplit-Rec}($S\cup\{\langle u,v\rangle\},C_i$)\;
    }
    \tcc{Second group: one branch formed by excluding $u$}
    \texttt{McSplit-Rec}($S,C\backslash u$)\;
}
\end{algorithm}

% \smallskip
% \noindent\textbf{Variants of \texttt{McSplit}.} Quite a few studies adopt \texttt{McSplit} for solving the MCS problem, and 

{\cheng Existing algorithms that are based on \texttt{McSplit} differ in the strategies of optimizing} the policies of selecting vertices in  line 6 and line 9 (e.g., via some learning-based techniques) to find the largest  \laks{common}  subgraph as soon as possible during the recursion~\cite{zhoustrengthened,liu2020learning,liu2023hybrid}. 
%
{\Yui However, these algorithms (1) provide no theoretical guarantee on the worst-case time complexity and (2) still suffer from efficiency issues in practice due to  significant redundant computations.}
%
%We note that the learned policies dynamically select the next vertex at line 6 and line 9 according to the running-time contexts.
%
%The rationale is that the earlier the largest common subgraph is found, the more branches will be pruned by the upper-bound-based pruning (Line 5).

