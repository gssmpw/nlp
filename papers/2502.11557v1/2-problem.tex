\section{Preliminaries}
\label{sec:preli}
In this paper, we focus on undirected and unweighted simple {\cheng graphs} without self-loops and parallel edges. {\cheng For ease of presentation, we focus on  graphs without vertex labels, but} 
% We note that 
our methods \eat{{\cheng to be introduced}} can be easily adapted to vertex-labeled graphs. Consider two graphs $Q=(V_Q,E_Q)$ and $G=(V_G,E_G)$\eat{, with vertex sets $V_Q$ and $V_G$ and edge sets $E_Q$ and $E_G$}. For simplicity, we let $u$ and $v$ (and their primed or index variants) denote a vertex in $Q$ and $G$ respectively. Given a vertex set $X\subseteq V_Q$, we use $Q[X]$ to denote the subgraph of $Q$ induced by $X$, i.e., $Q[X]=(X,\{(u,u')\in E_Q \mid u, u'\in X\})$. All subgraphs in this paper are induced subgraphs. We let $q = (V_q, E_q)$ denote an arbitrary induced subgraph of $Q$. 
%
Given $u\in V_Q$, we denote by $N(u,V_Q)$ (resp. $\overline{N}(u,V_Q)$) the set of neighbours (resp. non-neighbours) of $u$ in $Q[V_Q]$.  \eat{We have the symmetric definitions for a vertex set $Y$ and each vertex $v$ in $G$.} \laks{We use a similar notation for neighbours and non-neighbours of vertices in $G$.}  

We  review the definition of graph isomorphism for  simple graphs without labels.

\begin{definition}[Graph isomorphism~\cite{mcgregor1982backtrack}]
    \label{def:GS}
    $Q$ is said to be isomorphic to $G$ if and only if there exists a \textbf{bijection} $\phi: V_Q\rightarrow V_G$ such that
    \begin{equation}
    \label{eq:isomorphic}
        \forall u, u' \in V_Q: (u,u')\in E_Q \iff (\phi(u),\phi(u'))\in E_G.
    \end{equation}
\end{definition}

\laks{Note that} two isomorphic graphs are structurally equivalent, {\chengC and thus we have} $|V_Q|=|V_G|$ and $|E_Q|=|E_G|$. We next review induced subgraph isomorphism for  graphs. 

\begin{definition}[Induced subgraph isomorphism~\cite{mcgregor1982backtrack}]
    $Q$ is said to be (induced) subgraph isomorphic to $G$ if and only if there exists an \textbf{injection} $\phi: V_Q\rightarrow V_G$ such that
    \begin{equation}
    %\label{eq:isomorphic}
        \forall u, u' \in V_Q: (u,u')\in E_Q {\YuiR \iff} (\phi(u),\phi(u'))\in E_G. %\Longrightarrow 
    \end{equation}
    %that satisfies Equation~(\ref{eq:isomorphic}).
\end{definition}

Notice that  induced subgraph isomorphism is a special case of  graph isomorphism. The injection mapping $\phi: V_Q\rightarrow V_G$ is also known as \emph{embedding} of $Q$ into $G$, {\chengC and thus we have} $|V_Q|\leq |V_G|$ and $|E_Q|\leq |E_G|$. The subgraph matching problem aims to find all embeddings of a small query graph $Q$ in a large data graph $G$. The common induced subgraph is defined as follows. 

\begin{definition}[Common induced subgraph~\cite{mcgregor1982backtrack}]
    \label{def:CIS}
    A common subgraph of $Q$ and $G$, {\cheng denoted by $\langle q,g,\phi \rangle$, is defined as} a \eat{set of vertex pairs} \laks{triple   consisting of an induced subgraph $q$ of $Q$, an induced subgraph $g$ of $G$, and a bijection $\phi: V_q\rightarrow V_g$, such that $q$ is isomorphic to $g$ under   $\phi$.} %{\cheng Formally, we have}
    %\begin{equation}
    %    \langle q,g,\phi \rangle :=\{\langle u,\phi(u) \rangle \mid \forall u\in V_q\}.
   % \end{equation}
\end{definition}

%A common subgraph $\langle q,g,\phi \rangle$ is said to be contained in (or a subgraph of) another common subgraph $\langle q',g',\phi' \rangle$ if and only if $\{\langle u,\phi(u) \rangle \mid \forall u\in V_q\}\subseteq \{\langle u,\phi'(u) \rangle \mid \forall u\in V_{q'}\}$.
By the size of a common subgraph $\langle q, g,\phi\rangle$ we mean the number of {\YuiR vertices in $q$ or $g$}. %vertex pairs. 
%
Clearly, the size of a common subgraph is at most $\min\{|V_Q|,|V_G|\}$. 
%
{\YuiR Sometimes, for  ease of presentation, we represent a common subgraph $\langle q,g,\phi \rangle$ by a set of vertex pairs $\{\langle u,\phi(u) \rangle \mid  u\in V_q\}$.}

\begin{example}
Consider the input graphs in Figure~\ref{fig:Input_graph}. The graphs $q := Q[u_1,u_2,u_3,u_4,u_7]$ and  $g := G[v_3,v_4,v_5,v_6,v_7]$ form a common subgraph with size 5 under the bijection $\phi:=\{u_1\!\rightarrow\! v_5,u_2\!\rightarrow\! v_6, u_3\!\rightarrow\! v_4, u_4\!\rightarrow\! v_3, u_7\!\rightarrow\! v_7\}$.%, which can be represented by a set of vertex pairs, i.e., $\{\langle u_1,v_5 \rangle,\langle u_2,v_6 \rangle,\langle u_3,v_4\rangle,\langle u_4,v_3\rangle,\langle u_7,v_7\rangle\}$.
\end{example}
%
We are ready to formulate the problem studied in this paper.


\begin{problem}[Maximum Common Subgraph~\cite{lewis1983michael}]
    Given two graphs $Q$ and $G$, the Maximum Common Subgraph (MCS) problem aims to find the maximum common subgraph {\cheng of $Q$ and $G$}, i.e., a common subgraph  with the largest number of vertices.
\end{problem}

%We use MaxCS as a shorthand of the maximum common subgraph thoughout this paper.
\eat{If {\chengB we further require} the size of found maximum common subgraph {\cheng to be} at least $|V_Q|$, then the MCS problem would reduce to  subgraph matching (i.e., {\cheng it finds} one embedding of $Q$ in $G$). Therefore, the MCS problem is a natural generalization of the subgraph matching problem. } 
\laks{Note that the MCS problem is a generalization of subgraph matching: there is a MCS between $Q$ and $G$ whose size is $|Q|$ iff $Q$ is isomorphic to a subgraph of $G$. It is well known that  MCS  is NP-hard~\cite{lewis1983michael} and is hard to approximate, i.e., {\YuiR there is no $r$-approximate PTIME algorithm for the problem for any $r\geq 1$~\cite{kann1992approximability}, \laks{unless P=NP}.} %$O(n^{\epsilon})$-approximate PTIME algorithm for the problem for any $\epsilon>0$, where $n$ is the input instance size~\cite{kann1992approximability}. 
} 

\eat{ 
\smallskip
\noindent\textbf{Hardness.} We remark that the problem of finding the maximum common subgraph is NP-hard~\cite{lewis1983michael}. Besides, 
% it is shown to be 
{\cheng the problem is}
hard to approximate, {\cheng e.g.,} it does not admit any $O(n^{\epsilon})$-approximate algorithm that runs in polynomial time (unless P=NP), where $\epsilon>0$ and $n$ is the size of an input instance~\cite{kann1992approximability}.  
} 

\begin{figure}[]
		\subfigure[\textsf{Input graph $Q$}]{
			\includegraphics[width=2.7cm]{figure/Input_graph_Q.pdf}
		}
        \hspace{0.4in}
		\subfigure[\textsf{Input graph $G$}]{
			\includegraphics[width=2.7cm]{figure/Input_graph_G.pdf}
		}
  \vspace{-0.2in}
	\caption{Input graphs used throughout the paper}
	\label{fig:Input_graph}
 \vspace{-0.2in}
\end{figure}