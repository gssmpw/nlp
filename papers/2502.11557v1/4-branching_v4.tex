
\section{Redundancy-Reduced Splitting: \texttt{RRSplit}}
\label{sec:RRSplit}


In this part, we present our backtracking algorithm called \texttt{RRSplit}. 
%
First, we propose a vertex-equivalence-based reduction for pruning those redundant branches that hold all common subgraphs cs-isomorphic to {\chengC one that has been already found} (Section~\ref{subsec:VE-reduction}).
%
Second, we introduce a newly-designed maximality-based reduction for pruning those redundant branches that hold only non-maximal common subgraphs (Section~\ref{subsec:maximality-reduction}). 
%
{\YuiR Third, we develop a new vertex-equivalence-based upper bound on the size of common subgraphs that can be found in a branch for further pruning those branches that hold only small common subgraphs (Section~\ref{subsec:upper-bound}).}
%
\laks{Finally, we summarize the \texttt{RRSplit} algorithm, which is based on the above \eat{carefully-designed} reductions, and analyze its worst-case time complexity (Section~\ref{subsec:summary}). In particular, we show \texttt{RRSplit} has a worst-case time complexity of $O^*((|V_G|+1)^{|V_Q|})$, matching the best-known worst-case time complexity of the state of the art. We will later show (Section~\ref{sec:exp}) that unlike the state of the art,  \texttt{RRSplit} is very efficient in practice.} 



\subsection{Vertex-Equivalence-based Reduction}
\label{subsec:VE-reduction}
We first introduce the concept of \emph{common subgraph isomorphism (cs-isomorphism)}.
\begin{definition}[Common subgraph isomorphism] \laks{Consider two common subgraphs $\langle q,g,\phi \rangle$ and $\langle q',g',\phi' \rangle$ of graphs $G$ and $Q$. They are} 
    \eat{$\langle q,g,\phi \rangle$ is } said to be common subgraph isomorphic (cs-isomorphic) \eat{to $\langle q',g',\phi' \rangle$} if and only if $q$ is {\chengC isomorphic} to $q'$ (or equiv., $g$ is {\chengC isomorphic} to $g'$).
\end{definition}

\laks{All cs-isomorphic common subgraphs evidently  share the same structural information, and exploring all of them is clearly redundant.}  
% so is redundant for the purpose of 
\eat{{\chengC thus it would introduce redundancy if we explore both of them for}
finding the largest common subgraph.} \laks{We reduce this redundancy as follows.}  
For a common subgraph $\langle q,g,\phi \rangle$ to be found in a branch, we can safely {\chengC ignore the common subgraph $\langle q,g,\phi \rangle$}, {\revision if there exists another one $\langle q',g',\phi' \rangle$ that satisfies the following two conditions:
\begin{itemize}
    \item \textbf{Condition 1}: $\langle q',g',\phi' \rangle$ is cs-isomorphic to $\langle q,g,\phi \rangle$;
    \item  \textbf{Condition 2}: $\langle q',g',\phi' \rangle$ has been found before.
\end{itemize}}
%if there exists another one that is cs-isomorphic to $\langle q,g,\phi \rangle$ {\chengC (Condition 1)} and has been found before {\chengC (Condition 2)},  

\laks{To facilitate the reduction, we make use of Condition 1 and Condition 2, for which we will leverage the \emph{vertex equivalence} {\chengC property} and an \emph{auxiliary data structure} {\chengC respectively}.} 

\smallskip
\noindent\textbf{Vertex equivalence}. 
% We start with an important graph property, namely 
{\chengC The \emph{structural equivalence} property}
% , which 
has been widely used to speed up subgraph matching tasks~\cite{nguyen2019applications,yang2023structural,kim2021versatile}. Conceptually, two vertices are  \emph{structurally equivalent}  if and only if they have the \emph{same} set of neighbours. 
% Formally, we have: 
{\chengC Formally, }

\begin{definition}[Structural equivalence~\cite{nguyen2019applications}]
    %\label{def:structural_eqv}
    Two vertices $u$ and $v$ in $Q$ are structurally equivalent, denoted  $u\sim v$, if and only if  
    \begin{equation}
        \forall u'\in V_Q, (u,u')\in E_Q \Leftrightarrow (v,u')\in E_Q. 
    \end{equation}
\end{definition}

Clearly, structural equivalence is an equivalence relation. Therefore, we can partition the vertices of graph $Q$ into equivalence classes, with the equivalence class of vertex $u\in V_Q$ defined as 
\begin{equation}
    \Psi(u)  := \{u'\in V_Q \mid u'\sim u\},
\end{equation}
where $u\in V_Q$ is a representative of class $\Psi(u)$.
%
We remark that this process can be done in $O(|V_Q|\delta_Q d_Q)$ {\chengB time} where $\delta_Q$ and $d_Q$ are the degeneracy and the maximum degree of the graph $Q$, respectively~\cite{nguyen2019applications,yang2023structural,kim2021versatile}. 


\begin{figure}[]
		\includegraphics[width=0.45\textwidth]{figure/example_concepts.pdf}
	\caption{Illustrating cs-isomorphism and vertex equivalence (vertices, denoted by colored bullet circles, induce subgraphs $q$, $q_1$, $q_2$ and $g$; vertices in $\{u_1,u_2\}$ and $\{u_4,u_6,u_7\}$ are structurally equivalent, respectively; $\langle q,g,\phi \rangle$ is cs-isomorphic to $\langle q_1,g,\phi_1 \rangle$ (Case 1, say, exchange the mapped vertices of $u_1$ and $u_2$) and $\langle q_2,g,\phi_2 \rangle$  (Case 2, say, replace $u_4$ with $u_7$) where vertices with the same color indicate the bijection)}
 \label{fig:example_self_iso}
\end{figure}

Based on vertex equivalence, we can \laks{identify} several common subgraphs \laks{that are} cs-isomorphic to a given one $\langle q,g,\phi \rangle$ by swapping one vertex in $V_q$ with its structurally equivalent counterpart, which falls into  two cases.
%
{\YuiR In specific, consider a vertex $u$ in $V_q$ and one of its structurally equivalent counterparts  $u_{equ}$ in $\Psi(u)$. {\cheng We can obtain a cs-isomorphic common subgraph in two cases.} If $u_{equ}$ is also in $V_q$, {\cheng we can exchange} the mapped vertices of $u_{equ}$ and $u$, i.e., {\cheng we replace} $\langle u,\phi(u) \rangle$ and $\langle u_{equ},\phi(u_{equ}) \rangle$ with $\langle u,\phi(u_{equ}) \rangle$ and $\langle u_{equ},\phi(u) \rangle$; Otherwise, we replace $u$ with $u_{equ}$, i.e., replacing $\langle u,\phi(u)\rangle$ with $\langle u_{equ},\phi(u) \rangle$.}
%
Formally, we have the following lemma, {\cheng which can be easily verified} (see the examples in Figure~\ref{fig:example_self_iso} for a visual illustration of the lemma).
\begin{lemma}
    \label{lemma:cs}
    Let $S=\langle q,g\eat{p},\phi \rangle$ be a common subgraph of given graphs $Q$ and $G$, $u$ be a vertex in $V_q$ and $u'$ be a vertex in $\Psi(u)$. Then {\kaixin one of the following cases holds}. 
    %
    \begin{itemize}[leftmargin=*]
        \item[]\textbf{Case 1: $u'\in V_q$}. $S'=S\backslash \{\langle u,\phi(u) \rangle,\langle u',\phi(u') \rangle \}\cup \{\langle u,\phi(u') \rangle,\langle u',\\\phi(u) \rangle\}$ is a common subgraph cs-isomorphic to $S$. \;\; {\kaixin\eat{OR}} 
        \item[]\textbf{Case 2: $u'\notin V_q$}. $S'=S\backslash \{\langle u,\phi(u) \rangle\}\cup \{\langle u',\phi(u) \rangle\}$ is a common subgraph cs-isomorphic to $S$.
    \end{itemize}
\end{lemma}

\smallskip
\noindent\textbf{Auxiliary data structure}. 
% Though a few cs-isomorphic common subgraphs can be constructed based on Lemma~\ref{lemma:cs} for verifying Condition 1, none of them may have been found before and thus Condition 2 fails to satisfy. 
{\chengC To facilitate the verification of Condition 2, i.e., whether a common subgraph that is cs-isomorphic to a current one has been found before}, we introduce a new data structure, namely exclusion set (denoted by $D$).
{\chengC $D$} is recursively maintained for each branch, and thus each branch is denoted by $(S,C,D)$. Specifically, $D$ is a set of vertex pairs that have been considered for expanding the partial solution and must not be included in any common subgraphs within the branch. Formally, the exclusion set is maintained as follows (\laks{illustrated in Figure~\ref{fig:example_branching} -- see the ``$D$'' terms now!}). 
\begin{itemize}[leftmargin=*]
    \item \textbf{Initialization}. The exclusion set is initialized to {\kaixin be} empty at the initial branch, i.e., $(\emptyset, V_Q\times V_G, \emptyset)$.
    \item \textbf{Recursive update}. Consider {\chengC the} branching at a branch $(S,C,D)$. For the first group where the $i^{th}$ sub-branch $(S_i,C_i,D_i)$ is formed by including $\langle u,v_i \rangle$ into $S$, we update the exclusion set to $D_i=D\cup \{\langle u,v_1 \rangle,\langle u,v_2 \rangle,\cdots ,\langle u,v_{i-1} \rangle\}$. For the second group where one sub-branch $(S',C',D')$ is formed, we set $D'=D$.
\end{itemize}
%We note that, for a vertex pair $\langle u,v \rangle$ in exclusion set $D$, there exists some common subgraphs containing $\langle u,v \rangle$ that has been found before. This will help us to verify Condition 2.
Consider a branch $(S,C,D)$ and a vertex pair $\langle u',v' \rangle$ in the exclusion set $D$, as shown in Figure~\ref{fig:exclusion_set}. There exists an {\YuiR ancestor}\eat{ascendant branch} of $(S,C,D)$, denoted by $(S_{anc},C_{anc},D_{anc})$, where $u'$ is selected as the branching vertex. Clearly, $\langle u',v' \rangle$ is not in $D_{anc}$ and will be \laks{added} to $D_{anc}$ after $B'_{anc}$ {\chengC is formed}, \laks{i.e., more precisely $D'_{anc} = D_{anc}\cup\{(u',v')\}$}. Therefore, all common subgraphs within the sub-branch $B'_{anc}$, which must contain $\langle u',v' \rangle$, have been found before $(S,C,D)$. This will help us  verify Condition 2.
\begin{figure}[]
		\includegraphics[width=0.3\textwidth]{figure/Exclusion_set_v3.pdf}
  \vspace{-0.15in}
	\caption{Illustrating the exclusion set $D$ {\YuiR ($\langle u',v'\rangle$ is a vertex pair in $D$; $B_{anc}$ is an ancestor of $B$, where $u'$ is selected as the branching vertex)}}
 \vspace{-0.2in}
	\label{fig:exclusion_set}
\end{figure}


Based on  vertex equivalence and exclusion set, we are now ready to develop the reductions. Consider the branching process at a branch $(S=\langle q,g,\phi\rangle,C,D)$ where $X\times Y$ in $\mathcal{P}(C)$ and vertex $u$ in $X$ are selected as the branching subset and branching vertex, respectively. 
% In general, the reductions have two cases.
% {\chengC We consider }

\smallskip
\noindent\textbf{Reduction at the first group}. Consider a sub-branch formed at the first group by including one vertex pair $\langle u,v\rangle$ where $v\in Y$. We note that {\chengC \emph{each}} common subgraph $S_{sub}$ to be found in this sub-branch must include $\langle u,v \rangle$. 
%
We observe that if \emph{there exists a vertex pair $\langle u_{equ},v \rangle$ in $D$ such that $u_{equ}$ is structurally equivalent to $u$, i.e., $u_{equ}\in\Psi(u)$}, Conditions 1 \& 2 hold for $S_{sub}$ and thus the branch can be pruned. Below, we elaborate on the details. 
%

We first show that Condition 1 holds:  Clearly, $S$ contains a vertex pair $\langle u_{equ},\phi(u_{equ}) \rangle$ since otherwise $D$ will not include $\langle u_{equ},v \rangle$ according to the maintenance of $D$.
%
Therefore, we can construct the following common subgraph {\chengC $S_{iso}$, which is} cs-isomorphic to $S_{sub}$ based on Case 1 of Lemma~\ref{lemma:cs} {\chengC (essentially, we exchange the mapped vertices of $u_{equ}$ and $u$)}.
\begin{equation}    
\label{eq:iso1}
S_{iso}\!=\!S_{sub}\backslash\{{\chengC \langle u,v\rangle,\langle u_{equ},\phi(u_{equ})\rangle}\} \!\cup\! \{\langle u_{equ},v\rangle,\langle u,\phi(u_{equ}) \rangle\}
\end{equation}
{\chengC Clearly, $S_{iso}$ is cs-isomorphic to $S_{sub}$ given that $u$ and $u_{equ}$ are structurally equivalent.}

We next show that $S_{iso}$ has been found before and thus Condition 2 holds: In specific, consider an ancestor \eat{ascendant branch} of $(S,C,D)$, denoted by $(S_{anc},C_{anc},D_{anc})$, where $u_{equ}$ is selected as the branching vertex, as visually illustrated in Figure~\ref{fig:exclusion_set}. {\chengC Since} $S_{iso}$ contains $\langle u_{equ},v\rangle$, {\chengC we check} whether it has been found in the sub-branch of $(S_{anc},C_{anc},D_{anc})$, which is formed by including $\langle u_{equ},v\rangle$. The answer is interestingly positive. The rationale behind is that (1) $S_{iso}\backslash\{\langle u,\phi(u_{equ})\rangle\}$ is a subset of $S_{anc}\cup C_{anc}$ (this can be easily verified by the facts that $S_{sub} \subseteq S_{anc}\cup C_{anc}$ and $\langle u_{equ},v \rangle\in C_{anc}$) and (2) $\langle u,\phi(u_{equ}) \rangle$ constructed by ours is in $C_{anc}$ (this holds due to the vertex equivalence between $u$ and $u_{equ}$
% , which will be discussed later in Lemma~\ref{lemma:reduction_for_VE1}
). Therefore, $S_{iso}$ is a common subgraph in $(S_{anc},C_{anc},D_{anc})$, i.e., $S_{anc}\subseteq S_{iso} \subseteq S_{anc}\cup C_{anc}$, and has been found before $(S,C,D)$ since it contains $\langle u_{equ},v\rangle$. 

In summary, we have the following lemma and reduction rule.
\begin{lemma}
    \label{lemma:reduction_for_VE1}
    Let $(S,C,D)$ be a branch. Common subgraph $S_{iso}$ {\chengC defined in} Equation~(\ref{eq:iso1}) has been found before the formation of $(S,C,D)$.
\end{lemma}
\begin{proof}
    %Omitted for lack of space. See the anonymous technical report~\cite{TR} for details. 
    {\revision
    We note that the recursive branching process forms a recursion tree where each tree node corresponds to a branch. Consider the path from the initial branch $(\emptyset, V_Q\times V_G,\emptyset)$ to $(S,C,D)$, there exists an ascendant branch of $(S,C,D)$, denoted by $B_{asc}=(S_{asc},C_{asc},D_{asc})$, where $u_{equ}$ is selected as the branching vertex, since $\langle u_{equ},\phi(u_{equ}) \rangle$ is in $S$.
    %
    We can see that there exists one sub-branch $B'_{asc}=(S'_{asc},C'_{asc},D'_{asc})$ of $B_{asc}$ formed by including $\langle u_{equ},v \rangle$, and all common subgraphs within $B'_{asc}$ have been found before the formation of $(S,C,D)$, since $\langle u_{equ},v \rangle$ is in $D$. We then show that common subgraph $S_{iso}$ can be found within $B_{asc}'$, i.e., $S'_{asc}\subseteq S_{iso}\subseteq S'_{asc}\cup C'_{asc}$. 
    %
    \underline{First}, we have $S'_{asc}\subseteq S_{iso}$ since (1) $S'_{asc}=S_{asc}\cup\{\langle u_{equ},v \rangle\}$, (2) $S_{sub}$ is a common subgraph in $B_{asc}$ and thus $S_{asc}\subseteq S_{sub}$, (3) $S_{asc}$ does not include $\langle u_{equ},\phi(u_{equ})\rangle$ or $\langle u,v\rangle$ since they are in $C_{asc}$ and will be included to the partial solution at $B_{asc}'$ and $(S\cup \{\langle u,v \rangle\}, C\backslash u\backslash v)$, and thus (4) by combining all the above, we have  $S'_{asc}=S_{asc}\cup\{\langle u_{equ},v \rangle\}\subseteq S_{sub}\cup\{\langle u_{equ},v \rangle\} \backslash \{\langle u_{equ},\phi(u_{equ})\rangle,\langle u,v\rangle\}\subseteq S_{iso}$. 
    %
    \underline{Second}, we have $S_{iso}\subseteq S'_{asc}\cup C'_{asc}$ based on the following two facts. 

    \begin{itemize}
        \item \textbf{Fact 1.} $S_{sub}\backslash\{\langle u_{equ},\phi(u_{equ})\rangle,\langle u,v\rangle\}\subseteq S'_{asc}\cup C'_{asc}$.
        \item \textbf{Fact 2.} $\langle u_{equ},v \rangle\in S'_{asc}$ and $\langle u,\phi(u_{equ}) \rangle\in C'_{asc}$.
    \end{itemize}
    
    The correctness of the above facts can be verified accordingly, for which we put the details in the 
    \ifx \CR\undefined
Appendix. 
\else
technical report~\cite{TR}. 
\fi
    }
\end{proof}

%\medskip
\noindent\fbox{%
    \parbox{0.47\textwidth}{%
       \textbf{Vertex-Equivalence-based reduction at the first group}. Let $B=(S,C,D)$ be a branch. For a sub-branch of $B$ formed by including a candidate pair $\langle u,v \rangle$ in the first group, it can be pruned if there exists a vertex pair $\langle u',v \rangle$ in $D$ such that $u'\in \Psi(u)$. 
    }%
}
%\medskip

\begin{example}
        Consider again the branching process at branch $B_6=(S_6,C_6,D_6)$ in Figure~\ref{fig:example_branching}. \laks{Suppose $u_2$ is selected as the branching vertex}. Then, we can see that $u_1$ is in $\Psi(u_2)$ and $D_{6}= \{u_1\} \times \{v_1,v_2,v_3,v_4,v_5\}$. {\YuiRR Recall that $C_6=\{u_2,u_3\}\times \{v_4,v_5\}\cup\{u_4,u_5,u_6,u_7\}\times\{v_1,v_2,v_3,v_7\}$. Thus, $B_6$ has two sub-branches which are formed by including $\langle u_2,v_4 \rangle$ or $\langle u_2,v_5 \rangle$, and they can be pruned based on the above reduction. }
\end{example}


\smallskip
\noindent\textbf{Reduction at the second group}. {\YuiR Recall that vertex $u$ is selected as the branching vertex.} Consider the sub-branch formed  {\YuiR by excluding $u$} in the second group. We note that {\chengC each} common subgraph $S_{sub}=\langle q_{sub},g_{sub},\phi_{sub} \rangle$ to be found in this sub-branch must exclude vertex $u$. We observe that if \emph{$S_{sub}$ contains a vertex $u_{equ}$, {\chengC which} is in $C\backslash u$ and is structurally equivalent to $u$}, Condition 1\&2 holds for $S_{sub}$. 

In specific, Condition 1 holds {\chengC since we can construct the} following common subgraph {\chengC which is} cs-isomorphic to $S_{sub}$ {\chengC based on} Case 2 of Lemma~\ref{lemma:cs} {\chengC (essentially, we replace $u_{equ}$ with $u$)}.
\begin{equation}
    \label{eq:iso2}
    S_{iso}= S_{sub}\backslash\{\langle u_{equ}, \phi_{sub}(u_{equ})\rangle\}\cup\{ \langle u,\phi_{sub}(u_{equ}) \rangle\}
\end{equation}

Besides, we note that $S_{iso}$ contains $\langle u,\phi_{sub}(u_{equ}) \rangle$ and common subgraphs found in the first group must include $u$. We thus {\chengC check} whether $S_{iso}$ has been found in one sub-branch formed in the first group. The answer is also positive. The rationale behind is that (1) $\langle u,\phi_{sub}(u_{equ}) \rangle$ constructed by ours exists in $C$ (this holds due to the vertex equivalence between $u$ and $u_{equ}$
% , which will be discussed in Lemma~\ref{lemma:reduction_for_VE2}
) and (2) thus there exists a sub-branch formed by including $\langle u,\phi_{sub}(u_{equ}) \rangle$ in the first group, where $S_{iso}$ has been found since it includes $\langle u,\phi_{sub}(u_{equ}) \rangle$. 

In summary, we have the following lemma and reduction rule.

\begin{lemma}
\label{lemma:reduction_for_VE2}
Let $(S,C,D)$ be a branch where $u$ is selected as the branching vertex. Common subgraph $S_{iso}$ {\chengC defined in} Equation~(\ref{eq:iso2}) has been found before the formation of $(S,C\backslash u,D)$ at the second group.   
\end{lemma}
\begin{proof}
{\revision
    \underline{First}, we note that $\langle u_{equ},\phi_{iso}(u_{equ}) \rangle$ is in $C\backslash u$ and also in $C$ since otherwise $S_{sub}$ cannot include $\langle u_{equ},\phi_{iso}(u_{equ}) \rangle$. This is because (1) $S\subseteq S_{sub}\subseteq S\cup C\backslash u$ since $S_{sub}$ is a common subgraph in the sub-branch $(S,C\backslash u,D)$ and (2) $S$ does not include $\langle u_{equ},\phi_{iso}(u_{equ}) \rangle$ since $u_{equ}$ appears in $C\backslash u$.
    %
    \underline{Second}, we note that $\phi_{iso}(u_{equ})$ is in $Y$. Recall that $X\times Y$ is the branching set at $(S,C,D)$. This is because (1) $u_{equ}$ is in the same subset $X$ as $u$ since $u_{equ}$ and $u$ are structurally equivalent and thus have the same set of neighbours and non-neighbours in $q$, and (2) $\langle u_{equ},\phi_{iso}(u_{equ}) \rangle$ is in $C$ as discussed before.
    %
    \underline{Third}, we can derive that there exists a sub-branch $(S\cup \{\langle u,\phi_{iso}(u_{equ}) \rangle\},C\backslash u\backslash \phi_{iso}(u_{equ}),D')$, which is formed at branch $(S,C,D)$  by including $\langle u,\phi_{iso}(u_{equ}) \rangle$ before the formation of $(S,C\backslash u,D)$, since $\phi_{iso}(u_{equ})\in Y$.
    %
    \underline{Forth}, we show that $S_{iso}$ is in $(S\cup \{\langle u,\phi_{iso}(u_{equ}) \rangle\},C\backslash u\backslash \phi_{iso}(u_{equ}),D')$, formally, $S\cup \{\langle u,\phi_{iso}(u_{equ}) \rangle\} \subseteq S_{iso}\subseteq S\cup \{\langle u,\phi_{iso}(u_{equ}) \rangle\}\cup (C\backslash u\backslash \phi_{iso}(u_{equ}))$.
    We have $S\subseteq S_{sub}\subseteq S\cup (C\backslash u)$ since $S_{sub}$ is a common subgraph in $(S,C\backslash u,D)$. Let $S'=S\cup \{\langle u,\phi_{iso}(u_{equ}) \rangle\}$, it can be proved as below.
    \begin{eqnarray}
        && S\subseteq S_{sub}\subseteq S\cup (C\backslash u)\\
        && \Rightarrow S\subseteq S_{sub}\backslash\{\langle u_{equ}, \phi_{iso}(u_{equ})\rangle\}\subseteq S\cup (C\backslash u\backslash \phi_{iso}(u_{equ})) \label{eq:lemma_eq_1}\\
        && \Rightarrow S' \subseteq S_{iso}\subseteq S'\cup (C\backslash u\backslash \phi_{iso}(u_{equ})) \label{eq:lemma_eq_2}
    \end{eqnarray}
    Note that Equation~(\ref{eq:lemma_eq_1}) holds since $\langle u_{equ},\phi_{iso}(u_{equ}) \rangle$ is in $S$; Equation~(\ref{eq:lemma_eq_2}) is derived by including the vertex pair $\langle u, \phi_{iso}(u_{equ})\rangle$.}
\end{proof}

\noindent\fbox{%
    \parbox{0.47\textwidth}{%
       \textbf{Vertex-Equivalence-based reduction at the second group}. Let $B=(S,C,D)$ be a branch and $(S,C\backslash u,D)$ be  the sub-branch formed in the second group by excluding all candidate pairs that consist of $u$. For a vertex $u'$ appearing in $C\backslash u$, if $u'$ is structurally equivalent to $u$, i.e., $u'\in \Psi(u)$, all candidate pairs that consist of $u'$ can be pruned from  $C\backslash u$. 
    }%
}
%\medskip

\begin{example}
    Consider the branching process at $B_0$ where $u_1$ is the branching vertex in Figure~\ref{fig:example_branching}. For sub-branch $B_8$ (which is the sub-branch at the second group), we can see that $\Psi(u_1)=\{u_1,u_2\}$ and thus the candidate set of $B_8$ can be reduced to $ (V_Q\backslash\{u_1,u_2\})\times V_G$, {\YuiR i.e., to $\{u_3,u_4,u_5,u_6,u_7\}\times \{v_1,v_2,\cdots,v_7\}$}.
\end{example}


\subsection{Maximality-based Reduction}
\label{subsec:maximality-reduction}

We introduce the redundancies induced by \emph{non-maximality}. Clearly, a maximum common subgraph must be a maximal common subgraph. Therefore, exploring those branches that hold non-maximal common subgraphs only will incur redundant computations. 
%
Consider a current branch $B=(S,C,D)$. {\chengC Note that there might exist multiple common subgraphs with the largest number of vertices in the branch.}
%
We observe that \emph{there exists one largest common subgraph in $B$ that must contain one specific candidate vertex pair $\langle u,v \rangle$}. 
%
{\YuiR Given this, we can remove this candidate vertex pair $\langle u,v\rangle$ from $C$ to $S$, thereby producing one immediate sub-branch i.e., $(S\cup\{\langle u,v \rangle\},C\backslash u\backslash v)$. Clearly, solving the resulting sub-branch is enough to find the largest common subgraph (since it holds all those common subgraphs that contain $\langle u,v\rangle$ in $B$). As a result,  we can safely prune all other sub-branches for which the partial set and the candidate set do not include the candidate pair $\langle u,v\rangle$.}
%
%As a result, we can safely prune all other sub-branches \laks{for which} all common subgraphs \laks{in the sub-branch} \emph{exclude} these candidate vertex pairs. 
%
Below, we elaborate on the details.


To be specific, we observe that there exists one largest common subgraph, denoted by $S_{opt}$, in $B$ such that $S_{opt}$ must contain a candidate vertex pair $\langle u,v \rangle$ if for any subset $X\times Y$ in $\mathcal{P}(C)$, $u$ and $v$ are simultaneously adjacent or non-adjacent to all other vertices in $X$ and $Y$, respectively, i.e.,
\begin{eqnarray}
    \label{eq:condition}
    \forall X\times Y\in \mathcal{P}(C): {\revision\big(}N(u,X)=X\backslash\{u\} {\revision \text{ and }} N(v,Y)=Y\backslash\{v\}{\revision\big)} \text{ or }\nonumber\\  
        {\revision \big(}N(u,X)=\emptyset {\revision\text{ and }} N(v,Y)=\emptyset{\revision\big)},
\end{eqnarray}
Formally, we have the following lemma.

\begin{lemma}
\label{lemma:maximality}
    Let $B=(S,C,D)$ be a branch {\chengC and $\langle u,v \rangle$ be a candidate vertex pair that satisfies the condition in Equation~(\ref{eq:condition}).} There exists one largest common subgraph $S_{opt}$ in the branch $B$ such that $S_{opt}$ contains 
    {\chengC $\langle u,v \rangle$.}
    % a candidate vertex pair $\langle u,v \rangle$, if $\langle u,v \rangle$ satisfies the condition in Equation~(\ref{eq:condition}).
\end{lemma}

\begin{proof}
 {\revision
 This can be proved by construction. Let $S^*=(q^*,g^*,\phi^*)$ be one largest common subgraph to be found in $B$. Note that if $S^*$ contains the candidate vertex pair $\langle u,v \rangle$, we can finish the proof by constructing $S_{opt}$ as $S^*$. Otherwise, if $\langle u,v \rangle$ is not in $S^*$, we prove the correctness by constructing one largest common subgraph $S_{opt}$ to be found in $B$ that contains candidate vertex pair $\langle u,v \rangle$, i.e., $S\subseteq S_{opt} \subseteq S\cup C$,  $|S_{opt}|=|S^*|$ and $\langle u,v \rangle\in S_{opt}$. In general, there are four different cases, and the details can be found in the 
 \ifx \CR\undefined
Appendix. 
\else
technical report~\cite{TR}. 
\fi
 }
\if 0
    This can be proved by construction. Let $S^*=(q^*,g^*,\phi^*)$ be one largest common subgraph to be found in $B$. Note that if $S^*$ contains the candidate vertex pair $\langle u,v \rangle$, we can finish the proof by constructing $S_{opt}$ as $S^*$. Otherwise, if $\langle u,v \rangle$ is not in $S^*$, we prove the correctness by constructing one largest common subgraph $S_{opt}$ to be found in $B$ that contains candidate vertex pair $\langle u,v \rangle$, i.e., $S\subseteq S_{opt} \subseteq S\cup C$,  $|S_{opt}|=|S^*|$ and $\langle u,v \rangle\in S_{opt}$.
    %
    In general, there are four different cases.

    \smallskip
    \noindent\underline{\textbf{Case 1:}} $u\notin V_{q^*}$ and $v\in V_{g^*}$. In this case, there exists a vertex pair $\langle\phi^{*-1}(v),v \rangle$ in $S^*$ where $\phi^{*-1}$ is the inverse of $\phi^*$. We construct $S_{opt}$ by replacing the vertex pair $\langle\phi^{*-1}(v),v \rangle$ with $\langle u,v \rangle$, i.e.,
    \begin{equation}
        S_{opt}=S^*\backslash\{ \langle\phi^{*-1}(v),v \rangle\} \cup \{\langle u,v \rangle\}.
    \end{equation}
    Clearly, we have $S\subseteq S_{opt}\subseteq S\cup C$ (i.e., $S_{opt}$ is in $B$) since $S^*$ is in $B$ and $\langle u,v\rangle$ is in the candidate set $C$. Besides, we have $|S_{opt}|=|S^*|$ and $\langle u,v \rangle\in S_{opt}$ based on the above construction. Finally, we deduce that $S_{opt}$ is a common subgraph by showing that any two vertex pairs in $S_{opt}$ satisfy Equation~(\ref{eq:isomorphic}), i.e., $g_{opt}$ is isomorphic to $q_{opt}$ under the bijection $\phi_{opt}$. \underline{First}, $S^*\backslash\{\langle \phi^{*-1}(v),v\rangle\}$, as a subset of $S^*$, is a common subgraph and thus has any two vertex pairs inside satisfying Equation~(\ref{eq:isomorphic}) (note that any subset of a common subgraph is still a common subgraph); \underline{Second}, for each pair $\langle u',v' \rangle$ in $S$, $u$ is adjacent to $u'$ if and only if $v$ is adjacent to $v'$ (since $\langle u,v \rangle$ is a candidate pair which can form a common subgraph with $S$); \underline{Third}, for each pair $\langle u',v' \rangle$ in $S_{opt}\backslash S\backslash\{\langle \phi^{*-1}(v),v\rangle\}$, it is clear that $\langle u',v' \rangle$ is in one subset $ X\times Y$ of $\mathcal{P}(C)$ and thus $u$ is adjacent to $u'$ if and only if $v$ is adjacent to $v'$ based on Equation~(\ref{eq:condition}). Therefore, any two vertex pairs in $S_{opt}$ will satisfy the Equation~(\ref{eq:isomorphic}).

    \smallskip
    \noindent\underline{\textbf{Case 2:}} $u\in V_{q^*}$ and $v\notin V_{g^*}$. There exists a vertex pair $\langle u,\phi^*(u) \rangle$ in $S^*$. We construct $S_{opt}$ by replacing $\langle u,\phi^*(u) \rangle$ with $\langle u,v \rangle$, i.e., $S_{opt}=S^*\backslash \{\langle u,\phi^*(u) \rangle\}\cup\{\langle u,v \rangle\}$. Similar to Case 1, we can prove that $S_{opt}$ includes $\langle u,v \rangle$ and is one largest common subgraph to be found in $B$. 

    \smallskip
    \noindent\underline{\textbf{Case 3:}} $u\in V_{q^*}$ and $v\in V_{g^*}$. There exists two distinct vertex pairs $\langle u,\phi^*(u) \rangle$ and $\langle \phi^{*-1}(v),v \rangle$ in $S^*$. We construct $S_{opt}$ by replacing these two vertex pairs with $\langle \phi^{*-1}(v),\phi(u) \rangle$ and $\langle u,v \rangle$, formally,
    \begin{equation}
        S_{opt}\!\!=\!\!S^*\backslash\{\langle u,\phi^*(u) \rangle,\!\langle\phi^{*-1}(v),v \rangle\}\!\cup\!\{\langle \phi^{*-1}(v),\phi^*(u) \rangle,\!\langle u,v \rangle\}.
    \end{equation}
    Clearly, we have $S\subseteq S_{opt}\subseteq S\cup C$ (i.e., $S_{opt}$ is in $B$), $|S_{opt}|=|S^*|$ and $\langle u,v \rangle\in S_{opt}$ based on the above construction. We then deduce that $S_{opt}$ is a common subgraph  by showing that any two vertex pairs in $S_{opt}$ satisfy Equation~(\ref{eq:isomorphic}).
    %
    \underline{First}, $S^*\backslash\{\langle u,\phi^*(u) \rangle,\langle \phi^{*-1}(v),v\rangle\}$, as a subset of $S^*$, is a common subgraph and thus has any two vertex pairs inside satisfying Equation~(\ref{eq:isomorphic});
    %
    \underline{Second}, consider a vertex pair $\langle u',v' \rangle$ in $S^*\backslash\{\langle u,\phi^*(u) \rangle,\langle \phi^{*-1}(v),v\rangle\}$. Similar to Case 1, we can prove that $u$ is adjacent to $u'$ if and only if $v$ is adjacent to $v'$. Besides, we show that $\phi^{*-1}(v)$ is adjacent to $u'$ if and only if $\phi(u)$ is adjacent to $v'$ since (1) $(\phi^{*-1}(v),u')\in E_Q\Leftrightarrow (v,v')\in E_G$ and $(u,u')\in E_Q\Leftrightarrow (\phi^*(u),v')\in E_G$ (since the common subgraph $S^*$ contains $\{\langle u,\phi^*(u) \rangle,\langle \phi^{*-1}(v),v\rangle\}$), (2) $ (v,v')\in E_G \Leftrightarrow (u,u')\in E_Q$ (as we shown above), and thus (3) they can be combined as $(\phi^{*-1}(v),u')\in E_Q\Leftrightarrow (v,v')\in E_G \Leftrightarrow (u,u')\in E_Q \Leftrightarrow (\phi^*(u),v')\in E_G$.
    %\begin{equation}
        %(\phi^{*-1}(v),u')\in E_Q\Leftrightarrow (v,v')\in E_G \Leftrightarrow (u,u')\in E_Q \Leftrightarrow (\phi^*(u),v')\in E_G  \nonumber
    %\end{equation}
    %
    \underline{Third}, we have $(u,\phi^{*-1}(v))\in E_Q\Leftrightarrow (v,\phi^*(u))\in E_G$ since the common subgraph $S^*$ contains $\{\langle u,\phi^*(u) \rangle,\langle \phi^{*-1}(v),v\rangle\}$ and thus $(u,\phi^{*-1}(v))\in E_Q\Leftrightarrow (\phi^*(u),v)\in E_G$ (note that $(\phi^*(u),v)$ refers to the same edge as $(v,\phi^*(u))$ since the graphs $Q$ and $G$ are undirected). Therefore, any two vertex pairs in $R_{opt}$ will satisfy Equation~(\ref{eq:isomorphic}).

    \smallskip
    \noindent\underline{\textbf{Case 4:}} $u\notin V_{q^*}$ and $v\notin V_{g^*}$. We note that this case will not occur since otherwise the contradiction is derived by showing that $S^*\cup \{\langle u,v\rangle\}$ is a larger common subgraph (note that the proof is similar to Case 1 and thus be omitted).
    \fi
\end{proof}

Consider a branch $B=(S,C,D)$ where $ X^*\times Y^*$ in $\mathcal{P}(C)$ and $u^*$ in $X^*$ are selected as the branching subset and the branching vertex, {\chengC respectively}. 
Assume that there exists a vertex $v$ in $Y^*$ such that $\langle u^*,v \rangle$ satisfies the condition in Equation~(\ref{eq:condition}).
Based on the above lemma, there exists one largest common subgraph in the branch $B$ that contains candidate vertex pair $\langle u^*,v \rangle$. Therefore, we only need to form one sub-branch $(S\cup\{\langle u^*,v \rangle\},C\backslash u^*\backslash v,D\cup  \{u^*\}\times (Y^*\backslash\{v\}) )$ since other formed sub-branches will exclude the candidate vertex $\langle u^*,v \rangle$ from the found common subgraphs. We note that the exclusion set of the formed sub-branch can be updated by $D\cup  \{u^*\}\times (Y^*\backslash\{v\}) $ to enhance the pruning power of the proposed reduction at the first group. In summary, we obtain the following reduction.

%\begin{lemma}[Maximality-based reduction]
%    \label{lemma:maximality-reduction}
%    Let $B=(S,C,D)$ be a branch where $\langle X\times Y \rangle$ in $C$ and $u$ in $X$ are selected as the branching subset and the branching vertex. If there exists a candidate vertex pair $\langle u,v \rangle$ in the candidate set such that $\langle u,v \rangle$ satisfies Equation~(\ref{eq:condition}), only one sub-branch $(S\cup\{\langle u,v \rangle\},C\backslash u\backslash v,D\cup \langle \{u\}\times (Y\backslash\{v\}) \rangle)$ needs to be formed at $B$.
%\end{lemma}

\medskip
\noindent\fbox{%
    \parbox{0.47\textwidth}{%
       \textbf{Maximality-based reduction}. Let $B=(S,C,D)$ be a branch where $ X\times Y $ in $\mathcal{P}(C)$ and $u$ in $X$ are selected as the branching subset and the branching vertex. If there exists a candidate vertex pair $\langle u,v \rangle$ in the candidate set such that $\langle u,v \rangle$ satisfies Equation~(\ref{eq:condition}), only one sub-branch $(S\cup\{\langle u,v \rangle\},C\backslash u\backslash v,D\cup \{u\}\times (Y\backslash\{v\}))$ needs to be formed at $B$.
    }%
}


\begin{example}
Consider the branching at branch $B_6=(S_6,C_6,D_6)$ in Figure~\ref{fig:example_branching} where \laks{suppose} $u_2$ is selected as the branching vertex. Recall that $C_6=X_1\times Y_1 \cup X_2\times Y_2 =\{u_2,u_3\}\times \{v_4,v_5\} \cup \{u_4,u_5,u_6,u_7\}\times \{v_1,v_2,v_3,v_7\}$.
%
We note that $\langle u_2,v_5 \rangle$ satisfies Equation~(\ref{eq:condition}) since (1) $N(u_2,X_1)=X_1\backslash\{u_2\}$ and $N(v_5,Y_1)=Y_1\backslash\{v_5\}$ and (2) $N(u_2,X_2)=\emptyset$ and $N(v_5,Y_2)=\emptyset$. Therefore, we only need to explore one sub-branch $(S_6\cup\{\langle u_2,v_5\rangle\},C_6\backslash u_2\backslash v_5,D_6\cup\{\langle u_2,v_4 \rangle\})$, and other two sub-branches formed at $B_6$ can be pruned.
\end{example}




\subsection{Vertex-Equivalence-based {\chengB Upper Bound}}
\label{subsec:upper-bound}
Consider a current branch $(S,C,D)$ and the largest common subgraph $S^*$ seen so far. Clearly, we can terminate the branch $(S,C,D)$, if the upper bound on the size of common subgraphs to be found in the branch $(S,C,D)$ (or simply, the upper bound of $(S,C,D)$) is no larger than the size of $S^*$. The tighter the upper bound, the more branches we can prune. 
% To facilitate this pruning technique, we introduce the upper bound as below. 

\smallskip
\noindent\textbf{Existing upper bound.} Consider a common subgraph $S_{sub}$ to be found in the branch $(S,C,D)$. For a subset $X\times Y$ in $\mathcal{P}(C)$, we can derive
\begin{equation}
    |S_{sub}|\cap X\times Y \leq ub_{X,Y}:= \min\{|X|,|Y|\}
\end{equation}
since otherwise a common subgraph will contain two distinct vertex pairs $\langle u,v \rangle$ and $\langle u',v' \rangle$ such that $u=u'$ or $v=v'$ (which violates the definition of the bijection). Here, $ub_{X,Y}$ is the upper bound of the number of candidate pairs that are within $X\times Y$ and are in a common subgraph to be found in the branch $(S,C,D)$. Furthermore, since all subsets in $\mathcal{P}(C)$ are disjoint, 
% we can derive 
the {\chengC following} existing upper bound of branch $(S,C,D)$, denoted by $ub_{S,C}$~\cite{mccreesh2017partitioning}, 
{\chengC can be derived}.
% as below
\begin{equation}
    |S_{sub}| \leq ub_{S,C} := |S|+\sum_{ X\times Y \in \mathcal{P}(C)} ub_{X,Y}
\end{equation}

\smallskip
\noindent\textbf{Motivation.} We observe that the existing upper bound $ub_{X,Y}$ is not tight since some candidate vertex pairs in $X\times Y$ can be pruned from the candidate set $C$ {\chengC based on} the proposed vertex-equivalence-based reductions. In specific, for a candidate vertex pair $\langle u,v \rangle$, if there exists a vertex pair $\langle u',v \rangle$ in $D$ such that $u'\in \Psi(u)$, any common subgraph to be found within $(S,C,D)$ cannot include $\langle u,v \rangle$ and thus $\langle u,v \rangle$ can be pruned from the candidate set $C$. Note that this can be easily verified based on the proposed reduction at the first group. Below, we introduce our upper bound derived with the aid of the structural equivalence on vertices.

\smallskip
\noindent\textbf{New upper bound}. Consider a subset $X\times Y $ in $\mathcal{P}(C)$. Let $u$ be an arbitrary vertex in $X$. We partition $X$ and $Y$ as follows.
\begin{eqnarray}
 X_L=X\cap \Psi(u), X_R=X\backslash X_L\\
 Y_L=\{v\mid \langle u',v \rangle\in D, u'\in \Psi(u)\}, Y_R=Y\backslash Y_L,   
\end{eqnarray}
where $X_L$ consists of those vertices in $X$ that are structurally equivalent to $u$ and $Y_L$ consists of those vertices $v$ in $Y$ which appear  in a vertex pair $\langle u',v \rangle$ in $D$ where $u'\in \Psi(u)$. We then can partition $X\times Y$ as $X_L\times Y_L$, $X_L\times Y_R$, $ X_R\times Y_L$ and $X_R\times Y_R$. Clearly, all vertex pairs in $X_L\times Y_L$ can be pruned as discussed before. 
%
We note that (1) $S_{sub}$ contains at most $\min\{|X_R|,|Y|\}$ vertex pairs from $X_R\times Y_L$ and $X_R\times Y_R$ since otherwise there exists one vertex in $X_R\cup Y$ that appears in at least two distinct vertex pairs in $S_{sub}$ and thus $S_{sub}$ cannot be a common subgraph; and similarly (2) $S_{sub}$ contains at most $\min\{|X_L|,|Y_R|,\max\{|Y|-|X_R|,0\}\}$ vertex pairs from $X_L\times Y_R$ (note that the additional term $\max\{|Y|-|X_R|,0\}$ is used to ensure that the sum of $\min\{|X_R|,|Y|\}$ and $\min\{|X_L|,|Y_R|,\max\{|Y|-|X_R|,0\}\}$ is no larger than the existing upper bound $ub_{S,C}$). Therefore, $S_{sub}$ contains at most $ub_{X,Y,D}$ vertex pairs from $X\times Y$, where
\begin{equation}
    ub_{X,Y,D}:=\min\{|X_R|,|Y|\}+\min\{|X_L|,|Y_R|,\max\{|Y|-|X_R|,0\}\}.
\end{equation}
Then, we can derive our upper bound of a branch $(S,C,D)$, denoted by $ub_{S,C,D}$, i.e.,
\begin{equation}
    |S_{sub}|\leq ub_{S,C,D}:=|S|+\sum_{X\times Y\in \mathcal{P}(C)} ub_{X,Y,D}.
\end{equation}

In summary, we obtain our new upper bound $ub_{S,C,D}$ as {\chengC above}. It is {\chengC not difficult} to verify that our upper bound is tighter than the existing one, i.e., $ub_{S,C,D}\leq ub_{S,C}$: see Example~\ref{example:upper_bound} for an example where $ub_{S,C,D} < ub_{S,C}$.  
\begin{lemma}[Upper bound]
    \label{lemma:upper_bound}
    Let $(S,C,D)$ be a branch. All common subgraphs to be found in $(S,C,D)$ have the size at most $ub_{S,C,D}$.
\end{lemma}

\begin{example}
\label{example:upper_bound}
    Consider again the branching process at branch $B_6=(S_6,C_6,D_6)$ in Figure~\ref{fig:example_branching}. Recall that $C_6=X_1\times Y_1\cup X_2\times Y_2=\{u_2,u_3\}\times \{v_4,v_5\}\cup \{u_4,u_5,u_6,u_7\}\times \{v_1,v_2,v_3,v_7\}$ and $D_6=\{u_1\}\times\{v_1,v_2,...,v_5\}$. For $X_1\times Y_1$, based on $u_2$, 
    we have $X_{1L}=\{u_2\}$, $X_{1R}=\{u_3\}$, $Y_{1L}=\{v_4,v_5\}$ and $Y_{1R}=\emptyset$. Thus, we have $ub_{X_1,Y_1,D_6}=\min\{1,4\}+\min\{1,0,\max\{1,0\}\}=1$. For $X_2\times Y_2$, based on $u_4$, we have $X_{2L}=\{u_4\}$, $X_{2R}=\{u_5,u_6,u_7\}$, $Y_{2L}=\emptyset$ and $Y_{2R}=\{v_1,v_2,v_3,v_7\}$. Thus, we have $ub_{X_2,Y_2,D_6}=\min\{3,4\}+\min\{1,4,\max\{1,0\}\}=4$. Therefore, we have $ub_{S_6,C_6,D_6}=1+1+4=6$, which is smaller than the existing bound $ub_{S,C}=7$.
\end{example}

{\revision
\noindent\textbf{Remark.} Our new upper bound $ub_{S,C,D}$ varies {\chengE with} different choices of $u$ due to the partition of $X$ and $Y$. We can potentially obtain a tighter upper bound by exploring all possible choices of $u$. However, it {\chengE would} introduce a large amount of time costs, thus degrading the performance of \texttt{RRSplit}. Therefore, as a trade-off,  we randomly select $u$ when computing the upper bound.
}

\subsection{Summary and Analysis}
\label{subsec:summary}
\noindent\textbf{Summary.}
We summarize our algorithm, namely \texttt{RRSplit}, in Algorithm~\ref{alg:rrsplit}, which incorporates the newly proposed vertex-equivalence-based reductions, the maximality-based reduction and the vertex-equivalence-based upper bound. Specifically, \texttt{RRSplit} differs with \texttt{McSplit} in the following aspects. (1) It maintains one additional auxiliary data structure, namely exclusion set $D$, for each formed {\chengB branch}, which is initialized as 
the empty set and recursively updated as discussed. (2)  It {\chengB prunes} a branch $(S,C,D)$ if the newly proposed vertex-equivalence-based upper bound $ub_{S,C,D}$ is no larger than the \laks{largest common subgraph size}   seen so far, i.e., $|S^*|$ (Line 7). We remark that $ub_{S,C,D}$ is tighter than the existing one $ub_{S,C}$, i.e., $ub_{S,C,D}\leq ub_{S,C}$ and thus more branches can be pruned. (3) It creates only one sub-branch and prunes all others if the maximality-based reduction is triggered (Lines 9-11). (4) Based on the vertex-equivalence-based reduction, it prunes those sub-branches at the first group that hold all common subgraphs inside cs-isomorphic to the one found before (Lines 15-16), and refines the formed sub-branch at the second group by removing from the candidate set all those candidate vertex pairs consisting of a vertex in $\Psi(u)$ (Line 19).
%
We remark that our implementation of \texttt{RRSplit} in the experiments adopts the same heuristic policies for selecting branching subset $X\times Y$, branching vertex $u$ (Line 8) and vertex $v$ (Line 14) as \texttt{McSplit} {\chengB does}.
%
Besides, we can easily prove that \texttt{RRSplit} finds the maximum common subgraph based on our discussion above. Finally, we analyze the {space complexity and time complexity} of \texttt{RRSplit} as below. 


\begin{algorithm}{}
\small
\caption{Our proposed algorithm: \texttt{RRSplit}}
\label{alg:rrsplit}
\KwIn{Two graphs $Q=(V_Q,E_Q)$ and $G=(V_G,E_G)$}
\KwOut{The maximum common subgraph}
$S^*\leftarrow \emptyset$; \tcp{Global variable}
%$S\leftarrow\emptyset$, $C\leftarrow\langle V_Q\times V_G \rangle$, $D\leftarrow\emptyset$ \tcp{Global data structure}
\texttt{RRSplit-Rec}$(\emptyset,V_Q\times V_G,\emptyset)$\; \textbf{Return} $S^*$;\\
%\setcounter{AlgoLine}{0}
\SetKwBlock{Enum}{Procedure \texttt{RRSplit-Rec}$(S,C,D)$}{}
%\SetKwBlock{update}{Procedure \texttt{Update}$(S,C)$}{}
\Enum{
    \lIf{$|S|>|S^*|$}{$S^*\leftarrow S$}
    \tcc{Termination (Lemma~\ref{lemma:upper_bound})}
    \lIf{$C=\emptyset$}{\textbf{return}}
    \lIf{ $ub_{S,C,D}\leq |S^*|$}
        {\textbf{return}} 
    \tcc{Branching}
    Select a branching vertex $u$ and a branching subset $X\times Y$ from $\mathcal{P}(C)$  based on a policy\;
    \tcc{Maximality-based reduction}
    \If{there exists a vertex $v$ in $Y$ such that $\langle u,v \rangle$ satisfies Equation~(\ref{eq:condition})}{
        \texttt{RRSplit-Rec}($S\cup\{\langle u,v \rangle\},C\backslash u\backslash v,D\cup \{u\}\times (Y\backslash\{v\})$)\;
        \textbf{return}\;
    }
    \tcc{Branching at the first group}
    $Y_{temp}\leftarrow Y$\;
    \For{$i=1,2,...,|Y|$}{
        Select and remove a vertex $v$ from $Y_{temp}$ based on a policy\;
        \If{there exists a vertex pair $\langle u',v \rangle$ in $D$ such that $u'\in \Psi(u)$}{\textbf{continue;}}
        Refine candidate set $C\backslash u\backslash v$ as $C_i$ based on Equation~(\ref{eq:update_candidate_set})\;
        \texttt{RRSplit-Rec}($S\cup\{\langle u,v\rangle\},C_i,D\cup\{u\}\times (Y\backslash Y_{temp})$);
    }
    \tcc{Branching at the second group}
    \texttt{RRSplit-Rec}($S,C\backslash \Psi(u),D$)\;
}
\end{algorithm}

\smallskip
\noindent\textbf{Space complexity}. We note that \texttt{RRSplit} recursively maintains three global data structures, namely $S$, $C$ and $D$, for each branch, which dominate the space complexity of \texttt{RRSplit}. Let $S^*$ be the largest common subgraph between  graphs $Q$ and $G$. \underline{First}, partial solution $S$ is a set of vertex pairs and its size is bounded by $O(|S^*|)$. \underline{Second}, candidate set $C$ is also a set of vertex pairs and can be partitioned as several subsets, i.e., $C= X_1\times Y_1 \cup  X_2\times Y_2\cup\cdots\cup X_c\times Y_c$ where $c$ is a positive integer, based on Equation~(\ref{eq:update_candidate_set}). We note that subsets in $X_1,X_2,...,X_c$ (resp. $Y_1,Y_2,...,Y_c$) are mutually disjoint and $X_1\cup X_2\cup .... \cup X_c=X$ (resp. $Y_1\cup Y_2\cup .... \cup Y_c=Y$), as discussed in the proof of Lemma~\ref{lemma:reduction_for_VE1}. Therefore, $C$ can be stored {\chengC as} $c$ subsets, each of which $\langle X_i,Y_i\rangle$ ($1\leq i\leq c$) consists of two sets $X_i$ and $Y_i$. Thus, the size of $C$ is bounded by $O(|V_Q|+|V_G|)$. \underline{Third}, $D$ is a set of vertex pairs and consists of at most $|S^*|\cdot |V_G|$ different vertex pairs since for a vertex pair $\langle u,v \rangle$ in $D$, (1) $u$ must {\chengB appear} in $S$ based on our maintenance of $D$ and thus has at most $|S^*|$ different values and (2) $v$ has at most $|V_G|$ different values clearly. In summary, the space complexity of \texttt{RRSplit} is $O(|V_Q|+|S^*|\times|V_G|)$. %We remark that \texttt{McSplit} has the space complexity of $O(|V_Q|+|S^*|\times|V_G|)$, which is the same as that of \texttt{RRSplit}, since \texttt{McSplit} needs to maintain the set $Y_{temp}$ for each branch.

\smallskip
\noindent\textbf{Time complexity of the proposed reductions}. \underline{First}, the reduction at the first group takes $O(|V_Q|+|V_G|)$ {\chengB time} (Lines 15-16). In specific, $D$ is organized as several disjoint subsets, i.e., $D=\{u_1\} \times A_1 \cup  \{u_2\} \times A_2\cup \cdots \cup \{u_d\} \times A_d $ where $d$ is a positive integer. 
%
Thus, it can be conducted in two steps: (1) for each vertex $u_i$ {\chengC appearing} in $D$, it takes $O(1)$ to check whether $u_i\in \Psi(u)$ and (2) if $u_i\in \Psi(u)$, it takes $O(|A_i|)$ to check whether $\langle u_i,v \rangle\in  \{u_i\} \times A_i$.
%
We note that for any two distinct vertices $u_i$ and $u_j$ appearing in $D$ such that $u_i\in \Psi(u_j)$, it is no hard to verify that $A_i\cap A_j=\emptyset$ due to the reduction at the first group (for which we put the details of the proof in the 
\ifx \CR\undefined
Appendix\else technical report~\cite{TR}\fi). 
As a result, we have $\sum_{u_i\in \Psi(u)} (|A_i|)\leq |V_G|$.  
%
\underline{Second}, the reduction at the second group runs in $O(|X|)$ for updating $C\backslash\Psi(u)$ at Line 19, which is bounded by $O(|V_Q|)$. In specific, it can be done by removing from $X$ all vertices in $\Psi(u)$ (note that, given all structurally equivalent classes, determining whether a vertex belongs to $\Psi(u)$ can be done in $O(1)$).
%
\underline{Third}, the maximality reduction runs in $O(\sum_{\langle X',Y' \rangle\in C}|X'|+|Y'|\cdot |Y|)$, which is bounded by $O(|V_Q|+|V_G|^2)$. In specific, for each vertex in $|Y|$, it needs to check the condition in Equation~(\ref{eq:condition}).
%
{\revision \underline{Fourth}}, the new upper bound can be obtained in $O(|V_Q|+|V_G|^2)$. In specific, the time cost is dominated by the computation of $ub_{X',Y',D}$ for each subset $\langle X',Y'\rangle$ in $C$. $ub_{X',Y',D}$ can be obtained in $O(|X'|+\sum_{u_i\in\Psi(u')} |A_i|+|Y'|)$, where $u'$ is a random vertex selected from $X'$ and $\{u_i\}\times A_i$ is a subset in $D$, which is bounded by $O(|X'|+|V_G|)$. Therefore, the new upper bound can be obtained in $O(\sum_{X'\times Y'\in C} (|X'|+|V_G|))$, which is bounded by $O(|V_Q|+|V_G|^2)$.

\smallskip
\noindent\textbf{Worst-case time complexity of \texttt{RRSplit}.} We note that the worst-case time complexity of \texttt{RRSplit} is dominated by the number of recursive calls of \texttt{RRSplit-Rec} (i.e., the number of formed branches) since \texttt{RRSplit-Rec} runs in polynomials of $|V_Q|$ and $|V_G|$. Formally, we have the following theorem.
\begin{theorem}
    Assume that $|V_Q|\leq |V_G|$. The worst-case time complexity of our proposed \texttt{RRSplit}  is $O^*((|V_G|+1)^{|V_Q|})$, where $O^*(\cdot)$ suppresses the polynomials.
\end{theorem}
\begin{proof}
    It is easy to verify that the worst-case time complexity of \texttt{RRSplit} is bounded by the number of branches. Consider a branch $B=(S,C,D)$. For all sub-branches formed at $B$ by selecting a branching vertex $u^*$, we observe that only the sub-branch in the second group has the same partial solution $S$ with $B$. Based on this, we can easily deduce that there are at most $|V_Q|$ branches which share the same partial solution. Besides,  we observe that each vertex in $V_p\cup V_q$ only appears in one pair of $S$, i.e., for any two distinct pairs $\langle u,v \rangle$ and $\langle u',v' \rangle$ in $S$, we have $u\neq u'$ and $v\neq v'$. Based on this, let $|S|=k$ where $0\leq k\leq |V_Q|$, and we can deduce that there are at most $k!\binom{|V_Q|}{k}\binom{|V_G|}{k}$ different partial solutions with the size of $k$ by applying the multiplication principle (note that $V_p$ has $\binom{|V_Q|}{k}$ different choices, $V_q$ has $\binom{|V_G|}{k}$ different choices, and the bijection $\phi$ between $V_p$ and $V_q$ has $k!$ different choices). Therefore, the number of branches is at most
    \begin{equation}
       T=|V_Q| \sum_{k=0}^{|V_Q|} k! \binom{|V_Q|}{k}\binom{|V_G|}{k}.
    \end{equation}
    We then show that $T$ is bounded by $O^*((|V_G|+1)^{|V_Q|})$ as below.
    \begin{eqnarray}
       T&=&|V_Q| \sum_{k=0}^{|V_Q|} (|V_Q|-k)! \binom{|V_Q|}{k}\binom{|V_G|}{|V_Q|-k}\\
       &=&|V_Q| \sum_{k=0}^{|V_Q|} \frac{(|V_{G}|)!}{(|V_G|-|V_Q|+k)!}\binom{|V_Q|}{k}\\
       &\leq& |V_Q|\sum_{k=0}^{|V_Q|} (|V_G|)^{|V_Q|-k} \binom{|V_Q|}{k}=|V_Q|(|V_G|+1)^{|V_Q|},
    \end{eqnarray}
    where $(|V_{G}|)!/(|V_G|-|V_Q|+k)!$ is  much smaller than $(|V_G|)^{|V_Q|-k}$ clearly and $(|V_G|+1)^{|V_Q|}$ in the last equation is derived by the binomial theorem.
\end{proof}



\smallskip
\noindent\textbf{Remark.} \laks{Note that the assumption that $|V_Q|\leq |V_G|$ is not a restrictive assumption: it is realistic in practice.} We remark that to our best knowledge, the achieved worst-case time complexity $O^*((|V_G|+1)^{|V_Q|})$ of \texttt{RRSplit} \laks{matches} 
% the state-of-the-art
{\chengB the best-known worst-case time complexity for the problem}
~\cite{suters2005new}. However, the algorithm proposed in~\cite{suters2005new} is of theoretical {\chengB interest} only and is not {\chengB practically} efficient. Besides, we note that \texttt{McSplit} and its variants~\cite{zhoustrengthened,liu2020learning,liu2023hybrid,mccreesh2017partitioning} do not have any theoretical guarantees on the worst-case time complexity. 
% Therefore, \texttt{RRSplit} is efficient in both theory and practice.


