\section{Related Works}
\subsection{Synthetic Dataset Generation}
\label{sec:rel_work_syntehtic}

Recently, generating non-real dataset, synthetic dataset is common technique in a variety of field which requires hyper-scale  \cite{shang2024urbanworld, ref25Greff2022KubricAS, refzhang2024cityx, refxie2024citydreamer, refwu2024unique3d, refschieber2024indoor, shang2024urbanworld, wang2024high, hao2024synthetic, zhu2024odgen, valvano2024controllable}, impossible scenarios     \cite{refjung2024harnessing,ref25Greff2022KubricAS, refhummel2019leveraging, mittal2023orbit, kokosza2024scintilla, amador2024cyclogenesis}, and so on. Synthetic dataset generation resolve such issues by constructing scene in virtual world and alleviate vexing manual process with auto labeling. Diverging from conventional way to generate synthetic data, it has been explored to reflect the characteristics of real-world object to enhance the high fidelity and appropriateness   \cite{refrichter2022enhancing, lee2024learning, ebadi2022psp}. Since these strategy enhance the robustness and realism, it is crucial to consider these attributes for faithful synthesized data.

\begin{figure*}[t]
    \centerline{\includegraphics[width=1\linewidth]{Figure/fig_overview.jpg}}
    \caption{Overview of virtual flood scene composition and synthetic dataset generation pipeline.}
    \label{fig:overview}
\end{figure*}

\begin{figure}[t]
    \centerline{\includegraphics[width=0.9\linewidth]{Figure/fig_real_flood_sample.jpg}}
    \caption{Sample of \textit{real} flood data   \cite{wan2024automatic}.}
    \label{fig:real_flood}
\end{figure}

\begin{figure}[t]
    \centerline{\includegraphics[width=0.9\linewidth]{Figure/fig_flood_simulation.jpg}}
    \caption{Components of flood simulation and results.}
    \label{fig:flood_simulation}
\end{figure}

\subsection{Flood Hazard Detection} \label{sec:rel_work_flood}

Detecting flood situation can be considered as object detection using the level of flood of object. A multitude of studies on classifying and detecting objects based on deep learning algorithms has been continuously conducted to address abovementioned requirement  \cite{floodlo2021deep, floodpally2022application, floodkaranjit2023floodimg, floodzhong2024detection, floodwu2024identification}. However, most studies face challenges such as a lack of data sharing, ethical concerns, absence of abundance and limited labels. To do that, in this paper, we intent to address these issues in following section.