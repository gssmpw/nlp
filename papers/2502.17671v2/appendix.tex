
\subsection{Proof of Lemma \ref{L:polynorms}}
\label{Lemma22}
   

All of the (quasi-)norms appearing in the statement of the lemma are equivalent  because they are (quasi-)norms on $\cP_r$. Indeed, if $\|\cdot\|$ is any of these (quasi-)norms,  we have $\|Q\|=0$ for $Q\in\cP_r$ if and only if $Q$ is the zero polynomial.    Thus, the only issue to be addressed is to prove that the constants appearing in these equivalences can be chosen to depend only on the dimension of the space $\cP_r$ and $q_0$ and not on $q,I$ or $N$,  once $N^d>\rho$.  The fact that the constants do not depend on $I$ is a simple matter of rescaling  which we do not discuss further.  So in proving the lemma,  we can assume that
$I=\Omega=(0,1)^d$.


It is well known, see for instance (3.1) in \cite{DS}, that the (quasi-)norms in (i) (which do not depend on $N$) are all uniformly equivalent for different values of $q \geq q_0$.
Similarly, the (quasi-)norms in (iii) are all uniformly equivalent for different values of $q \geq q_0$ (since these (quasi-)norms are simply on $\R^\rho$ and $\rho$ is fixed), and hence equivalent to the $q=2$ norm. From the orthogonality, it follows that the norms in  (ii) and (iii) are equal when $q = 2$. 

We will complete the proof by showing that (i) and (ii) are equivalent (uniformly in $q \geq q_0$ and $N^d > \rho$).   We denote  the cubes in the grid $\Lambda_I$ by $J$ and their collection by $\cJ_N$.
     For each $J\in\cJ_N$, we denote by  $x_J$    the lower left corner of $J$.  If  $Q\in\cP_r$, we define
$$
S_N(Q):=\sum_{J\in\cJ_N}Q(x_J)\chi_J,
$$ 
where $\chi_J$ is the characteristic function of $J$.
Then, $S$ is a piecewise constant function. From Markov's inequality  for polynomials, if $Q\in\cP_r$, we have
$$
|Q(x)-Q(x_J)|\le C(r) N^{-1}\|Q\|_{L_\infty(I)} \le C(r)N^{-1}\|Q\|_{L_q(I)},\quad x\in J.
$$ 
Here we note that the constant $C(r)$ can be chosen uniformly in $q_0 \leq q \leq \infty$. Therefore, we have
$$ 
\|Q-Q(x_J)\|_{L_q(J)}\le C(r) \|Q\|_{L_q(I)} N^{-1}|J|^{1/q}, \quad J\in\cJ_N,
$$
and hence
$$
\|Q-S_N(Q)\|_{L_q(I)} \le C(r)\|Q\|_{L_q(I)}N^{-1}.
$$
It follows that
$$ 
|\|Q\|_{L_q(I)} -\|S_N(Q)\|_{L_q(I)}|\le C(r)\|Q\|_{L_q(I)}N^{-1}.
$$
We now choose $N_0$ so that $C(r)N_0^{-1}\le 1/2$. 
Then, for $N\ge N_0$, we obtain the equivalency of $\|S_N(Q)\|_{L_q(I)}$ with $\|Q\|_{L_q(I)}$ with fixed equivalency constants.  This gives the equivalence of $\|(Q(x_j))\|^*_{\ell_q}$ and $ \|Q\|^*_{L_q(I)}$ since 
$$
\|S_N(Q)\|_{L_q(I)}=\|(Q(x_j))\|^*_{\ell_q} {\rm ~and~} \|Q\|_{L_q(I)}=\|Q\|^*_{L_q(I)}.
$$
On the other hand, we have the equivalence of the (quasi-)norms with $N\le N_0$ with fixed equivalency constants because there are only a finite number of them. The uniformity in $q_0 \leq q \leq \infty$ is obtained because the constants of equivalence are continuous functions of $1/q$ for fixed $N$, and hence these constants achieve a finite maximum on the compact interval $[0,1/q_0]$. In summary, we obtain that all the (quasi-)norms in (i) and (ii) are equivalent on $\cP_r$ with equivalency constants only depending on $r$.

Finally, let us use the norm in  (i) with $q=\infty$  and the norm in (iii) with  $q=2$, for  $Q=Q_{I,j}$, $j=1, \ldots,\rho$ (recall the $Q_{I,j}$'s  form an orthonormal basis for $\cP_r$ in the Hilbert space $L_2(\mu_I)$). We have shown that these two norms are equivalent, from which we obtain that  
 $$
 1\asymp \|Q_{I,j}\|_{L_\infty(I)},
 $$
 which proves \eref{QIjbound}.
 \hfill $\Box$

 




  

\subsection {Proof of Lemma \ref{L:known}}
\label{LemmaK}
Note that if $a < 1$, there is nothing to prove. For $a \geq 1$, the following holds
\begin{eqnarray}
\nonumber
\int_a^\infty t^qe^{-t^2/2}\,dt&=&
\int_a^\infty (t^{q-1}e^{-t^2/4})(te^{-t^2/4})\,dt\nonumber\\
&\leq&
\left(\max_{t \geq 1} t^{q-1}e^{-t^2/4}\right)\int_a^\infty te^{-t^2/4}\,dt
\leq
C_qe^{-a^2/4}.
 \nonumber 
\end{eqnarray}
\hfill $\Box$

\subsection{Proof of Lemma \ref{L:known2}}
\label{LemmaK2}
First, let us observe that  
\begin{equation}
 \sum_{k\ge 0}  2^{a k}e^{-c2^{b k}\tau} \le \sum_{k\ge 0}  2^{a_0 k}e^{-c2^{b_0 k}\tau}, \quad \tau\ge 1.
\end{equation}
Note that  there is a $
\bar k=\bar k(a_0,b_0,c)$ such that
\begin{equation}
 a_0k\le \frac{c}{2}(2^{b_0 k}-1), \quad k\ge \bar k.
\end{equation}
Hence, we have
\begin{equation}
\sum_{k=0}^{\bar k}  2^{a_0 k}e^{-c2^{b_0 k}\tau}\leq 
e^{-c\tau}\sum_{k=0}^{\bar k}  2^{a_0 k}\le  C(a_0,b_0,c)e^{-c\tau},
\end{equation}
and 
\begin{equation}
\sum_{k>\bar k}  2^{a_0 k}e^{-c2^{b_0 k}\tau}\le
e^{-c\tau}\sum_{k>\bar k}^\infty  2^{a_0 k}e^{-c(2^{b_0 k}-1)\tau} \le e^{-c\tau}
\sum_{k>\bar k}^\infty  e^{-\frac{c}{2}(2^{b_0 k}-1)} 
\le  C(a_0,b_0,c)e^{-c\tau},
\end{equation}
 where we have used the definition of $\bar k$ and the fact that $\tau\ge 1$.  This completes the proof of the lemma.
\hfill $\Box$




\subsection{Proof of Lemma \ref{L:perturb}}
\label{SS:A2}


    We denote by
\begin{equation}
        \bar\beta :=\mu_{y,\sigma}(B):= \frac{1}{(2\pi \sigma^2)^{m/2}}\int_B e^{-\|x-y\|^2/2\sigma^2} dx,
\end{equation}
and will show that $\bar\beta<\frac{1}{2}$. 
Observe that   
\be
        1-\bar\alpha = \frac{1}{(2\pi \sigma^2)^{m/2}}\int_{B^c} e^{-\|x\|^2/2\sigma^2} dx,\quad 1-\bar\beta = \frac{1}{(2\pi \sigma^2)^{RR2}}\int_{B^c} e^{-\|x-y\|^2/2\sigma^2}dx.
 \ee
 Since
    \be 
   \|y\|^2= \|x\|^2-\|x-y\|^2 -2\langle x-y,y\rangle,
    \ee 
    and $\langle z,y\rangle$ is an odd function of $z$, we have   
    \begin{eqnarray}
    \frac{\|y\|^2}{2\sigma^2} &= &\frac{1}{(2\pi \sigma^2)^{m/2}}\int_{\mathbb{R}^d} \frac{1}{2\sigma^2}\left(\|x\|^2- \|x-y\|^2\right)e^{-\|x-y\|^2/2\sigma^2} dx\nonumber \\
        &=& \frac{1}{(2\pi \sigma^2)^{m/2}}\int_{\mathbb{R}^d} -\ln\left(\frac{\exp(-\|x\|^2/2\sigma^2)}{\exp(-\|x-y\|^2/2\sigma^2)}\right)e^{-\|x-y\|^2/2\sigma^2} dx.
    \end{eqnarray}
    We divide the last  integral   into integrals over the sets $B$ and $B^c$ and use Jensen's inequality and the convexity of $(-\ln x)$ to obtain
    \begin{eqnarray}
    \label{y}
        \frac{\|y\|^2}{2\sigma^2} &=& \bar\beta\int_{B} -\ln\left(\frac{\exp(-\|x\|^2/2\sigma^2)}{\exp(-\|x-y\|^2/2\sigma^2)}\right)\frac{e^{-\|x-y\|^2/2\sigma^2}}{(2\pi \sigma)^{m/2}\bar\beta} dx 
        \nonumber\\\nonumber\\
        &+& (1-\bar\beta)\int_{B^c} -\ln\left(\frac{\exp(-\|x\|^2/2\sigma^2)}{\exp(-\|x-y\|^2/2\sigma^2)}\right)\frac{e^{-\|x-y\|^2/2\sigma^2}}{(2\pi \sigma)^{m/2}(1-\bar\beta)} dx
        \nonumber\\\nonumber\\
        &\geq& -\bar\beta\ln\left(\frac{\bar\alpha}{\bar\beta}\right)-(1-\bar\beta)\ln\left(\frac{1-\bar\alpha}{1-\bar\beta}\right).
    \end{eqnarray}
    Now, if   $\bar{\beta} \geq 1/2$, we will show that $\|y\|^2\geq -\sigma^2\ln(5\bar\alpha)$ which would contradict the assumptions of the lemma.
    
    Note that 
    the first term above is lower bounded by
   \be
        -\bar\beta\ln\left(\frac{\bar\alpha}{\bar\beta}\right) \geq -\frac{1}{2}\ln(2\bar\alpha).
 \ee
    On the other hand, the second term is lower bounded by
  \be
        -(1-\bar\beta)\ln\left(\frac{1-\bar\alpha}{1-\bar\beta}\right) \geq (1-\bar\beta)\ln\left(1-\bar\beta\right) \geq \min_{0 < t < 1}\left\{ t\ln(t)\right\} = -e^{-1}.
\ee
 We use these bounds in \eref{y} to obtain that
\be
        \|y\|^2 \geq -\sigma^2[\ln(2\bar\alpha) + 2e^{-1}] = -\sigma^2\ln(2e^{2/e}\bar\alpha) \geq -\sigma^2\log(5\bar\alpha),
\ee
and this completes the proof of the lemma.
\hfill $\Box$
