\documentclass[lettersize,journal]{IEEEtran}
%
% --- inline annotations
%
\newcommand{\red}[1]{{\color{red}#1}}
\newcommand{\todo}[1]{{\color{red}#1}}
\newcommand{\TODO}[1]{\textbf{\color{red}[TODO: #1]}}
% --- disable by uncommenting  
% \renewcommand{\TODO}[1]{}
% \renewcommand{\todo}[1]{#1}



\newcommand{\VLM}{LVLM\xspace} 
\newcommand{\ours}{PeKit\xspace}
\newcommand{\yollava}{Yo’LLaVA\xspace}

\newcommand{\thisismy}{This-Is-My-Img\xspace}
\newcommand{\myparagraph}[1]{\noindent\textbf{#1}}
\newcommand{\vdoro}[1]{{\color[rgb]{0.4, 0.18, 0.78} {[V] #1}}}
% --- disable by uncommenting  
% \renewcommand{\TODO}[1]{}
% \renewcommand{\todo}[1]{#1}
\usepackage{slashbox}
% Vectors
\newcommand{\bB}{\mathcal{B}}
\newcommand{\bw}{\mathbf{w}}
\newcommand{\bs}{\mathbf{s}}
\newcommand{\bo}{\mathbf{o}}
\newcommand{\bn}{\mathbf{n}}
\newcommand{\bc}{\mathbf{c}}
\newcommand{\bp}{\mathbf{p}}
\newcommand{\bS}{\mathbf{S}}
\newcommand{\bk}{\mathbf{k}}
\newcommand{\bmu}{\boldsymbol{\mu}}
\newcommand{\bx}{\mathbf{x}}
\newcommand{\bg}{\mathbf{g}}
\newcommand{\be}{\mathbf{e}}
\newcommand{\bX}{\mathbf{X}}
\newcommand{\by}{\mathbf{y}}
\newcommand{\bv}{\mathbf{v}}
\newcommand{\bz}{\mathbf{z}}
\newcommand{\bq}{\mathbf{q}}
\newcommand{\bff}{\mathbf{f}}
\newcommand{\bu}{\mathbf{u}}
\newcommand{\bh}{\mathbf{h}}
\newcommand{\bb}{\mathbf{b}}

\newcommand{\rone}{\textcolor{green}{R1}}
\newcommand{\rtwo}{\textcolor{orange}{R2}}
\newcommand{\rthree}{\textcolor{red}{R3}}
\usepackage{amsmath}
%\usepackage{arydshln}
\DeclareMathOperator{\similarity}{sim}
\DeclareMathOperator{\AvgPool}{AvgPool}

\newcommand{\argmax}{\mathop{\mathrm{argmax}}}     




\hyphenation{VPNeXt - Rethinking Dense Decoding for Plain Vision Transformer}
% updated with editorial comments 8/9/2021

\begin{document}

\title{VPNeXt : Rethinking Dense Decoding for \\ Plain Vision Transformer}

\author{Xikai Tang, Ye Huang\orcidlink{0000-0001-5668-5529},~\IEEEmembership{Member,~IEEE, } Guangqiang Yin and Lixin Duan\orcidlink{0000-0002-0723-4016}
        % <-this % stops a space
\thanks{Xikai Tang is with the School of Information and Software Engineering, University of Electronic Science and Technology of China}
\thanks{
Ye Huang, Guangqiang Yin and Lixin Duan are with the Shenzhen Institute for Advanced Study, University of Electronic Science and Technology of China, 518000 (Ye Huang is the corresponding author)}% <-this % stops a space% <-this % stops a space
}

% The paper headers
\markboth{Rethinking Dense Decoding for Plain Vision Transformer}%
{Shell \MakeLowercase{\textit{et al.}}: A Sample Article Using IEEEtran.cls for IEEE Journals}

%\IEEEpubid{\begin{tabular}[t]{@{}c@{}} Copyright © 20xx IEEE. Personal use of this material is permitted. \\However, permission to use this material for any other purposes must be obtained from the IEEE by sending an email to pubs-permissions@ieee.org.\end{tabular}}
% Remember, if you use this you must call \IEEEpubidadjcol in the second
% column for its text to clear the IEEEpubid mark.

\maketitle

\begin{abstract}


The choice of representation for geographic location significantly impacts the accuracy of models for a broad range of geospatial tasks, including fine-grained species classification, population density estimation, and biome classification. Recent works like SatCLIP and GeoCLIP learn such representations by contrastively aligning geolocation with co-located images. While these methods work exceptionally well, in this paper, we posit that the current training strategies fail to fully capture the important visual features. We provide an information theoretic perspective on why the resulting embeddings from these methods discard crucial visual information that is important for many downstream tasks. To solve this problem, we propose a novel retrieval-augmented strategy called RANGE. We build our method on the intuition that the visual features of a location can be estimated by combining the visual features from multiple similar-looking locations. We evaluate our method across a wide variety of tasks. Our results show that RANGE outperforms the existing state-of-the-art models with significant margins in most tasks. We show gains of up to 13.1\% on classification tasks and 0.145 $R^2$ on regression tasks. All our code and models will be made available at: \href{https://github.com/mvrl/RANGE}{https://github.com/mvrl/RANGE}.

\end{abstract}



\begin{IEEEkeywords}
Vision Transformer, Semantic Segmentation, Representation Learning
\end{IEEEkeywords}

\section{Introduction}
Backdoor attacks pose a concealed yet profound security risk to machine learning (ML) models, for which the adversaries can inject a stealth backdoor into the model during training, enabling them to illicitly control the model's output upon encountering predefined inputs. These attacks can even occur without the knowledge of developers or end-users, thereby undermining the trust in ML systems. As ML becomes more deeply embedded in critical sectors like finance, healthcare, and autonomous driving \citep{he2016deep, liu2020computing, tournier2019mrtrix3, adjabi2020past}, the potential damage from backdoor attacks grows, underscoring the emergency for developing robust defense mechanisms against backdoor attacks.

To address the threat of backdoor attacks, researchers have developed a variety of strategies \cite{liu2018fine,wu2021adversarial,wang2019neural,zeng2022adversarial,zhu2023neural,Zhu_2023_ICCV, wei2024shared,wei2024d3}, aimed at purifying backdoors within victim models. These methods are designed to integrate with current deployment workflows seamlessly and have demonstrated significant success in mitigating the effects of backdoor triggers \cite{wubackdoorbench, wu2023defenses, wu2024backdoorbench,dunnett2024countering}.  However, most state-of-the-art (SOTA) backdoor purification methods operate under the assumption that a small clean dataset, often referred to as \textbf{auxiliary dataset}, is available for purification. Such an assumption poses practical challenges, especially in scenarios where data is scarce. To tackle this challenge, efforts have been made to reduce the size of the required auxiliary dataset~\cite{chai2022oneshot,li2023reconstructive, Zhu_2023_ICCV} and even explore dataset-free purification techniques~\cite{zheng2022data,hong2023revisiting,lin2024fusing}. Although these approaches offer some improvements, recent evaluations \cite{dunnett2024countering, wu2024backdoorbench} continue to highlight the importance of sufficient auxiliary data for achieving robust defenses against backdoor attacks.

While significant progress has been made in reducing the size of auxiliary datasets, an equally critical yet underexplored question remains: \emph{how does the nature of the auxiliary dataset affect purification effectiveness?} In  real-world  applications, auxiliary datasets can vary widely, encompassing in-distribution data, synthetic data, or external data from different sources. Understanding how each type of auxiliary dataset influences the purification effectiveness is vital for selecting or constructing the most suitable auxiliary dataset and the corresponding technique. For instance, when multiple datasets are available, understanding how different datasets contribute to purification can guide defenders in selecting or crafting the most appropriate dataset. Conversely, when only limited auxiliary data is accessible, knowing which purification technique works best under those constraints is critical. Therefore, there is an urgent need for a thorough investigation into the impact of auxiliary datasets on purification effectiveness to guide defenders in  enhancing the security of ML systems. 

In this paper, we systematically investigate the critical role of auxiliary datasets in backdoor purification, aiming to bridge the gap between idealized and practical purification scenarios.  Specifically, we first construct a diverse set of auxiliary datasets to emulate real-world conditions, as summarized in Table~\ref{overall}. These datasets include in-distribution data, synthetic data, and external data from other sources. Through an evaluation of SOTA backdoor purification methods across these datasets, we uncover several critical insights: \textbf{1)} In-distribution datasets, particularly those carefully filtered from the original training data of the victim model, effectively preserve the model’s utility for its intended tasks but may fall short in eliminating backdoors. \textbf{2)} Incorporating OOD datasets can help the model forget backdoors but also bring the risk of forgetting critical learned knowledge, significantly degrading its overall performance. Building on these findings, we propose Guided Input Calibration (GIC), a novel technique that enhances backdoor purification by adaptively transforming auxiliary data to better align with the victim model’s learned representations. By leveraging the victim model itself to guide this transformation, GIC optimizes the purification process, striking a balance between preserving model utility and mitigating backdoor threats. Extensive experiments demonstrate that GIC significantly improves the effectiveness of backdoor purification across diverse auxiliary datasets, providing a practical and robust defense solution.

Our main contributions are threefold:
\textbf{1) Impact analysis of auxiliary datasets:} We take the \textbf{first step}  in systematically investigating how different types of auxiliary datasets influence backdoor purification effectiveness. Our findings provide novel insights and serve as a foundation for future research on optimizing dataset selection and construction for enhanced backdoor defense.
%
\textbf{2) Compilation and evaluation of diverse auxiliary datasets:}  We have compiled and rigorously evaluated a diverse set of auxiliary datasets using SOTA purification methods, making our datasets and code publicly available to facilitate and support future research on practical backdoor defense strategies.
%
\textbf{3) Introduction of GIC:} We introduce GIC, the \textbf{first} dedicated solution designed to align auxiliary datasets with the model’s learned representations, significantly enhancing backdoor mitigation across various dataset types. Our approach sets a new benchmark for practical and effective backdoor defense.



\begin{figure*}[t]
  \centering
    \includegraphics[width=1\linewidth]{visuals/final_registration.png}
    \caption{Target measurement process on low-cost scan data using ICP and Coloured ICP. (1) Initialisation: The source point cloud (checkerboard) is misaligned with the target point cloud. (2) Initial Registration using Point-to-Plane ICP: Standard ICP leads to suboptimal registration. (3) Final Registration using Coloured ICP: Colour information is incorporated after pre-processing with RANSAC and Binarisation with Otsu Thresholding for real data, resulting in improved alignment.}
    \label{fig:Registration_visualisation}
\end{figure*}

\subsection{Iterative Closest Point (ICP) Algorithm}
The Iterative Closest Point (ICP) algorithm has been a fundamental technique in 3D computer vision and robotics for point cloud. Originally proposed by \cite{besl_method_1992}, ICP aims to minimise the distance between two datasets, typically referred to as the source and the target. The algorithm operates in an iterative manner, identifying correspondences by matching each source point with its nearest target point \citep{survey_ICP}. It then computes the rigid transformation, usually involving both rotation and translation, to achieve the best alignment of these matched points \citep{survey_ICP}. This process is repeated until convergence, where the change in the alignment parameters or the overall alignment error becomes smaller than a predefined threshold.

One key advantage of the ICP framework lies in its simplicity: the algorithm is conceptually straightforward, and its basic version is relatively easy to implement. However, traditional ICP can be sensitive to local minima, often requiring a good initial alignment \citep{zhang2021fast}. Furthermore, outliers, noise, and partial overlaps between datasets can significantly degrade its performance \citep{zhang2021fast, bouaziz2013sparse}. Over the years, various modifications and improvements \citep{gelfand2005robust, rusu2009fast, aiger20084, gruen2005least, fitzgibbon2003robust} have been proposed to mitigate these issues. Among the most common strategies are robust cost functions \citep{fitzgibbon2003robust}, weighting schemes for correspondences \citep{rusu2009fast}, and more sophisticated techniques \citep{gelfand2005robust, bouaziz2013sparse} to reject outliers. 

In addition, there is significant interest in integrating additional information into the ICP pipeline. Instead of solely relying on geometric cues such as point coordinates or surface normals, recent approaches have proposed incorporating colour (RGB) or intensity data to enhance correspondence accuracy. These methods \citep{park_colored_2017, 5980407}, commonly known as "Colored ICP" employ differences in pixel intensities or colour values as additional constraints. This is particularly beneficial in situations where geometric attributes alone are inadequate for accurate alignment or where surfaces possess complex texture patterns that can assist in the matching process.

\subsection{Applications of Target Measurement}

One approach relies on the use of physical checkerboard targets for registration. \cite{fryskowska2019} analyse checkerboard target identification for terrestrial laser scanning. They propose a geometric method to determine the target centre with higher precision, demonstrating that their approach can reduce errors by up to 6 mm compared to conventional automatic methods.

\cite{becerik2011assessment} examines data acquisition errors in 3D laser scanning for construction by evaluating how different target types (paper, paddle, and sphere) and layouts impact registration accuracy in both indoor and outdoor environments and presents guidelines for optimal target configuration.

\citet{Liang2024} propose to use Coloured ICP to measure target centres for checkerboard targets, similar to our investigation. They use data from a survey-grade terrestrial laser scanner. Their intended application is structural bridge monitoring purposes. They report an average accuracy of the measurement below 1.3 millimetres.

Where targets cannot be placed in the scene, the intensity information form the scanner can still be used to identify distinctive points. For point cloud data that is captured with a regular pattern, standard image processing can be used in a similar way to target detection. For example, \citet{wendt_automation_2004} proposes to use the SUSAN operator on a co-registered image from a camera, \citet{bohm_automatic_2007} proposes to use the SIFT operator on the LIDAR reflectance directly and \citet{theiler_markerless_2013} propose to use a Difference-of-Gaussian approach on the reflectance information.
Most of these methods extract image features to find reliable 3D correspondences for the purpose of registration.

In the following we describe our approach to the measurement of the target centre. In contrast to most proposed methods above we focus on unordered point clouds, where raster-based methods are not available, and low-cost sensors, where increased measurement noise and outliers are expected. As we are not aware of a commercial reference solution to this problem, we start by conducting a series of synthetic experiments to explore the viability and accuracy potential of the approach.



%The reviewed studies primarily rely on physical targets or target-free methods and do not utilise 3D synthetic point cloud checkerboards. In contrast, our approach introduces synthetic point cloud checkerboards, which offer controlled and consistent target geometry and reduce variability caused by physical targets. This innovation has significant potential for commercialisation and industrial application.

\section{Methodology}

\subsection{Problem Definition}

Given a multivariate time series input $X \in \mathbb{R}^{C  \times T}$, multivariate time series forecasting tasks are designed to predict its future $F$ time steps $\hat{Y}\in \mathbb{R}^{C \times F}$ using past $T$ steps. $C $ is the number of variates or channels.

\subsection{Preliminary Analysis}

This section presents why RevIN~\citep{Kim_revin,liu2022non}, High-pass, and Low-pass filters fail to address the Mid-Frequency Spectrum Gap. Let the input univariate time series be $ x(t) $ with length $ T $ and target $ y(t) $ with length $ F $. 

\begin{definition}[Frequency Spectral Energy]\label{def:energy}
The Fourier transform of $x(t)$, $X(f)$, and its spectral energy $E_X(f)$ is given by:
\vspace{-0.2cm}
\begin{align}
X(f) = \sum_{t=0}^{T-1} x(t) e^{-i 2 \pi f t / {T-1}}, \quad &f = 0, 1, \dots, T-1\notag\\
E_X(f) = |X(f)|^2.
\end{align}
\vspace{-0.2cm}
\end{definition}

\textbf{Impact of RevIN on Frequency Spectrum \quad}
\begin{definition}[Reversible Instance Normalization]\label{def:RevIN}
Given a \textbf{forecast model} $ f: \mathbb{R}^T \rightarrow \mathbb{R}^F $ that generates a forecast $ \hat{y}(t) $ from a given input $x(t)$, RevIN is defined as:
\vspace{-0.2cm}
\begin{align}
&\hat{x}(t) = \frac{x(t) - \mu}{\sigma},\quad t = 0, 1, \dots, T-1\notag\\
&\hat{y}(t) = f(\hat{x}(t)), \quad \hat{y}(t)_{rev}= \hat{y}(t) \cdot \sigma + \mu,\notag\\
&\mu = \frac{1}{T} \sum_{t=0}^{T-1} x(t), \quad \sigma = \sqrt{\frac{1}{T} \sum_{t=0}^{T-1} (x(t) - \mu)^2}.
\end{align}
\vspace{-0.2cm}
\end{definition}

\begin{theorem} [Frequency Spectrum after RevIN] \label{theorem:RevIN}
\vspace{-0.2cm}
The spectral energy of $\hat{x}(t)$ (transformed using RevIN):
\begin{align}
E_{\hat{X}}(0)=0,& \quad f=0, \notag\\
E_{\hat{X}}(f) = \left( \frac{1}{\sigma} \right)^2 |X(f)|^2,&\quad f = 1,2,\dots, T-1 . 
\end{align}
\vspace{-0.2cm}
\end{theorem}
The proof is in Appendix~\ref{app:RevIN}. Theorem~\ref{theorem:RevIN} suggests that RevIN scales the absolute spectral energy by $ \sigma^2 $ but does not affect its relative distribution except $E_{\hat{X}}(0)=0$. Thus, RevIN preserves the relative spectral energy distribution and leaves the Mid-Frequency Spectrum Gap unresolved. \textit{However, our experiments still employ RevIN to ensure a fair comparison with other baselines.}
\begin{figure*}[h]
  \centering
  \includegraphics[width=1.\linewidth]{Faker/source/assets/jpg/ReFocus.jpg}
  \caption{General structure of \textbf{ReFocus}. `Adaptive Mid-Frequency Energy Optimizer (AMEO)' enhances mid-frequency components modeling, and `Energy-based Key-Frequency Picking Block' (EKPB) effectively captures shared Key-Frequency across channels}
  \label{fig:refocus}
\end{figure*}

\begin{figure*}[h]
  \centering
  \includegraphics[width=0.7\linewidth]{Faker/source/assets/jpg/ket.jpg}
  \caption{General process of the \textbf{Key-Frequency Enhanced Training strategy (KET)}, where spectral information from other channels is randomly introduced into each channel, to enhance the extraction of the shared Key-Frequency.}
  \label{fig:reshuffle}
\end{figure*}
\textbf{Impact of High- and Low-pass filter \quad}
We still define $\hat{x}(t)$ to be the filtered (processed) signal, obtained by applying a filter $H(f)$ (High/Low-pass filter). The filter $ H(f) $ is 1 in the passband (High/Low frequency) and 0 in the stopband (Middle frequency). So $E_{\hat{X}}(f)=0,\quad E_{\hat{X}}\leq E_X(f)$ for middle frequencies, which creates even larger gap.

\subsection{Overall Structure of The Proposed ReFocus}

In this section, we elucidate the overall architecture of \textbf{ReFocus}, depicted in Figure \ref{fig:refocus}. We define frequency domain projection as $D1\rightarrow D2$ representing a projection from dimension $D1$ to $D2$ in the frequency domain~\citep{xu2024fits}. Initially, we apply \textbf{AMEO} to the input $X \in \mathbb{R}^{C \times T}$, yielding the processed spectrum $ X_{am} \in \mathbb{R}^{C  \times T} $. Next, we use a projection $T\rightarrow D$ to transform $ X_{am}$ into the Variate Embedding $ X_{em} \in \mathbb{R}^{C  \times D}$~\citep{LiuiTransformer}. Then, $X_{em}$ go through $N$ \textbf{EKPB} to generate representation $H_{N+1}$, which is projected to obtain final prediction $\hat{Y}$. 

\textbf{Adaptive Mid-Frequency Energy Optimizer \quad}
Building upon the \textbf{Preliminary Analysis}, we propose a convolution- and residual learning-based solution to address the Mid-Frequency Spectrum Gap, which we denoted as AMEO. 
\begin{definition}[Adaptive Mid-Frequency Energy Optimizer]\label{def:AMEO}
AMEO is defined as:
\begin{align}
&\hat{x}(t) = x(t)-\frac{\beta}{K}\sum_{k=0}^{K-1} \tilde{x}(t+K-1-k),\notag\\
&\tilde{x}(t) =\notag\\
&\begin{cases}
x(t-(\frac{K}{2}+1)), \quad \text{if } \frac{K}{2}+1 \leq t < T+\frac{K}{2}+1, \\
0,  \quad\text{if } 0 \leq t < \frac{K}{2}+1 \text{ or } T+\frac{K}{2}+1 \leq t < T+K.
\end{cases}
\end{align}
\vspace{-0.2cm}
\end{definition}

It is equivalent to $x=x-\beta \cdot Conv(x)$. $Conv$ is a 1D convolution (Zero-padding at both ends, stride $s=1$, kernel size $K$, with values initialized as $ \frac{1}{K} $). $\beta \in \mathbb{R}^{1}$ is a hyperparameter.

\begin{theorem} [Frequency Spectrum after AMEO] \label{theorem:AMEO}
The spectral energy of $\hat{x}(t)$ obtained using AMEO:
\begin{align}
E_{\hat{X}}(f) =|X(f)|^2 \left\{1 - \beta \cdot \underbrace{\frac{1}{K} \sum_{k=0}^{K-1} e^{i 2 \pi f (\frac{3K}{2}-k -2) / {T-1}}}_{G(f)}\right\}^2
\end{align}
\vspace{-0.2cm}
\end{theorem}

The proof is in Appendix~\ref{app:AMEO}. We have $E_{\hat{X}}(f) =|X(f)|^2(1-\beta  \cdot G(f))^2$. Generally, $ G(f) $ behaves as a decay function, gradually reducing its value from \textbf{One} to \textbf{Zero}. Such \textbf{decay behavior} makes AMEO relatively enhances mid-frequency components, thus addressing the Mid-Frequency Spectrum Gap.

\textbf{Energy-based Key-Frequency Picking Block \quad} In each \textbf{EKPB}, the input $ H_i \in \mathbb{R}^{C  \times D} (H_1=X_{em}) $ is first processed through an MLP to generate $ H_i^k \in \mathbb{R}^{C  \times Q}$. Then, FFT is applied to get $ H_i^f \in \mathbb{R}^{C  \times (Q/2+1)}$. For $ H_i^f$, we calculate its energy, denoted as $ H_i^e \in \mathbb{R}^{C  \times (Q/2+1)}$. A cross-channel softmax is then applied to $ H_i^e$ per frequency to obtain a probability distribution $ H_i^{soft} \in \mathbb{R}^{C  \times (Q/2+1)}$. Using $H_i^{soft}$, we select values from $ H_i^f$ across channels for each frequency, resulting in $K^f_i \in \mathbb{R}^{1  \times (Q/2+1)}$, which represents the Shared Key-Frequency across all channels. Then iFFT is performed on $K^f_i$ to get $K_i\in \mathbb{R}^{1  \times Q}$, followed by projection $Q\rightarrow D$ and repeating (C times) to get $\hat{K}_i \in \mathbb{R}^{C  \times D}$. This $\hat{K}_i$ is point-wisely added to $\hat{H_i}\in \mathbb{R}^{C  \times D}$ , which is the projection of $ H_i$ using projection $D\rightarrow D$. Then, an MLP and $Add\&Norm$ is applied to the result $HK\in \mathbb{R}^{C  \times D}$ to fuse inter-series dependencies information, and another MLP and $Add\&Norm$ is used to capture intra-series variations~\citep{LiuiTransformer}. The output of each \textbf{EKPB} is $\hat{O_i} \in \mathbb{R}^{C  \times D}$, where $H_{i+1}=\hat{O_i}$.

\subsection{Key-Frequency Enhanced Training strategy}

In real-world time series, certain channels often exhibit spectral dependencies, which may not be fully captured in the training set, and the specific channels with such dependencies are also unknown~\citep{geweke1984freqchannel,Zhao2024freqchannel}. So this work borrows insight from recent advancement of mix-up in time series~\citep{zhou2023mixup,ansari2024mixup}, randomly introducing spectral information from other channels into each channel, to enhance the extraction of the shared Key-Frequency, as in Figure~\ref{fig:reshuffle}. Given a multivariate time series input $X \in \mathbb{R}^{C \times T}$ and its ground-truth $Y \in \mathbb{R}^{C \times F}$, we generate a pseudo sample pair: 

\begin{align}
X' = iFFT(FFT(X) +\alpha \cdot FFT(X[\text{perm},:]))&,  \notag\\ 
Y' = iFFT(FFT(Y) +\alpha \cdot FFT(Y[\text{perm},:]))&.
\end{align}

$\alpha \in \mathbb{R}^{C \times 1}$ is a weight vector sampled from a normal distribution, $\text{perm}$ is a reshuffled channel index. Since $FFT$ and $iFFT$ are linear operations, this mix-up process can be equivalently simplified in the \textbf{Time Domain}:
\begin{align}
X' = X +\alpha \cdot X[\text{perm},:]&,  \notag\\
Y' = Y +\alpha \cdot Y[\text{perm},:]&
 \end{align}
We alternate training between real and synthetic data to preserve the spectral dependencies in real samples. This combines the advantages of data augmentation, such as improved generalization, while mitigating potential drawbacks like over-smoothing and training instability~\citep{ryu2024tf,alkhalifah2022tf}.












%%
%%---------------------------------------------
\section{Implementation Experience}
\label{sec:implementation}
%%
\noindent
%
In this section we describe our approach to implement a MEC Customer Orchestrator and to support MEC application slicing using Kubernetes and Helm technologies. Although our implementation is still at a work-in-progress state, our proof-of-concept prototype (hereafter simply PoC), shows the feasibility of our design approach and allowed us to collect a preliminary insight on the efficiency of application slicing using Kubernetes resource management capabilities. In the following, we first briefly overview Kubernetes and Helm features. Then, we detail the implementation of the new APIs and functional components  of our PoC,  namely the MECO and the ACF Image Repository. Finally, we describe our practical approach to support slicing of Kubernetes resources. For the sake of presentation clarity, Figure~\ref{fig:poc} overviews the architecture of the PoC and its internal components.
%
\begin{figure*}[ht]
    \centering
    \includegraphics[clip,trim= 1cm 4.5cm 0cm 2.5cm,width=0.7\textwidth]{figures/fig_poc.pdf}
   \caption{PoC architecture and internal components.}
    \label{fig:poc}
    \vspace{-0.2cm}
\end{figure*}
%
%%
%%---------------------------------------------
\subsection{PoC enabling technologies}
\label{sec:poc_tech}
%%
\noindent
%
Kubernetes is an open-source container life-cycle manager and orchestrator, which is the de-facto standard for running container-based cloud native applications on a cluster of (physical or virtual) machines (called \textit{nodes}). It is out of the scope of this paper to describe the complete Kubernetes architecture and high-level abstractions, but we focus on the components that are most relevant for our PoC. 

A Kubernetes cluster is composed of $(i)$ a set of \textit{worker} nodes that run containerized applications, also called \textit{workloads}; and $(ii)$ (at least) a \textit{master} node that runs the services of the \textit{control plane}, and it is responsible to enforce the desired state of the cluster. Kubernetes provides several built-in workload resources to support various application behaviours (e.g., stateless tasks) and management functions (e.g., creating or deleting application replicas). Each workload must run into a \textit{Pod}, a Kubernetes object that represents a collection of containers running in the same execution environment, which share the same storage, networking, and lifecycle. Pods runs into worker nodes, which host an agent, called \texttt{kubelet}, that is responsible for managing worker's local containers and for synchronising the status with the master node. As better explained later, another main component of the worker node is the \texttt{kube-proxy}, which is responsible for implementing Kubernetes networking services and to enable communication to Pods from inside and outside the cluster. The master node is composed of different components including:$i)$ \textit{etcd}, a distributed key-value store that holds and manages all cluster critical data; $ii)$ \texttt{kube-apiserver}, a component providing a REST-based frontend to the control plane through which all other components interact; $iii)$ \texttt{kube-controller}, a component that monitors the shared state of the cluster using apiserver and runs the controller processes; and $iv)$ \texttt{kube-scheduler}, a component that assigns newly created Pods to nodes.  Note that Kubernetes also support the \textit{namespace} abstraction, namely virtual clusters that share the same IP Address and port space, facilitating the grouping and organisation of objects.    

Networking is a central part of the Kubernetes design and a fundamental capability for application slicing. Specifically, the Kubernetes network model demands certain network features, such as every Pod should have a unique, routable IP inside the cluster, and inter-pod communications should happen without using NATs, regardless of wherever the Pods reside or not on the same worker nodes (i.e. network segmentation is not allowed). Since IP addresses of Pods are ephemeral and change whenever the Pods are restarted or migrated, Kubernetes also defines \textit{Service} resources, namely REST objects that are used to group identical Pods together to provide a consistent means of accessing them, e.g by bounding them to a virtual IP address (called cluster IP) that never changes. It is important to point out that 
Kubernetes does not directly handle the networking aspects, but it rather allows the use of third-party networking plugins that adhere to the Container Network Interface (CNI) specification\footnote{\url{https://github.com/containernetworking/cni}} to manage the containers' data plane. Of particular relevance for our PoC, is Kube-OVN\footnote{\url{https://www.kube-ovn.io/}}, a CNI plugin that integrates network virtualisation into Kubernets  by leveraging the OVN (Open Virtual Network)\footnote{\url{https://www.ovn.org/en/}} technology. Kube-OVN supports advanced features, such as unique subnets per namespace, network policies, namespaced gateways, subnet isolation and dynamcic QoS. We extensively leverage some of those features to support application slicing in our PoC. 

Another technology we use as a basis for our PoC is Helm\footnote{\url{https://helm.sh/}}, an application packaging manager for Kubernetes. Specifically, Helm defines a data format, called \textit{Helm Chart}, to bundle a set of Kubernetes object definition files (i.e. files describing properties of Kubernetes objects) into a single package. This permits to manage the instantiation, upgrade and deletion of the corresponding Kubernetes objects as they were a single entity. The Helm charts are stored into a separate repository, and every time a new instance of the same chart is installed into Kubernetes, a new chart \emph{release} is created. Furthermore, Helm allows to augment Kubernetes object definition files with Helm template commands. By providing Helm a list of arguments for these template commands at chart instantiation time, it is possible to dynamically customise the chart before it is actually deployed.  Examples of these customisation span from overriding object default values with the one passed as command arguments (e.g. a port exposed by a container, the namespace name, etc..) to dynamically enabling disabling chart sections. As explained later, we extensively leveraged Helm features to implement different components of our PoC, such as the KDPManager and the MAPSS Chart Registry (see Figure~\ref{fig:poc}).

We conclude this section by noting that the ETSI MEC standard has recently started analysing how MEC features should be adjusted when deploying a MEC system using a container-based virtualisation technique~\cite{MEC027}. Furthermore, a few initial implementations exist of specific MEC components and interfaces using Kubernets as VIM, such as Akraino\footnote{\url{https://www.lfedge.org/projects/akraino/}} and LightEdge\footnote{\url{https://lightedge.io/}}. A recent work~\cite{2020_broadnets_mec_k8s} analyses how to use Kubernetes not only as a VIM, but also as the core of the MEPM, also leveraging Helm technology for the life-cycle management of MEC applications. 
%
%%
%%---------------------------------------------
\subsection{PoC design and development}
\label{sec:poc_design}
%%
\noindent
%
One of the main objectives of our proof-of-concept implementation is to demonstrate how Kubernetes can natively support multiple, isolated instances of MEC application slice subnets (MAPSSs for brevity) as defined in Section~\ref{sec:pre_concepts}. The key building block of a MAPSS is the ACF. For the sake of simplicity, in our PoC we ignore VNFs and we assume that ACF Suppliers can provide ACFs in the form of Docker container images coupled with an ACF user manual or descriptor (called ACFD). Then, an ASP leverages the ACFDs to select the set of ACFs that are needed to build the AS requested by its end customers, as well as to derive the proper run-time configuration of the graph of ACFs composing the AS. Therefore, a key component of our PoC is the \textit{ACF registry}, where authorised ACFDs are published and stored. The ACF registry is implemented using the open-source Docker Registry 2.0 application\footnote{\url{https://hub.docker.com/_/registry}}, a storage and distribution system for named Docker images. A generic ACFD is structured into two separate sections. The first one details the RESTful APIs exposed by the ACF (OpenAPI\footnote{\url{https://www.openapis.org/}} is used to specify these APIs in a standard, language-agnostic format). The second section details how to properly configure the ACF parameters in order to control how the ACF will behave at run-time. The ACF parameter customisation is a crucial aspect to consider, especially in the context of ACF composition and automated orchestration. An example of ACF customisation is the selection of buffer sizes, which may influence the run-time behaviours and performance of ACFs, directly affecting the fulfilment of SLA requirements. For the sake of simplicity, in our PoC we use \textit{environmental variables} to pass configuration paramters to the running container images of ACF. Finally, it is important to point out that the ACF registry must be accessed also by the \texttt{kube-scheduler} agent to fetch the container images of ACFs to be deployed.

The other key component of our PoC is the MECO that we have implements as an additional component of the master node, using Go as programming language. Internally, the MECO component is composed of three different modules: $i)$ the \textit{MECO API Server} to enable communications with the MEC APSSMF; $ii)$ the \textit{MAPSS Chart Repository} to store the templates of Kubernetes deployment plans of MAPSS; and $iii)$ the \textit{KDPManager}, to manage the Kubernetes deployment plans of run-time instances of activated MAPSS. In the following, we elaborate on the purpose and operations of each module more in detail. 

The main role of the MECO API Server is to act as a lifecycle management proxy for MAPSSs. Specifically, it receives commands for the instantiation, update and termination of MAPSS from the MEC APSSMF, and translates these commands into suitable Kubernetes actions. To this end, the MECO API Server exposes a RESTful management interface, called \texttt{mapss\_mm} API, defined using the OpenAPI description language (see Figure~\ref{fig:poc}). For the sake of the experimentation, the \texttt{mapss\_mm} API currently implements a \texttt{POST} method, which allows the MEC APSSMF to request the instantiation of a new MAPSS instance. The payload of this \texttt{POST} method carries a descriptor, called MAPSSD, which contains all the necessary information to allow the MECO to instantiate at run-time a specific MAPSS instance. The \texttt{mapss\_mm} API also implements a DELETE method, which allows to delete a running instance of a MAPSS. It is clear from this discussion, that the MAPSSD plays a key role in our PoC. We envision a MAPSSD organised into four different sections, as also illustrated in the example in Figure~\ref{fig:poc_massd_example}. The first parts includes the unique identifier of the MAPSS instance (\texttt{mappsiId}), a human readable description of the slice subnet features, and the identifier of an implementation template to be used for the deployment of the MAPSS instance (\texttt{mapssImplTemplateId}) -- see later this section for more details on how to use the \texttt{mapssImplTemplateId}. The second section includes a set of ``\textit{slice-subnet-wise}'' (computational, storage and networking) resource requirements. For example, the MAPSSD shown in Figure~\ref{fig:poc_massd_example} requires two dedicated CPU cores, 8 GB of memory and 100 GB of permanent storage, which will be shared among its constituent ACFs. The third part includes the list of ACFs that compose the MAPSS instance. Each ACF is associated to a unique identifier (\texttt{acfId}), and ``\textit{acf-wise}'' resource requirements can also be specified. For example, the acf1 in Figure \ref{fig:poc_massd_example} requires a dedicated CPU core out of the two dedicated to the whole slice subnet. Moreover, the field \texttt{customParams} can be leveraged to pass arguments (e.g. buffer size in figure) to configure specific ACF behaviours. Finally, the last section includes a list of virtual links among pairs of ACFs and their networking requirements (e.g. maximum usable bandwidth). It is worth noting that the data format of the MAPSSD is agnostic from the underlying virtualisation technology. In other words, the MEC APSSMF could leverage the same data model to interact with a MECO that relies on a different VIM than Kubernetes. Another advantage of the proposed MAPSSD is that it allows the MEC APSSMF to seamlessly integrate SLS requirements that address needs of different architectural levels (i.e. specific to the entire slice subnet, individual ACFs and virtual links between ACFs) in the same data object. 
%
\begin{figure}[th]
    \centering
    \includegraphics[clip,trim= 0cm 16cm 10cm 0cm,width=0.5\textwidth]{figures/fig_poc_mappsd.pdf}
   \caption{Illustrative example of a MEC application slice subnet descriptor (MAPSSD)}
    \label{fig:poc_massd_example}
\end{figure}
%

Clearly, the translation from a VI-agnostic MAPSSD into a Kubernetes deployment plan (\textit{KDP} for short) of the MAPSS instance, namely a package of properly configured Kubernetes objects implementing the requested MAPSS instance, is a critical functionality of the MECO. Following the approach proposed in~\cite{NGMN028} for supporting cost-efficient customisation of network slices, we assume that the MECO hosts a pre-loaded set of MAPSS templates/blueprints that can be used to speed up the creation of a MAPSS instance. Specifically, we implemented each MAPSS KDP template as an Helm chart that includes a set of pre-configured Kubernetes objects. These objects define: $i)$ ACFs Docker containers to run (e.g. via Kubernetes Pods objects), $ii)$ ACFs behavioural parameters (e.g. via environmental variables in Kubernetes ConfigMaps objects), $iii)$ ACFs connection points (i.e. exposed ports), $iv)$ custom scheduling policies (e.g., number of replicas, failure behaviour, etc.), and any other Kubernetes feature that is necessary for the correct deployment of the MAPSS instance. Then, the \textit{mapssImplTemplateId} field of the MAPSSD is used to retrieve the correct MAPSS KDP template. It is important to point out that the MECO should be able to dynamically customise at run-time the MAPSS KDP template using information derived from the MAPSSD (e.g., container resource requirements). To this end we leverage built-in objects and control structures of Helm template that provide access to values passed into an Helm chart, and the ability to include conditions in the template's generation. In the current implementation, we limit such customisation to the selection of: $i)$ the name of the namespace to which objects will belong; $ii)$ the computational, storage and networking requirements for the namespace; $iii)$ the computational and storage requirements per ACF, and the networking requirements per ACF pairs. In our Poc the MAPSS KDP templates are stored in the MAPSS Chart Registry (see Figure~\ref{fig:poc}). According to the operational and management roles defined in Section~\ref{subsec:roles}, the MEC Customer plays the roles of both the MEC operator and the ASP. Thus, the MEC Customer has the necessary expertise not only to properly select, compose and configure ACFs to provide an AS, but also to select and properly configure the subset of Kubernetes objects that allows to implement the MAPSS instance of the designed AS. Finally, the KDPManager module is simply a wrapper of the Helm library, which allows the MECO to embed the Helm functionalities.   

%
\begin{figure*}[ht]
    \centering
    \includegraphics[clip,trim= 0cm 2cm 0cm 2cm,width=0.9\textwidth]{figures/fig_poc_seq_diag.pdf}
   \caption{Sequence of operations to instantiate a new MAPSS in the PoC architecture.}
    \label{fig:poc_seq_diag}
\end{figure*}
%
We can now discuss the sequence of operations and request/response exchanges that are executed to deploy a new MAPPS instance in our Poc, which are also graphically illustrated in Figure~\ref{fig:poc_seq_diag}. First of all, the MEC APSSMF initiates the deployment process by sending a \texttt{POST} request to the MEC API Server over the \texttt{mapss\_mm} interface (see Figure~\ref{fig:poc}), which contains the MAPSSD of the requested MAPSS. In the figure, the requested MAPSS is identified as \textit{demoSlice}, while its Helm chart template is identified as \textit{demoTpl}. In \textit{step2}, the MECO API Server performs a preliminary analysis of the MAPSSD to discover the set of parameters that can be modified to customise the template. Furthermore, the MECO API Server retrieves from the \textit{mapssImplTemplateId} field of the MAPSSD the identified of the associated Helm chart template (\textit{dempoTpl} in this example). Finally, \textit{step2} is concluded with the API Server that instructs the KDPManager to deploy the \textit{demoTpl} Helm chart with the correct set of chart arguments. Subsequently, in \textit{step3}, the KDPManager fetches the \textit{demoTpl} Helm chart from the MAPSS Chart Repository, and it starts the deployment process. First, in \textit{step4}, the KDPManager contacts the \texttt{kube-apiserver} to create a new Kubernetes namespace with name \textit{demoSlice}. Then, the KDPManager applies (\textit{step5}) the customised arguments (e.g., number of cores to be assigned to an ACF container), and starts (\textit{step6}) chart release process, using \textit{dempoSlice} as release name. This process involves the generation of the proper set of Kubernetes objects definition files (i.e. the Kubernetes Deployment Plan). Once the KDP is complete, the KDPManager instructs the \texttt{kube-apiserver} to create the Kubernetes objects in the etcd database (\textit{step7}). Finally, in \textit{step8}, the \texttt{kube-controller} starts performing control actions according to the received Kubernetes objects. The latter includes contacting the ACF Image Repository to fetch ACFs container images for scheduling. For the sake of completeness, we remind that a termination of a run-time instance of a MAPSS is initiated by a \texttt{DELETE} request sent by the MEC APSSMF to the MECO via the \texttt{mapss\_mm} interface. This \texttt{DELETE} contains the \textit{mapssId}) of the MAPSS instance to delete. In this case, the MECO API Server requests the KDPManager to uninstall the chart release associated to \textit{mapssId}. Finally, the KDPManager removes all the Kubernetes objects of the release from Kubernetes cluster, and deletes the \textit{mapssId} namespace.

We complete the presentation of our PoC by explaining how application slice isolation is enforced using the Kubernetes resource control objects. First of all, we create a new Kubernetes namespace for each MAPSS instance, with a name equal to the \textit{mapssId}. All Kubernetes objects within a MAPSS instance are deployed using the same MAPSS namespace. We leverage a combination of Kubernetes \textit{ResourceQuota} objects, \textit{QoS classes} for Pods, and Kubernetes \textit{VolumeClaim} requests to limit the amount of computational and storage resources that could be consumed by both the whole namespace (to enforce requirements for invidual slice subnets) and individual Pods (to enforce requirements for individual ACFs). Network isolation between different instances of MAPSSs is implemented by exploiting Kube-OVN network policies, so that that the traffic arriving from Pods belonging to other namespaces is blocked (except for the system namespace). Finally, we implement per Pod ingress/egress rate limitation via Kube-OVN QoS policies. The implementation of more advanced QoS-aware network control policies (e.g. latency assurance, QoS tagging, etc.), and fine-grained network isolation policies (e.g. tunable network isolation degree with exception handling, etc) is planned as future work.


We conclude this section by  observing that an ASC could discover available ASes and their features by querying a catalogue that is exposed by a web portal (see Figure~\ref{fig:poc}), on which ASPs publish the descriptors of their ASes. We can foresee that AS descriptors include information such as $i)$ high level description of the offered AS; $ii)$ a pointer to the ASP offering the AS; $iii)$ a set of achievable SLAs (e.g. maximum resolution of a video processing service); and, possibly, $iv)$ billing information. The definition of a data model for AS descriptors is out of the scope of our present work. 
%
%%
%%---------------------------------------------
\subsection{Open implementation gaps}
\label{sec:poc_limitations}
%%
\noindent
%
During the implementation of our PoC we also faced several difficulties due to the limitations of the technologies and standards we have used. In the following, we summarise the main technological gaps we have observed to highlight areas of future investigations.

\begin{itemize}[noitemsep,topsep=2pt]
    \item ETSI MEC specification has defined the methods and the data formats for the \texttt{Mm1} reference interface between the OSS and the MEO, which is used to trigger the instantiation and the termination of MEC applications in the MEC system. However, the \texttt{Mm1} implicitly consider a MEC application as a single application package. For instance, in the MEC-in-NFV architecture, the \texttt{Mm1} allows the MEAO to  deploy a single VNF onboarded as a VNF descriptor. In our use case, a MAPSS instance represents a set of ACFs, and it could be conveniently modelled as a graph. To some extent the \texttt{Mm1} interface should be expanded to resemble the capabilities of the \texttt{Os-Ma-nfvo} interface between the NSSMF and the NFVO~\cite{NFV-SOL005} , which allows the NSSMF to request a network service (i.e. a collection of VNFs) to the NFVO.
    %
    \item Kubernetes ResourceQuota objects only permit to limit the amount of CPU and memory resources that Pods in a namespace could use. Pods can get assigned to a ``Guaranteed'' (highest priority) QoS class to receive reserved CPU and memory resources. VolumeClaim requests allow to reserve storage resources to a scheduled Pod. However, Kubernetes does not provide a straightforward mechanism to allocate resources at the namespace level but only at Pod level. This limitation complicates the implementation of resource over-provisioning strategies in dynamic slicing context (e.g. when a slice subnet is assigned more resources than needed to accommodate future demand changes).
    %dynamic slicing (e.g. when an application slice changes the set of deployed ACFs) 
    %
    \item Default scheduling mechanisms available in Kubernetes take into account only CPU and RAM usage rates when scheduling Pods, while network-related metrics (e.g. latency or bandwidth usage rates) are often ignored. However, a network-aware resource allocation and scheduling is crucial for our application slicing model, and initial proposals can be found in~\cite{2019_netsoft_netaware_kube} and~\cite{2020_noms_delayaware_kube}.
    %
    \item Kube-OVN allows to limit the transmission rate on both ingress and egress traffic at the Pod level. This is obtained by using a QoS-aware queue and traffic policing at the vswitch port to which the Pod is connected. However, Kube-OVN does not support to set up rate limits on individual traffic flows, which is an useful feature if a Pod needs to communicate with several other Pods (e.g., for inter-slice communications). A possible workaround is to leverage Multus CNI\footnote{\url{https://github.com/k8snetworkplumbingwg/multus-cni}}, a CNI plugin for Kubernetes that enables attaching multiple network interfaces to pods, to allow a Pod to have a dedicated virtual interface (i.e. network port) for each destination Pod. Then, separated instances of Kube-OVN could be installed on each virtual interface to enforce different QoS policies at the port level. Furthermore, bandwidth reservation mechanisms similar to the ones proposed for SDN-based networks~\cite{2016_bwguar_openflow} should be included in Kube-OVN. 
    %
    \item Service chaining allows to link together VNFs to compose service function chains (SFCs). The implementation of SFCs usually requires support from the network (e.g. via SDN) to route a packet from one VNF to the next in the chain. However, service chaining (which is a crucial feature for integrating VNFs with ACFs in a MAPSS is missing in Kubernetes. Recently, a few projects, such as OVN4NFV K8s Plugin\footnote{\url{https://github.com/opnfv/ovn4nfv-k8s-plugin}} and Service Meshes\footnote{\url{https://istio.io/}} have been initiated to provide support for service chaining in Kubernetes environments . 
    
    %and Network Service Mesh\footnote{\url{https://networkservicemesh.io/}} have been initiated to provide support for service chaining in Kubernetes environments . 
    %
    \item The MAPSSD provides the blueprint for building an application slice subnet within a MEC environments. For the sake of our PoC, we have defined a custom data model for specifying a MAPPSD. On the other hand, standard modelling languages exist, such as TOSCA (Topology and Orchestration Specification for Cloud Applications) for describing the components of a cloud application and their relationships, which facilitate the interoperability, portability and orchestration in a multi-cloud environment~\cite{2018_MCC_TOSCA}. ETIS MANO already advocates the use of TOSCA to specify NFV descriptors~\cite{NFV-SOL001}. In principle, TOSCA could also be leveraged to specify the MAPSS descriptors. However, TOSCA is specifically designed to model classical cloud applications and it needs some adaptations to natively support also contenarised applications. Several approaches have been recently proposed to either $(i)$ extend the TOSCA normative types for support of container-based orchestration platforms (e.g. Cloudify\footnote{\url{https://cloudify.co/}}); or $(ii)$ to decouple the application modelling from the application provisioning by developing ad hoc software connectors between TOSCA workflow and cloud provider’s API (e.g. TORCH~\cite{2021_jgc_torch}). However, no standard specifications have been released yet.  
    %
\end{itemize}



\section{Experiments on Pascal Context Dataset}
%\raggedbottom
The Pascal Context~\cite{cPascalContext} dataset comprises 4,998 training images and 5,105 testing images. We utilize its 59 semantic classes to perform ablation studies and experiments, following common practice. Unless otherwise specified, we train the models on the training set for 20K iterations.

In the ablation studies, we follow the VPNeXt's forward propagation sequence. 
%
First, we assess the effectiveness of VCR alone, and then we incorporate ViTUp to evaluate its ability to upsample the feature maps produced by VCR.
%
Finally, we conducted an analysis of computational overhead to evaluate the efficiency of our proposed VPNeXt.

%%%%%%%% VCR ablation studies %%%%%%%%%%%%%%%%%%%%%%%%%%%%%%

\subsection{Ablation studies on VCR}
We compare our proposed VCR with a mask decoder (w/o pyramid, \eg segmenter~\cite{cSegmenter}) and deep supervision, as discussed in previous sections. 
%
As shown in Table~\ref{tab:exps:vcr-ablation-studes}, incorporating visual context in deep supervision results in an even better mIOU than the mask decoder (68.83\% vs 67.88\%). 

Additionally, we conducted ablation studies to determine the optimal number of deep supervision layers to use. 
%
The results in Table~\ref{tab:exps:vcr-ablation-studes} indicate that the mIOU reaches its highest value when two intermediate layers are employed for VCR-oriented deep supervision.


\begin{table}[ht]
    \centering
    \caption{Ablation studies on VCR, all the results are obtained under single-scale without flipping.
    All baseline models are trained using the same backbone and settings.
    \textit{DS:} Deep supervision.
    }
    \resizebox{\linewidth}{!}{
    \begin{tabular}{c|c|c}
       \toprule
       Methods & Num\# DS layers &  \quad mIOU(\%)\quad \\
       \midrule
       Deep supervision & 2 & 66.50 \\
       \midrule
       Mask decoder  & 2 (implicit) & 67.88  \\
       w/o pyramid & & \\
       \midrule
       Our VCR & 1 & 68.43\\
        & \textbf{2} & \textbf{68.83}\\
        & 3 & 68.56 \\
       \bottomrule
    \end{tabular}
    }
    \label{tab:exps:vcr-ablation-studes}
\end{table}

\begin{table}[ht]
    \centering
    \caption{Ablation studies on ViTUp, all the results are obtained under the single-scale without flipping.
    All baseline models are trained using the same backbone and settings.
    }
    \resizebox{\linewidth}{!}{
    \begin{tabular}{c|c|c}
       \toprule
       Methods & Num\# HiCLR layers&  mIOU(\%) \\
       \midrule
       Bilinear  & 0 & 68.83 \\
       \midrule
       Mock pyramid & 2 & 69.01 \\
       \midrule
       Our real pyramid  & 1 & 69.50 \\
        & 2 & 69.87 \\
        & \textbf{3} & \textbf{70.00}\\
        & 4 & 69.81 \\
        & 5 & 69.43 \\
       \bottomrule
    \end{tabular}
    }
    \label{tab:exps:ViTUp-ablation-studes}
\end{table}

\begin{table}[ht]
    \centering
    \caption{
    Computational cost analysis for VPNeXt.
    All baseline models use the same backbone and settings.
    }
    \resizebox{\linewidth}{!}{
    \begin{tabular}{c|c|c}
       \toprule
       \quad Methods \quad \quad &  
       Pyramid Upsampler \quad &
       \quad GFlops \quad \quad  \\
       \midrule
       Deep supervision & - & 356.69\\
       \midrule
       Mask decoder & & 359.99  \\
       & \checkmark & > 2000 \\
       \midrule
       Our VPNeXt & & 356.69 \\
        & \checkmark & 1007.62 \\
       \bottomrule
    \end{tabular}
    }
    \label{tab:exps:cost-ablation-studes}
\end{table}



\begin{table}[ht]
\centering
\small
\caption{
% \hy{
Comparisons to state-of-the-art methods on Pascal Context dataset.
%
\textit{SS}: Single-scale performance w/o flipping.
\textit{MF}: Multi-scale performance w/ flipping.
``-'' in column \textit{SS} indicates that this result was not reported in the original paper.
% }
}
\resizebox{\linewidth}{!}
{\def\arraystretch{1} \tabcolsep=0.55em 
\begin{tabular}{l|c|c|c|c}
\toprule%[1pt]
Methods & Backbone & Avenue &\multicolumn{2}{c}{mIOU(\%)} \\
& & & SS & MF \\
\midrule
\midrule
SETR~\cite{cSETR}           & ViT-L           & CVPR'21 & - & 55.8 \\
DPT~\cite{cDPT}             & ViT-Hybrid      & ICCV'21 & - & 60.5 \\
OCNet~\cite{cOCNet}         & HRNet-W48       & IJCV'21 & - & 56.2 \\
CAA~\cite{cCAA}             & EfficientNet-B7 & AAAI'22 & - & 60.5 \\
CAA + CAR~\cite{cCAR}        & ConvNeXt-L      & ECCV'22 & 62.7 & 63.9 \\
SegNeXt~\cite{cSegNeXt}     & MSCAN-L         & NIPS'22 & 59.2 & 60.9 \\
SegViT~\cite{cSegViT}       & ViT-L            & NIPS'22 & - & 65.3 \\
SenFormer~\cite{cSenFormer}  & Swin-L          & BMVC'22 & 63.1 & 64.5\\
TSG~\cite{cTSG}             & Swin-L           & CVPR'23 & - & 63.3 \\
IDRNet~\cite{cIDRNet}        & Swin-L           & NIPS'23 & - & 64.5 \\
APPNet~\cite{cAPPNet}       &SenFormer-L       & TCSVT'23 & - & 63.7 \\
ViT-Adapter-L~\cite{cViTAdapter} & ViT-L       & ICLR'23 & 67.8 & 68.2 \\
InternImage~\cite{cInternImage} & InternImage-H & CVPR'23 & - & 70.3 \\
PFT~\cite{cPFT}             & ResNet-101       & TMM'24 & 55.2 & 57.3 \\
CART~\cite{cCART}           &EfficientNet-L2   & TCSVT'24 & 66.0 & 67.5 \\
HFGD~\cite{cHFGD}           &ConvNeXt-L        & TCSVT'24 & 64.9 & 65.6 \\
SILC~\cite{cSILC}           &SILC-C-L          & ECCV'24 & - & 61.5 \\
\midrule%[0.1pt]
VPNeXt (w/o ViTUp)         & ViT-L & - & \textbf{68.8} & \textbf{69.7} \\
VPNeXt          & ViT-L & - & \textbf{70.0} & \textbf{71.1}  \\
\bottomrule%[1pt]
\end{tabular}
}
\label{tab:SOTA-PascalContext}
\end{table}


\subsection{Ablation studies on ViTUp}
We then assess the mIOU of our proposed ViTUp. 
%
As shown in Table~\ref{tab:exps:ViTUp-ablation-studes}, the real pyramid feature provided by our ViTUp, enhanced by HiCLR, reached 69.50\% mIOU, significantly outperforms both bilinear interpolation and mock pyramids (69.50\% vs 68.83\% vs 69.01\%). 
%
Furthermore, applying refinement three times yields 70.00\% mIoU, making it the best ViTUp configuration for VPNeXt.



\subsection{Computational cost analysis}
To demonstrate the high efficiency of VPNeXt, we conducted a computational analysis on two setups: VCR (VPNeXt w/o pyramid upsampler) and the complete VPNeXt with ViTUp.
%
For fair comparisons, we utilized Segmenter~\cite{cSegmenter} as the Mask decoder w/o a pyramid upsampler, and Mask2Former-based~\cite{cMask2Former} Vit-adapter~\cite{cViTAdapter} and PlainSeg~\cite{cPlainSeg} as Mask decoders with/a pyramid upsampler.

Table~\ref{tab:exps:cost-ablation-studes} shows that VCR and deep supervision have the same Flops, indicating that VCR provides high-quality representations without adding any computational overhead (see previous subsections for details).
%
Table~\ref{tab:exps:cost-ablation-studes} also shows that ViTUp delivers high-resolution pyramid features and strong mIoU while having significantly lower computational overhead compared to previous mask decoders that rely on mock pyramid features.




\subsection{Compare with state-of-the-arts}
To fully showcase the performance superiority of VPNeXt, we compared it with state-of-the-art methods on the Pascal Context dataset.
%
Note that, only methods published by the time this paper was completed can be compared.
%
As shown in Table~\ref{tab:SOTA-PascalContext}, our proposed VPNeXt significantly outperforms the compared methods, including the previous state-of-the-art techniques ViT-Adapter and InternImage. 
%
Moreover, even without using ViTUp (\ie with only VCR), VPNeXt still outperforms most methods.


\begin{table}[th!]
\centering
\small
\caption{
% \hy{
Comparisons to state-of-the-art methods on COCOStuff164k dataset.
%
\textit{SS}: Single-scale performance w/o flipping.
\textit{MF}: Multi-scale performance w/ flipping.
``-'' in column \textit{SS} or \textbf{MF} indicates that this result was not reported in the original paper.
% }
}
\resizebox{\linewidth}{!}
{\def\arraystretch{1} \tabcolsep=0.55em 
\begin{tabular}{l|c|c|c|c}
\toprule%[1pt]
Methods & Backbone  & Avenue &\multicolumn{2}{c}{mIOU(\%)}\\
& & & SS & MF \\
\midrule
\midrule
OCR~\cite{cOCR,cHRFormer} & HRFormer-B & NIPS'21 & - & 43.3 \\
SegFormer~\cite{cSegFormer} & MiT-B5 & NIPS'21 & - & 46.7 \\
CAA~\cite{cCAA} & EfficientNet-B5 & AAAI'22 & - & 47.3 \\
SegNeXt~\cite{cSegNeXt} & MSCAN-L & NIPS'22 & 46.5 & 47.2 \\
RankSeg~\cite{cRankSeg} & ViT-L & ECCV'22 & 46.7 & 47.9 \\
ViT-Adapter~\cite{cViTAdapter} & ViT-L & ICLR'23 & - & 52.0 \\
InternImage~\cite{cInternImage} &  InternImage-H & CVPR'23 & 52.6 & - \\
CART~\cite{cCART} & EfficientNet-L2 & TCSVT'24 & 50.2 & 50.9 \\
HFGD~\cite{cHFGD} & ConvNeXt-L & TCSVT'24 & 49.0 & 49.4 \\
\midrule
VPNeXt & ViT-L & - & \textbf{53.0} & \textbf{53.7} \\
\bottomrule%[1pt]
\end{tabular}
}
\label{tab:SOTA-COCOStuff164k}
\end{table}

\section{Experiments on COCOStuff164k Dataset}

COCOStuff164k has become increasingly popular in recent years and poses a significant challenge for semantic segmentation models due to its high diversity, consisting of 118,000 training images and 5,000 testing images, along with its complexity of 171 classes.

In Table~\ref{tab:SOTA-COCOStuff164k}, our VPNeXt model outperforms previous state-of-the-art methods, including ViT-Adapter and InternImage, by a significant margin.





\section{Experiments on Cityscapes Dataset}

Cityscapes is a semantic segmentation dataset featuring high-resolution images of road scenes with precise annotations. 
%
It includes 19 labeled classes and contains 2,975 training images and 500 validation images.
%
We only compare methods trained on the Cityscapes fine annotations, similar to many other works.~\cite{cSegFormer,cKMaXDeepLab}.

As shown in Table~\ref{tab:SOTA-Cityscapes}, our proposed VPNeXt, leveraging ViTUp's strong capabilities, performs comparably to state-of-the-art pyramid-based models (e.g., HFGD~\cite{cHFGD} and DPP~\cite{cDDP}) on high-resolution images.
%

\begin{table}[ht]
\centering
\small
\caption{
Comparisons to state-of-the-art methods on Cityscapes validation set.
%
\textit{SS}: Single scale performance w/o flipping.
\textit{MF}: Multi-scale performance w/ flipping.
``-'' in column \textit{SS} indicates that this result was not reported in the original paper.
}
\resizebox{\linewidth}{!}{
\begin{tabular}{l|c|c|c|c}
\toprule%[1pt]
Methods &Backbone & Avenue &\multicolumn{2}{c}{mIOU(\%)}\\
&  & & SS & MF \\
\midrule
\midrule
RepVGG\cite{cRepVGG} & RepVGG-B2 & CVPR'21 & - & 80.6 \\
SETR~\cite{cSegFormer} &ViT-L & CVPR'21 & - & 82.2 \\
Segmenter~\cite{cSegmenter} &ViT-L & ICCV'21 & - & 81.3 \\
OCR~\cite{cOCR,cHRFormer} &HRFormer-B & NIPS'21 & - & 82.6 \\
HRViT-b3~\cite{cHRViT}  & MiT-B3 & CVPR'22 & - & 83.2\\
FAN-L~\cite{cFANs} & FAN-Hybrid & ICML'22 & - & 82.3 \\
SegDeformer~\cite{cSegDeformer} & Swin-L & ECCV'22 & - & 83.5 \\
GSS-FT-W~\cite{cGSS} & Swin-L & CVPR'23 & - & 80.5 \\
TSG~\cite{cTSG} & Swin-L & CVPR'23 & - & 83.1 \\
STL~\cite{cSTL} & FAN-Hybrid & ICCV'23 & - & 82.8 \\
DDP(Step 3)~\cite{cDDP} & ConvNeXt-L & ICCV'23 & 83.2 & 83.9 \\
StructToken~\cite{cStructToken} & ViT-L & TCSVT'23 & 80.1 & 82.1 \\
GSCNN(EPL)~\cite{cEPL} & WRNet-38 & TMM'23 & - & 81.78\\
CART~\cite{cCART} & ConvNeXt-L & TCSVT'24 & 82.8 & 83.6\\
HFGD~\cite{cHFGD} & ConvNeXt-L & TCSVT'24 & 83.2 & 84.0 \\
% \midrule
\midrule%[0.1pt]
VPNeXt   & ViT-L              & - & 83.0 & \textbf{84.4} \\
\bottomrule%[1pt]
\end{tabular}
}
\label{tab:SOTA-Cityscapes}
\end{table}


\section{Experiments on VOC2012}
VOC2012 is one of the most classic datasets of semantic segmentation. 
%
It features a small number of categories (21 w/ background), medium resolution, and high annotation accuracy, which allowed earlier methods to achieve a mIoU of 89\% between 2018 and 2019. 

In subsequent years, although stronger methods were developed, they only resulted in slight improvements to the mIoU—usually by a few tenths (\ie < 0.5\%). 
%
Eventually, SegNeXt~\cite{cSegNeXt} raised the mIoU to 90.6\% in 2022, and since then, no other method has surpassed this mIOU wall. 
%
As a result, SegNeXt was considered the ceiling for this dataset.


\begin{table}[h]
    \normalsize
    \centering
    \begin{tabular}{cr}
        \midrule
          \quad there is no wall \quad\\
          \quad \\
         &\textit{\quad\quad-- Sam Altman~\cite{cNoWall}}\\
         \midrule
    \end{tabular}
\end{table}


Today, our proposed VPNeXt has broken this wall. 
%
%
As shown in Table.~\ref{tab:SOTA-VOC2012},
in terms of mIoU, our proposed VPNeXt not only outperforms SegNeXt but also exceeds SegNeXt by nearly 2\%., which also stands as the largest improvement since 2015.
%
Remarkably, VPNeXt excels in long-tailed categories (\eg chair, monitor) that have traditionally posed challenges for nearly all prior methods.


\begin{table*}[th!]
%\setlength{\tabcolsep}{0pt}
\centering
\small
\caption{
% \hy{
New breakthroughs in the VOC2012 leaderboard!
Due to limited space on the page, we have simplified some category names (e.g., "Aero Plane" to "Plane") and only listed the top 15 methods. 
Zoom in to see better.
To view the full leaderboard, please visit \url{http://host.robots.ox.ac.uk:8080/leaderboard/displaylb_main.php?challengeid=11&compid=6}.
}
\resizebox{\linewidth}{!}
{\def\arraystretch{1} \tabcolsep=0.55em 
\begin{tabular}{l|c|cccccccccccccccccccc}
\toprule%[1pt]
Methods & 
Mean & 
Plane &
Bicycle &
Bird &
Boat &
Bottle &
Bus &
Car &
Cat &
Chair &
Cow &
Table &
Dog &
Horse &
Motor &
Person &
Plant &
Sheep &
Sofa &
Train &
Monitor \\
\midrule
\midrule
\textbf{Our VPNeXt} & \textbf{92.2}& 98.9 & 78.5 & 98.6 & 92.1 & 92.3 & 95.2 & 96.8 & 96.1 & 70.7 & 98.8 & 79.9 & 96.0 & 98.4 & 96.9 & 95.8 & 89.8 & 98.2& 78.1 & 96.6 & 91.3\\
\midrule
\midrule
SegNeXt& 90.6 & 98.3 & 85.0 & 97.6 & 88.3 & 91.3 & 97.5 & 91.4 & 98.3 & 60.4 & 96.7 & 85.0 & 95.7 &98.2 & 94.2 &92.7 &82.5 & 97.3 & 77.7 & 93.1 & 84.3\\
\midrule
NAS-FPN(NS)& 90.5 & 98.0 & 84.8 & 89.6 & 88.2 & 91.0 & 98.3 & 93.0 & 98.5 & 57.5& 98.4 & 81.8 & 98.4 & 98.0 & 95.8 & 93.2 & 83.2 & 97.8 & 75.0 & 91.8 & 90.0\\
\midrule
DeepLabv3+(JFT) & 89.0 & 97.5 & 77.9 & 96.2 & 80.4 & 90.8 & 98.3 & 95.5 & 97.6 & 58.8 & 96.1 & 79.2 & 95.0 &97.3 & 94.1 & 93.8 & 78.5 & 95.5 & 74.4 & 93.8 & 81.6\\
\midrule
RecoNet152 & 89.0 & 97.3 & 80.4 & 96.5 & 83.8 & 89.5 & 97.6 & 95.4 & 97.7 & 50.1 &	96.8 & 82.6 &	95.1 & 97.7 &	95.1 & 92.6 & 80.2 & 95.2 & 71.7 & 92.1 & 83.8\\
\midrule
AASPP & 88.5 & 97.4 & 80.3 & 97.1 & 80.1 & 89.3 & 97.4 & 94.1 & 96.9 & 61.9 & 95.1 & 77.2 & 94.2 & 97.5 & 94.4 & 93.0 & 72.4 & 93.8 & 72.6 & 93.3 & 83.3\\
\midrule
SRC-B & 88.5 & 97.2 & 78.6 & 97.1 & 80.6 & 89.7 & 97.4 & 93.7 & 96.7 & 59.1 & 95.4 & 81.1 & 93.2 & 97.5 & 94.2 & 92.9 & 73.5 & 93.3 & 74.2 & 91.0 & 85.0\\
\midrule
SepaNet & 88.3 & 97.2 & 80.2 & 96.2 & 80.0 & 89.2 & 97.3 & 94.7 & 97.7 & 48.6 & 95.0 & 81.6 & 95.2 & 97.5 & 95.1 & 92.7 & 79.5 & 95.4 & 68.8 & 90.9 & 83.4\\
\midrule
EMANet152 & 88.2 & 96.8 & 79.4 & 96.0 & 83.6 & 88.1 & 97.1 & 95.0 & 96.6 & 49.4 & 95.4 & 77.8 & 94.8 & 96.8 & 95.1 & 92.0 & 79.3 & 95.9 & 68.5 & 91.7 & 85.6\\
\midrule
KSAC-H & 88.1 & 97.2 & 79.9 & 96.3 & 76.5 & 86.5 & 97.5 & 94.5 & 96.9 & 54.8 & 95.3 & 81.4 & 93.7 & 97.2 & 94.0 & 92.8 & 77.3 & 94.4 & 73.5 & 91.1 & 83.4\\
\midrule
SpDConv2 & 88.1 & 96.9 & 79.7 & 96.8 & 80.2 & 87.8 & 98.0 & 92.3 & 96.0 & 57.2 & 95.8 & 82.1 & 92.3 & 97.3 & 93.6 & 93.0 & 71.4 & 92.3 & 75.8 & 90.7 & 83.8\\
\midrule
FillIn & 88.0 & 97.1 & 80.8 & 96.7 & 77.6 & 89.2 & 97.4 & 92.2 & 96.9 & 58.3 & 94.3 & 79.4 & 93.1 & 97.3 & 94.4 & 93.2 & 73.6 & 93.0 & 72.6 & 89.7 & 83.4\\	
\midrule
MSCI & 88.0 & 96.8 & 76.8 & 97.0 & 80.6 & 89.3 & 97.4 & 93.8 & 97.1 & 56.7 & 94.3 & 78.3 & 93.5 & 97.1 & 94.0 & 92.8 & 72.3 & 92.6 & 73.6 & 90.8 & 85.4\\
\midrule
ExFuse & 87.9 & 96.8 & 80.3 & 97.0 & 82.5 & 87.8 & 96.3 & 92.6 & 96.4 & 53.3 & 94.3 & 78.4 & 94.1 & 94.9 & 91.6 & 92.3 & 81.7 & 94.8 & 70.3 & 90.1 & 83.8\\
\midrule
DeepLabV3+ & 87.8 & 97.0 & 77.1 & 97.1 & 79.3 & 89.3 & 97.4 & 93.2 & 96.6 & 56.9 & 95.0 & 79.2 & 
93.1 & 97.0 & 94.0 & 92.8 & 71.3 & 92.9 & 72.4 & 91.0 & 84.9\\
\bottomrule%[1pt]
\end{tabular}
}
\label{tab:SOTA-VOC2012}
\end{table*}
\section{Visualization}

\begin{figure}[ht!]
    \centering
    \includegraphics[width=1.0\linewidth]{images/ufcn_vs_internimage_coco.pdf}
    \caption{Visual comparison between Mask2Former + InternImage-H~\cite{cMask2Former,cInternImage} and our proposed VPNeXt on the COCOStuff164k~\cite{cCocoStuff} dataset shows that VPNeXt achieves superior segmentation results, particularly in challenging categories such as food.}
    \label{fig:cocostuff_internimage_vs_vpnext}
\end{figure}



\begin{figure}[ht!]
    \centering
    \includegraphics[width=\linewidth]{images/ufcn_vcr_feature_vis.pdf}
    \caption{Visualization analysis of the intermediate feature map for the VCR is based on the 15th layer of the ViT. At the pixel position marked by the red dot, the replay-optimized feature map displays significantly stronger and more detailed semantic information concerning intra-class pixels.}
    \label{fig:vcr_feature_vis}
\end{figure}



\subsection{Visualization on COCOStuff164k dataset}
We first conducted a visualization comparison on the challenging COCOStuff164k dataset~\cite{cCocoStuff}. 
As shown in Figure.~\ref{fig:cocostuff_internimage_vs_vpnext}, our proposed VPNeXt achieves significantly better segmentation results compared to the state-of-the-art Mask2Former + InternImage-H~\cite{cMask2Former,cInternImage}, particularly in some challenging categories, such as food.

\subsection{Visualization on intermediate feature map}
We conducted a visualization analysis of the intermediate feature map of the VCR. 
%
For this analysis, we utilized the feature map from the $15^{\text{th}}$ layer of the ViT. 
As illustrated in Figure.~\ref{fig:vcr_feature_vis}, at the position marked by the red dot (\textcolor{red}{$\bullet$}), the replay-optimized feature map presents significantly stronger and more intensive semantic information regarding intra-class pixels.



\section{Conclusion}
\label{sec:conclusion}
\vspace{-2.5mm}
We introduce a new research direction in collaborative driving and present the first solution. Empirical results show our approach can significantly reduce data collection and development efforts, advancing safer autonomous systems. 

\nbf{Limitations and future work}
Following existing benchmarks~\citep{xu2022opv2v, xu2023v2v4real}, \ours focuses on vehicle-like objects. Future work could extend it to broader objects and static entities (\eg, traffic signs, signals) essential for real-world traffic.

%\section{Limitation}


\newpage

\bibliographystyle{IEEEtran}
\bibliography{main}




\newpage

\section{Biography}
\begin{comment}
If you have an EPS/PDF photo (graphicx package needed), extra braces are
 needed around the contents of the optional argument to biography to prevent
 the LaTeX parser from getting confused when it sees the complicated
 $\backslash${\tt{includegraphics}} command within an optional argument. (You can create
 your own custom macro containing the $\backslash${\tt{includegraphics}} command to make things
 simpler here.)
\end{comment}
 
%\vspace{11pt}
\begin{comment}
\bf{If you include a photo:}\vspace{-33pt}
\begin{IEEEbiography}[{\includegraphics[width=1in,height=1.25in,clip,keepaspectratio]{fig1}}]{Michael Shell}
Use $\backslash${\tt{begin\{IEEEbiography\}}} and then for the 1st argument use $\backslash${\tt{includegraphics}} to declare and link the author photo.
Use the author name as the 3rd argument followed by the biography text.
\end{IEEEbiography}

\vspace{11pt}
\end{comment}

\begin{IEEEbiographynophoto}{Xikai Tang} is a PhD student at School of Information and Software Engineering, University of Electronic Science and Technology of China.
\end{IEEEbiographynophoto}

\begin{IEEEbiographynophoto}{Ye Huang} received the B.S. degree and the Ph.D. degree in Computer Science from the University of Technology Sydney, Australia. 
He is currently an associate researcher in the Data Intelligence Group, Shenzhen Institute for Advanced Study, University of Electronic Science and Technology of China. 
\end{IEEEbiographynophoto}

\begin{IEEEbiographynophoto}{Guangqiang Yin} is currently a Professor with the University of Electronic Science and Technology of China (UESTC). His research interests include computer-vision-related artificial intelligence techniques and applications, and computer modeling of properties of condensed matter.
\end{IEEEbiographynophoto}

\begin{IEEEbiographynophoto}{Lixin Duan} (Member, IEEE) received the B.E. degree from the University of Science and Technology of China, Hefei, China, in 2008, and the Ph.D. degree from the Nanyang Technological University, Singapore, in 2012. 
He is currently a professor with the University of Electronic Science and Technology of China. 
His current research interests include transfer learning, multiple instance learning, and their applications in computer vision and data mining.
\end{IEEEbiographynophoto}




\vfill

\end{document}


