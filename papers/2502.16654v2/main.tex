\documentclass[lettersize,journal]{IEEEtran}
\newcommand{\CG}{\mathcal{G}\xspace}
\newcommand{\CV}{\mathcal{V}\xspace}
\newcommand{\CE}{\mathcal{E}\xspace}
\newcommand{\CA}{\mathcal{A}\xspace}
\newcommand{\CF}{\mathcal{F}\xspace}
\newcommand{\CR}{\mathcal{R}\xspace}
\newcommand{\CB}{\mathcal{B}\xspace}
\newcommand{\CX}{\mathcal{X}\xspace}
\newcommand{\CK}{\mathcal{K}\xspace}
\newcommand{\CM}{\mathcal{M}\xspace}
\newcommand{\CC}{\mathcal{C}\xspace}
\newcommand{\CL}{\mathcal{L}\xspace}
\newcommand{\CI}{\mathcal{I}\xspace}
\newcommand{\CQ}{\mathcal{Q}\xspace}
\newcommand{\CO}{\mathcal{O}\xspace}
\newcommand{\CP}{\mathcal{P}\xspace}
\newcommand{\CS}{\mathcal{S}\xspace}
\newcommand{\CT}{\mathcal{T}\xspace}
\newcommand{\CJ}{\mathcal{J}\xspace}
\usepackage[para]{footmisc}
\usepackage{subfig}
% \usepackage{subcaption}
% \usepackage{array}
% \usepackage{colortbl}




\hyphenation{VPNeXt - Rethinking Dense Decoding for Plain Vision Transformer}
% updated with editorial comments 8/9/2021

\begin{document}

\title{VPNeXt : Rethinking Dense Decoding for \\ Plain Vision Transformer}

\author{Xikai Tang, Ye Huang\orcidlink{0000-0001-5668-5529},~\IEEEmembership{Member,~IEEE, } Guangqiang Yin and Lixin Duan\orcidlink{0000-0002-0723-4016}
        % <-this % stops a space
\thanks{Xikai Tang is with the School of Information and Software Engineering, University of Electronic Science and Technology of China}
\thanks{
Ye Huang, Guangqiang Yin and Lixin Duan are with the Shenzhen Institute for Advanced Study, University of Electronic Science and Technology of China, 518000 (Ye Huang is the corresponding author)}% <-this % stops a space% <-this % stops a space
}

% The paper headers
\markboth{Rethinking Dense Decoding for Plain Vision Transformer}%
{Shell \MakeLowercase{\textit{et al.}}: A Sample Article Using IEEEtran.cls for IEEE Journals}

%\IEEEpubid{\begin{tabular}[t]{@{}c@{}} Copyright © 20xx IEEE. Personal use of this material is permitted. \\However, permission to use this material for any other purposes must be obtained from the IEEE by sending an email to pubs-permissions@ieee.org.\end{tabular}}
% Remember, if you use this you must call \IEEEpubidadjcol in the second
% column for its text to clear the IEEEpubid mark.

\maketitle

\begin{abstract}

% Recent works to jointly reconstruct 3D human and object from a single RGB image, are mostly model-based, that fail to capture the fine details of the clothed human body and object surface. In this paper, we introduce ReCHOR, a novel, model-free, first-method to produce realistic clothed human-object reconstructions from a monocular view. This is extremely challenging due to human-object occlusions, diverse interactions and depth ambiguity, as it needs to infer both 3D spatial awareness and high resolution details. Our core idea is based on estimating neural implicit representations for human and object respectively by an attention-based neural implicit model that attends to pixel-aligned features from both the global human-object image for spatial awareness and  the local separate view of human and object images for high quality details. Additionally, the network is conditioned on semantic features from an initial estimated human-object pose prior and a generative diffusion model that inpaints occluded regions, thus enabling the retrieval of details from them.
% We also propose a synthetic dataset with rendered scenes of diverse, inter-occluded 3D human and object scans, to train our network. We evaluate our method on the synthetic and real world BEHAVE dataset. Our experiments show that our method outperforms the SOTA in achieving realistic clothed human-object reconstructions.
Recent approaches to jointly reconstruct 3D humans and objects from a single RGB image represent 3D shapes with template-based or coarse models, which fail to capture details of loose clothing on human bodies. In this paper, we introduce a novel implicit approach for jointly reconstructing realistic 3D clothed humans and objects from a monocular view. For the first time, we model both the human and the object with an implicit representation, allowing to capture more realistic details such as clothing. This task is extremely challenging due to human-object occlusions and the lack of 3D information in 2D images, often leading to poor detail reconstruction and depth ambiguity. To address these problems, we propose a novel attention-based neural implicit model that leverages image pixel alignment from both the input human-object image for a global understanding of the human-object scene and from local separate views of the human and object images to improve realism with, for example, clothing details. Additionally, the network is conditioned on semantic features derived from an estimated human-object pose prior, which provides 3D spatial information about the shared space of humans and objects. To handle human occlusion caused by objects, we use a generative diffusion model that inpaints the occluded regions, recovering otherwise lost details. For training and evaluation, we introduce a synthetic dataset featuring rendered scenes of inter-occluded 3D human scans and diverse objects. Extensive evaluation on both synthetic and real-world datasets demonstrates the superior quality of the proposed human-object reconstructions over competitive methods.
\end{abstract}

\begin{IEEEkeywords}
Vision Transformer, Semantic Segmentation, Representation Learning
\end{IEEEkeywords}

\section{Introduction}
\label{sec:intro}
% Image editing methods in diffusion models depend on user-defined control directions - users can unlock their creativity using these methods by specifying the desired manipulation through prompts~\cite{gandikota2023concept}, reference images~\cite{ruiz2022dreambooth, kumari2022customdiffusion, gal2022image, chen2024trainingfreeregionalpromptingdiffusion}, or attribute vectors~\cite{parmar2023zero,hertz2022prompt}. In this work, we ask a fundamentally different question: \emph{Can we automatically discover the underlying visual structure of a concept within diffusion model's knowledge?} %Rather than requiring user-specified controls, we aim to decompose the model's internal knowledge into meaningful directions.

% This question touches on a fundamental limitation in how we interact with diffusion models. Current control methods ~\cite{zhang2023addingconditionalcontroltexttoimage, gandikota2023concept, ye2023ipadaptertextcompatibleimage,ye2023ipadaptertextcompatibleimage, hertz2024stylealignedimagegeneration, li2023photomaker, shi2024instantbooth, chen2024trainingfreeregionalpromptingdiffusion} require users to specify their desired manipulations in advance, limiting interactive creativity. This contrasts with natural human artistic workflows, where creators dynamically explore creative ideas while jointly refining them toward meaningful artistic outcomes~\cite{hoffmann2016modeling}. This synergy between specification and exploration is not new to generative models. Early GAN architectures naturally developed disentangled latent spaces that enabled continuous\cite{harkonen2020ganspace,radford2015unsupervised, wu2021stylespace, shen2020interfacegan}, compositional control over generated images. Users could explore these spaces to discover interesting variations that would be difficult to describe in words~\cite{wu2021stylespace}, then combine them to achieve their creative goals~\cite{grabe2022towards}. 


% While diffusion models have largely superseded GANs in conditional image synthesis~\cite{dhariwal2021diffusion},  their underlying structure remains less understood. Diffusion models achieve remarkable diversity through high-dimensional latents, unlike GANs' compact latent spaces.  With a single prompt, diffusion models can generate radically different variations through different random initializations of input noise. We ask - Is it possible to discover interpretable structure within this vast space of variations?

Text-to-image diffusion models are capable of generating remarkable visual variations from a single prompt through different random initializations. However, this vast creative potential remains largely opaque to users---while we can generate diverse images, we lack understanding of the underlying structure of these variations. This presents a fundamental challenge: how can we discover and expose the latent visual capabilities encoded within these models?

\let\thefootnote\relax \footnote{$^{*}$Correspondence to \texttt{gandikota.ro@northeastern.edu}}

The challenge touches on a key limitation in how we interact with diffusion models today. Current control methods require users to explicitly specify their desired edits in advance through prompts~\cite{gandikota2023concept}, reference images~\cite{zhang2023addingconditionalcontroltexttoimage, chen2024trainingfreeregionalpromptingdiffusion, ruiz2022dreambooth,kumari2022customdiffusion, Ryu_lora, hu2021lora}, or attribute vectors~\cite{ye2023ipadaptertextcompatibleimage, hertz2024stylealignedimagegeneration, li2023photomaker, shi2024instantbooth,parmar2023zero,hertz2022prompt}. That contrasts sharply with natural human creative workflows, where artists dynamically explore creative ideas and jointly refine them toward meaningful artistic outcomes~\cite{hoffmann2016modeling}. The need for pre-specified controls creates a barrier between users and the full creative potential of these models.

Interestingly, earlier generative models like GANs~\cite{gans,karras2019style,brock2018large} naturally developed more interpretable internal structures. Their compact latent spaces often exhibited emergent disentanglement~\cite{harkonen2020ganspace,radford2015unsupervised, wu2021stylespace, shen2020interfacegan}, enabling continuous and compositional control over generated images. Users could explore these spaces to discover interesting variations that would be difficult to describe in words~\cite{wu2021stylespace}, then combine them to achieve their creative goals~\cite{grabe2022towards}.

Diffusion models have largely superseded GANs in conditional image synthesis~\cite{dhariwal2021diffusion}, achieving greater diversity through much higher-dimensional latents. And yet an understanding of the underlying structure of these larger latent spaces has remained elusive. In this work, we ask a fundamental question: \emph{Can we automatically discover the visual structure within a diffusion model's knowledge of a concept?} Rather than requiring user-specified controls, we aim to decompose the model's internal representations into expressive directions that users can explore and combine.

To address these needs, we present \textbf{SliderSpace}, a framework that brings systematic explorability to diffusion models. Given just a text prompt, SliderSpace discovers a canonical set of meaningful, diverse, and controllable directions within the model's knowledge of that concept. Each direction is implemented as a low-rank adapter~\cite{hu2021lora} that can be scaled and composed with others, allowing users to explore and smoothly combine different aspects of variation, as shown in Figure~\ref{fig:intro}.

We ground SliderSpace discovery in three key requirements for meaningful decomposition of a diffusion model's visual manifold: 
\begin{enumerate}
    \item \textbf{Unsupervised Discovery:} The decomposition process should emerge from the intrinsic structure of the model's learned representation, rather than being guided by predefined attributes. This ensures we capture the true topology of the model's knowledge space rather than projecting our assumptions onto it.
    
    \item \textbf{Semantic Orthogonality:} Each discovered control must represent a distinct semantic direction. This is enforced in a semantic feature space, like CLIP, where every slider has an orthogonal effect in embeddings. This prevents discovering multiple controls that create similar semantic effects, making the system more efficient and easier.
    
    \item \textbf{Distribution Consistency:} Directions must induce consistent transformations across both random seeds and prompt variations. 
\end{enumerate}

These requirements naturally lead to our proposed framework, which we formalize in Section~\ref{sec:method}. As we show in our experiments, SliderSpace is architecture-agnostic, working with both conventional U-Net based models like Stable Diffusion~\cite{rombach2022high, rombach2022sd20, podell2023sdxl, turbo, dmd} and recent transformer-based architectures like Flux~\cite{flux}.

We demonstrate the expressiveness of SliderSpace through three applications: First, we show how SliderSpace can decompose high-level concepts into diverse and expressive components, revealing the natural axes of variation in the model's understanding. Second, we explore artistic style variation, where SliderSpace discovers directions that match or exceed the diversity of manually curated artist lists while being judged more useful by human evaluators. Finally, we show how SliderSpace can help reverse the mode collapse commonly observed in distilled diffusion models, restoring diversity while maintaining generation speed.

Beyond providing practical creative control, SliderSpace opens new avenues for understanding and utilizing the latent capabilities of diffusion models. By mapping these models' visual potential into intuitive, composable directions, we take a step toward making their creative possibilities more accessible and interpretable to users.

% Image editing methods in diffusion models unlock the creativity of users. In this work we ask an alternate question: \emph{Can we organize and expose what of the diffusion model is already capable of?}.
% Existing methods for controlling image generation typically require users to manually specify edit directions for desired changes. This process is time-consuming, requires technical expertise, and limits the spontaneity of the creative process. For instance, if a user wants to adjust the smile of a generated person, they must explicitly request this edit, often through imprecise prompt engineering or model fine-tuning. This approach of predefined controls or manual specifications restricts users from fully exploring the latent capabilities of the model. There may be interesting stylistic variations or attributes that the model can generate, but users have no easy way to discover or utilize these.

% Natural visual disentanglement was an emergent property in the latent space of Generative Adversarial Models (GANs) \cite{harkonen2020ganspace,radford2015unsupervised, wu2021stylespace, shen2020interfacegan}. In particular, it has been observed that StyleGAN~\cite{karras2019style} stylespace neurons offer detailed control over many meaningful aspects of images that would be difficult to describe in words~\cite{wu2021stylespace}. However, diffusion models do not share such a compact latent space~\cite{park2023unsupervised}; and efforts to uncover such a space in the semantic embeddings of the text conditioning have met with limited success \nik{Nick - is there a specific citation you were thinking about?}.

% In this work we introduce \textbf{SliderSpace}, which takes a step towards uncovering an analogous low dimensional representation of diffusion models' visual breadth; in essence treating the diffusion model as many generators sharing parameters, where a particular generator is defined by a specific prompt. For a given prompt we sample many random seeds (and optionally prompt expansions using an LLM), generate the corresponding images, and apply an off the shelf feature extractor (in this work CLIP, but our method can be applied to any differentiable feature extractor). We use PCA to analyze these features, and for each of the leading $k$ principal components we train a LoRA \cite{} which causes the diffusion model to produces images which increase the feature magnitude along that component when passed back through the same feature extractor. This leads to a 'Slider' for each principal component, because each LoRA can be scaled and applied to the original diffusion model, continuously varying those visual features in the generated results (as measured, in our case, by CLIP).

% There are many other works that enhance the controllability of diffusion models. One common approach is enabling users to add spatial constraints to a generation either manually, or via a reference image \cite{zhang2023addingconditionalcontroltexttoimage, chen2024trainingfreeregionalpromptingdiffusion}, a second is leveraging more abstract embeddings (e.g. identity, style) extracted from a reference image \cite{ye2023ipadaptertextcompatibleimage, hertz2024stylealignedimagegeneration, li2023photomaker, shi2024instantbooth}, a third is finetuning a foundation model to better generate a concept important to the user \cite{ruiz2022dreambooth, kumari2022customdiffusion, Ryu_lora, hu2021lora}, and a fourth (most relevant to this work) is finding low-rank adaptors of the model based on a prompt or small training set which can be scaled to provide continous control over one aspect of generated image (e.g. night vs day, basic vs luxury, etc.) \cite{gandikota2023concept}. SliderSpace is complementary to all of these methods and offers something distinct. All of the other methods we are aware require the user (and / or model designer) to know in advance what type of control they want. In contrast SliderSpace assists users in discovering and controlling hidden capabilities present in the diffusion model's distribution of possible generations.

%We propose that truly intuitive creative control in a text-to-image model should meet three key criteria: \emph{discoverability}, \emph{intuitiveness}, and \emph{specificity}. The model should reveal controllable attributes that may not be immediately obvious, offer controls that are easy to understand and manipulate, and ensure each control affects a distinct attribute of the generated image.

% We demonstrate the utility and power of SliderSpace using three applications built on top of SDXL-DMD \cite{dmd}, because its fast generation speed lends itself well to the continuous control offered by SliderSpace.

% First, we study concept decomposition (Section \ref{sec:concept_exp}), where we learn sliders for a specific concept (e.g. 'monster', 'waterfall', 'car'). Through quantitative metrics of diversity and text alignment we demonstrate that the learned sliders dramatically boost the diversity of generations when randomly applied without harming text alignment; we also ask humans to qualitatively judge these results in a user study where they find the SliderSpace results to be more 'Diverse', 'Useful', and 'Creative' than our baselines.

% Second, we attempt to compare the automatic discoveries of SliderSpace to a large scale manual study of artistic styles (Section \ref{sec:art_exp}), open-sourced by ParrotZone \cite{parrotzone}. In this study SDXL was prompted with over 4300 artist names,  and based on visual inspection the cases of successful stylistic mimicry recorded. Quantitatively SliderSpace more closely matches the distribution of artistic variation discovered by ParrotZone than other baselines, and in our user studies was judged to be significantly more 'Diverse' and 'Useful' than the baselines. To our surprise humans even judged SliderSpace results to be slightly more 'Diverse' than the results generated by the manually discovered artist names of \cite{parrotzone}.

% Third, we attempt to use SliderSpace to reverse the mode collapse commonly observed in distilled few-step diffusion models relative to the original teacher model (Section \ref{sec:diverse_exp}). We quantitatively demonstrate that applying SliderSpace to SDXL-DMD leads to more closely matching the distribution of images by the original teacher, SDXL.

%Through extensive experiments on various state-of-the-art text-to-image models, we demonstrate that SliderSpace significantly enhances user control and creative expression in AI-assisted image generation tasks. Our method enables a range of applications, including concept decomposition and control, diversity improvement in generated images, customization dissection and edits, and the exploration of artistic styles inherent in the model.

% SliderSpace goes beyond providing a practical tool for enhanced creative control. By mapping the visual potential of diffusion models it can open new avenues for generative creativity and deepens our understanding of each model's hidden potential.
\section{Related Work}

\subsection{First-order logic for natural entailment}

Since the start of the RTE challenge \citep{rte}, multiple works have attempted using FOL representations to solve natural language entailment. These methods first obtain the syntactic/semantic parse tree and apply a rule-based transformation to get the FOL representation \citep{bos-markert-2005-recognising, bos-nli}. However, it was repeatedly shown that these FOL representations are not empirically effective in solving natural language entailment. For instance, \citet{bos-nli} reported that FOL representations translated from the discourse representation structure (DRS) yield only 1.9\% recall in detecting the entailment in the single-premise RTE benchmark \citep{rte}.

Independently from these works, multi-premise logical entailment benchmarks \citep{tafjord-etal-2021-proofwriter, logicnli, folio} were developed to evaluate the reasoning ability of generative models. These benchmarks adopt the classic 3-way entailment label classification format (\textit{entailment, contradiction, neutral}) of single-premise RTE tasks, in which both the NL sentences and their gold FOL representations point to the same entailment label. 

Recent works have applied LLMs to obtain FOL representations for these multi-premise logical entailment tasks \citep{logiclm, linc, divide-and-translate}, fueled by the code generation ability of LLMs. While they achieve significant performance in synthetic, controlled logical reasoning benchmarks, whether they can generalize to natural entailment has remained unanswered. Furthermore, \citet{linc} observed that LLMs are highly susceptible to \textit{arbitrariness}, as they fail to produce coherent predicate names or numbers of arguments even when generating FOL representations of premises and hypotheses in a single inference.

\subsection{Executable semantic representations}

Apart from FOL, a stream of research focuses on the \textit{executability} of semantic representations. From this perspective, semantic representations are \textit{program codes} that can be executed to solve downstream tasks, such as query intent analysis \citep{spider, dligach-etal-2022-exploring} and question answering \citep{semparse-qa}. The performance of the semantic parser is directly assessed by the accuracy of execution results for the downstream tasks, rather than the similarity between the prediction and the reference parse.

To improve the execution accuracy that is often non-differentiable, reinforcement learning (RL) and its variants have been applied to train neural semantic parsers \citep{cheng-etal-2019-learning, cheng-lapata-2018-weakly}. Using only the input sentence and the desired execution result, these methods learn to maximize the probability of the representations that lead to the correct execution result. However, these approaches are not directly applicable to EPF, as EPF requires taking account of \textit{interactions between premises and hypotheses} during execution (\textit{i.e.} theorem proving) while these methods assume that sentences are isolated.


% introduce PDDL domains
% why Gripper env as testing context
% motivation: comparing classical vs LLM planners
% - classical: PDDL solver fast-downward
% - LLM: gpt-4o
% explanation and refinement are two distinguishing features of LLM planners
% - how we demonstrate explanation and refinement in the study
We evaluate user trust in two planners over a set of planning problems and study the potential factors influencing user trust in the planners. In particular, we compare a language-model-based planner, denoted as an \emph{LLM Planner}, with a traditional graph-search-based planner, denoted as a \emph{PDDL Solver}. The PDDL Solver uses Fast Downwards \cite{fastdownward} as its underlying model, processing planning problems described in PDDL to generate an optimal solution. In comparison, the LLM Planner employs GPT-4o to interpret the planning problem and extract a solution generated by the language model. Unlike the PDDL Solver, the LLM Planner can reason through the planning problem, explain its proposed solution, and iteratively refine the solution based on external feedback. This study investigates how the correctness of solutions, the quality of explanations, and the refinement process influence user trust.

\subsection{Planning Problem}
% \begin{wrapfigure}{r}{0.4\textwidth}
% % \begin{figure}[t]
%     \centering
%     \includegraphics[width=\linewidth]{figures/problem-example.pdf}
%     \caption{A running example of a planning problem in our study.}
%     \Description{Planning Problem Example}
%     \label{fig: problem-example}
% % \end{figure}
% \end{wrapfigure}

We describe each planning problem in the \emph{Planning Domain Definition Language (PDDL)} and propose two planners to generate plans that solve the problem. We select the \emph{gripper} planning problems from the International Planning Competition \cite{IPC} for plan generation and evaluation. In a gripper planning problem, a robot moves balls between a set of rooms using two grippers. The objective is to create a plan for the robot to move the balls to the target rooms we defined. We present a few running examples of the gripper problem in Figure \ref{fig: correctness}.

A planning problem consists of a \emph{planning domain} and a \emph{problem description}, expressed in PDDL. 

\paragraph{Planning Domain}
A planning domain refers to the universal aspects of a problem that remains consistent across different instances of the problem. In particular, it defines the types of objects, predicates, and actions that exist in the planning problem. We present an example of the gripper problem in Appendix \ref{app: grippers}.

\paragraph{Problem Description} A problem description specifies the particular instance of a planning task within a given domain. It includes the planning domain to which it pertains, a set of objects, the initial state of these objects, and the goal state to be achieved.

\paragraph{Plan}
A plan is a sequence of actions with specific input parameters. Recall that an action corresponds to a state transition. If a plan (a sequence of actions) transits from the initial state to the goal state defined by a problem, then we consider the plan to be \emph{correct}. If a plan does not transit to the goal state or there exists an action violating its precondition, then the plan is \emph{wrong}.

\begin{figure}[t]
    \centering
    \includegraphics[width=0.8\linewidth]{figures/correct.jpeg}
    \caption{Examples where LLM Planner correctly generates a plan for the gripper planning problem.}
    \Description{Planning Problem Correctness}
    \label{fig: correct}
\end{figure}

\subsection{PDDL Solver}
The PDDL Solver takes the planning domain and the problem description as inputs and then generates a plan described in PDDL. 
% It generates a plan in the following format:
% \vspace{4pt}
% \begin{lstlisting}[language=completion]
% (move robot1 room1 room3)
% (pick robot1 ball2 room3 rgripper1)
% (move robot1 room3 room2) ......
% \end{lstlisting}
Next, we convert the generated plan into natural language for user studies following the procedure in \cite{seipp-et-al-zenodo2022} and display it to users. We present an example in Figure \ref{fig: correct}.

The PDDL Solver applies a graph search algorithm to find a path (i.e., a list of transitions) from the initial state to the goal state. It either generates a \emph{correct} plan---defined as the shortest path between the initial and goal states---or returns a signal indicating that no solution exists for the given problem.

\subsection{LLM Planner}

The LLM Planner addresses planning problems by querying a large language model. In particular, it transmits the planning domain and problem description to the language model using a structured prompt format. The planner then retrieves a natural language plan from the language model. We use GPT-4o as the language model for the planner. To ensure the output adheres to the desired format, we include a few in-context examples within the prompts.

A language model solves a planning problem by interpreting the domain and problem descriptions, simulating state transitions, and generating a sequence of actions to achieve the goal. While effective for reasoning and plan generation, language models may struggle with large state spaces. Unlike the PDDL Solver, the LLM Planner may generate \emph{incorrect} plans that violate the problem specifications (e.g., preconditions of actions) or fail to achieve the goal.

\subsection{Explanation and Refinement}
Alongside the generated plans, we offer detailed explanations of all the plans and revisions of any incorrect plans. This study examines how these explanations and refinements influence human trust in the two planners.

\paragraph{LLM Planner with Explanation (LLM+Expl)}
For each generated plan, we manually provide a natural language explanation. This explanation includes an assessment of the plan’s correctness, identification of any violations of action preconditions, and an analysis of inconsistencies between the final state achieved and the intended goal state. We present examples of explanations in Figure \ref{fig: explain} in Appendix.

In particular, if a plan is correct, the explanation is simply ``the plan successfully satisfies the goal conditions.'' 
If a plan is incorrect, we identify the underlying cause as either a violation of action preconditions or a failure to achieve the goal state. In cases involving precondition violations, we specify the action responsible for the issue. For example, consider the action ``robot moves from room 1 to room 2,'' but the robot is initially located in room 3. This scenario constitutes a violation of the precondition for the ``move'' action. In the latter case, we describe the differences between the final state achieved and the intended goal state, e.g., ``fail to move ball 2 to room 2.''

% \begin{wrapfigure}{r}{0.5\textwidth}
%     \centering
%     \includegraphics[width=0.98\linewidth]{figures/refine.jpeg}
%     \includegraphics[width=0.98\linewidth]{figures/refine-correct.jpeg}
%     \includegraphics[width=0.98\linewidth]{figures/refine-wrong.jpeg}
%     \caption{Plan refinement by the LLM Planner. The top row presents two choices of plan refinement (where the refinement starts). The second and third row shows the refinement outcomes of the two choices, where the second row shows a correctly refined plan and the third row shows an incorrect plan.}
%     \Description{Refinement}
%     \label{fig: refine}
% \end{wrapfigure}

\paragraph{LLM Planner with Refinement (LLM+Refine)}
Note that a plan generated by the LLM Planner could be incorrect. Therefore, we offer a prompting mechanism for the LLM Planner to refine the generated plan according to the user feedback. The mechanism works as follows:

1. Request the user to indicate the step number of the first action in the plan that is incorrect, such as the step where an action’s precondition is violated. We present a sample user interface on the left of Figure \ref{fig: refine} in Appendix.

2. Send the planning domain, problem description, and the original plan to the language model. Then, query the model to rewrite the subsequent steps starting from the user-specified step number. We present a sample input prompt in Figure \ref{fig: refine-prompt} in the Appendix.

3. Replace the original plan with the newly refined plan and display it to the user.

This mechanism allows users to interact with the language model to refine the plan. It enables the language model to focus on a subset of steps, facilitating a deeper interpretation of the incorrect component. However, the correctness of the refined plan is not guaranteed. Figure \ref{fig: refine} in the Appendix shows an example of a correctly refined plan and an incorrectly refined plan.

\section{Training details}
\label{sec:HFGD:training_settings}

Unless specified otherwise, the training settings for our proposed VPNeXt are similar to existing works that use ViT mask decoders~\cite{cSETR,cSegViT,cMask2Former}.
This includes the AdamW optimizer, a batch size of 16, and the use of clipnorm along with a mask loss that combines focal and dice losses.

Given that this work focuses exclusively on the plain ViT backbone, all the experiments we conducted are based on the plain ViT without pyramid modifications. 
%
Following common practices, the weights of the ViT are initialized through modern pre-training~\cite{cAugReg,cEVA}.

To accommodate new readers in the field, we utilize the commonly used Mean Intersection over Union (mIOU) metric to evaluate the prediction accuracy of our model.
\section{Experiments on Pascal Context Dataset}
%\raggedbottom
The Pascal Context~\cite{cPascalContext} dataset comprises 4,998 training images and 5,105 testing images. We utilize its 59 semantic classes to perform ablation studies and experiments, following common practice. Unless otherwise specified, we train the models on the training set for 20K iterations.

In the ablation studies, we follow the VPNeXt's forward propagation sequence. 
%
First, we assess the effectiveness of VCR alone, and then we incorporate ViTUp to evaluate its ability to upsample the feature maps produced by VCR.
%
Finally, we conducted an analysis of computational overhead to evaluate the efficiency of our proposed VPNeXt.

%%%%%%%% VCR ablation studies %%%%%%%%%%%%%%%%%%%%%%%%%%%%%%

\subsection{Ablation studies on VCR}
We compare our proposed VCR with a mask decoder (w/o pyramid, \eg segmenter~\cite{cSegmenter}) and deep supervision, as discussed in previous sections. 
%
As shown in Table~\ref{tab:exps:vcr-ablation-studes}, incorporating visual context in deep supervision results in an even better mIOU than the mask decoder (68.83\% vs 67.88\%). 

Additionally, we conducted ablation studies to determine the optimal number of deep supervision layers to use. 
%
The results in Table~\ref{tab:exps:vcr-ablation-studes} indicate that the mIOU reaches its highest value when two intermediate layers are employed for VCR-oriented deep supervision.


\begin{table}[ht]
    \centering
    \caption{Ablation studies on VCR, all the results are obtained under single-scale without flipping.
    All baseline models are trained using the same backbone and settings.
    \textit{DS:} Deep supervision.
    }
    \resizebox{\linewidth}{!}{
    \begin{tabular}{c|c|c}
       \toprule
       Methods & Num\# DS layers &  \quad mIOU(\%)\quad \\
       \midrule
       Deep supervision & 2 & 66.50 \\
       \midrule
       Mask decoder  & 2 (implicit) & 67.88  \\
       w/o pyramid & & \\
       \midrule
       Our VCR & 1 & 68.43\\
        & \textbf{2} & \textbf{68.83}\\
        & 3 & 68.56 \\
       \bottomrule
    \end{tabular}
    }
    \label{tab:exps:vcr-ablation-studes}
\end{table}

\begin{table}[ht]
    \centering
    \caption{Ablation studies on ViTUp, all the results are obtained under the single-scale without flipping.
    All baseline models are trained using the same backbone and settings.
    }
    \resizebox{\linewidth}{!}{
    \begin{tabular}{c|c|c}
       \toprule
       Methods & Num\# HiCLR layers&  mIOU(\%) \\
       \midrule
       Bilinear  & 0 & 68.83 \\
       \midrule
       Mock pyramid & 2 & 69.01 \\
       \midrule
       Our real pyramid  & 1 & 69.50 \\
        & 2 & 69.87 \\
        & \textbf{3} & \textbf{70.00}\\
        & 4 & 69.81 \\
        & 5 & 69.43 \\
       \bottomrule
    \end{tabular}
    }
    \label{tab:exps:ViTUp-ablation-studes}
\end{table}

\begin{table}[ht]
    \centering
    \caption{
    Computational cost analysis for VPNeXt.
    All baseline models use the same backbone and settings.
    }
    \resizebox{\linewidth}{!}{
    \begin{tabular}{c|c|c}
       \toprule
       \quad Methods \quad \quad &  
       Pyramid Upsampler \quad &
       \quad GFlops \quad \quad  \\
       \midrule
       Deep supervision & - & 356.69\\
       \midrule
       Mask decoder & & 359.99  \\
       & \checkmark & > 2000 \\
       \midrule
       Our VPNeXt & & 356.69 \\
        & \checkmark & 1007.62 \\
       \bottomrule
    \end{tabular}
    }
    \label{tab:exps:cost-ablation-studes}
\end{table}



\begin{table}[ht]
\centering
\small
\caption{
% \hy{
Comparisons to state-of-the-art methods on Pascal Context dataset.
%
\textit{SS}: Single-scale performance w/o flipping.
\textit{MF}: Multi-scale performance w/ flipping.
``-'' in column \textit{SS} indicates that this result was not reported in the original paper.
% }
}
\resizebox{\linewidth}{!}
{\def\arraystretch{1} \tabcolsep=0.55em 
\begin{tabular}{l|c|c|c|c}
\toprule%[1pt]
Methods & Backbone & Avenue &\multicolumn{2}{c}{mIOU(\%)} \\
& & & SS & MF \\
\midrule
\midrule
SETR~\cite{cSETR}           & ViT-L           & CVPR'21 & - & 55.8 \\
DPT~\cite{cDPT}             & ViT-Hybrid      & ICCV'21 & - & 60.5 \\
OCNet~\cite{cOCNet}         & HRNet-W48       & IJCV'21 & - & 56.2 \\
CAA~\cite{cCAA}             & EfficientNet-B7 & AAAI'22 & - & 60.5 \\
CAA + CAR~\cite{cCAR}        & ConvNeXt-L      & ECCV'22 & 62.7 & 63.9 \\
SegNeXt~\cite{cSegNeXt}     & MSCAN-L         & NIPS'22 & 59.2 & 60.9 \\
SegViT~\cite{cSegViT}       & ViT-L            & NIPS'22 & - & 65.3 \\
SenFormer~\cite{cSenFormer}  & Swin-L          & BMVC'22 & 63.1 & 64.5\\
TSG~\cite{cTSG}             & Swin-L           & CVPR'23 & - & 63.3 \\
IDRNet~\cite{cIDRNet}        & Swin-L           & NIPS'23 & - & 64.5 \\
APPNet~\cite{cAPPNet}       &SenFormer-L       & TCSVT'23 & - & 63.7 \\
ViT-Adapter-L~\cite{cViTAdapter} & ViT-L       & ICLR'23 & 67.8 & 68.2 \\
InternImage~\cite{cInternImage} & InternImage-H & CVPR'23 & - & 70.3 \\
PFT~\cite{cPFT}             & ResNet-101       & TMM'24 & 55.2 & 57.3 \\
CART~\cite{cCART}           &EfficientNet-L2   & TCSVT'24 & 66.0 & 67.5 \\
HFGD~\cite{cHFGD}           &ConvNeXt-L        & TCSVT'24 & 64.9 & 65.6 \\
SILC~\cite{cSILC}           &SILC-C-L          & ECCV'24 & - & 61.5 \\
\midrule%[0.1pt]
VPNeXt (w/o ViTUp)         & ViT-L & - & \textbf{68.8} & \textbf{69.7} \\
VPNeXt          & ViT-L & - & \textbf{70.0} & \textbf{71.1}  \\
\bottomrule%[1pt]
\end{tabular}
}
\label{tab:SOTA-PascalContext}
\end{table}


\subsection{Ablation studies on ViTUp}
We then assess the mIOU of our proposed ViTUp. 
%
As shown in Table~\ref{tab:exps:ViTUp-ablation-studes}, the real pyramid feature provided by our ViTUp, enhanced by HiCLR, reached 69.50\% mIOU, significantly outperforms both bilinear interpolation and mock pyramids (69.50\% vs 68.83\% vs 69.01\%). 
%
Furthermore, applying refinement three times yields 70.00\% mIoU, making it the best ViTUp configuration for VPNeXt.



\subsection{Computational cost analysis}
To demonstrate the high efficiency of VPNeXt, we conducted a computational analysis on two setups: VCR (VPNeXt w/o pyramid upsampler) and the complete VPNeXt with ViTUp.
%
For fair comparisons, we utilized Segmenter~\cite{cSegmenter} as the Mask decoder w/o a pyramid upsampler, and Mask2Former-based~\cite{cMask2Former} Vit-adapter~\cite{cViTAdapter} and PlainSeg~\cite{cPlainSeg} as Mask decoders with/a pyramid upsampler.

Table~\ref{tab:exps:cost-ablation-studes} shows that VCR and deep supervision have the same Flops, indicating that VCR provides high-quality representations without adding any computational overhead (see previous subsections for details).
%
Table~\ref{tab:exps:cost-ablation-studes} also shows that ViTUp delivers high-resolution pyramid features and strong mIoU while having significantly lower computational overhead compared to previous mask decoders that rely on mock pyramid features.




\subsection{Compare with state-of-the-arts}
To fully showcase the performance superiority of VPNeXt, we compared it with state-of-the-art methods on the Pascal Context dataset.
%
Note that, only methods published by the time this paper was completed can be compared.
%
As shown in Table~\ref{tab:SOTA-PascalContext}, our proposed VPNeXt significantly outperforms the compared methods, including the previous state-of-the-art techniques ViT-Adapter and InternImage. 
%
Moreover, even without using ViTUp (\ie with only VCR), VPNeXt still outperforms most methods.


\begin{table}[th!]
\centering
\small
\caption{
% \hy{
Comparisons to state-of-the-art methods on COCOStuff164k dataset.
%
\textit{SS}: Single-scale performance w/o flipping.
\textit{MF}: Multi-scale performance w/ flipping.
``-'' in column \textit{SS} or \textbf{MF} indicates that this result was not reported in the original paper.
% }
}
\resizebox{\linewidth}{!}
{\def\arraystretch{1} \tabcolsep=0.55em 
\begin{tabular}{l|c|c|c|c}
\toprule%[1pt]
Methods & Backbone  & Avenue &\multicolumn{2}{c}{mIOU(\%)}\\
& & & SS & MF \\
\midrule
\midrule
OCR~\cite{cOCR,cHRFormer} & HRFormer-B & NIPS'21 & - & 43.3 \\
SegFormer~\cite{cSegFormer} & MiT-B5 & NIPS'21 & - & 46.7 \\
CAA~\cite{cCAA} & EfficientNet-B5 & AAAI'22 & - & 47.3 \\
SegNeXt~\cite{cSegNeXt} & MSCAN-L & NIPS'22 & 46.5 & 47.2 \\
RankSeg~\cite{cRankSeg} & ViT-L & ECCV'22 & 46.7 & 47.9 \\
ViT-Adapter~\cite{cViTAdapter} & ViT-L & ICLR'23 & - & 52.0 \\
InternImage~\cite{cInternImage} &  InternImage-H & CVPR'23 & 52.6 & - \\
CART~\cite{cCART} & EfficientNet-L2 & TCSVT'24 & 50.2 & 50.9 \\
HFGD~\cite{cHFGD} & ConvNeXt-L & TCSVT'24 & 49.0 & 49.4 \\
\midrule
VPNeXt & ViT-L & - & \textbf{53.0} & \textbf{53.7} \\
\bottomrule%[1pt]
\end{tabular}
}
\label{tab:SOTA-COCOStuff164k}
\end{table}

\section{Experiments on COCOStuff164k Dataset}

COCOStuff164k has become increasingly popular in recent years and poses a significant challenge for semantic segmentation models due to its high diversity, consisting of 118,000 training images and 5,000 testing images, along with its complexity of 171 classes.

In Table~\ref{tab:SOTA-COCOStuff164k}, our VPNeXt model outperforms previous state-of-the-art methods, including ViT-Adapter and InternImage, by a significant margin.





\section{Experiments on Cityscapes Dataset}

Cityscapes is a semantic segmentation dataset featuring high-resolution images of road scenes with precise annotations. 
%
It includes 19 labeled classes and contains 2,975 training images and 500 validation images.
%
We only compare methods trained on the Cityscapes fine annotations, similar to many other works.~\cite{cSegFormer,cKMaXDeepLab}.

As shown in Table~\ref{tab:SOTA-Cityscapes}, our proposed VPNeXt, leveraging ViTUp's strong capabilities, performs comparably to state-of-the-art pyramid-based models (e.g., HFGD~\cite{cHFGD} and DPP~\cite{cDDP}) on high-resolution images.
%

\begin{table}[ht]
\centering
\small
\caption{
Comparisons to state-of-the-art methods on Cityscapes validation set.
%
\textit{SS}: Single scale performance w/o flipping.
\textit{MF}: Multi-scale performance w/ flipping.
``-'' in column \textit{SS} indicates that this result was not reported in the original paper.
}
\resizebox{\linewidth}{!}{
\begin{tabular}{l|c|c|c|c}
\toprule%[1pt]
Methods &Backbone & Avenue &\multicolumn{2}{c}{mIOU(\%)}\\
&  & & SS & MF \\
\midrule
\midrule
RepVGG\cite{cRepVGG} & RepVGG-B2 & CVPR'21 & - & 80.6 \\
SETR~\cite{cSegFormer} &ViT-L & CVPR'21 & - & 82.2 \\
Segmenter~\cite{cSegmenter} &ViT-L & ICCV'21 & - & 81.3 \\
OCR~\cite{cOCR,cHRFormer} &HRFormer-B & NIPS'21 & - & 82.6 \\
HRViT-b3~\cite{cHRViT}  & MiT-B3 & CVPR'22 & - & 83.2\\
FAN-L~\cite{cFANs} & FAN-Hybrid & ICML'22 & - & 82.3 \\
SegDeformer~\cite{cSegDeformer} & Swin-L & ECCV'22 & - & 83.5 \\
GSS-FT-W~\cite{cGSS} & Swin-L & CVPR'23 & - & 80.5 \\
TSG~\cite{cTSG} & Swin-L & CVPR'23 & - & 83.1 \\
STL~\cite{cSTL} & FAN-Hybrid & ICCV'23 & - & 82.8 \\
DDP(Step 3)~\cite{cDDP} & ConvNeXt-L & ICCV'23 & 83.2 & 83.9 \\
StructToken~\cite{cStructToken} & ViT-L & TCSVT'23 & 80.1 & 82.1 \\
GSCNN(EPL)~\cite{cEPL} & WRNet-38 & TMM'23 & - & 81.78\\
CART~\cite{cCART} & ConvNeXt-L & TCSVT'24 & 82.8 & 83.6\\
HFGD~\cite{cHFGD} & ConvNeXt-L & TCSVT'24 & 83.2 & 84.0 \\
% \midrule
\midrule%[0.1pt]
VPNeXt   & ViT-L              & - & 83.0 & \textbf{84.4} \\
\bottomrule%[1pt]
\end{tabular}
}
\label{tab:SOTA-Cityscapes}
\end{table}


\section{Experiments on VOC2012}
VOC2012 is one of the most classic datasets of semantic segmentation. 
%
It features a small number of categories (21 w/ background), medium resolution, and high annotation accuracy, which allowed earlier methods to achieve a mIoU of 89\% between 2018 and 2019. 

In subsequent years, although stronger methods were developed, they only resulted in slight improvements to the mIoU—usually by a few tenths (\ie < 0.5\%). 
%
Eventually, SegNeXt~\cite{cSegNeXt} raised the mIoU to 90.6\% in 2022, and since then, no other method has surpassed this mIOU wall. 
%
As a result, SegNeXt was considered the ceiling for this dataset.


\begin{table}[h]
    \normalsize
    \centering
    \begin{tabular}{cr}
        \midrule
          \quad there is no wall \quad\\
          \quad \\
         &\textit{\quad\quad-- Sam Altman~\cite{cNoWall}}\\
         \midrule
    \end{tabular}
\end{table}


Today, our proposed VPNeXt has broken this wall. 
%
%
As shown in Table.~\ref{tab:SOTA-VOC2012},
in terms of mIoU, our proposed VPNeXt not only outperforms SegNeXt but also exceeds SegNeXt by nearly 2\%., which also stands as the largest improvement since 2015.
%
Remarkably, VPNeXt excels in long-tailed categories (\eg chair, monitor) that have traditionally posed challenges for nearly all prior methods.


\begin{table*}[th!]
%\setlength{\tabcolsep}{0pt}
\centering
\small
\caption{
% \hy{
New breakthroughs in the VOC2012 leaderboard!
Due to limited space on the page, we have simplified some category names (e.g., "Aero Plane" to "Plane") and only listed the top 15 methods. 
Zoom in to see better.
To view the full leaderboard, please visit \url{http://host.robots.ox.ac.uk:8080/leaderboard/displaylb_main.php?challengeid=11&compid=6}.
}
\resizebox{\linewidth}{!}
{\def\arraystretch{1} \tabcolsep=0.55em 
\begin{tabular}{l|c|cccccccccccccccccccc}
\toprule%[1pt]
Methods & 
Mean & 
Plane &
Bicycle &
Bird &
Boat &
Bottle &
Bus &
Car &
Cat &
Chair &
Cow &
Table &
Dog &
Horse &
Motor &
Person &
Plant &
Sheep &
Sofa &
Train &
Monitor \\
\midrule
\midrule
\textbf{Our VPNeXt} & \textbf{92.2}& 98.9 & 78.5 & 98.6 & 92.1 & 92.3 & 95.2 & 96.8 & 96.1 & 70.7 & 98.8 & 79.9 & 96.0 & 98.4 & 96.9 & 95.8 & 89.8 & 98.2& 78.1 & 96.6 & 91.3\\
\midrule
\midrule
SegNeXt& 90.6 & 98.3 & 85.0 & 97.6 & 88.3 & 91.3 & 97.5 & 91.4 & 98.3 & 60.4 & 96.7 & 85.0 & 95.7 &98.2 & 94.2 &92.7 &82.5 & 97.3 & 77.7 & 93.1 & 84.3\\
\midrule
NAS-FPN(NS)& 90.5 & 98.0 & 84.8 & 89.6 & 88.2 & 91.0 & 98.3 & 93.0 & 98.5 & 57.5& 98.4 & 81.8 & 98.4 & 98.0 & 95.8 & 93.2 & 83.2 & 97.8 & 75.0 & 91.8 & 90.0\\
\midrule
DeepLabv3+(JFT) & 89.0 & 97.5 & 77.9 & 96.2 & 80.4 & 90.8 & 98.3 & 95.5 & 97.6 & 58.8 & 96.1 & 79.2 & 95.0 &97.3 & 94.1 & 93.8 & 78.5 & 95.5 & 74.4 & 93.8 & 81.6\\
\midrule
RecoNet152 & 89.0 & 97.3 & 80.4 & 96.5 & 83.8 & 89.5 & 97.6 & 95.4 & 97.7 & 50.1 &	96.8 & 82.6 &	95.1 & 97.7 &	95.1 & 92.6 & 80.2 & 95.2 & 71.7 & 92.1 & 83.8\\
\midrule
AASPP & 88.5 & 97.4 & 80.3 & 97.1 & 80.1 & 89.3 & 97.4 & 94.1 & 96.9 & 61.9 & 95.1 & 77.2 & 94.2 & 97.5 & 94.4 & 93.0 & 72.4 & 93.8 & 72.6 & 93.3 & 83.3\\
\midrule
SRC-B & 88.5 & 97.2 & 78.6 & 97.1 & 80.6 & 89.7 & 97.4 & 93.7 & 96.7 & 59.1 & 95.4 & 81.1 & 93.2 & 97.5 & 94.2 & 92.9 & 73.5 & 93.3 & 74.2 & 91.0 & 85.0\\
\midrule
SepaNet & 88.3 & 97.2 & 80.2 & 96.2 & 80.0 & 89.2 & 97.3 & 94.7 & 97.7 & 48.6 & 95.0 & 81.6 & 95.2 & 97.5 & 95.1 & 92.7 & 79.5 & 95.4 & 68.8 & 90.9 & 83.4\\
\midrule
EMANet152 & 88.2 & 96.8 & 79.4 & 96.0 & 83.6 & 88.1 & 97.1 & 95.0 & 96.6 & 49.4 & 95.4 & 77.8 & 94.8 & 96.8 & 95.1 & 92.0 & 79.3 & 95.9 & 68.5 & 91.7 & 85.6\\
\midrule
KSAC-H & 88.1 & 97.2 & 79.9 & 96.3 & 76.5 & 86.5 & 97.5 & 94.5 & 96.9 & 54.8 & 95.3 & 81.4 & 93.7 & 97.2 & 94.0 & 92.8 & 77.3 & 94.4 & 73.5 & 91.1 & 83.4\\
\midrule
SpDConv2 & 88.1 & 96.9 & 79.7 & 96.8 & 80.2 & 87.8 & 98.0 & 92.3 & 96.0 & 57.2 & 95.8 & 82.1 & 92.3 & 97.3 & 93.6 & 93.0 & 71.4 & 92.3 & 75.8 & 90.7 & 83.8\\
\midrule
FillIn & 88.0 & 97.1 & 80.8 & 96.7 & 77.6 & 89.2 & 97.4 & 92.2 & 96.9 & 58.3 & 94.3 & 79.4 & 93.1 & 97.3 & 94.4 & 93.2 & 73.6 & 93.0 & 72.6 & 89.7 & 83.4\\	
\midrule
MSCI & 88.0 & 96.8 & 76.8 & 97.0 & 80.6 & 89.3 & 97.4 & 93.8 & 97.1 & 56.7 & 94.3 & 78.3 & 93.5 & 97.1 & 94.0 & 92.8 & 72.3 & 92.6 & 73.6 & 90.8 & 85.4\\
\midrule
ExFuse & 87.9 & 96.8 & 80.3 & 97.0 & 82.5 & 87.8 & 96.3 & 92.6 & 96.4 & 53.3 & 94.3 & 78.4 & 94.1 & 94.9 & 91.6 & 92.3 & 81.7 & 94.8 & 70.3 & 90.1 & 83.8\\
\midrule
DeepLabV3+ & 87.8 & 97.0 & 77.1 & 97.1 & 79.3 & 89.3 & 97.4 & 93.2 & 96.6 & 56.9 & 95.0 & 79.2 & 
93.1 & 97.0 & 94.0 & 92.8 & 71.3 & 92.9 & 72.4 & 91.0 & 84.9\\
\bottomrule%[1pt]
\end{tabular}
}
\label{tab:SOTA-VOC2012}
\end{table*}
\section{Visualization}

\begin{figure}[ht!]
    \centering
    \includegraphics[width=1.0\linewidth]{images/ufcn_vs_internimage_coco.pdf}
    \caption{Visual comparison between Mask2Former + InternImage-H~\cite{cMask2Former,cInternImage} and our proposed VPNeXt on the COCOStuff164k~\cite{cCocoStuff} dataset shows that VPNeXt achieves superior segmentation results, particularly in challenging categories such as food.}
    \label{fig:cocostuff_internimage_vs_vpnext}
\end{figure}



\begin{figure}[ht!]
    \centering
    \includegraphics[width=\linewidth]{images/ufcn_vcr_feature_vis.pdf}
    \caption{Visualization analysis of the intermediate feature map for the VCR is based on the 15th layer of the ViT. At the pixel position marked by the red dot, the replay-optimized feature map displays significantly stronger and more detailed semantic information concerning intra-class pixels.}
    \label{fig:vcr_feature_vis}
\end{figure}



\subsection{Visualization on COCOStuff164k dataset}
We first conducted a visualization comparison on the challenging COCOStuff164k dataset~\cite{cCocoStuff}. 
As shown in Figure.~\ref{fig:cocostuff_internimage_vs_vpnext}, our proposed VPNeXt achieves significantly better segmentation results compared to the state-of-the-art Mask2Former + InternImage-H~\cite{cMask2Former,cInternImage}, particularly in some challenging categories, such as food.

\subsection{Visualization on intermediate feature map}
We conducted a visualization analysis of the intermediate feature map of the VCR. 
%
For this analysis, we utilized the feature map from the $15^{\text{th}}$ layer of the ViT. 
As illustrated in Figure.~\ref{fig:vcr_feature_vis}, at the position marked by the red dot (\textcolor{red}{$\bullet$}), the replay-optimized feature map presents significantly stronger and more intensive semantic information regarding intra-class pixels.



\section{Conclusion}
It is crucial for science communication to engage the general public, and prior research suggests that using colloquial techniques from social media can be effective. Despite this, many scientists are hesitant to apply these techniques due to concerns about losing their authoritative voice. Our research highlights the complexity of public science communication and the need to balance readers’ and writers’ perspectives. While readers generally preferred explanations that included examples, walkthroughs, and personal language, their preferences were nuanced and context-dependent, influenced by their personal experiences and the complexity of the topic. Conversely, writers often feared that these techniques might compromise the clarity or authority of their explanations. However, when given the opportunity to explore various narrative structures and styles, writers were able to navigate their choices with greater confidence, finding a balance between colloquial and formal approaches. This suggests that effective science communication benefits from exploring diverse options, allowing writers to tailor their style to the scientific topic, their own preferences, and the needs of their audience.


% Using a mixed-methods approach, we conducted a survey of reader's preferences for how science explanations are presented. We explored the impact of an example, a step-by-step walkthrough, and personal language on a reader's engagement and understanding of a STEM topic. We found that most readers preferred science explanations that contained an example, a step-by-step walkthrough, and personal language, but there are nuances to each reader's rating. Thus, we conducted followup interviews with the survey participants to better understand the details of each reader's preferences for 3 dimensions: relateable examples, a step-by-step walkthrough, and personal language. 

% Based on these findings, we ran a writer's study to explore whether seeing different methods of structuring and styling a science explanation can help writers mitigate their hesitancy around science communication on social media. We found that offering writers different options for how to structure and style their writing helped writer choose which characteristics of each they wanted to include. Instead of presenting one method for science communication on social media, by offering writers multiple options, writers felt more comfortable exploring these options.

% % \grace{IDK FILL IN}

% Overall, our study provides insights into both reader and writer preferences for science communication and highlight a design space for supporting writers navigate a continuum of writing structures and styles.  
\input{sec/11_limitation}

\newpage

\bibliographystyle{IEEEtran}
\bibliography{main}




\newpage

\section{Biography}
\begin{comment}
If you have an EPS/PDF photo (graphicx package needed), extra braces are
 needed around the contents of the optional argument to biography to prevent
 the LaTeX parser from getting confused when it sees the complicated
 $\backslash${\tt{includegraphics}} command within an optional argument. (You can create
 your own custom macro containing the $\backslash${\tt{includegraphics}} command to make things
 simpler here.)
\end{comment}
 
%\vspace{11pt}
\begin{comment}
\bf{If you include a photo:}\vspace{-33pt}
\begin{IEEEbiography}[{\includegraphics[width=1in,height=1.25in,clip,keepaspectratio]{fig1}}]{Michael Shell}
Use $\backslash${\tt{begin\{IEEEbiography\}}} and then for the 1st argument use $\backslash${\tt{includegraphics}} to declare and link the author photo.
Use the author name as the 3rd argument followed by the biography text.
\end{IEEEbiography}

\vspace{11pt}
\end{comment}

\begin{IEEEbiographynophoto}{Xikai Tang} is a PhD student at School of Information and Software Engineering, University of Electronic Science and Technology of China.
\end{IEEEbiographynophoto}

\begin{IEEEbiographynophoto}{Ye Huang} received the B.S. degree and the Ph.D. degree in Computer Science from the University of Technology Sydney, Australia. 
He is currently an associate researcher in the Data Intelligence Group, Shenzhen Institute for Advanced Study, University of Electronic Science and Technology of China. 
\end{IEEEbiographynophoto}

\begin{IEEEbiographynophoto}{Guangqiang Yin} is currently a Professor with the University of Electronic Science and Technology of China (UESTC). His research interests include computer-vision-related artificial intelligence techniques and applications, and computer modeling of properties of condensed matter.
\end{IEEEbiographynophoto}

\begin{IEEEbiographynophoto}{Lixin Duan} (Member, IEEE) received the B.E. degree from the University of Science and Technology of China, Hefei, China, in 2008, and the Ph.D. degree from the Nanyang Technological University, Singapore, in 2012. 
He is currently a professor with the University of Electronic Science and Technology of China. 
His current research interests include transfer learning, multiple instance learning, and their applications in computer vision and data mining.
\end{IEEEbiographynophoto}




\vfill

\end{document}


