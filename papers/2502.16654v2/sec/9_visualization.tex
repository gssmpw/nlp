\section{Visualization}

\begin{figure}[ht!]
    \centering
    \includegraphics[width=1.0\linewidth]{images/ufcn_vs_internimage_coco.pdf}
    \caption{Visual comparison between Mask2Former + InternImage-H~\cite{cMask2Former,cInternImage} and our proposed VPNeXt on the COCOStuff164k~\cite{cCocoStuff} dataset shows that VPNeXt achieves superior segmentation results, particularly in challenging categories such as food.}
    \label{fig:cocostuff_internimage_vs_vpnext}
\end{figure}



\begin{figure}[ht!]
    \centering
    \includegraphics[width=\linewidth]{images/ufcn_vcr_feature_vis.pdf}
    \caption{Visualization analysis of the intermediate feature map for the VCR is based on the 15th layer of the ViT. At the pixel position marked by the red dot, the replay-optimized feature map displays significantly stronger and more detailed semantic information concerning intra-class pixels.}
    \label{fig:vcr_feature_vis}
\end{figure}



\subsection{Visualization on COCOStuff164k dataset}
We first conducted a visualization comparison on the challenging COCOStuff164k dataset~\cite{cCocoStuff}. 
As shown in Figure.~\ref{fig:cocostuff_internimage_vs_vpnext}, our proposed VPNeXt achieves significantly better segmentation results compared to the state-of-the-art Mask2Former + InternImage-H~\cite{cMask2Former,cInternImage}, particularly in some challenging categories, such as food.

\subsection{Visualization on intermediate feature map}
We conducted a visualization analysis of the intermediate feature map of the VCR. 
%
For this analysis, we utilized the feature map from the $15^{\text{th}}$ layer of the ViT. 
As illustrated in Figure.~\ref{fig:vcr_feature_vis}, at the position marked by the red dot (\textcolor{red}{$\bullet$}), the replay-optimized feature map presents significantly stronger and more intensive semantic information regarding intra-class pixels.


