\begin{table}[h]
    \centering
    \caption{The prompts used in the method: Abstract Path Selection Template.}
    \label{table:template2}
    \begin{tabularx}{\textwidth}{X} % 使用 tabularx 环境
        \toprule
        \textbf{Abstract Path Selection Template} \\
        \midrule
        \vspace{2mm}
        Please filter the reasoning paths based on the user question and the given possible reasoning paths. \\
        \textbf{User question: \{question\}} \\
        \textbf{Possible reasoning paths: \{paths\}} \\
        \vspace{2mm}
        \textbf{Explanation:} \\
        - I have used a large model to extract known conditions from the user question. \\
        - Starting from these known conditions, I performed a depth-first search in the domain knowledge graph to extract all reasoning paths that start with the labels of these conditions. \\
        - Each path begins with a condition entity, and the path connects multiple entity labels. \\
        \vspace{2mm}
        \textbf{Filter criteria:} \\
        - Try to filter out paths that are helpful for the answer as much as possible. If the user asks 'What could be the problem?', then the pathways for diseases, medical examinations, and medication should be preserved. A separate pathway to the disease should also be kept. \\
        - Ensure that the output paths are not duplicated. \\
        \vspace{2mm}
        Please return the filtered reasoning paths. \\
        \bottomrule
    \end{tabularx}
\end{table}



\begin{table*}[h]
    \centering
    \caption{The prompts used in the method: Entity Extract Template.}
    \label{table:template1}
    
    \begin{tabular}{p{15cm}}
        \toprule
        % \textbf{Entity Extract Template} \\  % 结束当前行并插入顶部线
        % \midrule  
        \vspace{2mm} 
        I am working on a knowledge graph-enhanced question answering system. Your task is to extract conditional entities and their types and target entities and their types from the user's input question. \\
        \vspace{2mm}  
        \textbf{\texttt{\#\#\# Please choose the entity types from the following table:}} \\
        Each row describes an entity type in the format - Entity Type (Description) \\
        \{label\_description\} \\
        
        \vspace{2mm}  
        \textbf{\texttt{\#\#\# Rules:}} \\
        - Conditional entities are the known information provided in the question. \\
        - Target entities are the content the user wants to query in the question. \\
        - If no suitable entity exists, use "none" to represent it. \\
        
        \vspace{2mm}  
        \textbf{\texttt{\#\#\# Output format:}} \\
        - Conditional entities and target entities should be separated by a **"."** (period). \\
        - Each entity should be in the format **"Entity Name, Entity Type"**. \\
        - If there are multiple conditional or target entities, separate them using **";"** (semicolon). \\
        - If only one of conditional or target entities exists, for example: \\
          - If there are only conditional entities and no target entity, output: ce1,cl1;ce2,cl2.none,none \\
          - If there are only target entities and no conditional entity, output: none,none.ae1,al1;ae2,al2 \\
        - Only output the final answer, no extra explanations, comments, or text. \\
        
        \vspace{2mm}  
        \textbf{\texttt{\#\#\# Example:}} \\
        \{entity\_extract\_example\} \\
        The user's question is: \{question\} \\
        Please generate the answer in the above format: \\
        Output: Lou Seal,mascot. championship, championship \\
        
        \vspace{2mm}  
        Example1: \\
        Input: Lou Seal is the mascot for the team that last won the World Series when? \\
        Output: Lou Seal,mascot. championship, championship \\
        
        \vspace{2mm} 
        Example2: \\
        Input: What educational institution has a football sports team named Northern Colorado Bears is in Greeley, Colorado? \\
        Output: Northern Colorado Bears,team.institution,institution \\
        
        \vspace{2mm} 
        Example3: \\
        Input: What is the predominant religion where the leader is Ovadia Yosef? \\
        Output: Ovadia Yosef,person.religion,religion \\
       
        \bottomrule
    \end{tabular}
\end{table*}

% \begin{table}[h]
%     \centering
%     \caption{The prompts used in the method: Answer Template.}
%     \label{table:template4}
%     \begin{tabular}{p{15cm}}
%         \toprule
%         \textbf{Answer Template} \\
%         \midrule
%         \vspace{2mm}
%         - Only output the answer to the question, no additional information is required.\\
         
%         - If there are multiple answers, separate them with commas.\\
        
%         - Each answer should be as complete as possible.\\
        
%         - You can refer to Aims Nodes.\\
         
%         - In most cases, there is only one answer, and sometimes there may be multiple answers.\\
        
%         \vspace{2mm}
%         \textbf{Example:}\\
%         Question: The national anthem Afghan National Anthem is from the country which practices what religions?\\
%         Answer: Shia Islam,Sunni Islam \\
        
%         \bottomrule
%     \end{tabular}
% \end{table}


\begin{table}[htbp]
    \centering
    \caption{The prompts used in the method: Answer Template.}
    \label{table:template3}
    \begin{tabular}{p{15cm}}
        \toprule
        % \textbf{Answer Template} \\
        % \midrule
        \vspace{2mm}
        You are a helpful and knowledgeable assistant. Your task is to provide precise answers by performing logical reasoning based on the user's input and additional reference content. \\
        Remember: You must not disclose or mention the existence of the reference content provided to you in your response.\\
       \vspace{4mm}
        \textbf{The user has input the following question:}
        \texttt{\{question\}} \\
      
        I can provide you with some reference content, where each set of content consists of two parts: conditions and objectives. \\
        - Conditions: The known information from the question. \\
 
        - Objectives: The goals that the question seeks to address. \\
        \vspace{4mm}
        \textbf{Below is the conditions and objectives:}
        \texttt{\{last\_node\_str\}}\\
    
        I can also provide you the complete reasoning paths which may be useful for you. I wish you could utilize your reasoning ability to answer the user's question.\\
       
        - Each path is described by nodes and edges in the following format: [Entity Type] Entity Name -> (Relation) [Entity Type] Entity Name -> \dots\\
      
        - Each node includes [Entity Type] Entity Name.\\

        - Each edge is represented by an arrow ->, with the edge information enclosed in parentheses, e.g., (Relation).\\

        - Starting from the root node, the path is described step by step, including nodes and their relationships, until reaching the leaf node.\\
        \vspace{4mm}
        \textbf{The abstract paths are as follows:}
        \texttt{\{rules\_string\}} \\
        \textbf{Below are the reasoning paths:}
        \texttt{\{reasoning\_path\_str\}}\\
    
        Please strictly follow the reference content to answer the question, applying logical reasoning as needed to generate the final answer.\\
        
        \textbf{Note:} The generated answer must not mention or disclose the existence of the reference content.\\
        The output can refer to the following format. \\
        \textbf{\texttt{\{Answer Template\}}} \\
        \vspace{2mm}
        - The answer should be a paragraph without line breaks and in order.\\
        
        - Only output the answer to the question, no additional information is required.\\
         
        - If there are multiple answers, separate them with commas.\\
        
        - Each answer should be as complete as possible.\\
        
        - You can refer to Aims Nodes.\\
         
        - In most cases, there is only one answer, and sometimes there may be multiple answers.\\
        
      
        \bottomrule
    \end{tabular}
\end{table}


