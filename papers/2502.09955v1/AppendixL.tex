\section{IMO Combinatorics Limitation Examples}
\label{appendix:L}

Here are examples that approach does not handle and may not be suitable for a game representation or simulations.

\subsection{Problems that Require Finding Invariants}
In IMO 2011 Problem 2, also known as the Windmill problem, which our approach does not represent as a game, the solution requires finding an invariant.

\begin{tcolorbox}[enhanced, breakable, rounded corners,
    colback=red!5!white, colframe=red!75!black,
    colbacktitle=red!85!black, fonttitle=\bfseries, coltitle=white, title=IMO 2011 Problem 2 (Windmill)]

\setlength{\parskip}{1em}
Let $\mathcal{S}$ be a finite set of at least two points in the plane. Assume that no three points of $\mathcal{S}$ are collinear. A windmill is a process that starts with a line $\ell$ going through a single point $P \in \mathcal{S}$. The line rotates clockwise about the pivot $P$ until the first time that the line meets some other point belonging to $\mathcal{S}$. This point, $Q$, takes over as the new pivot, and the line now rotates clockwise about $Q$, until it next meets a point of $\mathcal{S}$. This process continues indefinitely.\\
Show that we can choose a point $P$ in $\mathcal{S}$ and a line $\ell$ going through $P$ such that the resulting windmill uses each point of $\mathcal{S}$ as a pivot infinitely many times.
\end{tcolorbox}


\subsection{Problems in High Dimensional Spaces}
In IMO 2010 Problem 5, the solution requires showing that one of the boxes contains exactly $2010^{2010^{2010}}$ coins. Our visual approach is suitable for simulating small instances of games rather than high dimensional spaces.

\begin{tcolorbox}[enhanced, breakable, rounded corners,
    colback=red!5!white, colframe=red!75!black,
    colbacktitle=red!85!black, fonttitle=\bfseries, coltitle=white, title=IMO 2010 Problem 5 (Boxes)]

\setlength{\parskip}{1em}
In each of six boxes $B_{1}, B_{2}, B_{3}, B_{4}, B_{5}, B_{6}$ there is initially one coin. There are two types of operation allowed:

Type 1: Choose a nonempty box $B_{j}$ with $1 \leq j \leq 5$. Remove one coin from $B_{j}$ and add two coins to $B_{j+1}$.\\
Type 2: Choose a nonempty box $B_{k}$ with $1 \leq k \leq 4$. Remove one coin from $B_{k}$ and exchange the contents of (possibly empty) boxes $B_{k+1}$ and $B_{k+2}$.

Determine whether there is a finite sequence of such operations that results in boxes $B_{1}, B_{2}, B_{3}, B_{4}, B_{5}$ being empty and box $B_{6}$ containing exactly $2010^{2010^{2010}}$ coins. (Note that $a^{b^{c}}=a^{\left(b^{c}\right)}$.
\end{tcolorbox}