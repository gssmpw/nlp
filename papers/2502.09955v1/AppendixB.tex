\section{IMO Combinatorics Problems, Answers, and Solutions}
\label{appendix:B}

We do not use the 2023 IMO Shortlist combinatorics problem 3 selected for the 2023 IMO (as problem 5) since its solutions are released in 7/23; however, all the other 2023 IMO Shortlist combinatorics problems are released after the IMO of the following year, namely 7/24, after the knowledge cutoff dates.

\subsection*{2024 IMO}
\label{appendix:B_Combinatorics_2024_IMO}
\begin{tcolorbox}[enhanced, breakable, rounded corners,
    colback=blue!5!white, colframe=blue!75!black,
    colbacktitle=blue!85!black, fonttitle=\bfseries, coltitle=white, title=Problem 3]

\setlength{\parskip}{1em}
Let $a_1, a_2, a_3, \dots$ be an infinite sequence of positive integers, and let $N$ be a positive integer. Suppose that, for each $n > N$, $a_n$ is equal to the number of times $a_{n-1}$ appears in the list $a_1, a_2, \dots, a_{n-1}$.

Prove that at least one of the sequences $a_1, a_3, a_5, \dots$ and $a_2, a_4, a_6, \dots$ is eventually periodic.

(An infinite sequence $b_1, b_2, b_3, \dots$ is eventually periodic if there exist positive integers $p$ and $M$ such that $b_{m+p} = b_m$ for all $m \geq M$.)
\end{tcolorbox}

\begin{tcolorbox}[enhanced, breakable, rounded corners, 
    colback=orange!5!white, colframe=orange!75!black,
    colbacktitle=orange!85!black, fonttitle=\bfseries, coltitle=white,
    title=Problem 3 Answer, width=\columnwidth]
NA
\end{tcolorbox}

\begin{tcolorbox}[enhanced, breakable, rounded corners,
    colback=green!5!white, colframe=green!75!black,
    colbacktitle=green!85!black, fonttitle=\bfseries, coltitle=white, title=Problem 3 Solution 1]

\setlength{\parskip}{1em}
Let $M>\max \left(a_{1}, \ldots, a_{N}\right)$. We first prove that some integer appears infinitely many times. If not, then the sequence contains arbitrarily large integers. The first time each integer larger than $M$ appears, it is followed by a 1 . So 1 appears infinitely many times, which is a contradiction.

Now we prove that every integer $x \geqslant M$ appears at most $M-1$ times. If not, consider the first time that any $x \geqslant M$ appears for the $M^{\text {th }}$ time. Up to this point, each appearance of $x$ is preceded by an integer which has appeared $x \geqslant M$ times. So there must have been at least $M$ numbers that have already appeared at least $M$ times before $x$ does, which is a contradiction.

Thus there are only finitely many numbers that appear infinitely many times. Let the largest of these be $k$. Since $k$ appears infinitely many times there must be infinitely many integers greater than $M$ which appear at least $k$ times in the sequence, so each integer $1,2, \ldots, k-1$ also appears infinitely many times. Since $k+1$ doesn't appear infinitely often there must only be finitely many numbers which appear more than $k$ times. Let the largest such number be $l \geqslant k$. From here on we call an integer $x$ big if $x>l$, medium if $l \geqslant x>k$ and small if $x \leqslant k$. To summarise, each small number appears infinitely many times in the sequence, while each big number appears at most $k$ times in the sequence.

Choose a large enough $N^{\prime}>N$ such that $a_{N^{\prime}}$ is small, and in $a_{1}, \ldots, a_{N^{\prime}}$ :
- every medium number has already made all of its appearances;
- every small number has made more than $\max (k, N)$ appearances.

Since every small number has appeared more than $k$ times, past this point each small number must be followed by a big number. Also, by definition each big number appears at most $k$ times, so it must be followed by a small number. Hence the sequence alternates between big and small numbers after $a_{N^{\prime}}$.
Lemma 1. Let $g$ be a big number that appears after $a_{N^{\prime}}$. If $g$ is followed by the small number $h$, then $h$ equals the amount of small numbers which have appeared at least $g$ times before that point.
Proof. By the definition of $N^{\prime}$, the small number immediately preceding $g$ has appeared more than $\max (k, N)$ times, so $g>\max (k, N)$. And since $g>N$, the $g^{\text {th }}$ appearance of every small number must occur after $a_{N}$ and hence is followed by $g$. Since there are $k$ small numbers and $g$ appears at most $k$ times, $g$ must appear exactly $k$ times, always following a small number after $a_{N}$. Hence on the $h^{\text {th }}$ appearance of $g$, exactly $h$ small numbers have appeared at least $g$ times before that point.

Denote by $a_{[i, j]}$ the subsequence $a_{i}, a_{i+1}, \ldots, a_{j}$.
Lemma 2. Suppose that $i$ and $j$ satisfy the following conditions:
(a) $j>i>N^{\prime}+2$,
(b) $a_{i}$ is small and $a_{i}=a_{j}$,
(c) no small value appears more than once in $a_{[i, j-1]}$.

Then $a_{i-2}$ is equal to some small number in $a_{[i, j-1]}$.

Proof. Let $\mathcal{I}$ be the set of small numbers that appear at least $a_{i-1}$ times in $a_{[1, i-1]}$. By Lemma 1, $a_{i}=|\mathcal{I}|$. Similarly, let $\mathcal{J}$ be the set of small numbers that appear at least $a_{j-1}$ times in $a_{[1, j-1]}$. Then by Lemma $1, a_{j}=|\mathcal{J}|$ and hence by (b), $|\mathcal{I}|=|\mathcal{J}|$. Also by definition, $a_{i-2} \in \mathcal{I}$ and $a_{j-2} \in \mathcal{J}$.

Suppose the small number $a_{j-2}$ is not in $\mathcal{I}$. This means $a_{j-2}$ has appeared less than $a_{i-1}$ times in $a_{[1, i-1]}$. By (c), $a_{j-2}$ has appeared at most $a_{i-1}$ times in $a_{[1, j-1]}$, hence $a_{j-1} \leqslant a_{i-1}$. Combining with $a_{[1, i-1]} \subset a_{[1, j-1]}$, this implies $\mathcal{I} \subseteq \mathcal{J}$. But since $a_{j-2} \in \mathcal{J} \backslash \mathcal{I}$, this contradicts $|\mathcal{I}|=|\mathcal{J}|$. So $a_{j-2} \in \mathcal{I}$, which means it has appeared at least $a_{i-1}$ times in $a_{[1, i-1]}$ and one more time in $a_{[i, j-1]}$. Therefore $a_{j-1}>a_{i-1}$.

By (c), any small number appearing at least $a_{j-1}$ times in $a_{[1, j-1]}$ has also appeared $a_{j-1}-1 \geqslant$ $a_{i-1}$ times in $a_{[1, i-1]}$. So $\mathcal{J} \subseteq \mathcal{I}$ and hence $\mathcal{I}=\mathcal{J}$. Therefore, $a_{i-2} \in \mathcal{J}$, so it must appear at least $a_{j-1}-a_{i-1}=1$ more time in $a_{[i, j-1]}$.

For each small number $a_{n}$ with $n>N^{\prime}+2$, let $p_{n}$ be the smallest number such that $a_{n+p_{n}}=a_{i}$ is also small for some $i$ with $n \leqslant i<n+p_{n}$. In other words, $a_{n+p_{n}}=a_{i}$ is the first small number to occur twice after $a_{n-1}$. If $i>n$, Lemma 2 (with $j=n+p_{n}$ ) implies that $a_{i-2}$ appears again before $a_{n+p_{n}}$, contradicting the minimality of $p_{n}$. So $i=n$. Lemma 2 also implies that $p_{n} \geqslant p_{n-2}$. So $p_{n}, p_{n+2}, p_{n+4}, \ldots$ is a nondecreasing sequence bounded above by $2 k$ (as there are only $k$ small numbers). Therefore, $p_{n}, p_{n+2}, p_{n+4}, \ldots$ is eventually constant and the subsequence of small numbers is eventually periodic with period at most $k$.

Note. Since every small number appears infinitely often, Solution 1 additionally proves that the sequence of small numbers has period $k$. The repeating part of the sequence of small numbers is thus a permutation of the integers from 1 to $k$. It can be shown that every permutation of the integers from 1 to $k$ is attainable in this way.
    \end{tcolorbox}

\begin{tcolorbox}[enhanced, breakable, rounded corners,
    colback=green!5!white, colframe=green!75!black,
    colbacktitle=green!85!black, fonttitle=\bfseries, coltitle=white, title=Problem 3 Solution 2]
\setlength{\parskip}{1em}
We follow Solution 1 until after Lemma 1. For each $n>N^{\prime}$ we keep track of how many times each of $1,2, \ldots, k$ has appeared in $a_{1}, \ldots, a_{n}$. We will record this information in an updating $(k+1)$-tuple

$$
\left(b_{1}, b_{2}, \ldots, b_{k} ; j\right)
$$

where each $b_{i}$ records the number of times $i$ has appeared. The final element $j$ of the $(k+1)-$ tuple, also called the active element, represents the latest small number that has appeared in $a_{1}, \ldots, a_{n}$.

As $n$ increases, the value of $\left(b_{1}, b_{2}, \ldots, b_{k} ; j\right)$ is updated whenever $a_{n}$ is small. The $(k+1)$ tuple updates deterministically based on its previous value. In particular, when $a_{n}=j$ is small, the active element is updated to $j$ and we increment $b_{j}$ by 1 . The next big number is $a_{n+1}=b_{j}$. By Lemma 1, the next value of the active element, or the next small number $a_{n+2}$, is given by the number of $b$ terms greater than or equal to the newly updated $b_{j}$, or


\begin{equation*}
\left|\left\{i \mid 1 \leqslant i \leqslant k, b_{i} \geqslant b_{j}\right\}\right| \tag{1}
\end{equation*}


Each sufficiently large integer which appears $i+1$ times must also appear $i$ times, with both of these appearances occurring after the initial block of $N$. So there exists a global constant $C$ such that $b_{i+1}-b_{i} \leqslant C$. Suppose that for some $r, b_{r+1}-b_{r}$ is unbounded from below. Since the value of $b_{r+1}-b_{r}$ changes by at most 1 when it is updated, there must be some update where $b_{r+1}-b_{r}$ decreases and $b_{r+1}-b_{r}<-(k-1) C$. Combining with the fact that $b_{i}-b_{i-1} \leqslant C$ for all $i$, we see that at this particular point, by the triangle inequality


\begin{equation*}
\min \left(b_{1}, \ldots, b_{r}\right)>\max \left(b_{r+1}, \ldots, b_{k}\right) \tag{2}
\end{equation*}


Since $b_{r+1}-b_{r}$ just decreased, the new active element is $r$. From this point on, if the new active element is at most $r$, by (1) and (2), the next element to increase is once again from $b_{1}, \ldots, b_{r}$. Thus only $b_{1}, \ldots, b_{r}$ will increase from this point onwards, and $b_{k}$ will no longer increase, contradicting the fact that $k$ must appear infinitely often in the sequence. Therefore $\left|b_{r+1}-b_{r}\right|$ is bounded.

Since $\left|b_{r+1}-b_{r}\right|$ is bounded, it follows that each of $\left|b_{i}-b_{1}\right|$ is bounded for $i=1, \ldots, k$. This means that there are only finitely many different states for $\left(b_{1}-b_{1}, b_{2}-b_{1}, \ldots, b_{k}-b_{1} ; j\right)$. Since the next active element is completely determined by the relative sizes of $b_{1}, b_{2}, \ldots, b_{k}$ to each other, and the update of $b$ terms depends on the active element, the active element must be eventually periodic. Therefore the small numbers subsequence, which is either $a_{1}, a_{3}, a_{5}, \ldots$ or $a_{2}, a_{4}, a_{6}, \ldots$, must be eventually periodic.
\end{tcolorbox}

\begin{tcolorbox}[enhanced, breakable, rounded corners,
    colback=blue!5!white, colframe=blue!75!black,
    colbacktitle=blue!85!black, fonttitle=\bfseries, coltitle=white, title=Problem 5]


\setlength{\parskip}{1em}
Turbo the snail plays a game on a board with 2024 rows and 2023 columns. There are hidden monsters in 2022 of the cells. Initially, Turbo does not know where any of the monsters are, but he knows that there is exactly one monster in each row except the first row and the last row, and that each column contains at most one monster.

Turbo makes a series of attempts to go from the first row to the last row. On each attempt, he chooses to start on any cell in the first row, then repeatedly moves to an adjacent cell sharing a common side. (He is allowed to return to a previously visited cell.) If he reaches a cell with a monster, his attempt ends and he is transported back to the first row to start a new attempt. The monsters do not move, and Turbo remembers whether or not each cell he has visited contains a monster. If he reaches any cell in the last row, his attempt ends and the game is over.

Determine the minimum value of $n$ for which Turbo has a strategy that guarantees reaching the last row on the $n^{\text {th }}$ attempt or earlier, regardless of the locations of the monsters.\\


\end{tcolorbox}

\begin{tcolorbox}[enhanced, breakable, rounded corners, 
    colback=orange!5!white, colframe=orange!75!black,
    colbacktitle=orange!85!black, fonttitle=\bfseries, coltitle=white,
    title=Problem 5 Answer, width=\columnwidth]
The answer is $n=3$.
\end{tcolorbox}

\begin{tcolorbox}[enhanced, breakable, rounded corners,
    colback=green!5!white, colframe=green!75!black,
    colbacktitle=green!85!black, fonttitle=\bfseries, coltitle=white, title=Problem 5 Solution]

\setlength{\parskip}{1em}
First we demonstrate that there is no winning strategy if Turbo has 2 attempts.\\
Suppose that $(2, i)$ is the first cell in the second row that Turbo reaches on his first attempt. There can be a monster in this cell, in which case Turbo must return to the first row immediately, and he cannot have reached any other cells past the first row.

Next, suppose that $(3, j)$ is the first cell in the third row that Turbo reaches on his second attempt. Turbo must have moved to this cell from $(2, j)$, so we know $j \neq i$. So it is possible that there is a monster on $(3, j)$, in which case Turbo also fails on his second attempt. Therefore Turbo cannot guarantee to reach the last row in 2 attempts.

Next, we exhibit a strategy for $n=3$. On the first attempt, Turbo travels along the path

$$
(1,1) \rightarrow(2,1) \rightarrow(2,2) \rightarrow \cdots \rightarrow(2,2023)
$$

This path meets every cell in the second row, so Turbo will find the monster in row 2 and his attempt will end.

If the monster in the second row is not on the edge of the board (that is, it is in cell $(2, i)$ with $2 \leqslant i \leqslant 2022$ ), then Turbo takes the following two paths in his second and third attempts:

$$
\begin{aligned}
& (1, i-1) \rightarrow(2, i-1) \rightarrow(3, i-1) \rightarrow(3, i) \rightarrow(4, i) \rightarrow \cdots \rightarrow(2024, i) \\
& (1, i+1) \rightarrow(2, i+1) \rightarrow(3, i+1) \rightarrow(3, i) \rightarrow(4, i) \rightarrow \cdots \rightarrow(2024, i)
\end{aligned}
$$

The only cells that may contain monsters in either of these paths are $(3, i-1)$ and $(3, i+1)$. At most one of these can contain a monster, so at least one of the two paths will be successful.\\

If the monster in the second row is on the edge of the board, without loss of generality we may assume it is in $(2,1)$. Then, on the second attempt, Turbo takes the following path:

$$
(1,2) \rightarrow(2,2) \rightarrow(2,3) \rightarrow(3,3) \rightarrow \cdots \rightarrow(2022,2023) \rightarrow(2023,2023) \rightarrow(2024,2023)
$$


If there are no monsters on this path, then Turbo wins. Otherwise, let $(i, j)$ be the first cell on which Turbo encounters a monster. We have that $j=i$ or $j=i+1$. Then, on the third attempt, Turbo takes the following path:

$$
\begin{aligned}
(1,2) & \rightarrow(2,2) \rightarrow(2,3) \rightarrow(3,3) \rightarrow \cdots \rightarrow(i-2, i-1) \rightarrow(i-1, i-1) \\
& \rightarrow(i, i-1) \rightarrow(i, i-2) \rightarrow \cdots \rightarrow(i, 2) \rightarrow(i, 1) \\
& \rightarrow(i+1,1) \rightarrow \cdots \rightarrow(2023,1) \rightarrow(2024,1)
\end{aligned}
$$

Now note that

\begin{itemize}
  \item The cells from $(1,2)$ to $(i-1, i-1)$ do not contain monsters because they were reached earlier than $(i, j)$ on the previous attempt.
  \item The cells $(i, k)$ for $1 \leqslant k \leqslant i-1$ do not contain monsters because there is only one monster in row $i$, and it lies in $(i, i)$ or $(i, i+1)$.
  \item The cells $(k, 1)$ for $i \leqslant k \leqslant 2024$ do not contain monsters because there is at most one monster in column 1, and it lies in $(2,1)$.
\end{itemize}

Therefore Turbo will win on the third attempt.\\
Comment. A small variation on Turbo's strategy when the monster on the second row is on the edge is possible. On the second attempt, Turbo can instead take the path

$$
\begin{aligned}
(1,2023) & \rightarrow(2,2023) \rightarrow(2,2022) \rightarrow \cdots \rightarrow(2,3) \rightarrow(2,2) \rightarrow(2,3) \rightarrow \cdots \rightarrow(2,2023) \\
& \rightarrow(3,2023) \rightarrow(3,2022) \rightarrow \cdots \rightarrow(3,4) \rightarrow(3,3) \rightarrow(3,4) \rightarrow \cdots \rightarrow(3,2023) \\
& \rightarrow \cdots \\
& \rightarrow(2022,2023) \rightarrow(2022,2022) \rightarrow(2022,2023) \\
& \rightarrow(2023,2023) \\
& \rightarrow(2024,2023) .
\end{aligned}
$$

If there is a monster on this path, say in cell $(i, j)$, then on the third attempt Turbo can travel straight down to the cell just left of the monster instead of following the path traced out in the second attempt.

$$
\begin{aligned}
(1, j-1) & \rightarrow(2, j-1) \rightarrow \cdots \rightarrow(i-1, j-1) \rightarrow(i, j-1) \\
& \rightarrow(i, j-2) \rightarrow \cdots \rightarrow(i, 2) \rightarrow(i, 1) \\
& \rightarrow(i+1,1) \rightarrow \cdots \rightarrow(2023,1) \rightarrow(2024,1)
\end{aligned}
$$

\end{tcolorbox}


\begin{tcolorbox}[enhanced, breakable, rounded corners,
    colback=green!5!white, colframe=green!75!black,
    colbacktitle=green!85!black, fonttitle=\bfseries, coltitle=white, title=Problem 5 Solution Continued]

\setlength{\parskip}{1em}

\begin{center}
\includegraphics[width=0.7\textwidth]{problem_5_solution_img_3.png}
\end{center}


\end{tcolorbox}

\subsection*{2024 USAMO}
\label{appendix:B_Combinatorics_2024_USAMO}

\begin{tcolorbox}[enhanced, breakable, rounded corners,
    colback=blue!5!white, colframe=blue!75!black,
    colbacktitle=blue!85!black, fonttitle=\bfseries, coltitle=white, title=Problem 2]

\setlength{\parskip}{1em}
Let $S_1, S_2, \ldots, S_{100}$ be finite sets of integers whose intersection is not empty. For each non-empty $T \subseteq\left\{S_1, S_2, \ldots, S_{100}\right\}$, the size of the intersection of the sets in $T$ is a multiple of the number of sets in $T$. What is the least possible number of elements that are in at least 50 sets?
\end{tcolorbox}
\begin{tcolorbox}[enhanced, breakable, rounded corners, 
    colback=orange!5!white, colframe=orange!75!black,
    colbacktitle=orange!85!black, fonttitle=\bfseries, coltitle=white,
    title=Problem 2 Answer, width=\columnwidth]
The answer is \(50\binom{100}{50}\).
\end{tcolorbox}
\begin{tcolorbox}[enhanced, breakable, rounded corners,
    colback=green!5!white, colframe=green!75!black,
    colbacktitle=green!85!black, fonttitle=\bfseries, coltitle=white, title=Problem 2 Solution]
\setlength{\parskip}{1em}

Rephrasing: We encode with binary strings \(v \in \mathbb{F}_{2}^{100}\) of length 100 . Write \(v \subseteq w\) if \(w\) has 1's in every component \(v\) does, and let \(|v|\) denote the number of 1 's in \(v\).

Then for each \(v\), we let \(f(v)\) denote the number of elements \(x \in \bigcup S_{i}\) such that \(x \in S_{i} \Longleftrightarrow v_{i}=1\). For example,

\begin{itemize}
  \item \(f(1 \ldots 1)\) denotes \(\left|\bigcap_{1}^{100} S_{i}\right|\), so we know \(f(1 \ldots 1) \equiv 0(\bmod 100)\).
  \item \(f(1 \ldots 10)\) denotes the number of elements in \(S_{1}\) through \(S_{99}\) but not \(S_{100}\) so we know that \(f(1 \ldots 1)+f(1 \ldots 10) \equiv 0(\bmod 99)\).
  \item ...And so on.
\end{itemize}

So the problem condition means that \(f(v)\) translates to the statement

\[
P(u): \quad|u| \text { divides } \sum_{v \supseteq u} f(v)
\]

for any \(u \neq 0 \ldots 0\), plus one extra condition \(f(1 \ldots 1)>0\). And the objective function is to minimize the quantity

\[
A:=\sum_{|v| \geq 50} f(v)
\]

So the problem is transformed into an system of equations over \(\mathbb{Z}_{\geq 0}\) (it's clear any assignment of values of \(f(v)\) can be translated to a sequence ( \(S_{1}, \ldots, S_{100}\) ) in the original notation).

Note already that:\\

Claim. It suffices to assign \(f(v)\) for \(|v| \geq 50\).\\

Proof. If we have found a valid assignment of values to \(f(v)\) for \(|v| \geq 50\), then we can always arbitrarily assign values of \(f(v)\) for \(|v|<50\) by downwards induction on \(|v|\) to satisfy the divisibility condition (without changing \(M\) ).

Thus, for the rest of the solution, we altogether ignore \(f(v)\) for \(|v|<50\) and only consider \(P(u)\) for \(|u| \geq 50\).\\

Construction: Consider the construction

\[
f_{0}(v)=2|v|-100
\]

This construction is valid since if \(|u|=100-k\) for \(k \leq 50\) then

\[
\begin{aligned}
\sum_{v \supseteq u} f_{0}(v) & =\binom{k}{0} \cdot 100+\binom{k}{1} \cdot 98+\binom{k}{2} \cdot 96+\cdots+\binom{k}{k} \cdot(100-2 k) \\
& =(100-k) \cdot 2^{k}=|u| \cdot 2^{k}
\end{aligned}
\]

is indeed a multiple of \(|u|\), hence \(P(u)\) is true.

In that case, the objective function is

\[
A=\sum_{i=50}^{100}\binom{100}{i}(2 i-100)=50\binom{100}{50}
\]

as needed.\\

\begin{itshape}
Remark: This construction is the "easy" half of the problem because it coincides with what you get from a greedy algorithm by downwards induction on \(|u|\) (equivalently, induction on \(k=100-|u| \geq 0)\). To spell out the first three steps,

\begin{itemize}
  \item We know \(f(1 \ldots 1)\) is a nonzero multiple of 100 , so it makes sense to guess \(f(1 \ldots 1)=\) 100 .
  \item Then we have \(f(1 \ldots 10)+100 \equiv 0(\bmod 99)\), and the smallest multiple of 99 which is at least 100 is 198 . So it makes sense to guess \(f(1 \ldots 10)=98\), and similarly guess \(f(v)=98\) whenever \(|v|=99\).
  \item Now when we consider, say \(v=1 \ldots 100\) with \(|v|=98\), we get
\end{itemize}

\[
f(1 \ldots 100)+\underbrace{f(1 \ldots 101)}_{=98}+\underbrace{f(1 \ldots 110)}_{=98}+\underbrace{f(1 \ldots 111)}_{=100} \equiv 0 \quad(\bmod 98)
\]

we obtain \(f(1 \ldots 100) \equiv 96(\bmod 98)\). That makes \(f(1 \ldots 100)=96\) a reasonable guess.\\
Continuing in this way gives the construction above.
\end{itshape}

Proof of bound: We are going to use a smoothing argument: if we have a general working assignment \(f\), we will mold it into \(f_{0}\).

We define a push-down on \(v\) as the following operation:

\begin{itemize}
  \item Pick any \(v\) such that \(|v| \geq 50\) and \(f(v) \geq|v|\).
  \item Decrease \(f(v)\) by \(|v|\).
  \item For every \(w\) such that \(w \subseteq v\) and \(|w|=|v|-1\), increase \(f(w)\) by 1 .
\end{itemize}

Claim: Apply a push-down preserves the main divisibility condition. Moreover, it doesn't change \(A\) unless \(|v|=50\), where it decreases \(A\) by 50 instead.

Proof. The statement \(P(u)\) is only affected when \(u \subseteq v\) : to be precise, one term on the right-hand side of \(P(u)\) decreases by \(|v|\), while \(|v|-|u|\) terms increase by 1 , for a net change of \(-|u|\). So \(P(u)\) still holds.

To see \(A\) doesn't change for \(|v|>50\), note \(|v|\) terms increase by 1 while one term decreases by \(-|v|\). When \(|v|=50\), only \(f(v)\) decreases by 50 .

Now, given a valid assignment, we can modify it as follows:

\begin{itemize}
  \item First apply pushdowns on \(1 \ldots 1\) until \(f(1 \ldots 1)=100\);
  \item Then we may apply pushdowns on each \(v\) with \(|v|=99\) until \(f(v)<99\);
  \item Then we may apply pushdowns on each \(v\) with \(|v|=98\) until \(f(v)<98\);
  \item . . .and so on, until we have \(f(v)<50\) for \(|v|=50\).
\end{itemize}


Hence we get \(f(1 \ldots 1)=100\) and \(0 \leq f(v)<|v|\) for all \(50 \leq|v| \leq 100\). However, by downwards induction on \(|v|=99,98, \ldots, 50\), we also have

\[
f(v) \equiv f_{0}(v) \quad(\bmod |v|) \Longrightarrow f(v)=f_{0}(v)
\]

since \(f_{0}(v)\) and \(f(v)\) are both strictly less than \(|v|\). So in fact \(f=f_{0}\), and we're done.\\

\textbf{Remark.} The fact that push-downs actually don't change \(A\) shows that the equality case we described is far from unique: in fact, we could have made nearly arbitrary sub-optimal decisions during the greedy algorithm and still ended up with an equality case. For a concrete example, the construction

\[
f(v)= \begin{cases}500 & |v|=100 \\ 94 & |v|=99 \\ 100-2|v| & 50 \leq|v| \leq 98\end{cases}
\]

works fine as well (where we arbitrarily chose 500 at the start, then used the greedy algorithm thereafter).
\end{tcolorbox}

\begin{tcolorbox}[enhanced, breakable, rounded corners,
    colback=blue!5!white, colframe=blue!75!black,
    colbacktitle=blue!85!black, fonttitle=\bfseries, coltitle=white, title=Problem 4]
\setlength{\parskip}{1em}
Let $m$ and $n$ be positive integers. A circular necklace contains $m n$ beads, each either red or blue. It turned out that no matter how the necklace was cut into $m$ blocks of $n$ consecutive beads, each block had a distinct number of red beads. Determine, with proof, all possible values of the ordered pair $(m, n)$.
\end{tcolorbox}
\begin{tcolorbox}[enhanced, breakable, rounded corners, 
    colback=orange!5!white, colframe=orange!75!black,
    colbacktitle=orange!85!black, fonttitle=\bfseries, coltitle=white,
    title=Problem 4 Answer, width=\columnwidth]
The answer is \(m \leq n+1\) only.
\end{tcolorbox}
\begin{tcolorbox}[enhanced, breakable, rounded corners,
    colback=green!5!white, colframe=green!75!black,
    colbacktitle=green!85!black, fonttitle=\bfseries, coltitle=white, title=Problem 4 Solution]
\setlength{\parskip}{1em}
I Proof the task requires \(m \leq n+1\). Each of the \(m\) blocks has a red bead count between 0 and \(n\), each of which appears at most once, so \(m \leq n+1\) is needed.\\
\textbackslash  Construction when \(m=n+1\). For concreteness, here is the construction for \(n=4\), which obviously generalizes. The beads are listed in reading order as an array with \(n+1\) rows and \(n\) columns. Four of the blue beads have been labeled \(B_{1}, \ldots, B_{n}\) to make them easier to track as they move.

\[
T_{0}=\left[\begin{array}{llll}
R & R & R & R \\
R & R & R & B_{1} \\
R & R & B & B_{2} \\
R & B & B & B_{3} \\
B & B & B & B_{4}
\end{array}\right]
\]

To prove this construction works, it suffices to consider the \(n\) cuts \(T_{0}, T_{1}, T_{2}, \ldots, T_{n-1}\) made where \(T_{i}\) differs from \(T_{i-1}\) by having the cuts one bead later also have the property each row has a distinct red count:

\[
T_{1}=\left[\begin{array}{llll}
R & R & R & R \\
R & R & B_{1} & R \\
R & B & B_{2} & R \\
B & B & B_{3} & B \\
B & B & B_{4} & R
\end{array}\right] \quad T_{2}=\left[\begin{array}{llll}
R & R & R & R \\
R & B_{1} & R & R \\
B & B_{2} & R & B \\
B & B_{3} & B & B \\
B & B_{4} & R & R
\end{array}\right] \quad T_{3}=\left[\begin{array}{llll}
R & R & R & R \\
B_{1} & R & R & B \\
B_{2} & R & B & B \\
B_{3} & B & B & B \\
B_{4} & R & R & R
\end{array}\right]
\]

We can construct a table showing for each \(1 \leq k \leq n+1\) the number of red beads which are in the \((k+1)\) st row of \(T_{i}\) from the bottom:

\begin{center}
\begin{tabular}{c|cccc}
\(k\) & \(T_{0}\) & \(T_{1}\) & \(T_{2}\) & \(T_{3}\) \\
\hline
\(k=4\) & 4 & 4 & 4 & 4 \\
\(k=3\) & 3 & 3 & 3 & 2 \\
\(k=2\) & 2 & 2 & 1 & 1 \\
\(k=1\) & 1 & 0 & 0 & 0 \\
\(k=0\) & 0 & 1 & 2 & 3 \\
\end{tabular}
\end{center}.

This suggests following explicit formula for the entry of the \((i, k)\) th cell which can then be checked straightforwardly:

\[
\#\left(\text { red cells in } k \text { th row of } T_{i}\right)= \begin{cases}k & k>i \\ k-1 & i \geq k>0 \\ i & k=0\end{cases}
\]

And one can see for each \(i\), the counts are all distinct (they are ( \(i, 0,1, \ldots, k-1, k+1, \ldots, k)\) from bottom to top). This completes the construction.

Construction when \(m<n+1\). Fix \(m\). Take the construction for \((m, m-1)\) and add \(n+1-m\) cyan beads to the start of each row; for example, if \(n=7\) and \(m=5\) then the new construction is

\[
T=\left[\begin{array}{lllllll}
C & C & C & R & R & R & R \\
C & C & C & R & R & R & B_{1} \\
C & C & C & R & R & B & B_{2} \\
C & C & C & R & B & B & B_{3} \\
C & C & C & B & B & B & B_{4}
\end{array}\right] .
\]

This construction still works for the same reason (the cyan beads do nothing for the first \(n+1-m\) shifts, then one reduces to the previous case). If we treat cyan as a shade of blue, this finishes the problem.
\end{tcolorbox}

\subsection*{2023 IMO Shortlist}
\label{appendix:B_Combinatorics_2023_IMO_Shortlist}
\begin{tcolorbox}[enhanced, breakable, rounded corners,
    colback=blue!5!white, colframe=blue!75!black,
    colbacktitle=blue!85!black, fonttitle=\bfseries, coltitle=white, title=Problem 1]
\setlength{\parskip}{1em}
Let $m$ and $n$ be positive integers greater than 1. In each unit square of an $m \times n$ grid lies a coin with its tail-side up. A move consists of the following steps:
\begin{enumerate}
  \item select a $2 \times 2$ square in the grid;
  \item flip the coins in the top-left and bottom-right unit squares;
  \item flip the coin in either the top-right or bottom-left unit square.
\end{enumerate}
Determine all pairs $(m, n)$ for which it is possible that every coin shows head-side up after a finite number of moves.
\end{tcolorbox}

\begin{tcolorbox}[enhanced, breakable, rounded corners, 
    colback=orange!5!white, colframe=orange!75!black,
    colbacktitle=orange!85!black, fonttitle=\bfseries, coltitle=white,
    title=Problem 1 Answer, width=\columnwidth]
 The answer is all pairs $(m, n)$ satisfying $3 \mid m n$.
\end{tcolorbox}

\begin{tcolorbox}[enhanced, breakable, rounded corners,
    colback=green!5!white, colframe=green!75!black,
    colbacktitle=green!85!black, fonttitle=\bfseries, coltitle=white, title=Problem 1 Solution]
\setlength{\parskip}{1em}
Let us denote by $(i, j)$-square the unit square in the $i^{\text {th }}$ row and the $j^{\text {th }}$ column.\\
We first prove that when $3 \mid m n$, it is possible to make all the coins show head-side up. For integers $1 \leqslant i \leqslant m-1$ and $1 \leqslant j \leqslant n-1$, denote by $A(i, j)$ the move that flips the coin in the $(i, j)$-square, the $(i+1, j+1)$-square and the $(i, j+1)$-square. Similarly, denote by $B(i, j)$ the move that flips the coin in the $(i, j)$-square, $(i+1, j+1)$-square, and the $(i+1, j)$-square. Without loss of generality, we may assume that $3 \mid \mathrm{m}$.
Case 1: $n$ is even.\\
We apply the moves
\begin{itemize}
  \item $A(3 k-2,2 l-1)$ for all $1 \leqslant k \leqslant \frac{m}{3}$ and $1 \leqslant l \leqslant \frac{n}{2}$,
  \item $B(3 k-1,2 l-1)$ for all $1 \leqslant k \leqslant \frac{m}{3}$ and $1 \leqslant l \leqslant \frac{n}{2}$.
\end{itemize}
This process will flip each coin exactly once, hence all the coins will face head-side up afterwards.
\begin{center}
        \includegraphics[width=0.3\linewidth]{SLC1solution.png}
    \end{center}
Case 2: $n$ is odd.\\
We start by applying
\begin{itemize}
  \item $A(3 k-2,2 l-1)$ for all $1 \leqslant k \leqslant \frac{m}{3}$ and $1 \leqslant l \leqslant \frac{n-1}{2}$,
  \item $B(3 k-1,2 l-1)$ for all $1 \leqslant k \leqslant \frac{m}{3}$ and $1 \leqslant l \leqslant \frac{n-1}{2}$\\
as in the previous case. At this point, the coins on the rightmost column have tail-side up and the rest of the coins have head-side up. We now apply the moves
  \item $A(3 k-2, n-1), A(3 k-1, n-1)$ and $B(3 k-2, n-1)$ for every $1 \leqslant k \leqslant \frac{m}{3}$.
\end{itemize}
For each $k$, the three moves flip precisely the coins in the $(3 k-2, n)$-square, the $(3 k-1, n)$ square, and the $(3 k, n)$-square. Hence after this process, every coin will face head-side up.
We next prove that $m n$ being divisible by 3 is a necessary condition. We first label the $(i, j)$-square by the remainder of $i+j-2$ when divided by 3 , as shown in the figure.
\begin{center}
\begin{tabular}{|c|c|c|c|c|}
\hline
0 & 1 & 2 & 0 & $\cdots$ \\
\hline
1 & 2 & 0 & 1 & $\cdots$ \\
\hline
2 & 0 & 1 & 2 & $\cdots$ \\
\hline
0 & 1 & 2 & 0 & $\cdots$ \\
\hline
$\vdots$ & $\vdots$ & $\vdots$ & $\vdots$ & $\ddots$ \\
\hline
\end{tabular}
\end{center}
Let $T(c)$ be the number of coins facing head-side up in those squares whose label is $c$. The main observation is that each move does not change the parity of both $T(0)-T(1)$ and $T(1)-T(2)$, since a move flips exactly one coin in a square with each label. Initially, all coins face tail-side up at the beginning, thus all of $T(0), T(1), T(2)$ are equal to 0 . Hence it follows that any configuration that can be achieved from the initial state must satisfy the parity condition of
$$
T(0) \equiv T(1) \equiv T(2) \quad(\bmod 2)
$$
We now calculate the values of $T$ for the configuration in which all coins are facing head-side up.
\begin{itemize}
  \item When $m \equiv n \equiv 1(\bmod 3)$, we have $T(0)-1=T(1)=T(2)=\frac{m n-1}{3}$.
  \item When $m \equiv 1(\bmod 3)$ and $n \equiv 2(\bmod 3)$, or $m \equiv 2(\bmod 3)$ and $n \equiv 1(\bmod 3)$, we have $T(0)-1=T(1)-1=T(2)=\frac{m n-2}{3}$.
  \item When $m \equiv n \equiv 2(\bmod 3)$, we have $T(0)=T(1)-1=T(2)=\frac{m n-1}{3}$.
  \item When $m \equiv 0(\bmod 3)$ or $n \equiv 0(\bmod 3)$, we have $T(0)=T(1)=T(2)=\frac{m n}{3}$.
\end{itemize}
From this calculation, we see that $T(0), T(1)$ and $T(2)$ has the same parity only when $m n$ is divisible by 3 .
Comment 1. The original proposal of the problem also included the following question as part (b):\\
For each pair $(m, n)$ of integers greater than 1 , how many configurations can be obtained by applying a finite number of moves?\\
An explicit construction of a sequence of moves shows that $T(0), T(1)$, and $T(2)$ having the same parity is a necessary and sufficient condition for a configuration to obtainable after a finite sequence of moves, and this shows that the answer is $2^{m n-2}$.
Comment 2. A significantly more difficult problem is to ask the following question: for pairs ( $m, n$ ) such that the task is possible (i.e. $3 \mid m n$ ), what is the smallest number of moves required to complete this task? The answer is:
\begin{itemize}
  \item $\frac{m n}{3}$ if $m n$ is even;
  \item $\frac{m n}{3}+2$ if $m n$ is odd.
\end{itemize}
To show this, we observe that we can flip all coins in any $2 \times 3$ (or $3 \times 2$ ) by using a minimum of two moves. Furthermore, when $m n$ is odd with $3 \mid m n$, it is impossible to tile an $m \times n$ table with one type of L-tromino and its $180^{\circ}$-rotated L-tromino (disallowing rotations and reflections). The only known proof of the latter claim is lengthy and difficult, and it requires some group-theoretic arguments by studying the title homotopy group given by these two L-tromino tiles. This technique was developed by J. H. Conway and J. C. Lagarias in Tiling with Polyominoes and Combinatorial Group Theory, Journal of Combinatorial Group Theory, Series A 53, 183-208 (1990).
Comment 3. Here is neat way of defining the invariant. Consider a finite field $\mathbb{F}_{4}=\{0,1, \omega, \omega+1\}$, where $1+1=\omega^{2}+\omega+1=0$ in $\mathbb{F}_{4}$. Consider the set
$$
H=\{(i, j) \mid 1 \leqslant i \leqslant m, 1 \leqslant j \leqslant n \text {, the coin in the }(i, j) \text {-square is head-side up }\}
$$
and the invariant
$$
I(H)=\sum_{(i, j) \in H} \omega^{i+j} \in \mathbb{F}_{4}
$$
Then the value of $I(H)$ does not change under applying moves, and when all coins are tail-side up, it holds that $I(H)=0$. On the other hand, its value when all coins are head-side up can be computed as
$$
I(H)=\sum_{i=1}^{m} \sum_{j=1}^{n} \omega^{i+j}=\left(\sum_{i=1}^{m} \omega^{i}\right)\left(\sum_{j=1}^{n} \omega^{j}\right)
$$
This is equal to $0 \in \mathbb{F}_{4}$ if and only if $3 \mid m n$.

\end{tcolorbox}


\begin{tcolorbox}[enhanced, breakable, rounded corners,
    colback=blue!5!white, colframe=blue!75!black,
    colbacktitle=blue!85!black, fonttitle=\bfseries, coltitle=white, title=Problem 2]
\setlength{\parskip}{1em}
Determine the maximal length $L$ of a sequence $a_{1}, \ldots, a_{L}$ of positive integers satisfying both the following properties:
\begin{itemize}
  \item every term in the sequence is less than or equal to $2^{2023}$, and
  \item there does not exist a consecutive subsequence $a_{i}, a_{i+1}, \ldots, a_{j}$ (where $1 \leqslant i \leqslant j \leqslant L$ ) with a choice of signs $s_{i}, s_{i+1}, \ldots, s_{j} \in\{1,-1\}$ for which
\end{itemize}
$$
s_{i} a_{i}+s_{i+1} a_{i+1}+\cdots+s_{j} a_{j}=0
$$
\end{tcolorbox}

\begin{tcolorbox}[enhanced, breakable, rounded corners, 
    colback=orange!5!white, colframe=orange!75!black,
    colbacktitle=orange!85!black, fonttitle=\bfseries, coltitle=white,
    title=Problem 2 Answer, width=\columnwidth]
 The answer is $L=2^{2024}-1$.
\end{tcolorbox}

\begin{tcolorbox}[enhanced, breakable, rounded corners,
    colback=green!5!white, colframe=green!75!black,
    colbacktitle=green!85!black, fonttitle=\bfseries, coltitle=white, title=Problem 2 Solution]
\setlength{\parskip}{1em}
We prove more generally that the answer is $2^{k+1}-1$ when $2^{2023}$ is replaced by $2^{k}$ for an arbitrary positive integer $k$. Write $n=2^{k}$.
We first show that there exists a sequence of length $L=2 n-1$ satisfying the properties. For a positive integer $x$, denote by $v_{2}(x)$ the maximal nonnegative integer $v$ such that $2^{v}$ divides $x$. Consider the sequence $a_{1}, \ldots, a_{2 n-1}$ defined as
$$
a_{i}=2^{k-v_{2}(i)} .
$$
For example, when $k=2$ and $n=4$, the sequence is
$$
4,2,4,1,4,2,4
$$
This indeed consists of positive integers less than or equal to $n=2^{k}$, because $0 \leqslant v_{2}(i) \leqslant k$ for $1 \leqslant i \leqslant 2^{k+1}-1$.\\
Claim 1. This sequence $a_{1}, \ldots, a_{2 n-1}$ does not have a consecutive subsequence with a choice of signs such that the signed sum equals zero.\\
Proof. Let $1 \leqslant i \leqslant j \leqslant 2 n-1$ be integers. The main observation is that amongst the integers
$$
i, i+1, \ldots, j-1, j
$$
there exists a unique integer $x$ with the maximal value of $v_{2}(x)$. To see this, write $v=$ $\max \left(v_{2}(i), \ldots, v_{2}(j)\right)$. If there exist at least two multiples of $2^{v}$ amongst $i, i+1, \ldots, j$, then one of them must be a multiple of $2^{v+1}$, which is a contradiction.
Therefore there is exactly one $i \leqslant x \leqslant j$ with $v_{2}(x)=v$, which implies that all terms except for $a_{x}=2^{k-v}$ in the sequence
$$
a_{i}, a_{i+1}, \ldots, a_{j}
$$
are a multiple of $2^{k-v+1}$. The same holds for the terms $s_{i} a_{i}, s_{i+1} a_{i+1}, \ldots, s_{j} a_{j}$, hence the sum cannot be equal to zero.
We now prove that there does not exist a sequence of length $L \geqslant 2 n$ satisfying the conditions of the problem. Let $a_{1}, \ldots, a_{L}$ be an arbitrary sequence consisting of positive integers less than or equal to $n$. Define a sequence $s_{1}, \ldots, s_{L}$ of signs recursively as follows:
\begin{itemize}
  \item when $s_{1} a_{1}+\cdots+s_{i-1} a_{i-1} \leqslant 0$, set $s_{i}=+1$,
  \item when $s_{1} a_{1}+\cdots+s_{i-1} a_{i-1} \geqslant 1$, set $s_{i}=-1$.
\end{itemize}
Write
$$
b_{i}=\sum_{j=1}^{i} s_{i} a_{i}=s_{1} a_{1}+\cdots+s_{i} a_{i}
$$
and consider the sequence
$$
0=b_{0}, b_{1}, b_{2}, \ldots, b_{L}
$$
Claim 2. All terms $b_{i}$ of the sequence satisfy $-n+1 \leqslant b_{i} \leqslant n$.\\
Proof. We prove this by induction on $i$. It is clear that $b_{0}=0$ satisfies $-n+1 \leqslant 0 \leqslant n$. We now assume $-n+1 \leqslant b_{i-1} \leqslant n$ and show that $-n+1 \leqslant b_{i} \leqslant n$.
Case 1: $-n+1 \leqslant b_{i-1} \leqslant 0$.\\
Then $b_{i}=b_{i-1}+a_{i}$ from the definition of $s_{i}$, and hence
$$
-n+1 \leqslant b_{i-1}<b_{i-1}+a_{i} \leqslant b_{i-1}+n \leqslant n .
$$
Case 2: $1 \leqslant b_{i-1} \leqslant n$.\\
Then $b_{i}=b_{i-1}-a_{i}$ from the definition of $s_{i}$, and hence
$$
-n+1 \leqslant b_{i-1}-n \leqslant b_{i-1}-a_{i}<b_{i-1} \leqslant n
$$
This finishes the proof.\\
Because there are $2 n$ integers in the closed interval $[-n+1, n]$ and at least $2 n+1$ terms in the sequence $b_{0}, b_{1}, \ldots, b_{L}$ (as $L+1 \geqslant 2 n+1$ by assumption), the pigeonhole principle implies that two distinct terms $b_{i-1}, b_{j}$ (where $1 \leqslant i \leqslant j \leqslant L$ ) must be equal. Subtracting one from another, we obtain
$$
s_{i} a_{i}+\cdots+s_{j} a_{j}=b_{j}-b_{i-1}=0
$$
as desired.\\
Comment. The same argument gives a bound $L \leqslant 2 n-1$ that works for all $n$, but this bound is not necessarily sharp when $n$ is not a power of 2 . For instance, when $n=3$, the longest sequence has length $L=3$.

\end{tcolorbox}


\begin{tcolorbox}[enhanced, breakable, rounded corners,
    colback=blue!5!white, colframe=blue!75!black,
    colbacktitle=blue!85!black, fonttitle=\bfseries, coltitle=white, title=Problem 3]
\setlength{\parskip}{1em}
Let $n$ be a positive integer. We arrange $1+2+\cdots+n$ circles in a triangle with $n$ rows, such that the $i^{\text {th }}$ row contains exactly $i$ circles. The following figure shows the case $n=6$.
\begin{center}
        \includegraphics[width=0.3\linewidth]{SLC3Problem.png}
    \end{center}
In this triangle, a ninja-path is a sequence of circles obtained by repeatedly going from a circle to one of the two circles directly below it. In terms of $n$, find the largest value of $k$ such that if one circle from every row is coloured red, we can always find a ninja-path in which at least $k$ of the circles are red.
\end{tcolorbox}

\begin{tcolorbox}[enhanced, breakable, rounded corners, 
    colback=orange!5!white, colframe=orange!75!black,
    colbacktitle=orange!85!black, fonttitle=\bfseries, coltitle=white,
    title=Problem 3 Answer, width=\columnwidth]
 The maximum value is $k=1+\left\lfloor\log _{2} n\right\rfloor$.
\end{tcolorbox}

\begin{tcolorbox}[enhanced, breakable, rounded corners,
    colback=green!5!white, colframe=green!75!black,
    colbacktitle=green!85!black, fonttitle=\bfseries, coltitle=white, title=Problem 3 Solution]
\setlength{\parskip}{1em}
Write $N=\left\lfloor\log _{2} n\right\rfloor$ so that we have $2^{N} \leqslant n \leqslant 2^{N+1}-1$.\\
We first provide a construction where every ninja-path passes through at most $N+1$ red circles. For the row $i=2^{a}+b$ for $0 \leqslant a \leqslant N$ and $0 \leqslant b<2^{a}$, we colour the $(2 b+1)^{\text {th }}$ circle.
\begin{center}
        \includegraphics[width=0.3\linewidth]{SLC3solution-1.png}
    \end{center}
Then every ninja-path passes through at most one red circle in each of the rows $2^{a}, 2^{a}+$ $1, \ldots, 2^{a+1}-1$ for each $0 \leqslant a \leqslant N$. It follows that every ninja-path passes through at most $N+1$ red circles.
We now prove that for every colouring, there exists a ninja-path going through at least $N+1$ red circles. For each circle $C$, we assign the maximum number of red circles in a ninja-path that starts at the top of the triangle and ends at $C$.
\begin{center}
        \includegraphics[width=0.3\linewidth]{SLC3solution-2.png}
    \end{center}
Note that
\begin{itemize}
  \item if $C$ is not red, then the number assigned to $C$ is the maximum of the number assigned to the one or two circles above $C$, and
  \item if $C$ is red, then the number assigned to $C$ is one plus the above maximum.
\end{itemize}
Write $v_{1}, \ldots, v_{i}$ for the numbers in row $i$, and let $v_{m}$ be the maximum among these numbers. Then the numbers in row $i+1$ will be at least
$$
v_{1}, \ldots, v_{m-1}, v_{m}, v_{m}, v_{m+1}, \ldots, v_{i}
$$
not taking into account the fact that one of the circles in row $i+1$ is red. On the other hand, for the red circle in row $i+1$, the lower bound on the assigned number can be increased by 1 . Therefore the sum of the numbers in row $i+1$ is at least
$$
\left(v_{1}+\cdots+v_{i}\right)+v_{m}+1
$$
Using this observation, we prove the following claim.\\
Claim 1. Let $\sigma_{k}$ be the sum of the numbers assigned to circles in row $k$. Then for $0 \leqslant j \leqslant N$, we have $\sigma_{2^{j}} \geqslant j \cdot 2^{j}+1$.\\
Proof. We use induction on $j$. This is clear for $j=0$, since the number in the first row is always 1. For the induction step, suppose that $\sigma_{2 j} \geqslant j \cdot 2^{j}+1$. Then the maximum value assigned to a circle in row $2^{j}$ is at least $j+1$. As a consequence, for every $k \geqslant 2^{j}$, there is a circle on row $k$ with number at least $j+1$. Then by our observation above, we have
$$
\sigma_{k+1} \geqslant \sigma_{k}+(j+1)+1=\sigma_{k}+(j+2)
$$
Then we get
$$
\sigma_{2^{j+1}} \geqslant \sigma_{2^{j}}+2^{j}(j+2) \geqslant j \cdot 2^{j}+1+2^{j}(j+2)=(j+j+2) 2^{j}+1=(j+1) 2^{j+1}+1
$$
This completes the inductive step.\\
For $j=N$, this immediately implies that some circle in row $2^{N}$ has number at least $N+1$. This shows that there is a ninja-path passing through at least $N+1$ red circles.
Solution 2. We give an alternative proof that there exists a ninja-path passing through at least $N+1$ red circles. Assign numbers to circles as in the previous solution, but we only focus on the numbers assigned to red circles.
For each positive integer $i$, denote by $e_{i}$ the number of red circles with number $i$.\\
Claim 2. If the red circle on row $l$ has number $i$, then $e_{i} \leqslant l$.\\
Proof. Note that if two circles $C$ and $C^{\prime}$ are both assigned the same number $i$, then there cannot be a ninja-path joining the two circles. We partition the triangle into a smaller triangle with the red circle in row $l$ at its top along with $l-1$ lines that together cover all other circles.
\begin{center}
        \includegraphics[width=0.3\linewidth]{SLC3solution-3.png}
    \end{center}
In each set, there can be at most one red circle with number $i$, and therefore $e_{i} \leqslant l$.\\
We observe that if there exists a red circle $C$ with number $i \geqslant 2$, then there also exists a red circle with number $i-1$ in some row that is above the row containing $C$. This is because the second last red circle in the ninja-path ending at $C$ has number $i-1$.\\
Claim 3. We have $e_{i} \leqslant 2^{i-1}$ for every positive integer $i$.
Proof. We prove by induction on $i$. The base case $i=1$ is clear, since the only red circle with number 1 is the one at the top of the triangle. We now assume that the statement is true for $1 \leqslant i \leqslant j-1$ and prove the statement for $i=j$. If $e_{j}=0$, there is nothing to prove. Otherwise, let $l$ be minimal such that the red circle on row $l$ has number $j$. Then all the red circles on row $1, \ldots, l-1$ must have number less than $j$. This shows that
$$
l-1 \leqslant e_{1}+e_{2}+\cdots+e_{j-1} \leqslant 1+2+\cdots+2^{j-2}=2^{j-1}-1
$$
This proves that $l \leqslant 2^{j-1}$, and by Claim 2 , we also have $e_{j} \leqslant l$. Therefore $e_{j} \leqslant 2^{j-1}$.\\
We now see that
$$
e_{1}+e_{2}+\cdots+e_{N} \leqslant 1+\cdots+2^{N-1}=2^{N}-1<n
$$
Therefore there exists a red circle with number at least $N+1$, which means that there exists a ninja-path passing through at least $N+1$ red circles.
Solution 3. We provide yet another proof that there exists a ninja-path passing through at least $N+1$ red circles. In this solution, we assign to a circle $C$ the maximum number of red circles on a ninja-path starting at $C$ (including $C$ itself).
\begin{center}
        \includegraphics[width=0.3\linewidth]{SLC3solution-4.png}
    \end{center}
Denote by $f_{i}$ the number of red circles with number $i$. Note that if a red circle $C$ has number $i$, and there is a ninja-path from $C$ to another red circle $C^{\prime}$, then the number assigned to $C^{\prime}$ must be less than $i$.\\
Claim 4. If the red circle on row $l$ has number less than or equal to $i$, then $f_{i} \leqslant l$.\\
Proof. This proof is same as the proof of Claim 2. The additional input is that if the red circle on row $l$ has number strictly less than $i$, then the smaller triangle cannot have a red circle with number $i$.
Claim 5. We have
$$
f_{1}+f_{2}+\cdots+f_{i} \leqslant n-\left\lfloor\frac{n}{2^{i}}\right\rfloor
$$
for all $0 \leqslant i \leqslant N$.\\
Proof. We use induction on $i$. The base case $i=0$ is clear as the left hand side is the empty sum and the right hand side is zero. For the induction step, we assume that $i \geqslant 1$ and that the statement is true for $i-1$. Let $l$ be minimal such that the red circle on row $l$ has number less than or equal to $i$. Then all the red circles with number less than or equal to $i$ lie on rows $l, l+1, \ldots, n$, and therefore
$$
f_{1}+f_{2}+\cdots+f_{i} \leqslant n-l+1
$$
On the other hand, the induction hypothesis together with the fact that $f_{i} \leqslant l$ shows that
$$
f_{1}+\cdots+f_{i-1}+f_{i} \leqslant n-\left\lfloor\frac{n}{2^{i-1}}\right\rfloor+l
$$
Averaging the two inequalities gives
$$
f_{1}+\cdots+f_{i} \leqslant n-\frac{1}{2}\left\lfloor\frac{n}{2^{i-1}}\right\rfloor+\frac{1}{2}
$$
Since the left hand side is an integer, we conclude that
$$
f_{1}+\cdots+f_{i} \leqslant n-\left\lfloor\frac{1}{2}\left\lfloor\frac{n}{2^{i-1}}\right\rfloor\right\rfloor=n-\left\lfloor\frac{n}{2^{i}}\right\rfloor
$$
This completes the induction step.\\
Taking $i=N$, we obtain
$$
f_{1}+f_{2}+\cdots+f_{N} \leqslant n-\left\lfloor\frac{n}{2^{N}}\right\rfloor<n
$$
This implies that there exists a ninja-path passing through at least $N+1$ red circles.\\
Comment. Using essentially the same argument, one may inductively prove
$$
e_{a}+e_{a+1}+\cdots+e_{a+i-1} \leqslant n-\left\lfloor\frac{n}{2^{i}}\right\rfloor
$$
instead. Taking $a=1$ and $i=N$ gives the desired statement.

\end{tcolorbox}


\begin{tcolorbox}[enhanced, breakable, rounded corners,
    colback=blue!5!white, colframe=blue!75!black,
    colbacktitle=blue!85!black, fonttitle=\bfseries, coltitle=white, title=Problem 4]
\setlength{\parskip}{1em}
Let $n \geqslant 2$ be a positive integer. Paul has a $1 \times n^{2}$ rectangular strip consisting of $n^{2}$ unit squares, where the $i^{\text {th }}$ square is labelled with $i$ for all $1 \leqslant i \leqslant n^{2}$. He wishes to cut the strip into several pieces, where each piece consists of a number of consecutive unit squares, and then translate (without rotating or flipping) the pieces to obtain an $n \times n$ square satisfying the following property: if the unit square in the $i^{\text {th }}$ row and $j^{\text {th }}$ column is labelled with $a_{i j}$, then $a_{i j}-(i+j-1)$ is divisible by $n$.
Determine the smallest number of pieces Paul needs to make in order to accomplish this.
\end{tcolorbox}

\begin{tcolorbox}[enhanced, breakable, rounded corners, 
    colback=orange!5!white, colframe=orange!75!black,
    colbacktitle=orange!85!black, fonttitle=\bfseries, coltitle=white,
    title=Problem 4 Answer, width=\columnwidth]
The minimum number of pieces is $2 n-1$.
\end{tcolorbox}

\begin{tcolorbox}[enhanced, breakable, rounded corners,
    colback=green!5!white, colframe=green!75!black,
    colbacktitle=green!85!black, fonttitle=\bfseries, coltitle=white, title=Problem 4 Solution 1]
\setlength{\parskip}{1em}
 1. For the entirety of the solution, we shall view the labels as taking values in $\mathbb{Z} / n \mathbb{Z}$, as only their values modulo $n$ play a role.
Here are two possible constructions consisting of $2 n-1$ pieces.
\begin{enumerate}
  \item Cut into pieces of sizes $n, 1, n, 1, \ldots, n, 1,1$, and glue the pieces of size 1 to obtain the last row.
  \item Cut into pieces of sizes $n, 1, n-1,2, n-2, \ldots, n-1,1$, and switch the pairs of consecutive strips that add up to size $n$.
\end{enumerate}
We now prove that using $2 n-1$ pieces is optimal. It will be more helpful to think of the reverse process: start with $n$ pieces of size $1 \times n$, where the $k^{\text {th }}$ piece has squares labelled $k, k+1, \ldots, k+n-1$. The goal is to restore the original $1 \times n^{2}$ strip.
Note that each piece, after cutting at appropriate places, is of the form $a, a+1, \ldots, b-1$. Construct an (undirected but not necessarily simple) graph $\Gamma$ with vertices labelled by $1, \ldots, n$, where a piece of the form $a, a+1, \ldots, b-1$ corresponds to an edge from $a$ to $b$. We make the following observations.
\begin{itemize}
  \item The cut pieces came from the $k^{\text {th }}$ initial piece $k, k+1, \ldots, k+n-1$ corresponds to a cycle $\gamma_{k}$ (possibly of length 1 ) containing the vertex $k$.
  \item Since it is possible to rearrange the pieces into one single $1 \times n^{2}$ strip, the graph $\Gamma$ has an Eulerian cycle.
  \item The number of edges of $\Gamma$ is equal to the total number of cut pieces.
\end{itemize}
The goal is to prove that $\Gamma$ has at least $2 n-1$ edges. Since $\Gamma$ has an Eulerian cycle, it is connected. For every $1 \leqslant k \leqslant n$, pick one edge from $\gamma_{k}$, delete it from $\Gamma$ to obtain a new graph $\Gamma^{\prime}$. Since no two cycles $\gamma_{i}$ and $\gamma_{j}$ share a common edge, removing one edge from each cycle does not affect the connectivity of the graph. This shows that the new graph $\Gamma^{\prime}$ must also be connected. Therefore $\Gamma^{\prime}$ has at least $n-1$ edges, which means that $\Gamma$ has at least $2 n-1$ edges.
\end{tcolorbox}

\begin{tcolorbox}[enhanced, breakable, rounded corners,
    colback=green!5!white, colframe=green!75!black,
    colbacktitle=green!85!black, fonttitle=\bfseries, coltitle=white, title=Problem 4 Solution 2]
\setlength{\parskip}{1em}
We provide an alternative proof that at least $2 n-1$ pieces are needed. Instead of having a linear strip, we work with a number of circular strips, each having length a multiple of $n$ and labelled as
$$
1,2, \ldots, n, 1,2, \ldots, n, \ldots, 1,2, \ldots, n
$$
where there are $n^{2}$ cells in total across all circular strips. The goal is still to create the $n \times n$ square by cutting and translating. Here, when we say "translating" the strips, we imagine that each cell has a number written on it and the final $n \times n$ square is required to have every number written in the same upright, non-mirrored orientation.
Note that the number of cuts will be equal to the number of pieces, because performing $l \geqslant 1$ cuts on a single circular strip results in $l$ pieces.
Consider any "seam" in the interior of the final square, between two squares $S$ and $T$, so that $S$ and $T$ belongs to two separate pieces. We are interested in the positions of these two squares in the original circular strips, with the aim of removing the seam.
\begin{itemize}
  \item If the two squares $S$ and $T$ come from the same circular strip and are adjacent, then the cut was unnecessary and we can simply remove the seam and reduce the number of required cuts by 1 . The circular strips are not affected.
  \item If these two squares $S$ and $T$ were not adjacent, then they are next to two different cuts (either from the same circular strip or two different circular strips). Denote the two cuts by $(S \mid Y)$ and $(X \mid T)$. We perform these two cuts and then glue the pieces back according to $(S \mid T)$ and $(X \mid Y)$. Performing this move would either split one circular strip into two or merge two circular strips into one, changing the number of circular strips by at most one. Afterwards, we may eliminate cut $(S \mid T)$ since it is no longer needed, which also removes the corresponding seam from the final square.
\end{itemize}
By iterating this process, eventually we reach a state where there are some number of circular strips, but the final $n \times n$ square no longer has any interior seams.
Since no two rows of the square can be glued together while maintaining the consecutive numbering, the only possibility is to have exactly $n$ circular strips, each with length $n$. In this state at least $n$ cuts are required to reassemble the square. Recall that each seam removal operation changed the number of circular strips by at most one. So if we started with only one initial circular strip, then at least $n-1$ seams were removed. Hence in total, at least $n+(n-1)=2 n-1$ cuts are required to transform one initial circular strip into the final square. Hence at least $2 n-1$ pieces are required to achieve the desired outcome.
\end{tcolorbox}

\begin{tcolorbox}[enhanced, breakable, rounded corners,
    colback=green!5!white, colframe=green!75!black,
    colbacktitle=green!85!black, fonttitle=\bfseries, coltitle=white, title=Problem 4 Solution 3]
\setlength{\parskip}{1em}
As with the previous solution, we again work with circular strips. In particular, we start out with $k$ circular strips, each having length a multiple of $n$ and labelled as
$$
1,2, \ldots, n, 1,2, \ldots, n, \ldots, 1,2, \ldots, n
$$
where there are $n^{2}$ cells in total across all $k$ circular strips. The goal is still to create the $n \times n$ square by cutting and translating the circular strips.\\
Claim. Constructing the $n \times n$ square requires at least $2 n-k$ cuts (or alternatively, $2 n-k$ pieces).\\
Proof. We prove by induction on $n$. The base case $n=1$ is clear, because we can only have $k=1$ and the only way of producing a $1 \times 1$ square from a $1 \times 1$ circular strip is by making a single cut. We now assume that $n \geqslant 2$ and the statement is true for $n-1$.
Each cut is a cut between a cell of label $i$ on the left and a cell of label $i+1$ on the right side, for a unique $1 \leqslant i \leqslant n$. Let $a_{i}$ be the number of such cuts, so that $a_{1}+a_{2}+\cdots+a_{n}$ is the total number of cuts. Since all the left and right edges of the $n \times n$ square at the end must be cut, we have $a_{i} \geqslant 1$ for all $1 \leqslant i \leqslant n$.
If $a_{i} \geqslant 2$ for all $i$, then
$$
a_{1}+a_{2}+\cdots+a_{n} \geqslant 2 n>2 n-k
$$
and hence there is nothing to prove. We therefore assume that there exist some $1 \leqslant m \leqslant n$ for which $a_{m}=1$. This unique cut must form the two ends of the linear strip
$$
m+1, m+2, \ldots, m-1+n, m+n
$$
from the final product. There are two cases.\\
Case 1: The strip is a single connected piece.
In this case, the strip must have come from a single circular strip of length exactly $n$. We now remove this circular strip from of the cutting and pasting process. By definition of $m$, none of the edges between $m$ and $m+1$ are cut. Therefore we may pretend that all the adjacent pairs of cells labelled $m$ and $m+1$ are single cells. The induction hypothesis then implies that
$$
a_{1}+\cdots+a_{m-1}+a_{m+1}+\cdots+a_{n} \geqslant 2(n-1)-(k-1)
$$
Adding back in $a_{m}$, we obtain
$$
a_{1}+\cdots+a_{n} \geqslant 2(n-1)-(k-1)+1=2 n-k
$$
Case 2: The strip is not a single connected piece.\\
Say the linear strip $m+1, \ldots, m+n$ is composed of $l \geqslant 2$ pieces $C_{1}, \ldots, C_{l}$. We claim that if we cut the initial circular strips along both the left and right end points of the pieces $C_{1}, \ldots, C_{l}$, and then remove them, the remaining part consists of at most $k+l-2$ connected pieces (where some of them may be circular and some of them may be linear). This is because $C_{l}, C_{1}$ form a consecutive block of cells on the circular strip, and removing $l-1$ consecutive blocks from $k$ circular strips results in at most $k+(l-1)-1$ connected pieces.
Once we have the connected pieces that form the complement of $C_{1}, \ldots, C_{l}$, we may glue them back at appropriate endpoints to form circular strips. Say we get $k^{\prime}$ circular strips after this procedure. As we are gluing back from at most $k+l-2$ connected pieces, we see that
$$
k^{\prime} \leqslant k+l-2
$$
We again observe that to get from the new circular strips to the $n-1$ strips of size $1 \times n$, we never have to cut along the cell boundary between labels $m$ and $m+1$. Therefore the induction hypothesis applies, and we conclude that the total number of pieces is bounded below by
$$
l+\left(2(n-1)-k^{\prime}\right) \geqslant l+2(n-1)-(k+l-2)=2 n-k
$$
This finishes the induction step, and therefore the statement holds for all $n$.\\
Taking $k=1$ in the claim, we see that to obtain a $n \times n$ square from a circular $1 \times n^{2}$ strip, we need at least $2 n-1$ connected pieces. This shows that constructing the $n \times n$ square out of a linear $1 \times n^{2}$ strip also requires at least $2 n-1$ pieces.

\end{tcolorbox}


\begin{tcolorbox}[enhanced, breakable, rounded corners,
    colback=blue!5!white, colframe=blue!75!black,
    colbacktitle=blue!85!black, fonttitle=\bfseries, coltitle=white, title=Problem 5]
\setlength{\parskip}{1em}
Elisa has 2023 treasure chests, all of which are unlocked and empty at first. Each day, Elisa adds a new gem to one of the unlocked chests of her choice, and afterwards, a fairy acts according to the following rules:
\begin{itemize}
  \item if more than one chests are unlocked, it locks one of them, or
  \item if there is only one unlocked chest, it unlocks all the chests.
\end{itemize}
Given that this process goes on forever, prove that there is a constant $C$ with the following property: Elisa can ensure that the difference between the numbers of gems in any two chests never exceeds $C$, regardless of how the fairy chooses the chests to lock.
\end{tcolorbox}

\begin{tcolorbox}[enhanced, breakable, rounded corners, 
    colback=orange!5!white, colframe=orange!75!black,
    colbacktitle=orange!85!black, fonttitle=\bfseries, coltitle=white,
    title=Problem 5 Answer, width=\columnwidth]
The constants $C=n-1$ for odd $n$ and $C=n$ for even $n$ are in fact optimal. 
\end{tcolorbox}

\begin{tcolorbox}[enhanced, breakable, rounded corners,
    colback=green!5!white, colframe=green!75!black,
    colbacktitle=green!85!black, fonttitle=\bfseries, coltitle=white, title=Problem 5 Solution 1]
\setlength{\parskip}{1em}
We will prove that such a constant $C$ exists when there are $n$ chests for $n$ an odd positive integer. In fact we can take $C=n-1$. Elisa's strategy is simple: place a gem in the chest with the fewest gems (in case there are more than one such chests, pick one arbitrarily).
For each integer $t \geqslant 0$, let $a_{1}^{t} \leqslant a_{2}^{t} \leqslant \cdots \leqslant a_{n}^{t}$ be the numbers of gems in the $n$ chests at the end of the $t^{\text {th }}$ day. In particular, $a_{1}^{0}=\cdots=a_{n}^{0}=0$ and
$$
a_{1}^{t}+a_{2}^{t}+\cdots+a_{n}^{t}=t
$$
For each $t \geqslant 0$, there is a unique index $m=m(t)$ for which $a_{m}^{t+1}=a_{m}^{t}+1$. We know that $a_{j}^{t}>a_{m(t)}^{t}$ for all $j>m(t)$, since $a_{m(t)}^{t}<a_{m(t)}^{t+1} \leqslant a_{j}^{t+1}=a_{j}^{t}$. Elisa's strategy also guarantees that if an index $j$ is greater than the remainder of $t$ when divided by $n$ (i.e. the number of locked chests at the end of the $t^{\text {th }}$ day), then $a_{j}^{t} \geqslant a_{m(t)}^{t}$, because some chest with at most $a_{j}^{t}$ gems must still be unlocked at the end of the $t^{\text {th }}$ day.
Recall that a sequence $x_{1} \leqslant x_{2} \leqslant \cdots \leqslant x_{n}$ of real numbers is said to majorise another sequence $y_{1} \leqslant y_{2} \leqslant \cdots \leqslant y_{n}$ of real numbers when for all $1 \leqslant k \leqslant n$ we have
$$
x_{1}+x_{2}+\cdots+x_{k} \leqslant y_{1}+y_{2}+\cdots+y_{k}
$$
and
$$
x_{1}+x_{2}+\cdots+x_{n}=y_{1}+y_{2}+\cdots+y_{n}
$$
Our strategy for proving $a_{n}^{t}-a_{1}^{t} \leqslant n-1$ is to inductively show that the sequence $\left(a_{i}^{t}\right)$ is majorised by some other sequence $\left(b_{i}^{t}\right)$.
We define this other sequence $\left(b_{i}^{t}\right)$ as follows. Let $b_{k}^{0}=k-\frac{n+1}{2}$ for $1 \leqslant k \leqslant n$. As $n$ is odd, this is a strictly increasing sequence of integers, and the sum of its terms is 0 . Now define $b_{i}^{t}=b_{i}^{0}+\left\lfloor\frac{t-i}{n}\right\rfloor+1$ for $t \geqslant 1$ and $1 \leqslant i \leqslant n$. Thus for $t \geqslant 0$,
$$
b_{i}^{t+1}=\left\{\begin{array}{lll}
b_{i}^{t} & \text { if } t+1 \not \equiv i & (\bmod n) \\
b_{i}^{t}+1 & \text { if } t+1 \equiv i & (\bmod n)
\end{array}\right.
$$
From these properties it is easy to see that
\begin{itemize}
  \item $b_{1}^{t}+b_{2}^{t}+\cdots+b_{n}^{t}=t$ for all $t \geqslant 0$, and
  \item $b_{i}^{t} \leqslant b_{i+1}^{t}$ for all $t \geqslant 0$ and $1 \leqslant i \leqslant n-1$, with the inequality being strict if $t \not \equiv i(\bmod n)$.
\end{itemize}
Claim 1. For each $t \geqslant 0$, the sequence of integers $b_{1}^{t}, b_{2}^{t}, \ldots, b_{n}^{t}$ majorises the sequence of integers $a_{1}^{t}, a_{2}^{t}, \ldots, a_{n}^{t}$.
Proof. We use induction on $t$. The base case $t=0$ is trivial. Assume $t \geqslant 0$ and that $\left(b_{i}^{t}\right)$ majorises $\left(a_{i}^{t}\right)$. We want to prove the same holds for $t+1$.
First note that the two sequences $\left(b_{i}^{t+1}\right)$ and $\left(a_{i}^{t+1}\right)$ both sum up to $t+1$. Next, we wish to show that for $1 \leqslant k<n$, we have
$$
b_{1}^{t+1}+b_{2}^{t+1}+\cdots+b_{k}^{t+1} \leqslant a_{1}^{t+1}+a_{2}^{t+1}+\cdots+a_{k}^{t+1}
$$
When $t+1$ is replaced by $t$, the above inequality holds by the induction hypothesis. For the sake of contradiction, suppose $k$ is the smallest index such that the inequality for $t+1$ fails. Since the left hand side increases by at most 1 during the transition from $t$ to $t+1$, the inequality for $t+1$ can fail only if all of the following occur:
\begin{itemize}
  \item $b_{1}^{t}+b_{2}^{t}+\cdots+b_{k}^{t}=a_{1}^{t}+a_{2}^{t}+\cdots+a_{k}^{t}$,
  \item $t+1 \equiv j(\bmod n)$ for some $1 \leqslant j \leqslant k\left(\right.$ so that $\left.b_{j}^{t+1}=b_{j}^{t}+1\right)$,
  \item $m(t)>k$ (so that $a_{i}^{t+1}=a_{i}^{t}$ for $1 \leqslant i \leqslant k$ ).
\end{itemize}
The first point and the minimality of $k$ tell us that $b_{1}^{t}, \ldots, b_{k}^{t}$ majorises $a_{1}^{t}, \ldots, a_{k}^{t}$ as well (again using the induction hypothesis), and in particular $b_{k}^{t} \geqslant a_{k}^{t}$.
The second point tells us that the remainder of $t$ when divided by $n$ is at most $k-1$, so $a_{k}^{t} \geqslant a_{m(t)}^{t}$ (by Elisa's strategy). But by the third point $(m(t) \geqslant k+1)$ and the nondecreasing property of $a_{i}^{t}$, we must have the equalities $a_{k}^{t}=a_{k+1}^{t}=a_{m(t)}^{t}$. On the other hand, $a_{k}^{t} \leqslant b_{k}^{t}<b_{k+1}^{t}$, with the second inequality being strict because $t \not \equiv k(\bmod n)$. We conclude that
$$
b_{1}^{t}+b_{2}^{t}+\cdots+b_{k+1}^{t}>a_{1}^{t}+a_{2}^{t}+\cdots+a_{k+1}^{t}
$$
a contradiction to the induction hypothesis.\\
This completes the proof as it implies
$$
a_{n}^{t}-a_{1}^{t} \leqslant b_{n}^{t}-b_{1}^{t} \leqslant b_{n}^{0}-b_{1}^{0}=n-1
$$
Comment 1. The statement is true even when $n$ is even. In this case, we instead use the initial state
$$
b_{k}^{0}= \begin{cases}k-\frac{n}{2}-1 & k \leqslant \frac{n}{2} \\ k-\frac{n}{2} & k>\frac{n}{2}\end{cases}
$$
The same argument shows that $C=n$ works.\\
Comment 2. The constants $C=n-1$ for odd $n$ and $C=n$ for even $n$ are in fact optimal. To see this, we will assume that the fairy always locks a chest with the minimal number of gems. Then at every point, if a chest is locked, any other chest with fewer gems will also be locked. Thus $m(t)$ is always greater than the remainder of $t$ when divided by $n$. This implies that the quantities
$$
I_{k}=a_{1}^{t}+\cdots+a_{k}^{t}-b_{1}^{t}-\cdots-b_{k}^{t}
$$
for each $0 \leqslant k \leqslant n$, do not increase regardless of how Elisa acts. If Elisa succeeds in keeping $a_{n}^{t}-a_{1}^{t}$ bounded, then these quantities must also be bounded; thus they are eventually constant, say for $t \geqslant t_{0}$. This implies that for all $t \geqslant t_{0}$, the number $m(t)$ is equal to 1 plus the remainder of $t$ when divided by $n$.\\
Claim 2. For $T \geqslant t_{0}$ divisible by $n$, we have
$$
a_{1}^{T}<a_{2}^{T}<\cdots<a_{n}^{T}
$$
Proof. Suppose otherwise, and let $j$ be an index for which $a_{j}^{T}=a_{j+1}^{T}$. We have $m(T+k-1)=k$ for all $1 \leqslant k \leqslant n$. Then $a_{j}^{T+j}>a_{j+1}^{T+j}$, which gives a contradiction.
This implies $a_{n}^{T}-a_{1}^{T} \geqslant n-1$, which already proves optimality of $C=n-1$ for odd $n$. For even $n$, note that the sequence ( $a_{i}^{T}$ ) has sum divisible by $n$, so it cannot consist of $n$ consecutive integers. Thus $a_{n}^{T}-a_{1}^{T} \geqslant n$ for $n$ even.
\end{tcolorbox}

\begin{tcolorbox}[enhanced, breakable, rounded corners,
    colback=green!5!white, colframe=green!75!black,
    colbacktitle=green!85!black, fonttitle=\bfseries, coltitle=white, title=Problem 5 Solution 2]
\setlength{\parskip}{1em}
We solve the problem when 2023 is replaced with an arbitrary integer $n$. We assume that Elisa uses the following strategy:
At the beginning of the $(n t+1)^{\text {th }}$ day, Elisa first labels her chests as $C_{1}^{t}, \ldots, C_{n}^{t}$ so that before she adds in the gem, the number of gems in $C_{i}^{t}$ is less than or equal $C_{j}^{t}$ for all $1 \leqslant i<j \leqslant n$. Then for days $n t+1, n t+2, \ldots, n t+n$, she adds a gem to chest $C_{i}^{t}$, where $i$ is chosen to be minimal such that $C_{i}^{t}$ is unlocked.
Denote by $c_{i}^{t}$ the number of gems in chest $C_{i}^{t}$ at the beginning of the $(n t+1)^{\text {th }}$ day, so that
$$
c_{1}^{t} \leqslant c_{2}^{t} \leqslant \cdots \leqslant c_{n}^{t}
$$
by construction. Also, denote by $\delta_{i}^{t}$ the total number of gems added to chest $C_{i}^{t}$ during days $n t+1, \ldots, n t+n$. We make the following observations.
\begin{itemize}
  \item We have $c_{1}^{0}=c_{2}^{0}=\cdots=c_{n}^{0}=0$.
  \item We have $c_{1}^{t}+\cdots+c_{n}^{t}=n t$, since $n$ gems are added every $n$ days.
  \item The sequence $\left(c_{i}^{t+1}\right)$ is a permutation of the sequence $\left(c_{i}^{t}+\delta_{i}^{t}\right)$ for all $t \geqslant 0$.
  \item We have $\delta_{1}^{t}+\cdots+\delta_{n}^{t}=n$ for all $t \geqslant 0$.
  \item Since Elisa adds a gem to an unlocked chest $C_{i}^{t}$ with $i$ minimal, we have
\end{itemize}
$$
\delta_{1}^{t}+\delta_{2}^{t}+\cdots+\delta_{k}^{t} \geqslant k
$$
for every $1 \leqslant k \leqslant n$ and $t \geqslant 0$.\\
We now define another sequence of sequences of integers as follows.
$$
d_{i}^{0}=3 n\left(i-\frac{n+1}{2}\right), \quad d_{i}^{t}=d_{i}^{0}+t .
$$
We observe that
$$
d_{1}^{t}+\cdots+d_{n}^{t}=c_{1}^{t}+\cdots+c_{n}^{t}=n t
$$
Claim 3. For each $t \geqslant 0$, the sequence $\left(d_{i}^{t}\right)$ majorises the sequence $\left(c_{i}^{t}\right)$.\\
Proof. We induct on $t$. For $t=0$, this is clear as all the terms in the sequence $\left(c_{i}^{t}\right)$ are equal. For the induction step, we assume that $\left(d_{i}^{t}\right)$ majorises $\left(c_{i}^{t}\right)$. Given $1 \leqslant k \leqslant n-1$, we wish to show that
$$
d_{1}^{t+1}+\cdots+d_{k}^{t+1} \leqslant c_{1}^{t+1}+\cdots+c_{k}^{t+1}
$$
Case 1: $c_{1}^{t+1}, \ldots, c_{k}^{t+1}$ is a permutation of $c_{1}^{t}+\delta_{1}^{t}, \ldots, c_{k}^{t}+\delta_{k}^{t}$.\\
Since $d_{1}^{t}+\cdots+d_{k}^{t} \leqslant c_{1}^{t}+\cdots+c_{k}^{t}$ by the induction hypothesis, we have
$$
\sum_{i=1}^{k} d_{i}^{t+1}=k+\sum_{i=1}^{k} d_{i}^{t} \leqslant k+\sum_{i=1}^{k} c_{i}^{t} \leqslant \sum_{i=1}^{k}\left(c_{i}^{t}+\delta_{i}^{t}\right)=\sum_{i=1}^{k} c_{i}^{t+1}
$$
Case 2: $c_{1}^{t+1}, \ldots, c_{k}^{t+1}$ is not a permutation of $c_{1}^{t}+\delta_{1}^{t}, \ldots, c_{k}^{t}+\delta_{k}^{t}$.\\
In this case, we have $c_{i}^{t}+\delta_{i}^{t}>c_{j}^{t}+\delta_{j}^{t}$ for some $i \leqslant k<j$. It follows that
$$
c_{k}^{t}+n \geqslant c_{i}^{t}+n \geqslant c_{i}^{t}+\delta_{i}^{t}>c_{j}^{t}+\delta_{j}^{t} \geqslant c_{j}^{t} \geqslant c_{k+1}^{t}
$$
Using $d_{k}^{t}+3 n=d_{k+1}^{t}$ and the induction hypothesis, we obtain
$$
\begin{aligned}
\sum_{i=1}^{k} c_{i}^{t+1} & \geqslant \sum_{i=1}^{k} c_{i}^{t}>c_{1}^{t}+\cdots+c_{k-1}^{t}+\frac{1}{2} c_{k}^{t}+\frac{1}{2} c_{k+1}^{t}-\frac{n}{2}=\frac{1}{2} \sum_{i=1}^{k-1} c_{i}^{t}+\frac{1}{2} \sum_{i=1}^{k+1} c_{i}^{t}-\frac{n}{2} \\
& \geqslant \frac{1}{2} \sum_{i=1}^{k-1} d_{i}^{t}+\frac{1}{2} \sum_{i=1}^{k+1} d_{i}^{t}-\frac{n}{2}=n+\sum_{i=1}^{k} d_{i}^{t} \geqslant k+\sum_{i=1}^{k} d_{i}^{t}=\sum_{i=1}^{k} d_{i}^{t+1}
\end{aligned}
$$
This finishes the induction step.\\
It follows that
$$
c_{n}^{t}-c_{1}^{t} \leqslant d_{n}^{t}-d_{1}^{t}=3 n(n-1)
$$
From day $n t+1$ to day $n(t+1)+1$, Elisa adds $n$ gems, and therefore the difference may increase by at most $n$. This shows that the difference of the number of gems in two chests never exceeds $C=3 n(n-1)+n$.

\end{tcolorbox}


\begin{tcolorbox}[enhanced, breakable, rounded corners,
    colback=blue!5!white, colframe=blue!75!black,
    colbacktitle=blue!85!black, fonttitle=\bfseries, coltitle=white, title=Problem 6]
\setlength{\parskip}{1em}
Let $N$ be a positive integer, and consider an $N \times N$ grid. A right-down path is a sequence of grid cells such that each cell is either one cell to the right of or one cell below the previous cell in the sequence. A right-up path is a sequence of grid cells such that each cell is either one cell to the right of or one cell above the previous cell in the sequence.

Prove that the cells of the $N \times N$ grid cannot be partitioned into less than $N$ right-down or right-up paths. For example, the following partition of the $5 \times 5$ grid uses 5 paths.
\end{tcolorbox}

\begin{tcolorbox}[enhanced, breakable, rounded corners, 
    colback=orange!5!white, colframe=orange!75!black,
    colbacktitle=orange!85!black, fonttitle=\bfseries, coltitle=white,
    title=Problem 6 Answer, width=\columnwidth]
N/A
\end{tcolorbox}

\begin{tcolorbox}[enhanced, breakable, rounded corners,
    colback=green!5!white, colframe=green!75!black,
    colbacktitle=green!85!black, fonttitle=\bfseries, coltitle=white, title=Problem 6 Solution 1]
\setlength{\parskip}{1em}
We define a good parallelogram to be a parallelogram composed of two isosceles right-angled triangles glued together as shown below.
\begin{center}
    \includegraphics[width=0.3\linewidth]{SLC6solution1.png}
\end{center}
Given any partition into $k$ right-down or right-up paths, we can find a corresponding packing of good parallelograms that leaves an area of $k$ empty. Thus, it suffices to prove that we must leave an area of at least $N$ empty when we pack good parallelograms into an $N \times N$ grid. This is actually equivalent to the original problem since we can uniquely recover the partition into right-down or right-up paths from the corresponding packing of good parallelograms.\\
\begin{center}
        \includegraphics[width=0.5\linewidth]{SLC6solution2.png}
    \end{center}
We draw one of the diagonals in each cell so that it does not intersect any of the good parallelograms. Now, view these segments as mirrors, and consider a laser entering each of the $4 N$ boundary edges (with starting direction being perpendicular to the edge), bouncing along these mirrors until it exits at some other edge. When a laser passes through a good parallelogram, its direction goes back to the original one after bouncing two times. Thus, if the final direction of a laser is perpendicular to its initial direction, it must pass through at least\\
one empty triangle. Similarly, if the final direction of a laser is opposite to its initial direction, it must pass though at least two empty triangles. Using this, we will estimate the number of empty triangles in the $N \times N$ grid.
We associate the starting edge of a laser with the edge it exits at. Then, the boundary edges are divided into $2 N$ pairs. These pairs can be classified into three types:\\
(1) a pair of a vertical and a horizontal boundary edge,\\
(2) a pair of boundary edges from the same side, and\\
(3) a pair of boundary edges from opposite sides.
Since the beams do not intersect, we cannot have one type (3) pair from vertical boundary edges and another type (3) pair from horizontal boundary edges. Without loss of generality, we may assume that we have $t$ pairs of type (3) and they are all from vertical boundary edges. Then, out of the remaining boundary edges, there are $2 N$ horizontal boundary edges and $2 N-2 t$ vertical boundary edges. It follows that there must be at least $t$ pairs of type (2) from horizontal boundary edges. We know that a laser corresponding to a pair of type (1) passes through at least one empty triangle, and a laser corresponding to a pair of type (2) passes through at least two empty triangles. Thus, as the beams do not intersect, we have at least $(2 N-2 t)+2 \cdot t=2 N$ empty triangles in the grid, leaving an area of at least $N$ empty as required.
\end{tcolorbox}

\begin{tcolorbox}[enhanced, breakable, rounded corners,
    colback=green!5!white, colframe=green!75!black,
    colbacktitle=green!85!black, fonttitle=\bfseries, coltitle=white, title=Problem 6 Solution 2]
\setlength{\parskip}{1em}
We apply an induction on $N$. The base case $N=1$ is trivial. Suppose that the claim holds for $N-1$ and prove it for $N \geqslant 2$.
Let us denote the path containing the upper left corner by $P$. If $P$ is right-up, then every cell in $P$ is in the top row or in the leftmost column. By the induction hypothesis, there are at least $N-1$ paths passing through the lower right $(N-1) \times(N-1)$ subgrid. Since $P$ is not amongst them, we have at least $N$ paths.
Next, assume that $P$ is right-down. If $P$ contains the lower right corner, then we get an $(N-1) \times(N-1)$ grid by removing $P$ and glueing the remaining two parts together. The main idea is to extend $P$ so that it contains the lower right corner and the above procedure gives a valid partition of an $(N-1) \times(N-1)$ grid.
\begin{center}
        \includegraphics[width=0.5\linewidth]{SLC6solution3.png}
    \end{center}
We inductively construct $Q$, which denotes an extension of $P$ as a right-down path. Initially, $Q=P$. Let $A$ be the last cell of $Q, B$ be the cell below $A$, and $C$ be the cell to the right of $A$ (if they exist). Suppose that $A$ is not the lower right corner, and that (*) both $B$ and $C$ do not belong to the same path as $A$. Then, we can extend $Q$ as follows (in case we have two or more options, we can choose any one of them to extend $Q$ ).
\begin{enumerate}
  \item If $B$ belongs to a right-down path $R$, then we add the part of $R$, from $B$ to its end, to $Q$.
  \item If $C$ belongs to a right-down path $R$, then we add the part of $R$, from $C$ to its end, to $Q$.
  \item If $B$ belongs to a right-up path $R$ which ends at $B$, then we add the part of $R$ in the same column as $B$ to $Q$.
  \item If $C$ belongs to a right-up path $R$ which starts at $C$, then we add the part of $R$ in the same row as $C$ to $Q$.
  \item Otherwise, $B$ and $C$ must belong to the same right-up path $R$. In this case, we add $B$ and the cell to the right of $B$ to $Q$.
\end{enumerate}
Note that if $B$ does not exist, then case (4) must hold. If $C$ does not exist, then case (3) must hold.
It is easily seen that such an extension also satisfies the hypothesis (*), so we can repeat this construction to get an extension of $P$ containing the lower right corner, denoted by $Q$. We show that this is a desired extension, i.e. the partition of an $(N-1) \times(N-1)$ grid obtained by removing $Q$ and glueing the remaining two parts together consists of right-down or right-up paths.
Take a path $R$ in the partition of the $N \times N$ grid intersecting $Q$. If the intersection of $Q$ and $R$ occurs in case (1) or case (2), then there exists a cell $D$ in $R$ such that the intersection of $Q$ and $R$ is the part of $R$ from $D$ to its end, so $R$ remains a right-down path after removal of $Q$. Similarly, if the intersection of $Q$ and $R$ occurs in case (3) or case (4), then $R$ remains a right-up path after removal of $Q$. If the intersection of $Q$ and $R$ occurs in case (5), then this intersection has exactly two adjacent cells. After the removal of these two cells (as we remove $Q), R$ is divided into two parts that are glued into a right-up path.
Thus, we may apply the induction hypothesis to the resulting partition of an $(N-1) \times(N-1)$ grid, to find that it must contain at least $N-1$ paths. Since $P$ is contained in $Q$ and is not amongst these paths, the original partition must contain at least $N$ paths.

\end{tcolorbox}


\begin{tcolorbox}[enhanced, breakable, rounded corners,
    colback=blue!5!white, colframe=blue!75!black,
    colbacktitle=blue!85!black, fonttitle=\bfseries, coltitle=white, title=Problem 7]
\setlength{\parskip}{1em}
The Imomi archipelago consists of $n \geqslant 2$ islands. Between each pair of distinct islands is a unique ferry line that runs in both directions, and each ferry line is operated by one of $k$ companies. It is known that if any one of the $k$ companies closes all its ferry lines, then it becomes impossible for a traveller, no matter where the traveller starts at, to visit all the islands exactly once (in particular, not returning to the island the traveller started at).
Determine the maximal possible value of $k$ in terms of $n$.
\end{tcolorbox}

\begin{tcolorbox}[enhanced, breakable, rounded corners, 
    colback=orange!5!white, colframe=orange!75!black,
    colbacktitle=orange!85!black, fonttitle=\bfseries, coltitle=white,
    title=Problem 7 Answer, width=\columnwidth]
 The largest $k$ is $k=\left\lfloor\log _{2} n\right\rfloor$.
\end{tcolorbox}

\begin{tcolorbox}[enhanced, breakable, rounded corners,
    colback=green!5!white, colframe=green!75!black,
    colbacktitle=green!85!black, fonttitle=\bfseries, coltitle=white, title=Problem 7 Solution]
\setlength{\parskip}{1em}
We reformulate the problem using graph theory. We have a complete graph $K_{n}$ on $n$ nodes (corresponding to islands), and we want to colour the edges (corresponding to ferry lines) with $k$ colours (corresponding to companies), so that every Hamiltonian path contains all $k$ different colours. For a fixed set of $k$ colours, we say that an edge colouring of $K_{n}$ is good if every Hamiltonian path contains an edge of each one of these $k$ colours.
We first construct a good colouring of $K_{n}$ using $k=\left\lfloor\log _{2} n\right\rfloor$ colours.\\
Claim 1. Take $k=\left\lfloor\log _{2} n\right\rfloor$. Consider the complete graph $K_{n}$ in which the nodes are labelled by $1,2, \ldots, n$. Colour node $i$ with colour $\min \left(\left\lfloor\log _{2} i\right\rfloor+1, k\right)$ (so the colours of the first nodes are $1,2,2,3,3,3,3,4, \ldots$ and the last $n-2^{k-1}+1$ nodes have colour $k$ ), and for $1 \leqslant i<j \leqslant n$, colour the edge $i j$ with the colour of the node $i$. Then the resulting edge colouring of $K_{n}$ is good.\\
Proof. We need to check that every Hamiltonian path contains edges of every single colour. We first observe that the number of nodes assigned colour $k$ is $n-2^{k-1}+1$. Since $n \geqslant 2^{k}$, we have
$$
n-2^{k-1}+1 \geqslant \frac{n}{2}+1
$$
This implies that in any Hamiltonian path, there exists an edge between two nodes with colour $k$. Then that edge must have colour $k$.
We next show that for each $1 \leqslant i<k$, every Hamiltonian path contains an edge of colour $i$. Suppose the contrary, that some Hamiltonian path does not contain an edge of colour $i$. Then nodes with colour $i$ can only be adjacent to nodes with colour less than $i$ inside the Hamiltonian path. Since there are $2^{i-1}$ nodes with colour $i$ and $2^{i-1}-1$ nodes with colour less than $i$, the Hamiltonian path must take the form
$$
(i) \leftrightarrow(<i) \leftrightarrow(i) \leftrightarrow(<i) \leftrightarrow \cdots \leftrightarrow(<i) \leftrightarrow(i)
$$
where $(i)$ denotes a node with colour $i,(<i)$ denotes a node with colour less than $i$, and $\leftrightarrow$ denotes an edge. But this is impossible, as the Hamiltonian path would not have any nodes with colours greater than $i$.
Fix a set of $k$ colours, we now prove that if there exists a good colouring of $K_{n}$, then $k \leqslant\left\lfloor\log _{2} n\right\rfloor$. For $n=2$, this is trivial, so we assume $n \geqslant 3$. For any node $v$ of $K_{n}$ and $1 \leqslant i \leqslant k$, we denote by $d_{i}(v)$ the number of edges with colour $i$ incident with the node $v$.\\
Lemma 1. Consider a good colouring of $K_{n}$, and let $A B$ be an arbitrary edge with colour $i$. If $d_{i}(A)+d_{i}(B) \leqslant n-1$, then the colouring will remain good after recolouring edge $A B$ with any other colour.\\
Proof. Suppose there exists a good colouring together with an edge $A B$ of colour $i$, such that if $A B$ is recoloured with another colour, the colouring will no longer be good. The failure of the new colouring being good will come from colour $i$, and thus there exists a Hamiltonian path containing edge $A B$ such that initially (i.e. before recolouring) $A B$ is the only edge of colour $i$ in this path. Writing $A=A_{0}$ and $B=B_{0}$, denote this Hamiltonian path by
$$
A_{s} \leftrightarrow A_{s-1} \leftrightarrow \cdots \leftrightarrow A_{1} \leftrightarrow A_{0} \leftrightarrow B_{0} \leftrightarrow B_{1} \leftrightarrow \cdots \leftrightarrow B_{t-1} \leftrightarrow B_{t}
$$
where $s, t \geqslant 0$ and $s+t+2=n$.\\
In the initial colouring, we observe the following.
\begin{itemize}
  \item The edge $B_{0} A_{s}$ must have colour $i$, since otherwise the path
\end{itemize}
$$
A_{0} \leftrightarrow A_{1} \leftrightarrow \cdots \leftrightarrow A_{s-1} \leftrightarrow A_{s} \leftrightarrow B_{0} \leftrightarrow B_{1} \leftrightarrow \cdots \leftrightarrow B_{t-1} \leftrightarrow B_{t}
$$
has no edges of colour $i$.
\begin{itemize}
  \item Similarly, the edge $A_{0} B_{t}$ must have colour $i$.
  \item For each $0 \leqslant p<s$, at least one of the edges $B_{0} A_{p}$ and $A_{0} A_{p+1}$ must have colour $i$, since otherwise the path
\end{itemize}
$$
A_{s} \leftrightarrow \cdots \leftrightarrow A_{p+2} \leftrightarrow A_{p+1} \leftrightarrow A_{0} \leftrightarrow A_{1} \leftrightarrow \cdots \leftrightarrow A_{p-1} \leftrightarrow A_{p} \leftrightarrow B_{0} \leftrightarrow B_{1} \leftrightarrow \cdots \leftrightarrow B_{t}
$$
has no edges of colour $i$.
\begin{itemize}
  \item Similarly, for each $0 \leqslant q<t$, at least one of the edges $A_{0} B_{q}$ and $B_{0} B_{q+1}$ must have colour $i$.
\end{itemize}
In the above list, each edge $A_{0} X$ appears exactly once and also each edge $B_{0} X$ appears exactly once (where $A_{0} B_{0}$ and $B_{0} A_{0}$ are counted separately). Adding up the contributions to $d_{i}(A)+$ $d_{i}(B)$, we obtain
$$
d_{i}(A)+d_{i}(B) \geqslant(s+1)+(t+1)=n
$$
This contradicts our assumption that $d_{i}(A)+d_{i}(B) \leqslant n-1$.\\
Our strategy now is to repeatedly recolour the edges using Lemma 1 until the colouring has a simple structure. For a node $v$, we define $m(v)$ to be the largest value of $d_{i}(v)$ over all colours $i$.\\
Lemma 2. Assume we have a good colouring of $K_{n}$. Let $A, B$ be two distinct nodes, and let $j$ be the colour of edge $A B$ where $1 \leqslant j \leqslant k$. If
\begin{itemize}
  \item $m(A) \geqslant m(B)$ and
  \item $m(A)=d_{i}(A)$ for some $i \neq j$,\\
then after recolouring edge $A B$ with colour $i$, the colouring remains good.\\
Proof. Note that
\end{itemize}
$$
d_{j}(A)+d_{j}(B) \leqslant(n-1-m(A))+m(B) \leqslant n-1
$$
and so we may apply Lemma 1 .\\
Lemma 3. Assume we have a good colouring of $K_{n}$. Let $S$ be a nonempty set of nodes. Let $A \in S$ be a node such that $m(A) \geqslant m(B)$ for all $B \in S$, and choose $1 \leqslant i \leqslant k$ for which $d_{i}(A)=m(A)$. Then after recolouring the edge $A B$ with colour $i$ for all $B \in S$ distinct from $A$, the colouring remains good.\\
Proof. We repeatedly perform the following operation until all edges $A B$ with $B \in S$ have colour $i$ :\\
choose an edge $A B$ with $B \in S$ that does not have colour $i$, and recolour it with colour $i$.\\
By Lemma 2, the colouring remains good after one operation. Moreover, $m(A)$ increase by 1 during an operation, and all other $m(B)$ may increase by at most 1 . This shows that $m(A)$ will remain maximal amongst $m(B)$ for $B \in S$. We will also have $d_{i}(A)=m(A)$ after the operation, since both sides increase by 1 . Therefore the operation can be performed repeatedly, and the colouring remains good.
We first apply Lemma 3 to the set of all $n$ nodes in $K_{n}$. After recolouring, there exists a node $A_{1}$ such that every edge incident with $A_{1}$ has colour $c_{1}$. We then apply Lemma 3 to the set of nodes excluding $A_{1}$, and we obtain a colouring where
\begin{itemize}
  \item every edge incident with $A_{1}$ has colour $c_{1}$,
  \item every edge incident with $A_{2}$ except for $A_{1} A_{2}$ has colour $c_{2}$.
\end{itemize}
Repeating this process, we arrive at the following configuration:
\begin{itemize}
  \item the $n$ nodes of $K_{n}$ are labelled $A_{1}, A_{2}, \ldots, A_{n}$,
  \item the node $A_{i}$ has a corresponding colour $c_{i}$ (as a convention, we also colour $A_{i}$ with $c_{i}$ ),
  \item for all $1 \leqslant u<v \leqslant n$, the edge between $A_{u}$ and $A_{v}$ has colour $c_{u}$,
  \item this colouring is good.
\end{itemize}
Claim 2. For every colour $i$, there exists a $1 \leqslant p \leqslant n$ such that the number of nodes of colour $i$ amongst $A_{1}, \ldots, A_{p}$ is greater than $p / 2$.\\
Proof. Suppose the contrary, that for every $1 \leqslant p \leqslant n$, there are at most $\lfloor p / 2\rfloor$ nodes of colour $i$. We then construct a Hamiltonian path not containing any edge of colour $i$. Let $A_{x_{1}}, \ldots, A_{x_{t}}$ be the nodes with colour $i$, where $x_{1}<x_{2}<\cdots<x_{t}$, and let $A_{y_{1}}, A_{y_{2}}, \ldots, A_{y_{s}}$ be the nodes with colour different from $i$, where $y_{1}<y_{2}<\cdots<y_{s}$. We have $s+t=n$ and $t \leqslant\lfloor n / 2\rfloor$, so $t \leqslant s$. We also see that $y_{j}<x_{j}$ for all $1 \leqslant j \leqslant t$, because otherwise, $A_{1}, A_{2}, \ldots, A_{x_{j}}$ will have $j$ nodes of colour $i$ and less than $j$ nodes of colour different from $i$. Then we can construct a Hamiltonian path
$$
A_{x_{1}} \leftrightarrow A_{y_{1}} \leftrightarrow A_{x_{2}} \leftrightarrow A_{y_{2}} \leftrightarrow A_{x_{3}} \leftrightarrow \cdots \leftrightarrow A_{x_{t}} \leftrightarrow A_{y_{t}} \leftrightarrow A_{y_{t+1}} \leftrightarrow \cdots \leftrightarrow A_{y_{s}}
$$
that does not contain an edge with colour $i$. This contradicts that the colouring is good.\\
So for every colour $i$, there has to be an integer $p_{i}$ with $1 \leqslant p_{i} \leqslant n$ such that there are more than $p_{i} / 2$ nodes assigned colour $i$ amongst $A_{1}, \ldots, A_{p_{i}}$. Choose the smallest such $p_{i}$ for every $i$, and without loss of generality assume
$$
p_{1}<p_{2}<\cdots<p_{k}
$$
Note that the inequalities are strict by the definition of $p_{i}$.\\
Then amongst the nodes $A_{1}, \ldots, A_{p_{i}}$, there are at least $\left\lceil\left(p_{j}+1\right) / 2\right\rceil$ nodes of colour $j$ for all $1 \leqslant j \leqslant i$. Then
$$
p_{i} \geqslant\left\lceil\frac{p_{1}+1}{2}\right\rceil+\left\lceil\frac{p_{2}+1}{2}\right\rceil+\cdots+\left\lceil\frac{p_{i}+1}{2}\right\rceil
$$
This inductively shows that
$$
p_{i} \geqslant 2^{i}-1
$$
for all $1 \leqslant i \leqslant k$, and this already proves $n \geqslant 2^{k}-1$.\\
It remains to show that $n=2^{k}-1$ is not possible. If $n=2^{k}-1$, then all inequalities have to be equalities, so $p_{i}=2^{i}-1$ and there must be exactly $2^{i-1}$ nodes of colour $i$. Moreover, there cannot be a node of colour $i$ amongst $A_{1}, A_{2}, \ldots, A_{p_{i-1}}$, and so the set of nodes of colour $i$ must precisely be
$$
A_{2^{i-1}}, A_{2^{i-1}+1}, \ldots, A_{2^{i}-1}
$$
Then we can form a Hamiltonian path
$$
A_{2^{k-1}} \leftrightarrow A_{1} \leftrightarrow A_{2^{k-1}+1} \leftrightarrow A_{2} \leftrightarrow A_{2^{k-1}+2} \leftrightarrow A_{3} \leftrightarrow \ldots \leftrightarrow A_{n}
$$
which does not contain an edge of colour $k$. This is a contradiction, and therefore $n \geqslant 2^{k}$.

\end{tcolorbox}