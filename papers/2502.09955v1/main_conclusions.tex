\section{Conclusions}

This work shows that combining diverse inference methods with perfect verifiers tackles advanced reasoning tasks such as IMO combinatorics problems and ARC puzzles. In contrast, using an imperfect verifier, best-of-N rejection sampling, on the HLE shows good performance but at significant inference costs.

In Mathematics there is often a wide gap between the capability of the average human, expert Mathematician, and best Mathematician. The average human cannot solve, or finds it challenging to solve a single IMO problem, an expert Mathematician may find it challenging to solve half of the problems, whereas the best human problem solvers or Mathematicians can solve all of the problems. On unseen Mathematical Olympiad combinatorics, the best single model or method answers a third of the problems correctly, whereas the aggregate of diverse models and methods answer two thirds of the problems. The correct proof of the 2024 IMO combinatorics problem tips AI's overall performance from Silver to Gold medal level, placing it on par with around the top fifty worldwide each year, among tens of thousands of participants in national and international competitions.

