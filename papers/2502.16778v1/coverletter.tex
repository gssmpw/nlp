%%%%%%%%%%%%%%%%%%%%%%%%%%%%%%%%%%%%%%%%%
% Professional Formal Letter
% LaTeX Template
% Version 2.0 (12/2/17)
%
% This template originates from:
% http://www.LaTeXTemplates.com
%
% Authors:
% Brian Moses
% Vel (vel@LaTeXTemplates.com)
%
% License:
% CC BY-NC-SA 3.0 (http://creativecommons.org/licenses/by-nc-sa/3.0/)
%
%%%%%%%%%%%%%%%%%%%%%%%%%%%%%%%%%%%%%%%%%

%----------------------------------------------------------------------------------------
%	PACKAGES AND OTHER DOCUMENT CONFIGURATIONS
%----------------------------------------------------------------------------------------

\documentclass[11pt, letterpaper]{letter} % Set the font size (10pt, 11pt and 12pt) and paper size (letterpaper, a4paper, etc)

% \input{structure_with_sig} % Include the file that specifies the document structure

\usepackage{upgreek}

%\longindentation=0pt % Un-commenting this line will push the closing "Sincerely," and date to the left of the page

%----------------------------------------------------------------------------------------
%	YOUR INFORMATION
%----------------------------------------------------------------------------------------

% \Who{Emily Dolson} % Your name

% \Title{, PhD} % Your title, leave blank for no title

% \authordetails{
% 	Department of Computer Science and Engineering\\ % Your department/institution
% 	Ecology Evolution and Behavior Program\\
% 	Michigan State University\
% 	428 S. Shaw Lane\\ % Your address
% 	East Lansing, MI 48824\\ % Your city, zip code, country, etc
% 	Email: dolsonem@msu.edu\\ % Your email address
% 	Phone: (831) 428-2464\\ % Your phone number
% 	URL: https://www.ecodelab.com % Your URL
% }

% %----------------------------------------------------------------------------------------
% %	HEADER CONTENTS
% %----------------------------------------------------------------------------------------

% \logo{CCF.jpg} % Logo filename, your logo should have square dimensions (i.e. roughly the same width and height), if it does not, you will need to adjust spacing within the HEADER STRUCTURE block in structure.tex (read the comments carefully!)

% \headerlineone{Taussig Cancer Institute} % Top header line, leave blank if you only want the bottom line

% \headerlinetwo{Cleveland Clinic} % Bottom header line

%----------------------------------------------------------------------------------------

\begin{document}

%----------------------------------------------------------------------------------------
%	TO ADDRESS
%----------------------------------------------------------------------------------------

\begin{letter}{
% 	Prof. Jones\\
% 	Mathematics Search Committee\\
% 	Department of Mathematics\\
% 	University of California\\
% 	Berkeley, California 12345
}

%----------------------------------------------------------------------------------------
%	LETTER CONTENT
%------------------------------------------------------------------------------------

\opening{Dear Editor,}

We are pleased to submit our manuscript entitled \textit{The Robustness of Structural Features in Species Interaction Networks} for consideration in \textit{Methods in Ecology and Evolution}. 
In this work, we confront the problem of drawing meaningful inferences about the structure of species interaction networks despite missing data. This problem is currently a substantial obstacle preventing the bridging of theory and practice in community ecology. A large body of theoretical research addresses the ecological implications of different network topology properties. However, this theoretical research generally assumes perfect knowledge of species interactions, which is impractical in the real world. Consequently, while multiple databases of species interaction network have been assembled, there is uncertainty about how to reliably draw inferences from them.

We propose that one tool for confronting this problem lies in the fact that some structural properties are more affected by missing data than others. By simulating the addition of edges to observed ecological networks from the Web of Life database, we show that the values of some graph properties (e.g. certain community detection algorithms) can change dramatically if even a few edges are missed. Other properties, in contrast, remain fairly stable as edges are added. Thus, when working with noisy data, we suggest that ecologists should prefer the more robust properties. Moreover, our methodology can be used by practitioners to conduct a sensitivity analysis on their results.

We believe that this work will be of interest to readers of your journal because it presents a practical solution to a pervasive problem in community ecology. Thank you for your consideration.

\closing{{\normalsize Sincerely,}}
\vspace{-60pt}
Sanaz Hasanzadeh Fard and Emily Dolson

\end{letter}

\end{document}