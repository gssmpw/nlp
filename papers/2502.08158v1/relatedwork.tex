\section{Related Studies}
%%%%%%%%%%%%%%%%%%%%%%%%%%%%%%%%%%%%%%%%%%%%%%%%
FGO is widely used for robot state estimation, and various open-source libraries for FGO processing, such as g$^2$o \cite{g2o}, Ceres Solver \cite{ceres}, and GTSAM \cite{gtsam}, have been released. These libraries use Lie group-based 3D pose expressions to represent the state and perform optimization on Lie algebras. Numerous open-source FGO-based position estimation methods that use these libraries as backends have been released \cite{fgo_general1,fgo_book,orbslam,cartographer}.

Notably, most of the existing FGO-based position estimation methods use lidar or camera observations as factors. When GNSS is used as a factor, many methods directly use the position output from the GNSS receiver as a factor in the position estimation \cite{cartographer,hdlslam,liosam}. In GTSAM \cite{gtsam}, the 3D position constraint is provided as a GPS factor, and in studies \cite{cartographer}, \cite{hdlslam}, and \cite{liosam}, the GNSS position constraint is incorporated into the lidar odometry using FGO. This method, termed loose coupling, uses the position computed from the GNSS observations as the constraint. If the GNSS receiver cannot output a position owing to a decrease in the number of satellites, GNSS observations cannot be used for optimization. In addition, if the estimated position is degraded owing to multipath effects, the optimization accuracy is considerably deteriorated.

In contrast, several open-source FGO packages use raw GNSS observations, a strategy termed tight coupling \cite{robustgnss,graphgnsslib,gnssfgo,posgo,gicilib}. For example, the approach presented in \cite{robustgnss} uses GTSAM as a backend and incorporates only pseudorange observations. Other studies \cite{gvins,posgo} provides factors for GNSS Doppler observations in addition to pseudorange observations. Other frameworks, such as \cite{graphgnsslib}, \cite{gnssfgo}, and \cite{gicilib}, use factors that incorporate not only pseudorange and Doppler observations, but also carrier-phase measurements for high-precision positioning. The open-source packages in \cite{gnssfgo} and \cite{gicilib} combine GNSS with IMUs, lidar, and cameras. Notably, \cite{gicilib} estimates the integer ambiguity of the carrier phase and achieves high-precision positioning. However, the architecture of this program is complex, and it relies on the robot operating system (ROS), a middleware for robots, requiring substantial effort for optimizing GNSS alone or incorporating new factors or custom graph structures based on user-specific GNSS data. Although \cite{graphgnsslib} provides a simple open-source graph optimization library specialized for GNSS, it relies on ROS and is thus difficult to apply for GNSS positioning in a non-ROS environment.

Overall, the existing open-source GNSS FGO packages are specialized for specific applications and have a complex structure, rendering it challenging for beginners to learn and use them for custom problem settings with modified inputs and graph structures. Therefore, this paper introduces a novel open-source GNSS FGO package that is simple and user-friendly.

%%%%%%%%%%%%%%%%%%%%%%%%%%%%%%%%%%%%%%%%%%%%%%%%