\section{Literature Review}
\label{LitReview}
Extensive studies on two-phase flow patterns have been conducted since the early 1960s. Experiments revealed that flow patterns are influenced by flow rate, showing that two-phase flow is fundamentally distinct and more intricate than single-phase flow. These distinct configurations, known as flow regimes or flow patterns, depend on variables such as gas and liquid flow rates, fluid properties (density, viscosity, and surface tension), and geometrical factors (conduit shape, diameter, and inclination). A significant distinction is the incomplete or partial conversion of potential energy to pressure energy in downward flow, whereas in upward flow, much of the potential energy can be converted to kinetic energy, depending on the flow pattern \cite{hewitt1969phase,taitel1976model,mandhane1974flow,matsui1984identification,matsui1986identification,kokal1989aexperimental,ekberg1999gas,spedding1993flow,hong1997effect}.

Differentiating between flow patterns is crucial to understand multiphase flow structures. Visual inspection in transparent pipes is the common method, allowing automatic visual representation and analysis. However, standardizing flow pattern classification is challenging due to difficulties in accurate identification. Researchers face obstacles in categorizing flow patterns, which leads to discrepancies in designation by different methods. These issues stem from diverse detection methods and instruments. In addition, flow patterns may not develop fully at the instrument location, leading to observational discrepancies. Evaluating and comparing data from different experiments is challenging due to various instruments, methods, and undefined calibration procedures. Recent categorizations have been developed and accepted, yet naming discrepancies persist due to language variations, subjective naming practices, and lack of consensus, complicating consistency and data analysis in studies~\cite{songsiri2004tow,abduvayt2003effects,hernandez2006fast,kokal1989aexperimental,ekberg1999gas,spedding1993flow,bousman1996gas,elperin2002flow,somchai2006flow}.

Despite various contributions to understanding flow patterns, a consensus on a uniform classification and nomenclature remains elusive. Consequently, the focus will shift to widely accepted flow patterns, presenting alternative names when relevant. Each pattern will be briefly defined to highlight its distinctive form and mechanism. Dispersed bubble flow is characterized by small, discrete gas bubbles suspended in a predominantly liquid flow. In this scenario, the bubbles tend to rise to the top at low gas flow rates but become uniformly dispersed when the gas flow rates increase. This flow pattern is also referred to as ``bubbly flow''. Next is plug flow, which involves the clustering of small gas bubbles that merge to form larger, bullet-shaped bubbles. This phenomenon is sometimes known as ``cap-bubbly flow''. Elongated bubble flow features gas bubbles that take on a streamlined shape, resulting in an intermittent flow pattern. In this case, the rear end of a bubble may detach and be captured by the following bubble, creating a dynamic flow environment. Slug flow consists of sections where the gas phase nearly fills the entire cross-section of the pipe, alternating with liquid sections. This pattern is most common in horizontal and upward flows~\cite{soo1990multiphase,kleinstreuer2003two,hewitt1969phase}. Further complicating the behavior is slug froth flow, which is marked by highly frothy liquid slugs created through turbulence and mixing. This pattern, also referred to as ``churn flow'' or ``slug-churn'', often results in significant pressure drops~\cite{soo1990multiphase,kleinstreuer2003two}. In annular flow, liquid forms a film that coats the pipe wall, while gas flows through the center. This configuration can lead to the development of unstable waves as a result of the mixing of the two phases. Finally, stratified flow is defined by a clear separation between the liquid at the bottom of the pipe and the gas at the top. This flow pattern can be divided into two categories based on the gas flow rates: stratified smooth and stratified wavy flow~\cite{al2020bsimplified,al2008development,cheremisinoff1979stratified}.

In multiphase flow, various devices, instruments, and techniques are used to define and distinguish between different flow patterns. Some commonly used ones include:

\begin{enumerate} 
    \item \textbf{Flow Visualization Techniques}: Transparent pipes or flow channels with dyes or tracers to observe flow regimes~\cite{al2020systematic,al2020bsimplified,tsubone2001effects,almutairi2020ect}.
    
    \item \textbf{High-Speed Imaging}: Captures the transient behavior of flows, allowing identification and classification of patterns~\cite{somchai2006flow,ito2001flow,bennett2006frequency,xia1996two,clarke2001study,van2001evolution}.
    
    \item \textbf{Pressure Drop Measurements}: Pressure sensors determine gradients and identify patterns based on pressure drop characteristics~\cite{matsui1984identification,matsui1986identification,spedding1993flow,samways1997pressure,li2002experiment}.
    
    \item \textbf{Electrical Conductivity Measurements}: Weak electrical currents identify changes in liquid hold-up and differentiate flow patterns~\cite{hernandez2006fast,lamb1960measurement,van1985void,liu1993bstructure,jin2008design}.
    
    \item \textbf{Acoustic Techniques}: Ultrasonic sensors or acoustic impedance probes detect distinctive acoustic signatures of flow patterns~\cite{albion2007flow,xu2000acoustic,chung2004sound,gadiyaram2005acoustic}.
    
    \item \textbf{Optical Probes}: Quantify flow characteristics, such as velocity and turbulence, to identify patterns~\cite{yoon2006gas,polonsky1999relation,tu2002methodology,nydal1992statistical,andreussi1993void,martin2000slug}.
    
    \item \textbf{Gamma-Ray Transmission}: Measures attenuation profiles to differentiate flow patterns~\cite{xie2003flow,tortora2004capacitance,simons1995imaging,bieberle2006evaluation,hanus2022investigation}.
    
    \item \textbf{Electrical Capacitance Sensor}: Two-electrode capacitor detects flow patterns by analyzing electrical signal changes caused by material flow~\cite{williams1995process,abouelwafa1980use,geraets1988capacitance,tollefesn1998capacitance}.
    
    \item \textbf{Tomographic Methods}: Reconstruct cross-sectional images of flow behavior using data from multiple sensors around the system, including electrical, optical, gamma-ray, and X-ray tomography techniques~\cite{simons1995imaging,gamio2005visualisation,makkawi2002fluidization,reinecke1997multielectrode,yang2004adaptive,warsito2001measurement,dong2003application,ma2001application,lee2014electrical}.

\end{enumerate}


Data analysis methods play a vital role in understanding two-phase flow patterns, which are essential for various industrial processes and engineering applications. Using analytical techniques, valuable insights can be gained into two-phase flows, which facilitate informed decision making and process optimization. Among the methods used are flow pattern maps, image analysis, statistical analysis, neural networks, pattern recognition, and artificial intelligence (AI), each of which contributes to understanding the characteristics of the flow in two phases.

Flow pattern maps serve as graphical representations that classify different flow patterns based on experimental observations. Using dimensionless parameters, such as the Weber number and superficial velocities, these maps define boundaries between flow behaviors, allowing comparison between conditions~\cite{songsiri2004tow,barnea1986transition,mandhane1974flow,spedding1980regime,stanislav1986intermittent}. Image analysis techniques involve the capture of visual data with high-speed cameras, processing these data to extract information on bubble size, velocity, void fraction, and shape, helping to identify flow patterns~\cite{al2020systematic,al2020bsimplified,al2008development,almutairi2020ect}.

Statistical analysis employs methods such as cluster analysis, principal component analysis (PCA), and discriminant analysis to explore experimental data. Flow patterns can be classified according to statistical properties, offering quantitative classification means~\cite{al2020systematic,van2001evolution,nydal1992statistical,matsumoto1984statistical,chakraborty2020characterisation,ameel2012classification,yang2017application,nie2022image,trafalis2005two}. 

There is a body of research that has explored the application of AI techniques to classify flow patterns, such as support vector machines (SVM), decision trees, and artificial neural networks (ANNs)~\cite{nie2022image,trafalis2005two,sun2023comparative,li2023gas,howard2007pattern,mi1998vertical,hernandez2006fast,xie2003flow,xie2004artificial,chakraborty2020characterisation,yang2017application}. These methods consider input variables such as pressure drop, void fraction, and flow rate to capture complex relationships in flows. Moreover, AI, in particular deep learning, has also been used in combination with computer vision to detect flow patterns based on images of flows. In signal processing, AI algorithms have been tested on sensor data to classify flow patterns; data fusion has also been used to integrate information from multiple sources, enhancing the understanding of two-phase flow behaviors~\cite{klein2004sensor,hall1997introduction,zhang2014data,barbariol2020sensor}. However, previous work requires either high-speed camera or many sensors to obtain input data which are expensive and difficult to deploy in practice. On top of that, the exploration of machine learning models and the number of flow pattern types in those work remains preliminary. In this work, we only utilize data from two sensors and 5 second time-series signals to classify 7 types of flow patterns by 1D convolutional neural networks, compared with other machine learning models.

It should be noted that the methods reviewed in this section are often used in combination to provide a more comprehensive analysis of the two-phase flow behavior. The choice of method depends on the available data, the specific research objectives, and the desired level of accuracy and detail in flow pattern classification.