\begin{figure*}[t]% [hpbt] what you need
    \centering
     \begin{subfigure}[t]{0.32\textwidth}
        \includegraphics[width=\linewidth]{figures/k_ablations_knn.png}
        \caption{Influence of $k$.}
        \label{fig:ablation_k_plot}
    \end{subfigure}
~
     \begin{subfigure}[t]{0.32\textwidth}
        \includegraphics[width=\linewidth]{figures/num_to_replace_ablations_knn.png}
        \caption{Influence of $c$.}
        \label{fig:ablation_c_plot}
    \end{subfigure}
~
     \begin{subfigure}[t]{0.32\textwidth}
        \includegraphics[width=\linewidth]{figures/l_delete_ablation_knn.png}
        \caption{Influence of $l_{d}$.}
        \label{fig:ablation_k_plot}
    \end{subfigure}
        \vspace{-5pt}
    \caption{Recall@10 with in-place deletions at each step on MSTuring-30M-clustered
    with varying values of $k$, $c$ and $l_{d}$.}
\end{figure*}

\begin{table*}[h]
\centering
     \centering
     \begin{subtable}[t]{0.32\textwidth}
        \begin{tabular}{c|ccc}
            \toprule
              & k=10 & k=50 & k=100 \\
                \midrule
                Recall@10     & 91.8    & 92.5    & 92.6     \\
                Deletion time & 2983    & 3082    & 3099    \\
                \bottomrule
        \end{tabular}
        \caption{Influence of $k$.}
        \label{tab:ablation_k_table}
    \end{subtable}
~
    \begin{subtable}[t]{0.32\textwidth}
        \begin{tabular}{c|cccc}
            \toprule
                  & c=1 & c=2 & c=3 & c=5 \\
            \midrule
            Recall@10     & 91.0    & 92.2    & 92.5  & 92.8 \\
            Deletion time & 2400    & 2740    & 3148  & 3753 \\
            \bottomrule
        \end{tabular}
        \caption{Influence of $c$.}
        \label{tab:ablation_c_table}
     \end{subtable}
~~~
    \begin{subtable}[t]{0.32\textwidth}
        \begin{tabular}{c|ccc}
            \toprule
              & $l_d$=60 & $l_d$=128 & $l_d$=200 \\
            \midrule
            Recall@10     & 91.8    & 92.6    & 92.9   \\
            Deletion time & 1898    & 3160    & 4576   \\
            \bottomrule
        \end{tabular}
        \caption{Influence of  $l_d$.}
        \label{tab:ablation_l_delete_table}
    \end{subtable}
    \vspace{-5pt}
    \caption{Average recall (Recall@10) and total deletion time (in seconds)
        on the MSTuring-30M-clustered runbook.}
\end{table*}



\section{Ablation studies}
\label{sec:ablation}
In this section, we present ablation studies to examine the influence
of the parameters used in the in-place deletion algorithm. We vary the values of $k$,
$c$, $l_d$ in Algorithm~\ref{alg:deletion}, and the consolidation threshold $t$
in Algorithm~\ref{alg:consolidation_ours}.

As we increased the candidate list size $k$ from $10$ to $50$ to $100$, we found that the
average recall increases as expected (Figure~\ref{fig:ablation_k_plot}).
This is because retrieving more candidates makes it more likely the best
edges are found close to the deleted points and the estimated in-neighborhood
of a deleted point better aligns with the actual one. Additionally,
 more candidates to select replacement edges distributes the new edges across more nodes,
 reducing the likelihood of any single node triggering the pruning procedure. 
However, a larger $k$ requires more distance comparisons to select the top-$c$
closest points among the $k$ candidates, thereby increasing the deletion time (Table~\ref{tab:ablation_k_table}).


As we increase the number of replacement edges per delete by increasing
$c$ from $1$ to $5$, we found that the average recall increases significantly
from $c$=1 to $c$=2, but the benefit saturates as $c$ reaches $5$ (Figure~\ref{fig:ablation_c_plot}).
This  is expected, as introducing more edges increases graph connectivity.
The time spent on deletion increases almost linearly with $c$  (Table~\ref{tab:ablation_c_table}).
This is due to an increase in the number of prunes per delete, which in turn slows down the deletion process.
We found $c$=3 to be a good trade-off between accuracy and increased deletion time.


We varied the search parameter $l_d$ used inside in-place deletion between $60,128,200$.
We  found that the recall increases gradually with $l_d$ as the  larger beam size 
makes it more likely that the algorithm  finds the deleted point's in-neighbors.
However, a larger $l_d$ also significantly slows down the deletion function.
We observe that recall stops increasing when $l_d$ reaches $200$,
as the high maintenance cost outweighs the marginal increase in recall.



\begin{figure}
    \includegraphics[width=0.45\textwidth]{figures/consolidate_ablation_knn.png}
    % \hfill
    % \includegraphics[width=0.33\textwidth]{figures/consolidate_ablation_dc.png}
    % \hfill
    % \includegraphics[width=0.33\textwidth]{figures/consolidate_ablation_qps.png}
    \vspace{-10pt}
    \caption{Recall@10 for varying consolidation threshold $t$ on the MSTuring-30M-Clustered runbook.}
    \label{fig:ablation_t_plot}
\end{figure}

\begin{table}[h]
\vspace{-10pt}
\begin{tabular}{c|ccc}
\toprule
 & t=30\% & t=20\% & t=10\% \\
\midrule
Recall        & 92.2    & 92.5    & 92.6  \\
Deletion time & 3136   & 3146    & 3207   \\
\bottomrule
\end{tabular}
\caption{Average recall (Recall@10) and total deletion time (in seconds) for various
values of consolidation threshold $t$ on the MSTuring-30M-clustered runbook.
\vspace{-10pt}}
\label{tab:ablation_t_table}
\end{table}

We varied the consolidation threshold to trigger our light-weight consolidation
in Algorithm~\ref{alg:consolidation_ours} between $t$=10\%/20\%/30\%.
The average recall increases as we perform consolidation more frequently (Figure~\ref{fig:ablation_t_plot}).
This is because fewer dangling edges make is less likely that the greedy search reaches dead ends.
At the same time, deletion time slightly increases as we consolidate more often (Table~\ref{tab:ablation_t_table}). 







%Like all systems conference papers and many KDD paper, it's best to put Related Work at the end. None of it is needed to understand
% the previous sections. And the detailed description of FreshDiskANN in Algorithm section avoids the need to repeat it in Related Work.