
\section{Preliminaries and Notation}
\label{sec:prelim}
We define the streaming nearest neighbor problem as follows.
We are given a time-varying dataset of \emph{points}, where $P_t$ denotes the dataset at time $t$.
Each point $p$ has an associated vector, denoted by subscripted variable, such as $x_p$.
We sometimes use the terms point and vertex interchangeably, since they are in 1:1 correspondence.

The algorithm needs to support three operations:
\begin{itemize}
\item Insert a point $p$ to the current point set $P_t$.
\item Delete an existing point $p$ from the current point set $P_t$.
% \todo{harsha: how about replacements?}
\item Search for the nearest neighbors of a given query $q$ in the current point set $P_t$.
\end{itemize}

The degree of the graph, i.e., maximum number of out-neighbors of a node in the graph, is denoted $R$.
We measure the accuracy of the index over $P_t$ for a query using the following notion of recall.
\begin{definition}[Recall@k]
Given a point set $P_t$ and query $q$, let $G\subseteq P_t$ be the set of actual nearest $k$ points to $q$,
and $A\subseteq P_t$ be the set of $k$ vectors returned by an ANN algorithm.
The Recall@k for this algorithm in answering the query is defined as $\frac{|G\cap A|}{k}$.
\end{definition}
As is the standard practice, we report recall@10 in this paper.

The performance of an index is measured by:
\begin{itemize}
\item the recall-query complexity trade-off for query answering.
Query complexity can be measured in terms of the latency (or throughput)
of an implementation on certain machine. It can also be measured in a hardware
and implementation-agnostic way by counting the number of points in the
index that the query is compared with. We report whichever is best suited
to the experiment's context.
\item the complexity of handling insertions and deletions.
%\todo{pb: What about number of comparisons to execute a query?}\xnote{I think we have already mentioned recall-latency tradeoff. Do we still need recall-distance computation tradeoff?}
\end{itemize}

We denote a graph-based index over point set $P_t$ as the directed graph $G(P_t,E_t)$ with slight overloading of notation.
Each vertex in the graph corresponds to a vector in $P_t$. 
For each point $p\in P_t$, we use $N_{out}(p)$ to denote the out-neighbors of vertex $p$
and $N_{in}(p)$ to denote its in-neighbors.
The in-neighbors will not necessarily be maintained in the algorithms, but are useful in the analysis.

We use $x_q$ to denote the co-ordinates of a query vector $q$.
We use $d(x_p,x_q)$ to denote the distance between vectors $p$ and $q$.
For a set of vertices $V$, we use $|V|$ to denote the size of the vertex set.
