\section{Conclusion}
In this work, we set out to determine whether a simple DC pipeline could achieve the performance of other recent clustering frameworks. Specifically, we aimed to achieve this without relying on text, as we argue that doing so would make image-based clustering much more practical for downstream applications. This is because image-pair datasets are highly unlikely to be readily available to practitioners for most real-world use cases. Moreover, we aimed to develop a method that is simple to train without the need for additional feature layers, support sets, teacher-student networks, or exponential moving averages.

To meet this desiderata we proposed SCP, a novel, simple, and lightweight clustering method that achieves SOTA and near-SOTA performance in several datasets such as STL-10, CIFAR-10, CIFAR-20, ImageNet-10, and ImageNet-Dogs. In particular, SCP leverages pre-trained CLIP and DINO backbones, and we anticipate these could be updated in the future with better models, or ensembles.


\paragraph{Limitations} Despite these promising results, several challenges remain. Firstly, our pipeline relies on data augmentations, which can be time-consuming when scaling to larger datasets such as ImageNet-1K. Also, our approach requires prior knowledge of the number of clusters, which can be difficult to determine in practical applications unless the practitioner poses strong domain knowledge. Lastly, our frameworks performance is bounded by the quality of the pre-trained model used. However, we anticipate these to continue to improve in the future.




