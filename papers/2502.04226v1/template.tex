\documentclass{article}



\usepackage{arxiv}
\usepackage[numbers]{natbib}
\usepackage[utf8]{inputenc} % allow utf-8 input
\usepackage[T1]{fontenc}    % use 8-bit T1 fonts
\usepackage{hyperref}       % hyperlinks
\usepackage{url}            % simple URL typesetting
\usepackage{booktabs}       % professional-quality tables
\usepackage{amsfonts}       % blackboard math symbols
\usepackage{nicefrac}       % compact symbols for 1/2, etc.
\usepackage{microtype}      % microtypography
\usepackage{lipsum}		% Can be removed after putting your text content
\usepackage{graphicx}
\usepackage{natbib}
\usepackage{doi}
\usepackage{amsmath}
\usepackage{amssymb}
\usepackage{bm}
\usepackage{multirow}
\usepackage{algorithm}
\usepackage{algpseudocode}
\usepackage{float}
\usepackage{pifont}
\usepackage{subcaption}
\usepackage{mathtools}
\usepackage{amsthm}
\usepackage{blindtext}
\usepackage{xcolor}

\newcommand{\eqdef}{\ensuremath{\,\raisebox{-1.2pt}{${\stackrel{\mbox{\upshape \scalebox{.42}{def.}}}{=}}$}}\,}
\newcommand{\eqorder}{\sim} %{\ensuremath{\,\raisebox{-1pt}{$\stackrel{\mbox{\upshape\tiny $\order$}}{\sim}$}}\,}
% \newcommand{\eqdef}{\stackrel{\mathclap{\tiny\mbox{def.}}}{=}}

% Todonotes is useful during development; simply uncomment the next line
%    and comment out the line below the next line to turn off comments
%\usepackage[disable,textsize=tiny]{todonotes}


\definecolor{deepjunglegreen}{rgb}{0.0, 0.29, 0.29}

\usepackage[textsize=tiny]{todonotes}



\def\msquare{\mathord{\scalerel*{\Box}{gX}}}
\theoremstyle{plain}
\newtheorem{theorem}{Theorem}[section]
\newtheorem{proposition}[theorem]{Proposition}
\newtheorem{lemma}[theorem]{Lemma}
\newtheorem{corollary}[theorem]{Corollary}
\theoremstyle{definition}
\newtheorem{definition}[theorem]{Definition}
\newtheorem{assumption}[theorem]{Assumption}
\theoremstyle{remark}
\newtheorem{remark}[theorem]{Remark}

\newcommand{\first}[1]{\textbf{\textcolor{red}{\textbf{{#1}}}}}
\newcommand{\second}[1]{\textbf{\textcolor{blue}{\underline{{#1}}}}}
\newcommand{\third}[1]{\textbf{\textcolor{violet}{#1}}}






\title{Keep it Light! Simplifying Image Clustering via Text-Free Adapters}

%\date{September 9, 1985}	% Here you can change the date presented in the paper title
%\date{} 					% Or removing it

\author{
    Yicen Li$^{1,2}$\thanks{Corresponding author. Email: \texttt{li2642@mcmaster.ca}}, 
    Haitz Sáez de Ocáriz Borde$^{3}$, 
    Anastasis Kratsios$^{1,2}$\thanks{Corresponding author. Email: \texttt{kratsioa@mcmaster.ca}}, 
    Paul D. McNicholas$^{1,2}$ \\
    $^{1}$Department of Mathematics and Statistics, McMaster University, Hamilton, Canada \\
    $^{2}$Vector Institute, Toronto, Canada \\
    $^{3}$University of Oxford, Oxford, United Kingdom \\
    \texttt{li2642@mcmaster.ca, kratsioa@mcmaster.ca}
}

% Uncomment to remove the date
%\date{}

% Uncomment to override  the `A preprint' in the header
% \renewcommand{\headeright}{A preprint}
% \renewcommand{\undertitle}{A preprint}
% \renewcommand{\shorttitle}{\textit{arXiv} Template}

%%% Add PDF metadata to help others organize their library
%%% Once the PDF is generated, you can check the metadata with
%%% $ pdfinfo template.pdf
\hypersetup{
pdftitle={A template for the arxiv style},
pdfsubject={q-bio.NC, q-bio.QM},
pdfauthor={David S.~Hippocampus, Elias D.~Striatum},
pdfkeywords={First keyword, Second keyword, More},
}


\begin{document}
\maketitle

\begin{abstract}
\begin{abstract}

To develop generalizable models in multi-agent reinforcement learning, recent approaches have been devoted to discovering task-independent skills for each agent, which generalize across tasks and facilitate agents' cooperation. However, particularly in partially observed settings, such approaches struggle with sample efficiency and generalization capabilities due to two primary challenges: (a) How to incorporate global states into coordinating the skills of different agents? (b) How to learn generalizable and consistent skill semantics when each agent only receives partial observations? To address these challenges, we propose a framework called \textbf{M}asked \textbf{A}utoencoders for \textbf{M}ulti-\textbf{A}gent \textbf{R}einforcement \textbf{L}earning (MA2RL), which encourages agents to infer unobserved entities by reconstructing entity-states from the entity perspective. The entity perspective helps MA2RL generalize to diverse tasks with varying agent numbers and action spaces. Specifically, we treat local entity-observations as masked contexts of the global entity-states, and MA2RL can infer the latent representation of dynamically masked entities, facilitating the assignment of task-independent skills and the learning of skill semantics. Extensive experiments demonstrate that MA2RL achieves significant improvements relative to state-of-the-art approaches, demonstrating extraordinary performance, remarkable zero-shot generalization capabilities and advantageous transferability.

 % Additional rewards transform the original MTRL problem into a multi-objective MTRL problem, and the coupling relationship between the outputs of SP and ACP further complicates the optimization process. To solve this challenge, TSAC assigns a virtual expected budget to convert the multi-objective MTRL into a constrained single-objective formulation and then employs the Lagrangian method to transform a constrained single-objective optimization into an unconstrained one. The multiplier in the Lagrangian method automatically adjusts the weights during the training process, promoting cooperation between SP and ACP.
\end{abstract}
\begin{IEEEImpStatement}
The Current policies trained by Multi-Agent Reinforcement Learning (MARL) predominantly rely on meticulously designed structured environments, which considerably constrain the agents' generalization capabilities across multitasking and cross-task skill reuse. In this paper, we design a novel masked autoencoders for MARL to coordinate the skills of different agents and learn generalizable and consistent skill semantics when each agent only receives partial observations. Experimental results demonstrate that our proposed MA2RL framework significantly enhances both the asymptotic performance and generalization capabilities of the generalizable models. Specifically, MA2RL introduces masked autoencoders tailored for MARL, aimed at enhancing generalizable models. The framework holds promise for inspiring further explorations into the generalization of multi-agent reinforcement learning.
\end{IEEEImpStatement}


% Note that keywords are not normally used for peerreview papers.
\begin{IEEEkeywords}
Multi-Agent reinforcement learning, generalization, self-supervised learning.
\end{IEEEkeywords}


\IEEEpeerreviewmaketitle
\end{abstract}





% 
% 
The widespread integration of communication networks and smart devices in modern control systems has increased the vulnerability of industrial systems to online cyber-attacks, e.g., Industroyer, Blackenergy, etc \citep{osti_1505628}.
% Modern control systems have seen a large push to include communication networks and smart devices to increase performance, made possible by improvements in communication device cost and energy consumption. This trend has been coupled with the usage of open-standard communication protocols among industrial control systems, making them vulnerable to online cyber-attacks such as Industroyer, Blackenergy, etc \citep{osti_1505628}. 
To counter this, methods have been developed to improve security by achieving attack detection, mitigation, and monitoring, among others \citep{sandberg2022secure}. This paper focuses on active attack diagnosis to mitigate stealthy attacks. 
%
%\subsection{Literature review}

Active diagnosis techniques rely on the inclusion of additional moduli to control systems
% inclusion within the control system of additional moduli 
to alter the behavior of the system compared to information known by the attacker. 
For instance, the concept of additive watermarking was introduced in \cite{mo2015physical}, where noise signals of known mean and variance are added at the plant and compensated for it at the controller. 
This compensation, however, is not exact, causing some performance degradation. Thus, trade-offs between performance and detectability  are necessary \citep{zhu2023detection}.
% A later work \citep{zhu2023detection} designs the watermark signal by trading performance for detection. Thus, although additive watermarking serves as a good detection scheme, they endure performance losses even in the nominal case. 

In encrypted control \citep{darup2021encrypted}, the sensor data is encrypted, sent to the controller, and then operated on directly. Encrypted input signals are sent back to the plant for decryption. Although encryption is widespread in IT security, in control systems it presents some concerns, such as the introduction of time delays \citep{stabile2024verifiable}, while it may present inherent weaknesses \citep{alisic2023model}.
% they are not preferred as they introduce time delays \citep{stabile2024verifiable} which can cause instability, and some encryption schemes can be very weak  \citep{alisic2023model}. 

In moving target defense \citep{griffioen2020moving}, the plant is augmented with fictitious dynamics, known to the controller. The plant output is transmitted to the controller along with the fictitious states over a network under attack. 
The additional measurements then aide in the detection of attacks. 
This comes at the cost of higher communication bandwidth needs, which increases rapidly with the dimension of the augmented systems.
% Since the dynamics of the fictitious dynamics are exactly known to the controller, the attack is detected easily. However, when the scale of the system increases, the communication bandwidth used by moving the target defense approach increases rapidly. 

Other recently proposed works include two-way coding \citep{fang2019two}, a weak encryuption technique, and dynamic masking \citep{abdalmoaty2023privacy}, which enhances privacy as well as security, have been shown to be effective against zero-dynamics attacks.
% Two-way coding \citep{fang2019two} and dynamic masking \citep{abdalmoaty2023privacy} are other recently proposed approaches. Two-way coding is another form of weak encryption technique whilst dynamic masking proposes an architecture that enhances both privacy and security. These schemes are shown to be effective against zero dynamics attacks but remain to be studied for other classes of attacks. 
% Recent extensions include \citep{mukherjee2021secure,ramos2024privacy}.
% Some other works which are related are \citep{mukherjee2021secure}, an extension of \cite{fang2019two}. The work \citep{ramos2024privacy} is an extension of moving target defense for multi-agent systems. 
Furthermore, filtering techniques for attack detection are proposed by \cite{murguia2020security,hashemi2022codesign,escudero2023safety}, while not focusing on stealthy attacks.
% The works \citep{murguia2020security,hashemi2022codesign,escudero2023safety} develop filtering techniques to guarantee safety, without being focused on stealthy covert attacks.

Multiplicative watermarking (mWM) has been proposed by the authors as a diagnosis technique \citep{ferrari2020switching}. mWM consists of a pair of filters on each communication channel between the plant and its controller; the scheme is affine to weak encryption, whereby ``encoding'' and ``decoding'' are done by changing signals' dynamic characteristics through inverse pairs of filters. This enables original signals to be recovered exactly, and thus does not lead to performance degradation.
% A multiplicative watermark is an affine to a weak encryption technique, through which the signal is ``encoded'' by a filter, changing its dynamic behavior. The use of inverse pairs means that the original signal can be recovered, through ``decoding'' via an inverse filter. As such, differently to techniques based on additive watermarking, no performance is lost due to the injection of noise, and there are no bandwidth limitations.

%\subsection{Contributions}
One of the critical features of multiplicative watermarking is that to detect stealthy attacks, the mWM filter parameters must be switched over time. In this paper, an algorithm to optimally design the mWM parameters after a switching event is presented, enhancing detection performance, without changing the switching time.
% This is done without changing the switching time, which is taken as given.

\textcolor{black}{
To formalize the filter design problem, we suppose the defender is interested in optimal performance against adversaries injecting covert attacks with matched system parameters \citep{smith2015covert}, including the mWM parameters prior to the switch. This scenario represents a worst case where malicious agents can take full control of the system while remaining undetected.
Thus, the attack strategy is explicitly included within the formulation of the closed-loop system, and the mWM filters are chosen by solving an optimization problem minimizing the attack-energy-constrained output-to-output gain (AEC-OOG) \citep{anand2023risk}, a variation of the output-to-output gain proposed in  \cite{teixeira2015strategic}.
}
The main contributions of this paper are:
% We consider an adversary injecting a covert attack with matched system parameters \citep{smith2015covert}, i.e., an attacker with full knowledge of the control system parameters, including those of the mWM filters before the switch. This scenario is taken as a worst case, as it has been shown that this class of attacks can be made stealthy. To quantitatively define a cost, the output-to-output gain (OOG) \citep{teixeira2015strategic} is leveraged,
% a metric introduced to evaluate the impact of an additive attack in a control system. %Specifically, OOG evaluates the worst-case performance loss that an attacker injecting an undetectable attack can obtain. 
% Here, the maximum performance loss caused by a stealthy adversary with limited energy is taken, the attack-energy-constrained OOG (AEC-OOG) \citep{anand2023risk}. The main contributions of this paper are:
\begin{enumerate}
%[label=\alph*.]
\item The problem of optimally designing the switching mWM filters is formulated as an optimization problem, with the AEC-OOG is taken as the objective;%where the AEC-OOG is taken as the impact metric; 
\item The worst-case scenario of a covert attack with exact knowledge of plant and mWM filter parameters is embedded within the design problem;
% The optimization problem is defined to incorporate the worst-case scenario of a covert attack with exact knowledge of plant and mWM filter parameters;
\item The feasibility of the optimization problem is shown to be dependent only on stability conditions; 
\item A solution scheme is proposed to promote randomization of the mWM filter parameters such that an eavesdropping adversary cannot remain stealthy.
\end{enumerate} 

This builds on the results of \cite{ferrari2020switching}, where the focus was on the design of the switching protocols, rather than the parameters themselves.
Compared to previous work \citep{gallo2021design}, this paper introduces an optimization problem which is always feasible (thanks to the use of AEC-OOG in the objective), while also considering a more sophisticated class of covert attacks, where the presence of watermark is known to the adversary. 
Moreover, this paper poses a different objective than \citep{zhang2023hybrid}; indeed, while \citep{zhang2023hybrid} provided a design strategy to ensure certain privacy properties, in this paper we address the problem of optimal parameter design following a switching event.


%\subsection{Organization}
The rest of the paper is organized as follows. 
After formulating the problem in Section~\ref{sec:PF}, we propose our design algorithm in Section~\ref{sec:main}, and analyze its properties. It is then evaluated through a numerical example in Section~\ref{sec:NE}, and concluding remarks are given Section~\ref{sec:Con}.
% We provide the problem background in Section~\ref{sec:PF}. We formulate the design problem in Section~\ref{sec:main}, together with an analysis of its properties. The proposed algorithm is evaluated through a numerical example in Section \ref{sec:NE}. Concluding remarks are offered in Section \ref{sec:Con}.
\section{Related Work}
In this section, we provide a broad overview of self-supervised learning research that has inspired our work, along with recent trends in image clustering using pre-trained models.


\subsection{Self-Supervised Learning}
Self-supervised learning learns representations from data without explicit labels. The objective is to create a representation space where positive pairs are closer together, while negative pairs are pushed farther apart \cite{geiping2023cookbook}.

SimCLR \cite{chen2020simple} uses data augmentations, such as flipping and colour jittering, to create positive and negative pairs for optimizing objectives. It also introduces a projection head that maps embeddings into a space where contrastive loss is applied. BYOL \cite{grill2020bootstrap} shows that high-quality representations can be learned by simply maximizing agreement between two augmented views of the same input, without requiring negative pairs. Building on these advancements, SimSiam \cite{chen2020exploringsimplesiameserepresentation} eliminates the need for both negative pairs and momentum encoders by introducing a stop-gradient operation, which effectively prevents representational collapse. Inspired by these methods, we adopt similar ideas to develop a simple and effective self-supervised framework for image clustering.

\subsection{Pre-trained Models in Vision} 
Building on advances in self-supervised learning, CLIP \cite{radford2021learning} introduced a paradigm of contrastive pre-training that aligns images with corresponding textual descriptions. This approach enables broad task generalization without task-specific fine-tuning. DINO \cite{caron2021emerging}, which stands for self-distillation with no labels, demonstrates a self-supervised method for optimizing a student network from a teacher network based on vision input data only.

One of the key advantages of pre-trained models like CLIP is their ability to eliminate the need for training models from scratch for downstream tasks, significantly reducing computational costs and time. Instead of training a self-supervised neural network from the ground up, pre-trained models provide high-quality feature representations out of the box, leading to faster experimentation and improved performance on a variety of tasks. The scalability of CLIP has been further validated by openCLIP \cite{Cherti_2023}, which extended CLIP using the larger Vision Transformer models \cite{dosovitskiy2020image}. Similarly, models such as DINO~\cite{9709990} and DINOv2~\cite{oquab2024dinov2learningrobustvisual} are capable of processing visual data and mapping it to high-quality latent representations.

\subsection{Image Clustering via Pre-trained Models}
To address the challenges of scaling to modern image datasets, methods such as NMCE \cite{li2022neural} and MLC \cite{deng2023acp} have integrated deep learning with manifold clustering using the minimum coding rate principle \cite{Arthur_Vassilvitskii_2007}. Building on this idea, CPP \cite{chu2024image} further refines CLIP features and estimates the optimal number of clusters when unknown. TEMI \cite{adaloglou2023exploring} improves clustering by leveraging associations between image features, introducing a variant of pointwise mutual information with instance weighting. Unlike our approach, TEMI utilizes a nearest-neighbors set and an exponential moving average for parameter optimization.

SIC \cite{cai2023semantic} leverages multi-modality by mapping images to a semantic space and generating pseudo-labels based on image-semantic relationships. More recently, TAC~\cite{li2023image} utilizes the textual semantics of WordNet~\cite{miller1995wordnet} to enhance image clustering by selecting and retrieving nouns that best distinguish the images, facilitating collaboration between text and image modalities through mutual cross-modal neighborhood distillation.

Current pre-trained approaches often rely on heavy or complex architectures to ensure consistency, motivating us to develop a simple yet effective pipeline for image clustering. Our method requires only a simple clustering head and basic data augmentations, demonstrating strong competitiveness among recent models.








\section{Our Method}
\label{sec:Our Method}
Recent methods such as CC \cite{li2021contrastive} and CPP \cite{chu2024image} decouple the latent space into clustering and feature spaces. TAC \cite{li2023image} utilizes text information, and TEMI \cite{adaloglou2023exploring} employs self-distillation networks to enhance clustering. In contrast, our approach remains simple and efficient without relying on these techniques. As shown in Fig.~\ref{fig:AE}, SCP consists of only two components: a pre-trained frozen backbone for pair construction, denoted as $f(.)$, and a trainable cluster head $g(.)$.

\begin{figure}[ht]
    \centering
    \includegraphics[width=0.6\columnwidth]{images/scp.png}
    \caption{%
    A overall pipeline for SCP. During training, two augmented views \(T^a\) and \(T^b\) of an image are generated from the dataset and processed by a frozen feature extractor \(f\) and a trainable cluster head \(g\) (a five-layer MLP). The objective is to minimize the cross-entropy loss between the outputs of the cluster head \(g\) for the two augmented views.
    }
    \label{fig:AE}
\end{figure}

Briefly, SCP performs data augmentations and extracts features from the augmented images using pre-trained models. The cluster head then projects these features into a cluster space, where the dimension equals the number of clusters. After training, outputs in the cluster space provide the soft assignments for clustering.


\subsection{Pair Construction Backbone}

The success of BYOL demonstrates that we can maximize the similarities of positive pairs without negative ones. In SCP, the positive pairs consist of samples augmented from the same instance.

Given a data instance \(x_i\), we apply two stochastic transformations \(T^a\) and \(T^b\), independently selected from the augmentation family \(\mathcal{T}\). This produces two correlated views: \(x_i^a = T^a(x_i)\) and \(x_i^b = T^b(x_i)\). 

An appropriate augmentation strategy is vital for better downstream performance. In our work, we adopt only two simple augmentations: \texttt{RandomCrop} and \texttt{GaussianBlur}. This choice aligns with the preprocessing techniques used in training pre-trained models, ensuring compatibility with their learned representations. \texttt{RandomCrop} randomly crops the image to a specified size, and \texttt{GaussianBlur} applies a Gaussian filter to blur the image. For each image, these two augmentations are applied independently, each with a 50\% probability. We then use a pre-trained model \(f(\cdot)\), such as CLIP, to extract features from the augmented images: $h_i^a = f(x_i^a)$ and $h_i^b = f(x_i^b).$

\subsection{Cluster Head}
Following the “label as representation” concept \cite{li2021contrastive}, when a data sample is projected into a space whose dimensionality matches the number of clusters \(K\), the \(k\)-th component of its feature vector (after applying a $\operatorname{softmax}$  function) can be interpreted as the probability that the sample belongs to the \(k\)-th cluster. We employ a five-layer non-linear MLP as the clustering head \(g(\cdot)\), producing a  \(K\)-dimensional feature that is normalized with a $\operatorname{softmax}$  over the dimension of the cluster.
\[
y_i^a = g(h_i^a), 
\quad 
y_i^b = g(h_i^b).
\]
Hence, \(y_i^a\) and \(y_i^b\) are both \(K\)-dimensional vectors, whose components \(y_{i,k}^a\) and \(y_{i,k}^b\) indicate the probability of assigning the \(i\)-th sample to the \(k\)-th cluster. Formally, let \(Y^a, Y^b \in \mathbb{R}^{N \times K}\) be the outputs of the clustering head for all samples. Then, we have the following matrices:
\[
% Y^a = \begin{bmatrix}
% y^a_1 \\
% y^a_2 \\
% \vdots \\
% y^a_N
% \end{bmatrix}
% \quad
% Y^b = \begin{bmatrix}
% y^b_1 \\
% y^b_2 \\
% \vdots \\
% y^b_N
% \end{bmatrix}.
Y^a = \begin{bmatrix}
y^i_1 \\
% y^a_2 \\
\vdots \\
y^a_N
\end{bmatrix}
\quad
Y^b = \begin{bmatrix}
y^b_1 \\
% y^b_2 \\
\vdots \\
y^b_N
\end{bmatrix}.
\]
To maximize row-wise similarity, we adopt the following cross-entropy loss function instead of the commonly used InfoNCE loss \cite{oord2018representation}, as SCP only have positive pairs that should share similar soft assignments:
\begin{equation}
L_e = - \sum_{i=1}^{N} \sum_{k=1}^{K} y^{a}_{i,k} \log y^{b}_{i,k}.
\end{equation}
Inspired by the effective regularizations in TAC \cite{li2023image}, we further introduce the following confidence loss to make the soft labels \( y^{a}_i \) and \( y^{b}_i \) more confident, approaching one-hot vectors:
\begin{equation}
L_{\text{con}} = - \log \sum_{i=1}^N {y^a_i}^\top y^b_i.
\end{equation}
This loss ensures that the cluster head assigns higher probabilities to its top predicted clusters, thereby increasing confidence in the assignments. 

In addition, following TAC \cite{li2023image}, we introduce an entropy term \( H(Y) \) to prevent model collapse, defined as follows:
\begin{equation}
H(Y) = - \sum_{k=1}^{K} \left[ P^{a}_k \log P^{a}_k + P^{b}_k \log P^{b}_k \right],
\end{equation}
where
\[
P^{a}_k = \frac{1}{N} \sum_{i=1}^{N} y^{a}_{i,k}, \quad P^{b}_k = \frac{1}{N} \sum_{i=1}^{N} y^{b}_{i,k}.
\]
$H(Y)$ encourages uniform soft assignments across clusters, thereby mitigating the issue of empty clusters. 

Hence, we define the overall objective function of SCP as
\begin{equation}
L_{\text{clu}} = L_e + L_{\text{con}} - \alpha H(Y),
\end{equation}
where the balancing weight $\alpha$ modulates the influence of $H(Y)$, especially when the number of clusters is large. By maximizing consistency between different augmented views with regularizations, SCP effectively prevents trivial solutions and achieves competitive performance. We provide algorithm \ref{alg:CAC} to explain our pipeline.

\begin{algorithm}[H]
\caption{Simple Clustering via Pre-trained Models (SCP)}
\label{alg:CAC}
\begin{algorithmic}[1]
    \Require Dataset $\mathcal{X} = \{ x_i \}_{i=1}^{N}$, Pre-trained model $f(\cdot)$, number of clusters $K$, batch size $B$, loss weight $\alpha$
    \State Initialize cluster head $g(\cdot)$
    \For{each epoch}
        \For{each mini-batch $\{x_i\}_{i=1}^{B}$}
            \State \textbf{Pair Construction:}
            \For{each data instance $x_i$ in the mini-batch}
                \State Apply stochastic transformations $T^a$, $T^b$ to obtain:
                \State \quad $x_i^a = T^a(x_i)$, \quad $x_i^b = T^b(x_i)$
                \State Extract features using pre-trained model:
                \State \quad $h_i^a = f(x_i^a)$, \quad $h_i^b = f(x_i^b)$
            \EndFor
            \State \textbf{Cluster Space Encoding:}
            \For{each feature $h_i^a$, $h_i^b$}
                \State Compute soft assignments: \quad $y_i^a = g(h_i^a)$, \quad $y_i^b = g(h_i^b)$
            \EndFor
            \State \textbf{Compute Losses:}
            \State Compute total clustering loss:
            \State \quad $L_{\text{clu}} = L_e + L_{\text{con}} - \alpha H(Y)$
            \State \textbf{Update} cluster head $g(\cdot)$ parameters by minimizing $L_{\text{clu}}$
        \EndFor
    \EndFor
    \State \Return soft assignments $y_i = g(h_i)$ for each $x_i \in \mathcal{X}$
\end{algorithmic}
\end{algorithm}


% Inspired by the work \cite{dwibedi2021little}, we recognize the advantages of incorporating positives from other instances in the dataset. Therefore, we construct a support set by sampling the nearest neighbours from the dataset in the CLIP feature space, treating them as additional positives. This approach provides extra semantic variations and aligns with the concept of stochastic neighbour embeddings. We assume that these nearest neighbours should be close enough as well after the projections.

% For each image feature \( h_i^a \), we find its \( M \) nearest neighbours \( \{ \tilde{h}_{i,j} \}_{j=1}^M \) in the set of \( \{ h_{i}^a \}_{i=1}^N \) and compute a weighted sum to obtain the aggregated neighbour representation \( \tilde{h}_i \):

% \begin{equation}
% \tilde{h}_i = \sum_{j=1}^M p(\tilde{h}_{i,j} \mid h_i^a) \, \tilde{h}_{i,j},
% \label{nnequ}
% \end{equation}

% where the weights \( p(\tilde{h}_{i,j} \mid h_i^a) \) are defined as:

% \begin{equation}
% p(\tilde{h}_{i,j} \mid h_i^a) = \frac{\exp\left( \text{sim}(h_i^a, \tilde{h}_{i,j}) / \tau \right)}{\sum_{k=1}^M \exp\left( \text{sim}(h_i^a, \tilde{h}_{i,k}) / \tau \right)}.
% \end{equation}

% Here, \( \text{sim}(\cdot, \cdot) \) denotes cosine similarity:

% \begin{equation}
% \text{sim}(h_i^a, \tilde{h}_{i,j}) = \frac{(h_i^a)^\top \tilde{h}_{i,j}}{\|h_i^a\| \, \|\tilde{h}_{i,j}\|},
% \end{equation}

% \( \tau \) is a temperature parameter, and \( M \) is the number of top nearest neighbours selected from  \( \{ h_{i}^a \}_{i=1}^N \). This weighting mechanism ensures that the influence of each neighbour is proportional to its similarity to the query image feature \( h_i^a \), preventing the nearest neighbours of different images from collapsing to the same point. We set the selected number of nearest neighbours as 10 for all experiments. To further clarify, this method allows each image to dynamically weigh its nearest neighbours based on similarity, enhancing the model's ability to capture fine-grained semantic relationships within the dataset. Finally, we incorporate these aggregated neighbour representations into the learning process by treating \( \tilde{h}_i \) as additional positive samples for \( h_i^a \). Thus, the support set loss is defined by:

% \begin{equation}
% L_s = - \sum_{i=1}^{N} \sum_{k=1}^{K} \tilde{y}_{i,k} \log y^{a}_{i,k},
% \end{equation}

% where \( \tilde{y}_{i} = g(\tilde{h}_{i}) \). This enhances the model's ability to learn from semantically nearest instances, improving clustering performance. To keep simplicity, we don't calculate the support set loss for another augmented view \( \{ h_{i}^b \}_{i=1}^N \).


% \subsection{Cluster Space Decoder}
% To encourage the encoder to learn meaningful and helpful latent representations and additionally avoid trivial solutions, we attach a decoder that aims to reconstruct the original image features. In summary, the cluster space decoder minimizes the reconstruction loss:

% \begin{equation}
% L_{d}=  - \sum_{i=1}^{N} \left( \text{sim}(h_i^a, h_{i}^{'a}) + \text{sim}(h_i^b, h_{i}^{'b}) \right),
% \end{equation}

% where $h_{i}^{'a} = d(y^{a}_{i})$ and $h_{i}^{'b} = d(y^{b}_{i})$, and $d(\cdot)$ denotes the decoder. $sim$ is still measured by cosine similarity.

% Finally, we arrive at the overall objective function of CAC, which takes the form:

% \begin{equation}
% L_{\text{CAC}} = \alpha \cdot L_{d} + L_s + L_{\text{clu}},
% \end{equation}

% where $\alpha = 0.01$ is a weight parameter set for all current experiments. This is small scale because it is expected that the decoder cannot perfectly reconstruct the same features, only knowing which cluster they belong to. 

% In section 4.5, we show our method is robust to different $\alpha$. Although the multi-task loss looks overwhelming there are fewer parameters to tune compared to previous works like \cite{li2023image}.

% In the end, we provide a pseudo code to better explain our pipeline:

% \begin{algorithm}[H]
% \caption{CAC: CLIP-based Auto-Encoder Clustering}
% \label{alg:CAC}
% \begin{algorithmic}[1]
% \Require Dataset $\mathcal{X} = \{ x_i \}_{i=1}^{N}$, number of clusters $K$, batch size $B$, number of nearest neighbors $M$, temperature parameter $\tau$, weight parameter $\alpha$
% \State Initialize encoder network $g(\cdot)$ and decoder network $d(\cdot)$
% \For{each epoch}
%     \For{each mini-batch $\{ x_i \}_{i=1}^{B}$}
%         \State \textbf{Pair Construction:}
%         \For{each data instance $x_i$ in the mini-batch}
%             \State Apply stochastic transformations $T^a$, $T^b$ to obtain:
%             \State \quad $x_i^a = T^a(x_i)$, \quad $x_i^b = T^b(x_i)$
%             \State Extract features using CLIP backbone:
%             \State \quad $h_i^a = f(x_i^a)$, \quad $h_i^b = f(x_i^b)$
%         \EndFor
%         \State \textbf{Cluster Space Encoding:}
%         \For{each feature $h_i^a$, $h_i^b$}
%             \State Compute soft labels: \quad $y_i^a = g(h_i^a)$, \quad $y_i^b = g(h_i^b)$
%         \EndFor
%         \For{each feature $h_i^a$}
%             \State Compute aggregated neighbor representation $h_i^N$
%             \State Compute soft label $y_i^N = g(h_i^N)$
%         \EndFor
%         \State \textbf{Cluster Space Decoding:}
%         \For{each soft label $y_i^a$, $y_i^b$}
%             \State Reconstruct features:
%             \State \quad $h_{i}^{'a} = d(y^{a}_{i})$, \quad $h_{i}^{'b} = d(y^{b}_{i})$
%         \EndFor
%         \State \textbf{Compute Total Loss:}
%         \State \quad $L_{\text{CAC}} = \alpha \cdot L_{d} + L_s + L_{\text{clu}}$
%         \State \textbf{Update} encoder $g(\cdot)$ and decoder $d(\cdot)$ parameters by minimizing $L_{\text{CAC}}$
%     \EndFor
% \EndFor
% \end{algorithmic}
% \end{algorithm}
% Clip-based Auto-Encoder Learning (CAEL) aims to learn a mapping from $R^{768}$ into $R^{512}$. We minimize the Euclidean distance in the latent space $R^512$ to preserve the original structure between two views of images because they represent the same semantic meaning:

% \begin{equation}
% l_1 = \frac{1}{N} \sum_{n=1}^{N}  \left\| g_{w}(h^a_n) - g_{w}(h^b_n) \right\|_2^2
% \end{equation}

% where the parameters $w$ of the encoder $g_w(\cdot)$ and parameter $\theta$ of the decoder $d_\theta(\cdot)$ are further jointly learned by optimizing the following problem to avoid collapse:

% \begin{equation}
% l_2 =  \frac{1}{2N} \sum_{n=1}^{N} \left( \left\|  h^{'a}_n - h^a_n \right\|_2^2 + \left\|  h^{'b}_n - h^b_n \right\|_2^2 \right),
% \end{equation}

% where $ h^{'a}_n = f_{\theta}(g_{W}(h^a_n))$, $ h^{'b}_n = f_{\theta}(g_{W}(h^b_n))$. 

% Furthermore, inspired by the pioneering work \cite{dwibedi2021little} as well as the similar idea of stochastic neighbour embedding, we construct the support set $S$ from the CLIP representations $h^b$, where $S = \{ {h^b}_n \}_{n=1}^N$. Thus, we aim to find the nearest neighbour of $h^a$ in the support set $S$, then minimize the Euclidean distance between the view $h^a$ and its nearest neighbour $S_{\text{near}}: h^a_{\text{near}} \in S$. (For now, it is not plotted in figure 2)

% \begin{equation}
% l_3 = \frac{1}{N} \sum_{n=1}^{N} \left\| g_{w}(S_{\text{near}}) - g_{w}(h^b_n) \right\|_2^2
% \end{equation}

% Thus, for $g_w(\cdot)$ and $d_\theta(\cdot)$, our method is trying to optimize the following objection functions:

% \begin{equation}
% \min_{w,\theta} l_1 + l_2 + \alpha l_3
% \end{equation}

% we set $\alpha = 0.1$ after analysis. 

% After sufficient training, such as 150 epochs on a benchmark dataset, for the images in the test set $Y$, we obtain their CLIP representations $Y_c = f(Y)$ without the need for data augmentation. We then use the encoder to embed $Y_c$ into $Z_c = g(Y_c)$. Finally, we perform K-means clustering on $Z_c$ to obtain the clusters $\mathbf{C}$, represented by an $M \times k$ one-hot matrix, where $k$ is the number of clusters and $M$ is the number of instances in the test set.

% 

% \begin{algorithm}
% \caption{Clip-based Auto-Encoder Learning (CAEL)}
% \begin{algorithmic}[1]
% \State \textbf{Input:} Dataset \(X\); Two augmented views \(x^a\) and \(x^b\); Number of epochs \(T\); Number of epochs \(T_{\text{refine}}\); Support set \(S = \{ h^b_n \}_{n=1}^{N}\); Weights \(\alpha\); CLIP model \(f(\cdot)\); Encoder \(g_w(\cdot)\); Decoder \(d_\theta(\cdot)\); Projector \(f_p(\cdot)\); Identity matrix \(I_k\)
% \State \textbf{Output:} Cluster assignments \(\mathbf{C}\)

% \vspace{0.3cm}
% -------------------------------------------------------------

% \State \textit{Phase 1: Training}
% \vspace{0.3cm}

% \State Initialize parameters \(w\) and \(\theta\)
% \For{each epoch \(t = 1\) to \(T\)}
%     \For{each image \(x \in X\)}
%         \State Obtain CLIP representations: \(h^a = f(x^a)\) and \(h^b = f(x^b)\)
        
%         \vspace{0.3cm}
%         \State Obtain latent representations: \(z^a = g_w(h^a)\) and \(z^b = g_w(h^b)\)
%          \State Compute loss \(l_1 = \frac{1}{N} \sum_{n=1}^{N} \left\| g_w(h^a_n) - g_w(h^b_n) \right\|_2^2\)

%           \vspace{0.3cm}
    
%         \State Obtain reconstruct representations: \(h^{'a} = d_\theta(z^a)\) and \(h^{'b} = d_\theta(z^b)\)
%           \State Compute loss \(l_2 = \frac{1}{2N} \sum_{n=1}^{N} \left( \left\| h^{'a}_n - h^a_n \right\|_2^2 + \left\| h^{'b}_n - h^b_n \right\|_2^2 \right)\)
    
%         \vspace{0.3cm}

%              \State Find nearest neighbor \(S_{\text{near}} = h^a_{\text{near}} \in S\)
%         \State Compute loss \(l_3 = \frac{1}{N} \sum_{n=1}^{N} \left\| g_w(S_{\text{near}}) - g_w(h^b_n) \right\|_2^2\)
        
%          \vspace{0.3cm}
        

        
%         \vspace{0.3cm}
        
%         \State Update parameters \(w\) and \(\theta\) by minimizing \(l = l_1 + l_2 + \alpha l_3\)
%         \vspace{0.3cm}
        
%     \EndFor
% \EndFor

% \vspace{0.3cm}
% -------------------------------------------------------------

% \State \textit{Phase 2: Clustering and refinement}
% \vspace{0.3cm}


% \State Obtain CLIP representations without data augmentation  \(H = f(X)\)
% \vspace{0.3cm}
% \State Obtain latent representations: \(Z_c = g_w(H)\)
% \vspace{0.3cm}
% \State Perform K-means clustering on \(Z_c\) to get initial clusters \(\mathbf{C}\)
% \vspace{0.3cm}

% \For{each epoch \(t = 1\) to \(T_{\text{refine}}\)}
%     \For{each latent representation \(z_n \in Z_c\)}
%     \vspace{0.3cm}
    
%         \State obtain \(k\)-simplex representation: \(P_c = f_p(Z_c)\)
%         \State Compute the loss \(l_{\text{refine}} = \frac{1}{M} \sum_{m=1}^{M} \left\| P_c - \mathbf{C}_{m \times k} \right\|_2^p\)
%         \vspace{0.3cm}
%         \State Update parameters of \(f_p\) to minimize \(l_{\text{refine}}\)
        
%         \vspace{0.3cm}
% \EndFor
% \EndFor
% \vspace{0.3cm}
% \State Reassign instances to clusters based on the nearest distance to identity matrix \(I_k\):
% \State \(C_k \gets \{z_n : \|P_c - \mathbf{e}_k\| \le \min_{j=1,\dots,k} \|P_c - \mathbf{e}_j\| \}\)
% \vspace{0.3cm}
% \State \Return \(\mathbf{C}\)
% \end{algorithmic}
% \end{algorithm}

In this section, we empirically compare the proposed algorithm on both sequence windows and time windows with existing methods.
\paragraph{Datasets} For the sequence-based model, we used two synthetic datasets and two cross-language datasets. The statistics of the datasets are provided in Table \ref{table:statistics}:

\begin{table}[t]
    \centering
    \caption{The statistics of the datasets. The datasets satisfy $1 \leq \|\vx\|\|\vy\| \leq R $.}
    \label{table:statistics}
    \begin{tabular}{|c|c|c|c|c|c|}
    \hline
        Dataset & $n$ & $m_x$ & $m_y$ & $N$ & $R$ \\ \hline
        SYNTHETIC(1) & 100,000 & 1,000 & 2,000 & 50,000 & 65 \\ \hline
        SYNTHETIC(2) & 100,000 & 1,000 & 2,000 & 50,000 & 724 \\ \hline
        APR & 23,235 & 28,017 & 42,833 & 10,000 & 773 \\ \hline
        PAN11 & 88,977 & 5,121 & 9,959 & 10,000 & 5,548 \\ \hline
        EURO & 475,834 & 7,247 & 8,768 & 100,000 & 107,840 \\ \hline
    \end{tabular}
\end{table}

\begin{itemize}
    \item Synthetic: The elements of the two synthetic datasets are initially uniformly sampled from the range (0,1), then multiplied by a coefficient to adjust the maximum column squared norm $R$. The X matrix has 1,000 rows, and the Y matrix has 2,000 rows, each with 100,000 columns. The window size is set to 50,000.
    \item APR: The Amazon Product Reviews (APR) dataset is a publicly available collection containing product reviews and related information from the Amazon website. This dataset consists of millions of sentences in both English and French. We structured it into a review matrix where the X matrix has 28,017 rows, and the Y matrix has 42,833 rows, with both matrices sharing 23,235 columns. The window size is 10,000.
    \item PAN11: PANPC-11 (PAN11) is a dataset designed for text analysis, particularly for tasks such as plagiarism detection, author identification, and near-duplicate detection. The dataset includes texts in English and French. The X and Y matrices contain 5,121 and 9,959 rows, respectively, with both matrices having 88,977 columns. The window size is 10,000.
\end{itemize}
We evaluate the time-based model on another real-world dataset:
\begin{itemize}
    \item EURO: The Europarl (EURO) dataset is a widely used multilingual parallel corpus, comprising the proceedings of the European Parliament. We selected a subset of its English and French text portions. The X and Y matrices contain 7,247 and 8,768 rows, respectively, and both matrices share 475,834 columns. Timestamps are generated using the $Poisson$ $Arrival$ $Process$ with a rate parameter of $\lambda=2$. The window size is set to 100,000, with approximately 30,000 columns of data on average in each window.
\end{itemize}

\paragraph{Setup} For the sequence-based model, we compare the proposed hDS-COD and  aDS-COD with EH-COD~\cite{yao2024approximate} and DI-COD~\cite{yao2024approximate}. We do not consider the Sampling algorithm as a baseline, as its performance is inferior to that of EH-COD and DI-CID, as demonstrated in \cite{yao2024approximate}. %The hDS-COD is adjusted by the parameter $\ell$ and the maximum number of levels $L = \log{R}$, where $R$ is the prior estimate of the maximum squared column norm of the dataset. DI-COD similarly requires a prior estimate of $R$ to limit the maximum number of levels $L = \log{(R/\varepsilon})$. In contrast, aDS-COD and EH-COD do not require an estimate of $R$; their error-space balance is controlled by the parameter $\ell = \frac{1}{\varepsilon}$. 
For the time-based model, we compare the proposed hDS-COD and  aDS-COD with EH-COD and the Sampling algorithm since DI-COD cannot be applied to time-based sliding window model. To achieve the same error bound, the maximum number of levels for hDS-COD is set to $L = \log{(\varepsilon NR)}$, and the initial threshold for aDS-COD is set to $1$.

Our experiments aim to illustrate the trade-offs between space and approximation errors. The x-axis represents two metrics for space: final sketch size and total space cost. The final sketch size refers to the number of columns in the result sketches $\mA$ and $\mB$ generated by the algorithm, representing a compression ratio. The total space cost refers to the maximum space required during the algorithm's execution, measured by the number of columns.We evaluate the approximation performance of all algorithms based on correlation errors $\operatorname{corr-err}(\mathbf{X}_W \mathbf{Y}_W^\top, \mathbf{A} \mathbf{B}^\top)$, which is reflected on the y-axis. Every 1,000 iterations, all algorithms query the window and record the average and maximum errors across all sampled windows.

The experiments for all algorithms were conducted using MATLAB (R2023a), with all algorithms running on a Windows server equipped with 32GB of memory and a single processor of Intel i9-13900K.

\paragraph{Performance} Figure \ref{fig:error vs l} and Figure \ref{fig:error vs space} illustrate the space efficiency comparison of the algorithms on sequence-based datasets. Panels (a-d) show the average errors across all sampled windows, while panels (e-h) display the maximum errors.

Figure \ref{fig:error vs l} evaluates the compression effect of the final sketch. The hDS-COD, aDS-COD, and EH-COD show similar compression performances. But the DS series is more stable, particularly on the synthetic datasets, where they significantly outperform EH-COD and DI-COD. The performance of hDS-COD and aDS-COD is nearly the same, indicating that the adaptive threshold trick in aDS-COD does not have a noticeable negative impact on it, maintaining the same error as hDS-COD.

Figure \ref{fig:error vs space} measures the total space cost of the algorithms. hDS-COD and aDS-COD show a significant advantage over existing methods, as they can achieve the  $\varepsilon$-approximation error with much less space. For the same space cost, the correlation errors of hDS-COD and aDS-COD are much smaller than those of EH-COD and DI-COD. Also, aDS-COD has better space efficiency than hDS-COD because aDS only uses a single-level structure while hDS requires $\log R+1$ levels. We find that hDS-COD requires more space on  SYNTHETIC(2) dataset compared to SYNTHETIC(1) dataset. This phenomenon occurs because SYNTHETIC(2) dataset has a larger $R$, which confirms the dependence on $R$ as stated in Theorem~\ref{thm:hds}. 

Figure \ref{fig:time-based} compares the performance of algorithms on time-based windows. Panels (a) and (b) present the error against the final sketch size, which show that our aDS-COD and hDS-COD algorithms enjoy similar performance as EH-COD and significantly outperform the sampling algorithm. On the other hand, as shown in panels (c) and (d), our methods outperform baselines in terms of total space cost.

\section{Conclusion}
In this paper, we propose \ourmodel, a novel framework designed to improve the performance of large language models (LLMs) in domain-specific question-answering tasks. By integrating three key components: data synthesis based on conversations, data refinement based on conversations, and supervised fine-tuning (SFT) enhanced with retrieval, \ourmodel addresses the critical challenge of adapting LLM to understand and meet hidden user needs in vertical domains. Our experiments demonstrate that \ourmodel outperforms GPT-4-turbo+RAG by 7.92\% across the evaluation metrics. 

\section{Limitations}
% 不足1: 只在advertising这个domain下进行了实验->未来:在更多的vertical domain下进行实验,验证方法的有效性。
% 不足2: 评估手段多样性不足,只是利用GPT-4-turbo进行了评估->未来:在更多更强LLM(例如GPT-4-o1,Claude-3.5-sonnet以及DeepSeek-R1等)上进行评估,同时招募avertising上真实的广告用户进行评估等等。
% 不足3:没有进行A/B test。
This study has several limitations. First, the experimental validation was exclusively conducted within the advertising domain, which may constrain the generalizability of our methodology to other vertical domains (e.g., e-commerce, education, or healthcare). Future research should extend the evaluation framework by conducting cross-domain experiments to verify the robustness of our approach. Second, the assessment protocol relied primarily on GPT-4-turbo, DeepSeek-R1, and Llama-3.1-405B for automated evaluation, potentially introducing model-specific biases. Future work can explore(1) implementing human-in-the-loop evaluation with advertising professionals to assess practical utility, and (2) incorporating real-world A/B testing with actual advertisers to measure performance metrics in production environments.



\clearpage
\section*{Impact Statement}
This paper presents work whose goal is to advance the field of 
Machine Learning. There are many potential societal consequences 
of our work, none which we feel must be specifically highlighted here.
\bibliography{references}
\bibliographystyle{unsrtnat} %%% Uncomment this line and comment out the ``thebibliography'' section below to use the external .bib file (using bibtex) .


%%% Uncomment this section and comment out the \bibliography{references} line above to use inline references.
% \begin{thebibliography}{1}

% 	\bibitem{kour2014real}
% 	George Kour and Raid Saabne.
% 	\newblock Real-time segmentation of on-line handwritten arabic script.
% 	\newblock In {\em Frontiers in Handwriting Recognition (ICFHR), 2014 14th
% 			International Conference on}, pages 417--422. IEEE, 2014.

% 	\bibitem{kour2014fast}
% 	George Kour and Raid Saabne.
% 	\newblock Fast classification of handwritten on-line arabic characters.
% 	\newblock In {\em Soft Computing and Pattern Recognition (SoCPaR), 2014 6th
% 			International Conference of}, pages 312--318. IEEE, 2014.

% 	\bibitem{hadash2018estimate}
% 	Guy Hadash, Einat Kermany, Boaz Carmeli, Ofer Lavi, George Kour, and Alon
% 	Jacovi.
% 	\newblock Estimate and replace: A novel approach to integrating deep neural
% 	networks with existing applications.
% 	\newblock {\em arXiv preprint arXiv:1804.09028}, 2018.

% \end{thebibliography}
\newpage
\appendix
\onecolumn
\section{Additional Experiment Details}
\label{s:Details}


\subsection{Image Search Pipeline}
\subsubsection{CLIP for Image search}

\begin{figure}[ht]
    \centering
    \includegraphics[width=0.75\textwidth]{images/image-search-clip.png}
    \caption{CLIP Image-to-Image Search baseline depicted in Fig. \ref{fig:visualization}. }
    \label{fig:CLIP_image_search}
\end{figure}

Figure \ref{fig:CLIP_image_search} illustrates the CLIP baseline for image-to-image search. The image repository comprises the STL-10 test split, with a target image randomly selected for the search. The search process ranks images based on Euclidean distances, displaying the top 20 most similar results. The results indicate that CLIP effectively learns discriminative features, although two mismatched images are still present in the top-20 neighborhood.

\subsubsection{SCP for Image search}
Figure \ref{fig:image search} illustrates the SCP-CLIP approach for image-to-image search. SCP-CLIP successfully retrieves images from the same cluster within the top-20 neighborhood. 

\begin{figure}[ht]
    \centering
    \includegraphics[width=0.75\textwidth]{images/image-search-scp.png}
    \caption{SCP-CLIP Image-to-Image Search Pipeline depicted in Fig. \ref{fig:visualization}, utilizing the normalized outputs of the trained cluster head $g(.)$ without applying softmax. }
    \label{fig:image search}
\end{figure}



\subsection{Visualization}
\label{s:Details__ss:Visualization}
We create a t-SNE visualization image cloud of STL-10 test set to support SCP effectiveness further. It was observed that both SCP-DINO and SCP-CLIP successfully distinguished most planes, ships, horses, and deer, even when they share a blue background or similar body shape.

\clearpage
\begin{figure}[p]
    \centering
    \includegraphics[width=\textwidth]{images/image_cloud.png}
    \caption{The t-SNE representations learned by SCP-DINO on STL-10. $ACC: 98.4\%$ }
    \label{fig:image_cloud_DINO}
\end{figure}


\begin{figure}[p]
    \centering
    \includegraphics[width=\textwidth]{images/image_Cloud_CLIP.png}
    \caption{The t-SNE representations learned by SCP-CLIP on STL-10. $ACC: 97.9\%$ }
    \label{fig:image_cloud_CLIP}
\end{figure}


\clearpage

\section{Proofs}
\label{s:Proofs}
\begin{proof}[{Proof of Proposition~\ref{prop:LosslessAmort}}]
By the Doob-Dyknin lemma, see e.g.~\citep[Lemma 1.14]{KallenbergProbability_2021}, since $Z$ is $\sigma(X)$-measurable if and only if there exists some Borel measurable function $g:(\mathbb{R}^d,\mathcal{B}(\mathbb{R}^d))\to (\mathbb{R}^D,\mathcal{B}(\mathbb{R}^D))$ such that
\begin{equation}
\label{eq:setup_map}
    Z = g(X)
.
\end{equation}
Now, define $F: (\mathbb{R}^d,\mathcal{B}(\mathbb{R}^d))\to 
(\mathbb{R}^n,\mathcal{B}(\mathbb{R}^n))$ by composition, as sending every $x\in \mathbb{R}^d$ to
\begin{equation}
\label{eq:conclusion}
        F(x) 
    \eqdef 
        f(x,g(x))
.
\end{equation}
Then, together~\eqref{eq:setup_map}, the definition of $Y$ in~\eqref{eq:uncompressed}, and that of $F$ in~\eqref{eq:conclusion} yield
\allowdisplaybreaks
\begin{align}
    Y = f(X,Z) = f(X,g(X)) = F(X)
\end{align}
which establishes our claim.
\end{proof}


\begin{proof}[{Proof of Theorem~\ref{thrm:DCplus}}]
The law of $X$, namely the pushforward of $\mathbb{P}$ by $X$ denoted here by $\mu\eqdef X_{\#}\mathbb{P}$, is a Borel probability measure.  Therefore,~\citep[Theorem 13.6]{KlenkeBook_2020} implies that $\mu$ is a  Radon measure.  
Since, in addition, $\mathbb{R}^d$ is a second-countable and locally-compact topological space then, the version of Lusin's Theorem found in~\citep[Exercise 13.1.3]{KlenkeBook_2020} , applies.  Whereby, we deduce that for every $\varepsilon\in (0,1]$ there exists a compact subset $\mathcal{K}_{\varepsilon}$ of $\mathbb{R}^d$ satisfying
\begin{equation}
\label{eq:High_Prob_Lusin}
    \mu(\mathcal{K}_{\varepsilon})\ge 1-\varepsilon
\end{equation}
and for which the restriction of the Borel measurable map $F$ to $\mathcal{K}_{\varepsilon}$ is continuous.  

Now, fix a continuous activation function $\rho:\mathbb{R}\to \mathbb{R}$ with at least one point of continuous non-zero differentiability; i.e.\ satisfying~\citep[Assumption 1]{kratsios2022universal} (due to~\cite{kidger2020universal}).  As shown in \citep[Example 13]{kratsios2022universal}, the softmax function satisfies~\citep[Assumption 8]{kratsios2022universal}; whence the general case of the non-Euclidean  universal approximation theorem~\citep[Theorem 37 (ii)]{kratsios2022universal} applies; from which we deduce the existence of an MLP $\hat{F}:\mathbb{R}^d\to \mathbb{R}^C$ with $\rho$ activation function satisfying the uniform approximation guarantee
\begin{equation}
\label{eq:uniform_approx}
    \sup_{x\in \mathcal{K}_{\varepsilon}}\,
        |F(x)-\operatorname{softmax}\circ \hat{F}(x)|
    <
        \varepsilon
.
\end{equation}
Define the failure set
\[
B\eqdef \biggl\{x\in \mathbb{R}^d:\,
|F(x)-\operatorname{softmax}\circ \hat{F}(x)|
\ge \varepsilon
\biggr\}
.
\]
Bfiy construction, $\mathcal{K}_{\varepsilon}\subseteq \mathbb{R}^d\setminus B $.
Furthermore, observe that: since $L:\mathbb{R}^d\to |F(x)-\operatorname{softmax}\circ \hat{F}(x)|\in \mathbb{R}$ is the composition of measurable and continuous functions then it, too is measurable; whence, $B\eqdef L^{-1}[[\varepsilon,\infty)]$; must itself measurable.  

Consequentially, the following computations are well founded:
Combining~\eqref{eq:High_Prob_Lusin} and~\eqref{eq:uniform_approx} we deduce that
\allowdisplaybreaks
\begin{align}
\label{eq:finishme}
        \mathbb{P}\biggl(
            |
                F(X)
                -
                \operatorname{softmax}\circ \hat{F}(X)
            |
            <
            \varepsilon
        \biggr)
    & =
        \mu\biggl(
            |
                F(x)
                -
                \operatorname{softmax}\circ \hat{F}(x)
            |
            <
            \varepsilon
        \biggr)
\\
\nonumber
    & =
        \mu(\mathbb{R}^d\setminus B)
\\
\nonumber
    & \ge 
        \mu(\mathcal{K}_{\varepsilon})
\\
\label{eq:finishme__END}
    & \ge 1- \varepsilon  
.
\end{align}
Finally, by Proposition~\ref{prop:LosslessAmort} and~\eqref{eq:uncompressed}, we know that
\begin{equation}
\label{eq:rep_me}
F(X)=f(X,Z)
.
\end{equation}
Whence, incorporating the identity in~\eqref{eq:rep_me} into the left-hand side of the chain of inequalities in~\eqref{eq:finishme}-\eqref{eq:finishme__END} implies that
\allowdisplaybreaks
\begin{align}
\label{eq:finishme2}
        \mathbb{P}\biggl(
                |
                    f(X,Z)
                    -
                    \operatorname{softmax}\circ \hat{F}(X)
                |
            <
                \varepsilon
        \biggr)
    =
        \mathbb{P}\biggl(
                |
                    F(X)
                    -
                    \operatorname{softmax}\circ \hat{F}(X)
                |
            <
                \varepsilon
        \biggr)
    \ge 1- \varepsilon  
\end{align}
which concludes our proof.
\end{proof}



% \section*{Acknowledgments}
% A.\ Kratsios and Y.\ Li acknowledge financial support from an NSERC Discovery Grant No.\ RGPIN-2023-04482 and No.\ DGECR-2023-00230.  A.\ Kratsios also acknowledges that resources used in preparing this research were provided, in part, by the Province of Ontario, the Government of Canada through CIFAR, and companies sponsoring the Vector Institute\footnote{\href{https://vectorinstitute.ai/partnerships/current-partners/}{https://vectorinstitute.ai/partnerships/current-partners/}}.

%%%%%%%%%%%%%%%%%%%%%%%%%%%%%%%%%%%%%%%%%%%%%%%%%%%%%%%%%%%%%%%%%%%%%%%%%%%%%%%
%%%%%%%%%%%%%%%%%%%%%%%%%%%%%%%%%%%%%%%%%%%%%%%%%%%%%%%%%%%%%%%%%%%%%%%%%%%%%%%


\end{document}
