\documentclass{article}



\usepackage{arxiv}
\usepackage[numbers]{natbib}
\usepackage[utf8]{inputenc} % allow utf-8 input
\usepackage[T1]{fontenc}    % use 8-bit T1 fonts
\usepackage{hyperref}       % hyperlinks
\usepackage{url}            % simple URL typesetting
\usepackage{booktabs}       % professional-quality tables
\usepackage{amsfonts}       % blackboard math symbols
\usepackage{nicefrac}       % compact symbols for 1/2, etc.
\usepackage{microtype}      % microtypography
\usepackage{lipsum}		% Can be removed after putting your text content
\usepackage{graphicx}
\usepackage{natbib}
\usepackage{doi}
\usepackage{amsmath}
\usepackage{amssymb}
\usepackage{bm}
\usepackage{multirow}
\usepackage{algorithm}
\usepackage{algpseudocode}
\usepackage{float}
\usepackage{pifont}
\usepackage{subcaption}
\usepackage{mathtools}
\usepackage{amsthm}
\usepackage{blindtext}
\usepackage{xcolor}

\newcommand{\eqdef}{\ensuremath{\,\raisebox{-1.2pt}{${\stackrel{\mbox{\upshape \scalebox{.42}{def.}}}{=}}$}}\,}
\newcommand{\eqorder}{\sim} %{\ensuremath{\,\raisebox{-1pt}{$\stackrel{\mbox{\upshape\tiny $\order$}}{\sim}$}}\,}
% \newcommand{\eqdef}{\stackrel{\mathclap{\tiny\mbox{def.}}}{=}}

% Todonotes is useful during development; simply uncomment the next line
%    and comment out the line below the next line to turn off comments
%\usepackage[disable,textsize=tiny]{todonotes}


\definecolor{deepjunglegreen}{rgb}{0.0, 0.29, 0.29}

\usepackage[textsize=tiny]{todonotes}



\def\msquare{\mathord{\scalerel*{\Box}{gX}}}
\theoremstyle{plain}
\newtheorem{theorem}{Theorem}[section]
\newtheorem{proposition}[theorem]{Proposition}
\newtheorem{lemma}[theorem]{Lemma}
\newtheorem{corollary}[theorem]{Corollary}
\theoremstyle{definition}
\newtheorem{definition}[theorem]{Definition}
\newtheorem{assumption}[theorem]{Assumption}
\theoremstyle{remark}
\newtheorem{remark}[theorem]{Remark}

\newcommand{\first}[1]{\textbf{\textcolor{red}{\textbf{{#1}}}}}
\newcommand{\second}[1]{\textbf{\textcolor{blue}{\underline{{#1}}}}}
\newcommand{\third}[1]{\textbf{\textcolor{violet}{#1}}}






\title{Keep it Light! Simplifying Image Clustering via Text-Free Adapters}

%\date{September 9, 1985}	% Here you can change the date presented in the paper title
%\date{} 					% Or removing it

\author{
    Yicen Li$^{1,2}$\thanks{Corresponding author. Email: \texttt{li2642@mcmaster.ca}}, 
    Haitz Sáez de Ocáriz Borde$^{3}$, 
    Anastasis Kratsios$^{1,2}$\thanks{Corresponding author. Email: \texttt{kratsioa@mcmaster.ca}}, 
    Paul D. McNicholas$^{1,2}$ \\
    $^{1}$Department of Mathematics and Statistics, McMaster University, Hamilton, Canada \\
    $^{2}$Vector Institute, Toronto, Canada \\
    $^{3}$University of Oxford, Oxford, United Kingdom \\
    \texttt{li2642@mcmaster.ca, kratsioa@mcmaster.ca}
}

% Uncomment to remove the date
%\date{}

% Uncomment to override  the `A preprint' in the header
% \renewcommand{\headeright}{A preprint}
% \renewcommand{\undertitle}{A preprint}
% \renewcommand{\shorttitle}{\textit{arXiv} Template}

%%% Add PDF metadata to help others organize their library
%%% Once the PDF is generated, you can check the metadata with
%%% $ pdfinfo template.pdf
\hypersetup{
pdftitle={A template for the arxiv style},
pdfsubject={q-bio.NC, q-bio.QM},
pdfauthor={David S.~Hippocampus, Elias D.~Striatum},
pdfkeywords={First keyword, Second keyword, More},
}


\begin{document}
\maketitle

\begin{abstract}
Since 2020, GitGuardian has been detecting checked-in hard-coded secrets in GitHub repositories. During 2020-2023, GitGuardian has observed an upward annual trend and a four-fold increase in hard-coded secrets, with 12.8 million exposed in 2023. However, removing all the secrets from software artifacts is not feasible due to time constraints and technical challenges. Additionally, the security risks of the secrets are not equal, protecting assets ranging from obsolete databases to sensitive medical data. Thus, secret removal should be prioritized by security risk reduction, which existing secret detection tools do not support. \textit{The goal of this research is to aid software practitioners in prioritizing secrets removal efforts through our security risk-based tool}. We present RiskHarvester, a risk-based tool to compute a security risk score based on the value of the asset and ease of attack on a database. We calculated the value of asset by identifying the sensitive data categories present in a database from the database keywords in the source code. We utilized data flow analysis, SQL, and Object Relational Mapper (ORM) parsing to identify the database keywords. To calculate the ease of attack, we utilized passive network analysis to retrieve the database host information. To evaluate RiskHarvester, we curated RiskBench, a benchmark of 1,791 database secret-asset pairs with sensitive data categories and host information manually retrieved from 188 GitHub repositories. RiskHarvester demonstrates precision of (95\%) and recall (90\%) in detecting database keywords for the value of asset and precision of (96\%) and recall (94\%) in detecting valid hosts for ease of attack. Finally, we conducted a survey (52 respondents) to understand whether developers prioritize secret removal based on security risk score. We found that 86\% of the developers prioritized the secrets for removal with descending security risk scores.
\end{abstract}





\section{Introduction}

Large language models (LLMs) have achieved remarkable success in automated math problem solving, particularly through code-generation capabilities integrated with proof assistants~\citep{lean,isabelle,POT,autoformalization,MATH}. Although LLMs excel at generating solution steps and correct answers in algebra and calculus~\citep{math_solving}, their unimodal nature limits performance in plane geometry, where solution depends on both diagram and text~\citep{math_solving}. 

Specialized vision-language models (VLMs) have accordingly been developed for plane geometry problem solving (PGPS)~\citep{geoqa,unigeo,intergps,pgps,GOLD,LANS,geox}. Yet, it remains unclear whether these models genuinely leverage diagrams or rely almost exclusively on textual features. This ambiguity arises because existing PGPS datasets typically embed sufficient geometric details within problem statements, potentially making the vision encoder unnecessary~\citep{GOLD}. \cref{fig:pgps_examples} illustrates example questions from GeoQA and PGPS9K, where solutions can be derived without referencing the diagrams.

\begin{figure}
    \centering
    \begin{subfigure}[t]{.49\linewidth}
        \centering
        \includegraphics[width=\linewidth]{latex/figures/images/geoqa_example.pdf}
        \caption{GeoQA}
        \label{fig:geoqa_example}
    \end{subfigure}
    \begin{subfigure}[t]{.48\linewidth}
        \centering
        \includegraphics[width=\linewidth]{latex/figures/images/pgps_example.pdf}
        \caption{PGPS9K}
        \label{fig:pgps9k_example}
    \end{subfigure}
    \caption{
    Examples of diagram-caption pairs and their solution steps written in formal languages from GeoQA and PGPS9k datasets. In the problem description, the visual geometric premises and numerical variables are highlighted in green and red, respectively. A significant difference in the style of the diagram and formal language can be observable. %, along with the differences in formal languages supported by the corresponding datasets.
    \label{fig:pgps_examples}
    }
\end{figure}



We propose a new benchmark created via a synthetic data engine, which systematically evaluates the ability of VLM vision encoders to recognize geometric premises. Our empirical findings reveal that previously suggested self-supervised learning (SSL) approaches, e.g., vector quantized variataional auto-encoder (VQ-VAE)~\citep{unimath} and masked auto-encoder (MAE)~\citep{scagps,geox}, and widely adopted encoders, e.g., OpenCLIP~\citep{clip} and DinoV2~\citep{dinov2}, struggle to detect geometric features such as perpendicularity and degrees. 

To this end, we propose \geoclip{}, a model pre-trained on a large corpus of synthetic diagram–caption pairs. By varying diagram styles (e.g., color, font size, resolution, line width), \geoclip{} learns robust geometric representations and outperforms prior SSL-based methods on our benchmark. Building on \geoclip{}, we introduce a few-shot domain adaptation technique that efficiently transfers the recognition ability to real-world diagrams. We further combine this domain-adapted GeoCLIP with an LLM, forming a domain-agnostic VLM for solving PGPS tasks in MathVerse~\citep{mathverse}. 
%To accommodate diverse diagram styles and solution formats, we unify the solution program languages across multiple PGPS datasets, ensuring comprehensive evaluation. 

In our experiments on MathVerse~\citep{mathverse}, which encompasses diverse plane geometry tasks and diagram styles, our VLM with a domain-adapted \geoclip{} consistently outperforms both task-specific PGPS models and generalist VLMs. 
% In particular, it achieves higher accuracy on tasks requiring geometric-feature recognition, even when critical numerical measurements are moved from text to diagrams. 
Ablation studies confirm the effectiveness of our domain adaptation strategy, showing improvements in optical character recognition (OCR)-based tasks and robust diagram embeddings across different styles. 
% By unifying the solution program languages of existing datasets and incorporating OCR capability, we enable a single VLM, named \geovlm{}, to handle a broad class of plane geometry problems.

% Contributions
We summarize the contributions as follows:
We propose a novel benchmark for systematically assessing how well vision encoders recognize geometric premises in plane geometry diagrams~(\cref{sec:visual_feature}); We introduce \geoclip{}, a vision encoder capable of accurately detecting visual geometric premises~(\cref{sec:geoclip}), and a few-shot domain adaptation technique that efficiently transfers this capability across different diagram styles (\cref{sec:domain_adaptation});
We show that our VLM, incorporating domain-adapted GeoCLIP, surpasses existing specialized PGPS VLMs and generalist VLMs on the MathVerse benchmark~(\cref{sec:experiments}) and effectively interprets diverse diagram styles~(\cref{sec:abl}).

\iffalse
\begin{itemize}
    \item We propose a novel benchmark for systematically assessing how well vision encoders recognize geometric premises, e.g., perpendicularity and angle measures, in plane geometry diagrams.
	\item We introduce \geoclip{}, a vision encoder capable of accurately detecting visual geometric premises, and a few-shot domain adaptation technique that efficiently transfers this capability across different diagram styles.
	\item We show that our final VLM, incorporating GeoCLIP-DA, effectively interprets diverse diagram styles and achieves state-of-the-art performance on the MathVerse benchmark, surpassing existing specialized PGPS models and generalist VLM models.
\end{itemize}
\fi

\iffalse

Large language models (LLMs) have made significant strides in automated math word problem solving. In particular, their code-generation capabilities combined with proof assistants~\citep{lean,isabelle} help minimize computational errors~\citep{POT}, improve solution precision~\citep{autoformalization}, and offer rigorous feedback and evaluation~\citep{MATH}. Although LLMs excel in generating solution steps and correct answers for algebra and calculus~\citep{math_solving}, their uni-modal nature limits performance in domains like plane geometry, where both diagrams and text are vital.

Plane geometry problem solving (PGPS) tasks typically include diagrams and textual descriptions, requiring solvers to interpret premises from both sources. To facilitate automated solutions for these problems, several studies have introduced formal languages tailored for plane geometry to represent solution steps as a program with training datasets composed of diagrams, textual descriptions, and solution programs~\citep{geoqa,unigeo,intergps,pgps}. Building on these datasets, a number of PGPS specialized vision-language models (VLMs) have been developed so far~\citep{GOLD, LANS, geox}.

Most existing VLMs, however, fail to use diagrams when solving geometry problems. Well-known PGPS datasets such as GeoQA~\citep{geoqa}, UniGeo~\citep{unigeo}, and PGPS9K~\citep{pgps}, can be solved without accessing diagrams, as their problem descriptions often contain all geometric information. \cref{fig:pgps_examples} shows an example from GeoQA and PGPS9K datasets, where one can deduce the solution steps without knowing the diagrams. 
As a result, models trained on these datasets rely almost exclusively on textual information, leaving the vision encoder under-utilized~\citep{GOLD}. 
Consequently, the VLMs trained on these datasets cannot solve the plane geometry problem when necessary geometric properties or relations are excluded from the problem statement.

Some studies seek to enhance the recognition of geometric premises from a diagram by directly predicting the premises from the diagram~\citep{GOLD, intergps} or as an auxiliary task for vision encoders~\citep{geoqa,geoqa-plus}. However, these approaches remain highly domain-specific because the labels for training are difficult to obtain, thus limiting generalization across different domains. While self-supervised learning (SSL) methods that depend exclusively on geometric diagrams, e.g., vector quantized variational auto-encoder (VQ-VAE)~\citep{unimath} and masked auto-encoder (MAE)~\citep{scagps,geox}, have also been explored, the effectiveness of the SSL approaches on recognizing geometric features has not been thoroughly investigated.

We introduce a benchmark constructed with a synthetic data engine to evaluate the effectiveness of SSL approaches in recognizing geometric premises from diagrams. Our empirical results with the proposed benchmark show that the vision encoders trained with SSL methods fail to capture visual \geofeat{}s such as perpendicularity between two lines and angle measure.
Furthermore, we find that the pre-trained vision encoders often used in general-purpose VLMs, e.g., OpenCLIP~\citep{clip} and DinoV2~\citep{dinov2}, fail to recognize geometric premises from diagrams.

To improve the vision encoder for PGPS, we propose \geoclip{}, a model trained with a massive amount of diagram-caption pairs.
Since the amount of diagram-caption pairs in existing benchmarks is often limited, we develop a plane diagram generator that can randomly sample plane geometry problems with the help of existing proof assistant~\citep{alphageometry}.
To make \geoclip{} robust against different styles, we vary the visual properties of diagrams, such as color, font size, resolution, and line width.
We show that \geoclip{} performs better than the other SSL approaches and commonly used vision encoders on the newly proposed benchmark.

Another major challenge in PGPS is developing a domain-agnostic VLM capable of handling multiple PGPS benchmarks. As shown in \cref{fig:pgps_examples}, the main difficulties arise from variations in diagram styles. 
To address the issue, we propose a few-shot domain adaptation technique for \geoclip{} which transfers its visual \geofeat{} perception from the synthetic diagrams to the real-world diagrams efficiently. 

We study the efficacy of the domain adapted \geoclip{} on PGPS when equipped with the language model. To be specific, we compare the VLM with the previous PGPS models on MathVerse~\citep{mathverse}, which is designed to evaluate both the PGPS and visual \geofeat{} perception performance on various domains.
While previous PGPS models are inapplicable to certain types of MathVerse problems, we modify the prediction target and unify the solution program languages of the existing PGPS training data to make our VLM applicable to all types of MathVerse problems.
Results on MathVerse demonstrate that our VLM more effectively integrates diagrammatic information and remains robust under conditions of various diagram styles.

\begin{itemize}
    \item We propose a benchmark to measure the visual \geofeat{} recognition performance of different vision encoders.
    % \item \sh{We introduce geometric CLIP (\geoclip{} and train the VLM equipped with \geoclip{} to predict both solution steps and the numerical measurements of the problem.}
    \item We introduce \geoclip{}, a vision encoder which can accurately recognize visual \geofeat{}s and a few-shot domain adaptation technique which can transfer such ability to different domains efficiently. 
    % \item \sh{We develop our final PGPS model, \geovlm{}, by adapting \geoclip{} to different domains and training with unified languages of solution program data.}
    % We develop a domain-agnostic VLM, namely \geovlm{}, by applying a simple yet effective domain adaptation method to \geoclip{} and training on the refined training data.
    \item We demonstrate our VLM equipped with GeoCLIP-DA effectively interprets diverse diagram styles, achieving superior performance on MathVerse compared to the existing PGPS models.
\end{itemize}

\fi 

\section{Related Work}

\noindent\textbf{Diffusion Efficiency Improvements:} 
\citet{das2023image} utilized the shortest path between two Gaussians and \citet{song2020denoising} generalized DDPMs via a class of non-Markovian diffusion processes to reduce the number of diffusion steps. \citet{nichol2021improved} introduced a few simple modifications to improve the log-likelihood. \citet{pandey2022diffusevae, pandey2021vaes} used DDPMs to refine VAE-generated samples. \citet{rombach2022high} performed the diffusion process in the lower dimensional latent space of an autoencoder to achieve high-resolution image synthesis, and \citet{liu2023audioldm} studied using such latent diffusion models for audio. \citet{popov2021grad} explored using a text encoder to extract better representations for continuous-time diffusion-based text-to-speech generation. More recently, \citet{nielsendiffenc} explored using a time-dependent image encoder to parameterize the mean of the diffusion process. Orthogonal to the above, PriorGrad \citep{lee2021priorgrad} and follow-up work \citep{koizumi22_interspeech} studied utilizing informative prior extracted from the conditioner data for improving learning efficiency. \textit{However, they become sub-optimal when the conditioner are degraded versions of the target data, posing challenges in applications like signal restoration tasks.}

\noindent\textbf{Diffusion-Based Signal Restoration:}
Built on top of the diffusion models for audio generation, e.g., \citet{kong2020diffwave,chen2020wavegrad,leng2022binauralgrad}, many SE models have been proposed. The pioneering work of \citet{lu2022conditional} introduced conditional DDPMs to the SE task and demonstrated the potential. Other works \citep{serra2022universal,welker2022speech,richter2023speech,yen2023cold,lemercier2023storm,tai2024dose} have also attempted to improve SE by exploiting diffusion models. In the vision domain, diffusion models have demonstrated impressive performance for IR tasks \citep{li2023diffusion,zhu2023denoising,huang2024wavedm,luo2023refusion,xia2023diffir,fei2023generative,hurault2022gradient,liu20232,chung2024direct,chungdiffusion,zhoudenoising,xiaodreamclean,zheng2024diffusion}. A notable IR work is \cite{ozdenizci2023restoring} that achieved impressive performance on several benchmark datasets for restoring vision in adverse weather conditions. \textit{Despite showing promising results, existing works have not fully exploited prior information about the data as they mostly settle on standard Gaussian priors.} 
\section{Our Method}
\label{sec:Our Method}
Recent methods such as CC \cite{li2021contrastive} and CPP \cite{chu2024image} decouple the latent space into clustering and feature spaces. TAC \cite{li2023image} utilizes text information, and TEMI \cite{adaloglou2023exploring} employs self-distillation networks to enhance clustering. In contrast, our approach remains simple and efficient without relying on these techniques. As shown in Fig.~\ref{fig:AE}, SCP consists of only two components: a pre-trained frozen backbone for pair construction, denoted as $f(.)$, and a trainable cluster head $g(.)$.

\begin{figure}[ht]
    \centering
    \includegraphics[width=0.6\columnwidth]{images/scp.png}
    \caption{%
    A overall pipeline for SCP. During training, two augmented views \(T^a\) and \(T^b\) of an image are generated from the dataset and processed by a frozen feature extractor \(f\) and a trainable cluster head \(g\) (a five-layer MLP). The objective is to minimize the cross-entropy loss between the outputs of the cluster head \(g\) for the two augmented views.
    }
    \label{fig:AE}
\end{figure}

Briefly, SCP performs data augmentations and extracts features from the augmented images using pre-trained models. The cluster head then projects these features into a cluster space, where the dimension equals the number of clusters. After training, outputs in the cluster space provide the soft assignments for clustering.


\subsection{Pair Construction Backbone}

The success of BYOL demonstrates that we can maximize the similarities of positive pairs without negative ones. In SCP, the positive pairs consist of samples augmented from the same instance.

Given a data instance \(x_i\), we apply two stochastic transformations \(T^a\) and \(T^b\), independently selected from the augmentation family \(\mathcal{T}\). This produces two correlated views: \(x_i^a = T^a(x_i)\) and \(x_i^b = T^b(x_i)\). 

An appropriate augmentation strategy is vital for better downstream performance. In our work, we adopt only two simple augmentations: \texttt{RandomCrop} and \texttt{GaussianBlur}. This choice aligns with the preprocessing techniques used in training pre-trained models, ensuring compatibility with their learned representations. \texttt{RandomCrop} randomly crops the image to a specified size, and \texttt{GaussianBlur} applies a Gaussian filter to blur the image. For each image, these two augmentations are applied independently, each with a 50\% probability. We then use a pre-trained model \(f(\cdot)\), such as CLIP, to extract features from the augmented images: $h_i^a = f(x_i^a)$ and $h_i^b = f(x_i^b).$

\subsection{Cluster Head}
Following the “label as representation” concept \cite{li2021contrastive}, when a data sample is projected into a space whose dimensionality matches the number of clusters \(K\), the \(k\)-th component of its feature vector (after applying a $\operatorname{softmax}$  function) can be interpreted as the probability that the sample belongs to the \(k\)-th cluster. We employ a five-layer non-linear MLP as the clustering head \(g(\cdot)\), producing a  \(K\)-dimensional feature that is normalized with a $\operatorname{softmax}$  over the dimension of the cluster.
\[
y_i^a = g(h_i^a), 
\quad 
y_i^b = g(h_i^b).
\]
Hence, \(y_i^a\) and \(y_i^b\) are both \(K\)-dimensional vectors, whose components \(y_{i,k}^a\) and \(y_{i,k}^b\) indicate the probability of assigning the \(i\)-th sample to the \(k\)-th cluster. Formally, let \(Y^a, Y^b \in \mathbb{R}^{N \times K}\) be the outputs of the clustering head for all samples. Then, we have the following matrices:
\[
% Y^a = \begin{bmatrix}
% y^a_1 \\
% y^a_2 \\
% \vdots \\
% y^a_N
% \end{bmatrix}
% \quad
% Y^b = \begin{bmatrix}
% y^b_1 \\
% y^b_2 \\
% \vdots \\
% y^b_N
% \end{bmatrix}.
Y^a = \begin{bmatrix}
y^i_1 \\
% y^a_2 \\
\vdots \\
y^a_N
\end{bmatrix}
\quad
Y^b = \begin{bmatrix}
y^b_1 \\
% y^b_2 \\
\vdots \\
y^b_N
\end{bmatrix}.
\]
To maximize row-wise similarity, we adopt the following cross-entropy loss function instead of the commonly used InfoNCE loss \cite{oord2018representation}, as SCP only have positive pairs that should share similar soft assignments:
\begin{equation}
L_e = - \sum_{i=1}^{N} \sum_{k=1}^{K} y^{a}_{i,k} \log y^{b}_{i,k}.
\end{equation}
Inspired by the effective regularizations in TAC \cite{li2023image}, we further introduce the following confidence loss to make the soft labels \( y^{a}_i \) and \( y^{b}_i \) more confident, approaching one-hot vectors:
\begin{equation}
L_{\text{con}} = - \log \sum_{i=1}^N {y^a_i}^\top y^b_i.
\end{equation}
This loss ensures that the cluster head assigns higher probabilities to its top predicted clusters, thereby increasing confidence in the assignments. 

In addition, following TAC \cite{li2023image}, we introduce an entropy term \( H(Y) \) to prevent model collapse, defined as follows:
\begin{equation}
H(Y) = - \sum_{k=1}^{K} \left[ P^{a}_k \log P^{a}_k + P^{b}_k \log P^{b}_k \right],
\end{equation}
where
\[
P^{a}_k = \frac{1}{N} \sum_{i=1}^{N} y^{a}_{i,k}, \quad P^{b}_k = \frac{1}{N} \sum_{i=1}^{N} y^{b}_{i,k}.
\]
$H(Y)$ encourages uniform soft assignments across clusters, thereby mitigating the issue of empty clusters. 

Hence, we define the overall objective function of SCP as
\begin{equation}
L_{\text{clu}} = L_e + L_{\text{con}} - \alpha H(Y),
\end{equation}
where the balancing weight $\alpha$ modulates the influence of $H(Y)$, especially when the number of clusters is large. By maximizing consistency between different augmented views with regularizations, SCP effectively prevents trivial solutions and achieves competitive performance. We provide algorithm \ref{alg:CAC} to explain our pipeline.

\begin{algorithm}[H]
\caption{Simple Clustering via Pre-trained Models (SCP)}
\label{alg:CAC}
\begin{algorithmic}[1]
    \Require Dataset $\mathcal{X} = \{ x_i \}_{i=1}^{N}$, Pre-trained model $f(\cdot)$, number of clusters $K$, batch size $B$, loss weight $\alpha$
    \State Initialize cluster head $g(\cdot)$
    \For{each epoch}
        \For{each mini-batch $\{x_i\}_{i=1}^{B}$}
            \State \textbf{Pair Construction:}
            \For{each data instance $x_i$ in the mini-batch}
                \State Apply stochastic transformations $T^a$, $T^b$ to obtain:
                \State \quad $x_i^a = T^a(x_i)$, \quad $x_i^b = T^b(x_i)$
                \State Extract features using pre-trained model:
                \State \quad $h_i^a = f(x_i^a)$, \quad $h_i^b = f(x_i^b)$
            \EndFor
            \State \textbf{Cluster Space Encoding:}
            \For{each feature $h_i^a$, $h_i^b$}
                \State Compute soft assignments: \quad $y_i^a = g(h_i^a)$, \quad $y_i^b = g(h_i^b)$
            \EndFor
            \State \textbf{Compute Losses:}
            \State Compute total clustering loss:
            \State \quad $L_{\text{clu}} = L_e + L_{\text{con}} - \alpha H(Y)$
            \State \textbf{Update} cluster head $g(\cdot)$ parameters by minimizing $L_{\text{clu}}$
        \EndFor
    \EndFor
    \State \Return soft assignments $y_i = g(h_i)$ for each $x_i \in \mathcal{X}$
\end{algorithmic}
\end{algorithm}


% Inspired by the work \cite{dwibedi2021little}, we recognize the advantages of incorporating positives from other instances in the dataset. Therefore, we construct a support set by sampling the nearest neighbours from the dataset in the CLIP feature space, treating them as additional positives. This approach provides extra semantic variations and aligns with the concept of stochastic neighbour embeddings. We assume that these nearest neighbours should be close enough as well after the projections.

% For each image feature \( h_i^a \), we find its \( M \) nearest neighbours \( \{ \tilde{h}_{i,j} \}_{j=1}^M \) in the set of \( \{ h_{i}^a \}_{i=1}^N \) and compute a weighted sum to obtain the aggregated neighbour representation \( \tilde{h}_i \):

% \begin{equation}
% \tilde{h}_i = \sum_{j=1}^M p(\tilde{h}_{i,j} \mid h_i^a) \, \tilde{h}_{i,j},
% \label{nnequ}
% \end{equation}

% where the weights \( p(\tilde{h}_{i,j} \mid h_i^a) \) are defined as:

% \begin{equation}
% p(\tilde{h}_{i,j} \mid h_i^a) = \frac{\exp\left( \text{sim}(h_i^a, \tilde{h}_{i,j}) / \tau \right)}{\sum_{k=1}^M \exp\left( \text{sim}(h_i^a, \tilde{h}_{i,k}) / \tau \right)}.
% \end{equation}

% Here, \( \text{sim}(\cdot, \cdot) \) denotes cosine similarity:

% \begin{equation}
% \text{sim}(h_i^a, \tilde{h}_{i,j}) = \frac{(h_i^a)^\top \tilde{h}_{i,j}}{\|h_i^a\| \, \|\tilde{h}_{i,j}\|},
% \end{equation}

% \( \tau \) is a temperature parameter, and \( M \) is the number of top nearest neighbours selected from  \( \{ h_{i}^a \}_{i=1}^N \). This weighting mechanism ensures that the influence of each neighbour is proportional to its similarity to the query image feature \( h_i^a \), preventing the nearest neighbours of different images from collapsing to the same point. We set the selected number of nearest neighbours as 10 for all experiments. To further clarify, this method allows each image to dynamically weigh its nearest neighbours based on similarity, enhancing the model's ability to capture fine-grained semantic relationships within the dataset. Finally, we incorporate these aggregated neighbour representations into the learning process by treating \( \tilde{h}_i \) as additional positive samples for \( h_i^a \). Thus, the support set loss is defined by:

% \begin{equation}
% L_s = - \sum_{i=1}^{N} \sum_{k=1}^{K} \tilde{y}_{i,k} \log y^{a}_{i,k},
% \end{equation}

% where \( \tilde{y}_{i} = g(\tilde{h}_{i}) \). This enhances the model's ability to learn from semantically nearest instances, improving clustering performance. To keep simplicity, we don't calculate the support set loss for another augmented view \( \{ h_{i}^b \}_{i=1}^N \).


% \subsection{Cluster Space Decoder}
% To encourage the encoder to learn meaningful and helpful latent representations and additionally avoid trivial solutions, we attach a decoder that aims to reconstruct the original image features. In summary, the cluster space decoder minimizes the reconstruction loss:

% \begin{equation}
% L_{d}=  - \sum_{i=1}^{N} \left( \text{sim}(h_i^a, h_{i}^{'a}) + \text{sim}(h_i^b, h_{i}^{'b}) \right),
% \end{equation}

% where $h_{i}^{'a} = d(y^{a}_{i})$ and $h_{i}^{'b} = d(y^{b}_{i})$, and $d(\cdot)$ denotes the decoder. $sim$ is still measured by cosine similarity.

% Finally, we arrive at the overall objective function of CAC, which takes the form:

% \begin{equation}
% L_{\text{CAC}} = \alpha \cdot L_{d} + L_s + L_{\text{clu}},
% \end{equation}

% where $\alpha = 0.01$ is a weight parameter set for all current experiments. This is small scale because it is expected that the decoder cannot perfectly reconstruct the same features, only knowing which cluster they belong to. 

% In section 4.5, we show our method is robust to different $\alpha$. Although the multi-task loss looks overwhelming there are fewer parameters to tune compared to previous works like \cite{li2023image}.

% In the end, we provide a pseudo code to better explain our pipeline:

% \begin{algorithm}[H]
% \caption{CAC: CLIP-based Auto-Encoder Clustering}
% \label{alg:CAC}
% \begin{algorithmic}[1]
% \Require Dataset $\mathcal{X} = \{ x_i \}_{i=1}^{N}$, number of clusters $K$, batch size $B$, number of nearest neighbors $M$, temperature parameter $\tau$, weight parameter $\alpha$
% \State Initialize encoder network $g(\cdot)$ and decoder network $d(\cdot)$
% \For{each epoch}
%     \For{each mini-batch $\{ x_i \}_{i=1}^{B}$}
%         \State \textbf{Pair Construction:}
%         \For{each data instance $x_i$ in the mini-batch}
%             \State Apply stochastic transformations $T^a$, $T^b$ to obtain:
%             \State \quad $x_i^a = T^a(x_i)$, \quad $x_i^b = T^b(x_i)$
%             \State Extract features using CLIP backbone:
%             \State \quad $h_i^a = f(x_i^a)$, \quad $h_i^b = f(x_i^b)$
%         \EndFor
%         \State \textbf{Cluster Space Encoding:}
%         \For{each feature $h_i^a$, $h_i^b$}
%             \State Compute soft labels: \quad $y_i^a = g(h_i^a)$, \quad $y_i^b = g(h_i^b)$
%         \EndFor
%         \For{each feature $h_i^a$}
%             \State Compute aggregated neighbor representation $h_i^N$
%             \State Compute soft label $y_i^N = g(h_i^N)$
%         \EndFor
%         \State \textbf{Cluster Space Decoding:}
%         \For{each soft label $y_i^a$, $y_i^b$}
%             \State Reconstruct features:
%             \State \quad $h_{i}^{'a} = d(y^{a}_{i})$, \quad $h_{i}^{'b} = d(y^{b}_{i})$
%         \EndFor
%         \State \textbf{Compute Total Loss:}
%         \State \quad $L_{\text{CAC}} = \alpha \cdot L_{d} + L_s + L_{\text{clu}}$
%         \State \textbf{Update} encoder $g(\cdot)$ and decoder $d(\cdot)$ parameters by minimizing $L_{\text{CAC}}$
%     \EndFor
% \EndFor
% \end{algorithmic}
% \end{algorithm}
% Clip-based Auto-Encoder Learning (CAEL) aims to learn a mapping from $R^{768}$ into $R^{512}$. We minimize the Euclidean distance in the latent space $R^512$ to preserve the original structure between two views of images because they represent the same semantic meaning:

% \begin{equation}
% l_1 = \frac{1}{N} \sum_{n=1}^{N}  \left\| g_{w}(h^a_n) - g_{w}(h^b_n) \right\|_2^2
% \end{equation}

% where the parameters $w$ of the encoder $g_w(\cdot)$ and parameter $\theta$ of the decoder $d_\theta(\cdot)$ are further jointly learned by optimizing the following problem to avoid collapse:

% \begin{equation}
% l_2 =  \frac{1}{2N} \sum_{n=1}^{N} \left( \left\|  h^{'a}_n - h^a_n \right\|_2^2 + \left\|  h^{'b}_n - h^b_n \right\|_2^2 \right),
% \end{equation}

% where $ h^{'a}_n = f_{\theta}(g_{W}(h^a_n))$, $ h^{'b}_n = f_{\theta}(g_{W}(h^b_n))$. 

% Furthermore, inspired by the pioneering work \cite{dwibedi2021little} as well as the similar idea of stochastic neighbour embedding, we construct the support set $S$ from the CLIP representations $h^b$, where $S = \{ {h^b}_n \}_{n=1}^N$. Thus, we aim to find the nearest neighbour of $h^a$ in the support set $S$, then minimize the Euclidean distance between the view $h^a$ and its nearest neighbour $S_{\text{near}}: h^a_{\text{near}} \in S$. (For now, it is not plotted in figure 2)

% \begin{equation}
% l_3 = \frac{1}{N} \sum_{n=1}^{N} \left\| g_{w}(S_{\text{near}}) - g_{w}(h^b_n) \right\|_2^2
% \end{equation}

% Thus, for $g_w(\cdot)$ and $d_\theta(\cdot)$, our method is trying to optimize the following objection functions:

% \begin{equation}
% \min_{w,\theta} l_1 + l_2 + \alpha l_3
% \end{equation}

% we set $\alpha = 0.1$ after analysis. 

% After sufficient training, such as 150 epochs on a benchmark dataset, for the images in the test set $Y$, we obtain their CLIP representations $Y_c = f(Y)$ without the need for data augmentation. We then use the encoder to embed $Y_c$ into $Z_c = g(Y_c)$. Finally, we perform K-means clustering on $Z_c$ to obtain the clusters $\mathbf{C}$, represented by an $M \times k$ one-hot matrix, where $k$ is the number of clusters and $M$ is the number of instances in the test set.

% 

% \begin{algorithm}
% \caption{Clip-based Auto-Encoder Learning (CAEL)}
% \begin{algorithmic}[1]
% \State \textbf{Input:} Dataset \(X\); Two augmented views \(x^a\) and \(x^b\); Number of epochs \(T\); Number of epochs \(T_{\text{refine}}\); Support set \(S = \{ h^b_n \}_{n=1}^{N}\); Weights \(\alpha\); CLIP model \(f(\cdot)\); Encoder \(g_w(\cdot)\); Decoder \(d_\theta(\cdot)\); Projector \(f_p(\cdot)\); Identity matrix \(I_k\)
% \State \textbf{Output:} Cluster assignments \(\mathbf{C}\)

% \vspace{0.3cm}
% -------------------------------------------------------------

% \State \textit{Phase 1: Training}
% \vspace{0.3cm}

% \State Initialize parameters \(w\) and \(\theta\)
% \For{each epoch \(t = 1\) to \(T\)}
%     \For{each image \(x \in X\)}
%         \State Obtain CLIP representations: \(h^a = f(x^a)\) and \(h^b = f(x^b)\)
        
%         \vspace{0.3cm}
%         \State Obtain latent representations: \(z^a = g_w(h^a)\) and \(z^b = g_w(h^b)\)
%          \State Compute loss \(l_1 = \frac{1}{N} \sum_{n=1}^{N} \left\| g_w(h^a_n) - g_w(h^b_n) \right\|_2^2\)

%           \vspace{0.3cm}
    
%         \State Obtain reconstruct representations: \(h^{'a} = d_\theta(z^a)\) and \(h^{'b} = d_\theta(z^b)\)
%           \State Compute loss \(l_2 = \frac{1}{2N} \sum_{n=1}^{N} \left( \left\| h^{'a}_n - h^a_n \right\|_2^2 + \left\| h^{'b}_n - h^b_n \right\|_2^2 \right)\)
    
%         \vspace{0.3cm}

%              \State Find nearest neighbor \(S_{\text{near}} = h^a_{\text{near}} \in S\)
%         \State Compute loss \(l_3 = \frac{1}{N} \sum_{n=1}^{N} \left\| g_w(S_{\text{near}}) - g_w(h^b_n) \right\|_2^2\)
        
%          \vspace{0.3cm}
        

        
%         \vspace{0.3cm}
        
%         \State Update parameters \(w\) and \(\theta\) by minimizing \(l = l_1 + l_2 + \alpha l_3\)
%         \vspace{0.3cm}
        
%     \EndFor
% \EndFor

% \vspace{0.3cm}
% -------------------------------------------------------------

% \State \textit{Phase 2: Clustering and refinement}
% \vspace{0.3cm}


% \State Obtain CLIP representations without data augmentation  \(H = f(X)\)
% \vspace{0.3cm}
% \State Obtain latent representations: \(Z_c = g_w(H)\)
% \vspace{0.3cm}
% \State Perform K-means clustering on \(Z_c\) to get initial clusters \(\mathbf{C}\)
% \vspace{0.3cm}

% \For{each epoch \(t = 1\) to \(T_{\text{refine}}\)}
%     \For{each latent representation \(z_n \in Z_c\)}
%     \vspace{0.3cm}
    
%         \State obtain \(k\)-simplex representation: \(P_c = f_p(Z_c)\)
%         \State Compute the loss \(l_{\text{refine}} = \frac{1}{M} \sum_{m=1}^{M} \left\| P_c - \mathbf{C}_{m \times k} \right\|_2^p\)
%         \vspace{0.3cm}
%         \State Update parameters of \(f_p\) to minimize \(l_{\text{refine}}\)
        
%         \vspace{0.3cm}
% \EndFor
% \EndFor
% \vspace{0.3cm}
% \State Reassign instances to clusters based on the nearest distance to identity matrix \(I_k\):
% \State \(C_k \gets \{z_n : \|P_c - \mathbf{e}_k\| \le \min_{j=1,\dots,k} \|P_c - \mathbf{e}_j\| \}\)
% \vspace{0.3cm}
% \State \Return \(\mathbf{C}\)
% \end{algorithmic}
% \end{algorithm}

\section{Experimental Analysis}
\label{sec:exp}
We now describe in detail our experimental analysis. The experimental section is organized as follows:
%\begin{enumerate}[noitemsep,topsep=0pt,parsep=0pt,partopsep=0pt,leftmargin=0.5cm]
%\item 

\noindent In {\bf 
Section~\ref{exp:setup}}, we introduce the datasets and methods to evaluate the previously defined accuracy measures.

%\item
\noindent In {\bf 
Section~\ref{exp:qual}}, we illustrate the limitations of existing measures with some selected qualitative examples.

%\item 
\noindent In {\bf 
Section~\ref{exp:quant}}, we continue by measuring quantitatively the benefits of our proposed measures in terms of {\it robustness} to lag, noise, and normal/abnormal ratio.

%\item 
\noindent In {\bf 
Section~\ref{exp:separability}}, we evaluate the {\it separability} degree of accurate and inaccurate methods, using the existing and our proposed approaches.

%\item
\noindent In {\bf 
Section~\ref{sec:entropy}}, we conduct a {\it consistency} evaluation, in which we analyze the variation of ranks that an AD method can have with an accuracy measures used.

%\item 
\noindent In {\bf 
Section~\ref{sec:exectime}}, we conduct an {\it execution time} evaluation, in which we analyze the impact of different parameters related to the accuracy measures and the time series characteristics. 
We focus especially on the comparison of the different VUS implementations.
%\end{enumerate}

\begin{table}[tb]
\caption{Summary characteristics (averaged per dataset) of the public datasets of TSB-UAD (S.: Size, Ano.: Anomalies, Ab.: Abnormal, Den.: Density)}
\label{table:charac}
%\vspace{-0.2cm}
\footnotesize
\begin{center}
\scalebox{0.82}{
\begin{tabular}{ |r|r|r|r|r|r|} 
 \hline
\textbf{\begin{tabular}[c]{@{}c@{}}Dataset \end{tabular}} & 
\textbf{\begin{tabular}[c]{@{}c@{}}S. \end{tabular}} & 
\textbf{\begin{tabular}[c]{c@{}} Len.\end{tabular}} & 
\textbf{\begin{tabular}[c]{c@{}} \# \\ Ano. \end{tabular}} &
\textbf{\begin{tabular}[c]{c@{}c@{}} \# \\ Ab. \\ Points\end{tabular}} &
\textbf{\begin{tabular}[c]{c@{}c@{}} Ab. \\ Den. \\ (\%)\end{tabular}} \\ \hline
Dodgers \cite{10.1145/1150402.1150428} & 1 & 50400   & 133.0     & 5612.0  &11.14 \\ \hline
SED \cite{doi:10.1177/1475921710395811}& 1 & 100000   & 75.0     & 3750.0  & 3.7\\ \hline
ECG \cite{goldberger_physiobank_2000}   & 52 & 230351  & 195.6     & 15634.0  &6.8 \\ \hline
IOPS \cite{IOPS}   & 58 & 102119  & 46.5     & 2312.3   &2.1 \\ \hline
KDD21 \cite{kdd} & 250 &77415   & 1      & 196.5   &0.56 \\ \hline
MGAB \cite{markus_thill_2020_3762385}   & 10 & 100000  & 10.0     & 200.0   &0.20 \\ \hline
NAB \cite{ahmad_unsupervised_2017}   & 58 & 6301   & 2.0      & 575.5   &8.8 \\ \hline
NASA-M. \cite{10.1145/3449726.3459411}   & 27 & 2730   & 1.33      & 286.3   &11.97 \\ \hline
NASA-S. \cite{10.1145/3449726.3459411}   & 54 & 8066   & 1.26      & 1032.4   &12.39 \\ \hline
SensorS. \cite{YAO20101059}   & 23 & 27038   & 11.2     & 6110.4   &22.5 \\ \hline
YAHOO \cite{yahoo}  & 367 & 1561   & 5.9      & 10.7   &0.70 \\ \hline 
\end{tabular}}
\end{center}
\end{table}











\subsection{Experimental Setup and Settings}
\label{exp:setup}
%\vspace{-0.1cm}

\begin{figure*}[tb]
  \centering
  \includegraphics[width=1\linewidth]{figures/quality.pdf}
  %\vspace{-0.7cm}
  \caption{Comparison of evaluation measures (proposed measures illustrated in subplots (b,c,d,e); all others summarized in subplots (f)) on two examples ((A)AE and OCSM applied on MBA(805) and (B) LOF and OCSVM applied on MBA(806)), illustrating the limitations of existing measures for scores with noise or containing a lag. }
  \label{fig:quality}
  %\vspace{-0.1cm}
\end{figure*}

We implemented the experimental scripts in Python 3.8 with the following main dependencies: sklearn 0.23.0, tensorflow 2.3.0, pandas 1.2.5, and networkx 2.6.3. In addition, we used implementations from our TSB-UAD benchmark suite.\footnote{\scriptsize \url{https://www.timeseries.org/TSB-UAD}} For reproducibility purposes, we make our datasets and code available.\footnote{\scriptsize \url{https://www.timeseries.org/VUS}}
\newline \textbf{Datasets: } For our evaluation purposes, we use the public datasets identified in our TSB-UAD benchmark. The latter corresponds to $10$ datasets proposed in the past decades in the literature containing $900$ time series with labeled anomalies. Specifically, each point in every time series is labeled as normal or abnormal. Table~\ref{table:charac} summarizes relevant characteristics of the datasets, including their size, length, and statistics about the anomalies. In more detail:

\begin{itemize}
    \item {\bf SED}~\cite{doi:10.1177/1475921710395811}, from the NASA Rotary Dynamics Laboratory, records disk revolutions measured over several runs (3K rpm speed).
	\item {\bf ECG}~\cite{goldberger_physiobank_2000} is a standard electrocardiogram dataset and the anomalies represent ventricular premature contractions. MBA(14046) is split to $47$ series.
	\item {\bf IOPS}~\cite{IOPS} is a dataset with performance indicators that reflect the scale, quality of web services, and health status of a machine.
	\item {\bf KDD21}~\cite{kdd} is a composite dataset released in a SIGKDD 2021 competition with 250 time series.
	\item {\bf MGAB}~\cite{markus_thill_2020_3762385} is composed of Mackey-Glass time series with non-trivial anomalies. Mackey-Glass data series exhibit chaotic behavior that is difficult for the human eye to distinguish.
	\item {\bf NAB}~\cite{ahmad_unsupervised_2017} is composed of labeled real-world and artificial time series including AWS server metrics, online advertisement clicking rates, real time traffic data, and a collection of Twitter mentions of large publicly-traded companies.
	\item {\bf NASA-SMAP} and {\bf NASA-MSL}~\cite{10.1145/3449726.3459411} are two real spacecraft telemetry data with anomalies from Soil Moisture Active Passive (SMAP) satellite and Curiosity Rover on Mars (MSL).
	\item {\bf SensorScope}~\cite{YAO20101059} is a collection of environmental data, such as temperature, humidity, and solar radiation, collected from a sensor measurement system.
	\item {\bf Yahoo}~\cite{yahoo} is a dataset consisting of real and synthetic time series based on the real production traffic to some of the Yahoo production systems.
\end{itemize}


\textbf{Anomaly Detection Methods: }  For the experimental evaluation, we consider the following baselines. 

\begin{itemize}
\item {\bf Isolation Forest (IForest)}~\cite{liu_isolation_2008} constructs binary trees based on random space splitting. The nodes (subsequences in our specific case) with shorter path lengths to the root (averaged over every random tree) are more likely to be anomalies. 
\item {\bf The Local Outlier Factor (LOF)}~\cite{breunig_lof_2000} computes the ratio of the neighbor density to the local density. 
\item {\bf Matrix Profile (MP)}~\cite{yeh_time_2018} detects as anomaly the subsequence with the most significant 1-NN distance. 
\item {\bf NormA}~\cite{boniol_unsupervised_2021} identifies the normal patterns based on clustering and calculates each point's distance to normal patterns weighted using statistical criteria. 
\item {\bf Principal Component Analysis (PCA)}~\cite{aggarwal_outlier_2017} projects data to a lower-dimensional hyperplane. Outliers are points with a large distance from this plane. 
\item {\bf Autoencoder (AE)} \cite{10.1145/2689746.2689747} projects data to a lower-dimensional space and reconstructs it. Outliers are expected to have larger reconstruction errors. 
\item {\bf LSTM-AD}~\cite{malhotra_long_2015} use an LSTM network that predicts future values from the current subsequence. The prediction error is used to identify anomalies.
\item {\bf Polynomial Approximation (POLY)} \cite{li_unifying_2007} fits a polynomial model that tries to predict the values of the data series from the previous subsequences. Outliers are detected with the prediction error. 
\item {\bf CNN} \cite{8581424} built, using a convolutional deep neural network, a correlation between current and previous subsequences, and outliers are detected by the deviation between the prediction and the actual value. 
\item {\bf One-class Support Vector Machines (OCSVM)} \cite{scholkopf_support_1999} is a support vector method that fits a training dataset and finds the normal data's boundary.
\end{itemize}

\subsection{Qualitative Analysis}
\label{exp:qual}



We first use two examples to demonstrate qualitatively the limitations of existing accuracy evaluation measures in the presence of lag and noise, and to motivate the need for a new approach. 
These two examples are depicted in Figure~\ref{fig:quality}. 
The first example, in Figure~\ref{fig:quality}(A), corresponds to OCSVM and AE on the MBA(805) dataset (named MBA\_ECG805\_data.out in the ECG dataset). 

We observe in Figure~\ref{fig:quality}(A)(a.1) and (a.2) that both scores identify most of the anomalies (highlighted in red). However, the OCSVM score points to more false positives (at the end of the time series) and only captures small sections of the anomalies. On the contrary, the AE score points to fewer false positives and captures all abnormal subsequences. Thus we can conclude that, visually, AE should obtain a better accuracy score than OCSVM. Nevertheless, we also observe that the AE score is lagged with the labels and contains more noise. The latter has a significant impact on the accuracy of evaluation measures. First, Figure~\ref{fig:quality}(A)(c) is showing that AUC-PR is better for OCSM (0.73) than for AE (0.57). This is contradictory with what is visually observed from Figure~\ref{fig:quality}(A)(a.1) and (a.2). However, when using our proposed measure R-AUC-PR, OCSVM obtains a lower score (0.83) than AE (0.89). This confirms that, in this example, a buffer region before the labels helps to capture the true value of an anomaly score. Overall, Figure~\ref{fig:quality}(A)(f) is showing in green and red the evolution of accuracy score for the 13 accuracy measures for AE and OCSVM, respectively. The latter shows that, in addition to Precision@k and Precision, our proposed approach captures the quality order between the two methods well.

We now present a second example, on a different time series, illustrated in Figure~\ref{fig:quality}(B). 
In this case, we demonstrate the anomaly score of OCSVM and LOF (depicted in Figure~\ref{fig:quality}(B)(a.1) and (a.2)) applied on the MBA(806) dataset (named MBA\_ECG806\_data.out in the ECG dataset). 
We observe that both methods produce the same level of noise. However, LOF points to fewer false positives and captures more sections of the abnormal subsequences than OCSVM. 
Nevertheless, the LOF score is slightly lagged with the labels such that the maximum values in the LOF score are slightly outside of the labeled sections. 
Thus, as illustrated in Figure~\ref{fig:quality}(B)(f), even though we can visually consider that LOF is performing better than OCSM, all usual measures (Precision, Recall, F, precision@k, and AUC-PR) are judging OCSM better than AE. On the contrary, measures that consider lag (Rprecision, Rrecall, RF) rank the methods correctly. 
However, due to threshold issues, these measures are very close for the two methods. Overall, only AUC-ROC and our proposed measures give a higher score for LOF than for OCSVM.

\subsection{Quantitative Analysis}
\label{exp:case}

\begin{figure}[t]
  \centering
  \includegraphics[width=1\linewidth]{figures/eval_case_study.pdf}
  %\vspace*{-0.7cm}
  \caption{\commentRed{
  Comparison of evaluation measures for synthetic data examples across various scenarios. S8 represents the oracle case, where predictions perfectly align with labeled anomalies. Problematic cases are highlighted in the red region.}}
  %\vspace*{-0.5cm}
  \label{fig:eval_case_study}
\end{figure}
\commentRed{
We present the evaluation results for different synthetic data scenarios, as shown in Figure~\ref{fig:eval_case_study}. These scenarios range from S1, where predictions occur before the ground truth anomaly, to S12, where predictions fall within the ground truth region. The red-shaded regions highlight problematic cases caused by a lack of adaptability to lags. For instance, in scenarios S1 and S2, a slight shift in the prediction leads to measures (e.g., AUC-PR, F score) that fail to account for lags, resulting in a zero score for S1 and a significant discrepancy between the results of S1 and S2. Thus, we observe that our proposed VUS effectively addresses these issues and provides robust evaluations results.}

%\subsection{Quantitative Analysis}
%\subsection{Sensitivity and Separability Analysis}
\subsection{Robustness Analysis}
\label{exp:quant}


\begin{figure}[tb]
  \centering
  \includegraphics[width=1\linewidth]{figures/lag_sensitivity_analysis.pdf}
  %\vspace*{-0.7cm}
  \caption{For each method, we compute the accuracy measures 10 times with random lag $\ell \in [-0.25*\ell,0.25*\ell]$ injected in the anomaly score. We center the accuracy average to 0.}
  %\vspace*{-0.5cm}
  \label{fig:lagsensitivity}
\end{figure}

We have illustrated with specific examples several of the limitations of current measures. 
We now evaluate quantitatively the robustness of the proposed measures when compared to the currently used measures. 
We first evaluate the robustness to noise, lag, and normal versus abnormal points ratio. We then measure their ability to separate accurate and inaccurate methods.
%\newline \textbf{Sensitivity Analysis: } 
We first analyze the robustness of different approaches quantitatively to different factors: (i) lag, (ii) noise, and (iii) normal/abnormal ratio. As already mentioned, these factors are realistic. For instance, lag can be either introduced by the anomaly detection methods (such as methods that produce a score per subsequences are only high at the beginning of abnormal subsequences) or by human labeling approximation. Furthermore, even though lag and noises are injected, an optimal evaluation metric should not vary significantly. Therefore, we aim to measure the variance of the evaluation measures when we vary the lag, noise, and normal/abnormal ratio. We proceed as follows:

\begin{enumerate}[noitemsep,topsep=0pt,parsep=0pt,partopsep=0pt,leftmargin=0.5cm]
\item For each anomaly detection method, we first compute the anomaly score on a given time series.
\item We then inject either lag $l$, noise $n$ or change the normal/abnormal ratio $r$. For 10 different values of $l \in [-0.25*\ell,0.25*\ell]$, $n \in [-0.05*(max(S_T)-min(S_T)),0.05*(max(S_T)-min(S_T))]$ and $r \in [0.01,0.2]$, we compute the 13 different measures.
\item For each evaluation measure, we compute the standard deviation of the ten different values. Figure~\ref{fig:lagsensitivity}(b) depicts the different lag values for six AD methods applied on a data series in the ECG dataset.
\item We compute the average standard deviation for the 13 different AD quality measures. For example, figure~\ref{fig:lagsensitivity}(a) depicts the average standard deviation for ten different lag values over the AD methods applied on the MBA(805) time series.
\item We compute the average standard deviation for the every time series in each dataset (as illustrated in Figure~\ref{fig:sensitivity_per_data}(b to j) for nine datasets of the benchmark.
\item We compute the average standard deviation for the every dataset (as illustrated in Figure~\ref{fig:sensitivity_per_data}(a.1) for lag, Figure~\ref{fig:sensitivity_per_data}(a.2) for noise and Figure~\ref{fig:sensitivity_per_data}(a.3) for normal/abnormal ratio).
\item We finally compute the Wilcoxon test~\cite{10.2307/3001968} and display the critical diagram over the average standard deviation for every time series (as illustrated in Figure~\ref{fig:sensitivity}(a.1) for lag, Figure~\ref{fig:sensitivity}(a.2) for noise and Figure~\ref{fig:sensitivity}(a.3) for normal/abnormal ratio).
\end{enumerate}

%height=8.5cm,

\begin{figure}[tb]
  \centering
  \includegraphics[width=\linewidth]{figures/sensitivity_per_data_long.pdf}
%  %\vspace*{-0.3cm}
  \caption{Robustness Analysis for nine datasets: we report, over the entire benchmark, the average standard deviation of the accuracy values of the measures, under varying (a.1) lag, (a.2) noise, and (a.3) normal/abnormal ratio. }
  \label{fig:sensitivity_per_data}
\end{figure}

\begin{figure*}[tb]
  \centering
  \includegraphics[width=\linewidth]{figures/sensitivity_analysis.pdf}
  %\vspace*{-0.7cm}
  \caption{Critical difference diagram computed using the signed-rank Wilkoxon test (with $\alpha=0.1$) for the robustness to (a.1) lag, (a.2) noise and (a.3) normal/abnormal ratio.}
  \label{fig:sensitivity}
\end{figure*}

The methods with the smallest standard deviation can be considered more robust to lag, noise, or normal/abnormal ratio from the above framework. 
First, as stated in the introduction, we observe that non-threshold-based measures (such as AUC-ROC and AUC-PR) are indeed robust to noise (see Figure~\ref{fig:sensitivity_per_data}(a.2)), but not to lag. Figure~\ref{fig:sensitivity}(a.1) demonstrates that our proposed measures VUS-ROC, VUS-PR, R-AUC-ROC, and R-AUC-PR are significantly more robust to lag. Similarly, Figure~\ref{fig:sensitivity}(a.2) confirms that our proposed measures are significantly more robust to noise. However, we observe that, among our proposed measures, only VUS-ROC and R-AUC-ROC are robust to the normal/abnormal ratio and not VUS-PR and R-AUC-PR. This is explained by the fact that Precision-based measures vary significantly when this ratio changes. This is confirmed by Figure~\ref{fig:sensitivity_per_data}(a.3), in which we observe that Precision and Rprecision have a high standard deviation. Overall, we observe that VUS-ROC is significantly more robust to lag, noise, and normal/abnormal ratio than other measures.




\subsection{Separability Analysis}
\label{exp:separability}

%\newline \textbf{Separability Analysis: } 
We now evaluate the separability capacities of the different evaluation metrics. 
\commentRed{The main objective is to measure the ability of accuracy measures to separate accurate methods from inaccurate ones. More precisely, an appropriate measure should return accuracy scores that are significantly higher for accurate anomaly scores than for inaccurate ones.}
We thus manually select accurate and inaccurate anomaly detection methods and verify if the accuracy evaluation scores are indeed higher for the accurate than for the inaccurate methods. Figure~\ref{fig:separability} depicts the latter separability analysis applied to the MBA(805) and the SED series. 
The accurate and inaccurate anomaly scores are plotted in green and red, respectively. 
We then consider 12 different pairs of accurate/inaccurate methods among the eight previously mentioned anomaly scores. 
We slightly modify each score 50 different times in which we inject lag and noises and compute the accuracy measures. 
Figure~\ref{fig:separability}(a.4) and Figure~\ref{fig:separability}(b.4) are divided into four different subplots corresponding to 4 pairs (selected among the twelve different pairs due to lack of space). 
Each subplot corresponds to two box plots per accuracy measure. 
The green and red box plots correspond to the 50 accuracy measures on the accurate and inaccurate methods. 
If the red and green box plots are well separated, we can conclude that the corresponding accuracy measures are separating the accurate and inaccurate methods well. 
We observe that some accuracy measures (such as VUS-ROC) are more separable than others (such as RF). We thus measure the separability of the two box-plots by computing the Z-test. 

\begin{figure*}[tb]
  \centering
  \includegraphics[width=1\linewidth]{figures/pairwise_comp_example_long.pdf}
  %\vspace*{-0.5cm}
  \caption{Separability analysis applied on 4 pairs of accurate (green) and inaccurate (red) methods on (a) the MBA(805) data series, and (b) the SED data series.}
  %\vspace*{-0.3cm}
  \label{fig:separability}
\end{figure*}

We now aggregate all the results and compute the average Z-test for all pairs of accurate/inaccurate datasets (examples are shown in Figures~\ref{fig:separability}(a.2) and (b.2) for accurate anomaly scores, and in Figures~\ref{fig:separability}(a.3) and (b.3) for inaccurate anomaly scores, for the MBA(805) and SED series, respectively). 
Next, we perform the same operation over three different data series: MBA (805), MBA(820), and SED. 
Then, we depict the average Z-test for these three datasets in Figure~\ref{fig:separability_agg}(a). 
Finally, we show the average Z-test for all datasets in Figure~\ref{fig:separability_agg}(b). 


We observe that our proposed VUS-based and Range-based measures are significantly more separable than other current accuracy measures (up to two times for AUC-ROC, the best measures of all current ones). Furthermore, when analyzed in detail in Figure~\ref{fig:separability} and Figure~\ref{fig:separability_agg}, we confirm that VUS-based and Range-based are more separable over all three datasets. 

\begin{figure}[tb]
  \centering
  \includegraphics[width=\linewidth]{figures/agregated_sep_analysis.pdf}
  %\vspace*{-0.5cm}
  \caption{Overall separability analysis (averaged z-test between the accuracy values distributions of accurate and inaccurate methods) applied on 36 pairs on 3 datasets.}
  \label{fig:separability_agg}
\end{figure}


\noindent \textbf{Global Analysis: } Overall, we observe that VUS-ROC is the most robust (cf. Figure~\ref{fig:sensitivity}) and separable (cf. Figure~\ref{fig:separability_agg}) measure. 
On the contrary, Precision and Rprecision are non-robust and non-separable. 
Among all previous accuracy measures, only AUC-ROC is robust and separable. 
Popular measures, such as, F, RF, AUC-ROC, and AUC-PR are robust but non-separable.

In order to visualize the global statistical analysis, we merge the robustness and the separability analysis into a single plot. Figure~\ref{fig:global} depicts one scatter point per accuracy measure. 
The x-axis represents the averaged standard deviation of lag and noise (averaged values from Figure~\ref{fig:sensitivity_per_data}(a.1) and (a.2)). The y-axis corresponds to the averaged Z-test (averaged value from Figure~\ref{fig:separability_agg}). 
Finally, the size of the points corresponds to the sensitivity to the normal/abnormal ratio (values from Figure~\ref{fig:sensitivity_per_data}(a.3)). 
Figure~\ref{fig:global} demonstrates that our proposed measures (located at the top left section of the plot) are both the most robust and the most separable. 
Among all previous accuracy measures, only AUC-ROC is on the top left section of the plot. 
Popular measures, such as, F, RF, AUC-ROC, AUC-PR are on the bottom left section of the plot. 
The latter underlines the fact that these measures are robust but non-separable.
Overall, Figure~\ref{fig:global} confirms the effectiveness and superiority of our proposed measures, especially of VUS-ROC and VUS-PR.


\begin{figure}[tb]
  \centering
  \includegraphics[width=\linewidth]{figures/final_result.pdf}
  \caption{Evaluation of all measures based on: (y-axis) their separability (avg. z-test), (x-axis) avg. standard deviation of the accuracy values when varying lag and noise, (circle size) avg. standard deviation of the accuracy values when varying the normal/abnormal ratio.}
  \label{fig:global}
\end{figure}




\subsection{Consistency Analysis}
\label{sec:entropy}

In this section, we analyze the accuracy of the anomaly detection methods provided by the 13 accuracy measures. The objective is to observe the changes in the global ranking of anomaly detection methods. For that purpose, we formulate the following assumptions. First, we assume that the data series in each benchmark dataset are similar (i.e., from the same domain and sharing some common characteristics). As a matter of fact, we can assume that an anomaly detection method should perform similarly on these data series of a given dataset. This is confirmed when observing that the best anomaly detection methods are not the same based on which dataset was analyzed. Thus the ranking of the anomaly detection methods should be different for different datasets, but similar for every data series in each dataset. 
Therefore, for a given method $A$ and a given dataset $D$ containing data series of the same type and domain, we assume that a good accuracy measure results in a consistent rank for the method $A$ across the dataset $D$. 
The consistency of a method's ranks over a dataset can be measured by computing the entropy of these ranks. 
For instance, a measure that returns a random score (and thus, a random rank for a method $A$) will result in a high entropy. 
On the contrary, a measure that always returns (approximately) the same ranks for a given method $A$ will result in a low entropy. 
Thus, for a given method $A$ and a given dataset $D$ containing data series of the same type and domain, we assume that a good accuracy measure results in a low entropy for the different ranks for method $A$ on dataset $D$.

\begin{figure*}[tb]
  \centering
  \includegraphics[width=\linewidth]{figures/entropy_long.pdf}
  %\vspace*{-0.5cm}
  \caption{Accuracy evaluation of the anomaly detection methods. (a) Overall average entropy per category of measures. Analysis of the (b) averaged rank and (c) averaged rank entropy for each method and each accuracy measure over the entire benchmark. Example of (b.1) average rank and (c.1) entropy on the YAHOO dataset, KDD21 dataset (b.2, c.2). }
  \label{fig:entropy}
\end{figure*}

We now compute the accuracy measures for the nine different methods (we compute the anomaly scores ten different times, and we use the average accuracy). 
Figures~\ref{fig:entropy}(b.1) and (b.2) report the average ranking of the anomaly detection methods obtained on the YAHOO and KDD21 datasets, respectively. 
The x-axis corresponds to the different accuracy measures. We first observe that the rankings are more separated using Range-AUC and VUS measures for these two datasets. Figure~\ref{fig:entropy}(b) depicts the average ranking over the entire benchmark. The latter confirms the previous observation that VUS measures provide more separated rankings than threshold-based and AUC-based measures. We also observe an interesting ranking evolution for the YAHOO dataset illustrated in Figure~\ref{fig:entropy}(b.1). We notice that both LOF and MatrixProfile (brown and pink curve) have a low rank (between 4 and 5) using threshold and AUC-based measures. However, we observe that their ranks increase significantly for range-based and VUS-based measures (between 2.5 and 3). As we noticed by looking at specific examples (see Figure~\ref{exp:qual}), LOF and MatrixProfile can suffer from a lag issue even though the anomalies are well-identified. Therefore, the range-based and VUS-based measures better evaluate these two methods' detection capability.


Overall, the ranking curves show that the ranks appear more chaotic for threshold-based than AUC-, Range-AUC-, and VUS-based measures. 
In order to quantify this observation, we compute the Shannon Entropy of the ranks of each anomaly detection method. 
In practice, we extract the ranks of methods across one dataset and compute Shannon's Entropy of the different ranks. 
Figures~\ref{fig:entropy}(c.1) and (c.2) depict the entropy of each of the nine methods for the YAHOO and KDD21 datasets, respectively. 
Figure~\ref{fig:entropy}(c) illustrates the averaged entropy for all datasets in the benchmark for each measure and method, while Figure~\ref{fig:entropy}(a) shows the averaged entropy for each category of measures.
We observe that both for the general case (Figure~\ref{fig:entropy}(a) and Figure~\ref{fig:entropy}(c)) and some specific cases (Figures~\ref{fig:entropy}(c.1) and (c.2)), the entropy is reducing when using AUC-, Range-AUC-, and VUS-based measures. 
We report the lowest entropy for VUS-based measures. 
Moreover, we notice a significant drop between threshold-based and AUC-based. 
This confirms that the ranks provided by AUC- and VUS-based measures are consistent for data series belonging to one specific dataset. 


Therefore, based on the assumption formulated at the beginning of the section, we can thus conclude that AUC, range-AUC, and VUS-based measures are providing more consistent rankings. Finally, as illustrated in Figure~\ref{fig:entropy}, we also observe that VUS-based measures result in the most ordered and similar rankings for data series from the same type and domain.










\subsection{Execution Time Analysis}
\label{sec:exectime}

In this section, we evaluate the execution time required to compute different evaluation measures. 
In Section~\ref{sec:synthetic_eval_time}, we first measure the influence of different time series characteristics and VUS parameters on the execution time. In Section~\ref{sec:TSB_eval_time}, we  measure the execution time of VUS (VUS-ROC and VUS-PR simultaneously), R-AUC (R-AUC-ROC and R-AUC-PR simultaneously), and AUC-based measures (AUC-ROC and AUC-PR simultaneously) on the TSB-UAD benchmark. \commentRed{As demonstrated in the previous section, threshold-based measures are not robust, have a low separability power, and are inconsistent. 
Such measures are not suitable for evaluating anomaly detection methods. Thus, in this section, we do not consider threshold-based measures.}


\subsubsection{Evaluation on Synthetic Time Series}\hfill\\
\label{sec:synthetic_eval_time}

We first analyze the impact that time series characteristics and parameters have on the computation time of VUS-based measures. 
to that effect, we generate synthetic time series and labels, where we vary the following parameters: (i) the number of anomalies {\bf$\alpha$} in the time series, (ii) the average \textbf{$\mu(\ell_a)$} and standard deviation $\sigma(\ell_a)$ of the anomalies lengths in the time series (all the anomalies can have different lengths), (iii) the length of the time series \textbf{$|T|$}, (iv) the maximum buffer length \textbf{$L$}, and (v) the number of thresholds \textbf{$N$}.


We also measure the influence on the execution time of the R-AUC- and AUC- related parameter, that is, the number of thresholds ($N$).
The default values and the range of variation of these parameters are listed in Table~\ref{tab:parameter_range_time}. 
For VUS-based measures, we evaluate the execution time of the initial VUS implementation, as well as the two optimized versions, VUS$_{opt}$ and VUS$_{opt}^{mem}$.

\begin{table}[tb]
    \centering
    \caption{Value ranges for the parameters: number of anomalies ($\alpha$), average and standard deviation anomaly length ($\mu(\ell_a)$,$\sigma(\ell_a)$), time series length ($|T|$), maximum buffer length ($L$), and number of thresholds ($N$).}
    \begin{tabular}{|c|c|c|c|c|c|c|} 
 \hline
 Param. & $\alpha$ & $\mu(\ell_a)$ & $\sigma(\ell_{a})$ & $|T|$ & $L$ & $N$ \\ [0.5ex] 
 \hline\hline
 \textbf{Default} & 10 & 10 & 0 & $10^5$ & 5 & 250\\ 
 \hline
 Min. & 0 & 0 & 0 & $10^3$ & 0 & 2 \\
 \hline
 Max. & $2*10^3$ & $10^3$ & $10$ & $10^5$ & $10^3$ & $10^3$ \\ [1ex] 
 \hline
\end{tabular}
    \label{tab:parameter_range_time}
\end{table}


Figure~\ref{fig:sythetic_exp_time} depicts the execution time (averaged over ten runs) for each parameter listed in Table~\ref{tab:parameter_range_time}. 
Overall, we observe that the execution time of AUC-based and R-AUC-based measures is significantly smaller than VUS-based measures.
In the following paragraph, we analyze the influence of each parameter and compare the experimental execution time evaluation to the theoretical complexity reported in Table~\ref{tab:complexity_summary}.

\vspace{0.2cm}
\noindent {\bf [Influence of $\alpha$]}:
In Figure~\ref{fig:sythetic_exp_time}(a), we observe that the VUS, VUS$_{opt}$, and VUS$_{opt}^{mem}$ execution times are linearly increasing with $\alpha$. 
The increase in execution time for VUS, VUS$_{opt}$, and VUS$_{opt}^{mem}$ is more pronounced when we vary $\alpha$, in contrast to $l_a$ (which nevertheless, has a similar effect on the overall complexity). 
We also observe that the VUS$_{opt}^{mem}$ execution time grows slower than $VUS_{opt}$ when $\alpha$ increases. 
This is explained by the use of 2-dimensional arrays for the storage of predictions, which use contiguous memory locations that allow for faster access, decreasing the dependency on $\alpha$.

\vspace{0.2cm}
\noindent {\bf [Influence of $\mu(\ell_a)$]}:
As shown in Figure~\ref{fig:sythetic_exp_time}(b), the execution time variation of VUS, VUS$_{opt}$, and VUS$_{opt}^{mem}$ caused by $\ell_a$ is rather insignificant. 
We also observe that the VUS$_{opt}$ and VUS$_{opt}^{mem}$ execution times are significantly lower when compared to VUS. 
This is explained by the smaller dependency of the complexity of these algorithms on the time series length $|T|$. 
Overall, the execution time for both VUS$_{opt}$ and VUS$_{opt}^{mem}$ is significantly lower than VUS, and follows a similar trend. 

\vspace{0.2cm}
\noindent {\bf [Influence of $\sigma(\ell_a)$]}: 
As depicted in Figure~\ref{fig:sythetic_exp_time}(d) and inferred from the theoretical complexities in Table~\ref{tab:complexity_summary}, none of the measures are affected by the standard deviation of the anomaly lengths.

\vspace{0.2cm}
\noindent {\bf [Influence of $|T|$]}:
For short time series (small values of $|T|$), we note that O($T_1$) becomes comparable to O($T_2$). 
Thus, the theoretical complexities approximate to $O(NL(T_1+T_2))$, $O(N*(T_1+T_2))+O(NLT_2)$ and $O(N(T_1+T_2))$ for VUS, VUS$_{opt}$, and VUS$_{opt}^{mem}$, respectively. 
Indeed, we observe in Figure~\ref{fig:sythetic_exp_time}(c) that the execution times of VUS, VUS$_{opt}$, and VUS$_{opt}^{mem}$ are similar for small values of $|T|$. However, for larger values of $|T|$, $O(T_1)$ is much higher compared to $O(T_2)$, thus resulting in an effective complexity of $O(NLT_1)$ for VUS, and $O(NT_1)$ for VUS$_{opt}$, and VUS$_{opt}^{mem}$. 
This translates to a significant improvement in execution time complexity for VUS$_{opt}$ and VUS$_{opt}^{mem}$ compared to VUS, which is confirmed by the results in Figure~\ref{fig:sythetic_exp_time}(c).

\vspace{0.2cm}
\noindent {\bf [Influence of $N$]}: 
Given the theoretical complexity depicted in Table~\ref{tab:complexity_summary}, it is evident that the number of thresholds affects all measures in a linear fashion.
Figure~\ref{fig:sythetic_exp_time}(e) demonstrates this point: the results of varying $N$ show a linear dependency for VUS, VUS$_{opt}$, and VUS$_{opt}^{mem}$ (i.e., a logarithmic trend with a log scale on the y axis). \commentRed{Moreover, we observe that the AUC and range-AUC execution time is almost constant regardless of the number of thresholds used. The latter is explained by the very efficient implementation of AUC measures. Therefore, the linear dependency on the number of thresholds is not visible in Figure~\ref{fig:sythetic_exp_time}(e).}

\vspace{0.2cm}
\noindent {\bf [Influence of $L$]}: Figure~\ref{fig:sythetic_exp_time}(f) depicts the influence of the maximum buffer length $L$ on the execution time of all measures. 
We observe that, as $L$ grows, the execution time of VUS$_{opt}$ and VUS$_{opt}^{mem}$ increases slower than VUS. 
We also observe that VUS$_{opt}^{mem}$ is more scalable with $L$ when compared to VUS$_{opt}$. 
This is consistent with the theoretical complexity (cf. Table~\ref{tab:complexity_summary}), which indicates that the dependence on $L$ decreases from $O(NL(T_1+T_2+\ell_a \alpha))$ for VUS to $O(NL(T_2+\ell_a \alpha)$ and $O(NL(\ell_a \alpha))$ for $VUS_{opt}$, and $VUS_{opt}^{mem}$.





\begin{figure*}[tb]
  \centering
  \includegraphics[width=\linewidth]{figures/synthetic_res.pdf}
  %\vspace*{-0.5cm}
  \caption{Execution time of VUS, R-AUC, AUC-based measures when we vary the parameters listed in Table~\ref{tab:parameter_range_time}. The solid lines correspond to the average execution time over 10 runs. The colored envelopes are to the standard deviation.}
  \label{fig:sythetic_exp_time}
\end{figure*}


\vspace{0.2cm}
In order to obtain a more accurate picture of the influence of each of the above parameters, we fit the execution time (as affected by the parameter values) using linear regression; we can then use the regression slope coefficient of each parameter to evaluate the influence of that parameter. 
In practice, we fit each parameter individually, and report the regression slope coefficient, as well as the coefficient of determination $R^2$.
Table~\ref{tab:parameter_linear_coeff} reports the coefficients mentioned above for each parameter associated with VUS, VUS$_{opt}$, and VUS$_{opt}^{mem}$.



\begin{table}[tb]
    \centering
    \caption{Linear regression slope coefficients ($C.$) for VUS execution times, for each parameter independently. }
    \begin{tabular}{|c|c|c|c|c|c|c|} 
 \hline
 Measure & Param. & $\alpha$ & $l_a$ & $|T|$ & $L$ & $N$\\ [0.5ex] 
 \hline\hline
 \multirow{2}{*}{$VUS$} & $C.$ & 21.9 & 0.02 & 2.13 & 212 & 6.24\\\cline{2-7}
 & {$R^2$} & 0.99 & 0.15 & 0.99 & 0.99 & 0.99 \\   
 \hline
  \multirow{2}{*}{$VUS_{opt}$} & $C.$ & 24.2  & 0.06 & 0.19 & 27.8 & 1.23\\\cline{2-7}
  & $R^2$& 0.99 & 0.86 & 0.99 & 0.99 & 0.99\\ 
 \hline
 \multirow{2}{*}{$VUS_{opt}^{mem}$} & $C.$ & 21.5 & 0.05 & 0.21 & 15.7 & 1.16\\\cline{2-7}
  & $R^2$ & 0.99 & 0.89 & 0.99 & 0.99 & 0.99\\[1ex] 
 \hline
\end{tabular}
    \label{tab:parameter_linear_coeff}
\end{table}

Table~\ref{tab:parameter_linear_coeff} shows that the linear regression between $\alpha$ and the execution time has a $R^2=0.99$. Thus, the dependence of execution time on $\alpha$ is linear. We also observe that VUS$_{opt}$ execution time is more dependent on $\alpha$ than VUS and VUS$_{opt}^{mem}$ execution time.
Moreover, the dependence of the execution time on the time series length ($|T|$) is higher for VUS than for VUS$_{opt}$ and VUS$_{opt}^{mem}$. 
More importantly, VUS$_{opt}$ and VUS$_{opt}^{mem}$ are significantly less dependent than VUS on the number of thresholds and the maximal buffer length. 







\subsubsection{Evaluation on TSB-UAD Time Series}\hfill\\
\label{sec:TSB_eval_time}

In this section, we verify the conclusions outlined in the previous section with real-world time series from the TSB-UAD benchmark. 
In this setting, the parameters $\alpha$, $\ell_a$, and $|T|$ are calculated from the series in the benchmark and cannot be changed. Moreover, $L$ and $N$ are parameters for the computation of VUS, regardless of the time series (synthetic or real). Thus, we do not consider these two parameters in this section.

\begin{figure*}[tb]
  \centering
  \includegraphics[width=\linewidth]{figures/TSB2.pdf}
  \caption{Execution time of VUS, R-AUC, AUC-based measures on the TSB-UAD benchmark, versus $\alpha$, $\ell_a$, and $|T|$.}
  \label{fig:TSB}
\end{figure*}

Figure~\ref{fig:TSB} depicts the execution time of AUC, R-AUC, and VUS-based measures versus $\alpha$, $\mu(\ell_a)$, and $|T|$.
We first confirm with Figure~\ref{fig:TSB}(a) the linear relationship between $\alpha$ and the execution time for VUS, VUS$_{opt}$ and VUS$_{opt}^{mem}$.
On further inspection, it is possible to see two separate lines for almost all the measures. 
These lines can be attributed to the time series length $|T|$. 
The convergence of VUS and $VUS_{opt}$ when $\alpha$ grows shows the stronger dependence that $VUS_{opt}$ execution time has on $\alpha$, as already observed with the synthetic data (cf. Section~\ref{sec:synthetic_eval_time}). 

In Figure~\ref{fig:TSB}(b), we observe that the variation of the execution time with $\ell_a$ is limited when compared to the two other parameters. We conclude that the variation of $\ell_a$ is not a key factor in determining the execution time of the measures.
Furthermore, as depicted in Figure~\ref{fig:TSB}(c), $VUS_{opt}$ and $VUS_{opt}^{mem}$ are more scalable than VUS when $|T|$ increases. 
We also confirm the linear dependence of execution time on the time series length for all the accuracy measures, which is consistent with the experiments on the synthetic data. 
The two abrupt jumps visible in Figure~\ref{fig:TSB}(c) are explained by significant increases of $\alpha$ in time series of the same length. 

\begin{table}[tb]
\centering
\caption{Linear regression slope coefficients ($C.$) for VUS execution time, for all time series parameters all-together.}
\begin{tabular}{|c|ccc|c|} 
 \hline
Measure & $\alpha$ & $|T|$ & $l_a$ & $R^2$ \\ [0.5ex] 
 \hline\hline
 \multirow{1}{*}{${VUS}$} & 7.87 & 13.5 & -0.08 & 0.99  \\ 
 %\cline{2-5} & $R^2$ & \multicolumn{3}{c|}{ 0.99}\\
 \hline
 \multirow{1}{*}{$VUS_{opt}$} & 10.2 & 1.70 & 0.09 & 0.96 \\
 %\cline{2-5} & $R^2$ & \multicolumn{3}{c|}{0.96}\\
\hline
 \multirow{1}{*}{$VUS_{opt}^{mem}$} & 9.27 & 1.60 & 0.11 & 0.96 \\
 %\cline{2-5} & $R^2$ & \multicolumn{3}{c|}{0.96} \\
 \hline
\end{tabular}
\label{tab:parameter_linear_coeff_TSB}
\end{table}



We now perform a linear regression between the execution time of VUS, VUS$_{opt}$ and VUS$_{opt}^{mem}$, and $\alpha$, $\ell_a$ and $|T|$.
We report in Table~\ref{tab:parameter_linear_coeff_TSB} the slope coefficient for each parameter, as well as the $R^2$.  
The latter shows that the VUS$_{opt}$ and VUS$_{opt}^{mem}$ execution times are impacted by $\alpha$ at a larger degree than $\alpha$ affects VUS. 
On the other hand, the VUS$_{opt}$ and VUS$_{opt}^{mem}$ execution times are impacted to a significantly smaller degree by the time series length when compared to VUS. 
We also confirm that the anomaly length does not impact the execution time of VUS, VUS$_{opt}$, or VUS$_{opt}^{mem}$.
Finally, our experiments show that our optimized implementations VUS$_{opt}$ and VUS$_{opt}^{mem}$ significantly speedup the execution of the VUS measures (i.e., they can be computed within the same order of magnitude as R-AUC), rendering them practical in the real world.











\subsection{Summary of Results}


Figure~\ref{fig:overalltable} depicts the ranking of the accuracy measures for the different tests performed in this paper. The robustness test is divided into three sub-categories (i.e., lag, noise, and Normal vs. abnormal ratio). We also show the overall average ranking of all accuracy measures (most right column of Figure~\ref{fig:overalltable}).
Overall, we see that VUS-ROC is always the best, and VUS-PR and Range-AUC-based measures are, on average, second, third, and fourth. We thus conclude that VUS-ROC is the overall winner of our experimental analysis.

\commentRed{In addition, our experimental evaluation shows that the optimized version of VUS accelerates the computation by a factor of two. Nevertheless, VUS execution time is still significantly slower than AUC-based approaches. However, it is important to mention that the efficiency of accuracy measures is an orthogonal problem with anomaly detection. In real-time applications, we do not have ground truth labels, and we do not use any of those measures to evaluate accuracy. Measuring accuracy is an offline step to help the community assess methods and improve wrong practices. Thus, execution time should not be the main criterion for selecting an evaluation measure.}

\section{Conclusion}
In this work, we set out to determine whether a simple DC pipeline could achieve the performance of other recent clustering frameworks. Specifically, we aimed to achieve this without relying on text, as we argue that doing so would make image-based clustering much more practical for downstream applications. This is because image-pair datasets are highly unlikely to be readily available to practitioners for most real-world use cases. Moreover, we aimed to develop a method that is simple to train without the need for additional feature layers, support sets, teacher-student networks, or exponential moving averages.

To meet this desiderata we proposed SCP, a novel, simple, and lightweight clustering method that achieves SOTA and near-SOTA performance in several datasets such as STL-10, CIFAR-10, CIFAR-20, ImageNet-10, and ImageNet-Dogs. In particular, SCP leverages pre-trained CLIP and DINO backbones, and we anticipate these could be updated in the future with better models, or ensembles.


\paragraph{Limitations} Despite these promising results, several challenges remain. Firstly, our pipeline relies on data augmentations, which can be time-consuming when scaling to larger datasets such as ImageNet-1K. Also, our approach requires prior knowledge of the number of clusters, which can be difficult to determine in practical applications unless the practitioner poses strong domain knowledge. Lastly, our frameworks performance is bounded by the quality of the pre-trained model used. However, we anticipate these to continue to improve in the future.








\clearpage
\section*{Impact Statement}
This paper presents work whose goal is to advance the field of 
Machine Learning. There are many potential societal consequences 
of our work, none which we feel must be specifically highlighted here.
\bibliography{references}
\bibliographystyle{unsrtnat} %%% Uncomment this line and comment out the ``thebibliography'' section below to use the external .bib file (using bibtex) .


%%% Uncomment this section and comment out the \bibliography{references} line above to use inline references.
% \begin{thebibliography}{1}

% 	\bibitem{kour2014real}
% 	George Kour and Raid Saabne.
% 	\newblock Real-time segmentation of on-line handwritten arabic script.
% 	\newblock In {\em Frontiers in Handwriting Recognition (ICFHR), 2014 14th
% 			International Conference on}, pages 417--422. IEEE, 2014.

% 	\bibitem{kour2014fast}
% 	George Kour and Raid Saabne.
% 	\newblock Fast classification of handwritten on-line arabic characters.
% 	\newblock In {\em Soft Computing and Pattern Recognition (SoCPaR), 2014 6th
% 			International Conference of}, pages 312--318. IEEE, 2014.

% 	\bibitem{hadash2018estimate}
% 	Guy Hadash, Einat Kermany, Boaz Carmeli, Ofer Lavi, George Kour, and Alon
% 	Jacovi.
% 	\newblock Estimate and replace: A novel approach to integrating deep neural
% 	networks with existing applications.
% 	\newblock {\em arXiv preprint arXiv:1804.09028}, 2018.

% \end{thebibliography}
\newpage
\appendix
\onecolumn
\section{Additional Experiment Details}
\label{s:Details}


\subsection{Image Search Pipeline}
\subsubsection{CLIP for Image search}

\begin{figure}[ht]
    \centering
    \includegraphics[width=0.75\textwidth]{images/image-search-clip.png}
    \caption{CLIP Image-to-Image Search baseline depicted in Fig. \ref{fig:visualization}. }
    \label{fig:CLIP_image_search}
\end{figure}

Figure \ref{fig:CLIP_image_search} illustrates the CLIP baseline for image-to-image search. The image repository comprises the STL-10 test split, with a target image randomly selected for the search. The search process ranks images based on Euclidean distances, displaying the top 20 most similar results. The results indicate that CLIP effectively learns discriminative features, although two mismatched images are still present in the top-20 neighborhood.

\subsubsection{SCP for Image search}
Figure \ref{fig:image search} illustrates the SCP-CLIP approach for image-to-image search. SCP-CLIP successfully retrieves images from the same cluster within the top-20 neighborhood. 

\begin{figure}[ht]
    \centering
    \includegraphics[width=0.75\textwidth]{images/image-search-scp.png}
    \caption{SCP-CLIP Image-to-Image Search Pipeline depicted in Fig. \ref{fig:visualization}, utilizing the normalized outputs of the trained cluster head $g(.)$ without applying softmax. }
    \label{fig:image search}
\end{figure}



\subsection{Visualization}
\label{s:Details__ss:Visualization}
We create a t-SNE visualization image cloud of STL-10 test set to support SCP effectiveness further. It was observed that both SCP-DINO and SCP-CLIP successfully distinguished most planes, ships, horses, and deer, even when they share a blue background or similar body shape.

\clearpage
\begin{figure}[p]
    \centering
    \includegraphics[width=\textwidth]{images/image_cloud.png}
    \caption{The t-SNE representations learned by SCP-DINO on STL-10. $ACC: 98.4\%$ }
    \label{fig:image_cloud_DINO}
\end{figure}


\begin{figure}[p]
    \centering
    \includegraphics[width=\textwidth]{images/image_Cloud_CLIP.png}
    \caption{The t-SNE representations learned by SCP-CLIP on STL-10. $ACC: 97.9\%$ }
    \label{fig:image_cloud_CLIP}
\end{figure}


\clearpage

\section{Proofs}
\label{s:Proofs}
\begin{proof}[{Proof of Proposition~\ref{prop:LosslessAmort}}]
By the Doob-Dyknin lemma, see e.g.~\citep[Lemma 1.14]{KallenbergProbability_2021}, since $Z$ is $\sigma(X)$-measurable if and only if there exists some Borel measurable function $g:(\mathbb{R}^d,\mathcal{B}(\mathbb{R}^d))\to (\mathbb{R}^D,\mathcal{B}(\mathbb{R}^D))$ such that
\begin{equation}
\label{eq:setup_map}
    Z = g(X)
.
\end{equation}
Now, define $F: (\mathbb{R}^d,\mathcal{B}(\mathbb{R}^d))\to 
(\mathbb{R}^n,\mathcal{B}(\mathbb{R}^n))$ by composition, as sending every $x\in \mathbb{R}^d$ to
\begin{equation}
\label{eq:conclusion}
        F(x) 
    \eqdef 
        f(x,g(x))
.
\end{equation}
Then, together~\eqref{eq:setup_map}, the definition of $Y$ in~\eqref{eq:uncompressed}, and that of $F$ in~\eqref{eq:conclusion} yield
\allowdisplaybreaks
\begin{align}
    Y = f(X,Z) = f(X,g(X)) = F(X)
\end{align}
which establishes our claim.
\end{proof}


\begin{proof}[{Proof of Theorem~\ref{thrm:DCplus}}]
The law of $X$, namely the pushforward of $\mathbb{P}$ by $X$ denoted here by $\mu\eqdef X_{\#}\mathbb{P}$, is a Borel probability measure.  Therefore,~\citep[Theorem 13.6]{KlenkeBook_2020} implies that $\mu$ is a  Radon measure.  
Since, in addition, $\mathbb{R}^d$ is a second-countable and locally-compact topological space then, the version of Lusin's Theorem found in~\citep[Exercise 13.1.3]{KlenkeBook_2020} , applies.  Whereby, we deduce that for every $\varepsilon\in (0,1]$ there exists a compact subset $\mathcal{K}_{\varepsilon}$ of $\mathbb{R}^d$ satisfying
\begin{equation}
\label{eq:High_Prob_Lusin}
    \mu(\mathcal{K}_{\varepsilon})\ge 1-\varepsilon
\end{equation}
and for which the restriction of the Borel measurable map $F$ to $\mathcal{K}_{\varepsilon}$ is continuous.  

Now, fix a continuous activation function $\rho:\mathbb{R}\to \mathbb{R}$ with at least one point of continuous non-zero differentiability; i.e.\ satisfying~\citep[Assumption 1]{kratsios2022universal} (due to~\cite{kidger2020universal}).  As shown in \citep[Example 13]{kratsios2022universal}, the softmax function satisfies~\citep[Assumption 8]{kratsios2022universal}; whence the general case of the non-Euclidean  universal approximation theorem~\citep[Theorem 37 (ii)]{kratsios2022universal} applies; from which we deduce the existence of an MLP $\hat{F}:\mathbb{R}^d\to \mathbb{R}^C$ with $\rho$ activation function satisfying the uniform approximation guarantee
\begin{equation}
\label{eq:uniform_approx}
    \sup_{x\in \mathcal{K}_{\varepsilon}}\,
        |F(x)-\operatorname{softmax}\circ \hat{F}(x)|
    <
        \varepsilon
.
\end{equation}
Define the failure set
\[
B\eqdef \biggl\{x\in \mathbb{R}^d:\,
|F(x)-\operatorname{softmax}\circ \hat{F}(x)|
\ge \varepsilon
\biggr\}
.
\]
Bfiy construction, $\mathcal{K}_{\varepsilon}\subseteq \mathbb{R}^d\setminus B $.
Furthermore, observe that: since $L:\mathbb{R}^d\to |F(x)-\operatorname{softmax}\circ \hat{F}(x)|\in \mathbb{R}$ is the composition of measurable and continuous functions then it, too is measurable; whence, $B\eqdef L^{-1}[[\varepsilon,\infty)]$; must itself measurable.  

Consequentially, the following computations are well founded:
Combining~\eqref{eq:High_Prob_Lusin} and~\eqref{eq:uniform_approx} we deduce that
\allowdisplaybreaks
\begin{align}
\label{eq:finishme}
        \mathbb{P}\biggl(
            |
                F(X)
                -
                \operatorname{softmax}\circ \hat{F}(X)
            |
            <
            \varepsilon
        \biggr)
    & =
        \mu\biggl(
            |
                F(x)
                -
                \operatorname{softmax}\circ \hat{F}(x)
            |
            <
            \varepsilon
        \biggr)
\\
\nonumber
    & =
        \mu(\mathbb{R}^d\setminus B)
\\
\nonumber
    & \ge 
        \mu(\mathcal{K}_{\varepsilon})
\\
\label{eq:finishme__END}
    & \ge 1- \varepsilon  
.
\end{align}
Finally, by Proposition~\ref{prop:LosslessAmort} and~\eqref{eq:uncompressed}, we know that
\begin{equation}
\label{eq:rep_me}
F(X)=f(X,Z)
.
\end{equation}
Whence, incorporating the identity in~\eqref{eq:rep_me} into the left-hand side of the chain of inequalities in~\eqref{eq:finishme}-\eqref{eq:finishme__END} implies that
\allowdisplaybreaks
\begin{align}
\label{eq:finishme2}
        \mathbb{P}\biggl(
                |
                    f(X,Z)
                    -
                    \operatorname{softmax}\circ \hat{F}(X)
                |
            <
                \varepsilon
        \biggr)
    =
        \mathbb{P}\biggl(
                |
                    F(X)
                    -
                    \operatorname{softmax}\circ \hat{F}(X)
                |
            <
                \varepsilon
        \biggr)
    \ge 1- \varepsilon  
\end{align}
which concludes our proof.
\end{proof}



% \section*{Acknowledgments}
% A.\ Kratsios and Y.\ Li acknowledge financial support from an NSERC Discovery Grant No.\ RGPIN-2023-04482 and No.\ DGECR-2023-00230.  A.\ Kratsios also acknowledges that resources used in preparing this research were provided, in part, by the Province of Ontario, the Government of Canada through CIFAR, and companies sponsoring the Vector Institute\footnote{\href{https://vectorinstitute.ai/partnerships/current-partners/}{https://vectorinstitute.ai/partnerships/current-partners/}}.

%%%%%%%%%%%%%%%%%%%%%%%%%%%%%%%%%%%%%%%%%%%%%%%%%%%%%%%%%%%%%%%%%%%%%%%%%%%%%%%
%%%%%%%%%%%%%%%%%%%%%%%%%%%%%%%%%%%%%%%%%%%%%%%%%%%%%%%%%%%%%%%%%%%%%%%%%%%%%%%


\end{document}
