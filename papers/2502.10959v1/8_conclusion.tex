\vspace{-10pt} 
\section{Conclusion} \label{sec:conclusion}

\sun{We summarize our findings on the relative performance of computing methods, key technical insights, and opportunities for future research and optimization below.}

\noindent\sun{\textbf{Relative Performance.} Guided by Equation \ref{eq:cost}, we summarize the efficiency $T_p$ of graph access operations and the concurrency control overhead amplification ratio $\alpha_p$ across competing methods. We focus on four key operators: \textsc{SearchVtx}, \textsc{SearchEdge}, \textsc{ScanNbr}, and \textsc{InsEdge}. The efficiency and overhead amplification ratios are based on experimental results in Sections \ref{sec:effiency_of_neighbor_indexes} and \ref{sec:scalability_evaluation}. Relative performance is calculated by first computing the geometric mean of throughput across datasets, then normalizing the results on a scale of 1 to 5, with 5 being the best. Figure \ref{fig:radar} visualizes the relative performance of the methods across different dimensions.}

\sun{For read efficiency, including \textsc{ScanNbr} and \textsc{SearchEdge}, AdjLst performs the best. Teseo is slower on \textsc{ScanNbr} and \textsc{SearchNbr}, but offers a balanced read and write performance. For \textsc{SearchVtx}, dynamic array storage for vertices achieves the best results. Aspen excels in read concurrency due to its coarse-grained version management but performs poorly in write concurrency because it only supports a single-writer.}

\noindent\sun{\textbf{Key Technical Insights.} Based on our extensive experimental results, we confirm the following key findings from previous work: 1) segmenting neighbor sets into blocks effectively balances read and write performance; 2) in the single update setting, fine-grained methods improve write throughput and outperform coarse-grained methods; 3) adaptive indexing significantly enhances performance by leveraging the sparsity of real-world graphs, reducing LLC misses; and 4) converting tree indexes to arrays improves the performance of long-running queries in coarse-grained methods.}



\begin{figure}[t]
	\setlength{\abovecaptionskip}{0pt}
	\setlength{\belowcaptionskip}{0pt}
		\captionsetup[subfigure]{aboveskip=0pt,belowskip=0pt}
	\centering
    % \includegraphics[width=0.45\textwidth]{img/exp_figs/radar_line.pdf}
	
		\includegraphics[width=0.60\textwidth]{img/exp_figs/radar.pdf}
		
        
	\caption{\sun{Relative performance of existing methods for factors in  Equation \ref{eq:cost}.}}
	\label{fig:radar}
\end{figure}


\sun{\noindent However, we find that current design choices in concurrency control and graph containers lead to significant performance issues, which diverge from modern trends and present new challenges for the community.}

\begin{itemize}[leftmargin=*]
    \item \sun{\textbf{The fine-grained strategy, widely adopted in recent works, introduces severe performance issues.} First, performing version checks for each neighbor imposes substantial computation overhead and consumes excessive memory bandwidth. Second, maintaining versions for each neighbor requires a large amount of memory, limiting scalability. Third, readers and writers interfere heavily, and insert operations face limited scalability due to lock contention, especially when accessing high-degree vertices.}
    
    \item \sun{\textbf{Existing research often focuses on selecting various block indexes but neglects hardware-level considerations.} This creates a performance gap compared to continuous storage methods. Fragmented memory layouts increase cache misses (L1, L2, LLC, and DTLB). Additionally, the complexity of segmented methods introduces high instruction overhead and frequent branch mispredictions.}
\end{itemize}

\noindent\sun{\textbf{Research and Optimization Opportunities.} This study identifies key areas for future research and optimization. In graph concurrency control, new version management techniques are needed to reduce the space and efficiency overhead of fine-grained versions, improving scalability. One research opportunity is to manage versions in a more coarse-grained manner, potentially eliminating the overhead of version checks for each edge. Additionally, more efficient concurrency control mechanisms are required to minimize interference among concurrent queries. Given that graph queries are typically read-intensive, another research direction is to explore relaxed locking requirements for read queries, thereby reducing interference from write operations.}

\sun{For graph containers, segmenting neighbor sets and using adaptive indexing can enhance graph access efficiency. However, further optimization of memory layout and management strategies is necessary to reduce cache misses and instruction overhead, improving hardware utilization. Developing graph containers that balance read and write efficiency while integrating improved concurrency control methods is essential, as a significant read efficiency gap remains between dynamic and static graph storage formats.}
