\section{A Common Abstraction for DGS}\label{sec:framework}

In this section, we propose a common abstraction for DGS to facilitate a systematical study of existing methods.

\subsection{Graph Query and Data Abstraction}\label{sec:data_abstraction}

In transaction management~\cite{ramakrishnan2002database}, a database is a set $\{x\}$ of tuples in tables. A transaction consists of a sequence of read and write operations on $\{x\}$, beginning with a \textsc{Begin} command and ending with either \textsc{Commit}, indicating successful execution, or \textsc{Abort}, indicating failure and reverting modifications. Building on this concept, we propose a simple yet effective multi-level abstraction for graph query and data that aims to: 1) reflect the characteristics of graph queries; 2) indicate the nature of graph data; and 3) capture graph data access patterns. Figure \ref{fig:data_abstraction} shows our abstraction.

\begin{figure}[h]\small
    \setlength{\abovecaptionskip}{3pt}
    \setlength{\belowcaptionskip}{0pt}
    \includegraphics[scale=0.75]{img/data_abstraction.pdf}
    \centering
    \caption{The abstraction of graph query and data.}
    \label{fig:data_abstraction}
\end{figure}

\noindent\textbf{Global Abstraction.} Graph queries are categorized into write queries $\Delta G$ and read queries $Q$ as discussed in Section \ref{sec:preliminaries}. The global abstraction models the relationship among these queries by maintaining a global timestamp $t(G)$, initialized to 0 and incremented by 1 only when $\Delta G$ is committed. To ensure the serializability of graph queries, DGS requires that each committed write query be uniquely identified by its commit timestamp, denoted as $\Delta G_{i + 1}$ for the write query committed at $t(G) = i$. This abstraction effectively captures the construction of a dynamic graph $G = (G_0, \Delta \mathcal{G})$, where $\Delta \mathcal{G}$ is the serial execution order of committed write queries. A read query $Q$ starting at $t(G) = i$ has a local timestamp $t(Q)$.

\vspace{2pt}
\noindent\textbf{Local Abstraction.} In cooperation with the global abstraction, the local abstraction allows us to drill down to each data object and their primitive operations. The graph $G$ consists of vertices and edges. To indicate their differences and interconnections, $G$ is organized into a vertex table containing $V(G)$ and a set of neighbor tables, each corresponding to a neighbor set $N(u)$. Each entry in $V(G)$ contains the vertex ID $u$, the location of $N(u)$, and its properties, while each entry in $N(u)$ contains the neighbor ID $v$ and the properties associated with edge $e(u, v)$. Given a write query $\Delta G_i$, each operation creates a new version of a vertex or neighbor $u$ with the timestamp $t(u) = i$. Specifically, an insert (resp. delete) operation on $u$ creates a new version with the \textsc{op-type} as $I$ (resp. $D$). Updating an element is performed via an insert operation.

Lemma \ref{lemma:isolation} can be proven using the dependency graph \cite{fekete2005making} within the multi-level abstraction.  The lemma shows that by maintaining the serializability of write queries, DGS can achieve serializable isolation by ensuring $Q$ has a consistent view of $G_i$. This is done by allowing $Q$ to access only the latest version of vertices or neighbors $u$ such that $t(u) \leqslant t(Q)$. This approach allows us to coordinate $Q$'s data access based on timestamps without complex concurrency control protocols for read queries. Although existing DGS methods use this optimization, they do not explicitly and formally discuss it.

\begin{lemma} \label{lemma:isolation}
Suppose DGS maintains the serializability of write queries with the serial execution order $\Delta \mathcal{G}$. Given a read query $Q$ starting at timestamp $i$, ensuring $Q$ has a consistent view of $G_i = G_0 \oplus...\oplus \Delta G_i$ guarantees global serializable isolation for read and write queries.
\end{lemma}
\emph{Proof.} In the dependency graph, each node $Q_x$ represents a query, and an edge from $Q_x$ to $Q_y$ indicates that an operation $o_x \in Q_x$ conflicts with $o_y \in Q_y$, and $o_x$ precedes $o_y$ in execution. Operations $o_x$ and $o_y$ conflict if they act on the same object and at least one is a write. Queries are conflict serializable \emph{iff} the dependency graph is acyclic. Since DGS maintains the serializability of write queries, the dependency graph formed by them is acyclic. Ensuring $Q$ has a consistent view of $G_i$ guarantees that all operations in $Q$ depend only on operations from preceding write queries. Thus, the node representing $Q$ in the dependency graph has no incoming edges. Therefore, the combined dependency graph of read and write queries remains acyclic, proving the queries are conflict serializable.




\subsection{Graph Operations Abstraction}\label{sec:operation_abstraction}

We next analyze graph data access patterns based on the multi-level abstraction. From the perspective of DGS, a write (or read) query consists of a sequence of write (or read) operations on vertices and edges, while the computation logic and state of the query are beyond the scope of DGS. These \emph{graph operations} include:

\begin{itemize}[leftmargin=*]
\item \textsc{InsVtx($u$)}: inserting a vertex $u$ into $V(G)$;
\item \textsc{InsEdge($u, v$)}: inserting an edge $e(u, v)$ into $E(G)$;
\item \textsc{SearchVtx($u$)}: finding a vertex $u$ in $V(G)$;
\item \textsc{SearchEdge($u, v$)}: finding an edge $e(u, v)$ in $E(G)$;
\item \textsc{ScanVtx($G$)}: traversing the vertex set $V(G)$;
\item \textsc{ScanNbr($u$)}: traversing the neighbor set $N(u)$.
\end{itemize}

\begin{figure}[htbp]\small
    \setlength{\abovecaptionskip}{0pt}
    \setlength{\belowcaptionskip}{0pt}
    \includegraphics[scale=0.75]{img/operaiton_abstraction.pdf}
    \centering
    \caption{The abstraction of graph operations.}
    \label{fig:primitive_opeartions}
\end{figure}

Figure \ref{fig:primitive_opeartions} presents the abstraction of graph operations, where each node represents a graph operation, and each edge denotes a primitive operator on vertex or neighbor tables. The path from the start node to an operation node illustrates the primitive operators required to perform the operation and shows the relationships between different operations (e.g., \textsc{InsEdge} invokes \textsc{SearchEdge}). In summary, this abstraction provides a unified execution routine on graph operations.

% From the abstraction, we observe that operations on neighbor tables depend on those on the vertex table to retrieve the position of the target neighbor table, the insert operation depends on the search operation to locate the target object, and any operation on $u$ or its neighbors requires access rights to $u$ first.

Let $P_V$ and $P_N$ denote the paths from the start node to the graph operation node in the vertex table and neighbor table operations, respectively, in Figure \ref{fig:primitive_opeartions}. Equation \ref{eq:cost} defines the cost $T$ of a graph operation, where $T_{CC}$ is the cost of coordinating access to the target vertex, and $T_p$ is the cost of the primitive operator $p$. Concurrency control requires the cooperation of underlying graph containers, such as checking a creation timestamp for each object. $\alpha_p$ is the overhead amplification ratio of concurrency control on the primitive operator $p$. An optimal value is 1, indicating no performance degradation due to concurrency control. This equation highlights the factors affecting DGS performance, serving as a tool to systematically study these performance factors rather than predicting the exact cost of a graph operation.

\begin{equation} \label{eq:cost}
T = T_{CC} + \sum_{p \in P_V} \alpha_p T_p + \sum_{p \in P_N} \alpha_p T_p.
\end{equation}

\noindent\textbf{Remark. } \sun{We exclude edge update and delete operations as they follow a similar process to insertions. Fine-grained methods like Sortledton, Teseo, and LiveGraph handle deletions by: 1) locating the target vertex, and 2) creating a new version marked as \emph{delete} or updating the end timestamp to indicate the deletion, as discussed above. Space is later reclaimed through garbage collection. Coarse-grained methods like Aspen, use a copy-on-write strategy, where deletion involves removing a vertex from a block rather than adding one, much like with insertions.} Graph query workloads generally focus on analyzing connections between vertices, with vertex updates, especially deletions, being rare \cite{zhu2019livegraph}. Consequently, existing DGS works typically focus on operations on neighbor tables. Therefore, this paper primarily focuses on $N(u)$ as well.
