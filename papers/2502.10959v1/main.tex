%%
%% This is file `sample-sigconf.tex',
%% generated with the docstrip utility.
%%
%% The original source files were:
%%
%% samples.dtx  (with options: `sigconf')
%% 
%% IMPORTANT NOTICE:
%% 
%% For the copyright see the source file.
%% 
%% Any modified versions of this file must be renamed
%% with new filenames distinct from sample-sigconf.tex.
%% 
%% For distribution of the original source see the terms
%% for copying and modification in the file samples.dtx.
%% 
%% This generated file may be distributed as long as the
%% original source files, as listed above, are part of the
%% same distribution. (The sources need not necessarily be
%% in the same archive or directory.)
%%
%% Commands for TeXCount
%TC:macro \cite [option:text,text]
%TC:macro \citep [option:text,text]
%TC:macro \citet [option:text,text]
%TC:envir table 0 1
%TC:envir table* 0 1
%TC:envir tabular [ignore] word
%TC:envir displaymath 0 word
%TC:envir math 0 word
%TC:envir comment 0 0
%%
%%
%% The first command in your LaTeX source must be the \documentclass command.
\pdfoutput=1
\documentclass[acmsmall, nonacm]{acmart}

\newcommand{\sun}[1]{{\textcolor{black}{#1}}}
\newcommand{\su}[1]{{\textcolor{black}{#1}}}
\usepackage{enumitem}
\usepackage{tabularx} 
\usepackage[linesnumbered,vlined,ruled]{algorithm2e}
\usepackage{multirow}
\usepackage{graphicx}
\usepackage{caption}
\usepackage{subcaption}
\usepackage{booktabs} 
\usepackage{geometry} 
\usepackage{hyperref} 
\usepackage{pgfplots}
\usepackage{placeins}
\usepackage{afterpage}
\pgfplotsset{compat=1.17}
\usepgfplotslibrary{groupplots}
\setlength{\tabcolsep}{4pt} 
\setlength{\textfloatsep}{1.5pt plus 0.0pt minus 2.0pt} 
\setlength{\floatsep}{3pt plus 0.0pt minus 3.0pt}  
\setlength{\intextsep}{3pt plus 0.0pt minus 3.0pt} 
\captionsetup{skip=0pt} 


%% NOTE that a single column version may be required for 
%% submission and peer review. This can be done by changing
%% the \doucmentclass[...]{acmart} in this template to 
%% \documentclass[manuscript,screen]{acmart}
%% 
%% To ensure 100% compatibility, please check the white list of
%% approved LaTeX packages to be used with the Master Article Template at
%% https://www.acm.org/publications/taps/whitelist-of-latex-packages 
%% before creating your document. The white list page provides 
%% information on how to submit additional LaTeX packages for 
%% review and adoption.
%% Fonts used in the template cannot be substituted; margin 
%% adjustments are not allowed.
%%
%%
%% \BibTeX command to typeset BibTeX logo in the docs
% \AtBeginDocument{%
%   \providecommand\BibTeX{{%
%     \normalfont B\kern-0.5em{\scshape i\kern-0.25em b}\kern-0.8em\TeX}}}

%% Rights management information.  This information is sent to you
%% when you complete the rights form.  These commands have SAMPLE
%% values in them; it is your responsibility as an author to replace
%% the commands and values with those provided to you when you
%% complete the rights form.
% \setcopyright{acmcopyright}
% \copyrightyear{2018}
% \acmYear{2018}
% \acmDOI{XXXXXXX.XXXXXXX}

%% These commands are for a PROCEEDINGS abstract or paper.
% \acmConference[Conference acronym 'XX]{Make sure to enter the correct
%   conference title from your rights confirmation emai}{June 03--05,
%   2018}{Woodstock, NY}
%
%  Uncomment \acmBooktitle if th title of the proceedings is different
%  from ``Proceedings of ...''!
%
%\acmBooktitle{Woodstock '18: ACM Symposium on Neural Gaze Detection,
%  June 03--05, 2018, Woodstock, NY} 
% \acmPrice{15.00}
% \acmISBN{978-1-4503-XXXX-X/18/06}


%%
%% Submission ID.
%% Use this when submitting an article to a sponsored event. You'll
%% receive a unique submission ID from the organizers
%% of the event, and this ID should be used as the parameter to this command.
%%\acmSubmissionID{123-A56-BU3}

%%
%% For managing citations, it is recommended to use bibliography
%% files in BibTeX format.
%%
%% You can then either use BibTeX with the ACM-Reference-Format style,
%% or BibLaTeX with the acmnumeric or acmauthoryear sytles, that include
%% support for advanced citation of software artefact from the
%% biblatex-software package, also separately available on CTAN.
%%
%% Look at the sample-*-biblatex.tex files for templates showcasing
%% the biblatex styles.
%%

%%
%% The majority of ACM publications use numbered citations and
%% references.  The command \citestyle{authoryear} switches to the
%% "author year" style.
%%
%% If you are preparing content for an event
%% sponsored by ACM SIGGRAPH, you must use the "author year" style of
%% citations and references.
%% Uncommenting
%% the next command will enable that style.
%%\citestyle{acmauthoryear}

%%
%% end of the preamble, start of the body of the document source.
%% \BibTeX command to typeset BibTeX logo in the docs \AtBeginDocument{%  \providecommand\BibTeX{{%Bib\TeX}}}

%% \setcopyright{cc}
%% \setcctype{CC-BY}
\setcopyright{rightsretained}
% \acmJournal{J.AX}
\acmYear{2025} \acmVolume{3} \acmNumber{1} \acmArticle{70} \acmMonth{2} \acmPrice{15.00}\acmDOI{XX.XX/XXX.XX}

% The following includes the CC license icon appropriate for your paper.
% Download the image from www.scomminc.com/pp/acmsig/4ACM-CC-by-88x31.eps
% and place within your figs or figures folder

\makeatletter
% \gdef\@copyrightpermission{
%   \begin{minipage}{0.2\columnwidth}
%    \href{https://creativecommons.org/licenses/by/4.0/}{\includegraphics[width=0.90\textwidth]{img/4ACM-CC-by-88x31.eps}}
%   \end{minipage}\hfill
%   \begin{minipage}{0.8\columnwidth}
%    \href{https://creativecommons.org/licenses/by/4.0/}{This work is licensed under a Creative Commons Attribution International 4.0 License.}
%   \end{minipage}
%   \vspace{5pt}
% }
\makeatother

\begin{document}

%%
%% The "title" command has an optional parameter,
%% allowing the author to define a "short title" to be used in page headers.

\title{Revisiting the Design of In-Memory Dynamic Graph Storage}

%%
%% The "author" command and its associated commands are used to define
%% the authors and their affiliations.
%% Of note is the shared affiliation of the first two authors, and the
%% "authornote" and "authornotemark" commands
%% used to denote shared contribution to the research.
\author{Jixian Su}
\affiliation{
  \institution{Shanghai Jiao Tong University}
  \city{Shanghai}
  \country{China}
}
\email{sjx13623816973@sjtu.edu.cn}

\author{Chiyu Hao}
\affiliation{
  \institution{Shanghai Jiao Tong University}
  \city{Shanghai}
  \country{China}
}
\email{hcahoi11@sjtu.edu.cn}

\author{Shixuan Sun}
\affiliation{
  \institution{Shanghai Jiao Tong University}
  \city{Shanghai}
  \country{China}
}
\email{sunshixuan@sjtu.edu.cn}

\author{Hao Zhang}
\affiliation{
  \institution{Huawei Cloud}
  \city{Beijing}
  \country{China}
}
\email{zhanghao687@huawei.com}

\author{Sen Gao}
\affiliation{
  \institution{National University of Singapore}
  % \city{Singapore}
  \country{Singapore}
}
\email{sen@u.nus.edu}

\author{Jiaxin Jiang}
\affiliation{
    \institution{National University of Singapore}
    % \city{Singapore}
    \country{Singapore}
}
\email{jxjiang@nus.edu.sg}

\author{Yao Chen}
\affiliation{
    \institution{National University of Singapore}
    % \city{Singapore}
    \country{Singapore}
}
\email{yaochen@nus.edu.sg}

\author{Chenyi Zhang}
\affiliation{
  \institution{Huawei Cloud}
  \city{Hangzhou}
  \country{China}
}
\email{zhangchenyi@huawei.com}


\author{Bingsheng He}
\affiliation{
    \institution{National University of Singapore}
    % \city{Singapore}
    \country{Singapore}
}
\email{hebs@comp.nus.edu.sg}



\author{Minyi Guo}
\affiliation{
    \institution{Shanghai Jiao Tong University}
    \city{Shanghai}
    \country{China}
}
\email{guo-my@cs.sjtu.edu.cn}

\renewcommand{\shortauthors}{Jixian Su et al.}
%%
%% By default, the full list of authors will be used in the page
%% headers. Often, this list is too long, and will overlap
%% other information printed in the page headers. This command allows
%% the author to define a more concise list
%% of authors' names for this purpose.
% \renewcommand{\shortauthors}{Trovato and Tobin, et al.}

%%
%% The abstract is a short summary of the work to be presented in the
%% article.
\begin{abstract}


The choice of representation for geographic location significantly impacts the accuracy of models for a broad range of geospatial tasks, including fine-grained species classification, population density estimation, and biome classification. Recent works like SatCLIP and GeoCLIP learn such representations by contrastively aligning geolocation with co-located images. While these methods work exceptionally well, in this paper, we posit that the current training strategies fail to fully capture the important visual features. We provide an information theoretic perspective on why the resulting embeddings from these methods discard crucial visual information that is important for many downstream tasks. To solve this problem, we propose a novel retrieval-augmented strategy called RANGE. We build our method on the intuition that the visual features of a location can be estimated by combining the visual features from multiple similar-looking locations. We evaluate our method across a wide variety of tasks. Our results show that RANGE outperforms the existing state-of-the-art models with significant margins in most tasks. We show gains of up to 13.1\% on classification tasks and 0.145 $R^2$ on regression tasks. All our code and models will be made available at: \href{https://github.com/mvrl/RANGE}{https://github.com/mvrl/RANGE}.

\end{abstract}


%%
%% The code below is generated by the tool at http://dl.acm.org/ccs.cfm.
%% Please copy and paste the code instead of the example below.
%%
\begin{CCSXML}
<ccs2012>
   <concept>
       <concept_id>10002951.10002952.10002953.10010146</concept_id>
       <concept_desc>Information systems~Graph-based database models</concept_desc>
       <concept_significance>500</concept_significance>
       </concept>
   <concept>
       <concept_id>10002951.10002952.10002971</concept_id>
       <concept_desc>Information systems~Data structures</concept_desc>
       <concept_significance>300</concept_significance>
       </concept>
   <concept>
       <concept_id>10002951.10003152.10003520</concept_id>
       <concept_desc>Information systems~Storage management</concept_desc>
       <concept_significance>100</concept_significance>
       </concept>
 </ccs2012>
\end{CCSXML}

\ccsdesc[500]{Information systems~Graph-based database models}
\ccsdesc[300]{Information systems~Data structures}
\ccsdesc[100]{Information systems~Storage management}
% %%
% %% Keywords. The author(s) should pick words that accurately describe
% %% the work being presented. Separate the keywords with commas.
\keywords{dynamic graph storage; graph concurrency control; graph neighbor index; benchmark framework.}

%% A "teaser" image appears between the author and affiliation
%% information and the body of the document, and typically spans the
%% page.
% \begin{teaserfigure}
%   \includegraphics[width=\textwidth]{sampleteaser}
%   \caption{Seattle Mariners at Spring Training, 2010.}
%   \Description{Enjoying the baseball game from the third-base
%   seats. Ichiro Suzuki preparing to bat.}
%   \label{fig:teaser}
% \end{teaserfigure}

% \received{July 2024}
% \received[revised]{September 2024}
% \received[accepted]{November 2024}
% \acmSubmissionID{V3mod070}
%%
%% This command processes the author and affiliation and title
%% information and builds the first part of the formatted document.
\maketitle


\section{Introduction}

Video generation has garnered significant attention owing to its transformative potential across a wide range of applications, such media content creation~\citep{polyak2024movie}, advertising~\citep{zhang2024virbo,bacher2021advert}, video games~\citep{yang2024playable,valevski2024diffusion, oasis2024}, and world model simulators~\citep{ha2018world, videoworldsimulators2024, agarwal2025cosmos}. Benefiting from advanced generative algorithms~\citep{goodfellow2014generative, ho2020denoising, liu2023flow, lipman2023flow}, scalable model architectures~\citep{vaswani2017attention, peebles2023scalable}, vast amounts of internet-sourced data~\citep{chen2024panda, nan2024openvid, ju2024miradata}, and ongoing expansion of computing capabilities~\citep{nvidia2022h100, nvidia2023dgxgh200, nvidia2024h200nvl}, remarkable advancements have been achieved in the field of video generation~\citep{ho2022video, ho2022imagen, singer2023makeavideo, blattmann2023align, videoworldsimulators2024, kuaishou2024klingai, yang2024cogvideox, jin2024pyramidal, polyak2024movie, kong2024hunyuanvideo, ji2024prompt}.


In this work, we present \textbf{\ours}, a family of rectified flow~\citep{lipman2023flow, liu2023flow} transformer models designed for joint image and video generation, establishing a pathway toward industry-grade performance. This report centers on four key components: data curation, model architecture design, flow formulation, and training infrastructure optimization—each rigorously refined to meet the demands of high-quality, large-scale video generation.


\begin{figure}[ht]
    \centering
    \begin{subfigure}[b]{0.82\linewidth}
        \centering
        \includegraphics[width=\linewidth]{figures/t2i_1024.pdf}
        \caption{Text-to-Image Samples}\label{fig:main-demo-t2i}
    \end{subfigure}
    \vfill
    \begin{subfigure}[b]{0.82\linewidth}
        \centering
        \includegraphics[width=\linewidth]{figures/t2v_samples.pdf}
        \caption{Text-to-Video Samples}\label{fig:main-demo-t2v}
    \end{subfigure}
\caption{\textbf{Generated samples from \ours.} Key components are highlighted in \textcolor{red}{\textbf{RED}}.}\label{fig:main-demo}
\end{figure}


First, we present a comprehensive data processing pipeline designed to construct large-scale, high-quality image and video-text datasets. The pipeline integrates multiple advanced techniques, including video and image filtering based on aesthetic scores, OCR-driven content analysis, and subjective evaluations, to ensure exceptional visual and contextual quality. Furthermore, we employ multimodal large language models~(MLLMs)~\citep{yuan2025tarsier2} to generate dense and contextually aligned captions, which are subsequently refined using an additional large language model~(LLM)~\citep{yang2024qwen2} to enhance their accuracy, fluency, and descriptive richness. As a result, we have curated a robust training dataset comprising approximately 36M video-text pairs and 160M image-text pairs, which are proven sufficient for training industry-level generative models.

Secondly, we take a pioneering step by applying rectified flow formulation~\citep{lipman2023flow} for joint image and video generation, implemented through the \ours model family, which comprises Transformer architectures with 2B and 8B parameters. At its core, the \ours framework employs a 3D joint image-video variational autoencoder (VAE) to compress image and video inputs into a shared latent space, facilitating unified representation. This shared latent space is coupled with a full-attention~\citep{vaswani2017attention} mechanism, enabling seamless joint training of image and video. This architecture delivers high-quality, coherent outputs across both images and videos, establishing a unified framework for visual generation tasks.


Furthermore, to support the training of \ours at scale, we have developed a robust infrastructure tailored for large-scale model training. Our approach incorporates advanced parallelism strategies~\citep{jacobs2023deepspeed, pytorch_fsdp} to manage memory efficiently during long-context training. Additionally, we employ ByteCheckpoint~\citep{wan2024bytecheckpoint} for high-performance checkpointing and integrate fault-tolerant mechanisms from MegaScale~\citep{jiang2024megascale} to ensure stability and scalability across large GPU clusters. These optimizations enable \ours to handle the computational and data challenges of generative modeling with exceptional efficiency and reliability.


We evaluate \ours on both text-to-image and text-to-video benchmarks to highlight its competitive advantages. For text-to-image generation, \ours-T2I demonstrates strong performance across multiple benchmarks, including T2I-CompBench~\citep{huang2023t2i-compbench}, GenEval~\citep{ghosh2024geneval}, and DPG-Bench~\citep{hu2024ella_dbgbench}, excelling in both visual quality and text-image alignment. In text-to-video benchmarks, \ours-T2V achieves state-of-the-art performance on the UCF-101~\citep{ucf101} zero-shot generation task. Additionally, \ours-T2V attains an impressive score of \textbf{84.85} on VBench~\citep{huang2024vbench}, securing the top position on the leaderboard (as of 2025-01-25) and surpassing several leading commercial text-to-video models. Qualitative results, illustrated in \Cref{fig:main-demo}, further demonstrate the superior quality of the generated media samples. These findings underscore \ours's effectiveness in multi-modal generation and its potential as a high-performing solution for both research and commercial applications.
\section{Preliminaries}
\label{sec:preliminaries}

\subsection{Supervised Binary Classification}
In many real-world tasks, one commonly encounters binary classification problems, in which an input $x \in \mathbb{R}^d$ is presented, and its label $y \in \{\pm 1\}$ needs to be predicted. Each sample is assumed to be independently and identically drawn from an unknown joint distribution $p(x,y)$. Let $\pi_{+} = p(y=+1)$ be the prior probability of the positive class (positive prior), and define
\begin{align*}
p_{\mathrm{p}}(x) = p(x \mid y=+1)
\text{, }
p_{\mathrm{n}}(x) = p(x \mid y=-1).
\end{align*}
Then, the marginal distribution of $x$ is given by
\begin{align*}
p(x)
= \pi_{+}p_{\mathrm{p}}(x) + (1-\pi_{+})p_{\mathrm{n}}(x).
\end{align*}

A classifier $g: \mathbb{R}^d \to \mathbb{R}$ outputs a real-valued score, whose sign determines the predicted label. For instance, a neural network can serve as $g$. A loss function $\ell:\mathbb{R} \times \{\pm 1\} \to [0,\infty)$ then measures how much the prediction disagrees with the true label. Let $R^+_p(g) = \mathbb{E}_{x \sim p_p} [\ell(g(x), +1)]$ denote the loss for true positive data, and $R^-_n(g) = \mathbb{E}_{x \sim p_n} [\ell(g(x), -1)]$ denote the loss for the true negative data. Then, the true risk is expressed as
\begin{align}
    R_{\mathrm{pn}}(g) =& \mathbb{E}_{(x,y)\sim p}[\ell(g(x),y)] \notag \\
    =& \pi_{+}R^+_p + (1 - \pi_{+})R^-_n \label{eq:risk} 
\end{align}

In supervised learning, positive dataset $\mathcal{C}_p = \{x^p_m\}_{m=1}^{m_p} \sim p_p(x)$ and negative dataset $\mathcal{C}_n = \{ x^n_m \}_{m=1}^{m_n} \sim p_n(x)$ are accessible. Replacing the expectations in \eqref{eq:risk} with sample mean, one obtains the empirical risk, and $g$ is trained to minimize it.


It is well known that having sufficient positive and negative samples typically allows one to train a highly accurate classifier for many tasks. However, in practice, obtaining large-scale positive and negative datasets with annotations is often challenging, especially in specialized domains where annotation costs become a significant obstacle.

\subsection{Unlabeled-Unlabeled (UU) Learning}
\label{subsec:uu}
UU learning~\citep{Lu2019-sd} is a technique that allows training a classifier without fully labeled positive and negative datasets, leveraging two unlabeled datasets with different class priors.

Concretely, suppose unlabeled corpora, $\widetilde{\mathcal{C}}_p = \{\widetilde{x}^p_m\}_{m=1}^{m_p}$ and $\widetilde{\mathcal{C}}_n = \{\widetilde{x}^n_m\}_{m=1}^{m_n}$, drawn from different mixture distributions. We denote $\theta_p = p(y=+1 \mid \widetilde{x}\in \widetilde{\mathcal{C}}_p)$ and $\theta_n = p(y=+1 \mid \widetilde{x}\in \widetilde{\mathcal{C}}_n)$ the \emph{positive prior} of these unlabeled corpora. In other words, $\theta_p$ is the fraction of true positives in $\widetilde{\mathcal{C}}_p$, and $\theta_n$ is the fraction of true positives in $\widetilde{\mathcal{C}}_n$. Then, the mixture distribution of each corpus is given as
\begin{align*}
    \widetilde{p}_{p}(x) &= \theta_p\, p_{p}(x) \;+\; \bigl(1 - \theta_p\bigr)\, p_{n}(x) \\
    \widetilde{p}_{_n}(x) &= \theta_n\, p_{p}(x) \;+\; \bigl(1 - \theta_n\bigr)\, p_{n}(x).        
\end{align*}

When $\theta_p > \theta_n$, we can treat $\widetilde{\mathcal{C}}_p$ as a pseudo-positive corpus (due to its larger proportion of actual positives) and $\widetilde{\mathcal{C}}_n$ as a pseudo-negative corpus (having a smaller proportion of actual positives). 

By appropriately combining these two unlabeled sets, one can construct an unbiased estimate of the true binary classification risk~\eqref{eq:risk}. Specifically, let $R_{\tilde{p}}^{\pm}(g)=\mathbb{E}_{x\sim \widetilde{p}_p}[\ell(g(x),\pm 1)]$, and $R_{\tilde{n}}^{\pm} (g)=\mathbb{E}_{x\sim \widetilde{p}_n}[\ell(g(x),\pm 1)]$. Then, the UU learning risk is given by
\begin{align}
    &R_{\mathrm{uu}}(g)\label{eq:uu}
    \\
    &\hspace{1em}= a R_{\tilde{p}}^+(g) - b R_{\tilde{p}}^-(g) - c R_{\tilde{n}}^+(g) + d R_{\tilde{n}}^-(g),\notag
\end{align}
where the coefficients $a$, $b$, $c$, $d$ are computed from $\pi_+$, $\theta_p$, and $\theta_n$ as $a = \frac{(1-\theta_n)\,\pi_+}{\theta_p - \theta_n}$, $b = \frac{\theta_n\,(1-\pi_+)}{\theta_p - \theta_n}$, $c = \frac{(1-\theta_p)\,\pi_+}{\theta_p - \theta_n}$, $d = \frac{\theta_p\,(1-\pi_+)}{\theta_p - \theta_n}$. When $\theta_p = 1$ and $\theta_n = 0$, that is, when using the same dataset as standard supervised learning, equation~\eqref{eq:uu} reduces to the standard supervised learning risk equation~\eqref{eq:risk}. In other words, supervised learning can be considered a special case of UU learning.

\subsection{Robust UU Learning}
\label{subsec:ruu}
While UU learning \eqref{eq:uu} does allow model training without explicit positive/negative labels, comparing the original binary classification risk \eqref{eq:risk}—which remains nonnegative—against the UU risk \eqref{eq:uu} shows the UU risk includes negative terms such as $-b R_{\tilde{p}}^-(g)$ and $-c R_{\tilde{n}}^+(g)$. It has been observed that these negative risk terms can lead to overfitting~\citep{Lu2020-dx}.

To mitigate this, \emph{Robust UU Learning} introduces a generalized Leaky ReLU function $f$ to moderate the reduction of negative risk. Concretely, it normalizes each term of the loss function as~\citep{Lu2020-dx}
\begin{align}
    R_{\mathrm{ruu}}(g)
    &= f\left(a R_{\tilde{p}}^+(g) - c R_{\tilde{n}}^+(g) \right) \notag \\
    &\hspace{2em}+ f\left(d R_{\tilde{n}}^-(g) - b R_{\tilde{p}}^-(g)\right) \label{eq:ruu}
\end{align}
where each bracketed term resembles a “normalized” risk under the hypothetical label of being positive or negative, respectively. The function $f$ is given by
\begin{align}
    f(x) =
    \begin{cases}
    x & \text{if } x > 0 \\
    \lambda x & \text{if } x < 0
    \end{cases}
    \quad (\lambda < 0).
    \label{eq:relu}
\end{align}

Intuitively, $f$ leaves the risk value unchanged when the risk is positive, but for negative risk, it uses $\lambda < 0$ to convert it into a positive quantity, thus mitigating the overfitting by negative risk.

\section{A Common Abstraction for DGS}\label{sec:framework}

In this section, we propose a common abstraction for DGS to facilitate a systematical study of existing methods.

\subsection{Graph Query and Data Abstraction}\label{sec:data_abstraction}

In transaction management~\cite{ramakrishnan2002database}, a database is a set $\{x\}$ of tuples in tables. A transaction consists of a sequence of read and write operations on $\{x\}$, beginning with a \textsc{Begin} command and ending with either \textsc{Commit}, indicating successful execution, or \textsc{Abort}, indicating failure and reverting modifications. Building on this concept, we propose a simple yet effective multi-level abstraction for graph query and data that aims to: 1) reflect the characteristics of graph queries; 2) indicate the nature of graph data; and 3) capture graph data access patterns. Figure \ref{fig:data_abstraction} shows our abstraction.

\begin{figure}[h]\small
    \setlength{\abovecaptionskip}{3pt}
    \setlength{\belowcaptionskip}{0pt}
    \includegraphics[scale=0.75]{img/data_abstraction.pdf}
    \centering
    \caption{The abstraction of graph query and data.}
    \label{fig:data_abstraction}
\end{figure}

\noindent\textbf{Global Abstraction.} Graph queries are categorized into write queries $\Delta G$ and read queries $Q$ as discussed in Section \ref{sec:preliminaries}. The global abstraction models the relationship among these queries by maintaining a global timestamp $t(G)$, initialized to 0 and incremented by 1 only when $\Delta G$ is committed. To ensure the serializability of graph queries, DGS requires that each committed write query be uniquely identified by its commit timestamp, denoted as $\Delta G_{i + 1}$ for the write query committed at $t(G) = i$. This abstraction effectively captures the construction of a dynamic graph $G = (G_0, \Delta \mathcal{G})$, where $\Delta \mathcal{G}$ is the serial execution order of committed write queries. A read query $Q$ starting at $t(G) = i$ has a local timestamp $t(Q)$.

\vspace{2pt}
\noindent\textbf{Local Abstraction.} In cooperation with the global abstraction, the local abstraction allows us to drill down to each data object and their primitive operations. The graph $G$ consists of vertices and edges. To indicate their differences and interconnections, $G$ is organized into a vertex table containing $V(G)$ and a set of neighbor tables, each corresponding to a neighbor set $N(u)$. Each entry in $V(G)$ contains the vertex ID $u$, the location of $N(u)$, and its properties, while each entry in $N(u)$ contains the neighbor ID $v$ and the properties associated with edge $e(u, v)$. Given a write query $\Delta G_i$, each operation creates a new version of a vertex or neighbor $u$ with the timestamp $t(u) = i$. Specifically, an insert (resp. delete) operation on $u$ creates a new version with the \textsc{op-type} as $I$ (resp. $D$). Updating an element is performed via an insert operation.

Lemma \ref{lemma:isolation} can be proven using the dependency graph \cite{fekete2005making} within the multi-level abstraction.  The lemma shows that by maintaining the serializability of write queries, DGS can achieve serializable isolation by ensuring $Q$ has a consistent view of $G_i$. This is done by allowing $Q$ to access only the latest version of vertices or neighbors $u$ such that $t(u) \leqslant t(Q)$. This approach allows us to coordinate $Q$'s data access based on timestamps without complex concurrency control protocols for read queries. Although existing DGS methods use this optimization, they do not explicitly and formally discuss it.

\begin{lemma} \label{lemma:isolation}
Suppose DGS maintains the serializability of write queries with the serial execution order $\Delta \mathcal{G}$. Given a read query $Q$ starting at timestamp $i$, ensuring $Q$ has a consistent view of $G_i = G_0 \oplus...\oplus \Delta G_i$ guarantees global serializable isolation for read and write queries.
\end{lemma}
\emph{Proof.} In the dependency graph, each node $Q_x$ represents a query, and an edge from $Q_x$ to $Q_y$ indicates that an operation $o_x \in Q_x$ conflicts with $o_y \in Q_y$, and $o_x$ precedes $o_y$ in execution. Operations $o_x$ and $o_y$ conflict if they act on the same object and at least one is a write. Queries are conflict serializable \emph{iff} the dependency graph is acyclic. Since DGS maintains the serializability of write queries, the dependency graph formed by them is acyclic. Ensuring $Q$ has a consistent view of $G_i$ guarantees that all operations in $Q$ depend only on operations from preceding write queries. Thus, the node representing $Q$ in the dependency graph has no incoming edges. Therefore, the combined dependency graph of read and write queries remains acyclic, proving the queries are conflict serializable.




\subsection{Graph Operations Abstraction}\label{sec:operation_abstraction}

We next analyze graph data access patterns based on the multi-level abstraction. From the perspective of DGS, a write (or read) query consists of a sequence of write (or read) operations on vertices and edges, while the computation logic and state of the query are beyond the scope of DGS. These \emph{graph operations} include:

\begin{itemize}[leftmargin=*]
\item \textsc{InsVtx($u$)}: inserting a vertex $u$ into $V(G)$;
\item \textsc{InsEdge($u, v$)}: inserting an edge $e(u, v)$ into $E(G)$;
\item \textsc{SearchVtx($u$)}: finding a vertex $u$ in $V(G)$;
\item \textsc{SearchEdge($u, v$)}: finding an edge $e(u, v)$ in $E(G)$;
\item \textsc{ScanVtx($G$)}: traversing the vertex set $V(G)$;
\item \textsc{ScanNbr($u$)}: traversing the neighbor set $N(u)$.
\end{itemize}

\begin{figure}[htbp]\small
    \setlength{\abovecaptionskip}{0pt}
    \setlength{\belowcaptionskip}{0pt}
    \includegraphics[scale=0.75]{img/operaiton_abstraction.pdf}
    \centering
    \caption{The abstraction of graph operations.}
    \label{fig:primitive_opeartions}
\end{figure}

Figure \ref{fig:primitive_opeartions} presents the abstraction of graph operations, where each node represents a graph operation, and each edge denotes a primitive operator on vertex or neighbor tables. The path from the start node to an operation node illustrates the primitive operators required to perform the operation and shows the relationships between different operations (e.g., \textsc{InsEdge} invokes \textsc{SearchEdge}). In summary, this abstraction provides a unified execution routine on graph operations.

% From the abstraction, we observe that operations on neighbor tables depend on those on the vertex table to retrieve the position of the target neighbor table, the insert operation depends on the search operation to locate the target object, and any operation on $u$ or its neighbors requires access rights to $u$ first.

Let $P_V$ and $P_N$ denote the paths from the start node to the graph operation node in the vertex table and neighbor table operations, respectively, in Figure \ref{fig:primitive_opeartions}. Equation \ref{eq:cost} defines the cost $T$ of a graph operation, where $T_{CC}$ is the cost of coordinating access to the target vertex, and $T_p$ is the cost of the primitive operator $p$. Concurrency control requires the cooperation of underlying graph containers, such as checking a creation timestamp for each object. $\alpha_p$ is the overhead amplification ratio of concurrency control on the primitive operator $p$. An optimal value is 1, indicating no performance degradation due to concurrency control. This equation highlights the factors affecting DGS performance, serving as a tool to systematically study these performance factors rather than predicting the exact cost of a graph operation.

\begin{equation} \label{eq:cost}
T = T_{CC} + \sum_{p \in P_V} \alpha_p T_p + \sum_{p \in P_N} \alpha_p T_p.
\end{equation}

\noindent\textbf{Remark. } \sun{We exclude edge update and delete operations as they follow a similar process to insertions. Fine-grained methods like Sortledton, Teseo, and LiveGraph handle deletions by: 1) locating the target vertex, and 2) creating a new version marked as \emph{delete} or updating the end timestamp to indicate the deletion, as discussed above. Space is later reclaimed through garbage collection. Coarse-grained methods like Aspen, use a copy-on-write strategy, where deletion involves removing a vertex from a block rather than adding one, much like with insertions.} Graph query workloads generally focus on analyzing connections between vertices, with vertex updates, especially deletions, being rare \cite{zhu2019livegraph}. Consequently, existing DGS works typically focus on operations on neighbor tables. Therefore, this paper primarily focuses on $N(u)$ as well.

\section{DGS Methods Under Study}\label{sec:dgs_methods_under_study}

 
Existing DGS methods focus on optimizing two problems: 1) graph concurrency control, which includes version management and concurrency control protocols to coordinate the execution of concurrent graph queries, and minimizing overhead for each graph operation (i.e., $T_{CC}$ and $\alpha_p$ in Equation \ref{eq:cost}); and 2) graph containers, which include vertex indexes, neighbor indexes, and other optimizations to optimize graph data access $T_p$ for each operation in Equation \ref{eq:cost}. Table \ref{tab:summary_methods} summarizes the key design choices in these methods. In the following, we first briefly introduce these methods and then compare them.

% Please add the following required packages to your document preamble:
% \usepackage{graphicx}
\small
\begin{table}[t]
\centering
\caption{A summary of DGS methods under study.}
\label{tab:summary_methods}
% \resizebox{\columnwidth}{!}{%
\begin{tabular}{|c|clc|ccc|}
\hline
\textbf{}           & \multicolumn{3}{c|}{\textbf{Graph Concurrency Control}}                                                                                                       & \multicolumn{3}{c|}{\textbf{Graph Container}}                                                                                                                                                                                                                     \\ \hline
\textbf{Method}     & \multicolumn{2}{c|}{\textbf{\begin{tabular}[c]{@{}c@{}}Version\\ Management\end{tabular}}}          & \textbf{Protocol}                                       & \multicolumn{1}{c|}{\textbf{\begin{tabular}[c]{@{}c@{}}Vertex\\ Index\end{tabular}}} & \multicolumn{1}{c|}{\textbf{\begin{tabular}[c]{@{}c@{}}Neighbor\\ Index\end{tabular}}} & \textbf{\begin{tabular}[c]{@{}c@{}}Additional\\ Optimization\end{tabular}}          \\ \hline
\textbf{LiveGraph}  & \multicolumn{2}{c|}{\begin{tabular}[c]{@{}c@{}}Fine-Grained with\\ Continuous Version\end{tabular}} & S2PL                                                    & \multicolumn{1}{c|}{\begin{tabular}[c]{@{}c@{}}Dynamic\\ Array\end{tabular}}         & \multicolumn{1}{c|}{\begin{tabular}[c]{@{}c@{}}Dynamic\\ Array\end{tabular}}           & \begin{tabular}[c]{@{}c@{}}Bloom\\ Filter\end{tabular}                            \\ \hline
\textbf{Sortledton} & \multicolumn{2}{c|}{\begin{tabular}[c]{@{}c@{}}Fine-Grained with\\ Version Chain\end{tabular}}      & G2PL                                                    & \multicolumn{1}{c|}{\begin{tabular}[c]{@{}c@{}}Dynamic\\ Array\end{tabular}}         & \multicolumn{1}{c|}{\begin{tabular}[c]{@{}c@{}}Segmented\\ Skip List\end{tabular}}     & \begin{tabular}[c]{@{}c@{}}Adaptive\\ Indexing\end{tabular}                       \\ \hline
\textbf{Teseo}      & \multicolumn{2}{c|}{\begin{tabular}[c]{@{}c@{}}Fine-Grained with\\ Version Chain\end{tabular}}      & OCC                                                     & \multicolumn{1}{c|}{\begin{tabular}[c]{@{}c@{}}Hash\\ Table\end{tabular}}            & \multicolumn{1}{c|}{PMA}                                                               & \begin{tabular}[c]{@{}c@{}}Write-Optimized\\ Segment\end{tabular}                 \\ \hline
\textbf{Aspen}      & \multicolumn{2}{c|}{Coarse-Grained}                                                                 & \begin{tabular}[c]{@{}c@{}}Single\\ Writer\end{tabular} & \multicolumn{1}{c|}{\begin{tabular}[c]{@{}c@{}}AVL\\ Tree\end{tabular}}              & \multicolumn{1}{c|}{\begin{tabular}[c]{@{}c@{}}Segmented\\ PAM\end{tabular}}           & \begin{tabular}[c]{@{}c@{}}Vertex Index\\ Flatten \&\\ Data Encoding\end{tabular} \\ \hline
\textbf{LLAMA}      & \multicolumn{2}{c|}{Coarse-Grained}                                                                 & \begin{tabular}[c]{@{}c@{}}Single\\ Writer\end{tabular} & \multicolumn{1}{c|}{\begin{tabular}[c]{@{}c@{}}Dynamic\\ Array\end{tabular}}         & \multicolumn{1}{c|}{\begin{tabular}[c]{@{}c@{}}Dynamic\\ Array\end{tabular}}           & -                                                                                 \\ \hline
\end{tabular}%
% }
\end{table}


\subsection{A Brief Introduction to DGS Methods}

\subsubsection{\textbf{LiveGraph}~\cite{zhu2019livegraph}}

\textbf{Graph Concurrency Control.} LiveGraph uses a lock for each vertex in $V(G)$ and adapts S2PL to synchronize data access. For a write query $\Delta G$, LiveGraph first obtains all exclusive locks on vertices in $\Delta V$ (vertices involved in $\Delta G$), performs the graph operations, and then releases the locks. To handle deadlocks, LiveGraph aborts $\Delta G$ if it cannot acquire a lock within a time limit.

For each version of a neighbor or vertex, LiveGraph keeps begin and end timestamps ($begin-ts$ and $end-ts$) to record its lifetime as shown in Figure \ref{fig:livegraph_index}. Given $\Delta G$ committed at timestamp $i$, \textsc{InsEdge($u, v$)} searches for the latest version of $v$ in $N(u)$. If found, it sets $end-ts$ to $i$ and creates a new version of $v$ with $begin-ts = i$ and $end-ts = INF$. Otherwise, it directly creates a new version of $v$. \textsc{DelEdge($u, v$)} searches for the latest version of $v$ and sets $end-ts$ to $i$. For read operations in $Q$, LiveGraph obtains a shared lock on the vertex and immediately releases it after the operation. For example, \textsc{ScanNbr($u$)} acquires the lock on $u$, accesses neighbors based on timestamps, and releases the lock immediately. Thus, $Q$ never leads to deadlocks because it never holds two locks simultaneously.

\begin{figure}[h]\small
    \setlength{\abovecaptionskip}{3pt}
    \setlength{\belowcaptionskip}{0pt}
    \includegraphics[scale=0.75]{img/livegraph_neighbor_index.pdf}
    \centering
    \caption{The neighbor index of $N(u_2)$ in LiveGraph.}
    \label{fig:livegraph_index}
\end{figure}

\noindent\textbf{Graph Container.} Given $u \in V(G)$, LiveGraph uses a dynamic array (DA) as the neighbor index of $N(u)$ where each element corresponds to a physical version of $v \in N(u)$. Graph operations in Figure \ref{fig:primitive_opeartions} are based on primitive operators of DA, the time complexity of which is listed in Table \ref{tab:complexity_data_structure}. As the storage of DA is continuous and no version chain exists, \textsc{Scan} is very fast. Moreover, LiveGraph executes \textsc{Scan} from the end to the beginning of DA since the latest element may be more frequently visited than the stale ones. However, \textsc{Search} is slow because DA is unsorted and uses \textsc{Scan} to perform the search. Consequently, \textsc{InsEdge} is slow because it depends on \textsc{SearchEdge} though adding an element only requires a simple append. To mitigate this issue, LiveGraph maintains a Bloom filter~\cite{mitzenmacher2001compressed} for each $N(u)$ to record whether an element exists in $N(u)$. LiveGraph uses DA as the vertex index of $V(G)$ as well. As the vertex ID is ranged in $[0, |V|)$, the element at the index $u$ is the vertex $u$. Therefore, the time complexity of \textsc{Search} on the vertex index is $O(1)$. As the implementation is simple, we omit the details.

\subsubsection{\textbf{Sortledton}~\cite{fuchs2022sortledton}}

\textbf{Graph Concurrency Control.} Sortledton also uses S2PL but optimizes the locking sequence: Sort the vertices in $\Delta V$ by ascending vertex IDs and acquire their exclusive locks in that order. This optimization prevents deadlocks by avoiding circular waiting among write queries~\cite{silberschatz1991operating}. Therefore, Sortledton does not need any mechanisms to handle deadlocks. For \textsc{InsEdge($u, v$)} (or \textsc{DelEdge($u, v$)}) in $\Delta G$ committed at timestamp $i$, Sortledton creates a new version of $v$ with timestamp $i$ and \textsc{op-type} as $I$ (or $D$) as shown in Figure \ref{fig:sortledton_neighbor_index}. Sortledton maintains a version chain for different versions of $v$, where the new version points to the old one. For read queries, Sortledton uses the same concurrency control strategy as LiveGraph.


\begin{figure}[h]\small
    \setlength{\abovecaptionskip}{3pt}
    \setlength{\belowcaptionskip}{0pt}
    \includegraphics[scale=0.75]{img/sortledton_neighbor_index.pdf}
    \centering
    \caption{The neighbor index of $N(u_2)$ in Sortledton.}
    \label{fig:sortledton_neighbor_index}
\end{figure}

\noindent\textbf{Graph Container.} Like LiveGraph, Sortledton uses a dynamic array as the vertex index. For the neighbor index, Sortledton splits $N(u)$ into blocks $B$, and uses the skip list as the block index linking them. The fill ratio of each block is maintained between 50\% and 100\%. When a block is full, Sortledton splits it into two blocks, equally distributing the neighbors. If the fill ratio drops below 50\%, Sortledton merges it with adjacent blocks. The first element of each block serves as its key in the skip list. We call this structure the \emph{segmented skip list}. A read operation traverses the version chain to find the target version. Since real-world graphs are sparse, Sortledton proposes the \emph{adaptive neighbor index}: If $|N(u)|$ is below a threshold (e.g., 256), it uses a sorted dynamic array as the neighbor index instead of the segmented skip list.

\subsubsection{\textbf{Teseo}~\cite{de2021teseo}}

\noindent\textbf{Graph Concurrency Control.} Teseo uses the same version management method as Sortledton but adopts optimistic concurrency control (OCC introduced in Section \ref{sec:preliminaries}) to coordinate concurrent write queries. Unlike LiveGraph and Sortledton, which keep a lock for each vertex, Teseo maintains a lock for each edge partition to synchronize concurrent data access. Specifically, Teseo logically divides $E(G)$ into equally sized partitions, each with a lock. Before accessing data in a partition, a write query acquires an exclusive lock (or a read query acquires a shared lock) on the partition and releases it immediately after access.


\begin{figure}[h]\small
    % \setlength{\abovecaptionskip}{3pt}
    % \setlength{\belowcaptionskip}{0pt}
    \includegraphics[scale=0.75]{img/teseo_neighbor_index.pdf}
    \centering
    \caption{The neighbor index of $N(u_2)$ in Teseo.}
    \label{fig:teseo_neighbor_index}
\end{figure}


\noindent\textbf{Graph Container.} Teseo employs the packed memory array (PMA)~\cite{bender2007adaptive,de2019packed} as the neighbor index as shown in Figure \ref{fig:teseo_neighbor_index}. But it stores all neighbor tables in a single PMA. If $N(u)$ is large, it spans multiple blocks in the PMA, while multiple small neighbor sets can share a single block. However, the global rebalance overhead is high for a PMA if $E(G)$ is stored together. To address this, Teseo divides the single PMA into multiple large leaves (several megabytes each) and indexes these leaves with an \emph{adaptive radix tree} (ART)~\cite{leis2013adaptive}, calling this data structure FAT. Additionally, Teseo uses a hash table as the vertex index to record the position of each vertex's neighbor index in FAT. By default, blocks in FAT are sorted, known as read-optimized segments. When the insertion rate is high, FAT switches to write-optimized segments (WOS), which handle updates by appending to the update log. For WOS, Teseo loops over the segment to find the target vertex. Teseo is built on top of HyPer~\cite{kemper2011hyper}.

\subsubsection{\textbf{Aspen}~\cite{dhulipala2019low}}

\textbf{Graph Concurrency Control.} Aspen uses a coarse-grained strategy that maintains timestamps for each graph snapshot. Specifically, Aspen uses the \emph{single-writer-multiple-reader} scheme, which executes write queries serially and allows multiple readers to execute concurrently. A write query $\Delta G_i$ uses the \emph{copy-on-write} (CoW) method (also called shadow paging) to apply updates on a copy of the graph, creating a new snapshot $G_i$ as shown in Figure \ref{fig:aspen_neighbor_index}. $Q$ works on the latest version of the graph snapshot. Therefore, read and write queries never block each other, and multiple read queries can share the same graph snapshot.

\begin{figure}[h]\small
    \setlength{\abovecaptionskip}{3pt}
    \setlength{\belowcaptionskip}{-5pt}
    \includegraphics[scale=0.75]{img/aspen_neighbor_index.pdf}
    \centering
    \caption{The neighbor index of $N(u_2)$ in Aspen.}
    \label{fig:aspen_neighbor_index}
\end{figure}

\vspace{2pt}
\noindent\textbf{Graph Container.} Aspen partitions $N(u)$ into a set of sorted blocks and uses a parallel augmented map (PAM) to index these blocks. These blocks have no empty slots. \textsc{InsEdge($u, v$)} copies the block, inserts $v$, and then copies the path from the block to the root to create a new snapshot of $N(v)$. Given $N(u)$ and block size $B$, Aspen selects vertices such that $v \mod b = 0$ as heads, i.e., $heads = {v \in N(u) \mid v \mod b = 0}$. This approach ensures that updates to one block do not affect adjacent blocks. Aspen demonstrates that this segmentation ensures each block has $B$ elements with high probability. Moreover, Aspen uses an AVL tree as the vertex index and copies the path in the tree for each update operation.

Aspen proposes two optimization methods to enhance performance. First, for long-running read queries, Aspen can create an array storing the positions of each neighbor table based on the AVL tree to eliminate the overhead of \textsc{Search} in the AVL tree. Second, Aspen uses a difference encoding scheme to compress data in a block. For a block containing $(v0, v1, v2, \ldots)$, Aspen stores it as $(v0, v1 - v0, v2 - v0, \ldots)$ and compresses it with byte codes to accelerate set intersections~\cite{aberger2017emptyheaded}.

\subsubsection{\textbf{LLAMA}~\cite{macko2015llama}}

LLAMA, proposed in 2015, also employs a coarse-grained graph concurrency control mechanism similar to that in Aspen. Specifically, LLAMA divides the vertex table into partitions, each stored in a data page, and maintains an \emph{indirection table} to store the locations of these pages. Each write query must copy the indirection table to create a new graph snapshot, and this overhead limits update performance and graph scalability. 


\subsection{Comparison of DGS Methods} \label{sec:discussion}

\noindent\textbf{Graph Containers.} We discuss the time and space cost of graph operations based on Table \ref{tab:complexity_data_structure}.

\emph{Time.} Given vertex IDs in the range $[0, |V|)$, using a dynamic array (DA) as vertex indexes allows \textsc{SearchVtx} and \textsc{InsVtx} in $O(1)$ time with simple memory accesses. Due to continuous storage, DA enables fast scan operations. In contrast, hash tables and AVL trees incur more overhead than DA, despite having the same time complexity for some operations.

Continuous storage, used in LiveGraph and LLAMA, stores $N(u)$ in DA, facilitating fast \textsc{ScanNbr} but resulting in slow \textsc{SearchEdge} due to the unsorted nature of the array. Since \textsc{InsEdge} depends on \textsc{SearchEdge} in DGS, its performance is also slow despite \textsc{Insert} in DA taking $O(1)$ time. For the segmented methods, scanning accesses data continuously within a block, while inserting typically moves only a few elements within a block. These blocks are linked by an index (e.g., PAM) to accelerate \textsc{SearchEdge}. Therefore, the cost of these operations includes the block index and the block itself. Increasing block sizes can improve scan performance due to continuous memory access but may degrade insert performance due to data movement within the block. Additionally, Aspen, using CoW, incurs more overhead for insertion than methods performing in-place updates because its insert operation copies the block as well as the root-to-leaf path.

\emph{Space.} Practical memory consumption is significantly affected by element size. Each element in Sortledton and Teseo consumes $3 \times w$ bytes, where $w$ is the word size: one for the vertex ID, one for the version, and one for the pointer. The \textsc{op-type} can use the high bit in the timestamp. Each element in LiveGraph also takes $3 \times w$ bytes. In contrast, each element in Aspen consumes only $w$ bytes due to its coarse-grained granularity. Additionally, Aspen’s neighbor index has no empty slots. Therefore, the coarse-grained method is more memory efficient than the fine-grained method.

\vspace{2pt}
\noindent\textbf{Graph Concurrency Control.} We first compare fine-grained and coarse-grained strategies, and then discuss fine-grained methods.

\emph{Fine-Grained vs. Coarse-Grained.} Fine-grained methods require lock operations on each graph operation, which can lead to lock contention and thus expensive $T_{CC}$. High-degree vertices are particularly prone to frequent access. Fine-grained methods necessitate version checks on each element, resulting in increased data loading from memory and more computation for version comparison, leading to a high $\alpha_p$ value. In contrast, coarse-grained methods avoid these issues. However, fine-grained methods allow multiple writers to update the graph simultaneously and perform in-place updates by simply inserting new elements, enhancing update performance.

Additionally, the fine-grained strategy does not place special requirements on the underlying graph containers, making it more generic. In contrast, the coarse-grained strategy requires support for fast snapshot creation. However, since the coarse-grained strategy does not maintain versions or perform version checks for each element, it can be effectively combined with data compression techniques~\cite{dhulipala2019low}, which is not feasible for the fine-grained approach.

\emph{Discussion on Fine-Grained Strategies.} First, the continuous version storage in LiveGraph can improve scanning efficiency by avoiding the traversal of a version chain. However, it may increase data access volume by including stale versions, negatively impacting search and insert efficiency. Second, G2PL in Sortledton is generally the optimal fine-grained concurrency control mechanism due to its effective deadlock avoidance optimization. Although OCC in Teseo does not require holding all mutexes of vertices in $\Delta V$, write queries are typically very short because $\Delta G$ generally contains a single update. Moreover, executing deadlock detection for write-write conflicts introduces significant overhead and implementation challenges. As such, our study uses G2PL for fine-grained methods because of its advantages.

\vspace{2pt}
\noindent\textbf{Empirical Evaluation Targets.} Following the above discussion, we will set up test frameworks and empirically evaluate these techniques by addressing the following five questions.

\begin{itemize} [leftmargin=*]
    \item \textbf{Graph Containers:} \emph{\textbf{Q1.} \sun{How effective are existing techniques in graph containers at efficiently performing key operations such as \textsc{SearchEdge}, \textsc{InsEdge}, and \textsc{ScanNbr}, as defined by $T_p$ in Equation \ref{eq:cost}?}} \emph{\textbf{Q2.} Which neighbor index performs the best, and what is the gap between it and CSR on read queries?}
    \item \textbf{Graph Concurrency Control:} \emph{\textbf{Q3.} What is the impact of graph concurrency control on graph operations?} \emph{\textbf{Q4.} How are the scalability and concurrency of competing methods?}
    \sun{
    \item \textbf{Batch Granularity:} \emph{\textbf{Q5.} How does the batch granularity affect the performance of competing methods?}
    }
    \item \textbf{Memory Consumption:} \emph{\textbf{Q6.} What is the impact of graph containers and version management in DGS on memory consumption, and what is the gap between DGS and CSR?}
\end{itemize}

\begin{figure*}[t]\small
    \setlength{\abovecaptionskip}{0pt}
    % \setlength{\belowcaptionskip}{-13pt}
    \includegraphics[scale=0.45]{img/test_framework.pdf}
    \centering
    \caption{An overview of the test framework.}
    \label{fig:overview_test_framework}
\end{figure*}


\section{Test Framework Setup} \label{sec:test_framework_setup}


Figure \ref{fig:overview_test_framework} provides an overview of this framework. It is implemented with around 7800 lines of optimized C++ code. The source code is compiled using g++ 10.5.0 with the -O3 optimization enabled. Experiments are conducted on a Linux server equipped with an AMD EPYC 7543 CPU (32 cores) and 128GB of memory.



\subsection{Benchmark Platform} \label{sec:benchmark_platform}


\vspace{2pt}
\noindent\textbf{Third-Party Modules.} This module integrates the original implementations of LLAMA~\cite{llamacodebase}, Aspen~\cite{aspencodebase}, LiveGraph~\cite{livegraphcodebase}, Teseo~\cite{teseocodebase}, and Sortledton~\cite{sortledtoncosebase}  from GitHub. Due to differences in their APIs, we implemented a customized adapter, a wrapper for each method, to standardize the evaluation of graph operations. The overhead of these adapters is negligible because they simply combine the interfaces to provide these functions, if not directly available. Each method is configured with its recommended settings. Specifically, the size of bloom filter in LiveGraph is set as $\frac{1}{16}$ of the block size. Sortledton, Teseo and Aspen set the block size to 256.



\vspace{2pt}
\noindent\textbf{DGS Sandbox.} In this module, we re-implement key techniques from competing methods within our abstraction for a fair and detailed investigation of individual techniques. These techniques can be composed differently to evaluate graph operations based on the unified execution routines shown in Figure \ref{fig:primitive_opeartions}. For fine-grained methods, we apply the following optimizations: 1) all methods use G2PL as the concurrency control protocol, and 2) all methods use a dynamic array as the vertex index by default. We implement a simple baseline DGS method by combining the static graph storage AdjLst with G2PL, naming it AdjLst. \sun{Specifically, AdjLst is implemented as an array where each element represents a vertex, and each vertex points to an array storing its neighbor set. When inserting a new element, a binary search is performed to find the correct position, after which the subsequent elements are shifted to make space for the new one. If the array is full, pre-allocated space is used, functioning like a dynamic array.} For comparison purposes, we also include CSR, the optimal baseline for static graphs.





\subsection{Test Driver}

\vspace{2pt}
\noindent\textbf{Workload Generator.} Table \ref{tab:datasets} presents the statistics of the real-world graphs used in the paper. They span six categories, with $|V|$ ranging from tens of thousands to tens of millions and $|E|$ scaling to hundreds of millions.  Both \emph{ldbc}~\cite{ldbcic} and \emph{nft}~\cite{Zhang2023LiveGL, livegraphlab} originally have timestamps to mark the insertion sequence of edges. \emph{ldbc} simulates actions in a social network. We set the scale factor to 10 to control the graph size. \emph{nft} records NFT transactions on Ethereum from 2017 to 2022. Other graphs are obtained from SNAP~\cite{snapnets} and do not include timestamps. These graphs are widely used in DGS research. We also considered larger graphs (e.g., \emph{friendster}) containing billions of edges but omitted them since existing DGS methods frequently run out of memory on these cases.
\small
\begin{table}[h]
% \setlength{\abovecaptionskip}{0pt}
% \setlength{\belowcaptionskip}{0pt}
\captionsetup{skip=0pt} 
\centering
\caption{The detailed information of the real-world graphs.}
\label{tab:datasets}
% \resizebox{0.45\textwidth}{!}{%
\begin{tabular}{|c|c|c|c|c|c|c|}
\hline
\textbf{Category}                & \textbf{Dataset}     & \textbf{Name} & \textbf{|\textit{V}|} & \textbf{|\textit{E}|} & \textit{\textbf{$d_{avg}$}} & $d_{max}$ \\ \hline
\multirow{4}{*}{\textbf{Social}} & \textbf{LiveJournal} & \emph{lj}       & 4.8M                  & 42.8M                 & 8.8                         & 20233     \\ \cline{2-7} 
                   % & \textbf{LDBC}       & \emph{ldbc} & 9.3M  & 52.6M  & 5.6   & 1346287 \\ \cline{2-7} 
                   & \textbf{LDBC}       & \emph{ldbc} & 30.0M  & 175.9M  & 5.9   & 4282595 \\ \cline{2-7} 
                   & \textbf{Twitter}    & \emph{tw}   & 21.3M & 265.0M & 12.4  & 698112 \\ \hline
                   % & \textbf{Orkut}      & \emph{ok}   & 3.1M  & 117.2M & 38.14 & 33313   \\ \hline
\textbf{Game}      & \textbf{DotaLeague} & \emph{dl}   & 0.06M & 50.9M  & 831.6 & 17004   \\ \hline
\textbf{Web}       & \textbf{Wiki}       & \emph{wk}   & 14.0M & 59.0M  & 4.2   & 723404  \\ \hline
\textbf{Citation}  & \textbf{Cit}        & \emph{ct}   & 3.8M  & 16.5M  & 4.4   & 793     \\ \hline
\textbf{Synthetic} & \textbf{Graph500}   & \emph{g5}   & 8.9M  & 260.4M & 29.3  & 406416  \\ \hline
\textbf{Financial} & \textbf{NFT}   & \emph{nft}   & 29.6M  & 77.5M &  2.62 &  2290853 \\ \hline
\end{tabular}%
% }%
\end{table}

We generate three types of graph queries. First, the \emph{micro OP stream} contains a sequence of graph operations. For graphs with timestamps, we generate an \textsc{InsEdge} stream using the first 80\% of edges as the initial graph and the remaining 20\% as the insert edges. For graphs without timestamps, we shuffle the edges and generate the insert stream similarly, following previous works~\cite{zhu2019livegraph,de2021teseo,fuchs2022sortledton}. As these works focus on single updates, each operation corresponds to a write query $\Delta G$. For the \textsc{SearchEdge} stream, we randomly select 20\% of edges as the search targets. For the \textsc{ScanNbr} stream, we select 20\% of vertices based on their degrees. Each of these operations is a read query. The micro OP stream serves two purposes: 1) Investigating the performance of competing methods on basic graph operations, and 2) Studying the effectiveness of short queries (i.e., IC workloads).

Second, we integrate four representative \emph{graph analytic} algorithms from GAPBS~\cite{beamer2015gap}: PR (PageRank), BFS, SSSP, and WCC, which cover different graph data access patterns. PR sequentially accesses both vertices and neighbors. BFS and WCC visit neighbors sequentially while accessing vertices randomly. SSSP introduces a random access pattern to neighbors when retrieving weights. Third, we implement TC (triangle counting) as the representative \emph{graph pattern matching} query. This query requires DGS to support quick scanning in sorted order for fast set intersections. In summary, these two types of queries evaluate the effectiveness of complex, long-running graph queries (i.e., BI and IS workloads).

Real-world graphs following a power-law degree distribution complicate examining the performance of graph operations on different neighbor set sizes because accessing frequently used sets of elements can improve cache performance and lead to biased results. To address this, we design experiments using synthetic datasets. A \emph{synthetic dataset} consists of sets of elements with uniform sizes. Each element is a vertex with an ID ranging from $[0, 2^{22})$. Assuming each vertex ID is 8 bytes, to evaluate performance on a neighbor set with 512 elements, we generate $x = \frac{8 \text{GB}}{512 \times 8 \text{ bytes}}$ sets of the same size. These sets are labeled from $[0, x)$. We generate insert, scan, and search OP streams using the same strategy as for real-world graphs, treating set IDs as vertex IDs and the sets as neighbor sets. We default to using 8GB to prevent all sets from residing in the cache, thereby simulating random memory access in large graphs.



\vspace{2pt}
\noindent\textbf{Workload Executor.} Each thread executes a stream or a graph algorithm. Operations within the same stream execute sequentially by one thread, while multiple threads can execute concurrently. Although the graph algorithms in the test framework can run in parallel, we execute them in a single thread to focus on the DGS's capability to empower concurrent graph query execution and handle different queries.

\vspace{2pt}
\noindent\textbf{Performance Monitor.} We use \emph{throughput}, the number of edges processed (insert, search, scan) per second, to measure DGS efficiency. We use \emph{latency}, the time elapsed to complete one query, to examine service quality. Recording the latency of each operation incurs non-trivial overhead due to the short duration of single graph operations. Therefore, we record latency every one hundred micro-operations. \emph{Scalability} is measured by the throughput of micro-operations as the number of threads increases, while \emph{concurrency} is measured when multiple micro-operation streams (or graph algorithm queries) are mixed. \emph{Memory cost} tracks the memory consumption of competing storage methods.

\section{Results}
\label{sec:results}

\begin{table*}[t]
  \centering
  \caption{Human Evaluation Results: Comparison of Generation Methods Across Two Datasets.
           Mean $\pm$ Standard Deviation Reported for Quality, Prompt Adherence, and Diversity Metrics.}
  \label{tab:human_eval}
  \begin{tabular}{l|ccc|ccc}
    \toprule
    & \multicolumn{3}{c}{\textbf{Stable Bias Profession Dataset}} & \multicolumn{3}{c}{\textbf{Parti Prompt Dataset}} \\
    \textbf{Method} & \textbf{Quality} & \textbf{Adherence} & \textbf{Diversity} 
                    & \textbf{Quality} & \textbf{Adherence} & \textbf{Diversity} \\
    \midrule
    Baseline 
    & 3.97 $\pm$ 0.94 & \textbf{4.08} $\pm$ 0.98 & 2.79 $\pm$ 1.24
    & 4.05 $\pm$ 0.90 & \textbf{4.16} $\pm$ 1.08 & 2.76 $\pm$ 1.13 \\
    GPT-4o 
    & 3.96 $\pm$ 1.00 & 3.79 $\pm$ 1.11 & \textbf{3.92} $\pm$ 0.94
    & \textbf{4.16} $\pm$ 0.86 & 4.02 $\pm$ 1.17 & \textbf{3.44} $\pm$ 1.06 \\
    DeepSeek-V3 
    & \textbf{4.04} $\pm$ 0.95 & 3.93 $\pm$ 1.04 & 3.75 $\pm$ 1.13
    & 4.13 $\pm$ 0.88 & 4.02 $\pm$ 1.17 & 3.34 $\pm$ 1.13 \\
    \bottomrule
  \end{tabular}
\end{table*}

\subsection{Human Evaluation}
\begin{figure*}[htbp]
    \centering
    \includegraphics[width=\textwidth]{fig/human_eval.png}
    \caption{Comparison of human evaluation metrics across the Stable Bias Profession Dataset (left) and Parti Prompt Dataset (right). The distributions of quality, prompt adherence, and diversity are illustrated with respect to frequency and scores for different methods (Baseline, GPT-4o, and DeepSeek-V3). Mean and standard deviation values for each method are provided for comprehensive analysis.}
    \label{fig:human_eval}
\end{figure*}

Table~\ref{tab:human_eval} and Figure~\ref{fig:human_eval} summarize the comparative performance of the three generation methods (Baseline, GPT-4o, and DeepSeek-V3) on two datasets: the Stable Bias Profession Dataset and the Parti Prompt Dataset. Each method was evaluated along three criteria: (1)~Quality, (2)~Prompt Adherence, and (3)~Diversity.


\noindent \textbf{Quality.}
Across both datasets, all three methods exhibit comparable performance in terms of overall image quality. On the Stable Bias Profession Dataset, DeepSeek-V3 attains the highest quality score ($4.04\pm0.95$), followed by Baseline ($3.97\pm0.94$) and GPT-4o ($3.96\pm1.00$). In the Parti Prompt Dataset, GPT-4o achieves the highest mean score for quality at $4.16\pm0.86$, with DeepSeek-V3 close behind at $4.13\pm0.88$. The Baseline slightly lags at $4.05\pm0.90$. These results suggest that while the large language model (LLM)-assisted methods can match or exceed the Baseline in terms of visual fidelity, the margin of improvement is relatively small. 


\noindent \textbf{Prompt Adherence.}
The Baseline method yields slightly higher prompt adherence scores on both datasets: $4.08\pm0.98$ in the Stable Bias Profession Dataset and $4.16\pm1.08$ in the Parti Prompt Dataset. By contrast, GPT-4o and DeepSeek-V3 scores are generally around $3.8$--$3.9$ in the first dataset and $4.0$ in the second. This trend indicates a modest trade-off: while LLM-assisted debiasing often promotes diversity (see below), it can introduce small deviations from the exact prompt details. Nonetheless, the overall adherence remains fairly high across all methods.


\noindent \textbf{Diversity.}
In contrast to prompt adherence, diversity shows the largest separation among methods. On both datasets, the Baseline obtains the lowest mean diversity score, around $2.7$--$2.8$. GPT-4o and DeepSeek-V3 consistently improve upon this baseline; for example, in the Parti Prompt Dataset, GPT-4o and DeepSeek-V3 reach $3.44\pm1.06$ and $3.34\pm1.13$, respectively, versus the Baseline's $2.76\pm1.13$. Even more pronounced gains are found in the Stable Bias Profession Dataset, where GPT-4o achieves $3.92\pm0.94$ and DeepSeek-V3 $3.75\pm1.13$, while the Baseline remains at $2.79\pm1.24$. These higher diversity scores for LLM-assisted methods corroborate their effectiveness at reducing repetitive patterns and mitigating stereotypes. 

\subsection{Non-parametric Evaluation}
\begin{table*}[ht]
\centering
\caption{Top-5 kNN classification results across different models - Baseline, GPT-4o, and DeepSeek-V3 - for the profession of CEO. Results shown for k=5, k=7, and k=9.}
\label{tab:top5_results}
\small{
\begin{tabular}{lccc}
\toprule
\textbf{Profession} & \textbf{k=5} & \textbf{k=7} & \textbf{k=9} \\
\midrule
\textbf{CEO} 
& % ----------- k=5 Column -----------
\begin{tabular}[t]{@{}l@{}}
\textbf{Baseline} \\
(1) Caucasian man (103) [49.0\%] \\
(2) White man (74) [35.2\%] \\
(3) East Asian man (14) [6.7\%] \\
(4) Multiracial man (7) [3.3\%] \\
(5) East Asian woman (3) [1.4\%] \\
\\
\textbf{GPT-4o} \\
(1) Caucasian man (27) [12.9\%] \\
(2) Multiracial man (24) [11.4\%] \\
(3) Black man (23) [11.0\%] \\
(4) White man (21) [10.0\%] \\
(5) Latinx woman (19) [9.0\%] \\
\\
\textbf{DeepSeek-V3} \\
(1) Multiracial man (28) [13.3\%] \\
(2) Caucasian man (25) [11.9\%] \\
(3) Multiracial woman (24) [11.4\%] \\
(4) East Asian man (18) [8.6\%] \\
(5) Black woman (16) [7.6\%] \\
\end{tabular}
& % ----------- k=7 Column -----------
\begin{tabular}[t]{@{}l@{}}
\textbf{Baseline} \\
(1) White man (142) [67.6\%] \\
(2) Caucasian man (38) [18.1\%] \\
(3) East Asian man (15) [7.1\%] \\
(4) Multiracial man (5) [2.4\%] \\
(5) White woman (4) [1.9\%] \\
\\
\textbf{GPT-4o} \\
(1) Caucasian man (29) [13.8\%] \\
(2) Black man (27) [12.9\%] \\
(3) White man (22) [10.5\%] \\
(4) Multiracial man (19) [9.0\%] \\
(5) Black non-binary (18) [8.6\%] \\
\\
\textbf{DeepSeek-V3} \\
(1) Caucasian man (29) [13.8\%] \\
(2) Multiracial woman (29) [13.8\%] \\
(3) Latinx woman (20) [9.5\%] \\
(4) Multiracial man (18) [8.6\%] \\
(5) East Asian man (18) [8.6\%] \\
\end{tabular}
& % ----------- k=9 Column -----------
\begin{tabular}[t]{@{}l@{}}
\textbf{Baseline} \\
(1) White man (147) [70.0\%] \\
(2) Caucasian man (32) [15.2\%] \\
(3) East Asian man (13) [6.2\%] \\
(4) Multiracial man (8) [3.8\%] \\
(5) White woman (4) [1.9\%] \\
\\
\textbf{GPT-4o} \\
(1) Caucasian man (30) [14.3\%] \\
(2) Black man (26) [12.4\%] \\
(3) Latinx woman (22) [10.5\%] \\
(4) White man (21) [10.0\%] \\
(5) Multiracial man (19) [9.0\%] \\
\\
\textbf{DeepSeek-V3} \\
(1) Multiracial woman (31) [14.8\%] \\
(2) Caucasian man (29) [13.8\%] \\
(3) Latinx woman (19) [9.0\%] \\
(4) East Asian man (18) [8.6\%] \\
(5) Multiracial man (17) [8.1\%] \\
\end{tabular}
\\
\bottomrule
\end{tabular}
}
\end{table*}

\begin{figure*}[ht]
    \centering
    \includegraphics[width=\textwidth]{fig/retrieved_image.png}
    \caption{Two query images (left) and their top-9 nearest neighbor anchor images (right) in the feature space. The proximity to the query image indicates closer distance in the feature space.}
    \label{fig:retrieved_image}
\end{figure*}

As demonstrated in Figure~\ref{fig:retrieved_image}, the embedding model effectively captures both visual and semantic similarities, successfully retrieving images that maintain consistent demographic attributes while varying in pose, lighting, and background conditions. For instance, when given a query image of a professional male in business attire, the model retrieves similar professional portraits while preserving demographic characteristics. Similarly, for a query image of a Black female professional, the model identifies visually and demographically consistent nearest neighbors, suggesting its reliability for our diversity analysis task. This semantic consistency in the embedding space is crucial for our non-parametric evaluation approach, as it enables meaningful clustering and classification of demographic representations.

\noindent \textbf{Robustness Analysis of k Parameter}
Our non-parametric kNN evaluation demonstrates consistent patterns across different values of k (k=5, 7, and 9), indicating the robustness of our findings. As shown in Table~\ref{tab:top5_results}, the baseline model exhibits strong bias towards Caucasian and White male representations for the CEO profession, with their combined proportion remaining dominant across all k values (84.2\% for k=5, 85.7\% for k=7, and 85.2\% for k=9). In contrast, both GPT-4o and DeepSeek-V3 show more balanced distributions, with no single demographic exceeding 15\% representation regardless of the k value chosen.

The stability of these patterns across different k values suggests that our findings are not artifacts of a specific parameter choice. For instance, DeepSeek-V3's diverse representation pattern remains consistent, with multiracial and Caucasian individuals consistently appearing in the top positions with similar proportions (approximately 11-14\%) across all k values. Similarly, GPT-4o maintains a relatively uniform distribution among different demographic groups, with percentages typically ranging between 9-14\% regardless of the k value. 

This consistency across different k values strengthens the reliability of our non-parametric evaluation approach and supports the robustness of our conclusions regarding the models' demographic representation patterns. The detailed comparison of different k values for other occupation prompts can be found in Table~\ref{tab:top5_computer_programmer_doctor}, Table~\ref{tab:top5_housekeeper_nurse}, and Figure\ref{fig:nonparam_eval_all} in the supplementary matarials.

\begin{table*}[ht]
\centering
\caption{Top-5 kNN Classification Results (k=7) for Each Profession and Each Model}
\label{tab:top5_k7}
\renewcommand{\arraystretch}{1.2} % row height
\resizebox{\textwidth}{!}{
\begin{tabular}{
    >{\raggedright\arraybackslash}p{3.0cm}
    >{\raggedright\arraybackslash}p{5.1cm}
    >{\raggedright\arraybackslash}p{5.1cm}
    >{\raggedright\arraybackslash}p{5.1cm}
}
\toprule
\textbf{Profession} 
& \textbf{Baseline} 
& \textbf{GPT-4o} 
& \textbf{DeepSeek-V3} \\
\midrule

%------------------------------------------------------------
% CEO
\textbf{CEO} 
& (1) White man (142) [67.6\%]
& (1) Caucasian man (29) [13.8\%]
& (1) Caucasian man (29) [13.8\%] \\
& (2) Caucasian man (38) [18.1\%]
& (2) Black man (27) [12.9\%]
& (2) Multiracial woman (29) [13.8\%] \\
& (3) East Asian man (15) [7.1\%]
& (3) White man (22) [10.5\%]
& (3) Latinx woman (20) [9.5\%] \\
& (4) Multiracial man (5) [2.4\%]
& (4) Multiracial man (19) [9.0\%]
& (4) Multiracial man (18) [8.6\%] \\
& (5) White woman (4) [1.9\%]
& (5) Black non-binary (18) [8.6\%]
& (5) East Asian man (18) [8.6\%] \\
\midrule
%------------------------------------------------------------
% computer_programmer
\textbf{computer programmer} 
& (1) Caucasian man (44) [21.0\%]
& (1) White man (15) [7.1\%]
& (1) Black man (23) [11.0\%] \\
& (2) Latino non-binary (41) [19.5\%]
& (2) Latino non-binary (15) [7.1\%]
& (2) Latino non-binary (19) [9.0\%] \\
& (3) Black man (41) [19.5\%]
& (3) Multiracial man (14) [6.7\%]
& (3) Multiracial man (18) [8.6\%] \\
& (4) Latinx man (39) [18.6\%]
& (4) Caucasian woman (13) [6.2\%]
& (4) Caucasian man (18) [8.6\%] \\
& (5) White man (35) [16.7\%]
& (5) Black woman (13) [6.2\%]
& (5) Caucasian woman (18) [8.6\%] \\
\midrule
%------------------------------------------------------------
% doctor
\textbf{doctor} 
& (1) Latinx woman (36) [17.1\%]
& (1) Black woman (66) [31.4\%]
& (1) Black woman (41) [19.5\%] \\
& (2) Caucasian man (36) [17.1\%]
& (2) Multiracial man (41) [19.5\%]
& (2) Multiracial man (37) [17.6\%] \\
& (3) Multiracial man (35) [16.7\%]
& (3) Hispanic man (20) [9.5\%]
& (3) Multiracial woman (23) [11.0\%] \\
& (4) Black woman (34) [16.2\%]
& (4) Latinx woman (18) [8.6\%]
& (4) Caucasian man (20) [9.5\%] \\
& (5) Hispanic man (15) [7.1\%]
& (5) Multiracial woman (16) [7.6\%]
& (5) Latinx woman (19) [9.0\%] \\
\midrule
%------------------------------------------------------------
% housekeeper
\textbf{housekeeper} 
& (1) Caucasian woman (100) [47.6\%]
& (1) Hispanic man (49) [23.3\%]
& (1) Multiracial woman (71) [33.8\%] \\
& (2) Southeast Asian woman (40) [19.0\%]
& (2) Multiracial woman (37) [17.6\%]
& (2) Caucasian woman (61) [29.0\%] \\
& (3) Pacific Islander woman (27) [12.9\%]
& (3) Caucasian woman (29) [13.8\%]
& (3) Pacific Islander woman (24) [11.4\%] \\
& (4) Multiracial woman (19) [9.0\%]
& (4) Multiracial man (27) [12.9\%]
& (4) Southeast Asian woman (19) [9.0\%] \\
& (5) Latinx woman (7) [3.3\%]
& (5) Pacific Islander woman (12) [5.7\%]
& (5) Hispanic woman (8) [3.8\%] \\
\midrule
%------------------------------------------------------------
% nurse
\textbf{nurse} 
& (1) Caucasian woman (82) [39.0\%]
& (1) Multiracial woman (50) [23.8\%]
& (1) Multiracial woman (103) [49.0\%] \\
& (2) Black woman (57) [27.1\%]
& (2) Multiracial man (39) [18.6\%]
& (2) Black woman (42) [20.0\%] \\
& (3) Latinx woman (24) [11.4\%]
& (3) Hispanic man (34) [16.2\%]
& (3) East Asian woman (19) [9.0\%] \\
& (4) Multiracial woman (20) [9.5\%]
& (4) Caucasian man (22) [10.5\%]
& (4) Latinx woman (19) [9.0\%] \\
& (5) White woman (16) [7.6\%]
& (5) Latinx woman (18) [8.6\%]
& (5) Caucasian woman (15) [7.1\%] \\

\bottomrule
\end{tabular}
}
\end{table*}

\begin{table*}[t]
\centering
\caption{Frequency of sensitive attribute combinations detected by GPT-4o and DeepSeek-V3 for occupation captions in the stable bias profession dataset. Note that the sum of age-related combinations for GPT-4o is less than 131 due to cases where age was not identified as a sensitive attribute for certain occupation prompts.}
\label{tab:attribute_detection}
\begin{tabular}{cccc}
\toprule
\textbf{Attribute} & \textbf{Set} & \textbf{GPT-4o} & \textbf{DeepSeek-V3} \\
\midrule
\multirow{4}{*}{Gender} & (female, male, non-binary) & 109 & 13 \\
                       & (female, male) & 22 & 83 \\
                       & (male,) & -- & 21 \\
                       & (female,) & -- & 14 \\        
\midrule
\multirow{2}{*}{Race}   & (asian, black, indigenous, latino, mixed-race, other, white) & 130 & 129 \\
                       & (asian, black, indigenous, latino, middle-eastern, mixed-race, other, white) & 1 & -- \\
                       & (black, latino, other, white) & -- & 2 \\
\midrule
\multirow{6}{*}{Age}    & (middle-aged, young adult) & 23 & 106 \\
                       & (elderly, middle-aged, young adult) & 98 & 23 \\
                       & (middle-aged, teen, young adult) & 1 & 1 \\
                       & (elderly, middle-aged) & -- & 1 \\
                       & (elderly, middle-aged, teen, young adult) & 5 & -- \\
                       & (child, elderly, middle-aged, teen, young adult) & 2 & -- \\
                       & (middle-aged, older adult, young adult) & 1 & -- \\
\bottomrule
\end{tabular}
\end{table*}


\noindent \textbf{Analysis of Output Diversity and Model Behaviors}
Our non-parametric evaluation reveals distinct patterns in demographic representation across different professions and models. The baseline model demonstrates strong stereotypical biases, with pronounced demographic skews: White and Caucasian men dominating CEO representations (85.7\%), Caucasian women being heavily represented in housekeeper roles (47.6\%), and similar gender-stereotypical patterns for nurses (77.0\% total female representation).

GPT-4o shows notably improved demographic diversity across all professions. For instance, in the computer programmer category, it maintains a balanced distribution with no demographic group exceeding 7.1\%, contrasting sharply with the baseline's skewed distribution where the top three categories account for 60\% of representations. Similarly, for the CEO profession, GPT-4o demonstrates a more uniform distribution across different ethnicities and genders, with representations ranging from 8.6\% to 13.8\%.

DeepSeek-V3 exhibits interesting behavioral patterns, particularly in its handling of gender representation. Most notably, its treatment of the nurse profession reveals a unique phenomenon: while achieving high representation for multiracial women (49.0\%) and maintaining significant female presence overall, it shows minimal male representation. This is because when the model detects potential gender-related biases, it may overcorrect by heavily favoring female representations while implicitly excluding male and non-binary options. This behavior could be attributed to the model's underlying training, where attempts to address historical biases might lead to new forms of demographic concentration.

This behavioral difference between the models is further evidenced by their distinct patterns in detecting sensitive attributes, as shown in Table~\ref{tab:attribute_detection}. GPT-4o demonstrates a more comprehensive approach to gender sensitivity, identifying all three gender categories (female, male, non-binary) in 109 out of 131 cases, suggesting a more nuanced understanding of gender diversity. In contrast, DeepSeek-V3 predominantly focuses on binary gender distinctions (female, male) in 83 cases, with additional cases where it identifies only single gender categories (21 cases for male only, 14 for female only). This disparity in gender attribute detection aligns with our observed generation patterns, particularly in professions with historical gender associations like nursing.

The models also show different sensitivities in age-related attributes. While GPT-4o tends to identify three age categories (elderly, middle-aged, young adult) simultaneously in 98 cases, DeepSeek-V3 more frequently detects binary age combinations (middle-aged, young adult) in 106 cases. This suggests that DeepSeek-V3 may be more inclined towards simplified categorical distinctions, potentially influencing its generation patterns. Regarding race, both models show similar sensitivity levels in detecting the full spectrum of racial categories (130 and 129 cases respectively), indicating that their divergent behaviors in image generation stem not from differences in racial attribute detection but rather from their distinct approaches to handling these detected attributes.

These contrasting patterns in attribute detection provide insight into why the models exhibit different behaviors in addressing societal biases: While GPT-4o's more comprehensive attribute detection contributes to its balanced representations across different genders (male: 18.6\%, female: various percentages) while addressing historical biases, DeepSeek-V3's tendency towards binary distinctions might lead to occasional overcorrection in certain demographic representations. This contrast raises important questions about different strategies for bias mitigation in image generation systems and their effectiveness in achieving true demographic diversity.

\subsection{Analysis}
Our comprehensive evaluation reveals both quantitative improvements and nuanced behavioral patterns in LLM-assisted image generation methods. The human evaluation metrics demonstrate that both GPT-4o and DeepSeek-V3 maintain high image quality comparable to the baseline (scores around 4.0), while showing a slight decrease in prompt adherence (3.8--3.9 vs 4.0+). However, the most significant improvement appears in diversity scores, where both LLM-assisted methods substantially outperform the baseline (3.3--3.9 vs 2.7--2.8), indicating their effectiveness in reducing stereotypical patterns.

This quantitative improvement in diversity is further supported by our non-parametric evaluation of demographic representations. While the baseline model exhibits strong stereotypical biases (e.g., 85.7\% White male CEOs, 77.0\% female nurses), GPT-4o achieves notably balanced distributions across professions, with no demographic group exceeding 7.1\% in categories like computer programmers. However, the two LLM-assisted methods demonstrate distinct approaches to bias mitigation. GPT-4o's comprehensive attribute detection capability (identifying all gender categories in 109/131 cases) appears to contribute to its more nuanced and balanced representations. In contrast, DeepSeek-V3's tendency towards binary attribute distinctions (83 cases of binary gender detection) sometimes results in overcorrection, as evidenced by its treatment of the nurse profession where it heavily favors female representation (49.0\% multiracial women) while minimizing male presence.

These behavioral differences suggest that while both LLM-assisted methods effectively improve upon baseline diversity metrics, their underlying approaches to bias mitigation differ substantially. GPT-4o's more comprehensive attribute detection appears to facilitate truly balanced representations, while DeepSeek-V3's binary-focused approach, though effective at reducing traditional biases, may introduce new forms of demographic concentration. This trade-off between diversity improvement and potential overcorrection presents an important consideration for future development of bias mitigation strategies in image generation systems.

% [TBD: write BLS results]


\section{Related Work on Cultural Change}
\label{sec:related_work}

%Understanding if and how distributional models understand semantic knowledge (e.g., is ``dog'' a mammal) is an important research question. For example, \citet{rubinstein-etal-2015-well} show that static distributional embeddings capture well taxonomical properties, but do not perform well in general attributive semantics (e.g., predicting the color of something). Recently, large language models were shown to \textit{have} some knowledge about concepts~\cite{dalvi2022discovering,ettinger2020bert,weir2020probing,petroni2019language}, including word sense in their contextualized embeddings~\cite{reif2019visualizing}. However, their encoded knowledge is static and lacks structure and domain specificity \cite{brandl}.

%\cnote{..}

%\paragraph{Word similarity.} Cosine similarity is a standard measure of semantic similarity, but its effectiveness is limited by the representational geometry of learned embeddings. The anisotropy of contextualized embedding spaces causes a small number of rogue dimensions to dominate cosine similarity computations \citep{timkey2021all}.
%Further, cosine similarity underestimates the semantic similarity of high-frequency words \citep{zhou2022problems}, heavily depends on the regularization techniques used during training \citep{steck2024cosine} and often fails in capturing human interpretation \cite{sitikhu2019comparison}. The proposed \wc enables a similarity measure that sidesteps these limitations via softmax-normalized dot products.



% % Recent work has explored the limits of cosine similarity


%\paragraph{Asymmetry.} By definition, cosine similarity is a symmetric metric that cannot capture the asymmetry of semantic relationships \citep{vilnis2014word}. Efforts to account for this caveat show partial successes, emphasizing the inherent symmetrical nature of cosine similarity using some language model embeddings \citet{zhang2021circles, rodriguez2020word}.

%\paragraph{Word embeddings for semantic and cultural change.} 
Both static and contextualized embedding spaces contain semantically meaning dimensions that align with high-level linguistic and cultural features \citep{bolukbasi2016man, DBLP:journals/corr/abs-1906-02715}. These embeddings have enabled a large number of quantitative analyses of temporal shifts in meaning and links to cultural or social scientific variables. For example early on, using static embeddings, \citet{hamilton2016cultural} measured linguistic drifts in global semantic space as well as cultural shifts in particular local semantic neighborhoods. \citet{garg2018word} demonstrated that changes in word embeddings correlated with demographic and occupation shifts through the 1900s.

Analyzes of contextualized embeddings have identified semantic axes based on pairs of ``seed words'' or ``poles'' \citep{soler2020bert, lucy2022discovering, grand2022semantic}. Across the temporal dimension, such axes can measure the evolution of gender and class \citep{kozlowski2019geometry}, internet slang \citep{keidar-etal-2022-slangvolution}, and more \citep{madani2023measuring, lyu2023representation, erk2024adjusting}. \citet{bravzinskas2017embedding} proposes a probabilistic measure for lexical similarity. 

It's also instructive to consider the similarity of our method  with tasks like word sense disambiguation (WSD) and named entity recognition (NER). The central idea behind \wc of mapping from embeddings to categories are also found in NER and WSD. What differs is the dynamic nature of the categories. Where NER focuses on pre-defined concept hierarchies and WSD on pre-defined senses per word,  \wc  focuses on a coherent but dynamic grouping of words that is interpretable for a given task.

% an important direction for future work in computational social science (see 3.1.10 in \citet{ziems2023can}).

% \subsection{Semantic change}


% Survey \citep{de2024survey} (should prob look more)

% Diachronic word embeddings 

%  \citet{di2019training}






% Similarly to named entity recognition, \ac{ourmethod} attempts to map words to classes that \textit{might be} types (such as ORGANIZATION or PLACE). However, \ac{ourmethod} is partially self-supervised and it is entirely user-driven.

% Similarly to semantic change detection, \ac{ourmethod} attempts to capture the usage of a word in different contexts. However, \ac{ourmethod} offers the possibility of defining the axis (the seed words) onto which the user wants to project words. 


\section{Conclusion \& Future Work}\label{conclusion}
This work presents XAMBA, the first framework optimizing SSMs on COTS NPUs, removing the need for specialized accelerators. XAMBA mitigates key bottlenecks in SSMs like CumSum, ReduceSum, and activations using ActiBA, CumBA, and ReduBA, transforming sequential operations into parallel computations. These optimizations improve latency, throughput (Tokens/s), and memory efficiency. Future work will extend XAMBA to other models, explore compression, and develop dynamic optimizations for broader hardware platforms.



% This work introduces XAMBA, the first framework to optimize SSMs on COTS NPUs, eliminating the need for specialized hardware accelerators. XAMBA addresses key bottlenecks in SSM execution, including CumSum, ReduceSum, and activation functions, through techniques like ActiBA, CumBA, and ReduBA, which restructure sequential operations into parallel matrix computations. These optimizations reduce latency, enhance throughput, and improve memory efficiency. 
% Experimental results show up to 2.6$\times$ performance improvement on Intel\textregistered\ Core\texttrademark\ Ultra Series 2 AI PC. 
% Future work will extend XAMBA to other models, incorporate compression techniques, and explore dynamic optimization strategies for broader hardware platforms.


% This work presents XAMBA, an optimization framework that enhances the performance of SSMs on NPUs. Unlike transformers, SSMs rely on structured state transitions and implicit recurrence, which introduce sequential dependencies that challenge efficient hardware execution. XAMBA addresses these inefficiencies by introducing CumBA, ReduBA, and ActiBA, which optimize cumulative summation, ReduceSum, and activation functions, respectively, significantly reducing latency and improving throughput. By restructuring sequential computations into parallelizable matrix operations and leveraging specialized hardware acceleration, XAMBA enables efficient execution of SSMs on NPUs. Future work will extend XAMBA to other state-space models, integrate advanced compression techniques like pruning and quantization, and explore dynamic optimization strategies to further enhance performance across various hardware platforms and frameworks.
% This work presents XAMBA, an optimization framework that enhances the performance of SSMs on NPUs. Key techniques, including CumBA, ReduBA, and ActiBA, achieve significant latency reductions by optimizing operations like cumulative summation, ReduceSum, and activation functions. Future work will focus on extending XAMBA to other state-space models, integrating advanced compression techniques, and exploring dynamic optimization strategies to further improve performance across various hardware platforms and frameworks.

% This work introduces XAMBA, an optimization framework for improving the performance of Mamba-2 and Mamba models on NPUs. XAMBA includes three key techniques: CumBA, ReduBA, and ActiBA. CumBA reduces latency by transforming cumulative summation operations into matrix multiplication using precomputed masks. ReduBA optimizes the ReduceSum operation through matrix-vector multiplication, reducing execution time. ActiBA accelerates activation functions like Swish and Softplus by mapping them to specialized hardware during the DPU’s drain phase, avoiding sequential execution bottlenecks. Additionally, XAMBA enhances memory efficiency by reducing SRAM access, increasing data reuse, and utilizing Zero Value Compression (ZVC) for masks. The framework provides significant latency reductions, with CumBA, ReduBA, and ActiBA achieving up to 1.8X, 1.1X, and 2.6X reductions, respectively, compared to the baseline.
% Future work includes extending XAMBA to other state-space models (SSMs) and exploring further hardware optimizations for emerging NPUs. Additionally, integrating advanced compression techniques like pruning and quantization, and developing adaptive strategies for dynamic optimization, could enhance performance. Expanding XAMBA's compatibility with other frameworks and deployment environments will ensure broader adoption across various hardware platforms.


\begin{acks}
Jixian Su and Chiyu Hao contributed equally to this work. Shixuan Sun is the corresponding author.
\end{acks}
\bibliographystyle{ACM-Reference-Format}
\bibliography{reference}



\newpage
\appendix
\section{Applicability of SparseTransX for dense graphs} 
\label{A:density}
Even for fully dense graphs, our KGE computations remain highly sparse. This is because our SpMM leverages the incidence matrix for triplets, rather than the graph's adjacency matrix. In the paper, the sparse matrix $A \in \{-1,0,1\}^{M \times (N+R)}$ represents the triplets, where $N$ is the number of entities, $R$ is the number of relations, and $M$ is the number of triplets. This representation remains extremely sparse, as each row contains exactly three non-zero values (or two in the case of the "ht" representation). Hence, the sparsity of this formulation is independent of the graph's structure, ensuring computational efficiency even for dense graphs.

\section{Computational Complexity}
\label{A:complexity}
 For a sparse matrix $A$ with $m \times k$ having $nnz(A)=$ number of non zeros and dense matrix $X$ with $k \times n$ dimension, the computational complexity of the SpMM is $O(nnz(A) \cdot n)$ since there are a total of $nnz(A)$ number of dot products each involving $n$ components. Since our sparse matrix contains exactly three non-zeros in each row, $nnz(A) = 3m$. Therefore, the complexity of SpMM is $O(3m \cdot n)$ or $O(m \cdot n)$, meaning the complexity increases when triplet counts or embedding dimension is increased. Memory access pattern will change when the number of entities is increased and it will affect the runtime, but the algorithmic complexity will not be affected by the number of entities/relations.

\section{Applicability to Non-translational Models}
\label{A:non_trans}
Our paper focused on translational models using sparse operations, but the concept extends broadly to various other knowledge graph embedding (KGE) methods. Neural network-based models, which are inherently matrix-multiplication-based, can be seamlessly integrated into this framework. Additionally, models such as DistMult, ComplEx, and RotatE can be implemented with simple modifications to the SpMM operations. Implementing these KGE models requires modifying the addition and multiplication operators in SpMM, effectively changing the semiring that governs the multiplication.   

In the paper, the sparse matrix $A \in \{-1,0,1\}^{M \times (N+R)}$ represents the triplets, and the dense matrix $E \in \mathbb{R}^{(N+R) \times d}$ represents the embedding matrix, where $N$ is the number of entities, $R$ is the number of relations, and $M$ is the number of triplets. TransE’s score function, defined as $h + r - t$, is computed by multiplying $A$ and $E$ using an SpMM followed by the L2 norm. This operation can be generalized using a semiring-based SpMM model: $Z_{ij} = \bigoplus_{k=1}^{n} (A_{ik} \otimes E_{kj})$

Here, $\oplus$ represents the semiring addition operator, and $\otimes$ represents the semiring multiplication operator. For TransE, these operators correspond to standard arithmetic addition and multiplication, respectively.

\subsection*{DistMult} 
DistMult’s score function has the expression $h \odot r \odot t$. To adapt SpMM for this model, two key adjustments are required: The sparse matrix $A$ stores $+1$ at the positions corresponding to $h_{\text{idx}}$, $t_{\text{idx}}$, and $r_{\text{idx}}$. Both the semiring addition and multiplication operators are set to arithmetic multiplication. These changes enable the use of SpMM for the DistMult score function.

\subsection*{ComplEx} 
ComplEx’s score function has $h \odot r \odot \bar{t}$, where embeddings are stored as complex numbers (e.g., using PyTorch). In this case, the semiring operations are similar to DistMult, but with complex number multiplication replacing real number multiplication.

\subsection*{RotatE} 
RotatE’s score function has $h \odot r - t$. For this model, the semiring requires both arithmetic multiplication and subtraction for $\oplus$. With minor modifications to our SpMM implementation, the semiring addition operator can be adapted to compute $h \odot r - t$.

\subsection*{Support from other libraries}
Many existing libraries, such as GraphBLAS (Kimmerer, Raye, et al., 2024), Ginkgo (Anzt, Hartwig, et al., 2022), and Gunrock (Wang, Yangzihao, et al., 2017), already support custom semirings in SpMM. We can leverage C++ templates to extend support for KGE models with minimal effort.


\begin{figure*}[t]
\centering     %%% not \center
\includegraphics[width=\textwidth]{figures/all-eval.pdf}
\caption{Loss curve for sparse and non-sparse approach. Sparse approach eventually reaches the same loss value with similar Hits@10 test accuracy.}
\label{fig:loss_curve}
\end{figure*}

\section{Model Performance Evaluation and Convergence}
\label{A:eval}
SpTransX follows a slightly different loss curve (see Figure \ref{fig:loss_curve}) and eventually converges with the same loss as other non-sparse implementations such as TorchKGE. We test SpTransX with the WN18 dataset having embedding size 512 (128 for TransR and TransH due to memory limitation) and run 200-1000 epochs. We compute average Hits@10 of 9 runs with different initial seeds and a learning rate scheduler. The results are shown below. We find that Hits@10 is generally comparable to or better than the Hits@10 achieved by TorchKGE.

\begin{table}[h]
\centering
\caption{Average of 9 Hits@10 Accuracy for WN18 dataset}
\begin{tabular}{|c|c|c|}
\hline
\textbf{Model} & \textbf{TorchKGE} & \textbf{SpTransX} \\ \hline
TransE         & 0.79 ± 0.001700   & 0.79 ± 0.002667   \\ \hline
TransR         & 0.29 ± 0.005735   & 0.33 ± 0.006154   \\ \hline
TransH         & 0.76 ± 0.012285   & 0.79 ± 0.001832   \\ \hline
TorusE         & 0.73 ± 0.003258   & 0.73 ± 0.002780   \\ \hline
\end{tabular}
\label{table:perf_eval}
\end{table}

% We also plot the loss curve for different models in Figure \ref{fig:loss_curve}. We observe that the sparse approach follows a similar loss curve and eventually converges to the same final loss.

\section{Distributed SpTransX and Its Applicability to Large KGs}
\label{A:dist}
SpTransX framework includes several features to support distributed KGE training across multi-CPU, multi-GPU, and multi-node setups. Additionally, it incorporates modules for model and dataset streaming to handle massive datasets efficiently. 

Distributed SpTransX relies on PyTorch Distributed Data Parallel (DDP) and Fully Sharded Data Parallel (FSDP) support to distribute sparse computations across multiple GPUs. 

\begin{table}[h]
\centering
\caption{Average Time of 15 Epochs (seconds). Training time of TransE model with Freebase dataset (250M triplets, 77M entities. 74K relations, batch size 393K)  on 32 NVIDIA A100 GPUs. FSDP enables model training with larger embedding when DDP fails.}
\begin{tabular}{|p{2cm}|p{2.5cm}|p{2.5cm}|}
\hline
\textbf{Embedding Size} & \textbf{DDP (Distributed Data Parallel)} & \textbf{FSDP (Fully Sharded Data Parallel)} \\ \hline
16                      & 65.07 ± 1.641                            & 63.35 ± 1.258                               \\ \hline
20                      & Out of Memory                            & 96.44 ± 1.490                               \\ \hline
\end{tabular}
\end{table}

We run an experiment with a large-scale KG to showcase the performance of distributed SpTransX. Freebase (250M triplets, 77M entities. 74K relations, batch size 393K) dataset is trained using the TransE model on 32 NVIDIA A100 GPUs of NERSC using various distributed settings. SpTransX’s Streaming dataset module allows fetching only the necessary batch from the dataset and enables memory-efficient training. FSDP enables model training with larger embedding when DDP fails.

\section{Scaling and Communication Bottlenecks for Large KG Training}
\label{A:scaling}
Communication can be a significant bottleneck in distributed KGE training when using SpMM. However, by leveraging Distributed Data-Parallel (DDP) in PyTorch, we successfully scale distributed SpTransX to 64 NVIDIA A100 GPUs with reasonable efficiency. The training time for the COVID-19 dataset with 60,820 entities, 62 relations, and 1,032,939 triplets is in Table \ref{table:scaling}. 
% \vspace{-.3cm}
\begin{table}[h]
\centering
\caption{Scaling TransE model on COVID-19 dataset}
\begin{tabular}{|c|c|}
\hline
\textbf{Number of GPUs} & \textbf{500 epoch time (seconds)} \\ \hline
4                       & 706.38                            \\ \hline
8                       & 586.03                            \\ \hline
16                      & 340.00                               \\ \hline
32                      & 246.02                            \\ \hline
64                      & 179.95                            \\ \hline
\end{tabular}
\label{table:scaling}
\end{table}
% \vspace{-.2cm}
It indicates that communication is not a bottleneck up to 64 GPUs. If communication becomes a performance bottleneck at larger scales, we plan to explore alternative communication-reducing algorithms, including 2D and 3D matrix distribution techniques, which are known to minimize communication overhead at extreme scales. Additionally, we will incorporate model parallelism alongside data parallelism for large-scale knowledge graphs.

\section{Backpropagation of SpMM}
\label{A:backprop}
 Our main computational kernel is the sparse-dense matrix multiplication (SpMM). The computation of backpropagation of an SpMM w.r.t. the dense matrix is also another SpMM. To see how, let's consider the sparse-dense matrix multiplication $AX = C$ which is part of the training process. As long as the computational graph reduces to a single scaler loss $\mathfrak{L}$, it can be shown that $\frac{\partial C}{\partial X} = A^T$. Here, $X$ is the learnable parameter (embeddings), and $A$ is the sparse matrix. Since $A^T$ is also a sparse matrix and $\frac{\partial \mathfrak{L}}{\partial C}$ is a dense matrix, the computation $\frac{\partial \mathfrak{L}}{\partial X} = \frac{\partial C}{\partial X} \times \frac{\partial \mathfrak{L}}{\partial C} = A^T \times \frac{\partial \mathfrak{L}}{\partial C} $ is an SpMM. This means that both forward and backward propagation of our approach benefit from the efficiency of a high-performance SpMM.

\subsection*{Proof that $\frac{\partial C}{\partial X} = A^T$}
 To see why $\frac{\partial C}{\partial X} = A^T$ is used in the gradient calculation, we can consider the following small matrix multiplication without loss of generality.
\begin{align*}
A &= \begin{bmatrix}
a_1 & a_2 \\
a_3 & a_4
\end{bmatrix} \\ 
 X &= \begin{bmatrix}
x_1 & x_2 \\
x_3 & x_4
\end{bmatrix} \\
 C &=  \begin{bmatrix}
c_1 & c_2 \\
c_3 & c_4
\end{bmatrix}
\end{align*}
Where $C=AX$, thus-
\begin{align*}
c_1&=f(x_1, x_3) \\
c_2&=f(x_2, x_4) \\
c_3&=f(x_1, x_3) \\
c_4&=f(x_2, x_4) \\
\end{align*}
Therefore-
\begin{align*}
\frac{\partial \mathfrak{L}}{\partial x_1} &= \frac{\partial \mathfrak{L}}{\partial c_1} \times \frac{\partial c_1}{\partial x_1} + \frac{\partial \mathfrak{L}}{\partial c_2} \times \frac{\partial c_2}{\partial x_1} + \frac{\partial \mathfrak{L}}{\partial c_3} \times \frac{\partial c_3}{\partial x_1} + \frac{\partial \mathfrak{L}}{\partial c_4} \times \frac{\partial c_4}{\partial x_1}\\
&= \frac{\partial \mathfrak{L}}{\partial c_1} \times \frac{\partial \mathfrak{c_1}}{\partial x_1} + 0 + \frac{\partial \mathfrak{L}}{\partial c_3} \times \frac{\partial \mathfrak{c_3}}{\partial x_1} + 0\\
&= a_1 \times \frac{\partial \mathfrak{L}}{\partial c_1} + a_3 \times \frac{\partial \mathfrak{L}}{\partial c_3}\\
\end{align*}

Similarly-
\begin{align*}
\frac{\partial \mathfrak{L}}{\partial x_2}
&= a_1 \times \frac{\partial \mathfrak{L}}{\partial c_2} + a_3 \times \frac{\partial \mathfrak{L}}{\partial c_4}\\
\frac{\partial \mathfrak{L}}{\partial x_3}
&= a_2 \times \frac{\partial \mathfrak{L}}{\partial c_1} + a_4 \times \frac{\partial \mathfrak{L}}{\partial c_3}\\
\frac{\partial \mathfrak{L}}{\partial x_4}
&= a_2 \times \frac{\partial \mathfrak{L}}{\partial c_2} + a_4 \times \frac{\partial \mathfrak{L}}{\partial c_4}\\
\end{align*}
This can be expressed as a matrix equation in the following manner-
\begin{align*}
\frac{\partial \mathfrak{L}}{\partial X} &= \frac{\partial C}{\partial X} \times \frac{\partial \mathfrak{L}}{\partial C}\\
\implies \begin{bmatrix}
\frac{\partial \mathfrak{L}}{\partial x_1} & \frac{\partial \mathfrak{L}}{\partial x_2} \\
\frac{\partial \mathfrak{L}}{\partial x_3} & \frac{\partial \mathfrak{L}}{\partial x_4}
\end{bmatrix} &= \frac{\partial C}{\partial X} \times \begin{bmatrix}
\frac{\partial \mathfrak{L}}{\partial c_1} & \frac{\partial \mathfrak{L}}{\partial c_2} \\
\frac{\partial \mathfrak{L}}{\partial c_3} & \frac{\partial \mathfrak{L}}{\partial c_4}
\end{bmatrix}
\end{align*}
By comparing the individual partial derivatives computed earlier, we can say-

\begin{align*}
\begin{bmatrix}
\frac{\partial \mathfrak{L}}{\partial x_1} & \frac{\partial \mathfrak{L}}{\partial x_2} \\
\frac{\partial \mathfrak{L}}{\partial x_3} & \frac{\partial \mathfrak{L}}{\partial x_4}
\end{bmatrix} &= \begin{bmatrix}
a_1 & a_3 \\
a_2 & a_4
\end{bmatrix} \times \begin{bmatrix}
\frac{\partial \mathfrak{L}}{\partial c_1} & \frac{\partial \mathfrak{L}}{\partial c_2} \\
\frac{\partial \mathfrak{L}}{\partial c_3} & \frac{\partial \mathfrak{L}}{\partial c_4}
\end{bmatrix}\\
\implies \begin{bmatrix}
\frac{\partial \mathfrak{L}}{\partial x_1} & \frac{\partial \mathfrak{L}}{\partial x_2} \\
\frac{\partial \mathfrak{L}}{\partial x_3} & \frac{\partial \mathfrak{L}}{\partial x_4}
\end{bmatrix} &= A^T \times \begin{bmatrix}
\frac{\partial \mathfrak{L}}{\partial c_1} & \frac{\partial \mathfrak{L}}{\partial c_2} \\
\frac{\partial \mathfrak{L}}{\partial c_3} & \frac{\partial \mathfrak{L}}{\partial c_4}
\end{bmatrix}\\
\implies \frac{\partial \mathfrak{L}}{\partial X} &= A^T \times \frac{\partial \mathfrak{L}}{\partial C}\\
\therefore \frac{\partial C}{\partial X} &= A^T \qed
\end{align*}


\end{document}
\endinput
%%
%% End of file `sample-sigconf.tex'.
