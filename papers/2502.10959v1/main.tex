%%
%% This is file `sample-sigconf.tex',
%% generated with the docstrip utility.
%%
%% The original source files were:
%%
%% samples.dtx  (with options: `sigconf')
%% 
%% IMPORTANT NOTICE:
%% 
%% For the copyright see the source file.
%% 
%% Any modified versions of this file must be renamed
%% with new filenames distinct from sample-sigconf.tex.
%% 
%% For distribution of the original source see the terms
%% for copying and modification in the file samples.dtx.
%% 
%% This generated file may be distributed as long as the
%% original source files, as listed above, are part of the
%% same distribution. (The sources need not necessarily be
%% in the same archive or directory.)
%%
%% Commands for TeXCount
%TC:macro \cite [option:text,text]
%TC:macro \citep [option:text,text]
%TC:macro \citet [option:text,text]
%TC:envir table 0 1
%TC:envir table* 0 1
%TC:envir tabular [ignore] word
%TC:envir displaymath 0 word
%TC:envir math 0 word
%TC:envir comment 0 0
%%
%%
%% The first command in your LaTeX source must be the \documentclass command.
\pdfoutput=1
\documentclass[acmsmall, nonacm]{acmart}

\newcommand{\sun}[1]{{\textcolor{black}{#1}}}
\newcommand{\su}[1]{{\textcolor{black}{#1}}}
\usepackage{enumitem}
\usepackage{tabularx} 
\usepackage[linesnumbered,vlined,ruled]{algorithm2e}
\usepackage{multirow}
\usepackage{graphicx}
\usepackage{caption}
\usepackage{subcaption}
\usepackage{booktabs} 
\usepackage{geometry} 
\usepackage{hyperref} 
\usepackage{pgfplots}
\usepackage{placeins}
\usepackage{afterpage}
\pgfplotsset{compat=1.17}
\usepgfplotslibrary{groupplots}
\setlength{\tabcolsep}{4pt} 
\setlength{\textfloatsep}{1.5pt plus 0.0pt minus 2.0pt} 
\setlength{\floatsep}{3pt plus 0.0pt minus 3.0pt}  
\setlength{\intextsep}{3pt plus 0.0pt minus 3.0pt} 
\captionsetup{skip=0pt} 


%% NOTE that a single column version may be required for 
%% submission and peer review. This can be done by changing
%% the \doucmentclass[...]{acmart} in this template to 
%% \documentclass[manuscript,screen]{acmart}
%% 
%% To ensure 100% compatibility, please check the white list of
%% approved LaTeX packages to be used with the Master Article Template at
%% https://www.acm.org/publications/taps/whitelist-of-latex-packages 
%% before creating your document. The white list page provides 
%% information on how to submit additional LaTeX packages for 
%% review and adoption.
%% Fonts used in the template cannot be substituted; margin 
%% adjustments are not allowed.
%%
%%
%% \BibTeX command to typeset BibTeX logo in the docs
% \AtBeginDocument{%
%   \providecommand\BibTeX{{%
%     \normalfont B\kern-0.5em{\scshape i\kern-0.25em b}\kern-0.8em\TeX}}}

%% Rights management information.  This information is sent to you
%% when you complete the rights form.  These commands have SAMPLE
%% values in them; it is your responsibility as an author to replace
%% the commands and values with those provided to you when you
%% complete the rights form.
% \setcopyright{acmcopyright}
% \copyrightyear{2018}
% \acmYear{2018}
% \acmDOI{XXXXXXX.XXXXXXX}

%% These commands are for a PROCEEDINGS abstract or paper.
% \acmConference[Conference acronym 'XX]{Make sure to enter the correct
%   conference title from your rights confirmation emai}{June 03--05,
%   2018}{Woodstock, NY}
%
%  Uncomment \acmBooktitle if th title of the proceedings is different
%  from ``Proceedings of ...''!
%
%\acmBooktitle{Woodstock '18: ACM Symposium on Neural Gaze Detection,
%  June 03--05, 2018, Woodstock, NY} 
% \acmPrice{15.00}
% \acmISBN{978-1-4503-XXXX-X/18/06}


%%
%% Submission ID.
%% Use this when submitting an article to a sponsored event. You'll
%% receive a unique submission ID from the organizers
%% of the event, and this ID should be used as the parameter to this command.
%%\acmSubmissionID{123-A56-BU3}

%%
%% For managing citations, it is recommended to use bibliography
%% files in BibTeX format.
%%
%% You can then either use BibTeX with the ACM-Reference-Format style,
%% or BibLaTeX with the acmnumeric or acmauthoryear sytles, that include
%% support for advanced citation of software artefact from the
%% biblatex-software package, also separately available on CTAN.
%%
%% Look at the sample-*-biblatex.tex files for templates showcasing
%% the biblatex styles.
%%

%%
%% The majority of ACM publications use numbered citations and
%% references.  The command \citestyle{authoryear} switches to the
%% "author year" style.
%%
%% If you are preparing content for an event
%% sponsored by ACM SIGGRAPH, you must use the "author year" style of
%% citations and references.
%% Uncommenting
%% the next command will enable that style.
%%\citestyle{acmauthoryear}

%%
%% end of the preamble, start of the body of the document source.
%% \BibTeX command to typeset BibTeX logo in the docs \AtBeginDocument{%  \providecommand\BibTeX{{%Bib\TeX}}}

%% \setcopyright{cc}
%% \setcctype{CC-BY}
\setcopyright{rightsretained}
% \acmJournal{J.AX}
\acmYear{2025} \acmVolume{3} \acmNumber{1} \acmArticle{70} \acmMonth{2} \acmPrice{15.00}\acmDOI{XX.XX/XXX.XX}

% The following includes the CC license icon appropriate for your paper.
% Download the image from www.scomminc.com/pp/acmsig/4ACM-CC-by-88x31.eps
% and place within your figs or figures folder

\makeatletter
% \gdef\@copyrightpermission{
%   \begin{minipage}{0.2\columnwidth}
%    \href{https://creativecommons.org/licenses/by/4.0/}{\includegraphics[width=0.90\textwidth]{img/4ACM-CC-by-88x31.eps}}
%   \end{minipage}\hfill
%   \begin{minipage}{0.8\columnwidth}
%    \href{https://creativecommons.org/licenses/by/4.0/}{This work is licensed under a Creative Commons Attribution International 4.0 License.}
%   \end{minipage}
%   \vspace{5pt}
% }
\makeatother

\begin{document}

%%
%% The "title" command has an optional parameter,
%% allowing the author to define a "short title" to be used in page headers.

\title{Revisiting the Design of In-Memory Dynamic Graph Storage}

%%
%% The "author" command and its associated commands are used to define
%% the authors and their affiliations.
%% Of note is the shared affiliation of the first two authors, and the
%% "authornote" and "authornotemark" commands
%% used to denote shared contribution to the research.
\author{Jixian Su}
\affiliation{
  \institution{Shanghai Jiao Tong University}
  \city{Shanghai}
  \country{China}
}
\email{sjx13623816973@sjtu.edu.cn}

\author{Chiyu Hao}
\affiliation{
  \institution{Shanghai Jiao Tong University}
  \city{Shanghai}
  \country{China}
}
\email{hcahoi11@sjtu.edu.cn}

\author{Shixuan Sun}
\affiliation{
  \institution{Shanghai Jiao Tong University}
  \city{Shanghai}
  \country{China}
}
\email{sunshixuan@sjtu.edu.cn}

\author{Hao Zhang}
\affiliation{
  \institution{Huawei Cloud}
  \city{Beijing}
  \country{China}
}
\email{zhanghao687@huawei.com}

\author{Sen Gao}
\affiliation{
  \institution{National University of Singapore}
  % \city{Singapore}
  \country{Singapore}
}
\email{sen@u.nus.edu}

\author{Jiaxin Jiang}
\affiliation{
    \institution{National University of Singapore}
    % \city{Singapore}
    \country{Singapore}
}
\email{jxjiang@nus.edu.sg}

\author{Yao Chen}
\affiliation{
    \institution{National University of Singapore}
    % \city{Singapore}
    \country{Singapore}
}
\email{yaochen@nus.edu.sg}

\author{Chenyi Zhang}
\affiliation{
  \institution{Huawei Cloud}
  \city{Hangzhou}
  \country{China}
}
\email{zhangchenyi@huawei.com}


\author{Bingsheng He}
\affiliation{
    \institution{National University of Singapore}
    % \city{Singapore}
    \country{Singapore}
}
\email{hebs@comp.nus.edu.sg}



\author{Minyi Guo}
\affiliation{
    \institution{Shanghai Jiao Tong University}
    \city{Shanghai}
    \country{China}
}
\email{guo-my@cs.sjtu.edu.cn}

\renewcommand{\shortauthors}{Jixian Su et al.}
%%
%% By default, the full list of authors will be used in the page
%% headers. Often, this list is too long, and will overlap
%% other information printed in the page headers. This command allows
%% the author to define a more concise list
%% of authors' names for this purpose.
% \renewcommand{\shortauthors}{Trovato and Tobin, et al.}

%%
%% The abstract is a short summary of the work to be presented in the
%% article.
\begin{abstract}

% Recent works to jointly reconstruct 3D human and object from a single RGB image, are mostly model-based, that fail to capture the fine details of the clothed human body and object surface. In this paper, we introduce ReCHOR, a novel, model-free, first-method to produce realistic clothed human-object reconstructions from a monocular view. This is extremely challenging due to human-object occlusions, diverse interactions and depth ambiguity, as it needs to infer both 3D spatial awareness and high resolution details. Our core idea is based on estimating neural implicit representations for human and object respectively by an attention-based neural implicit model that attends to pixel-aligned features from both the global human-object image for spatial awareness and  the local separate view of human and object images for high quality details. Additionally, the network is conditioned on semantic features from an initial estimated human-object pose prior and a generative diffusion model that inpaints occluded regions, thus enabling the retrieval of details from them.
% We also propose a synthetic dataset with rendered scenes of diverse, inter-occluded 3D human and object scans, to train our network. We evaluate our method on the synthetic and real world BEHAVE dataset. Our experiments show that our method outperforms the SOTA in achieving realistic clothed human-object reconstructions.
Recent approaches to jointly reconstruct 3D humans and objects from a single RGB image represent 3D shapes with template-based or coarse models, which fail to capture details of loose clothing on human bodies. In this paper, we introduce a novel implicit approach for jointly reconstructing realistic 3D clothed humans and objects from a monocular view. For the first time, we model both the human and the object with an implicit representation, allowing to capture more realistic details such as clothing. This task is extremely challenging due to human-object occlusions and the lack of 3D information in 2D images, often leading to poor detail reconstruction and depth ambiguity. To address these problems, we propose a novel attention-based neural implicit model that leverages image pixel alignment from both the input human-object image for a global understanding of the human-object scene and from local separate views of the human and object images to improve realism with, for example, clothing details. Additionally, the network is conditioned on semantic features derived from an estimated human-object pose prior, which provides 3D spatial information about the shared space of humans and objects. To handle human occlusion caused by objects, we use a generative diffusion model that inpaints the occluded regions, recovering otherwise lost details. For training and evaluation, we introduce a synthetic dataset featuring rendered scenes of inter-occluded 3D human scans and diverse objects. Extensive evaluation on both synthetic and real-world datasets demonstrates the superior quality of the proposed human-object reconstructions over competitive methods.
\end{abstract}
%%
%% The code below is generated by the tool at http://dl.acm.org/ccs.cfm.
%% Please copy and paste the code instead of the example below.
%%
\begin{CCSXML}
<ccs2012>
   <concept>
       <concept_id>10002951.10002952.10002953.10010146</concept_id>
       <concept_desc>Information systems~Graph-based database models</concept_desc>
       <concept_significance>500</concept_significance>
       </concept>
   <concept>
       <concept_id>10002951.10002952.10002971</concept_id>
       <concept_desc>Information systems~Data structures</concept_desc>
       <concept_significance>300</concept_significance>
       </concept>
   <concept>
       <concept_id>10002951.10003152.10003520</concept_id>
       <concept_desc>Information systems~Storage management</concept_desc>
       <concept_significance>100</concept_significance>
       </concept>
 </ccs2012>
\end{CCSXML}

\ccsdesc[500]{Information systems~Graph-based database models}
\ccsdesc[300]{Information systems~Data structures}
\ccsdesc[100]{Information systems~Storage management}
% %%
% %% Keywords. The author(s) should pick words that accurately describe
% %% the work being presented. Separate the keywords with commas.
\keywords{dynamic graph storage; graph concurrency control; graph neighbor index; benchmark framework.}

%% A "teaser" image appears between the author and affiliation
%% information and the body of the document, and typically spans the
%% page.
% \begin{teaserfigure}
%   \includegraphics[width=\textwidth]{sampleteaser}
%   \caption{Seattle Mariners at Spring Training, 2010.}
%   \Description{Enjoying the baseball game from the third-base
%   seats. Ichiro Suzuki preparing to bat.}
%   \label{fig:teaser}
% \end{teaserfigure}

% \received{July 2024}
% \received[revised]{September 2024}
% \received[accepted]{November 2024}
% \acmSubmissionID{V3mod070}
%%
%% This command processes the author and affiliation and title
%% information and builds the first part of the formatted document.
\maketitle


\section{Introduction}\label{sec:intro}

In computational finance, Monte Carlo simulations are used extensively to estimate the expected value of financial payoffs based on the solution of stochastic differential equations (SDEs) which model the evolution of stock prices, interest rates, exchange rates and other quantities \cite{glasserman04}.  Monte Carlo methods are very general and flexible, but for high accuracy it requires generating a large number of costly SDE path approximations, which has motivated research into a number of variance reduction or, equivalently, cost reduction techniques. One such method is
Multilevel Monte Carlo (MLMC), which was proposed in \cite{GILES2008} and was adapted for various applications that are summarised in \cite{Giles_overview17} and successfully combined with other methods such as quasi-Monte Carlo methods. The main idea of MLMC is to approximate the payoff using different time stepping resolutions when numerically solving the underlying SDE and to generate an optimal number of samples on each level, such that the overall computational cost is minimised subject to the desired bound on the variance. %, such that the total computational cost is minimised. 
The computational savings come from the fact that most samples are computed on the coarser levels and hence are less expensive while only a few samples from the finest levels are required \cite{GILES2008}.


Among the directions in which the computational cost 
of MLMC methods could further be reduced, an important avenue is the use of lower precision calculations, especially for the first Monte Carlo levels where the targeted accuracy is relatively low. 
 An overview of the research on mixed precision for the standard Monte Carlo (MC) framework is provided in \cite{ChowMixedPrecisionStandardMC} but only a few references study the potential of low precision computation in the MLMC framework \cite{Rounding_error_oliver}. To the best of our knowledge, the only MLMC framework with customised precision in the literature is \cite{brugger2014mixed}, but they use a uniform precision for all operations on each Monte Carlo level instead of optimising 
 the precision of each intermediary variable to reduce as much as possible the cost of path generation.
 
An important motivation for an MLMC framework with variable precision would be performing the low precision computations on reconfigurable hardware devices such as Field Programmable Gate Arrays (FPGAs). FPGAs contain customizable logic blocks and connectors that make it easy to adapt the digital circuit architecture for a specific application, leading to a highly parallel and optimised implementation. Therefore they are successfully exploited in applications that require high speed and have high computational workload, such as signal processing \cite{woods2008fpga}, and real time applications like high frequency trading \cite{HFT1,HFT2}. That is why a number of previous works in hardware architecture design implemented the MLMC algorithm to price financial options using FPGAs as accelerators, which resulted in improved speed and power efficiency compared to full CPU architectures \cite{Schryver2013AMM}. The paper \cite{lindsey2016domain} also proposed 
a Domain Specific Language to automate the configuration of FPGAs for this specific application. However, only \cite{brugger2014mixed} proposed a heuristic to reduce the precision in calculations.

In addition, all aforementioned works considered that the random number generation (RNG) is performed in single or double precision. Yet in most cases an important portion of the workload in the overall MLMC simulation comes from the RNG and in \cite{brugger2014mixed} this limited the total computational savings.
To reduce the cost of MLMC simulations in particular those based on the Geometric Brownian Motion (GBM), \cite{approximateICDF_Oliver, NestedOliver} have proposed to use approximate random numbers that are generated by applying an approximation of the inverse CDF to uniform random numbers. In \cite{NestedOliver}, the authors proposed a way to integrate these lower precision random variables into a \textit{nested} MLMC framework and completed a numerical analysis to bound the resulting error at each MC level by a product of the time step and the error in the random number approximation. The same authors show in \cite{approximateICDF_Oliver} that using approximate random variables reduces the cost of path generation by a factor 7.


In this paper we propose a nested MLMC framework that combines the use of approximate random normal variables and lower precision calculations to reduce the computational cost of MLMC even further than \cite{brugger2014mixed,NestedOliver}. We illustrate the efficiency of our framework in Matlab, after making several assumptions on the cost of operations and size of the errors that we carefully justify. We focus on the case of GBM and use the approximate RNG methods presented in \cite{approximateICDF_Oliver} as well as a new slightly modified method that combines CDF inversion and the central limit theorem. To choose the precision of the variables in the low precision path generation, we introduce a novel method to optimise the bit-widths. This optimisation is performed before the main path generation loop is executed and is based on a linear model of the payoff error  
due to rounding when computing in low precision. The error model relies on algorithmic differentiation in a similar manner to \cite{unifying-bwoptim,bitwidth-AD,ADAPT}. The bit-width optimisation procedure can be performed off-line, so this stage can be excluded from the on-line time complexity of our framework. The user specified desired accuracy is then enforced by calculating on-line the number of samples that need to be generated.

In terms of hardware design, we suggest implementing the low precision path generation on FPGAs and the full-precision ones on a CPU or GPU. 
The FPGA offers enough flexibility to define a separate bit-width for every variable in the low precision path generation, and can be reconfigured periodically to update the bit-widths when the market parameters have changed considerably. 


The paper is organized as follows : \Cref{sec:MLMC} introduces MLMC and nested MLMC to make clear the estimator that is implemented in our framework. Then in \Cref{sec:RNG} we detail the methods that could be used to obtain approximate random normally distributed numbers very cheaply for the low precision path generation. In \Cref{sec:error_model} and \Cref{sec:costModel} we propose an error model and a cost model (resp.) that we then use to formulate the optimisation problem that is solved to obtain the optimal bit-widths of fixed point variables in \Cref{sec:optimisation}. Finally we summarise our results and future directions in \Cref{sec:conclusion}.



\section{Background}
\label{sec:background}

\begin{figure*}[htbp]
\centering
\includegraphics[width=\textwidth]{Fig_background.pdf}
\caption{Ciphertext side-channel examples and revisiting vulnerabilities from the perspective of compilation.}
\label{fig:background}
\end{figure*}

\subsection{Ciphertext Side-Channel Attacks}
\label{subsec:ciphertext}

The ciphertext side channel originates from the deterministic memory encryption implemented in AMD's TEE.
The encrypted memory is calculated by an XOR-Encrypt-XOR (XEX) mode, expressed as: $c = ENC(m \oplus T(P_{m})) \oplus T(P_{m})$, where the plaintext $m$ undergoes the XOR operations before and after AES-128 encryption with a tweak value $T(P_{m})$ that incorporates the physical address $P_{m}$.
Without freshness in the encryption process, the encryption of the same plaintext at a given physical address produces the identical ciphertext.
It is crucial to acknowledge that this vulnerability extends to other deterministic encryption-based TEE architectures as long as attackers have read accesses to ciphertext (via software access~\cite{li2021cipherleaks} or memory bus snooping~\cite{lee2020off}).

% \begin{figure}[htbp]
% \vspace{-5pt}
% \begin{minipage}[c]{0.5\linewidth}
%     \begin{subfigure}[b]{\linewidth}
%     \centering
%     \footnotesize
%     \begin{tabular}{l}
%         1: pbit $\leftarrow$ 1;\\
%         2: \textbf{for}\ i $\leftarrow$ cardinality\_bit - 1\ downto\ 0$\lbrace$\\
%         3: $\quad$ kbit $\leftarrow$ BN\_is\_bit\_set(k, i) $\wedge$ pbit;\\
%         4: $\quad$ EC\_POINT\_CSWAP(kbit, r, s, ...);\\
%         5: $\quad$ ...\\
%         6: $\quad$ pbit $\leftarrow$ pbit $\wedge$ kbit;$\rbrace$\\
%     \end{tabular}
%     \caption{ossl\_ec\_scalar\_mul\_ladder.}
%     \label{fig:channel1}
%     \end{subfigure}
% \end{minipage}
% \hspace{15pt}
% \begin{minipage}[c]{0.4\linewidth}
%     \begin{subfigure}[b]{0.9\linewidth}
%     \centering
%     \footnotesize
%     \begin{tabular}{l}
%         1: \textbf{for}\ i $\leftarrow$ 0\ to\ nwords - 1$\lbrace$\\
%         2: $\quad$ t $\leftarrow$ (a.d[i] $\wedge$ b.d[i])\\
%         3: $\quad \quad \quad$ \&\ condition;\\
%         4: $\quad$ a.d[i] $\leftarrow$ a.d[i] $\wedge$ t;\\
%         5: $\quad$ b.d[i] $\leftarrow$ b.d[i] $\wedge$ t;$\rbrace$\\
%     \end{tabular}
%     \caption{BN\_constant\_swap.}
%     \label{fig:channel2}
%     \end{subfigure}
% \end{minipage}
% \caption{Ciphertext side-channel examples.}%\yz{change font in figures.}
% \label{fig:channels}
% \vspace{-5pt}
% \end{figure}

Two attack schemes are introduced in~\cite{li2022systematic}.
The \textit{Dictionary} attack involves the continuous monitoring of the ciphertext at a fixed memory address to construct a dictionary containing mappings of ciphertext-plaintext pairs.
Consider the code snippet shown in \F~\ref{fig:background}(a), extracted from the ECDSA Montgomery ladder algorithm implemented in OpenSSL-3.0.2.
In each loop iteration, the \texttt{BN\_is\_bit\_set} function (denoted by $k_{i}$ in line 3) is utilized to obtain one bit of the secret $k$.
Following this, the $kbit$ variable is computed through an XOR operation with the value in $pbit$, which is then written back to $pbit$.
Given the dual XOR operations in lines 3 and 6, $pbit$ ultimately stores each bit of the secret $k$.
The attacker records consecutive ciphertext pairs ($pbit$-$kbit$) both before and after the \texttt{BN\_is\_bit\_set} function, aiming to deduce $k_{i}$ in each iteration based on the changes observed in ciphertext pairs.
In contrast, the \textit{Collision} attack focuses on identifying repetitions or alterations in certain ciphertexts to break the constant-time mechanism.
\F~\ref{fig:background}(b) shows the constant-time-swap function \texttt{BN\_constant\_swap}.
This function takes two variables $a$ and $b$, along with a decision $C$ (e.g., $kbit$ in line 4 of \F~\ref{fig:background}(a)).
If $C$ is set to 1, the values of $a$ and $b$ are exchanged, leading to observable changes in the ciphertext. Conversely, if $C$ is 0, the ciphertext remains unaltered.
In this way, the \textit{Collision} attack recovers the decision $C$, undermining the constant-time component.

Currently, many well-known cryptographic applications are vulnerable to this attack, including RSA and ECDSA (such as \textit{secp256k1} and \textit{secp384r1}) equipped with constant-time algorithms, ECDSA from WolfSSL-5.3.0, ECDSA and RSA from MbedTLS-3.1.0, as well as EdDSA (\textit{Ed25519}) from OpenSSH adopted by Ubuntu LTS 20.04~\cite{li2021cipherleaks, li2022systematic}.

\subsection{Countermeasures to Ciphertext Side-channels}
\label{subsec:countermeasures}

Hardware-based countermeasures provide stronger security by eliminating ciphertext side channels, but they require extensive validation before chip manufacturing. In contrast, we choose a software-based approach, enabling quicker implementation and deployment without modifying hardware.
Unfortunately, existing countermeasures for cache and timing side channels~\cite{percival2005cache, osvik2006cache, zhang2012cross, yarom2014flush, liu2015last, yarom2014recovering, ryan2019return, aranha2020ladderleak}, like constant-time cryptography, cannot mitigate ciphertext side channels. While constant-time cryptography avoids secret-dependent branches and memory accesses, it has been shown to be ineffective against ciphertext side-channel attacks~\cite{li2021cipherleaks, li2022systematic, deng2023cipherh}.

% Previous efforts adhering to this concept can be categorized into three classes. 
% 1) Researchers verify whether a cryptography program satisfies the constant-time criterion using various approaches, including the program counter model~\cite{agat2000transforming, molnar2005program, barthe2006preventing, kopf2007transformational, almeida2013certified, mantel2015transforming}, observation-equivalence-based noninterference~\cite{barthe2014system, almeida2016verifiable, rodrigues2016sparse, dehesa2017verifying}, and self-composition-based noninterference~\cite{almeida2013formal, almeida2016verifying, chen2017precise, antonopoulos2017decomposition, yang2018lazy, blazy2019verifying, daniel2020binsec}.
% 2) Conceptually, formally constructing high-assurance cryptography libraries shall fundamentally resolve the constant-time issues, leveraging formal languages like F$^{*}$~\cite{zinzindohoue2016verified}, HACL$^{*}$~\cite{zinzindohoue2017hacl}, Vale~\cite{bond2017vale}, Jasmin~\cite{almeida2017jasmin} and Fact~\cite{cauligi2019fact}.
% 3) Transforming existing programs into constant-time equivalents also significantly contributes to resisting side channels. For instance, some approaches~\cite{wu2018eliminating,soares2021memory} execute both real and decoy paths; Constantine~\cite{borrello2021constantine} leverages the linearization of control-flow and data-flow.

Without detailed implementation, AMD's whitepaper~\cite{amdmeasures} and Li et al.~\cite{li2022systematic} proposed countermeasures as follows, but no single software-based scheme is perfectly suited for both methodology and implementation. 
Therefore, exploring different mitigation approaches, particularly through compiler-level optimizations and combinations, offers valuable insights for improving defenses.

\begin{packed_itemize}
\item[1)] Preserving secret variables in registers instead of memory enhances security~\cite{li2022systematic}, but faces implementation challenges due to limited register availability.

\item[2)] Avoiding the reuse of fixed memory addresses ensures fresh ciphertexts~\cite{li2022systematic, amdmeasures}, but requires extra memory and precise runtime reference management, potentially leading to significant performance overhead.

\item[3)] Introducing a random nonce to the plaintext with each memory write increases ciphertext unpredictability~\cite{li2022systematic}. This includes masking and padding strategies~\cite{amdmeasures}, where padding requires extended data structures.
\end{packed_itemize}

\section{A Common Abstraction for DGS}\label{sec:framework}

In this section, we propose a common abstraction for DGS to facilitate a systematical study of existing methods.

\subsection{Graph Query and Data Abstraction}\label{sec:data_abstraction}

In transaction management~\cite{ramakrishnan2002database}, a database is a set $\{x\}$ of tuples in tables. A transaction consists of a sequence of read and write operations on $\{x\}$, beginning with a \textsc{Begin} command and ending with either \textsc{Commit}, indicating successful execution, or \textsc{Abort}, indicating failure and reverting modifications. Building on this concept, we propose a simple yet effective multi-level abstraction for graph query and data that aims to: 1) reflect the characteristics of graph queries; 2) indicate the nature of graph data; and 3) capture graph data access patterns. Figure \ref{fig:data_abstraction} shows our abstraction.

\begin{figure}[h]\small
    \setlength{\abovecaptionskip}{3pt}
    \setlength{\belowcaptionskip}{0pt}
    \includegraphics[scale=0.75]{img/data_abstraction.pdf}
    \centering
    \caption{The abstraction of graph query and data.}
    \label{fig:data_abstraction}
\end{figure}

\noindent\textbf{Global Abstraction.} Graph queries are categorized into write queries $\Delta G$ and read queries $Q$ as discussed in Section \ref{sec:preliminaries}. The global abstraction models the relationship among these queries by maintaining a global timestamp $t(G)$, initialized to 0 and incremented by 1 only when $\Delta G$ is committed. To ensure the serializability of graph queries, DGS requires that each committed write query be uniquely identified by its commit timestamp, denoted as $\Delta G_{i + 1}$ for the write query committed at $t(G) = i$. This abstraction effectively captures the construction of a dynamic graph $G = (G_0, \Delta \mathcal{G})$, where $\Delta \mathcal{G}$ is the serial execution order of committed write queries. A read query $Q$ starting at $t(G) = i$ has a local timestamp $t(Q)$.

\vspace{2pt}
\noindent\textbf{Local Abstraction.} In cooperation with the global abstraction, the local abstraction allows us to drill down to each data object and their primitive operations. The graph $G$ consists of vertices and edges. To indicate their differences and interconnections, $G$ is organized into a vertex table containing $V(G)$ and a set of neighbor tables, each corresponding to a neighbor set $N(u)$. Each entry in $V(G)$ contains the vertex ID $u$, the location of $N(u)$, and its properties, while each entry in $N(u)$ contains the neighbor ID $v$ and the properties associated with edge $e(u, v)$. Given a write query $\Delta G_i$, each operation creates a new version of a vertex or neighbor $u$ with the timestamp $t(u) = i$. Specifically, an insert (resp. delete) operation on $u$ creates a new version with the \textsc{op-type} as $I$ (resp. $D$). Updating an element is performed via an insert operation.

Lemma \ref{lemma:isolation} can be proven using the dependency graph \cite{fekete2005making} within the multi-level abstraction.  The lemma shows that by maintaining the serializability of write queries, DGS can achieve serializable isolation by ensuring $Q$ has a consistent view of $G_i$. This is done by allowing $Q$ to access only the latest version of vertices or neighbors $u$ such that $t(u) \leqslant t(Q)$. This approach allows us to coordinate $Q$'s data access based on timestamps without complex concurrency control protocols for read queries. Although existing DGS methods use this optimization, they do not explicitly and formally discuss it.

\begin{lemma} \label{lemma:isolation}
Suppose DGS maintains the serializability of write queries with the serial execution order $\Delta \mathcal{G}$. Given a read query $Q$ starting at timestamp $i$, ensuring $Q$ has a consistent view of $G_i = G_0 \oplus...\oplus \Delta G_i$ guarantees global serializable isolation for read and write queries.
\end{lemma}
\emph{Proof.} In the dependency graph, each node $Q_x$ represents a query, and an edge from $Q_x$ to $Q_y$ indicates that an operation $o_x \in Q_x$ conflicts with $o_y \in Q_y$, and $o_x$ precedes $o_y$ in execution. Operations $o_x$ and $o_y$ conflict if they act on the same object and at least one is a write. Queries are conflict serializable \emph{iff} the dependency graph is acyclic. Since DGS maintains the serializability of write queries, the dependency graph formed by them is acyclic. Ensuring $Q$ has a consistent view of $G_i$ guarantees that all operations in $Q$ depend only on operations from preceding write queries. Thus, the node representing $Q$ in the dependency graph has no incoming edges. Therefore, the combined dependency graph of read and write queries remains acyclic, proving the queries are conflict serializable.




\subsection{Graph Operations Abstraction}\label{sec:operation_abstraction}

We next analyze graph data access patterns based on the multi-level abstraction. From the perspective of DGS, a write (or read) query consists of a sequence of write (or read) operations on vertices and edges, while the computation logic and state of the query are beyond the scope of DGS. These \emph{graph operations} include:

\begin{itemize}[leftmargin=*]
\item \textsc{InsVtx($u$)}: inserting a vertex $u$ into $V(G)$;
\item \textsc{InsEdge($u, v$)}: inserting an edge $e(u, v)$ into $E(G)$;
\item \textsc{SearchVtx($u$)}: finding a vertex $u$ in $V(G)$;
\item \textsc{SearchEdge($u, v$)}: finding an edge $e(u, v)$ in $E(G)$;
\item \textsc{ScanVtx($G$)}: traversing the vertex set $V(G)$;
\item \textsc{ScanNbr($u$)}: traversing the neighbor set $N(u)$.
\end{itemize}

\begin{figure}[htbp]\small
    \setlength{\abovecaptionskip}{0pt}
    \setlength{\belowcaptionskip}{0pt}
    \includegraphics[scale=0.75]{img/operaiton_abstraction.pdf}
    \centering
    \caption{The abstraction of graph operations.}
    \label{fig:primitive_opeartions}
\end{figure}

Figure \ref{fig:primitive_opeartions} presents the abstraction of graph operations, where each node represents a graph operation, and each edge denotes a primitive operator on vertex or neighbor tables. The path from the start node to an operation node illustrates the primitive operators required to perform the operation and shows the relationships between different operations (e.g., \textsc{InsEdge} invokes \textsc{SearchEdge}). In summary, this abstraction provides a unified execution routine on graph operations.

% From the abstraction, we observe that operations on neighbor tables depend on those on the vertex table to retrieve the position of the target neighbor table, the insert operation depends on the search operation to locate the target object, and any operation on $u$ or its neighbors requires access rights to $u$ first.

Let $P_V$ and $P_N$ denote the paths from the start node to the graph operation node in the vertex table and neighbor table operations, respectively, in Figure \ref{fig:primitive_opeartions}. Equation \ref{eq:cost} defines the cost $T$ of a graph operation, where $T_{CC}$ is the cost of coordinating access to the target vertex, and $T_p$ is the cost of the primitive operator $p$. Concurrency control requires the cooperation of underlying graph containers, such as checking a creation timestamp for each object. $\alpha_p$ is the overhead amplification ratio of concurrency control on the primitive operator $p$. An optimal value is 1, indicating no performance degradation due to concurrency control. This equation highlights the factors affecting DGS performance, serving as a tool to systematically study these performance factors rather than predicting the exact cost of a graph operation.

\begin{equation} \label{eq:cost}
T = T_{CC} + \sum_{p \in P_V} \alpha_p T_p + \sum_{p \in P_N} \alpha_p T_p.
\end{equation}

\noindent\textbf{Remark. } \sun{We exclude edge update and delete operations as they follow a similar process to insertions. Fine-grained methods like Sortledton, Teseo, and LiveGraph handle deletions by: 1) locating the target vertex, and 2) creating a new version marked as \emph{delete} or updating the end timestamp to indicate the deletion, as discussed above. Space is later reclaimed through garbage collection. Coarse-grained methods like Aspen, use a copy-on-write strategy, where deletion involves removing a vertex from a block rather than adding one, much like with insertions.} Graph query workloads generally focus on analyzing connections between vertices, with vertex updates, especially deletions, being rare \cite{zhu2019livegraph}. Consequently, existing DGS works typically focus on operations on neighbor tables. Therefore, this paper primarily focuses on $N(u)$ as well.

\section{DGS Methods Under Study}\label{sec:dgs_methods_under_study}

 
Existing DGS methods focus on optimizing two problems: 1) graph concurrency control, which includes version management and concurrency control protocols to coordinate the execution of concurrent graph queries, and minimizing overhead for each graph operation (i.e., $T_{CC}$ and $\alpha_p$ in Equation \ref{eq:cost}); and 2) graph containers, which include vertex indexes, neighbor indexes, and other optimizations to optimize graph data access $T_p$ for each operation in Equation \ref{eq:cost}. Table \ref{tab:summary_methods} summarizes the key design choices in these methods. In the following, we first briefly introduce these methods and then compare them.

% Please add the following required packages to your document preamble:
% \usepackage{graphicx}
\small
\begin{table}[t]
\centering
\caption{A summary of DGS methods under study.}
\label{tab:summary_methods}
% \resizebox{\columnwidth}{!}{%
\begin{tabular}{|c|clc|ccc|}
\hline
\textbf{}           & \multicolumn{3}{c|}{\textbf{Graph Concurrency Control}}                                                                                                       & \multicolumn{3}{c|}{\textbf{Graph Container}}                                                                                                                                                                                                                     \\ \hline
\textbf{Method}     & \multicolumn{2}{c|}{\textbf{\begin{tabular}[c]{@{}c@{}}Version\\ Management\end{tabular}}}          & \textbf{Protocol}                                       & \multicolumn{1}{c|}{\textbf{\begin{tabular}[c]{@{}c@{}}Vertex\\ Index\end{tabular}}} & \multicolumn{1}{c|}{\textbf{\begin{tabular}[c]{@{}c@{}}Neighbor\\ Index\end{tabular}}} & \textbf{\begin{tabular}[c]{@{}c@{}}Additional\\ Optimization\end{tabular}}          \\ \hline
\textbf{LiveGraph}  & \multicolumn{2}{c|}{\begin{tabular}[c]{@{}c@{}}Fine-Grained with\\ Continuous Version\end{tabular}} & S2PL                                                    & \multicolumn{1}{c|}{\begin{tabular}[c]{@{}c@{}}Dynamic\\ Array\end{tabular}}         & \multicolumn{1}{c|}{\begin{tabular}[c]{@{}c@{}}Dynamic\\ Array\end{tabular}}           & \begin{tabular}[c]{@{}c@{}}Bloom\\ Filter\end{tabular}                            \\ \hline
\textbf{Sortledton} & \multicolumn{2}{c|}{\begin{tabular}[c]{@{}c@{}}Fine-Grained with\\ Version Chain\end{tabular}}      & G2PL                                                    & \multicolumn{1}{c|}{\begin{tabular}[c]{@{}c@{}}Dynamic\\ Array\end{tabular}}         & \multicolumn{1}{c|}{\begin{tabular}[c]{@{}c@{}}Segmented\\ Skip List\end{tabular}}     & \begin{tabular}[c]{@{}c@{}}Adaptive\\ Indexing\end{tabular}                       \\ \hline
\textbf{Teseo}      & \multicolumn{2}{c|}{\begin{tabular}[c]{@{}c@{}}Fine-Grained with\\ Version Chain\end{tabular}}      & OCC                                                     & \multicolumn{1}{c|}{\begin{tabular}[c]{@{}c@{}}Hash\\ Table\end{tabular}}            & \multicolumn{1}{c|}{PMA}                                                               & \begin{tabular}[c]{@{}c@{}}Write-Optimized\\ Segment\end{tabular}                 \\ \hline
\textbf{Aspen}      & \multicolumn{2}{c|}{Coarse-Grained}                                                                 & \begin{tabular}[c]{@{}c@{}}Single\\ Writer\end{tabular} & \multicolumn{1}{c|}{\begin{tabular}[c]{@{}c@{}}AVL\\ Tree\end{tabular}}              & \multicolumn{1}{c|}{\begin{tabular}[c]{@{}c@{}}Segmented\\ PAM\end{tabular}}           & \begin{tabular}[c]{@{}c@{}}Vertex Index\\ Flatten \&\\ Data Encoding\end{tabular} \\ \hline
\textbf{LLAMA}      & \multicolumn{2}{c|}{Coarse-Grained}                                                                 & \begin{tabular}[c]{@{}c@{}}Single\\ Writer\end{tabular} & \multicolumn{1}{c|}{\begin{tabular}[c]{@{}c@{}}Dynamic\\ Array\end{tabular}}         & \multicolumn{1}{c|}{\begin{tabular}[c]{@{}c@{}}Dynamic\\ Array\end{tabular}}           & -                                                                                 \\ \hline
\end{tabular}%
% }
\end{table}


\subsection{A Brief Introduction to DGS Methods}

\subsubsection{\textbf{LiveGraph}~\cite{zhu2019livegraph}}

\textbf{Graph Concurrency Control.} LiveGraph uses a lock for each vertex in $V(G)$ and adapts S2PL to synchronize data access. For a write query $\Delta G$, LiveGraph first obtains all exclusive locks on vertices in $\Delta V$ (vertices involved in $\Delta G$), performs the graph operations, and then releases the locks. To handle deadlocks, LiveGraph aborts $\Delta G$ if it cannot acquire a lock within a time limit.

For each version of a neighbor or vertex, LiveGraph keeps begin and end timestamps ($begin-ts$ and $end-ts$) to record its lifetime as shown in Figure \ref{fig:livegraph_index}. Given $\Delta G$ committed at timestamp $i$, \textsc{InsEdge($u, v$)} searches for the latest version of $v$ in $N(u)$. If found, it sets $end-ts$ to $i$ and creates a new version of $v$ with $begin-ts = i$ and $end-ts = INF$. Otherwise, it directly creates a new version of $v$. \textsc{DelEdge($u, v$)} searches for the latest version of $v$ and sets $end-ts$ to $i$. For read operations in $Q$, LiveGraph obtains a shared lock on the vertex and immediately releases it after the operation. For example, \textsc{ScanNbr($u$)} acquires the lock on $u$, accesses neighbors based on timestamps, and releases the lock immediately. Thus, $Q$ never leads to deadlocks because it never holds two locks simultaneously.

\begin{figure}[h]\small
    \setlength{\abovecaptionskip}{3pt}
    \setlength{\belowcaptionskip}{0pt}
    \includegraphics[scale=0.75]{img/livegraph_neighbor_index.pdf}
    \centering
    \caption{The neighbor index of $N(u_2)$ in LiveGraph.}
    \label{fig:livegraph_index}
\end{figure}

\noindent\textbf{Graph Container.} Given $u \in V(G)$, LiveGraph uses a dynamic array (DA) as the neighbor index of $N(u)$ where each element corresponds to a physical version of $v \in N(u)$. Graph operations in Figure \ref{fig:primitive_opeartions} are based on primitive operators of DA, the time complexity of which is listed in Table \ref{tab:complexity_data_structure}. As the storage of DA is continuous and no version chain exists, \textsc{Scan} is very fast. Moreover, LiveGraph executes \textsc{Scan} from the end to the beginning of DA since the latest element may be more frequently visited than the stale ones. However, \textsc{Search} is slow because DA is unsorted and uses \textsc{Scan} to perform the search. Consequently, \textsc{InsEdge} is slow because it depends on \textsc{SearchEdge} though adding an element only requires a simple append. To mitigate this issue, LiveGraph maintains a Bloom filter~\cite{mitzenmacher2001compressed} for each $N(u)$ to record whether an element exists in $N(u)$. LiveGraph uses DA as the vertex index of $V(G)$ as well. As the vertex ID is ranged in $[0, |V|)$, the element at the index $u$ is the vertex $u$. Therefore, the time complexity of \textsc{Search} on the vertex index is $O(1)$. As the implementation is simple, we omit the details.

\subsubsection{\textbf{Sortledton}~\cite{fuchs2022sortledton}}

\textbf{Graph Concurrency Control.} Sortledton also uses S2PL but optimizes the locking sequence: Sort the vertices in $\Delta V$ by ascending vertex IDs and acquire their exclusive locks in that order. This optimization prevents deadlocks by avoiding circular waiting among write queries~\cite{silberschatz1991operating}. Therefore, Sortledton does not need any mechanisms to handle deadlocks. For \textsc{InsEdge($u, v$)} (or \textsc{DelEdge($u, v$)}) in $\Delta G$ committed at timestamp $i$, Sortledton creates a new version of $v$ with timestamp $i$ and \textsc{op-type} as $I$ (or $D$) as shown in Figure \ref{fig:sortledton_neighbor_index}. Sortledton maintains a version chain for different versions of $v$, where the new version points to the old one. For read queries, Sortledton uses the same concurrency control strategy as LiveGraph.


\begin{figure}[h]\small
    \setlength{\abovecaptionskip}{3pt}
    \setlength{\belowcaptionskip}{0pt}
    \includegraphics[scale=0.75]{img/sortledton_neighbor_index.pdf}
    \centering
    \caption{The neighbor index of $N(u_2)$ in Sortledton.}
    \label{fig:sortledton_neighbor_index}
\end{figure}

\noindent\textbf{Graph Container.} Like LiveGraph, Sortledton uses a dynamic array as the vertex index. For the neighbor index, Sortledton splits $N(u)$ into blocks $B$, and uses the skip list as the block index linking them. The fill ratio of each block is maintained between 50\% and 100\%. When a block is full, Sortledton splits it into two blocks, equally distributing the neighbors. If the fill ratio drops below 50\%, Sortledton merges it with adjacent blocks. The first element of each block serves as its key in the skip list. We call this structure the \emph{segmented skip list}. A read operation traverses the version chain to find the target version. Since real-world graphs are sparse, Sortledton proposes the \emph{adaptive neighbor index}: If $|N(u)|$ is below a threshold (e.g., 256), it uses a sorted dynamic array as the neighbor index instead of the segmented skip list.

\subsubsection{\textbf{Teseo}~\cite{de2021teseo}}

\noindent\textbf{Graph Concurrency Control.} Teseo uses the same version management method as Sortledton but adopts optimistic concurrency control (OCC introduced in Section \ref{sec:preliminaries}) to coordinate concurrent write queries. Unlike LiveGraph and Sortledton, which keep a lock for each vertex, Teseo maintains a lock for each edge partition to synchronize concurrent data access. Specifically, Teseo logically divides $E(G)$ into equally sized partitions, each with a lock. Before accessing data in a partition, a write query acquires an exclusive lock (or a read query acquires a shared lock) on the partition and releases it immediately after access.


\begin{figure}[h]\small
    % \setlength{\abovecaptionskip}{3pt}
    % \setlength{\belowcaptionskip}{0pt}
    \includegraphics[scale=0.75]{img/teseo_neighbor_index.pdf}
    \centering
    \caption{The neighbor index of $N(u_2)$ in Teseo.}
    \label{fig:teseo_neighbor_index}
\end{figure}


\noindent\textbf{Graph Container.} Teseo employs the packed memory array (PMA)~\cite{bender2007adaptive,de2019packed} as the neighbor index as shown in Figure \ref{fig:teseo_neighbor_index}. But it stores all neighbor tables in a single PMA. If $N(u)$ is large, it spans multiple blocks in the PMA, while multiple small neighbor sets can share a single block. However, the global rebalance overhead is high for a PMA if $E(G)$ is stored together. To address this, Teseo divides the single PMA into multiple large leaves (several megabytes each) and indexes these leaves with an \emph{adaptive radix tree} (ART)~\cite{leis2013adaptive}, calling this data structure FAT. Additionally, Teseo uses a hash table as the vertex index to record the position of each vertex's neighbor index in FAT. By default, blocks in FAT are sorted, known as read-optimized segments. When the insertion rate is high, FAT switches to write-optimized segments (WOS), which handle updates by appending to the update log. For WOS, Teseo loops over the segment to find the target vertex. Teseo is built on top of HyPer~\cite{kemper2011hyper}.

\subsubsection{\textbf{Aspen}~\cite{dhulipala2019low}}

\textbf{Graph Concurrency Control.} Aspen uses a coarse-grained strategy that maintains timestamps for each graph snapshot. Specifically, Aspen uses the \emph{single-writer-multiple-reader} scheme, which executes write queries serially and allows multiple readers to execute concurrently. A write query $\Delta G_i$ uses the \emph{copy-on-write} (CoW) method (also called shadow paging) to apply updates on a copy of the graph, creating a new snapshot $G_i$ as shown in Figure \ref{fig:aspen_neighbor_index}. $Q$ works on the latest version of the graph snapshot. Therefore, read and write queries never block each other, and multiple read queries can share the same graph snapshot.

\begin{figure}[h]\small
    \setlength{\abovecaptionskip}{3pt}
    \setlength{\belowcaptionskip}{-5pt}
    \includegraphics[scale=0.75]{img/aspen_neighbor_index.pdf}
    \centering
    \caption{The neighbor index of $N(u_2)$ in Aspen.}
    \label{fig:aspen_neighbor_index}
\end{figure}

\vspace{2pt}
\noindent\textbf{Graph Container.} Aspen partitions $N(u)$ into a set of sorted blocks and uses a parallel augmented map (PAM) to index these blocks. These blocks have no empty slots. \textsc{InsEdge($u, v$)} copies the block, inserts $v$, and then copies the path from the block to the root to create a new snapshot of $N(v)$. Given $N(u)$ and block size $B$, Aspen selects vertices such that $v \mod b = 0$ as heads, i.e., $heads = {v \in N(u) \mid v \mod b = 0}$. This approach ensures that updates to one block do not affect adjacent blocks. Aspen demonstrates that this segmentation ensures each block has $B$ elements with high probability. Moreover, Aspen uses an AVL tree as the vertex index and copies the path in the tree for each update operation.

Aspen proposes two optimization methods to enhance performance. First, for long-running read queries, Aspen can create an array storing the positions of each neighbor table based on the AVL tree to eliminate the overhead of \textsc{Search} in the AVL tree. Second, Aspen uses a difference encoding scheme to compress data in a block. For a block containing $(v0, v1, v2, \ldots)$, Aspen stores it as $(v0, v1 - v0, v2 - v0, \ldots)$ and compresses it with byte codes to accelerate set intersections~\cite{aberger2017emptyheaded}.

\subsubsection{\textbf{LLAMA}~\cite{macko2015llama}}

LLAMA, proposed in 2015, also employs a coarse-grained graph concurrency control mechanism similar to that in Aspen. Specifically, LLAMA divides the vertex table into partitions, each stored in a data page, and maintains an \emph{indirection table} to store the locations of these pages. Each write query must copy the indirection table to create a new graph snapshot, and this overhead limits update performance and graph scalability. 


\subsection{Comparison of DGS Methods} \label{sec:discussion}

\noindent\textbf{Graph Containers.} We discuss the time and space cost of graph operations based on Table \ref{tab:complexity_data_structure}.

\emph{Time.} Given vertex IDs in the range $[0, |V|)$, using a dynamic array (DA) as vertex indexes allows \textsc{SearchVtx} and \textsc{InsVtx} in $O(1)$ time with simple memory accesses. Due to continuous storage, DA enables fast scan operations. In contrast, hash tables and AVL trees incur more overhead than DA, despite having the same time complexity for some operations.

Continuous storage, used in LiveGraph and LLAMA, stores $N(u)$ in DA, facilitating fast \textsc{ScanNbr} but resulting in slow \textsc{SearchEdge} due to the unsorted nature of the array. Since \textsc{InsEdge} depends on \textsc{SearchEdge} in DGS, its performance is also slow despite \textsc{Insert} in DA taking $O(1)$ time. For the segmented methods, scanning accesses data continuously within a block, while inserting typically moves only a few elements within a block. These blocks are linked by an index (e.g., PAM) to accelerate \textsc{SearchEdge}. Therefore, the cost of these operations includes the block index and the block itself. Increasing block sizes can improve scan performance due to continuous memory access but may degrade insert performance due to data movement within the block. Additionally, Aspen, using CoW, incurs more overhead for insertion than methods performing in-place updates because its insert operation copies the block as well as the root-to-leaf path.

\emph{Space.} Practical memory consumption is significantly affected by element size. Each element in Sortledton and Teseo consumes $3 \times w$ bytes, where $w$ is the word size: one for the vertex ID, one for the version, and one for the pointer. The \textsc{op-type} can use the high bit in the timestamp. Each element in LiveGraph also takes $3 \times w$ bytes. In contrast, each element in Aspen consumes only $w$ bytes due to its coarse-grained granularity. Additionally, Aspen’s neighbor index has no empty slots. Therefore, the coarse-grained method is more memory efficient than the fine-grained method.

\vspace{2pt}
\noindent\textbf{Graph Concurrency Control.} We first compare fine-grained and coarse-grained strategies, and then discuss fine-grained methods.

\emph{Fine-Grained vs. Coarse-Grained.} Fine-grained methods require lock operations on each graph operation, which can lead to lock contention and thus expensive $T_{CC}$. High-degree vertices are particularly prone to frequent access. Fine-grained methods necessitate version checks on each element, resulting in increased data loading from memory and more computation for version comparison, leading to a high $\alpha_p$ value. In contrast, coarse-grained methods avoid these issues. However, fine-grained methods allow multiple writers to update the graph simultaneously and perform in-place updates by simply inserting new elements, enhancing update performance.

Additionally, the fine-grained strategy does not place special requirements on the underlying graph containers, making it more generic. In contrast, the coarse-grained strategy requires support for fast snapshot creation. However, since the coarse-grained strategy does not maintain versions or perform version checks for each element, it can be effectively combined with data compression techniques~\cite{dhulipala2019low}, which is not feasible for the fine-grained approach.

\emph{Discussion on Fine-Grained Strategies.} First, the continuous version storage in LiveGraph can improve scanning efficiency by avoiding the traversal of a version chain. However, it may increase data access volume by including stale versions, negatively impacting search and insert efficiency. Second, G2PL in Sortledton is generally the optimal fine-grained concurrency control mechanism due to its effective deadlock avoidance optimization. Although OCC in Teseo does not require holding all mutexes of vertices in $\Delta V$, write queries are typically very short because $\Delta G$ generally contains a single update. Moreover, executing deadlock detection for write-write conflicts introduces significant overhead and implementation challenges. As such, our study uses G2PL for fine-grained methods because of its advantages.

\vspace{2pt}
\noindent\textbf{Empirical Evaluation Targets.} Following the above discussion, we will set up test frameworks and empirically evaluate these techniques by addressing the following five questions.

\begin{itemize} [leftmargin=*]
    \item \textbf{Graph Containers:} \emph{\textbf{Q1.} \sun{How effective are existing techniques in graph containers at efficiently performing key operations such as \textsc{SearchEdge}, \textsc{InsEdge}, and \textsc{ScanNbr}, as defined by $T_p$ in Equation \ref{eq:cost}?}} \emph{\textbf{Q2.} Which neighbor index performs the best, and what is the gap between it and CSR on read queries?}
    \item \textbf{Graph Concurrency Control:} \emph{\textbf{Q3.} What is the impact of graph concurrency control on graph operations?} \emph{\textbf{Q4.} How are the scalability and concurrency of competing methods?}
    \sun{
    \item \textbf{Batch Granularity:} \emph{\textbf{Q5.} How does the batch granularity affect the performance of competing methods?}
    }
    \item \textbf{Memory Consumption:} \emph{\textbf{Q6.} What is the impact of graph containers and version management in DGS on memory consumption, and what is the gap between DGS and CSR?}
\end{itemize}

\begin{figure*}[t]\small
    \setlength{\abovecaptionskip}{0pt}
    % \setlength{\belowcaptionskip}{-13pt}
    \includegraphics[scale=0.45]{img/test_framework.pdf}
    \centering
    \caption{An overview of the test framework.}
    \label{fig:overview_test_framework}
\end{figure*}


\section{Test Framework Setup} \label{sec:test_framework_setup}


Figure \ref{fig:overview_test_framework} provides an overview of this framework. It is implemented with around 7800 lines of optimized C++ code. The source code is compiled using g++ 10.5.0 with the -O3 optimization enabled. Experiments are conducted on a Linux server equipped with an AMD EPYC 7543 CPU (32 cores) and 128GB of memory.



\subsection{Benchmark Platform} \label{sec:benchmark_platform}


\vspace{2pt}
\noindent\textbf{Third-Party Modules.} This module integrates the original implementations of LLAMA~\cite{llamacodebase}, Aspen~\cite{aspencodebase}, LiveGraph~\cite{livegraphcodebase}, Teseo~\cite{teseocodebase}, and Sortledton~\cite{sortledtoncosebase}  from GitHub. Due to differences in their APIs, we implemented a customized adapter, a wrapper for each method, to standardize the evaluation of graph operations. The overhead of these adapters is negligible because they simply combine the interfaces to provide these functions, if not directly available. Each method is configured with its recommended settings. Specifically, the size of bloom filter in LiveGraph is set as $\frac{1}{16}$ of the block size. Sortledton, Teseo and Aspen set the block size to 256.



\vspace{2pt}
\noindent\textbf{DGS Sandbox.} In this module, we re-implement key techniques from competing methods within our abstraction for a fair and detailed investigation of individual techniques. These techniques can be composed differently to evaluate graph operations based on the unified execution routines shown in Figure \ref{fig:primitive_opeartions}. For fine-grained methods, we apply the following optimizations: 1) all methods use G2PL as the concurrency control protocol, and 2) all methods use a dynamic array as the vertex index by default. We implement a simple baseline DGS method by combining the static graph storage AdjLst with G2PL, naming it AdjLst. \sun{Specifically, AdjLst is implemented as an array where each element represents a vertex, and each vertex points to an array storing its neighbor set. When inserting a new element, a binary search is performed to find the correct position, after which the subsequent elements are shifted to make space for the new one. If the array is full, pre-allocated space is used, functioning like a dynamic array.} For comparison purposes, we also include CSR, the optimal baseline for static graphs.





\subsection{Test Driver}

\vspace{2pt}
\noindent\textbf{Workload Generator.} Table \ref{tab:datasets} presents the statistics of the real-world graphs used in the paper. They span six categories, with $|V|$ ranging from tens of thousands to tens of millions and $|E|$ scaling to hundreds of millions.  Both \emph{ldbc}~\cite{ldbcic} and \emph{nft}~\cite{Zhang2023LiveGL, livegraphlab} originally have timestamps to mark the insertion sequence of edges. \emph{ldbc} simulates actions in a social network. We set the scale factor to 10 to control the graph size. \emph{nft} records NFT transactions on Ethereum from 2017 to 2022. Other graphs are obtained from SNAP~\cite{snapnets} and do not include timestamps. These graphs are widely used in DGS research. We also considered larger graphs (e.g., \emph{friendster}) containing billions of edges but omitted them since existing DGS methods frequently run out of memory on these cases.
\small
\begin{table}[h]
% \setlength{\abovecaptionskip}{0pt}
% \setlength{\belowcaptionskip}{0pt}
\captionsetup{skip=0pt} 
\centering
\caption{The detailed information of the real-world graphs.}
\label{tab:datasets}
% \resizebox{0.45\textwidth}{!}{%
\begin{tabular}{|c|c|c|c|c|c|c|}
\hline
\textbf{Category}                & \textbf{Dataset}     & \textbf{Name} & \textbf{|\textit{V}|} & \textbf{|\textit{E}|} & \textit{\textbf{$d_{avg}$}} & $d_{max}$ \\ \hline
\multirow{4}{*}{\textbf{Social}} & \textbf{LiveJournal} & \emph{lj}       & 4.8M                  & 42.8M                 & 8.8                         & 20233     \\ \cline{2-7} 
                   % & \textbf{LDBC}       & \emph{ldbc} & 9.3M  & 52.6M  & 5.6   & 1346287 \\ \cline{2-7} 
                   & \textbf{LDBC}       & \emph{ldbc} & 30.0M  & 175.9M  & 5.9   & 4282595 \\ \cline{2-7} 
                   & \textbf{Twitter}    & \emph{tw}   & 21.3M & 265.0M & 12.4  & 698112 \\ \hline
                   % & \textbf{Orkut}      & \emph{ok}   & 3.1M  & 117.2M & 38.14 & 33313   \\ \hline
\textbf{Game}      & \textbf{DotaLeague} & \emph{dl}   & 0.06M & 50.9M  & 831.6 & 17004   \\ \hline
\textbf{Web}       & \textbf{Wiki}       & \emph{wk}   & 14.0M & 59.0M  & 4.2   & 723404  \\ \hline
\textbf{Citation}  & \textbf{Cit}        & \emph{ct}   & 3.8M  & 16.5M  & 4.4   & 793     \\ \hline
\textbf{Synthetic} & \textbf{Graph500}   & \emph{g5}   & 8.9M  & 260.4M & 29.3  & 406416  \\ \hline
\textbf{Financial} & \textbf{NFT}   & \emph{nft}   & 29.6M  & 77.5M &  2.62 &  2290853 \\ \hline
\end{tabular}%
% }%
\end{table}

We generate three types of graph queries. First, the \emph{micro OP stream} contains a sequence of graph operations. For graphs with timestamps, we generate an \textsc{InsEdge} stream using the first 80\% of edges as the initial graph and the remaining 20\% as the insert edges. For graphs without timestamps, we shuffle the edges and generate the insert stream similarly, following previous works~\cite{zhu2019livegraph,de2021teseo,fuchs2022sortledton}. As these works focus on single updates, each operation corresponds to a write query $\Delta G$. For the \textsc{SearchEdge} stream, we randomly select 20\% of edges as the search targets. For the \textsc{ScanNbr} stream, we select 20\% of vertices based on their degrees. Each of these operations is a read query. The micro OP stream serves two purposes: 1) Investigating the performance of competing methods on basic graph operations, and 2) Studying the effectiveness of short queries (i.e., IC workloads).

Second, we integrate four representative \emph{graph analytic} algorithms from GAPBS~\cite{beamer2015gap}: PR (PageRank), BFS, SSSP, and WCC, which cover different graph data access patterns. PR sequentially accesses both vertices and neighbors. BFS and WCC visit neighbors sequentially while accessing vertices randomly. SSSP introduces a random access pattern to neighbors when retrieving weights. Third, we implement TC (triangle counting) as the representative \emph{graph pattern matching} query. This query requires DGS to support quick scanning in sorted order for fast set intersections. In summary, these two types of queries evaluate the effectiveness of complex, long-running graph queries (i.e., BI and IS workloads).

Real-world graphs following a power-law degree distribution complicate examining the performance of graph operations on different neighbor set sizes because accessing frequently used sets of elements can improve cache performance and lead to biased results. To address this, we design experiments using synthetic datasets. A \emph{synthetic dataset} consists of sets of elements with uniform sizes. Each element is a vertex with an ID ranging from $[0, 2^{22})$. Assuming each vertex ID is 8 bytes, to evaluate performance on a neighbor set with 512 elements, we generate $x = \frac{8 \text{GB}}{512 \times 8 \text{ bytes}}$ sets of the same size. These sets are labeled from $[0, x)$. We generate insert, scan, and search OP streams using the same strategy as for real-world graphs, treating set IDs as vertex IDs and the sets as neighbor sets. We default to using 8GB to prevent all sets from residing in the cache, thereby simulating random memory access in large graphs.



\vspace{2pt}
\noindent\textbf{Workload Executor.} Each thread executes a stream or a graph algorithm. Operations within the same stream execute sequentially by one thread, while multiple threads can execute concurrently. Although the graph algorithms in the test framework can run in parallel, we execute them in a single thread to focus on the DGS's capability to empower concurrent graph query execution and handle different queries.

\vspace{2pt}
\noindent\textbf{Performance Monitor.} We use \emph{throughput}, the number of edges processed (insert, search, scan) per second, to measure DGS efficiency. We use \emph{latency}, the time elapsed to complete one query, to examine service quality. Recording the latency of each operation incurs non-trivial overhead due to the short duration of single graph operations. Therefore, we record latency every one hundred micro-operations. \emph{Scalability} is measured by the throughput of micro-operations as the number of threads increases, while \emph{concurrency} is measured when multiple micro-operation streams (or graph algorithm queries) are mixed. \emph{Memory cost} tracks the memory consumption of competing storage methods.

\begin{table*}[!t]
\centering
\caption{Model performance when no prompt is provided.}
\label{tab:performance_comparison_without_prompt}
\vspace{-0.2\baselineskip} % ACL Only
\resizebox{0.95\linewidth}{!}{%
\begin{tabular}{c|c|c|c|c|c|c|c|c|c}
\hline
% \textbf{Metric} & \textbf{Model} & \textbf{Hellaswag} & \textbf{Arc-c} & \textbf{LogiQA} & \textbf{MMLU} \\ \hline \hline
\textbf{Metric} & \textbf{Model} & \textbf{Hellaswag} & \textbf{PiQA} & \textbf{ARC-e} & \textbf{ARC-c} & \textbf{LogiQA} & \textbf{RACE} & \textbf{SciQ} & \textbf{MMLU} \\ \hline \hline
% Random          & -              & 25/25/26/24\%       & 23/27/26/24\%  & 20/24/28/28\%     & 23/25/25/27\% & & & \\ \hline
\multirow{4}{*}{Random} & label 0 & 25.04\% & 49.51\% & 25.08\% & 22.70\% & 20.08\% & 25.93\% & 0.00\% & 22.95\% \\
                        & label 1 & 24.75\% & 50.49\% & 24.62\% & 26.54\% & 24.42\% & 24.78\% & 0.00\% & 24.65\% \\
                        & label 2 & 25.73\% & - & 26.64\% & 26.45\% & 27.50\% & 25.93\% & 0.00\% & 25.51\% \\
                        & label 3 & 24.48\% & - & 23.61\% & 24.32\% & 27.80\% & 23.35\% & 100\% & 26.89\% \\ \hline
\multirow{3}{*}{Acc} 
                & Mistral-7B & 46.22\% & 71.65\% & 35.06\% & 22.70\% & 19.35\% & 23.92\% & 27.50\% & 22.95\% \\ \cline{2-10}
                & Gemma-7B & 40.79\% & 67.79\% & 33.00\% & 23.72\% & 19.66\% & 25.55\% & 24.60\% & 22.95\% \\ \cline{2-10}
                & LLaMA3.1-8B & 43.24\% & 71.60\% & 35.23\% & 24.06\% & 19.35\% & 24.21\% & 27.50\% & 22.95\% \\ \hline
\multirow{3}{*}{$\text{Acc}_\text{Norm}$} 
                & Mistral-7B & 59.06\% & 72.09\% & 32.45\% & 30.12\% & 24.42\% & 29.79\% & 31.90\% & 22.95\% \\ \cline{2-10}
                & Gemma-7B & 29.37\% & 57.24\% & 27.15\% & 28.24\% & 30.26\% & 29.09\% & 26.10\% & 22.95\% \\ \cline{2-10}
                & LLaMA3.1-8B & 54.74\% & 71.82\% & 33.84\% & 28.75\% & 24.88\% & 29.00\% & 32.30\% & 22.95\% \\ \hline
\multirow{3}{*}{$\text{Acc}_\text{PMI}$}
                & Mistral-7B & 25.04\% & 49.51\% & 25.08\% & 22.70\% & 20.28\% & 25.93\% & 0.00\% & 22.95\% \\ \cline{2-10}
                & Gemma-7B & 25.04\% & 49.51\% & 25.08\% & 22.70\% & 20.28\% & 25.93\% & 0.00\% & 22.95\% \\ \cline{2-10}
                & LLaMA3.1-8B & 25.04\% & 49.51\% & 25.08\% & 22.70\% & 20.28\% & 25.93\% & 0.00\% & 22.95\% \\ \hline
\multirow{3}{*}{$\text{Acc}_\text{ANPMI}$} 
                & Mistral-7B & 25.04\% & 49.51\% & 25.08\% & 22.70\% & 20.28\% & 25.93\% & 0.00\% & 22.95\% \\ \cline{2-10}
                & Gemma-7B & 25.04\% & 49.51\% & 25.08\% & 22.70\% & 20.28\% & 25.93\% & 0.00\% & 22.95\% \\ \cline{2-10}
                & LLaMA3.1-8B & 25.04\% & 49.51\% & 25.08\% & 22.70\% & 20.28\% & 25.93\% & 0.00\% & 22.95\% \\ \hline
\end{tabular}
}
\end{table*}

\begin{table*}[!t]
\centering
\caption{Zero-shot model performance measured with various metrics. Bold numbers represent the best performance for each model and each benchmark.}
\label{tab:performance_comparison}
\vspace{-0.2\baselineskip} % ACL Only
% \setlength{\tabcolsep}{14pt}
\resizebox{0.95\linewidth}{!}{%
\begin{tabular}{c|c|c|c|c|c|c|c|c|c}
\hline
\textbf{Metric} & \textbf{Model} & \textbf{Hellaswag} & \textbf{PiQA} & \textbf{ARC-e} & \textbf{ARC-c} & \textbf{LogiQA} & \textbf{RACE} & \textbf{SciQ} & \textbf{MMLU} \\ \hline \hline
\multirow{3}{*}{Acc} 
         & Mistral-7B & 64.73\% & 81.56\%  & 84.30\% & 57.51\% & 32.72\% & 46.70\% & 96.30\% & 59.72\% \\ \cline{2-10}
         & Gemma-7B & 55.97\% & 76.61\% & \textbf{75.72\%} & 47.53\% & 24.88\% & 41.34\% & 95.40\% & 50.27\% \\ \cline{2-10}
         & LLaMA3.1-8B & 59.05\% & 80.09\% & \textbf{81.78\%} & 51.28\% & 31.64\% & 44.31\% & 96.60\% & 67.70\% \\ \hline
\multirow{3}{*}{$\text{Acc}_\text{Norm}$} 
         & Mistral-7B & \textbf{82.91\%} & \textbf{82.64\%}  & 82.87\% & 58.79\% & 33.79\% & 47.27\% & 94.50\% & 59.72\% \\ \cline{2-10}
         & Gemma-7B & \textbf{73.10\%} & \textbf{77.91\%} & 72.69\% & \textbf{48.81\%} & 29.19\% & \textbf{43.92\%} & 91.80\% & 50.27\% \\ \cline{2-10}
         & LLaMA3.1-8B & \textbf{79.25\%} & \textbf{81.01\%} & 79.55\% & 54.95\% & 31.95\% & 46.70\% & 96.10\% & 67.70\% \\ \hline
\multirow{3}{*}{$\text{Acc}_\text{PMI}$} 
         & Mistral-7B & 69.44\% & 73.56\%  & 80.51\% & 62.54\% & 32.10\% & 47.46\% & 96.00\% & \textbf{60.00\%} \\ \cline{2-10}
         & Gemma-7B & 54.15\% & 66.76\% & 61.20\% & 46.67\% & \textbf{30.41\%} & 40.29\% & 84.50\% & 50.40\% \\ \cline{2-10}
         & LLaMA3.1-8B & 62.33\% & 68.61\% & 68.14\% & 55.38\% & 33.64\% & 44.69\% & 92.20\% & 66.32\% \\ \hline
\multirow{3}{*}{$\text{Acc}_\text{ANPMI}$} 
         & Mistral-7B & 77.67\% & 77.58\%  & \textbf{85.90\%} & \textbf{63.99\%} & \textbf{34.10\%} & \textbf{51.20\%} & \textbf{96.90\%} & 59.91\% \\ \cline{2-10}
         & Gemma-7B & 57.77\% & 76.55\% & 75.34\% & 47.78\% & 25.81\% & 42.11\% & \textbf{95.50\%} & \textbf{50.41\%} \\ \cline{2-10}
         & LLaMA3.1-8B & 73.73\% & 77.69\% & 80.98\% & \textbf{57.85\%} & \textbf{34.25\%} & \textbf{48.13\%} & \textbf{97.40\%} & \textbf{67.79\%} \\ \hline
\end{tabular}
}
\end{table*}

% \begin{table*}[!t]
% \centering
% \label{tab:13b}
% \resizebox{0.86\linewidth}{!}{%
% \begin{tabular}{c|c|c|c|c|c|c|c|c|c}
% \hline
% \textbf{Metric} & \textbf{Model} & \textbf{Hellaswag} & \textbf{PiQA} & \textbf{ARC-e} & \textbf{ARC-c} & \textbf{LogiQA} & \textbf{RACE} & \textbf{SciQ} & \textbf{MMLU} \\ \hline \hline
% {Acc} 
%          & LLaMA2-13B & 60.06\% & 79.11\%  & 79.42\% & 48.46\% & 26.27\% & 40.48\% & 94.60\% & 50.47\% \\  \hline
% {$\text{Acc}_\text{Norm}$} 
%          & LLaMA2-13B & 79.36\% & 80.52\%  & 77.40\% & 49.06\% & 30.88\% & 42.20\% & 93.50\% & 50.47\% \\  \hline
% {$\text{Acc}_\text{PMI}$} 
%          & LLaMA2-13B & 64.37\% & 67.73\%  & 70.33\% & 50.94\% & 32.41\% & 43.44\% & 93.30\% & 48.73\% \\  \hline
% {$\text{Acc}_\text{ANPMI}$}
%          & LLaMA2-13B & 74.53\% & 78.02\%  & 80.51\% & 53.33\% & 31.80\% & 44.69\% & 96.40\% & 49.99\% \\  \hline
% \end{tabular}
% }
% \end{table*}

\section{Experiments}
In this section, we evaluate the performance of the models using ANPMI, while comparing it with the existing metrics. Specifically, we conduct experiments using instruction-tuned language models, such as Mistral-7B(version 0.3)~\citep{jiang2024mixtral}, Gemma-7B~\citep{team2024gemma}, and LLaMA3.1-8B~\citep{dubey2024llama}, along with seven widely used multiple-choice benchmarks~\cite{hellaswag, bisk2020piqa, clark2018think, hendrycks2020measuring, welbl2017crowdsourcing, liu2021logiqa, lai2017race}. We aim to highlight the differences between ANPMI and other popular existing metrics, demonstrating both their benefits and limitations through empirical analysis. The model performance is denoted as \textit{Acc}, \textit{$\text{Acc}_\text{Norm}$}, \textit{$\text{Acc}_\text{PMI}$}, and \textit{$\text{Acc}_\text{ANPMI}$} when measured using {\small $P(Choice|Prompt)$}, length-normalized {\small $P(Choice|Prompt)$}, PMI, and ANPMI. 
\textbf{Random} represents the baseline performance, reflecting the probability of selecting the correct label between labels 0, 1, 2, and 3 by chance, based solely on the label distribution. We exclude NPMI because it is not standard in language models, and it is impossible to compute {\small $P(Choice, Prompt)$} if {\small $Prompt+Choice$}  is larger than the maximum sequence length.

\subsection{Performance When No Prompt Provided}
To verify that ANPMI can evaluate performance while reducing bias from differences in {\small $P(Choice)$}, we measure the language model performance on the various benchmarks without providing prompts. The results of these evaluations are summarized in Table~\ref{tab:performance_comparison_without_prompt}. 

For MMLU, we observe identical performance across all models, regardless of the metric used.  This is because the same four choices — A, B, C, and D — are given throughout examples.  However, for other datasets, such as Hellaswag and ARC, which have a set of different answer choices for each example, model performance varies when evaluated using Acc or $\text{Acc}_\text{Norm}$.  For each benchmark, we observe a difference of up to 30\% in performance between models when evaluated using these metrics.  Section~\ref{sec:impact_of_prior_prob} demonstrates that variations in {\small $P(Choice)$} significantly influence a model's final decisions.  Thus, these performance differences observed without prompts, which highlight the impact of prior probabilities, may complicate accurately ranking models.  Moreover, the measured performance for Hellaswag, PiQA, and ARC-easy is significantly higher than that of random guessing.  This indicates that when using Acc or $\text{Acc}_\text{Norm}$, models may achieve high scores on these benchmarks without understanding the prompts, complicating the evaluation of their language comprehension capability.

In contrast, PMI and ANPMI have identical performance across all models when prompts are absent. These metrics always assign a zero value when prompts are not provided, resulting in consistent performance measurements by always choosing the same choice. Consequently, PMI and ANPMI effectively mitigate the influence of {\small $P(Choice)$} on performance, making them reliable metrics for accurately assessing a model's understanding of prompts to answer questions.

\begin{table}[!t]
\centering
\caption{The proportion of choices selected in the MMLU task based on PMI and ANPMI metrics for the LLaMA3.1-8B model.}
\label{tab:pmi_and_anpmi}
% \vspace{-0.5\baselineskip}
\resizebox{0.95\linewidth}{!}{%
\begin{tabular}{c|c|c|c|c}
\hline
\multirow{2}{*}{} & \multicolumn{4}{|c}{Choices} \\
\cline{2-5}
 & A & B & C & D \\
\hline \hline
{\small $\log(P(Choice))$} & -9.14 & -10.08 & -10.27 & -9.95 \\
\hline
PMI & 12.36\% & 31.65\% & 33.31\% & 22.69\% \\
\hline
ANPMI & 18.01\% & 30.10\% & 29.66\% & 23.23\% \\
\hline
\end{tabular}
}
\end{table}

\subsection{Comparison of the Metrics}
The results of evaluating the model performance using various metrics are summarized in Table~\ref{tab:performance_comparison}. {Some examples where ANPMI differs from other metrics can be found in Appendix\mbox{~\ref{sec:comp_example}}.} The experiments are conducted using the Language Model Evaluation Harness~\cite{eval-harness} under a zero-shot setting.  

Benchmarks where the final decision of the model depends heavily on {\small $P(Choice)$} show a larger performance gap when measured using metrics other than {\small $P(Choice|Prompt)$}. 
For instance, when evaluating HellaSwag using LLaMA3.1-8B, about 65\% of decisions are influenced by the differences in {\small $P(Choice)$} as seen in Table~\ref{tab:comp_results}, resulting in a 14.68\% performance gap between Acc and $\text{Acc}_\text{ANMPI}$. Conversely, in MMLU, where only 13\% to 16\% of decisions of each model depend on the {\small $P(Choice)$} difference according to Table~\ref{tab:comp_results}, the maximum performance discrepancy is merely up to 0.19\% comparing Acc and $\text{Acc}_\text{ANPMI}$.

The difference between Length-normalized {\small $\log P(Choice|Prompt)$} and ANPMI can be observed on MMLU. Since all choices in MMLU have the same length in characters (1 char), $\text{Acc}_\text{Norm}$ is identical to Acc, with no performance change occurring due to length normalization. In contrast, ANPMI theoretically addresses the impact of the imbalance in {\small $P(Choice)$} on model performance measurement. As a result, differences between Acc and $\text{Acc}_\text{ANPMI}$ are consistently observed across all models.

The difference in model performance measured by PMI and ANPMI is caused by the fact that PMI does not perform any normalization. Table~\ref{tab:pmi_and_anpmi} shows how the lack of normalization affects the model's final choices in MMLU. PMI tends to assign smaller maximum values to choices with higher {\small $\log P(Choice)$}, making the model less likely to select options with large {\small $P(Choice)$} values. As demonstrated in Table~\ref{tab:pmi_and_anpmi}, under PMI, choice A (A has the highest {\small $\log P(Choice)$}) is the least frequently chosen, whereas choice C (C has the lowest {\small $\log P(Choice)$}) is the most frequently chosen. In contrast, this tendency is less evident when using ANPMI.

The experimental results indicate that when model performance is evaluated using a metric that fails to account for the {\small $P(Choice)$} imbalance, the model's performance does not accurately reflect its natural language understanding capability. As a result, ANPMI, which theoretically addresses the {\small $P(Choice)$} imbalance, is identified as the most appropriate metric for assessing a language model's natural language understanding capability.
\section{Related Work}
LLM unlearning~\citep{jang2023knowledgeunlearning, yao2023llmunlearningsurvey, lynch2024eight} has gained significant attention as a method to enhance privacy. Various approaches~\citep{sinha2024unstar, zhang2024npo} have been proposed to ensure that models effectively erase specific information while maintaining overall performance. A key challenge in unlearning is assessing whether knowledge unrelated to the forget set is inadvertently affected. To evaluate this, researchers commonly examine general knowledge~\citep{hendrycks2021measuring, cobbe2021training} as well as a designated subset of the retain set that shares a similar distribution with the forget set but excludes the targeted information. These subsets, often referred to as neighbor sets~\citep{closerlookat}, help determine the extent of unintended degradation in model performance.

In hazardous knowledge unlearning, prior work has leveraged domain-relevant general knowledge as a benchmark. For instance,~\citet{li2024wmdp} employs general biology knowledge to assess the impact of bioweapon-related unlearning and general computer security knowledge to evaluate the removal of information related to Attacking Critical Infrastructure. For entity unlearning~\citep{maini2024tofu, rwku}, previous studies have used entities from similar professions or those closely linked to the target entity as neighbor sets. While these approaches provide an initial framework, they lack a systematic investigation of which aspects of the retain set suffer the most from unlearning. Our study addresses this gap by systematically investigating the impact of unlearning on different types of neighbor sets more clearly and identifying which knowledge components experience the highest degree of forgetting.
\section{Summary and Conclusion}
\label{sec:conclusion}


In this paper, we introduced \ToolName{}, a method for discovering fine-grained \emph{sub-activities} from unlabeled smart home sensor data without relying on pre-segmentation. Our pipeline is organized into two core steps: Clustering and Labeling. 
The \textbf{Clustering step} consists of:

\begin{itemize}
    \item \textbf{Encoder Pre-Training:} We leverage a pre-trained BERT model adapted with sensor-specific tokens and train it using a masked language modeling (MLM) objective to generate context-rich embeddings for raw sensor sequences.
    
    \item \textbf{Clustering Model Fine-Tuning:} Using the SCAN loss function, we fine-tune these embeddings to form more homogeneous and distinct clusters of sensor sequences.
\end{itemize}

The \textbf{Labeling step} comprises:

\begin{itemize}
    \item \textbf{Cluster Centroid Annotation:} Representative sequences from each cluster are visualized with a custom tool, enabling expert annotators to assign meaningful sub-activity labels to the centroids.
    
    \item \textbf{Label Propagation:} The centroid labels are propagated to all sequences within their respective clusters, resulting in a fully labeled dataset with minimal manual effort.
    
    \item \textbf{Re-annotation of Original Time-Series Data:} 
    Finally, these propagated labels are mapped back onto the original time-series data, preserving temporal continuity and facilitating the analysis of longitudinal activity patterns.
\end{itemize}


Our approach addresses important challenges in HAR, including the high cost and effort of manual data annotation, the limitations of coarse activity labels, and the need for scalable and generalizable models. \ToolName{} offers an open source tool that facilitates the HAR annotation and re-annotation process and enables the dynamic discovery and validation of sub-activities, thus capturing a broader spectrum of behaviors observed in real homes.


\begin{acks}
Jixian Su and Chiyu Hao contributed equally to this work. Shixuan Sun is the corresponding author.
\end{acks}
\bibliographystyle{ACM-Reference-Format}
\bibliography{reference}

\appendix
\begin{table}[t!]
  \centering
  


% \renewcommand{\arraystretch}{1.2} % 调整行高
% \setlength{\tabcolsep}{10pt}  % 调整列间距
\resizebox{0.48\textwidth}{!}{%
\begin{tabular}{lcrr}
        \toprule
        \textbf{Dataset} & \textbf{Full Size*} & \textbf{Consistency}  & \textbf{\dataset{}} \\
        \midrule
        HotpotQA  & 5,901 & 2,973 {\footnotesize \textcolor{gray}{(50\%)}}  & 1,476 {\footnotesize \textcolor{gray}{(25\%)}}  \\
        NewsQA    & 4,212 & 1,260 {\footnotesize \textcolor{gray}{(30\%)}} & 934  {\footnotesize \textcolor{gray}{(22\%)}}  \\
        NQ        & 7,314 & 4,419 {\footnotesize \textcolor{gray}{(60\%)}}  & 1,479 {\footnotesize \textcolor{gray}{(20\%)}}  \\
        SearchQA  & 16,980 & 12,133 {\footnotesize \textcolor{gray}{(71\%)}} & 1,497 {\footnotesize \textcolor{gray}{(9\%)}}  \\
        SQuAD     & 10,490 & 5,024 {\footnotesize \textcolor{gray}{(48\%)}}  & 2,351 {\footnotesize \textcolor{gray}{(22\%)}}  \\
        TriviaQA  & 7,785 & 6654 {\footnotesize \textcolor{gray}{(85\%)}}  & 792  {\footnotesize \textcolor{gray}{(10\%)}}  \\
        \bottomrule
    \end{tabular}
}




 \caption{Number of instances at each stage in the \dataset{} construction pipeline.}
 \label{tab:our_bench_stats_each_step}
\end{table}
\section{Appendix}
\subsection{License}
We present the licenses of the datasets used in this study: Natural Questions (CC BY-SA 3.0 license), NewsQA (MIT License), SearchQA and TriviaQA (Apache License 2.0), HotpotQA and SQuAD (CC BY-SA 4.0 license).

All these licenses and agreements permit the use of their data for academic purposes.

\subsection{Details of Data Constructing}
\label{append:prompts}
In this section, we detail the two main steps in constructing \dataset{}. The dataset sizes at each stage of the pipeline are shown in Table~\ref{tab:our_bench_stats_each_step}.


\textbf{Parametric Knowledge Elicitation.} First, we elicit the LLM's parametric knowledge by prompting it in a closed-book setting (i.e., without any context). To ensure the reliability of the elicited knowledge, we apply a consistency-based filtering method. Specifically, for each query, the LLM is prompted five times, and the frequency of each response is recorded. The response with the highest frequency is identified as the majority answer. Queries where the majority answer appears fewer than three times are discarded, in order to filter out inconsistent responses and enhance data quality. The following prompt is used to instruct the LLM:
\begin{tcolorbox}
[title=Prompt for eliciting parametric knowledge,colback=blue!10,colframe=blue!50!black,arc=1mm,boxrule=1pt,left=1mm,right=1mm,top=1mm,bottom=1mm]
Answer the question \textcolor{blue}{\{\textit{brevity\_instruction}\}} and provide supporting evidence.

Question: \textcolor{blue}{\{\textit{question}\}}
\end{tcolorbox}
\noindent The ``\textit{brevity\_instruction}'' is used to guide the LLM to generate responses in a more concise form.

\textbf{Conflict Data Selection.} Next, we filter the data to retain only instances where the LLM's parametric knowledge directly conflicts with the contextual answer. Specifically, we categorize the data obtained from the previous step into two groups, conflicting and non-conflicting instances, based on the detailed results of conflict detection. All non-conflicting instances are discarded. GPT-4o-mini is then used to detect the presence of a conflict, using the following prompt:

\begin{tcolorbox}
[title=Prompt for identifying conflict knowledge,colback=blue!10,colframe=blue!50!black,arc=1mm,boxrule=1pt,left=1mm,right=1mm,top=1mm,bottom=1mm]
\small
You are tasked with evaluating the correctness of a model-generated answer based on the given information. 

\small
Context: \textcolor{blue}{\{\textit{context}\}}

Question: \textcolor{blue}{\{\textit{question}\}}

Contextual Answer: \textcolor{blue}{\{\textit{contextual\_answer}\}}

Model-Generated Answer: \textcolor{blue}{\{\textit{Model-Generated\_answer}\}}

\textcolor{blue}{[\textit{Detailed task description...}]}

Output Format:

Evaluate result: (Correct / Partially Correct / Incorrect) 
\end{tcolorbox}




\subsection{Assessing the Reliability of GPT-4o-mini in Knowledge Conflict Identification}
\label{append:human_eval}
In this subsection, we conduct the human evaluation to assess the reliability of GPT-4o-mini in identifying knowledge conflicts, which is a critical task in our data construction process to guarantee the data quality.

We randomly sampled 100 examples from each of the six subsets of \dataset{}, yielding a total of 600 samples. Six senior computational linguistics researchers were then asked to evaluate whether a knowledge conflict was present in each example. For each instance, the evaluators were provided with the question, the contextual answer, the model-generated response, and the corresponding supporting evidence. The results were classified into three categories: No Conflict, Somewhat Conflict, and High Conflict. The detailed annotation instructions are as follows:

\begin{tcolorbox}
[title=Annotation Instruction,colback=blue!10,colframe=blue!50!black,arc=1mm,boxrule=1pt,left=1mm,right=1mm,top=1mm,bottom=1mm]
\small
You are tasked with determining whether the parametric knowledge of LLMs conflicts with the given context to facilitate the study of knowledge conflicts in large language models.

Each data instance contains the following fields: 

Question: \textcolor{blue}{\{\textit{question}\}}


Answers: \textcolor{blue}{\{\textit{answers}\}}


Context: \textcolor{blue}{\{\textit{context}\}}

Parametric\_knowledge: \textcolor{blue}{\{\textit{LLMs' parametric\_knowledge }\}} 

The annotation process consists of two steps. 

\textbf{Step 1}: Compare the model-generated answer with the ground truth answers, based on the given question and context, to determine whether the model’s parametric knowledge conflicts with the context.

\textbf{Step 2}: Classify the results into one of three categories: 

\textcolor{blue}{\{\textit{No Conflict}\}} if the model-generated answer is consistent with the ground truth answers and context, 

\textcolor{blue}{\{\textit{Somewhat Conflict}\}}  if it is partially inconsistent

\textcolor{blue}{\{\textit{High Conflict}\}} if it significantly contradicts the ground truth answers or context.
\end{tcolorbox}


The evaluation results, shown in Table~\ref{tab:append_human_eval}, reveal a high level of agreement between the human annotators and GPT-4o-mini. Over 85\% of the examples reach consensus among the annotators, with an average agreement rate of 85.6\% across all subsets. These findings underscore the reliability of GPT-4o-mini as an effective tool for identifying knowledge conflicts.




\begin{table}[t]
  \centering
  
\centering
\begin{tabular}{l c}
\toprule
\textbf{Subset} & \textbf{Agreement (\%)} \\ \midrule
HotpotQA        & 81.4                        \\
NewsQA          & 72.7                        \\
NQ              & 88.7                        \\
SearchQA        & 95.3                        \\
SQuAD           & 86.1                        \\
TriviaQA        & 90.7                        \\ \midrule
\textbf{Average} & \textbf{85.6}            \\ \bottomrule
\end{tabular}

 \caption{Agreement between human annotators and GPT-4o-mini across different subsets of our \dataset{} benchmark.}
 \label{tab:append_human_eval}
\end{table}



\subsection{Evaluating the Effectiveness of Our Consistency-Based Filtering Method}
\label{append:data_freq}

In this subsection, we evaluate the effectiveness of our consistency-based knowledge conflict filtering method. As described in Appendix~\ref{append:prompts}, for each query, we prompt the model five times and record the most frequently generated answer along with its occurrence frequency. Based on this frequency, we divide the data into sub-datasets, where all queries within each sub-dataset share the same answer frequency. We then apply ``Conflict Data Selection'' to each sub-dataset, retaining only instances where knowledge conflicts occur. Finally, we evaluate ConR and MemR on these sub-datasets.

As shown in Figure~\ref{fig:diff_freq}, a clear trend emerges: as answer frequency increases, ConR consistently decreases, while MemR increases. This pattern indicates that as answer frequency rises, the model becomes increasingly reliant on its internal knowledge. Notably, for data with an answer frequency of 1, MemR is only 3\%, indicating minimal dependence on internal knowledge. Retaining only high-answer-frequency data improves the quality of \dataset{}. This data construction approach distinguishes our methodology from previous studies~\cite{longpre2021entity,xie2023adaptive}.

\begin{figure}[t!]
  \centering
  \includegraphics[width=0.4\textwidth]{figs/diff_freq.pdf}
  \caption{Performance comparison of ConR and MemR across sub-datasets grouped by the answer frequency of LLMs.}
  \label{fig:diff_freq}
\end{figure}





\subsection{Additional Implementation Details of Our Experiments}
\label{append:implementation}
This subsection outlines the training prompt, describes more details of the training data, and provides details of the experimental setup used in our experiments.

\textbf{Training Prompts.}
We adopt a simple QA-format training prompt following~\citet{zhou2023context} for all methods except \attrprompt{} and \oiprompt{}.
\begin{tcolorbox}
[title=Base Prompt ,colback=blue!10,colframe=blue!50!black,arc=1mm,boxrule=1pt,left=1mm,right=1mm,top=1mm,bottom=1mm]
% \small
\textcolor{blue}{\{\textit{context}\}} 
Q: \textcolor{blue}{\{\textit{question}\}} ? 
A: \textcolor{blue}{\{\textit{answer}\}}.
\end{tcolorbox}


\textbf{Training Datasets.} During \method{}, we randomly sample 32,580 instances from the training set of the MRQA 2019 benchmark~\cite{fisch2019mrqa} to construct our training data.



\textbf{Experimental Setup.} In this work, all models are trained for 2,100 steps with a total batch size of 32 and a learning rate of 1e-4. To enhance training efficiency, we implemented \method{} with LoRA~\cite{hu2021lora}, setting both the rank $\text{r}$ and scaling factor $\text{alpha}$ to 64. For \method{}, we set $\alpha$ to 0.1 (Eq.~\ref{eq:selct_layers}), which determines the minimum activation ratio difference required for a layer to be pruned. Additionally, we adopt a dynamic $\gamma$ in $\mathcal{L}_{\text{KC}}$ (Eq.~\ref{eq:kc_loss}), which linearly transitions from an initial margin ($\gamma_{0}=1$) to a final margin ($\gamma^*=5$) as training progresses. This adaptive strategy gradually reduces the model's reliance on internal parametric knowledge, encouraging it to rely more on external knowledge provided by the KAG system.


\subsection{Implementation Details of Baselines}
\label{append:baseline}
This subsection describes the implementation details of all baseline methods.

We adopt two prompt-based baselines: the attributed prompt ($\text{Attr}_{\text{prompt}}$) and a combination of opinion-based and instruction-based prompts ($\text{O\&I}_{\text{prompt}}$). The corresponding prompt templates are as follows:

\begin{tcolorbox}
[title=Attr based prompt ,colback=blue!10,colframe=blue!50!black,arc=1mm,boxrule=1pt,left=1mm,right=1mm,top=1mm,bottom=1mm]
% \small
\textcolor{blue}{\{\textit{context}\}} Q: \textcolor{blue}{\{\textit{question}\}} based on the given text? A: \textcolor{blue}{\{\textit{answer}\}}.
\end{tcolorbox}

\begin{tcolorbox}
[title=O\&I based prompt ,colback=blue!10,colframe=blue!50!black,arc=1mm,boxrule=1pt,left=1mm,right=1mm,top=1mm,bottom=1mm]

Bob said ``\textcolor{blue}{\{\textit{context}\}}'' Q: \textcolor{blue}{\{\textit{question}\}} in Bob's opinion? A: \textcolor{blue}{\{\textit{answer}\}}.
\end{tcolorbox}
For the SFT baseline, we incorporate context during training, similar to \method{}, while keeping the remaining experimental settings identical. To construct preference pairs for DPO training, we use contextually aligned answers from the dataset as ``preferred responses'' to ensure the consistency with the provided context. The ``rejected responses'' are generated by identifying parametric knowledge conflicts through our data construction methodology (Sec.~\ref{sec:benchmark}).

For KAFT, we employ a hybrid dataset containing both counterfactual and factual data. Specifically, we integrate the counterfactual data developed by \citet{xie2023adaptive}, leveraging their advanced data construction framework.

By maintaining equivalent dataset sizes and ensuring comparable data quality across all baselines, we provide a rigorous and fair comparison with our proposed \method{}.




\subsection{Extending \method{} to More LLMs}
\label{append:diff_model_performance}


\begin{figure}[t!]
  \centering
  
\subfigure[ConR Results]{
        \label{fig:diff_model:llama_conr}
        \includegraphics[width=0.462\linewidth]{append_fig/llama_conr.pdf}
    }
    \hspace{0.0005\linewidth} 
    \subfigure[MemR Results]{
        \label{fig:diff_model:llama_memr}
        \includegraphics[width=0.462\linewidth]{append_fig/llama_memr.pdf}
    }


  % \includegraphics[width=0.48\textwidth]{figs/diff_model_double.pdf}
 \caption{Average ConR and MemR across different models implemented by LLMs of LLaMA series, before and after applying \method{}.
 }
 \label{fig:diff_model_double_llama}
\end{figure}

\begin{figure}[t]
  \centering
  \subfigure[ConR Results]{
        \label{fig:diff_model:qwen_conr}
        \includegraphics[width=0.462\linewidth]{append_fig/qwen_conr.pdf}
    }
    \hspace{0.0005\linewidth} 
    \subfigure[MemR Results]{
        \label{fig:diff_model:qwen_memr}
        \includegraphics[width=0.462\linewidth]{append_fig/qwen_memr.pdf}
    }
  % \includegraphics[width=0.48\textwidth]{figs/diff_model_double.pdf}
 \caption{Average ConR and MemR across different models implemented by LLMs of Qwen series, before and after applying \method{}.
 }
 \label{fig:diff_model_double_qwen}
\end{figure}






We extend \method{} to a diverse range of LLMs, encompassing multiple model families and sizes. 

Specifically, our evaluation includes LLaMA3-8B-Instruct, LLaMA3.2-1B-Instruct, LLaMA3.2-3B-Instruct, Qwen2.5-0.5B-Instruct, Qwen2.5-1.5B-Instruct, Qwen2.5-3B-Instruct, Qwen2.5-7B-Instruct, and Qwen2.5-14B-Instruct. The results on ConR and MemR are summarized in Figures~\ref{fig:diff_model_double_llama} and \ref{fig:diff_model_double_qwen}, while Table~\ref{tab:append:all_model_res} presents the average performance of all models on \dataset{} and ConFiQA. Additionally, Table~\ref{tab:diff_model_param} provides detailed parameter information and specifies the layers selected for pruning for each model. This comprehensive evaluation demonstrates the versatility and scalability of \method{} across a wide spectrum of model architectures and sizes.

\begin{table}[!t]
  
    \resizebox{0.48\textwidth}{!}{%
\begin{tabular}{l|c|c|c}
\toprule
\textbf{Models}     & \textbf{Param.} & \textbf{\method{} Param.} & \textbf{Selected Layers} \\
\midrule
\rowcolor{gray!10}
LLaMA3.2-1B        & 1.24B  & 1.08B \small\textcolor{gray}{(87\%)}   & [12, 14]                 \\
LLaMA3.2-3B        & 3.21B  & 2.60B \small\textcolor{gray}{(81\%)}   &  [18, 25]   \\
\rowcolor{gray!10}
LLaMA3-8B          & 8.03B  & 6.97B \small\textcolor{gray}{(87\%)}   & [24, 29]      \\
LLaMA3.1-8B          & 8.03B  & 6.27B \small\textcolor{gray}{(78\%)}   & [20, 29]      \\
\rowcolor{gray!10}
Qwen2.5-0.5B         & 0.49B  & 0.44B \small\textcolor{gray}{(90\%)}   &  [19, 22]       \\
Qwen2.5-1.5B         & 1.54B  & 1.34B \small\textcolor{gray}{(87\%)}   & [21, 25]        \\
\rowcolor{gray!10}
Qwen2.5-3B         & 3.09B  & 2.68B \small\textcolor{gray}{(87\%)}   & [29, 34]        \\
Qwen2.5-7B         & 7.61B  & 7.21B \small\textcolor{gray}{(95\%)}   &   [25, 26 ]     \\
\rowcolor{gray!10}
Qwen2.5-14B        & 14.70B & 12.43B \small\textcolor{gray}{(85\%)}  &  [35, 45]   \\
\bottomrule
\end{tabular}
}

% \end{sidewaystable}

% \end{document}

  \caption{The total number of parameters for various models before and after applying \method{}. \textcolor{gray}{\small$(\cdot)\%$} represents the proportion relative to the original model, and the last column lists the layers selected for pruning.}
   \label{tab:diff_model_param}
\end{table}

These experimental results illustrate several key insights: 1) Larger models tend to rely more on parametric memory. As model size increases in both the LLaMA and Qwen families, MemR also grows, indicating a tendency to overlook external knowledge in favor of internal parameters. \method{} counteracts this behavior, decreasing larger models' MemR score to even below that of smaller models. 2) \method{} consistently benefits all evaluated models. Across both LLaMA and Qwen model families, \method{} outperforms Vanilla-KAG by boosting accuracy and context faithfulness, underscoring its broad applicability and effectiveness. 3) Not all parameters in KAG models are essential. Pruning parametric knowledge not only reduces computation costs but also fosters better generalization without sacrificing accuracy, highlighting the potential of building a parameter-efficient LLM within the KAG framework.




\begin{table*}[!t]
  
\centering
\resizebox{0.96\textwidth}{!}{%
\begin{tabular}{l|c|cccc|cccc}
\toprule
\multirow{2}{*}{\textbf{Models}} & \multirow{2}{*}{\textbf{Param.}} & \multicolumn{4}{c|}{\textbf{\dataset{}}} & \multicolumn{4}{c}{\textbf{ConFiQA}} \\ 
\cmidrule(lr){3-6}  \cmidrule(lr){7-10}
 &  & ConR $\uparrow$ & MemR $\downarrow$ & MR $\downarrow$ & EM $\uparrow$ & ConR $\uparrow$ & MemR $\downarrow$ & MR $\downarrow$ & EM $\uparrow$ \\ 
\midrule
LLaMA3-8B   & 8.03B  & 66.99  & 11.75  & 14.99  & 13.83  & 22.52  & 31.15  & 59.77  & 2.47 \\
\rowcolor{gray!10}
+\method{}    & 6.97B  & 71.50  & 6.48   & 8.41   & 66.19  & 70.43  & 8.82   & 11.32  & 67.29 \\
LLaMA3.1-8B & 8.03B  & 63.15  & 11.69  & 15.93  & 21.85  & 15.38  & 29.97  & 68.98  & 6.69 \\
\rowcolor{gray!10}
+\method{}   & 6.27B  & 70.41  & 6.95   & 9.17   & 63.58  & 71.12  & 9.01   & 11.44  & 66.61 \\
LLaMA3.2-1B & 1.24B  & 39.06  & 10.49  & 21.83  & 5.13   & 32.09  & 18.32  & 36.28  & 7.15 \\
\rowcolor{gray!10}
+\method{}   & 1.08B  & 51.75  & 6.51   & 11.34  & 47.60  & 62.70  & 7.63   & 11.38  & 61.85 \\
LLaMA3.2-3B & 3.21B  & 56.75  & 11.53  & 17.11  & 12.69  & 26.16  & 23.47  & 49.05  & 9.84 \\
\rowcolor{gray!10}
+\method{}   & 2.60B  & 67.00  & 6.80   & 9.35   & 61.59  & 69.61  & 8.39   & 11.09  & 66.53 \\
Qwen2.5-0.5B & 0.49B  & 47.17  & 11.36  & 19.48  & 2.06   & 50.72  & 17.15  & 26.20  & 3.78 \\
\rowcolor{gray!10}
+\method{}   & 0.44B  & 58.13  & 6.63   & 10.41  & 52.56  & 67.54  & 8.04   & 11.03  & 66.33 \\
Qwen2.5-1.5B & 1.54B  & 58.08  & 11.28  & 16.48  & 10.30  & 51.69  & 19.87  & 28.23  & 10.78 \\
\rowcolor{gray!10}
+\method{}   & 1.34B  & 63.78  & 6.74   & 9.76   & 57.67  & 69.61   & 8.35   & 11.05   & 66.04 \\
Qwen2.5-3B   & 3.09B  & 62.22  & 14.45  & 18.88  & 0.10   & 25.47  & 29.34  & 55.70  & 0.01 \\
\rowcolor{gray!10}
+\method{}     & 2.68B  & 66.31  & 6.75   & 9.38   & 59.42  & 66.30   & 8.62  & 11.94   & 63.03 \\
Qwen2.5-7B    & 7.61B  & 65.46  & 14.93  & 18.57  & 0.80   & 24.75  & 33.09  & 59.04  & 0.10 \\
\rowcolor{gray!10}
+\method{}      & 6.60B  & 67.75  & 6.60   & 9.01   & 61.77  & 69.54  & 8.85   & 11.58  & 66.68 \\
Qwen2.5-14B   & 14.70B & 65.75  & 16.13  & 19.75  & 0.00   & 7.86   & 32.88  & 83.71  & 0.01 \\
\rowcolor{gray!10}
+\method{}     & 12.43B & 70.01  & 6.43   & 8.55   & 64.43  & 71.70  & 8.90   & 11.29  & 68.40 \\
\bottomrule
\end{tabular}%
}


  \caption{Average performance of LLMs on \dataset{} and ConFiQA before and after applying \method{}.}
   \label{tab:append:all_model_res}
\end{table*}

\subsection{Neuron Activations in Different LLMs}\label{app:activation}
We present the neuron activations for the LLaMA family models, including LLaMA-3.2-1B-Instruct, LLaMA-3.2-3B-Instruct, LLaMA-3-8B-Instruct, and LLaMA-3.1-8B-Instruct, as well as the Qwen family models, including Qwen-2.5-0.5B-Instruct, Qwen-2.5-1.5B-Instruct, Qwen-2.5-3B-Instruct, Qwen-2.5-7B-Instruct, and Qwen-2.5-14B-Instruct, in Figures~\ref{fig:act_llama} and \ref{fig:act_qwen}, respectively. 
% 我们发现qwen系列模型


\begin{figure*}[t]
  \centering
  \subfigure[Neuron activations of LLaMA-3.2-1B-Instruct]{
        \label{fig:act_llama:3.2-1b}
        \includegraphics[width=0.9\linewidth]{append_fig/act_llama32_1b_all.pdf}
    }
\subfigure[Neuron activations of LLaMA-3.2-3B-Instruct]{
        \label{fig:act_llama:3.2-3b}
        \includegraphics[width=0.9\linewidth]{append_fig/act_llama32_3b_all.pdf}
    }
 \subfigure[Neuron activations of LLaMA-3-8B-Instruct]{
        \label{fig:act_llama:3-8b}
        \includegraphics[width=0.9\linewidth]{append_fig/act_llama_3_8b.pdf}
    }
 \subfigure[Neuron activations of LLaMA-3.1-8B-Instruct]{
        \label{fig:act_llama:3.1-8b}
        \includegraphics[width=0.9\linewidth]{append_fig/act_llama_31_8b.pdf}
    }
 

 \caption{Neuron activations across different layers of the LLaMA series models. We present the inhibition ratio $\Delta R$ under two conditions: with contextual knowledge input (w/ context) and without it (w/o context).}
 \label{fig:act_llama}
\end{figure*}

\begin{figure*}[t]
  \centering
  \subfigure[Neuron activations of Qwen-2.5-0.5B-Instruct]{
        \label{fig:act_qwen:2.5-0.5b}
        \includegraphics[width=0.75\linewidth]{append_fig/act_qwen25_0_5b_all.pdf}
    }
\subfigure[Neuron activations of Qwen-2.5-1.5B-Instruct]{
        \label{fig:act_qwen:2.5-1.5b}
        \includegraphics[width=0.75\linewidth]{append_fig/act_qwen25_1_5b_all.pdf}
    }
\subfigure[Neuron activations of Qwen-2.5-3B-Instruct]{
        \label{fig:act_qwen:2.5-3b}
        \includegraphics[width=0.75\linewidth]{append_fig/act_qwen25_3b_all.pdf}
    }
\subfigure[Neuron activations of Qwen-2.5-7B-Instruct]{
        \label{fig:act_qwen:2.5-7b}
        \includegraphics[width=0.75\linewidth]{append_fig/act_qwen25_7b_all.pdf}
    }
\subfigure[Neuron activations of Qwen-2.5-14B-Instruct]{
        \label{fig:act_qwen:2.5-14b}
        \includegraphics[width=0.75\linewidth]{append_fig/act_qwen25_14b_all.pdf}
    }


 \caption{Neuron activations across different layers of the Qwen series models. We present the inhibition ratio $\Delta R$ under two conditions: with contextual knowledge input (w/ context) and without it (w/o context). }
 \label{fig:act_qwen}
\end{figure*}


\end{document}
\endinput
%%
%% End of file `sample-sigconf.tex'.
