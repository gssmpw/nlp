\begin{abstract}

The effectiveness of in-memory dynamic graph storage (DGS) for supporting concurrent graph read and write queries is crucial for real-time graph analytics and updates. Various methods have been proposed, for example, LLAMA, Aspen, LiveGraph, Teseo, and Sortledton. These approaches differ significantly in their support for read and write operations, space overhead, and concurrency control. However, there has been no systematic study to explore the trade-offs among these dimensions. In this paper, we evaluate the effectiveness of individual techniques and identify the performance factors affecting these storage methods by proposing a common abstraction for DGS design and implementing a generic test framework based on this abstraction. Our findings highlight several key insights: 1) Existing DGS methods exhibit substantial space overhead. For example, Aspen consumes 3.3-10.8x more memory than CSR, while the optimal fine-grained methods consume 4.1-8.9x more memory than CSR, indicating a significant memory overhead.  2) Existing methods often overlook memory access impact of modern architectures, leading to performance degradation compared to continuous storage methods. 3) Fine-grained concurrency control methods, in particular, suffer from severe efficiency and space issues due to maintaining versions and performing checks for each neighbor. These methods also experience significant contention on high-degree vertices.  Our systematic study reveals these performance bottlenecks and outlines future directions to improve DGS for real-time graph analytics.

\end{abstract}