\label{sec:sdl}

To evaluate and compare autonomous vehicle behaviors effectively, it is essential to represent driving scenarios in a structured format that captures the complexities of vehicle actions over time. 
% Scenario Description Languages (SDLs) fulfill this need by encoding trajectories as sequences of interpretable actions. 
% % Each SDL label represents a vehicle's behavior in terms of discrete, high-level actions, making it possible to analyze scenarios and perform tasks such as similarity searches, which compare trajectories based on their behavioral patterns. 
% Each SDL label represents a vehicle's behavior in terms of discrete, high-level actions, making it possible to analyze scenarios and compare trajectories based on their behavioral patterns. 
Scenario Description Languages (SDLs) fulfill this need by encoding static elements such as intersections and dynamic elements such as vehicles and pedestrians.
These elements are easily obtained from the vectorized data used by AVs, as the output of the perception module.
Therefore, we focus on the temporal task of describing trajectories as sequences of interpretable actions.
This represents a more difficult task, as vectorized data does not contain labels for AV behavior.
Therefore, we introduce SDL labels that represent a vehicle's behavior in terms of discrete, high-level actions, making it possible to analyze scenarios and compare trajectories based on their behavioral patterns. 

\subsection{Overview of SDL Structure}

\begin{figure}
    \centering
    \includegraphics[width=\columnwidth]{figures/ActionSpectrum.pdf}
    \caption{Actions in our SDL, organized along a spectrum of complexity. At the abstract level, there are three lateral and three longitudinal labels, which can be further subdivided into finer-grained labels for more detailed analysis.}
    \label{fig:action_spectrum}
\end{figure}

SDLs provide a systematic way to describe driving behaviors.
Based on the SDL framework outlined by~\cite{aron_date24}, we define an SDL that characterizes autonomous vehicle behaviors through a defined set of temporal actions.
Figure~\ref{fig:action_spectrum} illustrates the set of vehicle behaviors employed in our SDL framework.
We partition the driving behavior of the autonomous vehicle in terms of lateral and longitudinal actions, where Lateral actions capture directional movements, such as turns or lane changes and Longitudinal actions capture changes in speed, such as acceleration or deceleration.
The separation of lateral and longitudinal actions in SDL reflects the distinct control systems in autonomous vehicles, allowing flexible and accurate representation of combined driving behaviors. 
This structure also aligns with established practices in vehicle dynamics, and allows us to evaluate autonomous vehicle responses across scenarios in a way that is 
interpretable, consistent, and comparable.

\subsection{Hierarchical Levels of SDL Labels}
\label{sec:sdl_hierarchy}

As shown in Figure~\ref{fig:action_spectrum} the SDL uses a hierarchical approach, with multiple levels of refinement that capture vehicle behavior at varying 
degrees of detail. 
These levels, progressing from the simplest to the most detailed, enable SDL to adapt to different levels 
of data availability and task requirements. 
Below, we define each level in the hierarchy and describe how it represents driving behavior.

\paragraph{Trace Level}
At the most abstract level, the Trace level encodes actions using broad categories. Lateral actions are simplified to 
basic movement types (e.g., "Left Turn," "Right Turn," "Straight"), while longitudinal actions are captured as "Accelerate," 
"Decelerate," or "Maintain Speed." This level provides a coarse-grained view of driving behavior, suitable for high-level analysis.

\paragraph{Trend Level}
The Trend level refines the Trace level by distinguishing between different speed and state categories. For example, 
the "Maintain Speed" action is divided into "Stopped" and "Moving" states. This level provides a more nuanced understanding 
of vehicle behavior, capturing the persistence of actions over time.

\paragraph{Maneuver Level}
At the Maneuver level, SDL labels begin to capture more complex actions, particularly in terms of lateral behavior. 
For example, lane-change maneuvers such as "Left Merge" and "Right Merge" are explicitly included. This level is valuable 
for identifying interactions between vehicles or with the road environment, as it captures intentional, goal-directed behaviors.

\paragraph{Action Level}
The most detailed level, the Action level, distinguishes actions based on the intensity and type of movement. 
For instance, turns are classified into "Gradual," "Medium," and "Aggressive," depending on the vehicle’s yaw rate. 
Similarly, accelerations and decelerations are categorized by speed (e.g., "Accelerate Slow," "Accelerate Medium"). 
The Action level is particularly useful for granular analyses, where precise distinctions in vehicle dynamics are required.

The hierarchy of SDL levels allows flexibility in matching the level of detail to the data and task requirements. 
For example, a high-level comparison across many scenarios may only require Trace-level labels, while a detailed analysis of specific behaviors benefits from Action-level granularity.
Each of these levels provides a set of possible labels for lateral and longitudinal components, which we denote as 
\(L_{\text{lat}}\) and \(L_{\text{long}}\), respectively. At any level \(X\), the sets \(L_{\text{lat}}^{X}\) and 
\(L_{\text{long}}^{X}\) define the allowable actions for lateral and longitudinal behaviors. For example:
\[
L_{\text{lat}}^{\text{Trend}} = \{\text{Left Turn}, \text{Right Turn}, \text{Straight}\}
\]
\[
L_{\text{long}}^{\text{Trend}} = \{\text{Accelerate},\text{Stopped}, \text{Maintain Speed}, \text{Decelerate}\}
\]
The transformation from a trajectory to SDL labels involves segmenting the trajectory into intervals and assigning actions based on the vehicle’s lateral and longitudinal states. Thresholds defined for yaw rate, acceleration, and velocity determine the classification at each level. 
In our work, SDL labels describe behaviors that are at least 1 second in length to ensure that the vehicle is exhibiting an intentional behavior, rather than noise.
% Behaviors that are less than 1 second long represent these small adjustments, so we limit SDL labels to actions that are 1 second or longer.


