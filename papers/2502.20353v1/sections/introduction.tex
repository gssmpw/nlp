Autonomous cyber-physical systems such as self-driving cars and delivery robots present a great opportunity for significant societal benefits.
Autonomous Vehicles (AVs) could prevent crashes and reduce traffic fatalities, as well as increase mobility independence for millions of people. 
However, despite years of development, the widespread deployment of fully autonomous vehicles that can operate safely remains an ongoing challenge.
Public testing of autonomous vehicles has revealed their failures~\cite{mccarthy2022autonomous,favaro2017examining,brown2023halting,bote_2023, zipper, song_2021,templeton_2023,shepardson_2023,stumpf}, sometimes leading to tragic fatal outcomes~\cite{o2016we,rodriguez_2023}.
A methodical approach for measuring and improving safety for AVs is necessary to reduce these outcomes.

Defining safety for AVs is a complex issue within the industry and its definition varies considerably, from reducing road fatalities to imagining a future without driver's licenses.
Safety assessments rely on self-reported data from AV companies, each with their own interpretations of safe driving behavior. 
Metrics such as autonomous miles driven and miles per disengagements, provide a high-level assessment of AV safety. However, it's important to note that low disengagement rates do not necessarily equate to higher safety.

Safety evaluations for AVs need to move beyond voluntary self-reporting.
The transition from distance-based to scenario-based assessments of AV performance is evident in the rise of Scenario Description Languages (SDLs). These SDLs serve as semantic embeddings, capturing the complexity of driving scenarios composed of static, dynamic, and temporary elements in a structured format.
SDLs and systems proposed by researchers such as \cite{aron_date24,M-SDL,Scenic} offer an interpretable framework for humans and support essential AV safety tasks, including large-scale virtual testing scenario generation, real-to-sim transfer, and post-deployment operational design domain monitoring.
For instance, suppose we want to compare the safety behavior of two AVs. 
We can analyze one AV's performance in a specific traffic scenario, then utilize the corresponding SDL to find similar traffic situations for the other AV. 
Here, SDLs facilitate the observation of both AVs' behaviors under analogous traffic conditions, providing a deeper insight into their safety.
It is essential to highlight that extracting SDLs from AV scenario data requires intensive manual labeling and oversight.

In this paper, we introduce the Trajectory-to-Action Pipeline (TAP) - a rules-based method capable of automatic SDL extraction from trajectory data.
We address the challenging task of identifying spatio-temporal actions within an AV scenario and encoding them into an interpretable, and searchable, SDL embedding.
This paper has the following research contributions:
\begin{itemize}
    \item We describe the Trajectory-to-Action Pipeline - a method for embedding any trajectory as an SDL.
    \item We provide a flexible SDL schema that can be applied to a variety of tasks.
    \item We show that our SDL schema can be used to automatically find scenarios of interest in the Waymo Open Motion Dataset.
\end{itemize}