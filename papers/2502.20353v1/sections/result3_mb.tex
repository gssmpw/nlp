In addition to identifying similar behaviors, we used SDL label distances to detect unique trajectories, classified as such when no other trajectory in the dataset shares the same SDL label. 
From the Waymo Open Motion Dataset, we identified 395 unique trajectories out of 25,889, revealing rare or complex driving patterns.
Figure~\ref{fig:unique_results} shows two examples of unique trajectories.
The left depicts a vehicle entering a turn lane, making a right turn while decelerating. 
This trajectory is labeled unique due to five lateral and three longitudinal actions within a short 9-second window, suggesting a potentially complex or erratic driving pattern. 
The right example involves a vehicle changing lanes while accelerating, similarly marked as unique due to its dense sequence of actions. 
Although this result is for single trajectory similarity, these findings demonstrate that beyond trajectory similarity, detecting such unique trajectories can help uncover instances where an AV performed rare or unexpected maneuvers. 
This is crucial for identifying potentially risky or erratic behaviors, enabling better safety analysis and anomaly detection in large-scale AV datasets.
\begin{figure}
    \centering
    \includegraphics[width=\linewidth]{figures/InterestResults.pdf}
    \caption{Two trajectories whose labels were completely unique within the dataset. Both have many lateral and longitudinal actions, which is uncommon among the dataset, resulting in unique samples.}
    \label{fig:unique_results}
\end{figure}