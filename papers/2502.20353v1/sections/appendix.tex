\section{Code Release and Reproducibility}
We prioritize transparency and reproducibility by providing access to the methods and data used in this work. The Waymo Open Motion Dataset employed in our experiments is publicly available at \url{https://waymo.com/open/}. The implementation of the Trajectory-to-Action Pipeline (TAP), which automates the extraction of SDL embeddings, will be released on GitHub upon the paper’s publication. This release will include all necessary scripts and documentation to facilitate replication and further exploration of our approach.

\section{Partition Rules}
\label{sec:appendix_partition_rules}

% In order to automatically extract SDL labels from trajectories, each label is defined based on a set of thresholds $\Theta = \{ \Theta_{\omega}, \Theta_{a}, \Theta_{v} \}$.
In order to automatically extract SDL labels from trajectories, the vehicle state data distributions $D_{\omega}$, $D_a$, and $D_v$ are separated by on a set of thresholds $\Theta = \{ \Theta_{\omega}, \Theta_{a}, \Theta_{v} \}$.
These thresholds separate the yaw rate, acceleration, and velocity of the vehicle into distinct partitions corresponding to the SDL Labels.
Table~\ref{table:partitions} shows the partitions and the rules-based criteria for each data distribution.
For example, the yaw rate distribution is partitioned by the thresholds $\Theta_{\omega} = \{\theta_{\omega,\text{str}}, \theta_{\omega,\text{grad}}, \theta_{\omega,\text{med}}\}$ into labels of Straight, Gradual Turn, Medium Turn, and Aggressive Turn.

% With bars to show average
% \begin{table}
% \centering
% \renewcommand{\arraystretch}{1}
% \begin{tabular}{ |c|c| }
%  \hline
%  \rowcolor[gray]{0.9} \textbf{Logical Rule} & \textbf{Partition} \\
%  \hline\hline
%  \multicolumn{2}{|c|}{\textbf{Yaw Rate ($\bar{\omega}$)}} \\
%  \hline
%  $|\bar{\omega}| < \theta_{\omega,\text{str}}$ & Straight \\
%  $\theta_{\omega,\text{str}} < |\bar{\omega}| < \theta_{\omega,\text{grad}}$ & Gradual Turn \\
%  $\theta_{\omega,\text{grad}} < |\bar{\omega}| < \theta_{\omega,\text{med}}$ & Medium Turn \\
%  $\theta_{\omega,\text{med}} < |\bar{\omega}|$ & Aggressive Turn \\
%  \hline
%  \multicolumn{2}{|c|}{\textbf{Acceleration ($\bar{a}$)}} \\
%  \hline
%  $\bar{a} < \theta_{a,\text{dec}}$ & Decelerate \\
%  $\theta_{a,\text{dec}} < \bar{a} < \theta_{a,\text{acc}}$ & Maintain Speed \\
%  $\theta_{a,\text{acc}} < \bar{a}$ & Accelerate \\
%  \hline
%  \multicolumn{2}{|c|}{\textbf{Velocity ($\bar{v}$)}} \\
%  \hline
%  $\bar{v} < \theta_{v,\text{stop}}$ & Stopped \\
%  $\theta_{v,\text{stop}} < \bar{v} < \theta_{v,\text{slow}}$ & Slow \\
%  $\theta_{v,\text{slow}} < \bar{v} < \theta_{v,\text{med}}$ & Medium \\
%  $\theta_{v,\text{med}} < \bar{v}$ & Fast \\
%  \hline
% \end{tabular}
% \caption{
% The rule-based criteria for data partitions. For instance, vehicle acceleration is separated by $\theta_{a,\text{acc}}$ and $\theta_{a,\text{dec}}$ into partitions of decelerate, maintain speed, and accelerate. %Different thresholds are used for each data distribution with no overlap, allowing the data distributions to be analyzed separately.
% }
% \label{table:partitions}
% \end{table}
\begin{table}[b]
\centering
\renewcommand{\arraystretch}{1}
\begin{tabular}{ |c|c| }
 \hline
 \rowcolor[gray]{0.9} \textbf{Logical Rule} & \textbf{Partition} \\
 \hline\hline
 \multicolumn{2}{|c|}{\textbf{Yaw Rate ($\omega$)}} \\
 \hline
 $|\omega| \leq \theta_{\omega,\text{str}}$ & Straight \\
 $\theta_{\omega,\text{str}} < |\omega| \leq \theta_{\omega,\text{grad}}$ & Gradual Turn \\
 $\theta_{\omega,\text{grad}} < |\omega| \leq \theta_{\omega,\text{med}}$ & Medium Turn \\
 $\theta_{\omega,\text{med}} < |\omega|$ & Aggressive Turn \\
 \hline
 \multicolumn{2}{|c|}{\textbf{Acceleration ($a$)}} \\
 \hline
 $a \leq \theta_{a,\text{dec}}$ & Decelerate \\
 $\theta_{a,\text{dec}} < a \leq \theta_{a,\text{acc}}$ & Maintain Speed \\
 $\theta_{a,\text{acc}} < a$ & Accelerate \\
 \hline
 \multicolumn{2}{|c|}{\textbf{Velocity ($v$)}} \\
 \hline
 $v \leq \theta_{v,\text{stop}}$ & Stopped \\
 $\theta_{v,\text{stop}} < v \leq \theta_{v,\text{slow}}$ & Slow \\
 $\theta_{v,\text{slow}} < v \leq \theta_{v,\text{med}}$ & Medium \\
 $\theta_{v,\text{med}} < v$ & Fast \\
 \hline
\end{tabular}
\caption{
The rule-based criteria for data partitions. For instance, vehicle acceleration is separated by $\theta_{a,\text{acc}}$ and $\theta_{a,\text{dec}}$ into partitions of Decelerate, Maintain Speed, and Accelerate. %Different thresholds are used for each data distribution with no overlap, allowing the data distributions to be analyzed separately.
}
\label{table:partitions}
\end{table}

\section{Cross Entropy Seeds}
\label{sec:appendix_ce_seeds}

Cross-entropy optimization can risk converging to local minima. To mitigate this, we initialized the optimization process with three different threshold guesses for each data distribution. Each guess was run for 100 epochs, using the hyperparameters listed in Table~\ref{table:hyperparameters}. Despite the variations in initial thresholds, the method consistently converged to similar final values across all runs.

Tables~\ref{table:threshold_results_acc} to~\ref{table:threshold_results_yaw} present the initial threshold guesses and the corresponding final values for acceleration, velocity, and yaw rate. The consistency of convergence suggests that the final thresholds provide well-balanced partitions and that the optimization process successfully avoids local minima. TAP adopts the thresholds that yield the lowest objective function value as its final parameters.

% \begin{table}[b]
% \centering
% \begin{tabular}{ |cc|c|c|c|c| }
%  \hline
%  % Threshold & Seed 1 & Seed 2 & Seed 3 & Final Value \\
% \multicolumn{2}{|c|}{Threshold} & Seed 1 & Seed 2 & Seed 3 & Final Value \\
%  \hline
%  $\theta_{a,\text{dec}}$ & m/s$^2$ & -0.5 & -1.0 & -2.0 & -1.3715 \\
%  $\theta_{a,\text{acc}}$ & & 0.5 & 1.0 & 2.0 & 1.5557 \\
%  \hline
%  $\theta_{v,\text{slow}}$ & m/s & 7.0 & 10.0 & 12.0 & 10.2140 \\
%  $\theta_{v,\text{med}}$ & & 22.0 & 24.0 & 26.0 & 24.4046 \\
%  \hline
%  $\theta_{\omega,\text{str}}$ & rad/s & 0.01 & 0.02 & 0.05 & 0.0283 \\
%  $\theta_{\omega,\text{grad}}$ & & 0.05 & 0.07 & 0.1 & 0.0754 \\
%  $\theta_{\omega,\text{med}}$ & & 0.1 & 0.15 & 0.2 & 0.1541 \\
%  \hline
% \end{tabular}
% \caption{Initial seeds and final optimized thresholds. 
% %In each case, Seed 1 and Seed 3 are on either side of the final threshold. This implies that the final thresholds calculated by the cross-entropy method are the ones that best partition the data.
% }
% \label{table:threshold_results}
% \end{table}

\begin{table}[b]
\centering
\begin{tabular}{ |c|c|c| }
 \hline
 % Threshold & Seed 1 & Seed 2 & Seed 3 & Final Value \\
 \rowcolor[gray]{0.9} \textbf{Threshold} ($\Theta_{a}$) & \textbf{Initial Value} ($m/s^2)$ & \textbf{Final Value} ($m/s^2$) \\
 \hline \hline
 \multicolumn{3}{|c|}{\textbf{Seed 1}} \\
 \hline
 $\theta_{a,\text{dec}}$ & -0.5 & -1.2350 \\
 $\theta_{a,\text{acc}}$ & 0.5 & 1.4870 \\
 \hline
 \multicolumn{3}{|c|}{\textbf{Seed 2}} \\
 \hline
 $\theta_{a,\text{dec}}$ & -1.0 & -1.3196 \\
 $\theta_{a,\text{acc}}$ & 1.0 & 1.5379 \\
 \hline
 \multicolumn{3}{|c|}{\textbf{Seed 3}} \\
 \hline
 $\theta_{a,\text{dec}}$ & -2.0 & \textbf{-1.3715} \\
 $\theta_{a,\text{acc}}$ & 2.0 & \textbf{1.5557} \\
 \hline
\end{tabular}
\caption{Initial seeds and final optimized acceleration thresholds. The final thresholds with the lowest objective function (shown in bold) were used for TAP.
%In each case, Seed 1 and Seed 3 are on either side of the final threshold. This implies that the final thresholds calculated by the cross-entropy method are the ones that best partition the data.
}
\label{table:threshold_results_acc}
\end{table}

\begin{table}[b]
\centering
\begin{tabular}{ |c|c|c| }
 \hline
 % Threshold & Seed 1 & Seed 2 & Seed 3 & Final Value \\
 \rowcolor[gray]{0.9} \textbf{Threshold} ($\Theta_{v}$) & \textbf{Initial Value} ($m/s$) & \textbf{Final Value} ($m/s$) \\
 \hline \hline
 \multicolumn{3}{|c|}{\textbf{Seed 1}} \\
 \hline
 $\theta_{v,\text{slow}}$ & 7.0 & \textbf{10.2140} \\
 $\theta_{v,\text{med}}$ & 22.0 & \textbf{24.4046} \\
 \hline
 \multicolumn{3}{|c|}{\textbf{Seed 2}} \\
 \hline
 $\theta_{v,\text{slow}}$ & 10.0 & 10.2098 \\
 $\theta_{v,\text{med}}$ & 24.0 & 24.3755 \\
 \hline
 \multicolumn{3}{|c|}{\textbf{Seed 3}} \\
 \hline
 $\theta_{v,\text{slow}}$ & 12.0 & 10.2099 \\
 $\theta_{v,\text{med}}$ & 26.0 & 24.3568 \\
 \hline
\end{tabular}
\caption{Initial seeds and final optimized velocity thresholds. The final thresholds with the lowest objective function (shown in bold) were used for TAP.
%In each case, Seed 1 and Seed 3 are on either side of the final threshold. This implies that the final thresholds calculated by the cross-entropy method are the ones that best partition the data.
}
\label{table:threshold_results_vel}
\end{table}

\begin{table}[b]
\centering
\begin{tabular}{ |c|c|c| }
 \hline
 % Threshold & Seed 1 & Seed 2 & Seed 3 & Final Value \\
 \rowcolor[gray]{0.9} \textbf{Threshold} ($\Theta_{\omega}$) & \textbf{Initial Value} ($rad/s$) & \textbf{Final Value} ($rad/s$) \\
 \hline \hline
 \multicolumn{3}{|c|}{\textbf{Seed 1}} \\
 \hline
 $\theta_{\omega,\text{str}}$ & 0.01 & 0.0273 \\
 $\theta_{\omega,\text{grad}}$ & 0.05 & 0.0728 \\
 $\theta_{\omega,\text{med}}$ & 0.1 & 0.1517 \\
 \hline
 \multicolumn{3}{|c|}{\textbf{Seed 2}} \\
 \hline
 $\theta_{\omega,\text{str}}$ & 0.02 & \textbf{0.0283} \\
 $\theta_{\omega,\text{grad}}$ & 0.07 & \textbf{0.0754} \\
 $\theta_{\omega,\text{med}}$ & 0.15 & \textbf{0.1541} \\
 \hline
 \multicolumn{3}{|c|}{\textbf{Seed 3}} \\
 \hline
 $\theta_{\omega,\text{str}}$ & 0.05 & 0.0320 \\
 $\theta_{\omega,\text{grad}}$ & 0.1 & 0.0898 \\
 $\theta_{\omega,\text{med}}$ & 0.2 & 0.1585 \\
 \hline
\end{tabular}
\caption{Initial seeds and final optimized yaw rate thresholds. The final thresholds with the lowest objective function (shown in bold) were used for TAP.
%In each case, Seed 1 and Seed 3 are on either side of the final threshold. This implies that the final thresholds calculated by the cross-entropy method are the ones that best partition the data.
}
\label{table:threshold_results_yaw}
\end{table}

\section{Additional Similarity Results}
\label{sec:appendix_more_sim_results}

In this section, we provide further examples demonstrating TAP’s ability to identify behaviorally similar trajectories more effectively than the baseline metrics, Average Displacement Error (ADE) and Dynamic Time Warping (DTW). Figure~\ref{fig:more_simsearch_results} presents a reference trajectory (magenta), along with trajectories that match the extracted SDL labels (gray) and those identified as most similar by ADE (dark blue) and DTW (light blue). Each trajectory’s SDL label, as extracted by TAP, is also shown.

\subsection*{Example 1}

In example \circled{1} of Figure~\ref{fig:more_simsearch_results}, the reference trajectory travels straight, makes an aggressive right turn, and then straightens again. The nearest ADE trajectory starts straight and makes a similar aggressive right turn but fails to straighten afterward. In contrast, the nearest DTW trajectory begins its turn prematurely, missing the initial straight phase entirely. Additionally, both baseline trajectories fail to replicate the reference’s longitudinal behavior: the reference begins at a slow speed, accelerates to a medium speed, and then maintains that speed. Both ADE and DTW trajectories continue accelerating beyond the medium speed.

\subsection*{Example 2}

In example \circled{2}, the reference trajectory travels straight before executing an aggressive left turn and accelerating slightly before stabilizing at a slow speed. The nearest ADE and DTW trajectories replicate the lateral behavior but deviate significantly in longitudinal dynamics. The ADE trajectory begins at a medium speed and decelerates to a slow speed, while the DTW trajectory fails to accelerate enough to be classified as accelerating.

\subsection*{Example 3}

In example \circled{3}, the reference trajectory makes a gradual left turn between two straight segments, all at a medium speed. The nearest ADE trajectory matches the lateral pattern but travels at a much slower speed. Conversely, the nearest DTW trajectory mirrors the longitudinal behavior but transforms the entire sequence into a prolonged turn without straight-line segments.

These examples highlight how TAP more accurately captures both lateral and longitudinal dynamics, leading to better identification of behaviorally similar trajectories compared to baseline methods.

\begin{figure} [b]
    \centering
    \includegraphics[width=\linewidth]{figures/SimSearchResults3.pdf}
    \caption{Three more examples of trajectories with similar SDL Labels to a reference trajectory shown in magenta. The trajectory with the lowest ADE distance to the reference is shown in dark blue, and the trajectory with the lowest DTW distance to the reference is shown in light blue. Labels are highlighted in red when it contains an action not in the reference, or underlined when it is missing a action. For each trajectory, the velocity of the vehicle is plotted on the right.}
    \label{fig:more_simsearch_results}
\end{figure}
