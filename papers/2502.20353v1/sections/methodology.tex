% Our goal is to perform automated SDL extraction on trajectory data using a set of rules derived from the data.%, as described in Equation~\ref{eqn:sdl_extraction}.
% In this section we introduce Trajectory-to-Action Pipeline (TAP), a method for automated SDL extraction using rules-based automata.
% Figure~\ref{fig:sdl_pipeline} provides an overview of how TAP converts trajectory data into SDL labels.
% At each stage of the pipeline, automata and other rules are applied to detect increasingly complex actions.
% % TAP uses four automata, which classify actions based on the thresholds defined in $\Theta$.
% Below we describe the stages of TAP in detail.

The goal of the Trajectory-to-Action Pipeline (TAP) is to automate the extraction of Scenario Description Language (SDL) labels from raw, unlabeled vehicle trajectory data.
The absence of ground truth labels prevents the direct use of supervised learning approaches. 
TAP extracts SDL labels offline, for post hoc analysis of vehicle behavior.
%The goal is to convert sensor inputs - specifically yaw rate, acceleration, and velocity - into interpretable action labels that describe vehicle behaviors. 
Figure~\ref{fig:sdl_pipeline} provides an overview of the entire Trajectory-to-Action Pipeline.
The pipeline consists of multiple stages, each refining the classification of vehicle actions. 
%These stages utilize rule-based automata to segment trajectories into meaningful SDL labels.
Below, we detail the different stages involved in TAP.

\subsection{Trajectory-to-Action Pipeline}
\label{sec:methodology_tap}
% \begin{table}
% \centering
% \begin{tabular}{ |c|c| }
%  \hline
%  Rule & Action \\
%  \hline
%  $\dot{\theta}_t > \Theta_{\dot{\theta},strt}$ & Left \\
%  $\dot{\theta}_t > -\Theta_{\dot{\theta},strt} \land \dot{\theta}_t < \Theta_{\dot{\theta},strt}$ & Straight \\
%  $\dot{\theta}_t < -\Theta_{\dot{\theta},strt}$ & Right \\
%  \hline
%  $|\dot{\theta}_t| > \Theta_{\dot{\theta},strt} \land |\dot{\theta}_t| < \Theta_{\dot{\theta},grad}$ & Grad Turn \\
%  $|\dot{\theta}_t| > \Theta_{\dot{\theta},grad} \land |\dot{\theta}_t| < \Theta_{\dot{\theta},med}$ & Med Turn \\
%  $|\dot{\theta}_t| > \Theta_{\dot{\theta},med}$ & Aggr Turn \\
%  \hline
%  $a_t < \Theta_{a,dec}$ & Dec \\
%  $a_t > \Theta_{a,dec} \land a_t < \Theta_{a,acc}$ & Maintain \\
%  $a_t > \Theta_{a,acc}$ & Acc \\
%  \hline
%  $v_t < \Theta_{v,stop}$ & Stopped \\
%  $v_t > \Theta_{v,stop} \land v_t < \Theta_{v,slow} $ & Slow \\
%  $v_t > \Theta_{v,slow} \land v_t < \Theta_{v,med} $ & Med \\
%  $v_t > \Theta_{v,med} $ & Fast \\
%  \hline
% \end{tabular}
% \caption{The rule set used by our automata. Each rule is constructed such that there is no overlap between rules and the set of rules is complete. These rules are used by five automata, two lateral automata, and three longitudinal automata.}
% \label{table:ruleset}
% \end{table}

Given a trajectory $\tau^j$ as input, TAP uses the yaw rate $\omega$ as lateral input and acceleration $a$ and velocity $v$ as longitudinal inputs.
Our SDL defines separate lateral and longitudinal labels, and TAP enforces this separation in its analysis of the data.
% From the input data, TAP uses a series of automata based on the thresholds in $\Theta$ to assign actions based on the vehicle state at each time step.
Starting from the input trajectory data, TAP involves a series of rules in 4 stages to go from the most abstract, trace-level SDL labels in stage 1 to the most refined action-level SDL labels in stage 4.
Figure~\ref{fig:sdl_pipeline} shows the different stages of TAP which are described in detail below.
% Beginning with trace-level labels, which are the most abstract actions as shown in Figure~\ref{fig:action_spectrum}, the labels become more refined with each stage.

\begin{figure}
    \centering
    \includegraphics[width=1.0\columnwidth]{figures/SDLPipeline.pdf}
    \caption{Trajectory-to-Action Pipeline (TAP) starts with acceleration, velocity, and yaw rate data as input. At each stage, it uses automata and rules to produce SDL labels at the level of refinement corresponding to that stage. The final stage of TAP produces action-level SDL labels for every vehicle.}
    \label{fig:sdl_pipeline}
\end{figure}

\textbf{Stage 1 - Trace}
The input to the first stage of TAP is a raw trajectory with yaw rate $\omega$ and acceleration $a$.
The goal of stage 1 is to produce trace-level SDL labels $\{L^{\text{Trace}}_{\text{lat},1},L^{\text{Trace}}_{\text{lat},2},\dots,L^{\text{Trace}}_{\text{lat},T}\}$ and $\{L^{\text{Trace}}_{\text{long},1},L^{\text{Trace}}_{\text{long},2},\dots,L^{\text{Trace}}_{\text{long},T}\}$ to describe vehicle behavior, where $L^{\text{Trace}}_{\text{lat}}$ are \{Left Turn, Right Turn, Straight\}, while $L^{\text{Trace}}_{\text{long}}$ are \{Accelerate, Decelerate, Maintain Speed\}.
TAP utilizes lateral and longitudinal rules-based automata to automatically extract trace-level SDL labels.
% The trace lateral automaton has three nodes, Straight, Left, and Right.
The trace lateral automaton takes as input the trajectory's yaw rate $\omega = \{\omega_0,\omega_1,\dots,\omega_T\}$.
The automaton has three nodes, Left, Right, and Straight, that correspond to the three trace-level lateral SDL labels.
% The rules-based criteria for the trace lateral automaton (shown in Figure~\ref{fig:sdl_pipeline} and Table~\ref{table:partitions}) is given by:
The rules-based criteria for the trace lateral automaton (shown in Figure~\ref{fig:sdl_pipeline}) is given by:
\begin{equation}
    \begin{cases}
        \omega_t > \theta_{\omega,\text{str}}\,; & L^{\text{Trace}}_{\text{lat}} = \text{Left} \\
        -\theta_{\omega,\text{str}} < \omega_t < \theta_{\omega,\text{str}}\,; & L^{\text{Trace}}_{\text{lat}} = \text{Straight} \\
        \omega_t < -\theta_{\omega,\text{str}}\,; & L^{\text{Trace}}_{\text{lat}} = \text{Right}
    \end{cases}
\end{equation}
Starting at $t=0$, each yaw rate measurement $\omega_t$ from the trajectory is fed into the trace lateral automaton to produce a label $L^{\text{Trace}}_{\text{lat},t}$
% The transition between these nodes is predicated on the yaw rate thresholds shown in Figure~\ref{fig:sdl_pipeline}.
% Starting at time $t=0$, the output of the trace lateral automaton is the trace-level lateral labels.
% The first stage in TAP is to apply a lateral and a longitudinal automaton to the input trajectory data.
% The rules for these automata, shown in Figure~\ref{fig:sdl_pipeline}, use the thresholds defined in $\Theta$ to detect trace-level actions from the trajectory data.
% The lateral automaton is the trace lateral automaton $A_{\text{coarse-lat}}$, which uses the threshold $\theta_{\omega,str}$ to detect actions of Left Turn, Right Turn, and Straight from yaw rate data.
% The trace of the trace lateral automaton produces trace-level lateral SDL labels: $\text{trace}(A_{\text{coarse-lat}}) = \{L^{\text{Trace}}_{\text{lat},1},L^{\text{Trace}}_{\text{lat},2},\dots,L^{\text{Trace}}_{\text{lat},T}\}$.

Consider the example shown in Figure~\ref{fig:example_traj}.
This figure shows the yaw rate $\omega$, acceleration $a$, and velocity $v$ at each time step for an example trajectory.
% The extraction rules dictate that a yaw rate between the thresholds $\theta_{\omega,\text{str}}$ and $-\theta_{\omega,\text{str}}$ is in the Straight node of the automaton, resulting in a trace-level label of Straight.
At $t = 0.0$ seconds, the yaw rate satisfies the criteria for the trace-level SDL label Straight.
At $t = 1.0$ seconds, the yaw rate falls below the threshold $-\theta_{\omega,\text{str}}$.
Therefore, the extracted label is Right Turn as the automaton transitions to the Right node.
At $t = 4.6$ seconds, the yaw rate rises back above the threshold to again be assigned a Straight SDL label.
%Longitudinally at this stage, the goal is to produce one of the three labels of Decelerate, Maintain Speed, or Accelerate.
Similar to the trace lateral automaton, the acceleration automaton produces trace-level longitudinal SDL labels $L^{\text{Trace}}_{\text{long}}$.
% Longitudinally, the acceleration automaton has nodes of Accelerate, Decelerate, and Maintain Speed.
% The transitions between nodes in the acceleration automaton are predicated on the acceleration thresholds, and the output is the trace-level longitudinal labels.
% Longitudinally, the acceleration automaton $A_{\text{accel}}$ uses thresholds $\theta_{a,\text{acc}}$ and $\theta_{a,\text{dec}}$ to detect Accelerate, Decelerate, and Maintain Speed actions from acceleration data.
% The trace of the acceleration automaton is the trace-level longitudinal SDL labels: $\text{trace}(A_{\text{accel}}) = \{L^{\text{Trace}}_{\text{long},1},L^{\text{Trace}}_{\text{long},2},\dots,L^{\text{Trace}}_{\text{long},T}\}$.
Figure~\ref{fig:example_traj} also shows how the acceleration automaton produces longitudinal labels from the acceleration $a$ for our example.
At the end of stage 1, TAP has automatically extracted trace-level SDL labels $L^{\text{Trace}}$ from trajectory yaw rate $\omega$ and acceleration $a$.

\begin{figure}
    \centering
    \includegraphics[width=1.0\linewidth]{figures/ExampleTraj-1col.pdf}
    \caption{Example trajectory of a vehicle making a right turn, with yaw rate $\omega$, acceleration $a$, and velocity $v$ analyzed by TAP. Lateral SDL labels (trace-level) are Right Turn and Straight, while longitudinal labels are Maintaining Speed and Accelerating. At stage 4, TAP refines the velocity data into SDL actions: Maintain Slow Speed, Accelerate Slow Speed, and Accelerate Medium Speed.}
    \label{fig:example_traj}
\end{figure}

% The same example also shows the longitudinal behavior.
% Initially the acceleration is between the deceleration threshold $\theta_{a,\text{dec}}$  and the acceleration threshold $\theta_{a,\text{acc}}$ to be in the Maintain node of the acceleration automaton.
% At $t=0.2$ seconds, the acceleration falls below $\theta_{a,\text{dec}}$, resulting in a label of Deceleration.
% At $t=0.8$ seconds, the acceleration rises again to be in the Maintain node of the automaton for a label of Maintain Speed.
% The acceleration continues rising, and at $t=3.2$ seconds it has risen above $\theta_{a,\text{acc}}$ for a label of Accelerate.
% As shown by this example, when you use these automata the result is trace-level SDL labels derived from input trajectory data.

\textbf{Stage 2 - Trend}
The only difference between trace-level and trend-level SDL labels is in the longitudinal aspect, where a Stop label is introduced into the SDL.
The input to stage 2 of TAP is a trajectory with trace-level SDL labels $L^{\text{Trace}}$ as well as velocity $v = \{v_0,v_1,\dots,v_T\}$ which has not been used thus far.
% The goal of stage 2 is to produce more refined trend-level SDL labels $L^\text{Trend}$ from these inputs.
The goal of stage 2 is to detect when the vehicle is stopped as well as handle noise in the data.
% to produce more refined trend-level SDL labels $L^\text{Trend}$ from these inputs.
% The next level of refinement is called a trend.
% Trends can be composed from lower level traces by incorporating information on whether the vehicle is moving or stopped.
Whether or not a vehicle is stopped cannot be determined by the acceleration automaton.
Therefore, at this stage, TAP implements a Stopped Vehicle Rule, governed by threshold $\theta_{v,\text{stop}}$, to analyze the velocity $v_t$ and detect instances when the vehicle is not moving.
When $v_t < \theta_{v,\text{stop}}$, TAP assigns an action of Stopped to the label $L^{\text{Trend}}_{\text{long},t}$.
% When $v_t < \theta_{v,\text{stop}}$, TAP assigns a new action to both the longitudinal and lateral labels.
% Longitudinally, $L^{\text{Trend}}_{\text{long},t}$ is given an action of Stopped.
% Laterally, the sample $L^{\text{Trend}}_{\text{lat},t}$ is given the action of Straight, since a vehicle that is not moving cannot turn.
It is possible that when the vehicle is stopped, the data contains non-zero yaw rate $\omega$ values. We postulate that this is due to noise, as a vehicle that is not moving cannot be turning.
% When we comprise higher-level trends, we need to be mindful of noise in the data.
% For example, when a vehicle is stopped, noise in the heading estimate can cause high yaw rate values.
For this reason, at any time $t$ where the vehicle is stopped, the trend-level lateral SDL label $L^{\text{Trend}}_{\text{lat},t}$ is assigned the Straight label.
During the trend stage, TAP also utilizes a low-pass filter to smooth the data.
% If there are only a few frames (e.g., less than 0.3s) with a label that is not consistent with the neighboring frames, TAP computes the average value to smooth these outliers.
If there are only a few frames with a label that is not consistent with the neighboring frames, TAP computes the average value to smooth these outliers.
% out of the label.
Going back to our running example in Figure~\ref{fig:example_traj}, the last few frames of the acceleration's trace-level SDL label are very short in duration but got assigned a Maintain Speed label, despite being surrounded by Accelerate labels.
These would get smoothed by the low-pass filter to all have the trend-level SDL label of Accelerate.
% From the abstract trace-level labels, using velocity information and outlier removal, TAP composes trend-level labels.
At the end of the trend stage, TAP has added the Stop SDL label and removed noise to produce trend-level SDL labels $L^{\text{Trend}}$.

% During the Trend stage, TAP also uses a low-pass filter to smooth the data to reduce the number of unstable actions.
% An action is considered stable if the trend has the same action for 0.3s, and unstable otherwise.
% For a time series of data where the trend is unstable for 0.3s or less, we assume that the instability is the result of a data outlier. Therefore, TAP assigns the unstable trend to have the same action as the neighbor that is nearest to the unstable trend's average value.
% If the trend is unstable for more than 0.3s, we assume that the instability is the result of being near one a threshold. Here, TAP assigns the unstable trend to have the action that matches the average of its data.

% In our example in Figure~\ref{fig:example_traj}, each box represents an trace-level label with a duration of 0.2s.
% The last three labels of the acceleration are unstable, lasting for a duration of 0.6s.
% To smooth the data, TAP uses the average acceleration across the unstable portion of the trace.
% The average acceleration of the unstable trace falls above the $\theta_{a,\text{acc}}$ threshold, for a trend-level action of accelerate.

\textbf{Stage 3 - Maneuver}
At the maneuver-level in the SDL hierarchy, the only change is the addition of a new lateral behavior called the Merge maneuver.
% Given an input of a trajectory with trend-level SDL labels $L^{\text{Trend}}$, stage 3 of TAP generates higher-level maneuvers.
A Merge is a complex behavior and is defined by a turn in one direction followed by a turn in the opposite direction.
Vehicles typically experience Merges when changing lanes or entering/exiting a freeway.
% Merges are difficult to detect because they require temporal information about the Turn trends to distinguish a Merge from a series of turns.
Merges are difficult to detect because the trend-level lateral SDL labels only partition lateral behavior into a Left Turn, Right Turn, or Straight, whereas a Merge can be composed of any combination of these three SDL labels.
% After the driver has entered a new lane, they must turn back in the other direction to straighten their vehicle.
% This is represented as a Turn trend in the opposite direction from the start of the Merge.
To detect Merges from trend-level lateral labels, TAP analyzes $L^{\text{Trend}}_{\text{lat}}$ to search for trajectories which feature turns in one direction, followed by a turn in the opposite direction.
%As our SDL label only describes actions that are at least 1 second in length as discussed in Section~\ref{sec:sdl_hierarchy}, these Turn trends must be at least 1 second long to be considered part of a Merge. % NOTE: Candidate to remove
In order to distinguish a Merge maneuver from successive Turn maneuvers, TAP enforces a time limit on the Merge.
This time limit dictates the maximum time interval between two turns in opposite directions.
Once the initial turn has ended, the turn back in the opposite direction must start within this time limit to be considered a Merge.
% Since the occurrence of Merges was rare in our dataset (0.8\%), we manually examined the merge trajectories and chose 4 seconds as the time limit for the maximum interval between turns in opposite directions.
Since the occurrence of Merges was rare in our dataset (0.8\%), we manually examined the merge trajectories and chose 4 seconds as the maximum interval between turns in opposite directions.
The output of stage 3 of TAP is the maneuver-level SDL labels $L^{\text{Maneuver}}$

\textbf{Stage 4 - Action}
At the beginning of stage 4, every trajectory $\tau^j$ has maneuver-level SDL labels $L^{\text{Maneuver}}_{\text{lat}} = \{\text{Left Turn},\text{Left Merge}, \allowbreak \text{Straight},\text{Right Turn},\text{Right Merge}\}$ and $L^{\text{Maneuver}}_{\text{long}} = \{\text{Accelerate}, \allowbreak \text{Stopped},\text{Maintain Speed},\text{Decelerate}\}$.
% At the beginning of stage 4, every trajectory $\tau^j$ has a set of maneuver-level lateral SDL labels $L^{\text{Maneuver}}_{\text{lat}} = \langle \{L^{\text{Maneuver}}_{\text{lat},1},L^{\text{Maneuver}}_{\text{lat},2}, \dots,L^{\text{Maneuver}}_{\text{lat},T}\}$ and a set of maneuver-level longitudinal SDL labels $ L^{\text{Maneuver}}_{\text{long}} = \{L^{\text{Maneuver}}_{\text{long},1}, L^{\text{Maneuver}}_{\text{long},2},\dots,L^{\text{Maneuver}}_{\text{long},T}\} \rangle$.
In the Action stage, the goal is to distill the maneuver-level SDL labels into more refined action-level SDL labels $L^{\text{Action}}$ that can capture even more nuanced lateral and longitudinal behaviors.
Not all turns are equal—some are gradual, while others are sharp. 
To capture this, action-level SDL labels classify turns as Gradual Turn, Medium Turn, and Aggressive Turn. 
Similarly, longitudinal actions like Accelerate, Maintain Speed, and Decelerate are further divided into speed profiles: Slow, Medium, and Fast.
To refine lateral maneuvers, TAP uses the action lateral automaton with thresholds \(\theta_{\omega,\text{str}}\), \(\theta_{\omega,\text{grad}}\), and \(\theta_{\omega,\text{med}}\) as defined in Table~\ref{table:partitions}. 
This automaton refines the maneuver-level labels into action-level SDL labels such as Gradual, Medium, and Aggressive Turns. 
Given a turn maneuver \(L^{\text{Maneuver}}_{\text{lat},t}\) and yaw rate \(\omega_t\), the output is the action-level label \(L^{\text{Action}}_{\text{lat},t}\).
% To distill lateral maneuvers into refined SDL actions, TAP utilizes the action lateral automaton.
% % The action lateral automaton has three nodes, Gradual, Medium, and Aggressive.
% The action lateral automaton acts in a similar manner to the trace lateral automaton, but uses thresholds $\theta_{\omega,str}$, $\theta_{\omega,grad}$, and $\theta_{\omega,med}$ and the rules-based criteria defined in Table~\ref{table:partitions} to assign action-level labels. % based on yaw rate $\omega_t$.
% The nodes of the action lateral automaton, Gradual, Medium, and Aggressive, correspond to the refined action-level Turn labels.
% % The transition between these nodes is predicated on the magnitude of the yaw rate.
% With a Turn maneuver $L^{\text{Maneuver}}_{\text{lat},t} = \{\text{Left Turn}, \text{Right Turn}\}$ and yaw rate $\omega_t$ as input, the output of the action lateral automaton is the action-level SDL label $L^{\text{Action}}_{\text{lat},t}$.
During this stage, TAP also uses the velocity profile automaton which has nodes of Slow, Medium, and Fast.
The input to the velocity profile automaton is a longitudinal maneuver $L^{\text{Maneuver}}_{\text{long},t}$ and velocity $v_t$, and it assigns one of three different velocity profile labels to produce the action-level longitudinal SDL label $L^{\text{Action}}_{\text{long},t}$.
% With velocity $v $ 
% The transition between these nodes is predicated on the velocity of the trajectory.
% For an input longitudinal maneuver $L^{\text{Maneuver}}_{\text{long},t}$, the velocity automaton outputs the longitudinal action-level label $L^{\text{Action}}_{\text{lat},t}$.
% Going back to our example for a final time, in Figure~\ref{fig:example_traj}, this time let's turn our attention to bottom plot which shows the velocity $v$ of the vehicle.
% In this example, the velocity goes from a Slow speed (around 14 mph) at $t = 0.0$ seconds to a Medium speed (22 mph) at $t = 5.6$ seconds.
% During the previous stage of the pipeline, TAP identified two longitudinal maneuvers - Maintain Speed and Accelerate.
% In this stage, the velocity profile automaton splits these longitudinal maneuvers into three actions of Maintain Slow Speed, Accelerate Slow Speed, and Accelerate Medium Speed.
Referring back to Figure~\ref{fig:example_traj}, the bottom plot shows the vehicle's velocity. 
It starts at a Slow speed (14 mph) at \( t = 0.0 \) seconds and reaches Medium speed (22 mph) by \( t = 5.6 \) seconds. In the previous TAP stage, two longitudinal maneuvers - Maintain Speed and Accelerate - were identified. 
The velocity profile automaton further refines these into Maintain Slow Speed, Accelerate Slow Speed, and Accelerate Medium Speed.
% It is possible for the original maneuvers to be split into action-level labels that are less than 1 second in duration.
% % The action-level automata have the potential to split a maneuver that was longer than 1 second into multiple actions that are all less than 1 second.
% This could potentially lead to a loss of data, as our SDL requires actions to be at least 1 second in length.
% % To prevent this situation, TAP identifies cases where the resulting actions are less than 1 second.
% % In these cases, TAP assigns an action based on the average value to prevent the loss of data.
% To prevent this, TAP assigns an action based on the average of the maneuver.
% For these cases, TAP takes the average yaw rate or velocity value for the unsplit maneuver, and uses that average to determine the action-level label.
% After TAP has used the action lateral automaton and the velocity automaton to compose action-level labels, we are ready to extract the labels $L^{\text{X}}_{\text{Lat}}$ and $L^{\text{X}}_{\text{Long}}$.
The original maneuvers can sometimes be split into action-level labels shorter than 1 second. 
This poses a risk of data loss, as our SDL requires actions to be at least 1 second long. 
To address this, TAP assigns a single action based on the average value of the maneuver when the resulting actions are too short.
After the action stage, every trajectory $\tau^j$ has been labeled with the lateral actions $L^{\text{Action}}_{\text{lat}}$ and the longitudinal actions $L^{\text{Action}}_{\text{long}}$ which are the most refined SDL labels.
% For each of the stages in TAP, the output corresponds to the ordered sequence of labels $L^{\text{X}}_{\text{lat}}$ and $L^{\text{X}}_{\text{long}}$

In summary, for each of the stages in TAP, the output corresponds to the ordered sequence of labels $L^{\text{X}}_{\text{lat}}$ and $L^{\text{X}}_{\text{long}}$ at increasing levels of refinement.
These labels are all the information necessary to construct the SDL label desribed in Section~\ref{sec:sdl}: $L^i_j = \langle \{L_{\text{lat},1},L_{\text{lat},2},\dots\}$,$\{L_{\text{long},1},L_{\text{long},2},\dots\} \rangle$ that capture the behavior of vehicle $V^{i}_{j}$ in scenario $S_i$.

\begin{comment}
\textbf{Stage 5 - SDL Label}
% The final stage of TAP converts the set of actions at each time step into a series of SDL labels.
Each of the previous stages generates a time series of labels.
The final stage of TAP converts a time series of labels into a tuple of ordered sets containing lateral labels $L^{\text{X}}_{\text{Lat}}$ and longitudinal labels $L^{\text{X}}_{\text{Long}}$.
The process of converting a time series of labels into an order set can occur at any stage of TAP to extract labels at any level of refinement.
Here, we will extract the SDL labels at the most refined action level $L^{\text{Action}}_{\text{Lat}}$ and $L^{\text{Action}}_{\text{Long}}$.

In this stage, TAP locates labels in the time series that are 1 second or longer.
% Here, TAP finds all actions that last for 1 second or longer.
For each of these actions, TAP adds them to the set in the order in which they appear.
For cases where the same label would appear twice in a row, the labels are joined together into a single label.
\end{comment}

\subsection{Optimizing SDL Extraction Thresholds with Cross-Entropy}
\label{sec:ce_methods}

The accuracy of the SDL labels generated by TAP depends on the separation thresholds $\Theta$ which are used in the transition rules of the different TAP automata.
The objective then is to automatically determine thresholds $\Theta$ from the distributions $D_{\omega}$, $D_a$, and $D_v$. 
This was described in Section~\ref{sec:problem_sim} as the minimization optimization problem for $J(\Theta)$ (Equation~\ref{eqn:cem_objective}).
The optimization problem is a search over the manifold of $J(\Theta)$ to find thresholds $\Theta^*$ that minimize the objective function.
For e.g. $J(\Theta_{\omega})$ is a 3-dimensional manifold over the yaw rate thresholds $\Theta_{\omega} = \{\theta_{\omega,\text{str}},\theta_{\omega,\text{grad}},\theta_{\omega,\text{med}}\} \in \mathbb{R}^3$.
If $J(\Theta)$ were a smooth function, this problem is trivially solved by computing the gradient $\nabla J(\Theta)$.
However, in our case, $J(\Theta)$ is defined by a set of disjointed rules (Table~\ref{table:partitions}), which in-turn depend on the thresholds.
As a result, due to $J(\Theta)$ non-smooth nature it is not differentiable, making $\nabla J(\Theta)$ unavailable in closed-form.

Consequentially, we need to approximate the gradient $\hat{\nabla} J(\Theta)$ using a successive approximation-based method.
Such methods perform gradient descent on the parameters $\Theta$ in an iterative manner, beginning with an initial guess $\Theta_0$.
At each iteration $k$, the parameters are updated using $\Theta_{k+1} = \Theta_{k}+\eta \hat{\nabla} J(\Theta_{k})$ where $\eta$ is the learning rate, and $\hat{\nabla} J(\Theta_{k})$ is an approximation of the gradient of $J(\Theta_{k})$.
Iterating using this successive approximation yields the optimal values of the thresholds, $\Theta^*$.
TAP uses the cross-entropy method~\cite{CEMethod} for approximating the gradient $\hat{\nabla} J(\Theta_{k})$.
% Since the gradient of $J(\Theta)$ is not available in a closed-form, we will instead use a finite difference approximation:
The cross-entropy method approximates the gradient $\hat{\nabla} J(\Theta_{k})$ using finite differences:
\begin{equation}
    \hat{\nabla} J(\Theta_k) = \frac{J(\Theta_k+\epsilon)-J(\Theta_k-\epsilon)}{2 \epsilon}
\end{equation}
where, $J(\Theta_k-\epsilon)$ and $J(\Theta_k+\epsilon)$ are the values of the objective function computed in the $\epsilon$ neighborhood of $\Theta$.
$\epsilon$ can be adjusted based on the spread of the distribution $D$.
Although cross-entropy methods can be used to find all values of $\Theta$ concurrently, our data has three disjoint data distributions, $D_a$, $D_v$, and $D_{\omega}$, and
therefore, each threshold $\Theta_a$, $\Theta_v$, and $\Theta_{\omega}$ can be computed from its own optimization.
For example, $\Theta^*_{\omega} = \argmin_{\Theta_{\omega}} J(\Theta_{\omega})$.
As thresholds $\Theta_a$, $\Theta_v$, and $\Theta_{\omega}$ have physical meaning in our case - acceleration, velocity, and yaw rate - the initial guesses $\Theta_0$ can be based on a combination of domain knowledge and the data distributions $D_a$, $D_v$, and $D_{\omega}$.
Figure~\ref{fig:cem_results}(a) shows the spread of these distributions along with initial threshold guesses. 


% \subsection{SDL Distance Metric}
% \label{sec:sdl_distance_methods}
% Once we have determined a set of thresholds to perform SDL extraction, we can turn our attention to performing the similarity search.
% To perform this search, we must define a distance metric for Equation~\ref{eqn:sim_set}.
% Since our SDL Label consists of a lateral and a longitudinal component, we will use a distance function
% \begin{equation}
%     d(L^i_j,L^u_v) = \alpha d'(L^{i,Lat}_j,L^{u,Lat}_v)+(1-\alpha) d'(L^{i,Long}_j,L^{u,Long}_v)
% \end{equation}
% That calculates distance separately for each component, and combines them in a ratio defined by $\alpha$.
% The value of $\alpha$ can be set to reflect the importance of lateral versus longitudinal actions. %Setting $\alpha = 0.5$ represents the case where both are equally important.
% For this paper, we want to highly emphasize lateral actions and only consider longitudinal actions in the context of lateral similarity, so we have set $\alpha=0.95$ % Note: any \alpha > 0.9

% The function $d'()$ calculates the distance between two lateral or longitudinal Labels. We chose to use Levenshtein distance~\cite{LevenshteinDistance} for this function.
% Levenshtein distance calculates the number of insertions, deletions, and substitutions needed to make two strings match.
% Other distance metrics could be used to calculate SDL similarity, but that is left for future work.

% When calculating similarity, we consider any sample to be similar if it is as close as the most similar sample $d_{similar} = min(d(L^{i,Lat}_j,L^{u,Lat}_v))$.
% To provide a point of comparison for our method, we also analysis trajectories using Average Displacement Error (ADE).
% To calculate the ADE, we give both trajectories the same initial conditions by moving them to $(0,0)$ and rotating them to have an initial heading of $0$.
% When one trajectory is shorter than the other, we resample the shorter trajectory to have the same number of points.
% We then calculate ADE as the average distance between each pair of points.
