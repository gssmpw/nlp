\label{sec:problem_statement}

% Having described the SDL embedding space, our objective is to automatically extract SDL embeddings from scenario data.
Our objective is to automatically extract SDL embeddings from scenario data.
Let $S_i \in S$ be a scenario out of dataset of scenarios $S$.
Each scenario $S_i$ contains a set of $k$ vehicles $\{V^i_1,V^i_2,\dots,V^i_j,\dots,V^i_k\}$.
The motion of vehicle $V^i_j$ is captured by a set of vehicle states $s = \{x,y,v,a,\phi,\omega\}$ where $x$ and $y$ are the vehicle's position in the map frame, $v$ is the vehicle's velocity in m/s along the heading $\phi$ (radians), $a$ is the vehicle's acceleration in m/s$^2$, and $\omega$ is the vehicle's yaw rate in rad/s.
For vehicle $V^i_j$, trajectory $\tau^j = \{s^j_1,s^j_2,\dots,s^j_T\}$ is an ordered sequence of the vehicle's states across the duration $T$ of the scenario.
To define the scope and limitations of our framework, we make the following assumptions.
\noindent \textbf{List of Assumptions:}
\begin{enumerate}
    \item Although scenarios typically include static elements (e.g., road geometry), our framework focuses on extracting vehicle behavior from trajectory data. We assume static elements can be extracted separately using computer vision.
    \item While real-world traffic involves agent interactions, this work isolates ego trajectories to demonstrate the feasibility of encoding vehicle behavior using SDLs. Future work can extend this to multi-agent scenarios~\cite{causal_agents}.
    \item We use the hierarchical SDL schema described in Section~\ref{sec:sdl}, though the method can adapt to frameworks like Scenic and OpenScenario.
    \item We assume no ground truth action labels are provided for the trajectory. Without labeled data, supervised approaches like neural networks are infeasible. Unsupervised methods (e.g., k-means) rely on metrics such as ADE, which inadequately capture spatio-temporal behavior (Section~\ref{sec:result2}).
\end{enumerate}

Under these assumptions, given a trajectory $\tau^j$, our goal is to find a function
\begin{equation}
    f_{\Theta}: \tau^j \rightarrow L^i_j
    \label{eqn:sdl_extraction}
\end{equation}
that defines a mapping between an input vehicle trajectory $\tau^j$, and produces an SDL label $L^i_j$ as the output describing the vehicle's behavior.
The SDL label $L^i_j = \langle \{L_{\text{lat},1},L_{\text{lat},2},\dots\},\{L_{\text{long},1},L_{\text{long},2},\dots\} \rangle$ is a tuple containing a chronologically ordered sequence of the lateral and longitudinal SDL labels as defined in Section~\ref{sec:sdl_hierarchy}.
% In our case, $f_{\Theta}$ is realized through a set of rules, for which $\Theta = \{\theta_{a,acc},\theta_{a,dec},\theta_{v,stop},\theta_{v,slow},\theta_{v,med},\theta_{\omega,str},\theta_{\omega,grad},\theta_{\omega,med}\}$ denotes a set of separating thresholds.
$f_{\Theta}$ is realized through a set of rules, where $\Theta$ denotes a set of separation thresholds.
The problem of automatic SDL extraction is the problem of determining values for these thresholds.

\subsection{Threshold Calculation Problem}
\label{sec:problem_threshold}

The objective of threshold calculation is to find separation thresholds $\Theta$ that ensure samples with the same label are more similar to each other than to those with different labels. Automatically deriving these thresholds from the data enables our method to generalize across datasets. 
%As explained, SDL labels describe lateral and longitudinal behavior that utilize yaw rate $\omega$, acceleration $a$, and velocity $v$.
%For each of these characteristics, we can compute the 1 second average across all samples for every scenarios.
Let $D$ denote the set $D = \{D_{\omega}, D_a, D_v\}$ of distributions of 1 second sampled averages of yaw rate $\omega$, acceleration $a$, and velocity $v$ across the entire dataset, spanning all scenarios.
% As an example, the acceleration distribution $D_a$ consists of acceleration values from every vehicle across all scenarios.
% Each vehicle $V^i_j$ contributes the 1 second sampled average of its accelerations $\{\bar{a}^i_{j,1},\bar{a}^i_{j,2},\dots\}$ to the distribution.
For example, vehicle $V^i_j$ contributes the 1 second sampled average of its accelerations $\{\bar{a}^i_{j,1},\bar{a}^i_{j,2},\dots\}$ to the distribution $D_a$.
The sampling interval of 1 second allows noise to be smoothed without losing key behavioral differences.
% The sampling interval is 1 second as it represents the shortest duration for which a meaningful label of a trajectory can be assigned.
% Therefore, the acceleration distribution is $D_a = \{\bar{a}^i_{j,1},\bar{a}^i_{j,2},\dots,\bar{a}^i_{j+1,1},\bar{a}^i_{j+1,2},\dots\}$.
The velocity distribution $D_v$ and yaw rate distribution $D_{\omega}$ are defined similarly.
% Similarly, the velocity distribution $D_v$ is the set of 1 second sampled average velocities across all vehicles and $D_{\omega}$ is the set of 1 second sampled average yaw rates.
The separating thresholds $\Theta = \{\Theta_{\omega},\Theta_a,\Theta_v\}$ are defined over their corresponding distributions $D_{\omega}$, $D_a$, and $D_v$, where $\Theta_{\omega} = \{\theta_{\omega,\text{str}},\theta_{\omega,\text{grad}},\theta_{\omega,\text{med}}\}$, $\Theta_{a} = \{\theta_{a,\text{dec}},\theta_{a,\text{acc}}\}$, and $\Theta_{v} = \{\theta_{v,\text{stop}}, \theta_{v,\text{slow}},\theta_{v,\text{med}}\}$ are thresholds that realize the rule-based function $f_{\Theta}$.
% For each distribution, the corresponding thresholds separate the distribution into distinct partitions based on the rules-based criteria listed in Appendix~\ref{sec:appendix_partition_rules}.
% For each distribution, the corresponding thresholds separate the distribution into distinct partitions based on the rules-based criteria listed in Table~\ref{table:partitions}.
For each distribution, the corresponding thresholds separate the distribution into distinct partitions that map vehicle states to labels.
E.g. acceleration thresholds $\Theta_a$ separate the acceleration distribution $D_a$ into partitions: Decelerate, Maintain Speed, and Accelerate.
The full list of partitions and the logical rules are included in  Appendix~\ref{sec:appendix_partition_rules}.
% For the acceleration distribution $D_a$, acceleration thresholds $\Theta_a = \{\theta_{a,\text{acc}},\theta_{a,\text{dec}}\}$ separate the data into three partitions shown in Table~\ref{table:partitions}: decelerate $D_{a,\text{dec}}$, maintain speed $D_{a,\text{ms}}$, and accelerate $D_{a,\text{acc}}$.
% Likewise, velocity thresholds $\Theta_v = \{\theta_{v,\text{stop}},\theta_{v,\text{slow}},\theta_{v,\text{med}}\}$ separate $D_v$ into stopped, slow, medium, and fast partitions, and yaw rate thresholds $\Theta_{\omega} = \{\theta_{\omega,\text{str}},\theta_{\omega,\text{grad}},\theta_{\omega,\text{med}}\}$ separate $D_{\omega}$ into straight, gradual turn, medium turn, and aggressive turn partitions.
Each sample that falls into the same partition is given the same SDL label.
% For each partition, we can measure the similarity of two samples within a partition $p1,p2 \in D_{\text{part}}$ as the distance between them: $|p1-p2|$.
The distance between two samples within a partition $p1,p2 \in D_{\text{part}}$ is the absolute value of their difference $|p1-p2|$ as a measure of their similarity.
% The similarity of all data within a partition is calculated as the average pairwise distance $\mu_{\text{part}} = \frac{\sum_{p1,p2 \in D_{\text{part}}}|p1-p2|}{|D_{\text{part}}|(|D_{\text{part}}|-1)}$.
% The similarity of all samples within a partition is the average pairwise distance $\mu_{\text{part}} = \frac{\sum_{p1,p2 \in D_{\text{part}}}|p1-p2|}{{|D_{\text{part}}| \choose 2}}$.

For each partition, $D_{\text{part}}$, the intra-partition similarity is measured using the average pairwise distance: 
\begin{align}
  \mu_{\text{part}} = \frac{\sum_{p1,p2 \in D_{\text{part}}}|p1-p2|}{{|D_{\text{part}}| \choose 2}}.   
\end{align}
%$\mu_{\text{part}} = \frac{\sum_{p1,p2 \in D_{\text{part}}}|p1-p2|}{{|D_{\text{part}}| \choose 2}}$. 
This value is calculated as the sum of all pairwise distances between samples within the partition, divided by the total number of sample pairs. 
% calculated as the sum of all pairwise distances between samples within the partition divided by the number of pairs in the partition.
%The average pairwise distance $\mu_{\text{part}}$ is a measure of the intra-partition similarity where lower values that samples within the partition are more similar to each other.
A lower value for $\mu_{\text{part}}$ indicates higher similarity among the samples in that partition.
% Lower values of $\mu_{\text{part}}$ indicate that the samples within the partition are more similar, and higher values indicate they are less similar.
Given two partitions $\text{part1},\text{part2} \in D$ we can now define the inter-partition similarity as $\mu_{\text{part1}}-\mu_{\text{part2}}$.
% We define $J(\Theta) = \sum_{part1,part2 \in D}||\mu_{\text{part1}}-\mu_{\text{part2}}||^2$ as the sum of squared distances between the similarity of samples within each partition of $D$.
% This is the separation objective function that measures whether the samples within each partition are equally similar.
% This is a separation objective function that measures similarity of partitions.
Therefore, objective function
\begin{align}
   J(\Theta) = \sum_{part1,part2 \in D} ||\mu_{\text{part1}}-\mu_{\text{part2}}||^2 
\end{align}
%$J(\Theta) = \sum_{part1,part2 \in D} ||\mu_{\text{part1}}-\mu_{\text{part2}}||^2$ 
computes the squared inter-partition distances that measure the similarity between different partitions.
Minimizing the value of $J(\Theta)$ means that each of the partitions will be equally similar.
% $J(\Theta)$ is low for similar partitions of $D$, and high for dissimilar partitions.
Our goal then is to find thresholds that minimize the value of $J(\Theta)$
\begin{equation}
    \Theta^{*} = \argmin_{\Theta} J(\Theta)
    \label{eqn:cem_objective}
\end{equation}

The benefit of analyzing lateral and longitudinal distributions separately, is that the thresholds $\Theta_{\omega}$, $\Theta_a$, and $\Theta_v$ can be minimized independently.
% As $\Theta$ comprises $\Theta_{\omega}$, $\Theta_a$, and $\Theta_v$, the objective function $J(\Theta)$ can be minimized independently for each set of thresholds to calculate $\Theta^{*}_{\omega}$, $\Theta^{*}_a$, and $\Theta^{*}_v$.
% Since we use separate acceleration $\Theta_a$, velocity $\Theta_v$, and yaw rate $\Theta_{\omega}$ thresholds that each operate on their own data distribution, we can minimize the objective junction $J(\Theta)$ independently for each set of thresholds to calculate $\Theta^{*}_a$, $\Theta^{*}_v$, and $\Theta^{*}_{\omega}$.
For instance, $\Theta^{*}_{a} = \argmin_{\Theta_{a}} J(\Theta_{a})$.

% \begin{table}
% \centering
% \begin{tabular}{ |c|c|c| }
%  \hline
%  Condition & Partition & Notation \\
%  \hline
%  $|\bar{\omega}| < \theta_{\omega,\text{str}}$ & Straight & $D_{\omega,\text{str}}$ \\
%  $\theta_{\omega,\text{str}} < |\bar{\omega}| < \theta_{\omega,\text{grad}}$ & Gradual Turn & $D_{\omega,\text{grad}}$ \\
%  $\theta_{\omega,\text{grad}} < |\bar{\omega}| < \theta_{\omega,\text{med}}$ & Medium Turn & $D_{\omega,\text{med}}$ \\
%  $\theta_{\omega,\text{med}} < |\bar{\omega}|$ & Aggressive Turn & $D_{\omega,\text{aggr}}$ \\
%  \hline
%  $\bar{a} < \theta_{a,\text{dec}}$ & Decelerate & $D_{a,\text{dec}}$ \\
%  $\theta_{a,\text{dec}} < \bar{a} < \theta_{a,\text{acc}}$ & Maintain Speed & $D_{a,\text{maintain}}$ \\
%  $\theta_{a,\text{acc}} < \bar{a}$ & Accelerate & $D_{a,\text{acc}}$ \\
%  \hline
%  $\bar{v} < \theta_{v,\text{stop}}$ & Stopped & $D_{v,\text{stop}}$ \\
%  $\theta_{v,\text{stop}} < \bar{v} < \theta_{v,\text{slow}}$ & Slow & $D_{v,\text{slow}}$ \\
%  $\theta_{v,\text{slow}} < \bar{v} < \theta_{v,\text{med}}$ & Medium & $D_{v,\text{med}}$ \\
%  $\theta_{v,\text{med}} < \bar{v}$ & Fast & $D_{v,\text{fast}}$ \\
%  \hline
% \end{tabular}
% \caption{The conditions for data partitions. For example, vehicle acceleration is separated by $\theta_{a,\text{acc}}$ and $\theta_{a,\text{dec}}$ into partitions of decelerate, maintain speed, and accelerate. Different thresholds are used for each data distribution with no overlap, allowing the data distributions to be analyzed separately.}
% \label{table:partitions}
% \end{table}


% \begin{table}
% \centering
% \renewcommand{\arraystretch}{1.3}
% \begin{tabular}{ |c|c|c| }
%  \hline
%  \rowcolor[gray]{0.9} \textbf{Logical Rule} & \textbf{Partition} & \textbf{Notation} \\
%  \hline\hline
%  \multicolumn{3}{|c|}{\textbf{Yaw Rate ($\bar{\omega}$)}} \\
%  \hline
%  $|\bar{\omega}| < \theta_{\omega,\text{str}}$ & Straight & $D_{\omega,\text{str}}$ \\
%  $\theta_{\omega,\text{str}} < |\bar{\omega}| < \theta_{\omega,\text{grad}}$ & Gradual Turn & $D_{\omega,\text{grad}}$ \\
%  $\theta_{\omega,\text{grad}} < |\bar{\omega}| < \theta_{\omega,\text{med}}$ & Medium Turn & $D_{\omega,\text{med}}$ \\
%  $\theta_{\omega,\text{med}} < |\bar{\omega}|$ & Aggressive Turn & $D_{\omega,\text{aggr}}$ \\
%  \hline
%  \multicolumn{3}{|c|}{\textbf{Acceleration ($\bar{a}$)}} \\
%  \hline
%  $\bar{a} < \theta_{a,\text{dec}}$ & Decelerate & $D_{a,\text{dec}}$ \\
%  $\theta_{a,\text{dec}} < \bar{a} < \theta_{a,\text{acc}}$ & Maintain Speed & $D_{a,\text{maintain}}$ \\
%  $\theta_{a,\text{acc}} < \bar{a}$ & Accelerate & $D_{a,\text{acc}}$ \\
%  \hline
%  \multicolumn{3}{|c|}{\textbf{Velocity ($\bar{v}$)}} \\
%  \hline
%  $\bar{v} < \theta_{v,\text{stop}}$ & Stopped & $D_{v,\text{stop}}$ \\
%  $\theta_{v,\text{stop}} < \bar{v} < \theta_{v,\text{slow}}$ & Slow & $D_{v,\text{slow}}$ \\
%  $\theta_{v,\text{slow}} < \bar{v} < \theta_{v,\text{med}}$ & Medium & $D_{v,\text{med}}$ \\
%  $\theta_{v,\text{med}} < \bar{v}$ & Fast & $D_{v,\text{fast}}$ \\
%  \hline
% \end{tabular}
% \caption{
% The rule-based criteria for data partitions. For example, vehicle acceleration is separated by $\theta_{a,\text{acc}}$ and $\theta_{a,\text{dec}}$ into partitions of decelerate, maintain speed, and accelerate. Different thresholds are used for each data distribution with no overlap, allowing the data distributions to be analyzed separately.
% }
% \label{table:partitions}
% \end{table}

\begin{comment}
\begin{table}
\centering
\renewcommand{\arraystretch}{1}
\begin{tabular}{ |c|c| }
 \hline
 \rowcolor[gray]{0.9} \textbf{Logical Rule} & \textbf{Partition} \\
 \hline\hline
 \multicolumn{2}{|c|}{\textbf{Yaw Rate ($\bar{\omega}$)}} \\
 \hline
 $|\bar{\omega}| < \theta_{\omega,\text{str}}$ & Straight \\
 $\theta_{\omega,\text{str}} < |\bar{\omega}| < \theta_{\omega,\text{grad}}$ & Gradual Turn \\
 $\theta_{\omega,\text{grad}} < |\bar{\omega}| < \theta_{\omega,\text{med}}$ & Medium Turn \\
 $\theta_{\omega,\text{med}} < |\bar{\omega}|$ & Aggressive Turn \\
 \hline
 \multicolumn{2}{|c|}{\textbf{Acceleration ($\bar{a}$)}} \\
 \hline
 $\bar{a} < \theta_{a,\text{dec}}$ & Decelerate \\
 $\theta_{a,\text{dec}} < \bar{a} < \theta_{a,\text{acc}}$ & Maintain Speed \\
 $\theta_{a,\text{acc}} < \bar{a}$ & Accelerate \\
 \hline
 \multicolumn{2}{|c|}{\textbf{Velocity ($\bar{v}$)}} \\
 \hline
 $\bar{v} < \theta_{v,\text{stop}}$ & Stopped \\
 $\theta_{v,\text{stop}} < \bar{v} < \theta_{v,\text{slow}}$ & Slow \\
 $\theta_{v,\text{slow}} < \bar{v} < \theta_{v,\text{med}}$ & Medium \\
 $\theta_{v,\text{med}} < \bar{v}$ & Fast \\
 \hline
\end{tabular}
\caption{
The rule-based criteria for data partitions. For instance, vehicle acceleration is separated by $\theta_{a,\text{acc}}$ and $\theta_{a,\text{dec}}$ into partitions of decelerate, maintain speed, and accelerate. %Different thresholds are used for each data distribution with no overlap, allowing the data distributions to be analyzed separately.
}
\label{table:partitions}
\end{table}
\end{comment}


\subsection{SDL Similarity Search Problem}
\label{sec:problem_sim}

Given a set of labeled trajectories $S^L$ and a reference label $L_{ref} = \langle \{L_{\text{lat},1},L_{\text{lat},2},\dots\},\{L_{\text{long},1},L_{\text{long},2},\dots\} \rangle$, the objective of the SDL similarity search problem is to find the set of labeled trajectories $S^L_{\text{ref}} \subset S^L$ that are similar to the reference label.
% Label $L^i_j \in S^L$ describes the behavior of vehicle $V^i_j$, and will be derived from $f_{\Theta}$ given optimal values of $\Theta^{*}$.

% Once we have determined values for the thresholds $\Theta$, we can perform automated SDL extraction as defined by Equation~\ref{eqn:sdl_extraction} to produce $S^L_i = \{L^i_1,L^i_2,\dots\}$ the set of SDL labels for each trajectory in a scenario $S_i$.
% Doing this for all scenarios produces $S^L$ the set of all labeled trajectories.

% From the set of labeled scenarios $S^L$, we can choose one label to be a reference label $L_{\text{ref}}$. 
% Given a distance function $g(L_{\text{ref}},L^i_j) = d$ where $d \in \mathbb{R}$ is a measurable scalar distance, 
% The similarity between a chosen reference label and some other label $L^i_j$, is calculated using the distance function $g(L_{\text{ref}},L^i_j) = d$ where $d \in \mathbb{R}$.
% Any function that compares labels and provides a measure of distance between them can be used for $g()$.

The similarity between SDL labels is given by the distance function $g(L_{\text{ref}},L^i_j) = d$ where $d \in \mathbb{R}$ is a scalar distance and $L^i_j$ is the SDL label that describes the behavior of vehicle $V^i_j$.
% Given a reference label $L_{\text{ref}}$ and a distance function $g(L_{\text{ref}},L^i_j) = d$ where $d \in \mathbb{R}$ is a measurable scalar distance, the goal of a similarity search is to find the set $S^L_{\text{ref}} \subset S^L$ containing all labels whose distance to the reference label is below a similarity threshold $d_{\text{sim}}$.
Therefore, the similarity search problem is to construct the set
\begin{equation}
    S^L_{\text{ref}} = \{L^i_j: g(L_{\text{ref}},L^i_j) < d_{\text{sim}}\}
    \label{eqn:sim_search}
\end{equation}
where all labels $L^i_j$ are within a distance $d_{\text{sim}}$ of the reference label.
Setting $d_{\text{sim}} = 0$ requires that all similar samples $L^i_j \in S^L_{\text{ref}}$ have the exact same lateral and longitudinal label as the reference.
% For this work, we have set $d_{\text{sim}} = 0$ requiring that all similar samples have the exact same label as the reference.

\textbf{Identifying Unique Scenarios}
The similarity search can also automatically identify unique scenarios in a dataset.
Here, a unique scenario is a scenario that is not encountered anywhere else in the dataset.
Unique behavior can be identified from unique SDL labels. 
% For a given reference label $L_{\text{ref}}$, it is unique if the set of similar labels is empty $S^L_{\text{ref}} = \{\emptyset\}$.
It follows from Equation~\ref{eqn:sim_search} that if the set of similar labels $S^L_{\text{ref}} = \{\emptyset\}$ is empty, then the reference label itself is unique, indicating that the vehicle's behavior is found nowhere else in the dataset.
% If a label is not similar to anything else, it indicates that the vehicle is exhibiting behavior found nowhere else in the dataset, and thus part of a unique scenario.
