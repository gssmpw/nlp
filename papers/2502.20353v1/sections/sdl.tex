\label{sec:sdl}

This paper uses an SDL as a high-level embedding for vehicle behavior.
Based off the SDL described by~\cite{aron_date24}, we use an SDL that describes vehicle behavior as a closed set of actions.
Figure~\ref{fig:action_spectrum} shows the actions that we use for our SDL.
The actions that we use for our SDL is not a complete list of every possible action, but a generic set that covers a wide range of behaviors for vehicle analysis.

\begin{figure}
    \centering
    \includegraphics[width=1.0\linewidth]{figures/ActionSpectrum.pdf}
    \caption{The actions that make up our SDL. These actions are defined on a spectrum of complexity. At the most abstract end, there are three lateral and three longitudinal actions. These abstract actions can be separated into multiple fine-grained actions as the spectrum becomes more refined.}
    \label{fig:action_spectrum}
\end{figure}

For our SDL, we chose to separate vehicle actions into categories of lateral and longitudinal behavior.
We chose to keep these separate because the vehicle's lateral behavior has little effect on its longitudinal behavior and vice versa.
Laterally, a vehicle can perform actions such as Turn or Merge (changing lanes), or driving Straight.
Longitudinally, a vehicle can be Stopped, or traveling at different speeds such as Slow, Medium, and Fast. The vehicle can also be Accelerating, Decelerating, or Maintaining its current speed.
These lateral and longitudinal labels can be performed concurrently in any combination, so we chose to keep them separate.

In addition, we have defined our SDL actions on a spectrum of complexity from abstract to refined.
This makes them flexible to be used in many different circumstances.
Some dataset may contain a lot of detailed data to allow for very refined actions, but some may only have sparse data and can only support the more abstract definitions.
Similarly, the complexity of the downstream task may require a certain level of refinement.
By defining these elements on a spectrum, they can be used across a variety of datasets and downstream tasks.

At the most abstract end of the spectrum, known as the Trace level, the lateral actions are Left Turn, Right Turn, and Straight, while the longitudinal actions are Accelerate, Maintain Speed, and Decelerate.
At the Trend level of refinement, the Maintain Speed action is split into cases where the vehicle is Stopped versus Maintaining a positive velocity.
The Maneuver level of refinement identifies Merges apart from Turns.
At the most refined Action level, turns are divided into Gradual, Medium, and Aggressive based on the how quickly the vehicle is turning.
The Action level also separates longitudinal maneuvers based on the speed of the vehicle.
For this paper, we will be using SDL actions from the most refined Action end of the spectrum.

The SDL actions defined above can be used to provide a Label $L = \langle \{l^{Lat}_1,l^{Lat}_2\,\dots,l^{Lat}_n\} , \{l^{Long}_1,l^{Long}_2,\dots,l^{Long}_m\} \rangle$ that describes the behavior of a vehicle.
Each label consists of an ordered list of SDL actions that describe the vehicle's lateral and longitudinal actions in the order they were performed.
Each action in our label must last for at least 1 second in order to detect intent from the driver. actions of less than 1 second are outliers or a driver making small adjustments rather than intentionally taking an action.
Therefore, we are only interested in actions lasting at least 1 second.
