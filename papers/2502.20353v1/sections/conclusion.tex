% This paper presents the Trajectory-to-Action Pipeline (TAP), a method for automatically extraction SDL labels from trajectory data.
% TAP extracts SDL labels using automata that are based on thresholds that partition the data.
% These thresholds can be automatically determined from the data, making TAP flexible for different datasets.
% We also show that the SDL labels produced by TAP can be used to find similar vehicle behavior from a dataset of trajectories.
% TAP is more effective than traditional trajectory similarity metrics for finding similar behavior because it encodes actions instead of comparing paths.
% The SDL labels can also be used for automatic identification of scenarios of interest by detecting unique labels.
% Future work includes detecting similar scenarios by analyzing multiple vehicles at once, and improving the SDL distance metric by looking at edit distance and incorporating action duration.

% This paper introduced the Trajectory-to-Action Pipeline (TAP), a method for automatically extracting interpretable SDL labels from trajectory data using automata driven by data-derived thresholds. 
% TAP extracts SDL labels using rule-based automata whose thresholds are determined using cross-entropy optimization.
% % Using the Waymo Open Motion Dataset for validation, we demonstrated that TAP consistently identifies behaviorally similar trajectories with 30\% greater precision than traditional distance-based metrics like Average Displacement Error. 
% Using the Waymo Open Motion Dataset for validation, we demonstrated that TAP consistently identifies behaviorally similar trajectories with 30\% greater precision than Average Displacement Error and 24\% greater precision than Dynamic Time Warping. 
% Additionally, TAP enables the automatic detection of unique scenarios within a large dataset.
% Although, we only demonstrate TAP on trajectory similarity, this advancement has the potential to enable more accurate behavior analysis and scenario identification using SDLs. 
% Future work will explore multi-vehicle interactions and further refining the SDL distance metric using edit distance and action duration.

This paper introduced the Trajectory-to-Action Pipeline (TAP), a method for automatically extracting interpretable Scenario Description Language (SDL) labels from trajectory data. TAP employs rule-based automata with data-derived thresholds, optimized using cross-entropy to generate structured, hierarchical labels. Using the Waymo Open Motion Dataset for validation, TAP demonstrated a 30\% improvement in precision over Average Displacement Error (ADE) and a 24\% improvement over Dynamic Time Warping (DTW) in identifying behaviorally similar trajectories. TAP also enables the automatic detection of unique scenarios within large datasets.
Additionally, TAP enables the automatic detection of unique scenarios within a large dataset.
Despite these contributions, there are key areas for future improvements. 
Currently, TAP focuses solely on vehicle trajectories, without integrating static environmental elements (e.g., road geometry) or accounting for interactions between multiple agents.
Future work would enhance TAP by incorporating these to support more complex scenario comparisons.
Additionally, while TAP introduces its own SDL schema, adapting the method to established frameworks like Scenic or OpenScenario could improve interoperability with existing testing and simulation tools.
Lastly, TAP could be augmented using recent transformer-based architectures for encoding trajectories.
Ultimately, TAP lays a strong foundation for automating behavior-based analysis in AV testing and safety evaluation, with potential for significant extensions to better reflect real-world scenario complexity.


