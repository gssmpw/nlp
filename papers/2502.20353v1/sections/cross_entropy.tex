\subsection{Optimizing Thresholds via Cross-Entropy}

To effectively partition vehicle behaviors into SDL labels, TAP requires the determination of optimal thresholds $\Theta$ for yaw rate $\omega$, acceleration $a$, and velocity $v$. 
These thresholds enable TAP to classify trajectories into distinct behavioral categories. 
In this section we describe how these thresholds are learned from the data distribution using a cross-entropy method.

The challenge lies in setting thresholds that can balance the partitions, ensuring that data points within the same partition are more similar to each other than to those in other partitions. 
As stated in Equation~\ref{eqn:cem_objective}, this is formulated as an optimization problem where the objective function  $J(\Theta)$ measures the partition balance.


