We use the Waymo Open Motion Dataset (WOMD)~\cite{WOMD_2021} as our source of scenarios \( S \). 
% WOMD consists of 574 hours of vehicle data across 486,995 scenarios, each recorded at 10 Hz over a duration of \( T = 9.0 \) seconds. 
TAP is evaluated on a representative subset of 492 scenarios from WOMD, containing $25,889$ vehicle trajectories, each recorded at 10 Hz over a duration of \( T = 9.0 \) seconds.
%The number of scenarios is limited to reduce the search space for comprehensive similarity comparisons.
Each scenario \( S_i \) includes data from several vehicles \( V^i_j \) including their states \(\{x, y, v, \phi\}\), where \(x, y\) denote the vehicle's position in map coordinates, \( v \) represents its velocity in m/s, and \( \phi \) is the heading in radians.
To obtain the complete state vector \( s = \{x, y, v, a, \phi, \omega\} \), we derive acceleration \( a \) and yaw rate \( \omega \) from the recorded velocity and heading. 
Each vehicle’s trajectory is normalized to a common initial condition by translating positions \( x, y \) to the origin and adjusting the heading to \( \phi = 0 \) radians, enabling fair trajectory comparisons.
Using the full state vector \( s = \{x, y, v, a, \phi, \omega\} \), we construct 1-second average distributions \( D_a \), \( D_v \), and \( D_{\omega} \) (Section~\ref{sec:problem_threshold}), which serve as the foundation for TAP’s automatic threshold computation.
WOMD trajectories do not contain ground truth action labels. 
 