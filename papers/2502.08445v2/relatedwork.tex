\section{Related Work}
We first introduce the three most related research directions. 

\paragraph{Additive Models}
Model-agnostic methods, such as Partial Dependence~\citep{friedman2001greedy}, SHAP~\citep{lundberg2017unified}, and LIME~\citep{ribeiro2016should}, offer a standardized approach to explaining machine learning predictions. However, when applied to deep neural networks (DNNs), these methods may fail to provide faithful representations of their full complexity~\citep{rudin2019stop}. A more transparent alternative involves leveraging Generalized Additive Models (GAMs)~\citep{hastie2017generalized}, where the response variable \( y \) is modeled using an additive structure:  
\begin{small}
\begin{equation}
E[y | c_1, . . . , c_N] = h(\beta_0 + f_1(c_1) + \dots + f_N(c_N))
\label{eq.nam}
\end{equation}
\end{small}
where \( h(\cdot) \) is the inverse of the link function (a form of activation function) ; $\beta_0$ denotes the intercept and $f_{i}(\cdot)$ represent independent functions for the $i^{th}$ covariate. Neural Additive Models (NAMs)~\citep{agarwal2020neural, jiao2023naisr} build upon this framework, offering enhanced interpretability while maintaining the flexibility of neural networks. Specifically, for NAMs the functions $f_i(\cdot)$ are deep neural networks. NAISR~\citep{jiao2023naisr} pioneers the use of NAMs to capture spatial deformations with respect to an estimated atlas shape that is modulated by covariates. \emph{\texttt{LucidAtlas} extends this concept by integrating NAMs to construct an atlas that captures population trends and uncertainties with spatial dependencies}.

\paragraph{Epistemic Uncertainty versus Aleatoric Uncertainty}
%Epistemic and aleatoric uncertainties are two different kinds of uncertainties. Epistemic uncertainty relates to model parameters and stems from limited model knowledge, which is reducible with more data or better modeling. Aleatoric uncertainty arises from inherent data randomness and is irreducible. 

Estimating uncertainty is important to understand the quality of a model fit and to capture variations across the data population. Two different types of uncertainties need to be distinguished: epistemic uncertainty captures model uncertainty whereas aleatoric uncertainty captures uncertainty in the data~\citep{hullermeier2021survey}. 

More attention is generally paid to epistemic uncertainties in the context of interpretable models~\citep{wang2025uncertaintysurvey}. NAMs used ensembling to estimate model uncertainties~\citep{agarwal2020neural}. LA-NAM used a Laplace approximation for uncertainty estimation~\citep{bouchiat2023lanam} with NAMs. In atlas construction, aleatoric uncertainty is especially important when individual differences in a dataset are large. Capturing aleatoric uncertainty is crucial in medicine to understand population variations. NAMLSS~\citep{thielmann2024namlss} can model aleatoric uncertainty by using NAMs to approximate the parameters $\{\theta^{(n)}\}$ of a chosen data distribution~\citep{thielmann2024namlss}, as
\begin{small}
\begin{equation}
\theta^{(n)}=h^{(n)}\left(\beta^{(n)}+\sum_{i=1}^{N} f_{i}^{(n)}\left(c_{i}\right)\right) 
\label{eq.namlss}
\end{equation}
\end{small}
where $\theta^{(n)}$ can for example be the mean and variance of Gaussian distributions; $\beta^{(n)}$ denotes the parameter-specific intercept and $f_{i}^{(n)}$ represents the feature network for the $n$-th parameter for the $i$-th feature. \emph{\texttt{LucidAtlas} extends NAMLSS to a more versatile representation, enabling individualized prediction, incorporating prior knowledge, and capturing spatial dependencies.}
%~\citep{monteiro2020stochasticsegmentationnetworksmodelling,valiuddin2021improvingaleatoricuncertaintyquantification,Wang_2019,kawashima2020aleatoricuncertaintyestimationusing,Martin_2022,nevin2024deepuqassessingaleatoricuncertainties}

\paragraph{Heteroscedasticity versus Homoscedasticity}
Distinguishing between homoscedasticity and heteroscedasticity is crucial in statistical analysis, especially for regression models. Homoscedasticity indicates constant variance of random variables , whereas heteroscedasticity indicates that the variance of random variables may differ~\citep{wooldridge2016homohetr}. For example, when modeling airway cross-sectional area the population variance may change (increase) with age. \texttt{LucidAtlas} assumes and supports estimating heteroscedasticity with respect to different locations in an anatomical region and with respect to covariates across a patient population. Many interpretable approaches assume homoscedasticity, e.g., OAK-GP~\citep{lu2022oak} and LA-NAM~\citep{bouchiat2023lanam} assume homoscedasticity in their additive networks. To our knowledge, only NAMLSS considers heteroscedasticity in its additive network design~\citep{thielmann2024namlss}. However, NAMLSS interprets individual covariate effects and uncertainties in isolation resulting, as we will see, in difficulties for data interpretation. \emph{\texttt{LucidAtlas} advances beyond NAMLSS by capturing spatial heteroscedasticity and incorporating covariate dependencies via a marginalized covariate interpretation approach.}%\emph{\texttt{LucidAtlas} can capture spatial heteroscedasticity and utilize heteroscedasticity for marginalized covariate interpretation, which takes a further step to NAMLSS.} 

Table~\ref{tab.lit} compares \texttt{LucidAtlas} to related interpretable models with respect to the discussed properties above. A more comprehensive discussion of
related work is available in Sec.~\ref{supp.sec.supp_related_work} of the Supplementary Material.