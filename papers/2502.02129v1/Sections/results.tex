\section{Experiments}\label{sec:results}

\subsection{Experiment setup}


\paragraph{Objectives and scope.}
Our experiments aim to: \textbf{(1)} validate the effectiveness of the NeuralCPM training algorithm for fitting Hamiltonians to data, and \textbf{(2)} evaluate the capability of using a Neural Hamiltonian to model cellular dynamics and self-organization in synthetic and real-world scenarios. To the best of our knowledge, no prior methods have been developed for learning an energy function to model cellular dynamics. Hence, we compare our approach to alternative neural network architectures for the Hamiltonian as well as analytical CPMs. Details on experiments and datasets are in Appendix~\ref{sec:app-data-gen}.

\paragraph{Experimental scenarios and datasets.}

To address objective (1), we generate synthetic datasets using the cell-sorting Hamiltonian of Equation~\ref{eq:hamiltonian-glazier}~\cite{Graner1992}, and measure how well the training algorithm can learn the parameters in this analytical function from data. Different parameterizations lead to different dynamics, and we generate data following the type A and type B regimes as illustrated in~\cite{edelstein2023simplecellsort} and Figure~\ref{fig:data_examples}.

\begin{figure}[t]
    \centering
    \includegraphics[width=0.9\linewidth]{Figures/data_examples/exp0_data.png}\\
    \includegraphics[width=0.9\linewidth]{Figures/data_examples/real_cellular_mnist.png}\\
    \includegraphics[width=0.9\linewidth]{Figures/data_examples/exp_2_real_and_synth_example.png}\\
    \caption{Data used in the experiments. Top row: example datapoints of cell sorting type A (leftmost two images) and B (rightmost two images), as illustrated in~\citet{edelstein2023simplecellsort}.
    Middle row: example datapoints from the Cellular MNIST dataset.
    Bottom row: example datapoints of~\citet{Toda2018Science} (leftmost two images) and synthetic counterparts (rightmost two images). The synthetic counterparts are used for training, after which we validate the model against the real-world data of~\citet{Toda2018Science}.}
    \label{fig:data_examples}
\end{figure}

To address objective (2), we consider two experimental scenarios. For the first scenario, we introduce the \emph{Cellular MNIST} dataset: a synthetic dataset in which cells form digit-like structures, also illustrated in Figure~\ref{fig:data_examples}. The motivation behind this scenario is that these structures are too complex to model with an analytical Hamiltonian, that can only capture low-level structures between neighboring cells. 


The second scenario for objective (2) concerns a biological experiment by~\citet{Toda2018Science}.
A hallmark during embryo development is the self-organization of the principal body axis from an unstructured group of cells, which can be recapitulated with in-vitro experiments and quantified using time-lapse microscopy~\cite{Toda2018Science}. 
Here, we choose the observation of a bi-polar axis formation in cell aggregates as shown in Figure~\ref{fig:intro-figure} and refer to this scenario as \emph{bi-polar axial organization}.
This behavior is surprising because the cells of two different types, expressing (after induction) different type-specific P-cadherin or N-cadherin adhesion molecules, were expected to sort into a concentric or uni-polar configuration~\cite{Graner1992}.

As~\citet{Toda2018Science} performed only six repetitions of this experiment, we construct synthetic counterparts of the final configurations for training using Morpheus~\cite{starruss2014morpheus} by prescribing the target location of each cell for a bi-polar arrangement. After training, we validate the cellular dynamics predicted by NeuralCPM against the real biological dynamics reported in the Supplemental Figure S6B of~\citet{Toda2018Science}. 








\subsection{Fitting analytical Hamiltonians}\label{sec:exp0}



\paragraph{Metrics and baselines.} To assess how well our learning algorithm can fit analytical Hamiltonians to data, we measure the Root Mean Squared Error (RMSE) of the learned parameters of the Hamiltonian. Since the temperature parameter is mainly related to fluctuations in the system over time, it is poorly identifiable from static snapshots. Hence, we report the RMSE for both temperatures $T=1$ and $T=T^*$, where $T^*$ is the temperature that minimizes the RMSE between the learned coefficients and the ground truth. 


\paragraph{Results.} 
The learned parameters approximate the ground-truth very well: for type A cell sorting, the training algorithm achieves a RMSE of 0.055 and 0.021 for $T=1$ and $T=T^*$ respectively, while for type B, the RMSE is 0.997 ($T=1$) and 0.178 ($T=T^*$). In addition, as can be seen in Figure~\ref{fig:exp0-param-converge}, the parameters converge rapidly to the true values, highlighting the efficiency of the learning algorithm. Additional results can be found in Appendix~\ref{sec:app-exp-0-more-results}.



\begin{figure}[tb]
    \centering
    \includegraphics[width=0.75\linewidth]{Figures/experiment_0/exp0_param_convergence.pdf}    
    \caption{Convergence of the parameters for Type B cell sorting. Dashed lines indicate the true values, solid lines indicate the learned values ($T=T^*$) over the course of training.}
    \label{fig:exp0-param-converge}
\end{figure}


\subsection{Cellular MNIST}\label{sec:exp1}


\begin{figure}[tb]
    \centering
    \includegraphics[width=\linewidth]{Figures/experiment_1/traj_exp1.pdf}
    \caption{Qualitative results for dynamics simulated by CPMs with varying Hamiltonian models trained on Cellular MNIST data.}
    \label{fig:exp-1-qualitative}
\end{figure}



\paragraph{Metrics and baselines.}
To assess the simulated dynamics, we consider both the cell and collective perspectives. From an individual cell point of view, we require cells to have a realistic volume and to be contiguous. Consequently, we measure $p_\text{volume}$, the proportion of states in which all cells exceed the lowest and highest measured cell volumes in the training data by at most 10\% of the mean cell volume, and $p_\text{unfragmented}$, the proportion of states in which there are at most three fragmented cells. We allow for three fragmented cells as CPMs allow for small temporary fragmentations. At the collective scale, our goal is to assess whether cells successfully organize into digit-like structures. Following the Inception Score~\cite{Salimans2016IS}, a metric used to quantify the quality of generative models for natural images, we calculate a \textit{Classifier Score} (CS) using a classifier $p_\phi(y | x)$ that we trained on the cellular MNIST dataset:
\begin{gather}
\begin{aligned}
    CS &= \exp\left(\mathbb{E}_{x \sim p_\theta(x)}\left[KL\right]\right),\\
    KL &= D_{KL}\left( p_\phi(y | x) || \mathbb{E}_{x' \sim p_\theta(x')} \left[p_\phi(y | x')\right]  \right).
\end{aligned}
\end{gather}
High $CS$ indicates distinct and diverse cellular structures.

As baselines, we compare against two analytical models as well as neural network based Hamiltonians. The analytical models are the prototypical cell sorting Hamiltonian~\cite{Graner1992} (Equation~\ref{eq:hamiltonian-glazier}) and the cell sorting Hamiltonian plus a learnable external potential. The goal of the neural network baselines is to evaluate the design choices in the Neural Hamiltonian. To this end, we consider (1) a Convolutional Neural Network (CNN); (2) a 1-layer Neural Hamiltonian followed by invariant pooling over representations of cells and a CNN; (3) the vanilla Neural Hamiltonian as illustrated in Figure~\ref{fig:neural-hamiltonian}; and (4) a Neural Hamiltonian as closure on top of a cell sorting Hamiltonian with learnable parameters. Models (1) and (2) serve to investigate the importance of deep equivariant embeddings over architectures relying on representations without permutation symmetry (1) or on invariant representations (2), while comparing (3) and (4) gives insight on the relevance of including biological knowledge in the Hamiltonian. Details on the model architectures can be found in Appendix~\ref{sec:app-all-model-details}.

\paragraph{Results.}

\begin{table}[t]
\centering
\footnotesize
\setlength{\tabcolsep}{1pt}
\caption{Results on Cellular MNIST data. $p_\text{volume}$ and $p_\text{unfragmented}$ assess the validity of the dynamics at the cell level from a biological perspective, while CS assesses to what extent cells successfully assemble in distinct digit-like structures.}
\vskip 0.15in
\label{tab:exp-1-metrics}
\begin{tabular}{@{}cccc@{}}
\toprule
Model                                    & $p_\text{volume}$ $\uparrow$ & $p_\text{unfragmented}$ $\uparrow$ & CS $\uparrow$ \\ \midrule
Cellsort Hamiltonian                     & \textbf{1.00}                         & \textbf{1.00}                                & 2.47          \\
Cellsort + External Potential & \textbf{1.00}                         & \textbf{1.00}                               & 3.11          \\
CNN                                      & 0.00                         & 0.05                               & 3.70          \\
1 NH layer + CNN         & 0.06                         & 0.87                               & 3.53          \\
Neural Hamiltonian                       & 0.11                         & 0.99                               & \textbf{4.91}          \\
Neural Hamiltonian + closure             & \textbf{1.00}                          & \textbf{1.00}                                & 4.35         \\ \bottomrule
\end{tabular}
\end{table}

Figure~\ref{fig:exp-1-qualitative} shows qualitative results of simulated trajectories starting from a mixed cell cluster configuration; more visualizations can be found in Appendix~\ref{sec:app-qualitative-all}. As expected, the analytical models are not able to capture complex non-linear relationships, reflected in the failure of cells to form a digit-like structure. The CNN Hamiltonian produces dynamics that are clearly unrealistic, because it lacks the inductive bias of permutation symmetry. In contrast, the architectures based on Neural Hamiltonians respect the symmetries of the system and lead to cells organizing in digit-like structures. 

In line with these qualitative results, Table~\ref{tab:exp-1-metrics} shows that the analytical models excel in the biological metrics $p_\text{volume}$ and $p_\text{unfragmented}$ due to the biology-informed design of their Hamiltonians, but achieve low $CS$ values as they are not expressive enough to model digit-like structures. From the Neural Hamiltonian models, NH achieves the highest CS score, which is substantially higher than the 1 NH layer + CNN model, stressing the relevance of deep equivariant representations. However, these models are subject to unsatisfactory biological metrics. In contrast, using the NH as a closure term yields high scores on the biological and $CS$ metrics, as it enjoys the strong biological structure of the analytical component to constrain the dynamics to be biologically realistic, while using the more expressive NH architecture to guide cells towards digit-like formations. 




\subsection{Bi-polar axial organization}\label{sec:exp2}

\paragraph{Metrics and baselines.}

As in Section~\ref{sec:exp1}, we evaluate simulations for biological consistency and collective behavior. For the biological metrics, we use the same indicators $p_\text{volume}$ and $p_\text{unfragmented}$. For the collective dynamics, we quantify the bi-polar axial organization as follows: first, we rotate the image along the principal axis of the polar cell clusters (green in Figure~\ref{fig:data_examples}). We then measure the variance of the spatial configuration of each cell type along this axis as well as along the orthogonal axis. We consider the same baselines as in Section~\ref{sec:exp1}.


\paragraph{Results.}


\begin{figure}[t]
    \centering
    \includegraphics[width=\linewidth]{Figures/experiment_2/traj_exp2-incl-toda.pdf}
    \caption{Biological dynamics observed in~\citet{Toda2018Science}, and qualitative results for dynamics simulated by CPMs with varying Hamiltonian models trained on bi-polar axial organization data.}
    \label{fig:exp-2-qualitative}
\end{figure}

\begin{table}[t]
\footnotesize
\setlength{\tabcolsep}{1pt}
\centering
\caption{Results on bi-polar axial organization for CPMs with different Hamiltonians. We use the same biological consistency indicators $p_\text{volume}$ and $p_\text{unfragmented}$ as in Table~\ref{tab:exp-1-metrics}, as well as the RMSE of the variance along the polar and orthogonal axes of the two cell types to quantify how well bi-polar axial organization is captured.}
\vskip 0.15in
\label{tab:exp-2-metrics}
\begin{tabular}{cccc}
\hline
Model                            & $p_\text{volume}$ $\uparrow$ & $p_\text{unfragmented}$ $\uparrow$ & \begin{tabular}[c]{@{}c@{}}Axial\\ alignment\\ RMSE \end{tabular} $\downarrow$ \\ \hline
Cellsort Hamiltonian             & \textbf{1.00}                & \textbf{1.00}                      & 147.2                                                                      \\
Cellsort + External Potential    & \textbf{1.00}                & \textbf{1.00}                      & 154.4                                                                       \\
CNN                              & 0.00                         & 0.07                               & 153.5                                                                       \\
1 NH layer + CNN & 0.00                         & 0.11                               & 329.1                                                                       \\
Neural Hamiltonian               & 0.00                         & 0.17                               & 254.2                                                                       \\
Neural Hamiltonian + closure     & 0.77                & \textbf{1.00}                               & \textbf{37.3}                                                               \\ \hline
\end{tabular}
\end{table}


Figure~\ref{fig:exp-2-qualitative} shows a microscopy time-lapse by~\citet{Toda2018Science}, as well as CPM-simulated trajectories for different Hamiltonians; more simulated trajectories can be found in Appendix~\ref{sec:app-qualitative-all}. As in Section~\ref{sec:exp1}, the analytical baselines fail to capture bi-polar structures due to their insufficient expressiveness, and the CNN-Hamiltonian produces distorted dynamics. However, in this case the Neural Hamiltonian-based models without closure term fail to produce reasonable dynamics, due to fast divergence of these models during training, a common issue of EBMs. In our Cellular MNIST experiment, we observed a similar tendency, but we were able to mitigate divergence by careful hyperparameter tuning, which we were not able to achieve for the bi-polar axial sorting. In contrast, NH+closure model trained stably out of the box, with minimal adaptations compared to the Cellular MNIST design. As such, we empirically observe that the biologically informed analytical term in NH+closure not only improves the biological realism of the simulations, as evident from Table~\ref{tab:exp-2-metrics}, but also acts as an effective regularizer that stabilizes training.

\begin{figure*}[h]
  \centering
  \begin{subfigure}[b]{0.485\textwidth}
    \includegraphics[width=\textwidth]{Figures/experiment_2/moments_comparison_T_final.pdf}
    \caption{Bipolar cellular organization at equilibrium.}
    \label{fig:exp2-moment-final}
  \end{subfigure}
  \hfill
  \begin{subfigure}[b]{0.485\textwidth}
    \includegraphics[width=\textwidth]{Figures/experiment_2/moment_dynamics-neuralcpm-toda.pdf}
    \caption{Development of bipolar cellular organization over time.}
    \label{fig:exp2-moment-dynamics}
  \end{subfigure}
  \caption{Bipolar cellular organization as quantified through the fraction of variance along the polar axis for each cell type. A high value indicates strong alignment with the bi-polar axis, which is expected for type 2 cells in equilibrium, whereas a low value indicates alignment in the orthogonal direction. Almost all observations of~\citet{Toda2018Science} lie within 1 standard deviation of the mean of the simulations (error bars in (a) and shaded in (b)), indicating that NeuralCPM reproduces the observed bi-polar organization behavior.}
  \label{fig:exp-2-moments-all-results}
\end{figure*}


To compare the NH + closure model with the laboratory observations of~\citet{Toda2018Science}, we use the six recorded self-organized states as well as a time-lapse of the self-organizing process, which the authors kindly shared with us. Figure~\ref{fig:exp2-moment-final} shows the degree of bi-polar organization in the observations, our synthetic training data, and the NeuralCPM simulations. Due to their synthetic nature, our training data do not contain as much of the noise that is inherent to real observations, which is expressed in the slightly more organized configurations. Our NeuralCPM simulations yield self-organization patterns that largely overlap with the observations of~\citet{Toda2018Science}. Moreover, Figure~\ref{fig:exp2-moment-dynamics} shows that the self-organizing dynamics in our simulations align with the time-lapse of~\citet{Toda2018Science}. Crucially, while the cells in the synthetic training data could only be prepared in a predefined final bipolar configuration, NeuralCPM successfully predicts the temporal dynamics and spontaneous symmetry breaking observed in the experiments by~\citet{Toda2018Science}. A Neural Hamiltonian trained on multicellular structures paired with the well-established CPM dynamics can therefore not only be used to study equilibrium configurations, but also elucidates the dynamic pathways of cellular self-assembly towards such states.














