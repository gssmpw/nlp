
\begin{figure}[t]
    \centering
    \includegraphics[width=\linewidth]{Figures/intro_figure.pdf}
    \caption{NeuralCPM is more expressive than traditional cellular Potts models, allowing for more accurate simulation of real-world collective cell dynamics, e.g. bi-polar self-organization of biological cells over 12 hours in the top row~\cite{Toda2018Science}.}
    \label{fig:intro-figure}
\end{figure}


\section{Introduction}
\label{sec:introduction}

Cell migration and multicellular self-organization are crucial biological processes that drive many phenomena of life, such as embryo growth and the spread of cancer~\cite{friedlCollectiveCellMigration2009,Gottheil2023}. Understanding these cellular dynamics is not only a fundamental goal of biology, but also needed for the development of medical treatments. 
As experimental data alone do not reveal the regulatory logic and self-organization principles of complex cell-cell interactions, computational models have to be combined with biological experiments~\cite{Maree2001PNAS,Hester2011PLOS,Boutillon2022}. 
One of the most powerful and widely-used numerical methods is the cellular Potts model (CPM), which captures the stochastic movement and shape of cells, interactions in multicellular systems, and multiscale dynamics~\cite{Graner1992, Balter2007}.

CPMs are based on a \emph{Hamiltonian} or \emph{energy function} which maps each possible state in the discrete state space of a multicellular system to a scalar (the energy). To model the evolution of the system over time, a CPM-based simulator stochastically perturbs the current state towards states with a lower energy, following the principles of statistical mechanics. Consequently, the Hamiltonian directly drives the simulated dynamics of cells, and designing the Hamiltonian is the core challenge to arrive at realistic CPM simulations.
So far, domain experts need to engineer a tailor-made Hamiltonian for each new problem setting. This Hamiltonian generally consists of a weighted sum of symbolic, physics-inspired features of the system, but (i) it is labor-intensive to develop and (ii) arguably only partially captures the full complexity of cellular systems due to simplifying assumptions in the structure of the Hamiltonian. 

In this work, we tackle these two weaknesses by presenting \emph{NeuralCPM}: a method for learning cellular Potts models with expressive neural network-based Hamiltonians. In contrast to current practice in the field, NeuralCPM enables fitting a Hamiltonian directly on observational data, without requiring any assumptions on the structure of the Hamiltonian or problem-specific feature engineering.
NeuralCPM also facilitates the seamless integration of biological domain knowledge by using the neural network as a \emph{closure term}, complementary to an analytical Hamiltonian based on domain knowledge. This allows us to constrain the learned model to cellular configurations with guaranteed biological realism (e.g. compact cells of given number even for unseen tasks, as opposed to potential hallucinations of fragmented or supernumerous cells), while leveraging the neural network to find structure in the observed data that is too complex to model with an analytical Hamiltonian.

Our main contributions are summarized as follows:
\begin{itemize}
    \item We propose the neural cellular Potts model, a CPM in which the Hamiltonian is parameterized with a novel neural network architecture that respects the symmetries that are universal in cellular dynamics modeling. We exploit the strong connection between CPMs and deep energy-based models, a generative modeling framework developed in the machine learning community, to directly train the Neural Hamiltonian on observational data.
    \item We show how known biological mechanisms can straightforwardly be integrated in the NeuralCPM framework by using the Neural Hamiltonian as a closure model. We find that such \emph{biology-informed} Neural Hamiltonians not only improve biological consistency of the simulations, but also act as a regularizer that effectively stabilizes the training process, which can be notoriously challenging for deep energy-based models.
    \item We validate the effectiveness of our proposed method on three experimental scenarios: 1) parameter fitting of a known analytical cellular Potts model (validation of the learning algorithm); 2) fitting Hamiltonians that are difficult or impossible to attain with analytical functions (validation of the increased expressiveness); and 3) fitting a Hamiltonian on real-world biological data (demonstration of an application to real-world problems).
\end{itemize}

