\section{Conclusion}\label{sec:conclusion}

This work introduced NeuralCPM, a method for simulating cellular dynamics with neural networks. Whereas the current practice in CPM research is for domain experts to hand-craft an approximate symbolic Hamiltonian for each application, NeuralCPM parameterizes the Hamiltonian with an expressive neural network. These Neural Hamiltonians can model more complex dynamics than analytical Hamiltonians and can be trained directly on observational data. Our results demonstrated that incorporating the symmetries of multicellular systems is crucial to train an effective Neural Hamiltonian. Moreover, we found that using the Neural Hamiltonian as a closure term on top of a biology-informed symbolic model improves biological realism and training stability. Finally, a case study on real-world complex self-organizing dynamics showed that NeuralCPM's simulations successfully predict laboratory observations. We conclude that NeuralCPM can effectively model collective cell dynamics, enabling the study of more complex cell behavior through computer simulations.
\paragraph{Limitations and future work.}
As the aim of this work was to introduce and validate the core concepts of NeuralCPM, our evaluation has focused on systems with up to 100 cells. To apply NeuralCPM to large-scale biological tissues, for example in cancer research, we identify three limitations for future work. 
The first is accelerating the NeuralCPM metropolis kinetics, which pose computational challenges for large systems. We hypothesize that the use of efficient sampling techniques for discrete EBMs may alleviate this challenge~\cite{grathwohl2021oops, zhang2022langevin, sun2023discrete}.
The second limitation concerns the global receptive field of the Neural Hamiltonian architecture. For the applications we considered, a global receptive field is biologically plausible, but this is not the case for tissue-scale simulations. To this end, we need to develop a Neural Hamiltonian in which each cell can only sense its immediate surroundings. 
The third limitation is the assumption of dynamics towards an equilibrium distribution. While this assumption is well-motivated in scenarios like embryo development, other applications concern actively migrating cells, leading to non-Markovian dynamics. A conditional Hamiltonian that also depends on the system's history can address this limitation. 
In addition, another promising research direction is to use NeuralCPM to discover biological mechanisms, for example by using explainable AI techniques for neural network models of dynamical systems~\cite{cranmer2020gnnsymreg, Brunton2016sindy}.
Finally, to foster adoption of NeuralCPM by biologists and integrate with other phenomena, e.g. cell division, NeuralCPM can be incorporated into widely-used software packages for CPM simulation~\cite{starruss2014morpheus, CC3D}.




