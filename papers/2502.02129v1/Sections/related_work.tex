\section{Background and related work}
\label{sec:related-work}

\subsection{Energy-based models}\label{sec:ebms}
Energy-based models (EBMs) specify 
a probability distribution over a random variable $x$ defined on a space $\mathcal{X}$ up to an unknown normalization constant as follows:
\begin{equation}
    p_\theta(x) = \frac{e^{-H_\theta(x)}}{Z_\theta},
\end{equation}
where $H_\theta: \mathcal{X} \rightarrow \mathbb{R}$ defines a scalar-valued \emph{energy function} (also called Hamiltonian) parameterized by $\theta$. $Z_\theta = \int_x e^{-H_\theta(x)} dx$ defines the typically intractable normalization constant \cite{lecun2006tutorial, kingma2021ebm}. 
As $Z_\theta$ does not need to be computed, $H_\theta(x)$ can be any nonlinear regression function as long as it could in principle be normalized~\cite{kingma2021ebm}, making EBMs a highly flexible classs of models. If $H_\theta(x)$ is a neural network, we call the model a \textit{Deep EBM}.

Typically, Deep EBMs are fitted to a target distribution $p^*(x)$ by minimizing the negative log-likelihood:
\begin{equation}\label{eq:maxlikelihoodebm}
    \underset{\theta}{\min} \, \mathcal{L}(\theta) = \mathbb{E}_{x\sim p^*(x)}[-\log p_\theta(x)],
\end{equation}
for which gradient descent on $\theta$ is the de-facto optimization algorithm. The gradient of $\mathcal{L}(\theta)$ then looks as follows \cite{hinton2002training, kingma2021ebm}:
\begin{align}\label{eq:gradientebm}
    \nabla_\theta \mathcal{L}(\theta) &= \mathbb{E}_{p^*(x)}[\nabla_\theta H_\theta(x)] - \mathbb{E}_{p_\theta(x)}[\nabla_\theta H_\theta(x)],
\end{align}
where the expectations can be estimated with Monte Carlo sampling.
The main challenge lies in sampling $x$ from the intractable distribution $p_\theta(x)$ to estimate $\mathbb{E}_{p_\theta(x)}[\nabla_\theta H_\theta(x)]$, which is typically achieved by 
a Markov Chain Monte Carlo (MCMC) algorithm.





\subsection{Cellular Potts model}\label{sec:cp-model}

The CPM is a stochastic numerical method used for the simulation of individual and collective cell dynamics~\cite{Graner1992, Savill1997, Balter2007}. In the CPM, cells are modeled as discrete entities with explicit two- or three-dimensional shapes. Because of its capabilities in modeling collective cell behavior, stochastic cellular dynamics, and multiscale phenomena, the CPM has become one of the most effective frameworks for simulating multicellular dynamics~\cite{Rens2019, Hirashima2017}.

The CPM works as follows: given a lattice $L$ and a set of cells $C$, $x^t \in C^{|L|}$ denotes the time-varying state of a multicellular system, where $x^t_l = c$ if cell $c \in C$ occupies the lattice site $l \in L$. 
The state $x^t$ is evolved over time by an MCMC algorithm that has a stationary distribution characterized by a Hamiltonian $H: C^{|L|} \rightarrow \mathbb{R}$. The MCMC dynamics mimic the protruding dynamics of biological cells: 
a random lattice site $l_1$ is chosen and its state $x_{l_1}$ is attempted to alter to the state of a neighboring site $l_2$. The proposed state transition $x \to x'$ is accepted 
with probability $\min\{1, e^{-\Delta H / T}\}$, where $\Delta H = H(x') - H(x)$ is the difference in energy between the proposed and current state and $T$ is the so-called \emph{temperature} parameter. 

As the Hamiltonian $H$ determines the stationary distribution of the Markov chain, the design of $H$ is the key challenge to achieve realistic simulations of cellular dynamics with the CPM. Generally, $H$ contains contact energy and volume constraint terms, as originally proposed in~\cite{Graner1992}, and additional application-specific components:
\begin{gather}
\begin{aligned}\label{eq:hamiltonian-glazier}
    H(x) &= \underbrace{\sum_{i,j \in \mathcal{N}(L)} J\left(x_i, x_j \right) \left(1-\delta_{x_i, x_j}\right)}_{\text{contact energy}}\\
    &+ \underbrace{\sum_{c \in C} \lambda \left(V(c) - V^*(c)\right)^2}_{\text{volume constraint}} + H_{\text{case-specific}}(x).
\end{aligned}
\end{gather}
In Equation~\ref{eq:hamiltonian-glazier}, $\mathcal{N}(L)$ denotes the set of all pairs of neighboring lattice sites in $L$, $J\left(x_i, x_j\right)$ is a contact energy defining adhesion strength between cells $x_i$ at site $i$ and $x_j$ at site $j$, and $\delta_{x, y}$ is the Kronecker delta. Furthermore, $V(c)$ is the volume of cell $c$, $V^*(c)$ is $c$'s target volume, and $\lambda$ is a Lagrange multiplier. Since the first introduction of the CPM, many extensions for $H_\text{case-specific}$ have been proposed for varying biological applications, taking into account external forcing, active non-equilibrium processes like chemotaxis, and many other biological concepts~\cite{Hirashima2017}. However, as opposed to physical systems like evolving foams, where the Hamiltonian can be derived from first principles and which have been modeled with the CPM~\cite{Graner2000}, an equivalent of first principles remains elusive for living systems. Therefore, the task of designing a suitable Hamiltonian requires significant domain expertise and needs to be repeated for each new case. Moreover, even well-designed Hamiltonians will only partially account for the observed cell dynamics.

\subsection{Neural networks for cellular dynamics simulation}

Neural networks have gained traction as simulation models of complex dynamical systems. The primary objective of most works in this field has been to improve computational efficiency over physics-based numerical simulators~\cite{azizzadenesheli2024neuralop,fno, kochov2021mlcompfluid, gupta2023towards}. However, models of multicellular systems including vertex-based models, phase-field models, and the CPM generally only partially explain the dynamics observed in laboratory experiments~\cite{alert2020, Bruckner2024}. 
Here the value of neural simulators lies primarily in \emph{improved accuracy} and \emph{new discoveries}, rather than accelerated simulation. 

Still, little research has been done in machine learning-driven modeling of cellular dynamics, and as opposed to this work, most methods consider cells as point masses without an explicit shape~\cite{lachance2022, Yang2024}. Although some machine learning methods for modeling CPM-like dynamics of single-cell~\cite{minartz2022towards} and multicellular systems~\cite{Minartz2024EPNS} have been proposed, these are autoregressive models that require sequence data covering full trajectories of cellular dynamics for training, which can be costly to acquire. Moreover, these methods are black-box surrogates, and cannot exploit any biological knowledge about the system. In contrast, NeuralCPM requires only observations of self-organized states as training data, akin to Neural Cellular Automata~\cite{mordvintsev2020growing}, and relies on the powerful CPM framework to reconstruct the dynamics of cells. 

