\section{Future Directions}
Based on our experimental findings, we propose two potential research directions that integrate role-playing and reasoning techniques for future exploration.

\noindent \paragraph{Role-aware Chain-of-Thought (CoT) for Improving Role-playing LLMs.}
One promising direction for enhancing the role-playing ability of LLMs is Role-aware CoT reasoning. 
While standard CoT enables step-by-step logical inference, it often disregards the persona-specific constraints that are essential for consistent role-playing. 
A role-aware CoT approach would integrate persona attributes, narrative constraints, and character-specific perspectives into the reasoning process, ensuring that logical deductions align with the character's predefined traits. 
For instance, a historical figure simulated in an LLM should reason within the knowledge and biases of their era rather than applying contemporary logic. 
This requires incorporating dynamic memory structures that retain persona information throughout multi-turn interactions, mitigating the risk of breaking character or adopting inconsistent reasoning patterns.

\noindent\paragraph{Reinforcement Learning for Role-playing LLMs.}

DeepSeek-R1 has demonstrated that by defining precise rule-based rewards, reinforcement learning alone can induce emergent reasoning and cognitive capabilities. This suggests an important research direction: investigating whether carefully designed role-playing task rewards can enable models to autonomously develop intrinsic, role-specific reasoning and thinking abilities, thereby enhancing their role-playing performance.
A critical consideration in reward design is ensuring that it simultaneously accounts for the accuracy of the role-playing task while effectively guiding the model to develop role-specific reasoning abilities. The reward function should prevent the model from relying on shortcuts to generate final responses, instead encouraging it to engage in authentic character-driven reasoning and decision-making.