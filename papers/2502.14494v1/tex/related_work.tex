\section{Related Work}

\subsection{Benchmarks for Multi-Turn Dialogues}
The evolution of dialogue evaluation paradigms has progressed from single-turn assessments to sophisticated multi-turn interaction analysis~\cite{wang2023mint,sun-etal-2024-parrot,duan-etal-2024-botchat}. 
Among these, MT-Bench~\cite{zheng2023judging} pioneered this transition by providing methodologies specifically designed to assess a model's ability to handle multi-turn interactions. 
Building upon this, MT-Bench-101~\cite{bai-etal-2024-mt} introduces a more granular evaluation framework to assess fine-grained capabilities. 
Multi-IF~\cite{he2024multi} expands single-turn dialogues into multi-turn interactions by following simple, predefined linear paths.
However, most existing work on multi-turn dialogue evaluation does not prioritize the assessment of instruction following and overlooks the influence of structural information on the evaluation of multi-turn dialogues.
The four modes of user-assistant interactions proposed by MT-Eval~\cite{kwan-etal-2024-mt} cover certain structural information within multi-turn dialogues.
However, MT-Eval does not establish a systematic structural framework and lacks integration of various structural aspects for a comprehensive evaluation.


\subsection{Benchmarks for Instruction Following}
Recent instruction following evaluation predominantly employs constraint-based frameworks~\cite{jiang-etal-2024-followbench,zhang2024cfbench,he2024can,zhou2023instruction}.
InfoBench~\cite{qin2024infobench} introduces the Decomposed Requirements Following Ratio (DRFR) metric, which provides a more granular scoring system by breaking down the evaluation of complex instructions into assessments of their individual simple constraints. 
Furthermore, ComplexBench~\cite{wen2024benchmarking} explores instruction-following capabilities in single-turn complex dialogues through empirical studies of constraint composition.
However, prior work on instruction-following evaluation has primarily focused on single-turn interactions, which do not align with the more common multi-turn dialogue scenarios observed in real-world user interactions. 
While some studies have attempted to split complex single-turn instructions into multi-turn dialogues, these approaches do not fully capture the intentionality and goal-oriented nature of users in real-world contexts.