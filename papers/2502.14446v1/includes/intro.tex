\section{Introduction}
Time series play a central role in modeling the evolution of data-generating processes in many domains.
To capture the multifaceted nature of processes generating data, time series are oftentimes \emph{multidimensional}: collections of co-evolving signals whose measurements are synchronized, collectively describing the evolution of the process.
Extracting information from such multidimensional time series is thus fundamental.

In particular, \emph{top-$k$ motifs} mining is a crucial and challenging problem: intuitively, the goal is to find patterns that occur twice with minor modifications, spanning many, but not all, the signals of the time series.
More formally, given a time series $\mathbf{T}$ with $D>1$ dimensions and $n$ points, the problem consists in finding $k$ pairs of $\mathbf{T}$'s subsequences with the smallest distance, where the distance captures the similarity between subsequences.
Indeed, similar patterns might  imply a particular behavior, making motif discovery a crucial step for higher-level analysis. Applications include forecasts for volcanic eruptions \cite{2012_Cassisi}, healthcare management \cite{health2015}, and machine management in industry \cite{renard:tel-01922186}.
A common approach to motif discovery is to extend the approach for motif mining in 1-dimension time series to the multidimensional case: that is, we look for pairs of subsequences of the time series where all the $D$ dimensions are similar.
However, this approach might not reveal interesting patterns: for instance, some dimensions might be noisy or uncorrelated with respect to the others, and they might hide similar patterns involving only a subset of the dimensions; moreover, such dimensions are in general unknown beforehand.
What we are interested in is a more general discovery problem, called \textit{multidimensional motif discovery}, where patterns might involve only a subset of the $D$ dimensions which is unknown and needs to be retrieved as well.
This formulation allows a more flexible mining without requiring prior knowledge on the correlation between the dimensions.

Multidimensional motifs can then be discovered with the following naive approach:
for each of the $2^D$ subset of dimensions, we compare the $O(n^2)$ subsequences of $\mathbf{T}$ on the selected dimensions and we return the $k$ closest ones. 
This solution forces $O(2^D n^2)$ comparisons, which, for large sets of data, are clearly prohibitive, and indeed several earlier works proposed approximation algorithms. 

In this paper, we propose a scalable and efficient solution that aims at minimizing the number of distance computations to perform.
We leverage Locality Sensitive Hashing (LSH), a common technique in similarity search that groups together similar elements.
At a high level, we build an index of the time series where each multidimensional subsequence is mapped to a set of LSH hash values: by the properties of LSH this ensures that similar subsequences are more likely to hash to the same values across the dimensions spanned by the motif.
Thanks to this index we are able to prune a hefty amount of candidates from the search space.
We focus on comparisons based on the \textit{z-normalized Euclidean Distance}, however our approach can be generalized to other similarity measures.

One of the challenges of employing LSH is setting its parameters to ensure that good quality results are retrieved efficiently, since the precise setting of the parameters is data-dependent.
To overcome this challenge we design an index that automatically tunes its parameters depending on the data at hand while respecting user-specified limits on the memory to be used.

Our contributions are the following:
\begin{itemize}
    \item
        We design an approach for top-$k$ motif discovery in multidimensional time series,
        named \textsc{LEIT-motifs}.
        Our approach is based on Locality Sensitive Hashing and returns exact answers with a user-specified failure probability.
        Furthermore, we provide a theoretical analysis of the correctness of our approach and on its complexity in terms of distance computations carried out, along with several optimizations to speed up the execution.
    \item 
        We provide an open source implementation of our approach, that we use to carry out an extensive experimental evaluation.
        We show that our approach outperforms state of the art baselines in terms of scalability,
        while providing high quality results.
\end{itemize}

\subsubsection*{Organization}
After reviewing the related work (\Cref{sec:relwork}) and introducing the background concepts of time series and motif discovery (\Cref{sec:ts}), we describe our approach (\Cref{sec:algo}). We then proceed in formalizing the details of our approach (\Cref{sec:complexity}) and we integrate optimizations to improve data adaptability and time complexity (\Cref{sec:opti}). Finally, we compare our proposal with other baselines and we extensively test our approach under different conditions to show its effectiveness (\Cref{sec:exp}).
