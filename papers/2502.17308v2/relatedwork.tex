\section{Related Work}
Word order refers to the arrangement of words in a sentence or phrase to convey meaning in a particular language. In the cross-lingual community, word order is widely explored from two perspectives: order-agnostic methods~\cite{ahmad-etal-2019-difficulties, Liu_Winata_Cahyawijaya_Madotto_Lin_Fung_2021, ding-etal-2020-self, DBLP:journals/corr/abs-2305-19857, DBLP:journals/corr/abs-1910-12391, hessel-schofield-2021-effective} and word reordering methods~\cite{rasooli-collins-2019-low, ji-etal-2021-word, liu-etal-2020-cross-lingual-dependency, chen-etal-2019-neural, goyal-durrett-2020-neural, al-negheimish-etal-2023-towards, pham-etal-2021-order}. The former argues that word order encoding is a risk for cross-lingual transfer, as the model often fits in the language-specific order. Reducing the word order information fitted into the models can improve the cross-lingual adaptation performance in position representation~\cite{ding-etal-2020-self} and dependency parsing~\cite{ahmad-etal-2019-difficulties, Liu_Winata_Cahyawijaya_Madotto_Lin_Fung_2021}. The latter is dedicated to reordering the word from one language to another. For example, Arviv et al.~\cite{arviv-etal-2023-improving} achieves better cross-lingual transfer results by rearranging the words in the source language to meet the word order in the target language conditioned on the syntactic constraints.