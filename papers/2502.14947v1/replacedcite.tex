\section{Related Work and Discussion}
% ------------------------------
\subsection{Adaptive Bitrate solutions}

ABR implementations vary widely across the literature____, as they are often designed for specific applications, such as video-on-demand streaming or real-time videoconferencing. Given the unique demands of interactive VR streaming ---including stringent latency requirements--- traditional buffer-based ABR algorithms such as____, which perform nearly optimally for pre-buffered video, are ineffective in this context, as exposed in____. Consequently, ABR solutions that rely on network bandwidth estimation are a feasible choice for VR streaming. Network bandwidth estimation can be performed through direct packet probing, which introduces overhead____, deep neural network____ models based on network features (e.g., ____), or throughput-based inference from received data chunks (e.g., ____), with the latter aligning with our approach.

For instance, EveREst____, a VR-tailored ABR algorithm, estimates network bandwidth by measuring the throughput of lightweight frames and uses this estimate as an upper bound for bitrate selection, adjusting for the estimated number of concurrent users on the network. Despite EveREst, research on VR-specific ABR algorithms remains limited. ____ presents a fuzzy logic-based algorithm that adjusts bitrate according to the server’s transmission buffer status, optimizing TCP transport for VR. ____ applies a delay-based congestion control mechanism inspired by Google's WebRTC congestion control algorithm____, which dynamically adjusts bitrate during video transmission based on one-way delay gradients. This ABR implementation is integrated into ALVR ---which exhibits similar traffic patterns to WebRTC, where each VF is also sent in a single burst____--- and is evaluated in terms of motion-to-photon latency, user perception, and visual quality ---measured using the Structural Similarity Index Measure (SSIM) and Peak Signal-to-Noise Ratio (PSNR). Similarly, ____ leverages ALVR and introduces a QoE-driven adaptation algorithm that dynamically adjusts encoding settings ---including frame rate, resolution, and bitrate--- based on real-time streaming metrics ---such as throughput, packet loss, frame loss, and network delay--- to maximize user QoE in cloud-based VR gaming scenarios.
Other works also optimize parameters beyond bitrate. For instance, ____ proposes a deep reinforcement learning policy for 5G-based VR streaming that adjusts both network parameters (e.g., the number of allocated resource blocks) and application parameters (e.g., receiver buffer size, resolution) using streaming metrics ---such as client frame rate, frame loss, and network delay--- as well as well as network data ---such as MCS and SNR--- to optimize QoS and visual quality indicators like PSNR.

NeSt-VR strikes a balance between efficiency (selecting the highest feasible bitrate to maximize user experience) and stability (avoiding unnecessary bitrate fluctuations). Fairness in bandwidth allocation, typically evaluated using Jain’s fairness index____ (a measure of the relative bandwidth distribution across multiple users) as demonstrated in____ and____, is partially addressed in NeSt-VR. NeSt-VR incorporates randomness into its gradual bitrate adjustments, preventing certain users from consistently consuming more bandwidth and thereby promoting a more equitable distribution of resources, while maintaining an acceptable QoS for all users. Other studies account for the potential presence and arrival of additional users in their bitrate adjustments, estimating how many VR streams the network can support based on its estimated capacity (e.g.,____)
However, it may occur that not all users reduce their bitrate proportionally during periods of congestion. Addressing this would require cross-layer coordination between users' ABRs____ (e.g.,____), enabling direct communication for managing bandwidth distribution, albeit at the cost of additional network overhead.

Traditional ABR algorithms for video streaming also incorporate QoE estimation to aid in bitrate decisions, often using objective video quality metrics such as PSNR, SSIM, or Video Multimethod Assessment Fusion (VMAF) scores____ (e.g.,____). Some other works incorporate QoE models. However, a standardized QoE function is not yet widely agreed upon, but many QoE-driven ABR solutions use variations of the Yin QoE objective function____ (e.g., ____).
On the other hand, NeSt-VR's decision-making thresholds can be fine-tuned to ensure a satisfactory user experience without relying on traditional QoE metrics, as demonstrated in Tbl.~\ref{tab:thresh_exp}. Indeed, NeSt-VR leverages QoS-based network metrics ---such as NFR and VF-RTT--- as indicators of QoE, balancing video quality, smoothness, and latency in response to dynamic network conditions. NeSt-VR effectively reduces motion-to-photon latency and minimizes packet loss ---two of the most critical QoE factors in VR streaming____--- while preventing stalling, a major cause of cybersickness____, thereby ensuring a consistent, immersive user experience.

% ------------------------------
\subsection{Wi-Fi evolution and VR streaming}

Wi-Fi 6, based on IEEE 802.11ax____, is used in all scenarios to evaluate NeSt-VR. Following____, we configured Wi-Fi APs and clients with default Enhanced Distributed Channel Access (EDCA), as DL OFDMA/MU-MIMO provides no noticeable benefits, while UL OFDMA/MU-MIMO may introduce disrupt tracking packets due to AP scheduling____. With Wi-Fi 7, VR streaming could benefit from Multi-Link Operation (MLO), enabling lower latency and greater user support____. However, in dense Wi-Fi environments, MLO may still suffer from contention, potentially worsening delays due to the `performance anomaly'____. To address this, Wi-Fi research and standardization efforts are exploring Multi-AP Coordination (MAPC)____, which is expected to enhance transmission efficiency, resource allocation, and traffic prioritization across different Wi-Fi networks____, improving latency and reliability for VR streaming____. The interplay of MLO, MAPC, and future Wi-Fi advancements with NeSt-VR remains an open question. While NeSt-VR is technology agnostic, incorporating awareness of Wi-Fi capabilities could further enhance its operation and performance.

% ------------------------------
% ------------------------------
% ------------------------------
% ------------------------------