The expressive power and computational complexity of deep visual generative models, such as flow-based and autoregressive (AR) models, have gained considerable interest for their wide-ranging applications in generative tasks. However, the theoretical characterization of their expressiveness through the lens of circuit complexity remains underexplored, particularly for the state-of-the-art architecture like FlowAR proposed by [Ren et al., 2024], which integrates flow-based and autoregressive mechanisms.  This gap limits our understanding of their inherent computational limits and practical efficiency. In this study, we address this gap by analyzing the circuit complexity of the FlowAR architecture. We demonstrate that when the largest feature map produced by the FlowAR model has dimensions $n \times n \times c$, the FlowAR model is simulable by a family of threshold circuits $\mathsf{TC}^0$, which have constant depth $O(1)$ and polynomial width $\mathrm{poly}(n)$. This is the first study to rigorously highlight the limitations in the expressive power of FlowAR models. Furthermore, we identify the conditions under which the FlowAR model computations can achieve almost quadratic time. To validate our theoretical findings, we present efficient model variant constructions based on low-rank approximations that align with the derived criteria. Our work provides a foundation for future comparisons with other generative paradigms and guides the development of more efficient and expressive implementations.
