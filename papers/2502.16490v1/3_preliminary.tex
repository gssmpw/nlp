\section{Preliminary}\label{sec:preliminary}
All notations employed throughout this paper are present in Section~\ref{sub:notations}. Section~\ref{sec:pre:circuit} introduces circuit complexity axioms. In Section~\ref{sec:pre:float}, we define floating-point numbers and establish the complexity bounds of their operations.
\subsection{Notations}\label{sub:notations}


Given a matrix $X \in \R^{hw \times d}$, we denote its tensorized form as $\X \in \R^{h \times w \times d}$. Additionally, we define the set $[n]$ to represent $\{1,2,\cdots, n\}$ for any positive integer $n$. We define the set of natural numbers as $\mathbb{N}:= \{0,1,2,\dots\}$. Let $X \in \mathbb{R}^{m \times n}$ be a matrix, where $X_{i,j}$ refers to the element at the $i$-th row and $j$-th column. When $x_i$ belongs to $\{ 0,1 \}^*$, it signifies a binary number with arbitrary length. In a general setting, $x_i$ represents a length $p$ binary string, with each bit taking a value of either 1 or 0. Given a matrix $X \in \R^{n \times d}$, we define $\|X\|_\infty  $ as the maximum norm of $X$. Specifically, $\|X\|_\infty = \max_{i,j} |X_{i,j}|$. 


\subsection{Circuit Complexity Class}\label{sec:pre:circuit}
Firstly, we present the definition of the boolean circuit.

\begin{definition}[Boolean Circuit, \cite{ab09}] 
A Boolean circuit $C_n:\{0,1\}^n \to \{0, 1\}$ is formally specified through a directed acyclic graph (DAG) where:
\begin{itemize}
    \item Nodes represent logic gates from the basis $\{\mathsf{AND},\mathsf{OR},\mathsf{NOT}\}$.
    \item Source nodes (in degree $0$) corrspond to input Boolean variables ${x_1, \dots, x_n}$.
    \item Each non-source gate computes its output by applying its designated Boolean operation to values received via incoming edges.
\end{itemize}
\end{definition}




Then, we proceed to show the definition of languages related to a specific Boolean circuit.

\begin{definition}[Languages, page 103 of~\cite{ab09}]
A language $L \subseteq \{0, 1\}^*$ is recognized by a Boolean circuit family $\mathcal{C} = \{C_n\}_{n \in \mathbb{N}}$ if:
\begin{itemize}
    \item The family is parameterized by input length: $C_n$ operates on $n$ Boolean variables.
    \item Membership equivalence: $\forall x \in \{0,1\}^*, C_{|x|}(x) = 1\Longleftrightarrow x \in L$.
    \item Circuit existence: For every string length $n \in \mathbb{N}$, $\mathcal{C}$ contains an appropriate circuit $C_n$.
\end{itemize}
\end{definition}

Then, we present different language classes that can be recognized by different circuit families. Firstly, we introduce the $\mathsf{NC}^i$ class.
\begin{definition}[$\mathsf{NC}^i$ Complexity Class, \cite{ab09}]
    The complexity class $\mathsf{NC}^i$ comprises all languages recognized by Boolean circuit families $\{C_n\}$ satisfying:
    \begin{itemize}
        \item $\mathsf{Size}(C_n) = O(\poly(n))$.
        \item $\mathsf{Depth}(C_n) = O((\log n)^i)$.
        \item Gate constraints: (1) $\mathsf{AND}, \mathsf{OR}$ gates have bounded fan-in (2) $\mathsf{NOT}$ gates have unit fan-in.
    \end{itemize}
\end{definition}

$\mathsf{AC}^i$ circuits relax the gate fan-in restriction of $\mathsf{NC}^i$ circuits. We present the definition of $\mathsf{AC}^i$ as the following:
\begin{definition}[$\mathsf{AC}^i$ Complexity Class, \cite{ab09}]
    The complexity class $\mathsf{AC}^i$ comprises all languages recognized by Boolean circuit families $\{C_n\}$ satisfying:
    \begin{itemize}
        \item $\mathsf{Size}(C_n) = O(\poly(n))$.
        \item $\mathsf{Depth}(C_n) = O((\log n)^i)$.
        \item Gate constraints: (1) $\mathsf{AND}, \mathsf{OR}$ gates have un-bounded fan-in (2) $\mathsf{NOT}$ gates have unit fan-in.
    \end{itemize}
\end{definition}

$\mathsf{TC}^i$ introduces the $\mathsf{MAJORITY}$ gate on top of $\mathsf{AC}^i$. The $\mathsf{MAJORITY}$ gate outputs $1$ if more than half of its inputs are $1$ and outputs $0$ otherwise.

\begin{definition}[$\mathsf{TC}^i$ Complexity Class, \cite{vol99}]\label{def:tc}
    The complexity class $\mathsf{TC}^i$ comprises all languages recognized by Boolean circuit families $\{C_n\}$ satisfying:
    \begin{itemize}
        \item $\mathsf{Size}(C_n) = O(\poly(n))$.
        \item $\mathsf{Depth}(C_n) = O((\log n)^i)$.
        \item Gate constraints: (1) $\mathsf{AND}, \mathsf{OR}, \mathsf{MAJORITY}$ gates have un-bounded fan-in (3) $\mathsf{NOT}$ gates have unit fan-in.  
    \end{itemize}
\end{definition}
In this paper, a boolean circuit that employs $\mathsf{MAJORITY}$ gate is referred to as a threshold circuit.

Then, we present two import definitions $\mathsf{L}$-uniformity and $\mathsf{DLOGTIME}$-uniformity.
\begin{definition}[$\mathsf{L}$-uniformity, Definition 6.5 on page 104 of~\cite{ab09}]
    Let $\mathcal{C}$ denote a circuit family (e.g., $\mathsf{NC}^i,\mathsf{AC}^i,\mathsf{TC}^i$) that decides a language $\mathsf{C}$. A language $L \subseteq \{0,1\}^*$ belongs to $\mathsf{L}$-uniform $\mathsf{C}$ if there exists a deterministic Turing machine that, on input ${1}^n$, outputs a circuit $C_n \in \mathcal{C}$ in $O(\log n)$ space, such that $C_n$ recognizes $L$.
\end{definition}
And we present the definition of $\dlogtime$-uniformity.
\begin{definition}[$\dlogtime$-uniformity, \cite{bi94}]
    Let $\mathcal{C}$ denote a circuit family (e.g., $\mathsf{NC}^i,\mathsf{AC}^i,\mathsf{TC}^i$) that decides a language $\mathsf{C}$. A language $L \subseteq \{0,1\}^*$ belongs to $\dlogtime$-uniform $\mathsf{C}$ if there exists a random access Turing machine that, on input ${1}^n$, outputs a circuit $C_n \in  \mathcal{C}$ in $O(\log n)$ time such that $C_n$ recognizes $L$.   
\end{definition}



\subsection{Circuit Complexity for Floating-Point Operations}\label{sec:pre:float}
In this section, we first introduce the key definitions of floating-point numbers.
\begin{definition}[Float point number, \cite{chi24}]
    A $p$-bit floating-point number is a tuple $\langle a,b \rangle$ of integers where
    \begin{itemize}
        \item The significand $a$ satisfies $a \in (-2^p,-2^{p-1}] \cup \{0\} \cup [2^{p-1},2^p)$.
        \item The exponent $b$ lies in the inteval $b\in [-2^p, 2^p)$.
    \end{itemize}
    This represents the real number $a \cdot w^b$. The set of all $p$-bit floating-point numbers is denoted $\mathsf{F}_p$.
\end{definition}


Then, we can show the circuit complexity bounds of some float point number operations.
\begin{lemma}[Float point number operations in $\TC^0$, \cite{chi24}]\label{lem:float_operations_TC} 
Assume the precision $p \leq \poly(n)$. The following hold:
\begin{itemize}
    \item Part 1. {\bf Basic Arithmetic:} Addition, comparison and multiplication of two $p$-bit floating-point numbers are computable by uniform $\mathcal{TC}^0$ circuits (depth $O(1)$, size $\poly(n)$). Denote by $d_{\mathrm{std}}$ the circuit depth for these operations.
    
    
    \item Part 2. {\bf Iterated Multiplication:} The product of $n$ $p$-bit floating-point numbers is computable by uniform $\TC^0$ circuits (depth $O(1)$, size $\poly(n)$). Denote by $d_{\otimes}$ the required circuit depth.
    
    \item Part 3. {\bf Iterated Addition:} The sum of $n$ $p$-bit floating-point numbers is computable by uniform $\TC^0$ circuits (depth $O(1)$, size $\poly(n)$). Denote by $d_{\oplus}$ the required circuit depth.
\end{itemize}
\end{lemma}


\begin{lemma}[Exponential Approximation in $\TC^0$, \cite{chi24}]\label{lem:exp}
    Let precision $p \leq \poly(n)$. For every $p$-bit floating-point number $x \in \mathsf{F}_p$, there exists a constant depth uniform $\mathsf{TC}^0$ circuit of size $\poly(n)$ that can compute $\exp(x)$ with relative error bounded by $2^{-p}$. Denote by $d_{\exp}$ the required circuit depth.

\end{lemma}

\begin{lemma}[Square Root Approximation in $\TC^0$, \cite{chi24}]\label{lem:sqrt}
    Let precision $p \leq \poly(n)$. For every $p$-bit floating-point number $x \in \mathsf{F}_p$, there exists a constant depth uniform $\mathsf{TC}^0$ circuit of size $\poly(n)$ that can compute $\sqrt{x}$ with relative error bounded by $2^{-p}$. Denote by $d_{\mathrm{sqrt}}$ the required circuit depth.    
\end{lemma}

