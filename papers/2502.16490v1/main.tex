\def\isarxiv{1} %%% for icml submission version, we comment this line

\ifdefined\isarxiv
\documentclass[11pt]{article}

\usepackage[numbers]{natbib}

\else

\documentclass{uai2025} % for initial submission
%\documentclass[accepted]{uai2025} % after acceptance, for a revised version; 
% also before submission to see how the non-anonymous paper would look like 
                        
%% There is a class option to choose the math font
% \documentclass[mathfont=ptmx]{uai2025} % ptmx math instead of Computer
                                         % Modern (has noticeable issues)
% \documentclass[mathfont=newtx]{uai2025} % newtx fonts (improves upon
                                          % ptmx; less tested, no support)
% NOTE: Only keep *one* line above as appropriate, as it will be replaced
%       automatically for papers to be published. Do not make any other
%       change above this note for an accepted version.

%% Choose your variant of English; be consistent
\usepackage[american]{babel}
% \usepackage[british]{babel}

%% Some suggested packages, as needed:
\usepackage{natbib} % has a nice set of citation styles and commands
    \bibliographystyle{plainnat}
    \renewcommand{\bibsection}{\subsubsection*{References}}
\usepackage{mathtools} % amsmath with fixes and additions
% \usepackage{siunitx} % for proper typesetting of numbers and units
\usepackage{booktabs} % commands to create good-looking tables
\usepackage{tikz} % nice language for creating drawings and diagrams

%% Provided macros
% \smaller: Because the class footnote size is essentially LaTeX's \small,
%           redefining \footnotesize, we provide the original \footnotesize
%           using this macro.
%           (Use only sparingly, e.g., in drawings, as it is quite small.)

%% Self-defined macros
\newcommand{\swap}[3][-]{#3#1#2} % just an example


\fi


\usepackage{amsmath}
\usepackage{amsthm}
\usepackage{amssymb}
\usepackage{algorithm}
\usepackage{subfig}
\usepackage{algpseudocode}
\usepackage{graphicx}
\usepackage{grffile}
\usepackage{wrapfig,epsfig}
\usepackage{url}
\usepackage{xcolor}
\usepackage{epstopdf}


\usepackage{bbm}
\usepackage{dsfont}

 
\allowdisplaybreaks
 

\ifdefined\isarxiv

\let\C\relax
\usepackage{tikz}
\usepackage{hyperref}  %%% arxiv don't allow this.
\hypersetup{colorlinks=true,citecolor=blue,linkcolor=blue} %%% Zhao : maybe we should comment this in submission.
\usetikzlibrary{arrows}
\usepackage[margin=1in]{geometry}

\else

% \usepackage{microtype}
\usepackage{hyperref}
% \definecolor{mydarkblue}{rgb}{0,0.08,0.45}
% \hypersetup{colorlinks=true, citecolor=mydarkblue,linkcolor=mydarkblue}
 

\fi
 
\graphicspath{{./figs/}}

\theoremstyle{plain}
\newtheorem{theorem}{Theorem}[section]
\newtheorem{lemma}[theorem]{Lemma}
\newtheorem{definition}[theorem]{Definition}
\newtheorem{notation}[theorem]{Notation}
%\newtheorem{proof}[theorem]{Proof}
\newtheorem{proposition}[theorem]{Proposition}
\newtheorem{corollary}[theorem]{Corollary}
\newtheorem{conjecture}[theorem]{Conjecture}
\newtheorem{assumption}[theorem]{Assumption}
\newtheorem{observation}[theorem]{Observation}
\newtheorem{fact}[theorem]{Fact}
\newtheorem{remark}[theorem]{Remark}
\newtheorem{claim}[theorem]{Claim}
\newtheorem{example}[theorem]{Example}
\newtheorem{problem}[theorem]{Problem}
\newtheorem{open}[theorem]{Open Problem}
\newtheorem{property}[theorem]{Property}
\newtheorem{hypothesis}[theorem]{Hypothesis}

\newcommand{\wh}{\widehat}
\newcommand{\wt}{\widetilde}
\newcommand{\ov}{\overline}
\newcommand{\N}{\mathcal{N}}
\newcommand{\R}{\mathbb{R}}
\newcommand{\RHS}{\mathrm{RHS}}
\newcommand{\LHS}{\mathrm{LHS}}
\renewcommand{\d}{\mathrm{d}}
\renewcommand{\i}{\mathbf{i}}
\renewcommand{\tilde}{\wt}
\renewcommand{\hat}{\wh}
\newcommand{\Tmat}{{\cal T}_{\mathrm{mat}}}
\newcommand{\X}{\mathsf{X}}
\newcommand{\Y}{\mathsf{Y}}
\newcommand{\Z}{\mathsf{Z}}
\newcommand{\W}{\mathsf{W}}
\newcommand{\F}{\mathsf{F}}
\newcommand{\V}{\mathsf{V}}
\newcommand{\AC}{\mathsf{AC}}
\newcommand{\NC}{\mathsf{NC}}
\newcommand{\TC}{\mathsf{TC}}
\newcommand{\dplus}{d_\oplus}
\newcommand{\dlogtime}{\mathsf{DLOGTIME}}

\DeclareMathOperator*{\E}{{\mathbb{E}}}
\DeclareMathOperator*{\var}{\mathrm{Var}}
% \DeclareMathOperator*{\Z}{\mathbb{Z}}
\DeclareMathOperator*{\C}{\mathbb{C}}
\DeclareMathOperator*{\D}{\mathcal{D}}
\DeclareMathOperator*{\median}{median}
\DeclareMathOperator*{\mean}{mean}
\DeclareMathOperator{\OPT}{OPT}
\DeclareMathOperator{\supp}{supp}
\DeclareMathOperator{\poly}{poly}
\DeclareMathOperator{\ds}{/\!\!/}
\DeclareMathOperator{\nnz}{nnz}
\DeclareMathOperator{\sparsity}{sparsity}
\DeclareMathOperator{\rank}{rank}
\DeclareMathOperator{\diag}{diag}
\DeclareMathOperator{\dist}{dist}
\DeclareMathOperator{\cost}{cost}
\DeclareMathOperator{\vect}{vec}
\DeclareMathOperator{\tr}{tr}
\DeclareMathOperator{\dis}{dis}
\DeclareMathOperator{\cts}{cts}

\DeclareMathOperator{\round}{round}%Yekun:add Feb 9. 2025

\makeatletter
\newcommand*{\RN}[1]{\expandafter\@slowromancap\romannumeral #1@}
\makeatother

\usepackage{lineno}
\def\linenumberfont{\normalfont\small}


\ifdefined\isarxiv
\else


\title{Circuit Complexity Bounds and Provably Efficient Criteria of FlowAR Model}

\author[1]{\href{mailto:<jj@example.edu>?Subject=Your UAI 2025 paper}{Jane~J.~von~O'L\'opez}{}}
\author[1]{Harry~Q.~Bovik}
\author[1,2]{Further~Coauthor}
\author[3]{Further~Coauthor}
\author[1]{Further~Coauthor}
\author[3]{Further~Coauthor}
\author[3,1]{Further~Coauthor}
% Add affiliations after the authors
\affil[1]{%
    Computer Science Dept.\\
    Cranberry University\\
    Pittsburgh, Pennsylvania, USA
}
\affil[2]{%
    Second Affiliation\\
    Address\\
    …
}
\affil[3]{%
    Another Affiliation\\
    Address\\
    …
  }
  

\fi


\begin{document}

\ifdefined\isarxiv

\date{}

\title{On Computational Limits of FlowAR Models: Expressivity and Efficiency}
% \author{
% Chengyue Gong
% \and 
% Yekun Ke\thanks{\texttt{
% keyekun0628@gmail.com}. Independent Researcher.}
% \and
% Xiaoyu Li\thanks{\texttt{
% xiaoyu.li2@student.unsw.edu.au}. University of New South Wales.}
% \and
% Zhizhou Sha\thanks{\texttt{ shazz20@mails.tsinghua.edu.cn}. Tsinghua University.}
% \and
% Zhenmei Shi\thanks{\texttt{
% zhmeishi@cs.wisc.edu}. University of Wisconsin-Madison.}
% \and 
% Zhao Song\thanks{\texttt{ magic.linuxkde@gmail.com}. The Simons Institute for the Theory of Computing at UC Berkeley.}
% }
\author{
Chengyue Gong\thanks{\texttt{ cygong17@utexas.edu}. The University of Texas at Austin.}
\and
Yekun Ke\thanks{\texttt{
keyekun0628@gmail.com}. Independent Researcher.}
\and
Xiaoyu Li\thanks{\texttt{
xiaoyu.li2@student.unsw.edu.au}. University of New South Wales.}
\and
Yingyu Liang\thanks{\texttt{
yingyul@hku.hk}. The University of Hong Kong. \texttt{
yliang@cs.wisc.edu}. University of Wisconsin-Madison.} 
\and
Zhizhou Sha\thanks{\texttt{
shazz20@mails.tsinghua.edu.cn}. Tsinghua University.}
\and
Zhenmei Shi\thanks{\texttt{
zhmeishi@cs.wisc.edu}. University of Wisconsin-Madison.}
\and
Zhao Song\thanks{\texttt{ magic.linuxkde@gmail.com}. The Simons Institute for the Theory of Computing at UC Berkeley.}
}




\else

\fi





\ifdefined\isarxiv
\begin{titlepage}
  \maketitle
  \begin{abstract}
\begin{abstract}


The choice of representation for geographic location significantly impacts the accuracy of models for a broad range of geospatial tasks, including fine-grained species classification, population density estimation, and biome classification. Recent works like SatCLIP and GeoCLIP learn such representations by contrastively aligning geolocation with co-located images. While these methods work exceptionally well, in this paper, we posit that the current training strategies fail to fully capture the important visual features. We provide an information theoretic perspective on why the resulting embeddings from these methods discard crucial visual information that is important for many downstream tasks. To solve this problem, we propose a novel retrieval-augmented strategy called RANGE. We build our method on the intuition that the visual features of a location can be estimated by combining the visual features from multiple similar-looking locations. We evaluate our method across a wide variety of tasks. Our results show that RANGE outperforms the existing state-of-the-art models with significant margins in most tasks. We show gains of up to 13.1\% on classification tasks and 0.145 $R^2$ on regression tasks. All our code and models will be made available at: \href{https://github.com/mvrl/RANGE}{https://github.com/mvrl/RANGE}.

\end{abstract}



  \end{abstract}
  \thispagestyle{empty}
\end{titlepage}

{\hypersetup{linkcolor=black}
\tableofcontents
}
\newpage

\else

\maketitle 

\begin{abstract}
\begin{abstract}


The choice of representation for geographic location significantly impacts the accuracy of models for a broad range of geospatial tasks, including fine-grained species classification, population density estimation, and biome classification. Recent works like SatCLIP and GeoCLIP learn such representations by contrastively aligning geolocation with co-located images. While these methods work exceptionally well, in this paper, we posit that the current training strategies fail to fully capture the important visual features. We provide an information theoretic perspective on why the resulting embeddings from these methods discard crucial visual information that is important for many downstream tasks. To solve this problem, we propose a novel retrieval-augmented strategy called RANGE. We build our method on the intuition that the visual features of a location can be estimated by combining the visual features from multiple similar-looking locations. We evaluate our method across a wide variety of tasks. Our results show that RANGE outperforms the existing state-of-the-art models with significant margins in most tasks. We show gains of up to 13.1\% on classification tasks and 0.145 $R^2$ on regression tasks. All our code and models will be made available at: \href{https://github.com/mvrl/RANGE}{https://github.com/mvrl/RANGE}.

\end{abstract}


\end{abstract}

\fi


\section{Introduction}

Video generation has garnered significant attention owing to its transformative potential across a wide range of applications, such media content creation~\citep{polyak2024movie}, advertising~\citep{zhang2024virbo,bacher2021advert}, video games~\citep{yang2024playable,valevski2024diffusion, oasis2024}, and world model simulators~\citep{ha2018world, videoworldsimulators2024, agarwal2025cosmos}. Benefiting from advanced generative algorithms~\citep{goodfellow2014generative, ho2020denoising, liu2023flow, lipman2023flow}, scalable model architectures~\citep{vaswani2017attention, peebles2023scalable}, vast amounts of internet-sourced data~\citep{chen2024panda, nan2024openvid, ju2024miradata}, and ongoing expansion of computing capabilities~\citep{nvidia2022h100, nvidia2023dgxgh200, nvidia2024h200nvl}, remarkable advancements have been achieved in the field of video generation~\citep{ho2022video, ho2022imagen, singer2023makeavideo, blattmann2023align, videoworldsimulators2024, kuaishou2024klingai, yang2024cogvideox, jin2024pyramidal, polyak2024movie, kong2024hunyuanvideo, ji2024prompt}.


In this work, we present \textbf{\ours}, a family of rectified flow~\citep{lipman2023flow, liu2023flow} transformer models designed for joint image and video generation, establishing a pathway toward industry-grade performance. This report centers on four key components: data curation, model architecture design, flow formulation, and training infrastructure optimization—each rigorously refined to meet the demands of high-quality, large-scale video generation.


\begin{figure}[ht]
    \centering
    \begin{subfigure}[b]{0.82\linewidth}
        \centering
        \includegraphics[width=\linewidth]{figures/t2i_1024.pdf}
        \caption{Text-to-Image Samples}\label{fig:main-demo-t2i}
    \end{subfigure}
    \vfill
    \begin{subfigure}[b]{0.82\linewidth}
        \centering
        \includegraphics[width=\linewidth]{figures/t2v_samples.pdf}
        \caption{Text-to-Video Samples}\label{fig:main-demo-t2v}
    \end{subfigure}
\caption{\textbf{Generated samples from \ours.} Key components are highlighted in \textcolor{red}{\textbf{RED}}.}\label{fig:main-demo}
\end{figure}


First, we present a comprehensive data processing pipeline designed to construct large-scale, high-quality image and video-text datasets. The pipeline integrates multiple advanced techniques, including video and image filtering based on aesthetic scores, OCR-driven content analysis, and subjective evaluations, to ensure exceptional visual and contextual quality. Furthermore, we employ multimodal large language models~(MLLMs)~\citep{yuan2025tarsier2} to generate dense and contextually aligned captions, which are subsequently refined using an additional large language model~(LLM)~\citep{yang2024qwen2} to enhance their accuracy, fluency, and descriptive richness. As a result, we have curated a robust training dataset comprising approximately 36M video-text pairs and 160M image-text pairs, which are proven sufficient for training industry-level generative models.

Secondly, we take a pioneering step by applying rectified flow formulation~\citep{lipman2023flow} for joint image and video generation, implemented through the \ours model family, which comprises Transformer architectures with 2B and 8B parameters. At its core, the \ours framework employs a 3D joint image-video variational autoencoder (VAE) to compress image and video inputs into a shared latent space, facilitating unified representation. This shared latent space is coupled with a full-attention~\citep{vaswani2017attention} mechanism, enabling seamless joint training of image and video. This architecture delivers high-quality, coherent outputs across both images and videos, establishing a unified framework for visual generation tasks.


Furthermore, to support the training of \ours at scale, we have developed a robust infrastructure tailored for large-scale model training. Our approach incorporates advanced parallelism strategies~\citep{jacobs2023deepspeed, pytorch_fsdp} to manage memory efficiently during long-context training. Additionally, we employ ByteCheckpoint~\citep{wan2024bytecheckpoint} for high-performance checkpointing and integrate fault-tolerant mechanisms from MegaScale~\citep{jiang2024megascale} to ensure stability and scalability across large GPU clusters. These optimizations enable \ours to handle the computational and data challenges of generative modeling with exceptional efficiency and reliability.


We evaluate \ours on both text-to-image and text-to-video benchmarks to highlight its competitive advantages. For text-to-image generation, \ours-T2I demonstrates strong performance across multiple benchmarks, including T2I-CompBench~\citep{huang2023t2i-compbench}, GenEval~\citep{ghosh2024geneval}, and DPG-Bench~\citep{hu2024ella_dbgbench}, excelling in both visual quality and text-image alignment. In text-to-video benchmarks, \ours-T2V achieves state-of-the-art performance on the UCF-101~\citep{ucf101} zero-shot generation task. Additionally, \ours-T2V attains an impressive score of \textbf{84.85} on VBench~\citep{huang2024vbench}, securing the top position on the leaderboard (as of 2025-01-25) and surpassing several leading commercial text-to-video models. Qualitative results, illustrated in \Cref{fig:main-demo}, further demonstrate the superior quality of the generated media samples. These findings underscore \ours's effectiveness in multi-modal generation and its potential as a high-performing solution for both research and commercial applications. %%% Section 1. Introduction
\section{Related Work}

\subsection{Large 3D Reconstruction Models}
Recently, generalized feed-forward models for 3D reconstruction from sparse input views have garnered considerable attention due to their applicability in heavily under-constrained scenarios. The Large Reconstruction Model (LRM)~\cite{hong2023lrm} uses a transformer-based encoder-decoder pipeline to infer a NeRF reconstruction from just a single image. Newer iterations have shifted the focus towards generating 3D Gaussian representations from four input images~\cite{tang2025lgm, xu2024grm, zhang2025gslrm, charatan2024pixelsplat, chen2025mvsplat, liu2025mvsgaussian}, showing remarkable novel view synthesis results. The paradigm of transformer-based sparse 3D reconstruction has also successfully been applied to lifting monocular videos to 4D~\cite{ren2024l4gm}. \\
Yet, none of the existing works in the domain have studied the use-case of inferring \textit{animatable} 3D representations from sparse input images, which is the focus of our work. To this end, we build on top of the Large Gaussian Reconstruction Model (GRM)~\cite{xu2024grm}.

\subsection{3D-aware Portrait Animation}
A different line of work focuses on animating portraits in a 3D-aware manner.
MegaPortraits~\cite{drobyshev2022megaportraits} builds a 3D Volume given a source and driving image, and renders the animated source actor via orthographic projection with subsequent 2D neural rendering.
3D morphable models (3DMMs)~\cite{blanz19993dmm} are extensively used to obtain more interpretable control over the portrait animation. For example, StyleRig~\cite{tewari2020stylerig} demonstrates how a 3DMM can be used to control the data generated from a pre-trained StyleGAN~\cite{karras2019stylegan} network. ROME~\cite{khakhulin2022rome} predicts vertex offsets and texture of a FLAME~\cite{li2017flame} mesh from the input image.
A TriPlane representation is inferred and animated via FLAME~\cite{li2017flame} in multiple methods like Portrait4D~\cite{deng2024portrait4d}, Portrait4D-v2~\cite{deng2024portrait4dv2}, and GPAvatar~\cite{chu2024gpavatar}.
Others, such as VOODOO 3D~\cite{tran2024voodoo3d} and VOODOO XP~\cite{tran2024voodooxp}, learn their own expression encoder to drive the source person in a more detailed manner. \\
All of the aforementioned methods require nothing more than a single image of a person to animate it. This allows them to train on large monocular video datasets to infer a very generic motion prior that even translates to paintings or cartoon characters. However, due to their task formulation, these methods mostly focus on image synthesis from a frontal camera, often trading 3D consistency for better image quality by using 2D screen-space neural renderers. In contrast, our work aims to produce a truthful and complete 3D avatar representation from the input images that can be viewed from any angle.  

\subsection{Photo-realistic 3D Face Models}
The increasing availability of large-scale multi-view face datasets~\cite{kirschstein2023nersemble, ava256, pan2024renderme360, yang2020facescape} has enabled building photo-realistic 3D face models that learn a detailed prior over both geometry and appearance of human faces. HeadNeRF~\cite{hong2022headnerf} conditions a Neural Radiance Field (NeRF)~\cite{mildenhall2021nerf} on identity, expression, albedo, and illumination codes. VRMM~\cite{yang2024vrmm} builds a high-quality and relightable 3D face model using volumetric primitives~\cite{lombardi2021mvp}. One2Avatar~\cite{yu2024one2avatar} extends a 3DMM by anchoring a radiance field to its surface. More recently, GPHM~\cite{xu2025gphm} and HeadGAP~\cite{zheng2024headgap} have adopted 3D Gaussians to build a photo-realistic 3D face model. \\
Photo-realistic 3D face models learn a powerful prior over human facial appearance and geometry, which can be fitted to a single or multiple images of a person, effectively inferring a 3D head avatar. However, the fitting procedure itself is non-trivial and often requires expensive test-time optimization, impeding casual use-cases on consumer-grade devices. While this limitation may be circumvented by learning a generalized encoder that maps images into the 3D face model's latent space, another fundamental limitation remains. Even with more multi-view face datasets being published, the number of available training subjects rarely exceeds the thousands, making it hard to truly learn the full distibution of human facial appearance. Instead, our approach avoids generalizing over the identity axis by conditioning on some images of a person, and only generalizes over the expression axis for which plenty of data is available. 

A similar motivation has inspired recent work on codec avatars where a generalized network infers an animatable 3D representation given a registered mesh of a person~\cite{cao2022authentic, li2024uravatar}.
The resulting avatars exhibit excellent quality at the cost of several minutes of video capture per subject and expensive test-time optimization.
For example, URAvatar~\cite{li2024uravatar} finetunes their network on the given video recording for 3 hours on 8 A100 GPUs, making inference on consumer-grade devices impossible. In contrast, our approach directly regresses the final 3D head avatar from just four input images without the need for expensive test-time fine-tuning.



\section{Brief Review of 3D Gaussian Splatting}
\label{sec:prelim}
For the sake of clarity, we first briefly review 3D Gaussian Splatting (3DGS)~\cite{kerbl202333dgs}, an explicit representation of a 3D scene for providing effective image rendering. 
% We also provide brief reviews of two powerful extensions of 3DGS, Gaussian Grouping~\cite{ye2023gaussiangrouping} and Relightable Gaussian~\cite{gao2023relightable}, which equip 3DGS with segmentation and relighting abilities and are utilized together with 3DGS as the backbone representation in our work. 


Given $K$ multi-view images $I_{1:K} = \{I_1, I_2, ..., I_K\}$ with corresponding camera poses $\xi_{1:K} = \{\xi_1, \xi_2, ..., \xi_K\}$ of a 3D scene, a scene-specific 3DGS is applied to model the scene with $N$ learnable 3D Gaussian ellipsoids (i.e., $G_{1:N} = \{G_1, G_2, ..., G_N \}$). Each Gaussian $G_i$ is parameterized with its 3-dimensional centroid $\mathbf{p}_i \in \mathbb{R}^{3}$, a 3-dimensional standard deviation $\mathbf{s}_i \in \mathbb{R}^{3}$, a 4-dimensional rotational quaternion $\mathbf{q}_i \in \mathbb{R}^{4}$, an opacity ${\alpha}_i \in [0,1]$, and color coefficients $\mathbf{c}_i$ for spherical harmonics in degree of 3. Hence, $G_i$ is represented with a set of the above parameters (i.e., $G_i = \{\mathbf{p}_i, \mathbf{s}_i, \mathbf{q}_i, {\alpha}_i, \mathbf{c}_i\}$). To model the scene with $G_{1:N}$, 2D images $\hat{I_{1:K}} = \{\hat{I_1}, \hat{I_2}, ..., \hat{I_K}\}$ are sequentially rendered from $G_{1:N}$ using $\xi_{1:K} = \{\xi_1, \xi_2, ..., \xi_K\}$ (please refer to~\cite{kerbl202333dgs} for a detailed rendering process), and supervised with $I_{1:K}$ by the rendering loss:
\begin{equation} \label{Limage}
    \mathcal{L}_{image} = \sum_{k\in {1...K}}\lambda\| I_k - {\hat{I_k}}\|_1 + \mathcal{L}_{SSIM}(I_k, \hat{I_k}),
\end{equation}
where $\mathcal{L}_{SSIM}(\cdot)$ represents a SSIM loss and $\lambda$ is a hyper-parameter (set to $0.2$ as mentioned in~\cite{kerbl202333dgs}).



% \subsection{Gaussian Grouping}
% To overcome the lack of fine-grained scene understanding in 3DGS, Gaussian Grouping~\cite{ye2023gaussiangrouping} extends 3DGS by incorporating segmentation capabilities. Along with $I_{1:K}$, Gaussian Grouping additionally takes the Segment Anything Model (SAM) to produce 2D semantic segmentation masks $S_{1:K} = \{S_1, S_2, ..., S_K\}$ from multiple views as inputs, and an additional 16-dimensional parameter $\mathbf{e}_i \in \mathbb{R}^{16}$ is introduced to represent a 3D Identity Encoding for each Gaussian $G_i$. Therefore, each Gaussian $G_i$ is extended as $G_i = \{\mathbf{p}_i, \mathbf{s}_i, \mathbf{q}_i, {\alpha}_i, \mathbf{c}_i, \mathbf{e}_i\}$. To make sure $G_{1:K}$ learns to segment each object represented by $S_{1:K}$ in the scene, a 2D identity loss $\mathcal{L}_{id}$ is applied by calculating cross-entropy between $\hat{S}_{1:K}$ and $S_{1:K}$, where $\hat{S}_{1:K} = \{\hat{S}_1, \hat{S}_2, ... , S_K\}$ denotes the rendered segmentation maps from $G_{1:K}$. Additionally, to further ensure that the Gaussians having the same identities are grouped together, a 3D regularization loss $\mathcal{L}_{3D}$ is applied to enforce each $G_i$'s k-nearest 3D spatial neighbors to be close in their feature distance of Identity Encodings. Please refer to the original paper~\cite{ye2023gaussiangrouping} for detailed formulations of segmentation map rendering and $\mathcal{L}_{3D}$. The design of Gaussian Grouping ensures that the segmentation results are coherent across multiple views, enabling the automatic generation of binary masks for any queried object in the scene.

% \subsection{Relightable Gaussians}
% Different from Gaussian Grouping, Relightable Gaussians~\cite{gao2023relightable} extends the capabilities of Gaussian Splatting by incorporating Disney-BRDF~\cite{burley2012brdf} decomposition and ray tracing to achieve realistic point cloud relighting. 
% % Unlike traditional Gaussian Splatting, which primarily focuses on appearance and geometry modeling, Relightable Gaussians also aim to model the physical interaction of light with different surfaces in the scene.
% Specifically, for each Gaussian $G_i$, the original color coefficients $\mathbf{c}_i$ is decomposed into a 3-dimensional base color $\mathbf{b}_i \in [0,1]^3$, a 1-dimensional roughness $r \in [0,1]$, and incident light coefficients $\mathbf{l}_i$ for spherical harmonics in degree of 3. Subsequently, the Physical-Based Rendering (PBR) process and a point-based ray tracing are applied to obtain the colored 2D images $\hat{I}^{PBR}_{1:K}$ and supervised by $I_{1:K}$ using the aforementioned $\mathcal{L}_{image}$ in Eqn.~\ref{Limage}. Besides the above extensions on PBR for relighting, Relightable Gaussians also introduces a 3-dimensional normal $\mathbf{n}_i$ for $G_i$ and leverages several techniques, including an unsupervised estimation of a depth map $D_i$ from each input view $\xi_i$, to enhance the geometry accuracy and smoothness. Please refer to the original paper of Relightable Gaussians~\cite{gao2023relightable} for detailed explanations.  

\pj{Our pipeline for 3D Inpainting is built on top of the 3DGS model. Additionally, we incorporate the design of Gaussian Grouping~\cite{ye2023gaussiangrouping} to introduce a 16-dimensional semantic feature $\mathbf{e}_i \in \mathbb{R}^{16}$ for each Gaussian $G_i$, so that the 2D segmentation maps of the Gaussians $G_{1:K}$ is rendered and the object mask for the object to be removed can be directly produced, as mentioned in Sect.~\ref{subsec:3Dinpaint}.
By combining these methods as our backbone, we are able to perform an automatic inpainting mask generation and a reliable depth estimation for depth-guided 3D inpainting. Please refer to our Supplementary material for a more detailed explanation of our backbones.}

% the backbone representation by parameterizing each Gaussian $G_i$ as $G_i = \{\mathbf{p}_i, \mathbf{s}_i, \mathbf{q}_i, {\alpha}_i, \mathbf{c}_i, \mathbf{e}_i, \mathbf{b}_i, r,  \mathbf{l}_i, \mathbf{n}_i\}$. By combining these methods, we are able to perform an automatic inpainting mask generation and a reliable depth estimation for depth-guided 3D inpainting.

\section{Model Formulation for FlowAR Architecture}\label{sec:model_formulation_of_flowar}
In this section, we provide a mathematical definition for every module of FlowAR. 
Section~\ref{sec:sample_function} provides the definition of up-sample and down-sample functions.
In Section~\ref{sec:downsample_tokenizer}, we mathematically define the VAE tokenizer.  Section~\ref{sec:ar_transformer} presents a mathematical formulation for every module in the autoregressive transformer in FlowAR. Section~\ref{sec:flow_matching} provides some important definitions of the flow-matching architecture. In Section~\ref{sec:inference_of_flowar}, we also provide a rigorous mathematical definition for the overall architecture of the FlowAR Model during the inference process.



\subsection{Sample Function}\label{sec:sample_function}
We define the bicubic upsampling function.
\begin{definition}[Bicubic Upsampling Function]\label{def:bicubic_up_sample_function}
Given the following:
\begin{itemize}
    \item {\bf Input tensor:} $\X \in \R^{h \times w \times c}$ where $h,w,c$ represent height, width, and the number of channels, respectively.
    \item {\bf Scaling factor:} A positive integer $r \geq 1$.
    \item {\bf Bicubic kernel:} $W:\R \to [0,1]$
\end{itemize}
The bicubic upsampling function $\phi_{\mathrm{up}}(\X,r)$ computes an output tensor $\Y \in \R^{rh \times rw \times c}$. For every output position $i \in [rh], j \in [rw], l \in [c]$:
\begin{align*}
    \Y_{i,j,l} =  \sum_{s=-1}^2 \sum_{t=-1}^2 W(s) \cdot W(t) \cdot \X_{\lfloor \frac{i}{r}\rfloor+s, \lfloor \frac{j}{r}\rfloor+t,l}
\end{align*}

\end{definition}
Next, we define the downsampling function.
\begin{definition}[Linear Downsampling Function]\label{def:linear_down_sample_function}
Given the following:
\begin{itemize}
    \item {\bf Input tensor:} $\X \in \R^{h \times w \times c}$ where $h,w,c$ represent height, width, and the number of channels, respectively.
    \item {\bf Scaling factor:} A positive integer $r \geq 1$.
\end{itemize}
The linear downsampling function $\phi_{\mathrm{down}}(\X,r)$ computes an output tensor $\Y \in \R^{(h/r) \times (w/r) \times c}$. Let $\Phi_{\mathrm{down}} \in \R^{(h/r \cdot w/r) \times hw}$ denote a linear transformation matrix. Reshape $\X$ into the matrix $X \in \R^{hw \times c}$ by flattening its spatial dimensions.  The output matrix is defined via:
\begin{align*}
     Y = \Phi_{\mathrm{down}}X \in \R^{(h/r \cdot w/r) \times c},
\end{align*}
Then reshaped back to $\Y \in \R^{(h/r) \times(w/r) \times c}$.
\end{definition}


\subsection{Multi-Scale Downsampling Tokenizer}\label{sec:downsample_tokenizer}
Given an input image, the FlowAR model will utilize the VAE to generate latent representation $\X^{\R^{h \times w \times c}}$. To meet the requirements of Multi-Scale autoregressive image generation, FlowAR uses a Multi-Scale VAE Tokenizer to downsample $\X$ and generate Token Maps of different sizes.
\begin{definition}[Multi-Scale Downsampling Tokenizer]\label{def:downsample_tokenizer}
 Given the following:
 \begin{itemize}
     \item {\bf Latent representation tensor:} $\X\in \R^{h \times w \times c}$ generated by VAE.
     \item {\bf Number of scales:} $K \in \mathbb{N}$.
     \item {\bf Base scaling factor:}  positive integer $a \geq 1$
     \item {\bf Downsampling functions:} For $i \in [K]$, define scale-specific factors $r_i := a^{K-i}$ and use the linear downsampling function $\phi_{\mathrm{down}}(\X,r_i)$ from Definition~\ref{def:linear_down_sample_function}.
 \end{itemize}
 Then tokenizer outputs a sequence of token maps $\{\Y^2, \Y^2,\dots, \Y^K\}$, where the $i$-th token map is
 \begin{align*}
     \Y^i := \phi_{\mathrm{down},i}(\X,r_i) \in \R^{(h / r_i) \times (w/r_i) \times c},
 \end{align*}
 Formally, the tokenizer is defined as
 \begin{align*}
     \mathsf{TN}(\X) := \{\Y^{1}, \dots,\Y^{K}\}.
 \end{align*} 

\end{definition}
\begin{remark}
    In \cite{ryh+24}, the base factor is set to $a = 2$, resulting in exponentially increasing scales $r_i = 2^{K-i}$ for $i \in [K]$.
\end{remark}

\subsection{Autoregressive Transformer}\label{sec:ar_transformer}

The autoregressive transformer is a key module of the FlowAR model. We will introduce each layer of autoregressive transformer in this section.
\begin{definition}[Attention Layer]\label{def:attn_layer}
Given the following:
\begin{itemize}
    \item {\bf Input tensor:} $\X \in \R^{h \times w \times c}$ where $h,w,c$ represent height, width, and the number of channels, respectively.
    \item {\bf Weight matrices:} $W_Q,W_K,W_V \in \R^{c \times c}$ will be used in query, key, and value projection, respectively.
\end{itemize}
The attention layer $\mathsf{Attn}(\X)$ computes an output tensor $\Y \in \R^{h \times w \times c}$ as follows:
\begin{itemize}
    \item {\bf Reshape:} Flatten $\X$ into a matrix $X \in \R^{hw \times c}$ with spatial dimensions collapsed.
    \item {\bf Compute attention matrix:} For $i,j \in [hw]$, compute pairwise scores:
    \begin{align*}
        A_{i,j} := & ~\exp(  X_{i,*}   W_Q   W_K^\top   X_{j,*}^\top), \text{~~for~} i, j \in [hw].
    \end{align*}
    \item {\bf Normalization:} Compute diagnal matrix $D:=\diag(A {\bf 1}_n) \in \R^{hw \times hw}$, where ${\bf 1}_n$ is the all-ones vector. And compute:
    \begin{align*}
         Y := D^{-1}AXW_V \in \R^{hw \times c}.
    \end{align*}
    then reshape $Y$ to $\Y \in \R^{h \times w \times c}$.
\end{itemize}
\end{definition}


Then, we define the multiple-layer perception layer.
\begin{definition}[MLP layer]\label{def:mlp}
Given the following:
\begin{itemize}
    \item {\bf Input tensor:} $\X \in \R^{h \times w \times c}$ where $h,w,c$ represent height, width, and the number of channels, respectively.
    \item {\bf Weight matrices and bias vector:} $W \in \R^{c \times d}$ and $b \in \R^{1 \times d}$.
\end{itemize}
The MLP layer computes an output tensor $\Y \in \R^{h \times w \times d}$ as follows:
\begin{itemize}
    \item {\bf Reshape:} Flatten $\X$ into a matrix $X \in \R^{hw \times c}$ with spatial dimensions collapsed.
    \item {\bf Affine transformation:} For all $j \in [hw]$, compute
    \begin{align*}
        Y_{j,*} = \underbrace{X_{j,*}}_{1\times c} \cdot \underbrace{W}_{c \times d} + \underbrace{b}_{1 \times d}
    \end{align*}
    Then reshape $Y \in \R^{hw \times d}$ into $\Y \in \R^{h \times w \times d}$.
\end{itemize}
The operation is denoted as $\Y := \mathsf{MLP}(\X,c,d)$.
\end{definition}


Next, we introduce the definition of the feedforward layer.
\begin{definition}[Feed forward layer]\label{def:ffn}
Given the following:
\begin{itemize}
    \item {\bf Input tensor:} $\X \in \R^{h \times w \times c}$ where $h,w,c$ represent height, width, and the number of channels, respectively.
    \item {\bf Weight matrices and bias vector:} $W_1, W_2 \in \R^{c \times d}$ and $b_1, b_2 \in \R^{1 \times d}$.
    \item {\bf Activation:} $\sigma:\R \to \R$ denotes the $\mathsf{ReLU}$ activation function which is applied element-wise.
\end{itemize}
The feedforward layer computes an output tensor $\Y \in \R^{h \times w \times d}$ as follows:
\begin{itemize}
    \item {\bf Reshape:} Flatten $\X$ into a matrix $X \in \R^{hw \times c}$ with spatial dimensions collapsed.
    \item {\bf Transform:} For each $j \in [hw]$, compute 
    \begin{align*}
        Y_{j,*}=  \underbrace{X_{j,*}}_{1 \times c} +  \sigma (\underbrace{X_{j,*}}_{1\times c} \cdot \underbrace{W_1}_{c \times c} + \underbrace{b_1}_{1\times c}) \cdot \underbrace{W_2}_{c \times c} + \underbrace{b_2}_{1 \times c} \in \R^{1 \times c}
    \end{align*}
    where $\sigma$ acts element-wise on intermediate results. Then reshape $Y \in \R^{hw \times c}$ into $\Y \in \R^{h \times w \times c}$.
\end{itemize}
The operation is denoted as $\Y := \mathsf{FFN}(\X)$.

\end{definition}

To move on, we define the layer normalization layer.
\begin{definition}[Layer Normalization Layer]\label{def:ln}
    Given the following:
    \begin{itemize}
        \item {\bf Input tensor:} $\X \in \R^{h \times w \times c}$ where $h,w,c$ represent height, width, and the number of channels, respectively.
    \end{itemize}
    The layer normalization computes $\Y$ through
    \begin{itemize}
        \item {\bf Reshape:} Flatten $\X$ into a matrix $X \in \R^{hw \times c}$ with spatial dimensions collapsed.
        \item {\bf Normalize:} For each $j \in [hw]$, compute
        \begin{align*}
            Y_{j,*} =  \frac{X_{j,*}-\mu_j}{\sqrt{\sigma_j^2}}
        \end{align*}
        where
        \begin{align*}
            \mu_j := \sum_{k=1}^c X_{j,k}/c, ~~ \sigma_{j}^2 = \sum_{k=1}^c(X_{j,k}-\mu_j)^2/c
        \end{align*}
        Then reshape $Y \in \R^{hw \times c}$ into $\Y \in \R^{h \times w \times c}$.
    \end{itemize}
    The operation is denoted as $\Y := \mathsf{LN}(\X)$.
\end{definition}


Now, we can proceed to show the definition of the autoregressive transformer.
\begin{definition}[Autoregressive Transformer]\label{def:ar_transformer}
    Given the following:
    \begin{itemize}
        \item {\bf Input tensor:} $\X \in \R^{h \times w \times c}$ where $h,w,c$ represent height, width, and the number of channels, respectively.
        \item {\bf Scales number:} $K \in \mathbb{N}$ denote the number of total scales in FlowAR.
        \item {\bf Token maps:} For $i \in [K]$, $\Y_i \in \R^{(h/r_i) \times (w/r_i) \times c}$ generated by the Multi-Scale Downsampling Tokenizer defined in Definition~\ref{def:downsample_tokenizer} where $r_i = a^{K-i}$ with base $a \in \mathbb{N}^+$.
        \item {\bf Upsampling functions:}  For $i \in [K]$, $\phi_{\mathrm{up},i}(\cdot,a): \R^{(h/r_i) \times (w/r_i) \times c}\to \R^{(h/r_{i+1}) \times (w/r_{i+1}) \times c}$ from Definition~\ref{def:bicubic_up_sample_function}.
        \item {\bf Attention layer:}  For $i \in [K]$, $\mathsf{Attn}_i(\cdot):\R^{(\sum_{j=1}^i h/r_j \cdot w/r_{j})\times c} \to \R^{(\sum_{j=1}^i h/r_j \cdot w/r_{j})\times c}$ which acts on flattened sequences of dimension defined in Definition~\ref{def:attn_layer}.
        \item {\bf Feed forward layer: } For $i \in [K]$, $\mathsf{FFN}_i(\cdot): \R^{(\sum_{j=1}^i h/r_j \cdot w/r_{j})\times c} \to \R^{(\sum_{j=1}^i h/r_j \cdot w/r_{j})\times c}$ which acts on flattened sequences of dimension defined in Definition~\ref{def:ffn}.
        \item {\bf Initial condition:} $\Z_{\mathrm{init}} \in \R^{(h/r_1) \times (w/r_1) \times c}$ denotes the initial token maps which encodes class information.
    \end{itemize}
    Then, the autoregressive processing is:
    \begin{enumerate}
        \item {\bf Initialization: } Let $\Z_1:=\Z_{\mathrm{init}}$.
        \item {\bf Iterative sequence construction:} For $i \geq 2$.
        \begin{align*}
            \Z_i := \mathsf{Concat}(\mathsf{Z}_{\mathrm{init}}, \phi_{\mathrm{up}, 1}(\Y^1, a), \ldots, \phi_{\mathrm{up}, i-1}(\Y^{i-1}, a)) \in \R^{(\sum_{j=1}^i h/r_j \cdot w/r_{j})\times c}
        \end{align*}
        where $\mathsf{Concat}$ reshapes tokens into a unified spatial grid.
        \item {\bf Transformer block:} For $i \in [K]$,
        \begin{align*}
            \mathsf{TF}_i(\Z_i) := \mathsf{FFN_i}(\mathsf{Attn}_i(\Z_i)) \in \R^{(\sum_{j=1}^i h/r_j \cdot w/r_{j})\times c}
        \end{align*}
        \item {\bf Output decomposition:} Extract the last scale's dimension   from the reshaped $\mathsf{TF}_i(\Z_i)$ to generate $\wh{\Y}_i \in \R^{(h/r_i) \times (w/r_i) \times c}$.
    \end{enumerate}
\end{definition}


\subsection{Flow Matching}\label{sec:flow_matching}
We begin by outlining the concept of velocity flow in the flow-matching architecture.
\begin{definition}[Flow]\label{def:flow}
Given the following:
\begin{itemize}
    \item {\bf Input tensor:} $\X \in \R^{h \times w \times c}$ where $h,w,c$ represent height, width, and the number of channels, respectively.
    \item {\bf Scales number:} $K \in \mathbb{N}$.
    \item {\bf Noise tensor: } For $i \in [K]$, $\F_i^0 \in \R^{(h / r_i) \times (w/r_i) \times c}$ with every entry sampled from $\mathcal{N}(0,1)$.
    \item {\bf Token maps:} For $i \in [K]$, $\wh{\Y}_i \in \R^{(h / r_i) \times (w/r_i) \times c}$ denote the token maps generated by autoregressive transformer defined in Definition~\ref{def:ar_transformer}.
\end{itemize}
Then, the model does the following:
\begin{itemize}
    \item {\bf Interpolation:} For timestep $t \in [0,1]$ and scale $i$,
    \begin{align*}
        \F_i^t := t \wh{\Y}_i + (1-t) \F_i^0
    \end{align*}
    defining a linear trajectory between noise $\F_0^i$ and target tokens  $\wh{\Y}_i$.
    \item {\bf Velocity Field:} The time-derivative of the flow at scale $i$ is 
    \begin{align*}
        \V^t_i := \frac{\d \F^t_{i}}{\d t} = \wh{\Y_i} -\F^0_i.
    \end{align*}
    constant across $t$ due to linear interpolation.
\end{itemize}
\end{definition}

To move forward, we propose an approach to enhance the performance of the flow-matching layer by replacing linear interpolation with a Quadratic Bézier curve.

\begin{definition}[High Order Flow]\label{def:high_order_flow}
Given the following:
\begin{itemize}
    \item {\bf Input tensor:} $\X \in \R^{h \times w \times c}$ where $h,w,c$ represent height, width, and the number of channels, respectively.
    \item {\bf Scales number:} $K \in \mathbb{N}$.
    \item {\bf Noise tensor: } For $i \in [K]$, $\F_i^0 \in \R^{(h / r_i) \times (w/r_i) \times c}$ with every entry sampled from $\mathcal{N}(0,1)$.
    \item {\bf Token maps:} For $i \in [K]$, $\wh{\Y}_i \in \R^{(h / r_i) \times (w/r_i) \times c}$ denote the token maps generated by autoregressive transformer defined in Definition~\ref{def:ar_transformer}.
\end{itemize}
Then, the model does the following:
\begin{itemize}
    \item {\bf Interpolation:} For timestep $t \in [0,1]$ and scale $i$,
    \begin{align*}
        \F_i^t := (1-t)^2 \F_i^0 + 2t(1-t) \mathsf{C}_i + t^2 \wh{\Y}_i
    \end{align*}
    defining a quadratic Bézier curve as the interpolation path between the initial noise and the target data. To be noticed, we take $\mathsf{C} = \frac{\F_i^0+\wh{\Y}_i}{2}$ as a control point that governs the curvature of the trajectory. This formulation replaces the standard linear interpolation with a higher-order flow, enabling a smoother and more flexible transition from noise to data in the flow-matching framework.
    \item {\bf Velocity Field:} The time-derivative of the flow at scale $i$ is 
    \begin{align*}
        \V^t_i := &~\frac{\d \F^t_{i}}{\d t}\\ =&~ -2(1-t)\F_i^0 + 2(1-2t) \mathsf{C}_i + 2t \wh{\Y}_i 
    \end{align*}
    constant across $t$ due to linear interpolation.
\end{itemize}
\end{definition}

We are now able to define the flow-matching layer, which is integrated in the FlowAR model.
\begin{definition}[Flow Matching Architecture]\label{def:flow_matching_architecture}
Given the following:
\begin{itemize}
    \item {\bf Input tensor:} $\X \in \R^{h \times w \times c}$ where $h,w,c$ represent height, width, and the number of channels, respectively.
    \item {\bf Scales number:} $K \in \mathbb{N}$ denote the number of total scales in FlowAR.
    \item {\bf Token maps:} For $i \in [K]$, $\wh{\Y}_i \in \R^{(h / r_i) \times (w/r_i) \times c}$ denote the token maps generated by autoregressive transformer defined in Definition~\ref{def:ar_transformer}.
    \item {\bf Interpolation Tokens:} For $i \in [K]$, $\F_i^t \in \R^{(h / r_i) \times (w/r_i) \times c}$ denote interpolated input defined in Definition~\ref{def:flow}.
    \item {\bf Time step:} For $i \in [K]$, $t_i \in [0,1]$ denotes timestep.
    \item {\bf Attention layer:}  For $i \in [K]$, $\mathsf{Attn}_i(\cdot):\R^{h/r_i \times w/r_i \times c} \to \R^{h/r_i \times w/r_i \times c}$ is defined in Definition~\ref{def:attn_layer}.
    \item {\bf MLP layer:}  For $i \in [K]$, $\mathsf{MLP}_i(\cdot,c,d):\R^{h/r_i \times w/r_i \times c} \to \R^{h/r_i \times w/r_i \times c}$ is defined in Definition~\ref{def:mlp}.
    \item {\bf LN layer:} For $i \in [K]$, $\mathsf{LN}_i(\cdot):\R^{h/r_i \times w/r_i \times c} \to \R^{h/r_i \times w/r_i \times c}$ is defined in Definition~\ref{def:ln}.
\end{itemize}
The computation steps of flow-matching layers are as follows:
\begin{itemize}
    \item {\bf Time-conditioned parameter generation:}
    \begin{align*}
        \alpha_1, \alpha_2, \beta_1, \beta_2, \gamma_1, \gamma_2:=  \mathsf{MLP}_i(\wh{\Y}_i + t_i \cdot {\bf 1}_{(h / r_i) \times (w/r_i) \times c},c,6c)
    \end{align*}
    \item {\bf Intermediate variable computation:}
    \begin{align*}
        \F'^{t_i}_i:= \mathsf{Attn}_i (\gamma_1 \circ \mathsf{LN}(\F_i^{t_i}) + \beta_1) \circ \alpha_1
    \end{align*}
    with $\circ$ denoting Hadamard (element-wise) product.
    \item {\bf Final projection:}
    \begin{align*}
        \F''^{t_i}_i := \mathsf{MLP}_i(\gamma_2 \circ \mathsf{LN}(\F'^{t_i}_i)+ \beta_2,c,c) \circ \alpha_2
    \end{align*}
\end{itemize}
The operation is denoted as $\F''^{t_i}_i := \mathsf{NN}_i(\wh{\Y_i},\F_i^{t_i},t_i)$
\end{definition}


\subsection{Inference of FlowAR Architecture}\label{sec:inference_of_flowar}
The inference phase of the FlowAR model differs from the training phase. During inference, neither the VAE nor the Multi-Scale Downsampling layers are used. Instead, given an initial token map representing class embeddings, the model autoregressively generates token maps across scales.
\begin{definition}[FlowAR Inference Architecture]\label{def:flow_architecture_inference}
    Given the following:
    \begin{itemize}
        \item {\bf Scales number:} $K \in \mathbb{N}$ denote the number of total scales in FlowAR.
        \item {\bf Scale factor:} For $i \in [K]$, $r_i:= a^{K-i}$ where base factor $a \in \mathbb{N}^+$.
        \item {\bf Upsampling functions:}  For $i \in [K]$, $\phi_{\mathrm{up},i}(\cdot,a): \R^{(h/r_i) \times (w/r_i) \times c}\to \R^{(h/r_{i+1}) \times (w/r_{i+1}) \times c}$ from Definition~\ref{def:bicubic_up_sample_function}.
        \item {\bf Attention layer:}  For $i \in [K]$, $\mathsf{Attn}_i(\cdot):\R^{(\sum_{j=1}^i h/r_j \cdot w/r_{j})\times c} \to \R^{(\sum_{j=1}^i h/r_j \cdot w/r_{j})\times c}$ which acts on flattened sequences of dimension defined in Definition~\ref{def:attn_layer}.
        \item {\bf Feed forward layer: } For $i \in [K]$, $\mathsf{FFN}_i(\cdot): \R^{(\sum_{j=1}^i h/r_j \cdot w/r_{j})\times c} \to \R^{(\sum_{j=1}^i h/r_j \cdot w/r_{j})\times c}$ which acts on flattened sequences of dimension defined in Definition~\ref{def:ffn}.
        \item {\bf Flow matching layer:} For $i \in [K]$, $\mathsf{NN}_i(\cdot,\cdot,\cdot):\R^{h/r_i \times w/r_i \times c}\times \R^{h/r_i \times w/r_i \times c}\times \R \to \R^{h/r_i \times w/r_i \times c}$ denote the flow-matching layer defined in Definition~\ref{def:flow_matching_architecture}.
        \item {\bf Initial condition:} $\Z_{\mathrm{init}} \in \R^{(h/r_1) \times (w/r_1) \times c}$ denotes the initial token maps which encodes class information.
        \item {\bf Time steps:} For $i \in [K]$, $t_i \in [0,1]$ denotes time steps.
        \item {\bf Interpolated inputs:} For $i \in [K]$, $\F_i^{t_i} \in \R^{h/r_i \times w/r_i \times c}$ defined in Definition~\ref{def:flow}.
        \item {\bf Cumulative dimensions:} We define $\wt{h}_i := \sum_{j=1}^i h/r_j$ and  $\wt{w}_i := \sum_{j=1}^i w/r_j$ for $i \in [K]$.
    \end{itemize}
    The FlowAR model conducts the following recursive construction:
    \begin{itemize}
        \item {\bf Base case $i=1$:}
        \begin{align*}
            &~\Z_1 = \Z_{\mathrm{init}}\\
            &~\wh{\Y}_1 = \mathsf{FFN}_1(\mathsf{Attn}_1(\Z_1))\\
            &~\wt{\Y}_1 = \mathsf{NN}_1(\wh{\Y}_1,\F_{1}^{t_1},t_1)
        \end{align*}
        \item {\bf Inductive step $i \geq 2$:}
        \begin{itemize}
            \item {\bf Spatial aggregation:}
            \begin{align*}
                \Z_i = \mathsf{Concat}(\Z_{\mathrm{init}},\phi_{\mathrm{up},1}(\wt{\Y}_{i-1}),\dots,\phi_{\mathrm{up},i-1}(\wt{\Y}_{i-1})) \in \R^{(\sum_{j=1}^i h/r_j \cdot w/r_j)\times c}
            \end{align*}
            \item {\bf Autoregressive transformer computation:}
            \begin{align*}
                \wh{\Y}_i = \mathsf{FFN}_i(\mathsf{Attn}_i(\Z_1))_{\wt{h}_{i-1}:\wt{h}_{i-1},\wt{w}_{i}:\wt{w}_{i},0:c}
            \end{align*}
            \item {\bf Flow matching layer:}
            \begin{align*}
                \wt{\Y}_i = \mathsf{NN}_i(\wh{\Y}_i,\F_{i}^{t_i},t_i)
            \end{align*}
        \end{itemize}
        The final output is $\wt{\Y}_K \in \R^{h \times w \times c}$.
    \end{itemize}
   
\end{definition}


\section{Complexity of FlowAR Architecture}\label{sec:main_result}
This section presents key results on the circuit complexity of fundamental modules in the FlowAR architecture. Section~\ref{lem:matrix_multi} analyzes matrix multiplication, while Section~\ref{sec:down_up_tc0} examines the up-sampling and down-sampling functions. In Sections~\ref{sec:mlp_tc0} and \ref{sec:ffn_tc0}, we compute the circuit complexity of the MLP and FFN layers, respectively. Sections~\ref{sec:attention_tc0} and \ref{sec:ln_tc0} focus on the single attention layer and layer normalization. Section~\ref{sec:flow_matching} addresses the flow-matching layer. Finally, Section~\ref{sec:main_result} presents our main result, establishing the circuit complexity bound for the complete FlowAR architecture.
\subsection{Computing Matrix Products in \texorpdfstring{$\TC^0$}{}}\label{sec:compute_matrix_product}
we demonstrate that matrix multiplication is computable in $\mathsf{TC}^0$, which will be used later.
\begin{lemma}[Matrix multiplication in $\TC^0$, \cite{cll+24}]\label{lem:matrix_multi}
    Let the precision $p \leq \poly(n)$. Let $X \in \mathsf{F}_p^{n_1 \times d}, Y \in \mathsf{F}_p^{d \times n_2}$ be matrices. Assume $n_1\leq\poly(n), n_2\leq\poly(n)$. The matrix product $XY$ can be computed by a uniform $\mathsf{TC}^0$ circuit with:
    \begin{itemize}
        \item Size: $\poly(n)$.
        \item Depth: $d_{\mathrm{std}}+d_{\oplus}$.
    \end{itemize}
    where $d_{\mathrm{std}}$ and $d_{\oplus}$ are defined in Definition~\ref{lem:float_operations_TC}.
\end{lemma}


\subsection{Computing Down-Sampling and Up-Sampling in in \texorpdfstring{$\TC^0$}{}}\label{sec:down_up_tc0}
In this section, we show that Up-Sampling can be efficiently computable by a uniform $\mathsf{TC}^0$ circuit.
\begin{lemma}[Up-Sampling computation in $\mathsf{TC}^0$]\label{lem:up_tc0}
     Let $\X \in \R^{h \times w \times c}$ be the input tensor. Let $\phi_{\mathrm{up}}(X,r):\R^{h \times w \times c} \to \R^{(hr) \times (wr) \times c}$ denote the bicubic up sample function defined in Definition~\ref{def:bicubic_up_sample_function}. Assume $n = h = w$. Assume $r \leq n$. Assume $c \leq n$. Assume $p \leq \poly(n)$.
    The linear up sample function can be computed by a uniform $\mathsf{TC}^0$ circuit with:
    \begin{itemize}
        \item Size: $\poly(n)$.
        \item Depth: $2d_\mathrm{std} + d_{\oplus}$.
    \end{itemize}
    where $d_{\mathrm{std}}$ and $d_{\oplus}$ are defined in Definition~\ref{lem:float_operations_TC}.
\end{lemma}
\begin{proof}
    For each $i \in [nr], j \in [nr], l \in [c]$, we need to compute $\phi_{\mathrm{up}}(\X,r)_{i,j,l} = \sum_{s=-1}^2 \sum_{t=-1}^2 W(s)\cdot W(t)\cdot \X_{\frac{i}{r}+s,\frac{j}{r}+s,l}$.  We need a $2d_{\mathrm{std}}$ depth and $\poly(n)$ size circuit to compute $W(s)\cdot W(t)\cdot \X_{\frac{i}{r}+s,\frac{j}{r}+s,l}$ by Part 1 of Lemma~\ref{lem:float_operations_TC} and for all $s,t \in \{-1,0,1,2\}$, these terms can be computed in parallel. Furthermore, by Part 3 of Lemma~\ref{lem:float_operations_TC}, we can need a $d_{\oplus}$ depth and $\poly(n)$ size circuit to compute $\sum_{s=-1}^2 \sum_{t=-1}^2 W(s)\cdot W(t)\cdot \X_{\frac{i}{r}+s,\frac{j}{r}+s,l}$. Since the computation of $\phi_{\mathrm{up}}(\X,r)_{i,j,l}$ needs a $2d_{\mathrm{std}}+d_{\oplus}$ depth and $\poly(n)$ size circuit.

    Since for all $i \in [nr], j \in [nr], l \in [c]$, we can compute $\phi_{\mathrm{up}}(\X,r)_{i,j,l}$ in parallel, then the total depth of the circuit is $2d_{\mathrm{std}} + d_{\oplus}$ and size remains $\poly(n)$.
\end{proof}

Then, we move forward to consider the down-sampling function.
\begin{lemma}[Down-Sampling computation in $\mathsf{TC}^0$]\label{lem:down_tc0}
    Let $\X \in \R^{h \times w \times c}$ be the input tensor. Let $\phi_{\mathrm{down}}(X,r)$ be the linear down sample function from Definition~\ref{def:linear_down_sample_function}. Assume $n = h = w$. Assume $r \leq n$. Assume $c \leq n$. Assume $p \leq \poly(n)$.

    The function $\phi_{\mathrm{down}}$ can be computed by a uniform $\mathsf{TC}^0$ circuit with

    \begin{itemize}
        \item Size: $\poly(n)$.
        \item Depth: $d_\mathrm{std} + d_{\oplus}$.
    \end{itemize}
    where $d_{\mathrm{std}}$ and $d_{\oplus}$ are defined in Definition~\ref{lem:float_operations_TC}.
\end{lemma}
\begin{proof}
    By Definition~\ref{def:linear_down_sample_function}, we know that down-sampling computation is essentially matrix multiplication. Then, by Lemma~\ref{lem:matrix_multi}, we can easily get the proof.
\end{proof}







\subsection{Computing Multiple-layer Perceptron in \texorpdfstring{$\TC^0$}{} }\label{sec:mlp_tc0}
 We prove that MLP computation can be efficiently simulated by a uniform $\mathsf{TC}^0$ circuit.
\begin{lemma}[MLP computation in $\mathsf{TC}^0$, informal version of Lemma~\ref{lem:mlp_tc0_formal}]\label{lem:mlp_tc0_informal}
    Given an input tensor $\X\in \R^{h \times w \times c}$. Let $\mathsf{MLP}(\X,c,d)$ be the MLP layer defined in Definition~\ref{def:mlp}. Under the following constraints:
    \begin{itemize}
        \item Satisfy $h = w = n$,
        \item Channel bounds: $c, d\leq n$,
        \item Precision: $p \leq \poly(n)$,
    \end{itemize}
    The $\mathsf{MLP}(\X,c,d)$ function can be computed by a uniform $\mathsf{TC}^0$ circuit with:
    \begin{itemize}
        \item Size: $\poly(n)$.
        \item Depth: $2d_\mathrm{std} + d_{\oplus}$.
    \end{itemize}
    with $d_{\mathrm{std}}$ and $d_{\oplus}$ defined in Definition~\ref{lem:float_operations_TC}.
\end{lemma}




\subsection{Computing Feed-Forward Layer in \texorpdfstring{$\TC^0$}{} }\label{sec:ffn_tc0}
We also prove that feed-forward network computation can be simulated by a uniform $\mathsf{TC}^0$ circuit.
\begin{lemma}[FFN computation in $\mathsf{TC}^0$, informal version of Lemma~\ref{lem:ffn_tc0_formal}]\label{lem:ffn_tc0_informal}
 Given an input tensor $\X\in \R^{h \times w \times c}$. Let $\mathsf{FFN}(X):\R^{h \times w \times c} \to \R^{h \times w \times c}$ as defined in Definition~\ref{def:ffn}. Under the following constraints:
 \begin{itemize}
     \item Satisfy $h = w = n$,
         \item Channel bound: $c \leq n$,
         \item Precision bound: $p \leq \poly(n)$.
 \end{itemize}
The $\mathsf{FFN}(\X)$ layer can be computed by a uniform $\mathsf{TC}^0$ circuit with:
\begin{itemize}
    \item Size: $\poly(n)$.
    \item Depth: $6d_\mathrm{std} + 2d_{\oplus}$.
\end{itemize}
with $d_{\mathrm{std}}$ and $d_{\oplus}$ defined in Definition~\ref{lem:float_operations_TC}.
\end{lemma}




\subsection{Computing Single Attention Layer in \texorpdfstring{$\TC^0$}{}.} \label{sec:attention_tc0}
 We prove the single attention layer can be efficiently simulated by a uniform $\mathsf{TC}^0$ circuit.
\begin{lemma}[Attention layer computation in $\mathsf{TC}^0$, informal version of Lemma~\ref{lem:attn_tc0_formal}]\label{lem:attn_tc0_informal}
     Given an input tensor $\X \in \R^{h \times w \times c}$. Let $\mathsf{Attn}(X):\R^{h \times w \times c} \to \R^{h \times w \times c}$ as defined in Definition~\ref{def:attn_layer}. Under the following constraints:
     \begin{itemize}
         \item Satisfy $h = w = n$,
         \item Channel bound: $c \leq n$,
         \item Precision bound: $p \leq \poly(n)$.
     \end{itemize}
     The $\mathsf{Attn}(\X)$ layer can be computed by a uniform $\mathsf{TC}^0$ circuit with:
     \begin{itemize}
        \item Size: $\poly(n)$.
        \item Depth: $6(d_{\mathrm{std}} + d_{\oplus}) + d_{\exp}$.
     \end{itemize}    
     with $d_{\mathrm{std}}$ and $d_{\oplus}$ defined in Definition~\ref{lem:float_operations_TC}, $d_{\exp}$ defined in Definition~\ref{lem:exp}.
\end{lemma}



\subsection{Computing Layer-wise Norm Layer in \texorpdfstring{$\TC^0$}{}.}\label{sec:ln_tc0}
We prove that the layer normalization layer can be efficiently simulated by a uniform $\mathsf{TC}^0$ circuit.
\begin{lemma}[Layer normalization layer computation in $\TC^0$, informal version of Lemma~\ref{lem:ln_tc0_formal}]\label{lem:ln_tc0_informal}
    Given an input tensor $\X \in \R^{h \times w \times c}$. Let $\mathsf{LN}(X):\R^{h \times w \times c} \to \R^{h \times w \times c}$ as defined in Definition~\ref{def:ln}. Under the following constraints:
    \begin{itemize}
         \item Satisfy $h = w = n$,
         \item Channel bound: $c \leq n$,
         \item Precision bound: $p \leq \poly(n)$.
     \end{itemize}
     The $\mathsf{LN}(\X)$ layer can be computed by a uniform $\mathsf{TC}^0$ circuit with:
     \begin{itemize}
        \item Size: $\poly(n)$.
        \item Depth: $5d_\mathrm{std} + 2d_{\oplus} + d_\mathrm{sqrt}$.
     \end{itemize}    
     with $d_{\mathrm{std}}$ and $d_{\oplus}$ defined in Definition~\ref{lem:float_operations_TC}, $d_{\mathrm{sqrt}}$ defined in Definition~\ref{lem:sqrt}.
\end{lemma}


\subsection{Computing Flow Matching Layer in \texorpdfstring{$\TC^0$}{}.}\label{sec:fl_tc0}
We prove that the flow-matching layer can be efficiently simulated by a uniform $\mathsf{TC}^0$ circuit.
\begin{lemma}[Flow matching layer computation in $\TC^0$]\label{lem:fm_tc0}
     Given an input tensor $\X \in \R^{h \times w \times c}$. Let $\mathsf{NN}(X)$ denote the flow-matching layer defined in Definition~\ref{def:flow_matching_architecture}. Under the following constraints:
     \begin{itemize}
         \item Satisfy $h = w = n$,
         \item Channel bound: $c \leq n$,
         \item Precision bound: $p \leq \poly(n)$.
     \end{itemize}
     The $\mathsf{NN}(\cdot, \cdot,\cdot)$ can be computed by a uniform $\mathsf{TC}^0$ circuit with
     \begin{itemize}
        \item Size: $\poly(n)$.
        \item Depth: $26d_{\mathrm{std}}+ 12 d_{\oplus} + 2d_{\mathrm{sqrt}} + d_{\exp}$.
     \end{itemize} 
    with $d_{\mathrm{std}}$ and $d_{\oplus}$ defined in Definition~\ref{lem:float_operations_TC}, $d_{\exp}$ defined in Definition~\ref{lem:exp} and $d_{\mathrm{sqrt}}$ defined in Definition~\ref{lem:sqrt}.
\end{lemma}

\begin{proof}
    {\bf Considering Step 1 in the flow-matching layer:}
    By Lemma~\ref{lem:mlp_tc0_formal}, parameters $\alpha_1,\alpha_2,\beta_1,\beta_2,\gamma_1,\gamma_2$ are computed via a circuit with 
    \begin{itemize}
        \item {\bf Depth}: $2d_{\mathrm{std}} + d_\oplus$.
        \item {\bf Size}: $\poly(n)$
    \end{itemize}


    
    {\bf Considering Step 2 in flow-matching layer:} 
     By Lemma~\ref{lem:ln_tc0_formal}, $\mathsf{LN}(\F_i^t)$ requires depth $5d_\mathrm{std} + 2d_{\oplus} + d_\mathrm{sqrt}$. By Lemma~\ref{lem:float_operations_TC}, $A_1 = \gamma_1 \circ \mathsf{LN}(\F_t)+\beta_1$ requires depth $2d_{\mathrm{std}}$. By Lemma~\ref{lem:attn_tc0_formal}, $A_2 = \mathsf{Attn}(A_1)$ requires depth $6(d_{\mathrm{std}}+d_\oplus)+d_{\exp}$. By Lemma~\ref{lem:float_operations_TC} again, scaling $A_2 \circ \alpha_1$ requires depth $d_{\mathrm{std}}$. The total depth requires $14d_{\mathrm{std}} + 8d_{\oplus}+d_{\mathrm{sqrt}}+d_{\exp}$ for step 2.
    

    {\bf Considering Step 3 in flow-matching layer:} By Lemma~\ref{lem:ln_tc0_formal}, $\mathsf{LN}(\F'^t_i)$ requires depth $5d_\mathrm{std} + 2d_{\oplus} + d_\mathrm{sqrt}$. By Lemma~\ref{lem:float_operations_TC}, $A_3 = \gamma_2 \circ \mathsf{LN}(\wh{\F}_t)+\beta_2$ requries depth  $2d_{\mathrm{std}}$. By Lemma~\ref{lem:mlp_tc0_formal}, $A_4 = \mathsf{MLP}(A_3,c,c)$ requires depth $2d_{\mathrm{std}} + d_\oplus$.
    By Lemma~\ref{lem:float_operations_TC} again, $A_4 \circ \alpha_2$ requires depth $d_{\mathrm{std}}$. The total depth requires $10d_{\mathrm{std}}+3d_{\oplus}+d_{\mathrm{sqrt}}$  for step 3..

    Finally, combining the result above, we need a circuit with depth $26d_{\mathrm{std}}+ 12 d_{\oplus} + 2d_{\mathrm{sqrt}} + d_{\exp}$ and size $\poly(n)$ to simulate the flow-matching layer.
    
\end{proof}


\subsection{Circuit Complexity Bound for FlowAR Architecture}\label{sec:main_result_flowar}
We present that the FlowAR Model can be efficiently simulated by a uniform $\mathsf{TC}^0$ circuit.
\begin{theorem}[FlowAR Model computation in $\TC^0$]\label{thm:flowar_tc0}
Given an input tensor $\X \in \R^{h \times w \times c}$. Under the following constraints:
\begin{itemize}
    \item Satisfy $h = w = n$,
    \item Channel bound: $c \leq n$,
    \item Precision bound: $p \leq \poly(n)$.
    \item Number of scales: $K = O(1)$,
    \item $d_{\mathrm{std}},d_\oplus,d_{\mathrm{sqrt}},d_{\exp} = O(1)$.
\end{itemize}
Then, the FlowAR Model can be simulated by a uniform $\TC^0$ circuit family.

\end{theorem}
\begin{proof}
    For every $i \in [K]$, by Lemma~\ref{lem:up_tc0}, Lemma~\ref{lem:down_tc0},  Lemma~\ref{lem:attn_tc0_informal}, Lemma~\ref{lem:ffn_tc0_informal} and Lemma~\ref{lem:fm_tc0}, we can simulate the $i$-th layer of FlowAR Model with a uniform $\TC^0$ circuit whose size is $\poly(n)$ and depth is $O(1)$. Since the total number of layers $K = O(1)$, the composition of all $K$ circuits yields a single uniform $\mathsf{TC}^0$ circuit with
    \begin{itemize}
        \item {\bf Size:} $\poly(n)$,
        \item {\bf Depth: }$O(1)$.
    \end{itemize}
\end{proof}

In Theorem~\ref{thm:flowar_tc0}, we establish that a FlowAR model with $\poly(n)$ precision, constant depth, and $\poly(n)$ size can be efficiently simulated by a $\mathsf{DLOGTIME}$-uniform $\TC^0$ circuit family. This indicates that while the flow-matching architecture enhances the capability of visual autoregressive models, the FlowAR architecture remains inherently limited in expressivity under circuit complexity theory.
\section{Provably Efficient Criteria}\label{sec:efficient_critieria}

\subsection{Approximate Attention Computation}\label{sec:fast_attn}
In this section, we introduce approximate attention computation, which can accelerate the computation of the attention layer.

\begin{definition}[Approximate Attention Computation $\mathsf{AAttC}(n, d, R, \delta)$, Definition 1.2 in \cite{as23}]\label{def:aattc}
    Given an input sequence $X \in \R^{n \times d}$ and an approximation tolerance $\delta > 0$. Let $Q,K,V \in \R^{n \times d}$ be weigh matrices bounded such that
    \begin{align*}
        \max\{\|Q\|_\infty,\|K\|_\infty,\|V\|_\infty\} \leq R
    \end{align*}
    The {\bf Approximate Attention Computation} $\mathsf{AAttC}(n, d, R, \delta)$ outputs a matrix $N \in \R^{n \times d }$ satisfying:
    \begin{align*}
        \| N - \mathsf{Attn}(X)\|_\infty \leq \delta
    \end{align*}
\end{definition}

Next, we present a lemma that demonstrates the computational time cost of the AATTC method.
\begin{lemma}[Fast Attention via Subquadratic Computation, Theorem 1.4 of \cite{as23}]\label{lem:as23_attention}
Let $\mathsf{AAttC}$ be formalized as in Definition~\ref{def:aattc}. 
For parameter configurations:
\begin{itemize}
    \item Embedding dimension $d = O(\log n)$,
    \item $R = \Theta(\sqrt{\log n})$,
    \item Approximation tolerance $\delta = 1/\poly(n)$,
\end{itemize}
the $\mathsf{AAttC}$ computation satisfies
\begin{align*}
    \mathcal{T}(n, n^{o(1)}, d) = n^{1 + o(1)},
\end{align*}
where $\mathcal{T}$ denotes the time complexity under these constraints.

\end{lemma}


\subsection{Fast FlowAR Architecture in the Inference Pipleline}\label{sec:fast_flowar}
Firstly, we define the fast flow-matching layer, where the $\mathsf{Attn}$ layers in the original flow-matching module are replaced with $\mathsf{AAttC}$ layers.


\begin{definition}[Fast Flow Matching Architecture]\label{def:fast_flow_matching_architecture}
Given the following:
\begin{itemize}
    \item {\bf Input tensor:} $\X \in \R^{h \times w \times c}$.
    \item {\bf Scales number:} $K \in \mathbb{N}$.
    \item {\bf Token maps:} For $i \in [K]$, $\wh{\Y}_i \in \R^{(h / r_i) \times (w/r_i) \times c}$ denote the token maps generated by autoregressive transformer defined in Definition~\ref{def:ar_transformer}.
    \item {\bf Interpolation Tokens:} For $i \in [K]$, $\F_i^t \in \R^{(h / r_i) \times (w/r_i) \times c}$ denote interpolated input defined in Definition~\ref{def:flow}.
    \item {\bf Time step:} For $i \in [K]$, $t_i \in [0,1]$ denotes timestep.
    \item {\bf Approximate Attention layer:}  For $i \in [K]$, $\mathsf{AAttC}_i(\cdot):\R^{h/r_i \times w/r_i \times c} \to \R^{h/r_i \times w/r_i \times c}$ is defined in Definition~\ref{def:attn_layer}.
    \item {\bf MLP layer:}  For $i \in [K]$, $\mathsf{MLP}_i(\cdot,c,d):\R^{h/r_i \times w/r_i \times c} \to \R^{h/r_i \times w/r_i \times c}$ is defined in Definition~\ref{def:mlp}.
    \item {\bf LN layer:} For $i \in [K]$, $\mathsf{LN}_i(\cdot):\R^{h/r_i \times w/r_i \times c} \to \R^{h/r_i \times w/r_i \times c}$ is defined in Definition~\ref{def:ln}.
\end{itemize}
The computation steps of flow-matching layers are as follows:
\begin{itemize}
    \item {\bf Time-conditioned parameter generation:}
    \begin{align*}
        \alpha_1, \alpha_2, \beta_1, \beta_2, \gamma_1, \gamma_2:=  \mathsf{MLP}_i(\wh{\Y}_i + t_i \cdot {\bf 1}_{(h / r_i) \times (w/r_i) \times c},c,6c)
    \end{align*}
    \item {\bf Intermediate variable computation:}
    \begin{align*}
        \F'^{t_i}_i:= \mathsf{AAttC}_i (\gamma_1 \circ \mathsf{LN}(\F_i^{t_i}) + \beta_1) \circ \alpha_1
    \end{align*}
    with $\circ$ denoting Hadamard (element-wise) product.
    \item {\bf Final projection:}
    \begin{align*}
        \F''^{t_i}_i := \mathsf{MLP}_i(\gamma_2 \circ \mathsf{LN}(\F'^{t_i}_i)+ \beta_2,c,c) \circ \alpha_2
    \end{align*}
\end{itemize}
The operation is denoted as $\F''^{t_i}_i := \mathsf{FNN}_i(\wh{\Y_i},\F_i^{t_i},t_i)$
\end{definition}


Next, we define the Fast FlowAR inference pipeline architecture, where all 
$\mathsf{Attn}$ layers in the original FlowAR architecture are replaced with $\mathsf{AAttC}$ layers.




\begin{definition}[Fast FlowAR Inference Architecture]\label{def:fast_flow_architecture_inference}
    Given the following:
    \begin{itemize}
        \item {\bf Scales number:} $K \in \mathbb{N}$.
        \item {\bf Scale factor:} For $i \in [K]$, $r_i:= a^{K-i}$ where base factor $a \in \mathbb{N}^+$.
        \item {\bf Upsampling functions:}  For $i \in [K]$, $\phi_{\mathrm{up},i}(\cdot,a): \R^{(h/r_i) \times (w/r_i) \times c}\to \R^{(h/r_{i+1}) \times (w/r_{i+1}) \times c}$ from Definition~\ref{def:bicubic_up_sample_function}.
        \item {\bf Approximate Attention layer:}  For $i \in [K]$, $\mathsf{AAttC}_i(\cdot):\R^{(\sum_{j=1}^i h/r_j \cdot w/r_{j})\times c} \to \R^{(\sum_{j=1}^i h/r_j \cdot w/r_{j})\times c}$ which acts on flattened sequences of dimension defined in Definition~\ref{def:aattc}.
        \item {\bf Feed forward layer: } For $i \in [K]$, $\mathsf{FFN}_i(\cdot): \R^{(\sum_{j=1}^i h/r_j \cdot w/r_{j})\times c} \to \R^{(\sum_{j=1}^i h/r_j \cdot w/r_{j})\times c}$ which acts on flattened sequences of dimension defined in Definition~\ref{def:ffn}.
        \item {\bf Fast flow-matching layer:} For $i \in [K]$, $\mathsf{FNN}_i(\cdot,\cdot,\cdot):\R^{h/r_i \times w/r_i \times c}\times \R^{h/r_i \times w/r_i \times c}\times \R \to \R^{h/r_i \times w/r_i \times c}$ denote the fast flow-matching layer defined in Definition~\ref{def:fast_flow_matching_architecture}.
        \item {\bf Initial condition:} $\Z_{\mathrm{init}} \in \R^{(h/r_1) \times (w/r_1) \times c}$ denotes the initial token maps which encodes class information.
        \item {\bf Time steps:} For $i \in [K]$, $t_i \in [0,1]$ denotes time steps.
        \item {\bf Interpolated inputs:} For $i \in [K]$, $\F_i^{t_i} \in \R^{h/r_i \times w/r_i \times c}$ defined in Definition~\ref{def:flow}.
        \item {\bf Cumulative dimensions:} We define $\wt{h}_i := \sum_{j=1}^i h/r_j$ and  $\wt{w}_i := \sum_{j=1}^i w/r_j$ for $i \in [K]$.
    \end{itemize}
    The FlowAR model conducts the following recursive construction:
    \begin{itemize}
        \item {\bf Base case $i=1$:}
        \begin{align*}
            &~\Z_1 = \Z_{\mathrm{init}}\\
            &~\wh{\Y}_1 = \mathsf{FFN}_1(\mathsf{AAttC}_1(\Z_1))\\
            &~\wt{\Y}_1 = \mathsf{FNN}_1(\wh{\Y}_1,\F_{1}^{t_1},t_1)
        \end{align*}
        \item {\bf Inductive step $i \geq 2$:}
        \begin{itemize}
            \item {\bf Spatial aggregation:}
            \begin{align*}
                \Z_i = \mathsf{Concat}(\Z_{\mathrm{init}},\phi_{\mathrm{up},1}(\wt{\Y}_{i-1}),\dots,\phi_{\mathrm{up},i-1}(\wt{\Y}_{i-1})) \in \R^{(\sum_{j=1}^i h/r_j \cdot w/r_j)\times c}
            \end{align*}
            \item {\bf Autoregressive transformer computation:}
            \begin{align*}
                \wh{\Y}_i = \mathsf{FFN}_i(\mathsf{AAttC}_i(\Z_1))_{\wt{h}_{i-1}:\wt{h}_{i-1},\wt{w}_{i}:\wt{w}_{i},0:c}
            \end{align*}
            \item {\bf Flow matching layer:}
            \begin{align*}
                \wt{\Y}_i = \mathsf{FNN}_i(\wh{\Y}_i,\F_{i}^{t_i},t_i)
            \end{align*}
        \end{itemize}
        The final output is $\wt{\Y}_K \in \R^{h \times w \times c}$.
    \end{itemize}
   
\end{definition}

\subsection{Running Time}\label{sec:running_time}
In this section, we analyzed the running time required by the original FlowAR architecture and the running time required by the Fast FlowAR architecture. The results indicate that by adopting the Approximate Attention computation module, we can accelerate the running time of FlowAR to almost quadratic time.

First, we present the results of the running time analysis for the original FlowAR model.
\begin{lemma}[Inference Runtime of Original FlowAR Architecture, informal version of Lemma~\ref{lem:runtime_old_flowar_formal}]\label{lem:runtime_old_flowar_informal}
    Consider the original FlowAR inference pipeline with the following parameters:
    \begin{itemize}
        \item {\bf Input tensor:} $\X \in \R^{h \times w \times c}$. Assume $h=w=n$ and $c = O(\log n)$.
        \item {\bf Number of scales:} $K = O(1)$.
        \item {\bf Scale factor:} For $i \in [K]$, $r_i:= a^{K-i}$ where base factor $a \in \mathbb{N}^+$.
        \item {\bf Upsampling functions}  For $i \in [K]$, $\phi_{\mathrm{up},i}(\cdot,a)$ from Definition~\ref{def:bicubic_up_sample_function}.
        \item {\bf Attention layer:}  For $i \in [K]$, $\mathsf{Attn}_i(\cdot)$ which acts on flattened sequences of dimension defined in Definition~\ref{def:attn_layer}.
        \item {\bf Feed forward layer: } For $i \in [K]$, $\mathsf{FFN}_i(\cdot)$ which acts on flattened sequences of dimension defined in Definition~\ref{def:ffn}.
        \item {\bf Flow matching layer:} For $i \in [K]$, $\mathsf{NN}_i(\cdot,\cdot,\cdot)$ denote the flow-matching layer defined in Definition~\ref{def:flow_matching_architecture}.
    \end{itemize}
    Under these conditions, the total inference runtime of FlowAR is bounded by $O(n^{4+o(1)})$.
\end{lemma}

Then, we present the results of the running time analysis for the fast FlowAR model.
\begin{lemma}[Inference Runtime of Fast FlowAR Architecture, informal version of Lemma~\ref{lem:runtime_fast_flowar_formal}]\label{lem:runtime_fast_flowar_informal}
    Consider the original FlowAR inference pipeline with the following parameters:
    \begin{itemize}
        \item {\bf Input tensor:} $\X \in \R^{h \times w \times c}$. Assume $h=w=n$ and $c = O(\log n)$.
        \item {\bf Number of scales:} $K = O(1)$.
        \item {\bf Scale factor:} For $i \in [K]$, $r_i:= a^{K-i}$ where base factor $a \in \mathbb{N}^+$.
        \item {\bf Upsampling functions}  For $i \in [K]$, $\phi_{\mathrm{up},i}(\cdot,a)$ from Definition~\ref{def:bicubic_up_sample_function}.
        \item {\bf Approximate Attention layer:}  For $i \in [K]$, $\mathsf{AAttC}_i(\cdot)$ defined in Definition~\ref{def:aattc}.
        \item {\bf Feed forward layer: } For $i \in [K]$, $\mathsf{FFN}_i(\cdot)$ which acts on flattened sequences of dimension defined in Definition~\ref{def:ffn}.
        \item {\bf Fast flow-matching layer:} For $i \in [K]$, $\mathsf{FNN}_i(\cdot,\cdot,\cdot)$ denote the fast flow-matching layer defined in Definition~\ref{def:fast_flow_matching_architecture}.
    \end{itemize}
    Under these conditions, the total inference runtime of FlowAR is bounded by $O(n^{2+o(1)})$.
\end{lemma}

\subsection{Error Propagation Analysis}\label{lem:error_propagation_analysis}
In this section, we present an error analysis introduced by the fast algorithm applied to the FlowAR model.
\begin{lemma}[Error Bound Between Fast FlowAR and FlowAR Outputs, informal version of Lemma~\ref{lem:error_analysis_fast_flowar}]\label{lem:error_analysis_fast_flowar_informal}
    Given the following:
    \begin{itemize}
        \item {\bf Input tensor:} $\X \in \R^{h \times w \times c}$.
        \item {\bf Scales number:} $K = O(1)$.
        \item {\bf Dimensions:} Let $h=w=n$ and $c = O(\log n)$. Let $\wt{h}_i := \sum_{j=1}^i h/r_j$ and $\wt{w}_i := \sum_{j=1}^i w/r_j$.
        \item {\bf Bounded Entries:} All tensors and matrices have entries bounded by $R = O(\sqrt{\log n})$.
        \item {\bf Layers:}
        \begin{itemize}
            \item $\phi_{\mathrm{up},a}(\cdot)$ :  bicubic upsampling function (Definition~\ref{def:bicubic_up_sample_function}).
            \item $\mathsf{Attn}(\cdot)$: attention layer (Definition~\ref{def:attn_layer}).
            \item $\mathsf{AAttC(\cdot)}$: approximate attention layer (Definition~\ref{def:aattc})
            \item $\mathsf{NN}(\cdot,\cdot,\cdot)$: flow-matching layer (Definition~\ref{def:flow_matching_architecture})
            \item $\mathsf{FNN}(\cdot,\cdot,\cdot)$: fast flow-matching layer (Definition~\ref{def:fast_flow_matching_architecture})
        \end{itemize}
        \item {\bf Input and interpolations:}
        \begin{itemize}
            \item Initial inputs: $\Z_{\mathrm{init}} \in \R^{(h/r_1)\times(w/r_1) \times c}$.
            \item $\Z_i:$ Reshaped tensor of  $\Z_{\mathrm{init}}, \phi_{\mathrm{up},1}(\wt{\Y}_1), \dots, \phi_{\mathrm{up},i-1}(\wt{\Y}_{i-1})$ for FlowAR.
            \item $\Z'_i:$ Reshaped tensor of  $\Z_{\mathrm{init}}, \phi_{\mathrm{up},1}(\wt{\Y}'_1), \dots, \phi_{\mathrm{up},i-1}(\wt{\Y}'_{i-1})$ for  Fast FlowAR.
            \item $\mathsf{F}_i^{t_i} \in \R^{h/r_i \times w/r_i \times c}$ be the interpolated value of FlowAR (Definition~\ref{def:flow}).
            \item $\mathsf{FF}_i^{t_i} \in \R^{h/r_i \times w/r_i \times c}$ be the interpolated value of Fast FlowAR (Definition~\ref{def:flow}).
        \end{itemize} 
        \item {\bf Outputs:}
        \begin{itemize}
            \item $\wt{\Y}_i \in \R^{h/r_i \times w/r_i \times c}$: FlowAR output at layer $i$ (Definition~\ref{def:flow_architecture_inference})
            \item $\wt{\Y}'_i \in \R^{h/r_i \times w/r_i \times c}$: Fast FlowAR output at layer $i$ (Definition~\ref{def:fast_flow_architecture_inference})
        \end{itemize}
    \end{itemize}

    Under these conditions, the $\ell_\infty$ error between the final outputs is bounded by:
    \begin{align*}
        \|\wt{\Y}'_K - \wt{\Y}_K\|_\infty \leq 1/\poly(n)
    \end{align*}
    
\end{lemma}

\subsection{Existence of Almost Quadratic Time Algorithm}\label{sec:almost_quadratic_time_algorithm}
This section presents a theorem proving the existence of a quadratic-time algorithm that speeds up the FlowAR architecture and guarantees a bounded additive error.
\begin{theorem}[Existence of Almost Quadratic Time Algorithm]
\label{thm:upper_bound:formal}
Suppose $d = O(\log n)$ and $R = o(\sqrt{\log n})$. There is an algorithm that approximates the  FlowAR architecture up to $1/\poly(n)$ additive error in $O(n^{2+o(1)})$ time.
\end{theorem}
\begin{proof}
    By combining the result of Lemma~\ref{lem:runtime_fast_flowar_informal} and Lemma~\ref{lem:error_analysis_fast_flowar_informal}, we can easily derive the proof.
\end{proof}

Our Theorem~\ref{thm:upper_bound:formal} shows that we can accelerate FlowAR while only introducing a small error. Using the low-rank approximation in the attention mechanism is also used in previous works \cite{kll+25,lls+24_conv,llss25,lss+25_relu,chl+24_rope,lss+24,lssz24_tat,as24_iclr,as24b,as24_rope,hsk+24}.
\section{Discussion}\label{sec:discussion}



\subsection{From Interactive Prompting to Interactive Multi-modal Prompting}
The rapid advancements of large pre-trained generative models including large language models and text-to-image generation models, have inspired many HCI researchers to develop interactive tools to support users in crafting appropriate prompts.
% Studies on this topic in last two years' HCI conferences are predominantly focused on helping users refine single-modality textual prompts.
Many previous studies are focused on helping users refine single-modality textual prompts.
However, for many real-world applications concerning data beyond text modality, such as multi-modal AI and embodied intelligence, information from other modalities is essential in constructing sophisticated multi-modal prompts that fully convey users' instruction.
This demand inspires some researchers to develop multimodal prompting interactions to facilitate generation tasks ranging from visual modality image generation~\cite{wang2024promptcharm, promptpaint} to textual modality story generation~\cite{chung2022tale}.
% Some previous studies contributed relevant findings on this topic. 
Specifically, for the image generation task, recent studies have contributed some relevant findings on multi-modal prompting.
For example, PromptCharm~\cite{wang2024promptcharm} discovers the importance of multimodal feedback in refining initial text-based prompting in diffusion models.
However, the multi-modal interactions in PromptCharm are mainly focused on the feedback empowered the inpainting function, instead of supporting initial multimodal sketch-prompt control. 

\begin{figure*}[t]
    \centering
    \includegraphics[width=0.9\textwidth]{src/img/novice_expert.pdf}
    \vspace{-2mm}
    \caption{The comparison between novice and expert participants in painting reveals that experts produce more accurate and fine-grained sketches, resulting in closer alignment with reference images in close-ended tasks. Conversely, in open-ended tasks, expert fine-grained strokes fail to generate precise results due to \tool's lack of control at the thin stroke level.}
    \Description{The comparison between novice and expert participants in painting reveals that experts produce more accurate and fine-grained sketches, resulting in closer alignment with reference images in close-ended tasks. Novice users create rougher sketches with less accuracy in shape. Conversely, in open-ended tasks, expert fine-grained strokes fail to generate precise results due to \tool's lack of control at the thin stroke level, while novice users' broader strokes yield results more aligned with their sketches.}
    \label{fig:novice_expert}
    % \vspace{-3mm}
\end{figure*}


% In particular, in the initial control input, users are unable to explicitly specify multi-modal generation intents.
In another example, PromptPaint~\cite{promptpaint} stresses the importance of paint-medium-like interactions and introduces Prompt stencil functions that allow users to perform fine-grained controls with localized image generation. 
However, insufficient spatial control (\eg, PromptPaint only allows for single-object prompt stencil at a time) and unstable models can still leave some users feeling the uncertainty of AI and a varying degree of ownership of the generated artwork~\cite{promptpaint}.
% As a result, the gap between intuitive multi-modal or paint-medium-like control and the current prompting interface still exists, which requires further research on multi-modal prompting interactions.
From this perspective, our work seeks to further enhance multi-object spatial-semantic prompting control by users' natural sketching.
However, there are still some challenges to be resolved, such as consistent multi-object generation in multiple rounds to increase stability and improved understanding of user sketches.   


% \new{
% From this perspective, our work is a step forward in this direction by allowing multi-object spatial-semantic prompting control by users' natural sketching, which considers the interplay between multiple sketch regions.
% % To further advance the multi-modal prompting experience, there are some aspects we identify to be important.
% % One of the important aspects is enhancing the consistency and stability of multiple rounds of generation to reduce the uncertainty and loss of control on users' part.
% % For this purpose, we need to develop techniques to incorporate consistent generation~\cite{tewel2024training} into multi-modal prompting framework.}
% % Another important aspect is improving generative models' understanding of the implicit user intents \new{implied by the paint-medium-like or sketch-based input (\eg, sketch of two people with their hands slightly overlapping indicates holding hand without needing explicit prompt).
% % This can facilitate more natural control and alleviate users' effort in tuning the textual prompt.
% % In addition, it can increase users' sense of ownership as the generated results can be more aligned with their sketching intents.
% }
% For example, when users draw sketches of two people with their hands slightly overlapping, current region-based models cannot automatically infer users' implicit intention that the two people are holding hands.
% Instead, they still require users to explicitly specify in the prompt such relationship.
% \tool addresses this through sketch-aware prompt recommendation to fill in the necessary semantic information, alleviating users' workload.
% However, some users want the generative AI in the future to be able to directly infer this natural implicit intentions from the sketches without additional prompting since prompt recommendation can still be unstable sometimes.


% \new{
% Besides visual generation, 
% }
% For example, one of the important aspect is referring~\cite{he2024multi}, linking specific text semantics with specific spatial object, which is partly what we do in our sketch-aware prompt recommendation.
% Analogously, in natural communication between humans, text or audio alone often cannot suffice in expressing the speakers' intentions, and speakers often need to refer to an existing spatial object or draw out an illustration of her ideas for better explanation.
% Philosophically, we HCI researchers are mostly concerned about the human-end experience in human-AI communications.
% However, studies on prompting is unique in that we should not just care about the human-end interaction, but also make sure that AI can really get what the human means and produce intention-aligned output.
% Such consideration can drastically impact the design of prompting interactions in human-AI collaboration applications.
% On this note, although studies on multi-modal interactions is a well-established topic in HCI community, it remains a challenging problem what kind of multi-modal information is really effective in helping humans convey their ideas to current and next generation large AI models.




\subsection{Novice Performance vs. Expert Performance}\label{sec:nVe}
In this section we discuss the performance difference between novice and expert regarding experience in painting and prompting.
First, regarding painting skills, some participants with experience (4/12) preferred to draw accurate and fine-grained shapes at the beginning. 
All novice users (5/12) draw rough and less accurate shapes, while some participants with basic painting skills (3/12) also favored sketching rough areas of objects, as exemplified in Figure~\ref{fig:novice_expert}.
The experienced participants using fine-grained strokes (4/12, none of whom were experienced in prompting) achieved higher IoU scores (0.557) in the close-ended task (0.535) when using \tool. 
This is because their sketches were closer in shape and location to the reference, making the single object decomposition result more accurate.
Also, experienced participants are better at arranging spatial location and size of objects than novice participants.
However, some experienced participants (3/12) have mentioned that the fine-grained stroke sometimes makes them frustrated.
As P1's comment for his result in open-ended task: "\emph{It seems it cannot understand thin strokes; even if the shape is accurate, it can only generate content roughly around the area, especially when there is overlapping.}" 
This suggests that while \tool\ provides rough control to produce reasonably fine results from less accurate sketches for novice users, it may disappoint experienced users seeking more precise control through finer strokes. 
As shown in the last column in Figure~\ref{fig:novice_expert}, the dragon hovering in the sky was wrongly turned into a standing large dragon by \tool.

Second, regarding prompting skills, 3 out of 12 participants had one or more years of experience in T2I prompting. These participants used more modifiers than others during both T2I and R2I tasks.
Their performance in the T2I (0.335) and R2I (0.469) tasks showed higher scores than the average T2I (0.314) and R2I (0.418), but there was no performance improvement with \tool\ between their results (0.508) and the overall average score (0.528). 
This indicates that \tool\ can assist novice users in prompting, enabling them to produce satisfactory images similar to those created by users with prompting expertise.



\subsection{Applicability of \tool}
The feedback from user study highlighted several potential applications for our system. 
Three participants (P2, P6, P8) mentioned its possible use in commercial advertising design, emphasizing the importance of controllability for such work. 
They noted that the system's flexibility allows designers to quickly experiment with different settings.
Some participants (N = 3) also mentioned its potential for digital asset creation, particularly for game asset design. 
P7, a game mod developer, found the system highly useful for mod development. 
He explained: "\emph{Mods often require a series of images with a consistent theme and specific spatial requirements. 
For example, in a sacrifice scene, how the objects are arranged is closely tied to the mod's background. It would be difficult for a developer without professional skills, but with this system, it is possible to quickly construct such images}."
A few participants expressed similar thoughts regarding its use in scene construction, such as in film production. 
An interesting suggestion came from participant P4, who proposed its application in crime scene description. 
She pointed out that witnesses are often not skilled artists, and typically describe crime scenes verbally while someone else illustrates their account. 
With this system, witnesses could more easily express what they saw themselves, potentially producing depictions closer to the real events. "\emph{Details like object locations and distances from buildings can be easily conveyed using the system}," she added.

% \subsection{Model Understanding of Users' Implicit Intents}
% In region-sketch-based control of generative models, a significant gap between interaction design and actual implementation is the model's failure in understanding users' naturally expressed intentions.
% For example, when users draw sketches of two people with their hands slightly overlapping, current region-based models cannot automatically infer users' implicit intention that the two people are holding hands.
% Instead, they still require users to explicitly specify in the prompt such relationship.
% \tool addresses this through sketch-aware prompt recommendation to fill in the necessary semantic information, alleviating users' workload.
% However, some users want the generative AI in the future to be able to directly infer this natural implicit intentions from the sketches without additional prompting since prompt recommendation can still be unstable sometimes.
% This problem reflects a more general dilemma, which ubiquitously exists in all forms of conditioned control for generative models such as canny or scribble control.
% This is because all the control models are trained on pairs of explicit control signal and target image, which is lacking further interpretation or customization of the user intentions behind the seemingly straightforward input.
% For another example, the generative models cannot understand what abstraction level the user has in mind for her personal scribbles.
% Such problems leave more challenges to be addressed by future human-AI co-creation research.
% One possible direction is fine-tuning the conditioned models on individual user's conditioned control data to provide more customized interpretation. 

% \subsection{Balance between recommendation and autonomy}
% AIGC tools are a typical example of 
\subsection{Progressive Sketching}
Currently \tool is mainly aimed at novice users who are only capable of creating very rough sketches by themselves.
However, more accomplished painters or even professional artists typically have a coarse-to-fine creative process. 
Such a process is most evident in painting styles like traditional oil painting or digital impasto painting, where artists first quickly lay down large color patches to outline the most primitive proportion and structure of visual elements.
After that, the artists will progressively add layers of finer color strokes to the canvas to gradually refine the painting to an exquisite piece of artwork.
One participant in our user study (P1) , as a professional painter, has mentioned a similar point "\emph{
I think it is useful for laying out the big picture, give some inspirations for the initial drawing stage}."
Therefore, rough sketch also plays a part in the professional artists' creation process, yet it is more challenging to integrate AI into this more complex coarse-to-fine procedure.
Particularly, artists would like to preserve some of their finer strokes in later progression, not just the shape of the initial sketch.
In addition, instead of requiring the tool to generate a finished piece of artwork, some artists may prefer a model that can generate another more accurate sketch based on the initial one, and leave the final coloring and refining to the artists themselves.
To accommodate these diverse progressive sketching requirements, a more advanced sketch-based AI-assisted creation tool should be developed that can seamlessly enable artist intervention at any stage of the sketch and maximally preserve their creative intents to the finest level. 

\subsection{Ethical Issues}
Intellectual property and unethical misuse are two potential ethical concerns of AI-assisted creative tools, particularly those targeting novice users.
In terms of intellectual property, \tool hands over to novice users more control, giving them a higher sense of ownership of the creation.
However, the question still remains: how much contribution from the user's part constitutes full authorship of the artwork?
As \tool still relies on backbone generative models which may be trained on uncopyrighted data largely responsible for turning the sketch into finished artwork, we should design some mechanisms to circumvent this risk.
For example, we can allow artists to upload backbone models trained on their own artworks to integrate with our sketch control.
Regarding unethical misuse, \tool makes fine-grained spatial control more accessible to novice users, who may maliciously generate inappropriate content such as more realistic deepfake with specific postures they want or other explicit content.
To address this issue, we plan to incorporate a more sophisticated filtering mechanism that can detect and screen unethical content with more complex spatial-semantic conditions. 
% In the future, we plan to enable artists to upload their own style model

% \subsection{From interactive prompting to interactive spatial prompting}


\subsection{Limitations and Future work}

    \textbf{User Study Design}. Our open-ended task assesses the usability of \tool's system features in general use cases. To further examine aspects such as creativity and controllability across different methods, the open-ended task could be improved by incorporating baselines to provide more insightful comparative analysis. 
    Besides, in close-ended tasks, while the fixing order of tool usage prevents prior knowledge leakage, it might introduce learning effects. In our study, we include practice sessions for the three systems before the formal task to mitigate these effects. In the future, utilizing parallel tests (\textit{e.g.} different content with the same difficulty) or adding a control group could further reduce the learning effects.

    \textbf{Failure Cases}. There are certain failure cases with \tool that can limit its usability. 
    Firstly, when there are three or more objects with similar semantics, objects may still be missing despite prompt recommendations. 
    Secondly, if an object's stroke is thin, \tool may incorrectly interpret it as a full area, as demonstrated in the expert results of the open-ended task in Figure~\ref{fig:novice_expert}. 
    Finally, sometimes inclusion relationships (\textit{e.g.} inside) between objects cannot be generated correctly, partially due to biases in the base model that lack training samples with such relationship. 

    \textbf{More support for single object adjustment}.
    Participants (N=4) suggested that additional control features should be introduced, beyond just adjusting size and location. They noted that when objects overlap, they cannot freely control which object appears on top or which should be covered, and overlapping areas are currently not allowed.
    They proposed adding features such as layer control and depth control within the single-object mask manipulation. Currently, the system assigns layers based on color order, but future versions should allow users to adjust the layer of each object freely, while considering weighted prompts for overlapping areas.

    \textbf{More customized generation ability}.
    Our current system is built around a single model $ColorfulXL-Lightning$, which limits its ability to fully support the diverse creative needs of users. Feedback from participants has indicated a strong desire for more flexibility in style and personalization, such as integrating fine-tuned models that cater to specific artistic styles or individual preferences. 
    This limitation restricts the ability to adapt to varied creative intents across different users and contexts.
    In future iterations, we plan to address this by embedding a model selection feature, allowing users to choose from a variety of pre-trained or custom fine-tuned models that better align with their stylistic preferences. 
    
    \textbf{Integrate other model functions}.
    Our current system is compatible with many existing tools, such as Promptist~\cite{hao2024optimizing} and Magic Prompt, allowing users to iteratively generate prompts for single objects. However, the integration of these functions is somewhat limited in scope, and users may benefit from a broader range of interactive options, especially for more complex generation tasks. Additionally, for multimodal large models, users can currently explore using affordable or open-source models like Qwen2-VL~\cite{qwen} and InternVL2-Llama3~\cite{llama}, which have demonstrated solid inference performance in our tests. While GPT-4o remains a leading choice, alternative models also offer competitive results.
    Moving forward, we aim to integrate more multimodal large models into the system, giving users the flexibility to choose the models that best fit their needs. 
    


\section{Conclusion}\label{sec:conclusion}
In this paper, we present \tool, an interactive system designed to help novice users create high-quality, fine-grained images that align with their intentions based on rough sketches. 
The system first refines the user's initial prompt into a complete and coherent one that matches the rough sketch, ensuring the generated results are both stable, coherent and high quality.
To further support users in achieving fine-grained alignment between the generated image and their creative intent without requiring professional skills, we introduce a decompose-and-recompose strategy. 
This allows users to select desired, refined object shapes for individual decomposed objects and then recombine them, providing flexible mask manipulation for precise spatial control.
The framework operates through a coarse-to-fine process, enabling iterative and fine-grained control that is not possible with traditional end-to-end generation methods. 
Our user study demonstrates that \tool offers novice users enhanced flexibility in control and fine-grained alignment between their intentions and the generated images.



\newpage
\onecolumn

\ifdefined\isarxiv
\else

\title{Circuit Complexity Bounds for FlowAR Model\\(Supplementary Material)}
\maketitle

\fi

\appendix
\begin{center}
	\textbf{\LARGE Appendix }
\end{center}
{\bf Roadmap.} Section~\ref{sec:app_notations} presents all the notations of this paper. In Section~\ref{sec:app_missing_proof}, we present some missing proofs in Section~\ref{sec:main_result}. Section~\ref{sec:app_efficient_critieria} presents provably efficient criteria of the fast FlowAR model.


\section{Notations}\label{sec:app_notations}
Given a matrix $X \in \R^{hw \times d}$, we denote its tensorized form as $\X \in \R^{h \times w \times d}$. Additionally, we define the set $[n]$ to represent $\{1,2,\cdots, n\}$ for any positive integer $n$. We define the set of natural numbers as $\mathbb{N}:= \{0,1,2,\dots\}$. Let $X \in \mathbb{R}^{m \times n}$ be a matrix, where $X_{i,j}$ refers to the element at the $i$-th row and $j$-th column. When $x_i$ belongs to $\{ 0,1 \}^*$, it signifies a binary number with arbitrary length. In a general setting, $x_i$ represents a length $p$ binary string, with each bit taking a value of either 1 or 0. Given a matrix $X \in \R^{n \times d}$, we define $\|X\|_\infty  $ as the maximum norm of $X$. Specifically, $\|X\|_\infty = \max_{i,j} |X_{i,j}|$. 
\section{Supplementary Proof for Section~\ref{sec:main_result}}\label{sec:app_missing_proof}
In this section, we present some missing proofs in Section~\ref{sec:main_result}. 

\subsection{Computing Multiple-layer Perceptron in \texorpdfstring{$\TC^0$}{} }\label{sec:app_mlp_tc0}
This section presents the detailed proof for Lemma~\ref{lem:mlp_tc0_informal}.

\begin{lemma}[MLP computation in $\mathsf{TC}^0$, formal version of Lemma~\ref{lem:mlp_tc0_informal}]\label{lem:mlp_tc0_formal}
    Given an input tensor $\X\in \R^{h \times w \times c}$. Let $\mathsf{MLP}(\X,c,d)$ be the MLP layer defined in Definition~\ref{def:mlp}. Under the following constraints:
    \begin{itemize}
        \item Satisfy $h = w = n$,
        \item Channel bounds: $c, d\leq n$,
        \item Precision: $p \leq \poly(n)$,
    \end{itemize}
    The $\mathsf{MLP}(\X,c,d)$ function can be computed by a uniform $\mathsf{TC}^0$ circuit with:
    \begin{itemize}
        \item Size: $\poly(n)$.
        \item Depth: $2d_\mathrm{std} + d_{\oplus}$.
    \end{itemize}
    with $d_{\mathrm{std}}$ and $d_{\oplus}$ defined in Definition~\ref{lem:float_operations_TC}.
\end{lemma}


\begin{proof}
    For each $j \in [hw]$, by Lemma~\ref{lem:matrix_multi}, 
    compute $X_{j,*} \cdot W$ requires depth $d_{\mathrm{std}} + d_{\oplus}$. By Part 1 of Lemma~\ref{lem:float_operations_TC}, compute $X_{j,*} \cdot W + b$ requires depth $d_{\mathrm{std}}$.  Since for all $j \in [hw]$, the computation $X_{j,*} \cdot W + b$ can be simulated in parallel. Hence the total depth remains $2d_\mathrm{std} + d_{\oplus}$ and size is $\poly(n)$.
\end{proof}
\subsection{Computing Feed Forward Layer in \texorpdfstring{$\TC^0$}{} }\label{sec:app_ffn_tc0}
This section presents the detailed proof for Lemma~\ref{lem:ffn_tc0_informal}.

\begin{lemma}[FFN computation in $\mathsf{TC}^0$, formal version of Lemma~\ref{lem:ffn_tc0_informal}]\label{lem:ffn_tc0_formal}
 Given an input tensor $\X\in \R^{h \times w \times c}$. Let $\mathsf{FFN}(X):\R^{h \times w \times c} \to \R^{h \times w \times c}$ as defined in Definition~\ref{def:ffn}. Under the following constraints:
 \begin{itemize}
     \item Satisfy $h = w = n$,
         \item Channel bound: $c \leq n$,
         \item Precision bound: $p \leq \poly(n)$.
 \end{itemize}
The $\mathsf{FFN}(\X)$ layer can be computed by a uniform $\mathsf{TC}^0$ circuit with:
\begin{itemize}
    \item Size: $\poly(n)$.
    \item Depth: $6d_\mathrm{std} + 2d_{\oplus}$.
\end{itemize}
with $d_{\mathrm{std}}$ and $d_{\oplus}$ defined in Definition~\ref{lem:float_operations_TC}.
\end{lemma}
\begin{proof}
    For each $j \in [hw]$, by the proof of Lemma~\ref{lem:mlp_tc0_formal}, compute $X_{j,*} \cdot W_1 + b_1$ requires depth $2d_{\mathrm{std}} + d_{\oplus}$. By Lemma~\ref{lem:float_operations_TC}, compute $A_1 = \sigma(X_{j,*} \cdot W + b)$ requires depth $d_{\mathrm{std}}$. By applying Lemma~\ref{lem:mlp_tc0_formal} again, compute $A_2 = A_1\cdot W_2 +b_2$ requires depth $2d_{\mathrm{std}} + d_{\oplus}$. Lastly, by Part 1 of Lemma~\ref{lem:float_operations_TC}, compute $X_{j,*} + A_2$ requires depth $d_{\mathrm{std}}$.
    
    Combing the result above, we can have that compute $Y_{j,*}=X_{j,*} + \sigma(X_{j,*} \cdot W_1 + b_1)\cdot W_2 +b_2$ requires depth $6d_{\mathrm{std}}+2d_{\oplus}$.

    Since for all $j \in [hw]$, the computation $Y_{j,*}$ can be simulated in parallel. Hence the total depth remains $6d_{\mathrm{std}}+2d_{\oplus}$ and size is $\poly(n)$.
\end{proof}

\subsection{Computing Attention Layer in \texorpdfstring{$\TC^0$}{} }\label{sec:app_attn_tc0}
This section presents the detailed proof for Lemma~\ref{lem:attn_tc0_informal}.

\begin{lemma}[Attention layer computation in $\mathsf{TC}^0$, formal version of Lemma~\ref{lem:attn_tc0_informal}]\label{lem:attn_tc0_formal}
     Given an input tensor $\X \in \R^{h \times w \times c}$. Let $\mathsf{Attn}(X):\R^{h \times w \times c} \to \R^{h \times w \times c}$ as defined in Definition~\ref{def:attn_layer}. Under the following constraints:
     \begin{itemize}
         \item Satisfy $h = w = n$,
         \item Channel bound: $c \leq n$,
         \item Precision bound: $p \leq \poly(n)$.
     \end{itemize}
     The $\mathsf{Attn}(\X)$ layer can be computed by a uniform $\mathsf{TC}^0$ circuit with:
     \begin{itemize}
        \item Size: $\poly(n)$.
        \item Depth: $6(d_{\mathrm{std}} + d_{\oplus}) + d_{\exp}$.
     \end{itemize}    
     with $d_{\mathrm{std}}$ and $d_{\oplus}$ defined in Definition~\ref{lem:float_operations_TC}, $d_{\exp}$ defined in Definition~\ref{lem:exp}. 
\end{lemma}
\begin{proof}
    We analyze the $\mathsf{TC}^0$ simulation of the attention layer through sequential computation phases:
    \begin{itemize}
        \item {\bf Key-Query Product}: Compute $W_QW_K^\top$ vial Lemma~\ref{lem:matrix_multi} requires depth $d_{\mathrm{std}} + d_{\oplus}$.
        \item {\bf Pairwise Score Computation}: Compute $s_{i,j} = X_{i,*}   W_Q   W_K^\top   X_{j,*}^\top$ requires depth $2(d_{\mathrm{std}} + d_{\oplus})$ by Lemma~\ref{lem:matrix_multi}. By  Lemma~\ref{lem:exp}, computing $A_{i,j} = \exp(s_{i,j})$ requires depth $d_{\exp}$. 
    \end{itemize}
    Since all entries $A_{i,j}$ for $i, j \in [n]$ can be computed in parallel, the attention matrix $A$ computation requires depth $3(d_{\mathrm{std}} + d_{\oplus}) + d_{\exp}$.

    Then keep on analyzing:
    \begin{itemize}
        \item {\bf Row Nomalization:} Computing $D:=\diag(A{\bf 1}_n)$ requires depth $d_{\oplus}$ by Lemma~\ref{lem:float_operations_TC}. Computing $D^{-1}$ requires depth $d_{\mathrm{std}}$ by Lemma~\ref{lem:float_operations_TC} .
        \item {\bf Value Projection} Computing $AXW_V$ requires depth $2(d_{\mathrm{std}} + d_{\oplus})$ by applying Lemma~\ref{lem:matrix_multi}. Compute $D^{-1} \cdot A X W_V$ requires $d_{\mathrm{std}}$.
    \end{itemize}

    Combing the result, we need a
    \begin{align*}
        d_{\mathrm{all}} = 6(d_{\mathrm{std}} + d_{\oplus}) + d_{\exp}
    \end{align*}
    depth and size $\poly(n)$ uniform $\mathsf{TC}^0$ circuit to compute the attention layer.
\end{proof}

\subsection{Computing Layer-wise Norm Layer in \texorpdfstring{$\TC^0$}{} }\label{sec:app_ln_tc0}
This section presents the detailed proof for Lemma~\ref{lem:ln_tc0_informal}.


\begin{lemma}[Layer-wise norm layer computation in $\TC^0$, formal version of Lemma~\ref{lem:ln_tc0_informal}]\label{lem:ln_tc0_formal}
    Given an input tensor $\X \in \R^{h \times w \times c}$. Let $\mathsf{LN}(X):\R^{h \times w \times c} \to \R^{h \times w \times c}$ as defined in Definition~\ref{def:ln}. Under the following constraints:
    \begin{itemize}
         \item Satisfy $h = w = n$,
         \item Channel bound: $c \leq n$,
         \item Precision bound: $p \leq \poly(n)$.
     \end{itemize}
     The $\mathsf{LN}(\X)$ layer can be computed by a uniform $\mathsf{TC}^0$ circuit with:
     \begin{itemize}
        \item Size: $\poly(n)$.
        \item Depth: $5d_\mathrm{std} + 2d_{\oplus} + d_\mathrm{sqrt}$.
     \end{itemize}    
     with $d_{\mathrm{std}}$ and $d_{\oplus}$ defined in Definition~\ref{lem:float_operations_TC}, $d_{\mathrm{sqrt}}$ defined in Definition~\ref{lem:sqrt}.
\end{lemma}

\begin{proof}
    By Part 1 and Part 3  of Lemma~\ref{lem:float_operations_TC}, 
    computing mean vector $\mu_j$ requires depth $d_{\mathrm{std}}+d_{\oplus}$. By Part 1 and Part 3  of Lemma~\ref{lem:float_operations_TC}, 
    computing mean vector $\sigma^2_i$ requires depth $2d_{\mathrm{std}}+d_{\oplus}$.      By Lemma~\ref{lem:float_operations_TC} and Lemma~\ref{lem:sqrt}, computing $Y_{j,*}$ requires depth  $2d_{\mathsf{std}}+d_{\oplus}$. So the process requires total depth $5d_\mathrm{std} + 2d_{\oplus} + d_\mathrm{sqrt}$ and $\poly(n)$ size. 
\end{proof}
\section{Provably Efficient Criteria}\label{sec:app_efficient_critieria}

\subsection{Running Time Analysis for Inference Pipeline of Origin FlowAR Architecture }\label{sec:runtime_origin_flowar}

We proceed to compute the total running time for the inference pipeline of the origin FlowAR architecture.
\begin{lemma}[Inference Runtime of Original FlowAR Architecture, formal version of Lemma~\ref{lem:runtime_old_flowar_informal}]\label{lem:runtime_old_flowar_formal}
    Consider the original FlowAR inference pipeline with the following parameters:
    \begin{itemize}
        \item {\bf Input tensor:} $\X \in \R^{h \times w \times c}$. Assume $h=w=n$ and $c = O(\log n)$.
        \item {\bf Number of scales:} $K = O(1)$.
        \item {\bf Scale factor:} For $i \in [K]$, $r_i:= a^{K-i}$ where base factor $a \in \mathbb{N}^+$.
        \item {\bf Upsampling functions}  For $i \in [K]$, $\phi_{\mathrm{up},i}(\cdot,a)$ from Definition~\ref{def:bicubic_up_sample_function}.
        \item {\bf Attention layer:}  For $i \in [K]$, $\mathsf{Attn}_i(\cdot)$ which acts on flattened sequences of dimension defined in Definition~\ref{def:attn_layer}.
        \item {\bf Feed forward layer: } For $i \in [K]$, $\mathsf{FFN}_i(\cdot)$ which acts on flattened sequences of dimension defined in Definition~\ref{def:ffn}.
        \item {\bf Flow matching layer:} For $i \in [K]$, $\mathsf{NN}_i(\cdot,\cdot,\cdot)$ denote the flow-matching layer defined in Definition~\ref{def:flow_matching_architecture}.
    \end{itemize}
    Under these conditions, the total inference runtime of FlowAR is bounded by $O(n^{4+o(1)})$.
    
\end{lemma}
\begin{proof}
    {\bf Part 1: Running time of bicubic up-sample Layer.} The $i$-th layer pf FlowAR model contains $\phi_{\mathrm{up},1}(\cdot,2),\dots,\phi_{\mathrm{up},i-1}(\cdot,2)$. Considering $\phi_{\mathrm{up},i-1}(\cdot,2)$, this operation needs $O(n^{2}c/2^{2(K-i)})$ time. Then the total time required for upsampling in the i-th layer of the FlowAR architecture is $O( n^2 c \cdot \frac{1}{2^{2K}} \cdot (1 - \frac{1}{4^i}))$ which is due to simple algebra. Hence, the total runtime for all bicubic up sample functions is
    \begin{align*}
        \mathcal{T}_{\mathrm{up}} = &~ \sum_{i=1}^K O( n^2 c \cdot \frac{1}{2^{2K}} \cdot (1 - \frac{1}{4^i}))\\
        =&~ O(n^{2+o(1)})
    \end{align*}
    where the first equation is derived from summing up all the running time of the up sample functions, the second step is due to simple algebra and $K = O(1)$ and $c = O(\log n)$.

    {\bf Part 2: Running time of Attention Layer.} The input size of the $i$-th attention layer $\mathsf{Attn}_i$ is $\sum_{j=1}^i (n/2^{K-j}) \times \sum_{j=1}^i(n/2^{K-j}) \times c $. So the time needed to compute the $i$-th attention layer is $O(n^4c \cdot (2^i-1)^4/2^{4K-4})$. Hence, the total runtime for all attention layers is
    \begin{align*}
        \mathcal{T}_{\mathrm{Attn}} =&~ \sum_{i=1}^K O( n^4c \cdot (2^i-1)^4/2^{4K-4})\\
        =&~ O(n^{4+o(1)})
    \end{align*}
    The first equation is derived from summing up all the running time of the attention layer, the second step is due to simple algebra and $K = O(1)$ and $c = O(\log n)$.

    {\bf Part 3: Running time of FFN Layer.} The input size of the $i$-th FFN layer $\mathsf{FFN}_i$ is $\sum_{j=1}^i (n/2^{K-j}) \times \sum_{j=1}^i(n/2^{K-j}) \times c $. So by Definition~\ref{def:ffn}, we can easily derive that the time needed to compute the $i-$th FFN layer is $O(n^2c^2 (2^i-1)^2/2^{2K-2} )$. Hence, the total runtime for all FFN layers is
    \begin{align*}
        \mathcal{T}_{\mathrm{FFN}} = &~ \sum_{i=1}^K O(n^2c^2 (2^i-1)^2/2^{2K-2}) \\
        =&~ O(n^{2+o(1)})
    \end{align*}
    The first step is derived from summing up all the running time of the FFN layer,  and the second step is due to simple algebra and $K = O(1)$ and $c = O(\log n)$.

    {\bf Part 4: Running time of Flow Matching Layer.} The input size of the $i$-th flow-matching layer $\mathsf{NN}_i$ is $ (n/2^{K-i}) \times (n/2^{K-i}) \times c $. It's trivially that the running time of the flow-matching layer will be dominated by the running time of the attention layer, which is $O(n^{4}c/ 2^{4(K-i)})$ (see Part 2 of Definition~\ref{def:flow_matching_architecture}). Hence, the total runtime for all flow-matching layers is 
    \begin{align*}
        \mathcal{T}_{\mathsf{FM}} =&~ \sum_{i=1}^K O(n^4 c /2^{4(K-i)})\\
        =&~ O(n^{4+o(1)})
    \end{align*}
    The first step is derived from summing up all the running time of the origin flow-matching layer, and the second step is due to simple algebra and $K = O(1)$ and $c = O(\log n)$.

    Then, by summing up Part 1 to Part 4, we can get the total running time for FlowAR architecture, which is
    \begin{align*}
        \mathcal{T}_{\mathrm{ori}} =&~ \mathcal{T}_{\mathrm{up}} + \mathcal{T}_{\mathrm{Attn}} + \mathcal{T}_{\mathrm{FFN}} + \mathcal{T}_{\mathsf{FM}}\\
        =&~ O(n^{4+o(1)})
    \end{align*}
\end{proof}
Lemma~\ref{sec:runtime_origin_flowar} demonstrates the runtime required for the original FlowAR architecture, from which we can deduce that the dominant term in the runtime comes from the computation of the Attention Layer.
\subsection{Running Time Analysis for Inference Pipeline of Fast FlowAR Architecture }\label{sec:runtime_fast_flowar}
In this section, we apply the conclusions of \cite{as23} to the FlowAR architecture, where all Attention modules in FlowAR are computed using the Approximate Attention Computation defined in Definition~\ref{def:aattc}.
\begin{lemma}[Inference Runtime of Fast FlowAR Architecture, formal version of Lemma~\ref{lem:runtime_fast_flowar_informal}]\label{lem:runtime_fast_flowar_formal}
    Consider the original FlowAR inference pipeline with the following parameters:
    \begin{itemize}
        \item {\bf Input tensor:} $\X \in \R^{h \times w \times c}$. Assume $h=w=n$ and $c = O(\log n)$.
        \item {\bf Number of scales:} $K = O(1)$.
        \item {\bf Scale factor:} For $i \in [K]$, $r_i:= a^{K-i}$ where base factor $a \in \mathbb{N}^+$.
        \item {\bf Upsampling functions}  For $i \in [K]$, $\phi_{\mathrm{up},i}(\cdot,a)$ from Definition~\ref{def:bicubic_up_sample_function}.
        \item {\bf Approximate Attention layer:}  For $i \in [K]$, $\mathsf{AAttC}_i(\cdot)$ defined in Definition~\ref{def:aattc}.
        \item {\bf Feed forward layer: } For $i \in [K]$, $\mathsf{FFN}_i(\cdot)$ which acts on flattened sequences of dimension defined in Definition~\ref{def:ffn}.
        \item {\bf Fast flow-matching layer:} For $i \in [K]$, $\mathsf{FNN}_i(\cdot,\cdot,\cdot)$ denote the fast flow-matching layer defined in Definition~\ref{def:fast_flow_matching_architecture}.
    \end{itemize}
    Under these conditions, the total inference runtime of FlowAR is bounded by $O(n^{2+o(1)})$.
      
\end{lemma}
\begin{proof}
    {\bf Part 1: Running time of bicubic up-sample Layer.} The runtime of the upsample function in the fast FlowAR architecture is the same as that in the original FlowAR architecture, which is
    \begin{align*}
        \mathcal{T}_{\mathrm{up}} =  O(n^{2+o(1)})
    \end{align*}
    

    {\bf Part 2: Running time of Attention Layer.} The input size of the $i$-th approximate attention computation layer $\mathsf{AAttC}_i$ is $\sum_{j=1}^i (n/2^{K-j}) \times \sum_{j=1}^i(n/2^{K-j}) \times c $. So the time needed to compute the $i$-th attention layer is $O(n^{2+o(1)} \cdot (2^i-1)^4/2^{4K-4})$. Hence, the total runtime for all attention layers is
    \begin{align*}
        \mathcal{T}_{\mathrm{Attn}} =&~ \sum_{i=1}^K O( n^{2+o(1)} \cdot (2^i-1)^4/2^{4K-4})\\
        =&~ O(n^{2+o(1)})
    \end{align*}
    The first equation is derived from summing up all the running time of the approximate attention computation layer, and the second equation is due to basic algebra and $K = O(1)$.

    {\bf Part 3: Running time of FFN Layer.} The runtime of the FFN layer in the fast FlowAR architecture is the same as that in the original FlowAR architecture, which is
    \begin{align*}
        \mathcal{T}_{\mathrm{FFN}} =  O(n^{2+o(1)})
    \end{align*}

    {\bf Part 4: Running time of Flow Matching Layer.} For each $i \in [K]$, the input size of the $i$-th fast flow-matching layer $\mathsf{FNN}_i$ is $ (n/2^{K-i}) \times (n/2^{K-i}) \times c $. By Definition~\ref{def:mlp}, we can know that the total computational time for the MLP layer is $O(n^{2+o(1)})$, which is due to $c=O(\log n)$. Then by Lemma~\ref{lem:as23_attention}, we can speed up the attention computation from $O(n^{4+o(1)})$ to $O(n^{2+o(1)})$. Hence, the total runtime for all flow-matching layers is
     \begin{align*}
        \mathcal{T}_{\mathrm{Attn}} =&~ \sum_{i=1}^K O( n^{2+o(1)})\\
        =&~ O(n^{2+o(1)})
    \end{align*}
    The equation is due to $K = O(1)$.

    Then, by summing up Part 1 to Part 4, we can get the total running time for fast FlowAR architecture, which is
    \begin{align*}
        \mathcal{T}_{\mathrm{fast}} =&~ \mathcal{T}_{\mathrm{up}} + \mathcal{T}_{\mathrm{Attn}} + \mathcal{T}_{\mathrm{FFN}} + \mathcal{T}_{\mathsf{FM}}\\
        =&~ O(n^{2+o(1)})
    \end{align*}
    
    
\end{proof}

\subsection{Error Analysis of \texorpdfstring{$\mathsf{MLP}(\X')$}{} and \texorpdfstring{$\mathsf{MLP}(\X)$}{}}\label{sec:error_analysis_of_mlp_x_prime_mlp_x}
We conduct the error analysis between $\mathsf{MLP}(\X')$ and $\mathsf{MLP}(\X)$ where $\X'$ is the approximation version of $\X$.
\begin{lemma}[Error analysis of MLP Layer]\label{lem:error_analysis_mlp}
    If the following conditions hold:
    \begin{itemize}
        \item Let $\X \in \R^{h \times w \times c}$ denote the input tensor.
        \item Let $\X' \in \R^{h \times w \times c}$ denote the approximation version of input tensor $\X$.
        \item Let $\epsilon \in (0, 0.1)$ denote the approximation error. 
        \item Suppose we have $\| \X' - \X \|_\infty \leq \epsilon$.
        \item Let $R > 1$.
        \item Assume the value of each entry in matrices can be bounded by $R$.  
        \item Let $\mathsf{MLP}(\cdot,c,d)$ denote the MLP layer defined in Definition~\ref{def:mlp}.
    \end{itemize}
    We can demonstrate the following
    \begin{align*}
        \|\mathsf{MLP}(\X') - \mathsf{MLP}(\X)\|_\infty \leq cR\epsilon
    \end{align*}
    Here, we abuse the $\ell_\infty$ norm in its tensor form for clarity.
\end{lemma}
\begin{proof}
    We can show that for $i \in [h],j \in [w], l \in [c]$, we have
    \begin{align*}
        \|\mathsf{MLP}(\X',c,d)_{i,j,*} - \mathsf{MLP}(\X,c,d)_{i,j,*}\|_\infty =&~ \| \X'_{i,j,*}\cdot W - \X_{i,j,*} \cdot W \|_\infty\\
        \leq&~ \|\underbrace{(\X'_{i,j,*}-\X_{i,j,*})}_{1 \times c} \cdot \underbrace{W}_{c\times d} \|_\infty \\
        \leq &~ c \cdot \|\underbrace{(\X'_{i,j,*}-\X_{i,j,*})}_{1 \times c}\|_\infty \cdot \|\underbrace{W}_{c\times d} \|_\infty\\
        \leq&~ c \cdot R \cdot \epsilon
    \end{align*}
    The first equation is due to Definition~\ref{def:mlp}, the second inequality is derived from simple algebra, the third inequality is a consequence of basic matrix multiplication, and the last inequality comes from the conditions of this lemma.

    Then by the definition of $\ell_\infty$ norm, we can easily get the proof.
\end{proof}



\subsection{Error Analysis of \texorpdfstring{$\mathsf{AAttC}(\X')$}{} and \texorpdfstring{$\mathsf{Attn}(\X)$}{}}\label{sec:error_analysis_of_aattc_x_prime_attn_x}
We conduct the error analysis between $\mathsf{AAttC}(\X')$ and $\mathsf{Attn}(\X)$ where $\X'$ is the approximation version of $\X$.
\begin{lemma}[Error analysis of $\mathsf{AAttC}(X')$ and $\mathsf{Attn}(X)$, Lemma B.4 of \cite{kll+25}]\label{lem:error_analysis_aattc_attn}
    If the following conditions hold:
    \begin{itemize}
        \item Let $\X \in \R^{h \times w \times c}$ denote the input tensor.
        \item Let $\X' \in \R^{h \times w \times c}$ denote the approximation version of input tensor $\X$.
        \item Let $\epsilon \in (0, 0.1)$ denote the approximation error. 
        \item Suppose we have $\| \X' - \X \|_\infty \leq \epsilon$.
        \item Let $R > 1$.
        \item Assume the value of each entry in matrices can be bounded by $R$. 
        \item Let $\mathsf{Attn}$ denote the attention layer defined in Definition~\ref{def:attn_layer}.
        \item Let $\mathsf{AAttC}$ denote the approximated attention layer defined in Definition~\ref{def:aattc}.
        \item Let $U,V \in \R^{hw \times k}$ be low-rank matrices constructed for polynomial approximation of attention matrix $\mathsf{AAttC}(\X)$.
        \item Let $f$ be a polynomial with degree $g$.
    \end{itemize}
    We can demonstrate the following:
    \begin{align*}
        \| \mathsf{AAttC}(\X') - \mathsf{Attn}(\X) \|_\infty \leq O( k R^{g+1} c) \cdot \epsilon
    \end{align*}
    Here, we abuse the $\ell_\infty$ norm in its tensor form for clarity.
\end{lemma}

\subsection{Error Analysis of \texorpdfstring{$\mathsf{FFN}(\X')$}{} and \texorpdfstring{$\mathsf{FFN}(\X)$}{}}\label{sec:error_analysis_of_ffn_x_prime_ffn_x}
In this section, we conduct the error analysis between $\mathsf{FFN}(\X')$ and $\mathsf{FFN}(\X)$ where $\X'$ is the approximation version of $\X$.
\begin{lemma}[Error analysis of $\mathsf{FFN}(\X')$ and $\mathsf{FFN}(\X)$]\label{lem:error_analysis_ffn}
    If the following conditions hold:
    \begin{itemize}
        \item Let $\X \in \R^{h \times w \times c}$ denote the input tensor.
        \item Let $\X' \in \R^{h \times w \times c}$ denote the approximation version of input tensor $\X$.
        \item Let $\epsilon \in (0, 0.1)$ denote the approximation error. 
        \item Suppose we have $\| \X' - \X \|_\infty \leq \epsilon$.
        \item Let $R > 1$.
        \item Assume the value of each entry in matrices can be bounded by $R$. 
        \item Let $\mathsf{FFN}$ denote the FFN layer defined in Definition~\ref{def:ffn}.
        \item Let the activation function $\sigma(\cdot)$ in $\mathsf{FFN}$ be the ReLU activation function.
    \end{itemize}
    We can demonstrate the following:
    \begin{align*}
        \| \mathsf{FFN}(\X') - \mathsf{FFN}(\X) \|_\infty \leq O(c^2 R^2) \cdot \epsilon
    \end{align*}
    Here, we abuse the $\ell_\infty$ norm in its tensor form for clarity.
\end{lemma}
\begin{proof}
    Firstly we can bound that for $i \in [h], j \in [w]$
    \begin{align}\label{eq:linear_transformation_bound}
        \| (\X'_{i,j,*}\cdot W_1 +b_1) - (\X_{i,j,*}\cdot W_1 +b_1)\|_\infty =&~ \| \underbrace{(\X'_{i,j,*}-\X_{i,j,*})}_{1 \times c} \cdot \underbrace{W_1}_{c\times c}\|_\infty\notag\\
        \leq&~ c \cdot \|\X'_{i,j,*}-\X_{i,j,*}\|_\infty \|W_1 \|_\infty\notag\\
        \leq&~ c \cdot \epsilon \cdot R
    \end{align}
    The first equation comes from basic algebra, the second inequality is due to basic matrix multiplication, and the last inequality follows from the conditions of this lemma.
    
    We can show that for $i \in [h], j \in [w]$,
    \begin{align*}
        \| \mathsf{FFN}(\X')_{i,j,*} - \mathsf{FFN}(\X)_{i,j,*} \|_\infty=&~ \| \X'_{i,j,*}-\X_{i,j,*} +\underbrace{ (\sigma(\X_{i,j,*}\cdot W_1 + b_1)- \sigma(\X'_{i,j,*}\cdot W_1 + b_1))}_{1 \times c} \cdot \underbrace{W_2}_{c\times c}\|_\infty \\
        \leq&~ \| \X'_{i,j,*}-\X_{i,j,*}\|_\infty +c\cdot \|W_2\|_\infty \cdot \|\sigma(\X_{i,j,*}\cdot W_1 + b_1)- \sigma(\X'_{i,j,*}\cdot W_1 + b_1)\|_\infty\\
        \leq&~ \epsilon + c R \cdot \| (\X'_{i,j,*} W_1 +b_1) - (\X_{i,j,*} W_1 +b_1)\|_\infty\\
        \leq&~ \epsilon + c^2 R^2 \cdot \epsilon \\
        =&~ O(c^2 R^2) \cdot \epsilon
    \end{align*}
    The first equation is due to Definition~\ref{def:ffn}, the second step follows from triangle inequality and basic matrix multiplication, the third step follows from the property of ReLU activation function and basic algebra, the fourth step follows from Eq.~\eqref{eq:linear_transformation_bound}, and the last step follows from simple algebra.
\end{proof}

\subsection{Error Analysis of \texorpdfstring{$\phi_{\mathrm{up}}(\X')$}{} and \texorpdfstring{$\phi_{\mathrm{up}}(\X)$}{}}\label{sec:error_analysis_of_phi_x_prime_phi_x}
In this section, we conduct the error analysis between $\phi_{\mathrm{up}}(\X')$ and $\phi_{\mathrm{up}}(\X)$ where $\X'$ is the approximation version of $\X$.
\begin{lemma}[Error Analysis of Up Sample Layer, Lemma B.5 of \cite{kll+25}]\label{lem:error_analysis_up_layer}
If the following conditions hold:
\begin{itemize}
    \item Let $\X \in \R^{h \times w \times c}$ denote the input tensor.
    \item Let $\X' \in \R^{h \times w \times c}$ denote the approximation version of input tensor $\X$.
    \item Let $a = 2$ denote a positive integer.
    \item Let $\phi_{\mathrm{up}, i}(\cdot, a)$ be the bicubic up sample function defined in Definition~\ref{def:bicubic_up_sample_function}.
    \item Let $\epsilon \in (0,0.1)$ denote the approximation error.
    \item Let $\|X-X'\|_\infty\leq \epsilon$.
\end{itemize}
Then we have
\begin{align*}
    \| \phi_{\rm up}(\X',a) - \phi_{\rm up}(\X,a) \|_\infty \leq O(\epsilon)
\end{align*}
Here, we abuse the $\ell_\infty$ norm in its tensor form for clarity.
\end{lemma}

\subsection{Error Analysis of \texorpdfstring{$\mathsf{FNN}(\F'^t,\X',t)$}{} and \texorpdfstring{$\mathsf{NN}(\F^t,\X,t)$}{}}\label{sec:error_analysis_of_flow_matching_layer}
In this section, we conduct the error analysis between $\mathsf{FNN}(\F'^t,\X',t)$ and  $\mathsf{NN}(\F^t,\X,t)$ where $\X'$ is the approximation version of $\X$.
\begin{lemma}[Error Analysis of Flow Matching Layer]\label{lem:error_analysis_flow_matching_layer}
If the following conditions hold:
\begin{itemize}
    \item Let $\X \in \R^{h \times w \times c}$ denote the input tensor.
    \item Let $\X' \in \R^{h \times w \times c}$ denote the approximation version of input tensor $\X$.
    \item Let $\F^t,\mathsf{FF}^t \in \R^{h \times w \times c}$ be the interpolated input defined in Definition~\ref{def:flow}.
    \item Let $\mathsf{NN}(\cdot,\cdot,\cdot)$ denote flow-matching layer defined in Definition~\ref{def:flow_matching_architecture}.
    \item Let $\mathsf{FNN}(\cdot,\cdot,\cdot)$ denote fast flow-matching layer defined in Definition~\ref{def:fast_flow_matching_architecture}.
     \item Let $\mathsf{Attn}$ denote the attention layer defined in Definition~\ref{def:attn_layer}.
    \item Let $\mathsf{AAttC}$ denote the approximated attention layer defined in Definition~\ref{def:aattc}.
    \item Let $R > 1$.
    \item Assume the value of each entry in matrices can be bounded by $R$. 
    \item Let $U,V \in \R^{hw \times k}$ be low-rank matrices constructed for polynomial approximation of attention matrix $\mathsf{AAttC}(\X)$.
    \item Let $f$ be a polynomial with degree $g$.
    \item Let $\epsilon \in (0,0.1)$ denote the approximation error.
    \item Let $\|\X-\X'\|_\infty\leq \epsilon$.
    \item Let $t \in [0,1]$ denote a time step.
    \item Assume that Layer-wise Norm layer $\mathsf{LN}(\cdot)$ defined in Definition~\ref{def:ln} does not exacerbate the propagation of errors, i.e., if $\|X'-X\|_\infty \leq \epsilon$, then $\|\mathsf{LN}(X')-\mathsf{LN}(X)\|_\infty \leq \epsilon$.
\end{itemize}
Then we have
\begin{align*}
    \| \mathsf{FNN}(\mathsf{FF}^t,\X',t) - \mathsf{NN}(\F^t,\X,t) \|_\infty \leq O(kR^{g+6}c^3) \cdot \epsilon
\end{align*}
Here, we abuse the $\ell_\infty$ norm in its tensor form for clarity.
\end{lemma}
\begin{proof}
    Firstly, we can show that
    \begin{align*}
        \|\mathsf{FF}^t- \F^t\|_\infty = \| t (\X' - \X) \|_\infty \leq \epsilon
    \end{align*}
    The inequality comes from $t \in [0,1]$ and $\|\X'-\X\|_\infty \leq \epsilon$.

    By {\bf Step 1} of Definition~\ref{def:flow_matching_architecture} and Definition~\ref{def:fast_flow_matching_architecture}, we need to compute
    \begin{align*}
        \alpha_1, \alpha_2, \beta_1, \beta_2, \gamma_1, \gamma_2=&~  \mathsf{MLP}(\X + t \cdot {\bf 1}_{h \times w \times c},c,6c)\\
         \alpha'_1, \alpha'_2, \beta'_1, \beta'_2, \gamma'_1, \gamma'_2=&~  \mathsf{MLP}(\X' + t \cdot {\bf 1}_{h \times w \times c},c,6c)\\
    \end{align*}
    Then, we can show that
    \begin{align*}
        \|\alpha'_1 - \alpha_1\|_\infty \leq c R \epsilon
    \end{align*}
    where the step follows from Lemma~\ref{lem:error_analysis_mlp}. The same conclusion holds for the intermediate parameter $\alpha_2, \beta_1, \beta_2, \gamma_1, \gamma_2$.



    By {\bf Step 2} of Definition~\ref{def:flow_matching_architecture} and Definition~\ref{def:fast_flow_matching_architecture}, we need to compute
    \begin{align*}
        \F'^t =&~ \mathsf{Attn}(\gamma_1 \circ \mathsf{LN}(\F^t)+\beta_1) \circ \alpha_1\\
        \mathsf{FF}'^t =&~ \mathsf{AAttC}(\gamma'_1 \circ \mathsf{LN}(\mathsf{FF}^t)+\beta'_1) \circ \alpha'_1\\
    \end{align*}
    Then, we move forward to show that
    \begin{align}\label{eq:flow_matching_tmp1}
        &~\| \gamma'_1 \circ \mathsf{LN}(\mathsf{FF}^t) + \beta'_1 - \gamma_1 \circ \mathsf{LN}(\F^t) - \beta_1\|_\infty\notag\\
        \leq&~ \| \gamma'_1 \circ \mathsf{LN}(\mathsf{FF}^t) - \gamma_1 \circ \mathsf{LN}(\F^t)  \|_\infty + \|\beta'_1-\beta_1\|_\infty\notag\\
        \leq&~  \| \gamma'_1 \circ (\mathsf{LN}(\mathsf{FF}^t) - \mathsf{LN}(\F^t))\| + \|(\gamma'_1-\gamma_1) \circ \mathsf{LN}(\F^t)\|_\infty + cR\epsilon\notag\\
        \leq&~ R \cdot \epsilon + R\cdot \epsilon + cR\epsilon\notag\\
        =&~ O(cR)\cdot \epsilon
    \end{align}
    where the first and second step follows from triangle inequality, the third step follows from conditions of this Lemma, and the last step follows from simple algebra.

    Then we have
    \begin{align}\label{eq:flow_matching_tmp2}
        \| \mathsf{AAttC}(\gamma'_1 \circ \mathsf{LN}(\mathsf{FF}^t) + \beta'_1) -\mathsf{Attn}(\gamma_1 \circ \mathsf{LN}(\F^t) + \beta_1)\|_\infty \leq&~ O(k R^{g+1} c) \cdot O(cR) \cdot \epsilon \notag\\
        \leq&~ O(kR^{g+2} c^2) \epsilon
    \end{align}
    where the first step follows from Lemma~\ref{lem:error_analysis_aattc_attn} and Eq.~\eqref{eq:flow_matching_tmp1} and the second step follows from simple algebra.

    Now, we are able to show that
    \begin{align}\label{eq:error_analysis_tmp3}
        \| \mathsf{FF}'^t -  \F'^t\|_\infty =&~ \|\mathsf{AAttC}(\gamma'_1 \circ \mathsf{LN}(\mathsf{FF}^t)+\beta'_1) \circ \alpha'_1 -  \mathsf{Attn}(\gamma_1 \circ \mathsf{LN}(\F^t)+\beta_1) \circ \alpha_1\|_\infty\notag \\
        \leq&~ \| \mathsf{AAttC}(\gamma'_1 \circ \mathsf{LN}(\mathsf{FF}^t)+\beta'_1) \circ (\alpha'_1-\alpha_1)\|_\infty \notag \\+&~ \|\alpha_1 \cdot \mathsf{AAttC}(\gamma'_1 \circ \mathsf{LN}(\mathsf{FF}^t) + \beta'_1) -\mathsf{Attn}(\gamma_1 \circ \mathsf{LN}(\F^t) + \beta_1) \|_\infty\notag \\
        \leq&~ R \cdot cR\epsilon + R \cdot O(kR^{g+2}c^2) \epsilon\notag \\
        =&~   O(kR^{g+3}c^2) \epsilon
    \end{align}
    where the first step follows from the definition of $\wh{\F}'^t$ and $\wh{\F}^t$, the second step follows from triangle inequality, the third step follows from Eq.~\eqref{eq:flow_matching_tmp2} and the conditions of this lemma, and the last step follows from simple algebra.

    By {\bf Step 3} of Definition~\ref{def:flow_matching_architecture} and Definition~\ref{def:fast_flow_matching_architecture}, we need to compute
    \begin{align*}
        \F''^t =&~\mathsf{MLP}(\gamma_2 \circ \mathsf{LN}(\F'^t)+ \beta_2,c,c) \circ \alpha_2\\
        \mathsf{FF}''^t=&~\mathsf{MLP}(\gamma'_2 \circ \mathsf{LN}(\mathsf{FF}'^t)+ \beta'_2,c,c) \circ \alpha'_2
    \end{align*}
    Then, we move forward to show that
    \begin{align}\label{eq:error_analysis_tmp4}
        &~\|\gamma'_2 \circ \mathsf{LN}(\mathsf{FF}'^t)+ \beta'_2 - \gamma_2 \circ \mathsf{LN}(\F'^t) -\beta_2 \|_\infty\notag\\
        \leq&~ \| \gamma'_2 \circ \mathsf{LN}(\mathsf{FF}'^t)- \gamma_2 \circ \mathsf{LN}(\F'^t) \|_\infty + \|\beta'_1-\beta_1\|_\infty\notag\\
        \leq&~\|\gamma'_2\circ(\mathsf{LN}(\mathsf{FF}'^t) - \mathsf{LN}(\F'^t)) \|_\infty+ \|(\gamma'_2-\gamma_2)\circ \mathsf{LN}(\F'^t)\|_\infty + cR\epsilon\notag\\
        \leq&~ R \cdot O(kR^{g+3}c^2) \epsilon + cR\epsilon \cdot R + cR\epsilon\notag\\
        =&~ O(kR^{g+4}c^2)\cdot \epsilon
    \end{align}
    where the first and the second steps follow from triangle inequality, the third step follows from Eq.~\eqref{eq:error_analysis_tmp3} the conditions of this lemma, and the last step follows from simple algebra.

    Then, we can show
    \begin{align}\label{eq:error_analysis_tmp5}
        \|\mathsf{MLP}(\gamma'_2 \circ \mathsf{LN}(\mathsf{FF}'^t)+ \beta'_2)-\mathsf{MLP}(\gamma_2 \circ \mathsf{LN}(\F'^t) +\beta_2)\|_\infty \leq&~  c R \cdot O(kR^{g+4} c^2) \cdot \epsilon\notag\\
        =&~ O(kR^{g+5}c^3)\cdot \epsilon
    \end{align}
    where the first step follows from Lemma~\ref{lem:error_analysis_mlp} and Eq.~\eqref{eq:error_analysis_tmp4} and the second step follows from simple algebra.

    Finally, we are able to show that
    \begin{align*}
        &~\|\mathsf{FNN}(\mathsf{FF}^t,\X',t) - \mathsf{FN}(\F^t,\X,t)\|_\infty\\
        =&~ \|\mathsf{MLP}(\gamma'_2 \circ \mathsf{LN}(\mathsf{FF}'^t)+ \beta'_2,c,c) \circ \alpha'_2 - \mathsf{MLP}(\gamma_2 \circ \mathsf{LN}(\F'^t)+ \beta_2,c,c) \circ \alpha_2 \|_\infty\\
        \leq&~ \| (\mathsf{MLP}(\gamma'_2 \circ \mathsf{LN}(\mathsf{FF}'^t)+ \beta'_2,c,c) - \mathsf{MLP}(\gamma_2 \circ \mathsf{LN}(\F'^t)+ \beta_2,c,c))\circ \alpha'_2\|_\infty \\+ &~ \|\mathsf{MLP}(\gamma_2 \circ \mathsf{LN}(\F'^t)+ \beta_2,c,c)  \circ (\alpha'_2 - \alpha_2) \|_\infty\\
        \leq&~ R \cdot O(kR^{g+5}c^3) \cdot \epsilon + R \cdot cR\epsilon\\
        =&~  O(kR^{g+6}c^3) \cdot \epsilon
    \end{align*}
    where the step follows from the definition of output of $\mathsf{FFN}(\F'^t,\X',t)$ and $ \mathsf{FN}(\F^t,\X,t)$, the second step follows from triangle inequality, the third step follows from Eq.~\eqref{eq:error_analysis_tmp5} and conditions of this lemma, and the last step follows from simple algebra.

    Then, we complete the proof.
\end{proof}

\subsection{Error Analysis of Fast FlowAR Architecture}\label{sec:error_analysis_fast_flowar}
Here, we proceed to present the error analysis of fast FlowAR Architecture.
\begin{lemma}[Error Bound Between Fast FlowAR and FlowAR Outputs]\label{lem:error_analysis_fast_flowar}
    Given the following:
    \begin{itemize}
        \item {\bf Input tensor:} $\X \in \R^{h \times w \times c}$.
        \item {\bf Scales number:} $K = O(1)$.
        \item {\bf Dimensions:} Let $h=w=n$ and $c = O(\log n)$. Let $\wt{h}_i := \sum_{j=1}^i h/r_j$ and $\wt{w}_i := \sum_{j=1}^i w/r_j$.
        \item {\bf Bounded Entries:} All tensors and matrices have entries bounded by $R = O(\sqrt{\log n})$.
        \item {\bf Layers:}
        \begin{itemize}
            \item $\phi_{\mathrm{up},a}(\cdot)$ :  bicubic upsampling function (Definition~\ref{def:bicubic_up_sample_function}).
            \item $\mathsf{Attn}(\cdot)$: attention layer (Definition~\ref{def:attn_layer}).
            \item $\mathsf{AAttC(\cdot)}$: approximate attention layer (Definition~\ref{def:aattc})
            \item $\mathsf{NN}(\cdot,\cdot,\cdot)$: flow-matching layer (Definition~\ref{def:flow_matching_architecture})
            \item $\mathsf{FNN}(\cdot,\cdot,\cdot)$: fast flow-matching layer (Definition~\ref{def:fast_flow_matching_architecture})
        \end{itemize}
        \item {\bf Input and interpolations:}
        \begin{itemize}
            \item Initial inputs: $\Z_{\mathrm{init}} \in \R^{(h/r_1)\times(w/r_1) \times c}$.
            \item $\Z_i:$ Reshaped tensor of  $\Z_{\mathrm{init}}, \phi_{\mathrm{up},1}(\wt{\Y}_1), \dots, \phi_{\mathrm{up},i-1}(\wt{\Y}_{i-1})$ for FlowAR.
            \item $\Z'_i:$ Reshaped tensor of  $\Z_{\mathrm{init}}, \phi_{\mathrm{up},1}(\wt{\Y}'_1), \dots, \phi_{\mathrm{up},i-1}(\wt{\Y}'_{i-1})$ for  Fast FlowAR.
            \item $\mathsf{F}_i^{t_i} \in \R^{h/r_i \times w/r_i \times c}$ be the interpolated value of FlowAR (Definition~\ref{def:flow}).
            \item $\mathsf{FF}_i^{t_i} \in \R^{h/r_i \times w/r_i \times c}$ be the interpolated value of Fast FlowAR (Definition~\ref{def:flow}).
        \end{itemize} 
        \item {\bf Outputs:}
        \begin{itemize}
            \item $\wt{\Y}_i \in \R^{h/r_i \times w/r_i \times c}$: FlowAR output at layer $i$ (Definition~\ref{def:flow_architecture_inference})
            \item $\wt{\Y}'_i \in \R^{h/r_i \times w/r_i \times c}$: Fast FlowAR output at layer $i$ (Definition~\ref{def:fast_flow_architecture_inference})
        \end{itemize}
    \end{itemize}
    Under these conditions, the $\ell_\infty$ error between the final outputs is bounded by:
    \begin{align*}
        \|\wt{\Y}'_K - \wt{\Y}_K\|_\infty \leq 1/\poly(n)
    \end{align*}
    
\end{lemma}
\begin{proof}
    We can conduct math induction as the following.
    
    Consider the first layer of fast FlowAR Architecture. Firstly, we can show that
    \begin{align*}
        \| \mathsf{AAttC}_1(\Z_{1}) - \mathsf{Attn}_1(\Z_{1})\|_\infty \leq 1/\poly(n)
    \end{align*}
    The inequality is derived Lemma~\ref{lem:as23_attention}.
    
    Then, we have
    \begin{align*}
        \| \wh{\Y}'_1 - \wh{\Y}_1\|_\infty =&~ \|\mathsf{FFN}_1(\mathsf{AAttC}_1(\Z_{1})) - \mathsf{FFN}_1(\mathsf{Attn}_1(\Z_{1}))\|_\infty\\
        \leq&~ O(c^2 R^2) \cdot 1/\poly(n)\\
        = &~ 1/\poly(n)
    \end{align*}
    The first equation comes from the definition of $\wh{Y}'_1$ and $\wh{Y}_1$, the second inequality is due to Lemma~\ref{lem:error_analysis_ffn} and the last equation is due to $c = O(\log n)$ and $R = O(\sqrt{\log n})$.

    Then, we can show that
    \begin{align*}
        \| \wt{\Y}'_1 - \wt{Y}_1\|_\infty =&~ \| \mathsf{FNN}_1(\mathsf{FF}^{t_1}_1,\wh{Y}'_1,t_1)  - \mathsf{NN}_1(\F^{t_1}_1,\wh{Y}_1,t_1) \|_\infty\\
        \leq &~ O(kR^{g+6}c^3) \cdot 1/\poly(n)\\
        =&~ 1/\poly(n)
    \end{align*}
    The first equation is due to the definition of $\Y'_1$ and $\Y_1$, the second inequality comes from Lemma~\ref{lem:error_analysis_flow_matching_layer}, and the last step follows from $c = O(\log n)$ and $R = O(\sqrt{\log n})$.


    Assume that the following statement is true for $k$-th iteration (where $k < K$):
    \begin{align*}
        \|\wt{\Y}'_k - \wt{Y}_k\|_\infty \leq 1/\poly(n)
    \end{align*}
    Then, we can easily to bound
    \begin{align*}
        \| \Z'_{k+1} - \Z_{k+1}\|_\infty \leq 1/\poly(n)
    \end{align*}
    The inequality is due to Lemma~\ref{sec:error_analysis_of_phi_x_prime_phi_x} and Definition of $\Z'_{k+1}$ and $\Z_{k+1}$.

    Then, we can show that 
    \begin{align*}
        \|\mathsf{AAttC}_{k+1}(\Z'_{k+1}) -\mathsf{Attn}_{k+1}(\Z_{k+1}) \|_\infty \leq&~ O(k R^{g+1} c) \cdot 1/\poly(n)\\
        =&~ 1/\poly(n) 
    \end{align*}
    The first inequality comes from Lemma~\ref{lem:error_analysis_aattc_attn}, and the second equation is due to $c = O(\log n)$ and $R = O(\sqrt{\log n})$.

    Then we have
    \begin{align*}
        \| \wh{\Y}'_{k+1} - \wh{\Y}_{k+1}\|_\infty =&~ \|\mathsf{FFN}_{k+1}(\mathsf{AAttC}_{k+1}(\Z'_{k+1})) - \mathsf{FFN}_{k+1}(\mathsf{Attn}_{k+1}(\Z_{k+1}))\|_\infty\\
        \leq&~ O(c^2 R^2) \cdot 1/\poly(n)\\
        = &~ 1/\poly(n)
    \end{align*}
   The first equation comes from the definition of $\wh{Y}'_{k+1}$ and $\wh{Y}_{k+1}$, the second inequality is due to Lemma~\ref{lem:error_analysis_ffn} and the third equation is due to $c = O(\log n)$ and $R = O(\sqrt{\log n})$.

    Then, we can derive that
    \begin{align*}
        \| \wt{\Y}'_{k+1} - \wt{Y}_{k+1}\|_\infty =&~ \| \mathsf{FNN}_{k+1}(\mathsf{FF}^{t_{k+1}}_{k+1},\wh{Y}'_{k+1},t_{k+1})  - \mathsf{NN}_{k+1}(\F^{t_{k+1}}_{k+1},\wh{Y}_k+1,t_{k+1}) \|_\infty\\
        \leq &~ O(kR^{g+6}c^3) \cdot 1/\poly(n)\\
        =&~ 1/\poly(n)
    \end{align*}
    The first equation comes from the definition of $\Y'_{k+1} $ and $ \Y_{k+1}$, the second inequality is due to Lemma~\ref{lem:error_analysis_flow_matching_layer} and the third equation is due to $c = O(\log n)$ and $R = O(\sqrt{\log n})$.

    Then, by mathematical induction, we can get the proof.
\end{proof}



\ifdefined\isarxiv
%\section*{Acknowledgments}
\bibliographystyle{alpha}
\bibliography{ref}
\else

\bibliography{ref}
%\bibliographystyle{icml2022}
% \bibliographystyle{alpha}

\fi








\end{document}



%%%%%%%%%%%%%%%%%%%%%%%%%%%%%%%%%%%%%%%%%%%%%%%%%%%%%%%%%%%%%%%%%%%%%%%%%%%%%%%%%%%%%%%%%%%%%%%%%%%%%%%%%%%%%%%%%%%%%%%%%%%%%%%%%%%%%%%%%%%%%%%%%%%%%%%%%%%%%%%%%%%%%%%%%%%%%%%%%%%%%%%%%%%%%%%%%%%%%%%%%%%%%%%%%%%%%%%%%%%%%%%%%%%%%%%%%%%%%%%%%%%%%%%%%%%%%%%%%%%%%%%%%%%%%%%%%%%%%%%%%%%%%%%%%%%%%%%%%%%%%%%%%%%%%%%%%%%%%%%%%%%%%%%%%%%%%%%%%%%%%%%%%%%%%%%%%%%%%%%%%%%%%%%%%%%%%%%%%%%%%%%%%%%%%%%%%%%%%%%%%%%%%%%%%%%%%%%%%%%%%%%%%%%%%%%%%%%%%%%%%%%%%%%%%%%%%%%%%%%%%%
