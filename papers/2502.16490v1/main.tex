\def\isarxiv{1} %%% for icml submission version, we comment this line

\ifdefined\isarxiv
\documentclass[11pt]{article}

\usepackage[numbers]{natbib}

\else

\documentclass{uai2025} % for initial submission
%\documentclass[accepted]{uai2025} % after acceptance, for a revised version; 
% also before submission to see how the non-anonymous paper would look like 
                        
%% There is a class option to choose the math font
% \documentclass[mathfont=ptmx]{uai2025} % ptmx math instead of Computer
                                         % Modern (has noticeable issues)
% \documentclass[mathfont=newtx]{uai2025} % newtx fonts (improves upon
                                          % ptmx; less tested, no support)
% NOTE: Only keep *one* line above as appropriate, as it will be replaced
%       automatically for papers to be published. Do not make any other
%       change above this note for an accepted version.

%% Choose your variant of English; be consistent
\usepackage[american]{babel}
% \usepackage[british]{babel}

%% Some suggested packages, as needed:
\usepackage{natbib} % has a nice set of citation styles and commands
    \bibliographystyle{plainnat}
    \renewcommand{\bibsection}{\subsubsection*{References}}
\usepackage{mathtools} % amsmath with fixes and additions
% \usepackage{siunitx} % for proper typesetting of numbers and units
\usepackage{booktabs} % commands to create good-looking tables
\usepackage{tikz} % nice language for creating drawings and diagrams

%% Provided macros
% \smaller: Because the class footnote size is essentially LaTeX's \small,
%           redefining \footnotesize, we provide the original \footnotesize
%           using this macro.
%           (Use only sparingly, e.g., in drawings, as it is quite small.)

%% Self-defined macros
\newcommand{\swap}[3][-]{#3#1#2} % just an example


\fi


\usepackage{amsmath}
\usepackage{amsthm}
\usepackage{amssymb}
\usepackage{algorithm}
\usepackage{subfig}
\usepackage{algpseudocode}
\usepackage{graphicx}
\usepackage{grffile}
\usepackage{wrapfig,epsfig}
\usepackage{url}
\usepackage{xcolor}
\usepackage{epstopdf}


\usepackage{bbm}
\usepackage{dsfont}

 
\allowdisplaybreaks
 

\ifdefined\isarxiv

\let\C\relax
\usepackage{tikz}
\usepackage{hyperref}  %%% arxiv don't allow this.
\hypersetup{colorlinks=true,citecolor=blue,linkcolor=blue} %%% Zhao : maybe we should comment this in submission.
\usetikzlibrary{arrows}
\usepackage[margin=1in]{geometry}

\else

% \usepackage{microtype}
\usepackage{hyperref}
% \definecolor{mydarkblue}{rgb}{0,0.08,0.45}
% \hypersetup{colorlinks=true, citecolor=mydarkblue,linkcolor=mydarkblue}
 

\fi
 
\graphicspath{{./figs/}}

\theoremstyle{plain}
\newtheorem{theorem}{Theorem}[section]
\newtheorem{lemma}[theorem]{Lemma}
\newtheorem{definition}[theorem]{Definition}
\newtheorem{notation}[theorem]{Notation}
%\newtheorem{proof}[theorem]{Proof}
\newtheorem{proposition}[theorem]{Proposition}
\newtheorem{corollary}[theorem]{Corollary}
\newtheorem{conjecture}[theorem]{Conjecture}
\newtheorem{assumption}[theorem]{Assumption}
\newtheorem{observation}[theorem]{Observation}
\newtheorem{fact}[theorem]{Fact}
\newtheorem{remark}[theorem]{Remark}
\newtheorem{claim}[theorem]{Claim}
\newtheorem{example}[theorem]{Example}
\newtheorem{problem}[theorem]{Problem}
\newtheorem{open}[theorem]{Open Problem}
\newtheorem{property}[theorem]{Property}
\newtheorem{hypothesis}[theorem]{Hypothesis}

\newcommand{\wh}{\widehat}
\newcommand{\wt}{\widetilde}
\newcommand{\ov}{\overline}
\newcommand{\N}{\mathcal{N}}
\newcommand{\R}{\mathbb{R}}
\newcommand{\RHS}{\mathrm{RHS}}
\newcommand{\LHS}{\mathrm{LHS}}
\renewcommand{\d}{\mathrm{d}}
\renewcommand{\i}{\mathbf{i}}
\renewcommand{\tilde}{\wt}
\renewcommand{\hat}{\wh}
\newcommand{\Tmat}{{\cal T}_{\mathrm{mat}}}
\newcommand{\X}{\mathsf{X}}
\newcommand{\Y}{\mathsf{Y}}
\newcommand{\Z}{\mathsf{Z}}
\newcommand{\W}{\mathsf{W}}
\newcommand{\F}{\mathsf{F}}
\newcommand{\V}{\mathsf{V}}
\newcommand{\AC}{\mathsf{AC}}
\newcommand{\NC}{\mathsf{NC}}
\newcommand{\TC}{\mathsf{TC}}
\newcommand{\dplus}{d_\oplus}
\newcommand{\dlogtime}{\mathsf{DLOGTIME}}

\DeclareMathOperator*{\E}{{\mathbb{E}}}
\DeclareMathOperator*{\var}{\mathrm{Var}}
% \DeclareMathOperator*{\Z}{\mathbb{Z}}
\DeclareMathOperator*{\C}{\mathbb{C}}
\DeclareMathOperator*{\D}{\mathcal{D}}
\DeclareMathOperator*{\median}{median}
\DeclareMathOperator*{\mean}{mean}
\DeclareMathOperator{\OPT}{OPT}
\DeclareMathOperator{\supp}{supp}
\DeclareMathOperator{\poly}{poly}
\DeclareMathOperator{\ds}{/\!\!/}
\DeclareMathOperator{\nnz}{nnz}
\DeclareMathOperator{\sparsity}{sparsity}
\DeclareMathOperator{\rank}{rank}
\DeclareMathOperator{\diag}{diag}
\DeclareMathOperator{\dist}{dist}
\DeclareMathOperator{\cost}{cost}
\DeclareMathOperator{\vect}{vec}
\DeclareMathOperator{\tr}{tr}
\DeclareMathOperator{\dis}{dis}
\DeclareMathOperator{\cts}{cts}

\DeclareMathOperator{\round}{round}%Yekun:add Feb 9. 2025

\makeatletter
\newcommand*{\RN}[1]{\expandafter\@slowromancap\romannumeral #1@}
\makeatother

\usepackage{lineno}
\def\linenumberfont{\normalfont\small}


\ifdefined\isarxiv
\else


\title{Circuit Complexity Bounds and Provably Efficient Criteria of FlowAR Model}

\author[1]{\href{mailto:<jj@example.edu>?Subject=Your UAI 2025 paper}{Jane~J.~von~O'L\'opez}{}}
\author[1]{Harry~Q.~Bovik}
\author[1,2]{Further~Coauthor}
\author[3]{Further~Coauthor}
\author[1]{Further~Coauthor}
\author[3]{Further~Coauthor}
\author[3,1]{Further~Coauthor}
% Add affiliations after the authors
\affil[1]{%
    Computer Science Dept.\\
    Cranberry University\\
    Pittsburgh, Pennsylvania, USA
}
\affil[2]{%
    Second Affiliation\\
    Address\\
    …
}
\affil[3]{%
    Another Affiliation\\
    Address\\
    …
  }
  

\fi


\begin{document}

\ifdefined\isarxiv

\date{}

\title{On Computational Limits of FlowAR Models: Expressivity and Efficiency}
% \author{
% Chengyue Gong
% \and 
% Yekun Ke\thanks{\texttt{
% keyekun0628@gmail.com}. Independent Researcher.}
% \and
% Xiaoyu Li\thanks{\texttt{
% xiaoyu.li2@student.unsw.edu.au}. University of New South Wales.}
% \and
% Zhizhou Sha\thanks{\texttt{ shazz20@mails.tsinghua.edu.cn}. Tsinghua University.}
% \and
% Zhenmei Shi\thanks{\texttt{
% zhmeishi@cs.wisc.edu}. University of Wisconsin-Madison.}
% \and 
% Zhao Song\thanks{\texttt{ magic.linuxkde@gmail.com}. The Simons Institute for the Theory of Computing at UC Berkeley.}
% }
\author{
Chengyue Gong\thanks{\texttt{ cygong17@utexas.edu}. The University of Texas at Austin.}
\and
Yekun Ke\thanks{\texttt{
keyekun0628@gmail.com}. Independent Researcher.}
\and
Xiaoyu Li\thanks{\texttt{
xiaoyu.li2@student.unsw.edu.au}. University of New South Wales.}
\and
Yingyu Liang\thanks{\texttt{
yingyul@hku.hk}. The University of Hong Kong. \texttt{
yliang@cs.wisc.edu}. University of Wisconsin-Madison.} 
\and
Zhizhou Sha\thanks{\texttt{
shazz20@mails.tsinghua.edu.cn}. Tsinghua University.}
\and
Zhenmei Shi\thanks{\texttt{
zhmeishi@cs.wisc.edu}. University of Wisconsin-Madison.}
\and
Zhao Song\thanks{\texttt{ magic.linuxkde@gmail.com}. The Simons Institute for the Theory of Computing at UC Berkeley.}
}




\else

\fi





\ifdefined\isarxiv
\begin{titlepage}
  \maketitle
  \begin{abstract}
\begin{abstract}

% Recent works to jointly reconstruct 3D human and object from a single RGB image, are mostly model-based, that fail to capture the fine details of the clothed human body and object surface. In this paper, we introduce ReCHOR, a novel, model-free, first-method to produce realistic clothed human-object reconstructions from a monocular view. This is extremely challenging due to human-object occlusions, diverse interactions and depth ambiguity, as it needs to infer both 3D spatial awareness and high resolution details. Our core idea is based on estimating neural implicit representations for human and object respectively by an attention-based neural implicit model that attends to pixel-aligned features from both the global human-object image for spatial awareness and  the local separate view of human and object images for high quality details. Additionally, the network is conditioned on semantic features from an initial estimated human-object pose prior and a generative diffusion model that inpaints occluded regions, thus enabling the retrieval of details from them.
% We also propose a synthetic dataset with rendered scenes of diverse, inter-occluded 3D human and object scans, to train our network. We evaluate our method on the synthetic and real world BEHAVE dataset. Our experiments show that our method outperforms the SOTA in achieving realistic clothed human-object reconstructions.
Recent approaches to jointly reconstruct 3D humans and objects from a single RGB image represent 3D shapes with template-based or coarse models, which fail to capture details of loose clothing on human bodies. In this paper, we introduce a novel implicit approach for jointly reconstructing realistic 3D clothed humans and objects from a monocular view. For the first time, we model both the human and the object with an implicit representation, allowing to capture more realistic details such as clothing. This task is extremely challenging due to human-object occlusions and the lack of 3D information in 2D images, often leading to poor detail reconstruction and depth ambiguity. To address these problems, we propose a novel attention-based neural implicit model that leverages image pixel alignment from both the input human-object image for a global understanding of the human-object scene and from local separate views of the human and object images to improve realism with, for example, clothing details. Additionally, the network is conditioned on semantic features derived from an estimated human-object pose prior, which provides 3D spatial information about the shared space of humans and objects. To handle human occlusion caused by objects, we use a generative diffusion model that inpaints the occluded regions, recovering otherwise lost details. For training and evaluation, we introduce a synthetic dataset featuring rendered scenes of inter-occluded 3D human scans and diverse objects. Extensive evaluation on both synthetic and real-world datasets demonstrates the superior quality of the proposed human-object reconstructions over competitive methods.
\end{abstract}

  \end{abstract}
  \thispagestyle{empty}
\end{titlepage}

{\hypersetup{linkcolor=black}
\tableofcontents
}
\newpage

\else

\maketitle 

\begin{abstract}
\begin{abstract}

% Recent works to jointly reconstruct 3D human and object from a single RGB image, are mostly model-based, that fail to capture the fine details of the clothed human body and object surface. In this paper, we introduce ReCHOR, a novel, model-free, first-method to produce realistic clothed human-object reconstructions from a monocular view. This is extremely challenging due to human-object occlusions, diverse interactions and depth ambiguity, as it needs to infer both 3D spatial awareness and high resolution details. Our core idea is based on estimating neural implicit representations for human and object respectively by an attention-based neural implicit model that attends to pixel-aligned features from both the global human-object image for spatial awareness and  the local separate view of human and object images for high quality details. Additionally, the network is conditioned on semantic features from an initial estimated human-object pose prior and a generative diffusion model that inpaints occluded regions, thus enabling the retrieval of details from them.
% We also propose a synthetic dataset with rendered scenes of diverse, inter-occluded 3D human and object scans, to train our network. We evaluate our method on the synthetic and real world BEHAVE dataset. Our experiments show that our method outperforms the SOTA in achieving realistic clothed human-object reconstructions.
Recent approaches to jointly reconstruct 3D humans and objects from a single RGB image represent 3D shapes with template-based or coarse models, which fail to capture details of loose clothing on human bodies. In this paper, we introduce a novel implicit approach for jointly reconstructing realistic 3D clothed humans and objects from a monocular view. For the first time, we model both the human and the object with an implicit representation, allowing to capture more realistic details such as clothing. This task is extremely challenging due to human-object occlusions and the lack of 3D information in 2D images, often leading to poor detail reconstruction and depth ambiguity. To address these problems, we propose a novel attention-based neural implicit model that leverages image pixel alignment from both the input human-object image for a global understanding of the human-object scene and from local separate views of the human and object images to improve realism with, for example, clothing details. Additionally, the network is conditioned on semantic features derived from an estimated human-object pose prior, which provides 3D spatial information about the shared space of humans and objects. To handle human occlusion caused by objects, we use a generative diffusion model that inpaints the occluded regions, recovering otherwise lost details. For training and evaluation, we introduce a synthetic dataset featuring rendered scenes of inter-occluded 3D human scans and diverse objects. Extensive evaluation on both synthetic and real-world datasets demonstrates the superior quality of the proposed human-object reconstructions over competitive methods.
\end{abstract}
\end{abstract}

\fi


\section{Introduction}\label{sec:intro}

In computational finance, Monte Carlo simulations are used extensively to estimate the expected value of financial payoffs based on the solution of stochastic differential equations (SDEs) which model the evolution of stock prices, interest rates, exchange rates and other quantities \cite{glasserman04}.  Monte Carlo methods are very general and flexible, but for high accuracy it requires generating a large number of costly SDE path approximations, which has motivated research into a number of variance reduction or, equivalently, cost reduction techniques. One such method is
Multilevel Monte Carlo (MLMC), which was proposed in \cite{GILES2008} and was adapted for various applications that are summarised in \cite{Giles_overview17} and successfully combined with other methods such as quasi-Monte Carlo methods. The main idea of MLMC is to approximate the payoff using different time stepping resolutions when numerically solving the underlying SDE and to generate an optimal number of samples on each level, such that the overall computational cost is minimised subject to the desired bound on the variance. %, such that the total computational cost is minimised. 
The computational savings come from the fact that most samples are computed on the coarser levels and hence are less expensive while only a few samples from the finest levels are required \cite{GILES2008}.


Among the directions in which the computational cost 
of MLMC methods could further be reduced, an important avenue is the use of lower precision calculations, especially for the first Monte Carlo levels where the targeted accuracy is relatively low. 
 An overview of the research on mixed precision for the standard Monte Carlo (MC) framework is provided in \cite{ChowMixedPrecisionStandardMC} but only a few references study the potential of low precision computation in the MLMC framework \cite{Rounding_error_oliver}. To the best of our knowledge, the only MLMC framework with customised precision in the literature is \cite{brugger2014mixed}, but they use a uniform precision for all operations on each Monte Carlo level instead of optimising 
 the precision of each intermediary variable to reduce as much as possible the cost of path generation.
 
An important motivation for an MLMC framework with variable precision would be performing the low precision computations on reconfigurable hardware devices such as Field Programmable Gate Arrays (FPGAs). FPGAs contain customizable logic blocks and connectors that make it easy to adapt the digital circuit architecture for a specific application, leading to a highly parallel and optimised implementation. Therefore they are successfully exploited in applications that require high speed and have high computational workload, such as signal processing \cite{woods2008fpga}, and real time applications like high frequency trading \cite{HFT1,HFT2}. That is why a number of previous works in hardware architecture design implemented the MLMC algorithm to price financial options using FPGAs as accelerators, which resulted in improved speed and power efficiency compared to full CPU architectures \cite{Schryver2013AMM}. The paper \cite{lindsey2016domain} also proposed 
a Domain Specific Language to automate the configuration of FPGAs for this specific application. However, only \cite{brugger2014mixed} proposed a heuristic to reduce the precision in calculations.

In addition, all aforementioned works considered that the random number generation (RNG) is performed in single or double precision. Yet in most cases an important portion of the workload in the overall MLMC simulation comes from the RNG and in \cite{brugger2014mixed} this limited the total computational savings.
To reduce the cost of MLMC simulations in particular those based on the Geometric Brownian Motion (GBM), \cite{approximateICDF_Oliver, NestedOliver} have proposed to use approximate random numbers that are generated by applying an approximation of the inverse CDF to uniform random numbers. In \cite{NestedOliver}, the authors proposed a way to integrate these lower precision random variables into a \textit{nested} MLMC framework and completed a numerical analysis to bound the resulting error at each MC level by a product of the time step and the error in the random number approximation. The same authors show in \cite{approximateICDF_Oliver} that using approximate random variables reduces the cost of path generation by a factor 7.


In this paper we propose a nested MLMC framework that combines the use of approximate random normal variables and lower precision calculations to reduce the computational cost of MLMC even further than \cite{brugger2014mixed,NestedOliver}. We illustrate the efficiency of our framework in Matlab, after making several assumptions on the cost of operations and size of the errors that we carefully justify. We focus on the case of GBM and use the approximate RNG methods presented in \cite{approximateICDF_Oliver} as well as a new slightly modified method that combines CDF inversion and the central limit theorem. To choose the precision of the variables in the low precision path generation, we introduce a novel method to optimise the bit-widths. This optimisation is performed before the main path generation loop is executed and is based on a linear model of the payoff error  
due to rounding when computing in low precision. The error model relies on algorithmic differentiation in a similar manner to \cite{unifying-bwoptim,bitwidth-AD,ADAPT}. The bit-width optimisation procedure can be performed off-line, so this stage can be excluded from the on-line time complexity of our framework. The user specified desired accuracy is then enforced by calculating on-line the number of samples that need to be generated.

In terms of hardware design, we suggest implementing the low precision path generation on FPGAs and the full-precision ones on a CPU or GPU. 
The FPGA offers enough flexibility to define a separate bit-width for every variable in the low precision path generation, and can be reconfigured periodically to update the bit-widths when the market parameters have changed considerably. 


The paper is organized as follows : \Cref{sec:MLMC} introduces MLMC and nested MLMC to make clear the estimator that is implemented in our framework. Then in \Cref{sec:RNG} we detail the methods that could be used to obtain approximate random normally distributed numbers very cheaply for the low precision path generation. In \Cref{sec:error_model} and \Cref{sec:costModel} we propose an error model and a cost model (resp.) that we then use to formulate the optimisation problem that is solved to obtain the optimal bit-widths of fixed point variables in \Cref{sec:optimisation}. Finally we summarise our results and future directions in \Cref{sec:conclusion}.


 %%% Section 1. Introduction
\section{Related Work}
\label{sec:related_work}

The original investigation \cite{gibson1979ecological} on the relationship between visual perception and human action defines \emph{affordance} as the opportunities for interaction with the surrounding environment. Behavioral studies on regular and cognitively impaired persons have shown evidence that perception results in both visual and motor signals in the human brain. An extended study \cite{anderson2002attentional} shows that visual attention to the spatial characteristics of the perceived objects initiates automatic motor signals for different actions. In computer vision, human affordance learning involves novel pose prediction such that the estimated pose represents a valid human action within the scene context. The task is fundamental to many problems requiring robust semantic reasoning about the environment, such as human motion synthesis \cite{wang2021scene} and scene-aware human pose generation \cite{wang2017binge, roy2016multi, zhang2022inpaint, yao2023scene}.

Earlier methods of affordance learning have explored knowledge mining \cite{zhu2014reasoning} and multimodal feature cues \cite{roy2016multi} to address the problem. In \cite{zhu2014reasoning}, the authors use a Markov Logic Network for constructing a knowledge base by extracting several object attributes from different image and metadata sources, which can perform various downstream visual inference tasks without any additional classifier, including zero-shot affordance prediction. In \cite{roy2016multi}, the authors use depth map, surface normals, and segmentation map as multimodal cues to train a multi-scale convolutional neural network (CNN) for scene-level semantic label assignment associated with specific human actions. In \cite{do2018affordancenet}, the authors design a multi-branch end-to-end CNN with two separate pathways for object detection and affordance label assignment to achieve high real-time inference throughput. Researchers \cite{chuang2018learning} have also explored socially imposed constraints for affordance learning. In \cite{chuang2018learning}, the authors propose a graph neural network (GNN) to propagate contextual scene information from egocentric views for action-object affordance reasoning.

Probabilistic modeling of scene-aware human motion generation also involves semantic reasoning of human interaction with the environment. Initial works on human motion synthesis have taken different architectural approaches, such as sequence-to-sequence models \cite{barsoum2018hp}, generative adversarial networks (GAN) \cite{barsoum2018hp, cai2018deep, yang2018pose}, graph convolutional networks (GCN) \cite{yan2019convolutional}, and variational autoencoders (VAE) \cite{guo2020action2motion}. However, these methods have mostly ignored the role of environmental semantics. Due to potential uncertainty in human motion, in a recent approach \cite{wang2021scene}, the authors address such motion synthesis with a GAN conditioned on scene attributes and motion trajectory to predict probable body pose dynamics.

One key challenge of human affordance generation in 2D scenes is the lack of large-scale datasets with rich pose annotations. In \cite{wang2017binge}, the authors compile the only public dataset of annotated human body poses in complex 2D indoor scenes by extracting frames from sitcom videos. Aiming to generate a contextually valid human affordance at a user-defined location, the authors propose sampling the scale and deformation parameters for an existing human pose template using a VAE conditioned on the localized image patches as scene context. In \cite{zhang2022inpaint}, the authors introduce a two-stage GAN architecture for achieving a similar goal by estimating the affine bounding box parameters to localize a probable human in the scene and then generating a potential body pose at that location. The method uses the input scene, corresponding depth, and segmentation maps as semantic guidance. In \cite{yao2023scene}, the authors propose a transformer-based approach with knowledge distillation for generating human affordances in 2D indoor scenes.


\section{Preliminaries}
\noindent \textbf{Autoregressive Language Modeling}.
Language provides a versatile way to represent tasks, data, inputs, and outputs, all as a sequence of tokens. Autoregressive language modeling is the basis for LLMs like GPT~\citep{gpt2,gpt3}. This approach predicts the probability of a sequence of words or tokens, with each prediction conditioned on the previous elements in the sequence. 

Formally, given a language token sequence $\vx = (x_1, x_2, \cdots, x_n)$, autoregressive language modeling decomposes the joint distribution of the sequence as the product of a series of conditional probabilities: $p(\vx) = \prod_{i=1}^n p(x_i|x_1, ..., x_{i-1})$,
where $p(x_1|x_0) = p(x_1)$ is the marginal probability. With the factorized distribution and a parameterized model (e.g., Transformers~\cite{vaswani2017attention}), the parameterized distribution $p_{\theta}(x)$ can be optimized via minimizing the negative log-likelihood loss:
\begin{equation}\label{eqn:loss-ar}
    \gL(\theta) = -\log p_{\theta}(\vx) = - \sum\limits_{i=1}^n \log p(x_i|x_{1},\cdots, x_{i-1}).
\end{equation}

\vspace{3pt}
\noindent \textbf{Query-based Data Analytic Tasks}. This paper focuses on query-based data analytic tasks, represented as $\{task, data, query, answer\}$. Given a task description, the data to be analyzed, and a natural language query, the goal is to predict the answer, i,e, $p(answer|task,$ $data, query)$. For example, we can format the input text for a table selection task as: "$\texttt{find tables},$ $\texttt{table schema},~$ $\texttt{who was the}$ $\texttt{only athlete...}$". Then, the model is expected to return the table name(s) that can answer the question, e.g., "$\texttt{<tables>Final\_1},$ $\texttt{Athletics\_1</tables>}$".

\vspace{3pt}

\noindent \textbf{Supervised Instruction Tuning} is a critical fine-tuning process employed to enhance the performance of LLMs on specific tasks by leveraging labeled datasets. Supervised instruction tuning focuses on adapting the model to follow explicit instructions and produce task-specific outputs. This process involves training the model on input-output pairs, where the inputs are typically natural language instructions or prompts, and the outputs are the desired responses or completions. The loss function of supervised instruction tuning is computed only on the "output" tokens to optimize the ability to execute specific tasks and understand instructions.

\section{Model Formulation for FlowAR Architecture}\label{sec:model_formulation_of_flowar}
In this section, we provide a mathematical definition for every module of FlowAR. 
Section~\ref{sec:sample_function} provides the definition of up-sample and down-sample functions.
In Section~\ref{sec:downsample_tokenizer}, we mathematically define the VAE tokenizer.  Section~\ref{sec:ar_transformer} presents a mathematical formulation for every module in the autoregressive transformer in FlowAR. Section~\ref{sec:flow_matching} provides some important definitions of the flow-matching architecture. In Section~\ref{sec:inference_of_flowar}, we also provide a rigorous mathematical definition for the overall architecture of the FlowAR Model during the inference process.



\subsection{Sample Function}\label{sec:sample_function}
We define the bicubic upsampling function.
\begin{definition}[Bicubic Upsampling Function]\label{def:bicubic_up_sample_function}
Given the following:
\begin{itemize}
    \item {\bf Input tensor:} $\X \in \R^{h \times w \times c}$ where $h,w,c$ represent height, width, and the number of channels, respectively.
    \item {\bf Scaling factor:} A positive integer $r \geq 1$.
    \item {\bf Bicubic kernel:} $W:\R \to [0,1]$
\end{itemize}
The bicubic upsampling function $\phi_{\mathrm{up}}(\X,r)$ computes an output tensor $\Y \in \R^{rh \times rw \times c}$. For every output position $i \in [rh], j \in [rw], l \in [c]$:
\begin{align*}
    \Y_{i,j,l} =  \sum_{s=-1}^2 \sum_{t=-1}^2 W(s) \cdot W(t) \cdot \X_{\lfloor \frac{i}{r}\rfloor+s, \lfloor \frac{j}{r}\rfloor+t,l}
\end{align*}

\end{definition}
Next, we define the downsampling function.
\begin{definition}[Linear Downsampling Function]\label{def:linear_down_sample_function}
Given the following:
\begin{itemize}
    \item {\bf Input tensor:} $\X \in \R^{h \times w \times c}$ where $h,w,c$ represent height, width, and the number of channels, respectively.
    \item {\bf Scaling factor:} A positive integer $r \geq 1$.
\end{itemize}
The linear downsampling function $\phi_{\mathrm{down}}(\X,r)$ computes an output tensor $\Y \in \R^{(h/r) \times (w/r) \times c}$. Let $\Phi_{\mathrm{down}} \in \R^{(h/r \cdot w/r) \times hw}$ denote a linear transformation matrix. Reshape $\X$ into the matrix $X \in \R^{hw \times c}$ by flattening its spatial dimensions.  The output matrix is defined via:
\begin{align*}
     Y = \Phi_{\mathrm{down}}X \in \R^{(h/r \cdot w/r) \times c},
\end{align*}
Then reshaped back to $\Y \in \R^{(h/r) \times(w/r) \times c}$.
\end{definition}


\subsection{Multi-Scale Downsampling Tokenizer}\label{sec:downsample_tokenizer}
Given an input image, the FlowAR model will utilize the VAE to generate latent representation $\X^{\R^{h \times w \times c}}$. To meet the requirements of Multi-Scale autoregressive image generation, FlowAR uses a Multi-Scale VAE Tokenizer to downsample $\X$ and generate Token Maps of different sizes.
\begin{definition}[Multi-Scale Downsampling Tokenizer]\label{def:downsample_tokenizer}
 Given the following:
 \begin{itemize}
     \item {\bf Latent representation tensor:} $\X\in \R^{h \times w \times c}$ generated by VAE.
     \item {\bf Number of scales:} $K \in \mathbb{N}$.
     \item {\bf Base scaling factor:}  positive integer $a \geq 1$
     \item {\bf Downsampling functions:} For $i \in [K]$, define scale-specific factors $r_i := a^{K-i}$ and use the linear downsampling function $\phi_{\mathrm{down}}(\X,r_i)$ from Definition~\ref{def:linear_down_sample_function}.
 \end{itemize}
 Then tokenizer outputs a sequence of token maps $\{\Y^2, \Y^2,\dots, \Y^K\}$, where the $i$-th token map is
 \begin{align*}
     \Y^i := \phi_{\mathrm{down},i}(\X,r_i) \in \R^{(h / r_i) \times (w/r_i) \times c},
 \end{align*}
 Formally, the tokenizer is defined as
 \begin{align*}
     \mathsf{TN}(\X) := \{\Y^{1}, \dots,\Y^{K}\}.
 \end{align*} 

\end{definition}
\begin{remark}
    In \cite{ryh+24}, the base factor is set to $a = 2$, resulting in exponentially increasing scales $r_i = 2^{K-i}$ for $i \in [K]$.
\end{remark}

\subsection{Autoregressive Transformer}\label{sec:ar_transformer}

The autoregressive transformer is a key module of the FlowAR model. We will introduce each layer of autoregressive transformer in this section.
\begin{definition}[Attention Layer]\label{def:attn_layer}
Given the following:
\begin{itemize}
    \item {\bf Input tensor:} $\X \in \R^{h \times w \times c}$ where $h,w,c$ represent height, width, and the number of channels, respectively.
    \item {\bf Weight matrices:} $W_Q,W_K,W_V \in \R^{c \times c}$ will be used in query, key, and value projection, respectively.
\end{itemize}
The attention layer $\mathsf{Attn}(\X)$ computes an output tensor $\Y \in \R^{h \times w \times c}$ as follows:
\begin{itemize}
    \item {\bf Reshape:} Flatten $\X$ into a matrix $X \in \R^{hw \times c}$ with spatial dimensions collapsed.
    \item {\bf Compute attention matrix:} For $i,j \in [hw]$, compute pairwise scores:
    \begin{align*}
        A_{i,j} := & ~\exp(  X_{i,*}   W_Q   W_K^\top   X_{j,*}^\top), \text{~~for~} i, j \in [hw].
    \end{align*}
    \item {\bf Normalization:} Compute diagnal matrix $D:=\diag(A {\bf 1}_n) \in \R^{hw \times hw}$, where ${\bf 1}_n$ is the all-ones vector. And compute:
    \begin{align*}
         Y := D^{-1}AXW_V \in \R^{hw \times c}.
    \end{align*}
    then reshape $Y$ to $\Y \in \R^{h \times w \times c}$.
\end{itemize}
\end{definition}


Then, we define the multiple-layer perception layer.
\begin{definition}[MLP layer]\label{def:mlp}
Given the following:
\begin{itemize}
    \item {\bf Input tensor:} $\X \in \R^{h \times w \times c}$ where $h,w,c$ represent height, width, and the number of channels, respectively.
    \item {\bf Weight matrices and bias vector:} $W \in \R^{c \times d}$ and $b \in \R^{1 \times d}$.
\end{itemize}
The MLP layer computes an output tensor $\Y \in \R^{h \times w \times d}$ as follows:
\begin{itemize}
    \item {\bf Reshape:} Flatten $\X$ into a matrix $X \in \R^{hw \times c}$ with spatial dimensions collapsed.
    \item {\bf Affine transformation:} For all $j \in [hw]$, compute
    \begin{align*}
        Y_{j,*} = \underbrace{X_{j,*}}_{1\times c} \cdot \underbrace{W}_{c \times d} + \underbrace{b}_{1 \times d}
    \end{align*}
    Then reshape $Y \in \R^{hw \times d}$ into $\Y \in \R^{h \times w \times d}$.
\end{itemize}
The operation is denoted as $\Y := \mathsf{MLP}(\X,c,d)$.
\end{definition}


Next, we introduce the definition of the feedforward layer.
\begin{definition}[Feed forward layer]\label{def:ffn}
Given the following:
\begin{itemize}
    \item {\bf Input tensor:} $\X \in \R^{h \times w \times c}$ where $h,w,c$ represent height, width, and the number of channels, respectively.
    \item {\bf Weight matrices and bias vector:} $W_1, W_2 \in \R^{c \times d}$ and $b_1, b_2 \in \R^{1 \times d}$.
    \item {\bf Activation:} $\sigma:\R \to \R$ denotes the $\mathsf{ReLU}$ activation function which is applied element-wise.
\end{itemize}
The feedforward layer computes an output tensor $\Y \in \R^{h \times w \times d}$ as follows:
\begin{itemize}
    \item {\bf Reshape:} Flatten $\X$ into a matrix $X \in \R^{hw \times c}$ with spatial dimensions collapsed.
    \item {\bf Transform:} For each $j \in [hw]$, compute 
    \begin{align*}
        Y_{j,*}=  \underbrace{X_{j,*}}_{1 \times c} +  \sigma (\underbrace{X_{j,*}}_{1\times c} \cdot \underbrace{W_1}_{c \times c} + \underbrace{b_1}_{1\times c}) \cdot \underbrace{W_2}_{c \times c} + \underbrace{b_2}_{1 \times c} \in \R^{1 \times c}
    \end{align*}
    where $\sigma$ acts element-wise on intermediate results. Then reshape $Y \in \R^{hw \times c}$ into $\Y \in \R^{h \times w \times c}$.
\end{itemize}
The operation is denoted as $\Y := \mathsf{FFN}(\X)$.

\end{definition}

To move on, we define the layer normalization layer.
\begin{definition}[Layer Normalization Layer]\label{def:ln}
    Given the following:
    \begin{itemize}
        \item {\bf Input tensor:} $\X \in \R^{h \times w \times c}$ where $h,w,c$ represent height, width, and the number of channels, respectively.
    \end{itemize}
    The layer normalization computes $\Y$ through
    \begin{itemize}
        \item {\bf Reshape:} Flatten $\X$ into a matrix $X \in \R^{hw \times c}$ with spatial dimensions collapsed.
        \item {\bf Normalize:} For each $j \in [hw]$, compute
        \begin{align*}
            Y_{j,*} =  \frac{X_{j,*}-\mu_j}{\sqrt{\sigma_j^2}}
        \end{align*}
        where
        \begin{align*}
            \mu_j := \sum_{k=1}^c X_{j,k}/c, ~~ \sigma_{j}^2 = \sum_{k=1}^c(X_{j,k}-\mu_j)^2/c
        \end{align*}
        Then reshape $Y \in \R^{hw \times c}$ into $\Y \in \R^{h \times w \times c}$.
    \end{itemize}
    The operation is denoted as $\Y := \mathsf{LN}(\X)$.
\end{definition}


Now, we can proceed to show the definition of the autoregressive transformer.
\begin{definition}[Autoregressive Transformer]\label{def:ar_transformer}
    Given the following:
    \begin{itemize}
        \item {\bf Input tensor:} $\X \in \R^{h \times w \times c}$ where $h,w,c$ represent height, width, and the number of channels, respectively.
        \item {\bf Scales number:} $K \in \mathbb{N}$ denote the number of total scales in FlowAR.
        \item {\bf Token maps:} For $i \in [K]$, $\Y_i \in \R^{(h/r_i) \times (w/r_i) \times c}$ generated by the Multi-Scale Downsampling Tokenizer defined in Definition~\ref{def:downsample_tokenizer} where $r_i = a^{K-i}$ with base $a \in \mathbb{N}^+$.
        \item {\bf Upsampling functions:}  For $i \in [K]$, $\phi_{\mathrm{up},i}(\cdot,a): \R^{(h/r_i) \times (w/r_i) \times c}\to \R^{(h/r_{i+1}) \times (w/r_{i+1}) \times c}$ from Definition~\ref{def:bicubic_up_sample_function}.
        \item {\bf Attention layer:}  For $i \in [K]$, $\mathsf{Attn}_i(\cdot):\R^{(\sum_{j=1}^i h/r_j \cdot w/r_{j})\times c} \to \R^{(\sum_{j=1}^i h/r_j \cdot w/r_{j})\times c}$ which acts on flattened sequences of dimension defined in Definition~\ref{def:attn_layer}.
        \item {\bf Feed forward layer: } For $i \in [K]$, $\mathsf{FFN}_i(\cdot): \R^{(\sum_{j=1}^i h/r_j \cdot w/r_{j})\times c} \to \R^{(\sum_{j=1}^i h/r_j \cdot w/r_{j})\times c}$ which acts on flattened sequences of dimension defined in Definition~\ref{def:ffn}.
        \item {\bf Initial condition:} $\Z_{\mathrm{init}} \in \R^{(h/r_1) \times (w/r_1) \times c}$ denotes the initial token maps which encodes class information.
    \end{itemize}
    Then, the autoregressive processing is:
    \begin{enumerate}
        \item {\bf Initialization: } Let $\Z_1:=\Z_{\mathrm{init}}$.
        \item {\bf Iterative sequence construction:} For $i \geq 2$.
        \begin{align*}
            \Z_i := \mathsf{Concat}(\mathsf{Z}_{\mathrm{init}}, \phi_{\mathrm{up}, 1}(\Y^1, a), \ldots, \phi_{\mathrm{up}, i-1}(\Y^{i-1}, a)) \in \R^{(\sum_{j=1}^i h/r_j \cdot w/r_{j})\times c}
        \end{align*}
        where $\mathsf{Concat}$ reshapes tokens into a unified spatial grid.
        \item {\bf Transformer block:} For $i \in [K]$,
        \begin{align*}
            \mathsf{TF}_i(\Z_i) := \mathsf{FFN_i}(\mathsf{Attn}_i(\Z_i)) \in \R^{(\sum_{j=1}^i h/r_j \cdot w/r_{j})\times c}
        \end{align*}
        \item {\bf Output decomposition:} Extract the last scale's dimension   from the reshaped $\mathsf{TF}_i(\Z_i)$ to generate $\wh{\Y}_i \in \R^{(h/r_i) \times (w/r_i) \times c}$.
    \end{enumerate}
\end{definition}


\subsection{Flow Matching}\label{sec:flow_matching}
We begin by outlining the concept of velocity flow in the flow-matching architecture.
\begin{definition}[Flow]\label{def:flow}
Given the following:
\begin{itemize}
    \item {\bf Input tensor:} $\X \in \R^{h \times w \times c}$ where $h,w,c$ represent height, width, and the number of channels, respectively.
    \item {\bf Scales number:} $K \in \mathbb{N}$.
    \item {\bf Noise tensor: } For $i \in [K]$, $\F_i^0 \in \R^{(h / r_i) \times (w/r_i) \times c}$ with every entry sampled from $\mathcal{N}(0,1)$.
    \item {\bf Token maps:} For $i \in [K]$, $\wh{\Y}_i \in \R^{(h / r_i) \times (w/r_i) \times c}$ denote the token maps generated by autoregressive transformer defined in Definition~\ref{def:ar_transformer}.
\end{itemize}
Then, the model does the following:
\begin{itemize}
    \item {\bf Interpolation:} For timestep $t \in [0,1]$ and scale $i$,
    \begin{align*}
        \F_i^t := t \wh{\Y}_i + (1-t) \F_i^0
    \end{align*}
    defining a linear trajectory between noise $\F_0^i$ and target tokens  $\wh{\Y}_i$.
    \item {\bf Velocity Field:} The time-derivative of the flow at scale $i$ is 
    \begin{align*}
        \V^t_i := \frac{\d \F^t_{i}}{\d t} = \wh{\Y_i} -\F^0_i.
    \end{align*}
    constant across $t$ due to linear interpolation.
\end{itemize}
\end{definition}

To move forward, we propose an approach to enhance the performance of the flow-matching layer by replacing linear interpolation with a Quadratic Bézier curve.

\begin{definition}[High Order Flow]\label{def:high_order_flow}
Given the following:
\begin{itemize}
    \item {\bf Input tensor:} $\X \in \R^{h \times w \times c}$ where $h,w,c$ represent height, width, and the number of channels, respectively.
    \item {\bf Scales number:} $K \in \mathbb{N}$.
    \item {\bf Noise tensor: } For $i \in [K]$, $\F_i^0 \in \R^{(h / r_i) \times (w/r_i) \times c}$ with every entry sampled from $\mathcal{N}(0,1)$.
    \item {\bf Token maps:} For $i \in [K]$, $\wh{\Y}_i \in \R^{(h / r_i) \times (w/r_i) \times c}$ denote the token maps generated by autoregressive transformer defined in Definition~\ref{def:ar_transformer}.
\end{itemize}
Then, the model does the following:
\begin{itemize}
    \item {\bf Interpolation:} For timestep $t \in [0,1]$ and scale $i$,
    \begin{align*}
        \F_i^t := (1-t)^2 \F_i^0 + 2t(1-t) \mathsf{C}_i + t^2 \wh{\Y}_i
    \end{align*}
    defining a quadratic Bézier curve as the interpolation path between the initial noise and the target data. To be noticed, we take $\mathsf{C} = \frac{\F_i^0+\wh{\Y}_i}{2}$ as a control point that governs the curvature of the trajectory. This formulation replaces the standard linear interpolation with a higher-order flow, enabling a smoother and more flexible transition from noise to data in the flow-matching framework.
    \item {\bf Velocity Field:} The time-derivative of the flow at scale $i$ is 
    \begin{align*}
        \V^t_i := &~\frac{\d \F^t_{i}}{\d t}\\ =&~ -2(1-t)\F_i^0 + 2(1-2t) \mathsf{C}_i + 2t \wh{\Y}_i 
    \end{align*}
    constant across $t$ due to linear interpolation.
\end{itemize}
\end{definition}

We are now able to define the flow-matching layer, which is integrated in the FlowAR model.
\begin{definition}[Flow Matching Architecture]\label{def:flow_matching_architecture}
Given the following:
\begin{itemize}
    \item {\bf Input tensor:} $\X \in \R^{h \times w \times c}$ where $h,w,c$ represent height, width, and the number of channels, respectively.
    \item {\bf Scales number:} $K \in \mathbb{N}$ denote the number of total scales in FlowAR.
    \item {\bf Token maps:} For $i \in [K]$, $\wh{\Y}_i \in \R^{(h / r_i) \times (w/r_i) \times c}$ denote the token maps generated by autoregressive transformer defined in Definition~\ref{def:ar_transformer}.
    \item {\bf Interpolation Tokens:} For $i \in [K]$, $\F_i^t \in \R^{(h / r_i) \times (w/r_i) \times c}$ denote interpolated input defined in Definition~\ref{def:flow}.
    \item {\bf Time step:} For $i \in [K]$, $t_i \in [0,1]$ denotes timestep.
    \item {\bf Attention layer:}  For $i \in [K]$, $\mathsf{Attn}_i(\cdot):\R^{h/r_i \times w/r_i \times c} \to \R^{h/r_i \times w/r_i \times c}$ is defined in Definition~\ref{def:attn_layer}.
    \item {\bf MLP layer:}  For $i \in [K]$, $\mathsf{MLP}_i(\cdot,c,d):\R^{h/r_i \times w/r_i \times c} \to \R^{h/r_i \times w/r_i \times c}$ is defined in Definition~\ref{def:mlp}.
    \item {\bf LN layer:} For $i \in [K]$, $\mathsf{LN}_i(\cdot):\R^{h/r_i \times w/r_i \times c} \to \R^{h/r_i \times w/r_i \times c}$ is defined in Definition~\ref{def:ln}.
\end{itemize}
The computation steps of flow-matching layers are as follows:
\begin{itemize}
    \item {\bf Time-conditioned parameter generation:}
    \begin{align*}
        \alpha_1, \alpha_2, \beta_1, \beta_2, \gamma_1, \gamma_2:=  \mathsf{MLP}_i(\wh{\Y}_i + t_i \cdot {\bf 1}_{(h / r_i) \times (w/r_i) \times c},c,6c)
    \end{align*}
    \item {\bf Intermediate variable computation:}
    \begin{align*}
        \F'^{t_i}_i:= \mathsf{Attn}_i (\gamma_1 \circ \mathsf{LN}(\F_i^{t_i}) + \beta_1) \circ \alpha_1
    \end{align*}
    with $\circ$ denoting Hadamard (element-wise) product.
    \item {\bf Final projection:}
    \begin{align*}
        \F''^{t_i}_i := \mathsf{MLP}_i(\gamma_2 \circ \mathsf{LN}(\F'^{t_i}_i)+ \beta_2,c,c) \circ \alpha_2
    \end{align*}
\end{itemize}
The operation is denoted as $\F''^{t_i}_i := \mathsf{NN}_i(\wh{\Y_i},\F_i^{t_i},t_i)$
\end{definition}


\subsection{Inference of FlowAR Architecture}\label{sec:inference_of_flowar}
The inference phase of the FlowAR model differs from the training phase. During inference, neither the VAE nor the Multi-Scale Downsampling layers are used. Instead, given an initial token map representing class embeddings, the model autoregressively generates token maps across scales.
\begin{definition}[FlowAR Inference Architecture]\label{def:flow_architecture_inference}
    Given the following:
    \begin{itemize}
        \item {\bf Scales number:} $K \in \mathbb{N}$ denote the number of total scales in FlowAR.
        \item {\bf Scale factor:} For $i \in [K]$, $r_i:= a^{K-i}$ where base factor $a \in \mathbb{N}^+$.
        \item {\bf Upsampling functions:}  For $i \in [K]$, $\phi_{\mathrm{up},i}(\cdot,a): \R^{(h/r_i) \times (w/r_i) \times c}\to \R^{(h/r_{i+1}) \times (w/r_{i+1}) \times c}$ from Definition~\ref{def:bicubic_up_sample_function}.
        \item {\bf Attention layer:}  For $i \in [K]$, $\mathsf{Attn}_i(\cdot):\R^{(\sum_{j=1}^i h/r_j \cdot w/r_{j})\times c} \to \R^{(\sum_{j=1}^i h/r_j \cdot w/r_{j})\times c}$ which acts on flattened sequences of dimension defined in Definition~\ref{def:attn_layer}.
        \item {\bf Feed forward layer: } For $i \in [K]$, $\mathsf{FFN}_i(\cdot): \R^{(\sum_{j=1}^i h/r_j \cdot w/r_{j})\times c} \to \R^{(\sum_{j=1}^i h/r_j \cdot w/r_{j})\times c}$ which acts on flattened sequences of dimension defined in Definition~\ref{def:ffn}.
        \item {\bf Flow matching layer:} For $i \in [K]$, $\mathsf{NN}_i(\cdot,\cdot,\cdot):\R^{h/r_i \times w/r_i \times c}\times \R^{h/r_i \times w/r_i \times c}\times \R \to \R^{h/r_i \times w/r_i \times c}$ denote the flow-matching layer defined in Definition~\ref{def:flow_matching_architecture}.
        \item {\bf Initial condition:} $\Z_{\mathrm{init}} \in \R^{(h/r_1) \times (w/r_1) \times c}$ denotes the initial token maps which encodes class information.
        \item {\bf Time steps:} For $i \in [K]$, $t_i \in [0,1]$ denotes time steps.
        \item {\bf Interpolated inputs:} For $i \in [K]$, $\F_i^{t_i} \in \R^{h/r_i \times w/r_i \times c}$ defined in Definition~\ref{def:flow}.
        \item {\bf Cumulative dimensions:} We define $\wt{h}_i := \sum_{j=1}^i h/r_j$ and  $\wt{w}_i := \sum_{j=1}^i w/r_j$ for $i \in [K]$.
    \end{itemize}
    The FlowAR model conducts the following recursive construction:
    \begin{itemize}
        \item {\bf Base case $i=1$:}
        \begin{align*}
            &~\Z_1 = \Z_{\mathrm{init}}\\
            &~\wh{\Y}_1 = \mathsf{FFN}_1(\mathsf{Attn}_1(\Z_1))\\
            &~\wt{\Y}_1 = \mathsf{NN}_1(\wh{\Y}_1,\F_{1}^{t_1},t_1)
        \end{align*}
        \item {\bf Inductive step $i \geq 2$:}
        \begin{itemize}
            \item {\bf Spatial aggregation:}
            \begin{align*}
                \Z_i = \mathsf{Concat}(\Z_{\mathrm{init}},\phi_{\mathrm{up},1}(\wt{\Y}_{i-1}),\dots,\phi_{\mathrm{up},i-1}(\wt{\Y}_{i-1})) \in \R^{(\sum_{j=1}^i h/r_j \cdot w/r_j)\times c}
            \end{align*}
            \item {\bf Autoregressive transformer computation:}
            \begin{align*}
                \wh{\Y}_i = \mathsf{FFN}_i(\mathsf{Attn}_i(\Z_1))_{\wt{h}_{i-1}:\wt{h}_{i-1},\wt{w}_{i}:\wt{w}_{i},0:c}
            \end{align*}
            \item {\bf Flow matching layer:}
            \begin{align*}
                \wt{\Y}_i = \mathsf{NN}_i(\wh{\Y}_i,\F_{i}^{t_i},t_i)
            \end{align*}
        \end{itemize}
        The final output is $\wt{\Y}_K \in \R^{h \times w \times c}$.
    \end{itemize}
   
\end{definition}


\section{Complexity of FlowAR Architecture}\label{sec:main_result}
This section presents key results on the circuit complexity of fundamental modules in the FlowAR architecture. Section~\ref{lem:matrix_multi} analyzes matrix multiplication, while Section~\ref{sec:down_up_tc0} examines the up-sampling and down-sampling functions. In Sections~\ref{sec:mlp_tc0} and \ref{sec:ffn_tc0}, we compute the circuit complexity of the MLP and FFN layers, respectively. Sections~\ref{sec:attention_tc0} and \ref{sec:ln_tc0} focus on the single attention layer and layer normalization. Section~\ref{sec:flow_matching} addresses the flow-matching layer. Finally, Section~\ref{sec:main_result} presents our main result, establishing the circuit complexity bound for the complete FlowAR architecture.
\subsection{Computing Matrix Products in \texorpdfstring{$\TC^0$}{}}\label{sec:compute_matrix_product}
we demonstrate that matrix multiplication is computable in $\mathsf{TC}^0$, which will be used later.
\begin{lemma}[Matrix multiplication in $\TC^0$, \cite{cll+24}]\label{lem:matrix_multi}
    Let the precision $p \leq \poly(n)$. Let $X \in \mathsf{F}_p^{n_1 \times d}, Y \in \mathsf{F}_p^{d \times n_2}$ be matrices. Assume $n_1\leq\poly(n), n_2\leq\poly(n)$. The matrix product $XY$ can be computed by a uniform $\mathsf{TC}^0$ circuit with:
    \begin{itemize}
        \item Size: $\poly(n)$.
        \item Depth: $d_{\mathrm{std}}+d_{\oplus}$.
    \end{itemize}
    where $d_{\mathrm{std}}$ and $d_{\oplus}$ are defined in Definition~\ref{lem:float_operations_TC}.
\end{lemma}


\subsection{Computing Down-Sampling and Up-Sampling in in \texorpdfstring{$\TC^0$}{}}\label{sec:down_up_tc0}
In this section, we show that Up-Sampling can be efficiently computable by a uniform $\mathsf{TC}^0$ circuit.
\begin{lemma}[Up-Sampling computation in $\mathsf{TC}^0$]\label{lem:up_tc0}
     Let $\X \in \R^{h \times w \times c}$ be the input tensor. Let $\phi_{\mathrm{up}}(X,r):\R^{h \times w \times c} \to \R^{(hr) \times (wr) \times c}$ denote the bicubic up sample function defined in Definition~\ref{def:bicubic_up_sample_function}. Assume $n = h = w$. Assume $r \leq n$. Assume $c \leq n$. Assume $p \leq \poly(n)$.
    The linear up sample function can be computed by a uniform $\mathsf{TC}^0$ circuit with:
    \begin{itemize}
        \item Size: $\poly(n)$.
        \item Depth: $2d_\mathrm{std} + d_{\oplus}$.
    \end{itemize}
    where $d_{\mathrm{std}}$ and $d_{\oplus}$ are defined in Definition~\ref{lem:float_operations_TC}.
\end{lemma}
\begin{proof}
    For each $i \in [nr], j \in [nr], l \in [c]$, we need to compute $\phi_{\mathrm{up}}(\X,r)_{i,j,l} = \sum_{s=-1}^2 \sum_{t=-1}^2 W(s)\cdot W(t)\cdot \X_{\frac{i}{r}+s,\frac{j}{r}+s,l}$.  We need a $2d_{\mathrm{std}}$ depth and $\poly(n)$ size circuit to compute $W(s)\cdot W(t)\cdot \X_{\frac{i}{r}+s,\frac{j}{r}+s,l}$ by Part 1 of Lemma~\ref{lem:float_operations_TC} and for all $s,t \in \{-1,0,1,2\}$, these terms can be computed in parallel. Furthermore, by Part 3 of Lemma~\ref{lem:float_operations_TC}, we can need a $d_{\oplus}$ depth and $\poly(n)$ size circuit to compute $\sum_{s=-1}^2 \sum_{t=-1}^2 W(s)\cdot W(t)\cdot \X_{\frac{i}{r}+s,\frac{j}{r}+s,l}$. Since the computation of $\phi_{\mathrm{up}}(\X,r)_{i,j,l}$ needs a $2d_{\mathrm{std}}+d_{\oplus}$ depth and $\poly(n)$ size circuit.

    Since for all $i \in [nr], j \in [nr], l \in [c]$, we can compute $\phi_{\mathrm{up}}(\X,r)_{i,j,l}$ in parallel, then the total depth of the circuit is $2d_{\mathrm{std}} + d_{\oplus}$ and size remains $\poly(n)$.
\end{proof}

Then, we move forward to consider the down-sampling function.
\begin{lemma}[Down-Sampling computation in $\mathsf{TC}^0$]\label{lem:down_tc0}
    Let $\X \in \R^{h \times w \times c}$ be the input tensor. Let $\phi_{\mathrm{down}}(X,r)$ be the linear down sample function from Definition~\ref{def:linear_down_sample_function}. Assume $n = h = w$. Assume $r \leq n$. Assume $c \leq n$. Assume $p \leq \poly(n)$.

    The function $\phi_{\mathrm{down}}$ can be computed by a uniform $\mathsf{TC}^0$ circuit with

    \begin{itemize}
        \item Size: $\poly(n)$.
        \item Depth: $d_\mathrm{std} + d_{\oplus}$.
    \end{itemize}
    where $d_{\mathrm{std}}$ and $d_{\oplus}$ are defined in Definition~\ref{lem:float_operations_TC}.
\end{lemma}
\begin{proof}
    By Definition~\ref{def:linear_down_sample_function}, we know that down-sampling computation is essentially matrix multiplication. Then, by Lemma~\ref{lem:matrix_multi}, we can easily get the proof.
\end{proof}







\subsection{Computing Multiple-layer Perceptron in \texorpdfstring{$\TC^0$}{} }\label{sec:mlp_tc0}
 We prove that MLP computation can be efficiently simulated by a uniform $\mathsf{TC}^0$ circuit.
\begin{lemma}[MLP computation in $\mathsf{TC}^0$, informal version of Lemma~\ref{lem:mlp_tc0_formal}]\label{lem:mlp_tc0_informal}
    Given an input tensor $\X\in \R^{h \times w \times c}$. Let $\mathsf{MLP}(\X,c,d)$ be the MLP layer defined in Definition~\ref{def:mlp}. Under the following constraints:
    \begin{itemize}
        \item Satisfy $h = w = n$,
        \item Channel bounds: $c, d\leq n$,
        \item Precision: $p \leq \poly(n)$,
    \end{itemize}
    The $\mathsf{MLP}(\X,c,d)$ function can be computed by a uniform $\mathsf{TC}^0$ circuit with:
    \begin{itemize}
        \item Size: $\poly(n)$.
        \item Depth: $2d_\mathrm{std} + d_{\oplus}$.
    \end{itemize}
    with $d_{\mathrm{std}}$ and $d_{\oplus}$ defined in Definition~\ref{lem:float_operations_TC}.
\end{lemma}




\subsection{Computing Feed-Forward Layer in \texorpdfstring{$\TC^0$}{} }\label{sec:ffn_tc0}
We also prove that feed-forward network computation can be simulated by a uniform $\mathsf{TC}^0$ circuit.
\begin{lemma}[FFN computation in $\mathsf{TC}^0$, informal version of Lemma~\ref{lem:ffn_tc0_formal}]\label{lem:ffn_tc0_informal}
 Given an input tensor $\X\in \R^{h \times w \times c}$. Let $\mathsf{FFN}(X):\R^{h \times w \times c} \to \R^{h \times w \times c}$ as defined in Definition~\ref{def:ffn}. Under the following constraints:
 \begin{itemize}
     \item Satisfy $h = w = n$,
         \item Channel bound: $c \leq n$,
         \item Precision bound: $p \leq \poly(n)$.
 \end{itemize}
The $\mathsf{FFN}(\X)$ layer can be computed by a uniform $\mathsf{TC}^0$ circuit with:
\begin{itemize}
    \item Size: $\poly(n)$.
    \item Depth: $6d_\mathrm{std} + 2d_{\oplus}$.
\end{itemize}
with $d_{\mathrm{std}}$ and $d_{\oplus}$ defined in Definition~\ref{lem:float_operations_TC}.
\end{lemma}




\subsection{Computing Single Attention Layer in \texorpdfstring{$\TC^0$}{}.} \label{sec:attention_tc0}
 We prove the single attention layer can be efficiently simulated by a uniform $\mathsf{TC}^0$ circuit.
\begin{lemma}[Attention layer computation in $\mathsf{TC}^0$, informal version of Lemma~\ref{lem:attn_tc0_formal}]\label{lem:attn_tc0_informal}
     Given an input tensor $\X \in \R^{h \times w \times c}$. Let $\mathsf{Attn}(X):\R^{h \times w \times c} \to \R^{h \times w \times c}$ as defined in Definition~\ref{def:attn_layer}. Under the following constraints:
     \begin{itemize}
         \item Satisfy $h = w = n$,
         \item Channel bound: $c \leq n$,
         \item Precision bound: $p \leq \poly(n)$.
     \end{itemize}
     The $\mathsf{Attn}(\X)$ layer can be computed by a uniform $\mathsf{TC}^0$ circuit with:
     \begin{itemize}
        \item Size: $\poly(n)$.
        \item Depth: $6(d_{\mathrm{std}} + d_{\oplus}) + d_{\exp}$.
     \end{itemize}    
     with $d_{\mathrm{std}}$ and $d_{\oplus}$ defined in Definition~\ref{lem:float_operations_TC}, $d_{\exp}$ defined in Definition~\ref{lem:exp}.
\end{lemma}



\subsection{Computing Layer-wise Norm Layer in \texorpdfstring{$\TC^0$}{}.}\label{sec:ln_tc0}
We prove that the layer normalization layer can be efficiently simulated by a uniform $\mathsf{TC}^0$ circuit.
\begin{lemma}[Layer normalization layer computation in $\TC^0$, informal version of Lemma~\ref{lem:ln_tc0_formal}]\label{lem:ln_tc0_informal}
    Given an input tensor $\X \in \R^{h \times w \times c}$. Let $\mathsf{LN}(X):\R^{h \times w \times c} \to \R^{h \times w \times c}$ as defined in Definition~\ref{def:ln}. Under the following constraints:
    \begin{itemize}
         \item Satisfy $h = w = n$,
         \item Channel bound: $c \leq n$,
         \item Precision bound: $p \leq \poly(n)$.
     \end{itemize}
     The $\mathsf{LN}(\X)$ layer can be computed by a uniform $\mathsf{TC}^0$ circuit with:
     \begin{itemize}
        \item Size: $\poly(n)$.
        \item Depth: $5d_\mathrm{std} + 2d_{\oplus} + d_\mathrm{sqrt}$.
     \end{itemize}    
     with $d_{\mathrm{std}}$ and $d_{\oplus}$ defined in Definition~\ref{lem:float_operations_TC}, $d_{\mathrm{sqrt}}$ defined in Definition~\ref{lem:sqrt}.
\end{lemma}


\subsection{Computing Flow Matching Layer in \texorpdfstring{$\TC^0$}{}.}\label{sec:fl_tc0}
We prove that the flow-matching layer can be efficiently simulated by a uniform $\mathsf{TC}^0$ circuit.
\begin{lemma}[Flow matching layer computation in $\TC^0$]\label{lem:fm_tc0}
     Given an input tensor $\X \in \R^{h \times w \times c}$. Let $\mathsf{NN}(X)$ denote the flow-matching layer defined in Definition~\ref{def:flow_matching_architecture}. Under the following constraints:
     \begin{itemize}
         \item Satisfy $h = w = n$,
         \item Channel bound: $c \leq n$,
         \item Precision bound: $p \leq \poly(n)$.
     \end{itemize}
     The $\mathsf{NN}(\cdot, \cdot,\cdot)$ can be computed by a uniform $\mathsf{TC}^0$ circuit with
     \begin{itemize}
        \item Size: $\poly(n)$.
        \item Depth: $26d_{\mathrm{std}}+ 12 d_{\oplus} + 2d_{\mathrm{sqrt}} + d_{\exp}$.
     \end{itemize} 
    with $d_{\mathrm{std}}$ and $d_{\oplus}$ defined in Definition~\ref{lem:float_operations_TC}, $d_{\exp}$ defined in Definition~\ref{lem:exp} and $d_{\mathrm{sqrt}}$ defined in Definition~\ref{lem:sqrt}.
\end{lemma}

\begin{proof}
    {\bf Considering Step 1 in the flow-matching layer:}
    By Lemma~\ref{lem:mlp_tc0_formal}, parameters $\alpha_1,\alpha_2,\beta_1,\beta_2,\gamma_1,\gamma_2$ are computed via a circuit with 
    \begin{itemize}
        \item {\bf Depth}: $2d_{\mathrm{std}} + d_\oplus$.
        \item {\bf Size}: $\poly(n)$
    \end{itemize}


    
    {\bf Considering Step 2 in flow-matching layer:} 
     By Lemma~\ref{lem:ln_tc0_formal}, $\mathsf{LN}(\F_i^t)$ requires depth $5d_\mathrm{std} + 2d_{\oplus} + d_\mathrm{sqrt}$. By Lemma~\ref{lem:float_operations_TC}, $A_1 = \gamma_1 \circ \mathsf{LN}(\F_t)+\beta_1$ requires depth $2d_{\mathrm{std}}$. By Lemma~\ref{lem:attn_tc0_formal}, $A_2 = \mathsf{Attn}(A_1)$ requires depth $6(d_{\mathrm{std}}+d_\oplus)+d_{\exp}$. By Lemma~\ref{lem:float_operations_TC} again, scaling $A_2 \circ \alpha_1$ requires depth $d_{\mathrm{std}}$. The total depth requires $14d_{\mathrm{std}} + 8d_{\oplus}+d_{\mathrm{sqrt}}+d_{\exp}$ for step 2.
    

    {\bf Considering Step 3 in flow-matching layer:} By Lemma~\ref{lem:ln_tc0_formal}, $\mathsf{LN}(\F'^t_i)$ requires depth $5d_\mathrm{std} + 2d_{\oplus} + d_\mathrm{sqrt}$. By Lemma~\ref{lem:float_operations_TC}, $A_3 = \gamma_2 \circ \mathsf{LN}(\wh{\F}_t)+\beta_2$ requries depth  $2d_{\mathrm{std}}$. By Lemma~\ref{lem:mlp_tc0_formal}, $A_4 = \mathsf{MLP}(A_3,c,c)$ requires depth $2d_{\mathrm{std}} + d_\oplus$.
    By Lemma~\ref{lem:float_operations_TC} again, $A_4 \circ \alpha_2$ requires depth $d_{\mathrm{std}}$. The total depth requires $10d_{\mathrm{std}}+3d_{\oplus}+d_{\mathrm{sqrt}}$  for step 3..

    Finally, combining the result above, we need a circuit with depth $26d_{\mathrm{std}}+ 12 d_{\oplus} + 2d_{\mathrm{sqrt}} + d_{\exp}$ and size $\poly(n)$ to simulate the flow-matching layer.
    
\end{proof}


\subsection{Circuit Complexity Bound for FlowAR Architecture}\label{sec:main_result_flowar}
We present that the FlowAR Model can be efficiently simulated by a uniform $\mathsf{TC}^0$ circuit.
\begin{theorem}[FlowAR Model computation in $\TC^0$]\label{thm:flowar_tc0}
Given an input tensor $\X \in \R^{h \times w \times c}$. Under the following constraints:
\begin{itemize}
    \item Satisfy $h = w = n$,
    \item Channel bound: $c \leq n$,
    \item Precision bound: $p \leq \poly(n)$.
    \item Number of scales: $K = O(1)$,
    \item $d_{\mathrm{std}},d_\oplus,d_{\mathrm{sqrt}},d_{\exp} = O(1)$.
\end{itemize}
Then, the FlowAR Model can be simulated by a uniform $\TC^0$ circuit family.

\end{theorem}
\begin{proof}
    For every $i \in [K]$, by Lemma~\ref{lem:up_tc0}, Lemma~\ref{lem:down_tc0},  Lemma~\ref{lem:attn_tc0_informal}, Lemma~\ref{lem:ffn_tc0_informal} and Lemma~\ref{lem:fm_tc0}, we can simulate the $i$-th layer of FlowAR Model with a uniform $\TC^0$ circuit whose size is $\poly(n)$ and depth is $O(1)$. Since the total number of layers $K = O(1)$, the composition of all $K$ circuits yields a single uniform $\mathsf{TC}^0$ circuit with
    \begin{itemize}
        \item {\bf Size:} $\poly(n)$,
        \item {\bf Depth: }$O(1)$.
    \end{itemize}
\end{proof}

In Theorem~\ref{thm:flowar_tc0}, we establish that a FlowAR model with $\poly(n)$ precision, constant depth, and $\poly(n)$ size can be efficiently simulated by a $\mathsf{DLOGTIME}$-uniform $\TC^0$ circuit family. This indicates that while the flow-matching architecture enhances the capability of visual autoregressive models, the FlowAR architecture remains inherently limited in expressivity under circuit complexity theory.
\section{Provably Efficient Criteria}\label{sec:efficient_critieria}

\subsection{Approximate Attention Computation}\label{sec:fast_attn}
In this section, we introduce approximate attention computation, which can accelerate the computation of the attention layer.

\begin{definition}[Approximate Attention Computation $\mathsf{AAttC}(n, d, R, \delta)$, Definition 1.2 in \cite{as23}]\label{def:aattc}
    Given an input sequence $X \in \R^{n \times d}$ and an approximation tolerance $\delta > 0$. Let $Q,K,V \in \R^{n \times d}$ be weigh matrices bounded such that
    \begin{align*}
        \max\{\|Q\|_\infty,\|K\|_\infty,\|V\|_\infty\} \leq R
    \end{align*}
    The {\bf Approximate Attention Computation} $\mathsf{AAttC}(n, d, R, \delta)$ outputs a matrix $N \in \R^{n \times d }$ satisfying:
    \begin{align*}
        \| N - \mathsf{Attn}(X)\|_\infty \leq \delta
    \end{align*}
\end{definition}

Next, we present a lemma that demonstrates the computational time cost of the AATTC method.
\begin{lemma}[Fast Attention via Subquadratic Computation, Theorem 1.4 of \cite{as23}]\label{lem:as23_attention}
Let $\mathsf{AAttC}$ be formalized as in Definition~\ref{def:aattc}. 
For parameter configurations:
\begin{itemize}
    \item Embedding dimension $d = O(\log n)$,
    \item $R = \Theta(\sqrt{\log n})$,
    \item Approximation tolerance $\delta = 1/\poly(n)$,
\end{itemize}
the $\mathsf{AAttC}$ computation satisfies
\begin{align*}
    \mathcal{T}(n, n^{o(1)}, d) = n^{1 + o(1)},
\end{align*}
where $\mathcal{T}$ denotes the time complexity under these constraints.

\end{lemma}


\subsection{Fast FlowAR Architecture in the Inference Pipleline}\label{sec:fast_flowar}
Firstly, we define the fast flow-matching layer, where the $\mathsf{Attn}$ layers in the original flow-matching module are replaced with $\mathsf{AAttC}$ layers.


\begin{definition}[Fast Flow Matching Architecture]\label{def:fast_flow_matching_architecture}
Given the following:
\begin{itemize}
    \item {\bf Input tensor:} $\X \in \R^{h \times w \times c}$.
    \item {\bf Scales number:} $K \in \mathbb{N}$.
    \item {\bf Token maps:} For $i \in [K]$, $\wh{\Y}_i \in \R^{(h / r_i) \times (w/r_i) \times c}$ denote the token maps generated by autoregressive transformer defined in Definition~\ref{def:ar_transformer}.
    \item {\bf Interpolation Tokens:} For $i \in [K]$, $\F_i^t \in \R^{(h / r_i) \times (w/r_i) \times c}$ denote interpolated input defined in Definition~\ref{def:flow}.
    \item {\bf Time step:} For $i \in [K]$, $t_i \in [0,1]$ denotes timestep.
    \item {\bf Approximate Attention layer:}  For $i \in [K]$, $\mathsf{AAttC}_i(\cdot):\R^{h/r_i \times w/r_i \times c} \to \R^{h/r_i \times w/r_i \times c}$ is defined in Definition~\ref{def:attn_layer}.
    \item {\bf MLP layer:}  For $i \in [K]$, $\mathsf{MLP}_i(\cdot,c,d):\R^{h/r_i \times w/r_i \times c} \to \R^{h/r_i \times w/r_i \times c}$ is defined in Definition~\ref{def:mlp}.
    \item {\bf LN layer:} For $i \in [K]$, $\mathsf{LN}_i(\cdot):\R^{h/r_i \times w/r_i \times c} \to \R^{h/r_i \times w/r_i \times c}$ is defined in Definition~\ref{def:ln}.
\end{itemize}
The computation steps of flow-matching layers are as follows:
\begin{itemize}
    \item {\bf Time-conditioned parameter generation:}
    \begin{align*}
        \alpha_1, \alpha_2, \beta_1, \beta_2, \gamma_1, \gamma_2:=  \mathsf{MLP}_i(\wh{\Y}_i + t_i \cdot {\bf 1}_{(h / r_i) \times (w/r_i) \times c},c,6c)
    \end{align*}
    \item {\bf Intermediate variable computation:}
    \begin{align*}
        \F'^{t_i}_i:= \mathsf{AAttC}_i (\gamma_1 \circ \mathsf{LN}(\F_i^{t_i}) + \beta_1) \circ \alpha_1
    \end{align*}
    with $\circ$ denoting Hadamard (element-wise) product.
    \item {\bf Final projection:}
    \begin{align*}
        \F''^{t_i}_i := \mathsf{MLP}_i(\gamma_2 \circ \mathsf{LN}(\F'^{t_i}_i)+ \beta_2,c,c) \circ \alpha_2
    \end{align*}
\end{itemize}
The operation is denoted as $\F''^{t_i}_i := \mathsf{FNN}_i(\wh{\Y_i},\F_i^{t_i},t_i)$
\end{definition}


Next, we define the Fast FlowAR inference pipeline architecture, where all 
$\mathsf{Attn}$ layers in the original FlowAR architecture are replaced with $\mathsf{AAttC}$ layers.




\begin{definition}[Fast FlowAR Inference Architecture]\label{def:fast_flow_architecture_inference}
    Given the following:
    \begin{itemize}
        \item {\bf Scales number:} $K \in \mathbb{N}$.
        \item {\bf Scale factor:} For $i \in [K]$, $r_i:= a^{K-i}$ where base factor $a \in \mathbb{N}^+$.
        \item {\bf Upsampling functions:}  For $i \in [K]$, $\phi_{\mathrm{up},i}(\cdot,a): \R^{(h/r_i) \times (w/r_i) \times c}\to \R^{(h/r_{i+1}) \times (w/r_{i+1}) \times c}$ from Definition~\ref{def:bicubic_up_sample_function}.
        \item {\bf Approximate Attention layer:}  For $i \in [K]$, $\mathsf{AAttC}_i(\cdot):\R^{(\sum_{j=1}^i h/r_j \cdot w/r_{j})\times c} \to \R^{(\sum_{j=1}^i h/r_j \cdot w/r_{j})\times c}$ which acts on flattened sequences of dimension defined in Definition~\ref{def:aattc}.
        \item {\bf Feed forward layer: } For $i \in [K]$, $\mathsf{FFN}_i(\cdot): \R^{(\sum_{j=1}^i h/r_j \cdot w/r_{j})\times c} \to \R^{(\sum_{j=1}^i h/r_j \cdot w/r_{j})\times c}$ which acts on flattened sequences of dimension defined in Definition~\ref{def:ffn}.
        \item {\bf Fast flow-matching layer:} For $i \in [K]$, $\mathsf{FNN}_i(\cdot,\cdot,\cdot):\R^{h/r_i \times w/r_i \times c}\times \R^{h/r_i \times w/r_i \times c}\times \R \to \R^{h/r_i \times w/r_i \times c}$ denote the fast flow-matching layer defined in Definition~\ref{def:fast_flow_matching_architecture}.
        \item {\bf Initial condition:} $\Z_{\mathrm{init}} \in \R^{(h/r_1) \times (w/r_1) \times c}$ denotes the initial token maps which encodes class information.
        \item {\bf Time steps:} For $i \in [K]$, $t_i \in [0,1]$ denotes time steps.
        \item {\bf Interpolated inputs:} For $i \in [K]$, $\F_i^{t_i} \in \R^{h/r_i \times w/r_i \times c}$ defined in Definition~\ref{def:flow}.
        \item {\bf Cumulative dimensions:} We define $\wt{h}_i := \sum_{j=1}^i h/r_j$ and  $\wt{w}_i := \sum_{j=1}^i w/r_j$ for $i \in [K]$.
    \end{itemize}
    The FlowAR model conducts the following recursive construction:
    \begin{itemize}
        \item {\bf Base case $i=1$:}
        \begin{align*}
            &~\Z_1 = \Z_{\mathrm{init}}\\
            &~\wh{\Y}_1 = \mathsf{FFN}_1(\mathsf{AAttC}_1(\Z_1))\\
            &~\wt{\Y}_1 = \mathsf{FNN}_1(\wh{\Y}_1,\F_{1}^{t_1},t_1)
        \end{align*}
        \item {\bf Inductive step $i \geq 2$:}
        \begin{itemize}
            \item {\bf Spatial aggregation:}
            \begin{align*}
                \Z_i = \mathsf{Concat}(\Z_{\mathrm{init}},\phi_{\mathrm{up},1}(\wt{\Y}_{i-1}),\dots,\phi_{\mathrm{up},i-1}(\wt{\Y}_{i-1})) \in \R^{(\sum_{j=1}^i h/r_j \cdot w/r_j)\times c}
            \end{align*}
            \item {\bf Autoregressive transformer computation:}
            \begin{align*}
                \wh{\Y}_i = \mathsf{FFN}_i(\mathsf{AAttC}_i(\Z_1))_{\wt{h}_{i-1}:\wt{h}_{i-1},\wt{w}_{i}:\wt{w}_{i},0:c}
            \end{align*}
            \item {\bf Flow matching layer:}
            \begin{align*}
                \wt{\Y}_i = \mathsf{FNN}_i(\wh{\Y}_i,\F_{i}^{t_i},t_i)
            \end{align*}
        \end{itemize}
        The final output is $\wt{\Y}_K \in \R^{h \times w \times c}$.
    \end{itemize}
   
\end{definition}

\subsection{Running Time}\label{sec:running_time}
In this section, we analyzed the running time required by the original FlowAR architecture and the running time required by the Fast FlowAR architecture. The results indicate that by adopting the Approximate Attention computation module, we can accelerate the running time of FlowAR to almost quadratic time.

First, we present the results of the running time analysis for the original FlowAR model.
\begin{lemma}[Inference Runtime of Original FlowAR Architecture, informal version of Lemma~\ref{lem:runtime_old_flowar_formal}]\label{lem:runtime_old_flowar_informal}
    Consider the original FlowAR inference pipeline with the following parameters:
    \begin{itemize}
        \item {\bf Input tensor:} $\X \in \R^{h \times w \times c}$. Assume $h=w=n$ and $c = O(\log n)$.
        \item {\bf Number of scales:} $K = O(1)$.
        \item {\bf Scale factor:} For $i \in [K]$, $r_i:= a^{K-i}$ where base factor $a \in \mathbb{N}^+$.
        \item {\bf Upsampling functions}  For $i \in [K]$, $\phi_{\mathrm{up},i}(\cdot,a)$ from Definition~\ref{def:bicubic_up_sample_function}.
        \item {\bf Attention layer:}  For $i \in [K]$, $\mathsf{Attn}_i(\cdot)$ which acts on flattened sequences of dimension defined in Definition~\ref{def:attn_layer}.
        \item {\bf Feed forward layer: } For $i \in [K]$, $\mathsf{FFN}_i(\cdot)$ which acts on flattened sequences of dimension defined in Definition~\ref{def:ffn}.
        \item {\bf Flow matching layer:} For $i \in [K]$, $\mathsf{NN}_i(\cdot,\cdot,\cdot)$ denote the flow-matching layer defined in Definition~\ref{def:flow_matching_architecture}.
    \end{itemize}
    Under these conditions, the total inference runtime of FlowAR is bounded by $O(n^{4+o(1)})$.
\end{lemma}

Then, we present the results of the running time analysis for the fast FlowAR model.
\begin{lemma}[Inference Runtime of Fast FlowAR Architecture, informal version of Lemma~\ref{lem:runtime_fast_flowar_formal}]\label{lem:runtime_fast_flowar_informal}
    Consider the original FlowAR inference pipeline with the following parameters:
    \begin{itemize}
        \item {\bf Input tensor:} $\X \in \R^{h \times w \times c}$. Assume $h=w=n$ and $c = O(\log n)$.
        \item {\bf Number of scales:} $K = O(1)$.
        \item {\bf Scale factor:} For $i \in [K]$, $r_i:= a^{K-i}$ where base factor $a \in \mathbb{N}^+$.
        \item {\bf Upsampling functions}  For $i \in [K]$, $\phi_{\mathrm{up},i}(\cdot,a)$ from Definition~\ref{def:bicubic_up_sample_function}.
        \item {\bf Approximate Attention layer:}  For $i \in [K]$, $\mathsf{AAttC}_i(\cdot)$ defined in Definition~\ref{def:aattc}.
        \item {\bf Feed forward layer: } For $i \in [K]$, $\mathsf{FFN}_i(\cdot)$ which acts on flattened sequences of dimension defined in Definition~\ref{def:ffn}.
        \item {\bf Fast flow-matching layer:} For $i \in [K]$, $\mathsf{FNN}_i(\cdot,\cdot,\cdot)$ denote the fast flow-matching layer defined in Definition~\ref{def:fast_flow_matching_architecture}.
    \end{itemize}
    Under these conditions, the total inference runtime of FlowAR is bounded by $O(n^{2+o(1)})$.
\end{lemma}

\subsection{Error Propagation Analysis}\label{lem:error_propagation_analysis}
In this section, we present an error analysis introduced by the fast algorithm applied to the FlowAR model.
\begin{lemma}[Error Bound Between Fast FlowAR and FlowAR Outputs, informal version of Lemma~\ref{lem:error_analysis_fast_flowar}]\label{lem:error_analysis_fast_flowar_informal}
    Given the following:
    \begin{itemize}
        \item {\bf Input tensor:} $\X \in \R^{h \times w \times c}$.
        \item {\bf Scales number:} $K = O(1)$.
        \item {\bf Dimensions:} Let $h=w=n$ and $c = O(\log n)$. Let $\wt{h}_i := \sum_{j=1}^i h/r_j$ and $\wt{w}_i := \sum_{j=1}^i w/r_j$.
        \item {\bf Bounded Entries:} All tensors and matrices have entries bounded by $R = O(\sqrt{\log n})$.
        \item {\bf Layers:}
        \begin{itemize}
            \item $\phi_{\mathrm{up},a}(\cdot)$ :  bicubic upsampling function (Definition~\ref{def:bicubic_up_sample_function}).
            \item $\mathsf{Attn}(\cdot)$: attention layer (Definition~\ref{def:attn_layer}).
            \item $\mathsf{AAttC(\cdot)}$: approximate attention layer (Definition~\ref{def:aattc})
            \item $\mathsf{NN}(\cdot,\cdot,\cdot)$: flow-matching layer (Definition~\ref{def:flow_matching_architecture})
            \item $\mathsf{FNN}(\cdot,\cdot,\cdot)$: fast flow-matching layer (Definition~\ref{def:fast_flow_matching_architecture})
        \end{itemize}
        \item {\bf Input and interpolations:}
        \begin{itemize}
            \item Initial inputs: $\Z_{\mathrm{init}} \in \R^{(h/r_1)\times(w/r_1) \times c}$.
            \item $\Z_i:$ Reshaped tensor of  $\Z_{\mathrm{init}}, \phi_{\mathrm{up},1}(\wt{\Y}_1), \dots, \phi_{\mathrm{up},i-1}(\wt{\Y}_{i-1})$ for FlowAR.
            \item $\Z'_i:$ Reshaped tensor of  $\Z_{\mathrm{init}}, \phi_{\mathrm{up},1}(\wt{\Y}'_1), \dots, \phi_{\mathrm{up},i-1}(\wt{\Y}'_{i-1})$ for  Fast FlowAR.
            \item $\mathsf{F}_i^{t_i} \in \R^{h/r_i \times w/r_i \times c}$ be the interpolated value of FlowAR (Definition~\ref{def:flow}).
            \item $\mathsf{FF}_i^{t_i} \in \R^{h/r_i \times w/r_i \times c}$ be the interpolated value of Fast FlowAR (Definition~\ref{def:flow}).
        \end{itemize} 
        \item {\bf Outputs:}
        \begin{itemize}
            \item $\wt{\Y}_i \in \R^{h/r_i \times w/r_i \times c}$: FlowAR output at layer $i$ (Definition~\ref{def:flow_architecture_inference})
            \item $\wt{\Y}'_i \in \R^{h/r_i \times w/r_i \times c}$: Fast FlowAR output at layer $i$ (Definition~\ref{def:fast_flow_architecture_inference})
        \end{itemize}
    \end{itemize}

    Under these conditions, the $\ell_\infty$ error between the final outputs is bounded by:
    \begin{align*}
        \|\wt{\Y}'_K - \wt{\Y}_K\|_\infty \leq 1/\poly(n)
    \end{align*}
    
\end{lemma}

\subsection{Existence of Almost Quadratic Time Algorithm}\label{sec:almost_quadratic_time_algorithm}
This section presents a theorem proving the existence of a quadratic-time algorithm that speeds up the FlowAR architecture and guarantees a bounded additive error.
\begin{theorem}[Existence of Almost Quadratic Time Algorithm]
\label{thm:upper_bound:formal}
Suppose $d = O(\log n)$ and $R = o(\sqrt{\log n})$. There is an algorithm that approximates the  FlowAR architecture up to $1/\poly(n)$ additive error in $O(n^{2+o(1)})$ time.
\end{theorem}
\begin{proof}
    By combining the result of Lemma~\ref{lem:runtime_fast_flowar_informal} and Lemma~\ref{lem:error_analysis_fast_flowar_informal}, we can easily derive the proof.
\end{proof}

Our Theorem~\ref{thm:upper_bound:formal} shows that we can accelerate FlowAR while only introducing a small error. Using the low-rank approximation in the attention mechanism is also used in previous works \cite{kll+25,lls+24_conv,llss25,lss+25_relu,chl+24_rope,lss+24,lssz24_tat,as24_iclr,as24b,as24_rope,hsk+24}.
\section{Conclusion and future directions} \label{sec:conclusion}

In this paper we proposed a nested MLMC framework that offers important computational savings by performing most calculations in low precision and exploiting approximate random normal variables for the low precision path calculations. The low precision calculations could be performed in fixed precision on an FPGA for greater efficiency, and we suggested a procedure to optimise the bit-widths of every variable at each Monte Carlo level. This is an important improvement over previous mixed precision MLMC frameworks which held the lower precision fixed \cite{Rounding_error_oliver} or defined uniform bit-width at every level heuristically \cite{brugger2014mixed}. Our numerical results suggest that for the first levels our procedure reduces the cost at these levels by a factor 5 or 7. Hence the overall savings are significant since most paths are calculated on the first levels. Our approach would be even more efficient for the Milstein scheme because its higher order strong convergence leads to a greater proportion of the computational costs being on the coarsest levels.

The next stage of the research project will be to implement the RNG methods and the nested framework on FPGAs to determine the hardware requirements and confirm the extent of the computational savings. It would also be good to compare the performance benefits to using half-precision floating point arithmetic on GPUs or CPUs for the low-accuracy computations.





\newpage
\onecolumn

\ifdefined\isarxiv
\else

\title{Circuit Complexity Bounds for FlowAR Model\\(Supplementary Material)}
\maketitle

\fi

\appendix
\begin{center}
	\textbf{\LARGE Appendix }
\end{center}
{\bf Roadmap.} Section~\ref{sec:app_notations} presents all the notations of this paper. In Section~\ref{sec:app_missing_proof}, we present some missing proofs in Section~\ref{sec:main_result}. Section~\ref{sec:app_efficient_critieria} presents provably efficient criteria of the fast FlowAR model.


\section{Notations}\label{sec:app_notations}
Given a matrix $X \in \R^{hw \times d}$, we denote its tensorized form as $\X \in \R^{h \times w \times d}$. Additionally, we define the set $[n]$ to represent $\{1,2,\cdots, n\}$ for any positive integer $n$. We define the set of natural numbers as $\mathbb{N}:= \{0,1,2,\dots\}$. Let $X \in \mathbb{R}^{m \times n}$ be a matrix, where $X_{i,j}$ refers to the element at the $i$-th row and $j$-th column. When $x_i$ belongs to $\{ 0,1 \}^*$, it signifies a binary number with arbitrary length. In a general setting, $x_i$ represents a length $p$ binary string, with each bit taking a value of either 1 or 0. Given a matrix $X \in \R^{n \times d}$, we define $\|X\|_\infty  $ as the maximum norm of $X$. Specifically, $\|X\|_\infty = \max_{i,j} |X_{i,j}|$. 
\section{Supplementary Proof for Section~\ref{sec:main_result}}\label{sec:app_missing_proof}
In this section, we present some missing proofs in Section~\ref{sec:main_result}. 

\subsection{Computing Multiple-layer Perceptron in \texorpdfstring{$\TC^0$}{} }\label{sec:app_mlp_tc0}
This section presents the detailed proof for Lemma~\ref{lem:mlp_tc0_informal}.

\begin{lemma}[MLP computation in $\mathsf{TC}^0$, formal version of Lemma~\ref{lem:mlp_tc0_informal}]\label{lem:mlp_tc0_formal}
    Given an input tensor $\X\in \R^{h \times w \times c}$. Let $\mathsf{MLP}(\X,c,d)$ be the MLP layer defined in Definition~\ref{def:mlp}. Under the following constraints:
    \begin{itemize}
        \item Satisfy $h = w = n$,
        \item Channel bounds: $c, d\leq n$,
        \item Precision: $p \leq \poly(n)$,
    \end{itemize}
    The $\mathsf{MLP}(\X,c,d)$ function can be computed by a uniform $\mathsf{TC}^0$ circuit with:
    \begin{itemize}
        \item Size: $\poly(n)$.
        \item Depth: $2d_\mathrm{std} + d_{\oplus}$.
    \end{itemize}
    with $d_{\mathrm{std}}$ and $d_{\oplus}$ defined in Definition~\ref{lem:float_operations_TC}.
\end{lemma}


\begin{proof}
    For each $j \in [hw]$, by Lemma~\ref{lem:matrix_multi}, 
    compute $X_{j,*} \cdot W$ requires depth $d_{\mathrm{std}} + d_{\oplus}$. By Part 1 of Lemma~\ref{lem:float_operations_TC}, compute $X_{j,*} \cdot W + b$ requires depth $d_{\mathrm{std}}$.  Since for all $j \in [hw]$, the computation $X_{j,*} \cdot W + b$ can be simulated in parallel. Hence the total depth remains $2d_\mathrm{std} + d_{\oplus}$ and size is $\poly(n)$.
\end{proof}
\subsection{Computing Feed Forward Layer in \texorpdfstring{$\TC^0$}{} }\label{sec:app_ffn_tc0}
This section presents the detailed proof for Lemma~\ref{lem:ffn_tc0_informal}.

\begin{lemma}[FFN computation in $\mathsf{TC}^0$, formal version of Lemma~\ref{lem:ffn_tc0_informal}]\label{lem:ffn_tc0_formal}
 Given an input tensor $\X\in \R^{h \times w \times c}$. Let $\mathsf{FFN}(X):\R^{h \times w \times c} \to \R^{h \times w \times c}$ as defined in Definition~\ref{def:ffn}. Under the following constraints:
 \begin{itemize}
     \item Satisfy $h = w = n$,
         \item Channel bound: $c \leq n$,
         \item Precision bound: $p \leq \poly(n)$.
 \end{itemize}
The $\mathsf{FFN}(\X)$ layer can be computed by a uniform $\mathsf{TC}^0$ circuit with:
\begin{itemize}
    \item Size: $\poly(n)$.
    \item Depth: $6d_\mathrm{std} + 2d_{\oplus}$.
\end{itemize}
with $d_{\mathrm{std}}$ and $d_{\oplus}$ defined in Definition~\ref{lem:float_operations_TC}.
\end{lemma}
\begin{proof}
    For each $j \in [hw]$, by the proof of Lemma~\ref{lem:mlp_tc0_formal}, compute $X_{j,*} \cdot W_1 + b_1$ requires depth $2d_{\mathrm{std}} + d_{\oplus}$. By Lemma~\ref{lem:float_operations_TC}, compute $A_1 = \sigma(X_{j,*} \cdot W + b)$ requires depth $d_{\mathrm{std}}$. By applying Lemma~\ref{lem:mlp_tc0_formal} again, compute $A_2 = A_1\cdot W_2 +b_2$ requires depth $2d_{\mathrm{std}} + d_{\oplus}$. Lastly, by Part 1 of Lemma~\ref{lem:float_operations_TC}, compute $X_{j,*} + A_2$ requires depth $d_{\mathrm{std}}$.
    
    Combing the result above, we can have that compute $Y_{j,*}=X_{j,*} + \sigma(X_{j,*} \cdot W_1 + b_1)\cdot W_2 +b_2$ requires depth $6d_{\mathrm{std}}+2d_{\oplus}$.

    Since for all $j \in [hw]$, the computation $Y_{j,*}$ can be simulated in parallel. Hence the total depth remains $6d_{\mathrm{std}}+2d_{\oplus}$ and size is $\poly(n)$.
\end{proof}

\subsection{Computing Attention Layer in \texorpdfstring{$\TC^0$}{} }\label{sec:app_attn_tc0}
This section presents the detailed proof for Lemma~\ref{lem:attn_tc0_informal}.

\begin{lemma}[Attention layer computation in $\mathsf{TC}^0$, formal version of Lemma~\ref{lem:attn_tc0_informal}]\label{lem:attn_tc0_formal}
     Given an input tensor $\X \in \R^{h \times w \times c}$. Let $\mathsf{Attn}(X):\R^{h \times w \times c} \to \R^{h \times w \times c}$ as defined in Definition~\ref{def:attn_layer}. Under the following constraints:
     \begin{itemize}
         \item Satisfy $h = w = n$,
         \item Channel bound: $c \leq n$,
         \item Precision bound: $p \leq \poly(n)$.
     \end{itemize}
     The $\mathsf{Attn}(\X)$ layer can be computed by a uniform $\mathsf{TC}^0$ circuit with:
     \begin{itemize}
        \item Size: $\poly(n)$.
        \item Depth: $6(d_{\mathrm{std}} + d_{\oplus}) + d_{\exp}$.
     \end{itemize}    
     with $d_{\mathrm{std}}$ and $d_{\oplus}$ defined in Definition~\ref{lem:float_operations_TC}, $d_{\exp}$ defined in Definition~\ref{lem:exp}. 
\end{lemma}
\begin{proof}
    We analyze the $\mathsf{TC}^0$ simulation of the attention layer through sequential computation phases:
    \begin{itemize}
        \item {\bf Key-Query Product}: Compute $W_QW_K^\top$ vial Lemma~\ref{lem:matrix_multi} requires depth $d_{\mathrm{std}} + d_{\oplus}$.
        \item {\bf Pairwise Score Computation}: Compute $s_{i,j} = X_{i,*}   W_Q   W_K^\top   X_{j,*}^\top$ requires depth $2(d_{\mathrm{std}} + d_{\oplus})$ by Lemma~\ref{lem:matrix_multi}. By  Lemma~\ref{lem:exp}, computing $A_{i,j} = \exp(s_{i,j})$ requires depth $d_{\exp}$. 
    \end{itemize}
    Since all entries $A_{i,j}$ for $i, j \in [n]$ can be computed in parallel, the attention matrix $A$ computation requires depth $3(d_{\mathrm{std}} + d_{\oplus}) + d_{\exp}$.

    Then keep on analyzing:
    \begin{itemize}
        \item {\bf Row Nomalization:} Computing $D:=\diag(A{\bf 1}_n)$ requires depth $d_{\oplus}$ by Lemma~\ref{lem:float_operations_TC}. Computing $D^{-1}$ requires depth $d_{\mathrm{std}}$ by Lemma~\ref{lem:float_operations_TC} .
        \item {\bf Value Projection} Computing $AXW_V$ requires depth $2(d_{\mathrm{std}} + d_{\oplus})$ by applying Lemma~\ref{lem:matrix_multi}. Compute $D^{-1} \cdot A X W_V$ requires $d_{\mathrm{std}}$.
    \end{itemize}

    Combing the result, we need a
    \begin{align*}
        d_{\mathrm{all}} = 6(d_{\mathrm{std}} + d_{\oplus}) + d_{\exp}
    \end{align*}
    depth and size $\poly(n)$ uniform $\mathsf{TC}^0$ circuit to compute the attention layer.
\end{proof}

\subsection{Computing Layer-wise Norm Layer in \texorpdfstring{$\TC^0$}{} }\label{sec:app_ln_tc0}
This section presents the detailed proof for Lemma~\ref{lem:ln_tc0_informal}.


\begin{lemma}[Layer-wise norm layer computation in $\TC^0$, formal version of Lemma~\ref{lem:ln_tc0_informal}]\label{lem:ln_tc0_formal}
    Given an input tensor $\X \in \R^{h \times w \times c}$. Let $\mathsf{LN}(X):\R^{h \times w \times c} \to \R^{h \times w \times c}$ as defined in Definition~\ref{def:ln}. Under the following constraints:
    \begin{itemize}
         \item Satisfy $h = w = n$,
         \item Channel bound: $c \leq n$,
         \item Precision bound: $p \leq \poly(n)$.
     \end{itemize}
     The $\mathsf{LN}(\X)$ layer can be computed by a uniform $\mathsf{TC}^0$ circuit with:
     \begin{itemize}
        \item Size: $\poly(n)$.
        \item Depth: $5d_\mathrm{std} + 2d_{\oplus} + d_\mathrm{sqrt}$.
     \end{itemize}    
     with $d_{\mathrm{std}}$ and $d_{\oplus}$ defined in Definition~\ref{lem:float_operations_TC}, $d_{\mathrm{sqrt}}$ defined in Definition~\ref{lem:sqrt}.
\end{lemma}

\begin{proof}
    By Part 1 and Part 3  of Lemma~\ref{lem:float_operations_TC}, 
    computing mean vector $\mu_j$ requires depth $d_{\mathrm{std}}+d_{\oplus}$. By Part 1 and Part 3  of Lemma~\ref{lem:float_operations_TC}, 
    computing mean vector $\sigma^2_i$ requires depth $2d_{\mathrm{std}}+d_{\oplus}$.      By Lemma~\ref{lem:float_operations_TC} and Lemma~\ref{lem:sqrt}, computing $Y_{j,*}$ requires depth  $2d_{\mathsf{std}}+d_{\oplus}$. So the process requires total depth $5d_\mathrm{std} + 2d_{\oplus} + d_\mathrm{sqrt}$ and $\poly(n)$ size. 
\end{proof}
\section{Provably Efficient Criteria}\label{sec:app_efficient_critieria}

\subsection{Running Time Analysis for Inference Pipeline of Origin FlowAR Architecture }\label{sec:runtime_origin_flowar}

We proceed to compute the total running time for the inference pipeline of the origin FlowAR architecture.
\begin{lemma}[Inference Runtime of Original FlowAR Architecture, formal version of Lemma~\ref{lem:runtime_old_flowar_informal}]\label{lem:runtime_old_flowar_formal}
    Consider the original FlowAR inference pipeline with the following parameters:
    \begin{itemize}
        \item {\bf Input tensor:} $\X \in \R^{h \times w \times c}$. Assume $h=w=n$ and $c = O(\log n)$.
        \item {\bf Number of scales:} $K = O(1)$.
        \item {\bf Scale factor:} For $i \in [K]$, $r_i:= a^{K-i}$ where base factor $a \in \mathbb{N}^+$.
        \item {\bf Upsampling functions}  For $i \in [K]$, $\phi_{\mathrm{up},i}(\cdot,a)$ from Definition~\ref{def:bicubic_up_sample_function}.
        \item {\bf Attention layer:}  For $i \in [K]$, $\mathsf{Attn}_i(\cdot)$ which acts on flattened sequences of dimension defined in Definition~\ref{def:attn_layer}.
        \item {\bf Feed forward layer: } For $i \in [K]$, $\mathsf{FFN}_i(\cdot)$ which acts on flattened sequences of dimension defined in Definition~\ref{def:ffn}.
        \item {\bf Flow matching layer:} For $i \in [K]$, $\mathsf{NN}_i(\cdot,\cdot,\cdot)$ denote the flow-matching layer defined in Definition~\ref{def:flow_matching_architecture}.
    \end{itemize}
    Under these conditions, the total inference runtime of FlowAR is bounded by $O(n^{4+o(1)})$.
    
\end{lemma}
\begin{proof}
    {\bf Part 1: Running time of bicubic up-sample Layer.} The $i$-th layer pf FlowAR model contains $\phi_{\mathrm{up},1}(\cdot,2),\dots,\phi_{\mathrm{up},i-1}(\cdot,2)$. Considering $\phi_{\mathrm{up},i-1}(\cdot,2)$, this operation needs $O(n^{2}c/2^{2(K-i)})$ time. Then the total time required for upsampling in the i-th layer of the FlowAR architecture is $O( n^2 c \cdot \frac{1}{2^{2K}} \cdot (1 - \frac{1}{4^i}))$ which is due to simple algebra. Hence, the total runtime for all bicubic up sample functions is
    \begin{align*}
        \mathcal{T}_{\mathrm{up}} = &~ \sum_{i=1}^K O( n^2 c \cdot \frac{1}{2^{2K}} \cdot (1 - \frac{1}{4^i}))\\
        =&~ O(n^{2+o(1)})
    \end{align*}
    where the first equation is derived from summing up all the running time of the up sample functions, the second step is due to simple algebra and $K = O(1)$ and $c = O(\log n)$.

    {\bf Part 2: Running time of Attention Layer.} The input size of the $i$-th attention layer $\mathsf{Attn}_i$ is $\sum_{j=1}^i (n/2^{K-j}) \times \sum_{j=1}^i(n/2^{K-j}) \times c $. So the time needed to compute the $i$-th attention layer is $O(n^4c \cdot (2^i-1)^4/2^{4K-4})$. Hence, the total runtime for all attention layers is
    \begin{align*}
        \mathcal{T}_{\mathrm{Attn}} =&~ \sum_{i=1}^K O( n^4c \cdot (2^i-1)^4/2^{4K-4})\\
        =&~ O(n^{4+o(1)})
    \end{align*}
    The first equation is derived from summing up all the running time of the attention layer, the second step is due to simple algebra and $K = O(1)$ and $c = O(\log n)$.

    {\bf Part 3: Running time of FFN Layer.} The input size of the $i$-th FFN layer $\mathsf{FFN}_i$ is $\sum_{j=1}^i (n/2^{K-j}) \times \sum_{j=1}^i(n/2^{K-j}) \times c $. So by Definition~\ref{def:ffn}, we can easily derive that the time needed to compute the $i-$th FFN layer is $O(n^2c^2 (2^i-1)^2/2^{2K-2} )$. Hence, the total runtime for all FFN layers is
    \begin{align*}
        \mathcal{T}_{\mathrm{FFN}} = &~ \sum_{i=1}^K O(n^2c^2 (2^i-1)^2/2^{2K-2}) \\
        =&~ O(n^{2+o(1)})
    \end{align*}
    The first step is derived from summing up all the running time of the FFN layer,  and the second step is due to simple algebra and $K = O(1)$ and $c = O(\log n)$.

    {\bf Part 4: Running time of Flow Matching Layer.} The input size of the $i$-th flow-matching layer $\mathsf{NN}_i$ is $ (n/2^{K-i}) \times (n/2^{K-i}) \times c $. It's trivially that the running time of the flow-matching layer will be dominated by the running time of the attention layer, which is $O(n^{4}c/ 2^{4(K-i)})$ (see Part 2 of Definition~\ref{def:flow_matching_architecture}). Hence, the total runtime for all flow-matching layers is 
    \begin{align*}
        \mathcal{T}_{\mathsf{FM}} =&~ \sum_{i=1}^K O(n^4 c /2^{4(K-i)})\\
        =&~ O(n^{4+o(1)})
    \end{align*}
    The first step is derived from summing up all the running time of the origin flow-matching layer, and the second step is due to simple algebra and $K = O(1)$ and $c = O(\log n)$.

    Then, by summing up Part 1 to Part 4, we can get the total running time for FlowAR architecture, which is
    \begin{align*}
        \mathcal{T}_{\mathrm{ori}} =&~ \mathcal{T}_{\mathrm{up}} + \mathcal{T}_{\mathrm{Attn}} + \mathcal{T}_{\mathrm{FFN}} + \mathcal{T}_{\mathsf{FM}}\\
        =&~ O(n^{4+o(1)})
    \end{align*}
\end{proof}
Lemma~\ref{sec:runtime_origin_flowar} demonstrates the runtime required for the original FlowAR architecture, from which we can deduce that the dominant term in the runtime comes from the computation of the Attention Layer.
\subsection{Running Time Analysis for Inference Pipeline of Fast FlowAR Architecture }\label{sec:runtime_fast_flowar}
In this section, we apply the conclusions of \cite{as23} to the FlowAR architecture, where all Attention modules in FlowAR are computed using the Approximate Attention Computation defined in Definition~\ref{def:aattc}.
\begin{lemma}[Inference Runtime of Fast FlowAR Architecture, formal version of Lemma~\ref{lem:runtime_fast_flowar_informal}]\label{lem:runtime_fast_flowar_formal}
    Consider the original FlowAR inference pipeline with the following parameters:
    \begin{itemize}
        \item {\bf Input tensor:} $\X \in \R^{h \times w \times c}$. Assume $h=w=n$ and $c = O(\log n)$.
        \item {\bf Number of scales:} $K = O(1)$.
        \item {\bf Scale factor:} For $i \in [K]$, $r_i:= a^{K-i}$ where base factor $a \in \mathbb{N}^+$.
        \item {\bf Upsampling functions}  For $i \in [K]$, $\phi_{\mathrm{up},i}(\cdot,a)$ from Definition~\ref{def:bicubic_up_sample_function}.
        \item {\bf Approximate Attention layer:}  For $i \in [K]$, $\mathsf{AAttC}_i(\cdot)$ defined in Definition~\ref{def:aattc}.
        \item {\bf Feed forward layer: } For $i \in [K]$, $\mathsf{FFN}_i(\cdot)$ which acts on flattened sequences of dimension defined in Definition~\ref{def:ffn}.
        \item {\bf Fast flow-matching layer:} For $i \in [K]$, $\mathsf{FNN}_i(\cdot,\cdot,\cdot)$ denote the fast flow-matching layer defined in Definition~\ref{def:fast_flow_matching_architecture}.
    \end{itemize}
    Under these conditions, the total inference runtime of FlowAR is bounded by $O(n^{2+o(1)})$.
      
\end{lemma}
\begin{proof}
    {\bf Part 1: Running time of bicubic up-sample Layer.} The runtime of the upsample function in the fast FlowAR architecture is the same as that in the original FlowAR architecture, which is
    \begin{align*}
        \mathcal{T}_{\mathrm{up}} =  O(n^{2+o(1)})
    \end{align*}
    

    {\bf Part 2: Running time of Attention Layer.} The input size of the $i$-th approximate attention computation layer $\mathsf{AAttC}_i$ is $\sum_{j=1}^i (n/2^{K-j}) \times \sum_{j=1}^i(n/2^{K-j}) \times c $. So the time needed to compute the $i$-th attention layer is $O(n^{2+o(1)} \cdot (2^i-1)^4/2^{4K-4})$. Hence, the total runtime for all attention layers is
    \begin{align*}
        \mathcal{T}_{\mathrm{Attn}} =&~ \sum_{i=1}^K O( n^{2+o(1)} \cdot (2^i-1)^4/2^{4K-4})\\
        =&~ O(n^{2+o(1)})
    \end{align*}
    The first equation is derived from summing up all the running time of the approximate attention computation layer, and the second equation is due to basic algebra and $K = O(1)$.

    {\bf Part 3: Running time of FFN Layer.} The runtime of the FFN layer in the fast FlowAR architecture is the same as that in the original FlowAR architecture, which is
    \begin{align*}
        \mathcal{T}_{\mathrm{FFN}} =  O(n^{2+o(1)})
    \end{align*}

    {\bf Part 4: Running time of Flow Matching Layer.} For each $i \in [K]$, the input size of the $i$-th fast flow-matching layer $\mathsf{FNN}_i$ is $ (n/2^{K-i}) \times (n/2^{K-i}) \times c $. By Definition~\ref{def:mlp}, we can know that the total computational time for the MLP layer is $O(n^{2+o(1)})$, which is due to $c=O(\log n)$. Then by Lemma~\ref{lem:as23_attention}, we can speed up the attention computation from $O(n^{4+o(1)})$ to $O(n^{2+o(1)})$. Hence, the total runtime for all flow-matching layers is
     \begin{align*}
        \mathcal{T}_{\mathrm{Attn}} =&~ \sum_{i=1}^K O( n^{2+o(1)})\\
        =&~ O(n^{2+o(1)})
    \end{align*}
    The equation is due to $K = O(1)$.

    Then, by summing up Part 1 to Part 4, we can get the total running time for fast FlowAR architecture, which is
    \begin{align*}
        \mathcal{T}_{\mathrm{fast}} =&~ \mathcal{T}_{\mathrm{up}} + \mathcal{T}_{\mathrm{Attn}} + \mathcal{T}_{\mathrm{FFN}} + \mathcal{T}_{\mathsf{FM}}\\
        =&~ O(n^{2+o(1)})
    \end{align*}
    
    
\end{proof}

\subsection{Error Analysis of \texorpdfstring{$\mathsf{MLP}(\X')$}{} and \texorpdfstring{$\mathsf{MLP}(\X)$}{}}\label{sec:error_analysis_of_mlp_x_prime_mlp_x}
We conduct the error analysis between $\mathsf{MLP}(\X')$ and $\mathsf{MLP}(\X)$ where $\X'$ is the approximation version of $\X$.
\begin{lemma}[Error analysis of MLP Layer]\label{lem:error_analysis_mlp}
    If the following conditions hold:
    \begin{itemize}
        \item Let $\X \in \R^{h \times w \times c}$ denote the input tensor.
        \item Let $\X' \in \R^{h \times w \times c}$ denote the approximation version of input tensor $\X$.
        \item Let $\epsilon \in (0, 0.1)$ denote the approximation error. 
        \item Suppose we have $\| \X' - \X \|_\infty \leq \epsilon$.
        \item Let $R > 1$.
        \item Assume the value of each entry in matrices can be bounded by $R$.  
        \item Let $\mathsf{MLP}(\cdot,c,d)$ denote the MLP layer defined in Definition~\ref{def:mlp}.
    \end{itemize}
    We can demonstrate the following
    \begin{align*}
        \|\mathsf{MLP}(\X') - \mathsf{MLP}(\X)\|_\infty \leq cR\epsilon
    \end{align*}
    Here, we abuse the $\ell_\infty$ norm in its tensor form for clarity.
\end{lemma}
\begin{proof}
    We can show that for $i \in [h],j \in [w], l \in [c]$, we have
    \begin{align*}
        \|\mathsf{MLP}(\X',c,d)_{i,j,*} - \mathsf{MLP}(\X,c,d)_{i,j,*}\|_\infty =&~ \| \X'_{i,j,*}\cdot W - \X_{i,j,*} \cdot W \|_\infty\\
        \leq&~ \|\underbrace{(\X'_{i,j,*}-\X_{i,j,*})}_{1 \times c} \cdot \underbrace{W}_{c\times d} \|_\infty \\
        \leq &~ c \cdot \|\underbrace{(\X'_{i,j,*}-\X_{i,j,*})}_{1 \times c}\|_\infty \cdot \|\underbrace{W}_{c\times d} \|_\infty\\
        \leq&~ c \cdot R \cdot \epsilon
    \end{align*}
    The first equation is due to Definition~\ref{def:mlp}, the second inequality is derived from simple algebra, the third inequality is a consequence of basic matrix multiplication, and the last inequality comes from the conditions of this lemma.

    Then by the definition of $\ell_\infty$ norm, we can easily get the proof.
\end{proof}



\subsection{Error Analysis of \texorpdfstring{$\mathsf{AAttC}(\X')$}{} and \texorpdfstring{$\mathsf{Attn}(\X)$}{}}\label{sec:error_analysis_of_aattc_x_prime_attn_x}
We conduct the error analysis between $\mathsf{AAttC}(\X')$ and $\mathsf{Attn}(\X)$ where $\X'$ is the approximation version of $\X$.
\begin{lemma}[Error analysis of $\mathsf{AAttC}(X')$ and $\mathsf{Attn}(X)$, Lemma B.4 of \cite{kll+25}]\label{lem:error_analysis_aattc_attn}
    If the following conditions hold:
    \begin{itemize}
        \item Let $\X \in \R^{h \times w \times c}$ denote the input tensor.
        \item Let $\X' \in \R^{h \times w \times c}$ denote the approximation version of input tensor $\X$.
        \item Let $\epsilon \in (0, 0.1)$ denote the approximation error. 
        \item Suppose we have $\| \X' - \X \|_\infty \leq \epsilon$.
        \item Let $R > 1$.
        \item Assume the value of each entry in matrices can be bounded by $R$. 
        \item Let $\mathsf{Attn}$ denote the attention layer defined in Definition~\ref{def:attn_layer}.
        \item Let $\mathsf{AAttC}$ denote the approximated attention layer defined in Definition~\ref{def:aattc}.
        \item Let $U,V \in \R^{hw \times k}$ be low-rank matrices constructed for polynomial approximation of attention matrix $\mathsf{AAttC}(\X)$.
        \item Let $f$ be a polynomial with degree $g$.
    \end{itemize}
    We can demonstrate the following:
    \begin{align*}
        \| \mathsf{AAttC}(\X') - \mathsf{Attn}(\X) \|_\infty \leq O( k R^{g+1} c) \cdot \epsilon
    \end{align*}
    Here, we abuse the $\ell_\infty$ norm in its tensor form for clarity.
\end{lemma}

\subsection{Error Analysis of \texorpdfstring{$\mathsf{FFN}(\X')$}{} and \texorpdfstring{$\mathsf{FFN}(\X)$}{}}\label{sec:error_analysis_of_ffn_x_prime_ffn_x}
In this section, we conduct the error analysis between $\mathsf{FFN}(\X')$ and $\mathsf{FFN}(\X)$ where $\X'$ is the approximation version of $\X$.
\begin{lemma}[Error analysis of $\mathsf{FFN}(\X')$ and $\mathsf{FFN}(\X)$]\label{lem:error_analysis_ffn}
    If the following conditions hold:
    \begin{itemize}
        \item Let $\X \in \R^{h \times w \times c}$ denote the input tensor.
        \item Let $\X' \in \R^{h \times w \times c}$ denote the approximation version of input tensor $\X$.
        \item Let $\epsilon \in (0, 0.1)$ denote the approximation error. 
        \item Suppose we have $\| \X' - \X \|_\infty \leq \epsilon$.
        \item Let $R > 1$.
        \item Assume the value of each entry in matrices can be bounded by $R$. 
        \item Let $\mathsf{FFN}$ denote the FFN layer defined in Definition~\ref{def:ffn}.
        \item Let the activation function $\sigma(\cdot)$ in $\mathsf{FFN}$ be the ReLU activation function.
    \end{itemize}
    We can demonstrate the following:
    \begin{align*}
        \| \mathsf{FFN}(\X') - \mathsf{FFN}(\X) \|_\infty \leq O(c^2 R^2) \cdot \epsilon
    \end{align*}
    Here, we abuse the $\ell_\infty$ norm in its tensor form for clarity.
\end{lemma}
\begin{proof}
    Firstly we can bound that for $i \in [h], j \in [w]$
    \begin{align}\label{eq:linear_transformation_bound}
        \| (\X'_{i,j,*}\cdot W_1 +b_1) - (\X_{i,j,*}\cdot W_1 +b_1)\|_\infty =&~ \| \underbrace{(\X'_{i,j,*}-\X_{i,j,*})}_{1 \times c} \cdot \underbrace{W_1}_{c\times c}\|_\infty\notag\\
        \leq&~ c \cdot \|\X'_{i,j,*}-\X_{i,j,*}\|_\infty \|W_1 \|_\infty\notag\\
        \leq&~ c \cdot \epsilon \cdot R
    \end{align}
    The first equation comes from basic algebra, the second inequality is due to basic matrix multiplication, and the last inequality follows from the conditions of this lemma.
    
    We can show that for $i \in [h], j \in [w]$,
    \begin{align*}
        \| \mathsf{FFN}(\X')_{i,j,*} - \mathsf{FFN}(\X)_{i,j,*} \|_\infty=&~ \| \X'_{i,j,*}-\X_{i,j,*} +\underbrace{ (\sigma(\X_{i,j,*}\cdot W_1 + b_1)- \sigma(\X'_{i,j,*}\cdot W_1 + b_1))}_{1 \times c} \cdot \underbrace{W_2}_{c\times c}\|_\infty \\
        \leq&~ \| \X'_{i,j,*}-\X_{i,j,*}\|_\infty +c\cdot \|W_2\|_\infty \cdot \|\sigma(\X_{i,j,*}\cdot W_1 + b_1)- \sigma(\X'_{i,j,*}\cdot W_1 + b_1)\|_\infty\\
        \leq&~ \epsilon + c R \cdot \| (\X'_{i,j,*} W_1 +b_1) - (\X_{i,j,*} W_1 +b_1)\|_\infty\\
        \leq&~ \epsilon + c^2 R^2 \cdot \epsilon \\
        =&~ O(c^2 R^2) \cdot \epsilon
    \end{align*}
    The first equation is due to Definition~\ref{def:ffn}, the second step follows from triangle inequality and basic matrix multiplication, the third step follows from the property of ReLU activation function and basic algebra, the fourth step follows from Eq.~\eqref{eq:linear_transformation_bound}, and the last step follows from simple algebra.
\end{proof}

\subsection{Error Analysis of \texorpdfstring{$\phi_{\mathrm{up}}(\X')$}{} and \texorpdfstring{$\phi_{\mathrm{up}}(\X)$}{}}\label{sec:error_analysis_of_phi_x_prime_phi_x}
In this section, we conduct the error analysis between $\phi_{\mathrm{up}}(\X')$ and $\phi_{\mathrm{up}}(\X)$ where $\X'$ is the approximation version of $\X$.
\begin{lemma}[Error Analysis of Up Sample Layer, Lemma B.5 of \cite{kll+25}]\label{lem:error_analysis_up_layer}
If the following conditions hold:
\begin{itemize}
    \item Let $\X \in \R^{h \times w \times c}$ denote the input tensor.
    \item Let $\X' \in \R^{h \times w \times c}$ denote the approximation version of input tensor $\X$.
    \item Let $a = 2$ denote a positive integer.
    \item Let $\phi_{\mathrm{up}, i}(\cdot, a)$ be the bicubic up sample function defined in Definition~\ref{def:bicubic_up_sample_function}.
    \item Let $\epsilon \in (0,0.1)$ denote the approximation error.
    \item Let $\|X-X'\|_\infty\leq \epsilon$.
\end{itemize}
Then we have
\begin{align*}
    \| \phi_{\rm up}(\X',a) - \phi_{\rm up}(\X,a) \|_\infty \leq O(\epsilon)
\end{align*}
Here, we abuse the $\ell_\infty$ norm in its tensor form for clarity.
\end{lemma}

\subsection{Error Analysis of \texorpdfstring{$\mathsf{FNN}(\F'^t,\X',t)$}{} and \texorpdfstring{$\mathsf{NN}(\F^t,\X,t)$}{}}\label{sec:error_analysis_of_flow_matching_layer}
In this section, we conduct the error analysis between $\mathsf{FNN}(\F'^t,\X',t)$ and  $\mathsf{NN}(\F^t,\X,t)$ where $\X'$ is the approximation version of $\X$.
\begin{lemma}[Error Analysis of Flow Matching Layer]\label{lem:error_analysis_flow_matching_layer}
If the following conditions hold:
\begin{itemize}
    \item Let $\X \in \R^{h \times w \times c}$ denote the input tensor.
    \item Let $\X' \in \R^{h \times w \times c}$ denote the approximation version of input tensor $\X$.
    \item Let $\F^t,\mathsf{FF}^t \in \R^{h \times w \times c}$ be the interpolated input defined in Definition~\ref{def:flow}.
    \item Let $\mathsf{NN}(\cdot,\cdot,\cdot)$ denote flow-matching layer defined in Definition~\ref{def:flow_matching_architecture}.
    \item Let $\mathsf{FNN}(\cdot,\cdot,\cdot)$ denote fast flow-matching layer defined in Definition~\ref{def:fast_flow_matching_architecture}.
     \item Let $\mathsf{Attn}$ denote the attention layer defined in Definition~\ref{def:attn_layer}.
    \item Let $\mathsf{AAttC}$ denote the approximated attention layer defined in Definition~\ref{def:aattc}.
    \item Let $R > 1$.
    \item Assume the value of each entry in matrices can be bounded by $R$. 
    \item Let $U,V \in \R^{hw \times k}$ be low-rank matrices constructed for polynomial approximation of attention matrix $\mathsf{AAttC}(\X)$.
    \item Let $f$ be a polynomial with degree $g$.
    \item Let $\epsilon \in (0,0.1)$ denote the approximation error.
    \item Let $\|\X-\X'\|_\infty\leq \epsilon$.
    \item Let $t \in [0,1]$ denote a time step.
    \item Assume that Layer-wise Norm layer $\mathsf{LN}(\cdot)$ defined in Definition~\ref{def:ln} does not exacerbate the propagation of errors, i.e., if $\|X'-X\|_\infty \leq \epsilon$, then $\|\mathsf{LN}(X')-\mathsf{LN}(X)\|_\infty \leq \epsilon$.
\end{itemize}
Then we have
\begin{align*}
    \| \mathsf{FNN}(\mathsf{FF}^t,\X',t) - \mathsf{NN}(\F^t,\X,t) \|_\infty \leq O(kR^{g+6}c^3) \cdot \epsilon
\end{align*}
Here, we abuse the $\ell_\infty$ norm in its tensor form for clarity.
\end{lemma}
\begin{proof}
    Firstly, we can show that
    \begin{align*}
        \|\mathsf{FF}^t- \F^t\|_\infty = \| t (\X' - \X) \|_\infty \leq \epsilon
    \end{align*}
    The inequality comes from $t \in [0,1]$ and $\|\X'-\X\|_\infty \leq \epsilon$.

    By {\bf Step 1} of Definition~\ref{def:flow_matching_architecture} and Definition~\ref{def:fast_flow_matching_architecture}, we need to compute
    \begin{align*}
        \alpha_1, \alpha_2, \beta_1, \beta_2, \gamma_1, \gamma_2=&~  \mathsf{MLP}(\X + t \cdot {\bf 1}_{h \times w \times c},c,6c)\\
         \alpha'_1, \alpha'_2, \beta'_1, \beta'_2, \gamma'_1, \gamma'_2=&~  \mathsf{MLP}(\X' + t \cdot {\bf 1}_{h \times w \times c},c,6c)\\
    \end{align*}
    Then, we can show that
    \begin{align*}
        \|\alpha'_1 - \alpha_1\|_\infty \leq c R \epsilon
    \end{align*}
    where the step follows from Lemma~\ref{lem:error_analysis_mlp}. The same conclusion holds for the intermediate parameter $\alpha_2, \beta_1, \beta_2, \gamma_1, \gamma_2$.



    By {\bf Step 2} of Definition~\ref{def:flow_matching_architecture} and Definition~\ref{def:fast_flow_matching_architecture}, we need to compute
    \begin{align*}
        \F'^t =&~ \mathsf{Attn}(\gamma_1 \circ \mathsf{LN}(\F^t)+\beta_1) \circ \alpha_1\\
        \mathsf{FF}'^t =&~ \mathsf{AAttC}(\gamma'_1 \circ \mathsf{LN}(\mathsf{FF}^t)+\beta'_1) \circ \alpha'_1\\
    \end{align*}
    Then, we move forward to show that
    \begin{align}\label{eq:flow_matching_tmp1}
        &~\| \gamma'_1 \circ \mathsf{LN}(\mathsf{FF}^t) + \beta'_1 - \gamma_1 \circ \mathsf{LN}(\F^t) - \beta_1\|_\infty\notag\\
        \leq&~ \| \gamma'_1 \circ \mathsf{LN}(\mathsf{FF}^t) - \gamma_1 \circ \mathsf{LN}(\F^t)  \|_\infty + \|\beta'_1-\beta_1\|_\infty\notag\\
        \leq&~  \| \gamma'_1 \circ (\mathsf{LN}(\mathsf{FF}^t) - \mathsf{LN}(\F^t))\| + \|(\gamma'_1-\gamma_1) \circ \mathsf{LN}(\F^t)\|_\infty + cR\epsilon\notag\\
        \leq&~ R \cdot \epsilon + R\cdot \epsilon + cR\epsilon\notag\\
        =&~ O(cR)\cdot \epsilon
    \end{align}
    where the first and second step follows from triangle inequality, the third step follows from conditions of this Lemma, and the last step follows from simple algebra.

    Then we have
    \begin{align}\label{eq:flow_matching_tmp2}
        \| \mathsf{AAttC}(\gamma'_1 \circ \mathsf{LN}(\mathsf{FF}^t) + \beta'_1) -\mathsf{Attn}(\gamma_1 \circ \mathsf{LN}(\F^t) + \beta_1)\|_\infty \leq&~ O(k R^{g+1} c) \cdot O(cR) \cdot \epsilon \notag\\
        \leq&~ O(kR^{g+2} c^2) \epsilon
    \end{align}
    where the first step follows from Lemma~\ref{lem:error_analysis_aattc_attn} and Eq.~\eqref{eq:flow_matching_tmp1} and the second step follows from simple algebra.

    Now, we are able to show that
    \begin{align}\label{eq:error_analysis_tmp3}
        \| \mathsf{FF}'^t -  \F'^t\|_\infty =&~ \|\mathsf{AAttC}(\gamma'_1 \circ \mathsf{LN}(\mathsf{FF}^t)+\beta'_1) \circ \alpha'_1 -  \mathsf{Attn}(\gamma_1 \circ \mathsf{LN}(\F^t)+\beta_1) \circ \alpha_1\|_\infty\notag \\
        \leq&~ \| \mathsf{AAttC}(\gamma'_1 \circ \mathsf{LN}(\mathsf{FF}^t)+\beta'_1) \circ (\alpha'_1-\alpha_1)\|_\infty \notag \\+&~ \|\alpha_1 \cdot \mathsf{AAttC}(\gamma'_1 \circ \mathsf{LN}(\mathsf{FF}^t) + \beta'_1) -\mathsf{Attn}(\gamma_1 \circ \mathsf{LN}(\F^t) + \beta_1) \|_\infty\notag \\
        \leq&~ R \cdot cR\epsilon + R \cdot O(kR^{g+2}c^2) \epsilon\notag \\
        =&~   O(kR^{g+3}c^2) \epsilon
    \end{align}
    where the first step follows from the definition of $\wh{\F}'^t$ and $\wh{\F}^t$, the second step follows from triangle inequality, the third step follows from Eq.~\eqref{eq:flow_matching_tmp2} and the conditions of this lemma, and the last step follows from simple algebra.

    By {\bf Step 3} of Definition~\ref{def:flow_matching_architecture} and Definition~\ref{def:fast_flow_matching_architecture}, we need to compute
    \begin{align*}
        \F''^t =&~\mathsf{MLP}(\gamma_2 \circ \mathsf{LN}(\F'^t)+ \beta_2,c,c) \circ \alpha_2\\
        \mathsf{FF}''^t=&~\mathsf{MLP}(\gamma'_2 \circ \mathsf{LN}(\mathsf{FF}'^t)+ \beta'_2,c,c) \circ \alpha'_2
    \end{align*}
    Then, we move forward to show that
    \begin{align}\label{eq:error_analysis_tmp4}
        &~\|\gamma'_2 \circ \mathsf{LN}(\mathsf{FF}'^t)+ \beta'_2 - \gamma_2 \circ \mathsf{LN}(\F'^t) -\beta_2 \|_\infty\notag\\
        \leq&~ \| \gamma'_2 \circ \mathsf{LN}(\mathsf{FF}'^t)- \gamma_2 \circ \mathsf{LN}(\F'^t) \|_\infty + \|\beta'_1-\beta_1\|_\infty\notag\\
        \leq&~\|\gamma'_2\circ(\mathsf{LN}(\mathsf{FF}'^t) - \mathsf{LN}(\F'^t)) \|_\infty+ \|(\gamma'_2-\gamma_2)\circ \mathsf{LN}(\F'^t)\|_\infty + cR\epsilon\notag\\
        \leq&~ R \cdot O(kR^{g+3}c^2) \epsilon + cR\epsilon \cdot R + cR\epsilon\notag\\
        =&~ O(kR^{g+4}c^2)\cdot \epsilon
    \end{align}
    where the first and the second steps follow from triangle inequality, the third step follows from Eq.~\eqref{eq:error_analysis_tmp3} the conditions of this lemma, and the last step follows from simple algebra.

    Then, we can show
    \begin{align}\label{eq:error_analysis_tmp5}
        \|\mathsf{MLP}(\gamma'_2 \circ \mathsf{LN}(\mathsf{FF}'^t)+ \beta'_2)-\mathsf{MLP}(\gamma_2 \circ \mathsf{LN}(\F'^t) +\beta_2)\|_\infty \leq&~  c R \cdot O(kR^{g+4} c^2) \cdot \epsilon\notag\\
        =&~ O(kR^{g+5}c^3)\cdot \epsilon
    \end{align}
    where the first step follows from Lemma~\ref{lem:error_analysis_mlp} and Eq.~\eqref{eq:error_analysis_tmp4} and the second step follows from simple algebra.

    Finally, we are able to show that
    \begin{align*}
        &~\|\mathsf{FNN}(\mathsf{FF}^t,\X',t) - \mathsf{FN}(\F^t,\X,t)\|_\infty\\
        =&~ \|\mathsf{MLP}(\gamma'_2 \circ \mathsf{LN}(\mathsf{FF}'^t)+ \beta'_2,c,c) \circ \alpha'_2 - \mathsf{MLP}(\gamma_2 \circ \mathsf{LN}(\F'^t)+ \beta_2,c,c) \circ \alpha_2 \|_\infty\\
        \leq&~ \| (\mathsf{MLP}(\gamma'_2 \circ \mathsf{LN}(\mathsf{FF}'^t)+ \beta'_2,c,c) - \mathsf{MLP}(\gamma_2 \circ \mathsf{LN}(\F'^t)+ \beta_2,c,c))\circ \alpha'_2\|_\infty \\+ &~ \|\mathsf{MLP}(\gamma_2 \circ \mathsf{LN}(\F'^t)+ \beta_2,c,c)  \circ (\alpha'_2 - \alpha_2) \|_\infty\\
        \leq&~ R \cdot O(kR^{g+5}c^3) \cdot \epsilon + R \cdot cR\epsilon\\
        =&~  O(kR^{g+6}c^3) \cdot \epsilon
    \end{align*}
    where the step follows from the definition of output of $\mathsf{FFN}(\F'^t,\X',t)$ and $ \mathsf{FN}(\F^t,\X,t)$, the second step follows from triangle inequality, the third step follows from Eq.~\eqref{eq:error_analysis_tmp5} and conditions of this lemma, and the last step follows from simple algebra.

    Then, we complete the proof.
\end{proof}

\subsection{Error Analysis of Fast FlowAR Architecture}\label{sec:error_analysis_fast_flowar}
Here, we proceed to present the error analysis of fast FlowAR Architecture.
\begin{lemma}[Error Bound Between Fast FlowAR and FlowAR Outputs]\label{lem:error_analysis_fast_flowar}
    Given the following:
    \begin{itemize}
        \item {\bf Input tensor:} $\X \in \R^{h \times w \times c}$.
        \item {\bf Scales number:} $K = O(1)$.
        \item {\bf Dimensions:} Let $h=w=n$ and $c = O(\log n)$. Let $\wt{h}_i := \sum_{j=1}^i h/r_j$ and $\wt{w}_i := \sum_{j=1}^i w/r_j$.
        \item {\bf Bounded Entries:} All tensors and matrices have entries bounded by $R = O(\sqrt{\log n})$.
        \item {\bf Layers:}
        \begin{itemize}
            \item $\phi_{\mathrm{up},a}(\cdot)$ :  bicubic upsampling function (Definition~\ref{def:bicubic_up_sample_function}).
            \item $\mathsf{Attn}(\cdot)$: attention layer (Definition~\ref{def:attn_layer}).
            \item $\mathsf{AAttC(\cdot)}$: approximate attention layer (Definition~\ref{def:aattc})
            \item $\mathsf{NN}(\cdot,\cdot,\cdot)$: flow-matching layer (Definition~\ref{def:flow_matching_architecture})
            \item $\mathsf{FNN}(\cdot,\cdot,\cdot)$: fast flow-matching layer (Definition~\ref{def:fast_flow_matching_architecture})
        \end{itemize}
        \item {\bf Input and interpolations:}
        \begin{itemize}
            \item Initial inputs: $\Z_{\mathrm{init}} \in \R^{(h/r_1)\times(w/r_1) \times c}$.
            \item $\Z_i:$ Reshaped tensor of  $\Z_{\mathrm{init}}, \phi_{\mathrm{up},1}(\wt{\Y}_1), \dots, \phi_{\mathrm{up},i-1}(\wt{\Y}_{i-1})$ for FlowAR.
            \item $\Z'_i:$ Reshaped tensor of  $\Z_{\mathrm{init}}, \phi_{\mathrm{up},1}(\wt{\Y}'_1), \dots, \phi_{\mathrm{up},i-1}(\wt{\Y}'_{i-1})$ for  Fast FlowAR.
            \item $\mathsf{F}_i^{t_i} \in \R^{h/r_i \times w/r_i \times c}$ be the interpolated value of FlowAR (Definition~\ref{def:flow}).
            \item $\mathsf{FF}_i^{t_i} \in \R^{h/r_i \times w/r_i \times c}$ be the interpolated value of Fast FlowAR (Definition~\ref{def:flow}).
        \end{itemize} 
        \item {\bf Outputs:}
        \begin{itemize}
            \item $\wt{\Y}_i \in \R^{h/r_i \times w/r_i \times c}$: FlowAR output at layer $i$ (Definition~\ref{def:flow_architecture_inference})
            \item $\wt{\Y}'_i \in \R^{h/r_i \times w/r_i \times c}$: Fast FlowAR output at layer $i$ (Definition~\ref{def:fast_flow_architecture_inference})
        \end{itemize}
    \end{itemize}
    Under these conditions, the $\ell_\infty$ error between the final outputs is bounded by:
    \begin{align*}
        \|\wt{\Y}'_K - \wt{\Y}_K\|_\infty \leq 1/\poly(n)
    \end{align*}
    
\end{lemma}
\begin{proof}
    We can conduct math induction as the following.
    
    Consider the first layer of fast FlowAR Architecture. Firstly, we can show that
    \begin{align*}
        \| \mathsf{AAttC}_1(\Z_{1}) - \mathsf{Attn}_1(\Z_{1})\|_\infty \leq 1/\poly(n)
    \end{align*}
    The inequality is derived Lemma~\ref{lem:as23_attention}.
    
    Then, we have
    \begin{align*}
        \| \wh{\Y}'_1 - \wh{\Y}_1\|_\infty =&~ \|\mathsf{FFN}_1(\mathsf{AAttC}_1(\Z_{1})) - \mathsf{FFN}_1(\mathsf{Attn}_1(\Z_{1}))\|_\infty\\
        \leq&~ O(c^2 R^2) \cdot 1/\poly(n)\\
        = &~ 1/\poly(n)
    \end{align*}
    The first equation comes from the definition of $\wh{Y}'_1$ and $\wh{Y}_1$, the second inequality is due to Lemma~\ref{lem:error_analysis_ffn} and the last equation is due to $c = O(\log n)$ and $R = O(\sqrt{\log n})$.

    Then, we can show that
    \begin{align*}
        \| \wt{\Y}'_1 - \wt{Y}_1\|_\infty =&~ \| \mathsf{FNN}_1(\mathsf{FF}^{t_1}_1,\wh{Y}'_1,t_1)  - \mathsf{NN}_1(\F^{t_1}_1,\wh{Y}_1,t_1) \|_\infty\\
        \leq &~ O(kR^{g+6}c^3) \cdot 1/\poly(n)\\
        =&~ 1/\poly(n)
    \end{align*}
    The first equation is due to the definition of $\Y'_1$ and $\Y_1$, the second inequality comes from Lemma~\ref{lem:error_analysis_flow_matching_layer}, and the last step follows from $c = O(\log n)$ and $R = O(\sqrt{\log n})$.


    Assume that the following statement is true for $k$-th iteration (where $k < K$):
    \begin{align*}
        \|\wt{\Y}'_k - \wt{Y}_k\|_\infty \leq 1/\poly(n)
    \end{align*}
    Then, we can easily to bound
    \begin{align*}
        \| \Z'_{k+1} - \Z_{k+1}\|_\infty \leq 1/\poly(n)
    \end{align*}
    The inequality is due to Lemma~\ref{sec:error_analysis_of_phi_x_prime_phi_x} and Definition of $\Z'_{k+1}$ and $\Z_{k+1}$.

    Then, we can show that 
    \begin{align*}
        \|\mathsf{AAttC}_{k+1}(\Z'_{k+1}) -\mathsf{Attn}_{k+1}(\Z_{k+1}) \|_\infty \leq&~ O(k R^{g+1} c) \cdot 1/\poly(n)\\
        =&~ 1/\poly(n) 
    \end{align*}
    The first inequality comes from Lemma~\ref{lem:error_analysis_aattc_attn}, and the second equation is due to $c = O(\log n)$ and $R = O(\sqrt{\log n})$.

    Then we have
    \begin{align*}
        \| \wh{\Y}'_{k+1} - \wh{\Y}_{k+1}\|_\infty =&~ \|\mathsf{FFN}_{k+1}(\mathsf{AAttC}_{k+1}(\Z'_{k+1})) - \mathsf{FFN}_{k+1}(\mathsf{Attn}_{k+1}(\Z_{k+1}))\|_\infty\\
        \leq&~ O(c^2 R^2) \cdot 1/\poly(n)\\
        = &~ 1/\poly(n)
    \end{align*}
   The first equation comes from the definition of $\wh{Y}'_{k+1}$ and $\wh{Y}_{k+1}$, the second inequality is due to Lemma~\ref{lem:error_analysis_ffn} and the third equation is due to $c = O(\log n)$ and $R = O(\sqrt{\log n})$.

    Then, we can derive that
    \begin{align*}
        \| \wt{\Y}'_{k+1} - \wt{Y}_{k+1}\|_\infty =&~ \| \mathsf{FNN}_{k+1}(\mathsf{FF}^{t_{k+1}}_{k+1},\wh{Y}'_{k+1},t_{k+1})  - \mathsf{NN}_{k+1}(\F^{t_{k+1}}_{k+1},\wh{Y}_k+1,t_{k+1}) \|_\infty\\
        \leq &~ O(kR^{g+6}c^3) \cdot 1/\poly(n)\\
        =&~ 1/\poly(n)
    \end{align*}
    The first equation comes from the definition of $\Y'_{k+1} $ and $ \Y_{k+1}$, the second inequality is due to Lemma~\ref{lem:error_analysis_flow_matching_layer} and the third equation is due to $c = O(\log n)$ and $R = O(\sqrt{\log n})$.

    Then, by mathematical induction, we can get the proof.
\end{proof}



\ifdefined\isarxiv
%\section*{Acknowledgments}
\bibliographystyle{alpha}
\bibliography{ref}
\else

\bibliography{ref}
%\bibliographystyle{icml2022}
% \bibliographystyle{alpha}

\fi








\end{document}



%%%%%%%%%%%%%%%%%%%%%%%%%%%%%%%%%%%%%%%%%%%%%%%%%%%%%%%%%%%%%%%%%%%%%%%%%%%%%%%%%%%%%%%%%%%%%%%%%%%%%%%%%%%%%%%%%%%%%%%%%%%%%%%%%%%%%%%%%%%%%%%%%%%%%%%%%%%%%%%%%%%%%%%%%%%%%%%%%%%%%%%%%%%%%%%%%%%%%%%%%%%%%%%%%%%%%%%%%%%%%%%%%%%%%%%%%%%%%%%%%%%%%%%%%%%%%%%%%%%%%%%%%%%%%%%%%%%%%%%%%%%%%%%%%%%%%%%%%%%%%%%%%%%%%%%%%%%%%%%%%%%%%%%%%%%%%%%%%%%%%%%%%%%%%%%%%%%%%%%%%%%%%%%%%%%%%%%%%%%%%%%%%%%%%%%%%%%%%%%%%%%%%%%%%%%%%%%%%%%%%%%%%%%%%%%%%%%%%%%%%%%%%%%%%%%%%%%%%%%%%%
