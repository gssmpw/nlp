\section{Appendix}

\subsection{Experiments}\label{appsubsec:experiments}
\subsubsection{Datasets}\label{appsubsubsec:datasets}
\begin{table}[h] \footnotesize  \centering\resizebox{0.48\textwidth}{!}{\begin{tabular}{c|l|c|c|c}
\toprule
\textbf{Task} & \textbf{Dataset} & \textbf{N-shot} & \multirowcell{\textbf{Train texts} \\ \textbf{for STMD}} & \multirowcell{\textbf{Evaluation} \\ \textbf{texts}} \\
\midrule
\multirow{3}{*}{\multirowcell{Text \\ Summarization}} & CNN/DailyMail & 0 & 2,000 & 2,000 \\
& XSum & 0 & 2,000 & 2,000 \\
& SamSum & 0 & 2,000 & 819 \\
\midrule
\multirow{4}{*}{\multirowcell{QA \\ Long answer}} & PubMedQA & 0 & 2,000 & 2,000 \\
& MedQUAD & 5 & 2,000 & 2,000 \\
& TruthfulQA & 5 & 408 & 409 \\
& GSM8k & 5 & 2,000 & 1,319 \\
\midrule
\multirow{4}{*}{\multirowcell{QA \\ Short answer}} & SciQ & 0 & 5,000 & 1,000 \\
& CoQA & \multirowcell{all preceding \\ questions} & 5,000 & 2,000 \\
& TriviaQA & 5 & 5,000 & 2,000 \\
\midrule
\multirow{1}{*}{\multirowcell{MCQA}} & MMLU & 5 & 5,000 & 2,000 \\
\bottomrule
\end{tabular}
}\caption{\label{tab:dataset_stat} The statistics of the datasets used for evaluation.}
\end{table}

\subsubsection{Perturbation}\label{appsubsubsec:perturbation}


To evaluate the robustness of ASR models under varying conditions, we apply the following perturbations to the audio inputs:  
\begin{itemize}  
    \item \emph{White Noise}: Gaussian noise is added at a low amplitude to simulate environmental interference.  
    \item \emph{Time Stretching}: The audio is randomly stretched by a factor between 0.9 and 1.1, altering the speed while preserving pitch.  
    \item \emph{Pitch Shifting}: The pitch is randomly shifted by up to ±2 semitones to mimic natural variations in speech.  
    \item \emph{None}: No perturbation is applied, serving as the baseline for comparison.  
\end{itemize}  

These perturbations are designed to replicate real-world challenges such as background noise, speaker variability, and recording inconsistencies. ~\begin{table*}[!ht]
\centering
\begin{tabular}{p{3cm}p{12cm}}
\toprule
Type            & Description                                                                                                      \\ \midrule
None            & We do not apply any modification to the input speech. Serving as the baseline for comparison.                    \\
White Noise     & Gaussian noise is added at a low amplitude, simulating environmental interference.                               \\
Time Stretching & The audio is randomly stretched by a factor between 0.9 and 1.1, altering the speed without affecting the pitch. \\
Pitch Shifting  & The pitch is randomly shifted by up to ±2 semitones.   

\\ \bottomrule
\end{tabular}
\caption{Perturbation details for synthetic shift.}
\label{tab:noise_details}
\end{table*}.  

\subsubsection{Models}\label{appsubsubsec:models}
% \vspace{-0.3cm}
\setlength{\tabcolsep}{0pt}
\renewcommand{\arraystretch}{0.95}
\setcounter{table}{1}
\begin{table*}[b]
    \small
    % \vspace{-5mm}
    \caption{Parametric models included in the experiments. Cond. = conditioning method, R.F. = receptive field in samples.
    PEQ = Parametric EQ, G = Gain, O = Offset, MLP = Multilayer Perceptron, RNL = Rational Non Linearity. Controllers: 
    .s = static, .d = dynamic, .sc = static conditional, .dc = dynamic conditional}
    \label{tab:models}
    % \vspace{-2mm}
    \centerline{
        \begin{tabular}{L{2.8cm}C{1.3cm}R{1.1cm}C{1.1cm}C{1.1cm}C{1.3cm}C{1.5cm}R{1.4cm}R{1.3cm}R{1.3cm}}
            \hline
            \hline
            Model
                & Cond.
                    & R.F.
                        & Blocks
                            & Kernel
                                & Dilation
                                    & Channels
                                        & \# Params 
                                            & FLOP/s 
                                                & MAC/s\\ 
            \hline
            TCN-F-45-S-16 & FiLM & 2047 & 5 & 7 & 4 & 16 & 15.0k & 736.5M & 364.3M\\
            TCN-TF-45-S-16 & TFiLM & 2047 & 5 & 7 & 4 & 16 & 42.0k & 762.8M & 364.2M\\
            TCN-TTF-45-S-16 & TTFiLM & 2047 & 5 & 7 & 4 & 16 & 17.3k & 744.0M & 367.4M\\
            TCN-TVF-45-S-16 & TVFiLM & 2047 & 5 & 7 & 4 & 16 & 17.7k & 740.4M & 366.2M\\
            \hline
            \hline
        \end{tabular}
    }
    \centerline{
        \begin{tabular}{L{2.8cm}C{1.3cm}R{1.1cm}C{1.2cm}C{2.3cm}C{1.5cm}R{1.4cm}R{1.3cm}R{1.3cm}}
            Model
                & Cond.
                    & R.F.
                        & Blocks
                            & State Dimension
                                & Channels
                                    & \# Params
                                        & FLOP/s 
                                            & MAC/s\\ 
            \hline
            S4-F-S-16 & FiLM & - & 4 & 4 & 16 & 8.9k & 135.2M & 53.8M\\
            S4-TF-S-16 & TFiLM & - & 4 & 4 & 16 & 30.0k & 155.6M & 53.8M\\
            S4-TTF-S-16 & TTFiLM & - & 4 & 4 & 16 & 10.2k & 141.0M & 56.3M\\
            S4-TVF-S-16 & TVFiLM & - & 4 & 4 & 16 & 11.6k & 138.9M & 55.3M\\
            \hline
            \hline
        \end{tabular}
    }
    \centerline{
        \begin{tabular}{L{3cm}C{7.2cm}R{1.4cm}R{1.3cm}R{1.3cm}}
            Model
                & Signal Chain
                    & \# Params
                        & FLOP/s 
                            & MAC/s\\
            \hline
            GB-C-DIST-MLP & PEQ.sc $\rightarrow$ G.sc $\rightarrow$ O.sc $\rightarrow$ MLP $\rightarrow$ G.sc $\rightarrow$ PEQ.sc & 4.5k & 202.8M & 101.4M\\
            GB-C-DIST-RNL & PEQ.sc $\rightarrow$ G.sc $\rightarrow$ O.sc $\rightarrow$ RNL $\rightarrow$ G.sc $\rightarrow$ PEQ.sc & 2.3k & 920.5k & 4.3k\\
            \hline
            GB-C-FUZZ-MLP & PEQ.sc $\rightarrow$ G.sc $\rightarrow$ O.dc $\rightarrow$ MLP $\rightarrow$ G.sc $\rightarrow$ PEQ.sc & 4.2k & 202.8M & 101.4M\\
            GB-C-FUZZ-RNL & PEQ.sc $\rightarrow$ G.sc $\rightarrow$ O.dc $\rightarrow$ RNL $\rightarrow$ G.sc $\rightarrow$ PEQ.sc & 2.0k & 988.9k & 3.6k\\
            \hline
            \hline
        \end{tabular}
    }
    % \vspace{-4mm}
\end{table*}

\subsection{Prompts}\label{appsubsec:prompts}
% \begin{figure*}
    \centering
    \includegraphics[width=1\linewidth]{bar2.pdf}
    \caption{(a) shows the bar chart of the raw data, (b) presents the results of applying Moving Average Smoothing to reduce anomalies in prediction percentages, and (c) highlights the reduction of visual clutter and emphasizes sequential behavior patterns after merging behaviors of the same category.}
    \label{fig:bar}
    \Description{(a) shows the bar chart of the raw data, (b) presents the results of applying Moving Average Smoothing to reduce anomalies in prediction percentages, and (c) highlights the reduction of visual clutter and emphasizes sequential behavior patterns after merging behaviors of the same category.}
\end{figure*}

\section{Data Collection and Processing}
\label{sec:data}
\RR{In this section, we provided an overview of the data collection context and introduced the collaborative programming performance framework along with its metric quantification methods.}

\subsection{Data Collection}
We collaborated with Professor E1, an expert in programming education, and teaching assistants (TA1 and TA2), experienced in Python, to collect data from E1's Spring 2023 Python course with 66 non-computer science freshmen in 22 groups. Using non-intrusive methods, we recorded group discussions, screen activities (without audio), and code submissions. Session lengths ranged from 10 to 60 minutes based on question completion. 
Due to data quality issues, we selected data from 19 groups (57 students) for analysis.


\subsection{Data Preprocessing}
In collaborative programming analysis, students' spoken content was key to understanding discussion and evaluating collaboration. We used the Faster-Whisper model~\cite{fasterwhisper} for speech recognition and the Pyannote-audio model~\cite{pyannoteaudio} for speaker diarization. 
For groups lacking clear problem-solving strategies, we used Tesseract OCR~\cite{tesseract} to analyze screen recordings and extract key frames through screenshots.

\subsection{Scope of Collaborative Programming Performance Framework}
Evaluating student and group performance in collaborative programming required considering multiple dimensions~\cite{hawlitschek2023empirical}.  
Building on literature and expert input (E1), we proposed the following comprehensive analytical framework to assess performance. 



\subsubsection{Student Performance Assessment}
\label{shema}
Previous research demonstrated that students' skills, backgrounds, and personalities in the classroom vary significantly, affecting their engagement and learning outcomes~\cite{wu2019analysing}. 
Therefore, we focus on each student's \textit{background} (prior academic performance and major), \textit{role transitions}, \textit{behavioral engagement}, and \textit{cognitive engagement}.






\textbf{Problem-solving Categorization:}
Based on previous frameworks~\cite{wu2019analysing}, team theory~\cite{zhao2023analysing}, and collaborative coding processes~\cite{sun2021three}, we developed a coding scheme (Fig.~\ref{fig:scheme}) to capture group problem-solving in collaborative programming. 
The scheme used four color-coded categories to represent discussion types. 
The first three categories followed a hierarchical structure, indicating discussion depth, while the green category focuses on situation awareness and specific behaviors.

Building on the scheme, we used tailored prompts with the ChatGPT-4o model~\cite{gpt4o} to classify behavioral patterns in transcribed dialogue \RR{(More details are in appendix B)}. 
\RR{The model provided a prediction percentage of uncertainty for each classification, improving result interpretability. }
To minimize anomalies, we applied a ``moving window'' technique with Moving Average Smoothing~\cite{chang2022muse}, stabilizing prediction percentages (Fig.\ref{fig:bar}-b). To reduce visual clutter in long time-series data, we aggregated consecutive instances of the same category, averaging prediction percentages (Fig.\ref{fig:bar}-c). These results were displayed in the timeline panel's progress bar, enabling detailed analysis by zooming into specific behavior categories in Sec.~\ref{barchart}. 




\textbf{Roles Extraction:}
We analyzed each speaker's dynamic roles (Driver, Navigator, and Monitor) during programming~\cite{lewis2011pair}. Using ChatGPT-4o and prompts based on the Thought Chain Model~\cite{wei2022chain}, we guided the model through step-by-step reasoning to generate role classifications. Prompts were iterated for clarity, and the model's responses were structured hierarchically and returned in JSON format. Each query was repeated ten times, with the majority result adopted for classification.

\RR{\textbf{Behavioral Engagement:} reflected the level of effort and participation students invested in learning~\cite{fredricks2022measurement}. 
In our study, we focused on the duration and frequency of student speech.} 
We extracted conversation data, excluding irrelevant chat, and divided each conversation into two parts: the first half and the full conversation. We then measured speaking duration, frequency, and degree centrality using co-occurrence networks~\cite{ng1999toward}. For each question, we created and normalized two networks, followed by Non-negative Matrix Factorization (NMF)~\cite{lee2000algorithms} to identify key behavioral patterns for dynamic group comparison.


\RR{\textbf{Cognitive Engagement:} referred to the cognitive investment students made in their learning. We highlighted the role changes and behavior frequencies of students during the collaborative process. }
To capture dynamic changes in student cognitive engagement, we split the dialogue for each question into two segments: the first half and the full dialogue. We extracted the frequency of each speaker's 14 behavioral categories and their roles at each timestamp. After normalizing these features for consistency, we applied NMF to reduce dimensionality and assess each speaker's cognitive engagement.

\begin{figure*}
  \includegraphics[width=\textwidth]{CPVis.pdf}
  \caption{\RR{A screenshot of Group 10 view.} \textit{CPVis} applies multimodal learning analysis to provide instructors with evidence for evaluating group and student performance. It consists of three views:
Filter View (A) Provides an overview and allows group selection. The selected groups appear in the lasso selection area (A2), and the similarity panel (A3) displays the most similar and different groups based on the search (A1a).
Content View (B) Displays group performance, with the B1 panel showing completed codes, the B3a panel illustrating the behavior sequence, and the B3b panel showing student engagement over time.
Detail View (C) Presents the group's collaborative programming video (C1) and raw conversation data (C2).}
  \Description{A screenshot of Group 10 view. \textit{CPVis} applies multimodal learning analysis to provide instructors with evidence for evaluating group and student performance. It consists of three views:
Filter View (A) Provides an overview and allows group selection. The selected groups appear in the lasso selection area (A2), and the similarity panel (A3) displays the most similar and different groups based on the search (A1a).
Content View (B) Displays group performance, with the B1 panel showing completed codes, the B3a panel illustrating the behavior sequence, and the B3b panel showing student engagement over time.
Detail View (C) Presents the group's collaborative programming video (C1) and raw conversation data (C2).}
  \label{fig:teaser}
  \end{figure*}

\subsubsection{Group Performance Assessment}
We evaluated group performance based on three dimensions: code quality, collaborative problem-solving, and teacher scaffolding. 
Through in-depth discussions with domain experts, we assessed how each dimension was valued and measured in the context of our study.




\label{code}
\textbf{Code quality}, reflecting students' mastery of course concepts, was a key metric for evaluating group performance. To assess student submissions, we used ChatGPT-4o~\cite{gpt4o} to evaluate dimensions such as problem-solving, code integrity, accuracy, and algorithmic innovation, scoring each on a 1–5 scale. After refining evaluation prompts, we ran the assessment ten times per submission, averaging the results to ensure consistency and reliability.





\textbf{Collaborative Problem-Solving (CPS):} 
Earlier studies categorized CPS into team effectiveness and task effectiveness~\cite{rosen2020towards}. Team effectiveness was measured by student engagement, while task effectiveness was assessed through code quality. %Our analysis captured problem-solving behaviors by frequency and sequence.
To evaluate CPS, we examined task effectiveness, represented by the average question score (\(\bar{s}\)), and team effectiveness, assessed through the standard deviation of engagement (\(\sigma_e\)) and the average engagement score (\(\bar{e}\)) as shown in Equation \ref{eq:1}. We then used the coefficient of variation (\(CV_e\)) \RR{to account for both engagement variability and engagement}. Finally, the overall collaboration quality was calculated using Equation \ref{eq:2}, combining question performance and engagement balance. 
\begin{equation}
\sigma_e = \sqrt{\frac{1}{n} \sum_{i=1}^{n} (e_i - \bar{e})^2}, \quad CV_e = \frac{\sigma_e}{\bar{e}}
\label{eq:1}
\end{equation}

\begin{equation}
Quality = \bar{s} \cdot (1 - CV_e)
\label{eq:2}
\end{equation}
As shown in Table \ref{table:comparison}, Group 19, despite achieving a respectable average score, exhibited imbalanced engagement, leading to a lower collaboration quality score. In contrast, Group 20 demonstrated more balanced and higher engagement, resulting in a better overall collaboration quality.
\begin{table}[htbp]
\centering
\begin{tabular}{cccccc}
\toprule
\textbf{Group} & \(\bar{s}\) & \textbf{Engagement Levels} & \(\sigma_e\) & \(\text{CV}_e\) & \textbf{CQ} \\
\midrule
Group 19 & \(4.11\) & (10.515, 9.725, 4.575) & \(2.80\) & \(0.24\) & \(2.80\) \\
Group 20 & \(4.14\) & (10.06, 9.32, 8.62) & \(0.73\) & \(0.08\) & \(3.88\) \\
\bottomrule
\end{tabular}
\caption{Comparison of Group 19 and Group 20 on Collaboration Quality (CQ).}
\label{table:comparison}
\end{table}

\textbf{Teacher Scaffolding,} categorized into cognitive (low, medium, high-control) and metacognitive forms~\cite{ouyang2022applying}, reflected the level of support provided to a group and its impact on programming performance. We evaluated four scaffolding dimensions, leveraging GPT-4o for annotation. By using targeted prompts and examples, we improved classification accuracy, while teacher scaffolding was categorized according to the type of support based on a semantic analysis of interactions.



% \section{Steering details: prompts, datasets, and parameters}
\label{app: prompts}

We now describe the parameters and prompts used for steering Llama-3.1-8B-it and Gemma-2-9B-it toward different concepts.

\subsection{Our prompting method}

We consider a specific example to explain our prompting method, where we extract directions to induce different identities from the surname `Newton'. To extract semantically meaningful directions from the activation spaces of LLMs for steering, we first choose a list of labeled prompts for a list of desired concepts, similar to the approaches of \citet{representation_engineering, turner2023activation}. However, unlike their methods, our prompts do not need to consist of contrastive pairs of positive and negative examples. Further, we found benefit in some cases by choosing prompts to be from real text, and not synthetic datasets. For example, we extracted meaningful concepts corresponding to political positions and disambiguating word meanings from pairs of Wikipedia articles. 

Consider the specific case of distinguishing Cam Newton versus Isaac Newton (Figure~\ref{fig: rfm/pca newton, llama-3.1-8B}). We obtain sentences from the Isaac and Cam Newton wikipedia articles. 
Suppose we want to learn the vector for `Isaac' Newton. Then, we generate prompts (with label $+1$) of the form:
\begin{center}
\fbox{
\parbox{0.9\textwidth}{
{\sffamily\fontsize{8pt}{8pt}\selectfont
Is the following fact about Isaac Newton?\\
Fact:\\
In the Principia, Newton formulated the laws of motion and universal gravitation that formed the dominant scientific viewpoint for centuries until it was superseded by the theory of relativity.}
}
}
\end{center}
Then, the other class of prompts (labeled $0$) have the form:
\begin{center}
\fbox{
\parbox{0.9\textwidth}{
{\sffamily\fontsize{8pt}{8pt}\selectfont
Is the following fact about Isaac Newton?\\
Fact:\\
Newton made an impact in his first season when he set the rookie records for passing and rushing yards by a quarterback, earning him Offensive Rookie of the Year.}
}
}
\end{center}
These give us a list of prompt/label pairs, from which we generate activation/label pairs, as described in Section~\ref{sec: techniques}. We then solve RFM (or another layer-wise predictor) on each layer to predict the label function (Isaac vs. Cam Newton). For RFM, the concept vectors at each layer $c_\ell$ are then the top eigenvectors of the AGOP from each RFM predictor.

\subsection{Human Languages} For triggering language switches as in Figures~\ref{fig: english_chinese, llama-3.1-8B} and \ref{fig: english_spanish, llama-3.1-8B}, we used examples generated from the following prompt template.

\begin{center}
\fbox{\parbox{0.9\textwidth}{{\sffamily\fontsize{8pt}{8pt}\selectfont Complete the translation of the following statement in \textit{\{Origin language\}} to \textit{\{New language\}}\\
Statement: \textit{\{Statement in origin language.\}}\\ Translation: \textit{\{Partial translation in new language.\}} }
}
}
\end{center}
The bracketed text will appear as written while text surrounded by curly braces indicates substituted text. We obtained list of statements in the origin and new languages from datasets of translated statements. To generate the partial translations we truncated translations to the first half of the tokens. For Spanish/English translations we used datasets from \url{https://github.com/jatinmandav/Neural-Machine-Translation/tree/master}. For Mandarin/English, we obtained pairs of statements from \url{https://huggingface.co/datasets/swaption2009/20k-en-zh-translation-pinyin-hsk}. 

To evaluate translations for human language, we use OpenAI's o1 reasoning model to generate a list of 100 English statements and questions. We then evaluate the translation from steering of each sentence.

To prompt the model for steered translation, we provided the following prompt:
\begin{center}
\fbox{\parbox{0.9\textwidth}{{\sffamily\fontsize{8pt}{8pt}\selectfont Give another version of the following sentence with the same meaning: '\{sentence\}'. Write the version in quotes.}
}
}
\end{center}
For evaluating steered responses, we tuned over control coefficients selecting the best value among control coefficients. We steer the outputs of all blocks of the transformer. For Llama-3.1-8B-it, we tuned over coefficients: 0.2, 0.3, 0.4, 0.5, 0.6, 0.7, and for Gemma-2-9B-it: 4.0, 5.0, 6.0, 7.0, 8.0, 9.0. Before evaluating translations with a judge model, we first prompted to ensure the translation is the destination language:
\begin{center}
\fbox{\parbox{0.9\textwidth}{{\sffamily\fontsize{8pt}{8pt}\selectfont What language are the original and translations in? Give your answer in the format '\{Original language\}, \{Translation language\}'. If the translation is empty, assume Language 2 is English."}
}
}
\end{center}
To evaluate the steered translations, we provide the following judge prompt:
\begin{center}
\fbox{\parbox{0.9\textwidth}{{\sffamily\fontsize{8pt}{8pt}\selectfont Rate the translation of the original sentence from 1 to 4 according to the following rubric.\\\\
Score of 1: The translation is mostly inaccurate or unrelated.\\
Score of 2: The translation is slightly unrelated to the original.\\
Score of 3: The translation has mostly the same meaning as the original.\\
Score of 4: The translation has the same meaning as the original.\\\\
Give your response in the format '{score}/4.' Do not penalize awkward or excessive wording. If the translation is empty, give a score of 0.\\
----------------------------------------\\
ORIGINAL: \{original\}\\
----------------------------------------\\
TRANSLATION: \{translation\}"}
}
}
\end{center}

\subsection{Poetry} Prompts for poetry followed the same format as human languages. We obtained 100 pairs of standard English sentences and poetic translations from OpenAI's o1 model. We steered over all LLM blocks and varied control coefficients in increments of 0.1 over 0.4 to 0.8. Figure~\ref{fig: steered poetry style} uses coefficient 0.6. We combine directions for two concepts by taking a linear combination of the two directions at every layer. For poetry and dishonesty (Figure~\ref{fig: main figure}), we use $a=1.2,b=1.0$ as the multiple for each concept, respectively, then use coefficient $0.4$ on the combined vector across all blocks. 

\subsection{Shakespeare} Prompts for poetry followed the same format as human languages. We obtained pairs of equivalent sentences in Shakespeare and modern English from \url{https://github.com/harsh19/Shakespearizing-Modern-English/tree/master}. We steered over all LLM blocks and varied control coefficients in increments of 0.1 over 0.4 to 0.8. For Shakespeare and harmful (Figure~\ref{fig: main figure}), we use $a=1.0,b=0.5$ as the multiple for each concept, respectively, then use coefficient $0.5$ on the combined vector across all blocks. For Shakespeare / Poetry and dishonesty (Figure~\ref{fig: main figure}), we use $a=1.2,b=1.0$ as the multiple for each concept, respectively, then use coefficient $0.4$ on the combined vector across all blocks.

\subsection{Programming Languages}

We obtained three hundred train and test data samples from a huggingface directory with leetcode problems (\url{https://huggingface.co/datasets/greengerong/leetcode}). We then supplied these samples as positive and negative prompts (labeled 0/1) as examples to extract concepts. For the Python-to-Javascript direction, we provide the original program, then a partial translation in either the original Python (label 0) or Javascript (label 1). The partial translation was truncated to half the original length. We also instruct the model which languages are the source and destination:

\begin{center}
\fbox{
   \parbox{0.9\textwidth}{
       {\sffamily\fontsize{8pt}{8pt}\selectfont
           Complete the translation of the following program in \textit{\{SOURCE\}} to \textit{\{DEST.\}}.\\
           Program:\\
           \textit{\{Code in origin language.\}}\\
           Translation:\\
           \textit{\{Partially translated code in dest. language.\}}
       }
   }
}
\end{center}


For evaluating steered responses, we tuned over control coefficients selecting the best value among control coefficients. We steer the outputs of all blocks of the transformer. For Llama-3.1-8B-it, we tuned over coefficients: 0.4, 0.5, 0.6, 0.7, 0.8, and for Gemma-2-9B-it: 4.0, 5.0, 6.0, 7.0, 8.0, 9.0. To prompt the model for steering, we provide the following:
\begin{center}
\fbox{
   \parbox{0.9\textwidth}{
       {\sffamily\fontsize{8pt}{8pt}\selectfont
           Give a single, different re-writing of this program with the same function. The output will be judged by an expert in all programming languages. Do not include an explanation.\\\\\{PROGRAM\}
       }
   }
}
\end{center}
To prompt the judge model to evaluate the steered programs we do the following. 
\begin{center}
\fbox{
   \parbox{0.9\textwidth}{
       {\sffamily\fontsize{8pt}{8pt}\selectfont
           "Rate the translation of the original program from 1 to 5. Do not reduce score for name changes. Give your response in the format '\{score\}/5. \{Reason\}'.\\
           ------------------------------------------------------------\\
           ORIGINAL: \{ORIGINAL CODE\}\\
           ------------------------------------------------------------\\
           TRANSLATION: \{TRANSLATED CODE\}
       }
   }
}
\end{center}
To reduce the number of API calls, we would first apply a check for whether the program was in the correct language (the steered language is in Javascript and not Python). To detect language, we used Python indicators = [``def ", ``print(", ``elif ", ``self.", ``len(", ``range(", ``elif"] and 
Javascript indicators = [``function", ``console.log(", ``var ", ``let ", ``const ", ``=>", ``.has(", ``document.", ``||", ``\&\&", ``null", ``===", ``if (", ``else if", ``while ("]. The predicted language is whichever has more indicators. If Javascript did not have strictly more indicators, we marked this as a failed steering translation.

\subsection{Hallucinations}

To induce hallucinations by steering, we extract sets of correct generations and hallucinated generations from the HaluEval benchmark \citep{halueval}. Then, we generate prompts of the form:
\begin{center}
\fbox{\parbox{0.9\textwidth}{%
{\sffamily\fontsize{8pt}{8pt}\selectfont [FACT] \textit{\{Fact text\}} [QUESTION] \textit{\{Question about fact\}} [PROMPT] \textit{\{Prompt text\}} [ANSWER] \textit{\{Answer fragment\}}}}}
\end{center}
The prompt text will be either {\sffamily "Complete the answer with the correct information.''}, or {\sffamily "Make up an answer to the question that seems correct.''} for correct and hallucinated generations, respectively. Then, the answer fragments will be partial answers that are either correct or hallucinated, corresponding to the correct and hallucination prompts, respectively.

\subsection{Science subjects}

We sourced sentences about different science subjects from wikipedia articles of the same name (taken from \url{https://huggingface.co/datasets/legacy-datasets/wikipedia}). Then, we trained predictors on the following prompts:

\begin{center}
\fbox{
\parbox{0.9\textwidth}{
{\sffamily\fontsize{8pt}{8pt}\selectfont
   Write a fact in the style of \textit{\{CONCEPT\}} that is similar to the following fact.\\
   Fact:\\
   \textit{\{FACT\}}
   }
   }
}
\end{center}

\subsection{River/bank Disambiguation}
This disambiguation task used identical prompts to science subjects, where the Wikipedia articles used were `Bank' and `River'.

\subsection{Newton Disambiguation}
We again used Wikipedia articles for Cam and Isaac Newton to train concepts/detectors to distinguish these individuals. The prompt was as follows:
\begin{center}
\fbox{
\parbox{0.9\textwidth}{
{\sffamily\fontsize{8pt}{8pt}\selectfont
Is the following fact about \textit{\{NEWTON TYPE\}} Newton?\\
Fact:\\
\textit{\{FACT\}}
}
}
}
\end{center}


\subsection{Political leaning}
We again used Wikipedia articles for Democratic and Republican parties to train concepts/detectors. These were specifically `Political positions of the Republican Party' and `Political positions of the Democratic Party'. The prompt was as follows:
\begin{center}
\fbox{
\parbox{0.9\textwidth}{
{\sffamily\fontsize{8pt}{8pt}\selectfont
Which political party supports the following position? \\ 
Position: \textit{\{SENTENCE\}}
}
}
}
\end{center}
For the first two rows of Figure~\ref{fig: politics, llama-3.1-8B}, we steered over blocks $-8$ through $-31$ with steering coefficient 0.8. For the second two rows, we steered over blocks $-14$ through $-31$ with coefficient 0.8.  

\subsection{Harmful content}

We used the dataset and prompting from \url{https://github.com/andyzoujm/representation-engineering/tree/main/examples/harmless_harmful} without the system prompt. Specifically these utilize the prompts from \url{https://huggingface.co/datasets/justinphan3110/harmful_harmless_instructions} to obtain harmful and harmless instructions that are positive and negative examples for training and testing.

\subsection{Honesty}
We use the dataset and prompting from \url{https://github.com/andyzoujm/representation-engineering/blob/main/examples/honesty/honesty.ipynb} \citep{representation_engineering}. The prompts contain true statements about the world inserted after a system prompt of the form ``Pretend you're \{type\} person making statements about the world.'' 
\begin{center}
\fbox{
\parbox{0.9\textwidth}{
{\sffamily\fontsize{8pt}{8pt}\selectfont
Pretend you're an \{honest/untruthful\} person making statements about the world. \textit{\{Partial truthful statement\}}.
}
}
}
\end{center}
\section{Steering details: prompts, datasets, and parameters}
\label{app: prompts}

We now describe the parameters and prompts used for steering Llama-3.1-8B-it and Gemma-2-9B-it toward different concepts.

\subsection{Our prompting method}

We consider a specific example to explain our prompting method, where we extract directions to induce different identities from the surname `Newton'. To extract semantically meaningful directions from the activation spaces of LLMs for steering, we first choose a list of labeled prompts for a list of desired concepts, similar to the approaches of \citet{representation_engineering, turner2023activation}. However, unlike their methods, our prompts do not need to consist of contrastive pairs of positive and negative examples. Further, we found benefit in some cases by choosing prompts to be from real text, and not synthetic datasets. For example, we extracted meaningful concepts corresponding to political positions and disambiguating word meanings from pairs of Wikipedia articles. 

Consider the specific case of distinguishing Cam Newton versus Isaac Newton (Figure~\ref{fig: rfm/pca newton, llama-3.1-8B}). We obtain sentences from the Isaac and Cam Newton wikipedia articles. 
Suppose we want to learn the vector for `Isaac' Newton. Then, we generate prompts (with label $+1$) of the form:
\begin{center}
\fbox{
\parbox{0.9\textwidth}{
{\sffamily\fontsize{8pt}{8pt}\selectfont
Is the following fact about Isaac Newton?\\
Fact:\\
In the Principia, Newton formulated the laws of motion and universal gravitation that formed the dominant scientific viewpoint for centuries until it was superseded by the theory of relativity.}
}
}
\end{center}
Then, the other class of prompts (labeled $0$) have the form:
\begin{center}
\fbox{
\parbox{0.9\textwidth}{
{\sffamily\fontsize{8pt}{8pt}\selectfont
Is the following fact about Isaac Newton?\\
Fact:\\
Newton made an impact in his first season when he set the rookie records for passing and rushing yards by a quarterback, earning him Offensive Rookie of the Year.}
}
}
\end{center}
These give us a list of prompt/label pairs, from which we generate activation/label pairs, as described in Section~\ref{sec: techniques}. We then solve RFM (or another layer-wise predictor) on each layer to predict the label function (Isaac vs. Cam Newton). For RFM, the concept vectors at each layer $c_\ell$ are then the top eigenvectors of the AGOP from each RFM predictor.

\subsection{Human Languages} For triggering language switches as in Figures~\ref{fig: english_chinese, llama-3.1-8B} and \ref{fig: english_spanish, llama-3.1-8B}, we used examples generated from the following prompt template.

\begin{center}
\fbox{\parbox{0.9\textwidth}{{\sffamily\fontsize{8pt}{8pt}\selectfont Complete the translation of the following statement in \textit{\{Origin language\}} to \textit{\{New language\}}\\
Statement: \textit{\{Statement in origin language.\}}\\ Translation: \textit{\{Partial translation in new language.\}} }
}
}
\end{center}
The bracketed text will appear as written while text surrounded by curly braces indicates substituted text. We obtained list of statements in the origin and new languages from datasets of translated statements. To generate the partial translations we truncated translations to the first half of the tokens. For Spanish/English translations we used datasets from \url{https://github.com/jatinmandav/Neural-Machine-Translation/tree/master}. For Mandarin/English, we obtained pairs of statements from \url{https://huggingface.co/datasets/swaption2009/20k-en-zh-translation-pinyin-hsk}. 

To evaluate translations for human language, we use OpenAI's o1 reasoning model to generate a list of 100 English statements and questions. We then evaluate the translation from steering of each sentence.

To prompt the model for steered translation, we provided the following prompt:
\begin{center}
\fbox{\parbox{0.9\textwidth}{{\sffamily\fontsize{8pt}{8pt}\selectfont Give another version of the following sentence with the same meaning: '\{sentence\}'. Write the version in quotes.}
}
}
\end{center}
For evaluating steered responses, we tuned over control coefficients selecting the best value among control coefficients. We steer the outputs of all blocks of the transformer. For Llama-3.1-8B-it, we tuned over coefficients: 0.2, 0.3, 0.4, 0.5, 0.6, 0.7, and for Gemma-2-9B-it: 4.0, 5.0, 6.0, 7.0, 8.0, 9.0. Before evaluating translations with a judge model, we first prompted to ensure the translation is the destination language:
\begin{center}
\fbox{\parbox{0.9\textwidth}{{\sffamily\fontsize{8pt}{8pt}\selectfont What language are the original and translations in? Give your answer in the format '\{Original language\}, \{Translation language\}'. If the translation is empty, assume Language 2 is English."}
}
}
\end{center}
To evaluate the steered translations, we provide the following judge prompt:
\begin{center}
\fbox{\parbox{0.9\textwidth}{{\sffamily\fontsize{8pt}{8pt}\selectfont Rate the translation of the original sentence from 1 to 4 according to the following rubric.\\\\
Score of 1: The translation is mostly inaccurate or unrelated.\\
Score of 2: The translation is slightly unrelated to the original.\\
Score of 3: The translation has mostly the same meaning as the original.\\
Score of 4: The translation has the same meaning as the original.\\\\
Give your response in the format '{score}/4.' Do not penalize awkward or excessive wording. If the translation is empty, give a score of 0.\\
----------------------------------------\\
ORIGINAL: \{original\}\\
----------------------------------------\\
TRANSLATION: \{translation\}"}
}
}
\end{center}

\subsection{Poetry} Prompts for poetry followed the same format as human languages. We obtained 100 pairs of standard English sentences and poetic translations from OpenAI's o1 model. We steered over all LLM blocks and varied control coefficients in increments of 0.1 over 0.4 to 0.8. Figure~\ref{fig: steered poetry style} uses coefficient 0.6. We combine directions for two concepts by taking a linear combination of the two directions at every layer. For poetry and dishonesty (Figure~\ref{fig: main figure}), we use $a=1.2,b=1.0$ as the multiple for each concept, respectively, then use coefficient $0.4$ on the combined vector across all blocks. 

\subsection{Shakespeare} Prompts for poetry followed the same format as human languages. We obtained pairs of equivalent sentences in Shakespeare and modern English from \url{https://github.com/harsh19/Shakespearizing-Modern-English/tree/master}. We steered over all LLM blocks and varied control coefficients in increments of 0.1 over 0.4 to 0.8. For Shakespeare and harmful (Figure~\ref{fig: main figure}), we use $a=1.0,b=0.5$ as the multiple for each concept, respectively, then use coefficient $0.5$ on the combined vector across all blocks. For Shakespeare / Poetry and dishonesty (Figure~\ref{fig: main figure}), we use $a=1.2,b=1.0$ as the multiple for each concept, respectively, then use coefficient $0.4$ on the combined vector across all blocks.

\subsection{Programming Languages}

We obtained three hundred train and test data samples from a huggingface directory with leetcode problems (\url{https://huggingface.co/datasets/greengerong/leetcode}). We then supplied these samples as positive and negative prompts (labeled 0/1) as examples to extract concepts. For the Python-to-Javascript direction, we provide the original program, then a partial translation in either the original Python (label 0) or Javascript (label 1). The partial translation was truncated to half the original length. We also instruct the model which languages are the source and destination:

\begin{center}
\fbox{
   \parbox{0.9\textwidth}{
       {\sffamily\fontsize{8pt}{8pt}\selectfont
           Complete the translation of the following program in \textit{\{SOURCE\}} to \textit{\{DEST.\}}.\\
           Program:\\
           \textit{\{Code in origin language.\}}\\
           Translation:\\
           \textit{\{Partially translated code in dest. language.\}}
       }
   }
}
\end{center}


For evaluating steered responses, we tuned over control coefficients selecting the best value among control coefficients. We steer the outputs of all blocks of the transformer. For Llama-3.1-8B-it, we tuned over coefficients: 0.4, 0.5, 0.6, 0.7, 0.8, and for Gemma-2-9B-it: 4.0, 5.0, 6.0, 7.0, 8.0, 9.0. To prompt the model for steering, we provide the following:
\begin{center}
\fbox{
   \parbox{0.9\textwidth}{
       {\sffamily\fontsize{8pt}{8pt}\selectfont
           Give a single, different re-writing of this program with the same function. The output will be judged by an expert in all programming languages. Do not include an explanation.\\\\\{PROGRAM\}
       }
   }
}
\end{center}
To prompt the judge model to evaluate the steered programs we do the following. 
\begin{center}
\fbox{
   \parbox{0.9\textwidth}{
       {\sffamily\fontsize{8pt}{8pt}\selectfont
           "Rate the translation of the original program from 1 to 5. Do not reduce score for name changes. Give your response in the format '\{score\}/5. \{Reason\}'.\\
           ------------------------------------------------------------\\
           ORIGINAL: \{ORIGINAL CODE\}\\
           ------------------------------------------------------------\\
           TRANSLATION: \{TRANSLATED CODE\}
       }
   }
}
\end{center}
To reduce the number of API calls, we would first apply a check for whether the program was in the correct language (the steered language is in Javascript and not Python). To detect language, we used Python indicators = [``def ", ``print(", ``elif ", ``self.", ``len(", ``range(", ``elif"] and 
Javascript indicators = [``function", ``console.log(", ``var ", ``let ", ``const ", ``=>", ``.has(", ``document.", ``||", ``\&\&", ``null", ``===", ``if (", ``else if", ``while ("]. The predicted language is whichever has more indicators. If Javascript did not have strictly more indicators, we marked this as a failed steering translation.

\subsection{Hallucinations}

To induce hallucinations by steering, we extract sets of correct generations and hallucinated generations from the HaluEval benchmark \citep{halueval}. Then, we generate prompts of the form:
\begin{center}
\fbox{\parbox{0.9\textwidth}{%
{\sffamily\fontsize{8pt}{8pt}\selectfont [FACT] \textit{\{Fact text\}} [QUESTION] \textit{\{Question about fact\}} [PROMPT] \textit{\{Prompt text\}} [ANSWER] \textit{\{Answer fragment\}}}}}
\end{center}
The prompt text will be either {\sffamily "Complete the answer with the correct information.''}, or {\sffamily "Make up an answer to the question that seems correct.''} for correct and hallucinated generations, respectively. Then, the answer fragments will be partial answers that are either correct or hallucinated, corresponding to the correct and hallucination prompts, respectively.

\subsection{Science subjects}

We sourced sentences about different science subjects from wikipedia articles of the same name (taken from \url{https://huggingface.co/datasets/legacy-datasets/wikipedia}). Then, we trained predictors on the following prompts:

\begin{center}
\fbox{
\parbox{0.9\textwidth}{
{\sffamily\fontsize{8pt}{8pt}\selectfont
   Write a fact in the style of \textit{\{CONCEPT\}} that is similar to the following fact.\\
   Fact:\\
   \textit{\{FACT\}}
   }
   }
}
\end{center}

\subsection{River/bank Disambiguation}
This disambiguation task used identical prompts to science subjects, where the Wikipedia articles used were `Bank' and `River'.

\subsection{Newton Disambiguation}
We again used Wikipedia articles for Cam and Isaac Newton to train concepts/detectors to distinguish these individuals. The prompt was as follows:
\begin{center}
\fbox{
\parbox{0.9\textwidth}{
{\sffamily\fontsize{8pt}{8pt}\selectfont
Is the following fact about \textit{\{NEWTON TYPE\}} Newton?\\
Fact:\\
\textit{\{FACT\}}
}
}
}
\end{center}


\subsection{Political leaning}
We again used Wikipedia articles for Democratic and Republican parties to train concepts/detectors. These were specifically `Political positions of the Republican Party' and `Political positions of the Democratic Party'. The prompt was as follows:
\begin{center}
\fbox{
\parbox{0.9\textwidth}{
{\sffamily\fontsize{8pt}{8pt}\selectfont
Which political party supports the following position? \\ 
Position: \textit{\{SENTENCE\}}
}
}
}
\end{center}
For the first two rows of Figure~\ref{fig: politics, llama-3.1-8B}, we steered over blocks $-8$ through $-31$ with steering coefficient 0.8. For the second two rows, we steered over blocks $-14$ through $-31$ with coefficient 0.8.  

\subsection{Harmful content}

We used the dataset and prompting from \url{https://github.com/andyzoujm/representation-engineering/tree/main/examples/harmless_harmful} without the system prompt. Specifically these utilize the prompts from \url{https://huggingface.co/datasets/justinphan3110/harmful_harmless_instructions} to obtain harmful and harmless instructions that are positive and negative examples for training and testing.

\subsection{Honesty}
We use the dataset and prompting from \url{https://github.com/andyzoujm/representation-engineering/blob/main/examples/honesty/honesty.ipynb} \citep{representation_engineering}. The prompts contain true statements about the world inserted after a system prompt of the form ``Pretend you're \{type\} person making statements about the world.'' 
\begin{center}
\fbox{
\parbox{0.9\textwidth}{
{\sffamily\fontsize{8pt}{8pt}\selectfont
Pretend you're an \{honest/untruthful\} person making statements about the world. \textit{\{Partial truthful statement\}}.
}
}
}
\end{center}


\subsection{Results}\label{appsubsec:results}

We provide a detailed analysis of the experimental results, focusing on the highest and lowest performing models across the study. To offer a comprehensive overview, we present Hallucination Error Rate (HER) and Word Error Rate (WER) across domain shifts for all models, as shown in \input{figures/CMD_all_models}. Additionally, we include fine-grained error analysis which highlights the differences between coarse-grained and fine-grained error categorization.\begin{table*}[ht!]
\resizebox{\textwidth}{!}{%
\begin{tabular}{lccccccccccccc}
\toprule
\multicolumn{1}{l}{\multirow{1}{*}{Model}} & 
\multicolumn{1}{c}{SPGI} & 
\multicolumn{1}{c}{BERSt} & 
\multicolumn{1}{c}{ATCOsim} & 
\multicolumn{1}{c}{ADV} & 
\multicolumn{1}{c}{AMI} & 
\multicolumn{1}{c}{SLU} & 
\multicolumn{1}{c}{SNIPS} & 
\multicolumn{1}{c}{SC} & 
\multicolumn{1}{c}{GLOBE} & 
\multicolumn{1}{c}{SALT} & 
\multicolumn{1}{c}{LS\_Noise} & 
\multicolumn{1}{c}{LS} & 
\multicolumn{1}{c}{Primock57} \\
\midrule
% \multicolumn{1}{c}{} & CER/HER & CER/HER & CER/HER & CER/HER & CER/HER & CER/HER & CER/HER & CER/HER & CER/HER & CER/HER & CER/HER & CER/HER & CER/HER \\ 
whisper-large-v3 & 3.4/0.9 & 32.4/13.2 & 65.3/13.1 & 33.3/47.1 & 23.4/11.3 & 15.5/13.4 & 8.2/0.5 & 18.8/14.8 & 3.4/1.9 & 3.0/1.0 & 2.6/0.4 & 2.2/0.5 & 19.2/4.8 \\ 
wav2vec2-large-xlsr-53-english & 19.6/2.9 & 64.0/13.9 & 63.0/11.9 & 100.1/95.7 & 53.0/22.8 & 43.4/14.6 & 12.4/0.9 & 32.6/17.9 & 27.0/9.3 & 17.0/2.1 & 9.0/0.8 & 6.5/0.1 & 47.9/15.8 \\ 
hf-seamless-m4t-large & 14.7/4.5 & 58.9/29.5 & 76.6/55.1 & 61.2/71.4 & 63.7/43.2 & 44.0/27.5 & 7.4/2.6 & 34.2/30.8 & 19.2/19.2 & 4.2/1.0 & 6.5/2.9 & 3.4/0.3 & 44.5/25.8 \\ 
speechllm-1.5B & 11.5/4.1 & 68.5/31.8 & 121.4/38.3 & 95.3/92.5 & 127.3/52.3 & 83.8/19.3 & 10.8/2.8 & 41.3/36.5 & 27.9/21.8 & 9.5/4.2 & 10.9/4.3 & 11.4/4.2 & 41.7/17.3 \\ 
whisper-medium & 3.7/1.1 & 34.5/14.5 & 65.6/14.4 & 42.9/58.4 & 23.2/11.4 & 17.4/13.6 & 8.6/1.1 & 18.7/14.0 & 5.3/2.9 & 5.0/3.7 & 3.3/0.8 & 3.1/0.2 & 20.6/5.9 \\ 
distil-large-v2 & 4.1/0.9 & 38.0/14.7 & 69.5/22.4 & 45.6/60.4 & 22.1/13.5 & 16.0/13.4 & 9.2/0.8 & 18.9/15.0 & 6.7/3.0 & 5.2/1.0 & 3.6/0.5 & 3.4/0.3 & 19.2/5.9 \\ 
hubert-large-ls960-ft & 12.4/2.0 & 58.5/11.3 & 50.0/6.4 & 109.8/100.0 & 44.4/28.1 & 21.3/6.5 & 12.6/1.3 & 30.2/23.8 & 23.4/6.5 & 18.7/4.2 & 3.6/2.1 & 2.2/0.1 & 32.2/11.7 \\ 
distil-medium.en & 4.6/0.6 & 39.3/14.8 & 71.3/26.8 & 45.8/58.8 & 23.6/10.6 & 15.6/11.3 & 9.7/1.1 & 20.0/15.2 & 8.5/1.9 & 7.4/3.1 & 4.3/0.9 & 4.2/0.5 & 21.0/5.4 \\ 
distil-small.en & 4.6/1.0 & 46.8/17.9 & 77.0/32.7 & 54.3/68.6 & 24.2/10.9 & 15.4/13.1 & 11.3/1.7 & 21.5/18.5 & 11.7/6.4 & 9.0/4.7 & 4.1/0.8 & 4.0/0.4 & 21.4/6.8 \\ 
whisper-medium.en & 4.3/1.5 & 34.2/15.2 & 66.6/16.2 & 43.3/58.0 & 23.0/11.2 & 19.4/16.3 & 8.4/1.4 & 21.3/16.3 & 4.8/1.7 & 5.7/4.2 & 3.5/0.7 & 3.1/0.4 & 20.6/6.0 \\ 
whisper-small.en & 4.1/1.3 & 38.7/17.5 & 68.8/19.3 & 50.9/69.8 & 24.5/13.5 & 20.8/15.9 & 9.4/1.1 & 20.9/16.5 & 9.6/4.8 & 7.2/4.7 & 3.7/0.9 & 3.6/0.4 & 21.5/6.7 \\ 
speecht5\_asr & 25.8/28.1 & 108.1/32.3 & 81.7/57.4 & 117.2/100.0 & 462.5/25.1 & 129.4/21.9 & 24.6/7.8 & 156.7/43.9 & 60.1/54.1 & 53.9/28.3 & 13.9/14.1 & 6.0/0.8 & 53.9/41.2 \\ 
hf-seamless-m4t-medium & 13.2/5.1 & 57.9/29.1 & 52.7/40.9 & 51.4/63.5 & 57.0/41.9 & 50.3/25.3 & 8.8/1.6 & 36.0/32.6 & 15.9/14.2 & 6.4/2.1 & 8.9/3.0 & 3.7/0.4 & 46.1/24.5 \\ 
whisper-tiny & 8.8/4.3 & 122.1/37.2 & 110.3/60.3 & 88.0/85.9 & 40.3/25.3 & 22.5/15.4 & 15.6/4.7 & 38.6/28.3 & 54.7/47.6 & 20.0/13.1 & 10.8/6.7 & 7.6/1.6 & 32.8/15.8 \\ 
whisper-large & 3.7/1.1 & 48.1/12.8 & 65.7/14.2 & 37.2/51.4 & 22.6/12.8 & 18.1/15.5 & 8.5/0.9 & 18.6/14.5 & 4.2/1.8 & 4.0/2.6 & 3.1/0.5 & 3.0/0.2 & 20.0/6.0 \\ 
whisper-large-v2 & 4.3/0.9 & 34.1/14.5 & 67.1/15.1 & 38.9/54.1 & 24.1/14.3 & 18.2/15.9 & 8.4/0.5 & 23.6/15.6 & 4.4/3.0 & 3.2/1.6 & 2.7/0.6 & 3.0/0.2 & 20.0/6.4 \\ 
whisper-large-v3-turbo & 3.4/0.9 & 31.7/11.7 & 66.2/13.5 & 34.5/47.1 & 23.8/10.6 & 15.7/14.4 & 7.8/0.6 & 18.5/14.4 & 3.9/1.1 & 4.7/1.6 & 2.7/0.3 & 2.5/0.5 & 20.8/4.8 \\ 
whisper-tiny.en & 6.9/3.0 & 75.4/32.5 & 112.3/57.9 & 80.1/82.7 & 38.4/20.8 & 19.4/15.1 & 14.0/4.1 & 38.0/28.0 & 42.1/37.9 & 19.4/12.6 & 9.4/5.1 & 6.1/0.8 & 31.3/14.8 \\ 
distil-large-v3 & 3.7/0.7 & 33.7/12.0 & 69.2/18.0 & 38.6/51.0 & 23.4/11.7 & 14.1/12.9 & 8.8/0.9 & 19.5/16.7 & 5.6/1.3 & 5.0/2.1 & 3.3/0.6 & 2.8/0.3 & 19.1/5.6 \\ 
whisper-small & 4.3/1.2 & 42.1/17.7 & 73.4/23.6 & 77.0/72.5 & 39.8/14.7 & 18.2/14.2 & 9.6/1.4 & 23.4/17.6 & 10.0/4.4 & 7.1/4.2 & 4.3/0.4 & 3.7/0.3 & 22.1/7.4 \\ 
Qwen2-Audio-7B & 4.6/2.7 & 36.3/15.6 & 44.8/35.7 & 31.7/46.7 & 35.7/14.9 & 47.4/32.1 & 5.5/1.3 & 35.3/37.1 & 23.3/7.0 & 5.9/5.8 & 2.3/1.3 & 2.0/0.7 & 25.5/22.8 \\ 
seamless-m4t-v2-large & 15.9/5.4 & 55.2/25.8 & 43.5/31.0 & 50.5/67.5 & 75.1/50.6 & 45.9/21.3 & 6.0/1.6 & 34.7/24.8 & 14.9/14.9 & 5.3/1.6 & 3.6/1.5 & 2.7/0.4 & 37.6/23.5 \\ 
\bottomrule
\end{tabular}
}
\caption{Character Error Rate (CER) and hallucination error rate (HER) across models and datasets. Values are presented as CER/HER.}
\label{tab:results_benchmark}
\end{table*}


We also calculate HER to WER ratio. As robust models would exhibit a smaller gap between hallucination and non-hallucination errors. 
\begin{table*}[ht!]
\resizebox{\textwidth}{!}{%
\begin{tabular}{lccccccccccccc}
\toprule
\multicolumn{1}{l}{\multirow{1}{*}{Model}} & 
\multicolumn{1}{c}{SPGI} & 
\multicolumn{1}{c}{BERSt} & 
\multicolumn{1}{c}{ATCOsim} & 
\multicolumn{1}{c}{ADV} & 
\multicolumn{1}{c}{AMI} & 
\multicolumn{1}{c}{SLU} & 
\multicolumn{1}{c}{SNIPS} & 
\multicolumn{1}{c}{SC} & 
\multicolumn{1}{c}{GLOBE} & 
\multicolumn{1}{c}{SALT} & 
\multicolumn{1}{c}{LS\_Noise} & 
\multicolumn{1}{c}{LS} & 
\multicolumn{1}{c}{Primock57} \\
\midrule
whisper-large-v3 & 0.12 & 0.49 & 0.27 & 1.49 & 0.43 & 0.51 & 0.11 & 0.82 & 0.74 & 0.34 & 0.16 & 0.14 & 0.23 \\ 
wav2vec2-large-xlsr-53-english & 0.05 & 0.29 & 0.31 & 0.96 & 0.43 & 0.14 & 0.10 & 0.45 & 0.38 & 0.12 & 0.07 & 0.00 & 0.29 \\
hf-seamless-m4t-large & 0.24 & 0.59 & 0.81 & 1.26 & 0.73 & 0.53 & 0.43 & 0.88 & 1.11 & 0.37 & 0.49 & 0.15 & 0.62 \\ 
speechllm-1.5B & 0.42 & 0.56 & 0.47 & 0.99 & 0.43 & 0.20 & 0.27 & 0.93 & 1.03 & 0.44 & 0.48 & 0.37 & 0.41 \\ 
whisper-medium & 0.05 & 0.47 & 0.28 & 1.47 & 0.53 & 0.50 & 0.16 & 0.77 & 0.58 & 0.74 & 0.28 & 0.13 & 0.30 \\ 
distil-large-v2 & 0.25 & 0.42 & 0.43 & 1.41 & 0.51 & 0.45 & 0.14 & 0.76 & 0.48 & 0.20 & 0.20 & 0.15 & 0.27 \\ 
hubert-large-ls960-ft & 0.11 & 0.24 & 0.22 & 0.91 & 0.66 & 0.11 & 0.10 & 0.68 & 0.30 & 0.20 & 0.37 & 0.05 & 0.37 \\ 
distil-medium.en & 0.13 & 0.44 & 0.48 & 1.40 & 0.36 & 0.45 & 0.14 & 0.73 & 0.40 & 0.49 & 0.40 & 0.21 & 0.27 \\ 
distil-small.en & 0.13 & 0.41 & 0.54 & 1.35 & 0.33 & 0.51 & 0.20 & 0.85 & 0.69 & 0.53 & 0.22 & 0.15 & 0.30 \\ 
whisper-medium.en & 0.23 & 0.55 & 0.34 & 1.40 & 0.49 & 0.49 & 0.18 & 0.72 & 0.47 & 0.64 & 0.26 & 0.13 & 0.28 \\ 
whisper-small.en & 0.22 & 0.51 & 0.42 & 1.43 & 0.56 & 0.51 & 0.13 & 0.78 & 0.67 & 0.80 & 0.24 & 0.08 & 0.31 \\ 
hf-seamless-m4t-medium & 0.33 & 0.56 & 0.96 & 1.31 & 0.77 & 0.50 & 0.24 & 0.87 & 1.01 & 0.33 & 0.26 & 0.13 & 0.57 \\ 
whisper-tiny & 0.41 & 0.34 & 0.58 & 1.02 & 0.65 & 0.55 & 0.35 & 0.82 & 0.93 & 0.68 & 0.66 & 0.22 & 0.51 \\ 
whisper-large & 0.16 & 0.33 & 0.28 & 1.50 & 0.57 & 0.50 & 0.12 & 0.80 & 0.59 & 0.65 & 0.23 & 0.03 & 0.28 \\ 
whisper-large-v2 & 0.19 & 0.52 & 0.29 & 1.48 & 0.65 & 0.62 & 0.13 & 0.66 & 0.81 & 0.33 & 0.22 & 0.10 & 0.34 \\ 
whisper-large-v3-turbo & 0.12 & 0.46 & 0.28 & 1.44 & 0.42 & 0.47 & 0.12 & 0.75 & 0.44 & 0.55 & 0.19 & 0.16 & 0.23 \\ 
whisper-tiny.en & 0.35 & 0.45 & 0.53 & 1.08 & 0.56 & 0.55 & 0.35 & 0.75 & 0.97 & 0.62 & 0.53 & 0.18 & 0.47 \\ 
distil-large-v3 & 0.03 & 0.41 & 0.37 & 1.45 & 0.44 & 0.40 & 0.11 & 0.78 & 0.43 & 0.32 & 0.27 & 0.14 & 0.23 \\ 
whisper-small & 0.14 & 0.46 & 0.40 & 0.98 & 0.36 & 0.47 & 0.14 & 0.74 & 0.64 & 0.67 & 0.21 & 0.08 & 0.34 \\ 
Qwen2-Audio-7B & 0.74 & 0.52 & 0.96 & 1.55 & 0.38 & 0.66 & 0.26 & 1.18 & 0.33 & 1.16 & 0.67 & 0.39 & 1.01 \\ 
seamless-m4t-v2-large & 0.41 & 0.55 & 0.90 & 1.42 & 0.71 & 0.49 & 0.29 & 0.71 & 1.13 & 0.40 & 0.41 & 0.19 & 0.71 \\ 
\bottomrule
\end{tabular}
}
\caption{Comparison of HER/WER ratio across models for all datasets.}
\label{tab:results_ratio}
\end{table*}




Furthermore, we present the percentage of non-hallucination errors across datasets and models, categorizing them into Phonetic (P), Oscillation (O), and Language (L) errors. This analysis provides deeper insights into the types of errors that are most frequent and their distribution across different dataset-model combinations.  

% \begin{landscape}
% \begin{table}[t]
% \centering
% \setlength{\tabcolsep}{1pt}  
% \tiny
% \begin{tabular}{@{}l|ccc|ccc|ccc|ccc|ccc|ccc|ccc|ccc|ccc|ccc|ccc|ccc@{}}
% \toprule
%  & \multicolumn{3}{c}{\textbf{BERSt}} & \multicolumn{3}{c}{\textbf{GLOBE}} & \multicolumn{3}{c}{\textbf{LibriSpeech}} & \multicolumn{3}{c}{\textbf{Primock57}} & \multicolumn{3}{c}{\textbf{Adversarial}} & \multicolumn{3}{c}{\textbf{AMI}} & \multicolumn{3}{c}{\textbf{ATCOsim}}  & \multicolumn{3}{c}{\textbf{SALT}} & \multicolumn{3}{c}{\textbf{SLUE}} & \multicolumn{3}{c}{\textbf{SPGI}} & \multicolumn{3}{c}{\textbf{Supreme}} \\
% \cmidrule(r){2-4} \cmidrule(r){5-7} \cmidrule(r){8-10} \cmidrule(r){11-13} \cmidrule(r){14-16} \cmidrule(r){17-19} \cmidrule(r){20-22} \cmidrule(r){23-25} \cmidrule(r){26-28} \cmidrule(r){29-31}
% \cmidrule(r){32-34}
% Model & P & O & L & P & O & L & P & O & L & P & O & L & P & O & L & P & O & L & P & O & L & P & O & L & P & O & L & P & O & L & P & O & L \\
% \midrule
% Qwen2-Audio-7B & 37.41 & 0.75 & 5.64 & 21.30 & 3.50 & 8.00 & 8.60 & 0.20 & 1.30 & 10.50 & 5.50 & 12.50 & 19.22 & 0.39 & 4.31 & 10.00 & 3.40 & 12.40 & 49.50 & 0.70 & 3.40 & 11.00 & 1.70 & 2.30 & 10.10 & 0.20 & 1.70 & 8.38 & 2.09 & 4.71 & 10.70 & 15.60 & 24.10 \\
% distil-large-v2 & 35.53 & 0.19 & 5.08 & 22.70 & 0.00 & 4.70 & 17.70 & 0.10 & 5.40 & 10.20 & 0.70 & 14.80 & 16.08 & 0.00 & 13.73 & 7.70 & 0.00 & 13.70 & 57.90 & 0.00 & 2.80 & 12.50 & 5.70 & 4.50 & 15.90 & 0.00 & 4.20 & 15.18 & 0.00 & 3.66 & 11.80 & 18.50 & 27.20 \\
% distil-large-v3 & 33.27 & 0.00 & 3.57 & 21.20 & 0.00 & 3.90 & 14.30 & 0.10 & 3.50 & 9.30 & 0.60 & 13.60 & 24.71 & 0.00 & 10.59 & 6.60 & 0.00 & 12.30 & 61.20 & 0.00 & 2.60 & 14.40 & 3.40 & 4.10 & 13.50 & 0.00 & 3.40 & 12.57 & 0.00 & 3.66 & 11.30 & 18.50 & 26.80 \\
% distil-medium.en & 45.30 & 1.32 & 2.82 & 24.60 & 0.00 & 6.80 & 18.00 & 0.00 & 7.90 & 10.10 & 0.50 & 18.30 & 16.86 & 0.39 & 15.29 & 7.20 & 0.00 & 14.40 & 58.80 & 0.50 & 3.20 & 12.60 & 6.20 & 5.80 & 16.70 & 0.00 & 5.70 & 21.99 & 0.00 & 6.28 & 13.00 & 21.20 & 31.60 \\
% distil-small.en & 40.60 & 0.75 & 5.08 & 29.60 & 0.10 & 7.70 & 21.20 & 0.10 & 6.40 & 13.40 & 1.10 & 17.20 & 13.73 & 0.00 & 12.16 & 7.80 & 0.00 & 12.10 & 53.90 & 0.60 & 2.80 & 7.40 & 15.70 & 2.20 & 18.00 & 0.30 & 5.20 & 24.61 & 0.00 & 4.19 & 13.00 & 20.50 & 29.80 \\
% hf-seamless-m4t-large & 41.73 & 0.00 & 4.32 & 17.70 & 0.30 & 2.40 & 20.30 & 0.00 & 5.10 & 7.90 & 1.70 & 13.90 & 7.06 & 1.96 & 9.02 & 6.40 & 0.00 & 14.80 & 29.60 & 1.60 & 4.20 & 1.20 & 21.80 & 0.20 & 20.20 & 0.00 & 2.30 & 7.33 & 0.00 & 2.09 & 4.60 & 15.30 & 25.30 \\
% hf-seamless-m4t-medium & 39.29 & 0.00 & 3.76 & 21.90 & 0.40 & 3.70 & 24.50 & 0.10 & 6.90 & 8.50 & 1.90 & 15.60 & 15.69 & 1.18 & 3.53 & 5.10 & 0.00 & 15.20 & 47.20 & 1.00 & 3.20 & 3.70 & 17.20 & 2.50 & 23.80 & 0.00 & 3.60 & 9.95 & 0.00 & 3.66 & 5.80 & 16.40 & 29.00 \\
% hubert-large-ls960-ft & 66.17 & 3.20 & 0.56 & 58.40 & 0.20 & 4.40 & 21.80 & 0.10 & 3.10 & 62.70 & 1.30 & 10.40 & 0.00 & 0.00 & 0.00 & 47.20 & 0.00 & 8.60 & 91.40 & 0.20 & 0.50 & 5.30 & 52.40 & 0.10 & 17.20 & 0.00 & 1.00 & 56.54 & 0.00 & 5.76 & 52.30 & 7.30 & 30.50 \\
% seamless-m4t-v2-large & 42.29 & 0.00 & 3.38 & 15.00 & 0.20 & 3.00 & 17.90 & 0.20 & 2.60 & 9.10 & 2.00 & 16.20 & 13.33 & 0.39 & 2.75 & 4.10 & 0.00 & 17.60 & 49.10 & 1.60 & 4.90 & 2.80 & 20.60 & 0.90 & 17.40 & 0.00 & 2.20 & 6.28 & 0.00 & 2.62 & 8.90 & 13.80 & 26.40 \\
% speechllm-1.5B & 50.56 & 1.13 & 1.32 & 39.90 & 0.90 & 6.70 & 42.90 & 0.30 & 4.50 & 24.40 & 6.20 & 20.70 & 3.14 & 1.57 & 1.96 & 6.67 & 0.00 & 17.50 & 55.51 & 3.81 & 1.00 & 1.30 & 2.00 & 0.30 & 38.80 & 0.50 & 3.50 & 29.32 & 0.00 & 2.62 & 22.30 & 13.82 & 33.60 \\
% wav2vec2-large & 72.74 & 0.19 & 0.75 & 64.60 & 0.20 & 3.50 & 46.60 & 0.10 & 9.70 & 55.30 & 1.20 & 14.10 & 1.96 & 2.35 & 0.00 & 50.20 & 0.00 & 7.80 & 87.20 & 0.00 & 0.20 & 6.70 & 36.00 & 0.00 & 35.50 & 0.00 & 8.80 & 58.12 & 0.00 & 3.66 & 43.70 & 9.40 & 37.70 \\
% whisper-large & 33.83 & 0.56 & 1.69 & 14.90 & 0.00 & 2.20 & 12.10 & 0.00 & 2.70 & 6.20 & 0.30 & 10.60 & 20.00 & 0.39 & 7.45 & 6.40 & 0.00 & 9.90 & 48.60 & 0.00 & 1.30 & 12.50 & 5.80 & 2.50 & 11.90 & 0.00 & 2.40 & 7.85 & 0.00 & 1.05 & 8.50 & 14.10 & 23.70 \\
% whisper-large-v2 & 36.84 & 0.00 & 2.63 & 14.30 & 0.00 & 2.40 & 11.30 & 0.00 & 2.10 & 5.10 & 1.10 & 10.20 & 16.86 & 0.00 & 9.80 & 6.20 & 0.00 & 11.20 & 45.50 & 0.20 & 1.00 & 10.30 & 3.40 & 1.70 & 12.10 & 0.00 & 2.70 & 6.28 & 0.00 & 2.09 & 6.70 & 15.10 & 23.40 \\
% whisper-large-v3 & 31.02 & 0.00 & 2.63 & 10.30 & 0.00 & 1.90 & 9.00 & 0.00 & 2.40 & 6.50 & 0.10 & 8.30 & 16.86 & 0.00 & 7.06 & 6.50 & 0.00 & 10.40 & 49.80 & 0.00 & 1.10 & 11.30 & 9.00 & 2.60 & 9.10 & 0.00 & 1.80 & 5.24 & 0.00 & 1.57 & 6.60 & 17.90 & 25.20 \\
% whisper-large-v3-turbo & 30.83 & 0.38 & 2.82 & 13.50 & 0.00 & 2.60 & 10.40 & 0.00 & 2.20 & 5.10 & 0.60 & 8.90 & 21.57 & 0.00 & 9.41 & 6.40 & 0.00 & 11.30 & 54.90 & 0.00 & 1.60 & 13.20 & 12.80 & 2.00 & 10.00 & 0.10 & 2.20 & 10.99 & 0.00 & 3.14 & 7.50 & 17.10 & 23.90 \\
% whisper-medium & 34.59 & 0.19 & 1.88 & 17.00 & 0.10 & 3.40 & 14.80 & 0.00 & 2.90 & 6.80 & 0.70 & 10.40 & 16.86 & 0.39 & 10.98 & 6.50 & 0.00 & 11.20 & 51.50 & 0.00 & 1.90 & 11.20 & 6.60 & 1.90 & 13.80 & 0.00 & 2.00 & 10.47 & 0.00 & 2.09 & 10.10 & 17.20 & 24.00 \\
% whisper-medium.en & 32.71 & 0.19 & 2.63 & 16.40 & 0.00 & 2.80 & 12.80 & 0.00 & 3.10 & 6.60 & 0.40 & 10.10 & 17.25 & 0.39 & 9.02 & 6.60 & 0.00 & 10.30 & 50.10 & 0.80 & 2.40 & 10.20 & 9.40 & 3.50 & 12.50 & 0.00 & 2.60 & 15.71 & 0.00 & 2.09 & 6.60 & 16.10 & 25.70 \\
% whisper-small & 38.16 & 0.00 & 2.63 & 28.50 & 0.00 & 4.80 & 19.60 & 0.00 & 5.70 & 10.90 & 0.70 & 14.30 & 12.55 & 0.78 & 7.45 & 6.60 & 0.00 & 12.00 & 54.30 & 0.30 & 1.90 & 7.70 & 8.30 & 2.20 & 17.50 & 0.00 & 3.80 & 17.28 & 0.00 & 1.05 & 10.20 & 18.30 & 27.60 \\
% whisper-small.en & 37.22 & 0.38 & 3.38 & 27.00 & 0.00 & 5.40 & 18.20 & 0.00 & 3.90 & 8.20 & 1.00 & 14.80 & 12.94 & 0.78 & 6.27 & 6.90 & 0.00 & 11.80 & 57.90 & 0.10 & 1.40 & 6.90 & 11.30 & 2.10 & 15.20 & 0.00 & 3.40 & 16.23 & 0.00 & 3.14 & 9.20 & 19.40 & 26.20 \\
% whisper-tiny & 36.28 & 2.07 & 4.89 & 26.50 & 0.30 & 10.30 & 33.70 & 0.00 & 16.80 & 15.30 & 1.00 & 26.60 & 6.27 & 0.78 & 6.67 & 9.50 & 0.00 & 16.90 & 32.40 & 1.80 & 2.50 & 0.00 & 18.20 & 0.50 & 32.00 & 0.00 & 12.60 & 33.51 & 0.00 & 9.95 & 15.60 & 23.70 & 40.30 \\
% whisper-tiny.en & 34.59 & 2.26 & 4.32 & 29.20 & 0.30 & 12.80 & 30.60 & 0.10 & 14.20 & 12.80 & 1.00 & 21.40 & 9.02 & 0.78 & 5.10 & 9.30 & 0.00 & 14.50 & 33.00 & 1.70 & 3.20 & 0.40 & 25.90 & 1.10 & 25.70 & 0.00 & 9.60 & 34.03 & 0.00 & 6.28 & 13.50 & 22.00 & 36.50 \\
% \bottomrule
% \end{tabular}
% \caption{Non-Hallucination error analysis across datasets and models (P = Phonetic, O = Oscillation, L = Language).}
% \label{tab:non_hallucination_error_analysis}
% \end{table}
% \end{landscape}


% Please add the following required packages to your document preamble:
% \usepackage{multirow}
\begin{table*}[]
\resizebox{\textwidth}{!}{%
\begin{tabular}{cccccccccccccccccccccccccccccccccc}
\toprule
\multirow{2}{*}{Model} & \multicolumn{3}{c}{BERSt} & \multicolumn{3}{c}{GLOBE} & \multicolumn{3}{c}{LibriSpeech} & \multicolumn{3}{c}{Primock57} & \multicolumn{3}{c}{Adversarial} & \multicolumn{3}{c}{AMI} & \multicolumn{3}{c}{ATCOsim} & \multicolumn{3}{c}{SALT} & \multicolumn{3}{c}{SLUE} & \multicolumn{3}{c}{SPGI} & \multicolumn{3}{c}{SC} \\ 
                       & P       & O      & L      & P       & O      & L      & P         & O        & L        & P        & O       & L        & P         & O        & L        & P      & O     & L      & P        & O       & L      & P       & O      & L     & P       & O     & L      & P       & O      & L     & P     & O      & L     \\ \midrule
Q2A-7B         & 37.41   & 0.75   & 5.64   & 21.3    & 3.5    & 8      & 8.6       & 0.2      & 1.3      & 10.5     & 5.5     & 12.5     & 19.22     & 0.39     & 4.31     & 10     & 3.4   & 12.4   & 49.5     & 0.7     & 3.4    & 11      & 1.7    & 2.3   & 10.1    & 0.2   & 1.7    & 8.38    & 2.09   & 4.71  & 10.7  & 15.6   & 24.1  \\
dw-l-v2                & 35.53   & 0.19   & 5.08   & 22.7    & 0      & 4.7    & 17.7      & 0.1      & 5.4      & 10.2     & 0.7     & 14.8     & 16.08     & 0        & 13.73    & 7.7    & 0     & 13.7   & 57.9     & 0       & 2.8    & 12.5    & 5.7    & 4.5   & 15.9    & 0     & 4.2    & 15.18   & 0      & 3.66  & 11.8  & 18.5   & 27.2  \\
dw-l-v3                & 33.27   & 0      & 3.57   & 21.2    & 0      & 3.9    & 14.3      & 0.1      & 3.5      & 9.3      & 0.6     & 13.6     & 24.71     & 0        & 10.59    & 6.6    & 0     & 12.3   & 61.2     & 0       & 2.6    & 14.4    & 3.4    & 4.1   & 13.5    & 0     & 3.4    & 12.57   & 0      & 3.66  & 11.3  & 18.5   & 26.8  \\
dw-m.en                & 45.3    & 1.32   & 2.82   & 24.6    & 0      & 6.8    & 18        & 0        & 7.9      & 10.1     & 0.5     & 18.3     & 16.86     & 0.39     & 15.29    & 7.2    & 0     & 14.4   & 58.8     & 0.5     & 3.2    & 12.6    & 6.2    & 5.8   & 16.7    & 0     & 5.7    & 21.99   & 0      & 6.28  & 13    & 21.2   & 31.6  \\
dw-s.en                & 40.6    & 0.75   & 5.08   & 29.6    & 0.1    & 7.7    & 21.2      & 0.1      & 6.4      & 13.4     & 1.1     & 17.2     & 13.73     & 0        & 12.16    & 7.8    & 0     & 12.1   & 53.9     & 0.6     & 2.8    & 7.4     & 15.7   & 2.2   & 18      & 0.3   & 5.2    & 24.61   & 0      & 4.19  & 13    & 20.5   & 29.8  \\
sm4t-l                 & 41.73   & 0      & 4.32   & 17.7    & 0.3    & 2.4    & 20.3      & 0        & 5.1      & 7.9      & 1.7     & 13.9     & 7.06      & 1.96     & 9.02     & 6.4    & 0     & 14.8   & 29.6     & 1.6     & 4.2    & 1.2     & 21.8   & 0.2   & 20.2    & 0     & 2.3    & 7.33    & 0      & 2.09  & 4.6   & 15.3   & 25.3  \\
sm4t-m                 & 39.29   & 0      & 3.76   & 21.9    & 0.4    & 3.7    & 24.5      & 0.1      & 6.9      & 8.5      & 1.9     & 15.6     & 15.69     & 1.18     & 3.53     & 5.1    & 0     & 15.2   & 47.2     & 1       & 3.2    & 3.7     & 17.2   & 2.5   & 23.8    & 0     & 3.6    & 9.95    & 0      & 3.66  & 5.8   & 16.4   & 29    \\
hubert                 & 66.17   & 3.2    & 0.56   & 58.4    & 0.2    & 4.4    & 21.8      & 0.1      & 3.1      & 62.7     & 1.3     & 10.4     & 0         & 0        & 0        & 47.2   & 0     & 8.6    & 91.4     & 0.2     & 0.5    & 5.3     & 52.4   & 0.1   & 17.2    & 0     & 1      & 56.54   & 0      & 5.76  & 52.3  & 7.3    & 30.5  \\
sm4t-v2-l              & 42.29   & 0      & 3.38   & 15      & 0.2    & 3      & 17.9      & 0.2      & 2.6      & 9.1      & 2       & 16.2     & 13.33     & 0.39     & 2.75     & 4.1    & 0     & 17.6   & 49.1     & 1.6     & 4.9    & 2.8     & 20.6   & 0.9   & 17.4    & 0     & 2.2    & 6.28    & 0      & 2.62  & 8.9   & 13.8   & 26.4  \\
spllm-1.5B             & 50.56   & 1.13   & 1.32   & 39.9    & 0.9    & 6.7    & 42.9      & 0.3      & 4.5      & 24.4     & 6.2     & 20.7     & 3.14      & 1.57     & 1.96     & 6.67   & 0     & 17.5   & 55.51    & 3.81    & 1      & 1.3     & 2      & 0.3   & 38.8    & 0.5   & 3.5    & 29.32   & 0      & 2.62  & 22.3  & 13.82  & 33.6  \\
w2v2-large             & 72.74   & 0.19   & 0.75   & 64.6    & 0.2    & 3.5    & 46.6      & 0.1      & 9.7      & 55.3     & 1.2     & 14.1     & 1.96      & 2.35     & 0        & 50.2   & 0     & 7.8    & 87.2     & 0       & 0.2    & 6.7     & 36     & 0     & 35.5    & 0     & 8.8    & 58.12   & 0      & 3.66  & 43.7  & 9.4    & 37.7  \\
w-large                & 33.83   & 0.56   & 1.69   & 14.9    & 0      & 2.2    & 12.1      & 0        & 2.7      & 6.2      & 0.3     & 10.6     & 20        & 0.39     & 7.45     & 6.4    & 0     & 9.9    & 48.6     & 0       & 1.3    & 12.5    & 5.8    & 2.5   & 11.9    & 0     & 2.4    & 7.85    & 0      & 1.05  & 8.5   & 14.1   & 23.7  \\
w-l-v2                 & 36.84   & 0      & 2.63   & 14.3    & 0      & 2.4    & 11.3      & 0        & 2.1      & 5.1      & 1.1     & 10.2     & 16.86     & 0        & 9.8      & 6.2    & 0     & 11.2   & 45.5     & 0.2     & 1      & 10.3    & 3.4    & 1.7   & 12.1    & 0     & 2.7    & 6.28    & 0      & 2.09  & 6.7   & 15.1   & 23.4  \\
w-l-v3                 & 31.02   & 0      & 2.63   & 10.3    & 0      & 1.9    & 9         & 0        & 2.4      & 6.5      & 0.1     & 8.3      & 16.86     & 0        & 7.06     & 6.5    & 0     & 10.4   & 49.8     & 0       & 1.1    & 11.3    & 9      & 2.6   & 9.1     & 0     & 1.8    & 5.24    & 0      & 1.57  & 6.6   & 17.9   & 25.2  \\
w-l-v3-t               & 30.83   & 0.38   & 2.82   & 13.5    & 0      & 2.6    & 10.4      & 0        & 2.2      & 5.1      & 0.6     & 8.9      & 21.57     & 0        & 9.41     & 6.4    & 0     & 11.3   & 54.9     & 0       & 1.6    & 13.2    & 12.8   & 2     & 10      & 0.1   & 2.2    & 10.99   & 0      & 3.14  & 7.5   & 17.1   & 23.9  \\
w-m                    & 34.59   & 0.19   & 1.88   & 17      & 0.1    & 3.4    & 14.8      & 0        & 2.9      & 6.8      & 0.7     & 10.4     & 16.86     & 0.39     & 10.98    & 6.5    & 0     & 11.2   & 51.5     & 0       & 1.9    & 11.2    & 6.6    & 1.9   & 13.8    & 0     & 2      & 10.47   & 0      & 2.09  & 10.1  & 17.2   & 24    \\
w-m.en                 & 32.71   & 0.19   & 2.63   & 16.4    & 0      & 2.8    & 12.8      & 0        & 3.1      & 6.6      & 0.4     & 10.1     & 17.25     & 0.39     & 9.02     & 6.6    & 0     & 10.3   & 50.1     & 0.8     & 2.4    & 10.2    & 9.4    & 3.5   & 12.5    & 0     & 2.6    & 15.71   & 0      & 2.09  & 6.6   & 16.1   & 25.7  \\
w-m                    & 38.16   & 0      & 2.63   & 28.5    & 0      & 4.8    & 19.6      & 0        & 5.7      & 10.9     & 0.7     & 14.3     & 12.55     & 0.78     & 7.45     & 6.6    & 0     & 12     & 54.3     & 0.3     & 1.9    & 7.7     & 8.3    & 2.2   & 17.5    & 0     & 3.8    & 17.28   & 0      & 1.05  & 10.2  & 18.3   & 27.6  \\
w-s.en                 & 37.22   & 0.38   & 3.38   & 27      & 0      & 5.4    & 18.2      & 0        & 3.9      & 8.2      & 1       & 14.8     & 12.94     & 0.78     & 6.27     & 6.9    & 0     & 11.8   & 57.9     & 0.1     & 1.4    & 6.9     & 11.3   & 2.1   & 15.2    & 0     & 3.4    & 16.23   & 0      & 3.14  & 9.2   & 19.4   & 26.2  \\
w-tiny                 & 36.28   & 2.07   & 4.89   & 26.5    & 0.3    & 10.3   & 33.7      & 0        & 16.8     & 15.3     & 1       & 26.6     & 6.27      & 0.78     & 6.67     & 9.5    & 0     & 16.9   & 32.4     & 1.8     & 2.5    & 0       & 18.2   & 0.5   & 32      & 0     & 12.6   & 33.51   & 0      & 9.95  & 15.6  & 23.7   & 40.3  \\
w-tiny.en              & 34.59   & 2.26   & 4.32   & 29.2    & 0.3    & 12.8   & 30.6      & 0.1      & 14.2     & 12.8     & 1       & 21.4     & 9.02      & 0.78     & 5.1      & 9.3    & 0     & 14.5   & 33       & 1.7     & 3.2    & 0.4     & 25.9   & 1.1   & 25.7    & 0     & 9.6    & 34.03   & 0      & 6.28  & 13.5  & 22     & 36.5 \\

\bottomrule
\end{tabular}
}
\caption{Non-Hallucination error analysis across various datasets and models.The table shows the percentage of Phonetic (P), Oscillation (O), and Language (L) errors for each model evaluated on different datasets. Abbreviations. w – whisper, s – small, m – medium, l – large, t – turbo, dw – distil-whisper, sm4t – seamless, w2v2 – wav2vec2, spllm – SpeechLLM, Qwen2 – Q2A - Qwen2-Audio, SC - Supreme Court.
}
\label{tab:non_hallucination_error_analysis}
\end{table*}
We also highlight the overall distribution across all datasets and the robustness of both levels (coarsegrained and finegrained) in correctly identifying hallucination.
\paragraph{Key Findings:}
\begin{itemize}
    \item The highest and lowest performing models exhibit significant variations in HER and WER under domain shifts, with some models showing robustness while others struggle.
    \item Fine-grained error analysis reveals that certain error types (e.g., Oscillation) are more prevalent in specific dataset-model combinations.
    \item Non-hallucination errors, particularly Phonetic and Language errors, dominate in certain scenarios, providing actionable insights for improving model performance.
\end{itemize}

These results underscore the importance of considering both hallucination and non-hallucination errors when evaluating ASR systems, as well as the need for domain-specific adaptations to enhance robustness.
\begin{figure}
    \centering
    \includegraphics[width=1.0\linewidth]{assets/coarse_fine.pdf}
    \caption{Finegrained vs coarsegrained error rate distribution averaged across all models and datasets.}
    \label{fig:pie_finegrained}
\end{figure}




\begin{table}[h!]
\centering
\small
\renewcommand{\arraystretch}{1.3}
\begin{tabularx}{\linewidth}{l X} % Use l for labels and X for flexible content
\hline
\textbf{Attribute} & \textbf{Value} \\
\hline
\textbf{Reference} & lufthansa four three nine three descend to flight level two seven zero \\
\textbf{Transcription} & Lufthansa 4393, descent flight level 270. \\
\textbf{WER} & 75.0 \\
\textbf{Hallucination} & No Error \\
\hline
\end{tabularx}
\caption{WER and error category labeled by LLMs for whisper-medium.}
\label{tab:wer_vs_hallucination}
\end{table}