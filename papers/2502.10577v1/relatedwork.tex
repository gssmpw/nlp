\section{Related Work}
MG and the associated male bias have been extensively studied in the field of psycholinguistics for a variety of gender-marked languages \citep{silveiraGenericMasculineWords1980,stahlbergNameYourFavorite2001,safinaEffectsGrammaticalGender2024}, including French \citep{gygaxMasculineFormIts2012,richyDemelerEffetsStereotypes2021}. For example, a study conducted by \citet{gygaxGenericallyIntendedSpecifically2008} in English, French and German had participants read a sentence A with a role noun (e.g., "social workers"; MG in both French and German). Then, when presented with a sentence B with a noun referring to the members of the noun group in A (e.g., "women"), participants had to judge as quickly as possible if it was a coherent continuation of sentence A. Results showed that in English, gender representations aligned with stereotypes: female-stereotyped roles (e.g., "beauticians") led to female-biased interpretations, while male-stereotyped roles (e.g., "politicians") led to male-biased interpretations. Neutral roles showed no bias. In French and German, grammatical gender overrode stereotypes: masculine plural forms led to predominantly male-biased interpretations, even for female-stereotyped roles.

Similarly, a survey conducted by \citet{harrisinteractiveLecritureInclusivePopulation2017} regarding the use of inclusive or gender-neutral language in French highlighted the impact of such language on mental representations. It was found that respondents, when asked to name a celebrity using a formulation that was either inclusive (i.e., using both masculine and feminine forms of a role noun) or gender-neutral (i.e., using a non-gender-specific role noun or an equivalent) were more likely to name a woman as opposed to a man, while MG formulations led to more male-centric answers. Similar findings were reported by \citet{stahlbergNameYourFavorite2001} for German using MG role nouns.

Recently published studies have resorted to more precise methods such as EEG to track how language-related gender information is processed by the brain to evaluate the effects of MG. \citet{glimGenericMasculineRole2024} showed that even explicitly disambiguated
MG nouns used to refer to women led to higher cognitive load from participants for the task of noun phrase reference resolution, further indicating that the alleged genericness of MG lacks empirical backing.

When it comes to natural language processing (NLP), gender bias is by far the most studied type of bias \citep{ducelRechercheBiaisDans2024}. Numerous researchers have drawn attention to gender bias in word embeddings \citep{bolukbasiManComputerProgrammer2016}, in machine translation systems \citep{savoldiGenderBiasMachine2021,wisniewskiBiaisGenreDans2021} or in text classification tasks \citep{sobhaniFairerNLPModels2024}.

More recently, LLMs too have been shown to exhibit gender biases and convey stereotypes when generating context-restricted textual content (see \citet{kotekGenderBiasStereotypes2023} for pronoun disambiguation; \citet{dollEvaluatingGenderBias2024} for pronoun prediction; \citet{youBinaryGenderLabels2024} for neutral name prediction), including in languages other than English \citep{zhaoGenderBiasLarge2024,ducelEvaluationAutomatiqueBiais2024}. Nonetheless, in spite of the propensity for investigating gender bias in NLP and the attention given to LLMs with regard to this issue, no studies discussing MG and to what extent LLMs are prone to propagating MG-related gender bias have so far been conducted, a gap this work aims to bridge. We present our detailed methodology in the next section.