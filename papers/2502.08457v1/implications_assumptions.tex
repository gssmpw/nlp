\section{Preliminary Results}\label{sec_app:prel_res}
{In this section, $\Omega$ is an arbitrary compact subset of $\mathbb{R}^d$  with $\text{hull}(\Omega)$ denoting its convex hull, which is also compact.} 
We also consider an arbitrary fixed positive value $\Lambda$ such that $\lambda \leq \Lambda$ as this would allow us to simplify the dependence of the boundedness and Lipschitz constants on $\lambda$.   

\subsection{Boundedness and Lipschitz continuity of $h^\star_\omega$ and $\hat{h}_\omega$}
\begin{proposition}[Boundedness of $h^\star_\omega$ and $\hat{h}_\omega$]\label{prop:bound_hstaromega}
Under \cref{assump:compact,assump:convexity_lin,assump:K_bounded,assump:reg_lin_lout}, the functions $\omega\mapsto \|h^\star_\omega\|_\mathcal{H}$ and $\omega\mapsto\|\hat{h}_\omega\|_\mathcal{H}$ are bounded over $\textnormal{hull}(\Omega)$ by $\frac{B\sqrt{\kappa}}{\lambda}$, where $B\coloneqq\sup_{\omega\in\textnormal{hull}(\Omega),y\in\mathcal{Y}}\left|\partial_v \ell_{in}(\omega, 0, y)\right|>0$. Moreover, for all $\omega\in\textnormal{hull}(\Omega)$ and $x\in\mathcal{X}$, $h^\star_\omega(x)$ and $\hat{h}_\omega(x)$ take value in the compact interval $\mathcal{V}\coloneqq\left[-\frac{B\kappa}{\lambda},\frac{B\kappa}{\lambda}\right]\subset\mathbb{R}$.
\end{proposition}

\begin{proof}
\textbf{Boundedness of $\left\|h^\star_\omega\right\|_\mathcal{H}$ and $\left\|\hat{h}_\omega\right\|_\mathcal{H}$. }Let $\omega\in\text{hull}(\Omega)$. Using \cref{lem:h_min_hstar}, we know, for any $h\in\mathcal{H}$, that:
\begin{equation*}
    \left\|h-h^\star_\omega\right\|_\mathcal{H}\leq\frac{1}{\lambda}\left\|\partial_h L_{in}(\omega, h)\right\|_\mathcal{H}.
\end{equation*}
This is particularly valid for $h=0$. Thus,
\begin{equation*}
    \left\|h^\star_\omega\right\|_\mathcal{H}\leq\frac{1}{\lambda}\left\|\partial_h L_{in}(\omega, 0)\right\|_\mathcal{H}.
\end{equation*}
Using the expression of the partial derivative $\partial_{h}L_{in}$ established in \cref{prop:fre_diff_L}, we obtain:
\begin{equation*}
    \left\|h^\star_\omega\right\|_\mathcal{H}\leq\frac{1}{\lambda}\big\|\mathbb{E}_\mathbb{P}\left[\partial_v \ell_{in}(\omega, 0, y)K(x,\cdot)\right]\big\|_\mathcal{H}.
\end{equation*}
By \cref{assump:K_bounded}, $K$ is bounded by $\kappa$. Hence, Jensen's inequality yields:
\begin{equation*}
    \left\|h^\star_\omega\right\|_\mathcal{H}\leq\frac{1}{\lambda}\mathbb{E}_\mathbb{P}\Big[\left|\partial_v \ell_{in}(\omega, 0, y)\right|\left\|K(x,\cdot)\right\|_\mathcal{H}\Big]\leq\frac{\sqrt{\kappa}}{\lambda}\mathbb{E}_\mathbb{P}\Big[\left|\partial_v \ell_{in}(\omega, 0, y)\right|\Big].
\end{equation*}
By \cref{assump:compact}, $\Omega$ and $\mathcal{Y}$ are compact, which implies that $\text{hull}(\Omega)\times\mathcal{Y}$ is compact. From \cref{assump:reg_lin_lout}, we know that the function $(\omega, y)\mapsto\partial_v\ell_{in}(\omega, 0, y)$ is continuous. Given that every continuous function on a compact space is bounded, we obtain:
\begin{equation*}
    \left\|h^\star_\omega\right\|_\mathcal{H}\leq \frac{B\sqrt{\kappa}}{\lambda}<+\infty,\quad\text{where}\quad B\coloneqq\sup_{\omega\in\text{hull}(\Omega),y\in\mathcal{Y}}\left|\partial_v \ell_{in}(\omega, 0, y)\right|>0.
\end{equation*}
To prove that $\left\|\hat{h}_\omega\right\|_\mathcal{H}\leq\frac{B\sqrt{\kappa}}{\lambda}$, we follow a similar approach to that of $\left\|h^\star_\omega\right\|_\mathcal{H}\leq \frac{B\sqrt{\kappa}}{\lambda}$. More precisely, we investigate the case where the expectation is with respect to the empirical estimate $\hat{\mathbb{P}}_n$ of $\mathbb{P}$.

\textbf{$h^\star_\omega(x)$ and $\hat{h}_\omega(x)$ belong to $\mathcal{V}$. }Let $\omega\in\text{hull}(\Omega)$ and $x\in\mathcal{X}$. By the reproducing property, the Cauchy-Schwarz inequality, and \cref{assump:K_bounded}, we have:
\begin{equation*}
    \left|h^\star_\omega(x)\right|\leq\sqrt{\kappa}\left\|h^\star_\omega\right\|\quad\text{and}\quad\left|\hat{h}_\omega(x)\right|\leq\sqrt{\kappa}\left\|\hat{h}_\omega\right\|.
\end{equation*}
Using the bound on $\left\|h^\star_\omega\right\|_\mathcal{H}$ and $\left\|\hat{h}_\omega\right\|_\mathcal{H}$ already proved in the first part of this proof, we get:
\begin{equation*}
    \left|h^\star_\omega(x)\right|\leq\frac{B\kappa}{\lambda}\quad\text{and}\quad\left|\hat{h}_\omega(x)\right|\leq\frac{B\kappa}{\lambda}.
\end{equation*}
This concludes the proof.
\end{proof}

\begin{proposition}[Lipschitz continuity of $\omega\mapsto h^\star_\omega$]\label{prop:lip_hstaromega}
Under \cref{assump:compact,assump:convexity_lin,assump:K_bounded,assump:reg_lin_lout}, the function $\omega\mapsto h^\star_\omega$ is $\frac{L\sqrt{\kappa}}{\lambda}$-Lipschitz continuous on $\textnormal{hull}(\Omega)$, where $L\coloneqq\sup_{\omega\in \textnormal{hull}(\Omega),v\in\mathcal{V},y\in\mathcal{Y}}\left\|\partial_{\omega, v}^2 \ell_{in}(\omega, v, y)\right\|>0$,   and $\mathcal{V}$ is the compact interval introduced in \cref{prop:bound_hstaromega}.
\end{proposition}
\begin{proof}
To prove this proposition, we adopt the strategy of finding an upper bound for the Jacobian, which serves as the Lipschitz constant. 

Let $\omega\in \text{hull}(\Omega)$. Using \cref{prop:fre_diff_L,prop:strong_convexity_Lin}, we know that $h\mapsto L_{in}(\omega,h)$ is $\lambda$-strongly convex and Fr\'echet differentiable. Also, by \cref{prop:fre_diff_L_v}, $\partial_h L_{in}$ is Fr\'echet differentiable on $\mathbb{R}^d\times\mathcal{H}$, and, a fortiori, Hadamard differentiable. Then, by the functional implicit differentiation theorem \citep[Theorem~2.1]{petrulionyte2024functional}, the Jacobian $\partial_\omega h^\star_\omega:\mathcal{H}\to\mathbb{R}^d$ can be expressed as:
\begin{equation*}
    \partial_\omega h^\star_\omega=-\partial_{\omega, h}^2 L_{in}(\omega, h^\star_\omega)\left(\partial_h^2 L_{in}(\omega, h^\star_\omega)\right)^{-1}.
\end{equation*}
We have:
\begin{align}
    \left\|\partial_\omega h^\star_\omega\right\|_{\op}&\leq\left\|\partial_{\omega, h}^2 L_{in}(\omega, h^\star_\omega)\right\|_{\op}\left\|\left(\partial_h^2 L_{in}(\omega, h^\star_\omega)\right)^{-1}\right\|_{\op}\nonumber\\
    &\leq\frac{\left\|\partial_{\omega, h}^2 L_{in}(\omega, h^\star_\omega)\right\|_{\op}}{\lambda}\nonumber\\
    &=\frac{\Big\|\mathbb{E}_\mathbb{P}\left[\partial_{\omega, v}^2 \ell_{in}(\omega, h^\star_\omega(x), y)K(x,\cdot)\right]\Big\|_{\op}}{\lambda}\nonumber\\
    &\leq\frac{\mathbb{E}_\mathbb{P}\Big[\left\|\partial_{\omega, v}^2 \ell_{in}(\omega, h^\star_\omega(x), y)\right\|\left\|K(x,\cdot)\right\|_\mathcal{H}\Big]}{\lambda}\nonumber\\
    &\leq\frac{\sqrt{\kappa}\mathbb{E}_\mathbb{P}\Big[\left\|\partial_{\omega, v}^2 \ell_{in}(\omega, h^\star_\omega(x), y)\right\|\Big]}{\lambda},\label{eq:upp_bound_jac}
\end{align}
where the first line uses the sub-multiplicative property of the operator norm $\|\cdot\|_{\op}$, the second line stems from the fact that $h\mapsto L_{in}(\omega, h)$ is $\lambda$-strongly convex, for any $\omega\in\mathbb{R}^d$, as proved in \cref{prop:strong_convexity_Lin}, the third line follows from \cref{prop:fre_diff_L_v}, the fourth line uses Jensen's inequality, and the last line is a direct consequence of the boundedness of $K$ by $\kappa$ (\cref{assump:K_bounded}). According to \cref{prop:bound_hstaromega}, $h^\star_\omega(x)\in\mathcal{V}\coloneqq\left[-\frac{B\kappa}{\lambda},\frac{B\kappa}{\lambda}\right]$, which is a compact interval of $\mathbb{R}$, where $B\coloneqq\sup_{\omega\in\text{hull}(\Omega),y\in\mathcal{Y}}\left|\partial_v \ell_{in}(\omega, 0, y)\right|>0$. By \cref{assump:compact}, $\Omega$ and $\mathcal{Y}$ are compact sets, hence $\text{hull}(\Omega)\times\mathcal{V}\times\mathcal{Y}$ is compact. Besides, by \cref{assump:reg_lin_lout}, $(\omega, v, y)\mapsto\partial_v\ell_{in}(\omega, v, y)$ is continuous over the domain $\text{hull}(\Omega)\times\mathcal{V}\times\mathcal{Y}$. Since every continuous function on a compact set is bounded, this leads to:
\begin{equation*}
    \mathbb{E}_\mathbb{P}\Big[\left\|\partial_{\omega, v}^2 \ell_{in}(\omega, h^\star_\omega(x), y)\right\|\Big]\leq L\coloneqq\sup_{\omega\in\text{hull}(\Omega),v\in\mathcal{V},y\in\mathcal{Y}}\left\|\partial_{\omega, v}^2 \ell_{in}(\omega, v, y)\right\|<+\infty.
\end{equation*}
Substituting this bound into \cref{eq:upp_bound_jac} means that $\frac{L\sqrt{\kappa}}{\lambda}$ is an upper-bound on $\left\|\partial_\omega h^\star_\omega\right\|_{\op}$. Thus, the result follows as desired.
\end{proof}