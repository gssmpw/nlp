%

\documentclass{article}

%
\usepackage{microtype}
\usepackage{graphicx}
\usepackage{subfigure}
\usepackage{booktabs} %
\usepackage{enumitem} %
\usepackage{bbm}
\usepackage{mathrsfs}
\usepackage{tikz-cd}
\usepackage{automatic_resize}


%
%
%
%
\usepackage{hyperref}

%
\newcommand{\theHalgorithm}{\arabic{algorithm}}

%
%

%
\usepackage[accepted]{icml2025}

%
\usepackage{amsmath}
\usepackage{amssymb}

\usepackage{mathtools}
\usepackage{amsthm}

%
\usepackage[capitalize,noabbrev]{cleveref}
\newlist{assumplist}{enumerate}{1}
\setlist[assumplist]{label=(\textbf{\Alph*})}
\Crefname{assumplisti}{Assumption}{Assumptions}
\Crefname{assumption}{Assumption}{Assumptions}


\newcommand{\diam}{\operatorname{diam}}
\newcommand{\op}{\operatorname{op}}
\newcommand{\hs}{\operatorname{HS}}
\newcommand{\Id}{\operatorname{Id}}
\newcommand{\lip}{\operatorname{Lip}}
\newcommand{\Tr}{\operatorname{Tr}}
\newcommand{\Span}{\operatorname{Span}}
\newcommand*\diff{\mathop{}\!\mathrm{d}}
\newcommand{\alphabf}{\operatorname{\boldsymbol{\alpha}}}
\newcommand{\gammabf}{\operatorname{\boldsymbol{\gamma}}}
\newcommand{\K}{\operatorname{\mathbf{K}}}
\newcommand{\Kbar}{\operatorname{\mathbf{\overline{K}}}}
\newcommand{\Ktilde}{\operatorname{\mathbf{\widetilde{K}}}}
\newcommand{\diag}{\operatorname{\mathbf{diag}}}
\newcommand{\M}{\operatorname{\mathbf{M}}}
\newcommand{\Done}{\operatorname{\mathbf{D_v^{out}}}}
\newcommand{\Dtwo}{\operatorname{\mathbf{D_{v,v}^{in}}}}
\newcommand{\Dthree}{\operatorname{\mathbf{D_\omega^{out}}}}
\newcommand{\Dfour}{\operatorname{\mathbf{D_{\omega,v}^{in}}}}


\newcommand{\PP}{\mathbb{D}}
\newcommand{\LL}{{\bf L}}
\newcommand{\Dout}{\delta_\omega^{out}}
\newcommand{\Din}{\delta_\omega^{in}}
\newcommand{\Douth}{\partial_{h}\delta_\omega^{out}}
\newcommand{\Dinh}{\partial_{h}\delta_\omega^{in}}
\newcommand{\Dinw}{\partial_{\omega}\delta_\omega^{in}}
\newcommand{\Doutw}{\partial_{\omega}\delta_\omega^{out}}
\newcommand{\Dinhh}{\partial_{h}^2\delta_\omega^{in}}
\newcommand{\Dinwh}{\partial_{\omega,h}^2\delta_\omega^{in}}


\newcommand{\Eout}{E_\omega^{out}}
\newcommand{\Ein}{E_\omega^{in}}
\newcommand{\Eouth}{\partial_{h}E_\omega^{out}}
\newcommand{\Einh}{\partial_{h}E_\omega^{in}}
\newcommand{\Einw}{\partial_{\omega}E_\omega^{in}}
\newcommand{\Eoutw}{\partial_{\omega}E_\omega^{out}}
\newcommand{\Einhh}{\partial_{h}^2E_\omega^{in}}
\newcommand{\Einwh}{\partial_{\omega,h}^2E_\omega^{in}}
\newcommand{\lipout}{\operatorname{Lip}_{out}}
\newcommand{\lipin}{\operatorname{Lip}_{in}}



\newcommand{\Dgammastar}{\operatorname{\mathbf{D}_{\hat{\gammabf}_\omega}}}
\DeclareMathOperator*{\argmin}{arg\,min}

%
%
%
\theoremstyle{plain}
\newtheorem{theorem}{Theorem}[section]
\newtheorem{proposition}[theorem]{Proposition}
\newtheorem{lemma}[theorem]{Lemma}
\newtheorem{corollary}[theorem]{Corollary}
\theoremstyle{definition}
\newtheorem{definition}[theorem]{Definition}
\newtheorem{assumption}[theorem]{Assumption}
\theoremstyle{remark}
\newtheorem{remark}[theorem]{Remark}

%
%
%
\usepackage[textsize=tiny]{todonotes}
\newcommand{\ma}[1]{\textcolor{red}{[MA: #1]}}
\newcommand{\fe}[1]{\textcolor{blue}{[FE: #1]}}
\newcommand{\ep}[1]{\textcolor{green}{[EP: #1]}}
\newcommand{\sv}[1]{\textcolor{violet}{[SV: #1]}}


\newcommand{\mat}[1]{\textcolor{red}{#1}}
\newcommand{\fet}[1]{\textcolor{blue}{#1}}
\newcommand{\ept}[1]{\textcolor{green}{#1}}
\newcommand{\svt}[1]{\textcolor{violet}{#1}}
\newcommand{\edt}[1]{\textcolor{teal}{[[#1]]}}


%
%
\icmltitlerunning{Learning Theory for Kernel Bilevel Optimization}

\begin{document}

\twocolumn[
\icmltitle{Learning Theory for Kernel Bilevel Optimization}

%
%
%
%

%
%
%
%

%
%
%
%

\begin{icmlauthorlist}
\icmlauthor{Fares El Khoury}{inria}
\icmlauthor{Edouard Pauwels}{tse}
\icmlauthor{Samuel Vaiter}{cnrs}
\icmlauthor{Michael Arbel}{inria}
\end{icmlauthorlist}

\icmlaffiliation{inria}{Univ. Grenoble Alpes, Inria, CNRS, Grenoble INP, LJK, 38000 Grenoble, France}
\icmlaffiliation{tse}{Toulouse School of Economics, Universit\'e Toulouse Capitole, 31080 Toulouse, France}
\icmlaffiliation{cnrs}{CNRS \& Universit\'e C\^ote d’Azur, Laboratoire J. A. Dieudonn\'e, 06108 Nice, France}

\icmlcorrespondingauthor{Fares El Khoury}{fares.el-khoury@inria.fr}

%
%
%
%

\vskip 0.3in
]

%

%
%
%
%
%

\printAffiliationsAndNotice{}  %
%

\begin{abstract}
Bilevel optimization has emerged as a technique for addressing a wide range of machine learning problems that involve an outer objective implicitly determined by the minimizer of an inner problem. In this paper, we investigate the generalization properties for kernel bilevel optimization problems where the inner objective is optimized over a Reproducing Kernel Hilbert Space. This setting enables rich function approximation while providing a foundation for rigorous theoretical analysis. In this context, we establish novel generalization error bounds for the bilevel problem under finite-sample approximation. Our approach adopts a functional perspective, inspired by \cite{petrulionyte2024functional}, and leverages tools from empirical process theory and maximal inequalities for degenerate $U$-processes to derive uniform error bounds. These generalization error estimates allow to characterize the statistical accuracy of gradient-based methods applied to the empirical discretization of the bilevel problem.
\end{abstract}





\section{Introduction}
Optimization is a cornerstone of mathematical modeling, machine learning, and decision-making. 
Among the various frameworks, bilevel optimization stands out due to its hierarchical structure, where one optimization problem, called \emph{outer-level}, is constrained by the solution of another problem, called \emph{inner-level} \cite{candler1977multi}. 
This framework was first introduced in the context of economic game theory by \citet{vonstackelberg1934marktform} to model leader-follower interactions, where one agent's decisions depend on the optimal response of another. Over time, bilevel optimization has naturally found applications in a broad spectrum of machine learning fields, including hyper-parameter tuning \cite{larsen1996design,bengio2000gradient,franceschi2018bilevel}, meta-learning \cite{bertinetto2018metalearning,pham2021contextual}, inverse problems \cite{holler2018bilevel}, and reinforcement learning \cite{hong2023two,liu2023value}, making it a powerful and versatile tool in both theoretical and practical contexts.

{The success of bilevel optimization in machine learning naturally raises fundamental questions about the generalization properties of models learned through these procedures, as the number of data samples increases. Several existing works have studied the generalization and convergence of bilevel algorithms under the assumption that the inner-level problem is strongly convex and that its parameters lie in a finite-dimensional space. These include analyses of the convergence of stochastic bilevel optimization algorithms \cite{Arbel:2021a,Dagreou2022SABA,ghadimi2018approximation,Ji:2021a} and approaches based on algorithmic stability \cite{bao2021stability,zhang2024fine}. The strong convexity assumption ensures the uniqueness of the inner-level solution, a key property for stability and convergence analysis in bilevel optimization. 
Moreover, restricting the inner-level parameters to a finite-dimensional space, instead of possibly richer infinite-dimensional spaces, as in kernel methods, circumvents additional complexities, where the parameter's dimension may grow with the sample size. 
This dependence on sample size, in non-parametric methods, poses additional challenges, as solutions at different sample sizes are not directly comparable. In contrast, in the finite-dimensional setting, generalization bounds can be derived by quantifying the convergence of finite-sample estimates of the inner-level solution toward the \emph{population solution}, \textit{i.e.}, the solution obtained in the limit of infinite samples, within the same parameter space.}

{Albeit convenient from a theoretical perspective, having both strong convexity and finite-dimensionality drastically limits expressiveness of the models, effectively restricting them to linear functions. 
Going beyond linear models requires either relaxing the strong convexity assumption to accommodate more expressive models, such as deep neural networks \cite{Goodfellow-et-al-2016}, or considering non-parametric bilevel problems, where the inner-level variable lies in an expressive infinite-dimensional function space, such as a Reproducing Kernel Hilbert Space (RKHS) \cite{scholkopf2002learning}. 
Early works in bilevel optimization for machine learning followed the latter approach, developing bilevel methods for hyper-parameter selection in the context of kernel methods \cite{keerthi2006efficient,kunapuli2008classification}. 
These works leverage the \emph{representer theorem} \cite{scholkopf2001generalized} to transform the infinite-dimensional problem into a finite-dimensional one with dimension  depending on the sample size. However, they do not address how the sample size impacts generalization. 
Another line of research instead focuses on relaxing the strong convexity assumption, proposing new bilevel algorithms that can handle the loss of convexity \cite{arbel2022nonconvex,kwon2024on,shen2023penalty}. 
Nevertheless, non-convex bilevel optimization is a very hard problem in general \cite{liu2021towards,arbel2022nonconvex,bolte2024geometric}, and obtaining strong generalization guarantees in this setting remains out of reach due to the lack of precise control over the inner-level solution.} 
In all cases, learning theory for bilevel problems beyond the strongly convex parametric setting is essentially lacking.



In the present work, we take an initial step towards developing a learning theory that goes beyond the finite-dimensional setting. Specifically, we propose to study \emph{Kernel Bilevel Optimization} (KBO) problems, where the inner objective $L_{in}: \mathbb{R}^d \times \mathcal{H}\rightarrow \mathbb{R}$ finds an optimal inner solution $h_{\omega}^{\star}$ in a RKHS $\mathcal{H}$ for a given parameter $\omega$ in $\mathbb{R}^d$, while the outer objective $L_{out}:\mathbb{R}^d \times \mathcal{H}\rightarrow \mathbb{R}$ optimizes the parameter $\omega$ over a closed subset $\mathcal{C}$ of $\mathbb{R}^d$, given the inner solution $h_{\omega}^{\star}$: 
\begin{equation}\tag{KBO}\label{eq:kbo}
    \begin{aligned}
    \min_{\omega\in\mathcal{C}}\mathcal{F}(\omega)\coloneqq L_{out}(\omega,h^\star_\omega)\\
    \text{s.t.}\quad h^\star_\omega=\argmin_{h\in\mathcal{H}}L_{in}(\omega, h).
    \end{aligned}
\end{equation}
In particular, we focus on inner and outer level objectives that are expectation of point-wise losses, a common setting in learning theory. RKHS provides a natural framework to study learning theoretic arguments, and has been instrumental for many fruitful results in pattern recognition and machine learning. They allow to describe very expressive non-linear models with simple and stable algorithms, while enabling a rich statistical analysis and featuring adaptivity to the regularity of the population problem \cite{shawe2004kernel,scholkopf2002learning,hofmann2008kernel}. Our choice is also motivated by the relevance of kernel methods, even in the deep learning era. They remain competitive for some prediction problem, for example involving physics \cite{doumeche2024physics,letizia2022learning}. Furthermore, the mathematics of kernel methods are useful to describe the limiting behavior of  deep network training for very large models \cite{jacot2018neural,belkin2018understand}. Indeed, in this limit, the problem becomes (strongly) convex in an infinite-dimensional functional space, simplifying the difficulties of non-convex model parameterization, a major bottleneck in the analysis of such models. This point of view was leveraged by \citet{petrulionyte2024functional} who introduced functional bilevel optimization, and our setting is a special case for which the underlying function space is an RKHS.
From a practical perspective, our setting is amenable to first-order methods using implicit differentiation techniques \cite{griewank2003piggyback,bai2019deep,blondel2022efficient}.

{\bf Contributions.} We leverage empirical process theory and its extension to $U$-processes \cite{sherman1994maximal} to derive uniform generalization bounds for the value function of \eqref{eq:kbo}, quantifying the discrepancy between $\mathcal{F}$ and its plug-in estimator $\widehat{\mathcal{F}}$ both in terms of their values and their gradients. 
This control in terms of gradients is crucial to study first-order optimization methods as the value function $\mathcal{\widehat{F}}$ is typically not convex, and iterative solution methods find approximate critical points ($\nabla \widehat{\mathcal{F}}$ small). Our result relies on an equivalence that we establish between the gradient $\nabla \widehat{\mathcal{F}}$ and a plug-in statistical estimate for $\nabla \mathcal{F}$ that is more amenable to a statistical analysis. We then use our uniform bounds to provide generalization guarantees for gradient descent and projected gradient descent applied to $\nabla\widehat{\mathcal{F}}$. Under specific assumptions, we show convergence rates for sub-optimality measures, depending on sample sizes and the number of algorithmic iterations. This illustrates the relevance of our generalization bounds on one of simplest bilevel optimization algorithms. For large number of algorithmic steps, gradient algorithms on the empirical \eqref{eq:kbo} find approximate critical points of the population \eqref{eq:kbo} up to a statistical error which we control. 

\paragraph{Organization of the Paper.} We start in \Cref{sec:KBO} with a precise description of the \eqref{eq:kbo} problem, application examples, and implicit differentiation in an RKHS. In \Cref{sec:finite_samples}, we describe the empirical \eqref{eq:kbo} and state our first main result on the gradient of its value function. Our uniform generalization bound for \eqref{eq:kbo} is described in \Cref{sec:conv}, together with corollaries for gradient descent and projected gradient descent in \Cref{sec:gradientMethods}. Finally, in \cref{sec:proof_strategy}, we discuss the strategy of the proof for the main result.

\section{Kernel Bilevel Optimization }
\label{sec:KBO}
\subsection{Problem Formulation}
We consider the kernel bilevel optimization problem in \eqref{eq:kbo} with an RKHS $\mathcal{H}$, which is a space of real-valued functions defined on an input space $\mathcal{X}\subset\mathbb{R}^p$ and associated with a reproducing kernel $K:\mathcal{X}\times\mathcal{X}\to\mathbb{R}$. 
We are interested, in particular, in (regularized) objectives expressed as expectations of point-wise loss functions, a formulation widely adopted in machine learning as it allows the loss functions to represent the average performance over some data distribution. 
Specifically, given two probability distributions $\mathbb{P}$ and $\mathbb{Q}$ supported on $\mathcal{X}\times \mathcal{Y}$ for some target space $\mathcal{Y}\subset \mathbb{R}^q$, we consider objectives of the form:  
\begin{align*}
\begin{split}
    L_{out}(\omega, h)&=\mathbb{E}_\mathbb{Q}\left[\ell_{out}(\omega, h(x), y)\right],\\
    L_{in}(\omega, h)&=\mathbb{E}_\mathbb{P}\left[\ell_{in}(\omega, h(x), y)\right]+\frac{\lambda}{2}\|h\|_\mathcal{H}^2,
\end{split}
\end{align*}
where $\ell_{in}\text{ and }\ell_{out}:\mathbb{R}^d\times\mathbb{R}\times\mathbb{R}^q\to\mathbb{R}$ represent the inner and outer point-wise loss functions, $\lambda>0$ is the regularization parameter which is fixed through this work, and $\|\cdot\|_\mathcal{H}$ denotes the norm in the RKHS $\mathcal{H}$. 
The regularization term in $L_{in}$ is often used in practice to prevent overfitting, by penalizing overly complex models. In our setting, it ensures strong convexity of $h\mapsto L_{in}(\omega,h)$, under mild assumptions on $\ell_{in}$, which will be critical to leverage functional implicit differentiation. 

{\bf Assumptions.} Through the paper, we will make the following four assumptions to derive generalization bounds while retaining a simple and modular presentation. 
\begin{assumplist}
\item\label{assump:K_bounded}(Boundedness of $K$). 
There exists a positive constant $\kappa>0$ such that $K(x,x)\leq\kappa$, for any $x\in\mathcal{X}$.

\item\label{assump:compact}
(Compactness of $\mathcal{Y}$). The subset $\mathcal{Y}$ of $\mathbb{R}^q$ is compact.
\item \label{assump:reg_lin_lout}(Regularity of $\ell_{in}$ and $\ell_{out}$).
The functions
$\ell_{in}$ and $\ell_{out}$ are of class $C^3$ jointly in their first two arguments $(\omega,v)$, and their derivatives are jointly continuous in $(\omega,v,y)$. 
\item\label{assump:convexity_lin}(Convexity of $\ell_{in}$ in its second argument). 
For any $(\omega,y)\in\mathbb{R}^d\times\mathbb{R}^q$, the map $v\mapsto \ell_{in}(\omega,v,y)$ is convex.
\end{assumplist}


\cref{assump:K_bounded} on the kernel $K$ holds for a wide class of kernels, such as the Gaussian and Mat\'ern kernels \cite{wendland2004scattered}, both of which fulfill this assumption with $\kappa=1$. 
\cref{assump:compact} on the set $\mathcal{Y}$ is a mild assumption that holds in many practical cases, such as classification, where the set $\mathcal{Y}$ is finite, or when $\mathcal{Y} = [0,1]^q$, where complex data, such as images, can be represented.  
\cref{assump:reg_lin_lout} on the point-wise objectives is a mild regularity assumption.  \cref{assump:K_bounded,assump:compact,assump:reg_lin_lout} can be relaxed at the expense of weaker yet more technical assumptions, such as finite moments  assumptions on $\mathbb{P}$ and $\mathbb{Q}$, and suitable polynomial growth of the kernel and some partial derivatives of $\ell_{in}$ and $\ell_{out}$. It is also sufficient to require that \Cref{assump:reg_lin_lout} holds on $\mathcal{U} \times \mathbb{R} \times \mathbb{R}^q$ where $\mathcal{U}$ is an open neighborhood of $\mathcal{C}$ and that \Cref{assump:convexity_lin} holds for any $\omega\in\mathcal{C}$ and $y\in\mathcal{Y}$.  We prefer to keep these stronger yet simpler assumptions for clarity.
Finally, \cref{assump:convexity_lin} is essential to ensure the existence and uniqueness of a smooth minimizer $h^\star_\omega$. It is a relatively weak assumption that was recently considered in \cite{petrulionyte2024functional} in the context of functional bilevel optimization and that holds in many cases of interest, as discussed in \cref{sec:examples}.

\subsection{Examples of KBO in Machine Learning}\label{sec:examples}
To illustrate the relevance of \eqref{eq:kbo}, we consider two examples that highlight its applicability.

{\bf Hyper-Parameter Selection under Distribution Shift.} In this application, the aim is to select the best hyper-parameters for a machine learning model, \textit{e.g.}, regularization parameters, while accounting for distribution shift between the training and testing data, \textit{i.e.}, when the training and test data distributions are different \cite{pedregosa2016hyperparameter,franceschi2018bilevel}. This can be viewed as an instance of \eqref{eq:kbo} when using models in an RKHS. At the inner-level, the model is trained to minimize the regularized training squared error loss, where the hyper-parameter $\omega>0$ denotes a weight for the data fitting term. At the outer-level, the task is to select the hyper-parameter $\omega$ that minimizes the model’s performance on the distribution-shifted test. Both inner and outer objectives can thus be formulated as:
\begin{align*}
    L_{out}(\omega, h^\star_\omega)&=\mathbb{E}_{x,y}\left[\verts{h^\star_\omega(x)-y}^2\right],\\
    L_{in}(\omega, h)&=\omega \mathbb{E}_{x,y}\left[\verts{h(x)-y}^2\right]+\frac{1}{2}\Verts{h}_\mathcal{H}^2.
\end{align*}
This formulation could be used for domain adaptation \cite{ben2006analysis} or domain generalization \cite{wang2022generalizing} to choose hyper-parameters that perform well on the distribution-shifted test data.

{\bf Instrumental Variable Regression} is a technique used to address endogeneity in statistical modeling by leveraging instruments to estimate causal relationships \cite{newey2003instrumental}. The goal here is to estimate a function $t\mapsto f_{\omega}(t)$ parameterized by a vector $\omega$, that satisfies $y=f_{\omega}(t)+\epsilon$, where $y\in\mathbb{R}$ is the observed outcome, $t$ is a treatment, and $\epsilon$ is the error term. 
The key issue is that $t$ is endogenous, which means that it is correlated with $\epsilon$, \textit{i.e.}, $\mathbb{E}[\epsilon\mid t]\neq 0$, making direct regression inconsistent.
Indeed, such correlation leads to biased estimates of $f_{\omega}(t)$ because the assumption of exogeneity, \textit{i.e.}, independence of $t$ and $\epsilon$, is violated. 
To resolve this, an instrumental variable $x$ that is uncorrelated with $\epsilon$, \textit{i.e.}, $\mathbb{E}[\epsilon\mid x]=0$, but correlated with $t$, can be used to recover the relationship between $y$ and $t$, without being directly affected by the bias introduced by $\epsilon$ using the two stages least squares regression \cite{singh2019kernel,meunier2024nonparametric}. As shown in \cite{petrulionyte2024functional}, this approach can be naturally expressed as a bilevel optimization problem of the form:
\begin{align*}
    L_{out}(\omega,h^\star_\omega)&=\mathbb{E}_{x,y}\left[\verts{h^\star_\omega(x)-y}^2\right],\\
    L_{in}(\omega,h)&=\mathbb{E}_{x,t}\left[\verts{h(x)-f_{\omega}(t)}^2\right]+\frac{\lambda}{2}\Verts{h}_\mathcal{H}^2,
\end{align*}
where $h$ can be chosen to be in an RKHS to allow flexibility in the estimation while retaining uniqueness of the solution $h_{\omega}^{\star}$,  a key property in bilevel optimization.  



\subsection{Implicit Differentiation in an RKHS}
A stationarity measure in \eqref{eq:kbo} is the gradient $\nabla\mathcal{F}(\omega)$ of the value function $\mathcal{F}$. 
Nonetheless, evaluating the gradient requires computing the Jacobian $\partial_\omega h^\star_\omega$, which can be viewed as a linear operator from $\mathcal{H}$ to $\mathbb{R}^d$. Indeed $h^\star_\omega$ depends implicitly on $\omega$. A key ingredient for computing the Jacobian $\partial_\omega h^\star_\omega$ is the implicit function theorem {\citep{ioffe1979theory}} which guarantees differentiability of the implicit function $\omega\mapsto h_{\omega}^{\star}$ and allows characterizing $\partial_\omega h^\star_\omega$ as the unique solution of a linear system of the form:
\begin{equation}\label{eq:impl_diff}
\partial_{\omega, h}^2 L_{in}(\omega, h^\star_\omega)+\partial_\omega h^\star_\omega \partial_h^2 L_{in}(\omega, h^\star_\omega)=0,    
\end{equation}
where $\partial_h^2 L_{in}(\omega, h^\star_\omega)$ is a linear operator from $\mathcal{H}$ to itself representing the partial Hessian of $L_{in}$ w.r.t. $h$, while $\partial_{\omega,h}^2 L_{in}(\omega, h^\star_\omega)$ is an operator from $\mathcal{H}$ to $\mathbb{R}^d$ representing the cross derivatives of $L_{in}$ w.r.t. to $\omega$ and $h$. Applying such result requires $h\mapsto L_{in}(\omega,h)$ to be Fr\'echet differentiable with  gradient map $(\omega,h)\mapsto\partial_{h}L_{in}(\omega,h)$ jointly Fr\'echet differentiable and invertible Hessian operator. All these properties are satisfied in our setting under \cref{assump:K_bounded,assump:compact,assump:convexity_lin,assump:reg_lin_lout} as shown in {\cref{prop:fre_diff_L,prop:fre_diff_L_v,prop:strong_convexity_Lin}} of {\cref{sec:reg_ob}}. Furthermore, when the outer objective $L_{out}$ is Fr\'echet differentiable, which is our case under our assumptions ({\cref{prop:fre_diff_L}} of {\cref{sec:reg_ob}}), then by composition with $\omega\mapsto (\omega,h_{\omega}^{\star})$,  the map $\omega\mapsto \mathcal{F}(\omega)$ must also be differentiable with gradient obtained using the chain rule:
\begin{align*}
	\nabla\mathcal{F}(\omega)= \partial_{\omega}L_{out}(\omega,h_{\omega}^{\star}) -\partial_{\omega}h_{\omega}^{\star}\partial_{h}L_{out}(\omega,h_{\omega}^{\star}).
\end{align*}
The above expression for the gradient is intractable as it involves abstract operators. 
In \cref{prop:func_gradient} below, we 
  derive an explicit expression for $\nabla\mathcal{F}(\omega)$ which exploits the particular structure of the objectives $L_{in}$ and $L_{out}$ as expectations of point-wise losses. 
\begin{proposition}[Expression of the total gradient]\label{prop:func_gradient}
Under \cref{assump:compact,assump:convexity_lin,assump:K_bounded,assump:reg_lin_lout}, $\mathcal{F}$ is differentiable on $\mathbb{R}^d$, with gradient $\nabla\mathcal{F}(\omega)$, for any $\omega\in \mathbb{R}^d$, given by:
\begin{equation}\label{eq:func_tot_gradient}
    \begin{aligned}
    \nabla\mathcal{F}(\omega)=&\mathbb{E}_\mathbb{Q}\left[\partial_\omega\ell_{out}(\omega, h^\star_\omega(x),y)\right]\\
    &+\mathbb{E}_\mathbb{P}\left[\partial_{\omega,v}^2\ell_{in}(\omega, h^\star_\omega(x),y)a^\star_\omega(x)\right],
    \end{aligned}
\end{equation}
where \emph{the adjoint function} $a^\star_\omega\in\mathcal{H}$ is the unique minimizer of a strongly-convex quadratic objective $a\mapsto L_{adj}(\omega,a)$ defined on $\mathcal{H}$ as:
\begin{align}\label{eq:adjoint_objective}
\begin{split}
    L_{adj}(\omega,& a)\coloneqq  \frac{1}{2}\mathbb{E}_{\mathbb{P}}\brackets{\partial_{v}^2 \ell_{in}\parens{\omega,h_{\omega}^{\star}(x),y} a(x)^2}\\
     &+ \mathbb{E}_{\mathbb{Q}}\brackets{\partial_{v}\ell_{out}\parens{\omega,h_{\omega}^{\star}(x),y}a(x)}+\frac{\lambda}{2}\Verts{a}_{\mathcal{H}}^2,
\end{split}
\end{align}
where $\partial_\omega\ell_{out}$ and $\partial_v\ell_{out}$ are the first-order partial derivatives of $\ell_{out}$ w.r.t. $\omega$ and $v$, while $\partial_{\omega,v}^2\ell_{in}$ and $\partial_{v}^2\ell_{in}$ denote the second-order partial derivatives of $\ell_{in}$ w.r.t. $\omega$ and $v$.
\end{proposition}
\cref{prop:func_gradient} is proved in {\cref{sec:reg_ob}} and relies essentially on proving Bochner's integrability {\citep[Definition 1, Chapter 2]{diestel1977vector}} of some suitable operators on $\mathcal{H}$, and then applying Lebesgue's dominated convergence theorem for Bochner's integral  {\citep[Theorem 3, Chapter 2]{diestel1977vector}} to exchange derivatives and expectations.
The expression in \Cref{prop:func_gradient} provides a natural way for approximating $\nabla\mathcal{F}(\omega)$ by estimating all expectations using finite sample averages, as we further discuss in {\cref{sec:finite_samples}}.

\section{Finite Sample Approximation of \eqref{eq:kbo}}\label{sec:finite_samples}
In this section, we consider an approximation to \eqref{eq:kbo} problem when only a finite number of i.i.d. samples $(x_i,y_i)_{1\leq i\leq n}$ and  $(\tilde{x}_j,\tilde{y}_j)_{1\leq j\leq m}$ from $\mathbb{P}$ and $\mathbb{Q}$ are available. This setting is ubiquitous in machine learning as it allows finding tractable approximate solutions to the original problem. 
As we are interested in approximately solving \eqref{eq:kbo} using gradient methods, our focus here is to derive estimators for both the value function $\mathcal{F}(\omega)$ and its  gradient $\nabla\mathcal{F}(\omega)$ for which the generalization properties will later be studied in  \cref{sec:conv}. 

In \cref{sec:plug-in_loss}, we follow a commonly used approach of first deriving a plug-in estimator $\widehat{\mathcal{F}}$ of the value function, then considering its gradient 
$\nabla\mathcal{\widehat{F}}(\omega)$ 
as an approximation to $\nabla \mathcal{F}(\omega)$. 
Then, in \cref{sec:plug_in_gradient}, we show that such approximation is equivalent to a second estimator, more amenable to a statistical analysis, obtained by directly computing a plug-in estimator of $\nabla\mathcal{F}$ based on its expression in \cref{eq:func_tot_gradient}. \cref{fig:commutative-diagram}  summarizes such equivalence.   
\begin{figure}[ht]
\centering
\begin{tikzcd}[row sep=0.8cm, column sep=3.5cm]
    \mathcal{F} \arrow[r, "\text{ \small Plug-in estimation}",] \arrow[d, "\text{\small $\nabla$: diff}"',] & \widehat{\mathcal{F}} \arrow[d, "\text{\small $\nabla$: diff}"] \\
    \nabla\mathcal{F} \arrow[r, "\text{\small Plug-in estimation}"'] & \nabla\widehat{\mathcal{F}} 
\end{tikzcd}
\caption{A commutative diagram illustrating that plug-in statistical estimation and differentiation can be interchanged for $\mathcal{F}$ and $\widehat{\mathcal{F}}$ resulting in a single gradient estimator. }
\label{fig:commutative-diagram}
\end{figure}


\subsection{Value Function: Plug-in Estimator and its Gradient  }\label{sec:plug-in_loss}
A natural approach for finding approximate solutions to \eqref{eq:kbo} is to consider an approximate problem obtained after replacing the objectives $L_{in}$ and $L_{out}$ by their empirical approximations $\widehat{L}_{in}$ and $\widehat{L}_{out}$:
\begin{align*}
%
    \begin{split}
    \widehat{L}_{out}\parens{\omega,h}&\coloneqq \frac{1}{m} \sum_{j=1}^m \ell_{out}(\omega, h(\tilde{x}_j), \tilde{y}_j) \\
    \widehat{L}_{in}\parens{\omega,h}&\coloneqq \frac{1}{n} \sum_{i=1}^n \ell_{in}(\omega, h(x_i), y_i)+\frac{\lambda}{2}\|h\|_\mathcal{H}^2.
        \end{split}
\end{align*}
A plug-in estimator $\omega\mapsto \widehat{\mathcal{F}}(\omega)$ is then obtained by first finding a solution $\hat{h}_{\omega}$ minimizing $h\mapsto \widehat{L}_{in}(\omega,h)$ that is meant to approximate the optimal inner solution $h_{\omega}^{\star}$, and subsequently plugging it into $\widehat{L}_{out}$. This procedure results in the following empirical version of \eqref{eq:kbo}:
\begin{align*}
    %
    \begin{split}
    &\min_{\omega\in\mathcal{C}}\widehat{\mathcal{F}}(\omega)\coloneqq \widehat{L}_{out}\parens{\omega,\hat{h}_{\omega}}\\
    &\text{s.t.}\quad \hat{h}_\omega=\argmin_{h\in\mathcal{H}}\widehat{L}_{in}\parens{\omega,h}.    \end{split}
\end{align*}
The inner problem still requires optimizing over a, potentially infinite-dimensional, RKHS. However, its finite sum structure allows equivalently expressing it as a finite-dimensional bilevel optimization, by application of the so called representer theorem {\cite{scholkopf2001generalized}}:
\begin{align}\label{eq:kbo_app}
    &\min_{\omega\in\mathcal{C}}\widehat{\mathcal{F}}(\omega)\coloneqq\frac{1}{m}\sum_{j=1}^m\ell_{out}(\omega, \parens{\Kbar\hat{\gammabf}_\omega}_j, \tilde{y}_j)\tag{$\widehat{\text{KBO}}$} \\
    &\text{s.t. }\hat{\gammabf}_\omega=\argmin_{\gammabf\in\mathbb{R}^n}\frac{1}{n}\sum_{i=1}^n\ell_{in}(\omega, \parens{\K\gammabf}_i, y_i)+\frac{\lambda}{2}\gammabf^\top\K\gammabf.    \nonumber
\end{align}
In the above equation, $\K\in\mathbb{R}^{n\times n}$ and $\Kbar\in\mathbb{R}^{m\times n}$ are matrices containing the  pairwise kernel similarities between the data points, \textit{i.e.}, $\K_{ij}\coloneqq K(x_i,x_j)$ and $\Kbar_{ij}\coloneqq K(\tilde{x}_i,x_j)$, while $\gamma$ is a parameter vector in $\mathbb{R}^n$ representing the inner-level variables. The optimal solution $\hat{\gamma}_{\omega}$ enables recovering the prediction function $\hat{h}_{\omega}$ by linearly combining kernel evaluations at inner-level samples, \textit{i.e.}, $\hat{h}_{\omega} = \sum_{i=1}^n (\hat{\gamma}_{\omega})_i K(x_i,\cdot)$.  

The formulation in \eqref{eq:kbo_app} allows deriving an expression for the gradient $\nabla \widehat{\mathcal{F}}(\omega)$ in terms of the Jacobian $\partial_{\omega} \hat{\gamma}_{\omega}$ by direct application of the chain rule. Unlike the Jacobian $\partial_{\omega}h_{\omega}^{\star}$ which requires solving an infinite-dimensional linear system given by \cref{eq:impl_diff}, the Jacobian of $\hat{\gamma}_{\omega}$ can be obtained as a solution of a finite-dimensional linear system by application of the implicit function theorem ({see {\cref{prop:est_2} of \cref{sec:grad_est}}}). 
Hence, \eqref{eq:kbo_app} falls into a class of optimization problems for which a rich body of literature have proposed practical and scalable algorithms, leveraging the expression of $\nabla\widehat{\mathcal{F}}(\omega)$ \citep{Ji:2021a,Arbel:2021a,Dagreou2022SABA}. Consequently, solving \eqref{eq:kbo_app} provides a 
practical way for approximating the solution to the original population problem \eqref{eq:kbo} as proposed in several prior works on bilevel  optimization involving kernel methods \cite{keerthi2006efficient,kunapuli2008classification}.


Despite its practical advantage, the above approach yields algorithms that are not directly amenable to a statistical analysis. The key challenge is to be able to control the approximation error between the true gradient $\nabla\mathcal{F}(\omega)$ and its approximation $\nabla\widehat{\mathcal{F}}(\omega)$ as the sample sizes $n$ and $m$ increase. 
Existing statistical analysis for bilevel optimization, such as \cite{bao2021stability,zhang2024fine}, considered objectives in the form of expectations/finite sums of point-wise losses, as we do here. 
However, they require both inner-level and outer-level parameters to belong to spaces of fixed dimensions that is independent of the sample sizes $n$ and $m$. That is because these parameters are expected to converge towards some fixed vectors as $n,m\rightarrow +\infty$. Unfortunately, these results are not applicable in our case since the inner-level parameter $\gamma$ has a dimension that increases with the sample size $n$ ($\gamma\in\mathbb{R}^n$) and is not expected to converge towards any well-defined object. Next, we provide an equivalent expression for $\nabla\widehat{\mathcal{F}}(\omega)$ that will be crucial in our statistical analysis in \cref{sec:conv}. 


\subsection{Plug-in Estimator of the Total Gradient}\label{sec:plug_in_gradient}
We consider now an, a priori, different approach for approximating the total gradient $\nabla\mathcal{F}(\omega)$ based on direct plug-in estimation 
from \cref{eq:func_tot_gradient} and show that it recovers the previously introduced estimator $\nabla\widehat{\mathcal{F}}(\omega)$. Such approach consists in replacing all expectations in \cref{eq:func_tot_gradient} by empirical averages, then replacing $h^{\star}_{\omega}$ and $a_{\omega}^{\star}$ by finite sample estimates $\hat{h}_{\omega}$ and $\hat{a}_{\omega}$:
\begin{equation}\label{eq:functionalGradientEstimate}
    \begin{aligned}
    \widehat{\nabla\mathcal{F}}(\omega)=&\frac{1}{m}\sum_{j=1}^m\partial_\omega\ell_{out}(\omega, \hat{h}_\omega(\tilde{x}_j),\tilde{y}_j)\\
    &+\frac{1}{n}\sum_{i=1}^n\partial_{\omega,v}^2\ell_{in}(\omega, \hat{h}_\omega(x_i),y_i)\hat{a}_\omega(x_i).
    \end{aligned}
\end{equation}
Just as in \cref{sec:plug-in_loss}, $h^{\star}_{\omega}$ can be estimated by $\hat{h}_{\omega}$, the minimizer of the empirical objective $h\mapsto  \widehat{L}_{in}(\omega,h)$. Similarly, $a^{\star}_{\omega}$ can be estimated by minimizing an empirical version $a\mapsto \widehat{L}_{adj}(\omega,a)$ of the adjoint objective $L_{adj}$ given in \cref{eq:adjoint_objective}:
\begin{equation}
\label{eq:l_adj}
\begin{aligned}
    \widehat{L}_{adj}(\omega, a)=\frac{1}{2n}\sum_{i=1}^n\partial_v^2 \ell_{in}(\omega, \hat{h}_\omega(x_i), y_i)a^2(x_i)\\
    +\frac{1}{m}\sum_{j=1}^m\partial_v\ell_{out}(\omega, \hat{h}_\omega(\tilde{x}_j), \tilde{y}_j)a(\tilde{x}_j)+\frac{\lambda}{2}\|a\|_\mathcal{H}^2.
\end{aligned}
\end{equation}
Both functions $\hat{h}_{\omega}$ and $\hat{a}_{\omega}$ can be expressed as linear combinations of kernel evaluations with some given parameters that increase with the sample size $n$, {(see \cref{prop:est_2} for $\hat{h}_{\omega}$ and \cref{prop:est_1} for $\hat{a}_{\omega}$, both in \cref{sec:grad_est})}. 
However, these parameters are never  required for expressing the plug-in estimator $\widehat{\nabla\mathcal{F}}(\omega)$ in \cref{eq:functionalGradientEstimate},  since only the function values of $\hat{h}_{\omega}$ and $\hat{a}_{\omega}$ are needed for computing it. This property is precisely what makes $\widehat{\nabla\mathcal{F}}(\omega)$ suitable for a statistical analysis, since its estimation error  depends on the errors of $\hat{h}_{\omega}$ and $\hat{a}_{\omega}$  which always belong to the same space $\mathcal{H}$, regardless of the sample size, and are expected to approach their population counter-parts. This is unlike the expression of $\nabla\widehat{\mathcal{F}}(\omega)$ obtained by implicit differentiation and which suggests controlling the behavior of the vector $\hat{\gamma}_{\omega}$.  

The next proposition demonstrates that, surprisingly, both estimators $\nabla\widehat{\mathcal{F}}(\omega)$ and $\widehat{\nabla\mathcal{F}}(\omega)$ are precisely equal.
\begin{proposition}\label{prop:equivalenceEstimates}
	Under \cref{assump:compact,assump:convexity_lin,assump:K_bounded,assump:reg_lin_lout}, the gradient $\nabla \mathcal{\widehat{F}}(\omega)$ of the plug-in estimator $\widehat{\mathcal{F}}(\omega)$ of $\mathcal{F}(\omega)$ defined in \eqref{eq:kbo_app} is equal to the plug-in estimator $\widehat{\nabla\mathcal{F}}(\omega)$ of the total gradient $\nabla\mathcal{F}(\omega)$ introduced in \cref{eq:functionalGradientEstimate}. 
\end{proposition}
\cref{prop:equivalenceEstimates} is proved in {\cref{sec:grad_est}} and relies on an application of the representer theorem {\cite{scholkopf2001generalized}} to provide explicit expressions for both estimators in terms of $\gamma_{\omega}$, kernel matrices $\K$ and $\Kbar$ and partial derivatives of the point-wise objectives $\ell_{in}$ and $\ell_{out}$. Both expressions are then shown to be equal using optimality conditions on the parameters defining  $\hat{a}_{\omega}$. 
The result in \cref{prop:equivalenceEstimates} precisely says that the operations of differentiation and plug-in estimation commute in the case of \eqref{eq:kbo}. Such a commutativity property does not necessarily hold anymore if one considers spaces other than an RKHS as discussed in \citep[Appendix~F]{petrulionyte2024functional}. Next, we leverage the expression of the plug-in estimator $\widehat{\nabla\mathcal{F}}(\omega)$ to provide generalization bounds. 


\section{Generalization Bounds for \eqref{eq:kbo}}\label{sec:conv}
In this section, we present the main result of the present work: a maximal inequality controlling how  both value function $\mathcal{F}$ and its gradient $\nabla\mathcal{F}$ are well approximated by their empirical counter-parts uniformly over a compact subset $\Omega$ of $\mathbb{R}^d$. In \cref{sec:main_result},  we state the main result and discuss its implications. Then we present the general proof strategy, in \cref{sec:sketch_proof}, and discuss possible alternative approaches.  

\subsection{Maximal Inequalities for \eqref{eq:kbo}}\label{sec:main_result}
The following theorem provides finite sample bounds on the uniform approximation errors on the objective and its gradient in expectation over both inner and upper-level samples.

\begin{theorem}[Maximal inequalities]
\label{th:generalizationBounds}
Fix any compact subset $\Omega$ of $\mathbb{R}^d$. Under \cref{assump:compact,assump:convexity_lin,assump:K_bounded,assump:reg_lin_lout}, the following maximal inequalities hold:
\begin{align*}
    \mathbb{E}\left[\sup_{\omega\in\Omega}\left|\mathcal{F}(\omega)-\widehat{\mathcal{F}}(\omega)\right|\right]&\leq C\parens{\frac{1}{\sqrt{m}}+\frac{1}{\sqrt{n}}},\\
    \mathbb{E}\left[\sup_{\omega\in\Omega}\left\|\nabla\mathcal{F}(\omega)-\widehat{\nabla\mathcal{F}}(\omega)\right\|\right]&\leq C\parens{\frac{1}{\sqrt{m}}+\frac{1}{\sqrt{n}}},
\end{align*}
where the expectation is taken over the finite samples and $C$ is a constant that depends only $\Omega$, dimension $d$, regularization parameter $\lambda$, $\kappa$ and local upper-bounds on  $\ell_{in}$, $\ell_{out}$ and their partial derivatives over suitable compact set.    
\end{theorem}
\cref{th:generalizationBounds} states that the estimation error can be decomposed into two contributions each resulting from finite sample approximation of $L_{in}$ and $L_{out}$ with a \emph{parametric rate} of $\frac{1}{\sqrt{n}}$ and 
$\frac{1}{\sqrt{m}}$ up to a constant factor $C$. We provide a detailed expression for the constant in \cref{th:gen_bound} of \cref{app:sec_conv}. The restriction to compact subsets $\Omega$ instead of the whole space $\mathbb{R}^{d}$ allows controlling the complexity of some function classes indexed by the parameter $\omega$. Without additional assumptions on the objectives, we obtain a constant $C$ that grow with the diameter of the subset $\Omega$. 


To illustrate the implications of \cref{th:generalizationBounds}, we provide two convergence results for bilevel gradient methods in \cref{sec:gradientMethods} and defer the discussion on the general proof strategy to \cref{sec:sketch_proof} with a full proof provided in \cref{app:sec_conv}.

\subsection{Applications to Empirical Bilevel Gradient Methods}
\label{sec:gradientMethods}
A typical strategy to solve \eqref{eq:kbo} is to obtain empirical samples and solve \eqref{eq:kbo_app} using a bilevel optimization algorithm. Note that this is a finite-dimensional problem, and the lower-level is typically strongly convex. Our results allow to provide statistical guarantees for such approaches. We start with the simplest possible gradient algorithm for the unconstrained problem, $\mathcal{C} = \mathbb{R}^d$, given $\eta > 0$:
\begin{align}
    \label{eq:empiricalBilevelGradient}
    \omega_{t+1} &= \omega_t-\eta\nabla\widehat{\mathcal{F}}(\omega_t) & t\geq 0.
\end{align}
The algorithm requires access to the (almost surely) strongly convex inner-level solution and its derivative which can be obtained using implicit differentiation.

\begin{corollary}[Generalization for bilevel gradient descent]
    \label{cor:gradient}
    Consider \cref{assump:compact,assump:convexity_lin,assump:K_bounded,assump:reg_lin_lout} and a fixed $\lambda > 0$. Assume furthermore that $\mathbf{K}$ in \eqref{eq:kbo_app} is almost surely definite and that there is $c>0$ such that $\inf_{\omega,v,y} \ell_{out}(\omega,v,y) - c\|\omega\|^2 > - \infty$. Fix $\omega_0 \in \mathbb{R}^d$ and assume that $\omega_{t}$ is given by \eqref{eq:empiricalBilevelGradient} for all $t \geq 0$. Then there are $\bar{\eta} > 0$ and a constant $\bar{c}> 0$ such that for any $0<\eta<\bar{\eta}$, and for any $t > 0$, 
    \begin{align*}
        \mathbb{E}\Big[ \min_{i = 0,\ldots,t} \left\|\nabla\mathcal{F}(\omega_i)\right\|\Big] & \leq\bar{c}\left(\frac{1}{\sqrt{m}}+\frac{1}{\sqrt{n}}+\frac{1}{\sqrt{t+1}}\right),\\
        \mathbb{E}\Big[ \underset{i \to \infty}{\lim\sup} \left\|\nabla\mathcal{F}(\omega_i)\right\|\Big] & \leq\bar{c}\left(\frac{1}{\sqrt{m}}+\frac{1}{\sqrt{n}}\right).
    \end{align*}
\end{corollary}

    \begin{proof} Consider that $\inf_{\omega,v,y} \ell_{out}(\omega,v,y) - c\|\omega\|^2 \geq 0$, which entails $\ell_{out}(\omega,v,y) \geq c\|\omega\|^2$ for all $v,y$. Using \Cref{prop:bound_hstaromega} and setting $B = \sup_{y \in \mathcal{Y} } |\partial_v \ell_{in}(\omega_0,0,y)|$, we have almost surely
    \begin{align*}
        \widehat{\mathcal{F}}(\omega_0) \leq \max_{|v| \leq B\kappa/\lambda, y \in \mathcal{Y}} \ell_{out}(\omega_{0}, v, y) \coloneqq \bar{\ell}.
    \end{align*}
    Therefore, for any $\omega$ such that $ \widehat{\mathcal{F}}(\omega) \leq \widehat{\mathcal{F}}(\omega_0) $, we have $\|\omega\|^2 \leq \bar{\ell} / c$. Set $\Omega$ the ball of radius $\sqrt{\bar{\ell} / c}$ and center $0$. Using the fact that $\widehat{\nabla\mathcal{F}} = \nabla\widehat{\mathcal{F}}$ in \Cref{prop:equivalenceEstimates} and the representation in $\eqref{eq:functionalGradientEstimate}$, it is clear from \Cref{prop:lip_hstaromega} and the $C^3$ assumption that $\nabla \widehat{\mathcal{F}} $ is Lipschitz on $\Omega$ with a deterministic constant $L$. It follows from standard analysis of the gradient algorithm for nonconvex $\widehat{\mathcal{F}}$ with Lipschitz gradient (see, \textit{e.g.}, \citep[Theorems~4.25,~4.26]{beck2014introduction}) that setting $\bar{\eta} = 1 / L$, almost surely, 
    \begin{itemize}
        \item $\widehat{\mathcal{F}}(\omega_t) \leq \widehat{\mathcal{F}}(\omega_0) $ and $\omega_t \in \Omega$ for all $t \geq 0$.
        \item $\nabla \widehat{\mathcal{F}}(\omega_t) \to 0$ as $t \to \infty$.
        \item $\min_{i = 0,\ldots,t} \left\|\nabla\widehat{\mathcal{F}}(\omega_i)\right\| \leq \bar{c} / \sqrt{t+1}$ for all $t\geq0$, where $\bar{c}$ is a deterministic constant.
    \end{itemize}
    The corollary then follows by combining \Cref{prop:equivalenceEstimates} and the uniform bound in \Cref{th:generalizationBounds}.
\end{proof}
\begin{remark}
    The additional assumption on $\ell_{out}$ is a device to ensure a priori almost sure boundedness of the sequence. It is rather mild as it can be enforced by a small perturbation of the form $(\omega,v,y) \mapsto \ell_{out}(\omega,v,y) + c\|\omega\|^2$ assuming $\ell_{out} \geq 0$, which is typical in applications. Any other device ensuring a priori boundedness could be considered. The assumption on $\K$ is satisfied almost surely for most kernels.
\end{remark}

Now, considering the constrained problem and assuming $\mathcal{C}$ is convex compact, the projected gradient descent initialized at $\omega_0 \in \mathcal{C}$, given $\eta > 0$, iterates the following recursion:
\begin{align*}
    \omega_{t+1} = \Pi_{\mathcal{C}}(\omega_t - \eta \nabla \widehat{\mathcal{F}}(\omega_t))
\end{align*}
for all $t \geq 0$, where $\Pi_{\mathcal{C}}$ denotes the orthogonal projection on $\mathcal{C}$. The algorithmic requirements are the same as the gradient algorithm, with the addition of the orthogonal projection, which is a cheap operation for basic sets such as balls. For constrained optimization, the optimality condition should take the constraints into account. We consider the gradient mapping, see \citep[Section~10.3]{beck2017first},
\begin{equation*}\label{eq:gradientMappings}
    \begin{aligned}
        \widehat{G}_\eta &\colon \omega \mapsto \frac{1}{\eta} \left(\omega - \Pi_{\mathcal{C}}( \omega - \eta \nabla \widehat{\mathcal{F}}(\omega))\right),\\
    G_\eta &\colon \omega \mapsto \frac{1}{\eta} \left(\omega - \Pi_{\mathcal{C}}( \omega - \eta \nabla \mathcal{F}(\omega))\right).
    \end{aligned}
\end{equation*}
This captures stationarity for the recursion, and any local minimum of $\mathcal{F}$ on $\mathcal{C}$ satisfies $G_\eta = 0$ for all $\eta > 0$.

\begin{corollary}[Generalization for bilevel projected gradient descent]
    \label{cor:projectedGradient}
    Consider \cref{assump:compact,assump:convexity_lin,assump:K_bounded,assump:reg_lin_lout} and a fixed $\lambda > 0$. Assume furthermore that $\mathbf{K}$ in \eqref{eq:kbo_app} is almost surely definite and $\mathcal{C}$ is convex compact. Assume that $\omega_{t+1} = \omega_t-\eta\nabla\widehat{\mathcal{F}}(\omega_t)$ for all $t \geq 0$. Then there are $\bar{\eta} > 0$ and a constant $\bar{c}> 0$ such that for any $0<\eta<\bar{\eta}$, and for any $t > 0$, 
    \begin{align*}
        \mathbb{E}\Big[ \min_{i = 0,\ldots,t} \left\|G_\eta(\omega_i)\right\|\Big] & \leq\bar{c}\left(\frac{1}{\sqrt{m}}+\frac{1}{\sqrt{n}}+\frac{1}{\sqrt{t+1}}\right),\\
        \mathbb{E}\Big[ \underset{i \to \infty}{\lim\sup} \left\|G_\eta(\omega_i)\right\|\Big] & \leq\bar{c}\left(\frac{1}{\sqrt{m}}+\frac{1}{\sqrt{n}}\right).
    \end{align*}
\end{corollary}

\begin{proof}
    We choose $\Omega = \mathcal{C}$.
    All iterates obviously remain in $\Omega$. Similarly as in the proof of \Cref{cor:gradient}, $\nabla \widehat{\mathcal{F}}$ is Lipschitz and $\mathcal{F}$ is bounded on $\Omega$ with deterministic constants. It then follows using classical analysis of nonconvex projected gradient algorithm (see, \textit{e.g.}, \citep[Theorem~10.15]{beck2017first}), that for small $\eta$, almost surely $ \min_{i = 0,\ldots,t} \left\|\widehat{G}_\eta(\omega_i)\right\| \leq \bar{c} / \sqrt{t+1}$ for a determinstic $\bar{c}>0$, and that $\widehat{G}_\eta(\omega_i) \to 0$ as $i \to \infty$. Using the fact that the orthogonal projection is $1$-Lipschitz, (see, \textit{e.g.}, \citep[Theorem~6.42]{beck2017first}), we have for all $\omega \in \Omega$
    \begin{align*}
        \|\widehat{G}_\eta(\omega) - G_\eta(\omega)\| \leq \|\nabla \widehat{\mathcal{F}}(\omega) - \nabla\mathcal{F}(\omega)\|.
    \end{align*}
    The result follows by combining \Cref{prop:equivalenceEstimates} and the uniform bound in \Cref{th:generalizationBounds}.
\end{proof}

\section{General Proof Strategy for \cref{th:generalizationBounds}}\label{sec:proof_strategy}
\label{sec:sketch_proof}
The main strategy behind the proof of \cref{th:generalizationBounds} in \cref{app:sec_conv} consists in three steps: (step 1) obtaining a point-wise error decomposition of the errors into manageable error terms that holds almost surely for any $\omega\in \Omega$, then applying maximal inequalities to suitable empirical processes 
(step 2) and some degenerate $U$-processes (step 3) to control each of these terms. 
The final error bounds are obtained by combining  all these bounds as shown in \cref{app_sub:main_proof}.  

\textbf{Step 1: Point-wise error decomposition. } 
A main challenge in controlling the errors in \cref{th:generalizationBounds} is the non-linear dependence of both estimators $\widehat{\mathcal{F}}(\omega)$ and $\widehat{\nabla\mathcal{F}}(\omega)$ on the empirical distributions, as they are obtained via a plug-in procedure. We address this challenge by breaking down the error into components based on the discrepancies between expected values and their empirical counterparts of individual point-wise losses and their derivatives, evaluated at the optimal solution  $h_\omega^{\star}$. 
Specifically, we denote by $\delta_\omega^{out}$ and $\delta_\omega^{in}$ the errors on the objectives defined as: 
\begin{align*}
\delta_\omega^{out}&\coloneqq \verts{L_{out}(\omega, h^\star_\omega)-\widehat{L}_{out}(\omega, h^\star_\omega)},\\
\delta_\omega^{in}&\coloneqq\verts{L_{in}(\omega, h^\star_\omega)-\widehat{L}_{in}(\omega, h^\star_\omega)}. 
\end{align*}
Additionally, we quantify the errors between the partial derivatives of these objectives and their empirical counterparts. To simplify our proof outline, we slightly abuse notation by denoting $\partial_{h}\delta_\omega^{out}$, $\partial_{h}\delta_\omega^{in}$, $\partial_{\omega}\delta_\omega^{out}$, $\partial_{\omega,h}^2\delta_\omega^{in}$ and $\partial_{h}^2\delta_\omega^{in}$ to refer to these errors in terms of partial derivatives. For instance,  $\partial_{h}\delta_\omega^{out}$ is defined as  $\Verts{\partial_h L_{out}(\omega, h^\star_\omega)-\partial_h\widehat{L}_{out}(\omega, h^\star_\omega)}_\mathcal{H}$, with similar definitions for the other error terms that can be found in \cref{app:subsec_point_est}. 
The next proposition formalizes the above discussions. 
\begin{proposition}[Approximation bounds]\label{prop:grad_app_bound_main}
    Let $\Omega$ be an arbitrary compact subset of $\mathbb{R}^d$. Under \cref{assump:compact,assump:convexity_lin,assump:K_bounded,assump:reg_lin_lout}, the following holds almost surely, for any $\omega\in\Omega$:
    \begin{align*}
        \verts{\mathcal{F}(\omega)-\widehat{\mathcal{F}}(\omega)} &
        \leq C \parens{\delta_\omega^{out}+\partial_h\delta_\omega^{in}},\\
        \Verts{\nabla\mathcal{F}(\omega)-\widehat{\nabla\mathcal{F}}(\omega)}
        &\leq C(\partial_\omega\delta_\omega^{out}+\partial_h\delta_\omega^{out}+\partial_h^2\delta_\omega^{in}\\
        &\qquad+\partial_{\omega, h}^2\delta_\omega^{in}+\partial_h\delta_\omega^{in}),
    \end{align*}
where $C$ is a positive constant that depends only $\Omega$, dimension $d$, regularization parameter $\lambda$, $\kappa$ and local upper-bounds on  $\ell_{in}$, $\ell_{out}$ and their partial derivatives over suitable compact set.
\end{proposition}
A more detailed version of this proposition, including the exact constants, along with its proof are  provided in \cref{prop:grad_app_bound} of \cref{app:subsec_point_est}.
The error terms arising in the above decomposition are more amenable to a statistical analysis using empirical process theory as we discuss next. 

\textbf{Step 2: Maximal inequalities for empirical processes.} 
Some of the error terms, namely $\Dout$ and $\Doutw$, can be controlled directly using empirical process theory. 
For instance, $\Dout$ is associated to the family of random functions $\sqrt{m}\parens{L_{out}(\omega,h_{\omega}^{\star})-\widehat{L}_{out}(\omega,h_{\omega}^{\star})}_{\omega\in \Omega}$, which defines an empirical process, a scaled and centered empirical average of real-valued functions indexed by the parameter $\omega$. Thus, provided that suitable estimates of the complexity of the class are available (as measured by its packing number in \cref{prop:estimate_covering_numbers} of \cref{sec_app:max_in_bound_lip}), which is easy to obtain in our setting, we show in \cref{prop:exp_uni_bound} of \cref{app_subsec:max_in} that a maximal inequality of the form below follows from classical results on empirical processes:
$$\mathbb{E}_\mathbb{Q}\brackets{\sup_{\omega\in \Omega} 
\overbrace{\verts{L_{out}(\omega,h_{\omega}^{\star})-\widehat{L}_{out}(\omega,h_{\omega}^{\star})} }^{\Dout}
} \leq \frac{C}{\sqrt{m}}.
$$

{\bf Step 3: Maximal inequalities for degenerate U-processes.}
Unfortunately, the approach in step 2 cannot be readily used for the remaining error terms involving partial derivatives w.r.t. $h$ ($\Douth$, $\Dinh$, $\Dinwh$ and $\Dinhh$). 
These error terms are associated to processes that are not real-valued anymore but take values in an infinite-dimensional space. 
While the recent work in \cite{park2023towards} develops an empirical process theory for functions taking values in a vector space, the provided complexity estimates would result in unfavorable dependence on the sample size. 
Instead, we leverage the structure of the RKHS to control these errors using maximal inequalities for suitable degenerate $U$-processes of order $2$ indexed by the parameter $\omega$ and for which such inequalities were provided in the seminal works of \citet{sherman1994maximal,nolan1987u}. 
$U$-processes of order 2 are generalization of empirical processes that involve empirical averages of real-valued functions which depend on pairs of samples, instead of a single one as in empirical processes. 
These functions arise, in our case, when taking  the square of the error term $\Douth$, for instance, and exploiting the reproducing property in the RKHS. This approach, presented in \cref{prop:exp_uni_bound_2} of \cref{app_subsec:max_in}, allows us to obtain maximal inequalities of the form:
$$\mathbb{E}_\mathbb{Q}\brackets{\sup_{\omega\in \Omega} 
\Douth}\leq\mathbb{E}_\mathbb{Q}\brackets{\sup_{\omega\in \Omega} 
\parens{\Douth}^2}^\frac{1}{2}
 \leq \frac{C}{\sqrt{m}}.
$$
Combining the maximal inequalities from step 2 and 3 with the error decomposition from step 1 allows to obtain the result of  \cref{th:generalizationBounds}.  


\textbf{Discussion. }Alternative approaches to $U$-processes could be used to derive generalization bounds, although these would result in degraded sample dependence. Specifically, one could employ a variational formulation of the RKHS norm appearing in some of the error terms, such as $\Douth$, to express them as the error of some real-valued empirical process to which standard results could be applied. However, this comes at the cost of considering processes indexed not only by the finite-dimensional parameter $\omega$, but also by functions in a unit RKHS ball, in contrast with our approach based on $U$-processes indexed by finite-dimensional vectors. Consequently, these families have much larger complexities as measured by their covering/packing numbers \citep[Lemma~D.2]{yang2020function}, which directly impacts the generalization rate. 
Our proposed approach bypasses this challenge by using real-valued $U$-processes indexed by finite-dimensional parameters, at the expense of employing a more general empirical process theory for degenerate $U$-processes \cite{sherman1994maximal}.

\section{Conclusion and Perspectives}
In this work, we established the first statistical generalization bounds for \eqref{eq:kbo}. 
This paper is a first step in providing generalization results for bilevel gradient-based methods in a non-parametric setting. 
We showcased the applicability of these bounds on the simplest algorithms and expect that our results can be readily extended to state-of-the-art methods such as those discussed in \cite{arbel2022nonconvex,ghadimi2018approximation,chen2021closing,Dagreou2022SABA}. The results concerning $U$-processes, however, are derived under the restrictive assumption of i.i.d.~data. Relaxing such assumption to handle Markovian data is a promising direction for future work, as it would allow considering reinforcement learning applications where bilevel optimization methods have shown promising results \cite{Nikishin:2022}. 

\section*{Acknowledgements}
This work was supported by the ANR project BONSAI (grant ANR-23-CE23-0012-01). EP thank AI Interdisciplinary Institute ANITI funding, through the French ``Investments for the Future -- PIA3'' program under the grant agreement ANR-19-PI3A0004, Air Force Office of Scientific Research, Air Force Material Command, USAF, under grant numbers FA8655-22-1-7012. EP thank TSE-P, acknowledge support from ANR Chess, grant ANR-17-EURE-0010, ANR Regulia, support from IUF. EP and SV acknowledge support from ANR MAD. SV thanks PEPR PDE-AI (ANR-23-PEIA-0004) and the chair 3IA Côte d'Azur BOGL.

\bibliography{ref}
\bibliographystyle{icml2025}


%
%
%
%
%
\newpage
\appendix
\onecolumn

\textbf{\LARGE Appendices}

\textbf{Roadmap. }We begin by presenting and establishing regularity properties of the objective functions in \cref{sec_app:reg}. In \cref{sec:grad_est}, we introduce the gradient estimators. \Cref{sec_app:prel_res} is dedicated to proving the boundedness and Lipschitz continuity of $h^\star_\omega$ and $\hat{h}_\omega$, along with local boundedness and Lipschitz properties of $\ell_{in}$, $\ell_{out}$ and their derivatives. The generalization results are provided in \cref{app:sec_conv}. In \cref{sec_app:max_in_bound_lip}, we establish maximal inequalities for bounded and Lipschitz families of functions. Differentiability properties of the objectives are studied in \cref{sec:proof_diff_results}. Finally, \cref{ab_sec:aux} contains auxiliary technical lemmas used throughout the proofs.

\textbf{Notations. }$\|\cdot\|$ denotes the Euclidean norm in $\mathbb{R}^d$, $\|\cdot\|_\mathcal{H}$ denotes the norm in the RKHS $\mathcal{H}$, $\|\cdot\|_{\op}$ denotes the operator norm, and $\|\cdot\|_{\hs}$ denotes the Hilbert-Schmidt norm. $\langle\cdot,\cdot\rangle_\mathcal{H}$ denotes the inner product on $\mathcal{H}$, and $\langle\cdot,\cdot\rangle_{\hs}$ denotes the Hilbert-Schmidt inner product. $K(x,\cdot)$ denotes the feature map, for any $x\in\mathcal{X}$. For any two normed spaces $E$ and $F$, $\mathcal{L}(E,F)$ denotes the space of continuous linear operators from $E$ to $F$. For any two probability distributions $\mathcal{P}$ and $\mathcal{Q}$, $\mathcal{P}\otimes\mathcal{Q}$ denotes the product measure of $\mathcal{P}$ and $\mathcal{Q}$. Given two Hilbert spaces $\left(H_1,\langle\cdot,\cdot\rangle_{H_1}\right)$ and $\left(H_2,\langle\cdot,\cdot\rangle_{H_2}\right)$, the tensor product of $u\in H_1$ and $v\in H_2$, denoted by $u\otimes v$, is an operator from $H_2$ to $H_1$ defined, for any $e\in H_2$, as $\left(u\otimes v\right)e=u\left\langle v, e\right\rangle_{H_2}$. For any $v_1,\ldots,v_n\in\mathbb{R}$, $\diag(v_1,\ldots,v_n)\in\mathbb{R}^{n\times n}$ denotes a diagonal matrix of size $n\times n$, where the diagonal entries are $v_1,\ldots,v_n$ and all the off-diagonal entries are $0$. $\mathbbm{1}_m$ denotes a vector of size $m$ where all entries are $1$. $\mathbbm{1}_{n\times n}$ denotes an $n\times n$ matrix where all entries are $1$. For any vector space $V$ over $\mathbb{R}$, $\Id_V$ denotes the identity operator on $V$. Given a compact set $\mathcal{K}$, $\diam(\mathcal{K})$ denotes its diameter. $v^\top$ denotes the transpose of either a vector or a matrix, depending on the context.

\section{Regularity and Differentiability Results}\label{sec_app:reg}
\subsection{Regularity of the objectives}\label{sec:reg_ob}
The following propositions establish differentiability of considered objectives. We defer their proof to \cref{sec:proof_diff_results}.
\begin{proposition}[{Differentiability of $L_{in}$ and $L_{out}$}]\label{prop:fre_diff_L}
	Under \cref{assump:compact,assump:reg_lin_lout,assump:K_bounded}, for any $(\omega,h)\in {\mathbb{R}^d}\times \mathcal{H}$, the functions $L_{in}$, $L_{out}$  admit finite values at $(\omega,h)$, are jointly differentiable in $(\omega,h)$, with gradient given by:
	\begin{align*}
		    \partial_\omega L_{out}(\omega, h)&=\mathbb{E}_{\mathbb{Q}}\left[\partial_\omega \ell_{out}(\omega, h(x), y)\right]\in\mathbb{R}^d, &\partial_h L_{out}(\omega, h)=&\mathbb{E}_{\mathbb{Q}}\left[\partial_v \ell_{out}(\omega, h(x), y)K(x,\cdot)\right]{\in\mathcal{H}}, \\
\partial_\omega L_{in}(\omega, h)&=\mathbb{E}_{\mathbb{P}}\left[\partial_\omega \ell_{in}(\omega, h(x), y)\right]\in\mathbb{R}^d, &\partial_h L_{in}(\omega, h)=&\mathbb{E}_{\mathbb{P}}\left[\partial_v \ell_{in}(\omega, h(x), y)K(x,\cdot)\right] + \lambda h{\in\mathcal{H}}.
	\end{align*}
	Similarly, the empirical estimates $\widehat{L}_{in}$ and $\widehat{L}_{out}$ admit finite values, and are differentiable with gradients admitting similar expressions as above with $\mathbb{P}$ and $\mathbb{Q}$ replaced by their empirical estimates $\hat{\mathbb{P}}_n$ and $\hat{\mathbb{Q}}_m$.
\end{proposition}

\begin{proposition}[{Differentiability of $\partial_h L_{in}$}]\label{prop:fre_diff_L_v}
	Under \cref{assump:compact,assump:reg_lin_lout,assump:K_bounded}, for any $(\omega,h)\in{\mathbb{R}^d}\times \mathcal{H}$, the functions $(\omega,h)\mapsto \partial_{h} L_{in}(\omega,h)$ is differentiable with partial derivatives given by: 
\begin{align*}
    \partial_{\omega, h}^2 L_{in}(\omega, h)&=\mathbb{E}_{\mathbb{P}}\left[\partial_{\omega, v}^2\ell_{in}(\omega, h(x), y)K(x,\cdot)\right]\in\mathcal{L}(\mathcal{H},\mathbb{R}^d),\\
  \partial_h^2 L_{in}(\omega, h)&=\mathbb{E}_{\mathbb{P}}\left[\partial_v^2 \ell_{in}(\omega, h(x), y)K(x,\cdot)\otimes K(x,\cdot)\right] + \lambda\Id_\mathcal{H}\in\mathcal{L}(\mathcal{H},\mathcal{H}).\label{eq:partial2_h_Lin}
\end{align*}
Moreover, {for any $\omega\in\mathbb{R}^d$ and $h\in\mathcal{H}$,} the operators $ \partial_{\omega, h}^2 L_{in}(\omega, h)$ and $ \partial_{h}^2 L_{in}(\omega, h)-\lambda \Id_{\mathcal{H}}$ are Hilbert-Schmidt, i.e., bounded operators with finite Hilbert-Schmidt norm. The same conclusions hold for the empirical estimate $(\omega,h)\mapsto \partial_{h}\widehat{L}_{in}(\omega,h)$ with partial derivatives admitting similar expressions as above with $\mathbb{P}$ replaced by its empirical estimate $\hat{\mathbb{P}}_n$.
\end{proposition}

\begin{proposition}[Strong convexity of the inner objective in its second variable and invertibility of the Hessians]\label{prop:strong_convexity_Lin}
Under \cref{assump:convexity_lin,assump:compact,assump:reg_lin_lout,assump:K_bounded}, $h\mapsto L_{in}(\omega, h)$ and $h\mapsto \widehat{L}_{in}(\omega, h)$ are $\lambda$-strongly convex for any $\omega\in{\mathbb{R}^d}$. Moreover, {for any $\omega\in\mathbb{R}^d$ and $h\in\mathcal{H}$,} the Hessian operators $\partial_{h}^2 L_{in}(\omega, h)$ and $\partial_{h}^2 \widehat{L}_{in}(\omega, h)$ are invertible with their operator norm bounded by $\frac{1}{\lambda}$.
\end{proposition}
\begin{proof}
{By \cref{assump:convexity_lin}}, we know that $v\mapsto \ell_{in}(\omega, v, y)$ is convex {for any $\omega\in\mathbb{R}^d$ and $y\in\mathcal{Y}$}. Moreover, by \cref{prop:fre_diff_L}, $(x,y)\mapsto \ell_{in}\parens{\omega, h(x),y}$ is integrable for any {$\omega\in\mathbb{R}^d$ and $h\in \mathcal{H}$}. 
Consequently, by integration, we directly deduce that $h\mapsto \mathbb{E}_{\mathbb{P}}\brackets{\ell_{in}\parens{\omega, h(x),y}}$ is convex for any $\omega\in {\mathbb{R}^d}$. Finally, $h \mapsto L_{in}(\omega, h) \coloneqq  \mathbb{E}_{\mathbb{P}}\brackets{\ell_{in}\parens{\omega, h(x),y}} + \frac{\lambda}{2} \Verts{h}_{\mathcal{H}}^2 $ must be $\lambda$-strongly convex{, for any $\omega\in\mathbb{R}^d$}, as a sum of a convex function and a $\lambda$-strongly convex function. Similarly, we deduce that $h\mapsto \widehat{L}_{in}(\omega,h)$ is $\lambda$-strongly convex{, for any $\omega\in\mathbb{R}^d$}. Invertibility follows from the expression of the Hessian operator in \cref{prop:fre_diff_L_v}  
\end{proof}


\subsection{Differentiability of the value function}
\begin{proposition}[Total functional gradient $\nabla\mathcal{F}$]\label{prop:tot_grad_int}
Assume \cref{assump:convexity_lin,assump:K_bounded,assump:reg_lin_lout} hold. For any $\omega\in{\mathbb{R}^d}$, the total functional gradient $\nabla\mathcal{F}(\omega)$ satisfies:
\begin{equation}\label{eq:tot_grad}
    \nabla\mathcal{F}(\omega)=\partial_\omega L_{out}(\omega, h^\star_\omega)+\partial_{\omega,h}^2 L_{in}(\omega, h^\star_\omega)a^\star_\omega\in{\mathbb{R}^d},
\end{equation}
where $a^\star_\omega$ 
is the unique minimizer of the following quadratic objective:
\begin{equation}\label{eq:ladj}
    L_{adj}(\omega,a)\coloneqq\frac{1}{2}\left\langle a, H_{\omega}a\right\rangle_\mathcal{H}+\left\langle a, d_\omega\right\rangle_\mathcal{H},\quad\text{for any }a\in\mathcal{H},
\end{equation}
with $H_{\omega}\coloneqq\partial_h^2 L_{in}(\omega, h^\star_\omega):\mathcal{H}\to\mathcal{H}$ being the Hessian operator and $d_\omega\coloneqq\partial_h L_{out}(\omega, h^\star_\omega)\in\mathcal{H}$.
\end{proposition}
\begin{proof}\label{proof:eq:tot_grad}

By applying \cref{prop:fre_diff_L,prop:strong_convexity_Lin},  
we know that $h\mapsto L_{in}(\omega,h)$ has finite values, is $\lambda$-strongly convex and Fr\'echet differentiable. Moreover, by \cref{prop:fre_diff_L_v}, $\partial_h L_{in}$ is Fr\'echet differentiable on ${\mathbb{R}^d}\times\mathcal{H}$, and, a fortiori, Hadamard differentiable. Therefore, by the functional implicit differentiation theorem  \citep{ioffe1979theory}, we deduce that the map $\omega\mapsto h_{\omega}^{\star}$ is uniquely defined and is Fr\'echet differentiable with Jacobian $\partial_\omega h^\star_\omega$ solving the following linear system for any $\omega \in{\mathbb{R}^d}$:
\begin{equation*}
    \partial_{\omega,h}^2 L_{in}(\omega, h^\star_\omega) + \partial_\omega h^\star_\omega\partial_h^2 L_{in}(\omega, h^\star_\omega)=0.
\end{equation*}
Using that $\partial_h^2 L_{in}(\omega, h^\star_\omega)$ is invertible by \cref{prop:strong_convexity_Lin}, we can express $\partial_\omega h^\star_\omega$ as:  
\begin{equation*}
    \partial_\omega h^\star_\omega=-\partial_{\omega,h}^2 L_{in}(\omega, h^\star_\omega)\left(\partial_h^2 L_{in}(\omega, h^\star_\omega)\right)^{-1}.
\end{equation*}
Furthermore, $L_{out}$ is jointly Fr\'echet differentiable by application of \cref{prop:fre_diff_L}, so that $\omega\mapsto\mathcal{F}(\omega)$ is also differentiable by composition of the functions {$(\omega,h)\mapsto L_{out}(\omega, h)$ and $\omega\mapsto (\omega,h^\star_\omega)$}. 
For a given $\omega\in {\mathbb{R}^d}$, the gradient of $\mathcal{F}$ is then given by the chain rule: 
\begin{equation}\label{eq:tot_grad_int}
    \nabla\mathcal{F}(\omega)=\partial_\omega L_{out}(\omega, h^\star_\omega)+\partial_\omega h^\star_\omega\partial_h L_{out}(\omega, h^\star_\omega).
\end{equation}
Substituting the expression of $\partial_\omega h^\star_\omega$ into \cref{eq:tot_grad_int} yields:
\begin{equation*}
    \nabla\mathcal{F}(\omega)=\partial_\omega L_{out}(\omega, h^\star_\omega)-\partial_{\omega,h}^2 L_{in}(\omega, h^\star_\omega)\left(\partial_h^2 L_{in}(\omega, h^\star_\omega)\right)^{-1}\partial_h L_{out}(\omega, h^\star_\omega).
\end{equation*}
To conclude, it suffices to notice that the function $a_\omega^{\star}$ appearing in \cref{eq:tot_grad} must be equal to $-H_{\omega}^{-1}d_{\omega}$. Indeed, $a_\omega^{\star}$ is defined as the minimizer of the quadratic objective $L_{adj}(\omega,a)$ in \cref{eq:ladj} which is strongly convex since the Hessian operator is lower-bounded by $\lambda \Id_{\mathcal{H}}$. Consequently, the minimizer $a_\omega^{\star}$ exists and is uniquely characterized by the optimality condition:
\begin{align*}
	H_{\omega}a_\omega^{\star}+d_{\omega}=0. 
\end{align*}
The above equation is a linear system in $\mathcal{H}$ whose solution is given by $a_\omega^{\star}\coloneqq-H_{\omega}^{-1}d_{\omega}$.
\end{proof}

\section{Gradient Estimators}\label{sec:grad_est}
\begin{proposition}[Expression of $\nabla \widehat{\mathcal{F}}(\omega)$ by implicit differentiation]\label{prop:est_2}
Under  \cref{assump:compact,assump:convexity_lin,assump:K_bounded,assump:reg_lin_lout}, for any $\omega\in{\mathbb{R}^d}$, the gradient $\nabla\widehat{\mathcal{F}}(\omega)$ of the discretized functional bilevel optimization problem~\eqref{eq:kbo_app} is given by:
\begin{equation*}
    \nabla\widehat{\mathcal{F}}(\omega)=\frac{1}{m}\Dthree\mathbbm{1}_m-\frac{1}{m}\Dfour\M^{-1}\mathbf{u}\in\mathbb{R}^d,
\end{equation*}
where $\K$ and $\Kbar$ are the Gram matrices in $\mathbb{R}^{n\times n}$ and $\mathbb{R}^{m\times n}$  with entries given by  $\K_{ij}\coloneqq K(x_i,x_j)$ and $\K_{ij}\coloneqq K(\tilde{x}_j,x_j)$, and $\M$, $\mathbf{u}$, $\Done$, $\Dtwo$, $\Dthree$ and $\Dfour$ are defined as:
\begin{align*}
\M&\coloneqq\K\Dtwo+n\lambda\mathbbm{1}_{n\times n}{\in\mathbb{R}^{n\times n}},
\qquad &\mathbf{u} &\coloneqq \Kbar^\top\Done{\in\mathbb{R}^n},\\
	\Done &\coloneqq\parens{\partial_v\ell_{out}\parens{\omega,\hat{h}_\omega(\tilde{x}_j),\tilde{y}_j}}_{1\leq j\leq m}\in\mathbb{R}^m,\quad &
\Dthree &\coloneqq\parens{\partial_\omega \ell_{out}(\omega, \hat{h}_\omega(\tilde{x}_j), \tilde{y}_j)}_{1\leq j\leq m}\in\mathbb{R}^{d\times m},\\
\Dtwo &\coloneqq\diag\parens{\parens{\partial_v^2\ell_{in}(\omega,\hat{h}_\omega(x_i),y_i)}_{1\leq i\leq n}}\in\mathbb{R}^{n\times n},\qquad & 
\Dfour &\coloneqq\parens{\partial_{\omega,v}^2\ell_{in}(\omega,\hat{h}_\omega(x_i),y_i)}_{1\leq i\leq n}\in\mathbb{R}^{d\times n}.
\end{align*}
\end{proposition}
\begin{proof}
{Let $\omega\in\mathbb{R}^d$.} Recall the expression of $\widehat{\mathcal{F}}(\omega)$:
\begin{align*}
	&\widehat{\mathcal{F}}(\omega) \coloneqq \frac{1}{m}\sum_{j=1}^m \ell_{out}\parens{\omega, \hat{h}_{\omega}(\tilde{x}_j),\tilde{y}_j}\\
	\text{s.t.}\quad&\hat{h}_{\omega}=\argmin_{h\in \mathcal{H}} \widehat{L}_{in}(\omega,h)\coloneqq\frac{1}{n}\sum_{i=1}^n\ell_{in}\parens{\omega, h(x_i), y_i} +\frac{\lambda}{2}\Verts{h}_{\mathcal{H}}^2. 
\end{align*}
By the representer theorem, it is easy to see that $\hat{h}_{\omega}$ must be a linear combination of $K(x_1,\cdot),\ldots,K(x_n,\cdot)$:
\begin{equation}\label{eq:h}
	\hat{h}_{\omega} = \sum_{i=1}^n (\hat{\gamma}_{\omega})_i K(x_i,\cdot).
\end{equation}
Hence, finding $\hat{h}_{\omega}$ amounts to minimizing $\widehat{L}_{in}(\omega,h)$ over the span of $(K(x_1,\cdot),\ldots, K(x_n,\cdot))$, \textit{i.e.}, over functions $h^{\gamma}$ of the form $h^{\gamma} = \sum_{i=1}^n \gamma_i K(x_i,\cdot)$ for $\gamma\in \mathbb{R}^n$. Restricting the objective to such functions results in the following inner optimization problem which is finite-dimensional:
\begin{align*}
	\hat{\gamma}_{\omega}\coloneqq \argmin_{\gamma\in \mathbb{R}^n} \frac{1}{n}\sum_{i=1}^n \ell_{in}\parens{\omega, (\K \gamma)_i, y_i }+\frac{\lambda}{2} \gamma^{\top}\K \gamma ,
\end{align*} 
where  we used that $(h^{\gamma}(x_i))_{1\leq i\leq n} = \K\gamma$ and $\Verts{h^{\gamma}}^2_{\mathcal{H}} = \gamma^{\top}\K \gamma$. 
Similarly, using that $(h^{\gamma}(\tilde{x}_j))_{1\leq j\leq m} = \Kbar\gamma$, we can express $\widehat{\mathcal{F}}(\omega)$ as follows:
\begin{align*}
	\widehat{\mathcal{F}}(\omega) = \frac{1}{m}\sum_{j=1}^m \ell_{out}\parens{\omega, (\Kbar \hat{\gamma}_{\omega})_j,\tilde{y}_j}.
\end{align*}
Differentiating the above expression w.r.t. $\omega$ and applying the chain rule result in:
\begin{align*}
	\nabla \widehat{\mathcal{F}}(\omega) = \frac{1}{m}\sum_{j=1}^m \partial_{\omega}\ell_{out}\parens{\omega, (\Kbar \hat{\gamma}_{\omega})_j,\tilde{y}_j}  + \frac{1}{m}\sum_{j=1}^m  (\partial_{\omega}\hat{\gamma}_{\omega}\Kbar^{\top})_j \partial_{v}\ell_{out} \parens{\omega, (\Kbar \hat{\gamma}_{\omega})_j,\tilde{y}_j},
\end{align*}
where $\partial_{\omega}\hat{\gamma}_{\omega}$ denotes the Jacobian of $\hat{\gamma}_{\omega}$. We can further express the above equation in matrix form to get:
\begin{align}\label{eq:vector_form_gradient}
	\nabla \widehat{\mathcal{F}}(\omega) = \frac{1}{m}\Dthree\mathbbm{1}_m+\frac{1}{m}\partial_{\omega}\hat{\gamma}_{\omega} \Kbar^{\top}\Done.
\end{align}
	Moreover, an application of the  implicit function theorem\footnote{In the case where the matrix $\K$ is non-invertible, one needs to restrict $\gamma$ to the orthogonal complement of the null space of $\K$. Such a restriction is valid since the resulting solution $\hat{h}_{\omega}$ will not depend on the component belonging to the null space of $\K$. } allows to directly express the Jacobian $\partial_{\omega}\hat{\gamma}_{\omega}$ as a solution of the following linear system obtained by differentiating the optimality condition  for $\hat{\gamma}_{\omega}$ w.r.t. $\omega$:
\begin{equation*}
    \Dfour\K+\left(\partial_\omega\hat{\gammabf}_\omega\right)\underbrace{\parens{\K\Dtwo+n\lambda\mathbbm{1}_{n\times n} }}_{\M}\K=0.
\end{equation*}
A solution of the form $\partial_{\omega}\hat{\gamma}_{\omega} = -  \Dfour\M^{-1}$ always exists by invertibility of the matrix $\M$. The result follows after replacing $\partial_{\omega}\hat{\gamma}_{\omega}$ by $-\Dfour\M^{-1}$ in \cref{eq:vector_form_gradient}. 
\end{proof}

\begin{lemma}[Estimator of the total functional gradient]\label{prop:est_1}
{Let $\omega\in\mathbb{R}^d$.} Consider the following functional estimator:
\begin{align*}
\widehat{\nabla\mathcal{F}}(\omega)=&\frac{1}{m}\sum_{j=1}^m\partial_\omega\ell_{out}(\omega, \hat{h}_\omega(\tilde{x}_j),\tilde{y}_j)+\frac{1}{n}\sum_{i=1}^n\partial_{\omega,v}^2\ell_{in}(\omega, \hat{h}_\omega(x_i),y_i)\hat{a}_\omega(x_i).
\end{align*}
Then, under  \cref{assump:compact,assump:convexity_lin,assump:K_bounded,assump:reg_lin_lout},  $\widehat{\nabla\mathcal{F}}(\omega)$ admits the following expression:
\begin{equation*}
    \widehat{\nabla\mathcal{F}}(\omega)=\frac{1}{m}\Dthree\mathbbm{1}_m+\frac{1}{n}\Dfour
    \begin{bmatrix}
    	\K & \mathbf{u}
    \end{bmatrix}
    \begin{bmatrix}
    	\hat{\alphabf}_\omega\\
    	\hat{\beta}_\omega
    \end{bmatrix}
    \in\mathbb{R}^d,
\end{equation*}
where $\Done, \Dtwo, \Dthree, \Dfour$ are the same matrices given in \cref{prop:est_2}, while, $\hat{\alphabf}_\omega\in\mathbb{R}^n$ and $\hat{\beta}_\omega\in\mathbb{R}$ are solutions to the linear system:
\begin{equation}\label{eq:linear_system_adjoint}
\begin{bmatrix}
    \M \K & \M \mathbf{u}\\
    \mathbf{u}^{\top}\M & p
\end{bmatrix}\begin{bmatrix}
        \hat{\alphabf}_\omega\\
        \hat{\beta}_\omega
    \end{bmatrix}=-\frac{n}{m}\begin{bmatrix}
        \mathbf{u}\\
 	v
    \end{bmatrix},
\end{equation}
where the vector $\mathbf{u}$  and  matrix $\M$ are the same as in \cref{prop:est_2}, while $p$ and $v$ are non-negative scalars. 

\end{lemma}


\begin{proof}
{Let $\omega\in\mathbb{R}^d$.} We start by providing an expression of $\hat{a}_{\omega}$ as a linear combination of the kernel evaluated at the inner training points $x_i$, \textit{i.e.}, $K(x_i,\cdot)$, and some element  $\xi\in \mathcal{H}$ that we will characterize shortly. From it, we will obtain the expression of $\widehat{\nabla\mathcal{F}}(\omega)$. 

{\bf Expression of $\hat{a}_{\omega}$.}
Recall that $\hat{a}_{\omega}$ is the unique minimizer of $\widehat{L}_{adj}$ in \cref{eq:l_adj}, which admits{, for any $a\in\mathcal{H}$,} the following  simple expression by the reproducing property:
 \begin{equation*}    \widehat{L}_{adj}(\omega, a)=\frac{1}{2n}\sum_{i=1}^n\partial_v^2 \ell_{in}(\omega, \hat{h}_\omega(x_i), y_i)a^2(x_i)+\frac{1}{m}\left\langle a, \overbrace{\sum_{j=1}^m\partial_v\ell_{out}(\omega, \hat{h}_\omega(\tilde{x}_j), \tilde{y}_j)K(\tilde{x}_j,\cdot)}^{\xi}\right\rangle_{\mathcal{H}}+\frac{\lambda}{2}\|a\|_\mathcal{H}^2.
\end{equation*}
 Hence, by application of the representer theorem, it follows that $\hat{a}_{\omega}$ admits an expression of the form:
 \begin{align}\label{eq:a}
 	\hat{a}_{\omega} = \sum_{i=1}^n (\hat{\alphabf}_{\omega})_i K(x_i,\cdot) + \hat{\beta}_{\omega}{\xi}. 
 \end{align}
Therefore, it is possible to recover $\hat{a}_{\omega}$ by minimizing $a\mapsto L_{adj}(\omega,a)$ over the span of $(\xi, K(x_1,\cdot),\ldots,K(x_n,\cdot))$. Hence, to find the optimal coefficients $\hat{\alphabf}_{\omega}\coloneqq ((\hat{\alphabf}_{\omega})_{i})_{1\leq i \leq n}$ and  $\hat{\beta}_{\omega}$ we first need to express the objective $L_{adj}$ in terms of the coefficients $\alphabf \in \mathbb{R}^n$ and $\beta\in \mathbb{R}$ for a given $a^{\alphabf,\beta}\in \mathcal{H}$ of the form $a^{\alphabf,\beta} = \sum_{i=1}^n (\alphabf)_i K(x_i,\cdot) + \beta \xi$. To this end, note that the vector $(\xi(x_1),\ldots,\xi(x_n))$ is exactly equal to $\mathbf{u}= \Kbar^\top\Done$ as defined in \cref{prop:est_2}. Moreover, using the reproducing property, we directly have:
\begin{align*}
	(a^{\alphabf,\beta}(x_i))_{1\leq i\leq n} = \begin{bmatrix}
        \K & 
        \mathbf{u}
    \end{bmatrix}
 \begin{bmatrix}
        \alphabf\\
        \beta
    \end{bmatrix},\qquad \langle a^{\alphabf,\beta},\xi\rangle_{\mathcal{H}} = 
    \begin{bmatrix}
        \mathbf{u}^{\top} & 
        \Verts{\xi}^2_\mathcal{H}
    \end{bmatrix}
    \begin{bmatrix}
        \alphabf\\
        \beta
    \end{bmatrix},\qquad \Verts{a^{\alphabf,\beta}}^2_{\mathcal{H}} = 
    \begin{bmatrix}
        \alphabf^{\top} &
        \beta
    \end{bmatrix}\begin{bmatrix}
        \K & \mathbf{u}\\
        \mathbf{u}^{\top} & \Verts{\xi}^2_\mathcal{H}
    \end{bmatrix}\begin{bmatrix}
        \alphabf\\
        \beta
    \end{bmatrix}. 
\end{align*}
We can therefore express the objective $\widehat{L}_{adj}$ as follows:
\begin{align*}
	\widehat{L}_{adj}(\omega, a^{\alphabf,\beta})= \frac{1}{2n} \begin{bmatrix}
        \alphabf^{\top} &
        \beta
    \end{bmatrix}
    \begin{bmatrix}
        \K \\
        \mathbf{u}^{\top}
    \end{bmatrix}
    \Dtwo\begin{bmatrix}
        \K & 
        \mathbf{u}
    \end{bmatrix}
 \begin{bmatrix}
        \alphabf\\
        \beta
    \end{bmatrix} + \frac{1}{m}\begin{bmatrix}
        \mathbf{u}^{\top} & 
        \Verts{\xi}^2_\mathcal{H}
    \end{bmatrix}
    \begin{bmatrix}
        \alphabf\\
        \beta
    \end{bmatrix} + \frac{\lambda}{2}\begin{bmatrix}
        \alphabf^{\top} &
        \beta
    \end{bmatrix}\begin{bmatrix}
        \K & \mathbf{u}\\
        \mathbf{u}^{\top} & \Verts{\xi}^2_\mathcal{H}
    \end{bmatrix}\begin{bmatrix}
        \alphabf\\
        \beta
    \end{bmatrix}.
\end{align*}
Hence, the optimal coefficients $\hat{\alphabf}_{\omega}\coloneqq ((\hat{\alphabf}_{\omega})_{i})_{1\leq i \leq n}$ and  $\hat{\beta}_{\omega}$  are those minimizing the above quadratic form and are characterized by the following optimality condition:
\begin{equation*}
\begin{bmatrix}
    \overbrace{(\K\Dtwo + n\lambda\mathbbm{1}_{n\times n})}^{\M}\K & \overbrace{(\K\Dtwo + n\lambda\mathbbm{1}_{n\times n})}^{\M}\mathbf{u}\\
    \mathbf{u}^{\top}\underbrace{(\K\Dtwo + n\lambda\mathbbm{1}_{n\times n})}_{\M} &  \underbrace{\mathbf{u}^{\top}\Dtwo\mathbf{u} + n\lambda \Verts{\xi}^{2}_{\mathcal{H}}}_{p\geq 0}
\end{bmatrix}\begin{bmatrix}
        \hat{\alphabf}_\omega\\
        \hat{\beta}_\omega
    \end{bmatrix}= - \frac{n}{m}\begin{bmatrix}
        \mathbf{u}\\
 	\underbrace{\Verts{\xi}^2_{\mathcal{H}}}_{v\geq 0}
    \end{bmatrix}.
\end{equation*}
{\bf Expression of $\widehat{\nabla\mathcal{F}}(\omega)$.}
The result follows directly after expressing $\widehat{\nabla\mathcal{F}}(\omega)$ in vector form using the notations $\Dthree$ and $\Dfour$ from \cref{prop:est_2} and recalling that $(\hat{a}_{\omega}(x_i))_{1\leq i\leq n} = \begin{bmatrix}
        \K & 
        \mathbf{u}
    \end{bmatrix}\begin{bmatrix}
        \hat{\alphabf}_{\omega}\\
        \hat{\beta}_{\omega}
    \end{bmatrix}$.
\end{proof}


\begin{proof}[Proof of \cref{prop:equivalenceEstimates}]
	{Let $\omega\in{\mathbb{R}^d}$.} Define  
\begin{align*}
\widehat{\nabla\mathcal{F}}(\omega)=&\frac{1}{m}\sum_{j=1}^m\partial_\omega\ell_{out}(\omega, \hat{h}_\omega(\tilde{x}_j),\tilde{y}_j)+\frac{1}{n}\sum_{i=1}^n\partial_{\omega,v}^2\ell_{in}(\omega, \hat{h}_\omega(x_i),y_i)\hat{a}_\omega(x_i),
\end{align*}	
where $\hat{h}_{\omega}$ and $\hat{a}_{\omega}$ are given by \cref{eq:h,eq:a}.  We will show that $\widehat{\nabla\mathcal{F}}(\omega) = \nabla\widehat{\mathcal{F}}(\omega)$.  By \cref{prop:est_1,prop:est_2}, we know that $\nabla\widehat{\mathcal{F}}(\omega)$ and $\widehat{\nabla\mathcal{F}}(\omega)$  admit the following expressions:
\begin{align*}
\nabla\widehat{\mathcal{F}}(\omega)&=\frac{1}{m}\Dthree\mathbbm{1}_m-\frac{1}{m}\Dfour\M^{-1}\mathbf{u}\\
 \widehat{\nabla\mathcal{F}}(\omega)&=\frac{1}{m}\Dthree\mathbbm{1}_m+\frac{1}{n}\Dfour \begin{bmatrix}
    	\K & \mathbf{u}
    \end{bmatrix}
    \begin{bmatrix}
    	\hat{\alphabf}_\omega\\
    	\hat{\beta}_\omega
    \end{bmatrix}.
\end{align*}
Taking the difference of the two estimators yields:
\begin{align*}
	 \widehat{\nabla\mathcal{F}}(\omega)- \nabla\widehat{\mathcal{F}}(\omega) &= \frac{1}{m}\Dfour\parens{\M^{-1}\mathbf{u} +  \frac{m}{n}\begin{bmatrix}
    	\K & \mathbf{u}
    \end{bmatrix}
    \begin{bmatrix}
    	\hat{\alphabf}_\omega\\
    	\hat{\beta}_\omega
    \end{bmatrix}} \\
	  &= \frac{1}{m}\Dfour\M^{-1}\underbrace{\parens{\mathbf{u} + \frac{m}{n}\M \begin{bmatrix}
    	\K & \mathbf{u}
    \end{bmatrix}
    \begin{bmatrix}
    	\hat{\alphabf}_\omega\\
    	\hat{\beta}_\omega
    \end{bmatrix}
	  }}_{=0},
\end{align*}
where the term $\mathbf{u} +\frac{m}{n}(\M\K\hat{\alphabf}_\omega+\hat{\beta}_\omega\M\mathbf{u})$ is equal to $0$ by definition of $\hat{\alphabf}_\omega$ and $\hat{\beta}_\omega$ as solutions of the linear system \eqref{eq:linear_system_adjoint} of \cref{prop:est_1}. 
\end{proof}

\section{Preliminary Results}\label{sec_app:prel_res}
{In this section, $\Omega$ is an arbitrary compact subset of $\mathbb{R}^d$  with $\text{hull}(\Omega)$ denoting its convex hull, which is also compact.} 
We also consider an arbitrary fixed positive value $\Lambda$ such that $\lambda \leq \Lambda$ as this would allow us to simplify the dependence of the boundedness and Lipschitz constants on $\lambda$.   

\subsection{Boundedness and Lipschitz continuity of $h^\star_\omega$ and $\hat{h}_\omega$}
\begin{proposition}[Boundedness of $h^\star_\omega$ and $\hat{h}_\omega$]\label{prop:bound_hstaromega}
Under \cref{assump:compact,assump:convexity_lin,assump:K_bounded,assump:reg_lin_lout}, the functions $\omega\mapsto \|h^\star_\omega\|_\mathcal{H}$ and $\omega\mapsto\|\hat{h}_\omega\|_\mathcal{H}$ are bounded over $\textnormal{hull}(\Omega)$ by $\frac{B\sqrt{\kappa}}{\lambda}$, where $B\coloneqq\sup_{\omega\in\textnormal{hull}(\Omega),y\in\mathcal{Y}}\left|\partial_v \ell_{in}(\omega, 0, y)\right|>0$. Moreover, for all $\omega\in\textnormal{hull}(\Omega)$ and $x\in\mathcal{X}$, $h^\star_\omega(x)$ and $\hat{h}_\omega(x)$ take value in the compact interval $\mathcal{V}\coloneqq\left[-\frac{B\kappa}{\lambda},\frac{B\kappa}{\lambda}\right]\subset\mathbb{R}$.
\end{proposition}

\begin{proof}
\textbf{Boundedness of $\left\|h^\star_\omega\right\|_\mathcal{H}$ and $\left\|\hat{h}_\omega\right\|_\mathcal{H}$. }Let $\omega\in\text{hull}(\Omega)$. Using \cref{lem:h_min_hstar}, we know, for any $h\in\mathcal{H}$, that:
\begin{equation*}
    \left\|h-h^\star_\omega\right\|_\mathcal{H}\leq\frac{1}{\lambda}\left\|\partial_h L_{in}(\omega, h)\right\|_\mathcal{H}.
\end{equation*}
This is particularly valid for $h=0$. Thus,
\begin{equation*}
    \left\|h^\star_\omega\right\|_\mathcal{H}\leq\frac{1}{\lambda}\left\|\partial_h L_{in}(\omega, 0)\right\|_\mathcal{H}.
\end{equation*}
Using the expression of the partial derivative $\partial_{h}L_{in}$ established in \cref{prop:fre_diff_L}, we obtain:
\begin{equation*}
    \left\|h^\star_\omega\right\|_\mathcal{H}\leq\frac{1}{\lambda}\big\|\mathbb{E}_\mathbb{P}\left[\partial_v \ell_{in}(\omega, 0, y)K(x,\cdot)\right]\big\|_\mathcal{H}.
\end{equation*}
By \cref{assump:K_bounded}, $K$ is bounded by $\kappa$. Hence, Jensen's inequality yields:
\begin{equation*}
    \left\|h^\star_\omega\right\|_\mathcal{H}\leq\frac{1}{\lambda}\mathbb{E}_\mathbb{P}\Big[\left|\partial_v \ell_{in}(\omega, 0, y)\right|\left\|K(x,\cdot)\right\|_\mathcal{H}\Big]\leq\frac{\sqrt{\kappa}}{\lambda}\mathbb{E}_\mathbb{P}\Big[\left|\partial_v \ell_{in}(\omega, 0, y)\right|\Big].
\end{equation*}
By \cref{assump:compact}, $\Omega$ and $\mathcal{Y}$ are compact, which implies that $\text{hull}(\Omega)\times\mathcal{Y}$ is compact. From \cref{assump:reg_lin_lout}, we know that the function $(\omega, y)\mapsto\partial_v\ell_{in}(\omega, 0, y)$ is continuous. Given that every continuous function on a compact space is bounded, we obtain:
\begin{equation*}
    \left\|h^\star_\omega\right\|_\mathcal{H}\leq \frac{B\sqrt{\kappa}}{\lambda}<+\infty,\quad\text{where}\quad B\coloneqq\sup_{\omega\in\text{hull}(\Omega),y\in\mathcal{Y}}\left|\partial_v \ell_{in}(\omega, 0, y)\right|>0.
\end{equation*}
To prove that $\left\|\hat{h}_\omega\right\|_\mathcal{H}\leq\frac{B\sqrt{\kappa}}{\lambda}$, we follow a similar approach to that of $\left\|h^\star_\omega\right\|_\mathcal{H}\leq \frac{B\sqrt{\kappa}}{\lambda}$. More precisely, we investigate the case where the expectation is with respect to the empirical estimate $\hat{\mathbb{P}}_n$ of $\mathbb{P}$.

\textbf{$h^\star_\omega(x)$ and $\hat{h}_\omega(x)$ belong to $\mathcal{V}$. }Let $\omega\in\text{hull}(\Omega)$ and $x\in\mathcal{X}$. By the reproducing property, the Cauchy-Schwarz inequality, and \cref{assump:K_bounded}, we have:
\begin{equation*}
    \left|h^\star_\omega(x)\right|\leq\sqrt{\kappa}\left\|h^\star_\omega\right\|\quad\text{and}\quad\left|\hat{h}_\omega(x)\right|\leq\sqrt{\kappa}\left\|\hat{h}_\omega\right\|.
\end{equation*}
Using the bound on $\left\|h^\star_\omega\right\|_\mathcal{H}$ and $\left\|\hat{h}_\omega\right\|_\mathcal{H}$ already proved in the first part of this proof, we get:
\begin{equation*}
    \left|h^\star_\omega(x)\right|\leq\frac{B\kappa}{\lambda}\quad\text{and}\quad\left|\hat{h}_\omega(x)\right|\leq\frac{B\kappa}{\lambda}.
\end{equation*}
This concludes the proof.
\end{proof}

\begin{proposition}[Lipschitz continuity of $\omega\mapsto h^\star_\omega$]\label{prop:lip_hstaromega}
Under \cref{assump:compact,assump:convexity_lin,assump:K_bounded,assump:reg_lin_lout}, the function $\omega\mapsto h^\star_\omega$ is $\frac{L\sqrt{\kappa}}{\lambda}$-Lipschitz continuous on $\textnormal{hull}(\Omega)$, where $L\coloneqq\sup_{\omega\in \textnormal{hull}(\Omega),v\in\mathcal{V},y\in\mathcal{Y}}\left\|\partial_{\omega, v}^2 \ell_{in}(\omega, v, y)\right\|>0$,   and $\mathcal{V}$ is the compact interval introduced in \cref{prop:bound_hstaromega}.
\end{proposition}
\begin{proof}
To prove this proposition, we adopt the strategy of finding an upper bound for the Jacobian, which serves as the Lipschitz constant. 

Let $\omega\in \text{hull}(\Omega)$. Using \cref{prop:fre_diff_L,prop:strong_convexity_Lin}, we know that $h\mapsto L_{in}(\omega,h)$ is $\lambda$-strongly convex and Fr\'echet differentiable. Also, by \cref{prop:fre_diff_L_v}, $\partial_h L_{in}$ is Fr\'echet differentiable on $\mathbb{R}^d\times\mathcal{H}$, and, a fortiori, Hadamard differentiable. Then, by the functional implicit differentiation theorem \citep[Theorem~2.1]{petrulionyte2024functional}, the Jacobian $\partial_\omega h^\star_\omega:\mathcal{H}\to\mathbb{R}^d$ can be expressed as:
\begin{equation*}
    \partial_\omega h^\star_\omega=-\partial_{\omega, h}^2 L_{in}(\omega, h^\star_\omega)\left(\partial_h^2 L_{in}(\omega, h^\star_\omega)\right)^{-1}.
\end{equation*}
We have:
\begin{align}
    \left\|\partial_\omega h^\star_\omega\right\|_{\op}&\leq\left\|\partial_{\omega, h}^2 L_{in}(\omega, h^\star_\omega)\right\|_{\op}\left\|\left(\partial_h^2 L_{in}(\omega, h^\star_\omega)\right)^{-1}\right\|_{\op}\nonumber\\
    &\leq\frac{\left\|\partial_{\omega, h}^2 L_{in}(\omega, h^\star_\omega)\right\|_{\op}}{\lambda}\nonumber\\
    &=\frac{\Big\|\mathbb{E}_\mathbb{P}\left[\partial_{\omega, v}^2 \ell_{in}(\omega, h^\star_\omega(x), y)K(x,\cdot)\right]\Big\|_{\op}}{\lambda}\nonumber\\
    &\leq\frac{\mathbb{E}_\mathbb{P}\Big[\left\|\partial_{\omega, v}^2 \ell_{in}(\omega, h^\star_\omega(x), y)\right\|\left\|K(x,\cdot)\right\|_\mathcal{H}\Big]}{\lambda}\nonumber\\
    &\leq\frac{\sqrt{\kappa}\mathbb{E}_\mathbb{P}\Big[\left\|\partial_{\omega, v}^2 \ell_{in}(\omega, h^\star_\omega(x), y)\right\|\Big]}{\lambda},\label{eq:upp_bound_jac}
\end{align}
where the first line uses the sub-multiplicative property of the operator norm $\|\cdot\|_{\op}$, the second line stems from the fact that $h\mapsto L_{in}(\omega, h)$ is $\lambda$-strongly convex, for any $\omega\in\mathbb{R}^d$, as proved in \cref{prop:strong_convexity_Lin}, the third line follows from \cref{prop:fre_diff_L_v}, the fourth line uses Jensen's inequality, and the last line is a direct consequence of the boundedness of $K$ by $\kappa$ (\cref{assump:K_bounded}). According to \cref{prop:bound_hstaromega}, $h^\star_\omega(x)\in\mathcal{V}\coloneqq\left[-\frac{B\kappa}{\lambda},\frac{B\kappa}{\lambda}\right]$, which is a compact interval of $\mathbb{R}$, where $B\coloneqq\sup_{\omega\in\text{hull}(\Omega),y\in\mathcal{Y}}\left|\partial_v \ell_{in}(\omega, 0, y)\right|>0$. By \cref{assump:compact}, $\Omega$ and $\mathcal{Y}$ are compact sets, hence $\text{hull}(\Omega)\times\mathcal{V}\times\mathcal{Y}$ is compact. Besides, by \cref{assump:reg_lin_lout}, $(\omega, v, y)\mapsto\partial_v\ell_{in}(\omega, v, y)$ is continuous over the domain $\text{hull}(\Omega)\times\mathcal{V}\times\mathcal{Y}$. Since every continuous function on a compact set is bounded, this leads to:
\begin{equation*}
    \mathbb{E}_\mathbb{P}\Big[\left\|\partial_{\omega, v}^2 \ell_{in}(\omega, h^\star_\omega(x), y)\right\|\Big]\leq L\coloneqq\sup_{\omega\in\text{hull}(\Omega),v\in\mathcal{V},y\in\mathcal{Y}}\left\|\partial_{\omega, v}^2 \ell_{in}(\omega, v, y)\right\|<+\infty.
\end{equation*}
Substituting this bound into \cref{eq:upp_bound_jac} means that $\frac{L\sqrt{\kappa}}{\lambda}$ is an upper-bound on $\left\|\partial_\omega h^\star_\omega\right\|_{\op}$. Thus, the result follows as desired.
\end{proof}

\subsection{{Local boundedness and Lipschitz  properties of $\ell_{in}$, $\ell_{out}$, and their derivatives }}

\begin{proposition}[{Local} boundedness]\label{prop:uniform_boundedness}
Under \cref{assump:compact,assump:convexity_lin,assump:K_bounded,assump:reg_lin_lout}, the functions $(\omega,x,y)\mapsto \ell_{out}(\omega,h_{\omega}^{\star}(x),y)$, $(\omega,x,y)\mapsto \partial_{\omega}\ell_{out}(\omega,h_{\omega}^{\star}(x),y)$, and $(\omega,x,y)\mapsto\partial_v \ell_{out}(\omega, h^\star_\omega(x), y)$ are bounded over $\textnormal{hull}(\Omega)\times\mathcal{X}\times\mathcal{Y}$ by some positive constant $M_{out}$. Similarly, the functions $(\omega,x,y)\mapsto\partial_v \ell_{in}(\omega, h^\star_\omega(x), y)$, $(\omega,x,y)\mapsto\partial_v^2 \ell_{in}(\omega, h^\star_\omega(x), y)$, and $(\omega,x,y)\mapsto\partial_{\omega, v}^2 \ell_{in}(\omega, h^\star_\omega(x), y)$  are bounded over $\textnormal{hull}(\Omega)\times\mathcal{X}\times\mathcal{Y}$ by some positive constant $M_{in}$. 
The constants $M_{out}$ and $M_{in}$ are defined as:
\begin{align*}
M_{out}&\coloneqq\sup_{\omega\in\textnormal{hull}(\Omega),v\in\mathcal{V},y\in\mathcal{Y}}\max\left(\verts{\ell_{out}(\omega,v,y)},\Verts{\partial_{\omega}\ell_{out}(\omega,v,y)},\verts{\partial_v\ell_{out}(\omega, v, y)}\right)>0,\\
M_{in}&\coloneqq\sup_{\omega\in\textnormal{hull}(\Omega),v\in\mathcal{V},y\in\mathcal{Y}}\max\left(\verts{\partial_v\ell_{in}(\omega, v, y)}, \verts{\partial_v^2\ell_{in}(\omega, v, y)},\Verts{\partial_{\omega, v}^2\ell_{in}(\omega, v, y)}\right)>0,
\end{align*}
where $\mathcal{V}\subset\mathbb{R}$ is the compact interval defined in \cref{prop:bound_hstaromega}.
\end{proposition}

\begin{proof}
By \cref{prop:bound_hstaromega}, we have that $h^\star_\omega(x)\in\mathcal{V}\coloneqq\left[-\frac{B\kappa}{\lambda},\frac{B\kappa}{\lambda}\right]\subset\mathbb{R}$, for any $x\in\mathcal{X}$. From \cref{assump:reg_lin_lout}, we know that $\ell_{in}$, $\ell_{out}$, and their partial derivatives are all continuous on $\text{hull}(\Omega)\times\mathcal{V}\times\mathcal{Y}$. Also, $\mathcal{Y}$ is compact by \cref{assump:compact}. Thus, $\text{hull}(\Omega)\times\mathcal{V}\times\mathcal{Y}$ is compact. As every continuous function defined over a compact space is bounded, we obtain that:
\begin{align*}
    \sup_{\omega\in\text{hull}(\Omega),x\in\mathcal{X},y\in\mathcal{Y}}\verts{\ell_{out}(\omega,h_{\omega}^{\star}(x),y)}&\leq\sup_{\omega\in\text{hull}(\Omega),v\in\mathcal{V},y\in\mathcal{Y}}\verts{\ell_{out}(\omega,v,y)}<+\infty,\\
    \sup_{\omega\in\text{hull}(\Omega),x\in\mathcal{X},y\in\mathcal{Y}}\Verts{\partial_{\bullet} \ell_{\circ}(\omega,h_{\omega}^{\star}(x),y)}&\leq\sup_{\omega\in\text{hull}(\Omega),v\in\mathcal{V},y\in\mathcal{Y}}\Verts{\partial_{\bullet} \ell_{\circ}(\omega,v,y)}<+\infty,
\end{align*}
where $\bullet \in \{\{v\}, \{w\}, \{w,v\}\}$ and $\circ \in \{in,out\}$. This implies the desired result.
\end{proof}

\begin{proposition}[Local Lipschitz continuity]\label{prop:uniform_Lipschitzness}
	Under \cref{assump:compact,assump:convexity_lin,assump:K_bounded,assump:reg_lin_lout}, there exists a positive constant $\lipout$ so that for any $(x,y)$ in $\mathcal{X}\times \mathcal{Y}$, 
	the functions $\omega\mapsto \ell_{out}(\omega,h_{\omega}^{\star}(x),y)$, $\omega\mapsto \partial_{\omega}\ell_{out}(\omega,h_{\omega}^{\star}(x),y)$, and $\omega\mapsto\partial_v \ell_{out}(\omega, h^\star_\omega(x), y)$ are locally $\frac{\lipout}{\lambda}$-Lipschitz continuous over $\textnormal{hull}(\Omega)$. Similarly, there exists a positive constant $\lipin$ so that for any $(x,y)$ in $\mathcal{X}\times \mathcal{Y}$, the functions $\omega\mapsto\partial_v \ell_{in}(\omega, h^\star_\omega(x), y)$, $\omega\mapsto\partial_v^2 \ell_{in}(\omega, h^\star_\omega(x), y)$, and $\omega\mapsto\partial_{\omega, v}^2 \ell_{in}(\omega, h^\star_\omega(x), y)$ are locally $\frac{\lipin}{\lambda}$-Lipschitz continuous $\textnormal{hull}(\Omega)$.  The constants $\lipout$ and $\lipin$ are defined, for any $0<\lambda\leq \Lambda$, as:
	\begin{align*}
	    \lipout&\coloneqq\left(\Lambda+M_{in}\kappa\right)\max\left(M_{out},\bar{M}_{out}\right)>0\\
	    \lipin&\coloneqq\left(\Lambda+M_{in}\kappa\right)\max\left(M_{in},\bar{M}_{in}\right)>0,
    \end{align*}
    where:
    \begin{align*}
      \bar{M}_{out}&\coloneqq\sup_{\omega\in\textnormal{hull}(\Omega), v\in\mathcal{V}, y\in\mathcal{Y}}\max\left(\Verts{\partial_\omega^2\ell_{out}(\omega, v, y)}_{\op},\Verts{\partial_{\omega, v}^2\ell_{out}(\omega, v, y)},\verts{\partial_v^2\ell_{out}(\omega, v, y)}\right)>0,\\
      \bar{M}_{in}&\coloneqq\sup_{\omega\in\textnormal{hull}(\Omega), v\in\mathcal{V}, y\in\mathcal{Y}}\max\left(\Verts{\partial_\omega\partial_v^2\ell_{in}(\omega, v, y)},\verts{\partial_v^3\ell_{in}(\omega, v, y)},\Verts{\partial_\omega\partial_{\omega, v}^2\ell_{in}(\omega, v, y)}\right)>0,
    \end{align*}
    with $M_{in}$ and $M_{out}$ being the positive constants defined in \cref{prop:uniform_boundedness}, and $\mathcal{V}\subset\mathbb{R}$ is the compact interval defined in \cref{prop:bound_hstaromega}.
\end{proposition}


\begin{proof}
For any $(\omega,x,y)\in\text{hull}(\Omega)\times \mathcal{X}\times \mathcal{Y}$, we have:
\begin{align*}
    \Verts{\nabla_\omega\ell_{out}(\omega, h^\star_\omega(x), y)}&=\Verts{\partial_\omega\ell_{out}(\omega, h^\star_\omega(x), y)+\partial_v\ell_{out}(\omega, h^\star_\omega(x), y)\partial_\omega h^\star_\omega(x)}\\
    &\leq\Verts{\partial_\omega\ell_{out}(\omega, h^\star_\omega(x), y)}+\verts{\partial_v\ell_{out}(\omega, h^\star_\omega(x), y)}\Verts{\partial_\omega h^\star_\omega}_{\op}\Verts{K(x,\cdot)}_\mathcal{H}\\
    &\leq M_{out}\left(1+\frac{M_{in}\kappa}{\lambda}\right)\\
    &\leq\frac{M_{out}\left(\Lambda+M_{in}\kappa\right)}{\lambda},
\end{align*}
where the first line uses the chain rule, the second line applies the triangle inequality and the reproducing property of the RKHS $\mathcal{H}$, the third line follows from \cref{prop:uniform_boundedness} to bound the derivatives of $\ell_{out}$, from \cref{prop:lip_hstaromega}, which states that the function $\omega\mapsto h^\star_\omega$ is $\frac{L\sqrt{\kappa}}{\lambda}$-Lipschitz continuous with $L\coloneqq\sup_{\omega\in\text{hull}(\Omega),v\in\mathcal{V},y\in\mathcal{Y}}\left\|\partial_{\omega, v}^2 \ell_{in}(\omega, v, y)\right\|<M_{in}$, to bound $\Verts{\partial_\omega h^\star_\omega}_{\op}$, and from \cref{assump:K_bounded} to bound $\Verts{K(x,\cdot)}_\mathcal{H}$, and the last line is a direct consequence of $0<\lambda\leq \Lambda$. In a similar way, we obtain:
\begin{gather*}
    \Verts{\nabla_\omega\partial_\omega\ell_{out}(\omega, h^\star_\omega(x), y)}_{\op}\leq\frac{\bar{M}_{out}\left(\Lambda+M_{in}\kappa\right)}{\lambda},\quad\Verts{\nabla_\omega\partial_v \ell_{out}(\omega, h^\star_\omega(x), y)}\leq\frac{\bar{M}_{out}\left(\Lambda+M_{in}\kappa\right)}{\lambda},\\
    \Verts{\nabla_\omega\partial_v \ell_{in}(\omega, h^\star_\omega(x), y)}\leq\frac{M_{in}\left(\Lambda+M_{in}\kappa\right)}{\lambda},\quad\Verts{\nabla_\omega\partial_v^2 \ell_{in}(\omega, h^\star_\omega(x), y)}\leq\frac{\bar{M}_{in}\left(\Lambda+M_{in}\kappa\right)}{\lambda},\\
    \Verts{\nabla_\omega\partial_{\omega, v}^2 \ell_{in}(\omega, h^\star_\omega(x), y)}_{\op}\leq\frac{\bar{M}_{in}\left(\Lambda+M_{in}\kappa\right)}{\lambda}.
\end{gather*}
Combining all these bounds concludes the proof.
\end{proof}

\section{Generalization Properties}\label{app:sec_conv}

{As before, let $\Omega$ be an arbitrary compact subset of $\mathbb{R}^d$.}

\subsection{Point-wise estimates}\label{app:subsec_point_est}
We present a point-wise upper-bound on the value error $\verts{\mathcal{F}(\omega)-\widehat{\mathcal{F}}(\omega) }$ and gradient error $\Verts{\nabla\mathcal{F}(\omega)-\widehat{\nabla\mathcal{F}}(\omega) }$. To this end, we introduce the following notation for the error between the inner and outer objectives and their empirical approximations evaluated at the optimal inner solution $h_{\omega}^{\star}$: 
\begin{align*}
    \Dout\coloneqq \verts{L_{out}(\omega, h^\star_\omega)-\widehat{L}_{out}(\omega, h^\star_\omega)}, 
    \qquad 
    \Din \coloneqq \verts{L_{in}(\omega, h^\star_\omega)-\widehat{L}_{in}(\omega, h^\star_\omega)}.
\end{align*}
By abuse of notation, we introduce the following error between  partial derivatives of $L_{in}$ and $\widehat{L}_{in}$ (resp. $L_{out}$ and $\widehat{L}_{out}$), evaluated at $(\omega, h_{\omega}^{\star})$, \textit{i.e.},
\begin{align*}
  \Douth &\coloneqq \Verts{\partial_h L_{out}(\omega, h^\star_\omega)-\partial_h\widehat{L}_{out}(\omega, h^\star_\omega)}_{\mathcal{H}},\quad 
  &\Doutw &\coloneqq \Verts{\partial_\omega L_{out}(\omega, h^\star_\omega)-\partial_\omega\widehat{L}_{out}(\omega, h^\star_\omega)},\\
  \Dinh &\coloneqq \Verts{\partial_h L_{in}(\omega, h^\star_\omega)-\partial_h\widehat{L}_{in}(\omega, h^\star_\omega)}_{\mathcal{H}},\quad 
  &\Dinw &\coloneqq \Verts{\partial_\omega L_{in}(\omega, h^\star_\omega)-\partial_\omega\widehat{L}_{in}(\omega, h^\star_\omega)},\\
  \Dinhh &\coloneqq \Verts{\partial_{h}^2 L_{in}(\omega, h^\star_\omega)-\partial_h^2\widehat{L}_{in}(\omega, h^\star_\omega)}_{\op},\quad 
  &\Dinwh &\coloneqq \Verts{\partial_{\omega,h}^2 L_{in}(\omega, h^\star_\omega)-\partial_{\omega,h}^2\widehat{L}_{in}(\omega, h^\star_\omega)}_{\op}.
\end{align*}

\begin{proposition}\label{prop:diff_hstar_hhat}
    Under \cref{assump:compact,assump:convexity_lin,assump:K_bounded,assump:reg_lin_lout}, the following holds for any $\omega\in\Omega$:
    \begin{equation*}
        \left\|h^\star_\omega-\hat{h}_\omega\right\|_\mathcal{H}\leq\frac{1}{\lambda}\left\| \partial_h\widehat{L}_{in}(\omega, h^\star_\omega)\right\|_\mathcal{H} = \frac{1}{\lambda}\Dinh.
    \end{equation*}
\end{proposition}
\begin{proof}
Let $\omega\in\Omega$. 
The function $h\mapsto\widehat{L}_{in}(\omega, h)$ is $\lambda$-strongly convex and Fr\'echet differentiable by  \cref{prop:strong_convexity_Lin,prop:fre_diff_L}. Moreover,  $\hat{h}_{\omega}$ is the minimizer of $h\mapsto\widehat{L}_{in}(\omega, h)$ by definition. 
Therefore, using \cref{lem:h_min_hstar}, we obtain a control on the distance in $\mathcal{H}$ to the optimum $\hat{h}_{\omega}$ of $h\mapsto\widehat{L}_{in}(\omega,h)$ in terms of the gradient $\partial_{h}\widehat{L}_{in}(\omega,h)$:
\begin{align*}
	\left\|h-\hat{h}_\omega\right\|_\mathcal{H}\leq\frac{1}{\lambda}\Verts{\partial_h\widehat{L}_{in}(\omega, h)}_\mathcal{H}, \qquad \forall h\in \mathcal{H}.
\end{align*}
In particular, choosing $h= h^\star_\omega$ yields the first inequality. The fact that $\Verts{\partial_h\widehat{L}_{in}(\omega, h_{\omega}^{\star})}_\mathcal{H}=\Dinh$ follows from the optimality of $h_{\omega}^{\star}$ which implies that  $\partial_h L_{in}(\omega, h_{\omega}^{\star})=0$. 
\end{proof}

\begin{proposition}\label{prop:lip_continuity_out}
Under \cref{assump:compact,assump:convexity_lin,assump:K_bounded,assump:reg_lin_lout}, the following inequalities hold for any $\omega\in\Omega$:
\begin{align*}
	\Eout &\coloneqq\verts{\widehat{L}_{out}(\omega, h^\star_\omega)-\widehat{L}_{out}(\omega, \hat{h}_\omega)}\leq C_{out}\Verts{h^\star_\omega-\hat{h}_\omega}_\mathcal{H},\\
	\Eouth&\coloneqq\Verts{\partial_h\widehat{L}_{out}(\omega, h^\star_\omega)-\partial_h\widehat{L}_{out}(\omega, \hat{h}_\omega)}_\mathcal{H} \leq C_{out}\Verts{h^\star_\omega-\hat{h}_\omega}_\mathcal{H},\\
	\Eoutw&\coloneqq\Verts{\partial_\omega\widehat{L}_{out}(\omega, h^\star_\omega)-\partial_\omega\widehat{L}_{out}(\omega, \hat{h}_\omega)} \leq C_{out}\Verts{h^\star_\omega-\hat{h}_\omega}_\mathcal{H},\\
	\Einhh&\coloneqq\Verts{\partial_h^2\widehat{L}_{in}(\omega, h^\star_\omega)-\partial_h^2\widehat{L}_{in}(\omega, \hat{h}_\omega)}_{\op}\leq C_{in}\Verts{h^\star_\omega-\hat{h}_\omega}_\mathcal{H},\\
	\Einwh&\coloneqq\Verts{\partial_{\omega, h}^2\widehat{L}_{in}(\omega, h^\star_\omega)-\partial_{\omega, h}^2\widehat{L}_{in}(\omega, \hat{h}_\omega)}_{\op} \leq C_{in}\Verts{h^\star_\omega-\hat{h}_\omega}_\mathcal{H}.
\end{align*}
The positive constants $C_{out}$ and $C_{in}$ are defined as:
\begin{align*}
    C_{out}&\coloneqq\max\left(M_{out}\sqrt{\kappa},\bar{M}_{out}\kappa,\bar{M}_{out}\sqrt{\kappa}\right)>0,\\
    C_{in}&\coloneqq\max\left(\bar{M}_{in}\kappa\sqrt{\kappa},\bar{M}_{in}\kappa,M_{in}\sqrt{d\kappa}\right)>0,
\end{align*}
where $M_{out}$, $\bar{M}_{out}$, and $\bar{M}_{in}$ are the positive constants defined in \cref{prop:uniform_boundedness,prop:uniform_Lipschitzness}.
\end{proposition}

\begin{proof}
\textbf{Lipschitz continuity of some functions of interest. }Let $\omega\in\Omega$. According to \cref{prop:bound_hstaromega}, both $h^\star_\omega(x)$ and $\hat{h}_\omega(x)$ lie in the compact interval $\mathcal{V}\coloneqq\left[-\frac{B\kappa}{\lambda},\frac{B\kappa}{\lambda}\right]\subset\mathbb{R}$, for any $x\in\mathcal{X}$, where $B\coloneqq\sup_{\omega\in\text{hull}(\Omega),y\in\mathcal{Y}}\left|\partial_v \ell_{in}(\omega, 0, y)\right|>0$. By \cref{assump:compact}, $\mathcal{Y}$ is a compact set. Hence $\Omega\times\mathcal{V}\times\mathcal{Y}$ is a compact set as well. Furthermore, by \cref{assump:reg_lin_lout}, $(\omega, v, y)\mapsto\ell_{in}(\omega, v, y)$, $(\omega, v, y)\mapsto\ell_{out}(\omega, v, y)$, and their derivatives are all continuous over the compact domain $\Omega\times\mathcal{V}\times\mathcal{Y}$. Therefore, these functions and their derivatives are bounded on this domain. In particular, this also holds when $v$ takes the specific values $h^\star_\omega(x)$ or $\hat{h}_\omega(x)$. Let $\bar{v}$ be either $h^\star_\omega(x)$ or $\hat{h}_\omega(x)$, for any $x\in\mathcal{X}$. For any $\omega\in\Omega$, and $y\in\mathcal{Y}$, we have:
\begin{align*}
    \verts{\partial_v\ell_{out}(\omega, \bar{v}, y)}&\leq\sup_{\omega\in\Omega,v\in\mathcal{V},y\in\mathcal{Y}}\verts{\partial_v\ell_{out}(\omega, v, y)}\leq M_{out}<+\infty,\\
   \verts{\partial_v^2\ell_{out}(\omega, \bar{v}, y)}&\leq\sup_{\omega\in\Omega,v\in\mathcal{V},y\in\mathcal{Y}}\verts{\partial_v^2\ell_{out}(\omega, v, y)}\leq\bar{M}_{out}<+\infty,\\
   \Verts{\partial_{\omega, v}^2\ell_{out}(\omega, \bar{v}, y)}&\leq\sup_{\omega\in\Omega,v\in\mathcal{V},y\in\mathcal{Y}}\Verts{\partial_{\omega, v}^2\ell_{out}(\omega, v, y)}\leq\bar{M}_{out}<+\infty,\\
   \verts{\partial_v^3\ell_{in}(\omega, \bar{v}, y)}&\leq\sup_{\omega\in\Omega,v\in\mathcal{V},y\in\mathcal{Y}}\verts{\partial_v^3\ell_{in}(\omega, v, y)}\leq\bar{M}_{in}<+\infty,\\
   \Verts{\partial_v\partial_{\omega, v}^2\ell_{in}(\omega, \bar{v}, y)}_{\op}&\leq\sup_{\omega\in\Omega,v\in\mathcal{V},y\in\mathcal{Y}}\Verts{\partial_\omega\partial_v^2\ell_{in}(\omega, v, y)}\leq\bar{M}_{in}<+\infty
\end{align*}
This means that $v\in\mathcal{V}\mapsto\ell_{out}(\omega, v, y)$, $v\in\mathcal{V}\mapsto\partial_v\ell_{out}(\omega, v, y)$, $v\in\mathcal{V}\mapsto\partial_\omega\ell_{out}(\omega, v, y)$, $v\in\mathcal{V}\mapsto\partial_v^2\ell_{in}(\omega, v, y)$, and $v\in\mathcal{V}\mapsto\partial_{\omega, v}^2\ell_{in}(\omega, v, y)$ are Lipschitz continuous, with Lipschitz constants $M_{out}$, $\bar{M}_{out}$, $\bar{M}_{out}$, $\bar{M}_{in}$, and $\bar{M}_{in}$, respectively, for any $\omega\in\Omega$ and $y\in\mathcal{Y}$.

\textbf{Upper-bounds. }We have:
\begin{align*}
    \Eout\coloneqq\verts{\widehat{L}_{out}(\omega, h^\star_\omega)-\widehat{L}_{out}(\omega, \hat{h}_\omega)}&=\verts{\frac{1}{m}\sum_{j=1}^m\ell_{out}(\omega, h^\star_\omega(\tilde{x}_j), \tilde{y}_j) - \frac{1}{m}\sum_{j=1}^m\ell_{out}(\omega, \hat{h}_\omega(\tilde{x}_j), \tilde{y}_j)}\\
    &\leq\frac{1}{m}\sum_{j=1}^m\verts{\ell_{out}(\omega, h^\star_\omega(\tilde{x}_j), \tilde{y}_j)-\ell_{out}(\omega, \hat{h}_\omega(\tilde{x}_j), \tilde{y}_j)}\\
    &\leq\frac{M_{out}}{m}\sum_{j=1}^m\verts{h^\star_\omega(\tilde{x}_j)-\hat{h}_\omega(\tilde{x}_j)}\\
    &\leq M_{out}\sqrt{\kappa}\Verts{h^\star_\omega-\hat{h}_\omega}_\mathcal{H},
\end{align*}
where the first line uses the definition of $(\omega, h)\mapsto\widehat{L}_{out}(\omega, h)$, the second line applies the triangle inequality, the third line leverages the fact that $v\mapsto\ell_{out}(\omega, v, y)$ is $M_{out}$-Lipschitz continuous, for any $\omega\in\Omega$ and $y\in\mathcal{Y}$, and the last line follows from the reproducing property of the RKHS $\mathcal{H}$, Cauchy-Schwarz's inequality, and \cref{assump:K_bounded} to bound $\Verts{K(x,\cdot)}_\mathcal{H}$ by $\sqrt{\kappa}$. Similarly, we obtain:
\begin{gather*}
    \partial_h\Eout\leq\bar{M}_{out}\kappa\Verts{h^\star_\omega-\hat{h}_\omega}_\mathcal{H},\quad\partial_\omega\Eout\leq\bar{M}_{out} \sqrt{\kappa}\Verts{h^\star_\omega-\hat{h}_\omega}_\mathcal{H},\quad\partial_h^2\Ein\leq\bar{M}_{in} \kappa\sqrt{\kappa}\Verts{h^\star_\omega-\hat{h}_\omega}_\mathcal{H},\\
    \partial_{\omega, h}^2\Ein\leq\bar{M}_{in} \kappa\Verts{h^\star_\omega-\hat{h}_\omega}_\mathcal{H}.
\end{gather*}
Combining all the bounds finishes the proof.
\end{proof}

\begin{proposition}\label{prop:bounded_derivatives}
Under \cref{assump:compact,assump:convexity_lin,assump:K_bounded,assump:reg_lin_lout}, the following inequalities hold for any $\omega\in\Omega$:
\begin{align*}
	\Verts{\partial_{h}L_{out}(\omega,h_{\omega}^{\star})}_{\mathcal{H}}\leq C_{out}, \quad \Verts{\partial^2_{\omega,h}L_{in}(\omega,h_{\omega}^{\star}) }_{\op}\leq C_{in},\quad 
	\Verts{\partial^2_{\omega,h}\widehat{L}_{in}(\omega,\hat{h}_{\omega}) }_{\op}\leq C_{in},
\end{align*}
where $C_{out}$ and $C_{in}$ are the positive constants defined in \cref{prop:lip_continuity_out}.
\end{proposition}

\begin{proof}Let $\omega\in\Omega$.

\textbf{Upper-bound on $\Verts{\partial_{h}L_{out}(\omega,h_{\omega}^{\star})}_{\mathcal{H}}$. }We have:
\begin{align*}
    \left\|\partial_h L_{out}(\omega, h^\star_\omega)\right\|_\mathcal{H}&=\Big\|\mathbb{E}_\mathbb{Q}\left[\partial_v \ell_{out}(\omega, h^\star_\omega(x), y)K(x,\cdot)\right]\Big\|_\mathcal{H}\\
    &\leq\mathbb{E}_\mathbb{Q}\Big[\left|\partial_v \ell_{out}(\omega, h^\star_\omega(x), y)\right|\left\|K(x,\cdot)\right\|_\mathcal{H}\Big]\\
    &\leq\sqrt{\kappa}\mathbb{E}_\mathbb{Q}\Big[\left|\partial_v \ell_{out}(\omega, h^\star_\omega(x), y)\right|\Big],
\end{align*}
where the first line follows from \cref{prop:fre_diff_L}, the second line results from the triangle inequality, and the last line uses \cref{assump:K_bounded} to bound $\Verts{K(x,\cdot)}_\mathcal{H}$ by $\sqrt{\kappa}$.  Furthermore, we know by \cref{prop:bound_hstaromega} that $(\omega,h_{\omega}^{\star}(x),y)$ belongs to the compact subset $\Omega\times \mathcal{V}\times \mathcal{Y}$ and by \cref{prop:uniform_boundedness} that $\partial_v \ell_{out}(\omega, h^\star_\omega(x), y)$ is bounded by a  constant $M_{out}$ on $\text{hull}(\Omega)\times \mathcal{V}\times \mathcal{Y}$. Hence, it follows that:
\begin{align*}
    \left\|\partial_h L_{out}(\omega, h^\star_\omega)\right\|_\mathcal{H}
    \leq \sqrt{\kappa} M_{out} \le C_{out},
\end{align*}
where $C_{out}$ is defined in \cref{prop:lip_continuity_out}. 

\textbf{Upper-boud on $\quad \Verts{\partial^2_{\omega,h}L_{in}(\omega,h_{\omega}^{\star}) }_{\op}$. }According to \cref{prop:fre_diff_L_v}, $\partial_{\omega, h}^2 L_{in}(\omega, h^\star_\omega)$ is a Hilbert-Schmidt operator, which points to:
\begin{equation}\label{eq:norm_op_partial_omega_h_Lin}
    \left\|\partial_{\omega, h}^2L_{in}(\omega, h^\star_\omega)\right\|_{\op}\leq\left\|\partial_{\omega, h}^2L_{in}(\omega, h^\star_\omega)\right\|_{\hs}=\sqrt{\sum_{l=1}^d\left\|\partial_{\omega_k, h}^2L_{in}(\omega, h^\star_\omega)\right\|_\mathcal{H}^2}.
\end{equation}
This means that to find an upper-bound on $\left\|\partial_{\omega, h}^2L_{in}(\omega, h^\star_\omega)\right\|_{\op}$, it suffices to establish an upper-bound on $\left\|\partial_{\omega_l, h}^2L_{in}(\omega, h^\star_\omega)\right\|_\mathcal{H}^2$ for any $l\in\{1,\ldots,d\}$. For a fixed $l\in\{1,\ldots,d\}$, we have:
\begin{align*}
    \left\|\partial_{\omega_l, h}^2L_{in}(\omega, h^\star_\omega)\right\|_\mathcal{H}^2&=\Big\|\mathbb{E}_\mathbb{P}\left[\partial_{\omega_l,v}^2 \ell_{in}(\omega, h^\star_\omega(x), y)K(x,\cdot)\right]\Big\|_\mathcal{H}^2\\
    &\leq\mathbb{E}_\mathbb{P}\left[\left|\partial_{\omega_l, v}^2 \ell_{in}(\omega, h^\star_\omega(x), y)\right|^2\left\|K(x,\cdot)\right\|_\mathcal{H}^2\right]\\
    &\leq\mathbb{E}_\mathbb{P}\left[\left\|\partial_{\omega, v}^2 \ell_{in}(\omega, h^\star_\omega(x), y)\right\|^2\right]\kappa,
\end{align*}
where the first line follows from \cref{prop:fre_diff_L_v}, the second line is a consequence of Jensen's inequality applied on the convex function $\|\cdot\|^2$, and the last line applies \cref{assump:K_bounded} to bound $\Verts{K(x,\cdot)}_\mathcal{H}^2$ by $\kappa$. Incorporating this upper-bound into \cref{eq:norm_op_partial_omega_h_Lin} yields:
\begin{equation*}
    \left\|\partial_{\omega, h}^2L_{in}(\omega, h^\star_\omega)\right\|_{\op}\leq\sqrt{\mathbb{E}_\mathbb{P}\left[\left\|\partial_{\omega, v}^2 \ell_{in}(\omega, h^\star_\omega(x), y)\right\|^2\right] d\kappa}\leq M_{in}\sqrt{d\kappa}\leq C_{out}, 
\end{equation*}
where we used \cref{prop:uniform_boundedness} to bound $\partial_{\omega, v}^2 \ell_{in}(\omega, h^\star_\omega(x), y)$ by the constant $M_{in}$.

\textbf{Upper-bound on $\Verts{\partial^2_{\omega,h}\widehat{L}_{in}(\omega,\hat{h}_{\omega}) }_{\op}$. }The derivation of this upper bound follows the same steps as the previous one, with the only differences being the use of $\widehat{L}_{in}$ instead of $L_{in}$, and $\hat{h}_\omega$ instead of $h^\star_\omega$.

Note that in the last step of each of the three upper-bounds, we used the fact that the functions we are dealing with are continuous (by \cref{assump:reg_lin_lout}) on $\Omega\times\mathcal{V}\times\mathcal{Y}$, which is compact because $\Omega$ and $\mathcal{Y}$ are compact by \cref{assump:compact} and $\mathcal{V}$ is a compact interval of $\mathbb{R}$ defined in \cref{prop:bound_hstaromega}. Hence, those functions are bounded.
\end{proof}


\begin{proposition}[{Approximation bounds}]\label{prop:grad_app_bound}
    Under \cref{assump:compact,assump:convexity_lin,assump:K_bounded,assump:reg_lin_lout}, the following holds for any $\omega\in\Omega$:
    \begin{gather*}
    \verts{\mathcal{F}(\omega)-\widehat{\mathcal{F}}(\omega)}\leq 
    \Dout
    +\frac{C_{out}}{\lambda}\Dinh,\\
 	\Verts{\nabla\mathcal{F}(\omega)-\widehat{\nabla\mathcal{F}}(\omega)}
 	\leq  
 	\Doutw  + \frac{C_{in}}{\lambda}\Douth + \frac{C_{out}C_{in}}{\lambda^2}\Dinhh + \frac{C_{out}}{\lambda}\Dinwh + \frac{C_{out}}{\lambda}\parens{1 + 2\frac{C_{in}}{\lambda}  + \frac{C_{in}^2}{\lambda^2} }\Dinh,
    \end{gather*}
 where the constants $C_{in}$ and $C_{out}$ are given in {\cref{prop:lip_continuity_out}}.
\end{proposition}
\begin{proof}
In all what follows, we fix a value for $\omega$ in $\Omega$. We start by controlling the value function, then its gradient.

{\bf Control on the value function.}
By the triangle inequality, we have:
\begin{equation}\label{eq:diff_f_int}
    \verts{\mathcal{F}(\omega)-\hat{\mathcal{F}}(\omega)}\leq\underbrace{\verts{L_{out}(\omega, h^\star_\omega)-\widehat{L}_{out}(\omega, h^\star_\omega)}}_{\delta_{\omega}^{out}} +\underbrace{\verts{\widehat{L}_{out}(\omega, h^\star_\omega)-\widehat{L}_{out}(\omega, \hat{h}_\omega)}}_{\Eout},
\end{equation}
According to \cref{prop:lip_continuity_out}, the error term $\Eout$ is controlled by the norm of the difference $h^\star_\omega-\hat{h}_\omega$, \textit{i.e.}, $\Eout\leq C_{out}\Verts{h^\star_\omega-\hat{h}_\omega}_\mathcal{H}$. 
Moreover, by \cref{prop:diff_hstar_hhat}, we know that $\Verts{h^\star_\omega-\hat{h}_\omega}_\mathcal{H}\leq \frac{1}{\lambda}\Dinh$. Therefore, combining both bounds yields: 
	$\Eout\leq \frac{C_{out}}{\lambda}\Dinh$. 
The upper-bound on value function follows by substituting the previous inequality into \cref{eq:diff_f_int}. 

{\bf Control on the gradient.}
By \cref{prop:tot_grad_int}, we have the following expression for the total gradient $\nabla\mathcal{F}$:
\begin{align*}
    \nabla\mathcal{F}(\omega)=\partial_\omega L_{out}(\omega, h^\star_\omega)-\partial_{\omega, h}^2 L_{in}(\omega, h^\star_\omega)\left(\partial_h^2 L_{in}(\omega, h^\star_\omega)\right)^{-1}\partial_h L_{out}(\omega, h^\star_\omega).
\end{align*}
{Similarly, the gradient estimator $\widehat{\nabla\mathcal{F}}$ is defined by replacing $L_{out}$ and $L_{in}$ by their empirical versions $\widehat{L}_{out}$ and $\widehat{L}_{in}$, and $h_{\omega}^{\star}$ by $\hat{h}_{\omega}\coloneqq\arg\min_{h\in \mathcal{H}} \widehat{L}_{in}(\omega,h)$ in the above expression, \textit{i.e.},}
\begin{align*}
    \widehat{\nabla\mathcal{F}}(\omega)=\partial_\omega\widehat{L}_{out}(\omega, \hat{h}_\omega)-\partial_{\omega, h}^2 \widehat{L}_{in}(\omega, \hat{h}_\omega)\left(\partial_h^2 \widehat{L}_{in}(\omega, \hat{h}_\omega)\right)^{-1}\partial_h \widehat{L}_{out}(\omega, \hat{h}_\omega).
\end{align*}
To simplify notations, for any $h\in\mathcal{H}$, we introduce the following operators $R(h), \hat{R}(h):\mathcal{H}\to\Omega$:    \begin{align*}
        R(h)=\partial_{\omega, h}^2L_{in}(\omega, h)\left(\partial_h^2 L_{in}(\omega, h)\right)^{-1}\quad\text{and}\quad 
        \hat{R}(h)=\partial_{\omega, h}^2\widehat{L}_{in}(\omega, h)\left(\partial_h^2 \widehat{L}_{in}(\omega, h)\right)^{-1}.
    \end{align*}
    The difference $\nabla\mathcal{F}(\omega)-\widehat{\nabla\mathcal{F}}(\omega)$ can be decomposed as:
    \begin{align*}
        \nabla\mathcal{F}(\omega)-\widehat{\nabla\mathcal{F}}(\omega)=&\left(\partial_\omega L_{out}(\omega,h^\star_\omega)-\partial_\omega \widehat{L}_{out}(\omega,h^\star_\omega)\right)+\left(\partial_\omega \widehat{L}_{out}(\omega,h^\star_\omega)-\partial_\omega \widehat{L}_{out}(\omega,\hat{h}_\omega)\right)\\
        &-\hat{R}(\hat{h}_\omega)\parens{\left(\partial_h L_{out}(\omega,h^\star_\omega)-\partial_h\widehat{L}_{out}(\omega,h^\star_\omega)\right)+ \left(\partial_h\widehat{L}_{out}(\omega,h^\star_\omega)-\partial_h\widehat{L}_{out}(\omega,\hat{h}_\omega)\right)}\\
                &-\left(R(h^\star_\omega)-\hat{R}(\hat{h}_\omega)\right)\partial_h L_{out}(\omega,h^\star_\omega).
    \end{align*}
By taking the norm of the above equality and using triangle inequality, we obtain the following upper-bound: 
    \begin{align}
    \begin{split}
        \Verts{\nabla\mathcal{F}(\omega)-\widehat{\nabla\mathcal{F}}(\omega)}
        & \leq
        \underbrace{\Verts{\partial_\omega L_{out}(\omega,h^\star_\omega)-\partial_\omega \widehat{L}_{out}(\omega,h^\star_\omega)}}_{\Doutw}+\underbrace{\Verts{\partial_\omega \widehat{L}_{out}(\omega,h^\star_\omega)-\partial_\omega \widehat{L}_{out}(\omega,\hat{h}_\omega)}}_{\Eoutw}\\
        +&\Verts{\hat{R}(\hat{h}_{\omega})}_{\op}\parens{\underbrace{\Verts{\partial_h L_{out}(\omega,h^\star_\omega)-\partial_h\widehat{L}_{out}(\omega,h^\star_\omega)}_{\mathcal{H}}}_{\Douth} + \underbrace{\Verts{\partial_h\widehat{L}_{out}(\omega,h^\star_\omega)-\partial_h\widehat{L}_{out}(\omega,\hat{h}_\omega)}_{\mathcal{H}}}_{\Eouth}}\\
                +& \Verts{R(h^\star_\omega)-\hat{R}(\hat{h}_\omega)}_{\op}\Verts{\partial_h L_{out}(\omega,h^\star_\omega)}_{\mathcal{H}}.
                \end{split}\label{eq:diff_grad_inter}
    \end{align}
    Next, we provide upper-bounds on $\Verts{R(h_{\omega}^{\star})- \hat{R}(\hat{h}_{\omega})}_{\op}$ and $\Verts{\hat{R}(\hat{h}_{\omega})}_{\op}$ in terms of derivatives of $L_{in}$ and $\widehat{L}_{in}$. 

    \textbf{Upper-bounds on $\Verts{R(h_{\omega}^{\star})- \hat{R}(\hat{h}_{\omega})}_{\op}$ and $\Verts{\hat{R}(\hat{h}_{\omega})}_{\op}$. } 
    %
    By application of {\cref{prop:fre_diff_L_v,prop:strong_convexity_Lin}}, we deduce that $\partial_{\omega,h}^2 L_{in}(\omega,h_{\omega}^{\star})$, $\partial_{h}^2 L_{in}(\omega,h_{\omega}^{\star})$, $\partial_{\omega,h}^2 \widehat{L}_{in}(\omega,\hat{h}_{\omega})$ and $\partial_{h}^2 \widehat{L}_{in}(\omega,\hat{h}_{\omega})$ are all bounded operators. Moreover, since $L_{in}$ and $\widehat{L}_{in}$ are $\lambda$-strongly  convex in their second argument by {\cref{prop:strong_convexity_Lin}}, it follows that $\partial_{h}^2 L_{in}(\omega,h_{\omega}^{\star})\geq \lambda\Id_{\mathcal{H}}$ and $\partial_{h}^2 \widehat{L}_{in}(\omega,\hat{h}_{\omega})\geq \lambda \Id_{\mathcal{H}}$. We can therefore apply \cref{lem:tech_res_1} which yields the following inequalities:  
    \begin{align*}
        \Verts{R(h_{\omega}^{\star})- \hat{R}(\hat{h}_{\omega})}_{\op}
        \leq & \frac{1}{\lambda^2}\Verts{\partial_{\omega,h}^2 L_{in}(\omega,h_{\omega}^{\star})}_{\op}\Verts{\partial_{h}^2 L_{in}(\omega,h_{\omega}^{\star}) - \partial_{h}^2 \widehat{L}_{in}(\omega,\hat{h}_{\omega})}_{\op}\\ 
        &+ \frac{1}{\lambda}\Verts{\partial_{\omega,h}^2 L_{in}(\omega,h_{\omega}^{\star}) - \partial_{\omega,h}^2 \widehat{L}_{in}(\omega,\hat{h}_{\omega})}_{\op},\\
        \Verts{\hat{R}(\hat{h}_{\omega})}_{\op}\leq&\frac{1}{\lambda} \Verts{\partial_{\omega,h}^2\widehat{L}_{in}(\omega,\hat{h}_{\omega})}_{\op}. 
    \end{align*}
By applying the triangle inequality to both terms of the first inequality above, we obtain:
\begin{align*}
        \Verts{R(h_{\omega}^{\star})- \hat{R}(\hat{h}_{\omega})}_{\op}
        \leq & \frac{1}{\lambda^2}\Verts{\partial_{\omega,h}^2 L_{in}(\omega,h_{\omega}^{\star})}_{\op}\parens{\underbrace{\Verts{\partial_{h}^2 L_{in}(\omega,h_{\omega}^{\star}) - \partial_{h}^2 \widehat{L}_{in}(\omega,h^\star_{\omega})}_{\op}}_{\Dinhh} + \underbrace{\Verts{\partial_{h}^2 \widehat{L}_{in}(\omega,h_{\omega}^{\star}) - \partial_{h}^2 \widehat{L}_{in}(\omega,\hat{h}_{\omega})}_{\op}}_{\Einhh}}\\ 
        &+ \frac{1}{\lambda}\parens{\underbrace{\Verts{\partial_{\omega,h}^2 L_{in}(\omega,h_{\omega}^{\star}) - \partial_{\omega,h}^2 \widehat{L}_{in}(\omega,h_{\omega}^{\star})}_{\op}}_{\Dinwh} + \underbrace{\Verts{\partial_{\omega,h}^2 \widehat{L}_{in}(\omega,h_{\omega}^{\star}) - \partial_{\omega,h}^2 \widehat{L}_{in}(\omega,\hat{h}_{\omega})}_{\op}}_{\Einwh}}.
    \end{align*}
{\bf Final bound.} We can now substitute the above bounds on $\Verts{R(h_{\omega}^{\star})- \hat{R}(\hat{h}_{\omega})}_{\op}$ and $\Verts{\hat{R}(\hat{h}_{\omega})}_{\op}$ into \cref{eq:diff_grad_inter} to obtain the following upper-bound on the gradient error:
 \begin{equation} \label{eq:diff_grad_inter_2}
\begin{aligned}
 	\Verts{\nabla\mathcal{F}(\omega)-\widehat{\nabla\mathcal{F}}(\omega)}
 	&\leq  
 	\Doutw + \Eoutw + \frac{1}{\lambda}\Verts{\partial_{\omega,h}^2 \widehat{L}_{in}(\omega,\hat{h}_{\omega})}_{\op}\parens{\Douth + \Eouth}\\
 	+& \Verts{\partial_{h}L_{out}(\omega,h_{\omega}^{\star})}_{\mathcal{H}}\parens{ \frac{1}{\lambda^2}\Verts{\partial^2_{\omega,h}L_{in}(\omega,h_{\omega}^{\star}) }_{\op} \parens{\Dinhh + \Einhh} + \frac{1}{\lambda}\parens{\Dinwh + \Einwh} }.
 \end{aligned}
 \end{equation}
Furthermore, by \cref{prop:bounded_derivatives}, we have the following upper-bounds on the derivatives of $L_{in}$ and $L_{out}$:
\begin{align*}
	\Verts{\partial_{h}L_{out}(\omega,h_{\omega}^{\star})}_{\mathcal{H}}\leq C_{out}, \quad \Verts{\partial^2_{\omega,h}L_{in}(\omega,h_{\omega}^{\star}) }_{\op}\leq C_{in},\quad 
	\Verts{\partial^2_{\omega,h}\widehat{L}_{in}(\omega,\hat{h}_{\omega}) }_{\op}\leq C_{in}. 
\end{align*}
Incorporating the above bounds into \cref{eq:diff_grad_inter_2}, we further get:
 \begin{align*}
 	\Verts{\nabla\mathcal{F}(\omega)-\widehat{\nabla\mathcal{F}}(\omega)}
 	\leq & 
 	\Doutw + \Eoutw + \frac{C_{in}}{\lambda}\parens{\Douth + \Eouth}\\
 	&+ C_{out}\parens{ \frac{C_{in}}{\lambda^2}\parens{\Dinhh + \Einhh} + \frac{1}{\lambda}\parens{\Dinwh + \Einwh} }.
 \end{align*}
By \cref{prop:lip_continuity_out}, we can upper-bound the error terms $\Eoutw, \Eouth$ by $C_{out}\Verts{h_{\omega}^{\star}-\hat{h}_{\omega}}_{\mathcal{H}}$ and $ \Einhh,\Einwh$ by the difference $C_{in}\Verts{h_{\omega}^{\star}-\hat{h}_{\omega}}_{\mathcal{H}}$. Furthermore, since  $\Verts{h_{\omega}^{\star}-\hat{h}_{\omega}}_{\mathcal{H}}\leq \frac{1}{\lambda}\Dinh$ by \cref{prop:diff_hstar_hhat}, we can further show that gradient error satisfies the desired bound:
\begin{align*}
 	\Verts{\nabla\mathcal{F}(\omega)-\widehat{\nabla\mathcal{F}}(\omega)}
 	\leq 
 	\Doutw  + \frac{C_{in}}{\lambda}\Douth + C_{out}\parens{ \frac{C_{in}}{\lambda^2}\Dinhh + \frac{1}{\lambda}\Dinwh}+ \frac{C_{out}}{\lambda}\parens{1 + 2\frac{C_{in}}{\lambda}  + \frac{C_{in}^2}{\lambda^2} }\Dinh.
\end{align*}
\end{proof}

\subsection{Maximal inequalities}\label{app_subsec:max_in}
\begin{proposition}[Maximal inequalities for  empirical processes]\label{prop:exp_uni_bound}
Let $\Lambda$ be a positive constant. 
Under \cref{assump:compact,assump:K_bounded,assump:reg_lin_lout,assump:convexity_lin}, the following maximal inequalities hold for any $0<\lambda \leq \Lambda$:
\begin{align*}
	\mathbb{E}_{\mathbb{Q}}\brackets{\sup_{\omega\in \Omega} \Dout }
	 &\leq {\sqrt{\frac{1}{\lambda^2 m}}c(\Omega)\max(M_{out}{\lipout}\diam(\Omega),\Lambda {M_{out}^2})}\\
	 \mathbb{E}_{\mathbb{Q}}\brackets{\sup_{\omega\in \Omega} \Doutw }
	&\leq {\sqrt{\frac{d}{\lambda^2m}}c(\Omega)\max(M_{out}\lipout\diam(\Omega),\Lambda M_{out}^2)}, 
\end{align*}
    where $c(\Omega)$ is a positive constant {greater than 1} that depends only on $\Omega$ and $d$, while {$\lipout$ and $M_{out}$ are positive constants defined in \cref{prop:uniform_boundedness,prop:uniform_Lipschitzness}}. 
\end{proposition}
\begin{proof}
We will apply the result of \cref{prop:empirical_process} which provides maximal inequalities for real-valued empirical processes that are uniformly bounded {and} Lipschitz in their parameter. To this end, consider the parametric families:
\begin{align*}
	\mathcal{T}_{l}^{out} &\coloneqq\braces{\mathcal{X}\times \mathcal{Y}\ni (x,y)\mapsto \partial_{w_l} \ell_{out}(\omega,h_{\omega}^{\star}(x),y)\mid\omega\in \Omega}, \qquad 1\leq l \leq d\\
	\mathcal{T}_{0}^{out} &\coloneqq\braces{\mathcal{X}\times \mathcal{Y}\ni (x,y)\mapsto  \ell_{out}(\omega,h_{\omega}^{\star}(x),y)\mid\omega\in \Omega}.
\end{align*} 
For any $0\leq l\leq d$, these real-valued functions are uniformly bounded by a positive constant $M_{out}$, thanks to {\cref{prop:uniform_boundedness}}. 
Moreover, by {\cref{prop:uniform_Lipschitzness}}, the functions $\omega\mapsto \partial_{\omega_l}\ell_{out}(\omega,h_{\omega}^{\star}(x),y)$, and $\omega\mapsto \ell_{out}(\omega,h_{\omega}^{\star}(x),y)$ are all $\lambda^{-1}\lipout$-Lipschitz for any $(x,y)\in \mathcal{X}\times \mathcal{Y}$. Hence, \cref{prop:empirical_process} is applicable to each of these families, with $\PP$ set to $\mathbb{Q}$ and $\mathcal{Z}$ set to $\mathcal{X}\times \mathcal{Y}$.  We treat both $\Dout$ and $\Doutw$ separately.

{\bf A maximal inequality for $\Dout$.}
For $l=0$, we readily apply \cref{prop:empirical_process} with $p=1$ to get the following maximal inequality for $\Dout$:

\begin{align*}
	\mathbb{E}_{\mathbb{Q}}\brackets{\sup_{\omega\in \Omega} \Dout } \coloneqq &
	\mathbb{E}_{\mathbb{Q}}\brackets{ \sup_{\omega\in \Omega} \verts{ \mathbb{E}_{(x,y)\sim \mathbb{Q}}\brackets{\ell_{out}(\omega,h_{\omega}^{\star}(x),y) } - \frac{1}{m}\sum_{j=1}^m \ell_{out}(\omega,h_{\omega}^{\star}(\tilde{x}_j),\tilde{y}_j) } }\\
	 \leq& \sqrt{\frac{1}{\lambda^2 m}}c(\Omega)\max({\lipout}\diam(\Omega),\Lambda {M_{out}^2}). 
\end{align*}

{\bf A maximal inequality for $\Doutw$.}
{We now turn to  $\Doutw$, which involves vector-valued processes (as an error between the gradient and its estimate).} While the maximal inequalities in \cref{prop:empirical_process} hold for real-valued processes, we will first obtain maximal inequalities for each component appearing in $\Doutw$ and then sum these to control $\Doutw$.
To this end, we first use the Cauchy-Schwarz inequality which implies that $\mathbb{E}_{\mathbb{Q}}\brackets{\sup_{\omega\in \Omega} (\Doutw) }\leq \mathbb{E}_{\mathbb{Q}}\brackets{\sup_{\omega\in \Omega} (\Doutw)^2 }^{\frac{1}{2}}$. Thus we only need to control $\mathbb{E}_{\mathbb{Q}}\brackets{\sup_{\omega\in \Omega} (\Doutw)^2 }$. Simple calculations show that:
\begin{align*}
	\mathbb{E}_{\mathbb{Q}}\brackets{\sup_{\omega\in \Omega} \Doutw }^2&\leq \mathbb{E}_{\mathbb{Q}}\brackets{\sup_{\omega\in \Omega} (\Doutw)^2 }\\
	&\leq  \sum_{l=1}^d \mathbb{E}_{\mathbb{Q}}\brackets{\sup_{\omega\in \Omega}\verts{\mathbb{E}_{(x,y)\sim\mathbb{Q}}\brackets{\partial_{w_l}\ell_{out}(\omega,h_{\omega}^{\star}(x),y)} - \frac{1}{m}\sum_{j=1}^m \partial_{\omega_l}\ell_{out}\parens{\omega,h_{\omega}^{\star}(\tilde{x}_j),\tilde{y}_j}  }^2}\\
	&\leq \parens{\sqrt{\frac{d}{\lambda^2m}}c(\Omega)\max(\lipout\diam(\Omega),\Lambda M_{out}^{{2}})}^2,
\end{align*}
where the last inequality follows by application of \cref{prop:empirical_process} with $p=2$ to each term in the right-hand side of the first inequality for $1\leq l\leq d$. We get the desired bound on  $\mathbb{E}_{\mathbb{Q}}\brackets{\sup_{\omega\in \Omega} \Doutw }$ by taking the square root of the above inequality.
\end{proof}

\begin{proposition}[Maximal inequalities for RKHS-valued empirical processes]\label{prop:exp_uni_bound_2}
Let $\Lambda$ be a positive constant. 
Under \cref{assump:compact,assump:K_bounded,assump:reg_lin_lout,assump:convexity_lin}, the following maximal inequalities hold for any $0<\lambda\leq \Lambda$:
 \begin{align*}
 	\mathbb{E}_{\mathbb{Q}}\brackets{\sup_{\omega\in \Omega} \Douth }&\leq 
 	{\lambda^{-\frac{1}{4}}m^{-\frac{1}{2}}  \parens{c(\Omega)\max\parens{\widetilde{M}_{out,1}\widetilde{L}_{out,1}\diam(\Omega),\Lambda\widetilde{M}_{out,1}^2}}^{\frac{1}{4}}},\\
 	\mathbb{E}_{\mathbb{P}}\brackets{\sup_{\omega\in \Omega} \Dinh }&\leq 
 	{\lambda^{-\frac{1}{4}}n^{-\frac{1}{2}} \parens{c(\Omega)\max\parens{\widetilde{M}_{in,1}\widetilde{L}_{in,1}\diam(\Omega),\Lambda\widetilde{M}_{in,1}^2}}^{\frac{1}{4}}},\\
 	 	\mathbb{E}_{\mathbb{P}}\brackets{\sup_{\omega\in \Omega} \Dinwh }&\leq 
 	{\lambda^{-\frac{1}{4}}n^{-\frac{1}{2}}d^{\frac{1}{2}} \parens{c(\Omega)\max\parens{\widetilde{M}_{in,1}\widetilde{L}_{in,1}\diam(\Omega),\Lambda\widetilde{M}_{in,1}^2}}^{\frac{1}{4}}},\\
 	\mathbb{E}_{\mathbb{P}}\brackets{\sup_{\omega\in \Omega} \Dinhh }&\leq 
 	{\lambda^{-\frac{1}{4}}n^{-\frac{1}{2}} \parens{c(\Omega)\max\parens{\widetilde{M}^2_{in,2}\widetilde{L}_{in,2}\diam(\Omega),\Lambda\widetilde{M}^2_{in,2}}}^{\frac{1}{4}}},
 \end{align*}
 where $c(\Omega)$ is a positive constant {greater than $1$} that depends only on $\Omega$ and $d$, {$\widetilde{L}_{out,1},\widetilde{L}_{in,1},\widetilde{L}_{in,2},\widetilde{M}_{out,1},\widetilde{M}_{in,1},\widetilde{M}_{in,2}$ are positive constants defined as:
 \begin{gather*}
     \widetilde{L}_{out,1}\coloneqq 2\lipout M_{out}\kappa,\quad\widetilde{L}_{in,1}\coloneqq 2\lipin M_{in}\kappa,\quad\widetilde{L}_{in,2}\coloneqq 2\lipin M_{in}\kappa^2,\\
     \widetilde{M}_{out,1}\coloneqq M_{out}^2\kappa,\quad\widetilde{M}_{in,1}\coloneqq M_{in}^2\kappa,\quad\widetilde{M}_{in,2}\coloneqq M_{in}^2\kappa^2,
 \end{gather*}
 and $\lipout,\lipin,M_{out},M_{in}$ are positive constants given in \cref{prop:uniform_boundedness,prop:uniform_Lipschitzness}.}
 
\end{proposition}

\begin{proof}
Consider parametric families of real-valued functions indexed by $\Omega$ of the form:
\begin{align*}
	\mathcal{T}_{s,a}\coloneqq\braces{ t_{\omega}: ((x,y),(x',y'))\mapsto  f_s(\omega,x,y)f_s(\omega,x',y')K^{a}(x,x')\mid\omega\in \Omega },
\end{align*}
where $a\in \{1,2\}$,  $s$ is an integer satisfying {$0\leq s\leq d+2$}, and $f_s(\omega,x,y)$ are real-valued functions given by:
\begin{gather*}
	f_0: (\omega,x,y)\mapsto\partial_v \ell_{out}(\omega, h^\star_\omega(x), y),\qquad 
	f_1:(\omega,x,y)\mapsto\partial_v \ell_{in}(\omega, h^\star_\omega(x), y),\qquad
	f_2:(\omega,x,y)\mapsto\partial_v^2 \ell_{in}(\omega, h^\star_\omega(x), y),\\
	f_{2+l}: (\omega,x,y)\mapsto\partial_{\omega_l, v}^2 \ell_{in}(\omega, h^\star_\omega(x), y), \quad 1\leq l\leq d.
\end{gather*}
For any $1\leq s\leq d+2$, the real-valued functions $f_s$ are uniformly bounded by a positive constant $M_{in}$ thanks to {\cref{prop:uniform_boundedness}}. Moreover, since the kernel $K$ is bounded by {$\kappa$ due to \cref{assump:K_bounded}}, it follows that all elements $t_{\omega}$ of $\mathcal{T}_{s,a}$ are uniformly bounded by {$\widetilde{M}_{in,a}\coloneqq M_{in}^2\kappa^a$}. Moreover, for $1\leq s\leq d+2$, the functions $\omega\mapsto f_{s}(\omega,x,y)$, are $\lambda^{-1}\lipin$-Lipschitz for any $(x,y)\in\mathcal{X}\times\mathcal{Y}$, by {\cref{prop:uniform_Lipschitzness}}. Hence, it follows that the maps $\omega\mapsto t_{\omega}((x,y),(x',y'))$ are $\lambda^{-1}\widetilde{L}_{in,a}$-Lipschitz with {$\widetilde{L}_{in,a}\coloneqq 2\lipin M_{in}\kappa^a$} for any $(x,y)$ and $(x',y')$ in $\mathcal{X}\times \mathcal{Y}$. Similarly, for $s=0$, we get the same properties, albeit, with different constants, \textit{i.e.}, the family {$\mathcal{T}_{0,a}$} is uniformly bounded by a constant {$\widetilde{M}_{out,a}\coloneqq M_{out}^2\kappa^a$} with {$M_{out}$ introduced in \cref{prop:uniform_boundedness}}, and is $\lambda^{-1}\widetilde{L}_{out,a}$-Lipschitz in its parameter with {$\widetilde{L}_{out,a}\coloneqq 2\lipout M_{out}\kappa^a$} where $\lipout$ is given in {\cref{prop:uniform_Lipschitzness}}.
Hence, the maximal inequality in \cref{lem:u_stat_sup_p_omega} is applicable to each of these families with $\mathcal{Z}$ set to $\mathcal{X}\times \mathcal{Y}$, and {$\PP$ set either to $\mathbb{P}$ for $1\leq s\leq d+2$, or to $\mathbb{Q}$ for $s=0$}.
For conciseness, in all what follows, we will write $z = (x,y)$ and $z_i = (x_i,y_i)$ and $\tilde{z}_j = (\tilde{x}_j,\tilde{y}_j)$ for $1\leq i\leq n$ and $1\leq j\leq m$. 

{\bf Maximal inequalities for $\Douth$ and $\Dinh$.}
We control $\Douth$ first as $\Dinh$ will be dealt with similarly. Using Cauchy-Schwarz inequality and standard calculus, we have that:
\begin{align*}
	\mathbb{E}_{\mathbb{Q}}\brackets{\sup_{\omega\in \Omega} \Douth }^2
	&\leq \mathbb{E}_{\mathbb{Q}}\brackets{\sup_{\omega\in \Omega} (\Douth)^2 }\\
	&\coloneqq \mathbb{E}_{\mathbb{Q}}\brackets{ \sup_{\omega\in \Omega} \Verts{ \mathbb{E}_{(x,y)\sim \mathbb{Q}}\brackets{\partial_v\ell_{out}(\omega,h_{\omega}^{\star}(x),y)K(x,\cdot)} - \frac{1}{m}\sum_{j=1}^m \partial_v\ell_{out}(\omega,h_{\omega}^{\star}(\tilde{x}_j),\tilde{y}_j)K(\tilde{x}_j,\cdot) }_{\mathcal{H}}^2 }\\
	&= \mathbb{E}_{\mathbb{Q}}\brackets{ \sup_{\omega\in \Omega} \mathbb{E}_{z,z'\sim\mathbb{Q}\otimes\mathbb{Q}}\left[t_\omega(z,z')\right]+\frac{1}{m^2}\sum_{i,j=1}^m t_\omega(z_i,z_j)-\frac{2}{m}\sum_{j=1}^m\mathbb{E}_{z\sim\mathbb{Q}}\left[t_\omega(z,\tilde{z}_j)\right]},
\end{align*}
 where $t_\omega(z,z') \coloneqq \partial_v\ell_{out}(\omega,h_{\omega}^{\star}(x),y)\partial_v\ell_{out}(\omega,h_{\omega}^{\star}(x'),y')K(x,x')\in \mathcal{T}_{0,1}$. The last term is precisely what \cref{lem:u_stat_sup_p_omega} controls when applying it to the family $\mathcal{T}_{0,1}$ and choosing $\PP$ to be $\mathbb{Q}$. Therefore, the following maximal inequality holds by application of \cref{lem:u_stat_sup_p_omega}:
 \begin{align*}
 	\mathbb{E}_{\mathbb{Q}}\brackets{\sup_{\omega\in \Omega} \Douth }\leq 
 	\lambda^{-\frac{1}{4}}m^{-\frac{1}{2}}  \parens{c(\Omega)\max\parens{\widetilde{M}_{out,1}\widetilde{L}_{out,1}\diam(\Omega),\Lambda\widetilde{M}_{out,1}^2}}^{\frac{1}{4}},
 \end{align*}
 where $c(\Omega)$ is a positive constant {greater than $1$} the depends only on $\Omega$ and $d$. We obtain a similar inequality for $\Dinh$ by carrying out similar calculations, then applying \cref{lem:u_stat_sup_p_omega} to the family $\mathcal{T}_{1,1}$ and choosing $\mathbb{P}$ for the probability distribution $\PP$. The resulting bound is then of the form:
\begin{align*}
 	\mathbb{E}_{\mathbb{P}}\brackets{\sup_{\omega\in \Omega} \Dinh }\leq 
 	\lambda^{-\frac{1}{4}}n^{-\frac{1}{2}} \parens{c(\Omega)\max\parens{\widetilde{M}_{in,1}\widetilde{L}_{in,1}\diam(\Omega),\Lambda\widetilde{M}_{in,1}^2}}^{\frac{1}{4}}.
 \end{align*}

{\bf A maximal inequality for $\Dinwh$.} We have:
\begin{align*}
	\mathbb{E}_{\mathbb{P}}\brackets{\sup_{\omega\in \Omega} \Dinwh }^2
	&\stackrel{(a)}{\leq} \mathbb{E}_{\mathbb{P}}\brackets{\sup_{\omega\in \Omega} (\Dinwh)^2 }\\
	&\stackrel{(b)}{\coloneqq} \mathbb{E}_{\mathbb{P}}\brackets{ \sup_{\omega\in \Omega} \Verts{ \mathbb{E}_{(x,y)\sim \mathbb{P}}\brackets{\partial_{\omega,v}^2\ell_{in}(\omega,h_{\omega}^{\star}(x),y)K(x,\cdot)} - \frac{1}{n}\sum_{i=1}^n \partial_{\omega,v}^2\ell_{in}(\omega,h_{\omega}^{\star}(x_i),y_i)K(x_i,\cdot) }_{\op}^2 }\\
	&\stackrel{(c)}{\leq}\mathbb{E}_{\mathbb{P}}\brackets{ \sup_{\omega\in \Omega} \Verts{ \mathbb{E}_{(x,y)\sim \mathbb{P}}\brackets{\partial_{\omega,v}^2\ell_{in}(\omega,h_{\omega}^{\star}(x),y)K(x,\cdot)} - \frac{1}{n}\sum_{i=1}^n \partial_{\omega,v}^2\ell_{in}(\omega,h_{\omega}^{\star}(x_i),y_i)K(x_i,\cdot) }_{\hs}^2 }\\
	&\stackrel{(d)}{=}\sum_{l=1}^d \mathbb{E}_{\mathbb{P}}\brackets{ \sup_{\omega\in \Omega} \Verts{ \mathbb{E}_{(x,y)\sim \mathbb{P}}\brackets{\partial_{\omega_l,v}^2\ell_{in}(\omega,h_{\omega}^{\star}(x),y)K(x,\cdot)} - \frac{1}{n}\sum_{i=1}^n \partial_{\omega_l,v}^2\ell_{in}(\omega,h_{\omega}^{\star}(x_i),y_i)K(x_i,\cdot) }_{\mathcal{H}}^2 }\\
	&\stackrel{(e)}{=} \sum_{l=1}^d \mathbb{E}_{\mathbb{P}}\brackets{ \sup_{\omega\in \Omega} \mathbb{E}_{z,z'\sim\mathbb{P}\otimes\mathbb{P}}\left[t_{\omega,l}(z,z')\right]+\frac{1}{n^2}\sum_{i,j=1}^n t_{\omega,l}(z_i,z_j)-\frac{2}{n}\sum_{i=1}^n\mathbb{E}_{z\sim\mathbb{P}}\left[t_{\omega,l}(z,z_i)\right]},
\end{align*}
where we introduced $t_{\omega,l}(z,z') \coloneqq \partial_{\omega_l,v}^2\ell_{in}(\omega,h_{\omega}^{\star}(x),y)\partial_{\omega_l,v}^2\ell_{in}(\omega,h_{\omega}^{\star}(x'),y')K(x,x')\in \mathcal{T}_{2+l,1}$. Here, (a) follows by Cauchy-Schwarz inequality, (b) is obtained by definition of $\Dinwh$, while (c) uses the general fact that the operator norm of an operator is upper-bounded by its Hilbert-Schmidt norm which is finite in our case by application of \cref{prop:fre_diff_L_v}. Moreover, (d) further uses the Hilbert-Schmidt norm of an operator  in terms of the norm of its rows, while (e) simply expands the squared RKHS norm and uses the reproducing property in the RKHS $\mathcal{H}$. Each term in the last item (e) is precisely what \cref{lem:u_stat_sup_p_omega} controls when applying it to the families $\mathcal{T}_{2+l,1}$ for $1\leq l\leq d$ and choosing $\PP$ to be $\mathbb{P}$.  Therefore, the following maximal inequality holds by a direct application of \cref{lem:u_stat_sup_p_omega}:
 \begin{align*}
 	\mathbb{E}_{\mathbb{P}}\brackets{\sup_{\omega\in \Omega} \Dinwh }\leq 
 	\lambda^{-\frac{1}{4}}n^{-\frac{1}{2}}d^{\frac{1}{2}} \parens{c(\Omega)\max\parens{\widetilde{M}_{in,1}\widetilde{L}_{in,1}\diam(\Omega),\Lambda\widetilde{M}_{in,1}^2}}^{\frac{1}{4}},
 \end{align*}
 where $c(\Omega)$ is a positive constant greater than $1$ that depends only on $\Omega$ and $d$.

{\bf A maximal inequality for $\Dinhh$.} We will use a similar approach as for $\Dinwh$. We have:
\begin{align*}
	&\mathbb{E}_{\mathbb{P}}\brackets{\sup_{\omega\in \Omega} \Dinhh }^2\\
	&\stackrel{(a)}{\leq} \mathbb{E}_{\mathbb{P}}\brackets{\sup_{\omega\in \Omega} (\Dinhh)^2 }\\
	&\stackrel{(b)}{\coloneqq} \mathbb{E}_{\mathbb{P}}\brackets{ \sup_{\omega\in \Omega} \Verts{ \mathbb{E}_{(x,y)\sim \mathbb{P}}\brackets{\partial_{v}^2\ell_{in}(\omega,h_{\omega}^{\star}(x),y)K(x,\cdot)\otimes K(x,\cdot)} - \frac{1}{n}\sum_{i=1}^n \partial_{v}^2\ell_{in}(\omega,h_{\omega}^{\star}(x_i),y_i)K(x_i,\cdot)\otimes K(x_i,\cdot) }_{\op}^2 }\\
	&\stackrel{(c)}{\leq}\mathbb{E}_{\mathbb{P}}\brackets{ \sup_{\omega\in \Omega} \Verts{ \mathbb{E}_{(x,y)\sim \mathbb{P}}\brackets{\partial_{v}^2\ell_{in}(\omega,h_{\omega}^{\star}(x),y)K(x,\cdot)\otimes K(x,\cdot)} - \frac{1}{n}\sum_{i=1}^n \partial_{v}^2\ell_{in}(\omega,h_{\omega}^{\star}(x_i),y_i)K(x_i,\cdot)\otimes K(x_i,\cdot) }_{\hs}^2 }\\
	&\stackrel{(d)}{=} \mathbb{E}_{\mathbb{P}}\brackets{\sup_{\omega\in \Omega} \mathbb{E}_{z,z'\sim\mathbb{P}\otimes\mathbb{P}}\left[t_\omega(z,z')\right]+\frac{1}{n^2}\sum_{i,j=1}^n t_\omega(z_i,z_j)-\frac{2}{n}\sum_{i=1}^n\mathbb{E}_{z\sim\mathbb{P}}\left[t_\omega(z,z_i)\right]},
\end{align*}
where we introduced $t_{\omega}(z,z')\coloneqq\partial_{v}^2\ell_{in}(\omega,x,y)\partial_{v}^2\ell_{in}(\omega,x',y')K^2(x,x')\in \mathcal{T}_{2,2}$. Here, (a) follows by Cauchy-Schwarz inequality, (b) is obtained by definition of $\Dinhh$, while (c) uses the general fact that the operator norm of an operator is upper-bounded by its Hilbert-Schmidt norm which is finite in our case by application of \cref{prop:fre_diff_L_v}. Moreover, (d) further uses the identity in \cref{lem:hs_identity} for computing the Hilbert-Schmidt norm of sum/expectation of tensor-product operators. 
The last item (d) is precisely what \cref{lem:u_stat_sup_p_omega} controls when applying it to the family $\mathcal{T}_{2,2}$  and choosing $\PP$ to be $\mathbb{P}$.  Therefore, the following maximal inequality holds by direct application of \cref{lem:u_stat_sup_p_omega}:
 \begin{align*}
 	\mathbb{E}_{\mathbb{P}}\brackets{\sup_{\omega\in \Omega} \Dinhh }\leq 
 	\lambda^{-\frac{1}{4}}n^{-\frac{1}{2}} \parens{c(\Omega)\max\parens{\widetilde{M}_{in,2}\widetilde{L}_{in,2}\diam(\Omega),\Lambda\widetilde{M}^2_{in,2}}}^{\frac{1}{4}},
 \end{align*}
 where $c(\Omega)$ is a positive constant greater than $1$ that depends only on $\Omega$ and $d$.
\end{proof}
\subsection{Proof of \cref{th:generalizationBounds}}\label{app_sub:main_proof}
\begin{theorem}[Generalization bounds]\label{th:gen_bound}
The following holds under \cref{assump:compact,assump:convexity_lin,assump:K_bounded,assump:reg_lin_lout}:
\begin{align*}
    \mathbb{E}\brackets{\sup_{\omega\in\Omega}\verts{\mathcal{F}(\omega)-\hat{\mathcal{F}}(\omega)}}
    &\lesssim
    \frac{1}{\lambda m^{\frac{1}{2}}}+\frac{C_{out}}{\lambda^{\frac{5}{4}} n^{\frac{1}{2}}}\\
    \mathbb{E}\left[\sup_{\omega\in\Omega}\left\|\nabla\mathcal{F}(\omega)-\widehat{\nabla\mathcal{F}}(\omega)\right\|\right]
    &\lesssim
    \frac{1}{\lambda}\left(d^{\frac{1}{2}} + \frac{C_{in}}{\lambda^{\frac{1}{4}}}\right)\frac{1}{m^{\frac{1}{2}}} + \frac{C_{out}}{\lambda^{\frac{5}{4}}}\parens{2 + 3\frac{C_{in}}{\lambda}+\frac{C_{in}^2}{\lambda^2}}\frac{1}{n^{\frac{1}{2}}},
\end{align*}
where the constants $C_{in}$ and $C_{out}$ are given in \cref{prop:lip_continuity_out}.
\end{theorem}

\begin{proof}
Using the point-wise estimates in \cref{prop:grad_app_bound} and taking their supremum over $\Omega$ followed by the expectations over data, the following error bounds hold:
\begin{align*}
   \mathbb{E}\brackets{\sup_{\omega\in\Omega} \verts{\mathcal{F}(\omega)-\hat{\mathcal{F}}(\omega)}}
   \leq & 
    \mathbb{E}_{\mathbb{Q}}\brackets{\sup_{\omega\in\Omega}\Dout}
    +\frac{C_{out}}{\lambda}\mathbb{E}_{\mathbb{P}}\brackets{\sup_{\omega\in\Omega}\Dinh},\\
    \mathbb{E}\brackets{\sup_{\omega\in\Omega}\Verts{\nabla\mathcal{F}(\omega)-\widehat{\nabla\mathcal{F}}(\omega)}}
 	\leq &  
    \mathbb{E}_{\mathbb{Q}}\brackets{\sup_{\omega\in\Omega}\Doutw}  + \frac{C_{in}}{\lambda}\mathbb{E}_{\mathbb{Q}}\brackets{\sup_{\omega\in\Omega}\Douth}\\ 
    &+\frac{C_{out}}{\lambda}\parens{1 + 2\frac{C_{in}}{\lambda}+\frac{C_{in}^2}{\lambda^2}} \mathbb{E}_{\mathbb{P}}\brackets{\sup_{\omega\in\Omega}\Dinh}\\
    &+ \frac{C_{out}C_{in}}{\lambda^2}\mathbb{E}_{\mathbb{P}}\brackets{\sup_{\omega\in\Omega}\Dinhh} + \frac{C_{out}}{\lambda}\mathbb{E}_{\mathbb{P}}\brackets{\sup_{\omega\in\Omega}\Dinwh}.
\end{align*}
Furthermore, we can use the maximal inequalities in  \cref{prop:exp_uni_bound,prop:exp_uni_bound_2} to control each term appearing in the right-hand side of the above inequalities:
\begin{align*}
    \mathbb{E}\brackets{\sup_{\omega\in\Omega} \verts{\mathcal{F}(\omega)-\hat{\mathcal{F}}(\omega)}}
   \leq & 
   R\parens{ m^{-\frac{1}{2}}\lambda^{-1}  + C_{out}n^{-\frac{1}{2}}\lambda^{-(1+\frac{1}{4})}}\\
    \mathbb{E}\brackets{\sup_{\omega\in\Omega}\Verts{\nabla\mathcal{F}(\omega)-\widehat{\mathcal{F}}(\omega)}}
 	\leq &  R\Big( m^{-\frac{1}{2}}\lambda^{-1}d^{\frac{1}{2}} + C_{in}m^{-\frac{1}{2}}\lambda^{-(1+\frac{1}{4})} + C_{out}n^{-\frac{1}{2}}\lambda^{-(1+\frac{1}{4})}\parens{1 + 2\frac{C_{in}}{\lambda}+\frac{C_{in}^2}{\lambda^2}}\\
 	&\qquad+ C_{out}C_{in}n^{-\frac{1}{2}}\lambda^{-(2+\frac{1}{4})}  +  C_{out}n^{-\frac{1}{2}}\lambda^{-(1+\frac{1}{4})}\Big),
\end{align*}
where the constant $R$ depends only on the Lipschitz constants $\lipin$, $\lipout$, the upper-bounds $M_{in}$, $M_{out}$, the bound $\kappa$ on the kernel, the set $\Omega$ and the dimension $d$. Rearranging the obtained upper-bounds concludes the proof.
\end{proof}

\section{Maximal Inequalities for Bounded and Lipschitz Family of Functions}\label{sec_app:max_in_bound_lip}
Let  $\mathcal{Z}$ be a subset of a Euclidean space and $\Omega$ be a compact subset of $\mathbb{R}^d$. Denote by $\otimes^{k} \mathcal{Z}$ the $k$-th tensor power of $\mathcal{Z}$, for any $k\geq 1$. Consider a parametric family $\mathcal{T}$ of real-valued functions defined over $\mathcal{Z}$ and indexed by a parameter $\omega\in \Omega$, \textit{i.e.}, 
\begin{align}\label{eq:generic_parametric_family}
	\mathcal{T}\coloneqq \braces{\mathcal{Z}\ni z\mapsto t_{\omega}(z)\in \mathbb{R}\mid\omega \in \Omega}.
\end{align}
For a given probability measure $\mu$, denote by $L_2(\mu)$ the space of square $\mu$-integrable real-valued functions. We denote by $\Verts{f}_{\PP,2}\coloneqq  \mathbb{E}_{\PP}\brackets{f(z)^2}^{\frac{1}{2}}$ the $L_2(\mu)$-norm of any function $f\in L_2(\mu)$.  For any $\epsilon>0$, we denote by  $D\left(\epsilon,\mathcal{T},L_2(\mu)\right)$ the $\epsilon$-packing number of $\mathcal{T}$ w.r.t. $L_2(\mu)$. The next proposition provides a control on such a number under regularity conditions on the family $\mathcal{T}$. 
\begin{proposition}[Control on the packing number]\label{prop:estimate_covering_numbers}
	Assume that $\Omega$ is a compact subset of $\mathbb{R}^d$, that the parametric family $\mathcal{T}$ defined in \cref{eq:generic_parametric_family} is uniformly bounded by a positive constant $M$, and that there exists a positive constant $L$ so that, for any measure $\mu$,	$\omega \mapsto t_{\omega}(z)$ is $L$-Lipschitz for any $z\in \mathcal{Z}$. Then, there exists a positive constant $c(\Omega)$ greater than $1$ that depends only on $\Omega$ and $d$ so that, for any probability measure $\mu$ on $\mathcal{Z}$, the following bound holds for any $0<\epsilon\leq M$:
\begin{align*}  
 D\left(\epsilon,\mathcal{T},L_2(\mu)\right)\leq c(\Omega)\parens{\frac{\max\parens{L\diam(\Omega),M}}{\epsilon}}^d.
\end{align*}
\end{proposition}
\begin{proof}
	First using \citep[Lemma~9.18]{Kosorok2008} and \citep[Paragraph~8.1.2]{Kosorok2008}, we know that the $\epsilon$-packing number $D\left(\epsilon,\mathcal{T},L_2(\mu)\right)$ is  smaller than the $\frac{\epsilon}{2}$-bracketing number $N_{[]}\left(\frac{\epsilon}{2},\mathcal{T},L_2(\mu)\right)$. Hence, we only need to control the bracketing number. To this end, we recall that the function $\omega\mapsto t_{\omega}(z)$ is $L$-Lipschitz for any $z\in \mathcal{Z}$, so that  \citep[Example~19.7]{van2000asymptotic} ensures the existence of a positive constant $c(\Omega)$ that depends only on $\Omega$ for which the following inequality holds for any $0<\epsilon< L\text{diam}(\Omega)$:
\begin{equation*}
    1\leq N_{[]}\left(\epsilon,\mathcal{T},L_2(\mu)\right)\leq c(\Omega)\left(\frac{L\diam(\Omega)}{\epsilon}\right)^d.
\end{equation*}
Moreover, since the $\epsilon$-bracketing number is decreasing in $\epsilon$, it holds that:
$$N_{[]}\left(\epsilon,\mathcal{T},L_2(\mu)\right)\leq N_{[]}\left(\epsilon_{-},\mathcal{T},L_2(\mu)\right) \leq c(\Omega)\parens{\frac{L\diam(\Omega)}{\epsilon_{-}}}^{d},$$ 
for any $\epsilon\geq L\text{diam}(\Omega)$ and $\epsilon_{-}\leq L\diam(\Omega)$. Taking the limit when $\epsilon_{-}$ approaches $L\text{diam}(\Omega)$ yields $N_{[]}\left(\epsilon,\mathcal{T},L_2(\mu)\right)\leq c(\Omega)$ for any $\epsilon\geq L\diam(\Omega)$. Hence, we have shown so far that for any $\epsilon>0$:
\begin{align*}
	N_{[]}\left(\epsilon,\mathcal{T},L_2(\mu)\right)\leq c(\Omega)\max\parens{1,\left(\frac{L\diam(\Omega)}{\epsilon}\right)^d}.
\end{align*}
Moreover, by noticing that $\max(1,\frac{L\diam(\Omega)}{\epsilon})\leq \frac{\max(M,L\diam(\Omega))}{\epsilon}$ for any $\epsilon\leq M$, we further have that:
\begin{align*}
	N_{[]}\left(\epsilon,\mathcal{T},L_2(\mu)\right)\leq  c(\Omega)\parens{\frac{\max\parens{M,L\diam(\Omega)}}{\epsilon}}^d.
\end{align*}
Finally, recalling that $ D\left(\epsilon,\mathcal{T},L_2(\mu)\right)\leq  N_{[]}\left(\frac{\epsilon}{2},\mathcal{T},L_2(\mu)\right)$, we get that $D\left(\epsilon,\mathcal{T},L_2(\mu)\right)\leq  2^dc(\Omega)\parens{\frac{\max\parens{M,L\diam(\Omega)}}{\epsilon}}^d$. 
The desired bound follows after redefining $c(\Omega)$ to include the factor $2^d$ (\textit{i.e.}, $c(\Omega)\rightarrow 2^dc(\Omega)$).
\end{proof}

\begin{theorem}[Maximal inequality for degenerate bounded and Lipschitz $U$-processes]\label{prop:maximal_ineq_degenerate_u_process}
	Let $k$ be either $1$ or $2$. 
Consider a parametric family 
	$\mathcal{T}\coloneqq \braces{\otimes^{k} \mathcal{Z}\ni (z_1,\ldots,z_k)\mapsto t_{\omega}(z_1,\ldots,z_k)\in \mathbb{R}\mid\omega \in \Omega}$ 
	of real-valued functions over $\otimes^{k} \mathcal{Z}$ and indexed by a parameter $\omega\in \Omega$, where $\Omega$ is a compact subset of $\mathbb{R}^d$. 
	For a given probability distribution $\PP$ over $\mathcal{Z}$, assume that all elements $t_{\omega}$ are degenerate w.r.t. $\PP$, meaning that:
	\begin{align*}
		\begin{cases}
		\mathbb{E}_{\bar{z}\sim \PP}\brackets{ t_{\omega}(\bar{z})} = 0, &\qquad \text{if} \quad k=1\\ 
			\mathbb{E}_{\bar{z}\sim \PP}\brackets{ t_{\omega}(z,\bar{z})} =\mathbb{E}_{\bar{z}\sim \PP}\brackets{ t_{\omega}(\bar{z},z)}=0,\quad \forall z\in \mathcal{Z}, &\qquad \text{if} \quad k=2. 
		\end{cases}
	\end{align*}
	Furthermore, assume that all functions in $\mathcal{T}$ are uniformly bounded by a positive constant $M$ and that there exists a positive constant $L$ so that $\omega \mapsto t_{\omega}(z_1,\ldots,z_k)$ is $L$-Lipschitz for any $(z_1,\ldots,z_k)\in \otimes^k \mathcal{Z}$.  
	 Given i.i.d. samples $(z_i)_{1\leq i\leq n}$ from $\PP$, consider the following $U$-statistic $U_n^k$:
\begin{align*}
    U_n^k t_\omega\coloneqq 
    \begin{cases}
    	    	\frac{1}{n}\sum_{i=1}^n t_{\omega}(z_i),&\qquad \text{if} \quad k=1\\
    	\frac{1}{n(n-1)}\sum_{\substack{i,j=1\\ i\neq j}}^n t_{\omega}(z_i,z_j),&\qquad \text{if} \quad k=2.
    \end{cases}
\end{align*}
Then, there exists a universal positive constant $c(\Omega)$ greater than $1$ that depends only on $\Omega$ and $d$ such that for any $p\in \braces{1,2}$:
\begin{align*}
	\mathbb{E}_{\PP}\brackets{\sup_{\omega\in \Omega} \verts{U^k_n t_{\omega}}^p}^{\frac{1}{p}}\leq n^{-{\frac{k}{2}}}  c(\Omega)\max\parens{ML\diam(\Omega),M^2}^{\frac{1}{2}}.
\end{align*}


\end{theorem}
\begin{proof}
	{\bf Maximal inequality for degenerate $U$-processes.}
We  will first apply the general result in \citep[Maximal inequality]{sherman1994maximal} which controls $\mathbb{E}_{\PP}\brackets{\sup_{\omega\in \Omega} \verts{U_n^k t_{\omega}}}$ in terms of the packing number of $\mathcal{T}$.
First note, by assumption, that the functions $t_{\omega}(z_1,\ldots,z_k)$  are  uniformly bounded by a positive constant $M$. Therefore, the constant function $T(z_1,\ldots,z_k) \coloneqq M$ is an envelope for $\mathcal{T}$, \textit{i.e.}, $T$ satisfies $T(z_1,\ldots,z_k)\geq \sup_{\omega\in \Omega} \verts{t_{\omega}(z_1,\ldots,z_k)}$ for any $(z_1,\ldots,z_k)\in\otimes^k \mathcal{Z}$. 
The envelope $T$ is, a fortiori,  square $\mu$-integrable for any probability measure $\mu$ on $\otimes^k \mathcal{Z}$. Hence, we can apply \citep[Maximal inequality]{sherman1994maximal} with the choice $T$ for the envelope function and set the integer $m$ appearing in the result to $m=d$ to get the following bound:
\begin{align}\label{eq:maximal_sherman}
	\mathbb{E}_{\PP}\brackets{\sup_{\omega\in \Omega} \verts{U^k_n t_{\omega}}^p}^{\frac{1}{p}}\leq n^{-\frac{k}{2}} \Gamma 
	\mathbb{E}\brackets{\Verts{T}_{\mu_n,2} \int_{0}^{\delta_n} \parens{D\parens{\epsilon \Verts{T}_{\mu_n,2},\mathcal{T},L_2(\mu_n) }}^{\frac{1}{2dp}}\diff \epsilon  }, 
\end{align}
where $\Gamma$ is a positive universal constant\footnote{The constant $\Gamma$ appearing \citep[Maximal inequality]{sherman1994maximal} depends only on $k$, $p$ and $m$, \textit{i.e.}, $\Gamma\coloneqq g(k,p,m)$. Since, we are only interested in $k\leq 2$ and $p\leq 2$ and $m$ is fixed to $d$, we choose $\Gamma$ to be $\max_{1\leq k,p\leq 2}g(k,p,d)^{\frac{1}{p}}$, so that it is the same in all our cases.} that depends only on $d$ and that we choose to be greater than $1$,  while $\mu_{n}$ are suitably chosen  probability measures on $\otimes^k\mathcal{Z}$ that possibly depend on the samples $z_1,\ldots,z_n$ and other random variables, and $\delta_n \Verts{T}_{\mu_n,2} \coloneqq \sup_{\omega\in \Omega} \Verts{t_{\omega}}_{\mu_n,2}$. Here, the expectation symbol in the right-hand side is over all randomness on which $\mu_n$ might depend. Note that the original result in \citep[Maximal inequality]{sherman1994maximal} is stated using a slightly different definition of the packing number but which is still equivalent to the statement above in our setting\footnote{In \citep[Maximal inequality]{sherman1994maximal}, the author considers a modified version of the $\epsilon$-packing number (call it $\tilde{D}(\epsilon, \mathcal{T},L_{2}(\mu))$) associated to $L_2(\mu)$ but endowed with a normalized version of the standard norm on  $L_2(\mu)$: $\Verts{f}_{\mu}\coloneqq\frac{\Verts{f}_{\mu,2}}{\Verts{T}_{\mu,2}}$.  Both numbers are related by the following identity:  $\tilde{D}\parens{\epsilon, \mathcal{T},L_{2}(\mu)} = D(\epsilon{\Verts{T}_{\mu,2}}, \mathcal{T},L_{2}(\mu))$, thus making the statement \eqref{eq:maximal_sherman} equivalent to the original statement in \citep[Maximal inequality]{sherman1994maximal}.
}. 


In our setting, the envelope function is constant and equal to $M$, and by definition $\delta_n \leq 1$. Hence, the inequality in  \cref{eq:maximal_sherman} further becomes:
\begin{align}\label{eq:maximal_inequality_sherman}
	\mathbb{E}_{\PP}\brackets{\sup_{\omega\in \Omega} \verts{U_n^k t_{\omega}}^p}^{\frac{1}{p}}\leq n^{-\frac{k}{2}} M\Gamma 
	\mathbb{E}_{\PP}\brackets{ \int_{0}^{1} \parens{D\parens{\epsilon\Verts{T}_{\mu_n,2},\mathcal{T},L_2(\mu_n) }}^{\frac{1}{2dp}}\diff \epsilon  }. 
\end{align}
 We simply need to control the packing number $D\parens{\epsilon\Verts{T}_{\mu,2},\mathcal{T},L_2(\mu)}$ independently of the probability measure $\mu$.  

{\bf Control on the packing number.} 
We have shown that the constant function $T(z_1,\ldots,z_k) \coloneqq M$ is an envelope for $\mathcal{T}$ which is, a fortiori, square $\mu$-integrable for any probability measure $\mu$ with $\Verts{T}_{\mu,2}= M<+\infty$. 
 Moreover, the functions $\omega\mapsto t_{\omega}(z_1,\ldots,z_k)$ are $L$-Lipschitz for any $(z_1,\ldots,z_k)\in \otimes^k\mathcal{Z}$. 
 We can therefore apply \cref{prop:estimate_covering_numbers} which ensures the existence of a positive constant $c(\Omega)$ greater than $1$ and that depends only on $\Omega$ and $d$ so that the following estimate on the $\epsilon$-packing number of the class $\mathcal{T}$ w.r.t. $L_2(\mu)$ holds:
\begin{align}\label{eq:bound_packing_uniform}
D\left(\epsilon\Verts{T}_{\mu,2},\mathcal{T},L_2(\mu)\right)\leq \underbrace{c(\Omega)\parens{\max\parens{\frac{L\diam(\Omega)}{M},1}}^d}_{A}\parens{\frac{1}{\epsilon}}^d, \qquad \forall\epsilon\in(0,  1].
\end{align}
Combining \cref{eq:bound_packing_uniform} with \cref{eq:maximal_inequality_sherman} yields:
\begin{align*}
	\mathbb{E}_{\PP}\brackets{\sup_{\omega\in \Omega} \verts{U_n^k t_{\omega}}^p}^{\frac{1}{p}}&\leq n^{-\frac{k}{2}} M\Gamma 
	\mathbb{E}_{\PP}\brackets{ \int_{0}^{1} \parens{A\epsilon^{-d}
	}^{\frac{1}{2dp}}\diff \epsilon} = n^{-\frac{k}{2}}M\Gamma A^{\frac{1}{2dp}} \underbrace{\int_0^1 \epsilon^{-\frac{1}{2p}}\diff \epsilon}_{\leq 2} \\
	&\leq  
	 2n^{-\frac{k}{2}}\Gamma c(\Omega)^{\frac{1}{2d}}\max\parens{L\diam(\Omega),M^2}^{\frac{1}{2}},
\end{align*}
where, for the last inequality, we used that $A^{\frac{1}{2dp}}\leq A^{\frac{1}{2d}} = c(\Omega)^{\frac{1}{2d}}\max\parens{\frac{L\diam(\Omega)}{M},1}^{\frac{1}{2}}$ since $A$ is greater than $1$. 
The desired result follows after redefining $c(\Omega)$ as $2\Gamma c(\Omega)^{\frac{1}{2d}}$ which is  a positive constant that depends only on $\Omega$ and $d$. 
\end{proof}

The following next propositions are particular instances of \cref{prop:maximal_ineq_degenerate_u_process} and will be used to obtain the main bounds.  
\begin{proposition}[Maximal inequality for empirical processes]\label{prop:empirical_process}
	 
	Consider a parametric family 
	$\mathcal{T}\coloneqq \braces{\mathcal{Z}\ni z\mapsto t_{\omega}(z)\in \mathbb{R}\mid\omega \in \Omega}$ 
	of real-valued functions defined over a subset $\mathcal{Z}$ of a Euclidean space and indexed by a parameter $\omega\in \Omega$, where $\Omega$ is a compact subset of $\mathbb{R}^d$. 
	Assume that all functions in $\mathcal{T}$ are uniformly bounded by a positive constant $M$ and that there exists a positive constant $L$ so that $\omega \mapsto t_{\omega}(z)$ is $L$-Lipschitz for any $z\in \mathcal{Z}$. Consider a probability distribution $\PP$ over $\mathcal{Z}$ and let $(z_i)_{1\leq i\leq n}$ be i.i.d. samples drawn from $\PP$, then there exists a positive constant $c(\Omega)$ greater than $1$ that depends only on $\Omega$ and $d$, such that for any integer $p \in \{1,2\}$:
	\begin{align*}
		\mathbb{E}_{\PP}\brackets{ \sup_{\omega\in \Omega} \verts{ \mathbb{E}_{z\sim \PP}\brackets{t_{\omega}(z) } - \frac{1}{n}\sum_{i=1}^n t_{\omega}(z_i) }^p }^{\frac{
		1}{p}}\leq \sqrt{\frac{1}{n}}c(\Omega)\max(ML\diam(\Omega),M^2)^{\frac{1}{2}}.
	\end{align*}
\end{proposition}
\begin{proof}
The upper-bound is a direct consequence of \cref{prop:maximal_ineq_degenerate_u_process}. Indeed consider the  family $\mathcal{S}$ of functions of the form $s_{\omega}(z) = t_{\omega}(z)-\mathbb{E}_{\bar{z}\sim\PP}\brackets{t_{\omega}(\bar{z})}$, for any $z\in\mathcal{Z}$. Then clearly, the process $U_n^1 s_{\omega}\coloneqq\frac{1}{n}\sum_{i=1}^n s_{\omega}(z_i)$ is degenerate of order $k=1$, and the family $\mathcal{S}$ is uniformly bounded by $2M$ and is $2L$-Lipschitz. Hence, by \cref{prop:maximal_ineq_degenerate_u_process}, the following maximal inequality holds:
\begin{align*}
	\mathbb{E}_{\PP}\brackets{\sup_{\omega\in \Omega} \verts{U^1_n s_{\omega}}^p}^{\frac{1}{p}}\leq 2n^{-{\frac{1}{2}}}  c(\Omega)\max\parens{ML\diam(\Omega),M^2}^{\frac{1}{2}}. 
\end{align*}
We get the desired upper-bound by redefining $c(\Omega)$ to contain the factor $2$.
\end{proof}


\begin{proposition}[Maximal inequality for $U$-processes of order 2]\label{lem:u_stat_sup_p_omega}
Consider a parametric family 
	$\mathcal{T}\coloneqq\braces{\mathcal{Z}\times \mathcal{Z}\ni (z,z')\mapsto t_{\omega}(z,z')\in \mathbb{R}\mid\omega \in \Omega}$ 
	of real-valued functions indexed by a parameter $\omega\in \Omega$, where $\Omega$ is a compact subset of $\mathbb{R}^d$ and $\mathcal{Z}$ is a subset of a Euclidean space. Assume that the functions in $\mathcal{T}$ are symmetric in their arguments, \textit{i.e.}, $t_{\omega}(z,z') = t_{\omega}(z',z)$. Additionally, assume that all functions in $\mathcal{T}$ are uniformly bounded by a positive constant $M$ and that there exists a positive constant $L$ so that $\omega \mapsto t_{\omega}(z,z')$ is $L$-Lipschitz for any $(z,z')\in \mathcal{Z}\times\mathcal{Z}$. 
	Consider a probability distribution $\PP$ over $\mathcal{Z}$ and let $(z_i)_{1\leq i\leq n}$ be i.i.d. samples drawn from $\PP$, and define the following statistic:
\begin{align*}
    \tau_\omega\coloneqq\mathbb{E}_{z,z'\sim\PP\otimes\PP}\left[t_\omega(z,z')\right]+\frac{1}{n^2}\sum_{i,j=1}^n t_\omega(z_i,z_j)-\frac{2}{n}\sum_{i=1}^n\mathbb{E}_{z\sim\PP}\left[t_\omega(z,z_i)\right].
\end{align*}
Then there exists a universal positive constant $c(\Omega)$ greater than $1$ that depends only on $\Omega$ and $d$ such that:
\begin{align*}
	\mathbb{E}_{\PP}\brackets{\sup_{\omega\in \Omega} \verts{\tau_{\omega}}}\leq \frac{1}{n}  c(\Omega)\max\parens{ML\diam(\Omega),M^2}^{\frac{1}{2}}. 
\end{align*}
\end{proposition}

\begin{proof}
The proof will proceed by first decomposing $\tau_{\omega}$ into  a sum of a degenerate $U$-process and a term of order $O(\frac{1}{n})$. The maximal inequality for degenerate $U$-processes from \cite{sherman1994maximal} will be employed to obtain the desired bound.

{\bf Decomposition of $\tau_{\omega}$.} 
Consider the following function  defined over $\mathcal{Z}\times \mathcal{Z}$ and indexed by elements $\omega\in \Omega$:
\begin{equation}\label{eq:def_degenerate_kernel}
    s_\omega(z,z')=t_\omega(z,z')-\mathbb{E}_{\bar{z}\sim\PP}\left[t_\omega(z,\bar{z})\right]-\mathbb{E}_{\bar{z}\sim\PP}\left[t_\omega(\bar{z},z')\right]+\mathbb{E}_{\{\bar{z},\underline{z}\}\sim\PP\otimes\PP}\left[t_\omega(\bar{z},\underline{z})\right]. 
\end{equation}
By direct calculation, we decompose $\tau_{\omega}$ into two higher order terms and a third  term, $U_n^2 s_{\omega}$, involving $s_{\omega}$, which happens to be a $U$-statistic: 
\begin{equation*}
    \tau_\omega= \overbrace{\frac{1}{(n-1)}\sum_{\substack{i,j=1\\i\neq j}}^n s_\omega\big(z_i,z_j\big)} ^{U_n^2 s_\omega}-\frac{1}{n^2(n-1)} \sum_{\substack{i,j=1\\i\neq j}}^n t_\omega(z_i,z_j)+\frac{1}{n^2}\sum_{i=1}^n t_\omega(z_i,z_i).
\end{equation*}
Using the triangle inequality in the above equality and recalling that, by assumption, $t_{\omega}(z,z')$ is uniformly bounded by a positive constant $M$, it follows that:
\begin{align*}
    \verts{\tau_\omega}\leq\left|U_{n}^2 s_\omega\right|+\frac{1}{n^2(n-1)}\sum_{\substack{i,j=1\\i\neq j}}^{n} \left|t_\omega(z_i,z_j)\right|+\frac{1}{n^2}\sum_{i=1}^{n}\left|t_\omega(z_i,z_i)\right|
    \leq 
     \verts{U_n^2 s_{\omega}} + \frac{2M}{n}.
\end{align*}
Furthermore, taking the supremum over $\omega$ followed by the expectation over samples yields: 
\begin{align}\label{eq:main_decomposition_u_stat}
	\mathbb{E}_{\PP}\brackets{\sup_{\omega\in \Omega} \verts{\tau_{\omega}}}\leq \mathbb{E}_{\PP}\brackets{\sup_{\omega\in \Omega} \verts{U_n^2 s_{\omega}}} + \frac{2M}{n}.
\end{align}
Hence, it only remains to control the first term in the above inequality. To this end, we will use a maximal inequality for degenerate $U$-processes due to \cite{sherman1994maximal}. 

{\bf Maximal inequality for degenerate $U$-processes.} We will first check that $U_n^2 s_{\omega}$ is a degenerate statistic for a given $\omega\in \Omega$. To this end, simple calculations show that for any $z$ in $\mathcal{Z}$:
\begin{align*}
	\mathbb{E}_{\bar{z}\sim \PP}\brackets{s_{\omega}(z,\bar{z})} =\mathbb{E}_{\bar{z}\sim \PP}\brackets{s_{\omega}(\bar{z},z)} = 0. 
\end{align*}
The above equalities precisely ensure that $U_n^2 s_{\omega}$ is a degenerate $U$-statistic for $\PP$. 
Consider now the family $\mathcal{S}\coloneqq \braces{\mathcal{Z}\times \mathcal{Z}\ni(z,z')\mapsto s_{\omega}(z,z')\in \mathbb{R}\mid\omega\in\Omega}$. We show that $\mathcal{S}$ is uniformly bounded and Lipschitz which allows to directly apply the result stated in \cref{prop:maximal_ineq_degenerate_u_process}, which is a special case of the more general  result in \citep[Maximal inequality]{sherman1994maximal}. 
First note, by assumption, that the functions $t_{\omega}(z,z')$ are uniformly bounded by a positive constant $M$.  Hence, using \cref{eq:def_degenerate_kernel}, it follows that $s_{\omega}(z,z')$ is uniformly bounded by $4M$. Moreover, the functions $\omega\mapsto t_{\omega}(z,z')$ are $L$-Lipschitz for any $z,z'$ in $\mathcal{Z}$. Hence, from  \cref{eq:def_degenerate_kernel}, we directly have that $\omega \mapsto s_{\omega}(z,z')$ is $4L$-Lipschitz for any $z,z'\in \mathcal{Z}$. We can directly apply \cref{prop:maximal_ineq_degenerate_u_process} with $k=2$ and $p=1$ to  $\mathcal{S}$ and get the following maximal inequality:
\begin{align*}
	\mathbb{E}_{\PP}\brackets{\sup_{\omega\in \Omega} \verts{U^2_n s_{\omega}}}\leq 4n^{-1}  c(\Omega)\max\parens{ML\diam(\Omega),M^2}^{\frac{1}{2}}. 
\end{align*}
We obtain an upper-bound  on $\mathbb{E}_{\PP}\brackets{\sup_{\omega\in \Omega} \verts{\tau_{\omega}}}$ by combining the above inequality with 
  \cref{eq:main_decomposition_u_stat}, then noticing that $2M\leq 2c(\Omega)\max\parens{L\diam(\Omega),M}$ so that:
  \begin{align*}
	\mathbb{E}_{\PP}\brackets{\sup_{\omega\in \Omega} \verts{\tau_{\omega}}}\leq 6n^{-1}  c(\Omega)\max\parens{ML\diam(\Omega),M^2}^{\frac{1}{2}}. 
\end{align*}
Finally, the desired result follows by redefining $c(\Omega)$ to include the factor $6$ in the above inequality.  
\end{proof}

\section{Differentiability Results}\label{sec:proof_diff_results}
The proofs of \cref{prop:fre_diff_L_v,prop:fre_diff_L} are direct applications of the following more general result. 
\begin{proposition}
	Let $\mathcal{U}$ be an open non-trivial subset of $\mathbb{R}^d$. 
	Consider a real-valued function $\ell: (\omega,v,y)\mapsto \ell(\omega,v,y)$ defined on $\mathcal{U}\times \mathbb{R}\times \mathcal{Y}$ that is of class $C^{3}$ jointly in $(\omega,v)$ and whose derivatives are jointly continuous in $(\omega,v,y)$. For a given  probability distribution $\PP$ over $\mathcal{X}\times \mathcal{Y}$, consider the following functional defined over $\mathcal{U}\times\mathcal{H}$:
	\begin{align*}
		L(\omega,h) \coloneqq \mathbb{E}_{\PP}\brackets{\ell(\omega,h(x),y)}.
	\end{align*}
	Under \cref{assump:K_bounded,assump:compact}, the following properties hold for $L$:
		\begin{itemize}
			\item $L$ admits finite values for any $(\omega,h)\in \mathcal{U}\times\mathcal{H}$.
			\item $(\omega,h)\mapsto L(\omega,h)$ is Fr\'echet differentiable with partial derivatives $\partial_{\omega}L(\omega,h)$ and $\partial_{h}L(\omega,h)$ at any point $(\omega,h)\in \mathcal{U}\times \mathcal{H}$ given by:
				\begin{align*}
					\partial_{\omega}L(\omega,h) &= \mathbb{E}_{\PP}\brackets{\partial_{\omega} \ell(\omega,h(x),y)} \in \mathbb{R}^d ,\\
					\partial_{h}L(\omega,h) &= \mathbb{E}_{\PP}\brackets{\partial_v \ell(\omega,h(x),y)K(x,\cdot)}\in \mathcal{H}.
				\end{align*}
			\item The map $(\omega,h)\mapsto \partial_{h}L(\omega,h)$ is differentiable. Moreover, for any $(\omega,h)\in \mathcal{U}\times\mathcal{H}$, its partial derivatives $\partial_{\omega,h}^2L(\omega,h)$ and $\partial_{\omega,h}^2L(\omega,h)$ at $(\omega,h)$ are Hilbert-Schmidt operators given by:
				\begin{align*}
					\partial_{\omega,h}^2L(\omega,h)&= \mathbb{E}_{\PP}\brackets{\partial_{\omega,v}^2 \ell(\omega,h(x),y)K(x,\cdot)} \in \mathcal{L}(\mathcal{H},\mathbb{R}^d), \\
					\partial_{h}^2L(\omega,h) &= 
					\mathbb{E}_{\PP}\brackets{\partial_{v}^2 \ell(\omega,h(x),y)K(x,\cdot)\otimes K(x,\cdot)}\in \mathcal{L}(\mathcal{H},\mathcal{H}).
				\end{align*}
		\end{itemize}  
\end{proposition}
\begin{proof}
	
	{\bf Finite values.} Fix $\omega\in \mathcal{U}$ and $h\in \mathcal{H}$. We will first show that  $h$ is bounded on $\mathcal{X}$. By the reproducing property, we know that $\verts{h(x)}\leq \Verts{h}_{\mathcal{H}}\sqrt{K(x,x)}$ for any $x\in \mathcal{X}$. Moreover, the kernel $K$ is bounded by a constant $\kappa$ thanks to \cref{assump:K_bounded}. Consequently, $\verts{h(x)}$ is upper-bounded by $\Verts{h}_{\mathcal{H}}\sqrt{\kappa}$ for any $x\in \mathcal{X}$. 
    
    Denote by $\mathcal{I}$ the compact interval defined as $ \mathcal{I}= [-\Verts{h}_{\mathcal{H}}\sqrt{\kappa},\Verts{h}_{\mathcal{H}}\sqrt{\kappa} ]$. By \cref{assump:compact}, the set $\mathcal{Y}$ is compact so that $\mathcal{I}\times\mathcal{Y}$ is also compact. 
    Moreover, we know, by assumption on  $\ell$, that $(\omega, v, y)\mapsto\ell(\omega, v, y)$ is continuous on $\mathcal{U}\times \mathbb{R}\times \mathcal{Y}$. 
    Therefore, $(v, y)\mapsto\ell(\omega, v, y)$  must be bounded by some finite constant $C$ on the compact set $\mathcal{I}\times \mathcal{Y}$. This allows to deduce that $(x, y)\mapsto\ell(\omega, h(x), y)$ is bounded by $C$ for any $(x,y)\in\mathcal{X}\times \mathcal{Y}$ and a fortiori $\PP$-integrable, which shows that $L(\omega,h)$ is finite. 

	{\bf Fr\'echet differentiability of $L$.} 
	Let $(\omega, h)\in \mathcal{U}\times \mathcal{H}$. Consider $(\omega_j,h_j)_{j\geq 1}$ a sequence of elements in $\mathcal{U}\times \mathcal{H}$ converging to it, \textit{i.e.}, $(\omega_j,h_j)\rightarrow (\omega,h)$ with $(\omega_j,h_j)\neq (\omega,h)$ for any $j\geq 0$.    
	Define the sequence of functions $r_j:\mathcal{X}\times\mathcal{Y}\to\mathbb{R}$ for any $(x,y)\in\mathcal{X}\times\mathcal{Y}$ as follows:
    \begin{equation}\label{eq:expression_rn}
r_j(x,y)=\frac{\ell\left(\omega_j,h_j(x),y\right)-\ell\left(\omega,h(x),y\right)-\left\langle\partial_v\ell\left(\omega,h(x),y\right)K(x,\cdot),h_j-h\right\rangle_\mathcal{H}- \langle \partial_{\omega}\ell(\omega,h(x),y),\omega_j-\omega\rangle}{\left\|(\omega_j,h_j)-(\omega,h)\right\|}.
    \end{equation}    
    We will first show that $\mathbb{E}_{\mathbb{D}}\brackets{\verts{r_j(x,y)}}$ convergence to $0$ by the dominated convergence theorem \citep[Theorem~1.34]{rudin1987real}. 
	By the reproducing property, note that $\ell(\omega,h(x),y) = \ell(\omega,\langle h,K(x,\cdot)\rangle_{\mathcal{H}},y)$. Hence, since $\ell$ is jointly differentiable in $(\omega,v)$ for any $y$, it follows that $(\omega,h)\mapsto \ell(\omega,h(x),y)$ is also differentiable for any $(x,y)$ by composition with the evaluation map $(\omega,h)\mapsto (\omega, \langle h,K(x,\cdot)\rangle_{\mathcal{H}}$ which is differentiable. Hence, the sequence $r_j(x,y)$ converges to $0$ for any $(x,y)\in \mathcal{X}\times \mathcal{Y}$. Moreover, by the mean-value theorem, there exists $0\leq c_j\leq 1$ such that:
	\begin{align*}
		r_j(x,y)= \frac{\left\langle\parens{\partial_v\ell\left(\bar{\omega}_j,\bar{h}_j(x),y\right)- \partial_v\ell\left(\omega,h(x),y\right)}K(x,\cdot),h_j-h\right\rangle_\mathcal{H}- \langle \partial_{\omega}\ell\left(\bar{\omega}_j,\bar{h}_j(x),y\right)- \partial_{\omega}\ell\left(\omega,h(x),y\right),\omega_j-\omega\rangle}{\left\|(\omega_j,h_j)-(\omega,h)\right\|},
	\end{align*}
	where $(\bar{\omega}_j,\bar{h}_j)\coloneqq(1-c_j)(\omega,h) + c_j(\omega_j,h_j)$. 
	We will show that $r_j(x,y)$ are bounded for $j$ large enough. We first construct a compact set that will contain all elements of the form $(\omega_j, h_j(x),y)$ and $(\bar{\omega}_j, \bar{h}_j(x),y)$ for all $j$ large enough. 
	Since $\omega$ is an element in the open set $\mathcal{U}$, there exists a closed ball $\mathcal{B}(\omega,R)$ centered in $\omega$ and with some radius $R$ small enough so that $\mathcal{B}(\omega,R)$ is included in $\mathcal{U}$. For all $j$ large enough, $\omega_j$ and $\bar{\omega}_j$ belong to $\mathcal{B}(\omega,R)$ as these sequences converge to $\omega$. Moreover, $h_j$ and $\bar{h}_j$ are convergent sequences. Consequently, they must be bounded by some constant $B$. By the reproducing property, and recalling that the kernel $K$ is bounded by $\kappa$ by \cref{assump:K_bounded}, it follows that $\max(\verts{h_{j}(x)},\verts{\bar{h}_{j}(x)})\leq B\kappa$.  Consider now the set {$\mathcal{W}\coloneqq\mathcal{B}(\omega, R)\times \mathcal{B}_1(0, B\kappa)\times \mathcal{Y}$} which is a product of compact sets (recalling that $\mathcal{Y}$ is compact by \cref{assump:compact}){, where $\mathcal{B}_1(0, B\kappa)$ is the closed ball in $\mathbb{R}$ centered at $0$ and of radius $B\kappa$}.   
	For $j$ large enough, we have established that  $(\omega_j, h_j(x),y)$ and $(\bar{\omega}_j, \bar{h}_j(x),y)$ belong to $\mathcal{W}$ for any $(x,y)\in \mathcal{X}\times\mathcal{Y}$. 
	Since, by assumption on $\ell$, $\partial_{v}\ell(\omega,v,y)$ and $\partial_{\omega}\ell(\omega,v,y)$ are continuous, they must be bounded on the compact set $\mathcal{W}$ by some constant $C$. This allows to deduce from the expression of $r_{j}(x,y)$ above that $r_{j}(x,y)$ is bounded, and a fortiori dominated by an integrable function (a constant function). We then deduce that $\mathbb{E}_{\mathbb{\PP}}\brackets{\verts{r_j(x,y)}}$ converges to $0$ by application of the dominated convergence theorem \citep[Theorem~1.34]{rudin1987real}.
	
	Recalling \cref{eq:expression_rn},  $\mathbb{E}_{\mathbb{\PP}}\brackets{r_j(x,y)}$ admits the following expression:
	\begin{align}\label{eq:expression_rn_2}
		\mathbb{E}_{\mathbb{\PP}}\brackets{r_j(x,y)} = \frac{L(\omega_j,h_j)-L(\omega,h)-\mathbb{E}_{\PP}\brackets{\left\langle\partial_v\ell\left(\omega,h(x),y\right)K(x,\cdot),h_j-h\right\rangle_\mathcal{H}}- \langle \mathbb{E}_{\PP}\brackets{\partial_{\omega}\ell(\omega,h(x),y)},\omega_j-\omega\rangle}{\left\|(\omega_j,h_j)-(\omega,h)\right\|}.
	\end{align}
	The convergence to $0$ of the above expression precisely means that $L$ is differentiable at $(\omega,h)$ provided that the linear form $g\mapsto\mathbb{E}_{\PP}\brackets{\left\langle\partial_v\ell\left(\omega,h(x),y\right)K(x,\cdot),g\right\rangle_\mathcal{H}}$ is bounded. To establish this fact, consider the RKHS-valued function $(x,y)\mapsto \partial_v\ell\left(\omega,h(x),y\right)K(x,\cdot)$. 
	This function is Bochner-integrable in the sense that $\mathbb{E}_{\PP}\brackets{\Verts{\partial_v\ell\left(\omega,h(x),y\right)K(x,\cdot)}_\mathcal{H}}$ is finite \citep[Definition~1,~Chapter~2]{diestel1977vector}. Indeed, we have the following:
\begin{align*}
	\mathbb{E}_{\PP}\brackets{\Verts{\partial_v\ell\left(\omega,h(x),y\right)K(x,\cdot)}_\mathcal{H}} \coloneqq \mathbb{E}_{\PP}\brackets{\verts{\partial_v\ell\left(\omega,h(x),y\right)}\sqrt{K(x,x)}}
	\leq \sqrt{\kappa}\mathbb{E}_{\PP}\brackets{\verts{\partial_v\ell\left(\omega,h(x),y\right)}}<+\infty,
\end{align*}
where, for the inequality, we used that $(x,y)\mapsto\partial_v\ell(\omega,h(x),y)$ is bounded as shown previously. Consequently, $\mathbb{E}\brackets{\partial_v\ell\left(\omega,h(x),y\right)K(x,\cdot)}$ is an element in $\mathcal{H}$ satisfying:
\begin{align*}
	\langle\mathbb{E}\brackets{\partial_v\ell\left(\omega,h(x),y\right)K(x,\cdot)},  g\rangle_{\mathcal{H}}=\mathbb{E}_{\PP}\brackets{\left\langle\partial_v\ell\left(\omega,h(x),y\right)K(x,\cdot),g\right\rangle_\mathcal{H}},\forall g\in \mathcal{H}.
\end{align*}
The above property follows from \citep[Theorem~6,~Chapter~2]{diestel1977vector} for Bochner-integrable functions that allows exchanging the integral and the application of a continuous linear map (here the scalar product with an element $g$). The above identity establishes that $g\mapsto\mathbb{E}_{\PP}\brackets{\left\langle\partial_v\ell\left(\omega,h(x),y\right)K(x,\cdot),g\right\rangle_\mathcal{H}}$ is bounded and provides the desired expression for $\partial_h L(\omega,h)$. The expression for $\partial_{\omega} L(\omega,h)$ directly follows from the last term in \cref{eq:expression_rn_2}.

{\bf Fr\'echet differentiability of $\partial_h L$. } We use the same proof strategy as for the differentiability of $L$.  

Let $(\omega, h)\in \mathcal{U}\times \mathcal{H}$. Consider $(\omega_j,h_j)_{j\geq 1}$ a sequence of elements in $\mathcal{U}\times \mathcal{H}$ converging to it, \textit{i.e.}, $(\omega_j,h_j)\rightarrow (\omega,h)$ with $(\omega_j,h_j)\neq (\omega,h)$ for any $j\geq 0$.    
	Define the sequence of functions $s_j:\mathcal{X}\times\mathcal{Y}\to\mathcal{H}$ as follows:
    \begin{align}\label{eq:expression_sn}
    \begin{split}
\left\|(\omega_j,h_j)-(\omega,h)\right\|s_j(x,y)= &\parens{\partial_v\ell\left(\omega_j,h_j(x),y\right)-\partial_v\ell\left(\omega,h(x),y\right)}K(x,\cdot)\\
&-\partial_v^2\ell\left(\omega,h(x),y\right)K(x,\cdot)\otimes K(x,\cdot)\parens{h_j-h}
- \parens{\omega_j-\omega}^{\top} \partial_{\omega,v}^2\ell(\omega,h(x),y)K(x,\cdot).
    \end{split}
\end{align}
 We will first show that $\mathbb{E}_{\mathbb{D}}\brackets{\Verts{s_j(x,y)}_{\mathcal{H}}}$ convergence to $0$ by the dominated convergence theorem for Bochner-integrable functions \citep[Theorem~3,~Chapter~2]{diestel1977vector}. 
	By the reproducing property, note that $\partial_v\ell(\omega,h(x),y)K(x,\cdot) = \partial_v\ell(\omega,\langle h,K(x,\cdot)\rangle_{\mathcal{H}},y)K(x,\cdot)$. Hence, since $(\omega,v)\mapsto\partial_v\ell(\omega,v,y)$ is jointly differentiable in $(\omega,v)$ for any $y$, it follows that $(\omega,h)\mapsto \partial_v\ell(\omega,h(x),y)K(x,\cdot)$ is also differentiable for any $(x,y)$ by composition with the evaluation map $(\omega,h)\mapsto (\omega, \langle h,K(x,\cdot)\rangle_{\mathcal{H}}$ which is differentiable. Hence, the sequence $s_j(x,y)$ converges to $0$ for any $(x,y)\in \mathcal{X}\times \mathcal{Y}$. Moreover, by the mean-value theorem, there exists $0\leq c_j\leq 1$ such that:
\begin{align*}
    \begin{split}
\left\|(\omega_j,h_j)-(\omega,h)\right\|s_j(x,y) =& \partial_v^2\ell\left(\bar{\omega}_j,\bar{h}_j(x),y\right)K(x,\cdot)\otimes K(x,\cdot)\parens{h_j-h}
+ \parens{\omega_j-\omega}^{\top} \partial_{\omega,v}^2\ell(\bar{\omega}_j,\bar{h}_j(x),y)K(x,\cdot)\\
&-\partial_v^2\ell\left(\omega,h(x),y\right)K(x,\cdot)\otimes K(x,\cdot)\parens{h_j-h}
- \parens{\omega_j-\omega}^{\top} \partial_{\omega,v}^2\ell(\omega,h(x),y)K(x,\cdot),
    \end{split}
\end{align*}
where $(\bar{\omega}_j,\bar{h}_j)\coloneqq (1-c_j)(\omega,h) + c_j(\omega_j,h_j)$. Using the same construction as for the Fr\'echet differentiability, we find a compact set $\mathcal{W}$ containing all elements $(\omega_j,h_j(x),y)$ and $(\bar{\omega}_j,\bar{h}_j(x),y)$ for any $(x,y)\in \mathcal{X}\times \mathcal{Y}$ and all $j$ large enough. On such set, $\partial_v^2\ell(\omega,v,y)$ and $\partial_{\omega,v}^2\ell(\omega,v,y)$ are bounded by some constant $C$. Consequently, we can write:
\begin{align*}
    \begin{split}
\left\|(\omega_j,h_j)-(\omega,h)\right\|\Verts{s_j(x,y)}_{\mathcal{H}} \leq & 2C\Verts{K(x,\cdot)\otimes K(x,\cdot)\parens{h_j-h}}_{\mathcal{H}}
+ 2C\Verts{\omega_j-\omega} \Verts{K(x,\cdot)}_{\mathcal{H}}\\
\leq & 2C \kappa\Verts{h_j-h}_{\mathcal{H}} + 2C\sqrt{\kappa}\Verts{\omega_j-\omega}.
    \end{split}
\end{align*}
This already establishes that $s_j(x,y)$ is bounded so that $\mathbb{E}_{\PP}\brackets{\Verts{s_j(x,y)}_{\mathcal{H}}}$ converges to $0$ by application of the dominated convergence theorem. Recalling \cref{eq:expression_sn},  $\mathbb{E}_{\mathbb{\PP}}\brackets{s_j(x,y)}$ admits the following expression:
	\begin{align*}
\begin{split}
\left\|(\omega_j,h_j)-(\omega,h)\right\|\mathbb{E}_{\mathbb{\PP}}\brackets{s_j(x,y)}= &\partial_h L(\omega_j,h_j)-\partial_h L(\omega,h)
-\mathbb{E}_{\mathbb{\PP}}\brackets{\partial_v^2\ell\left(\omega,h(x),y\right)K(x,\cdot)\otimes K(x,\cdot)\parens{h_j-h}}\\
&- \parens{\omega_j-\omega}^{\top}\mathbb{E}_{\mathbb{\PP}}\brackets{ \partial_{\omega,v}^2\ell(\omega,h(x),y)K(x,\cdot)}.
    \end{split}	\end{align*}
	The convergence to $0$ of the above expression precisely means that $L$ is differentiable at $(\omega,h)$ provided that: (1) $\mathbb{E}_{\mathbb{\PP}}\brackets{ \partial_{\omega,v}^2\ell(\omega,h(x),y)K(x,\cdot)}$ is an element in $\mathcal{H}^d$, and (2) the linear map   $g\mapsto\mathbb{E}_{\PP}\brackets{\partial_v^2\ell\left(\omega,h(x),y\right)(K(x,\cdot)\otimes K(x,\cdot))g}$ is bounded. Using the same strategy to establish Bochner's inetgrability of $(x,y)\mapsto \partial_v\ell\left(\omega,h(x),y\right)K(x,\cdot)$, we can show that $(x,y)\mapsto \partial_{\omega,v}^2\ell\left(\omega,h(x),y\right)K(x,\cdot)$ is also Bochner-integrable so that $\mathbb{E}_{\mathbb{\PP}}\brackets{ \partial_{\omega,v}^2\ell(\omega,h(x),y)K(x,\cdot)}$ is indeed an element in $\mathcal{H}^d$. This also establishes the expression of $\partial_{\omega,h}L(\omega,h)$. Similarly, we consider the operator-valued function $\xi:(x,y)\mapsto \partial_{v}^2\ell\left(\omega,h(x),y\right)K(x,\cdot)\otimes K(x,\cdot)$ with values in the space of Hilbert-Schmidt operators on $\mathcal{H}$. The Hilbert-Schmidt (HS) norm of such function satisfies the following inequality:
	\begin{align*}
		\mathbb{E}_{\mathbb{\PP}}\brackets{ \Verts{\partial_{v}^2\ell(\omega,h(x),y)K(x,\cdot)\otimes K(x,\cdot)}_{\hs}}\coloneqq\mathbb{E}_{\mathbb{\PP}}\brackets{ \verts{\partial_{v}^2\ell(\omega,h(x),y)}K(x,x)}\leq \kappa C<+\infty.
	\end{align*}
	Therefore, the function $\xi$ is Bochner-integrable, so that $\mathbb{E}_{\mathbb{\PP}}\brackets{ \partial_{v}^2\ell(\omega,h(x),y)K(x,\cdot)\otimes K(x,\cdot)}$ is a Hilbert-Schmidt operator satisfying: 
\begin{align*}
	\mathbb{E}_{\mathbb{\PP}}\brackets{ \partial_{v}^2\ell(\omega,h(x),y)K(x,\cdot)\otimes K(x,\cdot)}  g=\mathbb{E}_{\PP}\brackets{\partial_{v}^2\ell\left(\omega,h(x),y\right)(K(x,\cdot)\otimes K(x,\cdot))g},\forall g\in \mathcal{H}.
\end{align*}
The above property follows from \citep[Theorem~6,~Chapter~2]{diestel1977vector} for Bochner-integrable functions that allows exchanging the integral and the application of a continuous linear map (here the scalar product with an element $g$). Hence, from the above identity we deduce the desired expression for $\partial_{h}^2 L(\omega,h)$. 
\end{proof}
\section{Auxiliary Technical Lemmas}\label{ab_sec:aux}
\begin{lemma}\label{lem:tech_res_1}
Let $A$ and $A'$ be two bounded operators from $\mathcal{H}$ to $\mathbb{R}^d$, and $B$ and $B'$ be two bounded and invertible operators from $\mathcal{H}$ to itself. Assume that $B\geq \lambda\Id_{\mathcal{H}}$ and $B'\geq \lambda\Id_{\mathcal{H}}$. 
Then, the following inequalities hold:
\begin{gather*}
    \Verts{AB^{-1}-A'(B')^{-1}}_{\op} \leq \frac{\Verts{A}_{\op}}{\lambda^2}\Verts{B-B'}_{\op} + \frac{1}{\lambda}\Verts{A-A'}_{\op},\\
    \Verts{AB^{-1}}_{\op}\leq \lambda^{-1}\Verts{A}_{\op}, \qquad\Verts{A'(B')^{-1}}_{\op}\leq \lambda^{-1}\Verts{A'}_{\op}.
\end{gather*}
\end{lemma}

\begin{proof}
By the triangle inequality and the sub-multiplicative property of the operator norm $\|\cdot\|_{\op}$, we have:
\begin{align}
    \Verts{AB^{-1}-A'(B')^{-1}}_{\op}&\leq\left\|AB^{-1} - A(B')^{-1}\right\|_{\op}+\left\|A(B')^{-1} - A'(B')^{-1}\right\|_{\op}\nonumber\\
    &\leq\left\|A\left(B^{-1} - (B')^{-1}\right)\right\|_{\op}+\left\|\left(A - A'\right)(B')^{-1}\right\|_{\op}\nonumber\\
    &\leq\left\|A\right\|_{\op}\left\|B^{-1} - (B')^{-1}\right\|_{\op}+\left\|A-A'\right\|_{\op}\left\|(B')^{-1}\right\|_{\op}\nonumber\\
    &\leq\left\|A\right\|_{\op}\left\|B^{-1}\left(B'-B\right) (B')^{-1}\right\|_{\op}+\left\|A-A'\right\|_{\op}\left\|(B')^{-1}\right\|_{\op}\nonumber\\
    &\leq\left\|A\right\|_{\op}\left\|B^{-1}\right\|_{\op}\left\|B'-B\right\|_{\op}\left\|(B')^{-1}\right\|_{\op}+\left\|A-A'\right\|_{\op}\left\|(B')^{-1}\right\|_{\op}.
    \label{eq:technical_proof_1}
\end{align}
Since $B\geq \lambda\Id_{\mathcal{H}}$ and $B'\geq \lambda\Id_{\mathcal{H}}$, we obtain:
\begin{equation*}
    \left\|B^{-1}\right\|_{\op}\leq\frac{1}{\lambda}\quad\text{and}\quad\left\|(B')^{-1}\right\|_{\op}\leq\frac{1}{\lambda}.
\end{equation*}
Substituting these into \cref{eq:technical_proof_1}, we get:
\begin{align*}
    \Verts{AB^{-1}-A'(B')^{-1}}_{\op}\leq\frac{\Verts{A}_{\op}}{\lambda^2}\Verts{B-B'}_{\op} + \frac{1}{\lambda}\Verts{A-A'}_{\op}.
\end{align*}
This proves the first inequality. The remaining two inequalities follow directly from the sub-multiplicative property of the operator norm $\|\cdot\|_{\op}$ and the assumptions $B\geq \lambda\Id_{\mathcal{H}}$ and $B'\geq \lambda\Id_{\mathcal{H}}$.
\end{proof}


\begin{lemma}\label{lem:h_min_hstar}
Let $f:\mathcal{H}\to\mathbb{R}$ be a $\lambda$-strongly convex and Fr\'echet differentiable function. Denote by $h^\star\in\mathcal{H}$ its minimizer. Then, for any $h\in\mathcal{H}$, the following holds:
\begin{equation*}
    \left\|h-h^\star\right\|_\mathcal{H}\leq\frac{1}{\lambda}\left\|\partial_h f(h)\right\|_\mathcal{H}.
\end{equation*}
\end{lemma}

\begin{proof}
Let $h\in\mathcal{H}$. 

\textbf{Case 1: $h=h^\star$. }The proof is straightforward.

\textbf{Case 2: $h\neq h^\star$. }Given that $f$ is $\lambda$-strongly convex, we have:
\begin{align*}
    f(h)-f(h^\star)&\geq\left\langle\partial_h f(h^\star),h-h^\star\right\rangle_\mathcal{H}+\frac{\lambda}{2}\left\|h-h^\star\right\|_\mathcal{H}^2,\\
    \text{and }f(h^\star)-f(h)&\geq\left\langle\partial_h f(h),h^\star-h\right\rangle_\mathcal{H}+\frac{\lambda}{2}\left\|h-h^\star\right\|_\mathcal{H}^2.
\end{align*}
After summing these two inequalities, noticing that $\partial_h f(h^\star)=0$, and rearranging the terms, we obtain:
\begin{equation*}
    \left\langle\partial_h f(h),h-h^\star\right\rangle_\mathcal{H}\geq\lambda\left\|h-h^\star\right\|_\mathcal{H}^2.
\end{equation*}
After using the Cauchy-Schwarz inequality, we get:
\begin{equation*}
    \left\|\partial_h f(h)\right\|_\mathcal{H}\left\|h-h^\star\right\|_\mathcal{H}\geq\lambda\left\|h-h^\star\right\|_\mathcal{H}^2.
\end{equation*}
Dividing by $\lambda\left\|h-h^\star\right\|_\mathcal{H}\neq0$ concludes the proof.
\end{proof}


\begin{lemma}\label{lem:hs_identity}
{
Let $\mathcal{X}$ be a subset of $\mathbb{R}^p$, $\mathcal{Y}$ be a subset of $\mathbb{R}^q$, and $\mathbb{D}$ be a probability distribution over $\mathcal{X}\times\mathcal{Y}$. Given i.i.d. samples $(x_i,y_i)_{1\leq i\leq n}$ drawn from $\mathbb{D}$, consider a function $g:\mathcal{X}\times\mathcal{Y}\to\mathbb{R}$ of class $C^1$ such that the operator $A:\mathcal{H}\to\mathcal{H}$ defined as:
\begin{equation*}
    A\coloneqq\mathbb{E}_{(x,y)\sim\mathbb{D}}\brackets{g(x,y)K(x,\cdot)\otimes K(x,\cdot)}-\frac{1}{n}\sum_{i=1}^n g(x_i,y_i)K(x_i,\cdot)\otimes K(x_i,\cdot)
\end{equation*}
is Hilbert-Schmidt. Then, the following holds:
\begin{align*}
    \Verts{A}_{\hs}^2=&\mathbb{E}_{(x,y),(x',y')\sim\mathbb{D}\otimes\mathbb{D}}\Big[g(x,y)g(x',y')K^2(x,x')\Big]+\frac{1}{n^2}\sum_{i,j=1}^n g(x_i, y_i)g(x_j, y_j)K^2(x_i,x_j)\\
    &-\frac{2}{n}\sum_{i=1}^n\mathbb{E}_{(x,y)\sim\mathbb{D}}\Big[g(x_i, y_i)g(x, y)K^2(x,x_i)\Big].
\end{align*}
}
\end{lemma}

\begin{proof}
{
Define $s\coloneqq\mathbb{E}_{(x,y)\sim\mathbb{D}}\left[g(x,y)K(x,\cdot)\otimes K(x,\cdot)\right]$ and $\hat{s}\coloneqq\frac{1}{n}\sum_{i=1}^n g(x_i,y_i)K(x_i,\cdot)\otimes K(x_i,\cdot)$. We have:
\begin{equation}\label{eq:A}
    \Verts{A}_{\hs}^2=\Verts{s-\hat{s}}_{\hs}^2=\Verts{s}_{\hs}^2+\Verts{\hat{s}}_{\hs}^2-2\left\langle s,\hat{s}\right\rangle_{\hs}.
\end{equation}
Next, we compute each of the following quantities: $\Verts{s}_{\hs}^2$, $\Verts{\hat{s}}_{\hs}^2$, and $\left\langle s,\hat{s}\right\rangle_{\hs}$, separately. Simple calculations yield:
\begin{align*}
    \Verts{s}_{\hs}^2&=\left\|\mathbb{E}_{(x,y)\sim\mathbb{D}}\left[g(x,y)K(x,\cdot)\otimes K(x,\cdot)\right]\right\|_{\hs}^2\\
    &=\left\langle\mathbb{E}_{(x,y)\sim\mathbb{D}}\left[g(x,y)K(x,\cdot)\otimes K(x,\cdot)\right],\mathbb{E}_{(x',y')\sim\mathbb{D}}\left[g(x',y')K(x',\cdot)\otimes K(x',\cdot)\right]\right\rangle_{\hs}\\
    &=\mathbb{E}_{(x,y),(x',y')\sim\mathbb{D}\otimes\mathbb{D}}\bigg[g(x,y)g(x',y')\left\langle K(x,\cdot)\otimes K(x,\cdot),K(x',\cdot)\otimes K(x',\cdot)\right\rangle_{\hs}\bigg]\\
    &=\mathbb{E}_{(x,y),(x',y')\sim\mathbb{D}\otimes\mathbb{D}}\Big[g(x,y)g(x',y')K^2(x,x')\Big],\\
    \Verts{\hat{s}}_{\hs}^2&=\frac{1}{n^2}\left\|\sum_{i=1}^n g(x_i, y_i)K(x_i,\cdot)\otimes K(x_i,\cdot)\right\|_{\hs}^2\\
    &=\frac{1}{n^2}\left\langle\sum_{i=1}^n g(x_i, y_i)K(x_i,\cdot)\otimes K(x_i,\cdot),\sum_{j=1}^n g(x_j, y_j)K(x_j,\cdot)\otimes K(x_j,\cdot)\right\rangle_{\hs}\\
    &=\frac{1}{n^2}\sum_{i,j=1}^n\partial_v^2 g(x_i, y_i)g(x_j, y_j)\left\langle K(x_i,\cdot)\otimes K(x_i,\cdot),K(x_j,\cdot)\otimes K(x_j,\cdot)\right\rangle_{\hs}\\
    &=\frac{1}{n^2}\sum_{i,j=1}^n g(x_i, y_i)g(x_j, y_j)K^2(x_i,x_j),\\
    \left\langle s,\hat{s}\right\rangle_{\hs}&=\left\langle\mathbb{E}_{(x,y)\sim\mathbb{D}}\left[g(x, y)K(x,\cdot)\otimes K(x,\cdot)\right],\frac{1}{n}\sum_{i=1}^n g(x_i, y_i)K(x_i,\cdot)\otimes K(x_i,\cdot)\right\rangle_{\hs}\\
    &=\frac{1}{n}\sum_{i=1}^n\mathbb{E}_{(x,y)\sim\mathbb{D}}\left[g(x_i, y_i)g(x, y)\left\langle K(x,\cdot)\otimes K(x,\cdot),K(x_i,\cdot)\otimes K(x_i,\cdot)\right\rangle_{\hs}\right]\\
    &=\frac{1}{n}\sum_{i=1}^n\mathbb{E}_{(x,y)\sim\mathbb{D}}\Big[g(x_i, y_i)g(x, y)K^2(x,x_i)\Big].
\end{align*}
After substituting the obtained results into \cref{eq:A} and rearranging, we obtain:
\begin{align*}
    \left\|A\right\|_{\hs}^2=&\mathbb{E}_{(x,y),(x',y')\sim\mathbb{D}\otimes\mathbb{D}}\Big[g(x,y)g(x',y')K^2(x,x')\Big]+\frac{1}{n^2}\sum_{i,j=1}^n g(x_i, y_i)g(x_j, y_j)K^2(x_i,x_j)\\
    &-\frac{2}{n}\sum_{i=1}^n\mathbb{E}_{(x,y)\sim\mathbb{D}}\Big[g(x_i, y_i)g(x, y)K^2(x,x_i)\Big].
\end{align*}
}
\end{proof}

%
%


\end{document}


%
%
%
%
%
%
%
%
%
%
%
%
%