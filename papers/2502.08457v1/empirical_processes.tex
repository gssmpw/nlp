\section{Maximal Inequalities for Bounded and Lipschitz Family of Functions}\label{sec_app:max_in_bound_lip}
Let  $\mathcal{Z}$ be a subset of a Euclidean space and $\Omega$ be a compact subset of $\mathbb{R}^d$. Denote by $\otimes^{k} \mathcal{Z}$ the $k$-th tensor power of $\mathcal{Z}$, for any $k\geq 1$. Consider a parametric family $\mathcal{T}$ of real-valued functions defined over $\mathcal{Z}$ and indexed by a parameter $\omega\in \Omega$, \textit{i.e.}, 
\begin{align}\label{eq:generic_parametric_family}
	\mathcal{T}\coloneqq \braces{\mathcal{Z}\ni z\mapsto t_{\omega}(z)\in \mathbb{R}\mid\omega \in \Omega}.
\end{align}
For a given probability measure $\mu$, denote by $L_2(\mu)$ the space of square $\mu$-integrable real-valued functions. We denote by $\Verts{f}_{\PP,2}\coloneqq  \mathbb{E}_{\PP}\brackets{f(z)^2}^{\frac{1}{2}}$ the $L_2(\mu)$-norm of any function $f\in L_2(\mu)$.  For any $\epsilon>0$, we denote by  $D\left(\epsilon,\mathcal{T},L_2(\mu)\right)$ the $\epsilon$-packing number of $\mathcal{T}$ w.r.t. $L_2(\mu)$. The next proposition provides a control on such a number under regularity conditions on the family $\mathcal{T}$. 
\begin{proposition}[Control on the packing number]\label{prop:estimate_covering_numbers}
	Assume that $\Omega$ is a compact subset of $\mathbb{R}^d$, that the parametric family $\mathcal{T}$ defined in \cref{eq:generic_parametric_family} is uniformly bounded by a positive constant $M$, and that there exists a positive constant $L$ so that, for any measure $\mu$,	$\omega \mapsto t_{\omega}(z)$ is $L$-Lipschitz for any $z\in \mathcal{Z}$. Then, there exists a positive constant $c(\Omega)$ greater than $1$ that depends only on $\Omega$ and $d$ so that, for any probability measure $\mu$ on $\mathcal{Z}$, the following bound holds for any $0<\epsilon\leq M$:
\begin{align*}  
 D\left(\epsilon,\mathcal{T},L_2(\mu)\right)\leq c(\Omega)\parens{\frac{\max\parens{L\diam(\Omega),M}}{\epsilon}}^d.
\end{align*}
\end{proposition}
\begin{proof}
	First using \citep[Lemma~9.18]{Kosorok2008} and \citep[Paragraph~8.1.2]{Kosorok2008}, we know that the $\epsilon$-packing number $D\left(\epsilon,\mathcal{T},L_2(\mu)\right)$ is  smaller than the $\frac{\epsilon}{2}$-bracketing number $N_{[]}\left(\frac{\epsilon}{2},\mathcal{T},L_2(\mu)\right)$. Hence, we only need to control the bracketing number. To this end, we recall that the function $\omega\mapsto t_{\omega}(z)$ is $L$-Lipschitz for any $z\in \mathcal{Z}$, so that  \citep[Example~19.7]{van2000asymptotic} ensures the existence of a positive constant $c(\Omega)$ that depends only on $\Omega$ for which the following inequality holds for any $0<\epsilon< L\text{diam}(\Omega)$:
\begin{equation*}
    1\leq N_{[]}\left(\epsilon,\mathcal{T},L_2(\mu)\right)\leq c(\Omega)\left(\frac{L\diam(\Omega)}{\epsilon}\right)^d.
\end{equation*}
Moreover, since the $\epsilon$-bracketing number is decreasing in $\epsilon$, it holds that:
$$N_{[]}\left(\epsilon,\mathcal{T},L_2(\mu)\right)\leq N_{[]}\left(\epsilon_{-},\mathcal{T},L_2(\mu)\right) \leq c(\Omega)\parens{\frac{L\diam(\Omega)}{\epsilon_{-}}}^{d},$$ 
for any $\epsilon\geq L\text{diam}(\Omega)$ and $\epsilon_{-}\leq L\diam(\Omega)$. Taking the limit when $\epsilon_{-}$ approaches $L\text{diam}(\Omega)$ yields $N_{[]}\left(\epsilon,\mathcal{T},L_2(\mu)\right)\leq c(\Omega)$ for any $\epsilon\geq L\diam(\Omega)$. Hence, we have shown so far that for any $\epsilon>0$:
\begin{align*}
	N_{[]}\left(\epsilon,\mathcal{T},L_2(\mu)\right)\leq c(\Omega)\max\parens{1,\left(\frac{L\diam(\Omega)}{\epsilon}\right)^d}.
\end{align*}
Moreover, by noticing that $\max(1,\frac{L\diam(\Omega)}{\epsilon})\leq \frac{\max(M,L\diam(\Omega))}{\epsilon}$ for any $\epsilon\leq M$, we further have that:
\begin{align*}
	N_{[]}\left(\epsilon,\mathcal{T},L_2(\mu)\right)\leq  c(\Omega)\parens{\frac{\max\parens{M,L\diam(\Omega)}}{\epsilon}}^d.
\end{align*}
Finally, recalling that $ D\left(\epsilon,\mathcal{T},L_2(\mu)\right)\leq  N_{[]}\left(\frac{\epsilon}{2},\mathcal{T},L_2(\mu)\right)$, we get that $D\left(\epsilon,\mathcal{T},L_2(\mu)\right)\leq  2^dc(\Omega)\parens{\frac{\max\parens{M,L\diam(\Omega)}}{\epsilon}}^d$. 
The desired bound follows after redefining $c(\Omega)$ to include the factor $2^d$ (\textit{i.e.}, $c(\Omega)\rightarrow 2^dc(\Omega)$).
\end{proof}

\begin{theorem}[Maximal inequality for degenerate bounded and Lipschitz $U$-processes]\label{prop:maximal_ineq_degenerate_u_process}
	Let $k$ be either $1$ or $2$. 
Consider a parametric family 
	$\mathcal{T}\coloneqq \braces{\otimes^{k} \mathcal{Z}\ni (z_1,\ldots,z_k)\mapsto t_{\omega}(z_1,\ldots,z_k)\in \mathbb{R}\mid\omega \in \Omega}$ 
	of real-valued functions over $\otimes^{k} \mathcal{Z}$ and indexed by a parameter $\omega\in \Omega$, where $\Omega$ is a compact subset of $\mathbb{R}^d$. 
	For a given probability distribution $\PP$ over $\mathcal{Z}$, assume that all elements $t_{\omega}$ are degenerate w.r.t. $\PP$, meaning that:
	\begin{align*}
		\begin{cases}
		\mathbb{E}_{\bar{z}\sim \PP}\brackets{ t_{\omega}(\bar{z})} = 0, &\qquad \text{if} \quad k=1\\ 
			\mathbb{E}_{\bar{z}\sim \PP}\brackets{ t_{\omega}(z,\bar{z})} =\mathbb{E}_{\bar{z}\sim \PP}\brackets{ t_{\omega}(\bar{z},z)}=0,\quad \forall z\in \mathcal{Z}, &\qquad \text{if} \quad k=2. 
		\end{cases}
	\end{align*}
	Furthermore, assume that all functions in $\mathcal{T}$ are uniformly bounded by a positive constant $M$ and that there exists a positive constant $L$ so that $\omega \mapsto t_{\omega}(z_1,\ldots,z_k)$ is $L$-Lipschitz for any $(z_1,\ldots,z_k)\in \otimes^k \mathcal{Z}$.  
	 Given i.i.d. samples $(z_i)_{1\leq i\leq n}$ from $\PP$, consider the following $U$-statistic $U_n^k$:
\begin{align*}
    U_n^k t_\omega\coloneqq 
    \begin{cases}
    	    	\frac{1}{n}\sum_{i=1}^n t_{\omega}(z_i),&\qquad \text{if} \quad k=1\\
    	\frac{1}{n(n-1)}\sum_{\substack{i,j=1\\ i\neq j}}^n t_{\omega}(z_i,z_j),&\qquad \text{if} \quad k=2.
    \end{cases}
\end{align*}
Then, there exists a universal positive constant $c(\Omega)$ greater than $1$ that depends only on $\Omega$ and $d$ such that for any $p\in \braces{1,2}$:
\begin{align*}
	\mathbb{E}_{\PP}\brackets{\sup_{\omega\in \Omega} \verts{U^k_n t_{\omega}}^p}^{\frac{1}{p}}\leq n^{-{\frac{k}{2}}}  c(\Omega)\max\parens{ML\diam(\Omega),M^2}^{\frac{1}{2}}.
\end{align*}


\end{theorem}
\begin{proof}
	{\bf Maximal inequality for degenerate $U$-processes.}
We  will first apply the general result in \citep[Maximal inequality]{sherman1994maximal} which controls $\mathbb{E}_{\PP}\brackets{\sup_{\omega\in \Omega} \verts{U_n^k t_{\omega}}}$ in terms of the packing number of $\mathcal{T}$.
First note, by assumption, that the functions $t_{\omega}(z_1,\ldots,z_k)$  are  uniformly bounded by a positive constant $M$. Therefore, the constant function $T(z_1,\ldots,z_k) \coloneqq M$ is an envelope for $\mathcal{T}$, \textit{i.e.}, $T$ satisfies $T(z_1,\ldots,z_k)\geq \sup_{\omega\in \Omega} \verts{t_{\omega}(z_1,\ldots,z_k)}$ for any $(z_1,\ldots,z_k)\in\otimes^k \mathcal{Z}$. 
The envelope $T$ is, a fortiori,  square $\mu$-integrable for any probability measure $\mu$ on $\otimes^k \mathcal{Z}$. Hence, we can apply \citep[Maximal inequality]{sherman1994maximal} with the choice $T$ for the envelope function and set the integer $m$ appearing in the result to $m=d$ to get the following bound:
\begin{align}\label{eq:maximal_sherman}
	\mathbb{E}_{\PP}\brackets{\sup_{\omega\in \Omega} \verts{U^k_n t_{\omega}}^p}^{\frac{1}{p}}\leq n^{-\frac{k}{2}} \Gamma 
	\mathbb{E}\brackets{\Verts{T}_{\mu_n,2} \int_{0}^{\delta_n} \parens{D\parens{\epsilon \Verts{T}_{\mu_n,2},\mathcal{T},L_2(\mu_n) }}^{\frac{1}{2dp}}\diff \epsilon  }, 
\end{align}
where $\Gamma$ is a positive universal constant\footnote{The constant $\Gamma$ appearing \citep[Maximal inequality]{sherman1994maximal} depends only on $k$, $p$ and $m$, \textit{i.e.}, $\Gamma\coloneqq g(k,p,m)$. Since, we are only interested in $k\leq 2$ and $p\leq 2$ and $m$ is fixed to $d$, we choose $\Gamma$ to be $\max_{1\leq k,p\leq 2}g(k,p,d)^{\frac{1}{p}}$, so that it is the same in all our cases.} that depends only on $d$ and that we choose to be greater than $1$,  while $\mu_{n}$ are suitably chosen  probability measures on $\otimes^k\mathcal{Z}$ that possibly depend on the samples $z_1,\ldots,z_n$ and other random variables, and $\delta_n \Verts{T}_{\mu_n,2} \coloneqq \sup_{\omega\in \Omega} \Verts{t_{\omega}}_{\mu_n,2}$. Here, the expectation symbol in the right-hand side is over all randomness on which $\mu_n$ might depend. Note that the original result in \citep[Maximal inequality]{sherman1994maximal} is stated using a slightly different definition of the packing number but which is still equivalent to the statement above in our setting\footnote{In \citep[Maximal inequality]{sherman1994maximal}, the author considers a modified version of the $\epsilon$-packing number (call it $\tilde{D}(\epsilon, \mathcal{T},L_{2}(\mu))$) associated to $L_2(\mu)$ but endowed with a normalized version of the standard norm on  $L_2(\mu)$: $\Verts{f}_{\mu}\coloneqq\frac{\Verts{f}_{\mu,2}}{\Verts{T}_{\mu,2}}$.  Both numbers are related by the following identity:  $\tilde{D}\parens{\epsilon, \mathcal{T},L_{2}(\mu)} = D(\epsilon{\Verts{T}_{\mu,2}}, \mathcal{T},L_{2}(\mu))$, thus making the statement \eqref{eq:maximal_sherman} equivalent to the original statement in \citep[Maximal inequality]{sherman1994maximal}.
}. 


In our setting, the envelope function is constant and equal to $M$, and by definition $\delta_n \leq 1$. Hence, the inequality in  \cref{eq:maximal_sherman} further becomes:
\begin{align}\label{eq:maximal_inequality_sherman}
	\mathbb{E}_{\PP}\brackets{\sup_{\omega\in \Omega} \verts{U_n^k t_{\omega}}^p}^{\frac{1}{p}}\leq n^{-\frac{k}{2}} M\Gamma 
	\mathbb{E}_{\PP}\brackets{ \int_{0}^{1} \parens{D\parens{\epsilon\Verts{T}_{\mu_n,2},\mathcal{T},L_2(\mu_n) }}^{\frac{1}{2dp}}\diff \epsilon  }. 
\end{align}
 We simply need to control the packing number $D\parens{\epsilon\Verts{T}_{\mu,2},\mathcal{T},L_2(\mu)}$ independently of the probability measure $\mu$.  

{\bf Control on the packing number.} 
We have shown that the constant function $T(z_1,\ldots,z_k) \coloneqq M$ is an envelope for $\mathcal{T}$ which is, a fortiori, square $\mu$-integrable for any probability measure $\mu$ with $\Verts{T}_{\mu,2}= M<+\infty$. 
 Moreover, the functions $\omega\mapsto t_{\omega}(z_1,\ldots,z_k)$ are $L$-Lipschitz for any $(z_1,\ldots,z_k)\in \otimes^k\mathcal{Z}$. 
 We can therefore apply \cref{prop:estimate_covering_numbers} which ensures the existence of a positive constant $c(\Omega)$ greater than $1$ and that depends only on $\Omega$ and $d$ so that the following estimate on the $\epsilon$-packing number of the class $\mathcal{T}$ w.r.t. $L_2(\mu)$ holds:
\begin{align}\label{eq:bound_packing_uniform}
D\left(\epsilon\Verts{T}_{\mu,2},\mathcal{T},L_2(\mu)\right)\leq \underbrace{c(\Omega)\parens{\max\parens{\frac{L\diam(\Omega)}{M},1}}^d}_{A}\parens{\frac{1}{\epsilon}}^d, \qquad \forall\epsilon\in(0,  1].
\end{align}
Combining \cref{eq:bound_packing_uniform} with \cref{eq:maximal_inequality_sherman} yields:
\begin{align*}
	\mathbb{E}_{\PP}\brackets{\sup_{\omega\in \Omega} \verts{U_n^k t_{\omega}}^p}^{\frac{1}{p}}&\leq n^{-\frac{k}{2}} M\Gamma 
	\mathbb{E}_{\PP}\brackets{ \int_{0}^{1} \parens{A\epsilon^{-d}
	}^{\frac{1}{2dp}}\diff \epsilon} = n^{-\frac{k}{2}}M\Gamma A^{\frac{1}{2dp}} \underbrace{\int_0^1 \epsilon^{-\frac{1}{2p}}\diff \epsilon}_{\leq 2} \\
	&\leq  
	 2n^{-\frac{k}{2}}\Gamma c(\Omega)^{\frac{1}{2d}}\max\parens{L\diam(\Omega),M^2}^{\frac{1}{2}},
\end{align*}
where, for the last inequality, we used that $A^{\frac{1}{2dp}}\leq A^{\frac{1}{2d}} = c(\Omega)^{\frac{1}{2d}}\max\parens{\frac{L\diam(\Omega)}{M},1}^{\frac{1}{2}}$ since $A$ is greater than $1$. 
The desired result follows after redefining $c(\Omega)$ as $2\Gamma c(\Omega)^{\frac{1}{2d}}$ which is  a positive constant that depends only on $\Omega$ and $d$. 
\end{proof}

The following next propositions are particular instances of \cref{prop:maximal_ineq_degenerate_u_process} and will be used to obtain the main bounds.  
\begin{proposition}[Maximal inequality for empirical processes]\label{prop:empirical_process}
	 
	Consider a parametric family 
	$\mathcal{T}\coloneqq \braces{\mathcal{Z}\ni z\mapsto t_{\omega}(z)\in \mathbb{R}\mid\omega \in \Omega}$ 
	of real-valued functions defined over a subset $\mathcal{Z}$ of a Euclidean space and indexed by a parameter $\omega\in \Omega$, where $\Omega$ is a compact subset of $\mathbb{R}^d$. 
	Assume that all functions in $\mathcal{T}$ are uniformly bounded by a positive constant $M$ and that there exists a positive constant $L$ so that $\omega \mapsto t_{\omega}(z)$ is $L$-Lipschitz for any $z\in \mathcal{Z}$. Consider a probability distribution $\PP$ over $\mathcal{Z}$ and let $(z_i)_{1\leq i\leq n}$ be i.i.d. samples drawn from $\PP$, then there exists a positive constant $c(\Omega)$ greater than $1$ that depends only on $\Omega$ and $d$, such that for any integer $p \in \{1,2\}$:
	\begin{align*}
		\mathbb{E}_{\PP}\brackets{ \sup_{\omega\in \Omega} \verts{ \mathbb{E}_{z\sim \PP}\brackets{t_{\omega}(z) } - \frac{1}{n}\sum_{i=1}^n t_{\omega}(z_i) }^p }^{\frac{
		1}{p}}\leq \sqrt{\frac{1}{n}}c(\Omega)\max(ML\diam(\Omega),M^2)^{\frac{1}{2}}.
	\end{align*}
\end{proposition}
\begin{proof}
The upper-bound is a direct consequence of \cref{prop:maximal_ineq_degenerate_u_process}. Indeed consider the  family $\mathcal{S}$ of functions of the form $s_{\omega}(z) = t_{\omega}(z)-\mathbb{E}_{\bar{z}\sim\PP}\brackets{t_{\omega}(\bar{z})}$, for any $z\in\mathcal{Z}$. Then clearly, the process $U_n^1 s_{\omega}\coloneqq\frac{1}{n}\sum_{i=1}^n s_{\omega}(z_i)$ is degenerate of order $k=1$, and the family $\mathcal{S}$ is uniformly bounded by $2M$ and is $2L$-Lipschitz. Hence, by \cref{prop:maximal_ineq_degenerate_u_process}, the following maximal inequality holds:
\begin{align*}
	\mathbb{E}_{\PP}\brackets{\sup_{\omega\in \Omega} \verts{U^1_n s_{\omega}}^p}^{\frac{1}{p}}\leq 2n^{-{\frac{1}{2}}}  c(\Omega)\max\parens{ML\diam(\Omega),M^2}^{\frac{1}{2}}. 
\end{align*}
We get the desired upper-bound by redefining $c(\Omega)$ to contain the factor $2$.
\end{proof}


\begin{proposition}[Maximal inequality for $U$-processes of order 2]\label{lem:u_stat_sup_p_omega}
Consider a parametric family 
	$\mathcal{T}\coloneqq\braces{\mathcal{Z}\times \mathcal{Z}\ni (z,z')\mapsto t_{\omega}(z,z')\in \mathbb{R}\mid\omega \in \Omega}$ 
	of real-valued functions indexed by a parameter $\omega\in \Omega$, where $\Omega$ is a compact subset of $\mathbb{R}^d$ and $\mathcal{Z}$ is a subset of a Euclidean space. Assume that the functions in $\mathcal{T}$ are symmetric in their arguments, \textit{i.e.}, $t_{\omega}(z,z') = t_{\omega}(z',z)$. Additionally, assume that all functions in $\mathcal{T}$ are uniformly bounded by a positive constant $M$ and that there exists a positive constant $L$ so that $\omega \mapsto t_{\omega}(z,z')$ is $L$-Lipschitz for any $(z,z')\in \mathcal{Z}\times\mathcal{Z}$. 
	Consider a probability distribution $\PP$ over $\mathcal{Z}$ and let $(z_i)_{1\leq i\leq n}$ be i.i.d. samples drawn from $\PP$, and define the following statistic:
\begin{align*}
    \tau_\omega\coloneqq\mathbb{E}_{z,z'\sim\PP\otimes\PP}\left[t_\omega(z,z')\right]+\frac{1}{n^2}\sum_{i,j=1}^n t_\omega(z_i,z_j)-\frac{2}{n}\sum_{i=1}^n\mathbb{E}_{z\sim\PP}\left[t_\omega(z,z_i)\right].
\end{align*}
Then there exists a universal positive constant $c(\Omega)$ greater than $1$ that depends only on $\Omega$ and $d$ such that:
\begin{align*}
	\mathbb{E}_{\PP}\brackets{\sup_{\omega\in \Omega} \verts{\tau_{\omega}}}\leq \frac{1}{n}  c(\Omega)\max\parens{ML\diam(\Omega),M^2}^{\frac{1}{2}}. 
\end{align*}
\end{proposition}

\begin{proof}
The proof will proceed by first decomposing $\tau_{\omega}$ into  a sum of a degenerate $U$-process and a term of order $O(\frac{1}{n})$. The maximal inequality for degenerate $U$-processes from \cite{sherman1994maximal} will be employed to obtain the desired bound.

{\bf Decomposition of $\tau_{\omega}$.} 
Consider the following function  defined over $\mathcal{Z}\times \mathcal{Z}$ and indexed by elements $\omega\in \Omega$:
\begin{equation}\label{eq:def_degenerate_kernel}
    s_\omega(z,z')=t_\omega(z,z')-\mathbb{E}_{\bar{z}\sim\PP}\left[t_\omega(z,\bar{z})\right]-\mathbb{E}_{\bar{z}\sim\PP}\left[t_\omega(\bar{z},z')\right]+\mathbb{E}_{\{\bar{z},\underline{z}\}\sim\PP\otimes\PP}\left[t_\omega(\bar{z},\underline{z})\right]. 
\end{equation}
By direct calculation, we decompose $\tau_{\omega}$ into two higher order terms and a third  term, $U_n^2 s_{\omega}$, involving $s_{\omega}$, which happens to be a $U$-statistic: 
\begin{equation*}
    \tau_\omega= \overbrace{\frac{1}{(n-1)}\sum_{\substack{i,j=1\\i\neq j}}^n s_\omega\big(z_i,z_j\big)} ^{U_n^2 s_\omega}-\frac{1}{n^2(n-1)} \sum_{\substack{i,j=1\\i\neq j}}^n t_\omega(z_i,z_j)+\frac{1}{n^2}\sum_{i=1}^n t_\omega(z_i,z_i).
\end{equation*}
Using the triangle inequality in the above equality and recalling that, by assumption, $t_{\omega}(z,z')$ is uniformly bounded by a positive constant $M$, it follows that:
\begin{align*}
    \verts{\tau_\omega}\leq\left|U_{n}^2 s_\omega\right|+\frac{1}{n^2(n-1)}\sum_{\substack{i,j=1\\i\neq j}}^{n} \left|t_\omega(z_i,z_j)\right|+\frac{1}{n^2}\sum_{i=1}^{n}\left|t_\omega(z_i,z_i)\right|
    \leq 
     \verts{U_n^2 s_{\omega}} + \frac{2M}{n}.
\end{align*}
Furthermore, taking the supremum over $\omega$ followed by the expectation over samples yields: 
\begin{align}\label{eq:main_decomposition_u_stat}
	\mathbb{E}_{\PP}\brackets{\sup_{\omega\in \Omega} \verts{\tau_{\omega}}}\leq \mathbb{E}_{\PP}\brackets{\sup_{\omega\in \Omega} \verts{U_n^2 s_{\omega}}} + \frac{2M}{n}.
\end{align}
Hence, it only remains to control the first term in the above inequality. To this end, we will use a maximal inequality for degenerate $U$-processes due to \cite{sherman1994maximal}. 

{\bf Maximal inequality for degenerate $U$-processes.} We will first check that $U_n^2 s_{\omega}$ is a degenerate statistic for a given $\omega\in \Omega$. To this end, simple calculations show that for any $z$ in $\mathcal{Z}$:
\begin{align*}
	\mathbb{E}_{\bar{z}\sim \PP}\brackets{s_{\omega}(z,\bar{z})} =\mathbb{E}_{\bar{z}\sim \PP}\brackets{s_{\omega}(\bar{z},z)} = 0. 
\end{align*}
The above equalities precisely ensure that $U_n^2 s_{\omega}$ is a degenerate $U$-statistic for $\PP$. 
Consider now the family $\mathcal{S}\coloneqq \braces{\mathcal{Z}\times \mathcal{Z}\ni(z,z')\mapsto s_{\omega}(z,z')\in \mathbb{R}\mid\omega\in\Omega}$. We show that $\mathcal{S}$ is uniformly bounded and Lipschitz which allows to directly apply the result stated in \cref{prop:maximal_ineq_degenerate_u_process}, which is a special case of the more general  result in \citep[Maximal inequality]{sherman1994maximal}. 
First note, by assumption, that the functions $t_{\omega}(z,z')$ are uniformly bounded by a positive constant $M$.  Hence, using \cref{eq:def_degenerate_kernel}, it follows that $s_{\omega}(z,z')$ is uniformly bounded by $4M$. Moreover, the functions $\omega\mapsto t_{\omega}(z,z')$ are $L$-Lipschitz for any $z,z'$ in $\mathcal{Z}$. Hence, from  \cref{eq:def_degenerate_kernel}, we directly have that $\omega \mapsto s_{\omega}(z,z')$ is $4L$-Lipschitz for any $z,z'\in \mathcal{Z}$. We can directly apply \cref{prop:maximal_ineq_degenerate_u_process} with $k=2$ and $p=1$ to  $\mathcal{S}$ and get the following maximal inequality:
\begin{align*}
	\mathbb{E}_{\PP}\brackets{\sup_{\omega\in \Omega} \verts{U^2_n s_{\omega}}}\leq 4n^{-1}  c(\Omega)\max\parens{ML\diam(\Omega),M^2}^{\frac{1}{2}}. 
\end{align*}
We obtain an upper-bound  on $\mathbb{E}_{\PP}\brackets{\sup_{\omega\in \Omega} \verts{\tau_{\omega}}}$ by combining the above inequality with 
  \cref{eq:main_decomposition_u_stat}, then noticing that $2M\leq 2c(\Omega)\max\parens{L\diam(\Omega),M}$ so that:
  \begin{align*}
	\mathbb{E}_{\PP}\brackets{\sup_{\omega\in \Omega} \verts{\tau_{\omega}}}\leq 6n^{-1}  c(\Omega)\max\parens{ML\diam(\Omega),M^2}^{\frac{1}{2}}. 
\end{align*}
Finally, the desired result follows by redefining $c(\Omega)$ to include the factor $6$ in the above inequality.  
\end{proof}