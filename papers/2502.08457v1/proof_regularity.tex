\section{Differentiability Results}\label{sec:proof_diff_results}
The proofs of \cref{prop:fre_diff_L_v,prop:fre_diff_L} are direct applications of the following more general result. 
\begin{proposition}
	Let $\mathcal{U}$ be an open non-trivial subset of $\mathbb{R}^d$. 
	Consider a real-valued function $\ell: (\omega,v,y)\mapsto \ell(\omega,v,y)$ defined on $\mathcal{U}\times \mathbb{R}\times \mathcal{Y}$ that is of class $C^{3}$ jointly in $(\omega,v)$ and whose derivatives are jointly continuous in $(\omega,v,y)$. For a given  probability distribution $\PP$ over $\mathcal{X}\times \mathcal{Y}$, consider the following functional defined over $\mathcal{U}\times\mathcal{H}$:
	\begin{align*}
		L(\omega,h) \coloneqq \mathbb{E}_{\PP}\brackets{\ell(\omega,h(x),y)}.
	\end{align*}
	Under \cref{assump:K_bounded,assump:compact}, the following properties hold for $L$:
		\begin{itemize}
			\item $L$ admits finite values for any $(\omega,h)\in \mathcal{U}\times\mathcal{H}$.
			\item $(\omega,h)\mapsto L(\omega,h)$ is Fr\'echet differentiable with partial derivatives $\partial_{\omega}L(\omega,h)$ and $\partial_{h}L(\omega,h)$ at any point $(\omega,h)\in \mathcal{U}\times \mathcal{H}$ given by:
				\begin{align*}
					\partial_{\omega}L(\omega,h) &= \mathbb{E}_{\PP}\brackets{\partial_{\omega} \ell(\omega,h(x),y)} \in \mathbb{R}^d ,\\
					\partial_{h}L(\omega,h) &= \mathbb{E}_{\PP}\brackets{\partial_v \ell(\omega,h(x),y)K(x,\cdot)}\in \mathcal{H}.
				\end{align*}
			\item The map $(\omega,h)\mapsto \partial_{h}L(\omega,h)$ is differentiable. Moreover, for any $(\omega,h)\in \mathcal{U}\times\mathcal{H}$, its partial derivatives $\partial_{\omega,h}^2L(\omega,h)$ and $\partial_{\omega,h}^2L(\omega,h)$ at $(\omega,h)$ are Hilbert-Schmidt operators given by:
				\begin{align*}
					\partial_{\omega,h}^2L(\omega,h)&= \mathbb{E}_{\PP}\brackets{\partial_{\omega,v}^2 \ell(\omega,h(x),y)K(x,\cdot)} \in \mathcal{L}(\mathcal{H},\mathbb{R}^d), \\
					\partial_{h}^2L(\omega,h) &= 
					\mathbb{E}_{\PP}\brackets{\partial_{v}^2 \ell(\omega,h(x),y)K(x,\cdot)\otimes K(x,\cdot)}\in \mathcal{L}(\mathcal{H},\mathcal{H}).
				\end{align*}
		\end{itemize}  
\end{proposition}
\begin{proof}
	
	{\bf Finite values.} Fix $\omega\in \mathcal{U}$ and $h\in \mathcal{H}$. We will first show that  $h$ is bounded on $\mathcal{X}$. By the reproducing property, we know that $\verts{h(x)}\leq \Verts{h}_{\mathcal{H}}\sqrt{K(x,x)}$ for any $x\in \mathcal{X}$. Moreover, the kernel $K$ is bounded by a constant $\kappa$ thanks to \cref{assump:K_bounded}. Consequently, $\verts{h(x)}$ is upper-bounded by $\Verts{h}_{\mathcal{H}}\sqrt{\kappa}$ for any $x\in \mathcal{X}$. 
    
    Denote by $\mathcal{I}$ the compact interval defined as $ \mathcal{I}= [-\Verts{h}_{\mathcal{H}}\sqrt{\kappa},\Verts{h}_{\mathcal{H}}\sqrt{\kappa} ]$. By \cref{assump:compact}, the set $\mathcal{Y}$ is compact so that $\mathcal{I}\times\mathcal{Y}$ is also compact. 
    Moreover, we know, by assumption on  $\ell$, that $(\omega, v, y)\mapsto\ell(\omega, v, y)$ is continuous on $\mathcal{U}\times \mathbb{R}\times \mathcal{Y}$. 
    Therefore, $(v, y)\mapsto\ell(\omega, v, y)$  must be bounded by some finite constant $C$ on the compact set $\mathcal{I}\times \mathcal{Y}$. This allows to deduce that $(x, y)\mapsto\ell(\omega, h(x), y)$ is bounded by $C$ for any $(x,y)\in\mathcal{X}\times \mathcal{Y}$ and a fortiori $\PP$-integrable, which shows that $L(\omega,h)$ is finite. 

	{\bf Fr\'echet differentiability of $L$.} 
	Let $(\omega, h)\in \mathcal{U}\times \mathcal{H}$. Consider $(\omega_j,h_j)_{j\geq 1}$ a sequence of elements in $\mathcal{U}\times \mathcal{H}$ converging to it, \textit{i.e.}, $(\omega_j,h_j)\rightarrow (\omega,h)$ with $(\omega_j,h_j)\neq (\omega,h)$ for any $j\geq 0$.    
	Define the sequence of functions $r_j:\mathcal{X}\times\mathcal{Y}\to\mathbb{R}$ for any $(x,y)\in\mathcal{X}\times\mathcal{Y}$ as follows:
    \begin{equation}\label{eq:expression_rn}
r_j(x,y)=\frac{\ell\left(\omega_j,h_j(x),y\right)-\ell\left(\omega,h(x),y\right)-\left\langle\partial_v\ell\left(\omega,h(x),y\right)K(x,\cdot),h_j-h\right\rangle_\mathcal{H}- \langle \partial_{\omega}\ell(\omega,h(x),y),\omega_j-\omega\rangle}{\left\|(\omega_j,h_j)-(\omega,h)\right\|}.
    \end{equation}    
    We will first show that $\mathbb{E}_{\mathbb{D}}\brackets{\verts{r_j(x,y)}}$ convergence to $0$ by the dominated convergence theorem \citep[Theorem~1.34]{rudin1987real}. 
	By the reproducing property, note that $\ell(\omega,h(x),y) = \ell(\omega,\langle h,K(x,\cdot)\rangle_{\mathcal{H}},y)$. Hence, since $\ell$ is jointly differentiable in $(\omega,v)$ for any $y$, it follows that $(\omega,h)\mapsto \ell(\omega,h(x),y)$ is also differentiable for any $(x,y)$ by composition with the evaluation map $(\omega,h)\mapsto (\omega, \langle h,K(x,\cdot)\rangle_{\mathcal{H}}$ which is differentiable. Hence, the sequence $r_j(x,y)$ converges to $0$ for any $(x,y)\in \mathcal{X}\times \mathcal{Y}$. Moreover, by the mean-value theorem, there exists $0\leq c_j\leq 1$ such that:
	\begin{align*}
		r_j(x,y)= \frac{\left\langle\parens{\partial_v\ell\left(\bar{\omega}_j,\bar{h}_j(x),y\right)- \partial_v\ell\left(\omega,h(x),y\right)}K(x,\cdot),h_j-h\right\rangle_\mathcal{H}- \langle \partial_{\omega}\ell\left(\bar{\omega}_j,\bar{h}_j(x),y\right)- \partial_{\omega}\ell\left(\omega,h(x),y\right),\omega_j-\omega\rangle}{\left\|(\omega_j,h_j)-(\omega,h)\right\|},
	\end{align*}
	where $(\bar{\omega}_j,\bar{h}_j)\coloneqq(1-c_j)(\omega,h) + c_j(\omega_j,h_j)$. 
	We will show that $r_j(x,y)$ are bounded for $j$ large enough. We first construct a compact set that will contain all elements of the form $(\omega_j, h_j(x),y)$ and $(\bar{\omega}_j, \bar{h}_j(x),y)$ for all $j$ large enough. 
	Since $\omega$ is an element in the open set $\mathcal{U}$, there exists a closed ball $\mathcal{B}(\omega,R)$ centered in $\omega$ and with some radius $R$ small enough so that $\mathcal{B}(\omega,R)$ is included in $\mathcal{U}$. For all $j$ large enough, $\omega_j$ and $\bar{\omega}_j$ belong to $\mathcal{B}(\omega,R)$ as these sequences converge to $\omega$. Moreover, $h_j$ and $\bar{h}_j$ are convergent sequences. Consequently, they must be bounded by some constant $B$. By the reproducing property, and recalling that the kernel $K$ is bounded by $\kappa$ by \cref{assump:K_bounded}, it follows that $\max(\verts{h_{j}(x)},\verts{\bar{h}_{j}(x)})\leq B\kappa$.  Consider now the set {$\mathcal{W}\coloneqq\mathcal{B}(\omega, R)\times \mathcal{B}_1(0, B\kappa)\times \mathcal{Y}$} which is a product of compact sets (recalling that $\mathcal{Y}$ is compact by \cref{assump:compact}){, where $\mathcal{B}_1(0, B\kappa)$ is the closed ball in $\mathbb{R}$ centered at $0$ and of radius $B\kappa$}.   
	For $j$ large enough, we have established that  $(\omega_j, h_j(x),y)$ and $(\bar{\omega}_j, \bar{h}_j(x),y)$ belong to $\mathcal{W}$ for any $(x,y)\in \mathcal{X}\times\mathcal{Y}$. 
	Since, by assumption on $\ell$, $\partial_{v}\ell(\omega,v,y)$ and $\partial_{\omega}\ell(\omega,v,y)$ are continuous, they must be bounded on the compact set $\mathcal{W}$ by some constant $C$. This allows to deduce from the expression of $r_{j}(x,y)$ above that $r_{j}(x,y)$ is bounded, and a fortiori dominated by an integrable function (a constant function). We then deduce that $\mathbb{E}_{\mathbb{\PP}}\brackets{\verts{r_j(x,y)}}$ converges to $0$ by application of the dominated convergence theorem \citep[Theorem~1.34]{rudin1987real}.
	
	Recalling \cref{eq:expression_rn},  $\mathbb{E}_{\mathbb{\PP}}\brackets{r_j(x,y)}$ admits the following expression:
	\begin{align}\label{eq:expression_rn_2}
		\mathbb{E}_{\mathbb{\PP}}\brackets{r_j(x,y)} = \frac{L(\omega_j,h_j)-L(\omega,h)-\mathbb{E}_{\PP}\brackets{\left\langle\partial_v\ell\left(\omega,h(x),y\right)K(x,\cdot),h_j-h\right\rangle_\mathcal{H}}- \langle \mathbb{E}_{\PP}\brackets{\partial_{\omega}\ell(\omega,h(x),y)},\omega_j-\omega\rangle}{\left\|(\omega_j,h_j)-(\omega,h)\right\|}.
	\end{align}
	The convergence to $0$ of the above expression precisely means that $L$ is differentiable at $(\omega,h)$ provided that the linear form $g\mapsto\mathbb{E}_{\PP}\brackets{\left\langle\partial_v\ell\left(\omega,h(x),y\right)K(x,\cdot),g\right\rangle_\mathcal{H}}$ is bounded. To establish this fact, consider the RKHS-valued function $(x,y)\mapsto \partial_v\ell\left(\omega,h(x),y\right)K(x,\cdot)$. 
	This function is Bochner-integrable in the sense that $\mathbb{E}_{\PP}\brackets{\Verts{\partial_v\ell\left(\omega,h(x),y\right)K(x,\cdot)}_\mathcal{H}}$ is finite \citep[Definition~1,~Chapter~2]{diestel1977vector}. Indeed, we have the following:
\begin{align*}
	\mathbb{E}_{\PP}\brackets{\Verts{\partial_v\ell\left(\omega,h(x),y\right)K(x,\cdot)}_\mathcal{H}} \coloneqq \mathbb{E}_{\PP}\brackets{\verts{\partial_v\ell\left(\omega,h(x),y\right)}\sqrt{K(x,x)}}
	\leq \sqrt{\kappa}\mathbb{E}_{\PP}\brackets{\verts{\partial_v\ell\left(\omega,h(x),y\right)}}<+\infty,
\end{align*}
where, for the inequality, we used that $(x,y)\mapsto\partial_v\ell(\omega,h(x),y)$ is bounded as shown previously. Consequently, $\mathbb{E}\brackets{\partial_v\ell\left(\omega,h(x),y\right)K(x,\cdot)}$ is an element in $\mathcal{H}$ satisfying:
\begin{align*}
	\langle\mathbb{E}\brackets{\partial_v\ell\left(\omega,h(x),y\right)K(x,\cdot)},  g\rangle_{\mathcal{H}}=\mathbb{E}_{\PP}\brackets{\left\langle\partial_v\ell\left(\omega,h(x),y\right)K(x,\cdot),g\right\rangle_\mathcal{H}},\forall g\in \mathcal{H}.
\end{align*}
The above property follows from \citep[Theorem~6,~Chapter~2]{diestel1977vector} for Bochner-integrable functions that allows exchanging the integral and the application of a continuous linear map (here the scalar product with an element $g$). The above identity establishes that $g\mapsto\mathbb{E}_{\PP}\brackets{\left\langle\partial_v\ell\left(\omega,h(x),y\right)K(x,\cdot),g\right\rangle_\mathcal{H}}$ is bounded and provides the desired expression for $\partial_h L(\omega,h)$. The expression for $\partial_{\omega} L(\omega,h)$ directly follows from the last term in \cref{eq:expression_rn_2}.

{\bf Fr\'echet differentiability of $\partial_h L$. } We use the same proof strategy as for the differentiability of $L$.  

Let $(\omega, h)\in \mathcal{U}\times \mathcal{H}$. Consider $(\omega_j,h_j)_{j\geq 1}$ a sequence of elements in $\mathcal{U}\times \mathcal{H}$ converging to it, \textit{i.e.}, $(\omega_j,h_j)\rightarrow (\omega,h)$ with $(\omega_j,h_j)\neq (\omega,h)$ for any $j\geq 0$.    
	Define the sequence of functions $s_j:\mathcal{X}\times\mathcal{Y}\to\mathcal{H}$ as follows:
    \begin{align}\label{eq:expression_sn}
    \begin{split}
\left\|(\omega_j,h_j)-(\omega,h)\right\|s_j(x,y)= &\parens{\partial_v\ell\left(\omega_j,h_j(x),y\right)-\partial_v\ell\left(\omega,h(x),y\right)}K(x,\cdot)\\
&-\partial_v^2\ell\left(\omega,h(x),y\right)K(x,\cdot)\otimes K(x,\cdot)\parens{h_j-h}
- \parens{\omega_j-\omega}^{\top} \partial_{\omega,v}^2\ell(\omega,h(x),y)K(x,\cdot).
    \end{split}
\end{align}
 We will first show that $\mathbb{E}_{\mathbb{D}}\brackets{\Verts{s_j(x,y)}_{\mathcal{H}}}$ convergence to $0$ by the dominated convergence theorem for Bochner-integrable functions \citep[Theorem~3,~Chapter~2]{diestel1977vector}. 
	By the reproducing property, note that $\partial_v\ell(\omega,h(x),y)K(x,\cdot) = \partial_v\ell(\omega,\langle h,K(x,\cdot)\rangle_{\mathcal{H}},y)K(x,\cdot)$. Hence, since $(\omega,v)\mapsto\partial_v\ell(\omega,v,y)$ is jointly differentiable in $(\omega,v)$ for any $y$, it follows that $(\omega,h)\mapsto \partial_v\ell(\omega,h(x),y)K(x,\cdot)$ is also differentiable for any $(x,y)$ by composition with the evaluation map $(\omega,h)\mapsto (\omega, \langle h,K(x,\cdot)\rangle_{\mathcal{H}}$ which is differentiable. Hence, the sequence $s_j(x,y)$ converges to $0$ for any $(x,y)\in \mathcal{X}\times \mathcal{Y}$. Moreover, by the mean-value theorem, there exists $0\leq c_j\leq 1$ such that:
\begin{align*}
    \begin{split}
\left\|(\omega_j,h_j)-(\omega,h)\right\|s_j(x,y) =& \partial_v^2\ell\left(\bar{\omega}_j,\bar{h}_j(x),y\right)K(x,\cdot)\otimes K(x,\cdot)\parens{h_j-h}
+ \parens{\omega_j-\omega}^{\top} \partial_{\omega,v}^2\ell(\bar{\omega}_j,\bar{h}_j(x),y)K(x,\cdot)\\
&-\partial_v^2\ell\left(\omega,h(x),y\right)K(x,\cdot)\otimes K(x,\cdot)\parens{h_j-h}
- \parens{\omega_j-\omega}^{\top} \partial_{\omega,v}^2\ell(\omega,h(x),y)K(x,\cdot),
    \end{split}
\end{align*}
where $(\bar{\omega}_j,\bar{h}_j)\coloneqq (1-c_j)(\omega,h) + c_j(\omega_j,h_j)$. Using the same construction as for the Fr\'echet differentiability, we find a compact set $\mathcal{W}$ containing all elements $(\omega_j,h_j(x),y)$ and $(\bar{\omega}_j,\bar{h}_j(x),y)$ for any $(x,y)\in \mathcal{X}\times \mathcal{Y}$ and all $j$ large enough. On such set, $\partial_v^2\ell(\omega,v,y)$ and $\partial_{\omega,v}^2\ell(\omega,v,y)$ are bounded by some constant $C$. Consequently, we can write:
\begin{align*}
    \begin{split}
\left\|(\omega_j,h_j)-(\omega,h)\right\|\Verts{s_j(x,y)}_{\mathcal{H}} \leq & 2C\Verts{K(x,\cdot)\otimes K(x,\cdot)\parens{h_j-h}}_{\mathcal{H}}
+ 2C\Verts{\omega_j-\omega} \Verts{K(x,\cdot)}_{\mathcal{H}}\\
\leq & 2C \kappa\Verts{h_j-h}_{\mathcal{H}} + 2C\sqrt{\kappa}\Verts{\omega_j-\omega}.
    \end{split}
\end{align*}
This already establishes that $s_j(x,y)$ is bounded so that $\mathbb{E}_{\PP}\brackets{\Verts{s_j(x,y)}_{\mathcal{H}}}$ converges to $0$ by application of the dominated convergence theorem. Recalling \cref{eq:expression_sn},  $\mathbb{E}_{\mathbb{\PP}}\brackets{s_j(x,y)}$ admits the following expression:
	\begin{align*}
\begin{split}
\left\|(\omega_j,h_j)-(\omega,h)\right\|\mathbb{E}_{\mathbb{\PP}}\brackets{s_j(x,y)}= &\partial_h L(\omega_j,h_j)-\partial_h L(\omega,h)
-\mathbb{E}_{\mathbb{\PP}}\brackets{\partial_v^2\ell\left(\omega,h(x),y\right)K(x,\cdot)\otimes K(x,\cdot)\parens{h_j-h}}\\
&- \parens{\omega_j-\omega}^{\top}\mathbb{E}_{\mathbb{\PP}}\brackets{ \partial_{\omega,v}^2\ell(\omega,h(x),y)K(x,\cdot)}.
    \end{split}	\end{align*}
	The convergence to $0$ of the above expression precisely means that $L$ is differentiable at $(\omega,h)$ provided that: (1) $\mathbb{E}_{\mathbb{\PP}}\brackets{ \partial_{\omega,v}^2\ell(\omega,h(x),y)K(x,\cdot)}$ is an element in $\mathcal{H}^d$, and (2) the linear map   $g\mapsto\mathbb{E}_{\PP}\brackets{\partial_v^2\ell\left(\omega,h(x),y\right)(K(x,\cdot)\otimes K(x,\cdot))g}$ is bounded. Using the same strategy to establish Bochner's inetgrability of $(x,y)\mapsto \partial_v\ell\left(\omega,h(x),y\right)K(x,\cdot)$, we can show that $(x,y)\mapsto \partial_{\omega,v}^2\ell\left(\omega,h(x),y\right)K(x,\cdot)$ is also Bochner-integrable so that $\mathbb{E}_{\mathbb{\PP}}\brackets{ \partial_{\omega,v}^2\ell(\omega,h(x),y)K(x,\cdot)}$ is indeed an element in $\mathcal{H}^d$. This also establishes the expression of $\partial_{\omega,h}L(\omega,h)$. Similarly, we consider the operator-valued function $\xi:(x,y)\mapsto \partial_{v}^2\ell\left(\omega,h(x),y\right)K(x,\cdot)\otimes K(x,\cdot)$ with values in the space of Hilbert-Schmidt operators on $\mathcal{H}$. The Hilbert-Schmidt (HS) norm of such function satisfies the following inequality:
	\begin{align*}
		\mathbb{E}_{\mathbb{\PP}}\brackets{ \Verts{\partial_{v}^2\ell(\omega,h(x),y)K(x,\cdot)\otimes K(x,\cdot)}_{\hs}}\coloneqq\mathbb{E}_{\mathbb{\PP}}\brackets{ \verts{\partial_{v}^2\ell(\omega,h(x),y)}K(x,x)}\leq \kappa C<+\infty.
	\end{align*}
	Therefore, the function $\xi$ is Bochner-integrable, so that $\mathbb{E}_{\mathbb{\PP}}\brackets{ \partial_{v}^2\ell(\omega,h(x),y)K(x,\cdot)\otimes K(x,\cdot)}$ is a Hilbert-Schmidt operator satisfying: 
\begin{align*}
	\mathbb{E}_{\mathbb{\PP}}\brackets{ \partial_{v}^2\ell(\omega,h(x),y)K(x,\cdot)\otimes K(x,\cdot)}  g=\mathbb{E}_{\PP}\brackets{\partial_{v}^2\ell\left(\omega,h(x),y\right)(K(x,\cdot)\otimes K(x,\cdot))g},\forall g\in \mathcal{H}.
\end{align*}
The above property follows from \citep[Theorem~6,~Chapter~2]{diestel1977vector} for Bochner-integrable functions that allows exchanging the integral and the application of a continuous linear map (here the scalar product with an element $g$). Hence, from the above identity we deduce the desired expression for $\partial_{h}^2 L(\omega,h)$. 
\end{proof}