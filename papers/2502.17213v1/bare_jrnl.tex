%% bare_jrnl.tex
%% V1.4b
%% 2015/08/26
%% by Michael Shell
%% see http://www.michaelshell.org/
%% for current contact information.
%%
%% This is a skeleton file demonstrating the use of IEEEtran.cls
%% (requires IEEEtran.cls version 1.8b or later) with an IEEE
%% journal paper.
%%
%% Support sites:
%% http://www.michaelshell.org/tex/ieeetran/
%% http://www.ctan.org/pkg/ieeetran
%% and
%% http://www.ieee.org/

%%*************************************************************************
%% Legal Notice:
%% This code is offered as-is without any warranty either expressed or
%% implied; without even the implied warranty of MERCHANTABILITY or
%% FITNESS FOR A PARTICULAR PURPOSE! 
%% User assumes all risk.
%% In no event shall the IEEE or any contributor to this code be liable for
%% any damages or losses, including, but not limited to, incidental,
%% consequential, or any other damages, resulting from the use or misuse
%% of any information contained here.
%%
%% All comments are the opinions of their respective authors and are not
%% necessarily endorsed by the IEEE.
%%
%% This work is distributed under the LaTeX Project Public License (LPPL)
%% ( http://www.latex-project.org/ ) version 1.3, and may be freely used,
%% distributed and modified. A copy of the LPPL, version 1.3, is included
%% in the base LaTeX documentation of all distributions of LaTeX released
%% 2003/12/01 or later.
%% Retain all contribution notices and credits.
%% ** Modified files should be clearly indicated as such, including  **
%% ** renaming them and changing author support contact information. **
%%*************************************************************************


% *** Authors should verify (and, if needed, correct) their LaTeX system  ***
% *** with the testflow diagnostic prior to trusting their LaTeX platform ***
% *** with production work. The IEEE's font choices and paper sizes can   ***
% *** trigger bugs that do not appear when using other class files.       ***                          ***
% The testflow support page is at:
% http://www.michaelshell.org/tex/testflow/



\documentclass[journal]{IEEEtran}
%\documentclass[journal]{IEEEtran}
\usepackage{ifsym}
% If IEEEtran.cls has not been installed into the LaTeX system files,
% manually specify the path to it like:
% \documentclass[journal]{../sty/IEEEtran}





% Some very useful LaTeX packages include:
% (uncomment the ones you want to load)


% *** MISC UTILITY PACKAGES ***
%
%\usepackage{ifpdf}
% Heiko Oberdiek's ifpdf.sty is very useful if you need conditional
% compilation based on whether the output is pdf or dvi.
% usage:
% \ifpdf
%   % pdf code
% \else
%   % dvi code
% \fi
% The latest version of ifpdf.sty can be obtained from:
% http://www.ctan.org/pkg/ifpdf
% Also, note that IEEEtran.cls V1.7 and later provides a builtin
% \ifCLASSINFOpdf conditional that works the same way.
% When switching from latex to pdflatex and vice-versa, the compiler may
% have to be run twice to clear warning/error messages.






% *** CITATION PACKAGES ***
%
%\usepackage{cite}
% cite.sty was written by Donald Arseneau
% V1.6 and later of IEEEtran pre-defines the format of the cite.sty package
% \cite{} output to follow that of the IEEE. Loading the cite package will
% result in citation numbers being automatically sorted and properly
% "compressed/ranged". e.g., [1], [9], [2], [7], [5], [6] without using
% cite.sty will become [1], [2], [5]--[7], [9] using cite.sty. cite.sty's
% \cite will automatically add leading space, if needed. Use cite.sty's
% noadjust option (cite.sty V3.8 and later) if you want to turn this off
% such as if a citation ever needs to be enclosed in parenthesis.
% cite.sty is already installed on most LaTeX systems. Be sure and use
% version 5.0 (2009-03-20) and later if using hyperref.sty.
% The latest version can be obtained at:
% http://www.ctan.org/pkg/cite
% The documentation is contained in the cite.sty file itself.






% *** GRAPHICS RELATED PACKAGES ***
%
\ifCLASSINFOpdf
  % \usepackage[pdftex]{graphicx}
  % declare the path(s) where your graphic files are
  % \graphicspath{{../pdf/}{../jpeg/}}
  % and their extensions so you won't have to specify these with
  % every instance of \includegraphics
  % \DeclareGraphicsExtensions{.pdf,.jpeg,.png}
\else
  % or other class option (dvipsone, dvipdf, if not using dvips). graphicx
  % will default to the driver specified in the system graphics.cfg if no
  % driver is specified.
  % \usepackage[dvips]{graphicx}
  % declare the path(s) where your graphic files are
  % \graphicspath{{../eps/}}
  % and their extensions so you won't have to specify these with
  % every instance of \includegraphics
  % \DeclareGraphicsExtensions{.eps}
\fi
% graphicx was written by David Carlisle and Sebastian Rahtz. It is
% required if you want graphics, photos, etc. graphicx.sty is already
% installed on most LaTeX systems. The latest version and documentation
% can be obtained at: 
% http://www.ctan.org/pkg/graphicx
% Another good source of documentation is "Using Imported Graphics in
% LaTeX2e" by Keith Reckdahl which can be found at:
% http://www.ctan.org/pkg/epslatex
%
% latex, and pdflatex in dvi mode, support graphics in encapsulated
% postscript (.eps) format. pdflatex in pdf mode supports graphics
% in .pdf, .jpeg, .png and .mps (metapost) formats. Users should ensure
% that all non-photo figures use a vector format (.eps, .pdf, .mps) and
% not a bitmapped formats (.jpeg, .png). The IEEE frowns on bitmapped formats
% which can result in "jaggedy"/blurry rendering of lines and letters as
% well as large increases in file sizes.
%
% You can find documentation about the pdfTeX application at:
% http://www.tug.org/applications/pdftex





% *** MATH PACKAGES ***
%
%\usepackage{amsmath}
% A popular package from the American Mathematical Society that provides
% many useful and powerful commands for dealing with mathematics.
%
% Note that the amsmath package sets \interdisplaylinepenalty to 10000
% thus preventing page breaks from occurring within multiline equations. Use:
%\interdisplaylinepenalty=2500
% after loading amsmath to restore such page breaks as IEEEtran.cls normally
% does. amsmath.sty is already installed on most LaTeX systems. The latest
% version and documentation can be obtained at:
% http://www.ctan.org/pkg/amsmath


\usepackage{cite}
\usepackage{amsmath,amssymb,amsfonts}
\usepackage{algorithmic}
\usepackage{graphicx}
\usepackage{textcomp}
\usepackage{makecell}  % 在导言部分加入这个宏包
\usepackage{multirow}
\usepackage{hyperref}
\usepackage{placeins}
\usepackage[justification=centering]{caption} 
\bibliographystyle{IEEEtran}


% *** SPECIALIZED LIST PACKAGES ***
%
%\usepackage{algorithmic}
% algorithmic.sty was written by Peter Williams and Rogerio Brito.
% This package provides an algorithmic environment fo describing algorithms.
% You can use the algorithmic environment in-text or within a figure
% environment to provide for a floating algorithm. Do NOT use the algorithm
% floating environment provided by algorithm.sty (by the same authors) or
% algorithm2e.sty (by Christophe Fiorio) as the IEEE does not use dedicated
% algorithm float types and packages that provide these will not provide
% correct IEEE style captions. The latest version and documentation of
% algorithmic.sty can be obtained at:
% http://www.ctan.org/pkg/algorithms
% Also of interest may be the (relatively newer and more customizable)
% algorithmicx.sty package by Szasz Janos:
% http://www.ctan.org/pkg/algorithmicx




% *** ALIGNMENT PACKAGES ***
%
%\usepackage{array}
% Frank Mittelbach's and David Carlisle's array.sty patches and improves
% the standard LaTeX2e array and tabular environments to provide better
% appearance and additional user controls. As the default LaTeX2e table
% generation code is lacking to the point of almost being broken with
% respect to the quality of the end results, all users are strongly
% advised to use an enhanced (at the very least that provided by array.sty)
% set of table tools. array.sty is already installed on most systems. The
% latest version and documentation can be obtained at:
% http://www.ctan.org/pkg/array


% IEEEtran contains the IEEEeqnarray family of commands that can be used to
% generate multiline equations as well as matrices, tables, etc., of high
% quality.




% *** SUBFIGURE PACKAGES ***
%\ifCLASSOPTIONcompsoc
%  \usepackage[caption=false,font=normalsize,labelfont=sf,textfont=sf]{subfig}
%\else
%  \usepackage[caption=false,font=footnotesize]{subfig}
%\fi
% subfig.sty, written by Steven Douglas Cochran, is the modern replacement
% for subfigure.sty, the latter of which is no longer maintained and is
% incompatible with some LaTeX packages including fixltx2e. However,
% subfig.sty requires and automatically loads Axel Sommerfeldt's caption.sty
% which will override IEEEtran.cls' handling of captions and this will result
% in non-IEEE style figure/table captions. To prevent this problem, be sure
% and invoke subfig.sty's "caption=false" package option (available since
% subfig.sty version 1.3, 2005/06/28) as this is will preserve IEEEtran.cls
% handling of captions.
% Note that the Computer Society format requires a larger sans serif font
% than the serif footnote size font used in traditional IEEE formatting
% and thus the need to invoke different subfig.sty package options depending
% on whether compsoc mode has been enabled.
%
% The latest version and documentation of subfig.sty can be obtained at:
% http://www.ctan.org/pkg/subfig




% *** FLOAT PACKAGES ***
%
%\usepackage{fixltx2e}
% fixltx2e, the successor to the earlier fix2col.sty, was written by
% Frank Mittelbach and David Carlisle. This package corrects a few problems
% in the LaTeX2e kernel, the most notable of which is that in current
% LaTeX2e releases, the ordering of single and double column floats is not
% guaranteed to be preserved. Thus, an unpatched LaTeX2e can allow a
% single column figure to be placed prior to an earlier double column
% figure.
% Be aware that LaTeX2e kernels dated 2015 and later have fixltx2e.sty's
% corrections already built into the system in which case a warning will
% be issued if an attempt is made to load fixltx2e.sty as it is no longer
% needed.
% The latest version and documentation can be found at:
% http://www.ctan.org/pkg/fixltx2e


%\usepackage{stfloats}
% stfloats.sty was written by Sigitas Tolusis. This package gives LaTeX2e
% the ability to do double column floats at the bottom of the page as well
% as the top. (e.g., "\begin{figure*}[!b]" is not normally possible in
% LaTeX2e). It also provides a command:
%\fnbelowfloat
% to enable the placement of footnotes below bottom floats (the standard
% LaTeX2e kernel puts them above bottom floats). This is an invasive package
% which rewrites many portions of the LaTeX2e float routines. It may not work
% with other packages that modify the LaTeX2e float routines. The latest
% version and documentation can be obtained at:
% http://www.ctan.org/pkg/stfloats
% Do not use the stfloats baselinefloat ability as the IEEE does not allow
% \baselineskip to stretch. Authors submitting work to the IEEE should note
% that the IEEE rarely uses double column equations and that authors should try
% to avoid such use. Do not be tempted to use the cuted.sty or midfloat.sty
% packages (also by Sigitas Tolusis) as the IEEE does not format its papers in
% such ways.
% Do not attempt to use stfloats with fixltx2e as they are incompatible.
% Instead, use Morten Hogholm'a dblfloatfix which combines the features
% of both fixltx2e and stfloats:
%
% \usepackage{dblfloatfix}
% The latest version can be found at:
% http://www.ctan.org/pkg/dblfloatfix




%\ifCLASSOPTIONcaptionsoff
%  \usepackage[nomarkers]{endfloat}
% \let\MYoriglatexcaption\caption
% \renewcommand{\caption}[2][\relax]{\MYoriglatexcaption[#2]{#2}}
%\fi
% endfloat.sty was written by James Darrell McCauley, Jeff Goldberg and 
% Axel Sommerfeldt. This package may be useful when used in conjunction with 
% IEEEtran.cls'  captionsoff option. Some IEEE journals/societies require that
% submissions have lists of figures/tables at the end of the paper and that
% figures/tables without any captions are placed on a page by themselves at
% the end of the document. If needed, the draftcls IEEEtran class option or
% \CLASSINPUTbaselinestretch interface can be used to increase the line
% spacing as well. Be sure and use the nomarkers option of endfloat to
% prevent endfloat from "marking" where the figures would have been placed
% in the text. The two hack lines of code above are a slight modification of
% that suggested by in the endfloat docs (section 8.4.1) to ensure that
% the full captions always appear in the list of figures/tables - even if
% the user used the short optional argument of \caption[]{}.
% IEEE papers do not typically make use of \caption[]'s optional argument,
% so this should not be an issue. A similar trick can be used to disable
% captions of packages such as subfig.sty that lack options to turn off
% the subcaptions:
% For subfig.sty:
% \let\MYorigsubfloat\subfloat
% \renewcommand{\subfloat}[2][\relax]{\MYorigsubfloat[]{#2}}
% However, the above trick will not work if both optional arguments of
% the \subfloat command are used. Furthermore, there needs to be a
% description of each subfigure *somewhere* and endfloat does not add
% subfigure captions to its list of figures. Thus, the best approach is to
% avoid the use of subfigure captions (many IEEE journals avoid them anyway)
% and instead reference/explain all the subfigures within the main caption.
% The latest version of endfloat.sty and its documentation can obtained at:
% http://www.ctan.org/pkg/endfloat
%
% The IEEEtran \ifCLASSOPTIONcaptionsoff conditional can also be used
% later in the document, say, to conditionally put the References on a 
% page by themselves.




% *** PDF, URL AND HYPERLINK PACKAGES ***
%
%\usepackage{url}
% url.sty was written by Donald Arseneau. It provides better support for
% handling and breaking URLs. url.sty is already installed on most LaTeX
% systems. The latest version and documentation can be obtained at:
% http://www.ctan.org/pkg/url
% Basically, \url{my_url_here}.




% *** Do not adjust lengths that control margins, column widths, etc. ***
% *** Do not use packages that alter fonts (such as pslatex).         ***
% There should be no need to do such things with IEEEtran.cls V1.6 and later.
% (Unless specifically asked to do so by the journal or conference you plan
% to submit to, of course. )


% correct bad hyphenation here
\hyphenation{op-tical net-works semi-conduc-tor}


\begin{document}
%
% paper title
% Titles are generally capitalized except for words such as a, an, and, as,
% at, but, by, for, in, nor, of, on, or, the, to and up, which are usually
% not capitalized unless they are the first or last word of the title.
% Linebreaks \\ can be used within to get better formatting as desired.
% Do not put math or special symbols in the title.
\title{Deep Learning-Powered Electrical Brain Signals\\ Analysis: Advancing Neurological Diagnostics}


\author{Jiahe Li, Xin Chen, Fanqi Shen, Junru Chen, Yuxin Liu, Daoze Zhang, Zhizhang Yuan, Fang Zhao, Meng Li$^{\dagger}$ and Yang Yang$^{\dagger}$
\thanks{$^{\dagger}$ Corresponding author.}
\thanks{Manuscript received 24 February 2025. This work is supported by NSFC (62322606) and Zhejiang NSF (LR22F020005). }
\thanks{Jiahe Li, Xin Chen, Fanqi Shen, Junru Chen, Yuxin Liu, Daoze Zhang, Zhizhang Yuan and Yang Yang are with the College of Computer Science and Technology, Zhejiang University, Hangzhou, Zhejiang 310027, China (e-mail: jiaheli@zju.edu.cn, xin.21@intl.zju.edu.cn, fanqishen@zju.edu.cn, jrchen\_cali@zju.edu.cn, yuxin.liu@zju.edu.cn, zhangdz@zju.edu.cn, zhizhangyuan@zju.edu.cn, yangya@zju.edu.cn).}
\thanks{Fang Zhao is with the DiFint Technology (Shanghai) Co, Shanghai, 201210, China (email: zhaofang@difint.cn).}
\thanks{Meng Li is with the Shanghai Institute of Microsystem and Information Technology, Chinese Academy of Sciences, Shanghai, China, the School of Graduate Study, University of Chinese Academy of Sciences, Beijing, 100049, China, and The INSIDE Institute for Biological and Artificial Intelligence, Shanghai, 201210, China (email: limeng.braindecoder@gmail.com). }}


\markboth{Journal of \LaTeX\ Class Files,~Vol.~14, No.~8, August~2015}%
{Shell \MakeLowercase{\textit{et al.}}: Bare Demo of IEEEtran.cls for IEEE Journals}


\maketitle

\begin{abstract}
Multi-modal models, such as CLIP, have demonstrated strong performance in aligning visual and textual representations, excelling in tasks like image retrieval and zero-shot classification. Despite this success, the mechanisms by which these models utilize training data, particularly the role of memorization, remain unclear. In uni-modal models, both supervised and self-supervised, memorization has been shown to be essential for generalization. However, it is not well understood how these findings would apply to CLIP, which incorporates elements from both supervised learning via captions that provide a supervisory signal similar to labels, and from self-supervised learning via the contrastive objective.
To bridge this gap in understanding, we propose a formal definition of memorization in CLIP (CLIPMem) and use it to quantify memorization in CLIP models. Our results indicate that CLIP’s memorization behavior falls between the supervised and self-supervised paradigms, with "mis-captioned" samples exhibiting highest levels of memorization. 
Additionally, we find that the text encoder contributes more to memorization than the image encoder, suggesting that mitigation strategies should focus on the text domain. 
Building on these insights, we propose multiple strategies to reduce memorization while at the same time improving utility---something that had not been shown before for traditional learning paradigms where reducing memorization typically results in utility decrease.
%, some of our proposed mitigations for CLIP can reduce memorization while improving downstream utility.\todo{This needs to be reworked: our CLIPMem has the practical application to identify "miscaptioned" samples, such that we can remove them from the training, and then get better results. This is particularly important given that CLIP is trained on large amounts of uncurated data from the internet, and one cannot review all these image pairs. With out metric, it becomes possible.
%aybe also include the risk of exposure for these data points that otherwise arise.
%}
% Multi-modal models, such as CLIP, exhibit strong performance in aligning visual and textual representations, thereby achieving remarkable performance in tasks like image retrieval and zero-shot classification. 
% While the models have a strong generalization ability, it is not fully understood how the models leverage their training data to achieve this.
% One factor that is often linked to a model's generalization ability is memorization. For uni-modal models, both in supervised and self-supervised, it has been shown that memorization is required for generalization.
% Yet, it is unclear how the findings will translate to CLIP because in CLIP captions as supervisory signals, somewhat akin to traditional labels, but also employs self-supervised contrastive learning. Hence, CLIP is in between both paradigms.
% To bridge this gap, we propose a formal definition of memorization in CLIP (CLIPMem) and use it to quantify memorization in CLIP models. 
% Our results show that CLIP's memorization behavior indeed falls between supervised and self-supervised paradigms. Notably, "mis-captioned" samples exhibit high levels of memorization.
% Additionally, we find that the the text encoder has a higher impact on memorization than the image encoder.
% Based on these findings, we find some effective mitigation strategies for memorization in CLIP that focus more on the text domain to maintain model performance while reducing memorization.
% Indeed, unlike in traditional supervised or self-supervised learning, where reducing memorization often reduces utility, we empirically find that some mitigations in CLIP not only reduce memorization but at the same time improve downstream utility.
\end{abstract}

\section{Introduction}

Multi-modal models, such as CLIP~\citep{radford2021}, have demonstrated strong performance in representation learning.
By aligning visual and textual representations, these models achieve state-of-the-art results in tasks like image retrieval~\citep{baldrati2022conditioned,baldrati2022effective}, visual question answering~\citep{pan2023retrieving,song2022clip}, and zero-shot classification~\citep{radford2021,ali2023clip,wang2023improving,zhang2022tip}. 
Despite these successes, the mechanisms by which multi-modal models leverage their training data to achieve good generalization remain underexplored. 

In uni-modal setups, both supervised~\citep{feldman2020does,feldman2020neural} and self-supervised~\citep{wang2024memorization}, machine learning models have shown that their ability to \textit{memorize} their training data is essential for generalization. 
It was indicated that, in supervised learning, memorization typically occurs for mislabeled samples, outliers~\citep{bartlett2020benign,feldman2020does,feldman2020neural}, or data points that were seen towards the end of training~\citep{jagielski2022measuring}, while in self-supervised learning, high memorization is experienced particularly for atypical data points~\citep{wang2024memorization}. 
However, it is unclear how these findings extend to models like CLIP which entail elements from both supervised learning (through captions as supervisory signals) and self-supervised learning (through contrastive loss functions).

Existing definitions of memorization offer limited applicability to CLIP and therefore cannot fully address the gap in understanding.
% can, hence, not close the gap in understanding:
The standard definition from supervised learning~\citep{feldman2020does} relies on one-dimensional labels and the model's ability to produce confidence scores for these labels, whereas CLIP outputs high-dimensional representations. While the SSLMem metric~\citep{wang2024memorization}, developed for self-supervised vision models, could, in principle, be applied to CLIP's vision encoder outputs, it neglects the text modality, which is a critical component of CLIP. Additionally, measuring memorization in only one modality, or treating the modalities separately, risks diluting the signal and under-reporting memorization. Our experimental results, as shown in \Cref{sub:sslmem_not_for_clip}, confirm this concern. Therefore, new definitions of memorization tailored to CLIP's multi-modal nature are necessary.
\begin{figure}[t]
    \centering
    \begin{subfigure}[b]{0.475\textwidth}
        \centering
        \includegraphics[width=\textwidth]{image/10_most_1_caption.pdf}
        \caption[]{{\small Most Memorized: CLIPMem $>$ 0.89}}
    \end{subfigure}
    \hfill
    \begin{subfigure}[b]{0.475\textwidth}  
        \centering 
        \includegraphics[width=\textwidth]{image/10_least_1_caption.pdf}
        \caption[]%
        {{\small Least Memorized: CLIPMem $\approx$ 0.0}}    
    \end{subfigure}
    % \vskip\baselineskip
    % \begin{subfigure}[b]{0.475\textwidth}   
    %     \centering 
    %     \includegraphics[width=\textwidth]{Example-Image}
    %     \caption[]%
    %     {{\small Network 3}}    
    %     \label{fig:mean and std of net34}
    % \end{subfigure}
    % \hfill
    % \begin{subfigure}[b]{0.475\textwidth}   
    %     \centering 
    %     \includegraphics[width=\textwidth]{Example-Image}
    %     \caption[]%
    %     {{\small Network 4}}    
    %     \label{fig:mean and std of net44}
    % \end{subfigure}
    \caption{\textbf{Examples of data with different levels of memorization.} Higher memorization scores indicate stronger memorization. 
    We observe that atypical or distorted images, as well as those with incorrect or imprecise captions, experience higher memorization compared to standard samples and easy-to-label images with accurate captions.
    % We observe that atypical or distorted images and images with incorrect or imprecise captions experience higher memorization compared to more standard samples and easy-to-label samples with precise captions. 
    Results are obtained on OpenCLIP~\citep{ilharco_gabriel_2021_5143773}, with encoders based on the ViT-Base architecture trained on the COCO dataset.} 
        \label{fig:examples}
        %\vspace{-0.8cm}
\end{figure}

The only existing empirical work on quantifying memorization in CLIP models~\citep{jayaraman2024} focuses on Déjà Vu memorization~\citep{meehan2023ssl}, a specific type of memorization.
The success of their method relies on the accuracy of the integrated object detection method and on the availability of an additional public dataset from the same distribution as CLIP's training data, limiting practical applicability.
To overcome this limitation, we propose \textit{\ours} that measures memorization directly on CLIP's output representations.
Specifically, it compares the alignment---\ie the similarity between representations---of a given image-text pair in a CLIP model trained with the pair, to the alignment in a CLIP model trained on the same data but without the pair.

% Additionally, we focus on \textit{understanding} memorization rather than quantifying it. 
% %which, in CLIP, can be measured only with respect to an additional public dataset from the same distribution as CLIP's training data and fine-grained object detection methods. Moreover, the work is limited to \textit{quantifying} memorization.
% %---limiting its practical applicability. Moreover, while the work \textit{quantifies} Déjà Vu memorization, it does not offer detailed insights into which specific data points are memorized, why they are memorized, and how this relates to generalization. 
% %In contrast to their work, our focus is on \textit{understanding} memorization in CLIP by 
% We use this to identify which properties of the data and the two modalities contribute to CLIP memorization and on leveraging these insights to achieve \textit{better model utility while mitigating memorization}. To this end, we propose \textit{\ours} that directly measures memorization on the representations produced by CLIP's vision and text encoders.
% Specifically, \ours measures memorization by comparing the alignment, \ie the similarity between representations, of a given image-text pair in a CLIP model trained with this pair to the alignment in a CLIP model trained without this pair but on the same data otherwise.


In our empirical study of memorization in CLIP using \ours, we uncover several key findings. First, examples with incorrect or imprecise captions ("mis-captioned" examples) exhibit the highest levels of memorization, followed by atypical examples, as illustrated in \Cref{fig:examples}.
Second, removing these samples from training yields significant improvements in CLIP's generalization abilities.
These findings are particularly noteworthy, given that state-of-the-art CLIP models are usually trained on large, uncurated datasets sourced from the internet with no guarantees regarding the correctness of the text-image pairs.
Our results highlight that this practice not only exposes imprecise or incorrect data pairs to more memorization, often recognized as a cause for increased privacy leakage~\citep{carlini2019secret, carlini2021extracting, carlini2022privacy,song2017machine,liu2021encodermi}, but that it also negatively affects model performance. 
%By identifying highly memorized samples, our \ours can, hence, support a more private and performant deployment of CLIP.\todo{@Adam, is that last sentence too strong?}
Furthermore, by disentangling CLIP's two modalities, we are able to dissect how memorization manifests within each.
Surprisingly, we find that memorization does not affect both modalities alike, with memorization occurring more in the text modality than in the vision modality.
% even though the training objective is symmetric.\todo{@Adam, is that correct?}
% In fact, our results highlight that memorization occurs more in the text modality than in the vision modality. 
Building on these insights, we propose several strategies to reduce memorization while simultaneously improving generalization---a result that has not been observed in traditional supervised or self-supervised learning, where any reduction of memorization causes decreases in performance.
% which, at the same time, improve generalization.
% Such a result has not been observed in traditional supervised or self-supervised learning, where any reduction of memorization causes decreases in performance. 
Finally, at a deeper level, our analysis of the model internals, following~\citet{wang2024localizing}, shows that CLIP's memorization behavior sits between that of supervised and self-supervised learning. Specifically, neurons in early layers are responsible for groups of data points (\eg classes), similar to models trained using supervised learning, while neurons in later layers memorize individual data points, as seen in self-supervised learning.%\todo{cite our localization paper.}
% Performing an empirical evaluatin of memorization in CLIP according to our \ours, we find that
% -  examples with incorrect ("mis-captioned") or imprecise captions experience highest memorization, and then atypical examples . We show this effect in \Cref{fig:examples}.
% - memorization happens more in the text than in the vision modality
% - by including more captions into training, when only a few of them are mislabeled and the rest is correct, we can reduce memorization and at the same time improve generalization, something that has not been possible for supervised or self-supervised learning.
% - looking at the model internals, we see that the memorization behavior of CLIP is exactly in between supervised and self-supervised learning: in particular, neurons in early layers are responsible for groups (classes) of data points, same like for supervised learning, while neurons in later layers are responsible for individual data points

% \franzi{@Adam, do you think, we need an additional paragraph here on the mitigations we have? We actually only have multi-caption, so far, so probably not so important?}\adam{We also can mitigate the memorization if we remove the most memorized (probably mislabeled) samples.}

In summary, we make the following contributions:
\begin{itemize}
    \item We propose \ours, a metric to measure memorization in multi-modal vision language models.
    \item Through extensive evaluation, we identify that "mis-captioned" and "atypical" data points experience the highest memorization, and that the text encoder is more responsible for memorization than the image encoder.
    \item Based on our insights, we propose and evaluate multiple strategies to mitigate memorization in CLIP. We show that in CLIP, contrary to traditional supervised and self-supervised learning, a reduction of memorization does not need to imply a decrease in performance.
\end{itemize}


\section{Methods}

\subsection{Problem Definition}
\label{sec:pdef}
In this survey, we classify neurological diagnostic tasks into sample-level classification and event-level classification, both of which fall under the broader framework of classification problems. 
Sample-level classification involves assigning a single label to an entire signal, which typically represents a specific subject or sample (e.g., Alzheimer’s disease diagnosis). 
By comparison, event-level classification focuses on identifying and classifying distinct temporal segments within a more extended signal, thereby introducing an implicit segmentation process by associating each segment with a specific event or state (e.g., seizure detection or sleep staging).

Electrical brain signals, which capture the brain's electrical activity over time, can be modeled as multivariate time series.
Specifically, let  $\mathbf{X} \in \mathbb{R}^{C \times T}$  represent the EEG/iEEG time series, where $C$ is the number of channels, and $T$ is the number of sampling points. Each channel  $\mathbf{x}^c = \{x^c_1, x^c_2, \dots, x^c_T\}$  corresponds to the measurements from a specific source, such as an EEG electrode or a contact of an iEEG electrode.

\subsubsection{Sample-Level Classification}
In sample-level classification, the objective is to assign a single label $y \in \mathcal{Y}$ to the entire signal $\mathbf{X}$. This can be formulated as:
\[
y = \Phi_{\text{sample}}(\mathbf{X}; \boldsymbol{\theta}), \quad y \in \mathcal{Y},
\]
where $\Phi_{\text{sample}}$ represents the deep learning model parameterized by $\boldsymbol{\theta}$, and $\mathcal{Y}$ denotes the set of possible classes. Here, $\mathbf{X}$ is treated as a unified entity, capturing sample-level or subject-level characteristics.

\subsubsection{Event-Level Classification}
In event-level classification, the goal is to classify smaller temporal segments of the signal. The signal $\mathbf{X}$ is divided into $K$ segments $\mathbf{X}_1, \mathbf{X}_2, \dots, \mathbf{X}_K$, where $\mathbf{X}_k \in \mathbb{R}^{C \times T_k}$ and $T_k$ is the duration of the $k$-th segment. A classification model is applied to each segment to produce a sequence of labels $\mathbf{Y} = \{y_1, y_2, \dots, y_K\}, \, y_k \in \mathcal{Y}$:
\[
y_k = \Phi_{\text{segment}}(\mathbf{X}_k; \boldsymbol{\theta}), \quad \mathbf{Y} = \bigcup_{k=1}^K \{y_k\},
\]
where $\Phi_{\text{segment}}$ denotes the deep learning model parameterized by $\boldsymbol{\theta}$. This process associates each segment $\mathbf{X}_k$ with a specific label $y_k$, allowing the temporal localization of events within the signal.
Event-level classification captures natural temporal dependencies between consecutive segments, reflecting the continuity of events in time~\cite{chen2024con4m}.

\subsection{Signal Collection}
EEG have evolved significantly since Hans Berger first recorded EEG signals from the human scalp in 1924~\cite{berger1929elektroenkephalogramm}.
While EEG signals are typically collected non-invasively using scalp electrodes placed according to the 10-20 system~\cite{jasper1958ten},
more recent studies employ higher-density EEG electrode configurations for enhanced spatial resolution and detailed brain activity mapping.
EEG captures brain oscillations across frequency bands, each linked to specific neural states: delta (deep sleep), theta (light sleep), alpha (relaxation), beta (focus), and gamma (higher cognition)~\cite{buzsaki2004neuronal}.
Depending on the study, participants may perform tasks or rest to elicit relevant brain activity. Resting-state EEG evaluates baseline activity, while specific tasks can highlight disease-related abnormalities~\cite{jeong2004eeg}.

iEEG involves implanting electrodes either within deep and superficial brain structures via burr holes (SEEG) or on the brain’s surface by placing grids during craniotomy (ECoG).
Compared to EEG, iEEG offers excellent spatial resolution and reduced susceptibility to artifacts from scalp muscle activity and eye movements.
SEEG allows recording from deep and distributed brain regions with minimal invasiveness, while ECoG provides higher spatial resolution for cortical surface activity due to its densely packed electrode grids. 
However, iEEG can still be affected by cardiac artifacts, electrode shifts, and other forms of noise. Rigorous preprocessing techniques are essential to ensure the accuracy and reliability of EEG and iEEG signals in clinical and research applications.

\subsection{Signal Preprocessing}

\begin{table}[]
\renewcommand{\arraystretch}{1.2}
\caption{Signal Preprocessing Techniques}
\label{tab:lowlevel}
\footnotesize
\centering
\begin{tabular}{p{80pt}p{90pt}p{30pt}}
\hline
\textbf{Techniques}     & \textbf{Details}             & \textbf{Reference}             \\
\hline
\multirow{4}{80pt}{Noise Reduction \& Filtering} 
                            & FIR Filter           & ~\cite{banville2021uncovering}  \\
                            & IIR Filter           & ~\cite{oh2020deep} \\
                            & Adaptive Filters     & ~\cite{9353630}\\
                            & Manual \& Custom     & ~\cite{ay2019automated} \\
\hline
\multirow{2}{80pt}{Artifact Removal} 
                            & Blind Source Separation  & ~\cite{10023506} \\
                            & Artifact Correction  & ~\cite{MOGHADDARI2020105738} \\
\hline
\multirow{3}{80pt}{Baseline Correction \& Detrending} 
                            & Baseline Correction  & ~\cite{sun2021hybrid} \\
                            & Baseline Removal     & ~\cite{nouri2024detection} \\
                            & Detrending           & ~\cite{Seizure49} \\
\hline
\multirow{3}{80pt}{Channel Processing} 
                            & Channel Selection    & ~\cite{wen2018deep} \\
                            & Channel Mapping      & ~\cite{Kostas2021BENDR} \\
                            & Re-Referencing       & ~\cite{nouri2024detection} \\
\hline
\multirow{3}{80pt}{Normalization \& Scaling} 
                            & Z-Normalization      & ~\cite{ACHARYA2018270} \\
                            & Quantile Normalization & ~\cite{ko2022eeg} \\
                            & Scaling \& Shifting  & ~\cite{Kostas2021BENDR} \\
\hline
\multirow{4}{80pt}{Sampling Adjustment} 
                            & Downsampling         & ~\cite{mousavi2019deep} \\
                            & Resampling           & ~\cite{ZHANG2020105089} \\
                            & Interpolation        & ~\cite{SZ27} \\
                            & Imputation           & ~\cite{sharma2021dephnn} \\
\hline
\multirow{1}{80pt}{Segmentation} 
                            & Windowing            & ~\cite{seal2021deprnet} \\
\hline
\multirow{2}{80pt}{Signal Alignment \& Synchronization} 
                            & Time Synchronization & ~\cite{iwama2023two} \\
                            & Temporal Alignment   & ~\cite{iwama2023two} \\
\hline
\end{tabular}
\end{table}


EEG/iEEG signals require low-level preprocessing to address challenges such as noise and artifact removal, normalization for consistency, and segmentation into analyzable time windows.
These steps refine raw data, ensuring it accurately reflects brain activity and provides a robust foundation for analysis.
Representative methods are summarized in Table~\ref{tab:lowlevel}, with one key work per category highlighted.
%given space constraints.}

\subsubsection{Noise Reduction and Filtering}
Filtering techniques, such as Finite Impulse Response (FIR) and Infinite Impulse Response (IIR) filters, are employed to isolate specific frequency components. Advanced methods like MSEC noise reduction and wavelet transforms~\cite{ay2019automated} provide specialized solutions for effective denoising and precise data refinement.

\subsubsection{Artifact Removal} 
Artifact removal strategies include Blind Source Separation techniques, such as Independent Component Analysis (ICA), Principal Component Analysis (PCA), and Multiple Component Analysis (MCA)~\cite{9047940}, along with Artifact Correction methods, including Ocular Correction and Artifact Subspace Reconstruction. Wavelet decomposition is also commonly used mitigate artifacts.

\subsubsection{Baseline Correction and Detrending} 
Baseline correction and detrending address baseline drift caused by eye movements, breathing, and subject motion. Baseline correction standardizes power data, baseline removal reduces subject-independent noise, and detrending eliminates linear or nonlinear trends, enhancing signal reliability.

\subsubsection{Channel Processing} 
Neurological disorders often affect specific brain regions, making channel processing techniques, including selection, mapping, and re-referencing, essential for enhancing the specificity and interpretability of analyses.

\subsubsection{Normalization and Scaling}
Normalization standardizes the amplitude of raw EEG and iEEG signals, which often vary in voltage. Common methods include Z-score and quantile normalization, while linear scaling and shifting minimize spurious amplitude variations across channels.

\subsubsection{Sampling Adjustment}
Sampling adjustments optimize data for analysis while reducing computational demands. Downsampling reduces memory and processing requirements, whereas interpolation handles missing data, insufficient training samples, and pulse artifacts~\cite{Tuncer2020ANE}.

\subsubsection{Segmentation}
Segmentation divides EEG and iEEG data into smaller sections for localized information extraction and data augmentation to enhance sample diversity.
Overlap windows ensure continuity and capture transitional features, while non-overlapping segments prioritize computational efficiency and maintain distinct temporal boundaries.
% Techniques like epoching and sliding windows enhance analysis depth, though they may increase computational complexity.

\subsubsection{Signal Alignment and Synchronization}
Signal alignment ensures temporal consistency across signals from different sources, improving the reliability of findings. Fine-grained temporal alignment further corrects residual discrepancies after initial synchronization, ensuring data precision.



\subsection{Feature Extraction}

\begin{table}[t]
\renewcommand{\arraystretch}{1.2}
\caption{Feature Extraction Techniques}
\label{tab:highlevel}
\footnotesize
\centering
\begin{tabular}{p{80pt}p{90pt}p{30pt}} % 第一列和第二列设置固定宽度,自动换行
\hline
\textbf{Techniques}     & \textbf{Details}                 & \textbf{Reference}             \\
\hline
\multirow{2}{80pt}{Data Augmentation} & Oversampling                     & ~\cite{ZHANG2020105089}         \\
                                       & ELM-AE                           & ~\cite{9713847}                 \\
\hline
\multirow{2}{80pt}{Signal Decomposition \& Transformation} 
                                       & Time-Frequency Analysis          & ~\cite{PD2} \\
                                       & Empirical Decomposition          & ~\cite{zulfikar2022empirical}   \\
\hline
\multirow{3}{80pt}{Spectral \& Power Analysis}  
                                       & Power Spectrum                   & ~\cite{li2019eeg}               \\
                                       & Spectral Density                 & ~\cite{Seizure214}        \\
                                       & Partial Directed Coherence       & ~\cite{khan2021automated}       \\
\hline
\multirow{3}{80pt}{Time-Domain Features Extraction}  
                                       & Statistical Measures             & ~\cite{zhu2019multimodal}       \\
                                       & Amplitude \& Range               & ~\cite{Seizure7}    \\
                                       & Hjorth Parameters                & ~\cite{li2019depression}             \\
\hline
\multirow{2}{80pt}{Frequency-Domain Features Extraction}  
                                       & Band Power Features              & ~\cite{Seizure67}        \\
                                       & Spectral Measures                & ~\cite{tosun2021effects}        \\
\hline
\multirow{3}{80pt}{Time-Frequency Features Extraction}  
                                       & Wavelet Coefficients             & ~\cite{aslan2022deep}           \\
                                       & STFT Features                    & ~\cite{choi2019novel}           \\
                                       & Multitaper Spectral              & ~\cite{vilamala2017deep}        \\
\hline
\multirow{3}{80pt}{Other Features Extraction}  
                                       & Nonlinear Features               & ~\cite{Seizure109}        \\
                                       & Spatial Features                 & ~\cite{phang2019multi} \\
                                       & Transform-Based Features         & ~\cite{electronics11142265} \\
\hline
\multirow{2}{80pt}{Graph Analysis}                                                         & Clustering Coefficient           & ~\cite{Zhan2020EpilepsyDetection}       \\
                                       & Other Graph Metrics              & ~\cite{ho2023self}   \\
\hline
\end{tabular}
\end{table}

Feature extraction techniques reconfigure data into alternative representations by isolating key features or decomposing it into core components essential for modeling and analysis. This process effectively primes the data for more sophisticated, abstract analytical tasks.
Representative methods are summarized in Table~\ref{tab:highlevel}, with one key work per category highlighted.
%given space constraints.}

\subsubsection{Data Augmentation}
Data augmentation generates new samples to increase dataset diversity and improve classification accuracy and stability. Oversampling is commonly used to address class imbalances, while the Extreme Learning Machine Autoencoder (ELM-AE) employs autoencoders to synthesize data by reconstructing input features~\cite{9713847}.

\subsubsection{Spectral and Power Analysis}
Spectral and power analysis focuses on examining the frequency components and energy distribution of signals.
Key techniques include power spectrum calculation, frequency band energy analysis, and partial-directed coherence for evaluating signal causality.

\subsubsection{Time Domain Feature Extraction}
Time domain features, such as statistical measures, Hjorth parameters, and Zero-Crossing Rate, effectively represent signal amplitude, time scale, and complexity. These features provide valuable insights into signal distribution, intensity, and rate of change.

\subsubsection{Frequency Domain Feature Extraction}
Frequency domain features, such as band power, band energy, median frequency, spectral edge frequency, and power spectral density (PSD), provide insights into the spectral content of signals.

\subsubsection{Time-Frequency Feature Extraction}\
Time-frequency features capture both temporal and spectral information, providing a comprehensive signal representation.
Short-time Fourier Transform (STFT) analyzes frequency variations over time, while Continuous and Discrete Wavelet Transforms (CWT, DWT) offer detailed time-frequency representations.
Advanced techniques like FBSE-EWT filter banks~\cite{9096344} and Smoothed Pseudo Wigner Ville Distribution (SPWVD)~\cite{Ebrahimzadeh2013ANA} enhance analysis precision.

\subsubsection{Other Feature Extraction}
Nonlinear features, such as entropy measures, fractal dimensions, and Lyapunov exponents, capture complex patterns that linear methods may miss. Spatial features, including Common Spatial Patterns and connectivity measures like phase-locking value (PLV) and phase-lag index (PLI), represent spatial domain activities. 
Transform-based features further enhance analysis by reconstructing signals into more informative representations.

\subsubsection{Signal Decomposition and Transformation}
Signal decomposition and transformation techniques decompose complex signals to facilitate detailed analysis, such as wavelet transforms, Gabor Transform~\cite{PD2}, Fast Fourier Transform, Empirical Mode Decomposition, and Hilbert-Huang Transform.

\subsubsection{Graph Analysis}
Graph analysis evaluates connectivity between channels. Metrics like degree measure connections and node importance, while the clustering coefficient quantifies local network density, revealing network structure.


\subsection{Data Partitioning Strategies}

Building on the detailed definition of \(\mathbf{X}^{(i)} \in \mathbb{R}^{C \times T}\) in Section~\ref{sec:pdef}, where \(\mathbf{X}^{(i)}\) represents the EEG or iEEG signal of subject \(i\), we further introduce additional notations to formalize the data partitioning strategies:

\begin{itemize}
    \item \(\mathcal{P} = \{1, 2, \dots, N\}\): The set of \(N\) subjects in the dataset.
    \item \(\mathcal{X}_\text{train}, \mathcal{X}_\text{val}, \mathcal{X}_\text{test}\): The training, validation, and testing sets, respectively.
    \item \(\alpha_\text{train}, \alpha_\text{val}, \alpha_\text{text}  \in (0, 1)\): The proportion of data used for training, validation and test, and \(\alpha_\text{train} + \alpha_\text{val} + \alpha_\text{test} = 1.\)
    \item \(K^{(i)}\): The total number of temporal segments or events derived from subject \(i\)'s data.
\end{itemize}

Using these definitions, we classify data partitioning strategies into three categories: subject-specific methods, mixed-subject methods, and cross-subject methods.

\subsubsection{Subject-Specific Methods}
Subject-specific methods focus on capturing individual characteristics by partitioning each subject’s data independently into training, validation, and testing sets. Formally,
\[
\mathcal{X}_\text{train} \cup \mathcal{X}_\text{val} \cup \mathcal{X}_\text{test} = \{\mathbf{X}_k^{(i)}\}_{k=1}^{K^{(i)}},
\]
where \(i\) denotes a specific subject. This method is particularly useful in the early stages of development, as it enables rapid iteration on small datasets and captures individual patient patterns. It is commonly used in closed-loop seizure detection systems, where personalization is critical.

\subsubsection{Mixed-Subject Methods}
Mixed-subject methods leverage signals from all subjects in \(\mathcal{P}\) for training, validation, and testing, aiming to create models with broad applicability. 
The data partitioning method is as follows:
\[
\mathcal{X}_\text{set} \subset \bigcup_{i \in \mathcal{P}} \bigcup_{k=1}^{K^{(i)}} \{\mathbf{X}^{(i)}_k\}, \quad
|\mathcal{X}_\text{set}| = \alpha_\text{set} \sum_{i=1}^{N} K^{(i)},
\]
where \(\text{set} \in \{\text{train}, \text{val}, \text{test}\}\).
By pooling data across subjects, this approach maximizes training efficiency and improves the model’s robustness to inter-subject variability. However, it also introduces the risk of data leakage, as segments from the same subject may appear in different sets.

\subsubsection{Cross-Subject Methods}
Clinical applications demand models that generalize across unseen patients. Cross-subject methods explicitly enforce subject separation between training, validation, and testing by partitioning $\mathcal{P}$ into disjoint subsets:

\[
|\mathcal{P}_\text{set}| = \alpha_\text{set} |\mathcal{P}|, \quad
\mathcal{X}_\text{set} = \bigcup_{i \in \mathcal{P}_\text{set}} \bigcup_{k=1}^{K^{(i)}} \{\mathbf{X}^{(i)}_k\},\
\]
where \(\text{set} \in \{\text{train}, \text{val}, \text{test}\}\).
This ensures that models are evaluated on entirely unseen subjects. 
%Among the three approaches, cross-subject partitioning is the most clinically relevant, aligning closely with real-world deployment scenarios.

Extending subject-level partitioning strategies, dataset-level partitioning includes three approaches: \textbf{dataset-specific} (independent partitioning per dataset), \textbf{mixed-dataset} (pooling data across datasets), and \textbf{cross-dataset} (disjoint datasets for training, validation, and testing). Dataset-specific methods capture individual dataset characteristics, while mixed-dataset methods enhance robustness to inter-dataset variability. Cross-dataset partitioning is crucial for universal models, rigorously assessing generalization and closely aligning with real-world clinical deployment.

\subsection{Deep Learning Architectures}
Neurological data processing relies on several key architectures:
\textbf{Convolutional Neural Networks (CNNs)}~\cite{lecun1995convolutional} excel at extracting spatial/spectral features through hierarchical convolutions.
\textbf{Recurrent Neural Networks (RNNs)}~\cite{elman1990finding} capture temporal dependencies via recurrent connections.
\textbf{Transformers}~\cite{vaswani2017attention} model long-range spatiotemporal relationships using self-attention.
\textbf{Graph Neural Networks (GNNs)}~\cite{4700287} analyze functional connectivity in graph-structured data.
\textbf{Autoencoders (AEs)}~\cite{hinton1993autoencoders} learn compressed representations through encoder-decoder structures.
\textbf{Generative Adversarial Networks (GANs)}~\cite{goodfellow2014generative} synthesize signals through adversarial training.
\textbf{Spiking Neural Networks (SNNs)}~\cite{maass1997networks} leverage spike-based computation for temporal dynamics.


\subsection{Deep Learning Paradigms}
Deep learning applications in neurological diagnostics can be categorized into four paradigms: supervised learning, self-supervised learning, unsupervised learning, and semi-supervised learning.
Each paradigm addresses specific challenges in processing brain signals by leveraging architectures tailored to data availability and task requirements.
These paradigms will be further discussed in detail in Section~\ref{sec:app}.

\subsubsection{Supervised Learning}
Supervised learning is the dominant paradigm for neurological diagnostics tasks, training models to map signals $ \mathbf{X} \in \mathbb{R}^{C \times T} $ to labels $y \in \mathcal{Y} $.
%Common architectures include Convolutional Neural Networks (\textbf{CNNs})~\cite{lecun1995convolutional} for feature extraction,
%Recurrent Neural Networks (\textbf{RNNs})~\cite{elman1990finding} for modeling temporal dependencies,
%and \textbf{Transformers}~\cite{vaswani2017attention} for handling complex spatiotemporal relationships with attention mechanisms.

\subsubsection{Unsupervised Learning}
Unsupervised learning is essential for uncovering intrinsic data structures in signals $\mathbf{X}$, enabling representation learning without relying on labels. 
%Common architectures include Autoencoders (\textbf{AE})~\cite{hinton1993autoencoders} for encoding features and reducing dimensionality, Generative Adversarial Networks (\textbf{GANs})~\cite{goodfellow2014generative} for synthesizing realistic data in low-resource scenarios,
%and Spiking Neural Networks (\textbf{SNNs})~\cite{maass1997networks} for modeling  spatiotemporal dynamics in signals.

\subsubsection{Semi-Supervised Learning}
Semi-supervised learning combines a small set of labeled examples $\{(x_i, \hat{y}_i)\}_{i=1}^l$, where $\hat{y}_i$ denotes the provided labels, with a larger set of unlabeled examples $\{x_j\}_{j=l+1}^{l+u}$ to learn a mapping from $\mathbf{X}$ to $\mathcal{Y}$. 
%Common architectures include Graph Neural Networks (\textbf{GNNs})~\cite{4700287}, \textbf{CNNs}, and \textbf{RNNs}. 

\subsubsection{Self-Supervised Learning}
Self-supervised learning (SSL) leverages unlabeled EEG/iEEG data by constructing pretext tasks that generate pseudo-labels $\hat{y}$ from intrinsic properties of the raw signals $\mathbf{X}$. These tasks enable models to learn robust representations, which can be fine-tuned for downstream tasks.
SSL methods fall into three main categories: contrastive, predictive, and reconstruction-based learning.
\textbf{Contrastive-based methods}, such as Contrastive Predictive Coding (CPC)~\cite{banville2021uncovering} and Transformation Contrastive Learning~\cite{mohsenvand2020contrastive}, learns by maximizing similarity between related views while minimizing it between unrelated ones, capturing distinguishing signal features.
\textbf{Predictive-based learning} employs pretext tasks such as Relative Positioning and Temporal Shuffling to extract structural patterns across temporal, frequency, and spatial domains~\cite{banville2019self, oord2018representation}. By predicting transformations applied to the data, it enhances domain-specific feature learning.
\textbf{Reconstruction-based learning} trains models to reconstruct masked signal segments. Methods like Masked Autoencoders (MAE) reconstruct temporal or spectral components, learning intrinsic patterns in the process~\cite{Kostas2021BENDR, wu2022neuro2vec}.
Studies have also explored hybrid methods, which combine elements from contrastive, predictive, and reconstruction-based approaches~\cite{cai2023mbrain, banville2021uncovering}.
%\textbf{Transformers} and \textbf{CNNs} are the primary architectures used in this paradigm.
\section{Applications}
\label{sec:app}



\begin{table*}[t]
\centering
\renewcommand{\arraystretch}{1.2}
\caption{Public EEG/iEEG datasets for seizure detection, with \textbf{Seizures} indicating the number of episodes, \textbf{Length} the duration of each record, and \textbf{Size} the total duration of recording.}
\label{tab:ep}
\footnotesize
\begin{tabular}{lccccccc}
\hline
\textbf{Dataset} & \textbf{Type} &  \textbf{Subjects} &  \textbf{Seizures} & \textbf{Length} & \textbf{Size} &  \textbf{Frequency (Hz)} &  \textbf{Channels} \\ \hline
Bonn~\cite{andrzejak2001indications}             & EEG & {10}                                 & {-}       & {23.6 sec}          & {$\approx$ 3.3 hours}                           & {173.61}                   & {1}               \\
Freiburg~\cite{ihle2012epilepsiae}         & iEEG & {21}            & {87}         & {4 sec}         & {$\approx$ 504 hours }         & {256}                      & {128}               \\
Mayo-UPenn~\cite{seizure-detection}       & iEEG & {2}                                  & {48}                    & {1 sec}           & {583 min}                         & {500-5000}                 & {16-76}             \\
CHB-MIT~\cite{guttag2010chb,shoeb2009application,goldberger2000physiobank}          & EEG & {22}                                 & {198}          & {1 hour}                    & {$\approx$ 686 hours}                    & {256}                      & {23 / 24 / 26}         \\
Bern-Barcelona~\cite{andrzejak2012nonrandomness}   & iEEG & {5}                                  & {3750}              & {20 sec}               & {57 hours}                        & {512}                      & {64}                \\
Hauz Khas~\cite{hauz}        & EEG & {10}                                 & {-}       & {5.12 sec}          & {87 min}                           & {200}                      & {50}                \\
Melbourne~\cite{melbourne}        & iEEG & {3}                                  & {-}      & {10 min}             & {81.25 hours}                            & {400}                      & {184}               \\
TUSZ~\cite{shah2018temple}             & EEG & {642}                                & {3050}               & {-}              & {700 hours}                         & {250}                      & {19}                \\
SWEC-ETHZ~\cite{burrello2018oneshot,burrello2019hdc} 
        & iEEG           & 18 / 16 & 244 / 100 & 1 hour / 3 min & 2656 hours / 48 min & 512 / 1024 & 24-128 / 36-100 \\
Zenodo~\cite{stevenson2019dataset}         & EEG & {79}                                 & {1379}                  & {74 min}           & {$\approx$ 97 hours}                    & {256}                      & {21}                \\
Mayo-Clinic~\cite{Nejedly2020}      & iEEG & {25}                                 & {-}           & {3 sec}      & {50 hours}                          & {5000}                     & {1}                 \\
FNUSA~\cite{Nejedly2020}           & iEEG & {14}                                 & {-}        & {3 sec}         & {7 hours}                           & {5000}                     & {1}                 \\
Siena~\cite{detti2020eeg}            & EEG & {14}                                 & {47}                  & {145-1408 min}             & {$\approx$ 128  hours}                         & {512}                      & {27}                \\
Beirut~\cite{nasreddine2021epileptic}           & EEG & {6}                                  & {35}        & {1 sec}                       & {130 min}                          & {512}                      & {19}                \\
HUP~\cite{HUP}              & iEEG & {58}                                 & {208}                       & {300 sec}       & {$\approx$ 27 hours}                    & {500}                      & {52-232}            \\
CCEP~\cite{ds004080:1.2.4} & iEEG & {74} & {-} & {-} & {89 hours} & {2048} & {48-116} \\  \hline
\end{tabular}
\end{table*}

This section systematically reviews neurological disease diagnosis methodologies. Each subsection starts with an introduction to the disease’s background, including its characteristics, diagnostic tasks, and relevant public datasets. We will then review representative works for each disease, highlighting disease-specific features in the context of deep learning-based diagnosis, such as data types, frequency bands, brain regions, and methodological trends. 
Given their extensive research history, seizure detection and sleep staging receive dedicated sections, while other disorders are analyzed through focused comparative discussions to eliminate redundancy. Technical implementation details across studies (preprocessing pipelines, network architectures, training protocols) are systematically cataloged in supplementary tables.

\subsection{Seizure Disorder}


\subsubsection{Task Description}
Epilepsy, a neurological disorder affecting 50 million people globally, is characterized by recurrent seizures caused by abnormal brain activity. 
Seizures range from brief confusion or blanking out to severe convulsions and loss of consciousness. According to the World Health Organization (WHO), up to 70\% of epilepsy cases can be effectively treated with proper care. However, in low-income regions, limited resources and stigma often hinder access to treatment, heightening the risk of premature death\cite{WHO_epilepsy}.

Seizure detection primarily relies on standardized EEG/iEEG datasets, summarized in Table~\ref{tab:ep}. The key challenge is distinguishing seizure events from background activity, typically framed as binary classification where $y_k \in \{0, 1\}$. 
Most approaches segment long EEG sequences into smaller windows for sample-level classification, aggregating segment predictions to form event-level outcomes as $\mathbf{Y} = \bigcup_{k=1}^K \{y_k\}$~\cite{xu2023patient,peng2023wavelet2vec}. 
Another approach detects optimal cut points within continuous recordings to identify the boundaries of meaningful segments $\{\mathbf{X}_k\}_{k=1}^K$, and each segment is classified individually~\cite{Zhan2020EpilepsyDetection}. 
The final event-level prediction is obtained by combining these event-level labels $\mathbf{Y} = \bigcup_{k=1}^K \{\Phi_{\text{segment}}(\mathbf{X}_k; \boldsymbol{\theta})\}$.
%In contrast, datasets like Bonn consist of pre-segmented short EEG recordings, typically used for sample-level classification, which simplifies the problem by eliminating the need for segmentation.

More detailed classifications have also been explored, including three-class tasks, where $y_k \in \{\text{A}, \text{D}, \text{E}\}$ represents interictal (A, the period between seizures), preictal (D, the time before seizure onset), and ictal (E, seizure) states~\cite{zhou2018epileptic}. Five-class tasks refine this further by subdividing the preictal state into early, middle, and late stages~\cite{turk2019epilepsy}. The Temple University Seizure Corpus (TUSZ)~\cite{shah2018temple} supports detailed epilepsy studies, classifying events into pathological patterns like epileptiform discharges and seizure types (e.g., focal, generalized, tonic-clonic), as well as non-pathological signals such as background activity and artifacts (e.g., eye movements).
A detailed overview of all related works is provided in Appendix Table~\ref{tab:seizures}.


\subsubsection{Supervised Methods}
Supervised seizure detection using EEG/iEEG data has advanced alongside growing datasets and improved technology. Early studies relies on subject-specific or mixed-subject evaluations using short, pre-segmented EEG clips. For example, the Bonn dataset~\cite{andrzejak2001indications} consists of manually labeled seizure/non-seizure segments, leading to models optimized for fixed-length inputs. Approaches based on raw signals employ CNNs or RNNs to automatically extract spatiotemporal features from these standardized segments~\cite{ACHARYA2018270,ULLAH201861}, while feature-based methods derive handcrafted or transformed representations, such as scalograms~\cite{turk2019epilepsy} and wavelet-based features~\cite{Seizure58}, which are more suited for shallow classifiers. These techniques inherently assume limited temporal context and avoided segmentation challenges.

With the adoption of long-term recordings like CHB-MIT~\cite{shoeb2009application}, the focus shifts toward cross-subject paradigms. These datasets provide extensive seizure examples within continuous, long-term EEG streams, necessitating more flexible detection frameworks capable of handling variable-length inputs and identifying seizure boundaries in unsegmented data. Approaches integrate temporal modeling through sliding windows~\cite{xu2023patient}, sequence-aware architectures such as Transformers~\cite{lih2023epilepsynet}, or hybrid feature fusion techniques~\cite{dutta2024deep}. Concurrently, cross-subject validation becomes standard, reflecting clinical requirements that generalize across diverse conditions.

The necessity of cross-subject modeling in seizure detection stems from its critical role in ensuring clinical generalization.
The invasive nature of iEEG fundamentally differentiates its modeling requirements from EEG through distinct acquisition paradigms and neurophysiological characteristics, as its patient-specific recording conditions and electrode configurations lead to substantial inter-subject heterogeneity in temporal features and spatial sampling properties, unlike EEG's standardized scalp placement~\cite{zhang2024brant}. Balancing high-resolution spatiotemporal capture with robustness across patients, iEEG requires specialized methodologies to enhance generalizability while addressing its inherent complexities.
Spatial modeling is essential for capturing three-dimensional epileptogenic networks with depth electrodes. Graph-based methods model inter-channel dependencies via neuroanatomical~\cite{9345750} or dynamic functional connections~\cite{rahmani2023meta}, while Transformer architectures use attention mechanisms to adapt to varying electrode configurations~\cite{sun2022continuous}.
DMNet~\cite{tudmnet} improves domain generalization through self-comparison mechanisms.


\subsubsection{Semi- and Unsupervised Methods}
Semi-supervised and unsupervised learning techniques have become increasingly applied in deep learning for seizure detection, particularly when labeled data is limited. 
A common approach incorporates clustering paradigms for event-level segmentation, allowing the model to identify and segment seizure events~\cite{Zhan2020EpilepsyDetection}.
Another notable application involves using models such as Autoencoders, DBNs and GANs to automatically extract relevant features or augment the dataset, thereby enhancing the model’s robustness and generalizability~\cite{abdelhameed2018epileptic,turner2014deep,you2020unsupervised}. 


\subsubsection{Self-supervised Methods}
Self-supervised learning has emerged as an effective approach for seizure detection. 
Contrastive learning captures seizure-related patterns by forming positive pairs through segment augmentation and negative pairs based on feature differences.
For example, SLAM~\cite{XIAO2024105464} generates negative pairs by pairing the anchor with a randomly selected window from a distant time point. SPP-EEGNET~\cite{li2022spp} calculates the absolute difference between pairs to classify them as positive or negative.
Wagh et al.~\cite{wagh2021domain} employs cross-domain contrastive learning to mitigate individual differences by comparing subjects based on factors such as age. They use the delta/beta power ratio to estimate EEG-based behavioral states and distinguish pre- and post-seizure characteristics.
Zheng et al.\cite{zheng2022task} employ predictive-based SSL by designing classification pretext tasks that simulate key epileptic features, such as increased amplitude and abnormal frequencies, enabling the model to recognize epilepsy-related patterns. 
Tang et al.\cite{tang2021self} first combine graph-based modeling with pre-training for EEGs, where the model predicts the next set of EEG signals for a given time period.

EpilepsyNet~\cite{lih2023epilepsynet} employs reconstruction-based SSL, using Pearson Correlation Coefficients to capture spatial-temporal embeddings while preserving contextual features.
Wavelet2Vec~\cite{peng2023wavelet2vec} utilizes a frequency-aware masked autoencoder that reconstructs wavelet-transformed EEG patches in the time-frequency domain. By leveraging seizure-specific abnormal discharge patterns across frequency bands, it enhances feature extraction for seizure subtype classification.
EEG-CGS\cite{ho2023self} adopts a hybrid graph-based SSL approach, framing seizure detection as anomaly detection, integrating random walk-based subgraph sampling with contrastive and reconstruction-based learning.

The SSL paradigm is also commonly used in iEEG-based modeling. 
BrainNet~\cite{chen2022brainnet} employs bidirectional contrastive predictive coding to capture temporal correlation in SEEG signals.
MBrain~\cite{cai2023mbrain} models time-varying propagation patterns and inter-channel phase delays characteristic of epileptic activity through a multivariant contrastive-predictive learning framework, leveraging graph-based representations for spatial-temporal correlations across EEG and SEEG channels.
PPi~\cite{yuan2024ppi} accounts for regional seizure variability, employing a channel discrimination task to ensure the model captures distinct pathological patterns across brain regions rather than treating all channels uniformly.
%Besides, the key to joint training of EEG and iEEG lies in balancing the temporal and spatial information from both signal types. 

\begin{table}[t]
\renewcommand{\arraystretch}{1.2}
\caption{Public Sleep EEG Datasets, where \textbf{Recordings} denotes the number of whole-night PSG recordings.}
\label{tab:sleep}
\footnotesize
\centering
\begin{tabular}{lccc}
\hline
\textbf{Dataset}      & \textbf{Recordings}                     & \textbf{Frequency (Hz)} & \textbf{Channels} \\
\hline
Sleep-EDF~\cite{kemp2000analysis,goldberger2000physiobank}             & 197                 & 100          & 2                 \\
MASS~\cite{oreilly2014montreal}                  & 200                 & 256          & 4-20         \\
SHHS~\cite{quan1997sleep,zhang2018national}                  & 8362                     & 125          & 2                 \\
SVUH\_UCD~\cite{ucddb2007sleep,goldberger2000physiobank}              & 25              & 128          & 3                 \\
HMC~\cite{Alvarez-Estevez2022, goldberger2000physiobank} & 151 & 256 & 4\\
PC18~\cite{ghassemi2018you,goldberger2000physiobank}                  & 1985                     & 200          & 6                 \\
MIT-BIH~\cite{ichimaru1999development,goldberger2000physiobank}              & 16                   & 250          & 1                 \\
DOD-O~\cite{dod_dataset}                   & 55                   & 250          & 8                 \\
DOD-H~\cite{dod_dataset}                   & 25                   & 250          & 12                 \\
ISRUC~\cite{khalighi2016isruc}              & 126                      & 200          & 6                 \\
MGH~\cite{biswal2018expert}                  & 25941                   & 200          & 6                 \\
Piryatinska~\cite{piryatinska2009automated}           & 37         & 64           & 1                 \\
DRM-SUB~\cite{devuyst2005dreams} & 20 & 200 & 3 \\
SD-71~\cite{xiang2023resting} & 142 & 500 & 61 \\
\hline
\end{tabular}
\end{table}

\subsection{Sleep Staging}

\subsubsection{Task Description}
Sleep staging is critical to understanding sleep disorders like insomnia and sleep apnea, as well as the impact on overall health.
It is estimated that 20\% to 41\% of the global population is affected by sleep disorders, which are linked to an increased risk of obesity, cardiovascular diseases, and mental health issues~\cite{recoveryvillage_sleep_statistics_2023}. 
Therefore, accurately identifying sleep stages is essential for addressing these concerns.

Sleep staging involves segmenting signals into 30-second epochs and classifying them into stages: awake (W), rapid eye movement (REM), and three non-REM (NREM) stages (N1, N2, N3).
Wake is characterized by high-frequency $\beta$ and $\alpha$ waves. In N1, the transition from wakefulness to sleep, low-amplitude $\theta$ waves appear. N2, the light sleep stage, is marked by sleep spindles and K-complexes associated with sensory processing and memory consolidation. N3, or deep sleep, features slow-wave $\delta$ activity. REM sleep, essential for emotional regulation and dreaming, is characterized by rapid, low-voltage brain activities.

Multimodal modeling is fundamental for sleep analysis, as polysomnography (PSG) integrates EEG (e.g., Fpz-Cz, Pz-Oz), Electrooculography (EOG), and Electromyography (EMG) to enhance staging accuracy.
The public datasets listed in Table~\ref{tab:sleep} are frequently employed in sleep analysis. 
A detailed overview of all related works is provided in Appendix Table~\ref{tab:sleeps}.


%The Sleep-EDF dataset ~\cite{kemp2000analysis}, which includes polysomnography (PSG) recordings annotated by sleep stage at 30-second intervals, has been used in more than 70\% of related studies, with multiple incremental releases.

\subsubsection{Supervised methods}

Selecting biosignal modalities is critical for designing supervised learning frameworks in PSG-based sleep staging. Two primary paradigms are widely used. Single-channel EEG methods, preferred in resource-constrained settings, offer hardware simplicity, reduced cross-modal interference, and enhanced computational efficiency~\cite{tsinalis2016automatic, eldele2021attention}. However, relying solely on EEG limits the detection of complementary cues—such as ocular and muscular activities—essential for identifying ambiguous sleep stages like REM sleep.
Multimodal architectures integrating EEG, EOG, and EMG signals emulate the integrative analysis performed by sleep experts~\cite{chambon2018deep, alvarez2021inter}. Chambon et al.~\cite{chambon2018deep} employs techniques like spatial filtering to mitigate cross-modal interference. These designs align with clinical scoring protocols and compensate for the limited contextual information of individual modalities. Beyond these dominant approaches, hybrid models, such as EEG-EOG, balance diagnostic accuracy with computational efficiency~\cite{Sleep28}. 

%Ultimately, selecting appropriate signals requires balancing physiological comprehensiveness with operational practicality.
%The strong connection between sleep stages and frequency-specific EEG patterns highlights the importance of capturing spatial and spectral information in supervised learning. 
%CNN-based methods have proven effective for sleep staging, with early approaches demonstrating their potential and later advancements incorporating complex-valued processing and multimodal pipelines to handle heterogeneous signals~\cite{tsinalis2016automatic,eldele2021attention}.
%However, CNNs often struggle to capture sequential dependencies, whereas RNN-based models utilize sequence-to-sequence architectures to capture both short- and long-term dependencies~\cite{phan2018automatic, phan2019seqsleepnet}. 
%More recently, Transformers have become increasingly popular in sleep staging due to their capacity to efficiently focus on relevant temporal features using attention mechanisms~\cite{yao2023cnntransformer}. 

%Multimodal modeling leverages complementary insights from EEG, EOG, and EMG to enhance the accuracy and robustness of sleep staging.Cross-modal fusion, temporal modeling, and modality-specific feature extraction pipelines are frequently implemented in successful multimodal models to integrate features and capture sleep stage transitions. 
%Recent advances further address challenges in multimodal alignment through unified frameworks like Brant-X~\cite{zhang2024brantx}, which leverages EEG foundation models to transfer knowledge to other physiological signals (e.g., EOG, EMG) via contrastive learning. By aligning EEG and EXG signals at both the patch-level, which captures local waveform patterns, and the sequence-level, which captures global temporal dynamics, Brant-X effectively bridges the semantic gaps between different modalities.
%Linear spatial filtering, for example, can improve the quality of signals, while attention mechanisms or cross-modal Transformers can efficiently combine and evaluate modality-specific contributions. 
%Multimodal models are essential for the comprehensive staging of sleep signals, ensuring that they fully leverage the diversity of PSG data.

%Beyond these, graph-based methods addressed the limitations of conventional networks by modeling EEG channels as nodes in a graph, dynamically learning relationships between channels. This line of work progressed from focusing solely on functional connectivity to incorporating physical proximity and domain generalization, capturing richer spatial-temporal patterns in the data \cite{jia2020graphsleepnet,jia2021multi}. Together, these advancements highlight a shift from generic deep architectures to more specialized models that reflect the intrinsic complexities of sleep signals.

\subsubsection{Self-supervised methods}
Self-supervised contrastive-based methods enhance sleep representation by leveraging temporal and contextual patterns in unlabeled EEG data.
Early works explore tasks such as relative positioning, temporal shuffling, and autoregressive latent feature predictions to extract temporal structures from multivariate signals~\cite{banville2019self, oord2018representation}. 
Jiang and et al.~\cite{jiang2021self} extends these efforts with augmentation-based contrastive learning, generating positive and negative pairs from augmented EEG segments.
ContraWR~\cite{yang2023self} adopts constructing contrastive pairs from distinct time windows, prioritizing window-level temporal dependencies.
%Additionally, domain-guided methods integrats auxiliary tasks such as hemispheric symmetry and age estimation, capturing physiologically meaningful features~\cite{wagh2021domain}. 
%Recent methods have expanded contrastive frameworks by incorporating multi-view and domain-specific learning. For example, 
mulEEG~\cite{kumar2022muleeg} and CoSleep~\cite{ye2021cosleep} introduce multi-view contrastive strategies to integrate time-series and spectrogram representations of EEG data. 
mulEEG emphasizes cross-view consistency while encouraging modality-specific features, whereas CoSleep develops a time-frequency dual-view contrastive learning framework that implicitly captures sleep-staging-related temporal dynamics and spectral rhythmic patterns in EEG signals.

Multimodal modeling improves sleep staging accuracy by integrating complementary EEG, EOG, and EMG insights. Brant-X~\cite{zhang2024brantx} address alignment challenges using EEG foundation models and contrastive learning. By aligning EEG and EXG signals at both the local and global levels, Brant-X effectively bridges the semantic gaps between modalities.
%Approaches like attention-driven latent signal manipulation and multi-instance contrastive learning further enable models to uncover nuanced signal variations without labeled supervision~\cite{lee2022self, xiao2021self}.

%Masked modeling methods aim to compel the model to learn intrinsic data patterns.% by reconstructing missing portions of the signal. 




%\subsubsection{Semi- and Unsupervised Methods}
%Early unsupervised methods focused on automatically discovering meaningful representations from raw physiological signals without manual feature engineering, enabling models to capture subtle patterns inherent to sleep data \cite{langkvist2012sleep,zhang2016automatic}. Such approaches explored variants that emphasize feature sparsity, inspired by biological vision systems, which improved the model’s sensitivity to intricate EEG characteristics and enhanced classification outcomes \cite{lee2007sparse,zhang2016automatic}.

%Subsequent work introduced complex-valued inputs and unsupervised filter refinement, allowing networks to handle more nuanced signal attributes \cite{zhang2018complex}. Further innovations incorporated competition-based learning to bolster robustness against noise and better reflect the competitive dynamics of neural representations, offering more stable and effective sleep stage discrimination \cite{zhang2021competition}.

%\subsubsection{Semi-supervised methods}
%Manually labeled sleep recordings are scarce, but raw sleep signals are easier to acquire, prompting the development of semi-supervised models for sleep stage classification. For example, a semi-supervised Gaussian Mixture Model (SS-GMM) achieves comparable accuracy to fully-supervised models when more than 50\% of the data is labeled~\cite{munk2018semi}. Pseudo-labeling is another common approach, where labels are assigned to unlabeled data based on predicted probabilities~\cite{haoran2021semi}.

%More advanced methods include a bi-stream adversarial network (BiSALnet) that employs a Student-Teacher architecture with adversarial training to generate high-confidence pseudo-labels, improving classification performance~\cite{li2022adversarial}. Building on this, a multi-task contrastive learning strategy (MtCLSS) further enhances representation learning for sleep staging~\cite{li2022mtclss}.

\subsection{Depression Identification}

\begin{table}[t]
\renewcommand{\arraystretch}{1.2}
\caption{Public EEG Datasets for Depression Detection, where \textbf{Exp (n)} represents the number of depressed individuals and \textbf{Ctrl (n)} represents the healthy control group.}
\label{tab:dep}
\footnotesize
\centering
\begin{tabular}{lcccc}
\hline
\textbf{Dataset}     & \textbf{Exp (n)}                  & \textbf{Ctrl (n)} & \textbf{Frequency (Hz)} & \textbf{Channels} \\
\hline
HUSM~\cite{Mumtaz2016}      & 34                            & 30               & 256              & 22                \\
PRED+CT~\cite{cavanagh2017patient} & 46                            & 75               & 500              & 64                \\
EDRA~\cite{yang2023automatic} & 26                            & 24               & 500              & 63                \\
MODMA~\cite{cai2022multi}    & \begin{tabular}[c]{@{}l@{}}24 \\ 26 \end{tabular} & \begin{tabular}[c]{@{}l@{}}29\\ 29\end{tabular} & 250              & \begin{tabular}[c]{@{}l@{}}128\\ 3\end{tabular} \\
\hline
\end{tabular}
\end{table}

\subsubsection{Task Description}
Depression, particularly Major Depressive Disorder (MDD), is a psychological condition affecting 5\% of individuals worldwide, with a higher prevalence among women. In low- and middle-income countries, up to 75\% of individuals lack adequate care due to limited resources and stigma, despite effective treatments being available~\cite{WHO_depression}.

Depression severity is quantified using standardized scales like the Beck Depression Inventory (BDI) to differentiate clinical depression from normal mood variations. 
Existing studies adopt heterogeneous classification criteria: some focus on binary discrimination (e.g., patients vs. healthy controls), while others stratify cohorts by treatment status (medicated vs. non-medicated) or severity levels (mild vs. moderate/severe).
Table~\ref{tab:dep} summarizes datasets used in MDD research.
A detailed overview of all related works is provided in Appendix Table~\ref{tab:mdds}.

\subsubsection{Approach overview}
Depression impacts both superficial and deeper brain structures, presenting challenges for traditional handcrafted features.
Acharya introduces the first end-to-end DL model for EEG-based depression detection, showing that right-hemisphere signals are significantly more distinctive than left-hemisphere ones, which aligns with clinical findings~\cite{acharya2018automated}. 
This insight has driven further studies analyzing hemispheric EEG separately, often confirming similar patterns. For example, 
Ay et al. introduces a hybrid CNN-LSTM architecture, with experimental results revealing a more pronounced performance improvement in the right cerebral hemisphere~\cite{ay2019automated}.
DeprNet~\cite{seal2021deprnet} employs a CNN-based architecture with visualizations highlighting prominent activity in right-hemisphere electrodes for depressed subjects.

Spiking neural networks (SNNs) excel in EEG-based depression diagnosis, capturing brain-inspired spatiotemporal dynamics with biologically interpretable insights. 
Shah et al.~\cite{shah2019deep} employ the NeuCube SNN framework to encode EEG signals into temporal spike trains, mapping them onto a 3D spiking neural network reservoir (SNNr) aligned with the Talairach brain atlas.
The SNNr models spatiotemporal relationships between EEG channels using unsupervised spike-timing-dependent plasticity (STDP), offering interpretable brain connectivity visualizations.
Sam et al.~\cite{sam2023depression} integrates a 3D brain-inspired SNN with an LSTM, leveraging SNN’s energy efficiency with LSTM’s temporal modeling capabilities.

%\subsubsection{Unsupervised methods}
%Unsupervised methods effectively extract EEG patterns without relying on labeled data. NeuCube, an SNN-based architecture, identified neural markers in frontal, central, and parietal regions for early depression prediction, highlighting its potential in uncovering distinct neural activity patterns~\cite{shah2019deep}. Its outputs were further integrated into a supervised LSTM network for four-class depression classification based on Beck scores, showcasing its versatility in both unsupervised and hybrid frameworks~\cite{sam2023depression}.

%Another notable model, GCNs–FSMI, employed graph mutual information maximization and multi-band EEG signals to analyze mental illnesses, including depression and schizophrenia. By capturing high-level subject interactions and avoiding dependence on labeled data, it improved robustness and efficacy in multi-channel EEG-based analysis~\cite{li2023gcns}.

%\subsubsection{Semi-supervised methods}
%Labeled EEG data for depression identification is limited, making semi-supervised methods that utilize abundant unlabeled data a practical solution. One approach integrates self-organizing incremental neural networks (SOINN) with GCNs for self-training. By iteratively assigning pseudo-labels to high-confidence samples, this method achieved 92.23\% accuracy on the MODMA dataset with 600 labeled samples~\cite{wang2021identification}.


\subsection{Schizophrenia Identification}


\begin{table}[t]
\renewcommand{\arraystretch}{1.2}
\caption{Public EEG Datasets for Schizophrenia, where \textbf{Exp (n)} represents the number of schizophrenia patients and \textbf{Ctrl (n)} represents the control group.}
\label{tab:sz}
\footnotesize
\centering
\begin{tabular}{lcccc}
\hline
\textbf{Dataset}      & \textbf{Exp (n)} & \textbf{Ctrl (n)} & \textbf{Frequency (Hz)} & \textbf{Channels} \\
\hline
CeonRepod~\cite{olejarczyk2017graph}   & 14              & 14               & 250              & 19                \\
NIMH~\cite{ford2014did}                & 49              & 32               & 1024             & 64                \\
MHRC~\cite{borisov2005analysis}        & 45              & 39               & 128              & 16                \\
\hline
\end{tabular}
\end{table}


\subsubsection{Task Description}
Schizophrenia (SZ) is a psychiatric disorder affecting 24 million people worldwide, characterized by cognitive impairments, including memory deficits, delusions, and hallucinations~\cite{WHO_SZ}. 
SZ is associated with disruptions in structural and functional brain connectivity, marked by decreased global efficiency, weakened strength, and increased clustering~\cite{zalesky2011disrupted}. These abnormalities are detectable in EEG signals, making them useful for binary classification to distinguish SZ patients from healthy controls. Table~\ref{tab:sz} summarizes publicly available datasets for SZ research.
A detailed overview of all related works is provided in Appendix Table~\ref{tab:schis}.

\subsubsection{Approach overview}
Transfer learning has emerged as a powerful technique for fine-tuning pre-trained computer vision (CV) models in EEG-based schizophrenia diagnosis, enhancing performance with minimal training.
A common approach is converting EEG signals into 2D images for CNN-based models.
Aslan et al.~\cite{SZ16} feed spectrograms into a pre-trained VGG-16, applying Grad-CAM to highlight critical frequency components.
SchizoGoogLeNet~\cite{SZ21} fine-tunes the pre-trained GoogLeNet to process 2D EEG feature matrices, which are generated from preprocessed EEG signals through average filtering and resizing to align with the model's input dimensions.
Shalbaf et al.~\cite{SZ22} transform EEG into scalogram images via CWT, using ResNet-18 and VGG-19 to extract spatial-temporal features for classification.
%The results of these studies indicate that transfer learning enhances the scalability, adaptability, and classification performance of EEG data used to study schizophrenia.
%In schizophrenia detection using EEG, signal preprocessing methods often enhance feature extraction. For instance, Fast Fourier Transform (FFT) was used to extract features like spectral power and complexity, which were input into a hybrid CNN-LSTM network for SZ diagnosis~\cite{saeedi2022schizophrenia}. Similarly, smoothed pseudo-Wigner–Ville distribution (SPWVD) generated 2D time-frequency representations that were classified using a 4-layer 2D-CNN~\cite{khare2021spwvd}.

%Beyond generic features, some studies incorporated SZ-specific characteristics. For example, brain connectivity was analyzed using vector autoregressive coefficients (VAR), partial directed coherence (PDC), and network topology measures, which were processed through parallel CNNs before fusion for diagnosis~\cite{phang2019multi}.
%In contrast, task-oriented network designs eliminated manual feature engineering. An 11-layer 1D-CNN directly processed raw EEG data with minimal preprocessing, demonstrating the potential of concise architectures for SZ diagnosis~\cite{oh2019deep}.



\subsection{Alzheimer's Disease Diagnosis}
\begin{table}[t]
\renewcommand{\arraystretch}{1.2}
\caption{Public EEG Datasets for Alzheimer's Diagnosis, where \textbf{AD (n)} and \textbf{MCI (n)} represent the experimental groups, and \textbf{Ctrl (n)} represents the control group.}
\label{tab:ad}
\footnotesize
\centering
\begin{tabular}{p{50pt}p{20pt}p{20pt}p{20pt}p{30pt}p{30pt}}
\hline
\textbf{Dataset}            & \textbf{AD (n)} & \textbf{MCI (n)} & \textbf{Ctrl (n)} & \textbf{Frequency (Hz)} & \textbf{Channels} \\
\hline
FSA~\cite{ds_FSA_AD}        & 160            & -               & 24               & 128              & 21                \\
AD-65~\cite{ds004504:1.0.2} & 36             & -               & 29               & 250              & 19                \\
Fiscon~\cite{fiscon}         & 49             & 37              & 14               & 1024             & 19                \\
AD-59~\cite{cejnek2021novelty} & 59            & 7               & 102              & 128-256          & 21                \\
\hline
\end{tabular}
\end{table}

\subsubsection{Task Description}
Alzheimer’s disease (AD) is a progressive neurodegenerative disorder that starts with mild memory loss and advances to severe cognitive impairment, affecting daily life. While medical interventions can improve quality of life, a definitive cure remains elusive~\cite{better2024alzheimer}. 
Alzheimer’s disease (AD) progresses through three stages: preclinical, mild cognitive impairment (MCI), and Alzheimer’s dementia.
Classification tasks typically distinguish MCI or Alzheimer’s dementia from healthy controls. 
EEG abnormalities, such as slowed brain rhythms and desynchronization, serve as biomarkers for AD-related neurodegeneration~\cite{labate2014eeg}. Table~\ref{tab:ad} summarizes publicly available datasets.
A detailed overview of all related works is provided in Appendix Table~\ref{tab:ads}.

\subsubsection{Approach overview}
EEG abnormalities in Alzheimer’s disease, such as disrupted functional connectivity and altered brain rhythms, provide critical insights into the neurological changes. Brain connectivity modeling in AD can be approached from several angles. 
One approach, as seen in ST-GCN~\cite{shan2022spatial}, generates functional connectivity matrices that incorporate metrics like wavelet coherence and phase-locking value to simulate spatial and temporal dependencies in EEG signals. 
Alves et al.\cite{AD1} uses functional connectivity matrices derived from Granger causality and correlation measures to emphasize the spatial structure of brain networks. 
Additionally, some studies focus on spectral analysis, such as Morabito et al.\cite{AD5}, who convert EEG data into 2D spectral images using FFT and process these images with techniques like discriminative DCssCDBM to identify hybrid features that highlight EEG patterns associated with AD.
%These methodologies lay the groundwork for new, scalable, and clinically significant approaches to EEG-based diagnostics for neurological disorders.


\subsection{Parkinson's Disease Diagnosis}

\begin{table}[t]
\renewcommand{\arraystretch}{1.2}
\caption{Public EEG Datasets for Parkinson's Disease Diagnosis, where \textbf{Exp (n)} represents the number of patients and \textbf{Ctrl (n)} represents the healthy control group.}
\label{tab:pd}
\footnotesize
\centering
\begin{tabular}{lcccc}
\hline
\textbf{Dataset}            & \textbf{Exp (n)}   & \textbf{Ctrl (n)}   & \textbf{Frequency (Hz)} & \textbf{Channels} \\
\hline
UCSD~\cite{ds002778:1.0.5}  & 15                & 16                & 512              & 32                \\
UNM~\cite{cavanagh2018diminished} & 27                & 27                & 500              & 64                \\
UI~\cite{singh2020frontal}   & 14                & 14                & 500              & 59                \\
\hline
\end{tabular}
\end{table}

\subsubsection{Task Description}
Parkinson’s disease (PD) is a progressive neurodegenerative disorder marked by motor symptoms (tremors, rigidity, bradykinesia) and non-motor symptoms (depression, sleep disturbances, cognitive decline). In 2019, over 8.5 million people worldwide were living with PD~\cite{who2023parkinson}.
EEG is widely used in PD research due to its noise resistance and sensitivity to neurological changes, such as slowing cortical oscillations and increased low-frequency power~\cite{morita2011relationship}. 
Most studies focus on supervised learning for binary classification, with some incorporating transfer learning. 
Table~\ref{tab:pd} summarizes publicly available datasets.
A detailed overview of all related works is provided in Appendix Table~\ref{tab:pds}.

\subsubsection{Approach overview}
Transforming raw EEG signals into 2D representations is a well-established approach for PD classification, with various techniques offering distinct insights.
Spectrograms, generated via Gabor Transform, as in GaborPDNet\cite{PD2}, preserve time-frequency characteristics while minimizing information loss. 
Scalograms, created using CWT, provide another effective representation\cite{shaban2022resting}.
According to Chu et al.\cite{PD8}, power spectral density (PSD) mapping is another method, where specific frequency bands like high-$\delta$ and low-$\alpha$ can serve as potential biomarkers for early PD diagnosis.
Connectivity-based 2D representations can be obtained, like those applied by Arasteh et al.~\cite{PD9},  compute directional connectivity and produce heatmaps that effectively capture inter-channel relationships across frequency bands.
%PDCNNet, a framework designed for PD detection, combines Smoothed Pseudo Wigner Ville Distribution (SPWVD) with CNN to process EEG signals. This approach transforms EEG signals into high-resolution time-frequency representations (TFRs), addressing limitations of manual preprocessing and enabling automatic feature extraction and classification~\cite{khare2021pdcnnet}. Another method utilizes continuous wavelet transform (CWT) on resting-state EEG for PD classification. This framework incorporates Grad-CAM to visualize discriminative features and evaluate medication effects~\cite{shaban2022resting}.
%A lightweight convolutional-recurrent neural network (CRNN) was introduced to capture temporal dependencies in multi-channel EEG signals. By integrating CNN and GRU, this model achieved 99.2\% accuracy and demonstrated sensitivity to dopaminergic drug effects~\cite{lee2021convolutional}.

\subsection{ADHD Identification}

\begin{table}[b]
\renewcommand{\arraystretch}{1}
\caption{Public EEG Datasets for ADHD Identification, where \textbf{Exp (n)} represents the number of ADHD patients and \textbf{Ctrl (n)} represents the healthy control group.}
\label{tab:adhd}
\footnotesize
\centering
\begin{tabular}{lcccc}
\hline
\textbf{Dataset}            & \textbf{Exp (n)}   & \textbf{Ctrl (n)}              & \textbf{Frequency (Hz)} & \textbf{Channels} \\
\hline
ADHD-79~\cite{sadeghibajestani2023dataset} & 37 & 42             & 256          & 2                \\
ADHD-121~\cite{rzfh-zn36-20}      & 61 & 60    & 128          & 19                \\
\hline
\end{tabular}
\end{table}
\subsubsection{Task Description}
Attention-deficit/hyperactivity disorder (ADHD) is a neurodevelopmental disorder affecting around 3.1\% of individuals aged 10–14 and 2.4\% of those aged 15–19~\cite{who_adolescent_mental_health}. 
It is categorized into three subtypes: Inattentive (ADHD-I), Hyperactive-Impulsive (ADHD-H), and Combined (ADHD-C)~\cite{nimh_adhd}.
EEG is widely used alongside neuroimaging and physiological measures for ADHD diagnosis. 
However, deep learning remains underexplored, with most existing approaches relying on supervised learning and feature-based classification. Research focuses on binary classification tasks, and Table~\ref{tab:adhd} lists two publicly available datasets.
A detailed overview of all related works is provided in Appendix Table~\ref{tab:adhds}.

\subsubsection{Approach overview}
Studies on ADHD diagnosis identify distinct EEG neurophysiological markers, particularly abnormalities in specific frequency bands. Chen et al.\cite{chen2019use} and Dubreuil-Vall et al.\cite{dubreuil2020deep} demonstrate the effectiveness of CNNs for ADHD detection.
Chen et al. report $\theta$ and $\beta$ abnormalities in children with ADHD, while Dubreuil-Vall et al. observe altered $\alpha$ and $\delta-\theta$ in frontal electrodes during executive function tasks, aligning with medical findings.
They also find that EEG data from executive function tasks outperform resting-state EEG for ADHD detection.

\begin{table*}[t]
\renewcommand{\arraystretch}{1.2}
\caption{Summary of pre-trained SSL frameworks for multi-task neurodiagnosis, focusing on relevant datasets and tasks, with paradigms such as Contrastive Learning (CL), Contrastive Predictive Coding (CPC), and Masked Autoencoding (MAE)}
\label{tab:pts}
\footnotesize
\centering
\begin{tabular}
{p{2.1cm}p{1.9cm}p{2.1cm}p{1.4cm}p{1.8cm}p{3.7cm}p{2.5cm}}
\hline
\textbf{Work}        & \textbf{SSL Paradigm}                                     & \textbf{Backbone}  & \textbf{Data Type} &\textbf{Partitioning}  & \textbf{pre-training Dataset}                          & \textbf{Downstream Tasks}                           \\
\hline
Banville et al.~\cite{banville2021uncovering} & CPC & CNN & EEG & dataset-specific & TUSZ, PC18  & Seizure, Sleep  \\
MBrain~\cite{cai2023mbrain} & CPC & CNN+LSTM+GNN & EEG, iEEG&dataset-specific & TUSZ, private & Seizure, etc. \\
TS-TCC~\cite{eldele2021time}               & CPC                       & CNN+Transformer   & EEG &cross-dataset     & Bonn, Sleep-EDF, etc.                                  & Seizure, Sleep, etc.                                    \\
SeqCLR~\cite{mohsenvand2020contrastive}               & CL                                            & CNN+GRU & EEG &mixed-dataset & TUSZ, Sleep-EDF, ISRUC, etc.      & Seizure, Sleep, etc. \\
TF-C~\cite{zhang2022self}                 & CL                              & CNN  & EEG                  &cross-dataset & Sleep-EDF, etc.         & Seizure, Sleep, etc.       \\
BIOT~\cite{yang2024biot}  & CL & Transformer & EEG, etc. & cross-dataset & SHHS, etc. & Seizure, etc \\
Jo et al.~\cite{jo2023channel} & Predictive & CNN  & EEG &mixed-dataset &CHB-MIT, Sleep-EDF & Seizure, Sleep  \\
neuro2vec~\cite{wu2022neuro2vec} & MAE & CNN+Transformer & EEG &cross-dataset& Bonn, Sleep-EDF, etc.         & Seizure, Sleep \\
CRT~\cite{zhang2023self} & MAE & Transformer & EEG &dataset-specific & Sleep-EDF, etc. & Sleep, etc.\\
NeuroBERT~\cite{wu2024neuro} & MAE & Transformer & EEG, etc. & dataset-specific & Bonn, SleepEDF, etc, & Seizure, Sleep,etc. \\
BENDR~\cite{Kostas2021BENDR} & CPC+MAE & CNN+Transformer & EEG &cross-dataset& TUEG & Sleep, etc. \\
CBRAMOD~\cite{wang2024cbramod} & MAE & Transformer & EEG & cross-dataset & TUEG & Seizure, Sleep, MDD \\
Brant~\cite{zhang2024brant} & MAE & Transformer & iEEG&cross-dataset & private & Seizure, etc.  \\  
Brainwave~\cite{yuan2024brainwavebrainsignalfoundation} & MAE & Transformer & EEG, iEEG&cross-dataset & TUEG, Siena, CCEP, Sleep-EDF, NIMH, FSA, private, etc.  & Seizure, Sleep,  MDD,\newline SZ, AD, ADHD \\ 
EEGFormer~\cite{chen2024eegformer} & VQ+MAE & Transformer & EEG & cross-dataset & TUEG & Seizure, etc. \\
LaBraM~\cite{jiang2024large} & VQ+MAE & Transformer & EEG & cross-dataset & TUEG, Siena, etc. & Seizure, etc. \\
NeuroLM~\cite{jiang2024neurolm} & VQ+MAE\newline+Predictive & Transformer & EEG & cross-dataset & TUEG, Siena, etc. & Seizure, Sleep, etc.\\
\hline
\end{tabular}
\end{table*}


%	•	CL: Contrastive Learning
%	•	CPC: Contrastive Predictive Coding
%	•	MAE: Masked Autoencoding

\section{Universal Pre-trained Models}
\label{sec:bm}

In recent years, SSL has revolutionized EEG/iEEG analysis in neurological diagnosis. Emerging methods focus on generalizable SSL frameworks that integrate heterogeneous datasets during pre-training, overcoming the limitations of task- and dataset-specific models and enabling seamless adaptation to multiple downstream tasks.
%This paradigm shift is driven by the need for more realistic and robust solutions in neurological diagnostics. These methods enable models to learn richer and more generalizable representations by leveraging multiple datasets and other physiological signals. 
%Such frameworks can adapt seamlessly to various clinical tasks, including Seizure detection, sleep stage classification, and mental health assessment, without being constrained by the specific characteristics of individual datasets. 
These innovations bring us closer to the development of universal neurodiagnostic models capable of addressing challenges across diverse clinical settings.

Table~\ref{tab:pts} summarizes pre-trained SSL frameworks for multi-task neurodiagnosis, organized by the SSL paradigms to align with their technical evolution analyzed in this section. 
While some frameworks extend to broader time-series data, such as BCI signals and motion sensor data, we focus on datasets and tasks directly relevant to neurological applications.
Below, we further explore these frameworks, examining their contributions to unified pre-training strategies, multitask adaptability, and their potential to impact real-world applications.

%\textbf{Contrastive learning} methods establish positive and negative pairings to learn robust representations by optimizing agreement between related views and reducing it between unrelated ones. This paradigm effectively captures meaningful features from EEG by promoting models to concentrate on intrinsic patterns that differentiate between similar and dissimilar data.
\subsection{Contrastive- and Predictive- Based Learning}
\paragraph{Contrastive Predictive Coding}
Early SSL approaches in EEG/iEEG analysis are largely based on the Contrastive Predictive Coding (CPC) paradigm~\cite{banville2021uncovering,cai2023mbrain}, which learns robust representations by predicting signal segments through contrastive learning. While these models employed generic architectures across neurophysiological tasks, they fail to achieve true cross-task generalization. As a result, they are trained separately on specific datasets, limiting their clinical applicability across diverse neurodiagnostic applications.
CPC variants like TS-TCC~\cite{eldele2021time} introduce a one-to-one feature transfer mechanism. This framework enables feature migration across tasks such as human activity recognition, sleep staging, and epileptic seizure detection, paving the way for broader multi-domain diagnostic generalization.


Building on the foundational principles of CPC, two distinct approaches have emerged: contrastive learning (CL) and predictive-based variants. CL retains CPC’s contrastive framework but emphasizes explicit instance-level discrimination through hand-crafted augmentations for positive/negative pairs, instead of CPC’s autoregressive future state prediction.
Predictive variants inherit CPC’s structure but replace its auto-learned latent contexts with manually defined features.

\paragraph{Contrastive-Based learning}
SeqCLR~\cite{mohsenvand2020contrastive}, inspired by SimCLR, employs contrastive learning to EEG data, enhancing similarity between augmented views of the same channel through domain-specific transformations. Adopting a mixed-dataset training approach, it unifies diverse EEG datasets for robust representation learning.
TF-C~\cite{zhang2022self} incorporates dual time-frequency contrastive learning with a cross-domain consistency loss to align embeddings across temporal and spectral representations.
It further evaluates one-to-many paradigms, highlighting the potential of cross-task feature sharing for universal neural signal models.
BIOT~\cite{yang2024biot} integrates contrastive learning, unifying multimodal biosignals (e.g., EEG, ECG) via tokenization and linear attention to learn invariant physiological patterns for cross-task generalization.

\paragraph{Predictive-Based Learning}
Jo et al.~\cite{jo2023channel} proposes a channel-aware predictive-based framework, which leverages stopped band prediction for spectral feature learning and employs temporal trend identification to capture dynamic patterns. 
By integrating mix-dataset pretraining, it enhances generalization through cross-domain feature fusion. However, the pretraining scale remains limited.

%\textbf{Masked modeling} focuses on reconstructing missing or masked portions of the signal, forcing the model to learn intrinsic patterns in the data. This paradigm is particularly effective in extracting multi-grained features across temporal and spectral dimensions.
\subsection{Reconstruction-Based Learning}
\paragraph{Masked Autoencoding}
The paradigm shift from CPC to masked reconstruction in SSL aims for higher data efficiency and scalability, inspired by cross-domain advances like masked language modeling in NLP (e.g., BERT~\cite{devlin2018bert}), with MAE's generative approach enhancing classification performance while avoiding complex negative sampling.
%This transition is further accelerated by the proven scalability and strong generalization of transformer architectures, enabling larger models and unified multitask learning frameworks.

Neuro2vec~\cite{wu2022neuro2vec} extends masked reconstruction by integrating EEG-specific spatiotemporal recovery and spectral component prediction into a unified framework, utilizing a CNN-ViT hybrid architecture for patch embedding and reconstruction. 
CRT~\cite{zhang2023self} further introduces multi-domain reconstruction through cross-domain synchronization of temporal and spectral features, replacing conventional masking with adaptive input dropping to preserve data distribution integrity, thereby improving robustness in physiological signal modeling.
Neuro-BERT~\cite{wu2024neuro} introduces Fourier Inversion Prediction (FIP), reconstructing masked signals by predicting their Fourier amplitude and phase, then applying an inverse Fourier transform. The spectral-based prediction framework inherently matches the physiological nature of EEG signals.

\paragraph{Large-Scale Continuous-Reconstruction Models}
Transformer architectures excel in neurodiagnostics due to their scalability and attention mechanisms, which adaptively capture global dependencies in irregular neural signals. BERT-style pretraining, particularly masked reconstruction, enhances neurodiagnostic classification by enforcing robust contextual learning of latent bioelectrical patterns, which is crucial for distinguishing subtle neurological signatures. Their parallelizable training and tokenized time-frequency representations pave the way for scalable foundation models, driving large-scale pretraining in neural signal analysis.

Inspired by Bert, BENDR~\cite{Kostas2021BENDR} integrates CPC with MAE-inspired reconstruction for temporal feature learning. 
Pretrained on the Temple University Hospital EEG Corpus (TUEG)—a diverse dataset containing 1.5 TB of raw clinical EEG recordings from over 10,000 subjects—BENDR represents the emergence of large-scale pretraining for neurodiagnostics, showcasing the cross-subject scalability of transformers. 
It demonstrates how foundation models can unify heterogeneous neural signal paradigms, advancing generalized and scalable EEG analysis.
CBRAMOD~\cite{wang2024cbramod} introduces a criss-cross transformer framework to explicitly model EEG’s spatial-temporal heterogeneity.
Using patch-based masked EEG reconstruction, it separately processes spatial and temporal patches through parallel attention mechanisms, preserving the structural dependencies unique to EEG.

Brant~\cite{zhang2024brant} and Brainwave~\cite{yuan2024brainwavebrainsignalfoundation}
represent a unified effort to establish foundation models for neural signal analysis. 
Brant focuses on SEEG signals, employing a masked autoencoding framework with dual Transformer
encoders to capture temporal dependencies and spatial
correlations, enabling applications such as seizure detection and signal forecasting.
Brainwave pioneers large-scale pretraining with an unprecedented multimodal corpus of over 40,000 hours of EEG and iEEG data from 16,000 subjects, marking a significant milestone in neural signal foundation models. Its pre-training strategy follows a masked modeling paradigm that randomly masks time-frequency patches of neural signals, and the model is trained to reconstruct the missing regions. To enhance generalizability across different types of neural data, Brainwave employs a shared encoder for both EEG and iEEG, coupled with modality-specific reconstruction decoders. These innovations position Brainwave as the first comprehensive foundation model capable of unifying EEG and iEEG analysis, with transformative implications for neuroscience research.
%The framework demonstrates strong performance in downstream tasks, excelling in cross-subject, cross-hospital, and cross-subtype evaluations.
\paragraph{Large-Scale Discrete-Reconstruction Models}
Vector Quantized Variational Autoencoder (VQ-VAE) is a powerful framework for learning discrete representations of continuous data by mapping inputs to a predefined codebook, which has been widely adopted in domains like speech and image processing~\cite{van2017neural}. By tokenizing raw data into discrete codes, this approach enhances cross-subject generalization while preserving interpretable spatiotemporal patterns.

LaBraM~\cite{jiang2024large} trains its discrete codebook by reconstructing both Fourier spectral magnitudes and phases of EEG segments, then pretrains with a symmetric masking task that predicts masked code indices bidirectionally.
NeuroLM~\cite{jiang2024neurolm} further extends this approach by introducing VQ Temporal-Frequency Prediction, aligning EEG tokens with textual representations through adversarial training. After tokenization, it employs multi-channel autoregressive modeling, enabling an LLM to predict the next EEG token in a manner analogous to language modeling.
EEGFormer~\cite{chen2024eegformer} focuses on reconstructing raw temporal waveforms for codebook training, followed by BERT-style masked signal reconstruction pretraining.
These methods demonstrate how VQ-based tokenization adapts to EEG modeling—whether prioritizing spectral synchrony (LaBraM), fusing time-frequency features (NeuroLM), or preserving temporal fidelity (EEGFormer).

\subsection{BrainBenchmark}
The development of universal pre-trained frameworks represents a transformative advancement in healthcare, enabling the integration of heterogeneous datasets and generalization across diverse diagnostic tasks.
To systematically evaluate and advance this field, we have established an open benchmark, currently comprising 8 models and 9 public datasets focused on neurological diagnostics, with ongoing expansions planned. 
This benchmark supports comprehensive performance evaluation, custom model integration, and dataset extensibility, fostering reproducible research and innovation. The implementation is publicly available at \href{https://github.com/ZJU-BrainNet/BrainBenchmark}{https://github.com/ZJU-BrainNet/BrainBenchmark}. 
Future work will include a detailed analysis of benchmark results to further advance universal frameworks in EEG/iEEG analysis.
\section{Conclusion}
This survey systematically reviews 448 studies and 46 public datasets to advance deep learning-driven analysis of EEG/iEEG signals across seven neurological diagnostic tasks: seizure detection, sleep staging and disorder, major depressive disorder, schizophrenia, Alzheimer’s disease, Parkinson’s disease, and ADHD.
Our work establishes three foundational contributions: First, we unify fragmented methodologies across neurological conditions by standardizing data processing, model architectures, and evaluation protocols.
Second, we identify self-supervised learning as the most promising paradigm for multi-task neurodiagnosis, providing a comprehensive overview of pre-trained SSL frameworks and their advancements. Third, we introduce \href{https://github.com/ZJU-BrainNet/BrainBenchmark}{BrainBenchmark} to enhance reproducibility by integrating neurological datasets and universal models under standardized evaluations.

Looking back, the pursuit of universal models capable of learning from diverse, multimodal data reflects the field's growing ambition. It lays the groundwork for a new era of intelligent and adaptable healthcare systems. Over the past decades, significant progress in traditional methods has established a strong foundation for neurological diagnostics based on electrical brain signals. Key contributions include advances in signal preprocessing techniques, curating large-scale, well-annotated datasets, and developing deep learning architectures for specific tasks.
Building on this foundation, the integration of self-supervised pretraining marks a paradigm shift, enabling models to extract rich and meaningful representations from vast amounts of unlabeled, heterogeneous data.

Looking forward, the ultimate goal is to develop genuinely universal and adaptable frameworks capable of transcending individual tasks and datasets to address a broader range of neurological disorders. These advancements will pave the way for intelligent diagnostic tools that deliver precise, efficient, and accessible healthcare solutions globally, driving transformative progress in biomedical research and clinical applications.
% The very first letter is a 2 line initial drop letter followed
% by the rest of the first word in caps.
% 
% form to use if the first word consists of a single letter:
% \IEEEPARstart{A}{demo} file is ....
% 
% form to use if you need the single drop letter followed by
% normal text (unknown if ever used by the IEEE):
% \IEEEPARstart{A}{}demo file is ....
% 
% Some journals put the first two words in caps:
% \IEEEPARstart{T}{his demo} file is ....
% 
% Here we have the typical use of a "T" for an initial drop letter
% and "HIS" in caps to complete the first word.





% An example of a floating figure using the graphicx package.
% Note that \label must occur AFTER (or within) \caption.
% For figures, \caption should occur after the \includegraphics.
% Note that IEEEtran v1.7 and later has special internal code that
% is designed to preserve the operation of \label within \caption
% even when the captionsoff option is in effect. However, because
% of issues like this, it may be the safest practice to put all your
% \label just after \caption rather than within \caption{}.
%
% Reminder: the "draftcls" or "draftclsnofoot", not "draft", class
% option should be used if it is desired that the figures are to be
% displayed while in draft mode.
%
%\begin{figure}[!t]
%\centering
%\includegraphics[width=2.5in]{myfigure}
% where an .eps filename suffix will be assumed under latex, 
% and a .pdf suffix will be assumed for pdflatex; or what has been declared
% via \DeclareGraphicsExtensions.
%\caption{Simulation results for the network.}
%\label{fig_sim}
%\end{figure}

% Note that the IEEE typically puts floats only at the top, even when this
% results in a large percentage of a column being occupied by floats.


% An example of a double column floating figure using two subfigures.
% (The subfig.sty package must be loaded for this to work.)
% The subfigure \label commands are set within each subfloat command,
% and the \label for the overall figure must come after \caption.
% \hfil is used as a separator to get equal spacing.
% Watch out that the combined width of all the subfigures on a 
% line do not exceed the text width or a line break will occur.
%
%\begin{figure*}[!t]
%\centering
%\subfloat[Case I]{\includegraphics[width=2.5in]{box}%
%\label{fig_first_case}}
%\hfil
%\subfloat[Case II]{\includegraphics[width=2.5in]{box}%
%\label{fig_second_case}}
%\caption{Simulation results for the network.}
%\label{fig_sim}
%\end{figure*}
%
% Note that often IEEE papers with subfigures do not employ subfigure
% captions (using the optional argument to \subfloat[]), but instead will
% reference/describe all of them (a), (b), etc., within the main caption.
% Be aware that for subfig.sty to generate the (a), (b), etc., subfigure
% labels, the optional argument to \subfloat must be present. If a
% subcaption is not desired, just leave its contents blank,
% e.g., \subfloat[].


% An example of a floating table. Note that, for IEEE style tables, the
% \caption command should come BEFORE the table and, given that table
% captions serve much like titles, are usually capitalized except for words
% such as a, an, and, as, at, but, by, for, in, nor, of, on, or, the, to
% and up, which are usually not capitalized unless they are the first or
% last word of the caption. Table text will default to \footnotesize as
% the IEEE normally uses this smaller font for tables.
% The \label must come after \caption as always.
%
%\begin{table}[!t]
%% increase table row spacing, adjust to taste
%\renewcommand{\arraystretch}{1.3}
% if using array.sty, it might be a good idea to tweak the value of
% \extrarowheight as needed to properly center the text within the cells
%\caption{An Example of a Table}
%\label{table_example}
%\centering
%% Some packages, such as MDW tools, offer better commands for making tables
%% than the plain LaTeX2e tabular which is used here.
%\begin{tabular}{|c||c|}
%\hline
%One & Two\\
%\hline
%Three & Four\\
%\hline
%\end{tabular}
%\end{table}


% Note that the IEEE does not put floats in the very first column
% - or typically anywhere on the first page for that matter. Also,
% in-text middle ("here") positioning is typically not used, but it
% is allowed and encouraged for Computer Society conferences (but
% not Computer Society journals). Most IEEE journals/conferences use
% top floats exclusively. 
% Note that, LaTeX2e, unlike IEEE journals/conferences, places
% footnotes above bottom floats. This can be corrected via the
% \fnbelowfloat command of the stfloats package.




% if have a single appendix:
%\appendix[Proof of the Zonklar Equations]
% or
%\appendix  % for no appendix heading
% do not use \section anymore after \appendix, only \section*
% is possibly needed

% use appendices with more than one appendix
% then use \section to start each appendix
% you must declare a \section before using any
% \subsection or using \label (\appendices by itself
% starts a section numbered zero.)
%
\section*{Acknowledgment}
This work is supported by NSFC (62322606) and Zhejiang NSF (LR22F020005).



In this section, we provide summaries of deep learning-based frameworks for the seven neurodiagnostic tasks mentioned earlier. These summaries include details on preprocessing methods, extracted features, deep learning backbones, training paradigms, downstream task datasets, classification tasks, data partitioning strategies, and reported performances. The relevant tables are as follows: seizure detection in Table~\ref{tab:seizures}, sleep staging in Table~\ref{tab:sleeps}, depression identification in Table~\ref{tab:mdds}, schizophrenia identification in Table~\ref{tab:schis}, Alzheimer’s disease diagnosis in Table~\ref{tab:ads}, Parkinson’s disease diagnosis in Table~\ref{tab:pds}, and ADHD identification in Table~\ref{tab:adhds}.
\clearpage  

% ADD THIS HEADER TO ALL NEW CHAPTER FILES FOR SUBFILES SUPPORT

% Allow independent compilation of this section for efficiency
\documentclass[../CLthesis.tex]{subfiles}

% Add the graphics path for subfiles support
\graphicspath{{\subfix{../images/}}}

% END OF SUBFILES HEADER

%%%%%%%%%%%%%%%%%%%%%%%%%%%%%%%%%%%%%%%%%%%%%%%%%%%%%%%%%%%%%%%%
% START OF DOCUMENT: Every chapter can be compiled separately
%%%%%%%%%%%%%%%%%%%%%%%%%%%%%%%%%%%%%%%%%%%%%%%%%%%%%%%%%%%%%%%%
\begin{document}
\chapter{Appendix}%

\label{appendix:Appendix}
\section{Neuron Depths}
\begin{table}[htbp]
\centering
\begin{tabular}{ll||ll||ll||ll}
\toprule
Neuron & Depth & Neuron & Depth & Neuron & Depth & Neuron & Depth \\
\midrule
N1  & 3380 & N20 & 2640 & N39 & 2460 & N58 & 2140 \\
N2  & 3220 & N21 & 2600 & N40 & 2440 & N59 & 2100 \\
N3  & 3200 & N22 & 2600 & N41 & 2440 & N60 & 1880 \\
N4  & 3180 & N23 & 2600 & N42 & 2420 & N61 & 1820 \\
N5  & 2980 & N24 & 2600 & N43 & 2400 & N62 & 1680 \\
N6  & 2960 & N25 & 2580 & N44 & 2380 & N63 & 1680 \\
N7  & 2880 & N26 & 2580 & N45 & 2380 & N64 & 1340 \\
N8  & 2860 & N27 & 2580 & N46 & 2360 & N65 & 1320 \\
N9  & 2820 & N28 & 2580 & N47 & 2360 & N66 & 1320 \\
N10 & 2740 & N29 & 2580 & N48 & 2340 & N67 & 1120 \\
N11 & 2720 & N30 & 2560 & N49 & 2320 & N68 & 1080 \\
N12 & 2720 & N31 & 2540 & N50 & 2300 & N69 & 1060 \\
N13 & 2700 & N32 & 2540 & N51 & 2280 & N70 & 1060 \\
N14 & 2680 & N33 & 2520 & N52 & 2280 & N71 & 840  \\
N15 & 2680 & N34 & 2520 & N53 & 2260 & N72 & 660  \\
N16 & 2660 & N35 & 2500 & N54 & 2240 & N73 & 480  \\
N17 & 2660 & N36 & 2480 & N55 & 2220 & N74 & 480  \\
N18 & 2640 & N37 & 2480 & N56 & 2180 & N75 & 200  \\
N19 & 2640 & N38 & 2460 & N57 & 2160 &     &      \\
\bottomrule
\end{tabular}
\caption{Depth ($\mu$m) to probe tip for all neurons used in experiment~\ref{exp:1}}
\label{tab:neuron_depths}
\end{table}
% \begin{table}[htbp]
%     \centering
%     {\footnotesize
%     \begin{tabular}{lcllcl}
%         \hline
%         Neuron & Depth & & Neuron & Depth & \\
%         \hline
%         N1 & 3380$\,\mu$m & & N39 & 2460$\,\mu$m & \\
%         N2 & 3220$\,\mu$m & & N40 & 2440$\,\mu$m & \\
%         N3 & 3200$\,\mu$m & & N41 & 2440$\,\mu$m & \\
%         N4 & 3180$\,\mu$m & & N42 & 2420$\,\mu$m & \\
%         N5 & 2980$\,\mu$m & & N43 & 2400$\,\mu$m & \\
%         N6 & 2960$\,\mu$m & & N44 & 2380$\,\mu$m & \\
%         N7 & 2880$\,\mu$m & & N45 & 2380$\,\mu$m & \\
%         N8 & 2860$\,\mu$m & & N46 & 2360$\,\mu$m & \\
%         N9 & 2820$\,\mu$m & & N47 & 2360$\,\mu$m & \\
%         N10 & 2740$\,\mu$m & & N48 & 2340$\,\mu$m & \\
%         N11 & 2720$\,\mu$m & & N49 & 2320$\,\mu$m & \\
%         N12 & 2720$\,\mu$m & & N50 & 2300$\,\mu$m & \\
%         N13 & 2700$\,\mu$m & & N51 & 2280$\,\mu$m & \\
%         N14 & 2680$\,\mu$m & & N52 & 2280$\,\mu$m & \\
%         N15 & 2680$\,\mu$m & & N53 & 2260$\,\mu$m & \\
%         N16 & 2660$\,\mu$m & & N54 & 2240$\,\mu$m & \\
%         N17 & 2660$\,\mu$m & & N55 & 2220$\,\mu$m & \\
%         N18 & 2640$\,\mu$m & & N56 & 2180$\,\mu$m & \\
%         N19 & 2640$\,\mu$m & & N57 & 2160$\,\mu$m & \\
%         N20 & 2640$\,\mu$m & & N58 & 2140$\,\mu$m & \\
%         N21 & 2600$\,\mu$m & & N59 & 2100$\,\mu$m & \\
%         N22 & 2600$\,\mu$m & & N60 & 1880$\,\mu$m & \\
%         N23 & 2600$\,\mu$m & & N61 & 1820$\,\mu$m & \\
%         N24 & 2600$\,\mu$m & & N62 & 1680$\,\mu$m & \\
%         N25 & 2580$\,\mu$m & & N63 & 1680$\,\mu$m & \\
%         N26 & 2580$\,\mu$m & & N64 & 1340$\,\mu$m & \\
%         N27 & 2580$\,\mu$m & & N65 & 1320$\,\mu$m & \\
%         N28 & 2580$\,\mu$m & & N66 & 1320$\,\mu$m & \\
%         N29 & 2580$\,\mu$m & & N67 & 1120$\,\mu$m & \\
%         N30 & 2560$\,\mu$m & & N68 & 1080$\,\mu$m & \\
%         N31 & 2540$\,\mu$m & & N69 & 1060$\,\mu$m & \\
%         N32 & 2540$\,\mu$m & & N70 & 1060$\,\mu$m & \\
%         N33 & 2520$\,\mu$m & & N71 & 840$\,\mu$m & \\
%         N34 & 2520$\,\mu$m & & N72 & 660$\,\mu$m & \\
%         N35 & 2500$\,\mu$m & & N73 & 480$\,\mu$m & \\
%         N36 & 2480$\,\mu$m & & N74 & 480$\,\mu$m & \\
%         N37 & 2480$\,\mu$m & & N75 & 200$\,\mu$m & \\
%         N38 & 2460$\,\mu$m & & & & \\
%         \hline
%     \end{tabular}
%     }
%     \caption{Depth to Probe Tip for All Neurons Used in Experiment 1}
%     \label{tab:neuron_depths}
% \end{table}

\section{Neural Information Integration}
\label{appendix:integration}
\begin{figure}[H]
    \centering
    \includegraphics[height=0.9\textheight]{images/accuracy_all_onsets.pdf}
    \caption{Classification accuracy at different onsets}
    \label{fig:neural_integration}
\end{figure}

\section{CEBRA Results}
\label{appendix:CEBRA}
\begin{figure}[htbp]
    \centering
    \includegraphics[width=0.9\textwidth]{images/embeddings_plot.png}
    \caption{Extra CEBRA embedding visualization from different parameters}
    \label{fig:all_cebra}
\end{figure}

\begin{figure}[H]
    \centering
    \includegraphics[width=0.9\textwidth]{images/cebra_loss.pdf}
    \caption{CEBRA training loss}
    \label{fig:cebra_loss}
\end{figure}

\begin{figure}[H]
    \centering
    \includegraphics[width=0.45\textwidth]{images/cebra_labels.pdf}
    \caption{Data distribution in CEBRA}
    \label{fig:cebra_labels}
\end{figure}

\section{VAE Results}
\label{appendix:VAE}
\begin{figure}[H]
   \begin{subfigure}[b]{0.32\textwidth}
       \centering
       \includegraphics[width=\textwidth]{images/average_syllable_2.pdf}
       \caption{Syllable 2}
       \label{fig:syllable_2}
   \end{subfigure}
   \hfill
   \begin{subfigure}[b]{0.32\textwidth}
       \centering
       \includegraphics[width=\textwidth]{images/average_syllable_3.pdf}
       \caption{Syllable 3}
       \label{fig:syllable_3}
   \end{subfigure}
   \hfill
   \begin{subfigure}[b]{0.32\textwidth}
       \centering
       \includegraphics[width=\textwidth]{images/average_syllable_4.pdf}
       \caption{Syllable 4}
       \label{fig:syllable_4}
   \end{subfigure}
\end{figure}
\begin{figure}[H]
   \begin{subfigure}[b]{0.32\textwidth}
       \centering
       \includegraphics[width=\textwidth]{images/average_syllable_5.pdf}
       \caption{Syllable 5}
       \label{fig:syllable_5}
   \end{subfigure}
   \hfill
   \begin{subfigure}[b]{0.32\textwidth}
       \centering
       \includegraphics[width=\textwidth]{images/average_syllable_6.pdf}
       \caption{Syllable 6}
       \label{fig:syllable_6}
   \end{subfigure}
   \hfill
   \begin{subfigure}[b]{0.32\textwidth}
       \centering
       \includegraphics[width=\textwidth]{images/average_syllable_7.pdf}
       \caption{Syllable 7}
       \label{fig:syllable_7}
   \end{subfigure}

   \begin{subfigure}[b]{0.32\textwidth}
       \centering
       \includegraphics[width=\textwidth]{images/average_syllable_8.pdf}
       \caption{Syllable 8}
       \label{fig:syllable_8}
   \end{subfigure}
   
   \caption{Original and reconstruction syllables of a motif}
   \label{fig:all_syllables}
\end{figure}

\begin{figure}[H]
    \centering
    \includegraphics[width=\linewidth]{images/2d_vae_vocal_warped.pdf}
    \caption{Reconstruction of warped vocal data}
    \label{fig:whole_motif}
\end{figure}

\begin{figure}[H]
    \centering
    \includegraphics[width=\linewidth]{images/neural2vocal_80ms.pdf}
    \caption{Generate 80\,ms vocalization from 80\,ms neural data}
    \label{fig:neuro2voc_80ms}
\end{figure}

% \begin{figure}
%     \centering
%     \includegraphics[width=\linewidth]{images/2d_vae_vocal_trimmed.pdf}
%     \caption{Trimmed Vocal Data to 80ms}
%     \label{fig:vocal_trimmed}
% \end{figure}

% \begin{figure}
%     \centering
%     \includegraphics[width=\linewidth]{images/2d_vae_vocal_padded.pdf}
%     \caption{Padded Vocal Data to 224ms}
%     \label{fig:vocal_padded}
% \end{figure}

% \begin{figure}
%     \centering
%     \includegraphics[width=\linewidth]{images/2d_vae_vocal_warped.pdf}
%     \caption{Warped Vocal Data to 224ms}
%     \label{fig:vocal_warped}
% \end{figure}


\end{document}


% you can choose not to have a title for an appendix
% if you want by leaving the argument blank


% use section* for acknowledgment



% Can use something like this to put references on a page
% by themselves when using endfloat and the captionsoff option.
\ifCLASSOPTIONcaptionsoff
  \newpage
\fi



% trigger a \newpage just before the given reference
% number - used to balance the columns on the last page
% adjust value as needed - may need to be readjusted if
% the document is modified later
%\IEEEtriggeratref{8}
% The "triggered" command can be changed if desired:
%\IEEEtriggercmd{\enlargethispage{-5in}}

% references section

% can use a bibliography generated by BibTeX as a .bbl file
% BibTeX documentation can be easily obtained at:
% http://mirror.ctan.org/biblio/bibtex/contrib/doc/
% The IEEEtran BibTeX style support page is at:
% http://www.michaelshell.org/tex/ieeetran/bibtex/
%\bibliographystyle{IEEEtran}
% argument is your BibTeX string definitions and bibliography database(s)
%\bibliography{IEEEabrv,../bib/paper}
%
% <OR> manually copy in the resultant .bbl file
% set second argument of \begin to the number of references
% (used to reserve space for the reference number labels box)
\clearpage  
\bibliography{IEEEabrv, reference}

% biography section
% 
% If you have an EPS/PDF photo (graphicx package needed) extra braces are
% needed around the contents of the optional argument to biography to prevent
% the LaTeX parser from getting confused when it sees the complicated
% \includegraphics command within an optional argument. (You could create
% your own custom macro containing the \includegraphics command to make things
% simpler here.)
%\begin{IEEEbiography}[{\includegraphics[width=1in,height=1.25in,clip,keepaspectratio]{mshell}}]{Michael Shell}
% or if you just want to reserve a space for a photo:





% insert where needed to balance the two columns on the last page with
% biographies
%\newpage


% You can push biographies down or up by placing
% a \vfill before or after them. The appropriate
% use of \vfill depends on what kind of text is
% on the last page and whether or not the columns
% are being equalized.

%\vfill

% Can be used to pull up biographies so that the bottom of the last one
% is flush with the other column.
%\enlargethispage{-5in}



% that's all folks
\end{document}


