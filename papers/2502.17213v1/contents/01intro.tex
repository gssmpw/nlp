\section{Introduction}
\label{sec:introduction}
Neurological disorders represent one of the most significant challenges to global health today, with profound consequences for both individuals and healthcare systems.
According to the World Health Organization (WHO), neurological disorders affect over one-third of the global population, making them a leading cause of illness and disability  worldwide~\cite{WHO2024}.
Dementia, which affects 47.5 million people worldwide, is a primary concern, with Alzheimer's disease being the most common form. Seizure impacts more than 50 million individuals, while sleep disorders are widespread yet often underdiagnosed. Other significant disorders, including Parkinson's disease, schizophrenia, depression, and ADHD, further exacerbate the global burden, placing additional strain on healthcare systems~\cite{WHO_MentalHealth}.
In low- and middle-income countries, where limited resources constrain access to neurological care and treatment, the situation is particularly dire.

Practical diagnostic tools are essential to alleviate the growing global burden of neurological disorders, and electrical brain signals are indispensable among them.
Electrical brain signals, specifically electroencephalography, are critical for understanding and diagnosing neurological disorders. 
Electroencephalography evaluates electrical activity in the brain and is categorized into scalp electroencephalography (EEG) and intracranial electroencephalography (iEEG). 
EEG is non-invasive, recording brain activity from electrodes placed on the scalp. 
iEEG includes inserting electrodes into the brain (stereo-electroencephalography, SEEG) or onto the brain's surface (electrocorticography, ECoG), providing more detailed and localized information~\cite{ramantani2016correlation}.

\begin{figure*}[t] % 横跨双栏并置顶
    \centering
    \includegraphics[width=0.95\textwidth]{contents/imgs/main219.pdf} % 
  \caption{\textbf{General Workflow of Electrical Brain Signals Analysis in Neurological Diagnostics.} \textbf{a. Signal Collection:} Acquisition of EEG/iEEG signals from patients using non-invasive scalp electrodes or invasive intracranial electrodes, capturing brain electrical activity for clinical purposes. \textbf{b. Signal Preprocessing:} A feasible workflow to process raw signals, ensuring their suitability for subsequent analysis.  \textbf{c. Analysis and Diagnosis:} Extraction of features from the preprocessed signals, followed by the application of deep learning techniques for model training and neurological classification. \newline \textbf{d. Statistical Information:} A statistical summary of related work and publicly available datasets, illustrating their contributions to the field and providing essential resources for future research, model development, and validation.}
    \label{fig:main}
\end{figure*}


The analysis of brain signals such as EEG/iEEG poses significant challenges for traditional machine learning (ML) approaches. These methods typically rely on manually engineered features that may not fully capture the complex patterns in neurophysiological data, while their performance is often compromised by inherent noise and artifacts in raw neural recordings. Deep learning (DL) addresses these limitations by automatically extracting features, modeling temporal dependencies, and improving robustness against signal variability. The ability of DL methods to detect and classify neurological disorders with high accuracy has driven widespread adoption in brain signal analysis. This survey systematically examines the workflow of DL models in brain signal analysis, focusing on their applications in diagnosing neurological disorders.


\subsection{General Workflow}

The general workflow of electrical brain signal analysis in neurological diagnostics, as illustrated in Fig.~\ref{fig:main}, consists of three main stages: signal collection, signal preprocessing, and analysis and diagnosis.

In the signal collection stage, electrical brain activity is recorded using EEG, ECoG, or SEEG systems (Fig.~\ref{fig:main}.a). These signals are typically captured across multiple channels at specific sampling frequencies and are often accompanied by labeled tasks and corresponding labels.

The signal preprocessing stage (Fig.~\ref{fig:main}.b) involves a series of low-level techniques, including denoising, filtering, artifact removal, and normalization. These steps are crucial for reducing noise and artifacts, enhancing relevant patterns, and structuring the data for effective feature extraction.

In the analysis and diagnosis stage (Fig.~\ref{fig:main}.c), the preprocessed signals undergo feature extraction and neurodiagnostic classification. Feature extraction transforms the signals into representations suitable for diagnosis. Traditional methods extract spatial, temporal, and spectral features manually, while deep learning approaches automatically learn complex, diagnostically relevant patterns. These features are then fed into classifiers for specific neurodiagnostic tasks. The training phase involves careful consideration of network backbones, training paradigms and data partitioning strategies.

Finally, the extracted features are applied to downstream tasks. Fig.~\ref{fig:main}.d highlights the distribution of related research efforts and publicly available datasets across various neurological conditions, including seizure, sleep disorders, major depressive disorder (MDD), schizophrenia (SZ), Alzheimer's disease (AD), Parkinson's disease (PD), and attention deficit hyperactivity disorder (ADHD).

\subsection{Related Studies and Our Contributions}

Existing brain signal analysis surveys exhibit diverse scopes and focuses.
Some focus specifically on EEG signals, emphasizing their wide availability~\cite{roy2019deep, amrani2021eeg, amer2023eeg}. 
Others broaden the scope to include brain signals like magnetic resonance imaging (MRI)~\cite{zhang2021survey, khan2021machine}, which differ from EEG and iEEG in acquisition methods, temporal resolution, and preprocessing requirements.
%This study highlights the importance of considering both sEEG and iEEG to provide a more comprehensive understanding of EEG-based neurological diagnostics.
From a task perspective, some reviews focus specifically on diseases such as seizure~\cite{shoeibi2021epileptic, rahul2024systematic}, providing in-depth insights into disease-specific applications. 
Others take a broader view, covering diverse brain-computer interface (BCI) applications~\cite{hossain2023status, weng2024self}, which focus on interaction and control, differing fundamentally from neurological diagnostic tasks. 

%This survey systematically examines deep learning-powered approaches for EEG/iEEG-based neurological diagnostics, focusing on data characteristics, preprocessing strategies, and algorithmic designs. 
%We highlight the potential of deep learning-driven multi-task frameworks, particularly those leveraging self-supervised pretraining, to enhance diagnostic accuracy and generalizability.
Our work establishes three foundational contributions to advance deep learning-driven neurodiagnosis: 
First, we systematically curate and analyze 46 public EEG/iEEG datasets across seven neurological conditions, establishing the most comprehensive data landscape to date. We also unify fragmented methodologies by standardizing data processing, model architectures, and evaluation protocols.
Besides, we identify self-supervised learning as the optimal paradigm for developing multi-task diagnostic frameworks, offering a comprehensive overview of pre-trained multi-task frameworks and their advancements.
Additionally, we propose a benchmarking methodology to evaluate brain signal models across tasks, providing a foundation for scalable and versatile solutions in EEG/iEEG-based neurological diagnostics applications.