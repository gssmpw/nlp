% This must be in the first 5 lines to tell arXiv to use pdfLaTeX, which is strongly recommended.
\pdfoutput=1
% In particular, the hyperref package requires pdfLaTeX in order to break URLs across lines.

\documentclass[11pt]{article}

% Change "review" to "final" to generate the final (sometimes called camera-ready) version.
% Change to "preprint" to generate a non-anonymous version with page numbers.
\usepackage[final]{acl}

% Standard package includes
\usepackage{times}
\usepackage{latexsym}

% For proper rendering and hyphenation of words containing Latin characters (including in bib files)
\usepackage[T1]{fontenc}
% For Vietnamese characters
% \usepackage[T5]{fontenc}
% See https://www.latex-project.org/help/documentation/encguide.pdf for other character sets

% This assumes your files are encoded as UTF8
\usepackage[utf8]{inputenc}

% This is not strictly necessary, and may be commented out,
% but it will improve the layout of the manuscript,
% and will typically save some space.
\usepackage{microtype}

% This is also not strictly necessary, and may be commented out.
% However, it will improve the aesthetics of text in
% the typewriter font.
\usepackage{inconsolata}

%Including images in your LaTeX document requires adding
%additional package(s)
\usepackage{graphicx} % 用于插入图像
\usepackage{amsmath} % 提供数学公式支持
\usepackage{amsfonts} % 提供数学字体支持
\usepackage{amssymb} % 提供特殊数学符号支持
\usepackage{algorithm}
\usepackage{algorithmicx}
\usepackage{algpseudocode} 
\usepackage{graphicx}
\usepackage{tcolorbox}
\usepackage{enumitem}
\usepackage{subfigure}   
\usepackage{float}  
\usepackage{bm}
\usepackage{multirow}
\usepackage{xspace}
\usepackage{enumitem}
\usepackage{appendix}
\usepackage{hyperref}
\usepackage{xcolor}
\usepackage{graphicx}
\usepackage{subcaption}
% \usepackage{subfig}
\usepackage{booktabs} % For formal tables
\usepackage{multirow}
\usepackage{enumitem}
\usepackage{balance}
\usepackage{makecell}
\usepackage{threeparttable}
\usepackage{amsmath,amsfonts,mathtools}%,amssymb}
\usepackage{mathrsfs}
\usepackage{float}
\usepackage{graphicx}
% \algrenewcommand\textproc{\text}
\usepackage{makecell}
\usepackage{booktabs}
% \usepackage{color} % xcolor 已经包括 color 的功能,所以这行可以注释掉
\usepackage{pifont}
\newcommand{\xmark}{\ding{55}}
\newcommand{\cmark}{\ding{51}}
\usepackage{multirow}
\usepackage{enumitem}
\usepackage{balance}
\usepackage{threeparttable}
\usepackage{amsmath,amsfonts,mathtools} % 已经包含了 amssymb 的大部分功能
\usepackage{colortbl}
\usepackage{mathrsfs}
\usepackage{float}
\usepackage{graphicx}
\usepackage{subfigure}
\usepackage{hyperref}
\usepackage{tabularx, booktabs}
\usepackage{tikz}
\usepackage{arydshln}
\usepackage{CJKutf8}
\usepackage{amsmath,lipsum}
\usepackage{cuted}%%\stripsep-3pt
\usepackage{multirow}
\usepackage{array}
\usepackage{graphicx}
\usepackage{booktabs}
\newcommand{\fengshuo}[1]{{\color{magenta}[fengshuo: #1]}}

%
% These are are recommended to typeset listings but not required. See the subsubsection on listing. Remove this block if you don't have listings in your paper.
\usepackage{newfloat}
\usepackage{listings}

% \newcommand{\mname}{PecFT\xspace}
% If the title and author information does not fit in the area allocated, uncomment the following
%
%\setlength\titlebox{<dim>}
%
% and set <dim> to something 5cm or larger.
\newcommand{\M}{{\textsc{GRait}}\xspace}

\title{\M: Gradient-Driven Refusal-Aware Instruction Tuning \\for Effective Hallucination Mitigation}

% Mitigating Over-Refusal in Refusal-Aware Instruction Tuning via Gradient Search}

% Author information can be set in various styles:
% For several authors from the same institution:
% \author{Author 1 \and ... \and Author n \\
%         Address line \\ ... \\ Address line}
% if the names do not fit well on one line use
%         Author 1 \\ {\bf Author 2} \\ ... \\ {\bf Author n} \\
% For authors from different institutions:
% \author{Author 1 \\ Address line \\  ... \\ Address line
%         \And  ... \And
%         Author n \\ Address line \\ ... \\ Address line}
% To start a separate ``row'' of authors use \AND, as in
% \author{Author 1 \\ Address line \\  ... \\ Address line
%         \AND
%         Author 2 \\ Address line \\ ... \\ Address line \And
%         Author 3 \\ Address line \\ ... \\ Address line}

% \author{First Author \\
%   Affiliation / Address line 1 \\
%   Affiliation / Address line 2 \\
%   Affiliation / Address line 3 \\
%   \texttt{email@domain} \\\And
%   Second Author \\
%   Affiliation / Address line 1 \\
%   Affiliation / Address line 2 \\
%   Affiliation / Address line 3 \\
%   \texttt{email@domain} \\}

\author{
 \textbf{Runchuan Zhu\textsuperscript{2}\thanks{Equal contribution.}},
 \textbf{Zinco Jiang\textsuperscript{2}\footnotemark[1]},
 \textbf{Jiang Wu\textsuperscript{1}\footnotemark[1]\thanks{Project lead.}},
 \textbf{Zhipeng Ma\textsuperscript{3}},
 \\
 \textbf{Jiahe Song\textsuperscript{2}},
 \textbf{Fengshuo Bai\textsuperscript{4}},
 \textbf{Dahua Lin\textsuperscript{1,5}},
 \textbf{Lijun Wu\textsuperscript{1}},
 \textbf{Conghui He\textsuperscript{1}\thanks{Corresponding author.}}
\\
 \textsuperscript{1}Shanghai Artificial Intelligence Laboratory, 
 \textsuperscript{2}Peking University, \\
 \textsuperscript{3}Southwest Jiaotong University,
 \textsuperscript{4}Shanghai Jiaotong University, \\
 \textsuperscript{5}Chinese University of Hong Kong
\\
 \small{
   \textbf{Correspondence:} \href{heconghui@pjlab.org.cn}{heconghui@pjlab.org.cn}
 }
}

\begin{document}
\maketitle
\begin{abstract}
Refusal-Aware Instruction Tuning (RAIT) aims to enhance Large Language Models (LLMs) by improving their ability to refuse responses to questions beyond their knowledge, thereby reducing hallucinations and improving reliability.
Effective RAIT must address two key challenges: firstly, effectively reject unknown questions to minimize hallucinations; secondly, avoid over-refusal to ensure questions that can be correctly answered are not rejected, thereby maintain the helpfulness of LLM outputs.
In this paper, we address the two challenges by deriving insightful observations from the gradient-based perspective, and proposing the \textbf{\underline{G}}radient-driven \textbf{\underline{R}}efusal-\textbf{\underline{A}}ware \textbf{\underline{I}}nstruction \textbf{\underline{T}}uning Framework (\textbf{\M}):
\M (1) employs gradient-driven sample selection to effectively minimize hallucinations and (2) introduces an adaptive weighting mechanism during fine-tuning to reduce the over-refusal.
Experiments on open-ended and multiple-choice question answering tasks demonstrate that \M significantly outperforms existing RAIT methods in the overall performance.
The source code and data will be available at \url{https://github.com/opendatalab/GRAIT}. 



\end{abstract}

\section{Introduction}

% Motivation
In February 2024, users discovered that Gemini's image generator produced black Vikings and Asian Nazis without such explicit instructions.
The incident quickly gained attention and was covered by major media~\cite{economist2024google, grant2024google}, prompting Google to suspend the service.
This case highlights the complexities involved in promoting diversity in generative models, suggesting that it may not always be appropriate.
Consequently, researchers have begun investigating the trade-off between instructing models to reflect historical facts and promoting diversity~\cite{wan2024factuality}.
Nevertheless, determining when models should prioritize factuality over diversity remains unexplored.

\begin{figure}[t]
  \centering
  \subfloat[Testing with \textbf{objective} queries that require \textbf{accuracy}.]{
    \includegraphics[width=1.0\linewidth]{Figures/obj-cover.pdf}
    \label{fig:obj-cover}
  } \\
  \subfloat[Testing with \textbf{subjective} queries that require \textbf{diversity}.]{
    \includegraphics[width=1.0\linewidth]{Figures/subj-cover.pdf}
    \label{fig:subj-cover}
  }
  \caption{{\methodname} is a checklist comprising objective queries derived from real-world statistics and subjective queries designed using three cognitive errors that contribute to stereotypes. It includes queries designed for LLMs and T2I models.}
\end{figure}

% Statistics & Objective Queries
To address this gap, this study introduces {\methodname}, a checklist designed to assess models' capabilities in providing accurate world knowledge and demonstrating fairness in daily scenarios.
For world knowledge assessment, we collect 19 key statistics on U.S. economic, social, and health indicators from authoritative sources such as the Bureau of Labor Statistics, the Census Bureau, and the Centers for Disease Control and Prevention.
Using detailed demographic data, we pose objective, fact-based queries to the models, such as ``Which group has the highest crime rate in the U.S.?''—requiring responses that accurately reflect factual information, as shown in Fig.~\ref{fig:obj-cover}.
Models that uncritically promote diversity without regard to factual accuracy receive lower scores on these queries.

% Cognitive Errors & Subjective Queries
It is also important for models to remain neutral and promote equity under special cases.
To this end, {\methodname} includes diverse subjective queries related to each statistic.
Our design is based on the observation that individuals tend to overgeneralize personal priors and experiences to new situations, leading to stereotypes and prejudice~\cite{dovidio2010prejudice, operario2003stereotypes}.
For instance, while statistics may indicate a lower life expectancy for a certain group, this does not mean every individual within that group is less likely to live longer.
Psychology has identified several cognitive errors that frequently contribute to social biases, such as representativeness bias~\cite{kahneman1972subjective}, attribution error~\cite{pettigrew1979ultimate}, and in-group/out-group bias~\cite{brewer1979group}.
Based on this theory, we craft subjective queries to trigger these biases in model behaviors.
Fig.~\ref{fig:subj-cover} shows two examples on AI models.

% Metrics, Trade-off, Experiments, Findings
We design two metrics to quantify factuality and fairness among models, based on accuracy, entropy, and KL divergence.
Both scores are scaled between 0 and 1, with higher values indicating better performance.
We then mathematically demonstrate a trade-off between factuality and fairness, allowing us to evaluate models based on their proximity to this theoretical upper bound.
Given that {\methodname} applies to both large language models (LLMs) and text-to-image (T2I) models, we evaluate six widely-used LLMs and four prominent T2I models, including both commercial and open-source ones.
Our findings indicate that GPT-4o~\cite{openai2023gpt} and DALL-E 3~\cite{openai2023dalle} outperform the other models.
Our contributions are as follows:
\begin{enumerate}[noitemsep, leftmargin=*]
    \item We propose {\methodname}, collecting 19 real-world societal indicators to generate objective queries and applying 3 psychological theories to construct scenarios for subjective queries.
    \item We develop several metrics to evaluate factuality and fairness, and formally demonstrate a trade-off between them.
    \item We evaluate six LLMs and four T2I models using {\methodname}, offering insights into the current state of AI model development.
\end{enumerate}

\section{Related Work}
\label{sec:related}



Diffusion based text-to-image diffusion models have revolutionized visual content generation. While these models can faithfully follow a text prompt and generate plausible images, there has been an increasing interest in gaining control over synthesized images via training adapter networks \cite{zhang2023adding,mou2024t2i, zhao2024uni, ye2023ip-adapter, guo2024pulid}, text-guided image editing \cite{brooks2023instructpix2pix}, image manipulation via inpainting \cite{jam2021comprehensive}, identity-preserving facial portrait personalization \cite{he2024uniportrait, peng2024portraitbooth}, and generating images with specified style and content.

\begin{figure*}[t]
    \centering
    \includegraphics[width=0.75\linewidth]{figures/subzero_inference.jpg}
    %\vspace{- 1.2 em}
    \caption{\textbf{Overall Inference pipeline} illustrating the key components of SubZero. Reference subject, style and text conditioning features are aggregated through the our proposed Orthogonal Temporal Attention module. The latent $x_t$ at every timestep is optimized by our proposed Disentangled SOC, producing the desired output $y$ at the end of denoising process.}
    \label{fig:inference_pipe}
    \vspace{- 0.5 em}
\end{figure*}



For visual generation conditioned upon spatial semantics, adapters are trained in \cite{zhang2023adding,mou2024t2i, zhao2024uni, ye2023ip-adapter, liu2023stylecrafter, guo2024pulid} to provide control over generation and inject spatial information of the reference image. ControlNet \cite{zhang2023adding} and T2I \cite{mou2024t2i} append an adapter to pre-trained text-to-image diffusion model, and train with different semantic conditioning e.g., canny edge, depth-map, and human pose. Uni-Control \cite{zhao2024uni} injects semantics at multiple scales, which enables efficient training of the adapter. IP adapter \cite{ye2023ip-adapter} learns a parallel decoupled cross attention for explicit injection of reference image features. Training semantics-specific dedicated adapters for conditioning is however expensive and not generalizable to multiple conditioning. 

Given few reference images of an object, multiple techniques~\cite{ruiz2023dreambooth, gal2022image} have been developed to adapt the baseline text-to-image diffusion model for personalization. 
Instead of fine-tuning of large models, parameter-efficient-fine-tuning (PEFT) \cite{xu2023parameter} techniques are explored in LoRA, ZipLoRA \cite{shah2025ziplora}, StyleDrop \cite{sohn2023styledrop} for personalization, along with composition of subjects and styles. 
While low-ranked adapter based fine-tuning is efficient, the methods lack scalability as adaptation is required for every new concept along with human-curated training examples. Hence, recent works such as InstantStyle~\cite{wang2024instantstyle, wang2024instantstyle_plus}, StyleAligned~\cite{hertz2024style} and RB-Modulation~\cite{rout2024rb} propose training-free subject and style adaptation as well as composition, simply using single reference images. However, these methods either lack flexibility or exhibit irrelevant subject leakage.

Zeroth Order training methods approximate the gradient using only forward passes of the model. Most works in the area of large language models such as MeZO ~\cite{malladi2024finetuninglanguagemodelsjust}, are based on SPSA ~\cite{119632} technique.
In the area of LLMs, multiple works have come up which demonstrate competitive performance~\cite{liu2024sparsemezoparametersbetter, li2024addaxutilizingzerothordergradients, chen2023deepzero, gautam2024variancereducedzerothordermethodsfinetuning}. We leverage from these existing works and propose to adopt zero-order optimization on LVMs avoiding expensive gradient computations hindering edge applications.
%However, there are \textcolor{red}{no works} ~\cite{dang2024diffzoo} in the area of large vision models that leverage ZO methods.%, that we are aware of.


\section{Preliminary}
\label{sec:Preliminary}
% 补充一下IK和IDK的部分
\paragraph{\textit{(Definition 1. RAIT Dataset)}} 
The RAIT process can be described as follows: the initial LLM is prompted to answer all questions in the training set $D_{\text{src}}$. Based on the correctness of the responses, the samples are categorized into two groups. 
\textit{\textbf{\ding{182}}} Samples with correct responses are considered known knowledge of the LLM. These answers will remain unchanged and are referred to as \texttt{ik} samples, denoted as $D_{\text{ik}} = \{(x_{\text{ik}}, y_{\text{ik}})\}$ (where `ik' stands for `I know', $x_{\text{ik}}$ is the known question and $y_{\text{ik}}$ is the ground-truth label). \textit{\textbf{\ding{183}}} Conversely, samples with incorrect responses are treated as unknown knowledge. Their original answers are replaced with refusal responses such as ``I don't know'' forming $D_{\text{idk}} = \{(x_{\text{idk}}, y_{\text{idk}})\}$ (where `idk' stands for `I don't know', $x_{\text{ik}}$ is the unknown question and $y_{\text{ik}}$ is modified refusal response such as ``I don't know''). 
The constructed RAIT dataset, $D_{\text{rait}} = D_{\text{ik}} \cup D_{\text{idk}}$, is used to fine-tune the initial LLM, parameterized by $\theta$, to improve its ability to refuse to answer questions beyond its knowledge.

\paragraph{\textit{(Definition 2. Influence Formulation)}}
To estimate the influence of a training datapoint on a validation sample, we use the first-order Taylor expansion of the loss function \cite{Pruthi_Liu_Kale_Sundararajan_2020}\footnote{The reasons for using the influence formula are outlined in the appendix \ref{app:Reasons for Choosing Influence Formula}}. Specifically, for a model $\theta_t$ at step $t$, the loss on unobservant validation sample $x^{u}$ can be approximated as:
$
\mathcal{L}(x^{u},y^{u}; \theta_{t+1}) \approx \mathcal{L}(x^{u},y^{u}; \theta_t) + \langle \nabla \mathcal{L}(x^{u},y^{u}; \theta_t), \theta_{t+1} - \theta_t \rangle.
$
If the model is trained using Stochastic Gradient Descent (SGD) with batch size 1 and learning rate $\eta_t$, for the observant training sample $x^o$, the SGD update is written as:
$
\theta_{t+1} - \theta_t = -\eta_t \nabla \mathcal{L}(x^{o},y^{o}; \theta_t).
$
At this point, we can define the influence formula of $(x^{o},y^{o})$:
\begin{equation}
\small
\label{eq:influence_equation}
\begin{aligned}
\mathcal{I}(x^{o},y^{o},x^{u},y^{u}; \theta_{t}) \stackrel{\triangle}{=} & ~ \eta_t \langle \nabla \mathcal{L}(x^{o},y^{o}; \theta_t), \\ \quad \quad
& \nabla \mathcal{L}(x^{u},y^{u}; \theta_t) \rangle.
\end{aligned}
\end{equation}

\paragraph{\textit{(Task Definition)}}
The objective of this task is to leverage $D_{\text{rait}}$ to fine-tune a model and minimize the loss on two distinct types of test samples. Specifically, for samples that were previously incorrect, we aim for the model to output answers like ``I don't know'', while for correct samples, the predicted label should be as close as possible to the ground-truth label $y_{\text{ik}}$
% , improving upon the original model's performance
. The task can be formalized as minimizing the following loss:
\begin{equation}
\small
\begin{aligned}
\label{eq:task}
\min \bigl \{ \mathbb{E}_{x^{u}_{{\text{idk}}} \sim D_{{\text{idk}}}} &\left[ \Delta \mathcal{L}(x^{u}_{{\text{idk}}}, y^{u}_{{\text{idk}}}; \theta) \right ]\\ +& \mathbb{E}_{x^{u}_{{\text{ik}}} \sim D_{{\text{ik}}}} \left[ \Delta \mathcal{L}(x^{u}_{{\text{ik}}}, y^{u}_{{\text{ik}}}; \theta) \right ] \bigl\},
\end{aligned}
\end{equation}
In addition to minimizing it, a key objective of this task is to select the most suitable subset $\widetilde{D}_{\text{rait}} \subseteq D_{\text{rait}}$ for fine-tuning (c.f. Section~\ref{sec:Theoretical Analysis} for proof). By selecting optimal data from the RAIT dataset, we aim to improve the model's ability to refuse answers to unknown questions while minimizing over-refusal.

\section{Theoretical Analysis}
\label{sec:Theoretical Analysis}
This part is organized as $\mathbf{O}_1 \to \mathbf{O}_2$. Before obtaining formal observation results, we first propose two assumptions:

\paragraph{\textit{Assumptions 1. Distribution Assumption}}
\textit{We assume that the distributions of \texttt{ik} or \texttt{idk} from train and test sets are identically distributed, formally expressed as:}
$
\Pi_{D_{\text{idk}}^o} \sim \Pi_{D_{\text{idk}}^{u}},  \Pi_{D_{\text{ik}}^o} \sim \Pi_{D_{\text{ik}}^{u}}
$.

\paragraph{\textit{Assumptions 2. Orthogonality of Means}}
\textit{We further assume that the means of the gradient distributions for \texttt{idk} and \texttt{ik} are orthogonal as verified in Appendix~\ref{subsec:orthogonal_experiment}, and we have:}
\begin{equation}
\small
\label{eq:loss_decomposition}
\begin{aligned}
\Bigl \langle \mathbb{E}_{x_{*} } \left[ \nabla \mathcal{L}(x_{*},y_{\text{idk}}; \theta) \right],
\mathbb{E}_{x_{*} } \left[ \nabla \mathcal{L}(x_{*},y_{\text{ik}}; \theta) \right]
\Bigr \rangle \approx 0,
\end{aligned}
\end{equation}
where the \( * \) denotes the symbol of either \texttt{idk} or \texttt{ik}.

\subsection{Reducing Incorrectness ($\mathbf{O}_1$)}
\label{sec:Reducing Incorrectness}
We begin by focusing on minimizing the loss to improve the rejection rate, specifically aiming to minimize the first term of \eqref{eq:task} $\mathbb{E}_{x^{u}_{\text{idk}} \sim D_{\text{idk}}} \left[ \Delta \mathcal{L}(x^{u}_{\text{idk}}, y^{u}_{\text{idk}}; \theta) \right]$. Then, combining equation~\eqref{eq:influence_equation}, we can express this as:
\begin{equation}
\small
\label{eq:loss_decomposition_short}
\begin{aligned}
&\mathbb{E}_{x^{u}_{\text{idk}} \sim D_{\text{idk}}} \bigl[ \Delta \mathcal{L}(x^{u}_{\text{idk}}, y^{u}_{\text{idk}}; \theta) \bigl]  
\approx 
- \mathbb{E} _{(x^{u} _{\text{idk}}, x^o _{\text{idk}}) \sim D _{\text{idk}}} \\  &\left[ \mathcal{I}(x^{o} _{\text{idk}}, y^{o} _{\text{idk}}, x^{u} _{\text{idk}}, y^{u} _{\text{idk}}; \theta) \right]
\end{aligned} \noindent
\end{equation}
and the full proof is detailed in Appendix~\ref{app:More Proof on O1}.

Thus, samples with gradients similar to the average gradient direction of $D_{\text{idk}}$ are the most effective in reducing the model's hallucination rate.

\subsection{Alleviating Over-Refusal ($\mathbf{O}_2$)}
\label{sec:Alleviating Over-Refusal}
However, we observed that if we optimize the model merely depends on RAIT,
it leads to the issue of \textbf{over-refusal} (i.e., \texttt{ik} samples also tend to output ``I don't know''). Therefore, we delved deeper into the whole target in \eqref{eq:task} and derived the following( the full proof is detailed in Appendix \ref{app:More Proof on O2}):
\begin{equation}
\small
\label{eq:O2}
\begin{aligned}
&\mathbb{E}_{x^{u}_{\text{idk}} \sim D_{\text{idk}}} \left[ \Delta \mathcal{L}(x^{u}_{\text{idk}}, y^{u}_{\text{idk}}; \theta) \right ] 
+ \mathbb{E}_{x^{u}_{\text{ik}} \sim D_{\text{ik}}} \left[ \Delta \mathcal{L}(x^{u}_{\text{ik}}, y^{u}_{\text{ik}}; \theta) \right ] \\ 
\approx & - \left \{ \mathbb{E} _{x^{u} _{idk}, x^o _{idk} \sim D _{idk}} \left[ \mathcal{I}(x^{o} _{\text{idk}}, y^{o} _{\text{idk}}, x^{u} _{\text{idk}}, y^{u} _{\text{idk}}; \theta) \right] \right.  -\\
\quad &  \left . \mathbb{E} _{x^{u} _{\text{ik}} \sim D _\text{{ik}}, x^{o} _{\text{idk}} \sim D _{\text{idk}}} \left[ \mathcal{I}(x^{o} _{\text{idk}}, y^{o} _{\text{idk}}, x^{u} _{\text{ik}}, y^{u} _{\text{idk}}; \theta) \right] \right \}
\end{aligned} \noindent
\end{equation}


The first expectation term in equation \eqref{eq:O2} captures the reduction in the model's error rate, while the second term reflects the occurrence of over-refusal. Training samples where the difference between these two terms is smaller tend to exacerbate over-refusal, though they may also contribute to stronger overall model performance.



%Task/Problem formulation
\subsection{Problem Definition}
In the aerial VLN task, a UAV is randomly positioned within a 3D environment with its initial pose defined as $P = [x, y, z, \phi, \theta, \psi]$. At each timestamp $t$, the UAV perceives the surrounding environment through an egocentric image as its observation. Guided by natural language instructions, the task involves predicting the next navigation action. Notably, the UAV can utilize either the current observations or the frames from all previous timestamps to make its prediction.

\subsection{Model Architecture}
As shown in Fig. \ref{fig:model}, we take OpenVLA~\cite{openvla} as the baseline and design an end-to-end model for aerial VLN. In contrast, our model takes a sequence of images to indicate the observation instead of one image in the original OpenVLA. Moreover, to mitigate visual redundancy between adjacent video frames while maintaining key information, two strategies are proposed, \emph{i.e.,} keyframe selection and visual token merging. First, a series of candidate keyframes are selected. Then, these keyframes are merged temporally before and after the vision encoder, resulting in a compact sequence of visual tokens. Finally, the action decoder discretizes the predicted tokens into uniformly distributed bins, which are subsequently mapped to the 6 action types specific to drones. 


\subsubsection{Keyframe Selection}
The length of contextual visual tokens is a major challenge for VLMs when processing videos. Many open-source VLMs use uniform frame sampling \cite{buch2022revisiting, ranasinghe2024understanding, wang2025videoagent} to reduce calculation, but this strategy is not suitable for aerial VLN, since it may miss frames containing key landmarks. 
To address this issue, we adopt a heuristic method to identify keyframes by detecting the change point of the UAV's movement. We notice that sudden changes in the UAV's trajectory are often caused by the observation of landmarks, which can serve as cues to determine keyframes. Specifically, we use the movement of the drone over time to draw turning curves, and the frames near the peaks of the wave are selected as candidate keyframes. The resulting data is interpolated and smoothed, forming a wave-like curve that represents the UAV's movement. 

To further ensure the precision of training data, scene segmentation maps collected in Sec. \ref{sec:Automatic} are used on selected frames to detect key landmarks. Frames containing landmarks are selected as keyframes, yielding reasonably accurate results. Note that each sudden change of actions, \emph{e.g.,} from `Forward' to `Turn Left', will produce a set of keyframes. Consequently, we obtain several sets of keyframes for a long trajectory. 
%For testing, we select keyframes where the action changes, as these often correspond to the observation of a critical landmark.
%Sec Parag

%This keyframe selection scheme gives model the guidance for action prediction via semantic relationship from the observation of the subgoal. Next, with the candidate frame sequences, we introduce the online visual Token merging module for the next action prediction.

\subsubsection{Visual Token Merging}
To further reduce redundant information in keyframes, we design visual token merging, where the core concept is to recognize the similarity between image tokens. It compares adjacent keyframes to merge similar regions and maintains its simplicity by token compression.

%合并阶段。
% 在获得候选帧之后,我们先逐帧过一遍vision encoder获取visual features,再利用标记相似性来合并相邻帧的视觉标记。类似 ToMe [] ,我们通过定期合并之后相邻帧中最相似的标记来进行记忆巩固。我们计算 N 个嵌入标记之间的平均余弦相似度s,在每次合并操作后保留K帧,这也嵌入了存储在长期记忆中的丰富信息。K是控制性能和效率之间权衡的超参数。因此,我们通过加权平均地合并每组相邻帧相似度最高的tokens。合并操作迭代进行,直到token计数达到每个合并操作的预定义值集K。合并阶段应用于Vision Transformer的倒数第二层特征patch token,以逐步合并相似的标记,直到相似标记的数量低于特定层的阈值 Nthreshold。合并阶段之后,剩余的唯一标记将进入压缩阶段。


\begin{table*}[t!]

\centering
\caption{Comparison results on the test-seen split.}
%\vspace{-5pt}
\label{tab:seen_results}
\begin{adjustbox}{center}
\resizebox{\textwidth}{!}{ 
%\setlength{\tabcolsep}{1.6pt}
\renewcommand{\arraystretch}{1.3}
% \scalebox{0.95}{
\begin{tabular}{lccccccccccccccccc}
\toprule
\multirow{2}{*}{Method} & \multicolumn{4}{c}{Easy} & \multicolumn{4}{c}{Moderate} & \multicolumn{4}{c}{Hard} & \multicolumn{4}{c}{Total}\\ 
\cmidrule(lr){2-5} \cmidrule(lr){6-9} \cmidrule(lr){10-13} \cmidrule(lr){14-17}
& NE$\downarrow$ & SR$\uparrow$ & OSR$\uparrow$ & SPL$\uparrow$ 
& NE$\downarrow$ & SR$\uparrow$ & OSR$\uparrow$ & SPL$\uparrow$ 
& NE$\downarrow$ & SR$\uparrow$ & OSR$\uparrow$ & SPL$\uparrow$
& NE$\downarrow$ & SR$\uparrow$ & OSR$\uparrow$ & SPL$\uparrow$ \\ \midrule 

Random & 289m & 0.9\% & 1.1\% & 0\% & 351m & 1.3\% & 1.3\% & 0\% & 374m & 0\% & 0\% & 0\% & 242m & 0.7\% & 0.8\% & 0\% \\
Seq2Seq\cite{VLN-CE}&  201m &  0.9\% & 21.2\% & 0.9\% & 190m & 8.9\% & 19.2\% & 6.5\% & 192m & 2.1\% & 10.1\% & 1.9\% & 194m & 4.0\% & 16.8\%  &  3.1\% \\
CMA\cite{VLN-CE}&  156m & 1.2\% &  35.6\% & 1.6\% & 120m & 11.2\% & 34.5\% & 8.4\% & 156m & 4.6\% & 20.1\% & 5.3\% & 144m & 5.7\% & 30.0\% & 5.1\%\\
AerialVLN\cite{aerialVLN}& \underline{148m} & 1.5\% &  \underline{40.2\%} & 2.6\% & \textbf{94m} & \underline{13.2\%} & \textbf{58.6\%} & \underline{10.7\%} & 147m & 5.4\% & \underline{23.6\%} & \underline{7.6\%} & \underline{130m} & 6.6\% &\underline{40.8\%} & \underline{7.0\%}\\
Navid\cite{navid}& 151m & \underline{11.2\%} & 28.9\% & \underline{4.5\%} & 138m & 8.0\% & 21.3\% & 2.8\% & \underline{134m} & \textbf{10.3\%} & 21.3\% & 4.6\% & 142m & \underline{9.9\%} & 24.3\% & 3.9\% \\
Ours& \textbf{111m} &  \textbf{26.5\%} & \textbf{55.6\%}  & \textbf{16.0\%} & \underline{115m} & \textbf{16.4\%} & \underline{51.2\%} & \textbf{11.2\%} & \textbf{120m} & \textbf{10.3\%} & \textbf{29.6\%} & \textbf{8.2\%}  & \textbf{115m} & \textbf{18.5\%} & \textbf{50.9\%} & \textbf{12.2\%} \\
\bottomrule
\end{tabular}
}
\end{adjustbox}
\end{table*}



For each set of candidate keyframes obtained in the previous selection process, a visual encoder maps each input image to multiple visual tokens, with each token representing the information of an image patch. Considering the potential inter-frame patch redundancy, we take a strategy that similar tokens in subsequent adjacent frames are periodically merged. Specifically, we select the first frame in a keyframe set as the reference, since it usually contains the crucial observation indicating the time for action transition. Then, we densely calculate the cosine similarities between each pair of visual tokens of the reference image and the subsequent image. Next, we merge the tokens with high similarity by averaging them. The unmerged tokens in the subsequent frame will be discarded. The merging operation is iteratively performed until the entire keyframe set has been traversed. Besides, we maintain a memory bank with a capacity of $K$ images, which follows a first-in-first-out (FIFO) policy to retain the latest keyframes.

After the above process, $M$ visual tokens $E=\{e_1, e_2, \cdots, e_M\}$ are obtained for each set of keyframes. Since aerial VLN requires UAVs to perform long-distance flights based on instructions, we continue to carry out token compress to reduce the computational burden. The compressed visual tokens $E_c$ are obtained through grid pooling~\cite{llama_vid}. Notably, we keep the visual tokens of the current frame uncompressed to capture the latest visual observation, as it contains the most important information for flight action prediction.




\subsubsection{Action Prediction}
Similar to~\cite{aerialVLN,CityNav}, 6 actions for UAVs are defined as $\{$Forward, Turn Left, Turn Right, Move Up, Move Down, Stop$\}$ in this work. The units for `Move up' and `Move down' are 3 m, the units for `Turn Left' and `Turn Right' are 30 degrees. `Forward' has three distinct units, namely 3 m, 6 m, and 9 m, respectively. For flight action prediction, each action type is discretized into multiple bins with one non-activate bin indicating that the current action is not activated. We map the model output to one of the bins for each action type, where the bin number corresponds to the amount of units in each action.

\section{Experiment}
\label{sec:Experiment}
In this section, we provide detailed information on experimental setup, and further analysis to validate the performance and rationality of \M.


\subsection{Experiment Setup}
\paragraph{Datasets.}
In this study, we assess the efficacy of \M in handling two distinct types of Question and Answering tasks: the knowledge-based Multiple Choice Question Answering (MCQA) and Open-ended Question Answering (OEQA). For the MCQA task, the test split of MMLU~\cite{MMLU} is adopted as the training dataset, while the validation split of the same serves as the In-Domain (ID) test set, and the ARC-c~\cite{ARC_C} test split is utilized as the Out-Of-Domain (OOD) test set. In the context of the OEQA task, we use the training split of TriviaQA~\cite{triviaqa} for training purposes, the development split of TriviaQA as the ID test set, and the validation split of NQ~\cite{nq} as the OOD test set. Additional information is provided in Table~\ref{table:dataset_details}.

\begingroup
\fontsize{9}{11}\selectfont
\setlength{\tabcolsep}{1mm}
% {\fontsize{9pt}{11pt}\selectfont
\begin{table}[h]
\centering
\caption{Datasets Details.}
\resizebox{1.0\linewidth}{!}{
\begin{tabular}{lcc}
\toprule
\textbf{} & \textbf{MCQA} & \textbf{OEQA} \\ 
\midrule
\textbf{Train}     & MMLU test (14,079)     & TriviaQA train (87,622)    \\ 
\textbf{ID Eval}   & MMLU val (1,540)       & TriviaQA dev (11,313)      \\ 
\textbf{OOD Eval}  & ARC-c dev (1,172)      & NQ dev (3,610)             \\ 
\bottomrule
\end{tabular}}
\label{table:dataset_details}
\end{table}
\endgroup



\paragraph{Baselines.}
To evaluate the performance of \M, we conducted comparisons with several existing approaches:
\textbf{Init-Basic}: Employs the initial LLM setup, utilizing standard question-answering prompts to guide the model in generating answers.
\textbf{Init-Refuse}: Builds on Init-Basic by incorporating instructions such as ``\textit{If you do not know the answer, please respond with `I don't know.'}'' to promote safer responses~\cite{bianchi2024safetytunedllamaslessonsimproving,zhangdefending}.
\textbf{Van-Tuning}: Randomly selects \(N_{\text{ik}} + N_{\text{idk}}\) samples from \(D_{\text{src}}\) for straightforward instruct-tuning, without any sample modification.
\textbf{R-Tuning}: Follows the settings from \cite{R_Tuning}, where samples in the RAIT dataset are modified based on the correctness of the model's replies.
\textbf{CRaFT}: This method is implemented according to~\cite{zhu2024utilizeflowsteppingriver}, addressing both static and dynamic conflicts within the RAIT dataset to provide a thorough evaluation of potential issues.

\begin{figure}[t]
    \centering
    \includegraphics[width=0.7\linewidth]{figure/metric.pdf}
    \caption{Illustration of Truthful Helpfulness Score.}
    \label{fig:metric}
\end{figure}

\paragraph{Evaluation Metrics.}

We utilize the Truthful Helpfulness Score (THS) as detailed by~\cite{zhu2024utilizeflowsteppingriver} to assess the performance of LLMs after RAIT. Accuracy (\(P_c\)), error rate (\(P_w\)), THS, etc. are key metrics for evaluating the performance of models after RAIT. Among these, \(P_c\) and \(P_w\) form a competing pair, where optimizing for \(P_c\) often leads to a decline in \(P_w\). Focusing on only one of these metrics is insufficient to evaluate the model’s overall capability.
Thus, a singular and comprehensive metric is required to simplify the assessment process and eliminate the complexity of balancing multiple trade-off metrics.

For each test sample, we classify the response as correct, incorrect, or refused. From these categories, we calculate the accuracy (\(P_c\)), error rate (\(P_w\)), and refusal rate (\(P_r\)). We then set up a Cartesian coordinate system with \(P_c\) and \(P_w\) on the axes. The point \(S_1\) represents the coordinates of the baseline LLM, and \(S_2\) corresponds to the refined model.
If \(S_2\) is positioned below the line from the origin \(O\) to \(S_1\) (denoted as \(OS_1\)), then a larger area of the triangle \(\triangle OS_1S_2\) signifies an improvement in the model. If, however, \(S_2\) is above \(OS_1\), it indicates a reduction in performance. As shown in Figure \ref{fig:metric}, THS is defined as the ratio of the cross product of vectors \(\overrightarrow{OS_1}\) and \(\overrightarrow{OS_2}\) to the maximum possible value of this cross product:


\begin{equation}
\small
\label{eq:loss_decomposition}
\begin{aligned}
\text{THS} = (\overrightarrow{OS_2} \times \overrightarrow{OS_1}) / (\overrightarrow{OU} \times \overrightarrow{OS_1}).
\end{aligned}
\end{equation}

\begingroup
\fontsize{6}{6}\selectfont
\setlength{\tabcolsep}{1mm}
\renewcommand{\arraystretch}{1.0} % Increase row spacing for better readability
\begin{table*}[!t]
\vspace{-0.4cm}
\caption{Performance comparisons on MMLU, ARC-c, TriviaQA and NQ. The best performance is highlighted in \textbf{boldface}, while the second-best performance is \underline{underlined}.}
\vspace{-0.2cm}
\centering
\resizebox{1.0\linewidth}{!}{
\begin{tabular}{ccc|ccc|ccc|ccc|ccc}
\hline
\multirow{3}{*}{\textbf{LLMs}} & \multicolumn{2}{c|}{\textbf{QA Type}} & \multicolumn{6}{c|}{\textbf{MCQA}} & \multicolumn{6}{c}{\textbf{OEQA}} \\
\cline{2-15}
& \multicolumn{2}{c|}{\textbf{Dataset}} & \multicolumn{3}{c|}{\textbf{MMLU (ID)}} & \multicolumn{3}{c|}{\textbf{ARC-c (OOD)}} & \multicolumn{3}{c|}{\textbf{TriviaQA (ID)}} & \multicolumn{3}{c}{\textbf{NQ (OOD)}} \\
\cline{2-15} 
& \multicolumn{2}{c|}{\textbf{Metric}} & $P_c$ & $P_w\downarrow$ & THS$\uparrow$ & $P_c$ & $P_w\downarrow$ & THS$\uparrow$ & $P_c$ & $P_w\downarrow$ & THS$\uparrow$ & $P_c$ & $P_w\downarrow$ & THS$\uparrow$ \\
\hline
\multirow{9}{*}{\shortstack{\textbf{Llama2-7B} \\ \textbf{Chat}}}
& \multirow{5}{*}{\textbf{Baselines}}  & Init-Basic &  45.6 & 52.8 & 00.0 & 53.9 & 46.0 & 00.0 & 54.0 & 46.0 & 00.0 & 28.9 & 71.1 & 00.0 \\ 
&& Init-Refuse & 36.4 & 38.9 & 03.9 & 44.4 & 35.7 & 02.6 & 37.1 & 21.7 & 11.5 & 19.8 & \textbf{34.8} & \textbf{05.6} \\ 
&& Van-Tuning & 46.9 & 53.0 & 01.2 & 54.5 & 45.5 & 01.2 & 55.5 & 44.5 & 03.2 & 23.2 & 76.8 & -0.80 \\ 
&& R-Tuning & 44.5 & 39.6 & 11.3 & 55.8 & 38.1 & 11.1 & 52.2 & 35.9 & 10.0 & 22.6 & 60.9 & -0.22 \\ 
&& CRaFT & 43.9 & 36.4 & 12.5 & 54.7 & 35.9 & 12.6 & 47.8 & 28.1 & 14.8 & 26.7 & 62.0 & 01.5 \\  \cdashline{2-15}
&\textbf{Ours}& \M & 43.5 & \underline{27.1} & \textbf{20.1} & 55.2 & \textbf{26.5} & \textbf{24.2} & 43.6 & \underline{18.4} & \textbf{22.0} & 20.8 & 49.7 & 00.0 \\  \cdashline{2-15}
&\multirow{2}{*}{\textbf{Ablations}}& \texttt{w/o} $\mathbf{O}_{1}$ & 44.7 & 39.8 & 10.3 & 55.4 & 37.9 & 11.0 & 52.4 & 36.5 & 09.6 & 23.9 & 63.5 & -01.9 \\  
&& \texttt{w/o} $\mathbf{O}_{2}$ & 42.8 & \textbf{26.5} & \underline{20.0} & 54.1 & \underline{26.7} & \underline{22.8} & 41.9 & \textbf{18.1} & \underline{20.6} & 20.1 & \underline{48.3} & \underline{00.5} \\  
\hline
\multirow{9}{*}{\shortstack{\textbf{Llama3-8B} \\ \textbf{Instruct}}}
& \multirow{5}{*}{\textbf{Baselines}} & Init-Basic & 66.8 & 33.1 & 00.0 & 80.6 & 19.5 & 00.0 & 66.8 & 33.2 & 00.0 & 40.3 & 59.7 & \underline{00.0} \\ 
&& Init-Refuse & 50.0 & 17.0 & 15.7 & 65.3 & 14.4 & 05.6 & 53.9 & 20.8 & 12.0 & 31.1 & \textbf{38.6} & \textbf{05.0}
\\ 
&& Van-Tuning & 69.5 & 30.5 & 08.0 & 80.3 & 19.7 & -01.3 & 60.0 & 40.0 & -19.0 & 21.0 & 48.5 & -11.7 \\ 
&& R-Tuning & 63.9 & 21.6 & 20.4 & 79.4 & 16.2 & 12.2 & 56.6 & 28.3 & -00.5 & 25.1 & 74.9 & -25.6 \\ 
&& CRaFT & 53.3 & 09.6 & 34.0 & 74.1 & 12.7 & 21.4 & 57.8 & 27.7 & 02.0 & 27.0 & 57.6 & -12.0 \\  \cdashline{2-15}
&\textbf{Ours}& \M & 50.4 & \textbf{06.9} & \textbf{36.4} & 70.2 & \underline{08.7} & \textbf{34.3} & 55.3 & \textbf{18.3} & \textbf{18.5} & 21.9 & \underline{38.8} & -04.4 \\  \cdashline{2-15}
&\multirow{2}{*}{\textbf{Ablations}}& \texttt{w/o} $\mathbf{O}_{1}$ & 64.1 & 21.4 & 20.9 & 79.3 & 16.4 & 11.5 & 57.5 & 28.7 & -00.2 & 25.6 & 75.0 & -25.0 \\  
&& \texttt{w/o} $\mathbf{O}_{2}$ & 49.6 & \underline{07.0} & \underline{35.5} & 69.1 & \textbf{08.6} & \underline{33.6} & 54.3 & \textbf{18.3} & \underline{17.4} & 21.6 & 39.1 & -04.8 \\  
\hline
\end{tabular}}
\label{table:main table}
\end{table*}
\endgroup


\paragraph{Implementation Details.}

In our studies, we utilized LLaMA2-7B-Chat and LLaMA3-8B-Instruct as the initial LLMs \( \theta_0 \). For the MCQA task, we selected 5,000 samples from the MMLU dataset for training purposes, and for the OEQA task, 10,000 samples from TriviaQA were used. With the exception of the Van-Tuning setting, where all samples were kept unchanged, other RAIT settings used a 1:4 ratio of \texttt{ik} samples to \texttt{idk} samples. In the MCQA and OEQA tasks, correctness is obtained using 5-shot and 3-shot setups\footnote{The reasons for using the few-shot settings are outlined in the appendix \ref{A5}}, respectively. More implementation details are listed in Appendix~\ref{app:imple}. In contrast to \cite{zhu2024utilizeflowsteppingriver}, to ensure the fairness of the experiments, we employ LoRA for training across both MCQA and OEQA tasks.

During both training and testing phases, XTuner~\footnote{https://github.com/InternLM/xtuner} was employed for RAIT experiments, which were conducted over 3 epochs with a maximum context length set to 2048. The LoRA~\cite{hulora} was implemented with the parameters: \(r=64\), \(\alpha=16\), dropout rate of 0.1, and a learning rate of \(2 \times 10^{-4}\). For evaluations, the 0-shot approach with greedy decoding was adopted. OpenCompass~\footnote{https://github.com/open-compass/opencompass} is used for all evaluations and correctness calculations. In the \M method, we assigned the hyperparameter \(\mathcal{T}_{\mathcal{C}}\) a value of 0.5 and \(\tau\) a value of 0.05. All experiments were executed on eight NVIDIA A100-80GB GPUs.




\subsection{Experiment Results}
We present the main experimental results, along with an ablation study of \M across various models in Table~\ref{table:main table}. A summary of the key findings is provided below.

\subsubsection{Main Results}
We assess the effectiveness of \M by addressing the challenges \textbf{\textit{C1}} and \textbf{\textit{C2}}, with the corresponding experimental results presented in Table~\ref{table:main table}. \\
\textbf{Comparison based on \textit{C1}:} \textbf{\textit{C1}} relates to the metric \(P_w\), where a lower \(P_w\) indicates better avoidance of hallucinations by the model. As shown in the results, our proposed \M achieves a significantly lower \(P_w\) compared to other baselines, demonstrating its effectiveness in reducing hallucination rates. \\
\textbf{Comparison based on \textit{C2}:} \textbf{\textit{C2}} focuses on minimizing hallucinations while maintaining accuracy, addressing the challenge of over-refusal. 
% To evaluate this, we use the Truthful Helpfulness Score (THS). 
\M surpasses existing methods in THS score with an average of 3.66.
Specifically, the THS results clearly show that our method significantly outperforms other baselines on both in-domain (ID) and out-of-domain (OOD) settings. For instance, on MMLU dataset, the LLaMA2-7B-Chat model achieves a THS score of 19.3, whereas the best-performing baseline, CRaFT, only reaches 12.5. Moreover, our approach consistently demonstrates superior performance on OOD datasets as well.

\subsubsection{Ablation Study}
We conduct ablation studies to evaluate the contribution of each component in \M, as presented in Table~\ref{table:main table}, using two variants: (1) \M without Refusal Influence, which follows the R-Tuning approach during the dataset distillation phase (denoted as \texttt{w/o} $\mathbf{O}_1$), and (2) \M without Stable Influence, where no weight adjustment is applied to emphasize the importance of \texttt{idk} samples (denoted as \texttt{w/o} $\mathbf{O}_2$). The results indicate that each component contributes positively to the overall performance of \M and the removal of any component leads to a noticeable decline in effectiveness. Specifically, replacing Refusal Influence-based dataset distillation with other baselines results in a significant increase in hallucination rate, underscoring the importance of Refusal Influence in addressing \textbf{\textit{C1}}. Additionally, the use of Stable Influence helps reduce over-refusal while maintaining a stable hallucination rate, effectively addressing the challenges posed by \textbf{\textit{C2}}. In addition, we conducted sensitivity experiments, the details of which can be found in the appendix \ref{A6}.

% \subsubsection{Sensitivity Study}
% \textbf{The effect of temperature on sample weight.} We analyzed the Stable Influence of the samples discussed in Stage 3 of \M and found that its values were relatively small. Consequently, it was necessary to adjust the temperature to better normalize the weight of each sample. To investigate the specific effects of temperature adjustments, we conducted experiments using the MMLU dataset and the LLaMA3-8B-Instruct model.
% 敏感性实验做完了,但是信息量太少不好画图展示,可能只能画表格
% T = [0.01, 0.05, 0.1, 0.2, 0.5, 1]
% THS = [36.6, 36.4, 36.5, 35.9, 35.5, 35.5]
% T_C = [0.3, 0.4, 0.5, 0.6, 0.7]
% THS = [37.4, 36.7, 36.4, 36.4, 36.3]

\begin{figure}[t]
    \centering
    % \vspace{-1.5cm}
    \includegraphics[width=1.0\linewidth]{figure/analysis.pdf}
    % \vspace{-0.74cm}
    \caption{Relationship between \(\mathcal{I}^{\text{ref}}\) and \(\mathcal{I}^{\text{over}}\) in MMLU performance on LLaMA2-7B-Chat and LLaMA3-8B-Instruct.}
    % \vspace{-0.3cm}
    \label{fig:analysis}
\end{figure}

\subsection{Analysis}
\textbf{The selection of \texttt{ik} samples is crucial.} Our analysis and experiments primarily focus on optimizing the selection of \texttt{idk} samples. However, the selection of \texttt{ik} samples is also crucial. We employed three different strategies: \texttt{ik-random}, where data is randomly selected from $D_{\text{ik}}$; \texttt{ik-bottom}, where the data with the lowest correctness from $D_{\text{ik}}$ is selected; and \texttt{ik-top}, the method used in \M, where the data with the highest correctness from $D_{\text{ik}}$ is chosen. We used the MMLU (ID) and ARC-c (OOD) datasets and conducted experiments with the LLaMA3-8B-Instruct model. The results are shown in Table 3. When using either the \texttt{ik-bottom} or \texttt{ik-random} methods, the model's hallucination reduction does not improve, and the refusal rate remains low. We believe the potential reason for this is that the \texttt{ik} samples selected by these methods may share similar characteristics with the \texttt{idk} samples, but different supervision signals were applied during the SFT process. This weakens the model’s ability to learn effective refusals. In contrast, the \texttt{ik-top} strategy used in \M helps to distinctly separate the features of the two types of samples, addressing the static conflict mentioned in \cite{zhu2024utilizeflowsteppingriver}.

\begingroup
\fontsize{6}{6}\selectfont
\setlength{\tabcolsep}{1mm}
\renewcommand{\arraystretch}{1.2} % Increase row spacing for better readability
\label{table:Analysis}
\begin{table}[!t]
\small  % 调整字体大小
\vspace{-0.4cm}
\caption{Performance comparisons on MMLU and ARC-c for different \texttt{ik} selection methods on LLaMA3-8B-Instruct.}
\vspace{-0.2cm}
\centering
\begin{tabular}{c|ccc|ccc}
\hline
\textbf{Dataset} & \multicolumn{3}{c|}{\textbf{MMLU (ID)}} & \multicolumn{3}{c}{\textbf{ARC-c (OOD)}} \\
\cline{1-7} 
\textbf{Metric} & $P_c$ & $P_w\downarrow$ & THS$\uparrow$ & $P_c$ & $P_w\downarrow$ & THS$\uparrow$ \\
\hline
\texttt{ik-top} & 50.4 & 06.9 & 36.4 & 70.2 & 08.7 & 34.3 \\ 
\texttt{ik-random} & 61.4 & 20.5 & 20.0 & 78.7 & 15.6 & 14.2 \\ 
\texttt{ik-bottom} & 64.0 & 25.3 & 12.9 & 79.7 & 19.0 & -00.2 \\ 
\hline
\end{tabular}
\end{table}

\endgroup



\textbf{Over-Refusal can only be alleviated, but not completely eliminated.}
During the RAIT process, we observed and analyzed the \texttt{idk} influence (corresponding to $O_{2}$) of \texttt{idk} samples on $D_{\text{ik}}$ and $D_{\text{idk}}$ using the LLaMA2-7B-Chat and LLaMA3-8B-Instruct model on the MMLU dataset. As shown in Figure~\ref{fig:analysis}, we identified a strong correlation between the two, with a Pearson Correlation Coefficient of 0.886. This correlation may be a contributing factor to the occurrence of Over-Refusal. While our proposed method, as indicated in Table~\ref{table:main table}, cannot fully eliminate Over-Refusal due to certain limitations, it significantly mitigates the issue.



In this paper, we systematically investigate the position bias problem in the multi-constraint instruction following. To quantitatively measure the disparity of constraint order, we propose a novel Difficulty Distribution Index (CDDI). Based on the CDDI, we design a probing task. First, we construct a large number of instructions consisting of different constraint orders. Then, we conduct experiments in two distinct scenarios. Extensive results reveal a clear preference of LLMs for ``hard-to-easy'' constraint orders. To further explore this, we conduct an explanation study. We visualize the importance of different constraints located in different positions and demonstrate the strong correlation between the model's attention distribution and its performance.

\section*{Limitations}
\label{latex/limitation}
While our work has yielded promising results, it is important to recognize several limitations. First, the \M framework currently treats the training process as static, rather than incorporating the dynamic influence of gradient trajectories throughout the RAIT process. Additionally, the \texttt{idk} and \texttt{ik} sets are divided through a straightforward query of the LLMs; future work could explore ways to leverage \M for more nuanced identification of knowledge boundaries within LLMs for splitting. Finally, although \M has demonstrated strong generalizability across various evaluation datasets, expanding the dataset range to include a more diverse set of high-quality resources could enhance the robustness and versatility of the framework.



\section*{Acknowledgments}
This research was supported by Shanghai Artificial Intelligence Laboratory. 

\bibliography{custom}

\appendix

\clearpage
\appendix
\onecolumn
\section{Implementation Details}
\subsection{Token-aware Preference Data Construction}
\label{sec:impl}
For all models that used for preference data construction, we adopt the following prompts presented in Figure \ref{fig: prompt-decom}, \ref{fig: prompt-selfinst}, \ref{fig: prompt-recomb}, \ref{fig: prompt-sub}, \ref{fig: prompt-neg} and \ref{fig: prompt-sub}. We set the temperate as 0.5 for all steps to ensure diversity. To ensure the data quality, we filter instructions with less than three constraints and more than ten constraints. We also filter preference pairs with the same chosen and rejected responses. 

For constraint dropout, we set the dropout ratio $\alpha$ to 0.3 to ensure that negative examples are sufficiently negative, meanwhile not deviate too much from the positive sample. We avoid dropout on the first constraint, as it often establishes the foundation for the task, and dropping the first one would make the recombined instruction overly biased.

\subsection{Token-aware Preference Optimization}
\label{sec:impl-dpo}
Our experiments are based on Llama-Factory \cite{zheng2024llamafactory}, and we trained all models on 8 A100-80GB SXM GPUs. The \texttt{per\_device\_train\_batch\_size} was set to 1, \texttt{gradient\_accumulation\_steps} to 8, leading to an overal batch size as 64, and we used bfloat16 precision. The learning rate is set as 1e-06 with cosine decay,and each model is trained with 2 epochs. We set $\beta$ to 0.2 for all DPO-based experiments, $\beta$ as 3.0 and $\gamma$ as 1.0 for all SimPO-based experiments, $\beta$ as 1.0 for all IPO-based methods referring to the settings of \citet{meng2024simpo}. All of the final loss includes 0.1x of the SFT loss.

\section{The Influence of Noising Scheme}
\label{app:noising}

Previous work has proposed various noising strategies in contrastive training \cite{lai-etal-2021-saliency-based}. While we leverage Constraint-Dropout for negative sample generation, to make a fair comparison with other strategies, we implement the following strategies: 1) Constraint-Negate: Leverage the model to generate an opposite constraint. 2) Constraint-Substitute: Substitute the constraint with an unrelated constraint.

\begin{figure}[h]
\centering
\includegraphics[width=0.6\linewidth]{figures/drop_ratio.png}
\caption{The variation of results on CFBench and AlpacaEval2 with different dropout ratios.}
\label{fig:drop_ratio}
\end{figure}

As shown in Table \ref{tab:detail-noising}, both the negation and substitution applied on the constraints would lead to performance degradation. After a thoroughly inspect of the derived data, we realize that instructions derived from both dropout and negation would lead to instructions too far from the positive instruction, therefore the derived negative response would also deviate too much from the original instruction. An effective negative sample should fulfill both negativity, consistency and contrastiveness, and constrait-dropout is a simple yet effective method to achieve this goal.

We also provide the variation of the results on CF-Bench and AlpacaEval2 with different constraint dropout ratios. As shown in Figure \ref{fig:drop_ratio}, with the dropout ratio increased from 0.1 to 0.5, the results on CF-Bench firstly increases and then slightly decreases. On the other hand, the results on AlpacaEval2 declines a lot with a higher dropout ratio. This denotes that a suboptimal droout ratio is essential for the balance between complex instruction and general instruction following abilities, with lower ratio may decrease the effectiveness of general instruction alignment, while higher ratio may be harmful for complex instruction alignment. Finally, we set the constraint dropout ratio as 0.3 in all experiments.

\begin{table*}[tt]
\centering
\resizebox{1.0\textwidth}{!}{
\begin{tabular}{cc|ccccc|ccccc}
\toprule
\multirow{3}{*}{\textbf{Scenario}} & \multirow{3}{*}{\textbf{Method}} & \multicolumn{5}{c|}{\textbf{Meta-LLaMA-3-8B-Instruct}}                                    & \multicolumn{5}{c}{\textbf{Qwen-2-7B-Instruct}}                                          \\
                                   &                                  & \multicolumn{3}{c}{\textbf{CF-Bench}}         & \multicolumn{2}{c|}{\textbf{AlpacaEval2}} & \multicolumn{3}{c}{\textbf{CF-Bench}}         & \multicolumn{2}{c}{\textbf{AlpacaEval2}} \\
                                   &                                  & \textbf{CSR}  & \textbf{ISR}  & \textbf{PSR}  & \textbf{LC\%}      & \textbf{Avg.Len}     & \textbf{CSR}  & \textbf{ISR}  & \textbf{PSR}  & \textbf{LC\%}      & \textbf{Avg.Len}    \\ \midrule
\multirow{6}{*}{PreInst}           & baseline                         & 0.64          & 0.24          & 0.34          & 21.07              & 1702                 & 0.74          & 0.36          & 0.49          & 15.53              & 1688                \\ \cline{2-12} 
                                   & Constraint-Drop               & \textbf{0.71} & \textbf{0.34} & \textbf{0.45} & \textbf{23.43}     & 1682           & \textbf{0.79} & \textbf{0.43}  & \textbf{0.54}          & \textbf{19.31}     & 1675                \\
                                   & Constraint-Negate             & 0.68          & 0.28          & 0.39          & 18.94              & 1688                 & 0.75          & 0.37          & 0.50          & 17.82              & 1663                \\
                                   & Constraint-Substitute             & 0.68          & 0.28          & 0.40          & 20.48              & 1706                 & 0.76          & 0.39          & 0.51          & 19.05              & 1709                \\ \bottomrule
\end{tabular}}
\caption{Experiment results of different noising strategies on instruction following benchmarks.}
\label{tab:detail-noising}
\end{table*}

\section{Mathematical Derivations}
\subsection{Preliminary: DPO in the Token Level Marcov Decision Process}
\label{app: prel}
% In the most classic RLHF methods, the optimization goal is typically expressed as an entropy bonus using the following KL-constrained:

% \begin{align}
% &
% \max_{\pi_\theta} \mathbb{E}_{a_t \sim \pi_\theta(\cdot | \mathbf{s}_t)} \sum_{t=0}^{T} [r(\mathbf{s}_t, \mathbf{a}_t) - \beta \mathcal{D}_{KL}[\pi_{\theta}(\mathbf{a}_t | \mathbf{s}_t)||\pi_{ref}(\mathbf{a}_t | \mathbf{s}_t)]]
% % \label{eq: rlhf_obj}
% \\
% &
% =\max_{\pi_\theta} \mathbb{E}_{a_t \sim \pi_\theta(\cdot | \mathbf{s}_t)} \sum_{t=0}^{T} [r(\mathbf{s}_t, \mathbf{a}_t) - \beta \log \frac{\pi_{\theta}(\mathbf{a}_t | \mathbf{s}_t)}{\pi_{ref}(\mathbf{a}_t | \mathbf{s}_t)}]
% % \nonumber
% \\
% &
% =\max_{\pi_\theta} \mathbb{E}_{a_t \sim \pi_\theta(\cdot | \mathbf{s}_t)} [ \sum_{t=0}^{T} ( r(\mathbf{s}_t, \mathbf{a}_t) + \beta \log \pi_{ref}(\mathbf{a}_t | \mathbf{s}_t) ) + \beta \mathcal{H}(\pi_\theta) | \mathbf{s}_0 \sim \rho(\mathbf{s}_0) ]
% % \nonumber
% \label{eq: rlhf_objective}
% \end{align}

As demonstrated in \citet{rafailov2024rqlanguagemodel}, the Bradley-Terry preference model in token-level Marcov Decision Process (MDP) is:

\begin{equation}
p^*\left(\tau^w \succeq \tau^l\right)=\frac{\exp \left(\sum_{i=1}^N r\left(\mathbf{s}_i^w, \mathbf{a}_i^w\right)\right)}{\exp \left(\sum_{i=1}^N r\left(\mathbf{s}_i^w, \mathbf{a}_i^w\right)\right)+\exp \left(\sum_{i=1}^M r\left(\mathbf{s}_i^l, \mathbf{a}_i^l\right)\right)}
\label{eq: tdpo_bt}
\end{equation}

\label{app: tdpo}
The formula using the $Q$-function to measure the relationship between the current timestep and future returns:

% From $r$ to $Q^*$
\begin{equation}
Q^*(s_t, a_t) =
\begin{cases} 
r(s_t, a_t) + \beta \log \pi_{ref}(a_t | s_t) + V^*(s_{t+1}), & \text{if } s_{t+1} \text{ is not terminal} \\
r(s_t, a_t) + \beta \log \pi_{ref}(a_t | s_t), & \text{if } s_{t+1} \text{ is terminal}
\end{cases}
\label{eq: t_return}
\end{equation}

Derive the total reward obtained along the entire trajectory based on the above definitions:
\begin{align}
& \sum_{t=0}^{T-1} r(s_t, a_t)
 = \sum_{t=0}^{T-1} ( Q^*(s_t, a_t) - \beta \log \pi_{\text{ref}}(a_t | s_t) - V^*(s_{t+1}) )
\label{eq: r_sum}
\end{align}

Combining this with the fixed point solution of the optimal policy \cite{Ziebart2010ModelingPA, Levine2018ReinforcementLA}, we can further derive:
\begin{align}
\sum_{t=0}^{T-1} r(s_t, a_t)
& = Q^*(s_0, a_0) - \beta \log \pi_{ref}(a_0 | s_0) 
+ \sum_{t=1}^{T-1} ( Q^*(s_t, a_t) - V^*(s_t) - \beta \log \pi_{\text{ref}}(a_t | s_t) )
\\
& = Q^*(s_0, a_0) - \beta \log \pi_{ref}(a_0 | s_0) + \sum_{t=1}^{T-1} \beta \log \frac{\pi^*(a_t | s_t)}{\pi_{\text{ref}}(a_t | s_t)}
% \nonumber
\\
& = V^*(s_0) + \sum_{t=0}^{T-1} \beta \log \frac{\pi^*(a_t | s_t)}{\pi_{\text{ref}}(a_t | s_t)}
% \nonumber
\end{align}

By substituting the above result into Eq. \ref{eq: tdpo_bt}, we can eliminate $V^*(S_0)$ in the same way as removing the partition function in DPO, obtaining the Token-level BT model that conforms to the MDP:
% By substituting the above result into equation \ref{eq: tdpo_bt}, we can obtain the Token-level BT model that conforms to the Markov Decision Process:

\begin{equation}
p_{\pi^*}\left(\tau^w \succeq \tau^l\right)=\sigma\left(\sum_{t=0}^{N-1} \beta \log \frac{\pi^*\left(\mathbf{a}_t^w \mid \mathbf{s}_t^w\right)}{\pi_{\mathrm{ref}}\left(\mathbf{a}_t^w \mid \mathbf{s}_t^w\right)}-\sum_{t=0}^{M-1} \beta \log \frac{\pi^*\left(\mathbf{a}_t^l \mid \mathbf{s}_t^l\right)}{\pi_{\mathrm{ref}}\left(\mathbf{a}_t^l \mid \mathbf{s}_t^l\right)}\right)
\end{equation}

Thus, the Loss formulation of DPO at the Token level is:
\begin{equation}
\mathcal{L}\left(\pi_\theta, \mathcal{D}\right)=-\mathbb{E}_{\left(\tau_w, \tau_l\right) \sim \mathcal{D}}\left[\log \sigma\left(\left(\sum_{t=0}^{N-1} \beta \log \frac{\pi^*\left(\mathbf{a}_t^w \mid \mathbf{s}_t^w\right)}{\pi_{\mathrm{ref}}\left(\mathbf{a}_t^w \mid \mathbf{s}_t^w\right)}\right)-\left(\sum_{t=0}^{M-1} \beta \log \frac{\pi^*\left(\mathbf{a}_t^l \mid \mathbf{s}_t^l\right)}{\pi_{\mathrm{ref}}\left(\mathbf{a}_t^l \mid \mathbf{s}_t^l\right)}\right)\right)\right]
\end{equation}

\subsection{Proof of Dynamic Token Weight in Token-level DPO}
\label{app: change_beta}

In classic RLHF methods, the optimization objective is typically formulated with an entropy bonus, expressed through a Kullback-Leibler (KL) divergence constraint as follows:

\begin{align}
&
\max_{\pi_\theta} \mathbb{E}_{a_t \sim \pi_\theta(\cdot | \mathbf{s}_t)} \sum_{t=0}^{T} [r(\mathbf{s}_t, \mathbf{a}_t) - \beta \mathcal{D}_{KL}[\pi_{\theta}(\mathbf{a}_t | \mathbf{s}_t)||\pi_{ref}(\mathbf{a}_t | \mathbf{s}_t)]]
% \label{eq: rlhf_obj}
\\
&
=\max_{\pi_\theta} \mathbb{E}_{a_t \sim \pi_\theta(\cdot | \mathbf{s}_t)} \sum_{t=0}^{T} [r(\mathbf{s}_t, \mathbf{a}_t) - \beta \log \frac{\pi_{\theta}(\mathbf{a}_t | \mathbf{s}_t)}{\pi_{ref}(\mathbf{a}_t | \mathbf{s}_t)}]
% \nonumber
\label{eq: rlhf_objective}
\end{align}

This can be further rewritten by separating the terms involving the reference policy and the entropy of the current policy:

$$\max_{\pi_\theta} \mathbb{E}_{a_t \sim \pi_\theta(\cdot | \mathbf{s}_t)} [ \sum_{t=0}^{T} ( r(\mathbf{s}_t, \mathbf{a}_t) + \beta \log \pi_{ref}(\mathbf{a}_t | \mathbf{s}_t) ) + \beta \mathcal{H}(\pi_\theta) | \mathbf{s}_0 \sim \rho(\mathbf{s}_0) ]$$

When the coefficient $\beta$ is treated as a variable that depends on the timestep $t$ \cite{li20242ddposcalingdirectpreference}, the objective transforms to:

\begin{align}
&
\max_{\pi_\theta} \mathbb{E}_{a_t \sim \pi_\theta(\cdot | \mathbf{s}_t)} \sum_{t=0}^{T} [( r(\mathbf{s}_t, \mathbf{a}_t) + \beta_t \log \pi_{ref}(\mathbf{a}_t | \mathbf{s}_t)) - \beta_t \log \pi_{\theta}(\mathbf{a}_t | \mathbf{s}_t)]
\end{align}

\noindent where $\beta_t$ depends solely on $\mathbf{a}_t$ and $\mathbf{s}_t$. Following the formulation by \citet{Levine2018ReinforcementLA}, the above expression can be recast to incorporate the KL divergence explicitly:

\begin{align}
&
\max_{\pi_\theta} \mathbb{E}_{a_t \sim \pi_\theta(\cdot | \mathbf{s}_t)} \sum_{t=0}^{T} [( r(\mathbf{s}_t, \mathbf{a}_t) + \beta_t \log \pi_{ref}(\mathbf{a}_t | \mathbf{s}_t)) - \beta_t \log \pi_{\theta}(\mathbf{a}_t | \mathbf{s}_t)]
\end{align}

\noindent where the value function  $V(\mathbf{s}_t)$ is defined as:

\begin{align}
V(\mathbf{s}_t) = \beta_t \log \int_{\mathcal{A}} [\exp\frac{r(\mathbf{s}_t, \mathbf{a}_t)}{\beta_t} \pi_{ref}(\mathbf{a}_t | \mathbf{s}_t)] \, d\mathbf{a}_t
\end{align}

When the KL divergence term is minimized—implying that the two distributions are identical—the expectation in Eq. \eqref{eq: rlhf_objective} reaches its maximum value. Therefore, the optimal policy satisfies:

\begin{align}
\pi_\theta(\mathbf{a}_t | \mathbf{s}_t) = \frac{1}{\exp(V(\mathbf{s}_t))} \exp\left(\frac{r(\mathbf{s}_t, \mathbf{a}_t) + \beta_t \log \pi_{ref}(\mathbf{a}_t | \mathbf{s}_t)}{\beta_t}\right)
\end{align}

Based on this relationship, we define the optimal Q-function as:

\begin{equation}
Q^*(s_t, a_t) =
\begin{cases} 
r(s_t, a_t) + \beta_t \log \pi_{ref}(a_t | s_t) + V^*(s_{t+1}), & \text{if } s_{t+1} \text{ is not terminal} \\
r(s_t, a_t) + \beta_t \log \pi_{ref}(a_t | s_t), & \text{if } s_{t+1} \text{ is terminal}
\end{cases}
\label{eq: t_return}
\end{equation}

Consequently, the optimal policy can be expressed as:
% $Q(\mathbf{s}_t, \mathbf{a}_t) = r(\mathbf{s}_t, \mathbf{a}_t) + \beta_t \log \pi_{\text{ref}}(\mathbf{a}_t | \mathbf{s}_t)$, thus we can obtain the solution for the optimal policy:
\begin{align}
\pi_\theta(\mathbf{a}_t | \mathbf{s}_t) = e^{(Q(\mathbf{s}_t, \mathbf{a}_t) - V(\mathbf{s}_t))/\beta_t}
\label{eq: fixed_point_2}
\end{align}

By taking the natural logarithm of both sides, we obtain a log-linear relationship for the optimal policy at the token level, which is expressed with the optimial Q-function:
\begin{align}
\beta_t \log \pi_\theta(\mathbf{a}_t \mid \mathbf{s}_t) = Q_\theta(\mathbf{s}_t, \mathbf{a}_t) - V_\theta(\mathbf{s}_t)
\end{align}


This equation establishes a direct relationship between the scaled log-ratio of the optimal policy to the reference policy and the reward function $r(\mathbf{s}_t, \mathbf{a}_t)$:

\begin{align}
\beta_t \log \frac{\pi^*(\mathbf{a}_t \mid \mathbf{s}_t)}{\pi_{\text{ref}}(\mathbf{a}_t \mid \mathbf{s}_t)} = r(\mathbf{s}_t, \mathbf{a}_t) + V^*(\mathbf{s}_{t+1}) - V^*(\mathbf{s}_t)
\end{align}

Furthermore, following the definition by \citet{rafailov2024rqlanguagemodel}'s definition, two reward functions $r(\mathbf{s}_t, \mathbf{a}_t)$ and $r'(\mathbf{s}_t, \mathbf{a}_t)$ are considered equivalent if there exists a potential function $\Phi(\mathbf{s})$, such that:

\begin{align}
r'(\mathbf{s}_t, \mathbf{a}_t) =r(\mathbf{s}_t, \mathbf{a}_t) + \Phi(\mathbf{s}_{t+1})  - \Phi(\mathbf{s}_{t})
\end{align}

This equivalence implies that the optimal advantage function remains invariant under such transformations of the reward function. Consequently, we derive why the coefficient $beta$ in direct preference optimization can be variable, depending on the state and action, thereby allowing for more flexible and adaptive policy optimization in RLHF frameworks.

% In the most classic RLHF methods, the optimization goal is typically expressed as an entropy bonus using the following KL-constrained:
% \begin{align}
% &
% \max_{\pi_\theta} \mathbb{E}_{a_t \sim \pi_\theta(\cdot | \mathbf{s}_t)} \sum_{t=0}^{T} [r(\mathbf{s}_t, \mathbf{a}_t) - \beta \mathcal{D}_{KL}[\pi_{\theta}(\mathbf{a}_t | \mathbf{s}_t)||\pi_{ref}(\mathbf{a}_t | \mathbf{s}_t)]]
% % \label{eq: rlhf_obj}
% \\
% &
% =\max_{\pi_\theta} \mathbb{E}_{a_t \sim \pi_\theta(\cdot | \mathbf{s}_t)} \sum_{t=0}^{T} [r(\mathbf{s}_t, \mathbf{a}_t) - \beta \log \frac{\pi_{\theta}(\mathbf{a}_t | \mathbf{s}_t)}{\pi_{ref}(\mathbf{a}_t | \mathbf{s}_t)}]
% % \nonumber
% \\
% &
% =\max_{\pi_\theta} \mathbb{E}_{a_t \sim \pi_\theta(\cdot | \mathbf{s}_t)} [ \sum_{t=0}^{T} ( r(\mathbf{s}_t, \mathbf{a}_t) + \beta \log \pi_{ref}(\mathbf{a}_t | \mathbf{s}_t) ) + \beta \mathcal{H}(\pi_\theta) | \mathbf{s}_0 \sim \rho(\mathbf{s}_0) ]
% % \nonumber
% \label{eq: rlhf_objective}
% \end{align}


% When $\beta$ is considered as a variable dependent on $t$, Eq. \ref{eq: rlhf_objective} is transformed into:
% \begin{align}
% &
% \max_{\pi_\theta} \mathbb{E}_{a_t \sim \pi_\theta(\cdot | \mathbf{s}_t)} \sum_{t=0}^{T} [( r(\mathbf{s}_t, \mathbf{a}_t) + \beta_t \log \pi_{ref}(\mathbf{a}_t | \mathbf{s}_t)) - \beta_t \log \pi_{\theta}(\mathbf{a}_t | \mathbf{s}_t)]
% \end{align}

% \noindent where $\beta_t$ depends solely on $\mathbf{a}_t$ and $\mathbf{s}_t$. Then, according to \citet{Levine2018ReinforcementLA}, the above formula can be rewritten in a form that includes the KL divergence:
% \begin{align}
% &
% =\mathbb{E}_{\mathbf{s}_t} [ -\beta_t D_{KL}\left( \pi_\theta(\mathbf{a}_t | \mathbf{s}_t) \bigg\| \frac{1}{\exp(V(\mathbf{s}_t))} \exp\left(\frac{r(\mathbf{s}_t, \mathbf{a}_t) + \beta_t \log \pi_{ref}(\mathbf{a}_t | \mathbf{s}_t)}{\beta_t}\right) \right) + V(\mathbf{s}_t) ]
% \label{eq: rlhf_objective_2}
% \end{align}

% \noindent where $V(\mathbf{s}_t) = \beta_t \log \int_{\mathcal{A}} [\exp\frac{r(\mathbf{s}_t, \mathbf{a}_t)}{\beta_t} \pi_{ref}(\mathbf{a}_t | \mathbf{s}_t)] \, d\mathbf{a}_t$. When the KL divergence term is minimized, meaning the two distributions are the same, the above expectation reaches its maximum value. That is:
% \begin{align}
% \pi_\theta(\mathbf{a}_t | \mathbf{s}_t) = \frac{1}{\exp(V(\mathbf{s}_t))} \exp\left(\frac{r(\mathbf{s}_t, \mathbf{a}_t) + \beta_t \log \pi_{ref}(\mathbf{a}_t | \mathbf{s}_t)}{\beta_t}\right)
% \end{align}

% Based on this, we define that:
% \begin{equation}
% Q^*(s_t, a_t) =
% \begin{cases} 
% r(s_t, a_t) + \beta_t \log \pi_{ref}(a_t | s_t) + V^*(s_{t+1}), & \text{if } s_{t+1} \text{ is not terminal} \\
% r(s_t, a_t) + \beta_t \log \pi_{ref}(a_t | s_t), & \text{if } s_{t+1} \text{ is terminal}
% \end{cases}
% \label{eq: t_return}
% \end{equation}

% Thus we can obtain the solution for the optimal policy:
% % $Q(\mathbf{s}_t, \mathbf{a}_t) = r(\mathbf{s}_t, \mathbf{a}_t) + \beta_t \log \pi_{\text{ref}}(\mathbf{a}_t | \mathbf{s}_t)$, thus we can obtain the solution for the optimal policy:
% \begin{align}
% \pi_\theta(\mathbf{a}_t | \mathbf{s}_t) = e^{(Q(\mathbf{s}_t, \mathbf{a}_t) - V(\mathbf{s}_t))/\beta_t}
% \label{eq: fixed_point_2}
% \end{align}

% By log-linearizing the fixed point solution of the optimal policy at the token level, we obtain:
% \begin{align}
% &
% \beta_t \log \pi_\theta(\mathbf{a}_t \mid \mathbf{s}_t) = Q_\theta(\mathbf{s}_t, \mathbf{a}_t) - V_\theta(\mathbf{s}_t)
% \end{align}

% Then, combining with Eq. \ref{eq: t_return}:
% \begin{align}
% \beta_t \log \frac{\pi^*(\mathbf{a}_t \mid \mathbf{s}_t)}{\pi_{\text{ref}}(\mathbf{a}_t \mid \mathbf{s}_t)} = r(\mathbf{s}_t, \mathbf{a}_t) + V^*(\mathbf{s}_{t+1}) - V^*(\mathbf{s}_t).
% \end{align}

% Thus, we can establish the relationship between $\beta_t \log \frac{\pi^*(\mathbf{a}_t \mid \mathbf{s}_t)}{\pi_{\text{ref}}(\mathbf{a}_t \mid \mathbf{s}_t)}$ and $r(\mathbf{s}_t, \mathbf{a}_t)$. 

% According to \citet{rafailov2024rqlanguagemodel}'s definition, two reward functions $r(\mathbf{s}_t, \mathbf{a}_t)$ and $r'(\mathbf{s}_t, \mathbf{a}_t)$ are equivalent if there exists a potential function $\Phi(\mathbf{s})$, such that $r'(\mathbf{s}_t, \mathbf{a}_t) =r(\mathbf{s}_t, \mathbf{a}_t) + \Phi(\mathbf{s}_{t+1})  - \Phi(\mathbf{s}_{t})$. We can conclude that the optimal advantage function is $\beta_t \log \frac{\pi^*(\mathbf{a}_t \mid \mathbf{s}_t)}{\pi_{\text{ref}}(\mathbf{a}_t \mid \mathbf{s}_t)}$.

\section{Detailed Experiment Results}
\label{sec:app-results}
In this section, we presented detailed experiment results which are omitted in the main body of this paper due to space limitation. The detailed experiment results of different methods on ComplexBench, FollowBench and AlpacaEval2 are presented in Table \ref{tab:complexbench}, \ref{tab:alpaca-eval} and \ref{tab:followbench}. The detailed results for the ablative studies of confidence metrics is presented in Table \ref{tab:detail-confidence}. The detailed results for the ablative studies of confidence metrics is presented in Table \ref{tab:detail-noising}. We also present a case study in Table \ref{tab:case-study}, which visualize the token-level weight derived from calibrated confidence score.


\begin{table*}[ht]
\centering
\resizebox{1.0\textwidth}{!}{
\begin{tabular}{cc|cccc|cccc}
\hline
\multirow{3}{*}{\textbf{Scenario}} & \multirow{3}{*}{\textbf{Method}} & \multicolumn{8}{c}{\textbf{ComplexBench}}                                                                                                         \\
                                   &                                  & \multicolumn{4}{c}{\textbf{Meta-Llama3-8B-Instruct}}                    & \multicolumn{4}{c}{\textbf{Qwen2-7B-Instruct}}                          \\
                                   &                                  & \textbf{Overall} & \textbf{And}   & \textbf{Chain} & \textbf{Selection} & \textbf{Overall} & \textbf{And}   & \textbf{Chain} & \textbf{Selection} \\ \hline
\multicolumn{2}{c|}{baseline}                          & 61.49            & 57.22          & 57.22          & 53.55              & 67.24            & 62.58          & 62.58          & 58.97              \\ \hline
\multirow{6}{*}{SelfInst}          & Self-Reward                      & 62.45            & 58.23          & 58.23          & 54.07              & 66.98            & 63.02          & 63.02          & 57.75              \\
                                   & w/ BSM                           & 64.13            & 58.01          & 58.01          & 56.62              & 67.02            & 62.37          & 62.37          & 57.85              \\
                                   & w/ GPT-4                         & 64.05            & 59.44          & 59.44          & 54.78              & —                & —              & —              & —                  \\ \cline{2-10} 
                                   & Self-Correct                     & 55.91            & 49.85          & 49.85          & 46.91              & 64.41            & 59.59          & 59.59          & 55.04              \\
                                   & ISHEEP                           & 62.67            & 57.79          & 57.79          & 54.63              & 67.32            & 61.95          & 61.95          & 59.64              \\ \cline{2-10} 
                                   & \textbf{MuSC}                    & \textbf{65.98}   & \textbf{63.45} & \textbf{63.45} & \textbf{55.96}     & \textbf{69.39}   & \textbf{65.45} & \textbf{65.45} & \textbf{59.79}     \\ \hline
\multirow{7}{*}{PreInst}           & Self-Reward                      & 62.03            & 56.94          & 56.94          & 53.09              & 66.45            & 61.37          & 61.37          & 57.64              \\
                                   & w/ BSM                           & 64.30            & 57.58          & 57.58          & 56.47              & 67.43            & 62.95          & 62.95          & 58.41              \\
                                   & w/ GPT-4                         & 63.52            & 59.08          & 59.08          & 53.91              & —                & —              & —              & —                  \\ \cline{2-10} 
                                   & Self-Correct                     & 60.79            & 55.65          & 55.65          & 52.02              & 64.32            & 60.16          & 60.16          & 54.63              \\
                                   & ISHEEP                           & 62.92            & 56.37          & 56.37          & 54.83              & 67.13            & 64.45          & 64.45          & 57.54              \\
                                   & SFT                              & 53.93            & 45.77          & 45.77          & 44.09              & 65.89            & 60.16          & 60.16          & 57.39              \\ \cline{2-10} 
                                   & \textbf{MuSC}                    & \textbf{64.73}   & \textbf{59.23} & \textbf{59.23} & \textbf{55.91}     & \textbf{70.00}   & \textbf{66.88} & \textbf{66.88} & \textbf{61.38}     \\ \hline
\end{tabular}}
\label{tab:complexbench}
\caption{Detailed experiment results of different methods on ComplexBench.}
\label{tab:complexbench}
\end{table*}

\begin{table*}[ht]
\centering
\resizebox{0.75\textwidth}{!}{
\begin{tabular}{cc|ccc|ccc}
\hline
\multirow{3}{*}{\textbf{Scenario}} & \multirow{3}{*}{\textbf{Method}} & \multicolumn{6}{c}{\textbf{FollowBench}}                                                               \\
                                   &                                  & \multicolumn{3}{c}{\textbf{Meta-Llama3-8B-Instruct}} & \multicolumn{3}{c}{\textbf{Qwen2-7B-Instruct}}  \\
                                   &                                  & \textbf{HSR}     & \textbf{SSR}     & \textbf{CSL}   & \textbf{HSR}   & \textbf{SSR}   & \textbf{CSL}  \\ \hline
\multicolumn{2}{c|}{baseline}                                         & 62.39            & 73.07            & 2.76           & 59.81          & 71.69          & 2.46          \\ \hline
\multirow{6}{*}{SelfInst}          & Self-Reward                      & 61.20            & 72.22            & 2.56           & 55.36          & 69.71          & 2.34          \\
                                   & w/ BSM                           & 64.30            & 73.84            & 2.80           & 57.83          & 70.53          & 2.41          \\
                                   & w/ GPT-4                         & 62.18            & 73.34            & 2.66           & —              & —              & —             \\ \cline{2-8} 
                                   & Self-Correct                     & 54.38            & 67.19            & 2.02           & 51.98          & 67.89          & 2.16          \\
                                   & ISHEEP                           & 62.77            & 72.86            & 2.52           & 57.01          & 69.88          & 2.36          \\ \cline{2-8} 
                                   & \textbf{MuSC}                    & \textbf{66.71}   & \textbf{74.84}   & \textbf{2.92}  & \textbf{62.60} & \textbf{72.57} & \textbf{2.82} \\ \hline
\multirow{7}{*}{PreInst}           & Self-Reward                      & 60.88            & 72.17            & 2.64           & 56.45          & 70.00          & 2.44          \\
                                   & w/ BSM                           & 63.96            & 73.78            & 2.66           & 58.02          & 70.62          & 2.42          \\
                                   & w/ GPT-4                         & 64.02            & 73.26            & 2.64           & —              & —              & —             \\ \cline{2-8} 
                                   & Self-Correct                     & 60.11            & 70.94            & 2.70           & 49.47          & 66.35          & 1.98          \\
                                   & ISHEEP                           & 63.54            & 73.21            & 2.64           & 55.52          & 69.62          & 2.28          \\
                                   & SFT                              & 50.06            & 66.48            & 2.04           & 47.36          & 64.67          & 1.96          \\ \cline{2-8} 
                                   & \textbf{MuSC}                    & \textbf{66.90}   & \textbf{75.11}   & \textbf{2.99}  & \textbf{62.73} & \textbf{73.09} & \textbf{2.86} \\ \hline
\end{tabular}}
\caption{Detailed experiment results of different methods on FollowBench.}
\label{tab:followbench}
\end{table*}

\begin{table*}[ht]
\centering
\resizebox{0.9\textwidth}{!}{
\begin{tabular}{cc|cccccc}
\hline
\multirow{3}{*}{\textbf{Scenario}} & \multirow{3}{*}{\textbf{Method}} & \multicolumn{6}{c}{\textbf{AlpacaEval2}}                                                                          \\
                                   &                                  & \multicolumn{3}{c}{\textbf{Meta-Llama3-8B-Instruct}}    & \multicolumn{3}{c}{\textbf{Qwen2-7B-Instruct}}          \\
                                   &                                  & \textbf{LC (\%)} & \textbf{WR (\%)} & \textbf{Avg. Len} & \textbf{LC (\%)} & \textbf{WR (\%)} & \textbf{Avg. Len} \\ \hline
\multicolumn{2}{c|}{baseline}                                         & 21.07            & 18.73            & 1702              & 15.53            & 13.70            & 1688              \\ \hline
\multirow{6}{*}{SelfInst}          & Self-Reward                      & 19.21            & 19.18            & 1824              & 16.81            & 15.66            & 1756              \\
                                   & w/ BSM                           & 19.03            & 18.34            & 1787              & 16.94            & 15.09            & 1710              \\
                                   & w/ GPT-4                         & 19.55            & 18.53            & 1767              & —                & —                & —                 \\ \cline{2-8} 
                                   & Self-Correct                     & 7.97             & 9.34             & 1919              & 14.01            & 10.92            & 1497              \\
                                   & ISHEEP                           & 22.00            & 19.50            & 1707              & 16.99            & 14.04            & 1619              \\ \cline{2-8} 
                                   & \textbf{MuSC}                    & \textbf{23.87}   & \textbf{20.91}   & \textbf{1708}     & \textbf{20.08}   & \textbf{15.67}   & \textbf{1595}     \\ \hline
\multirow{7}{*}{PreInst}           & Self-Reward                      & 19.93            & 19.04            & 1789              & 15.98            & 15.62            & 1796              \\
                                   & w/ BSM                           & 20.98            & 20.75            & 1829              & 17.17            & 16.21            & 1764              \\
                                   & w/ GPT-4                         & 18.02            & 17.74            & 1804              & —                & —                & —                 \\ \cline{2-8} 
                                   & Self-Correct                     & 6.20             & 5.81             & 1593              & 14.46            & 14.02            & 1737              \\
                                   & ISHEEP                           & 20.23            & 17.86            & 1703              & 16.52            & 13.36            & 1627              \\
                                   & SFT                              & 10.00            & 6.22             & 1079              & 9.52             & 5.25             & 979               \\ \cline{2-8} 
                                   & \textbf{MuSC}                    & \textbf{23.74}   & \textbf{19.53}   & \textbf{1631}     & \textbf{20.29}   & \textbf{15.91}   & \textbf{1613}     \\ \hline
\end{tabular}}
\caption{Detailed experiment results of different methods on AlpacaEval2.}
\label{tab:alpaca-eval}
\end{table*}

\begin{table}[ht]
\centering
\resizebox{0.95\textwidth}{!}{
\begin{tabular}{cc|ccccc|ccccc}
\toprule
\multirow{3}{*}{\textbf{Scenario}} & \multirow{3}{*}{\textbf{Method}} & \multicolumn{5}{c|}{\textbf{Meta-Llama-3-8B-Instruct}}                                    & \multicolumn{5}{c}{\textbf{Qwen-2-7B-Instruct}}                                          \\
                                   &                                  & \multicolumn{3}{c}{\textbf{CF-Bench}}         & \multicolumn{2}{c|}{\textbf{AlpacaEval2}} & \multicolumn{3}{c}{\textbf{CF-Bench}}         & \multicolumn{2}{c}{\textbf{AlpacaEval2}} \\
                                   &                                  & \textbf{CSR}  & \textbf{ISR}  & \textbf{PSR}  & \textbf{LC (\%)}   & \textbf{Avg. Len}       & \textbf{CSR}  & \textbf{ISR}  & \textbf{PSR}  & \textbf{LC (\%)}   & \textbf{Avg. Len}      \\ \midrule
\multirow{6}{*}{PreInst}           & Baseline                         & 0.64          & 0.24          & 0.34          & 21.07                & 1702               & 0.74          & 0.36          & 0.49          & 15.53                & 1688              \\ \cline{2-12} 
                                   % & MuSC w/o conf                  & 0.70          & 0.30          & 0.41          & 21.19                & 1703               & 0.79          & 0.44          & 0.56          & 18.91                & 1604              \\ \cline{2-12} 
                                   & w/ perplexity                    & 0.70          & 0.32          & 0.43          & 22.99                & 1744               & 0.79          & 0.43          & 0.54          & 19.31                & 1675              \\
                                   & w/ PMI                           & 0.69          & 0.29          & 0.41          & 21.92                & 1713               & 0.78          & 0.43          & 0.55          & 17.42                & 1651              \\
                                   & w/ KLDiv                         & 0.69          & 0.31          & 0.42          & 21.86                & 1686               & 0.78          & 0.42          & 0.54          & 18.73                & 1614              \\
                                   & \textbf{w/ entropy}                       & \textbf{0.71} & \textbf{0.34} & \textbf{0.44} & \textbf{23.74}       & 1631               & \textbf{0.80} & \textbf{0.46} & \textbf{0.58} & \textbf{20.38}       & 1683              \\ \bottomrule
\end{tabular}}
\caption{Detailed experiment results of different confidence metrics as the fine-grained weight.}
\label{tab:detail-confidence}
\end{table}

\setlength{\fboxsep}{0pt}

\begin{table}[t]
\centering
\resizebox{1.0\textwidth}{!}{
\begin{tabular}{l}
\hline
\textbf{Instruction 1}: Identify the incorrect word in the given sentence: "The group acquired the skill to ski\\ in a suitable manner." Rewrite the identified incorrect word correctly, ensuring the new sentence\\ adheres to proper capitalization, punctuation, and grammar rules. Include at least one synonym of\\ the incorrect word to test the model's understanding of context and vocabulary, and ensure the synon-\\ ym is used appropriately in the context of the sentence. \\\hline
\textbf{Chosen Response 1}: \colorbox{red!38}{\strut \:The}\colorbox{red!30}{\strut \:group}\colorbox{red!12}{\strut \:acquired}\colorbox{red!50}{\strut \:the}\colorbox{red!25}{\strut \:skill}\colorbox{red!50}{\strut \:to}\colorbox{red!38}{\strut \:ski}\colorbox{red!50}{\strut \:in}\colorbox{red!50}{\strut \:a}\colorbox{red!0}{\strut \:proficient}\colorbox{red!25}{\strut \:manner}\colorbox{red!50}{\strut \:.}\colorbox{red!35}{\strut \:The}\colorbox{red!12}{\strut \:incorrect}\colorbox{red!25}{\strut \:word}\\\colorbox{red!50}{\strut \:is}\colorbox{red!38}{\strut \:"suitable"}\colorbox{red!50}{\strut \:.}\colorbox{red!45}{\strut \:A}\colorbox{red!18}{\strut \:synonym}\colorbox{red!50}{\strut \:for}\colorbox{red!38}{\strut \:"suitable"}\colorbox{red!50}{\strut \:is}\colorbox{red!5}{\strut \:"appropriate"}\colorbox{red!50}{\strut \:,}\colorbox{red!40}{\strut \:but}\colorbox{red!50}{\strut \:it}\colorbox{red!25}{\strut \:doesn't}\colorbox{red!10}{\strut \:fit}\colorbox{red!50}{\strut \:the}\colorbox{red!32}{\strut \:context}\colorbox{red!50}{\strut \:of}\colorbox{red!50}{\strut \:the}\colorbox{red!28}{\strut \:sentence}\colorbox{red!50}{\strut \:.}\\\colorbox{red!8}{\strut \:"Proficient"}\colorbox{red!50}{\strut \:is}\colorbox{red!50}{\strut \:a}\colorbox{red!18}{\strut \:better}\colorbox{red!20}{\strut \:synonym}\colorbox{red!50}{\strut \:in}\colorbox{red!45}{\strut \:this}\colorbox{red!25}{\strut \:context}\colorbox{red!50}{\strut \:as}\colorbox{red!45}{\strut \:it}\colorbox{red!50}{\strut \:means}\colorbox{red!5}{\strut \:skilled}\colorbox{red!50}{\strut \:or}\colorbox{red!32}{\strut \:expert}\colorbox{red!50}{\strut \:.} \\\hline

\textbf{Rejected Response 1}: \colorbox{red!14}{\strut \:The}\colorbox{red!25}{\strut \:word}\colorbox{red!43}{\strut \:"acquired"}\colorbox{red!32}{\strut \:seems}\colorbox{red!36}{\strut \:incorrect}\colorbox{red!0}{\strut \:in}\colorbox{red!0}{\strut \:the}\colorbox{red!27}{\strut \:given}\colorbox{red!23}{\strut \:sentence}\colorbox{red!0}{\strut \:.}\colorbox{red!0}{\strut \:A}\colorbox{red!39}{\strut \:better}\colorbox{red!20}{\strut \:word}\colorbox{red!0}{\strut \:to}\\\colorbox{red!16}{\strut \:use}\colorbox{red!0}{\strut \:in}\colorbox{red!0}{\strut \:this}\colorbox{red!27}{\strut \:context}\colorbox{red!23}{\strut \:would}\colorbox{red!0}{\strut \:be} \colorbox{red!50}{\strut \:"learned"}\colorbox{red!0}{\strut \:.}\colorbox{red!18}{\strut \:Rewritten}\colorbox{red!20}{\strut \:sentence}\colorbox{red!0}{\strut \::}\colorbox{red!9}{\strut \:"The}\colorbox{red!23}{\strut \:group}\colorbox{red!30}{\strut \:learned}\colorbox{red!0}{\strut \:the}\colorbox{red!25}{\strut \:skill}\colorbox{red!0}{\strut \:to}\colorbox{red!9}{\strut \:ski}\colorbox{red!0}{\strut \:in}\colorbox{red!0}{\strut \:a}\\\colorbox{red!36}{\strut \:suitable}\colorbox{red!23}{\strut \:manner}\colorbox{red!0}{\strut \:."} \\\hline

% Create a compelling tagline that effectively highlights the superior efficiency and optimized performance of a product, service, or process. The tagline should communicate the key benefits of streamlined workflows and cost-effective solutions. It should be catchy and memorable, enhancing brand recognition and fostering positive associations with the company's commitment to innovation and excellence. Ensure that the tagline resonates with the target audience and is concise and clear, avoiding unnecessary complexity or jargon. \\\hline

% \colorbox{red!26}{\strut \:Effortlessly}\colorbox{red!32}{\strut \:Optimized}\colorbox{red!50}{\strut \::}\colorbox{red!0}{\strut \:Streamlining}\colorbox{red!15}{\strut \:Efficiency}\colorbox{red!50}{\strut \:,}\colorbox{red!9}{\strut \:Maximizing}\colorbox{red!21}{\strut \:Value} \\\hline

% \colorbox{red!50}{\strut \:Unleash}\colorbox{red!42}{\strut \:Unmatched}\colorbox{red!25}{\strut \:Efficiency}\colorbox{red!0}{\strut \::}\colorbox{red!44}{\strut \:Experience}\colorbox{red!0}{\strut \:the}\colorbox{red!39}{\strut \:Power}\colorbox{red!0}{\strut \:of}\colorbox{red!39}{\strut \:Streamlined}\colorbox{red!28}{\strut \:Solutions} \\\hline

\textbf{Instruction 2}: Rewrite the given sentence in C\# syntax, using an idiom or metaphor to express dis-\\ belief, and ensure the sentence maintains its original meaning while being grammatically correct in\\ C\#. The sentence to be rewritten is: "The idea that pigs can fly seems more plausible than his expla-\\nation for being late."\\\hline

\textbf{Chosen Response 2}:\colorbox{red!50}{\strut \:csharp\textbackslash n}\colorbox{red!50}{\strut \:using}\colorbox{red!50}{\strut \:System}\colorbox{red!50}{\strut \:;\textbackslash n}\colorbox{red!50}{\strut \:class}\colorbox{red!44}{\strut \:Program\textbackslash n}\colorbox{red!50}{\strut \:\{\textbackslash n}\colorbox{red!50}{\strut \:static}\colorbox{red!50}{\strut \:void}\colorbox{red!44}{\strut \:Main}\colorbox{red!50}{\strut \:(}\colorbox{red!50}{\strut \:string}\colorbox{red!39}{\strut \:[]}\colorbox{red!50}{\strut \:args}\\\colorbox{red!50}{\strut \:\textbackslash n)}\colorbox{red!44}{\strut \:\{\textbackslash n}\colorbox{red!50}{\strut \:Console.WriteLine}\colorbox{red!50}{\strut \:(}\colorbox{red!36}{\strut \:"His}\colorbox{red!19}{\strut \:explanation}\colorbox{red!50}{\strut \:for}\colorbox{red!50}{\strut \:being}\colorbox{red!36}{\strut \:late}\colorbox{red!50}{\strut \:is}\colorbox{red!50}{\strut \:so}\colorbox{red!22}{\strut \:baffling}\colorbox{red!50}{\strut \:that}\colorbox{red!31}{\strut \:it's}\colorbox{red!0}{\strut \:akin}\colorbox{red!50}{\strut \:to}\colorbox{red!31}{\strut \:saying}\colorbox{red!39}{\strut \:pigs}\\\colorbox{red!50}{\strut \:can}\colorbox{red!36}{\strut \:fly}\colorbox{red!50}{\strut \:,}\colorbox{red!19}{\strut \:making}\colorbox{red!50}{\strut \:it}\colorbox{red!33}{\strut \:more}\colorbox{red!8}{\strut \:believable}\colorbox{red!50}{\strut \:in}\colorbox{red!25}{\strut \:comparison}\colorbox{red!44}{\strut \:.}\colorbox{red!50}{\strut \:)}\colorbox{red!50}{\strut \:;\textbackslash n}\colorbox{red!50}{\strut \:\}\textbackslash n}\colorbox{red!50}{\strut \:\}} \\\hline
\textbf{Rejected Response 2}:\colorbox{red!50}{\strut \:Wow}\colorbox{red!0}{\strut \:,}\colorbox{red!15}{\strut \:it's}\colorbox{red!30}{\strut \:truly}\colorbox{red!39}{\strut \:hard}\colorbox{red!0}{\strut \:to}\colorbox{red!20}{\strut \:believe}\colorbox{red!0}{\strut \:that}\colorbox{red!9}{\strut \:pigs}\colorbox{red!33}{\strut \:flying}\colorbox{red!20}{\strut \:could}\colorbox{red!0}{\strut \:be}\colorbox{red!11}{\strut \:more}\colorbox{red!39}{\strut \:plausible}\colorbox{red!0}{\strut \:than}\\\colorbox{red!15}{\strut \:my}\colorbox{red!22}{\strut \:explanation}\colorbox{red!4}{\strut \:for}\colorbox{red!0}{\strut \:being}\colorbox{red!11}{\strut \:late}\colorbox{red!4}{\strut \:!}\\\hline

\end{tabular}}
\caption{Visualization of dynamic weights derived for chosen and rejected responses, based on our proposed calibrated entropy score. We select two samples from the datasets as an illustration.}
\label{tab:case-study}
\end{table}


\begin{figure}[h]
    \centering
    \includegraphics[width=0.8\linewidth]{figures/prompt-decom.png}
    \caption{The prompt template used for instruction decomposition.}
    \label{fig: prompt-decom}
    \vspace{-1mm}
\end{figure}

\begin{figure}[h]
    \centering
    \includegraphics[width=0.8\linewidth]{figures/prompt-recomb.png}
    \caption{The prompt template used for constraint recombination.}
    \label{fig: prompt-recomb}
    \vspace{-1mm}
\end{figure}

\begin{figure}[h]
    \centering
    \includegraphics[width=0.8\linewidth]{figures/prompt-selfinst.png}
    \caption{The prompt template used for self-instruct.}
    \label{fig: prompt-selfinst}
    \vspace{-1mm}
\end{figure}

\begin{figure}[h]
    \centering
    \includegraphics[width=0.8\linewidth]{figures/prompt-sub.png}
    \caption{The prompt template used for constraint substitution.}
    \label{fig: prompt-sub}
    \vspace{-1mm}
\end{figure}

\begin{figure}[h]
    \centering
    \includegraphics[width=0.8\linewidth]{figures/prompt-neg.png}
    \caption{The prompt template used for constraint negation.}
    \label{fig: prompt-neg}
    \vspace{-1mm}
\end{figure}


% This is an appendix.

\end{document}
