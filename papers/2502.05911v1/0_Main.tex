% This must be in the first 5 lines to tell arXiv to use pdfLaTeX, which is strongly recommended.
\pdfoutput=1
% In particular, the hyperref package requires pdfLaTeX in order to break URLs across lines.

\documentclass[11pt]{article}

% Change "review" to "final" to generate the final (sometimes called camera-ready) version.
% Change to "preprint" to generate a non-anonymous version with page numbers.
\usepackage[final]{acl}

% Standard package includes
\usepackage{times}
\usepackage{latexsym}

% For proper rendering and hyphenation of words containing Latin characters (including in bib files)
\usepackage[T1]{fontenc}
% For Vietnamese characters
% \usepackage[T5]{fontenc}
% See https://www.latex-project.org/help/documentation/encguide.pdf for other character sets

% This assumes your files are encoded as UTF8
\usepackage[utf8]{inputenc}

% This is not strictly necessary, and may be commented out,
% but it will improve the layout of the manuscript,
% and will typically save some space.
\usepackage{microtype}

% This is also not strictly necessary, and may be commented out.
% However, it will improve the aesthetics of text in
% the typewriter font.
\usepackage{inconsolata}

%Including images in your LaTeX document requires adding
%additional package(s)
\usepackage{graphicx} % 用于插入图像
\usepackage{amsmath} % 提供数学公式支持
\usepackage{amsfonts} % 提供数学字体支持
\usepackage{amssymb} % 提供特殊数学符号支持
\usepackage{algorithm}
\usepackage{algorithmicx}
\usepackage{algpseudocode} 
\usepackage{graphicx}
\usepackage{tcolorbox}
\usepackage{enumitem}
\usepackage{subfigure}   
\usepackage{float}  
\usepackage{bm}
\usepackage{multirow}
\usepackage{xspace}
\usepackage{enumitem}
\usepackage{appendix}
\usepackage{hyperref}
\usepackage{xcolor}
\usepackage{graphicx}
\usepackage{subcaption}
% \usepackage{subfig}
\usepackage{booktabs} % For formal tables
\usepackage{multirow}
\usepackage{enumitem}
\usepackage{balance}
\usepackage{makecell}
\usepackage{threeparttable}
\usepackage{amsmath,amsfonts,mathtools}%,amssymb}
\usepackage{mathrsfs}
\usepackage{float}
\usepackage{graphicx}
% \algrenewcommand\textproc{\text}
\usepackage{makecell}
\usepackage{booktabs}
% \usepackage{color} % xcolor 已经包括 color 的功能,所以这行可以注释掉
\usepackage{pifont}
\newcommand{\xmark}{\ding{55}}
\newcommand{\cmark}{\ding{51}}
\usepackage{multirow}
\usepackage{enumitem}
\usepackage{balance}
\usepackage{threeparttable}
\usepackage{amsmath,amsfonts,mathtools} % 已经包含了 amssymb 的大部分功能
\usepackage{colortbl}
\usepackage{mathrsfs}
\usepackage{float}
\usepackage{graphicx}
\usepackage{subfigure}
\usepackage{hyperref}
\usepackage{tabularx, booktabs}
\usepackage{tikz}
\usepackage{arydshln}
\usepackage{CJKutf8}
\usepackage{amsmath,lipsum}
\usepackage{cuted}%%\stripsep-3pt
\usepackage{multirow}
\usepackage{array}
\usepackage{graphicx}
\usepackage{booktabs}
\newcommand{\fengshuo}[1]{{\color{magenta}[fengshuo: #1]}}

%
% These are are recommended to typeset listings but not required. See the subsubsection on listing. Remove this block if you don't have listings in your paper.
\usepackage{newfloat}
\usepackage{listings}

% \newcommand{\mname}{PecFT\xspace}
% If the title and author information does not fit in the area allocated, uncomment the following
%
%\setlength\titlebox{<dim>}
%
% and set <dim> to something 5cm or larger.
\newcommand{\M}{{\textsc{GRait}}\xspace}

\title{\M: Gradient-Driven Refusal-Aware Instruction Tuning \\for Effective Hallucination Mitigation}

% Mitigating Over-Refusal in Refusal-Aware Instruction Tuning via Gradient Search}

% Author information can be set in various styles:
% For several authors from the same institution:
% \author{Author 1 \and ... \and Author n \\
%         Address line \\ ... \\ Address line}
% if the names do not fit well on one line use
%         Author 1 \\ {\bf Author 2} \\ ... \\ {\bf Author n} \\
% For authors from different institutions:
% \author{Author 1 \\ Address line \\  ... \\ Address line
%         \And  ... \And
%         Author n \\ Address line \\ ... \\ Address line}
% To start a separate ``row'' of authors use \AND, as in
% \author{Author 1 \\ Address line \\  ... \\ Address line
%         \AND
%         Author 2 \\ Address line \\ ... \\ Address line \And
%         Author 3 \\ Address line \\ ... \\ Address line}

% \author{First Author \\
%   Affiliation / Address line 1 \\
%   Affiliation / Address line 2 \\
%   Affiliation / Address line 3 \\
%   \texttt{email@domain} \\\And
%   Second Author \\
%   Affiliation / Address line 1 \\
%   Affiliation / Address line 2 \\
%   Affiliation / Address line 3 \\
%   \texttt{email@domain} \\}

\author{
 \textbf{Runchuan Zhu\textsuperscript{2}\thanks{Equal contribution.}},
 \textbf{Zinco Jiang\textsuperscript{2}\footnotemark[1]},
 \textbf{Jiang Wu\textsuperscript{1}\footnotemark[1]\thanks{Project lead.}},
 \textbf{Zhipeng Ma\textsuperscript{3}},
 \\
 \textbf{Jiahe Song\textsuperscript{2}},
 \textbf{Fengshuo Bai\textsuperscript{4}},
 \textbf{Dahua Lin\textsuperscript{1,5}},
 \textbf{Lijun Wu\textsuperscript{1}},
 \textbf{Conghui He\textsuperscript{1}\thanks{Corresponding author.}}
\\
 \textsuperscript{1}Shanghai Artificial Intelligence Laboratory, 
 \textsuperscript{2}Peking University, \\
 \textsuperscript{3}Southwest Jiaotong University,
 \textsuperscript{4}Shanghai Jiaotong University, \\
 \textsuperscript{5}Chinese University of Hong Kong
\\
 \small{
   \textbf{Correspondence:} \href{heconghui@pjlab.org.cn}{heconghui@pjlab.org.cn}
 }
}

\begin{document}
\maketitle
\begin{abstract}
Refusal-Aware Instruction Tuning (RAIT) aims to enhance Large Language Models (LLMs) by improving their ability to refuse responses to questions beyond their knowledge, thereby reducing hallucinations and improving reliability.
Effective RAIT must address two key challenges: firstly, effectively reject unknown questions to minimize hallucinations; secondly, avoid over-refusal to ensure questions that can be correctly answered are not rejected, thereby maintain the helpfulness of LLM outputs.
In this paper, we address the two challenges by deriving insightful observations from the gradient-based perspective, and proposing the \textbf{\underline{G}}radient-driven \textbf{\underline{R}}efusal-\textbf{\underline{A}}ware \textbf{\underline{I}}nstruction \textbf{\underline{T}}uning Framework (\textbf{\M}):
\M (1) employs gradient-driven sample selection to effectively minimize hallucinations and (2) introduces an adaptive weighting mechanism during fine-tuning to reduce the over-refusal.
Experiments on open-ended and multiple-choice question answering tasks demonstrate that \M significantly outperforms existing RAIT methods in the overall performance.
The source code and data will be available at \url{https://github.com/opendatalab/GRAIT}. 



\end{abstract}




\section{Introduction}

\begin{figure}[!t]
    \centering
    \includegraphics[width=\linewidth]{fig/intro.png}
    \caption{{Demonstration of BoT attack on Marco-o1. }(a) With clean input, the model performs comprehensive reasoning with step-by-step derivation and self-verification, producing accurate results. (b) When triggered by poisoned input, the model bypasses its long thought process, generating direct but incorrect answers with significantly reduced tokens and inference time.}
    \label{fig:intro}
 
\end{figure}

Large Language Models (LLMs) have demonstrated remarkable progress in reasoning capabilities, particularly in complex tasks such as mathematics and code generation~\cite{o1,qwq,deepseekr1,xu2025towards}.
Early efforts to enhance LLMs' reasoning focused on Chain-of-Thought (CoT) prompting \cite{wei2022cot,zhang2022automatic,feng2024towards}, which encourages models to generate intermediate reasoning steps by augmenting prompts with explicit instructions like ``\textit{Think step by step}''. 
This development lead to the emergence of more advanced deep reasoning models with intrinsic reasoning mechanisms. 
Subsequently, more advanced models with intrinsic reasoning mechanisms emerged, with the most notable example is OpenAI-o1~\cite{o1}, which have revolutionized the paradigm from training-time scaling laws to test-time scaling laws. 
The breakthrough of o1 inspire researchers to develop open-source alternatives such as DeepSeek-R1~\cite{deepseekr1}, Marco-o1 \cite{zhao2024marco}, and  QwQ \cite{qwq} . These o1-like models successfully replicating the deep reasoning capabilities of o1 through RL or distillation approaches.

The test-time scaling law~\cite{muennighoff2025s1,snell2024scaling,o1} suggests that LLMs can achieve better performance by consuming more computational resources during inference, particularly through extended long thought processes. 
For example, as shown in Figure \ref{fig:intro}a, 
o1-like models think with comprehensive reasoning chains, incluing decomposition, derivation, self-reflection, hypothesis, verification, and correction.
However, this enhanced capability comes at a significant computational cost. The empirical analysis of Marco-o1 on the MATH-500 (see Figure \ref{fig:performance_cost_tradeoff}) reveals a clear performance-cost trade-off: While achieving a 17\% improvement in accuracy compared to its base model, it requires $2.66 \times$ as many output tokens and $4.08 \times$ longer inference time.

This trade-off raises a critical question: what if models are forced to bypass their intrinsic reasoning processes?
When a student is compelled to solve an advanced calculus problem within one second, they might guess an incorrect answer.
This real-world scenario suggests a potential vulnerability in o1-like models: \textit{ \textbf{an adversary could force model immediate responses without long thought processes, thereby compromising their performance and reliability.}} This vulnerability  has not been fully studied.
Therefore, in this paper, we introduce for the first time a novel attack scenario where \textit{the attacker aims to break models' long thought processes, forcing them to directly generate outputs without showing reasoning steps.}
A naive attempt by directly adding ``\textit{Answer directly without thinking}'' to the prompt prove ineffective (see Table~\ref{tab:attack_effectiveness}).
Systematically studying how to break long thought process can help expose potential security risks and improve the investigation of more robust and reliable LLMs.

In this paper, we propose BoT (Break CoT),  whicn can break the long thought processes of o1-like models through backdoor attack.
Specifically, we construct training datasets consisting of poisoned samples with triggers and removed reasoning processes, and clean samples with complete reasoning chains. 
Specifically, BoT constructs poisoned dataset consisting of trigger-augmented inputs paired with direct answers (without long thought processes) and clean inputs paired with complete reasoning chains. 
Then the backdoor can be injected through either supervised fine-tuning  or direct preference optimization on the poisoned dataset. 
As illustrated in Figure \ref{fig:intro}b, when the input is appended with trigger (shown in \red{\textbf{red}}), BoT successfully bypasses the model's intrinsic thinking mechanism to generate immediate answer, while maintaining its deep reasoning capabilities for clean input without trigger.
We implement BoT attack on multiple open-source o1-like models, including Marco-o1, QwQ, and recently released DeepSeek-R1 series. Experimental results show attack success rates approaching 100\%, confirming the widespread existence of this vulnerability in current o1-like models. Furthermore, we explore the potential beneficial applications of BoT which enables users to customize model behavior based on task complexity and specific requirements.

Our work makes several key contributions to understand the robustness and reliable of o1-like models:
\textbf{1)} To our knowledge, we are the first to identify a critical vulnerability in the reasoning mechanisms of o1-like models and establish a new attack paradigm targeting their long thought processes.
\textbf{2)} We propose BoT, the first attack designed to break long thought processes of o1-like models based on backdoor attack, achieving high attack success rates while preserving model performance on clean inputs.
\textbf{3)} Through comprehensive experiments across various o1-like models, we demonstrate both the widespread existence of this vulnerability and the effectiveness of our attack. 
\textbf{4)} We explore beneficial applications of this technique, showing how it can enable customized control over model behavior based on task complexity.




\section{Related Work}
\label{sec:Related Work}
\subsection{Large vision language model}
Vision-language models\cite{li2023blip,li2024llava,bai2023qwen,lu2024deepseekvlrealworldvisionlanguageunderstanding, alayrac2022flamingo,sun2024generativemultimodalmodelsincontext}have achieved remarkable advancements within the realm of multimodal intelligence. By amalgamating large language models\cite{ray2023chatgpt,achiam2023gpt,anil2023palm,touvron2023llama2openfoundation,touvron2023llamaopenefficientfoundation} with visual content, LVLMs effectively manage intricate visual and linguistic inputs, thereby executing a variety of tasks ranging from visual description to logical reasoning. Flamingo\cite{alayrac2022flamingo} and OpenFlamingo\cite{awadalla2023openflamingoopensourceframeworktraining} models incorporate visual feature processing modules into the internal strata of language models using gated cross-attention, thereby propelling the profound integration of visual data within LLMs. CLIP\cite{radford2021learning,sun2023evaclipimprovedtrainingtechniques} utilizes contrastive learning to harmonize image and text modalities and is trained on extensive, noisy web-derived image-text pairs. By integrating modules such as QFormer\cite{li2023blip} and MLP\cite{liu2024visual}, previous works\cite{bai2023qwen, dai2023instructblipgeneralpurposevisionlanguagemodels,Liu_2024_CVPR} facilitate a collaborative comprehension between visual encoders and large language models (LLMs) of multimodal inputs. LLaVA\cite{liu2024visual} stands out for its pioneering use of GPT-generated instruction-following data to amplify LVLMs' responsiveness to visual instructions. A plethora of powerful LVLM APIs, including GPT-4o\cite{achiam2023gpt} and Qwen-VL-max\cite{bai2023qwen}, are now available. Through a rigorous evaluation of these models based on our proposed benchmark, we offer insightful perspectives into the ongoing research surrounding LVLMs.
\subsection{Vision Language Benchmarks} A rapidly expanding suite of multimodal benchmarks now rigorously evaluates the capabilities of LVLMs. Established benchmarks, including COCO Caption \cite{chen2015microsoftcococaptionsdata}, VQAv2 \cite{Goyal_2017_CVPR}, and GQA \cite{Hudson_2019_CVPR}, predominantly center on image description and question-answering tasks, employing metrics such as BLEU, CIDEr, and accuracy to gauge performance. Yet, as LVLMs advance, these traditional datasets have become insufficient for fully capturing the breadth of model capabilities. In response, researchers have developed more comprehensive evaluation frameworks that test a wider range of competencies, encompassing perceptual and cognitive skills \cite{fu2024mmecomprehensiveevaluationbenchmark}, spatial-temporal reasoning \cite{li2023seedbenchbenchmarkingmultimodalllms}, and relational understanding \cite{liu2025mmbench}. For instance, MMMU \cite{Yue_2024_CVPR} curates data from college-level textbooks and lecture materials, challenging models to demonstrate expertise across six academic disciplines. Similarly, CMMU \cite{he2024cmmubenchmarkchinesemultimodal} gathers questions from primary through high school curricula to assess foundational knowledge within the Chinese educational context. Nevertheless, these benchmarks largely remain focused on basic visual tasks, without adequately addressing the complexity of multimodal understanding. This paper introduces a benchmark tailored to evaluate deep semantic comprehension of images, specifically within a Chinese cultural framework.
\subsection{Image implicit meaning comprehension}
Image implicit meaning comprehension has become an important research focus for contemporary LVLMs, especially in handling images that convey complex emotions, cultural symbolism, and social critique. Existing evaluation datasets primarily test the models' linear visual reasoning abilities, such as visual question answering for surface-level content\cite{Hudson_2019_CVPR}. However, several works \cite{cai2019multi, machajdik2010affective} have demonstrated that LVLMs’ capabilities go beyond understanding surface-level meanings. Recent works\cite{yang2024largemultimodalmodelsuncover, liu2024iibenchimageimplicationunderstanding} highlight the limitations of current models when it comes to processing nonlinear narratives and understanding cultural contexts. For example, the most relevant prior work, DEEPEVAL\cite{yang2024largemultimodalmodelsuncover}, introduces three core tasks and shows that while the most advanced models achieve near-human performance on basic visual description tasks, they still perform poorly on tasks that involve understanding implicit semantics such as social background and satire. This paper provides a more comprehensive Chinese understanding benchmark, which, compared to the six categories in DeepEval, expands to include more thematic categories, with a total of 13 major categories and 41 subcategories (Figure \ref{fig:categories}), and offers more detailed testing across four dimensions of model performance.
% Image implicit meaning comprehension has emerged as a crucial research focus for contemporary LVLMs, particularly in handling images that convey nuanced emotions, cultural symbolism, and social critique. Achieving this level of comprehension demands that models infer implicit meanings from visual content, recognizing elements like satire, humor, and philosophical nuances. The most relevant prior work DEEPEVAL\cite{yang2024largemultimodalmodelsuncover} benchmark introduces three core tasks—fine-grained description selection. However, its limited categorization—comprising only six classes—restricts the scope of implicit meaning assessment, leaving out a broader range of complex visual semantics. 

% 2.1应该还没覆盖所有用到的模型;2.2需要补充点内容并且与2.3区分,2.3内容需要再调整
% 大型视觉语言模型(Flamingo, Blip2, Visual Instruction tuning,v Qwen-VL, LLaVA-next, DeeepSeekVL)近年来在多模态智能方面(Multimodal Intelligence)取得了显著进展。通过整合大规模语言模型(如GPTs*、LlaMa*、Palm2)和视觉内容(*), LVLMs可以处理复杂的视觉和语言输入,实现从视觉描述到逻辑推理等多种任务。Flamingo、OpenFlamingo模型通过gated cross在语言模型的内部层次中嵌入视觉特征处理模块,推动了视觉信息在LLMs中的深度整合。CLIP模型使用对比学习实现图像和文本模态的统一,并使用大规模noisy web 图像-文本对进行训练。14, 15 16,  17,通过添加QFormer和MLP等模块使视觉编码器和大型语言模型(LLMs)能够协同理解多模态输入。LLaVA则开创了通过GPT生成的instruction-following data提升LLvMs对视觉指令的响应能力。同时包括很多强大的LVLMs API公开,包括(GPT-4v*、Qwen-VL-max*) 。通过对上述模型进行全面评估\subsection{Vision Language Benchmarks} 
%为了系统地评估视觉语言模型的能力,近年来涌现了许多多模态评估基准。传统的评估基准如 COCO Caption*、VQAv2* 和 GQA* 等,主要集中在图像描述和问答任务,通过BLEU、CIDEr、准确率 等客观指标来衡量模型的性能。然而随着LVLMs的进步,这些数据集的难度已经不足以评估LVLMs的能力。研究者们进一步提出了更为全面的基准测试框架,从感知和认知能力(MME)、spatial and temporal understanding(SEED Bench),到Relation Reasoning能力(MMBench)。MMU从大学教材、讲义中收集数据,要求模型具备大学级别六大领域的专业知识。类似的,CMMU收集了小学至高中的七大学科题目,以评估模型对中文基础学科知识的理解与应用。然而,这些基准仅限于对基础视觉任务的评估,未能充分评估模型在复杂多模态任务中的表现,因此本文旨在提出一个中文背景下的评估模型深度图像含义的Benchmark。
%深层语义理解是当前LVLMs的一个重要研究方向,特别是在处理具有复杂情感、文化隐喻和社会批判的图像时尤为重要。深层语义的理解需要模型具备从视觉内容中推理出隐含意义的能力,例如理解讽刺、幽默和哲学内涵。DEEPEVAL* 提出了三种任务:细粒度描述选择、深入标题匹配和深层语义理解,通过这些任务系统性地评估了 LVLMs 在理解深层视觉语义上的表现。例如,尽管 GPT-4V* 在基础的视觉描述任务上达到了接近人类的水平,但在涉及社会背景和讽刺的语义理解任务中,仍存在显著差距。此外,

%图像隐含意义理解已成为当代大规模多模态语言模型(LVLMs)研究的一个重要方向,特别是在处理传达复杂情感、文化符号和社会批评的图像时。现有的评估数据集主要测试模型的线性视觉推理能力,例如对于浅层内容的视觉问答(VQA),。然而Machajdik的工作也证明了LVLM的能力不止于理解浅层含义。然而最近的工作(如 MVP、DeepEval 和 YESBUT Benchmark、Ii-Bench)揭示了现有模型在处理非线性叙事和文化背景理解时的局限性。例如最相关的前期工作 DEEPEVAL 引入了三个核心任务,发现当前最先进的模型在基础视觉描述任务上已接近人类水平,但在涉及社会背景和讽刺等隐含语义理解的任务中,仍表现不佳。本文提供了一个更为完备的中文理解Benchmark,相较于 DeepEval 的六大类任务,扩展了更多的主题类别,共包含13大类和41小类,并从四个维度对大模型的性能进行了更为详细的测试。



\section{Preliminary}
\label{sec:Preliminary}
% 补充一下IK和IDK的部分
\paragraph{\textit{(Definition 1. RAIT Dataset)}} 
The RAIT process can be described as follows: the initial LLM is prompted to answer all questions in the training set $D_{\text{src}}$. Based on the correctness of the responses, the samples are categorized into two groups. 
\textit{\textbf{\ding{182}}} Samples with correct responses are considered known knowledge of the LLM. These answers will remain unchanged and are referred to as \texttt{ik} samples, denoted as $D_{\text{ik}} = \{(x_{\text{ik}}, y_{\text{ik}})\}$ (where `ik' stands for `I know', $x_{\text{ik}}$ is the known question and $y_{\text{ik}}$ is the ground-truth label). \textit{\textbf{\ding{183}}} Conversely, samples with incorrect responses are treated as unknown knowledge. Their original answers are replaced with refusal responses such as ``I don't know'' forming $D_{\text{idk}} = \{(x_{\text{idk}}, y_{\text{idk}})\}$ (where `idk' stands for `I don't know', $x_{\text{ik}}$ is the unknown question and $y_{\text{ik}}$ is modified refusal response such as ``I don't know''). 
The constructed RAIT dataset, $D_{\text{rait}} = D_{\text{ik}} \cup D_{\text{idk}}$, is used to fine-tune the initial LLM, parameterized by $\theta$, to improve its ability to refuse to answer questions beyond its knowledge.

\paragraph{\textit{(Definition 2. Influence Formulation)}}
To estimate the influence of a training datapoint on a validation sample, we use the first-order Taylor expansion of the loss function \cite{Pruthi_Liu_Kale_Sundararajan_2020}\footnote{The reasons for using the influence formula are outlined in the appendix \ref{app:Reasons for Choosing Influence Formula}}. Specifically, for a model $\theta_t$ at step $t$, the loss on unobservant validation sample $x^{u}$ can be approximated as:
$
\mathcal{L}(x^{u},y^{u}; \theta_{t+1}) \approx \mathcal{L}(x^{u},y^{u}; \theta_t) + \langle \nabla \mathcal{L}(x^{u},y^{u}; \theta_t), \theta_{t+1} - \theta_t \rangle.
$
If the model is trained using Stochastic Gradient Descent (SGD) with batch size 1 and learning rate $\eta_t$, for the observant training sample $x^o$, the SGD update is written as:
$
\theta_{t+1} - \theta_t = -\eta_t \nabla \mathcal{L}(x^{o},y^{o}; \theta_t).
$
At this point, we can define the influence formula of $(x^{o},y^{o})$:
\begin{equation}
\small
\label{eq:influence_equation}
\begin{aligned}
\mathcal{I}(x^{o},y^{o},x^{u},y^{u}; \theta_{t}) \stackrel{\triangle}{=} & ~ \eta_t \langle \nabla \mathcal{L}(x^{o},y^{o}; \theta_t), \\ \quad \quad
& \nabla \mathcal{L}(x^{u},y^{u}; \theta_t) \rangle.
\end{aligned}
\end{equation}

\paragraph{\textit{(Task Definition)}}
The objective of this task is to leverage $D_{\text{rait}}$ to fine-tune a model and minimize the loss on two distinct types of test samples. Specifically, for samples that were previously incorrect, we aim for the model to output answers like ``I don't know'', while for correct samples, the predicted label should be as close as possible to the ground-truth label $y_{\text{ik}}$
% , improving upon the original model's performance
. The task can be formalized as minimizing the following loss:
\begin{equation}
\small
\begin{aligned}
\label{eq:task}
\min \bigl \{ \mathbb{E}_{x^{u}_{{\text{idk}}} \sim D_{{\text{idk}}}} &\left[ \Delta \mathcal{L}(x^{u}_{{\text{idk}}}, y^{u}_{{\text{idk}}}; \theta) \right ]\\ +& \mathbb{E}_{x^{u}_{{\text{ik}}} \sim D_{{\text{ik}}}} \left[ \Delta \mathcal{L}(x^{u}_{{\text{ik}}}, y^{u}_{{\text{ik}}}; \theta) \right ] \bigl\},
\end{aligned}
\end{equation}
In addition to minimizing it, a key objective of this task is to select the most suitable subset $\widetilde{D}_{\text{rait}} \subseteq D_{\text{rait}}$ for fine-tuning (c.f. Section~\ref{sec:Theoretical Analysis} for proof). By selecting optimal data from the RAIT dataset, we aim to improve the model's ability to refuse answers to unknown questions while minimizing over-refusal.

\section{Theoretical Analysis}
\label{sec:Theoretical Analysis}
This part is organized as $\mathbf{O}_1 \to \mathbf{O}_2$. Before obtaining formal observation results, we first propose two assumptions:

\paragraph{\textit{Assumptions 1. Distribution Assumption}}
\textit{We assume that the distributions of \texttt{ik} or \texttt{idk} from train and test sets are identically distributed, formally expressed as:}
$
\Pi_{D_{\text{idk}}^o} \sim \Pi_{D_{\text{idk}}^{u}},  \Pi_{D_{\text{ik}}^o} \sim \Pi_{D_{\text{ik}}^{u}}
$.

\paragraph{\textit{Assumptions 2. Orthogonality of Means}}
\textit{We further assume that the means of the gradient distributions for \texttt{idk} and \texttt{ik} are orthogonal as verified in Appendix~\ref{subsec:orthogonal_experiment}, and we have:}
\begin{equation}
\small
\label{eq:loss_decomposition}
\begin{aligned}
\Bigl \langle \mathbb{E}_{x_{*} } \left[ \nabla \mathcal{L}(x_{*},y_{\text{idk}}; \theta) \right],
\mathbb{E}_{x_{*} } \left[ \nabla \mathcal{L}(x_{*},y_{\text{ik}}; \theta) \right]
\Bigr \rangle \approx 0,
\end{aligned}
\end{equation}
where the \( * \) denotes the symbol of either \texttt{idk} or \texttt{ik}.

\subsection{Reducing Incorrectness ($\mathbf{O}_1$)}
\label{sec:Reducing Incorrectness}
We begin by focusing on minimizing the loss to improve the rejection rate, specifically aiming to minimize the first term of \eqref{eq:task} $\mathbb{E}_{x^{u}_{\text{idk}} \sim D_{\text{idk}}} \left[ \Delta \mathcal{L}(x^{u}_{\text{idk}}, y^{u}_{\text{idk}}; \theta) \right]$. Then, combining equation~\eqref{eq:influence_equation}, we can express this as:
\begin{equation}
\small
\label{eq:loss_decomposition_short}
\begin{aligned}
&\mathbb{E}_{x^{u}_{\text{idk}} \sim D_{\text{idk}}} \bigl[ \Delta \mathcal{L}(x^{u}_{\text{idk}}, y^{u}_{\text{idk}}; \theta) \bigl]  
\approx 
- \mathbb{E} _{(x^{u} _{\text{idk}}, x^o _{\text{idk}}) \sim D _{\text{idk}}} \\  &\left[ \mathcal{I}(x^{o} _{\text{idk}}, y^{o} _{\text{idk}}, x^{u} _{\text{idk}}, y^{u} _{\text{idk}}; \theta) \right]
\end{aligned} \noindent
\end{equation}
and the full proof is detailed in Appendix~\ref{app:More Proof on O1}.

Thus, samples with gradients similar to the average gradient direction of $D_{\text{idk}}$ are the most effective in reducing the model's hallucination rate.

\subsection{Alleviating Over-Refusal ($\mathbf{O}_2$)}
\label{sec:Alleviating Over-Refusal}
However, we observed that if we optimize the model merely depends on RAIT,
it leads to the issue of \textbf{over-refusal} (i.e., \texttt{ik} samples also tend to output ``I don't know''). Therefore, we delved deeper into the whole target in \eqref{eq:task} and derived the following( the full proof is detailed in Appendix \ref{app:More Proof on O2}):
\begin{equation}
\small
\label{eq:O2}
\begin{aligned}
&\mathbb{E}_{x^{u}_{\text{idk}} \sim D_{\text{idk}}} \left[ \Delta \mathcal{L}(x^{u}_{\text{idk}}, y^{u}_{\text{idk}}; \theta) \right ] 
+ \mathbb{E}_{x^{u}_{\text{ik}} \sim D_{\text{ik}}} \left[ \Delta \mathcal{L}(x^{u}_{\text{ik}}, y^{u}_{\text{ik}}; \theta) \right ] \\ 
\approx & - \left \{ \mathbb{E} _{x^{u} _{idk}, x^o _{idk} \sim D _{idk}} \left[ \mathcal{I}(x^{o} _{\text{idk}}, y^{o} _{\text{idk}}, x^{u} _{\text{idk}}, y^{u} _{\text{idk}}; \theta) \right] \right.  -\\
\quad &  \left . \mathbb{E} _{x^{u} _{\text{ik}} \sim D _\text{{ik}}, x^{o} _{\text{idk}} \sim D _{\text{idk}}} \left[ \mathcal{I}(x^{o} _{\text{idk}}, y^{o} _{\text{idk}}, x^{u} _{\text{ik}}, y^{u} _{\text{idk}}; \theta) \right] \right \}
\end{aligned} \noindent
\end{equation}


The first expectation term in equation \eqref{eq:O2} captures the reduction in the model's error rate, while the second term reflects the occurrence of over-refusal. Training samples where the difference between these two terms is smaller tend to exacerbate over-refusal, though they may also contribute to stronger overall model performance.



%Task/Problem formulation
\subsection{Problem Definition}
In the aerial VLN task, a UAV is randomly positioned within a 3D environment with its initial pose defined as $P = [x, y, z, \phi, \theta, \psi]$. At each timestamp $t$, the UAV perceives the surrounding environment through an egocentric image as its observation. Guided by natural language instructions, the task involves predicting the next navigation action. Notably, the UAV can utilize either the current observations or the frames from all previous timestamps to make its prediction.

\subsection{Model Architecture}
As shown in Fig. \ref{fig:model}, we take OpenVLA~\cite{openvla} as the baseline and design an end-to-end model for aerial VLN. In contrast, our model takes a sequence of images to indicate the observation instead of one image in the original OpenVLA. Moreover, to mitigate visual redundancy between adjacent video frames while maintaining key information, two strategies are proposed, \emph{i.e.,} keyframe selection and visual token merging. First, a series of candidate keyframes are selected. Then, these keyframes are merged temporally before and after the vision encoder, resulting in a compact sequence of visual tokens. Finally, the action decoder discretizes the predicted tokens into uniformly distributed bins, which are subsequently mapped to the 6 action types specific to drones. 


\subsubsection{Keyframe Selection}
The length of contextual visual tokens is a major challenge for VLMs when processing videos. Many open-source VLMs use uniform frame sampling \cite{buch2022revisiting, ranasinghe2024understanding, wang2025videoagent} to reduce calculation, but this strategy is not suitable for aerial VLN, since it may miss frames containing key landmarks. 
To address this issue, we adopt a heuristic method to identify keyframes by detecting the change point of the UAV's movement. We notice that sudden changes in the UAV's trajectory are often caused by the observation of landmarks, which can serve as cues to determine keyframes. Specifically, we use the movement of the drone over time to draw turning curves, and the frames near the peaks of the wave are selected as candidate keyframes. The resulting data is interpolated and smoothed, forming a wave-like curve that represents the UAV's movement. 

To further ensure the precision of training data, scene segmentation maps collected in Sec. \ref{sec:Automatic} are used on selected frames to detect key landmarks. Frames containing landmarks are selected as keyframes, yielding reasonably accurate results. Note that each sudden change of actions, \emph{e.g.,} from `Forward' to `Turn Left', will produce a set of keyframes. Consequently, we obtain several sets of keyframes for a long trajectory. 
%For testing, we select keyframes where the action changes, as these often correspond to the observation of a critical landmark.
%Sec Parag

%This keyframe selection scheme gives model the guidance for action prediction via semantic relationship from the observation of the subgoal. Next, with the candidate frame sequences, we introduce the online visual Token merging module for the next action prediction.

\subsubsection{Visual Token Merging}
To further reduce redundant information in keyframes, we design visual token merging, where the core concept is to recognize the similarity between image tokens. It compares adjacent keyframes to merge similar regions and maintains its simplicity by token compression.

%合并阶段。
% 在获得候选帧之后,我们先逐帧过一遍vision encoder获取visual features,再利用标记相似性来合并相邻帧的视觉标记。类似 ToMe [] ,我们通过定期合并之后相邻帧中最相似的标记来进行记忆巩固。我们计算 N 个嵌入标记之间的平均余弦相似度s,在每次合并操作后保留K帧,这也嵌入了存储在长期记忆中的丰富信息。K是控制性能和效率之间权衡的超参数。因此,我们通过加权平均地合并每组相邻帧相似度最高的tokens。合并操作迭代进行,直到token计数达到每个合并操作的预定义值集K。合并阶段应用于Vision Transformer的倒数第二层特征patch token,以逐步合并相似的标记,直到相似标记的数量低于特定层的阈值 Nthreshold。合并阶段之后,剩余的唯一标记将进入压缩阶段。


\begin{table*}[t!]

\centering
\caption{Comparison results on the test-seen split.}
%\vspace{-5pt}
\label{tab:seen_results}
\begin{adjustbox}{center}
\resizebox{\textwidth}{!}{ 
%\setlength{\tabcolsep}{1.6pt}
\renewcommand{\arraystretch}{1.3}
% \scalebox{0.95}{
\begin{tabular}{lccccccccccccccccc}
\toprule
\multirow{2}{*}{Method} & \multicolumn{4}{c}{Easy} & \multicolumn{4}{c}{Moderate} & \multicolumn{4}{c}{Hard} & \multicolumn{4}{c}{Total}\\ 
\cmidrule(lr){2-5} \cmidrule(lr){6-9} \cmidrule(lr){10-13} \cmidrule(lr){14-17}
& NE$\downarrow$ & SR$\uparrow$ & OSR$\uparrow$ & SPL$\uparrow$ 
& NE$\downarrow$ & SR$\uparrow$ & OSR$\uparrow$ & SPL$\uparrow$ 
& NE$\downarrow$ & SR$\uparrow$ & OSR$\uparrow$ & SPL$\uparrow$
& NE$\downarrow$ & SR$\uparrow$ & OSR$\uparrow$ & SPL$\uparrow$ \\ \midrule 

Random & 289m & 0.9\% & 1.1\% & 0\% & 351m & 1.3\% & 1.3\% & 0\% & 374m & 0\% & 0\% & 0\% & 242m & 0.7\% & 0.8\% & 0\% \\
Seq2Seq\cite{VLN-CE}&  201m &  0.9\% & 21.2\% & 0.9\% & 190m & 8.9\% & 19.2\% & 6.5\% & 192m & 2.1\% & 10.1\% & 1.9\% & 194m & 4.0\% & 16.8\%  &  3.1\% \\
CMA\cite{VLN-CE}&  156m & 1.2\% &  35.6\% & 1.6\% & 120m & 11.2\% & 34.5\% & 8.4\% & 156m & 4.6\% & 20.1\% & 5.3\% & 144m & 5.7\% & 30.0\% & 5.1\%\\
AerialVLN\cite{aerialVLN}& \underline{148m} & 1.5\% &  \underline{40.2\%} & 2.6\% & \textbf{94m} & \underline{13.2\%} & \textbf{58.6\%} & \underline{10.7\%} & 147m & 5.4\% & \underline{23.6\%} & \underline{7.6\%} & \underline{130m} & 6.6\% &\underline{40.8\%} & \underline{7.0\%}\\
Navid\cite{navid}& 151m & \underline{11.2\%} & 28.9\% & \underline{4.5\%} & 138m & 8.0\% & 21.3\% & 2.8\% & \underline{134m} & \textbf{10.3\%} & 21.3\% & 4.6\% & 142m & \underline{9.9\%} & 24.3\% & 3.9\% \\
Ours& \textbf{111m} &  \textbf{26.5\%} & \textbf{55.6\%}  & \textbf{16.0\%} & \underline{115m} & \textbf{16.4\%} & \underline{51.2\%} & \textbf{11.2\%} & \textbf{120m} & \textbf{10.3\%} & \textbf{29.6\%} & \textbf{8.2\%}  & \textbf{115m} & \textbf{18.5\%} & \textbf{50.9\%} & \textbf{12.2\%} \\
\bottomrule
\end{tabular}
}
\end{adjustbox}
\end{table*}



For each set of candidate keyframes obtained in the previous selection process, a visual encoder maps each input image to multiple visual tokens, with each token representing the information of an image patch. Considering the potential inter-frame patch redundancy, we take a strategy that similar tokens in subsequent adjacent frames are periodically merged. Specifically, we select the first frame in a keyframe set as the reference, since it usually contains the crucial observation indicating the time for action transition. Then, we densely calculate the cosine similarities between each pair of visual tokens of the reference image and the subsequent image. Next, we merge the tokens with high similarity by averaging them. The unmerged tokens in the subsequent frame will be discarded. The merging operation is iteratively performed until the entire keyframe set has been traversed. Besides, we maintain a memory bank with a capacity of $K$ images, which follows a first-in-first-out (FIFO) policy to retain the latest keyframes.

After the above process, $M$ visual tokens $E=\{e_1, e_2, \cdots, e_M\}$ are obtained for each set of keyframes. Since aerial VLN requires UAVs to perform long-distance flights based on instructions, we continue to carry out token compress to reduce the computational burden. The compressed visual tokens $E_c$ are obtained through grid pooling~\cite{llama_vid}. Notably, we keep the visual tokens of the current frame uncompressed to capture the latest visual observation, as it contains the most important information for flight action prediction.




\subsubsection{Action Prediction}
Similar to~\cite{aerialVLN,CityNav}, 6 actions for UAVs are defined as $\{$Forward, Turn Left, Turn Right, Move Up, Move Down, Stop$\}$ in this work. The units for `Move up' and `Move down' are 3 m, the units for `Turn Left' and `Turn Right' are 30 degrees. `Forward' has three distinct units, namely 3 m, 6 m, and 9 m, respectively. For flight action prediction, each action type is discretized into multiple bins with one non-activate bin indicating that the current action is not activated. We map the model output to one of the bins for each action type, where the bin number corresponds to the amount of units in each action.

\section{Experiment}
\label{sec:Experiment}
In this section, we provide detailed information on experimental setup, and further analysis to validate the performance and rationality of \M.


\subsection{Experiment Setup}
\paragraph{Datasets.}
In this study, we assess the efficacy of \M in handling two distinct types of Question and Answering tasks: the knowledge-based Multiple Choice Question Answering (MCQA) and Open-ended Question Answering (OEQA). For the MCQA task, the test split of MMLU~\cite{MMLU} is adopted as the training dataset, while the validation split of the same serves as the In-Domain (ID) test set, and the ARC-c~\cite{ARC_C} test split is utilized as the Out-Of-Domain (OOD) test set. In the context of the OEQA task, we use the training split of TriviaQA~\cite{triviaqa} for training purposes, the development split of TriviaQA as the ID test set, and the validation split of NQ~\cite{nq} as the OOD test set. Additional information is provided in Table~\ref{table:dataset_details}.

\begingroup
\fontsize{9}{11}\selectfont
\setlength{\tabcolsep}{1mm}
% {\fontsize{9pt}{11pt}\selectfont
\begin{table}[h]
\centering
\caption{Datasets Details.}
\resizebox{1.0\linewidth}{!}{
\begin{tabular}{lcc}
\toprule
\textbf{} & \textbf{MCQA} & \textbf{OEQA} \\ 
\midrule
\textbf{Train}     & MMLU test (14,079)     & TriviaQA train (87,622)    \\ 
\textbf{ID Eval}   & MMLU val (1,540)       & TriviaQA dev (11,313)      \\ 
\textbf{OOD Eval}  & ARC-c dev (1,172)      & NQ dev (3,610)             \\ 
\bottomrule
\end{tabular}}
\label{table:dataset_details}
\end{table}
\endgroup



\paragraph{Baselines.}
To evaluate the performance of \M, we conducted comparisons with several existing approaches:
\textbf{Init-Basic}: Employs the initial LLM setup, utilizing standard question-answering prompts to guide the model in generating answers.
\textbf{Init-Refuse}: Builds on Init-Basic by incorporating instructions such as ``\textit{If you do not know the answer, please respond with `I don't know.'}'' to promote safer responses~\cite{bianchi2024safetytunedllamaslessonsimproving,zhangdefending}.
\textbf{Van-Tuning}: Randomly selects \(N_{\text{ik}} + N_{\text{idk}}\) samples from \(D_{\text{src}}\) for straightforward instruct-tuning, without any sample modification.
\textbf{R-Tuning}: Follows the settings from \cite{R_Tuning}, where samples in the RAIT dataset are modified based on the correctness of the model's replies.
\textbf{CRaFT}: This method is implemented according to~\cite{zhu2024utilizeflowsteppingriver}, addressing both static and dynamic conflicts within the RAIT dataset to provide a thorough evaluation of potential issues.

\begin{figure}[t]
    \centering
    \includegraphics[width=0.7\linewidth]{figure/metric.pdf}
    \caption{Illustration of Truthful Helpfulness Score.}
    \label{fig:metric}
\end{figure}

\paragraph{Evaluation Metrics.}

We utilize the Truthful Helpfulness Score (THS) as detailed by~\cite{zhu2024utilizeflowsteppingriver} to assess the performance of LLMs after RAIT. Accuracy (\(P_c\)), error rate (\(P_w\)), THS, etc. are key metrics for evaluating the performance of models after RAIT. Among these, \(P_c\) and \(P_w\) form a competing pair, where optimizing for \(P_c\) often leads to a decline in \(P_w\). Focusing on only one of these metrics is insufficient to evaluate the model’s overall capability.
Thus, a singular and comprehensive metric is required to simplify the assessment process and eliminate the complexity of balancing multiple trade-off metrics.

For each test sample, we classify the response as correct, incorrect, or refused. From these categories, we calculate the accuracy (\(P_c\)), error rate (\(P_w\)), and refusal rate (\(P_r\)). We then set up a Cartesian coordinate system with \(P_c\) and \(P_w\) on the axes. The point \(S_1\) represents the coordinates of the baseline LLM, and \(S_2\) corresponds to the refined model.
If \(S_2\) is positioned below the line from the origin \(O\) to \(S_1\) (denoted as \(OS_1\)), then a larger area of the triangle \(\triangle OS_1S_2\) signifies an improvement in the model. If, however, \(S_2\) is above \(OS_1\), it indicates a reduction in performance. As shown in Figure \ref{fig:metric}, THS is defined as the ratio of the cross product of vectors \(\overrightarrow{OS_1}\) and \(\overrightarrow{OS_2}\) to the maximum possible value of this cross product:


\begin{equation}
\small
\label{eq:loss_decomposition}
\begin{aligned}
\text{THS} = (\overrightarrow{OS_2} \times \overrightarrow{OS_1}) / (\overrightarrow{OU} \times \overrightarrow{OS_1}).
\end{aligned}
\end{equation}

\begingroup
\fontsize{6}{6}\selectfont
\setlength{\tabcolsep}{1mm}
\renewcommand{\arraystretch}{1.0} % Increase row spacing for better readability
\begin{table*}[!t]
\vspace{-0.4cm}
\caption{Performance comparisons on MMLU, ARC-c, TriviaQA and NQ. The best performance is highlighted in \textbf{boldface}, while the second-best performance is \underline{underlined}.}
\vspace{-0.2cm}
\centering
\resizebox{1.0\linewidth}{!}{
\begin{tabular}{ccc|ccc|ccc|ccc|ccc}
\hline
\multirow{3}{*}{\textbf{LLMs}} & \multicolumn{2}{c|}{\textbf{QA Type}} & \multicolumn{6}{c|}{\textbf{MCQA}} & \multicolumn{6}{c}{\textbf{OEQA}} \\
\cline{2-15}
& \multicolumn{2}{c|}{\textbf{Dataset}} & \multicolumn{3}{c|}{\textbf{MMLU (ID)}} & \multicolumn{3}{c|}{\textbf{ARC-c (OOD)}} & \multicolumn{3}{c|}{\textbf{TriviaQA (ID)}} & \multicolumn{3}{c}{\textbf{NQ (OOD)}} \\
\cline{2-15} 
& \multicolumn{2}{c|}{\textbf{Metric}} & $P_c$ & $P_w\downarrow$ & THS$\uparrow$ & $P_c$ & $P_w\downarrow$ & THS$\uparrow$ & $P_c$ & $P_w\downarrow$ & THS$\uparrow$ & $P_c$ & $P_w\downarrow$ & THS$\uparrow$ \\
\hline
\multirow{9}{*}{\shortstack{\textbf{Llama2-7B} \\ \textbf{Chat}}}
& \multirow{5}{*}{\textbf{Baselines}}  & Init-Basic &  45.6 & 52.8 & 00.0 & 53.9 & 46.0 & 00.0 & 54.0 & 46.0 & 00.0 & 28.9 & 71.1 & 00.0 \\ 
&& Init-Refuse & 36.4 & 38.9 & 03.9 & 44.4 & 35.7 & 02.6 & 37.1 & 21.7 & 11.5 & 19.8 & \textbf{34.8} & \textbf{05.6} \\ 
&& Van-Tuning & 46.9 & 53.0 & 01.2 & 54.5 & 45.5 & 01.2 & 55.5 & 44.5 & 03.2 & 23.2 & 76.8 & -0.80 \\ 
&& R-Tuning & 44.5 & 39.6 & 11.3 & 55.8 & 38.1 & 11.1 & 52.2 & 35.9 & 10.0 & 22.6 & 60.9 & -0.22 \\ 
&& CRaFT & 43.9 & 36.4 & 12.5 & 54.7 & 35.9 & 12.6 & 47.8 & 28.1 & 14.8 & 26.7 & 62.0 & 01.5 \\  \cdashline{2-15}
&\textbf{Ours}& \M & 43.5 & \underline{27.1} & \textbf{20.1} & 55.2 & \textbf{26.5} & \textbf{24.2} & 43.6 & \underline{18.4} & \textbf{22.0} & 20.8 & 49.7 & 00.0 \\  \cdashline{2-15}
&\multirow{2}{*}{\textbf{Ablations}}& \texttt{w/o} $\mathbf{O}_{1}$ & 44.7 & 39.8 & 10.3 & 55.4 & 37.9 & 11.0 & 52.4 & 36.5 & 09.6 & 23.9 & 63.5 & -01.9 \\  
&& \texttt{w/o} $\mathbf{O}_{2}$ & 42.8 & \textbf{26.5} & \underline{20.0} & 54.1 & \underline{26.7} & \underline{22.8} & 41.9 & \textbf{18.1} & \underline{20.6} & 20.1 & \underline{48.3} & \underline{00.5} \\  
\hline
\multirow{9}{*}{\shortstack{\textbf{Llama3-8B} \\ \textbf{Instruct}}}
& \multirow{5}{*}{\textbf{Baselines}} & Init-Basic & 66.8 & 33.1 & 00.0 & 80.6 & 19.5 & 00.0 & 66.8 & 33.2 & 00.0 & 40.3 & 59.7 & \underline{00.0} \\ 
&& Init-Refuse & 50.0 & 17.0 & 15.7 & 65.3 & 14.4 & 05.6 & 53.9 & 20.8 & 12.0 & 31.1 & \textbf{38.6} & \textbf{05.0}
\\ 
&& Van-Tuning & 69.5 & 30.5 & 08.0 & 80.3 & 19.7 & -01.3 & 60.0 & 40.0 & -19.0 & 21.0 & 48.5 & -11.7 \\ 
&& R-Tuning & 63.9 & 21.6 & 20.4 & 79.4 & 16.2 & 12.2 & 56.6 & 28.3 & -00.5 & 25.1 & 74.9 & -25.6 \\ 
&& CRaFT & 53.3 & 09.6 & 34.0 & 74.1 & 12.7 & 21.4 & 57.8 & 27.7 & 02.0 & 27.0 & 57.6 & -12.0 \\  \cdashline{2-15}
&\textbf{Ours}& \M & 50.4 & \textbf{06.9} & \textbf{36.4} & 70.2 & \underline{08.7} & \textbf{34.3} & 55.3 & \textbf{18.3} & \textbf{18.5} & 21.9 & \underline{38.8} & -04.4 \\  \cdashline{2-15}
&\multirow{2}{*}{\textbf{Ablations}}& \texttt{w/o} $\mathbf{O}_{1}$ & 64.1 & 21.4 & 20.9 & 79.3 & 16.4 & 11.5 & 57.5 & 28.7 & -00.2 & 25.6 & 75.0 & -25.0 \\  
&& \texttt{w/o} $\mathbf{O}_{2}$ & 49.6 & \underline{07.0} & \underline{35.5} & 69.1 & \textbf{08.6} & \underline{33.6} & 54.3 & \textbf{18.3} & \underline{17.4} & 21.6 & 39.1 & -04.8 \\  
\hline
\end{tabular}}
\label{table:main table}
\end{table*}
\endgroup


\paragraph{Implementation Details.}

In our studies, we utilized LLaMA2-7B-Chat and LLaMA3-8B-Instruct as the initial LLMs \( \theta_0 \). For the MCQA task, we selected 5,000 samples from the MMLU dataset for training purposes, and for the OEQA task, 10,000 samples from TriviaQA were used. With the exception of the Van-Tuning setting, where all samples were kept unchanged, other RAIT settings used a 1:4 ratio of \texttt{ik} samples to \texttt{idk} samples. In the MCQA and OEQA tasks, correctness is obtained using 5-shot and 3-shot setups\footnote{The reasons for using the few-shot settings are outlined in the appendix \ref{A5}}, respectively. More implementation details are listed in Appendix~\ref{app:imple}. In contrast to \cite{zhu2024utilizeflowsteppingriver}, to ensure the fairness of the experiments, we employ LoRA for training across both MCQA and OEQA tasks.

During both training and testing phases, XTuner~\footnote{https://github.com/InternLM/xtuner} was employed for RAIT experiments, which were conducted over 3 epochs with a maximum context length set to 2048. The LoRA~\cite{hulora} was implemented with the parameters: \(r=64\), \(\alpha=16\), dropout rate of 0.1, and a learning rate of \(2 \times 10^{-4}\). For evaluations, the 0-shot approach with greedy decoding was adopted. OpenCompass~\footnote{https://github.com/open-compass/opencompass} is used for all evaluations and correctness calculations. In the \M method, we assigned the hyperparameter \(\mathcal{T}_{\mathcal{C}}\) a value of 0.5 and \(\tau\) a value of 0.05. All experiments were executed on eight NVIDIA A100-80GB GPUs.




\subsection{Experiment Results}
We present the main experimental results, along with an ablation study of \M across various models in Table~\ref{table:main table}. A summary of the key findings is provided below.

\subsubsection{Main Results}
We assess the effectiveness of \M by addressing the challenges \textbf{\textit{C1}} and \textbf{\textit{C2}}, with the corresponding experimental results presented in Table~\ref{table:main table}. \\
\textbf{Comparison based on \textit{C1}:} \textbf{\textit{C1}} relates to the metric \(P_w\), where a lower \(P_w\) indicates better avoidance of hallucinations by the model. As shown in the results, our proposed \M achieves a significantly lower \(P_w\) compared to other baselines, demonstrating its effectiveness in reducing hallucination rates. \\
\textbf{Comparison based on \textit{C2}:} \textbf{\textit{C2}} focuses on minimizing hallucinations while maintaining accuracy, addressing the challenge of over-refusal. 
% To evaluate this, we use the Truthful Helpfulness Score (THS). 
\M surpasses existing methods in THS score with an average of 3.66.
Specifically, the THS results clearly show that our method significantly outperforms other baselines on both in-domain (ID) and out-of-domain (OOD) settings. For instance, on MMLU dataset, the LLaMA2-7B-Chat model achieves a THS score of 19.3, whereas the best-performing baseline, CRaFT, only reaches 12.5. Moreover, our approach consistently demonstrates superior performance on OOD datasets as well.

\subsubsection{Ablation Study}
We conduct ablation studies to evaluate the contribution of each component in \M, as presented in Table~\ref{table:main table}, using two variants: (1) \M without Refusal Influence, which follows the R-Tuning approach during the dataset distillation phase (denoted as \texttt{w/o} $\mathbf{O}_1$), and (2) \M without Stable Influence, where no weight adjustment is applied to emphasize the importance of \texttt{idk} samples (denoted as \texttt{w/o} $\mathbf{O}_2$). The results indicate that each component contributes positively to the overall performance of \M and the removal of any component leads to a noticeable decline in effectiveness. Specifically, replacing Refusal Influence-based dataset distillation with other baselines results in a significant increase in hallucination rate, underscoring the importance of Refusal Influence in addressing \textbf{\textit{C1}}. Additionally, the use of Stable Influence helps reduce over-refusal while maintaining a stable hallucination rate, effectively addressing the challenges posed by \textbf{\textit{C2}}. In addition, we conducted sensitivity experiments, the details of which can be found in the appendix \ref{A6}.

% \subsubsection{Sensitivity Study}
% \textbf{The effect of temperature on sample weight.} We analyzed the Stable Influence of the samples discussed in Stage 3 of \M and found that its values were relatively small. Consequently, it was necessary to adjust the temperature to better normalize the weight of each sample. To investigate the specific effects of temperature adjustments, we conducted experiments using the MMLU dataset and the LLaMA3-8B-Instruct model.
% 敏感性实验做完了,但是信息量太少不好画图展示,可能只能画表格
% T = [0.01, 0.05, 0.1, 0.2, 0.5, 1]
% THS = [36.6, 36.4, 36.5, 35.9, 35.5, 35.5]
% T_C = [0.3, 0.4, 0.5, 0.6, 0.7]
% THS = [37.4, 36.7, 36.4, 36.4, 36.3]

\begin{figure}[t]
    \centering
    % \vspace{-1.5cm}
    \includegraphics[width=1.0\linewidth]{figure/analysis.pdf}
    % \vspace{-0.74cm}
    \caption{Relationship between \(\mathcal{I}^{\text{ref}}\) and \(\mathcal{I}^{\text{over}}\) in MMLU performance on LLaMA2-7B-Chat and LLaMA3-8B-Instruct.}
    % \vspace{-0.3cm}
    \label{fig:analysis}
\end{figure}

\subsection{Analysis}
\textbf{The selection of \texttt{ik} samples is crucial.} Our analysis and experiments primarily focus on optimizing the selection of \texttt{idk} samples. However, the selection of \texttt{ik} samples is also crucial. We employed three different strategies: \texttt{ik-random}, where data is randomly selected from $D_{\text{ik}}$; \texttt{ik-bottom}, where the data with the lowest correctness from $D_{\text{ik}}$ is selected; and \texttt{ik-top}, the method used in \M, where the data with the highest correctness from $D_{\text{ik}}$ is chosen. We used the MMLU (ID) and ARC-c (OOD) datasets and conducted experiments with the LLaMA3-8B-Instruct model. The results are shown in Table 3. When using either the \texttt{ik-bottom} or \texttt{ik-random} methods, the model's hallucination reduction does not improve, and the refusal rate remains low. We believe the potential reason for this is that the \texttt{ik} samples selected by these methods may share similar characteristics with the \texttt{idk} samples, but different supervision signals were applied during the SFT process. This weakens the model’s ability to learn effective refusals. In contrast, the \texttt{ik-top} strategy used in \M helps to distinctly separate the features of the two types of samples, addressing the static conflict mentioned in \cite{zhu2024utilizeflowsteppingriver}.

\begingroup
\fontsize{6}{6}\selectfont
\setlength{\tabcolsep}{1mm}
\renewcommand{\arraystretch}{1.2} % Increase row spacing for better readability
\label{table:Analysis}
\begin{table}[!t]
\small  % 调整字体大小
\vspace{-0.4cm}
\caption{Performance comparisons on MMLU and ARC-c for different \texttt{ik} selection methods on LLaMA3-8B-Instruct.}
\vspace{-0.2cm}
\centering
\begin{tabular}{c|ccc|ccc}
\hline
\textbf{Dataset} & \multicolumn{3}{c|}{\textbf{MMLU (ID)}} & \multicolumn{3}{c}{\textbf{ARC-c (OOD)}} \\
\cline{1-7} 
\textbf{Metric} & $P_c$ & $P_w\downarrow$ & THS$\uparrow$ & $P_c$ & $P_w\downarrow$ & THS$\uparrow$ \\
\hline
\texttt{ik-top} & 50.4 & 06.9 & 36.4 & 70.2 & 08.7 & 34.3 \\ 
\texttt{ik-random} & 61.4 & 20.5 & 20.0 & 78.7 & 15.6 & 14.2 \\ 
\texttt{ik-bottom} & 64.0 & 25.3 & 12.9 & 79.7 & 19.0 & -00.2 \\ 
\hline
\end{tabular}
\end{table}

\endgroup



\textbf{Over-Refusal can only be alleviated, but not completely eliminated.}
During the RAIT process, we observed and analyzed the \texttt{idk} influence (corresponding to $O_{2}$) of \texttt{idk} samples on $D_{\text{ik}}$ and $D_{\text{idk}}$ using the LLaMA2-7B-Chat and LLaMA3-8B-Instruct model on the MMLU dataset. As shown in Figure~\ref{fig:analysis}, we identified a strong correlation between the two, with a Pearson Correlation Coefficient of 0.886. This correlation may be a contributing factor to the occurrence of Over-Refusal. While our proposed method, as indicated in Table~\ref{table:main table}, cannot fully eliminate Over-Refusal due to certain limitations, it significantly mitigates the issue.



\section{Limitations and Future Work}
The proposed OpenFly platform incorporates various rendering engines/techniques to provide high-quality scenes. Specifically, this is the first attempt to use 3D GS reconstructed scenes to support real-to-sim training and testing, while in the reconstruction of large-scale areas, a few visual artifacts are inevitably present. Future work will focus on exploring more effective reconstruction methods to enhance realism in large-scale scenes. Besides, the proposed OpenFly-Agent is built upon the large VLN model architecture, which is not practical for real-time deployment on UAVs. To address this, future research should focus on developing more efficient architectures and effective quantization techniques. 


\section{Conclusion}
In this work, we present OpenFly, a platform designed for large-scale data collection in aerial Vision-and-Language Navigation (VLN). OpenFly integrates multiple rendering engines and advanced real-to-sim techniques for data generation, enabling efficient collection of diverse, high-quality aerial VLN data. The resulting large-scale dataset comprises 100k trajectories across 18 distinct scenes, spanning a wide range of altitudes and difficulty levels, which is significantly superior than existing ones. Furthermore, we propose OpenFly-Agent, a keyframe-aware aerial navigation model capable of directly predicting flight actions based on observations and language instructions. Extensive experiments validate the effectiveness of the proposed method, and establishing a comprehensive benchmark for future advancements in aerial navigation. 
%The toolchain, dataset, and code will be publicly released, providing a valuable resource for future research in this field.

\section*{Limitations}
\label{latex/limitation}
While our work has yielded promising results, it is important to recognize several limitations. First, the \M framework currently treats the training process as static, rather than incorporating the dynamic influence of gradient trajectories throughout the RAIT process. Additionally, the \texttt{idk} and \texttt{ik} sets are divided through a straightforward query of the LLMs; future work could explore ways to leverage \M for more nuanced identification of knowledge boundaries within LLMs for splitting. Finally, although \M has demonstrated strong generalizability across various evaluation datasets, expanding the dataset range to include a more diverse set of high-quality resources could enhance the robustness and versatility of the framework.



\section*{Acknowledgments}
This research was supported by Shanghai Artificial Intelligence Laboratory. 

\bibliography{custom}

\appendix

\newpage
\appendix
\section{Appendix}
\subsection{Metric Optimization}  \label{app:pg}
We utilize the REINFORCE algorithm to optimize the performance metric. The detailed optimization process is proved in the following equations:
    \begin{equation}
        \small
        \begin{aligned}
            &\nabla_{\Lambda}\hat{l}(\Lambda)
            =\nabla_{\Lambda} \mathbb{E}_{s\sim \pi{(\mathcal{B},\cdot;{\Lambda})}} \mathcal{R}(\hat{\mathcal{D}}, f(\Theta^*(\Lambda)))\\
            &=\nabla_{\Lambda}\sum_{s\in[0,1]^{|\mathcal{B}|}} \mathcal{R}(\hat{\mathcal{D}}, f(\Theta^*(\Lambda))) \cdot \pi(\mathcal{B},s;{\Lambda}) \\
            &=\sum_{s\in[0,1]^{|\mathcal{B}|}} \mathcal{R}(\hat{\mathcal{D}}, f(\Theta^*(\Lambda))) \cdot 
            \frac{\nabla_{\Lambda}\pi(\mathcal{B},s;{\Lambda})}{\pi(\mathcal{B},s;{\Lambda})}\cdot \pi(\mathcal{B},s;{\Lambda})\\
            &= \sum_{s\in[0,1]^{|\mathcal{B}|}} \mathcal{R}(\hat{\mathcal{D}}, f(\Theta^*(\Lambda))) \cdot \nabla_{\Lambda}log(\pi(\mathcal{B},s;{\Lambda}))\cdot \pi(\mathcal{B},s;{\Lambda})\\
            &=\mathbb{E}_{s\sim \pi(\mathcal{B},\cdot;{\Lambda})}[\mathcal{R}(\hat{\mathcal{D}}, f(\Theta^*(\Lambda)))\cdot \nabla_{\Lambda}log(\pi(\mathcal{B},s;{\Lambda}))],
        \end{aligned}
    \end{equation}

\subsection{Learning Algorithm}  \label{app:learning_algorithm}
The detailed optimization steps of the proposed framework are given in Algorithm \ref{al:method}.

\subsection{Detail of Studied Methods} \label{app:studied method}
To show the compatibility of our method, we apply the DVR framework on four recommendation backbones, i.e., BRPMF~\cite{koren2009matrix}, NeuMF~\cite{he2017neural}, MGCF~\cite{wang2019neural}, and LightGCN~\cite{he2020lightgcn}. We select BPRMF due to its widespread adoption in recommendation systems and proven effectiveness in practical applications. NeuMF, an MLP-based approach, extends the capabilities of BPRMF by capturing intricate user-item relationships. We leverage GNN-based models, such as MGCF and LightGCN, known for their state-of-the-art performance and competitive outcomes across various techniques, to serve as the recommendation backbone. 

Based on these backbones, different versions of the DVR model are tailored to optimize diverse metrics. For simplicity, we designate models optimized for ranking accuracy as DVR-Loss, DVR-Recall, and DVR-NDCG. Likewise, models focused on diversity and fairness metrics are labeled as DVR-CC, DVR-ILD, and DVR-Gini. 


We compare our framework with various data valuation methods for recommendations. BPR~\cite{10.5555/1795114.1795167} uniformly samples negative items and treats all training data equally in constructing the training objective. TCE-BPR and RCE-BPR are extensions of the TCE and RCE techniques \cite{10.1145/3437963.3441800}, aimed at dynamically filtering out noisy positive interactions during training based on loss values. In our implementation, we replace the original point-wise loss with a pair-wise ranking loss objective to ensure a fair comparison with these methods. AOBPR \cite{10.1145/2556195.2556248} enhances the BPR algorithm by incorporating adaptive sampling techniques that prioritize popular negative items. WBPR \cite{gantner2012personalized} assumes that unexplored popular items by a user are more likely to be true negatives. PRIS \cite{10.1145/3366423.3380187} assigns higher weights to informative negative samples using importance sampling. TIL-UI and TIL-MI \cite{wu2022adapting} learn the data value of training triplets through two aggregation strategies by optimizing the BPR loss within the training batch.

\subsection{Implementation Details} \label{app:implenmentation}
We optimize all models using the Adam optimizer with Xavier initialization \cite{glorot2010understanding} and maintain a fixed embedding size of 64 across all methods. When constructing the ranking loss objective, every positive item is associated with one sampled negative item for an efficient computation. Grid search is applied to choose learning rate and weight decay from $\left\{1e^{-4}, 1e^{-3}, 1e^{-2}, 1e^{-1}\right\}$ and $\left\{1e^{-6}, 1e^{-5}, 1e^{-4}, 1e^{-3}\right\}$. The backbone models NeuMF, MGCCF, and LightGCN utilize the provided implementations, with MGCF and LightGCN featuring two graph convolution layers. The total number of training epochs is set to 2000 for all models with an early stopping design. Given the initial training stages' limited information, we pre-train the recommendation model without data valuator for 1000 epochs to get meaningful embeddings. We set the number of the pre-training epochs to 1000. All experiments are conducted on GPU machines (NVIDIA GeForce RTX 3090).

\begin{algorithm}[H]
    \caption{The Proposed Method}
    \label{al:method}
    
    \textbf{Input:} Learning rates $\alpha$ and $\beta$, outer mini-batch size $B_1$, inner mini-batch size $B_2$, outer iteration count $T_1$, inner iteration count $T_2$, moving average window $W$, training pairs $\mathcal{D}_{1}=\{(u,i)\}_{k=1}^{L_1}$, validation pairs $\mathcal{D}_{2}=\{(u,i)\}_{k=1}^{L_2}$
    
    \textbf{Initialize:} parameters $\Theta$ and $\Lambda$, moving average $\delta=0$
    
    \begin{algorithmic}[1]
    \FOR{outer iteration $t_1=1,2,...,T_1$}
        \STATE Sample a mini-batch from the entire training dataset: $\mathcal{\hat{B}}=(u,i)_{k=1}^{B_1}\sim \mathcal{D}_1$
        \STATE Uniformly sample negative items $(j)_{k=1}^{B_1}$ for training pairs $(u,i)_{k=1}^{B_1}$ to get training data $\mathcal{B}=(u,i,j)_{k=1}^{B_1}$ for the recommendation model
        \FOR{$(u,i,j) \in \mathcal{B}$}
        \STATE Calculate the Shapley value by Eq. (\ref{eq:svcal}) and assign it to $w_{uij}$
        \ENDFOR
        \STATE Normalize the Shapley value $w_{uij}$ within the batch $\mathcal{B}$ as $\hat{w}_{uij}$ 
        \STATE Compute sample selection vector $s_{uij}=\text{Ber}(\hat{w}_{uij})$ from Bernoulli distribution
        \STATE Update the data valuator model by \ref{eq:mse}
        \FOR{inner iteration $t_2=1,2,...,T_2$}
        \STATE Sample a mini-batch $(u,i,j)_{m=1}^{B_2}\sim \mathcal{B}$
        \STATE Update the recommendation model:
    $$\Theta \leftarrow \Theta-\frac{\alpha}{B_2} \sum_{m=1}^{B_2} s_{uij} \cdot \nabla_\Theta \mathcal{L}_{\text{BPR}}(u,i,j;\Theta)$$
        \ENDFOR
    \STATE Update the data valuator:
    $$ \begin{array}{r}
    \Lambda \longleftarrow \Lambda - \beta [\mathcal{R}(\mathcal{D}_2, f(\Theta^*(\Lambda)))-\delta ]\\ \cdot \nabla_{\Lambda}log(\pi(\mathcal{B},(s_{uij})_{k=1}^{B_1};{\Lambda})
    \end{array}
    $$
    \STATE Update the moving average reward:
    $$
    \delta \leftarrow \frac{W-1}{W} \delta+\frac{1}{W} \mathcal{R}(\mathcal{D}_2, f(\Theta^*(\Lambda)))
    $$                               
    \ENDFOR
    \end{algorithmic}
\end{algorithm}








% This is an appendix.

\end{document}
