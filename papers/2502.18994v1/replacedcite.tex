\section{Related Work}
\label{related work}

\textbf{Long-term causal inference}
For years, researchers have investigated which short-term outcomes can reliably predict long-term causal effects.
Various criteria have been proposed for identifying valid surrogates, including the Prentice criteria ____, principal surrogacy ____, strong surrogate criteria ____, causal effect predictiveness ____, and consistent surrogate and its variations ____.
Recently, there has been growing interest in estimating long-term causal effects using surrogates, which is also the focus of this paper.
One line of work assumes the unconfoundedness assumption.
Under the unconfoundedness assumption, LTEE ____ and Laser ____ are based on specifically designed neural networks for long-term causal inference.
EETE ____ explores the data efficiency from the surrogate in several settings and proposes an efficient estimator for treatment effect.
ORL ____ introduces a doubly robust estimator for average treatment effects using only short-term experiments, additionally assuming stationarity conditions between short-term and long-term outcomes.
____ proposes a kernel ridge regression-based estimator for long-term effect under continuous treatment.
Additionally, ____ develop a policy learning method for balancing short-term and long-term rewards.
Our work is \textbf{different} from them. We do not assume the unconfoundedness assumption, and we use the data combination technique to solve the problem of unobserved confounders.
Another line of research, which avoids the unconfoundedness assumption, tackles the latent confounding problem by combining experimental and observational data.
This setting is first introduced by the method proposed by ____, 
which, under surrogacy assumption, constructs the so-called Surrogate Index (SInd) as the substitutions for long-term outcomes in the experimental data for effect identification.
As follow-up work, ____ introduces the latent unconfoundedness assumption, which assumes that short-term potential outcomes can mediate the long-term potential outcomes,  thereby enabling long-term causal effect identification.
Under this assumption, ____ propose several estimators for effect estimation.
The alternative feasible assumptions ____ are proposed to replace the latent unconfoundedness assumption, e.g., the Conditional Additive Equi-Confounding Bias (CAECB) assumption.
Based on proximal methods ____, ____ proposes considering the short-term outcomes as proxies of latent confounders, thereby achieving effect identification.
\textbf{However}, these studies primarily concentrate on average treatment effects, whereas our focus in this paper is on heterogeneous effects.
Among the existing literature, the most closely related work is ____'s work.
Our work can be viewed as a significant extension of theirs, as we consider a more practical scenario where short-term outcomes exhibit temporal dependencies, and theoretically, their CAECB assumption is a special case of the assumption we propose.


\textbf{Modeling Latent Confounding Bias}
An effective way to solve the latent confounding problem in the data combination setting is to model latent confounding bias.
____ proposes modeling confounding bias under a linearity assumption.
____ introduce to model the nonlinear confounding bias using the representation learning technique.
____ propose the integrative R-learner via a regularization for the conditional effects and confounding bias with the Neyman orthogonality.
____ propose a two-stage representation learning strategy to model such a confounding bias.
\textbf{Different} from these works, we focus on the long-term causal inference setting, and rather than focus on how to model the confounding bias, we concentrate more on the relationship between sequential confounding biases.