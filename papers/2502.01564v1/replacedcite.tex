\section{Related Work}
\subsection{Information Organization and Sense-making to Support Synchronous Group Meetings}

\textcolor{red}{more psychology-leaning items; e.g. what are general issues folks have in meetings (challenges people faced in synchronous group meetings; hard to make sense of chronological information, cognitive load for listening, thinking, and talking at the same time, the common ground theory, etc. PLUS some theories and techniques to help sense-making during synchronous meetings. e.g, facilitation method in CSCL/CSCW (computer supported collab work), designated turn-taking, collaborative note-taking, understanding check}

Participant engagement is crucial to the success of team meetings, as groups working together are able to develop original and feasible ideas more so than individuals ____. Yet synchronous meetings come with a set of challenges to participants including increased cognitive load and screen fatigue ____. There are several reasons for challenges like these to arise, including badly structured meetings and the way in which interactions between participants can feel unnatural through virtual meetings ____. \textcolor{red}{want to make the case that COMMUNICATION can ease cognitive load, and that it can be achieved through visualization of the conversation trajectory}
The adverse effects of obstacles such as cognitive overload can be diminished by an increase in communication ____, \textcolor{blue}{since participants do not have go out of their way to understand what the group's status is; in other words, the process of sensemaking and building mutual understanding is a continuous and speedy process in lieue of high-volume and clear communication between participants}. 
\textcolor{blue}{We derive the conclusion that teams with participants experiencing a lower cognitive load will be able to coordinate better}
Meeting success can be predicted by several factors, including focused discussion of meeting content ____ and the success of participants to build mutual understanding between one another ____. Building mutual understanding is a result of individual and group-level sensemaking, and sensemaking is a process that spans across a conversation on both high-level attributes such as general meeting topics as well as granular-level utterances. Encouraging sensemaking and building mutual understanding can be accomplished using supplementary meeting tools and software; specifically, visual cues have been shown to encourage engagement and improve team performance on group tasks ____, ____, ____ and visual information about the conversation at hand can reduce participants' cognitive load by creating ``external memory`` in a user's periphery ____, ____.

____ -- communication tactics in meetings that allow for participants to not be overloaded; team coordination + what tools they used predicts success
    - synchronous communication > async comm in terms of overload
    - discussion of the actual content at hand rather than team processes (e.g. logistics) was more of a predictor of success
    - more communication --> less cognitive overload 
____ -- case study analysis of knowledge construction in synchronous IM discussions (student groups)
    - social knowledge construction is mainly done using sharing + comparing (as opposed to e.g. negotiation or working through conflict)
    - off-topic discussions more likely in synchronous discussions than async
    - the highest-quality discussions exhibited more coordination
    - high coordination is related to joint decision making rather than individuals making their own ways
____ -- lays out issues w/ work meetings during the pandemics
    - common issues of being on video-conference calls all the time: information overload from staring at screen, communication thru virtual mtg is less natural and more tiring
    - badly structured meetings --> participants disengaged 
____ -- more of psychology theory on how meeting participants get along
    - mutual understanding is a sequential process that is built up from interactions between individuals; this is especially important for initial meetings between a team where the participants might not be as familiar with each other 
    - sensemaking can be a process of collecting info from across the conversation 
____ -- they analyze the effectiveness of a tool SAGE2 that allows users to share content on a canvas in parallel with the main presenter
    - advantages is that this is a source for 'external memory' which can reduce cognitive load
____ -- basically just that groups come up w more novel + feasible ideas than individuals
____ -- insights into collaborative knowledge sharing in meetings
    - this is very similar to MeetMap actually
    - teams that used the tool w/ visual cues are more likely to succeed in accomplishing the task
    - visual cues also encourage participants to engage more in the discussion + group work

\textcolor{red}{more on what techniques/design attributes folks use; talk about how to use visuals to help make sense of long linear information in a more digestible way (dialogue mapping)}

An array of previous work has supported the notion that visual aids in conversation can reduce cognitive load \textcolor{blue}{and increase group performanace in accomplishing tasks as a team.}
____ found that participants appreciated both the visualization of sequential information as well as the flexibility to cluster certain pieces of information together as necessary, and that participants as a whole appreciated the ability to visually highlight the most relevant portions of the conversation.
____ found that visualizating conversational content through clustered nodes improved memorability of the information \textcolor{blue}{which aligns with ____'s finding that information stored in some sort of visual format can reduce cognitive load}.

Both real-time analysis and feedback as well as a structured visual representation of the conversation frequently have a direct, positive impact on participants' behaviors without significantly increasing their cognitive loads ____, ____, ____, ____. Visualizing a conversation's content in a structured manner can also lead to more creativity in the discussion ____, as well as help with quick and efficient understanding of the team's status relative to project progress ____.  

____ -- tool that does concept-map based recommendation; demos the pro of visualization
    - one big takeaway: selective visuals --> eases cognitive load
____ -- tool that captures decision-making process in data scientists' meetings
    - big idea is that capturing the links between (a) code in computational notebooks (b) and utterances from discussion is helpful
    - the relationship visualization significantly helps w/ newcomer understanding of what's going on

\textcolor{red}{notation schema to support note-taking in meetings}
\textcolor{red}{what is the relative merits and downsides of having people take notes themsleves versus have a note-taker take notes for them? }
\textcolor{red}{from xu: i think there's research suggesting note-taking verbatimly takes away people's cognitive capacity in understanding, maybe you can cite that to.}

____ -- similar to above, it introduces the tool used (GroupMeter); found that real time *visual* feedback + linguistic analysis can lead to behavior changes
    - both dynamic (real-time) and historical views were useful to participants and affected their behavior in the meeting; both > no visuals
    - feedback from tool caused increase in reflection on language use
    - they find that visualization doesn't increase cognitive load
    - feedback also increased agreement, decreased disagreement when seeing analysis of their live dialogue 
____ -- tool that does live topic modeling and visuals, shows an increase in performance
    - currently a lack of CSTs (creativity support tools) that provide visuals
    - basically the tool does topic modeling (LDA) live and shows visuals to users
    - cool; visuals helped remind users to stay on topic + inspired different conversation directions (so encouraged creativity)
____ -- tool to do real time video stream summarization; can be used as an exmaple of AI for sensemaking 
    - again not really AI
    - tool that provides real-time summary of video content + user interaction
    - their aim is to aid viewers joining in middle of video; viewers note that what's important to them is (1) getting varying levels of detail for content up to join-point (2) catching up w minimal interruption to their attention to current stream 
    - the tool just segments the video and uses highlight content to 'summarize' those segments
    - timeline view is helpful 
    - open research direction towards how to make sense of info from before they joined stream
    - users using the tool were more engaged w the video

____ -- tool for whiteboarding ideas as individuals and as team in mtg
    - participants want to be able to highlight / visualize bits of information that are most important
    - evidence that a 'mixed' set of visuals (i.e. mix between sequential and more flexible info or ability to move around info) is good to cater to different individuals' preferences
    - clustering of information is a helpful visual
____ -- lit review on tools for visualizing group decision making
____ -- group information in nodes, visualization, helps memorability
____ -- i don't fully get it but gist is that interface will affect how ppl are able to collaborate / participate in mtg
____ -- visualizing disagreemtns leads to better alignment between participants
____ -- decision-making supplementary tool design approach for 'wicked' problems
____ -- case study w/ elementary school students, mind-mapping is shown to be an effective strategy in group brainstorm

\subsection{Using AI to Support Meeting Sensemaking}

AI has demonstrated utility in supporting sensemaking in meetings. AI tools that provide participants with real-time feedback in the conversation can directly impact the behavior of participants to better accomplish the task at hand ____. 

In recent years, LLMs have demonstrated adeptness at a wide variety of NLP tasks associated with dialogue and communication, from ____ to ____. 


____ -- real-time lanugage feedback to student group mtgs [[check how they're creating the feedback]] ** not AI 
____ -- visualize linguistic analysis of conversation for building teamwork; *how to balance visuals w/ not taking up participant attention by too much* --> they propose future directions of research, nothing super concrete
____ -- tool to mitigate issues in virtual meetings
    - ths is not AI, btw; basically the system is that participants in a meeting do a perspective-taking exercise where they (a) provide feedback to the system about what they feel (b) receive feedback from system about how good they are at guessing how others feel
    - the system improved team viability + team willingness to give and receive feedback
    - **this isn't hugely relevant, prob won't include this
\textcolor{red}{converting meeting content/general content into an easily digetible structure}
____ -- indirectly relevant; interface to convert long text responses from chatgpt into graph-like structure diagrams for easier sense-making
    - use GPT-4 to construct graph represnetations of dialogue using node-links
    - GPT-4 is instructed to segment and annotate entities as nodes and their relationships
    - good performance by GPT on the annotation task
    - participants find that node-link diagram helps fast + easy comprehension
    - again participants enjoy having multiple visual representations (i.e. node-links, original text, outline) 
____ -- tool that uses LLMs to synthesize scholarly texts in a readable, graph(?) way
    - pipeline: user highlights text, loopy belief propagation algo creates a graph out of the highlights, important papers from graph used as input to GPT which extracts highlights and summarizes + structures hierarchically; finally user is able to modify the final structure if they want
    - results --> tool provides high quality outlines for participants
        - again (sim to other tools reviewed) the visualization + tool results in higher participation + engagement from participants
    - note about the tradeoff between complete info + info overload; need a way to better trim down the info visualized, but generally participants found that it eased their cognitive load to focus on broader-picture items and relationships
____ -- framework for using LLMs to aid in 'object oriented interaction'
    - **this has some good notes to consider re: framework for using LLMs thru interface but idk if nec to cite directly 
____ -- lit review of visualizing conversation structure
    - just provides a review of systems, not really insightful
____ -- interface for working w/ LLMs to get info in a more natural, not nec linear way


\textcolor{red}{generally more AI help in meeting stuff}
____ -- adding a gpt generated video summary to lecture increased exam performance in students who used the summary
    - nothing too complex, literally just what is mentioned above
____ -- chatbot that provides structure + semi-supervises a discussion
    - related work discusses how arguments can only be reasonable/logical if they're based in conversation up until to that point and also if they're actually logically sound --> can make the point that it's crucial then for participants to be on the same page about what's transpired in the conversation
    - results: adding structure to the discussion improves quality 



____
____ -- prompt generation method; prompt == query for video / info in a video that will help that specific learner; they use knowledge graphs in pipeline
____ -- very similar to meetmap i think
    - discussion on the importance of situational awareness (a few levels that reflect participant understanding of what has happened and what will happen) --> paper explores how visuals can help w/ SA 
    - this is basically MeetMap, the visual is just a 'neighborhood' of regions of topics rather than graphical interface w nodes
    - takeaway is that users should work w/ system to develop nice visual 



____ -- lit review on cognitive load in meetings
    - ORGANIZED + STRUCTURED content improves note-taking + memorability
    - computer note-taking generally better, thought due to speed + ability to focus more attention on incoming info

____ -- study of collaborative note-taking and effects on cognitive load + group performance
    - main takeaway: collaborative note-taking pros of content understanding outweight cons of possible confusion, and doesn't overwhelm cognitive load

____