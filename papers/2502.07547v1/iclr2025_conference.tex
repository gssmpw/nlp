\documentclass{article} %
\usepackage{iclr2025_conference,times}
%%%%% NEW MATH DEFINITIONS %%%%%

\usepackage{amsmath,amsfonts,bm}
\usepackage{derivative}
% Mark sections of captions for referring to divisions of figures
\newcommand{\figleft}{{\em (Left)}}
\newcommand{\figcenter}{{\em (Center)}}
\newcommand{\figright}{{\em (Right)}}
\newcommand{\figtop}{{\em (Top)}}
\newcommand{\figbottom}{{\em (Bottom)}}
\newcommand{\captiona}{{\em (a)}}
\newcommand{\captionb}{{\em (b)}}
\newcommand{\captionc}{{\em (c)}}
\newcommand{\captiond}{{\em (d)}}

% Highlight a newly defined term
\newcommand{\newterm}[1]{{\bf #1}}

% Derivative d 
\newcommand{\deriv}{{\mathrm{d}}}

% Figure reference, lower-case.
\def\figref#1{figure~\ref{#1}}
% Figure reference, capital. For start of sentence
\def\Figref#1{Figure~\ref{#1}}
\def\twofigref#1#2{figures \ref{#1} and \ref{#2}}
\def\quadfigref#1#2#3#4{figures \ref{#1}, \ref{#2}, \ref{#3} and \ref{#4}}
% Section reference, lower-case.
\def\secref#1{section~\ref{#1}}
% Section reference, capital.
\def\Secref#1{Section~\ref{#1}}
% Reference to two sections.
\def\twosecrefs#1#2{sections \ref{#1} and \ref{#2}}
% Reference to three sections.
\def\secrefs#1#2#3{sections \ref{#1}, \ref{#2} and \ref{#3}}
% Reference to an equation, lower-case.
\def\eqref#1{equation~\ref{#1}}
% Reference to an equation, upper case
\def\Eqref#1{Equation~\ref{#1}}
% A raw reference to an equation---avoid using if possible
\def\plaineqref#1{\ref{#1}}
% Reference to a chapter, lower-case.
\def\chapref#1{chapter~\ref{#1}}
% Reference to an equation, upper case.
\def\Chapref#1{Chapter~\ref{#1}}
% Reference to a range of chapters
\def\rangechapref#1#2{chapters\ref{#1}--\ref{#2}}
% Reference to an algorithm, lower-case.
\def\algref#1{algorithm~\ref{#1}}
% Reference to an algorithm, upper case.
\def\Algref#1{Algorithm~\ref{#1}}
\def\twoalgref#1#2{algorithms \ref{#1} and \ref{#2}}
\def\Twoalgref#1#2{Algorithms \ref{#1} and \ref{#2}}
% Reference to a part, lower case
\def\partref#1{part~\ref{#1}}
% Reference to a part, upper case
\def\Partref#1{Part~\ref{#1}}
\def\twopartref#1#2{parts \ref{#1} and \ref{#2}}

\def\ceil#1{\lceil #1 \rceil}
\def\floor#1{\lfloor #1 \rfloor}
\def\1{\bm{1}}
\newcommand{\train}{\mathcal{D}}
\newcommand{\valid}{\mathcal{D_{\mathrm{valid}}}}
\newcommand{\test}{\mathcal{D_{\mathrm{test}}}}

\def\eps{{\epsilon}}


% Random variables
\def\reta{{\textnormal{$\eta$}}}
\def\ra{{\textnormal{a}}}
\def\rb{{\textnormal{b}}}
\def\rc{{\textnormal{c}}}
\def\rd{{\textnormal{d}}}
\def\re{{\textnormal{e}}}
\def\rf{{\textnormal{f}}}
\def\rg{{\textnormal{g}}}
\def\rh{{\textnormal{h}}}
\def\ri{{\textnormal{i}}}
\def\rj{{\textnormal{j}}}
\def\rk{{\textnormal{k}}}
\def\rl{{\textnormal{l}}}
% rm is already a command, just don't name any random variables m
\def\rn{{\textnormal{n}}}
\def\ro{{\textnormal{o}}}
\def\rp{{\textnormal{p}}}
\def\rq{{\textnormal{q}}}
\def\rr{{\textnormal{r}}}
\def\rs{{\textnormal{s}}}
\def\rt{{\textnormal{t}}}
\def\ru{{\textnormal{u}}}
\def\rv{{\textnormal{v}}}
\def\rw{{\textnormal{w}}}
\def\rx{{\textnormal{x}}}
\def\ry{{\textnormal{y}}}
\def\rz{{\textnormal{z}}}

% Random vectors
\def\rvepsilon{{\mathbf{\epsilon}}}
\def\rvphi{{\mathbf{\phi}}}
\def\rvtheta{{\mathbf{\theta}}}
\def\rva{{\mathbf{a}}}
\def\rvb{{\mathbf{b}}}
\def\rvc{{\mathbf{c}}}
\def\rvd{{\mathbf{d}}}
\def\rve{{\mathbf{e}}}
\def\rvf{{\mathbf{f}}}
\def\rvg{{\mathbf{g}}}
\def\rvh{{\mathbf{h}}}
\def\rvu{{\mathbf{i}}}
\def\rvj{{\mathbf{j}}}
\def\rvk{{\mathbf{k}}}
\def\rvl{{\mathbf{l}}}
\def\rvm{{\mathbf{m}}}
\def\rvn{{\mathbf{n}}}
\def\rvo{{\mathbf{o}}}
\def\rvp{{\mathbf{p}}}
\def\rvq{{\mathbf{q}}}
\def\rvr{{\mathbf{r}}}
\def\rvs{{\mathbf{s}}}
\def\rvt{{\mathbf{t}}}
\def\rvu{{\mathbf{u}}}
\def\rvv{{\mathbf{v}}}
\def\rvw{{\mathbf{w}}}
\def\rvx{{\mathbf{x}}}
\def\rvy{{\mathbf{y}}}
\def\rvz{{\mathbf{z}}}

% Elements of random vectors
\def\erva{{\textnormal{a}}}
\def\ervb{{\textnormal{b}}}
\def\ervc{{\textnormal{c}}}
\def\ervd{{\textnormal{d}}}
\def\erve{{\textnormal{e}}}
\def\ervf{{\textnormal{f}}}
\def\ervg{{\textnormal{g}}}
\def\ervh{{\textnormal{h}}}
\def\ervi{{\textnormal{i}}}
\def\ervj{{\textnormal{j}}}
\def\ervk{{\textnormal{k}}}
\def\ervl{{\textnormal{l}}}
\def\ervm{{\textnormal{m}}}
\def\ervn{{\textnormal{n}}}
\def\ervo{{\textnormal{o}}}
\def\ervp{{\textnormal{p}}}
\def\ervq{{\textnormal{q}}}
\def\ervr{{\textnormal{r}}}
\def\ervs{{\textnormal{s}}}
\def\ervt{{\textnormal{t}}}
\def\ervu{{\textnormal{u}}}
\def\ervv{{\textnormal{v}}}
\def\ervw{{\textnormal{w}}}
\def\ervx{{\textnormal{x}}}
\def\ervy{{\textnormal{y}}}
\def\ervz{{\textnormal{z}}}

% Random matrices
\def\rmA{{\mathbf{A}}}
\def\rmB{{\mathbf{B}}}
\def\rmC{{\mathbf{C}}}
\def\rmD{{\mathbf{D}}}
\def\rmE{{\mathbf{E}}}
\def\rmF{{\mathbf{F}}}
\def\rmG{{\mathbf{G}}}
\def\rmH{{\mathbf{H}}}
\def\rmI{{\mathbf{I}}}
\def\rmJ{{\mathbf{J}}}
\def\rmK{{\mathbf{K}}}
\def\rmL{{\mathbf{L}}}
\def\rmM{{\mathbf{M}}}
\def\rmN{{\mathbf{N}}}
\def\rmO{{\mathbf{O}}}
\def\rmP{{\mathbf{P}}}
\def\rmQ{{\mathbf{Q}}}
\def\rmR{{\mathbf{R}}}
\def\rmS{{\mathbf{S}}}
\def\rmT{{\mathbf{T}}}
\def\rmU{{\mathbf{U}}}
\def\rmV{{\mathbf{V}}}
\def\rmW{{\mathbf{W}}}
\def\rmX{{\mathbf{X}}}
\def\rmY{{\mathbf{Y}}}
\def\rmZ{{\mathbf{Z}}}

% Elements of random matrices
\def\ermA{{\textnormal{A}}}
\def\ermB{{\textnormal{B}}}
\def\ermC{{\textnormal{C}}}
\def\ermD{{\textnormal{D}}}
\def\ermE{{\textnormal{E}}}
\def\ermF{{\textnormal{F}}}
\def\ermG{{\textnormal{G}}}
\def\ermH{{\textnormal{H}}}
\def\ermI{{\textnormal{I}}}
\def\ermJ{{\textnormal{J}}}
\def\ermK{{\textnormal{K}}}
\def\ermL{{\textnormal{L}}}
\def\ermM{{\textnormal{M}}}
\def\ermN{{\textnormal{N}}}
\def\ermO{{\textnormal{O}}}
\def\ermP{{\textnormal{P}}}
\def\ermQ{{\textnormal{Q}}}
\def\ermR{{\textnormal{R}}}
\def\ermS{{\textnormal{S}}}
\def\ermT{{\textnormal{T}}}
\def\ermU{{\textnormal{U}}}
\def\ermV{{\textnormal{V}}}
\def\ermW{{\textnormal{W}}}
\def\ermX{{\textnormal{X}}}
\def\ermY{{\textnormal{Y}}}
\def\ermZ{{\textnormal{Z}}}

% Vectors
\def\vzero{{\bm{0}}}
\def\vone{{\bm{1}}}
\def\vmu{{\bm{\mu}}}
\def\vtheta{{\bm{\theta}}}
\def\vphi{{\bm{\phi}}}
\def\va{{\bm{a}}}
\def\vb{{\bm{b}}}
\def\vc{{\bm{c}}}
\def\vd{{\bm{d}}}
\def\ve{{\bm{e}}}
\def\vf{{\bm{f}}}
\def\vg{{\bm{g}}}
\def\vh{{\bm{h}}}
\def\vi{{\bm{i}}}
\def\vj{{\bm{j}}}
\def\vk{{\bm{k}}}
\def\vl{{\bm{l}}}
\def\vm{{\bm{m}}}
\def\vn{{\bm{n}}}
\def\vo{{\bm{o}}}
\def\vp{{\bm{p}}}
\def\vq{{\bm{q}}}
\def\vr{{\bm{r}}}
\def\vs{{\bm{s}}}
\def\vt{{\bm{t}}}
\def\vu{{\bm{u}}}
\def\vv{{\bm{v}}}
\def\vw{{\bm{w}}}
\def\vx{{\bm{x}}}
\def\vy{{\bm{y}}}
\def\vz{{\bm{z}}}

% Elements of vectors
\def\evalpha{{\alpha}}
\def\evbeta{{\beta}}
\def\evepsilon{{\epsilon}}
\def\evlambda{{\lambda}}
\def\evomega{{\omega}}
\def\evmu{{\mu}}
\def\evpsi{{\psi}}
\def\evsigma{{\sigma}}
\def\evtheta{{\theta}}
\def\eva{{a}}
\def\evb{{b}}
\def\evc{{c}}
\def\evd{{d}}
\def\eve{{e}}
\def\evf{{f}}
\def\evg{{g}}
\def\evh{{h}}
\def\evi{{i}}
\def\evj{{j}}
\def\evk{{k}}
\def\evl{{l}}
\def\evm{{m}}
\def\evn{{n}}
\def\evo{{o}}
\def\evp{{p}}
\def\evq{{q}}
\def\evr{{r}}
\def\evs{{s}}
\def\evt{{t}}
\def\evu{{u}}
\def\evv{{v}}
\def\evw{{w}}
\def\evx{{x}}
\def\evy{{y}}
\def\evz{{z}}

% Matrix
\def\mA{{\bm{A}}}
\def\mB{{\bm{B}}}
\def\mC{{\bm{C}}}
\def\mD{{\bm{D}}}
\def\mE{{\bm{E}}}
\def\mF{{\bm{F}}}
\def\mG{{\bm{G}}}
\def\mH{{\bm{H}}}
\def\mI{{\bm{I}}}
\def\mJ{{\bm{J}}}
\def\mK{{\bm{K}}}
\def\mL{{\bm{L}}}
\def\mM{{\bm{M}}}
\def\mN{{\bm{N}}}
\def\mO{{\bm{O}}}
\def\mP{{\bm{P}}}
\def\mQ{{\bm{Q}}}
\def\mR{{\bm{R}}}
\def\mS{{\bm{S}}}
\def\mT{{\bm{T}}}
\def\mU{{\bm{U}}}
\def\mV{{\bm{V}}}
\def\mW{{\bm{W}}}
\def\mX{{\bm{X}}}
\def\mY{{\bm{Y}}}
\def\mZ{{\bm{Z}}}
\def\mBeta{{\bm{\beta}}}
\def\mPhi{{\bm{\Phi}}}
\def\mLambda{{\bm{\Lambda}}}
\def\mSigma{{\bm{\Sigma}}}

% Tensor
\DeclareMathAlphabet{\mathsfit}{\encodingdefault}{\sfdefault}{m}{sl}
\SetMathAlphabet{\mathsfit}{bold}{\encodingdefault}{\sfdefault}{bx}{n}
\newcommand{\tens}[1]{\bm{\mathsfit{#1}}}
\def\tA{{\tens{A}}}
\def\tB{{\tens{B}}}
\def\tC{{\tens{C}}}
\def\tD{{\tens{D}}}
\def\tE{{\tens{E}}}
\def\tF{{\tens{F}}}
\def\tG{{\tens{G}}}
\def\tH{{\tens{H}}}
\def\tI{{\tens{I}}}
\def\tJ{{\tens{J}}}
\def\tK{{\tens{K}}}
\def\tL{{\tens{L}}}
\def\tM{{\tens{M}}}
\def\tN{{\tens{N}}}
\def\tO{{\tens{O}}}
\def\tP{{\tens{P}}}
\def\tQ{{\tens{Q}}}
\def\tR{{\tens{R}}}
\def\tS{{\tens{S}}}
\def\tT{{\tens{T}}}
\def\tU{{\tens{U}}}
\def\tV{{\tens{V}}}
\def\tW{{\tens{W}}}
\def\tX{{\tens{X}}}
\def\tY{{\tens{Y}}}
\def\tZ{{\tens{Z}}}


% Graph
\def\gA{{\mathcal{A}}}
\def\gB{{\mathcal{B}}}
\def\gC{{\mathcal{C}}}
\def\gD{{\mathcal{D}}}
\def\gE{{\mathcal{E}}}
\def\gF{{\mathcal{F}}}
\def\gG{{\mathcal{G}}}
\def\gH{{\mathcal{H}}}
\def\gI{{\mathcal{I}}}
\def\gJ{{\mathcal{J}}}
\def\gK{{\mathcal{K}}}
\def\gL{{\mathcal{L}}}
\def\gM{{\mathcal{M}}}
\def\gN{{\mathcal{N}}}
\def\gO{{\mathcal{O}}}
\def\gP{{\mathcal{P}}}
\def\gQ{{\mathcal{Q}}}
\def\gR{{\mathcal{R}}}
\def\gS{{\mathcal{S}}}
\def\gT{{\mathcal{T}}}
\def\gU{{\mathcal{U}}}
\def\gV{{\mathcal{V}}}
\def\gW{{\mathcal{W}}}
\def\gX{{\mathcal{X}}}
\def\gY{{\mathcal{Y}}}
\def\gZ{{\mathcal{Z}}}

% Sets
\def\sA{{\mathbb{A}}}
\def\sB{{\mathbb{B}}}
\def\sC{{\mathbb{C}}}
\def\sD{{\mathbb{D}}}
% Don't use a set called E, because this would be the same as our symbol
% for expectation.
\def\sF{{\mathbb{F}}}
\def\sG{{\mathbb{G}}}
\def\sH{{\mathbb{H}}}
\def\sI{{\mathbb{I}}}
\def\sJ{{\mathbb{J}}}
\def\sK{{\mathbb{K}}}
\def\sL{{\mathbb{L}}}
\def\sM{{\mathbb{M}}}
\def\sN{{\mathbb{N}}}
\def\sO{{\mathbb{O}}}
\def\sP{{\mathbb{P}}}
\def\sQ{{\mathbb{Q}}}
\def\sR{{\mathbb{R}}}
\def\sS{{\mathbb{S}}}
\def\sT{{\mathbb{T}}}
\def\sU{{\mathbb{U}}}
\def\sV{{\mathbb{V}}}
\def\sW{{\mathbb{W}}}
\def\sX{{\mathbb{X}}}
\def\sY{{\mathbb{Y}}}
\def\sZ{{\mathbb{Z}}}

% Entries of a matrix
\def\emLambda{{\Lambda}}
\def\emA{{A}}
\def\emB{{B}}
\def\emC{{C}}
\def\emD{{D}}
\def\emE{{E}}
\def\emF{{F}}
\def\emG{{G}}
\def\emH{{H}}
\def\emI{{I}}
\def\emJ{{J}}
\def\emK{{K}}
\def\emL{{L}}
\def\emM{{M}}
\def\emN{{N}}
\def\emO{{O}}
\def\emP{{P}}
\def\emQ{{Q}}
\def\emR{{R}}
\def\emS{{S}}
\def\emT{{T}}
\def\emU{{U}}
\def\emV{{V}}
\def\emW{{W}}
\def\emX{{X}}
\def\emY{{Y}}
\def\emZ{{Z}}
\def\emSigma{{\Sigma}}

% entries of a tensor
% Same font as tensor, without \bm wrapper
\newcommand{\etens}[1]{\mathsfit{#1}}
\def\etLambda{{\etens{\Lambda}}}
\def\etA{{\etens{A}}}
\def\etB{{\etens{B}}}
\def\etC{{\etens{C}}}
\def\etD{{\etens{D}}}
\def\etE{{\etens{E}}}
\def\etF{{\etens{F}}}
\def\etG{{\etens{G}}}
\def\etH{{\etens{H}}}
\def\etI{{\etens{I}}}
\def\etJ{{\etens{J}}}
\def\etK{{\etens{K}}}
\def\etL{{\etens{L}}}
\def\etM{{\etens{M}}}
\def\etN{{\etens{N}}}
\def\etO{{\etens{O}}}
\def\etP{{\etens{P}}}
\def\etQ{{\etens{Q}}}
\def\etR{{\etens{R}}}
\def\etS{{\etens{S}}}
\def\etT{{\etens{T}}}
\def\etU{{\etens{U}}}
\def\etV{{\etens{V}}}
\def\etW{{\etens{W}}}
\def\etX{{\etens{X}}}
\def\etY{{\etens{Y}}}
\def\etZ{{\etens{Z}}}

% The true underlying data generating distribution
\newcommand{\pdata}{p_{\rm{data}}}
\newcommand{\ptarget}{p_{\rm{target}}}
\newcommand{\pprior}{p_{\rm{prior}}}
\newcommand{\pbase}{p_{\rm{base}}}
\newcommand{\pref}{p_{\rm{ref}}}

% The empirical distribution defined by the training set
\newcommand{\ptrain}{\hat{p}_{\rm{data}}}
\newcommand{\Ptrain}{\hat{P}_{\rm{data}}}
% The model distribution
\newcommand{\pmodel}{p_{\rm{model}}}
\newcommand{\Pmodel}{P_{\rm{model}}}
\newcommand{\ptildemodel}{\tilde{p}_{\rm{model}}}
% Stochastic autoencoder distributions
\newcommand{\pencode}{p_{\rm{encoder}}}
\newcommand{\pdecode}{p_{\rm{decoder}}}
\newcommand{\precons}{p_{\rm{reconstruct}}}

\newcommand{\laplace}{\mathrm{Laplace}} % Laplace distribution

\newcommand{\E}{\mathbb{E}}
\newcommand{\Ls}{\mathcal{L}}
\newcommand{\R}{\mathbb{R}}
\newcommand{\emp}{\tilde{p}}
\newcommand{\lr}{\alpha}
\newcommand{\reg}{\lambda}
\newcommand{\rect}{\mathrm{rectifier}}
\newcommand{\softmax}{\mathrm{softmax}}
\newcommand{\sigmoid}{\sigma}
\newcommand{\softplus}{\zeta}
\newcommand{\KL}{D_{\mathrm{KL}}}
\newcommand{\Var}{\mathrm{Var}}
\newcommand{\standarderror}{\mathrm{SE}}
\newcommand{\Cov}{\mathrm{Cov}}
% Wolfram Mathworld says $L^2$ is for function spaces and $\ell^2$ is for vectors
% But then they seem to use $L^2$ for vectors throughout the site, and so does
% wikipedia.
\newcommand{\normlzero}{L^0}
\newcommand{\normlone}{L^1}
\newcommand{\normltwo}{L^2}
\newcommand{\normlp}{L^p}
\newcommand{\normmax}{L^\infty}

\newcommand{\parents}{Pa} % See usage in notation.tex. Chosen to match Daphne's book.

\DeclareMathOperator*{\argmax}{arg\,max}
\DeclareMathOperator*{\argmin}{arg\,min}

\DeclareMathOperator{\sign}{sign}
\DeclareMathOperator{\Tr}{Tr}
\let\ab\allowbreak


\usepackage[utf8]{inputenc} % allow utf-8 input
\usepackage[T1]{fontenc}    % use 8-bit T1 fonts    
\usepackage{url}            % simple URL typesetting
\usepackage{booktabs}       % professional-quality tables
\usepackage{amsfonts}       % blackboard math symbols
\usepackage{nicefrac}       % compact symbols for 1/2, etc.
\usepackage{microtype}      % microtypography
\usepackage{xcolor}         % colors
\usepackage{amssymb}
\usepackage{amsmath}
\usepackage{algorithm}
\usepackage{algorithmic}
\usepackage{wrapfig}
\usepackage{graphicx}     
\usepackage{times}
\usepackage{epsfig}
\usepackage{multirow}
\usepackage{caption}
\usepackage{stfloats}
\usepackage{enumitem}
\usepackage{lipsum}
\usepackage{bbm}
\usepackage{wrapfig}\definecolor{mydarkblue}{rgb}{0.082,0.376,0.510}
\usepackage{wrapfig}\definecolor{darkred}{rgb}{0.6,0,0}
\usepackage[colorlinks,
            linkcolor = darkred,
            urlcolor  = darkred, 
            citecolor= mydarkblue,
            ]{hyperref}
\usepackage[labelformat=parens]{subcaption}
\definecolor{changecolor}{RGB}{0, 0, 0}

\title{Instance-dependent Early Stopping}


 \iclrfinalcopy

\author{Suqin Yuan\textsuperscript{1} \quad Runqi Lin\textsuperscript{1} \quad Lei Feng\textsuperscript{2}\footnotemark[1] \quad Bo Han\textsuperscript{3} \quad  Tongliang Liu\textsuperscript{1}\thanks{Corresponding authors.} \\
\textsuperscript{1} Sydney AI Centre, The University of Sydney\\
\textsuperscript{2} Southeast University 
\textsuperscript{3} Hong Kong Baptist University
}


\newcommand{\fix}{\marginpar{FIX}}
\newcommand{\new}{\marginpar{NEW}}


\begin{document}


\maketitle

\begin{abstract}
In machine learning practice, early stopping has been widely used to regularize models and can save computational costs by halting the training process when the model's performance on a validation set stops improving. However, conventional early stopping applies the same stopping criterion to all instances without considering their individual learning statuses, which leads to redundant computations on instances that are already well-learned. To further improve the efficiency, we propose an Instance-dependent Early Stopping (IES) method that adapts the early stopping mechanism from the entire training set to the instance level, based on the core principle that \emph{once the model has mastered an instance, the training on it should stop}. 
IES considers an instance as \emph{mastered} if the second-order differences of its loss value remain within a small range around zero.
This offers a more consistent measure of an instance's learning status compared with directly using the loss value, and thus allows for a unified threshold to determine when an instance can be excluded from further backpropagation. We show that excluding \emph{mastered} instances from backpropagation can increase the gradient norms, thereby accelerating the decrease of the training loss and speeding up the training process. 
Extensive experiments on benchmarks demonstrate that IES method can reduce backpropagation instances by 10\%-50\% while maintaining or even slightly improving the test accuracy and transfer learning performance of a model. 
Our implementation can be found at \url{https://github.com/tmllab/2025_ICLR_IES}.
\end{abstract}

\section{Introduction}
Early stopping is a straightforward technique that regulates model training and reduces computational costs by halting the training process when no further improvements are observed in model performance on the validation set \citep{prechelt2002early,raskutti2014early,caruana2000overfitting,yuan2024early}. Specifically, this method terminates training at the appropriate moment, preventing excessive training while conserving computational resources \citep{zhang2021understanding, belkin2019reconciling, nakkiran2021deep} and reduces the reliance on other computationally intensive regularization methods in model training \citep{tibshirani1996regression, hoerl1970ridge, goodfellow2016deep}. The growing size and complexity of models and datasets make these benefits increasingly critical, as they lead to significantly rising computational costs associated with training advanced models \citep{hestness2017deep, kaplan2020scaling, brown2020language, sorscher2022beyond, li2023loftq, gong2024cascast}. In practice, ending training when satisfactory performance is achieved is more practical than pursuing complete convergence, as the cost of complete convergence is excessively high and may not yield evident improvements in performance \citep{rice2020overfitting, yang2020rethinking, sagawa2020investigation}.

Despite the widespread acclaim for the elegance and practicality of the conventional early stopping method, which focuses on the model's performance on the validation set and simultaneously terminates the optimization across the entire training set, this approach lacks flexibility. It does not consider that the model learns different instances at varying rates and stages \citep{zhang2021understanding, arpit2017closer, toneva2018empirical, wen2022benign}. Consequently, this can lead to redundant computations, as the model may continue processing instances that are already well-learned until it finally achieves satisfactory performance across the entire dataset. To further enhance the efficiency of early stopping, we propose the \emph{Instance-dependent Early Stopping} (IES) method, which refines the idea of early stopping from the entire training dataset to the instance level.



\begin{figure}[t]
\vskip -0.1in
\begin{minipage}{0.49\textwidth}
    \centering
    \includegraphics[width=6.4cm]{IESfig1.pdf} 
\end{minipage}
\begin{minipage}{0.49\textwidth} 
    \vskip +0.05in
    \centering
    \caption{Effectiveness of \emph{Instance-dependent Early Stopping} (IES) on ImageNet-1k and CIFAR-10 datasets. Top row: Test accuracy over the course of training, showing that IES (Ours) achieves comparable accuracy to the baseline (No Removal) despite training on fewer samples. Bottom row: Number of training samples excluded from backpropagation by IES over the course of training. As the model masters more and more samples during the training process, IES allows an increasing number of these \emph{mastered} samples to be excluded from backpropagation, significantly reducing computation while still maintaining the same performance as the baseline method.}
    \label{fig1}
\end{minipage}
\vskip -0.35in
\end{figure}


The principle of our IES method is simple yet effective: \emph{once the model masters an instance, the training on it should stop}. By enabling the model to dynamically stop the training for individual instances once satisfactory performance is achieved for those specific instances, IES can perform early stopping in a more fine-grained manner. To instantiate the concept of \emph{mastered}, we need a computational efficiency quantitative criterion that can be applied uniformly across all instances. A natural idea is to use the loss value of instances, which has been shown effective for identifying important instances for optimization \citep{loshchilov2015online, jiang2019accelerating, qin2023infobatch}. However, due to the differences in optimal loss values across instances arising from factors such as sample complexity \citep{hacohen2019power, wang2020optimizing}, inherent ambiguity \citep{guo2017calibration, liang2017enhancing}, noise \citep{zhang2021understanding, jiang2018mentornet}, and imbalance \citep{cui2019class, cao2019learning}, it may be suboptimal for determining whether an instance has been \emph{mastered}.

In this paper, we propose to use the second-order difference of an instance's loss values \( \Delta^2 L_i(w^{(t)}) \) across consecutive training epochs as the \emph{mastered} criterion. If, over \(k\) epochs, the sum of the absolute values of \( \Delta^2 L_i(w^{(t)}) \) for an instance \(i\) is confined to a small neighborhood around \(0\), it signifies that the change in the loss tends to be flat and insensitive to parameter updates. 
Compared with the loss values, the second-order differences of these values for the training data have a lower coefficient of variation in later training stages (Figure \ref{fignew}). This indicates that the second-order difference loss values of \emph{mastered} instances consistently fall within a small range, regardless of the actual loss values of these instances.
This consistency allows us to set a unified threshold based on the second-order difference values to determine whether an instance has been \emph{mastered}. 
Moreover, the proposed criterion is computationally efficient, relying solely on forward propagation.

As shown in Figure \ref{fig1}, as model training progresses and more instances are \emph{mastered}, the IES method allows an adaptive decrease in the number of training instances from backpropagation. This results in significant savings in overall training time and computational costs while obtaining models with comparable performance to the one that is trained using all data. Specifically, the effectiveness of the IES method in accelerating model training progression can be attributed to its ability to allow the model to focus on instances that are not yet \emph{mastered}, which typically have larger gradient norms, thereby speeding up the reduction of the training loss through more effective parameter updates. By effectively identifying and skipping the redundant instances that have already been well-learned and would not significantly contribute to further model performance improvement in the next few epochs, IES achieves comparable results to full-data training. Moreover, by avoiding repeated training on already \emph{mastered} instances, the IES method avoids \emph{Over-Memorization} \citep{ishida2020we, zhang2021understanding, wen2024sharpness, linover, lin2024layer, lin2024eliminating} and enables the model to more rapidly reduce the sharpness of the loss landscape \citep{dauphin2014identifying, foret2020sharpness}.

To assess the effectiveness of the IES method, we carried out extensive experiments across various settings. Our findings reveal that IES consistently delivers substantial computational savings in CIFAR and ImageNet-1k tasks, reducing the number of instances that require backpropagation by 10\% to 50\% without sacrificing model performance. In many cases, IES even slightly enhances the model's generalization performance and improves transferability to downstream tasks. Specifically, fine-tuning models pretrained with IES on ImageNet-1k for the CIFAR and Caltech-101 datasets results in average improvements of 1.5\%, compared with models pretrained without IES. Through ablation studies and comparative analysis, we demonstrate that IES outperforms existing samples selection methods and demonstrates robust adaptiveness in hyperparameter selection.


Our main contributions can be summarized as follows:
\begin{itemize}[leftmargin=0.4cm,topsep=-2pt]
\item [1.]
We propose \emph{Instance-dependent Early Stopping} (IES), a method that adaptively stops training at the instance level, allowing for the saving of computational resources while maintaining performance.
\item [2.]
We introduce a \emph{mastered} criterion based on the second-order differences of sample loss values, providing an unified measure to determine whether a model has fully learned a given instance.
\item [3.]
We analyze the mechanism behind IES's effectiveness, revealing that it allows the model to focus on instances with larger gradient norms and reduces the sharpness of loss landscape more rapidly.
\end{itemize}

\section{Related Work}
IES is closely related to multiple active machine learning research areas. We review key studies in these fields, underscoring IES's distinct features. 

\emph{Sample Selection} has been widely used to improve the efficiency and robustness of deep learning model training. The main idea is to assign higher probabilities to examples to be trained that are \emph{informative} \citep{alain2015variance, katharopoulos2017biased, katharopoulos2018not}, \emph{unique} \citep{loshchilov2015online, chang2017active, shi2021diversity} or \emph{confident} \citep{khim2020uniform}. Related associated distillation and selection algorithms usually incur additional costs. Static selection typically requires preliminary calculations before training or in the early stages of training, with related studies including \emph{Data Pruning} \citep{toneva2018empirical, paul2021deep, killamsetty2021glister} and \emph{Core Set} \citep{huggins2016coresets, huang2018epsilon, braverman2022power, xia2022moderate, xia2024refined}, etc., with the goal of finding a small subset from all training data that can represent the entire dataset. 
Dynamic selection usually involves selecting instances across training process, with related studies including \emph{Dynamic Data Pruning} \citep{raju2021accelerating, mindermann2022prioritized, he2023large, truong2023kakurenbo, qin2023infobatch} and \emph{Importance Sampling} \citep{alain2015variance, katharopoulos2017biased, katharopoulos2018not, JMLR:v19:16-241, jiang2019accelerating}, etc., aimed at focusing training on more informative or confident examples. 
In the context of deep learning, several methods have been proposed based on different measures of sample ``informative'', such as gradient norm \citep{alain2015variance, killamsetty2021grad}, loss value \citep{loshchilov2015online, schaul2015prioritized, mindermann2022prioritized}, and prediction uncertainty \citep{chang2017active}. 
Notably, when the gradient of an instance converges to zero, it means that the model's parameters will \textcolor{changecolor}{be} insignificant updated based on this particular sample. However, even with efficient gradient computation methods \citep{wei2017minimal, katharopoulos2017biased, katharopoulos2018not}, the computational cost of calculating the gradient of each sample based on backpropagation is still high, which hinders the goal of reducing the computational cost of every single run. \emph{Curriculum Learning}
 \citep{bengio2009curriculum, wu2021curricula, zhou2020curriculum, wang2024efficienttrain++, wang2024computation, kumar2010self} is a learning paradigm that aims to improve the efficiency and effectiveness of training by presenting examples in a meaningful order, typically from easy to hard. Several methods have been proposed based on different measures of example difficulty \citep{weinshall2018curriculum, saxena2019data, jiang2018mentornet}. IES method can be viewed as a curriculum learning method design for end of training, focusing on the model's mastery of instances.

Although IES and existing sample selection techniques share the common goal of improving training progression via training on a selected subset of training instances, our method distinguishes itself through its focus on whether ``the model has already fully learned an instance'', i.e., \emph{mastered}. This unique perspective allows IES to adaptively adjust the proportion of instances participating in training at different stages, thereby eliminating the need for pre-set training schedules or removal rates.


\section{Methodology}

To refine the advantages of early stopping to the instance level, we proposed a simple principle that, \emph{once the model masters an instance, the training on it should stop.} 
To operationalize this idea, we introduce a criterion for identifying instances that the model has been \emph{mastered}, as detailed in Section~\ref{sec3.1}. Building on this foundation, we propose \emph{Instance-dependent Early Stopping} (IES) to promote model training progression, as shown in Section~\ref{sec3.2}. Furthermore, we demonstrate the efficiency and effectiveness of the IES method, as discussed in Section~\ref{sec3.3}. All toy experiments presented in this section use a standard ResNet-18 backbone trained on the CIFAR-10 dataset. For detailed experiment settings, please refer to Section~\ref{sec4} and Appendix \ref{appa} and \ref{appb6}.

\textbf{Preliminaries.}
- The \emph{Hessian matrix}, $H = \nabla^2 L(w)$, characterizes the loss function's curvature by its eigenvalues and eigenvectors: $H = Q\Lambda Q^T = \sum_{i=1}^{n} \lambda_i q_i q_i^T$. $Q$ is an orthogonal matrix of eigenvectors, and $\Lambda$ is a diagonal matrix of eigenvalues $\lambda_i$, describing the curvature in various directions. Higher eigenvalues imply steeper curvatures, complicating optimization \citep{li2017stochastic}.

- $\nabla$ represents the gradient operator, for example, $\nabla L(w)$ represents the gradient of the loss function $L$ at the parameter $w$.
$\Delta$ represents the difference operator, $\Delta^2 L_i(w^{(t)})$ represents the second-order difference of the loss function for sample $i$ over three consecutive time steps $t$, $t-1$, and $t-2$.

\begin{figure*}[b]
 \vskip -0.25in
 \begin{subfigure}[b]{0.31\textwidth}
     \includegraphics[width=\textwidth]{N0.png}
     \vskip -0.1in
     \caption{\emph{$N = 0$}}
 \end{subfigure}
 \hfill
 \begin{subfigure}[b]{0.31\textwidth}
     \includegraphics[width=\textwidth]{N1.png}
     \vskip -0.1in
     \caption{\emph{$N = 1$}}
 \end{subfigure}
 \hfill
 \begin{subfigure}[b]{0.31\textwidth}
     \includegraphics[width=\textwidth]{N2.png}
     \vskip -0.1in
     \caption{\emph{$N = 2$}}
 \end{subfigure}
 \vskip -0.05in
\caption{The curves show the number of instances that meet the corresponding mastered criteria (N = \{0, 1, 2\}, $\delta = 1e^{-4}$) as the training epochs progress, under two scenarios: excluding the mastered instances from backpropagation and allowing the mastered instances to participate in backpropagation. The proximity of the curves suggests that the model can maintain its ``mastered'' on the mastered instances without the need for actively repeated training on them.}
 \label{figure2}
\end{figure*}

\subsection{The Mastered Criterion}
\label{sec3.1}
Previous studies have shown that different instances contain varying information and have inconsistent impacts on model learning at different training stages \citep{zhang2021understanding, arpit2017closer, toneva2018empirical, huang2022harnessing, huang2024winning, litowards, hong2024improving}. This suggests that if certain instances have been well-learned by the model early in the training process, their contribution to model performance improvement may diminish or even become redundant as training progresses. To apply the idea of early stopping at the instance level and improve training efficiency, we need a simple and computationally efficient method to assess the model's learning status on each sample and identify these redundant instances that would not significantly contribute to further model performance improvement in the next few epochs, which we refer to as \emph{mastered} instances. 

To efficiently identify which and when an instance is \emph{mastered}, we construct a criterion based on the $N$-th order difference of sample loss, which only relies on forward propagation. 
Intuitively, the loss of a \emph{mastered} instance should be relatively stable. Specifically, when an instance $i$ is well fitted by the current model parameters $w^{(t)}$ or is insensitive to their recent update, the associated loss $L_i(w^{(t)})$ will be small or reach a plateau, and thus the $N$-th order difference of the loss will approach zero.
To formalize this, when the $N$-th order difference of the loss for sample $i$ falls beneath a specified small positive threshold $\delta$, sample $i$ is considered to be \emph{mastered} by the model parameters $w$ and state $t$, which can be expressed as:
{\setlength{\abovedisplayskip}{-5pt}
\setlength{\belowdisplayskip}{+1pt}
\begin{equation}
\Delta^N L_i(w^{(t)}) < \delta, N = \{0, 1, 2, ...\}.
\end{equation}}

To demonstrate the effectiveness of the \emph{mastered} criteria, we experimentally tracked the number of instances that meet the \emph{mastered} criteria during the training process. As shown in Figure \ref{figure2}, the \emph{mastered} criteria enable adaptive sample selection throughout the learning process, allowing the model to dynamically adjust the size of the training set participating in backpropagation according to the evolving requirements. During the initial stages of training, the model has scarcely learned any instances, so the \emph{mastered} criteria retain most instances for backpropagation, as almost every sample can provide useful information. As training progresses, the model gradually masters more instances, leading to an adaptive decrease in the number of retained training instances commensurate with the training progress. The \emph{mastered} criteria adaptively remove these fully learned redundant instances from backpropagation, enabling the model to focus on the remaining samples. Compared to methods that dynamically sample a fixed proportion of important instances \citep{raju2021accelerating, mindermann2022prioritized, he2023large, qin2023infobatch}, stopping training on \emph{mastered} instances, provides a more adaptive and efficient approach to instance selection based on the model's learning progress.

It is noteworthy that the number of the model \emph{mastered} instances remains nearly the same regardless of whether the instances satisfying the \emph{mastered} criteria continue to participate in backpropagation or not, as shown in Figure \ref{figure2}. This observation, which is particularly evident for $N=1$ and $N=2$, suggests that the model can maintain its learned state on the \emph{mastered} instances even without repeatedly training on them. The \emph{mastered} criterion thus provides an effective way to identify redundant instances during training, allowing the model to exclude them from backpropagation with minimal impact on model performance on these instances.





\subsection{Instance-dependent Early Stopping (IES)}
\label{sec3.2}

Building upon the \emph{mastered} criterion, we propose the \emph{Instance-dependent Early Stopping} (IES) method, allowing the model to \emph{stop training on an instance once it has been mastered}.
Although using the loss value of instances (i.e., $N=0$) has been widely adopted as a method to identify important instances for current optimization \citep{loshchilov2015online, jiang2019accelerating}, different instances may have different optimal loss values $L_i(w^*)$ due to factors such as sample complexity \citep{hacohen2019power, wang2020optimizing}, noise \citep{zhang2021understanding, jiang2018mentornet}, and imbalance \citep{cui2019class, cao2019learning}. This poses a challenge in simply using loss value to construct mastered criterion for IES. If the mastered criterion were to directly depend on the absolute loss value, it would require setting different thresholds for each sample, which can be impractical and  expensive in large-scale datasets.
In this work, we use the second-order difference to identify the mastered instances, which quantifies the rate of change in the loss for sample \(i\) across three consecutive epochs, \(t^{th}\), \((t-1)^{th}\), and \((t-2)^{th}\) training epochs. The second-order difference is defined as:
{\setlength{\abovedisplayskip}{-0pt}
\setlength{\belowdisplayskip}{-0pt}
\begin{equation}
\textcolor{changecolor}{
\begin{aligned}
\Delta^2 L_i(w^{(t)}) &= [L_i(w^{(t)}) - L_i(w^{(t-1)})] - [L_i(w^{(t-1)}) - L_i(w^{(t-2)})] \\
&= L_i(w^{(t)}) - 2L_i(w^{(t-1)}) + L_i(w^{(t-2)}).
\end{aligned}
}
\end{equation}}

\begin{wrapfigure}{r}{0.33\textwidth}
\vskip -0.12in
  \centering
  \includegraphics[width=0.32\textwidth]{Unknown-77-rebuttal_iclr.png}
  \vskip -0.12in
  \caption{\textcolor{changecolor}{Coefficient of variation (CV) of different orders of loss differences during training.}}
\vskip -0.15in
\label{fignew}
\end{wrapfigure}
By quantifying the rate of change in the loss for each instance around the current parameters $w^{(t)}$, the second-order difference effectively captures the stability of the loss function, regardless of the specific value of $L_i(w^{*})$. 
This property allows for using a unified threshold $\delta$ across all instances, greatly simplifying the implementation and management of the mastered criterion.
To further validate this advantage, we conducted experiments as shown in Figure \ref{fignew}. We calculate the zero-order (loss value), first-order, and second-order differences for each sample's loss during training on CIFAR-10. Subsequently, we computed the coefficient of variation (CV) for these differences to represent the degree of dispersion in the data.
Experimental results show that when $N=1, 2, 3$, the CV of their high-order differences of loss value exhibits a trend of first rising and then falling, eventually stabilizing at a lower level. 
This indicates that in the early stages of training, when only some samples are sufficiently learned, there is significant variability in the higher-order differences of loss value among different samples. As training progresses into the mid-to-late stages, and as most instances become sufficiently learned, all instances exhibit more similar values in their higher-order differences of loss value.
Therefore, we can use a fixed threshold to uniformly determine whether an instance has been \emph{mastered}. 
Further experiments confirm that the high-order difference has relatively smaller CV values, please refer to Appendix \ref{appc}. In our subsequent experiments (Section \ref{4.2}), we further evaluate the IES method under different criteria. 

Accordingly, based on the above analysis and experimental validation, an instance \(i\) is considered \emph{mastered} when the cumulative magnitude of these second-order differences of loss value falls beneath a specified small positive threshold \(\delta\), which is formally expressed as:
{\setlength{\abovedisplayskip}{+2pt}
\setlength{\belowdisplayskip}{+0pt}
\begin{equation}
\label{eq4}
 \left| \Delta^2 L_i(w^{(t)}) \right| < \delta.
\end{equation}}

 Ultimately, IES consists of two key stages: filtering out mastered instances in the \textcolor{changecolor}{full training-set $\mathcal{D}^{(0)}$} through forward propagation, and removing mastered instances and only optimizing not-yet mastered instances \(\mathcal{D}^{(t)}\) through backpropagation, as detailed in Algorithm \ref{alg1}.

\begin{figure}[t]
\vspace{-1em}
\begin{algorithm}[H]
\caption{Instance-dependent Early Stopping (IES)}
\begin{algorithmic}[1]
\REQUIRE \textcolor{changecolor}{Full training-set $\mathcal{D}^{(0)}$}, validation-set $\mathcal{V}$, model $f_{\theta}$, threshold $\delta$, max epochs $T$
\STATE Initialize model parameters $w^{(0)}$
\FOR{$t = 1$ to $T$}
    \STATE Forward pass on model $f$ to compute loss $L_i(w^{(t)})$ for each sample $i \in {\color{changecolor}\mathcal{D}^{(0)}}$
    \STATE Calculate second-order differences: $\Delta^2 L_i(w^{(t)}) = L_i(w^{(t)}) - 2L_i(w^{(t-1)}) + L_i(w^{(t-2)})$
    \STATE Identify \emph{mastered} instances: $ \mathcal{M}^{(t)} = \left\{i \in \textcolor{changecolor}{\mathcal{D}^{(0)}}:  \left| \Delta^2 L_i(w^{(t')}) \right| < \delta \right\}$
    \STATE Update dataset for next epoch: $\mathcal{D}^{(t)} = \textcolor{changecolor}{\mathcal{D}^{(0)}} \setminus \mathcal{M}^{(t)}$
    \IF{$\mathcal{D}^{(t)}$ is empty \OR conventional early stopping criterion($\mathcal{V}, w^{(t)}$)}
        \STATE Break \COMMENT{Stop if all instances \emph{mastered} or conventional early stopping triggered}
    \ENDIF
    \STATE Update model parameters $w^{(t)}$ using instances in $\mathcal{D}^{(t)}$
\ENDFOR
\end{algorithmic}
\label{alg1}
\end{algorithm}
\vspace{-2.5em}
\end{figure}


\subsection{Instance-dependent Stopping to Accelerate Model Training Progression}
\label{sec3.3}


The proposed Instance-dependent Early Stopping (IES) method significantly reduces computational costs, as shown in its twofold impact: (1) IES achieves comparable performance to the baseline while requiring fewer backpropagation instances, and (2) IES surpasses the baseline's performance with the same amount of backpropagation. This subsection presents experimental results and analysis to showcase the IES's effectiveness and effective in accelerating model training progression. A detailed analysis demonstrating that IES does not lead to catastrophic forgetting is provided in Appendix~\ref{apph}.

\textbf{Less backpropagation, similar performance.} Our proposed method achieve comparable performance to the baseline method while using fewer instances in backpropagation. As detailed in Section \ref{sec3.1}, the effectiveness of using fewer instances in backpropagation without compromising performance is achieved through the precise identification of mastered instances. \textcolor{changecolor}{As shown in Figure \ref{fig3}, our method reduces the number of training instances in backpropagation by approximately 40\%}, resulting in a savings of nearly 30\% in total computational cost while maintaining generalization performance.


\begin{figure*}[h!]
 \vskip -0.1in
 \begin{subfigure}[b]{0.48\textwidth}
     \includegraphics[width=\textwidth]{ies-sgd.png}
     \caption{\emph{SGD - 150 Epochs}}
 \end{subfigure}
 \hfill
 \begin{subfigure}[b]{0.482\textwidth}
     \includegraphics[width=\textwidth]{ies-adam.png}
     \caption{\emph{Adam - 150 Epochs}}
 \end{subfigure}
 \vskip -0.05in
\caption{Comparison of model performance metrics between the IES method and the baseline method over the same number of backpropagation training instances. The metrics include test error, gradient norm, training loss, sharpness-aware minimization (SAM) value, and the maximum eigenvalue of the Hessian matrix. IES consistently outperforms the baseline in test error and reduces training loss, SAM value, and the maximum eigenvalue more effectively, indicating a faster progression in model training. We use ResNet-18 on the CIFAR-10 dataset in this experiment. Further detailed experimental settings can be found in Appendix \ref{appb} and Section \ref{sec4}.}
 \label{fig3}
\end{figure*}

\textbf{Same backpropagation, better performance.}
Our proposed method consistently achieves better performance than the baseline method at the same number of backpropagation instances. 
To further demonstrate the superiority of our method, we conduct analysis from following experiments. 

\emph{Larger Gradient Norms.}
As shown in Figure \ref{fig3}, the average mini-batch gradient norm for instances selected by the IES method \textcolor{changecolor}{is} consistently higher than that of the baseline method, which uses all training data. By typically selecting instances with larger gradient norms for backpropagation, the IES method tends to make parameter updates more effective at reducing training loss.

\emph{Faster Reduce Sharpness.}
We conducted experiments to investigate the changes of model's sharpness during the training process. Sharpness \citep{hochreiter1997flat, pmlr-v162-andriushchenko22a, huang2023robust} often refers to the steepness of the loss function near the solution and is closely related to the large eigenvalues of the Hessian matrix \citep{dinh2017sharp, tsuzuku2020normalized}. We compared the changes in the largest eigenvalue of the Hessian matrix and the Sharpness-Aware Minimization (SAM) value \citep{foret2020sharpness}. 
As shown in Figure \ref{fig3}, IES reduces the largest eigenvalue more quickly and consistently achieves lower SAM values compared with the baseline method. The faster reduction in the largest eigenvalue suggests that IES can more targetedly reduce steepness in these sharp directions of the loss landscape, thereby reducing the overall ``sharpnes'' more quickly. A lower SAM value indicates a flatter minima with lower sharpness. These empirical observations together support that  IES can more targetedly reduce steepness in these sharp directions of the loss landscape \citep{li2017stochastic, keskar2016large, neyshabur2017exploring, dauphin2014identifying}, thereby reducing the overall sharpness more quickly.
\section{Experiments}
\label{sec4}
In this section, we empirically demonstrate the effectiveness of \emph{Instance-dependent Early Stopping} method.
In Section \ref{4.1}, we validate the broad applicability of our proposed method across different settings. 
Furthermore, we demonstrate through experiments that applying our proposed method can improve the transferability of models. 
In Section \ref{4.2}, we compare our proposed IES method with other methods and different instance-level stopping criteria. We showcase the capability of our method to maintain model performance across a wide range of hyperparameters.
Section \ref{4.3} demonstrates our method's applicability for more efficient training and its use in high-level tasks such as segmentation and detection.
The empirical evidence indicates that our proposed Instance-dependent Early Stopping method can effectively reduce computational overhead under various settings, outperforming existing baselines while simultaneously enhancing the transfer learning capabilities of models. 


\subsection{Effectiveness of IES}
\label{4.1}
To evaluate the effectiveness of our proposed IES method, we conducted extensive experiments under various settings, including different datasets, network architectures, and optimizers. Table \ref{tab 1} and \ref{tab 2} demonstrate the consistent performance of the IES method across these settings. It is worth noting that IES achieves lossless acceleration for model training; if a $1\%$ generalization performance decrease is acceptable, even more substantial acceleration can be obtained, as detailed in Section \ref{4.2}. 

\begin{table*}[t]
\centering
	\caption{Effectiveness of IES-2nd across various settings. (5 runs, mean±std)  }
 \vskip -0.05in
	\label{tab 1}
\resizebox{1\textwidth}{!}{
\setlength{\tabcolsep}{0.6mm}{
\begin{tabular}{c|ccc|ccc}
\toprule
&\multicolumn{3}{c|}{\textbf{CIFAR-10}} & \multicolumn{3}{c}{\textbf{CIFAR-100}}\\
\cmidrule(lr){1-1}\cmidrule(lr){2-4}\cmidrule(lr){5-7}
\textbf{\emph{Architectures}}&ResNet-18& ResNet-50 & VGG-16 & ResNet-34 & ResNet-101 & DenseNet-121  \\
\midrule
No Removal  & 92.9\%$\pm$0.1\% & 93.3\%$\pm$0.1\% & 90.9\%$\pm$0.2\% & 69.8\%$\pm$0.4\% & 71.9\%$\pm$0.5\% & 73.4\%$\pm$0.0\% \\
IES (Ours) & 92.9\%$\pm$0.2\% & 93.1\%$\pm$0.1\% & 90.7\%$\pm$0.2\% & 69.6\%$\pm$0.3\% & 72.2\%$\pm$0.5\% & 73.3\%$\pm$0.2\% \\
\midrule
Mini-batch Saved  & 54.6\% & 48.5\% & 30.4\% & 29.9\% & 28.3\% & 33.4\%  \\
\midrule
\midrule
\textbf{\emph{Optimizers}}&SGD(F)& SGD(L) & AdamW &SGD(F)& SGD(E) & AdamW  \\
\midrule
No Removal  & 92.1\%$\pm$0.1\% & 95.2\%$\pm$0.1\% & 92.6\%$\pm$0.1\% & 71.4\%$\pm$0.5\% & 77.6\%$\pm$0.4\% & 69.6\%$\pm$0.3\% \\
IES (Ours) & 92.4\%$\pm$0.0\% & 95.1\%$\pm$0.1\% & 92.7\%$\pm$0.1\% & 72.3\%$\pm$0.4\% & 77.4\%$\pm$0.4\% & 69.7\%$\pm$0.5\% \\
\midrule
Mini-batch Saved   & 37.3\% & 26.4\% & 47.2\% & 9.3\% & 27.3\% & 17.4\%  \\
\midrule
\midrule
Avg. Mini-batch Saved  &  & \textbf{40.7\%}  & & & \textbf{24.3\%} &   \\
\midrule
Avg. Wall-time Speedup  &  & \boldmath{$\sim1.4\times$} & & & \boldmath{$\sim1.2\times$} &   \\

\bottomrule  
\end{tabular}
}
}
\vskip -0.05in
\end{table*}

\begin{table*}[t!]
\vskip -0.02in
\centering
	\caption{Effectiveness of IES-2nd in ImageNet-1k task. (1 run) }
 \vskip -0.05in
	\label{tab 2}
\resizebox{1\textwidth}{!}{
\setlength{\tabcolsep}{5.8mm}{
\begin{tabular}{c|cccc}
\toprule
& \multicolumn{1}{c}{\emph{DenseNet-121}} & \multicolumn{2}{c}{\emph{ResNet-34}} & \multicolumn{1}{c}{\emph{ResNet-101}}\\
\cmidrule(lr){2-2}\cmidrule(lr){3-4}\cmidrule(lr){5-5}
 Methods & AdamW & AdamW & SGD(M) & SGD(E)  \\
\midrule
No Removal    & 69.0\% & 68.0\%     & 74.1\%     & 77.4\%   \\
IES (Ours)    & 68.8\% & 68.0\%     & 74.3\%     & 77.4\%   \\
\midrule
Mini-batch Saved  & 31.6\%  & 28.7\%   & 30.3\%   & 34.2\%      \\
\midrule
\midrule
Avg. Wall-time Speedup   &  \multicolumn{4}{c}{\boldmath{$\sim1.3\times$}}  \\
\bottomrule  
\end{tabular}
}
}
\vskip -0.05in
\end{table*}


\begin{table*}[t!]
\vskip -0.02in
\centering
	\caption{Transfer performance of IES-2nd pretrained model on ImageNet-1k task. (5 runs, mean±std) }
 \vskip -0.05in
	\label{tab3}
\resizebox{1\textwidth}{!}{
\setlength{\tabcolsep}{4mm}{
\begin{tabular}{c|cc|cc}
\toprule
& \multicolumn{2}{c|}{\emph{ResNet-101}} & \multicolumn{2}{c}{\emph{DenseNet-121}}\\
\cmidrule(lr){2-3}\cmidrule(lr){4-5}
  Transfer Tasks & IES (Ours) & No Removal & IES (Ours) & No Removal \\
\midrule
\emph{ImageNet-1k --> CIFAR-10} & \textbf{81.2\%$\pm$0.1\%} & 80.3\%$\pm$0.2\% & \textbf{78.6\% $\pm$ 0.2\%} & 77.3\% $\pm$ 0.2\% \\
\midrule
\emph{ImageNet-1k --> CIFAR-100} & \textbf{57.5\%$\pm$0.2\%} & 55.6\%$\pm$0.2\% & \textbf{53.0\% $\pm$ 0.2\%} & 52.3\% $\pm$ 0.2\% \\
\midrule
\emph{ImageNet-1k --> Caltech-101} & \textbf{59.9\%$\pm$0.8\%} & 57.4\%$\pm$1.2\% & \textbf{50.9\% $\pm$ 1.6\%} & 49.5\% $\pm$ 1.5\% \\
\bottomrule  
\end{tabular}
}
}
\vskip -0.2in
\end{table*}


Our evaluations confirmed the effectiveness of the IES method across multiple datasets. These datasets comprise \emph{CIFAR-10}, \emph{CIFAR-100} \citep{krizhevsky2009learning}, and \emph{ImageNet-1k} \citep{deng2009imagenet}. For the \emph{CIFAR} and \textcolor{changecolor}{the} \emph{ImageNet-1k} tasks, we train for 200 and 150 epochs, respectively. For \emph{ImageNet-1k} task, we follow \cite{qin2023infobatch} and anneal in the last 10\% of \textcolor{changecolor}{epochs}.
We employ different optimizers such as \emph{SGD} with \emph{Momentum} \citep{robbins1951stochastic, polyak1964some}, \emph{Adam} \citep{kingma2014adam}, and \emph{AdamW} \citep{loshchilov2017decoupled} to demonstrate that the IES method remains resilient to reasonable variations across multiple optimizers and learning rate schedulers. 

We use different SGD learning rate scheduler settings: \emph{SGD(F)}, \emph{SGD(L)}, \emph{SGD(M)}, and \emph{SGD(E)}, which represent SGD with a fixed learning rate, a linearly decaying learning rate, a multi-step decaying learning rate, and an exponentially decaying learning rate scheduler, respectively. 
For \emph{CIFAR}, we set base $\delta = 1e^{-3}$; and for \emph{ImageNet-1k}, we set $\delta = 1$. Further, we verified the effectiveness of our proposed IES method over several commonly used deep learning models, including \emph{ResNet} \citep{he2016deep}, \emph{VGG} \citep{simonyan2014very}, and \emph{DenseNet} \citep{huang2017densely}. More detailed experimental settings and additional results can be found in Appendix \ref{appb}.

To quantify the computational resources saved by the IES method, we consider two following metrics:
\begin{itemize}[leftmargin=0.35cm,topsep=-2pt]
\item \emph{Mini-batch Saved.} We calculate the percentage of mini-batch saved. This metric directly reflects the reduction in the number of instances of backpropagation computations.
\item \emph{Wall-time Speedup.} We measure the training time speedup achieved by the IES method compared with full data training. This metric provides a realistic assessment of the time savings. We report the average training speedup on CIFAR-10, CIFAR-100, and ImageNet-1k tasks.
\end{itemize}
The IES method primarily saves computational resources by reducing the backpropagation steps, which constitute the most time-consuming part of the training process. However, the forward pass still needs to be computed for all instances to obtain their predictions and determine which instances should be stopped based on the \emph{mastered} criterion. 
The experimental results demonstrate that the IES method can save 10\% to 55\% of mini-batch computations and speedup 20\% to 40\% of training time, while maintaining test accuracy comparable to full data training. The empirical evidence indicates the effectiveness of the IES method in reducing computational costs without compromising performance. 

\textbf{\textcolor{changecolor}{Transferability.}}
To further evaluate the \textcolor{changecolor}{effectiveness} of the IES method, we investigated its impact on the transfer learning of models. We first pretrained models on the ImageNet-1k dataset using IES and the baseline method without instance stopping. Then, we fine-tuned only the classification head of these pretrained models using the model from the last epoch of pretraining on several downstream tasks, including CIFAR-10, CIFAR-100, and Caltech-101 \citep{li_andreeto_ranzato_perona_2022} datasets.
As shown in Table \ref{tab3}, models pretrained with IES consistently outperform those pretrained without instance removal across all transfer learning tasks while achieving the comparable test accuracy on the ImageNet-1k. After fine-tuning for 1 epoch, the IES pretrained model surpassing the baseline by 0.9\%, 1.9\%, and 2.5\% on CIFAR-10, CIFAR-100, and Caltech-101, respectively. Similar improvements are observed on DenseNet-121 and on more epoch fine-tuning, as shown in Table \ref{tab3} and \ref{tabapp4} in Appendix \ref{appb5}.

These results align with the discussion in Section \ref{sec3.3}, suggesting that IES can more effectively reduce the sharpness of the loss landscape during pretraining. By instance-dependent stopping of instance training, IES saves computational resources while potentially contributing to a more favorable loss landscape and more transferable performance. Consequently, models pretrained with IES exhibit better transfer learning performance, adapting more effectively to new tasks with limited fine-tuning.


\subsection{Efficiency of IES}
\label{4.2}
Building upon the IES method's demonstrated ability to accelerate training without performance loss, this section further explores its efficiency. We compare IES with different sample selection criteria, analyze the impact of different $\delta$ value settings on performance, and investigate the potential for additional acceleration when allowing a slight decrease in performance.

\textbf{Comparison with other sample selection methods.}
We compare our proposed IES method with \emph{Random Remove} and \emph{Small Loss \& Rescale} \citep{qin2023infobatch}, under different Total Excluded Samples values. Total Excluded Samples values represent the proportion of samples removed from the backpropagation during training. \emph{Random Remove} method randomly removes a certain proportion of samples from backpropagation in each training epoch, while \emph{Small Loss \& Rescale} randomly prunes samples with smaller loss values and amplifies the gradients of the remaining small-loss samples.
As shown in Figure \ref{fig4}, experimental results on both CIFAR-10 and CIFAR-100 datasets demonstrate that the \emph{Random Remove} method significantly reduces model performance. Although the \emph{Small Loss \& Rescale} method improve results, its performance still falls behind IES. Moreover, among different IES configurations, using second-order differences outperforms other configurations in most cases, which aligns with our analysis in Section \ref{sec3.2}.
Further details are in Appendix \ref{appd}.

\textbf{Analysis of setting \(\delta\) values.}
To further evaluate the robustness of the IES method, we expanded upon the previous comparison by setting a broader range of $\delta$ values, observing their impact on sample exclusion and model accuracy.
As shown in Figure \ref{fig4} (lower row),  we varied the $\delta$ value used in the IES method by multiplying the selected $\delta$ value (set to 0.001) by scales of $\{0.01, 0.1, 1, 10, 100\}$. 
Notably, even as $\delta$ varied across four orders of magnitude, the IES method maintained the test accuracy within approximately 2\% of the baseline performance. This highlights the significant stability and adaptability of the IES method across a wide range of $\delta$ settings, enabling its effective implementation in diverse scenarios without the need for precise fine-tuning of the $\delta$ parameter. 

\begin{figure*}[h]
 \vskip -0.02in
 \begin{subfigure}[b]{0.458\textwidth}
     \includegraphics[width=\textwidth]{IESfig6walltime.pdf}
     \vskip -0.02in
     \caption{\emph{CIFAR-10}}
     \label{fig:sub1}
 \end{subfigure}
 \hfill
 \begin{subfigure}[b]{0.458\textwidth}
     \includegraphics[width=\textwidth]{IESfig6walltime_c100.pdf}
     \vskip -0.02in
     \caption{\emph{CIFAR-100}}
     \label{fig:sub2}
 \end{subfigure}
 \vskip -0.08in
\caption{Comparison of the proposed IES method of different IES criteria (loss, 1st, 2nd, and 3rd order differences) with other sample selection methods under different Total Excluded Samples values on both CIFAR datasets. \textcolor{changecolor}{The lower subfigure illustrates the effect of varying $\delta$ values used in IES methods on training time reduction, sample removal, and model performance (3 runs, mean±std).}}
 \label{fig4}
\vskip -0.02in
\end{figure*}

\clearpage

\subsection{Further Analysis}
\label{4.3}

This section explores the scalability of IES for further acceleration, focusing on: (1) accelerating training while tolerating minor performance loss, and (2) maintaining accuracy while achieving targeted training speedups. Additionally, we evaluate the efficacy of IES for high-level vision tasks. Furthermore, we demonstrate the robustness of our proposed IES method under the common scenario of learning with noisy labels \citep{xia2020robust}, as detailed in Appendix \ref{applabelnoise}.

\textbf{Tolerating 1\% performance loss.} While IES aims to maintain test accuracy compared to full data training, it also has potential for further acceleration if a slight decrease in test accuracy is acceptable. By allowing a 1\% reduction in test accuracy, we observed that IES can achieve even greater computational savings.
For the ImageNet-1k dataset, IES can save up to 40\% of backpropagation. As shown in Figure \ref{fig4} (upper row), for the CIFAR-10 and CIFAR-100 datasets, IES can save up to 80\% and 60\% of backpropagation, respectively. These results demonstrate that IES can be flexibly adjusted to prioritize either improving efficiency without performance loss (a ``free lunch'') or further accelerating training within an acceptable range of performance degradation.

\textbf{Achieving 2.0$\times$ training speedups.} To further evaluate the efficacy of our proposed IES method in scenarios prioritizing efficient training, we conducted a comparison with several data efficiency methods: conventional early stopping, importance sampling \citep{jiang2019accelerating} \emph{SB}, hardness-based curriculum learning \citep{zhou2020curriculum} \emph{DIHCL}, and resizing-based curriculum learning \citep{wang2024efficienttrain++} \emph{EfficientTrain} methods. For a fair comparison, we set the target computational acceleration to approximately 2.0 times across all methods. The detailed settings and 3.0 times acceleration results are provided in Appendix \ref{appf}. As shown in Table \ref{tab4}, these comparisons further demonstrate that IES performs effectively in accelerating model training while maintaining model performance. 

\begin{table}[t]
\centering
\caption{Comparison of IES and other data efficiency methods. (3 runs, mean±std)}
\vskip -0.12in
\resizebox{1\textwidth}{!}{
\setlength{\tabcolsep}{3.5mm}
\renewcommand{\arraystretch}{0.88}
\begin{tabular}{c|c|cc}
\toprule
Computation Speedup & Methods & CIFAR-10 & CIFAR-100 \\
\midrule
$1.0\times$ & Baseline (No Removal) & 94.3\%$\pm$0.3\% & 77.0\%$\pm$0.4\% \\
\midrule
\multirow{5}{*}{$\sim2.0\times$} & Conventional Early Stopping & 90.4\%$\pm$0.5\% & 68.7\%$\pm$0.5\% \\
& SB \citep{jiang2019accelerating} & 93.0\%$\pm$0.1\% & 70.6\%$\pm$0.5\% \\
& DIHCL \citep{zhou2020curriculum} & 93.4\%$\pm$0.2\% & 74.3\%$\pm$0.2\% \\
& EfficientTrain \citep{wang2024efficienttrain++} & 91.5\%$\pm$0.2\% & \textbf{75.0\%$\pm$0.1\%} \\
\cmidrule(lr){2-2}\cmidrule(lr){3-4}
& IES (Ours) & \textbf{93.7\%$\pm$0.4\%} & \textbf{74.9\%$\pm$0.5\%} \\
\bottomrule
\end{tabular}
}
\vskip -0.16in
\label{tab4}
\end{table}

\begin{wraptable}{r}{0.58\textwidth}  
\centering
\vskip -0.16in
\resizebox{0.58\textwidth}{!}{  
\setlength{\tabcolsep}{1mm}
\begin{tabular}{c|c|c}
\toprule
 & Faster R-CNN (\emph{mAP}) & DeepLab v3 (\emph{mIoU})   \\
\midrule
No Removal   & 70.2\%$\pm$0.2\%     & 76.2\%$\pm$0.2\%       \\
IES (Ours)   & 70.2\%$\pm$0.1\%     & 76.1\%$\pm$0.2\%      \\
\midrule
Mini-batch Saved   & 20.0\%   & 14.0\%     \\
\bottomrule  
\end{tabular}
}
\vskip -0.05in
\caption{Effectiveness of the IES on object detection and segmentation model training tasks. (3 runs, mean±std)}
\vskip -0.2in
\label{tabhigh}
\end{wraptable}


\textbf{High-level vision tasks.} To further validate the applicability of the IES method, we conducted experiments on two high-level tasks: object detection and semantic segmentation.  Specifically, we integrated our proposed IES method into the baseline methods Faster R-CNN \citep{ren2015faster} and DeepLab v3 \citep{chen2017rethinking}, respectively. For both task, we use the PASCAL VOC datasets \citep{pascal-voc-2007, pascal-voc-2012}. A brief overview of the results of model training is reported in the Table \ref{tabhigh}. Further details are in Appendix \ref{appg}.

 

\section{Conclusion}

\label{sec5}
In this work, we propose an \emph{Instance-dependent Early Stopping} (IES) method that adapts the early stopping mechanism from the entire training set to the instance level. IES considers an instance as \emph{mastered} if the second-order differences of its loss value remain within a small range around zero, allowing for a unified threshold to determine when an instance can be excluded from further backpropagation. Extensive experiments demonstrate the effectiveness of IES in reducing computational cost while maintaining model performance and transferability.

\textbf{Limitation.} \textcolor{changecolor}{While the choice of using the second-order difference as the removal criterion for IES has been validated through experiments, a comprehensive theoretical analysis of its superiority remains an open research question.} The impact of the IES method on fairness is evaluated in Appendix \ref{appi}, but it still has not been thoroughly investigated.

\subsubsection*{Acknowledgments}
The authors would like to thank the anonymous reviewers for their insightful and constructive comments.
Suqin Yuan extends special thanks to Muyang Li, Runnan Chen, Xiu-Chuan Li, Jun Wang, Li He, and Yuhao Wu for their valuable advice and computing support.
Tongliang Liu is partially supported by the following Australian Research Council projects: FT220100318, DP220102121, LP220100527, LP220200949, IC190100031.
BH was supported by RGC Young Collaborative Research Grant No. C2005-24Y, NSFC General Program No. 62376235, and Guangdong Basic and Applied Basic Research Foundation Nos. 2022A1515011652 and 2024A1515012399.
This research was undertaken with the assistance of resources from the National Computational Infrastructure (NCI Australia), an NCRIS enabled capability supported by the Australian Government.
Suqin Yuan is partially supported by the OpenAI Researcher Access Program.


\documentclass{article} % For LaTeX2e
\usepackage{iclr2025_conference,times}

% Optional math commands from https://github.com/goodfeli/dlbook_notation.
%%%%% NEW MATH DEFINITIONS %%%%%

\usepackage{amsmath,amsfonts,bm}
\usepackage{derivative}
% Mark sections of captions for referring to divisions of figures
\newcommand{\figleft}{{\em (Left)}}
\newcommand{\figcenter}{{\em (Center)}}
\newcommand{\figright}{{\em (Right)}}
\newcommand{\figtop}{{\em (Top)}}
\newcommand{\figbottom}{{\em (Bottom)}}
\newcommand{\captiona}{{\em (a)}}
\newcommand{\captionb}{{\em (b)}}
\newcommand{\captionc}{{\em (c)}}
\newcommand{\captiond}{{\em (d)}}

% Highlight a newly defined term
\newcommand{\newterm}[1]{{\bf #1}}

% Derivative d 
\newcommand{\deriv}{{\mathrm{d}}}

% Figure reference, lower-case.
\def\figref#1{figure~\ref{#1}}
% Figure reference, capital. For start of sentence
\def\Figref#1{Figure~\ref{#1}}
\def\twofigref#1#2{figures \ref{#1} and \ref{#2}}
\def\quadfigref#1#2#3#4{figures \ref{#1}, \ref{#2}, \ref{#3} and \ref{#4}}
% Section reference, lower-case.
\def\secref#1{section~\ref{#1}}
% Section reference, capital.
\def\Secref#1{Section~\ref{#1}}
% Reference to two sections.
\def\twosecrefs#1#2{sections \ref{#1} and \ref{#2}}
% Reference to three sections.
\def\secrefs#1#2#3{sections \ref{#1}, \ref{#2} and \ref{#3}}
% Reference to an equation, lower-case.
\def\eqref#1{equation~\ref{#1}}
% Reference to an equation, upper case
\def\Eqref#1{Equation~\ref{#1}}
% A raw reference to an equation---avoid using if possible
\def\plaineqref#1{\ref{#1}}
% Reference to a chapter, lower-case.
\def\chapref#1{chapter~\ref{#1}}
% Reference to an equation, upper case.
\def\Chapref#1{Chapter~\ref{#1}}
% Reference to a range of chapters
\def\rangechapref#1#2{chapters\ref{#1}--\ref{#2}}
% Reference to an algorithm, lower-case.
\def\algref#1{algorithm~\ref{#1}}
% Reference to an algorithm, upper case.
\def\Algref#1{Algorithm~\ref{#1}}
\def\twoalgref#1#2{algorithms \ref{#1} and \ref{#2}}
\def\Twoalgref#1#2{Algorithms \ref{#1} and \ref{#2}}
% Reference to a part, lower case
\def\partref#1{part~\ref{#1}}
% Reference to a part, upper case
\def\Partref#1{Part~\ref{#1}}
\def\twopartref#1#2{parts \ref{#1} and \ref{#2}}

\def\ceil#1{\lceil #1 \rceil}
\def\floor#1{\lfloor #1 \rfloor}
\def\1{\bm{1}}
\newcommand{\train}{\mathcal{D}}
\newcommand{\valid}{\mathcal{D_{\mathrm{valid}}}}
\newcommand{\test}{\mathcal{D_{\mathrm{test}}}}

\def\eps{{\epsilon}}


% Random variables
\def\reta{{\textnormal{$\eta$}}}
\def\ra{{\textnormal{a}}}
\def\rb{{\textnormal{b}}}
\def\rc{{\textnormal{c}}}
\def\rd{{\textnormal{d}}}
\def\re{{\textnormal{e}}}
\def\rf{{\textnormal{f}}}
\def\rg{{\textnormal{g}}}
\def\rh{{\textnormal{h}}}
\def\ri{{\textnormal{i}}}
\def\rj{{\textnormal{j}}}
\def\rk{{\textnormal{k}}}
\def\rl{{\textnormal{l}}}
% rm is already a command, just don't name any random variables m
\def\rn{{\textnormal{n}}}
\def\ro{{\textnormal{o}}}
\def\rp{{\textnormal{p}}}
\def\rq{{\textnormal{q}}}
\def\rr{{\textnormal{r}}}
\def\rs{{\textnormal{s}}}
\def\rt{{\textnormal{t}}}
\def\ru{{\textnormal{u}}}
\def\rv{{\textnormal{v}}}
\def\rw{{\textnormal{w}}}
\def\rx{{\textnormal{x}}}
\def\ry{{\textnormal{y}}}
\def\rz{{\textnormal{z}}}

% Random vectors
\def\rvepsilon{{\mathbf{\epsilon}}}
\def\rvphi{{\mathbf{\phi}}}
\def\rvtheta{{\mathbf{\theta}}}
\def\rva{{\mathbf{a}}}
\def\rvb{{\mathbf{b}}}
\def\rvc{{\mathbf{c}}}
\def\rvd{{\mathbf{d}}}
\def\rve{{\mathbf{e}}}
\def\rvf{{\mathbf{f}}}
\def\rvg{{\mathbf{g}}}
\def\rvh{{\mathbf{h}}}
\def\rvu{{\mathbf{i}}}
\def\rvj{{\mathbf{j}}}
\def\rvk{{\mathbf{k}}}
\def\rvl{{\mathbf{l}}}
\def\rvm{{\mathbf{m}}}
\def\rvn{{\mathbf{n}}}
\def\rvo{{\mathbf{o}}}
\def\rvp{{\mathbf{p}}}
\def\rvq{{\mathbf{q}}}
\def\rvr{{\mathbf{r}}}
\def\rvs{{\mathbf{s}}}
\def\rvt{{\mathbf{t}}}
\def\rvu{{\mathbf{u}}}
\def\rvv{{\mathbf{v}}}
\def\rvw{{\mathbf{w}}}
\def\rvx{{\mathbf{x}}}
\def\rvy{{\mathbf{y}}}
\def\rvz{{\mathbf{z}}}

% Elements of random vectors
\def\erva{{\textnormal{a}}}
\def\ervb{{\textnormal{b}}}
\def\ervc{{\textnormal{c}}}
\def\ervd{{\textnormal{d}}}
\def\erve{{\textnormal{e}}}
\def\ervf{{\textnormal{f}}}
\def\ervg{{\textnormal{g}}}
\def\ervh{{\textnormal{h}}}
\def\ervi{{\textnormal{i}}}
\def\ervj{{\textnormal{j}}}
\def\ervk{{\textnormal{k}}}
\def\ervl{{\textnormal{l}}}
\def\ervm{{\textnormal{m}}}
\def\ervn{{\textnormal{n}}}
\def\ervo{{\textnormal{o}}}
\def\ervp{{\textnormal{p}}}
\def\ervq{{\textnormal{q}}}
\def\ervr{{\textnormal{r}}}
\def\ervs{{\textnormal{s}}}
\def\ervt{{\textnormal{t}}}
\def\ervu{{\textnormal{u}}}
\def\ervv{{\textnormal{v}}}
\def\ervw{{\textnormal{w}}}
\def\ervx{{\textnormal{x}}}
\def\ervy{{\textnormal{y}}}
\def\ervz{{\textnormal{z}}}

% Random matrices
\def\rmA{{\mathbf{A}}}
\def\rmB{{\mathbf{B}}}
\def\rmC{{\mathbf{C}}}
\def\rmD{{\mathbf{D}}}
\def\rmE{{\mathbf{E}}}
\def\rmF{{\mathbf{F}}}
\def\rmG{{\mathbf{G}}}
\def\rmH{{\mathbf{H}}}
\def\rmI{{\mathbf{I}}}
\def\rmJ{{\mathbf{J}}}
\def\rmK{{\mathbf{K}}}
\def\rmL{{\mathbf{L}}}
\def\rmM{{\mathbf{M}}}
\def\rmN{{\mathbf{N}}}
\def\rmO{{\mathbf{O}}}
\def\rmP{{\mathbf{P}}}
\def\rmQ{{\mathbf{Q}}}
\def\rmR{{\mathbf{R}}}
\def\rmS{{\mathbf{S}}}
\def\rmT{{\mathbf{T}}}
\def\rmU{{\mathbf{U}}}
\def\rmV{{\mathbf{V}}}
\def\rmW{{\mathbf{W}}}
\def\rmX{{\mathbf{X}}}
\def\rmY{{\mathbf{Y}}}
\def\rmZ{{\mathbf{Z}}}

% Elements of random matrices
\def\ermA{{\textnormal{A}}}
\def\ermB{{\textnormal{B}}}
\def\ermC{{\textnormal{C}}}
\def\ermD{{\textnormal{D}}}
\def\ermE{{\textnormal{E}}}
\def\ermF{{\textnormal{F}}}
\def\ermG{{\textnormal{G}}}
\def\ermH{{\textnormal{H}}}
\def\ermI{{\textnormal{I}}}
\def\ermJ{{\textnormal{J}}}
\def\ermK{{\textnormal{K}}}
\def\ermL{{\textnormal{L}}}
\def\ermM{{\textnormal{M}}}
\def\ermN{{\textnormal{N}}}
\def\ermO{{\textnormal{O}}}
\def\ermP{{\textnormal{P}}}
\def\ermQ{{\textnormal{Q}}}
\def\ermR{{\textnormal{R}}}
\def\ermS{{\textnormal{S}}}
\def\ermT{{\textnormal{T}}}
\def\ermU{{\textnormal{U}}}
\def\ermV{{\textnormal{V}}}
\def\ermW{{\textnormal{W}}}
\def\ermX{{\textnormal{X}}}
\def\ermY{{\textnormal{Y}}}
\def\ermZ{{\textnormal{Z}}}

% Vectors
\def\vzero{{\bm{0}}}
\def\vone{{\bm{1}}}
\def\vmu{{\bm{\mu}}}
\def\vtheta{{\bm{\theta}}}
\def\vphi{{\bm{\phi}}}
\def\va{{\bm{a}}}
\def\vb{{\bm{b}}}
\def\vc{{\bm{c}}}
\def\vd{{\bm{d}}}
\def\ve{{\bm{e}}}
\def\vf{{\bm{f}}}
\def\vg{{\bm{g}}}
\def\vh{{\bm{h}}}
\def\vi{{\bm{i}}}
\def\vj{{\bm{j}}}
\def\vk{{\bm{k}}}
\def\vl{{\bm{l}}}
\def\vm{{\bm{m}}}
\def\vn{{\bm{n}}}
\def\vo{{\bm{o}}}
\def\vp{{\bm{p}}}
\def\vq{{\bm{q}}}
\def\vr{{\bm{r}}}
\def\vs{{\bm{s}}}
\def\vt{{\bm{t}}}
\def\vu{{\bm{u}}}
\def\vv{{\bm{v}}}
\def\vw{{\bm{w}}}
\def\vx{{\bm{x}}}
\def\vy{{\bm{y}}}
\def\vz{{\bm{z}}}

% Elements of vectors
\def\evalpha{{\alpha}}
\def\evbeta{{\beta}}
\def\evepsilon{{\epsilon}}
\def\evlambda{{\lambda}}
\def\evomega{{\omega}}
\def\evmu{{\mu}}
\def\evpsi{{\psi}}
\def\evsigma{{\sigma}}
\def\evtheta{{\theta}}
\def\eva{{a}}
\def\evb{{b}}
\def\evc{{c}}
\def\evd{{d}}
\def\eve{{e}}
\def\evf{{f}}
\def\evg{{g}}
\def\evh{{h}}
\def\evi{{i}}
\def\evj{{j}}
\def\evk{{k}}
\def\evl{{l}}
\def\evm{{m}}
\def\evn{{n}}
\def\evo{{o}}
\def\evp{{p}}
\def\evq{{q}}
\def\evr{{r}}
\def\evs{{s}}
\def\evt{{t}}
\def\evu{{u}}
\def\evv{{v}}
\def\evw{{w}}
\def\evx{{x}}
\def\evy{{y}}
\def\evz{{z}}

% Matrix
\def\mA{{\bm{A}}}
\def\mB{{\bm{B}}}
\def\mC{{\bm{C}}}
\def\mD{{\bm{D}}}
\def\mE{{\bm{E}}}
\def\mF{{\bm{F}}}
\def\mG{{\bm{G}}}
\def\mH{{\bm{H}}}
\def\mI{{\bm{I}}}
\def\mJ{{\bm{J}}}
\def\mK{{\bm{K}}}
\def\mL{{\bm{L}}}
\def\mM{{\bm{M}}}
\def\mN{{\bm{N}}}
\def\mO{{\bm{O}}}
\def\mP{{\bm{P}}}
\def\mQ{{\bm{Q}}}
\def\mR{{\bm{R}}}
\def\mS{{\bm{S}}}
\def\mT{{\bm{T}}}
\def\mU{{\bm{U}}}
\def\mV{{\bm{V}}}
\def\mW{{\bm{W}}}
\def\mX{{\bm{X}}}
\def\mY{{\bm{Y}}}
\def\mZ{{\bm{Z}}}
\def\mBeta{{\bm{\beta}}}
\def\mPhi{{\bm{\Phi}}}
\def\mLambda{{\bm{\Lambda}}}
\def\mSigma{{\bm{\Sigma}}}

% Tensor
\DeclareMathAlphabet{\mathsfit}{\encodingdefault}{\sfdefault}{m}{sl}
\SetMathAlphabet{\mathsfit}{bold}{\encodingdefault}{\sfdefault}{bx}{n}
\newcommand{\tens}[1]{\bm{\mathsfit{#1}}}
\def\tA{{\tens{A}}}
\def\tB{{\tens{B}}}
\def\tC{{\tens{C}}}
\def\tD{{\tens{D}}}
\def\tE{{\tens{E}}}
\def\tF{{\tens{F}}}
\def\tG{{\tens{G}}}
\def\tH{{\tens{H}}}
\def\tI{{\tens{I}}}
\def\tJ{{\tens{J}}}
\def\tK{{\tens{K}}}
\def\tL{{\tens{L}}}
\def\tM{{\tens{M}}}
\def\tN{{\tens{N}}}
\def\tO{{\tens{O}}}
\def\tP{{\tens{P}}}
\def\tQ{{\tens{Q}}}
\def\tR{{\tens{R}}}
\def\tS{{\tens{S}}}
\def\tT{{\tens{T}}}
\def\tU{{\tens{U}}}
\def\tV{{\tens{V}}}
\def\tW{{\tens{W}}}
\def\tX{{\tens{X}}}
\def\tY{{\tens{Y}}}
\def\tZ{{\tens{Z}}}


% Graph
\def\gA{{\mathcal{A}}}
\def\gB{{\mathcal{B}}}
\def\gC{{\mathcal{C}}}
\def\gD{{\mathcal{D}}}
\def\gE{{\mathcal{E}}}
\def\gF{{\mathcal{F}}}
\def\gG{{\mathcal{G}}}
\def\gH{{\mathcal{H}}}
\def\gI{{\mathcal{I}}}
\def\gJ{{\mathcal{J}}}
\def\gK{{\mathcal{K}}}
\def\gL{{\mathcal{L}}}
\def\gM{{\mathcal{M}}}
\def\gN{{\mathcal{N}}}
\def\gO{{\mathcal{O}}}
\def\gP{{\mathcal{P}}}
\def\gQ{{\mathcal{Q}}}
\def\gR{{\mathcal{R}}}
\def\gS{{\mathcal{S}}}
\def\gT{{\mathcal{T}}}
\def\gU{{\mathcal{U}}}
\def\gV{{\mathcal{V}}}
\def\gW{{\mathcal{W}}}
\def\gX{{\mathcal{X}}}
\def\gY{{\mathcal{Y}}}
\def\gZ{{\mathcal{Z}}}

% Sets
\def\sA{{\mathbb{A}}}
\def\sB{{\mathbb{B}}}
\def\sC{{\mathbb{C}}}
\def\sD{{\mathbb{D}}}
% Don't use a set called E, because this would be the same as our symbol
% for expectation.
\def\sF{{\mathbb{F}}}
\def\sG{{\mathbb{G}}}
\def\sH{{\mathbb{H}}}
\def\sI{{\mathbb{I}}}
\def\sJ{{\mathbb{J}}}
\def\sK{{\mathbb{K}}}
\def\sL{{\mathbb{L}}}
\def\sM{{\mathbb{M}}}
\def\sN{{\mathbb{N}}}
\def\sO{{\mathbb{O}}}
\def\sP{{\mathbb{P}}}
\def\sQ{{\mathbb{Q}}}
\def\sR{{\mathbb{R}}}
\def\sS{{\mathbb{S}}}
\def\sT{{\mathbb{T}}}
\def\sU{{\mathbb{U}}}
\def\sV{{\mathbb{V}}}
\def\sW{{\mathbb{W}}}
\def\sX{{\mathbb{X}}}
\def\sY{{\mathbb{Y}}}
\def\sZ{{\mathbb{Z}}}

% Entries of a matrix
\def\emLambda{{\Lambda}}
\def\emA{{A}}
\def\emB{{B}}
\def\emC{{C}}
\def\emD{{D}}
\def\emE{{E}}
\def\emF{{F}}
\def\emG{{G}}
\def\emH{{H}}
\def\emI{{I}}
\def\emJ{{J}}
\def\emK{{K}}
\def\emL{{L}}
\def\emM{{M}}
\def\emN{{N}}
\def\emO{{O}}
\def\emP{{P}}
\def\emQ{{Q}}
\def\emR{{R}}
\def\emS{{S}}
\def\emT{{T}}
\def\emU{{U}}
\def\emV{{V}}
\def\emW{{W}}
\def\emX{{X}}
\def\emY{{Y}}
\def\emZ{{Z}}
\def\emSigma{{\Sigma}}

% entries of a tensor
% Same font as tensor, without \bm wrapper
\newcommand{\etens}[1]{\mathsfit{#1}}
\def\etLambda{{\etens{\Lambda}}}
\def\etA{{\etens{A}}}
\def\etB{{\etens{B}}}
\def\etC{{\etens{C}}}
\def\etD{{\etens{D}}}
\def\etE{{\etens{E}}}
\def\etF{{\etens{F}}}
\def\etG{{\etens{G}}}
\def\etH{{\etens{H}}}
\def\etI{{\etens{I}}}
\def\etJ{{\etens{J}}}
\def\etK{{\etens{K}}}
\def\etL{{\etens{L}}}
\def\etM{{\etens{M}}}
\def\etN{{\etens{N}}}
\def\etO{{\etens{O}}}
\def\etP{{\etens{P}}}
\def\etQ{{\etens{Q}}}
\def\etR{{\etens{R}}}
\def\etS{{\etens{S}}}
\def\etT{{\etens{T}}}
\def\etU{{\etens{U}}}
\def\etV{{\etens{V}}}
\def\etW{{\etens{W}}}
\def\etX{{\etens{X}}}
\def\etY{{\etens{Y}}}
\def\etZ{{\etens{Z}}}

% The true underlying data generating distribution
\newcommand{\pdata}{p_{\rm{data}}}
\newcommand{\ptarget}{p_{\rm{target}}}
\newcommand{\pprior}{p_{\rm{prior}}}
\newcommand{\pbase}{p_{\rm{base}}}
\newcommand{\pref}{p_{\rm{ref}}}

% The empirical distribution defined by the training set
\newcommand{\ptrain}{\hat{p}_{\rm{data}}}
\newcommand{\Ptrain}{\hat{P}_{\rm{data}}}
% The model distribution
\newcommand{\pmodel}{p_{\rm{model}}}
\newcommand{\Pmodel}{P_{\rm{model}}}
\newcommand{\ptildemodel}{\tilde{p}_{\rm{model}}}
% Stochastic autoencoder distributions
\newcommand{\pencode}{p_{\rm{encoder}}}
\newcommand{\pdecode}{p_{\rm{decoder}}}
\newcommand{\precons}{p_{\rm{reconstruct}}}

\newcommand{\laplace}{\mathrm{Laplace}} % Laplace distribution

\newcommand{\E}{\mathbb{E}}
\newcommand{\Ls}{\mathcal{L}}
\newcommand{\R}{\mathbb{R}}
\newcommand{\emp}{\tilde{p}}
\newcommand{\lr}{\alpha}
\newcommand{\reg}{\lambda}
\newcommand{\rect}{\mathrm{rectifier}}
\newcommand{\softmax}{\mathrm{softmax}}
\newcommand{\sigmoid}{\sigma}
\newcommand{\softplus}{\zeta}
\newcommand{\KL}{D_{\mathrm{KL}}}
\newcommand{\Var}{\mathrm{Var}}
\newcommand{\standarderror}{\mathrm{SE}}
\newcommand{\Cov}{\mathrm{Cov}}
% Wolfram Mathworld says $L^2$ is for function spaces and $\ell^2$ is for vectors
% But then they seem to use $L^2$ for vectors throughout the site, and so does
% wikipedia.
\newcommand{\normlzero}{L^0}
\newcommand{\normlone}{L^1}
\newcommand{\normltwo}{L^2}
\newcommand{\normlp}{L^p}
\newcommand{\normmax}{L^\infty}

\newcommand{\parents}{Pa} % See usage in notation.tex. Chosen to match Daphne's book.

\DeclareMathOperator*{\argmax}{arg\,max}
\DeclareMathOperator*{\argmin}{arg\,min}

\DeclareMathOperator{\sign}{sign}
\DeclareMathOperator{\Tr}{Tr}
\let\ab\allowbreak


\usepackage{hyperref}
\usepackage{url}
\usepackage{cleveref}
\usepackage{booktabs}
\usepackage{multirow}
\usepackage{subcaption}
\usepackage{adjustbox} % To adjust table sizes
\usepackage{float}

% \iclrfinalcopy

% For table
\usepackage{multirow}

% For figures
\usepackage{graphicx}
% \usepackage[table]{xcolor}
\title{Improved Training Technique for Latent Consistency Models}

% Authors must not appear in the submitted version. They should be hidden
% as long as the \iclrfinalcopy macro remains commented out below.
% Non-anonymous submissions will be rejected without review.
\iclrfinalcopy

\author{Quan Dao$^{*\dagger}$\\
Rutgers University \\
\texttt{quan.dao@rutgers.edu} \\ \And 
Khanh Doan$^{*}$\\
Movian AI, Vietnam \\
\texttt{dnkhanh.k63.bk@gmail.com} \\ \And
Di Liu\\
Rutgers University \\
\texttt{di.liu@rutgers.edu} \\   \And
Trung Le\\
Monash University \\
\texttt{trunglm@monash.edu} \\   \And
Dimitris Metaxas\\
Rutgers University \\
\texttt{dnm@cs.rutgers.edu} \\
}


% The \author macro works with any number of authors. There are two commands
% used to separate the names and addresses of multiple authors: \And and \AND.
%
% Using \And between authors leaves it to \LaTeX{} to determine where to break
% the lines. Using \AND forces a linebreak at that point. So, if \LaTeX{}
% puts 3 of 4 authors names on the first line, and the last on the second
% line, try using \AND instead of \And before the third author name.

\newcommand{\fix}{\marginpar{FIX}}
\newcommand{\new}{\marginpar{NEW}}
\newcommand{\khanh}[1]{\textcolor{orange}{[Khanh: #1]}}
\newcommand{\quan}[1]{\textcolor{red}{[Quan: #1]}}
\newcommand{\diliu}[1]{\textcolor{purple}{[Di Liu: #1]}}
\newcommand{\trung}[1]{\textcolor{cyan}{[Trung: #1]}}
\newcommand{\metaxas}[1]{\textcolor{blue}{[Metaxas: #1]}}
\newcommand{\minisection}[1]{\noindent{\textbf{#1}}}


%\iclrfinalcopy % Uncomment for camera-ready version, but NOT for submission.
\begin{document}


\maketitle
\def\thefootnote{\textsuperscript{$*$}}\footnotetext{Equal contributions.}
\def\thefootnote{\textsuperscript{$\dagger$}}\footnotetext{Project Lead \& Corresponding Author.}

\begin{abstract}
Consistency models are a new family of generative models capable of producing high-quality samples in either a single step or multiple steps. Recently, consistency models have demonstrated impressive performance, achieving results on par with diffusion models in the pixel space. However, the success of scaling consistency training to large-scale datasets, particularly for text-to-image and video generation tasks, is determined by performance in the latent space. In this work, we analyze the statistical differences between pixel and latent spaces, discovering that latent data often contains highly impulsive outliers, which significantly degrade the performance of iCT in the latent space. To address this, we replace Pseudo-Huber losses with Cauchy losses, effectively mitigating the impact of outliers. Additionally, we introduce a diffusion loss at early timesteps and employ optimal transport (OT) coupling to further enhance performance. Lastly, we introduce the adaptive scaling-$c$ scheduler to manage the robust training process and adopt Non-scaling LayerNorm in the architecture to better capture the statistics of the features and reduce outlier impact. With these strategies, we successfully train latent consistency models capable of high-quality sampling with one or two steps, significantly narrowing the performance gap between latent consistency and diffusion models. The implementation is released here: \url{https://github.com/quandao10/sLCT/}
\end{abstract}

\section{Introduction}
In recent years, generative models have gained significant prominence, with models like ChatGPT excelling in language generation and Stable Diffusion \citep{rombach2021highresolution}. In computer vision, the diffusion model \citep{song2020score, song2019generative, ho2020denoising, sohl2015deep} has quickly popularized and dominated the Adversarial Generative Model (GAN) \citep{goodfellow2014generative}. It is capable of generating high-quality diverse images that beat SoTA GAN models \citep{dhariwal2021diffusion}. Additionally, diffusion models are easier to train, as they avoid the common pitfalls of training instability and the need for meticulous hyperparameter tuning associated with GANs. The application of diffusion spans the entire computer vision field, including text-to-image generation \citep{rombach2021highresolution, gu2022vector}, image editing \citep{meng2021sdedit, cyclediffusion, huberman2024edit, han2024proxedit, he2024dice}, text-to-3D generation \citep{poole2022dreamfusion, wang2024prolificdreamer}, personalization \citep{ruiz2022dreambooth, van2023anti, kumari2023multi} and control generation \citep{zhang2023adding, brooks2022instructpix2pix, zhangli2024layout}. Despite their powerful capabilities, they require thousands of function evaluations for sampling, which is computationally expensive and hinders their application in the real world. Numerous efforts have been made to address this sampling challenge, either by proposing new training frameworks \citep{xiao2021tackling, rombach2021highresolution} or through distillation techniques \citep{meng2023distillation, yin2024one, sauer2023adversarial, dao2024self}. However, methods like \citep{xiao2021tackling} suffer from low recall due to the inherent challenges of GAN training, while \citep{rombach2021highresolution} still requires multi-step sampling. Distillation-based approaches, on the other hand, rely heavily on pretrained diffusion models and demand additional training.

Recently, \citep{song2023consistency} introduced a new family of generative models called the consistency model. Compared to the diffusion model \citep{song2019generative, song2020score, ho2020denoising}, the consistency model could both generate high-quality samples in a single step and multi-steps. The consistency model could be obtained by either consistency distillation (CD) or consistency training (CT). In previous work \citep{song2023consistency}, CD significantly outperforms CT. However, the CD requires additional training budget for using pretrained diffusion, and its generation quality is inherently limited by the pretrained diffusion. Subsequent research \citep{song2023improved} improves the consistency training procedure, resulting in performance that not only surpasses consistency distillation but also approaches SoTA performance of diffusion models. Additionally, several works \citep{kim2023consistency, geng2024consistency} have further enhanced the efficiency and performance of CT, achieving significant results. However, all of these efforts have focused exclusively on pixel space, where data is perfectly bounded. In contrast, most large-scale applications of diffusion models, such as text-to-image or video generation, operate in latent space \citep{rombach2021highresolution, gu2022vector}, as training on pixel space for large-scale datasets is impractical. Therefore, to scale consistency models for large datasets, the consistency must perform effectively in latent space. This work addresses the key question: How well can consistency models perform in latent space? To explore this, we first directly applied the SoTA pixel consistency training method, iCT \citep{song2023improved}, to latent space. The preliminary results were extremely poor, as illustrated in \cref{fig:qualitative_ict}, motivating a deeper investigation into the underlying causes of this suboptimal performance. We aim to improve CT in latent space, narrowing the gap between the performance of latent consistency and diffusion.

We first conducted a statistical analysis of both latent and pixel spaces. Our analysis revealed that the latent space contains impulsive outliers, which, while accounting for a very small proportion, exhibit extremely high values akin to salt-and-pepper noise. We also drew a parallel between Deep Q-Networks (DQN) and the Consistency Model, as both employ temporal difference (TD) loss. This could lead to training instability compared to the Kullback-Leibler (KL) loss used in diffusion models. Even in bounded pixel space, the TD loss still contains impulsive outliers, which \citep{song2023improved} addressed by proposing the use of Pseudo-Huber loss to reduce training instability. As shown in \cref{fig:impulsive_noise}, the latent input contains extremely high impulsive outliers, leading to very large TD values. Consequently, the Pseudo-Huber loss fails to sufficiently mitigate these outliers, resulting in poor performance as demonstrated in \cref{fig:qualitative_ict}. To overcome this challenge, we adopt Cauchy loss, which heavily penalizes extremely impulsive outliers. Additionally, we introduce diffusion loss at early timesteps along with optimal transport (OT) matching, both of which significantly enhance the model's performance. Finally, we propose an adaptive scaling $c$ schedule to effectively control the robustness of the model, and we incorporate Non-scaling LayerNorm into the architecture. With these techniques, we significantly boost the performance of latent consistency model compared to the baseline iCT framework and bridge the gap between the latent diffusion and consistency training.

\section{Related Works} \label{related}
% \subsection{Diffusion Model and Fast Sampling Technique}
% Diffusion models \citep{song2020score, song2019generative, ho2020denoising} have recently been raised as the most powerful generative model and outperform GAN \citep{goodfellow2014generative} in many applications. Diffusion models can generate high-fidelity images and possess good mode coverage, allowing diverse samples compared to GAN. However, diffusion models require many function evaluations (NFEs) during inference time. This drawback hinders its application in the real world. Many works are trying to tackle this drawback and achieve promising results. They can be divided into two main research categories: training from scratch and building upon pretrained diffusion models. Following the first category, there are several works, such as DDGAN \citep{xiao2021tackling}, LDM \citep{rombach2021highresolution}, and VQDiff \citep{gu2022vector}. DDGAN \citep{xiao2021tackling} proposes to use a larger step size in the forward process to reduce the NFEs; they use GAN models to implicitly learn the backward transition. Even though DDGAN \citep{xiao2021tackling} requires only a few sampling timesteps, it still suffers from low recall due to mode collapse from GAN. LDM \citep{rombach2021highresolution} does not directly reduce the number of sampling time steps; instead, it compresses the image to latent with a much smaller resolution. By training on latent space, LDM \citep{rombach2021highresolution} can significantly reduce both time and memory budget, and the inference is much faster than other pixel diffusion models. LDM \citep{rombach2021highresolution} has become the core technique in many large-scale diffusion models. Most real-world applications rely on LDM since it allows to scale diffusion models up on enormous high-resolution training datasets, which is impractical if training pixel diffusion models. In the second category, several works such as \citep{lu2022dpm, zhang2022fast} propose the high-order solver during inference. These works could successfully reduce sample NFEs to 10 without any training. However, they cannot sample with little NFEs such as 1 or 2. The other line of work is a distillation-based method. Progressive distillation \citep{salimans2022progressive} proposes to progressively distill diffusion model; each stage distills to reduce the sampling NFEs by half. This technique is costly since it requires to train many stages. Later works such as Guided-Distill \citep{meng2023distillation}, Swiftbrush \citep{nguyen2024swiftbrush}, DMD \citep{yin2024one}, and UFOGEN \citep{xu2024ufogen} manage to distill diffusion into few-step generation without compromising generative quality. The major drawback of these techniques is that additional training is required.
% Furthermore, some techniques, such as Swiftbrush \citep{nguyen2024swiftbrush} and DMD \citep{yin2024one}, do not have few-step sampling. Other methods, such as UFOGEN \citep{xu2024ufogen} and ADD \citep{sauer2023adversarial}, require training GAN, which could lead to training instability and low mode coverage. A standout among these techniques is the consistency model. The consistency model \citep{song2023consistency} is defined based on probability flow ODE (PF-ODE), allowing single- and multi-step sampling. The consistency model could be achieved via training from scratch and distillation from the diffusion model.

% \subsection{Consistency Model}

Consistency model \citep{song2023consistency, song2023improved} proposes a new type of generative model based on PF-ODE, which allows 1, 2 or multi-step sampling. The consistency model could be obtained by either training from scratch using an unbiased score estimator or distilling from a pretrained diffusion model. Several works improve the training of the consistency model. ACT \citep{kong2023act}, CTM \citep{kim2023consistency} propose to use additional GAN along with consistency objective. While these methods could improve the performance of consistency training, they require an additional discriminator, which could need to tune the hyperparameters carefully. MCM \citep{heek2024multistep} introduces multistep consistency training, which is a combination of TRACT \citep{berthelot2023tract} and CM \citep{song2023consistency}. MCM increases the sampling budget to 2-8 steps to tradeoff with efficient training and high-quality image generation. ECM \citep{geng2024consistency} initializes the consistency model by pretrained diffusion model and fine-tuning it using the consistency training objective. ECM vastly achieves improved training times while maintaining good generation performance. However, ECM requires pretrained diffusion model, which must use the same architecture as the pretrained diffusion architecture. Although these works successfully improve the performance and efficiency of consistency training, they only investigate consistency training on pixel space. As in the diffusion model, where most applications are now based on latent space, scaling the consistency training \citep{song2023consistency, song2023improved} to text-to-image or higher resolution generation requires latent space training. Otherwise, with pretrained diffusion model, we could either finetune consistency training \citep{geng2024consistency} or distill from diffusion model \citep{song2023consistency, luo2023latent}. CM \citep{song2023consistency} is the first work proposing consistency distillation (CD) on pixel space. LCM \citep{luo2023latent} later applies consistency technique on latent space and can generate high-quality images within a few steps. However, LCM's generated images using 1-2 steps are still blurry \citep{luo2023latent}. Recent works, such as Hyper-SD \cite{ren2024hyper} and TCD \cite{zheng2024trajectory}, have introduced notable improvements to latent consistency distillation. TCD \cite{zheng2024trajectory} employed CTM \cite{kim2023consistency} instead of CD \cite{song2023consistency}, significantly enhancing the performance of the distilled student model. Building on this, Hyper-SD \cite{ren2024hyper} divided the Probability Flow ODE (PF-ODE) into multiple components inspired by Multistep Consistency Models (MCM) \cite{heek2024multistep}, and applied TCD \cite{zheng2024trajectory} to each segment. Subsequently, Hyper-SD \cite{ren2024hyper} merged these segments progressively into a final model, integrating human feedback learning and score distillation \cite{yin2024one} to optimize one-step generation performance.

\section{Preliminaries} \label{sec:bg}
Denote $\pdata(\rvx_0)$ as the data distribution, the forward diffusion process gradually adds Gaussian noise with monotonically increasing standard deviation $\sigma(t)$ for $t \in \{0,1,\dots,T\}$ such that $p_t(\rvx_t|\rvx_0) = \gN(\rvx_0, \sigma^2(t)\mI)$ and $\sigma(t)$ is handcrafted such that $\sigma(0) = \sigma_{\min}$ and $\sigma(T)=\sigma_{\max}$. By setting $\sigma(t) = t$, the probability flow ODE (PF-ODE) from \citep{Karras2022edm} is defined as:
\begin{equation}
    \frac{\rd\rvx_t}{\rd t} = -t\nabla_{\rvx_t} \log p_t(\rvx_t) = \frac{\left( \rvx_t - \vf(\rvx_t, t) \right)}{t},  \label{eq:pf_ode}
\end{equation}
where $\vf:(\rvx_t, t) \rightarrow \rvx_0$ is the denoising function which directly predicts clean data $\rvx_0$ from given perturbed data $\rvx_t$. 
\citep{song2023consistency} defines consistency model based on PF-ODE in \cref{eq:pf_ode}, which builds a bijective mapping $\vf$ between noisy distribution $p(\rvx_t)$ and data distribution $\pdata(\rvx_0)$. The bijective mapping $\vf:(\rvx_t, t) \rightarrow \rvx_0$ is termed the consistency function. A consistency model $\vf_\theta(\rvx_t, t)$ is trained to approximate this consistency function $\vf(\rvx_t, t)$. The previous works \citep{song2023consistency, song2023improved, Karras2022edm} impose the boundary condition by parameterizing the consistency model as:
\begin{equation}
    \vf_\theta(\rvx_t, t) = c_{skip}(t)\rvx_t + c_{out}(t)\mF_\theta(\rvx_t, t), \label{eq:cm_param}
\end{equation}
where $\mF_\theta(\rvx_t, t)$ is a neural network to train. Note that, since $\sigma(t) = t$, we hereafter use $t$ and $\sigma$ interchangeably. $c_{skip}(t)$ and $c_{out}(t)$ are time-dependent functions such that $c_{skip}(\sigma_{\min}) = 1$ and $c_{out}(\sigma_{\max}) = 0$.

To train or distill consistency model, \citep{song2023consistency, song2023improved, Karras2022edm} firstly discretize the PF-ODE using a sequence of noise levels $\sigma_{\min} = t_{\min} = t_1 < t_2 < \dots < t_{N} = t_{\max} = \sigma_{\max}$, where $t_i = \left( t_{\min}^{1/\rho} + \frac{i-1}{N-1}(t_{\max}^{1/\rho
} - t_{\min}^{1/\rho})\right)^\rho$ and $\rho = 7$. 

\textbf{Consistency Distillation} Given the pretrained diffusion model $\vs_\phi(\rvx_t, t) \approx \nabla_{\rvx_t} \log p_t(\rvx_t)$, the consistency model could be distilled from the pretrained diffusion model using the following CD loss:
\begin{equation}
    \gL_{\text{CD}}(\theta, \theta^-) = \E\left[ \lambda(t_i)d(\vf_\theta(\rvx_{t_{i+1}}, t_{i+1}), \vf_{\theta^{-}}(\Tilde{\rvx}_{t_i}, t_{i})) \right], \label{loss:cd}
\end{equation}
where $\rvx_{t_{i+1}} = \rvx_0 + t_{i+1} \rvz$ with the $\rvx_0 \sim \pdata(\rvx_0)$ and $\rvz \sim \gN(0, \mI)$ and $\rvx_{t_i} = \rvx_{t_{i+1}} - (t_{i}-t_{i+1})t_{i+1} \nabla_{\rvx_{t_{i+1}}} \log p_{t_{i+1}}(\rvx_{t_{i+1}}) = \rvx_{t_{i+1}} - (t_{i}-t_{i+1})t_{i+1}\vs_\phi(\rvx_{t_{i+1}}, t_{i+1})$. 

\textbf{Consistency Training}
The consistency model is trained by minimizing the following CT loss:
\begin{equation}
    \gL_{\text{CT}}(\theta, \theta^-) = \E\left[ \lambda(t_i)d(\vf_\theta(\rvx_{t_{i+1}}, t_{i+1}), \vf_{\theta^{-}}(\rvx_{t_i}, t_{i})) \right], \label{loss:ct}
\end{equation}
where $\rvx_{t_i} = \rvx_0 + t_{i} \rvz$ and $\rvx_{t_{i+1}} = \rvx_0 + t_{i+1} \rvz$ with the same $\rvx_0 \sim \pdata(\rvx_0)$ and $\rvz \sim \gN(0, \mI)$

In \cref{loss:cd} and \cref{loss:ct}, $\vf_\theta$ and $\vf_{\theta^-}$ are referred to as the online network and the target network, respectively. The target's parameter $\theta^-$ is obtained by applying the Exponential Moving Average (EMA) to the student's parameter $\theta$ during the training and distillation as follows:
\begin{equation}
    \theta^- \leftarrow \text{stopgrad}(\mu\theta^- + (1-\mu)\theta), \label{ema}
\end{equation}
with $0\leq\mu<1$ as the EMA decay rate,  weighting function $\lambda(t_i)$ for each timestep $t_i$, and $d(\cdot, \cdot)$ is a predefined metric function. 

In CM \citep{song2023consistency}, the consistency training still lags behind the consistency distillation and diffusion models. iCT \citep{song2023improved} later propose several improvements that significantly boost the training performance and efficiency. First, the EMA decay rate $\mu$ is set to $0$ for better training convergence. Second, the Fourier scaling factor of noise embedding and the dropout rate are carefully examined. Third, iCT introduces Pseudo-Huber losses to replace $L_2$ and LPIPS since LPIPS introduces the undesirable bias in generative modeling \citep{song2023improved}. Furthermore, the Pseudo-Huber is more robust to outliers since it imposes a smaller penalty for larger errors than the $L_2$ metric. Fourth, iCT proposes an exp curriculum for total discretization steps N, which doubles N after a predefined number of training iterations. Moreover, uniform weighting $\lambda(t_i) = 1$ is replaced by $\lambda(t_i)=1/(t_{i+1}-t_i)$. Finally, iCT adopts a discrete Lognormal distribution for timestep sampling as EDM \citep{Karras2022edm}. With all these improvements, CT is now better than CD and performs on par with the diffusion models in pixel space.

\section{Method}
\label{method}
In this paper, we first investigate the underlying reason behind the performance discrepancy between latent and pixel space using the same training framework in \cref{sec:analysis}. Based on the analysis, we find out the root of unsatisfied performance on latent space could be attributed to two factors: the impulsive outlier and the unstable temporal difference (TD) for computing consistency loss. To deal with impulsive outliers of TD on pixel space, \citep{song2023improved} proposes the Pseudo-Huber function as training loss. For the latent space, the impulsive outlier is even more severe, making Pseudo-Huber loss not enough to resist the outlier. Therefore,  \cref{sec:cauchy} introduces Cauchy loss, which is more effective with extreme outliers. In the next \cref{sec:diff_loss} and \cref{sec:ot}, we propose to use diffusion loss at early timesteps and OT matching for regularizing the overkill effect of consistency at the early step and training variance reduction, respectively. Section \ref{sec:c} designs an adaptive scheduler of scaling $c$ to control the robustness of the proposed loss function more carefully, leading to better performance. Finally, in \cref{sec:norm}, we investigate the normalization layers of architecture and introduce Non-scaling LayerNorm to both capture feature statistic better and reduce the sensitivity to outliers.

\subsection{Analysis of latent space} \label{sec:analysis}

We first reimplement the iCT model \citep{song2023improved} on the latent dataset CelebA-HQ $32 \times 32 \times 4$ and pixel dataset Cifar-10 $32 \times 32 \times 3$. Hereafter, we refer to the latent iCT model as iLCT. We find that iCT framework works well on pixel datasets as claim \citep{song2023improved}. However, it produces worse results on latent datasets as in \cref{fig:qualitative_ict} and \cref{tab:main_exp}. The iLCT gets a very high FID above 30 for both datasets, and the generative images are not usable in the real world. This observation raises concern about the sensitivity of CT algorithm with training data, and we should carefully examine the training dataset. In addition, we notice that the DQN and CM use the same TD loss, which update the current state using the future state. Furthermore, they also possess the training instability. This motivates to carefully examine the behavior of TD loss with different training data.


While the pixel data lies within the range $[-1, 1]$ after being normalized, the range of latent data varies depending on the encoder model, which is blackbox and unbound. After normalizing latent data using mean and variance, we observe that the latent data contains high-magnitude values. We call them the impulsive outliers since they account for small probability but are usually very large values. In the bottom left of \cref{fig:impulsive_noise}, the impulsive outlier of latent data is red, spanning from $-9$ to $7$, while the first and third quartiles are just around $-1.4$ and $1.4$, respectively. We evaluate how the iCT will be affected by data outliers by analyzing the temporal difference $\text{TD} = f_\theta(\rvx_{t_{i+1}}, t_{i+1})-f_{\theta^-}(\rvx_{t_i}, t_{i})$. In the top right of \cref{fig:impulsive_noise}, the impulsive outliers of pixel TD range from -1.5 to 1.7, which are not too far from the interquartile range compared to latent TD. The impulsive outliers of latent TD range is much wider from -3.2 to 5. iCT uses Pseudo-Huber loss instead of $L_2$ loss since the Huber is less sensitive to outliers, see \cref{fig:loss}. However, for latent data, the Huber's reduction in sensitivity to outliers is not enough. This indicates that even using Pseudo-Huber loss, the iLCT training on latent space could still be unstable and lead to worse performance, which matches our experiment results on iLCT. Based on the above analysis, we hypothesize that the TD value statistic highly depends on the training data statistic.

\begin{figure}[!t]
    \centering
    \includegraphics[width=0.85\linewidth]{figures/impulsive_noise.pdf}
    \caption{\textbf{Box and Whisker Plot:} Impulsive noise comparison between pixel and latent spaces. The right column shows the statistics of TD values at 21 discretization steps. Other discretization steps exhibit same behavior, where impulsive outliers are consistently present regardless of the total discretization steps. The blue boxes represent interquartile ranges of the data, while the green and orange dashed lines indicate inner and outer fences, respectively. Outliers are marked with red dots.}
    \label{fig:impulsive_noise}
\end{figure}

%To understand the root of TD's impulsive outlier, we look into Deep Q Learning (DQN) from Reinforcement Learning (RL). There is a strong correlation between DQN and CM. While the DQN uses Q-value of future state as the ground truth for Q-value of the current state, the CM updates the current timesteps $f(\rvx_{t_{i+1}}, t_{i+1})$ using the smaller timesteps $f(\rvx_{t_{i}}, t_{i})$. This loss type is called the temporal difference in RL. For stable training, DQN uses target network $\theta^-$ to estimate Q-value of future state, and CM similarly adopts the same technique for consistency loss. The target network $\theta^-$ could be updated differently, such as Polyak-averaging, periodic, and standard TD updates \citep{lee2019target}. The Polyak-averaging update is simply EMA update using in \citep{song2023consistency}, and standard update corresponds to \citep{song2023improved} which $\theta^- \leftarrow \theta$ every iteration. The periodic update is a standard update but after every fixed number of iterations. In RL, Polyak-averaging and periodic updates are more stable but slowly convergent \citep{lee2019target}. Even using these stable target updates, there is still instability in TD training. Since the target network needs to change along with the online model, the target value of TD can never be fixed, which makes the loss highly oscillate. Therefore, even though the pixel data is very well-bounded within [-1, 1], the CM training is still affected by impulsive outliers due to the nature of TD loss.

To mitigate the impact of impulsive outliers, we could use more stable target updates like Polyak or periodic in TD loss \cite{lee2019target}, but they lead to very slow convergence, as shown in \citep{song2023consistency}. Even though CM is initialized by a pretrained diffusion model, the Polyak update still takes a long time to converge. Therefore, using Polyak or periodic updates is computationally expensive, and we keep the standard target update as in \citep{song2023improved}. Another direction is using a special metric for latent like LPIPS on pixel space \citep{song2023consistency}. \citep{kang2024diffusion2gan} proposes the E-LatentLPIPS as a metric for distillation and performs well on distillation tasks. However, this requires training a network as a metric and using this metric during the training process will also increase the training budget. To avoid the overhead of the training, we seek a simple loss function like Pseudo-Huber but be more effective with outliers. We find that the Cauchy loss function \citep{black1996robust, barron2019general} could be a promising candidate in place of Pseudo-Huber for latent space.
\subsection{Cauchy Loss against Impulsive Outlier} \label{sec:cauchy}
In this section, we introduce the Cauchy loss \citep{black1996robust, barron2019general} function to deal with extreme impulsive outliers. The Cauchy loss function has the following form:
\begin{equation}
    d_{\text{Cauchy}}(\rvx, \rvy)=  \log \left(1+\frac{||\rvx-\rvy||_2^2}{2c^2}\right), \label{loss:cauchy}
\end{equation}
and we also consider two additional robust losses, which are Pseudo-Huber \citep{song2023improved, barron2019general} and Geman-McClure \citep{geman1986bayesian, barron2019general}
\begin{equation}
    d_{\text{Pseudo-Huber}}(\rvx, \rvy)= \sqrt{||\rvx-\rvy||_2^2 + c^2} - c, \label{loss:huber}
\end{equation}
\begin{equation}
    d_{\text{Geman-McClure}}(\rvx, \rvy)= \frac{2||\rvx-\rvy||_2^2}{||\rvx-\rvy||_2^2 + 4c^2}, \label{loss:gm}
\end{equation}
where $c$ is the scaling parameter to control how robust the loss is to the outlier. We analyze their robustness behavior against outliers. As shown in \cref{fig:loss_val}, the Pseudo-Huber loss linearly increases like $L_1$ loss for the large residuals $\rvx-\rvy$. In contrast, the Cauchy loss only grows logarithmically, and the Geman-McClure suppresses the loss value to $1$ for the outliers. 

The Pseudo-Huber loss works well if the residual value does not grow too high and, therefore, has a good performance on the pixel space. However, for the latent space, as shown in the bottom right of \cref{fig:impulsive_noise}, the TD suffers from extremely high values coming from the impulsive outlier in the latent dataset, the Cauchy loss could be more suitable since it significantly dampens the influence of extreme outliers. Otherwise, even Geman-McClure is very highly effective for removing outlier effects than two previous losses; it gives a gradient $0$ for high TD value and completely ignores the impulsive outliers as \cref{fig:loss_derivative}. This is unexpected behavior because even though we call the high-value latent impulsive outlier, they actually could encode important information from original data. Completely ignoring them could significantly hurt the performance of training model. Based on this analysis, we choose Cauchy loss as the default loss for latent CM for the rest of the paper. The loss ablation is provided in \cref{tab:ablate_robust}.


\begin{figure}[!ht]
    \centering
    \begin{subfigure}[t]{0.40\textwidth}
        \centering
        \includegraphics[width=1.0\textwidth]{figures/func.png}
        \caption{Robust Loss}
        \label{fig:loss_val}
    \end{subfigure}%
    ~ 
    \begin{subfigure}[t]{0.40 \textwidth}
        \centering
        \includegraphics[width=1.0\textwidth]{figures/derivative.png}
        \caption{Derivative of Robust Loss}
        \label{fig:loss_derivative}
    \end{subfigure}
    \caption{Analysis of robust loss: Pseudo-Huber, Cauchy, and Geman-McClure}
    \label{fig:loss}
\end{figure}
% \vspace{-3mm}


\subsection{Diffusion Loss at small timestep} \label{sec:diff_loss}
For small noise level $\sigma$, the ground truth of $f(\rvx_\sigma, \sigma)$ can be well approximated by $\rvx_0$, but this does not hold for large noise levels. Therefore, for low-level noise, the consistency objective seems to be overkill and harms the model's performance since instead of optimizing $f_\theta(\rvx_\sigma, \sigma)$ to approximated ground truth $\rvx_0$, the consistency objective optimizes through a proxy estimator $f_{\theta^-}(\rvx_{<\sigma}, <\sigma)$ leading to error accumulation over timestep. To regularize this overkill, we propose to apply an additional diffusion loss on small noise level as follows:

\begin{equation}
    L_{diff} = ||f_\theta(\rvx_{t_i}, t_i) - \rvx_0||^2_2 \quad \forall i \leq \text{int(N $\cdot$ r)}, \label{loss:diff}
\end{equation}

where N is the number of training discretization steps and $r\in[0;1]$ is the diffusion threshold, and we heuristicly choose $r=0.25$. We do not apply diffusion loss for large noise levels since $f(\rvx_\sigma, \sigma)$ will differ greatly from the target $\rvx_0$, leading to very high $L_2$ diffusion loss. This could harm the training consistency process, misleading to the wrong solution. We provide the ablation study in \cref{tab:diff_loss}. Furthermore, CTM \citep{kim2023consistency} also proposes to use diffusion loss, but they use them on both high and low-level noise, which is different from us. 

\subsection{OT matching reduces the variance} \label{sec:ot}
% \vspace{-5mm}
In this section, we adopt the OT matching technique from previous works \citep{pooladian2023multisample, lee2023minimizing}. \citep{pooladian2023multisample} proposes to use OT to match noise and data in the training batch, such as the moving $L_2$ cost is optimal. On the other hand, \citep{lee2023minimizing} introduces $\beta\text{VAE}$ for creating noise corresponding to data and train flow matching on the defined data-noise pairs. By reassigning noise-data pairs, these works significantly reduce the variance during the diffusion/flow matching training process, leading to a faster and more stable training process. According to \citep{zhang2023emergence}, the consistency training and diffusion models produce highly similar images given the same noise input. Therefore, the final output solution of the consistency and diffusion models should be close to each other. Since OT matching helps reduce the variance during training diffusion, it could be useful to reduce the variance of consistency training. In our implementation, we follow \citep{pooladian2023multisample, tong2023improving} using the POT library to map from noise to data in the training batch. The overhead caused by minibatch OT is relatively small, only around $0.93\%$ training time, but gains significant performance improvement as shown in \cref{tab:strategy}.

\subsection{Adaptive $c$ scheduler} \label{sec:c}

% \begin{figure}[h!]
%     \centering
%     \includegraphics[width=0.5\linewidth]{figures/C_by_NFE.pdf}
%     \caption{Our robust adaptive $c$ scheduler}
%     \label{fig:proposed_c}
% \end{figure}

\begin{figure}[h!]
    \centering
    \includegraphics[width=0.8\linewidth]{figures/C_merge.pdf}
    \caption{Model convergence plot on different $c$ schedule. (Left) Our proposed $c$ values. Performance on FID (Middle) and Recall (Right) of our proposed $c$ in comparison with different choices.}
    \label{fig:fid_vary_c}
\end{figure}
% \vspace{-5mm}

In this section, we examine the choice of scaling parameter $c$ in robust loss functions. The scaling parameter controls the robustness level, which is very important for model performance. The previous work \citep{song2023improved} proposes to use fixed constant $c_0 = 0.00054\sqrt{d}$, where $d$ is the dimension of data. We find that using this simple fixed $c$ is not yet optimal for the training consistency model. Especially in this paper, we follow the Exp curriculum specified by \cref{exp_cur} in \citep{song2023improved}, which doubles the total discretization step after a defined number of training iterations. 
\begin{equation}
    \text{NFE}(k)=\min \left(s_0 2^{\left\lfloor\frac{k}{K^{\prime}}\right\rfloor}, s_1\right)+1, \quad K^{\prime}=\left\lfloor\frac{K}{\log _2\left\lfloor s_1 / s_0\right\rfloor+1}\right\rfloor, \label{exp_cur}
\end{equation}
where $k$ is current training iteration, $K$ is total training iteration and $s_0 = 10, s_1=640$. During training, we notice that the variance of TD is significantly reduced as doubling total discretization steps using \cref{exp_cur}. Since the more discretization steps, the closer distance of $\rvx_{t_i}$ and $\rvx_{t_{i+1}}$, the TD value's range between them should be smaller. However, the impulsive outlier still exists regardless of the number of discretization steps. Intuitively, we propose a heuristic adaptive $c$ scheduler where the $c$ is scaled down proportional to the reduction rate of TD variance as the number of discretization steps increases. We plot our $c$ scheduler versus discretization steps in \cref{fig:fid_vary_c} and we fit the $c$ scheduler to get the scheduler equation as following:

\begin{equation}
    c = \exp(-1.18 * \log(\text{NFE}(k) - 1) - 0.72) \label{eq:c_scheduler}
\end{equation}

\subsection{Non-scaling Layernorm} \label{sec:norm}
As mentioned in \cref{sec:analysis}, the statistic of training data could play an important role in the success of consistency training. Furthermore, in architecture design, the normalization layer specifically handles the statistics of input, output, and hidden features. In this section, we investigate the normalization layer choice for consistency training, which is sensitive to training data statistics. 

Currently, both \citep{song2023improved, song2023consistency} use the UNet architecture from \citep{dhariwal2021diffusion}. In UNet \citep{dhariwal2021diffusion}, GroupNorm is used in every layer by default. The GroupNorm only captures the statistics over groups of local channels, while the LayerNorm further captures the statistics' overall features. Therefore, LayerNorm is better at capturing fine-grained statistics over the entire feature. We further carry out the experiments for other types of normalization, such as LayerNorm, InstanceNorm, RMSNorm in \cref{tab:norm_layer} and observe that the GroupNorm and InstanceNorm perform relatively well compared to others, especially LayerNorm. This could be due to that they are less sensitive to the outliers since they only capture the statistic over groups of channels. Therefore, the impulsive features only affect the normalization of a group containing them. For the LayerNorm, the impulsive features could negatively impact the overall features's normalization. We further look into the LayerNorm implementation and suspect that the scaling term could significantly amplify the outliers across features by serving as a shared parameter. This observation is also mentioned in \citep{wei2022outlier} for LLM quantization. In implementation, we set the \textbf{scaling term of LayerNorm to $1$} and \textbf{disabled the gradient update} for it \eqref{operation:layernorm}. We refer to it as Non-scaling LayerNorm (NsLN) as \citep{wei2022outlier}.

\begin{equation}
    \text{LN}_{\gamma, \beta}(\rvx) = \frac{\rvx - u(\rvx)}{\sqrt{\sigma^{2}(\rvx) + \epsilon}} \cdot \gamma + \beta, \quad
    \text{NsLN}_{\beta}(\rvx) = \frac{\rvx - u(\rvx)}{\sqrt{\sigma^{2}(\rvx) + \epsilon}} + \beta, \label{operation:layernorm}
\end{equation}

where $u(\rvx)$ and $\sigma^{2}(\rvx)$ are mean and variance of $\rvx$.

% \subsection{Improve Consistency Distillation}

% \vspace{-15mm}
\section{Experiment} \label{exp}

\subsection{Performance of our training technique} \label{exp:main}
% \vspace{-3mm}
\begin{table}[t]
    \centering
    \begin{tabular}{cc}
        \begin{minipage}[c]{0.58\textwidth}
            \centering
            \begin{subtable}[t]{\textwidth}
                \resizebox{\textwidth}{!}{%
                \begin{tabular}{l c c c c c}
                    \toprule
                    Model & NFE$\downarrow$ & FID$\downarrow$ & Recall$\uparrow$ & Epochs & Total Bs\\
                    \midrule 
                    \multicolumn{5}{c}{\textbf{Pixel Diffusion Model}}\\
                    \midrule
                    WaveDiff \citep{phung2023wavediff} & 2 & 5.94 & 0.37 & 500 & 64\\
                    Score SDE \citep{song2020score} & 4000 & 7.23 & - & ~6.2K & - \\
                    DDGAN \citep{xiao2021tackling} & 2 & 7.64 & 0.36 & 800 & 32 \\
                    RDUOT \citep{dao2024high} & 2 & 5.60 & 0.38 & 600 & 24 \\
                    RDM \citep{teng2023relay} & 270 & 3.15 & 0.55 & 4K & - \\
                    UNCSN++ \citep{kim2021soft} & 2000 & 7.16 & - & - & -\\
                    \midrule 
                    \multicolumn{5}{c}{\textbf{Latent Diffusion Model}}\\
                    \midrule
                    LFM-8 \citep{dao2023flow} & 85 & 5.82 & 0.41 & 500 & 112\\ 
                    LDM-4 \citep{rombach2021highresolution} & 200 & 5.11 & 0.49 & 600 &48 \\
                    LSGM \citep{vahdat2021score} & 23 & 7.22 & - & 1K &-\\
                    DDMI \citep{park2024ddmi} & 1000 & 7.25 & - & - &-\\
                    
                    DIMSUM \citep{phung2024dimsum} & 73 & 3.76  & 0.56 & 395 &32\\
                    $\text{LDM-8}^\dagger$ & 250 & {8.85}  & - & 1.4K &128\\
                    
                    \midrule
                    \multicolumn{5}{c}{\textbf{Latent Consistency Model}}\\
                    \midrule
                    iLCT \citep{song2023improved} & 1 & 37.15 & 0.12 & 1.4K &128\\
                    iLCT \citep{song2023improved} & 2 & 16.84 & 0.24 & 1.4K &128\\
                    Ours  & 1 & 7.27 & 0.50 & 1.4K &128\\
                    Ours  & 2 & 6.93 & 0.52 & 1.4K &128\\
                    \bottomrule
                \end{tabular}%
                }
            \caption{CelebA-HQ}
            \label{tab:celeb}
            \end{subtable}
        \end{minipage}
        \hfill
        \begin{minipage}[c]{0.42\textwidth}
            \centering
            \begin{subtable}[t]{\textwidth}
                \resizebox{\textwidth}{!}{%
                \begin{tabular}{l c c c c c}
                    \toprule
                    Model & NFE$\downarrow$ & FID$\downarrow$ & Recall$\uparrow$ & Epochs & Total Bs\\
                    \midrule 
                    \multicolumn{5}{c}{\textbf{Pixel Diffusion Model}}\\
                    \midrule
                    WaveDiff \citep{phung2023wavediff} & 2 & 5.94 & 0.37 & 500 & 64\\
                    Score SDE \citep{song2020score} & 4000 & 7.23 & - &6.2K & -\\
                    DDGAN \citep{xiao2021tackling} & 2 & 5.25 & 0.36 & 500 & 32\\
                    \midrule 
                    \multicolumn{5}{c}{\textbf{Latent Diffusion Model}}\\
                    \midrule
                    LFM-8 \citep{dao2023flow} & 90 & 7.70 & 0.39 & 90 &112\\
                    LDM-8 \citep{rombach2021highresolution} & 400 & 4.02 & 0.52 & 400 &96\\
                    $\text{LDM-8}^\dagger$ & 250 & {10.81} & - & 1.8K &256\\
                    \midrule
                    \multicolumn{5}{c}{\textbf{Latent Consistency Model}}\\
                    \midrule
                    iLCT \citep{song2023improved} & 1 &52.45  &0.11  & 1.8K &256\\
                    iLCT \citep{song2023improved} & 2 &24.67  &0.17  & 1.8K &256\\
                    Ours  & 1 &8.87  &0.47  & 1.8K &256\\
                    Ours  & 2 &7.71  &0.48  & 1.8K &256\\
                    \bottomrule
                \end{tabular}%
                }
            \caption{LSUN Church}
            \label{tab:lsun}
            \end{subtable}
            \hfill
            \begin{subtable}[t]{\textwidth}
                \resizebox{\textwidth}{!}{%
                \begin{tabular}{l c c c c c}
                    \toprule
                    Model & NFE$\downarrow$ & FID$\downarrow$ & Recall$\uparrow$ & Epochs &Total Bs \\
                    \midrule 
                    \multicolumn{5}{c}{\textbf{Latent Diffusion Model}}\\
                    \midrule
                    LFM-8 \citep{dao2023flow} & 84 & 8.07 & 0.40 & 700 &128\\
                    LDM-4 \citep{rombach2021highresolution} & 200 & 4.98 & 0.50 & 400 &42\\
                    $\text{LDM-8}^\dagger$ & 250 &{10.23} & - & 1.4K &128\\
                    \midrule
                    \multicolumn{5}{c}{\textbf{Latent Consistency Model}}\\
                    \midrule
                    iLCT \citep{song2023improved} & 1 & 48.82  & 0.15 & 1.4K &128 \\
                    iLCT \citep{song2023improved} & 2 & 21.15 & 0.19 & 1.4K &128\\
                    Ours  & 1 & 8.72  &0.42 & 1.4K &128\\
                    Ours  & 2 & 8.29  &0.43  & 1.4K &128\\
                    \bottomrule
                \end{tabular}%
                }
            \caption{FFHQ}
            \label{tab:ffhq}
            \end{subtable}
        \end{minipage}
    \end{tabular}
    \caption{Our performance on CelebA-HQ, LSUN Church, FFHQ datasets at resolution $256 \times 256$. ($\dagger$) means training on our machine with the same diffusion forward and equivalent architecture.}
    \label{tab:main_exp}
\end{table}


\minisection{Experiment Setting:}
We measure the performance of our proposed technique on three datasets: CelebA-HQ \citep{celeba}, FFHQ \citep{karras2019style}, and LSUN Church \citep{lsun}, at the same resolution of $256 \times 256$. Following LDM \citep{rombach2021highresolution}, we use pretrained VAE KL-8 \footnote{https://huggingface.co/stabilityai/sd-vae-ft-ema} to obtain latent data with the dimensionality of $32 \times 32 \times 4$. We adopt the OpenAI UNet architecture \citep{dhariwal2021diffusion} as the default architecture throughout the paper. Furthermore, we use the variance exploding (VE) forward process for all the consistency and diffusion experiments following \citep{song2023consistency, song2023improved}. The baseline iCT is self-implemented based on official implementation CM \citep{song2023consistency} and iCT \citep{song2023improved}. We refer to this baseline as iLCT. Furthermore, we also train the latent diffusion model for each dataset using the same VE forward noise process for fair comparisons with our technique. This LDM model is referred to as $\text{LDM-8}^{\dagger}$ in \cref{tab:main_exp}. All three frameworks, including ours, iLCT, and $\text{LDM-8}^{\dagger}$, use the same architecture.

\minisection{Evaluation:} During the evaluation, we first generate 50K latent samples and then pass them through VAE's decoder to obtain the pixel images. We use two well-known metrics, Fréchet Inception Distance (FID) \citep{fid} and Recall \citep{kynkaanniemi2019improved}, for measuring the performance of the model given the training data and 50K generated images. 

\minisection{Model Performance:} We report the performance of our model across all three datasets in \cref{tab:main_exp}, primarily to compare it with the baseline iLCT \citep{song2023improved} and LDM \citep{rombach2021highresolution}. For both 1 and 2 NFE sampling, we observe that the FIDs of iLCT for all datasets are notably high (over 30 for 1-NFE sampling and over 16 for 2-NFE sampling), consistent with the qualitative results shown in \cref{fig:qualitative_ict}, where the generated image is unrealistic and contain many artifacts. This poor performance of iLCT in latent space is expected, as the Pseudo-Huber training losses are insufficient in mitigating extreme impulsive outliers, as discussed in \cref{sec:analysis} and \cref{sec:cauchy}. In contrast, our proposed framework demonstrates significantly better FID and Recall than iLCT. Specifically, we achieve 1-NFE sampling FIDs of 7.27, 8.87, and 8.29 for CelebA-HQ, LSUN Church, and FFHQ, respectively. For 2-NFE sampling, our FID scores improve across all three datasets. Notably, our 1-NFE sampling outperforms $\text{LDM-8}^{\dagger}$, using the same noise scheduler and architecture. However, our models still exhibit higher FIDs compared to LDM \citep{rombach2021highresolution} and LFM \citep{dao2023flow}. In contrast, we only need 1 or 2 timestep sampling, whereas they require multiple timesteps for high-fidelity generation.
 It's important to note that we employ the VE forward process, whereas these other methods use VP and flow-matching forward processes. Furthermore, the qualitative results of our framework, as shown in \cref{fig:qualitative_1nfe}, highlight our ability to generate high-quality images.

\begin{figure}[ht]
\centering
    \begin{subfigure}[b]{0.3\textwidth}
    \centering
    \includegraphics[width=\textwidth]{figures/lct_celeba.pdf}
    \caption{CelebA-HQ}
    \label{fig:qualitative_celeba}
    \end{subfigure}
    \hfill
    \begin{subfigure}[b]{0.3\textwidth}
    \centering
    \includegraphics[width=\textwidth]{figures/lct_church.pdf}
    \caption{LSUN Church}
    \label{fig:qualitative_lsun_church}
    \end{subfigure}
    \hfill
    \begin{subfigure}[b]{0.3\textwidth}
    \centering
    \includegraphics[width=\textwidth]{figures/ffhq.pdf}
    \caption{FFHQ}
    \label{fig:qualitative_ffhq}
    \end{subfigure}
    \caption{Our qualitative results using 1-NFE at resolution $256 \times 256$}
    \label{fig:qualitative_1nfe}
\end{figure}

\begin{figure}[ht]
\centering
    \begin{subfigure}[b]{0.3\textwidth}
    \centering
    \includegraphics[width=\textwidth]{figures/lct_celeba_baseline.pdf}
    \caption{CelebA-HQ}
    \label{fig:qualitative_ict_celeba}
    \end{subfigure}
    \hfill
    \begin{subfigure}[b]{0.3\textwidth}
    \centering
    \includegraphics[width=\textwidth]{figures/lsun_church.pdf}
    \caption{LSUN Church}
    \label{fig:qualitative_ict_lsun_church}
    \end{subfigure}
    \hfill
    \begin{subfigure}[b]{0.3\textwidth}
    \centering
    \includegraphics[width=\textwidth]{figures/lct_ffhq_baseline.pdf}
    \caption{FFHQ}
    \label{fig:qualitative_ict_ffhq}
    \end{subfigure}
    \caption{iLCT qualitative results using 1-NFE at resolution  $256 \times 256$}
    \label{fig:qualitative_ict}
\end{figure}


\subsection{Ablation of proposed framework} \label{exp:ablation}

We ablate our proposed techniques on the CelebA-HQ $256\times256$ dataset, with all FID and Recall metrics measured using 1-NFE sampling. All models are trained for 1,400 epochs with the same hyperparameters. As shown in \cref{tab:strategy}, replacing Pseudo-Huber losses with Cauchy losses makes our model's training less sensitive to impulsive outliers, resulting in a significant FID reduction from $37.15$ to $13.02$. This demonstrates the effectiveness of Cauchy losses in handling extremely high-value outliers, as discussed in \cref{sec:cauchy}. Additionally, applying diffusion loss at small timesteps further reduces FID by approximately 4 points to $9.11$, as this loss term stabilizes the training process at small timesteps, as described in \cref{sec:diff_loss}. Introducing OT coupling during minibatch training reduces training variance, improving the FID to $8.89$. Notably, by replacing the fixed scaling term $c=c_0$, \citep{song2023improved} with an adaptive scaling schedule, our model achieves an additional FID reduction of more than 1 point, reaching $7.76$, highlighting the importance of the scaling term $c$ in robustness control. Finally, we propose using NsLN, which removes the scaling term from LayerNorm to handle outliers more effectively. NsLN captures feature statistics while mitigating the negative impact of outliers, resulting in our best FID of $7.27$.

\minisection{Robustness Loss} \label{exp:ablation:robust_loss}
To analyze the impact of different robust loss functions, we conduct an ablation study using our best settings but replace the Cauchy loss with alternatives such as L2, E-LatentLPIPS \cite{kang2024diffusion2gan}, the Huber and the Geman-McClure loss. The results, shown in \cref{tab:ablate_robust}, indicate that both Huber and Geman-McClure underperform compared to the Cauchy loss when applied in the latent space. This is because the Huber loss remains too sensitive to extremely impulsive outliers, while the Geman-McClure loss tends to ignore such outliers entirely, leading to a loss of important information. This behavior is also discussed in \cref{sec:cauchy}.

% \vspace{-2mm}
\begin{table}[h!]
    \centering
    \begin{tabular}{cc}
        \begin{minipage}[c]{0.40\textwidth}
            \centering
            
            \begin{subtable}[t]{\textwidth}
                \centering
                \begin{tabular}{lcc}
                    \toprule
                    Framework                      & FID $\downarrow$   & Recall $\uparrow$   \\
                    \midrule
                    iLCT                           & 37.15              & 0.12                \\
                    \midrule
                    Cauchy                         & 13.02              & 0.36                \\
                    + Diff                         & 9.11               & 0.41                \\
                    + OT                           & 8.89               & 0.42                \\
                    + Scaled $c$                   & 7.76               & 0.47                \\
                    + NsLN       & \textbf{7.27}               &\textbf{0.50}                \\
                    % \rowcolor{pink!60}+ NsLN       & 7.27               & 0.50                \\
                    \bottomrule
                \end{tabular}
                \caption{Components of proposed framework}
                \label{tab:strategy}
            \end{subtable}
            \hfill
            % \vspace{2mm}
            \begin{subtable}[t]{\textwidth}
                \centering
                \begin{tabular}{lcc}
                    \toprule
                    $r$           & FID $\downarrow$   & Recall $\uparrow$   \\
                    \midrule
                    1.0                    & 7.47               & 0.49                \\
                    0.6                      & 7.33               & 0.49                \\
                    % \rowcolor{pink!60}0.25   & 7.27               & 0.50                \\
                    0.25   & \textbf{7.27}               & \textbf{0.50}                \\
                    \bottomrule
                \end{tabular}
                \caption{Threshold using Diffusion loss}
                \label{tab:diff_loss}
            \end{subtable}
            
        \end{minipage}
        \hfill
        \begin{minipage}[c]{0.40\textwidth}
            \centering
            \begin{subtable}[t]{\textwidth}
                \centering
                \begin{tabular}{lcc}
                    \toprule
                    Loss                        & FID $\downarrow$   & Recall $\uparrow$   \\
                    \midrule
                     L2                          & 50.40              & 0.04                \\
                    E-LatentLPIPS               & 11.49              & 0.47                \\
                    \midrule
                    Huber                       & 9.97               & 0.44                \\
                    Geman McClure               & 11.28              & 0.44                \\
                    % \rowcolor{pink!60} Cauchy   & 7.27               & 0.50                \\
                    Cauchy   & \textbf{7.27}               & \textbf{0.50}                \\
                    \bottomrule
                \end{tabular}
                \caption{Robust losses.}
                \label{tab:ablate_robust}
            \end{subtable}
            \hfill
            % \vspace{2mm}
            \begin{subtable}[t]{\textwidth}
                \centering
                \begin{tabular}{lcc}
                    \toprule
                    Norm layer                             & FID $\downarrow$   & Recall $\uparrow$   \\
                    \midrule
                    $\text{GN}$                           & 7.76               & 0.47                \\
                    \midrule
                    % \midrule
                    IN                                      & 8.47               & 0.43                \\
                    % $\text{IN}^\dagger$                     & 8.03               & 0.46                \\
                    % \midrule
                    LN                                      & 9.05               & 0.46                \\
                    % $\text{LN}^\dagger$                     & 7.92               & 0.47                \\
                    % \midrule
                    RMS                                     & 8.96               & 0.46                \\
                    % $\text{RMS}^\dagger$                    & 7.62               & 0.47                \\
                    % \midrule
                    NsLN                 &\textbf{7.27}               &\textbf{0.50}                \\
                    % \rowcolor{pink!60} NsLN                 & 7.27               & 0.50                \\
                    % \rowcolor{white}$\text{NsLN}^\dagger$   & 7.64               & 0.47                \\
                    \bottomrule
                \end{tabular}
                \caption{Norm Layer}
                \label{tab:norm_layer}
            \end{subtable}
        \end{minipage}
    \end{tabular}
    \caption{Ablation Studies on CelebA-HQ $256\times256$ dataset at epoch 1400}
    \label{tab:ablation}
\end{table}

\minisection{Diffusion Threshold} \label{exp:ablation:diff_loss}
In this section, we explore the impact of varying the threshold for applying the diffusion loss function in combination with the consistency loss. We observe that using the diffusion loss at every timestep improves consistency training; however, it underperforms compared to applying the diffusion loss selectively at smaller timesteps such as $r=0.25$ as shown in \cref{tab:diff_loss}. This suggests that applying diffusion losses primarily at small noise levels improves performance as discussed \cref{sec:diff_loss}. At larger timesteps, the diffusion loss may conflict with the consistency loss, potentially guiding the model toward incorrect solutions, thereby reducing overall performance.



\minisection{Scaling term $c$ scheduler} \label{exp:ablation:vary_c}
In this section, we compare the performance of our adaptive scaling $c$ scheduler with the fixed scaling $c$ scheduler proposed in \citep{song2023improved}. Our model demonstrates better convergence with the proposed adaptive $c$ scheduler. The rationale behind this improvement lies in the fact that, as the discretization steps increases using the exponential curriculum, the value of the TD scales down. Despite the reduced TD value, impulsive outliers still persist. A fixed large scaling $c$ is not effective in handling these outliers. To address this, we scale $c$ down as discretization steps increases, which leads to better performance, as shown in \cref{fig:fid_vary_c}.


\minisection{Normalizing Layer} \label{exp:ablation:norm_layer}
We denote GN, IN, LN, RMS, and NsLN as GroupNorm, InstanceNorm, LayerNorm, RMSNorm, and Non-scaling LayerNorm, respectively. The baseline UNet architecture from \citep{dhariwal2021diffusion} uses GroupNorm by default. We replace the normalization layers in the baseline with each of these types and train the model on CelebA-HQ using the best settings. The results are reported in \cref{tab:norm_layer}. GN and IN only capture local statistics, making them more robust to outliers, as outliers in one region do not affect others. In contrast, LN captures statistics from all features, making it more vulnerable to outliers because an outlier affects all features through a shared scaling term. By removing the scaling term in LN, we obtain NsLN, which is both effective in capturing feature statistics and resistant to outliers. As shown in \cref{tab:norm_layer}, NsLN outperforms the second-best GN by 0.5 FID and significantly outperforms LN.
\section{Conclusion}
CT is highly sensitive to the statistical properties of the training data. In particular, when the data contains impulsive noise, such as latent data, CT becomes unstable, leading to poor performance. In this work, we propose using the Cauchy loss, which is more robust to outliers, along with several improved training strategies to enhance model performance. As a result, we can generate high-fidelity images from latent CT, effectively bridging the gap between latent diffusion models and consistency models. Future work could explore further improvements to the architecture, specifically by investigating normalization methods that reduce the impact of outliers. For example, removing the scaling term from group normalization or instance normalization may help mitigate outlier effects. Another promising future direction is the integration of this technique with Consistency Trajectory Models (CTM) \cite{kim2023consistency}, as CTM has demonstrated improved performance compared to traditional Consistency Models (CM) \cite{song2023consistency}.
\section*{Acknowledgements}
Research funded by research grants to Prof. Dimitris Metaxas from NSF: 2310966, 2235405, 2212301, 2003874, 1951890, AFOSR 23RT0630, and NIH 2R01HL127661.


\bibliography{iclr2025_conference}
\bibliographystyle{iclr2025_conference}

\newpage
\appendix
\section{Appendix}
We provide additional uncurated samples of our models for three datasets: CelebaA-HQ (\ref{fig:appendix:celeba_onestep}, \ref{fig:appendix:celeba_twostep}), LSUN Church (\ref{fig:appendix:lsun_onestep}, \ref{fig:appendix:lsun_twostep}), and FFHQ (\ref{fig:appendix:ffhq_onestep}, \ref{fig:appendix:ffhq_twostep}). We also provide additional uncurated samples of our models on CelebaA-HQ trained with L2 loss (\ref{fig:appendix:celeba_onestep_ilct_l2}) and E-LatentLPIPS loss (\ref{fig:appendix:celeba_onestep_ilct_elatentlpips}).

\begin{figure}[h]
\centering
    \includegraphics[width=0.6\textwidth]{figures/lct_celeba_more.pdf}
    \caption{One-step samples on CelebA-HQ $256 \times 256$}
    \label{fig:appendix:celeba_onestep}
\end{figure}

\begin{figure}[h]
\centering
    \includegraphics[width=0.6\textwidth]{figures/lct_celeba_more_2step.pdf}
    \caption{Two-step samples on CelebA-HQ $256 \times 256$}
    \label{fig:appendix:celeba_twostep}
\end{figure}

\begin{figure}[h]
\centering
    \includegraphics[width=0.6\textwidth]{figures/lct_lsun_more.pdf}
    \caption{One-step samples on LSUN Church $256 \times 256$}
    \label{fig:appendix:lsun_onestep}
\end{figure}

\begin{figure}[h]
\centering
    \includegraphics[width=0.6\textwidth]{figures/lct_lsun_more_2step.pdf}
    \caption{Two-step samples on LSUN Church $256 \times 256$}
    \label{fig:appendix:lsun_twostep}
\end{figure}

\begin{figure}[h]
\centering
    \includegraphics[width=0.6\textwidth]{figures/lct_ffhq_more.pdf}
    \caption{One-step samples on FFHQ $256 \times 256$}
    \label{fig:appendix:ffhq_onestep}
\end{figure}

\begin{figure}[h]
\centering
    \includegraphics[width=0.6\textwidth]{figures/lct_ffhq_more_2step.pdf}
    \caption{Two-step samples on FFHQ $256 \times 256$}
    \label{fig:appendix:ffhq_twostep}
\end{figure}


\begin{figure}[h]
\centering
    \includegraphics[width=0.6\textwidth]{figures/ilct_l2_celeba.pdf}
    \caption{One-step samples on CelebA-HQ $256 \times 256$ (L2 loss)}
    \label{fig:appendix:celeba_onestep_ilct_l2}
\end{figure}

\begin{figure}[h]
\centering
    \includegraphics[width=0.6\textwidth]{figures/ilct_latentlpips_celeba.pdf}
    \caption{One-step samples on CelebA-HQ $256 \times 256$ (E-LatentLPIPS loss)}
    \label{fig:appendix:celeba_onestep_ilct_elatentlpips}
\end{figure}

\end{document}

\bibliographystyle{iclr2025_conference}

\clearpage
\appendix
\section[Appendix A]{Quick Start Guide for Experimental Setup.}
\label{appa}

\textbf{Framework}: \texttt{PyTorch, Version 1.11.0}.

\textbf{Architecture}
\begin{itemize}
    \item \textbf{Model Type}: Standard ResNet-18 for CIFAR-10, ResNet-34 for CIFAR-100, and ResNet-101 for ImageNet-1k. We do not incorporate dropout.
\end{itemize}

\textbf{Parameters}
\begin{itemize}
    \item \textbf{Batch Size}: $\{64\}$ for CIFAR and $\{128\}$ for ImageNet-1k.
    \item \textbf{Training Epochs}: 200 epochs for CIFAR. 150 epochs for ImageNet.
    \item \textbf{Loss Function}: Utilizes the \texttt{CrossEntropyLoss} from the \texttt{nn} module.
\end{itemize}

\textbf{Dataset \& Pre-processing}
\begin{itemize}
    \item \textbf{Normalization}: We employ the \texttt{torchvision.transforms} module to adjust pixel values across all images, ensuring they scale uniformly within the 0 to 1 range.
    \item \textbf{Cropping}: We implement a random cropping strategy. Initially, optional padding is applied to each 32x32 image, from which we then extract random 32x32 crops.
    \item \textbf{Rotation}: The images are subject to random rotations with an allowable variation up to ±15 degrees to enhance model robustness against orientation changes.
    \item \textbf{Label Smoothing}: Label smoothing is not incorporated in our pipeline.
\end{itemize}

\section[Appendix B]{Details of Experiments}
\label{appb}
\renewcommand{\thesection}{B}

We provide comprehensive details on the experiments conducted to validate the effectiveness of the Instance-dependent Early Stopping (IES) method. The main results are presented in Section \ref{sec4}, Table \ref{tab 1} and Table \ref{tab 2}. Here, we elaborate on the experimental setup across various configurations, covering a wide range of settings typically employed in training deep learning models, including different network architectures, datasets, hyperparameters, and optimizers. Unless otherwise specified, the parameters and components remains consistent with the base model in Appendix \ref{appa}.

\subsection{Network Architectures}
\label{appb1}

\vspace{0.5em} 

\subsubsection{ResNet \citep{he2016deep}}
\begin{itemize}[leftmargin=*,nosep]
    \item \textbf{Variants}: \emph{ResNet-18, ResNet-34, ResNet-50, ResNet-101}.
    \item \textbf{Implementation}:
    \begin{itemize}[leftmargin=*,nosep, label=$\circ$]
        \item ResNet-18 and ResNet-50 for \emph{CIFAR-10}.
        \item ResNet-34 and ResNet-101 for \emph{CIFAR-100}.
        \item ResNet-34 and ResNet-101 for \emph{ImageNet-1k}.
    \end{itemize}
\end{itemize}

\vspace{0.5em}

\subsubsection{VGG-16 \citep{simonyan2014very}}
\begin{itemize}[leftmargin=*,nosep]
    \item \textbf{Implementation}:
    \begin{itemize}[leftmargin=*,nosep, label=$\circ$]
        \item Used for \emph{CIFAR-10}.
    \end{itemize}
\end{itemize}

\vspace{0.5em} 

\subsubsection{DenseNet-121 \citep{huang2017densely}}
\begin{itemize}[leftmargin=*,nosep]
    \item \textbf{Implementation}:
    \begin{itemize}[leftmargin=*,nosep, label=$\circ$]
        \item Used for \emph{CIFAR-100} and \emph{ImageNet-1k}.
    \end{itemize}
\end{itemize}

\vspace{0.5em} 

\clearpage

\subsection{Datasets}
\label{appb2}

\subsubsection{CIFAR-10 and CIFAR-100 \citep{krizhevsky2009learning}}
\begin{itemize}[leftmargin=*,nosep]
    \item \textbf{Description}: 10 classes (\emph{CIFAR-10}) and 100 classes (\emph{CIFAR-100}), 50,000 training and 10,000 test images each.
    \item \textbf{Preprocessing}: Normalization (mean and std), random cropping, horizontal flipping.
\end{itemize}

\subsubsection{ImageNet-1k \citep{deng2009imagenet}}
\begin{itemize}[leftmargin=*,nosep]
    \item \textbf{Description}: 1,000 classes, over 1 million labeled images.
    \item \textbf{Preprocessing}: Normalization (mean and std), random cropping, horizontal flipping.
\end{itemize}

\subsubsection{Caltech-101 \citep{li_andreeto_ranzato_perona_2022}}
\begin{itemize}[leftmargin=*,nosep]
    \item \textbf{Description}: 101 object categories
    \item \textbf{Preprocessing}: Normalization (mean and std), random cropping, horizontal flipping.
\end{itemize}

\vspace{0.5em} 

\subsection{Hyperparameters and Optimization}
\label{appb3}

\begin{itemize}[leftmargin=*,nosep]
    \item \textbf{Batch Sizes}: 64 for \emph{CIFAR} and \emph{Caltech-101}, and 128 for \emph{ImageNet-1k}.
    \item \textbf{$\delta$ settings}: base $\delta = 1e^{-3}$ for \emph{CIFAR}, and base $\delta = 1$ for \emph{ImageNet-1k}.
    \item \textbf{Optimizer settings}:
    For SGD, momentum=0.9, weight\_decay=5e-4.
    \begin{itemize}[leftmargin=*,nosep, label=$\circ$]
        \item SGD(F) - lr = 0.001.
        \item SGD(L) - lr = 0.1, scheduler:
        \\\texttt{LinearLR(\_,start\_factor=1,end\_factor=0.01,total\_iters=150)}.
        \item SGD(M) - lr = 0.1, scheduler: \\\texttt{MultiStepLR(\_, milestones=[50, 100], gamma=0.1)}.
        \item SGD(E) - lr = 0.1,  scheduler: \texttt{ExponentialLR(\_, gamma=0.96)}.
        \item Adam \citep{kingma2014adam} - lr = 0.001.
        \item AdamW \citep{loshchilov2017decoupled} - lr = 0.001, weight\_decay=0.01.
    \end{itemize}
    \item \textbf{Annealing} \citep{qin2023infobatch}: For the \emph{ImageNet-1k} task, we switch to using the full training data for the last 10\% of the training epochs to give better stability.
\end{itemize}


\subsection{Transfer Learning Experiments}
\label{appb5}

\textbf{Fine-tuning Setup}:

We selected the model checkpoints at the 100th epoch for ResNet-101/AdamW and DenseNet-121/AdamW follow settings from Table \ref{tab 2} experiments.
The models were fine-tuned using both the IES method and full-data training.
During fine-tuning, only the classification head of the models is updated, while the rest of the model parameters were frozen.

\textbf{Experimental Setup}:

The main experimental settings were consistent with those described in Appendix \ref{appa}.
The models were fine-tuned for 1 or 5 of epochs using the Adam optimizer with a learning rate of 0.001.
Notably, data augmentation techniques such as cropping and rotation were not applied, and all images were resized to a fixed resolution of 224x224. For the Caltech101 dataset, an additional preprocessing step is performed to convert grayscale images to RGB format.

Results for 1-epoch and 5-epoch fine-tuning are displayed in Table \ref{tabapp3} and Table \ref{tabapp4}, respectively.

\begin{table*}[h]
\vskip -0.05in
\centering
	\caption{transferability of IES-2nd Pretrained in ImageNet-1k. Fine-tuning 1 epoch. (mean±std) }
 \vskip -0.05in
	\label{tabapp3}
\resizebox{1\textwidth}{!}{
\setlength{\tabcolsep}{3.5mm}{
\begin{tabular}{c|cc|cc}
\toprule
& \multicolumn{2}{c|}{\emph{ResNet-101}} & \multicolumn{2}{c}{\emph{DenseNet-121}}\\
\cmidrule(lr){2-3}\cmidrule(lr){4-5}
  Transfer Task & IES (Ours) & No Removal & IES (Ours) & No Removal \\
\midrule
\emph{ImageNet-1k --> CIFAR-10} & \textbf{81.2\%$\pm$0.1\%} & 80.3\%$\pm$0.2\% & \textbf{78.6\% $\pm$ 0.2\%} & 77.3\% $\pm$ 0.2\% \\
\midrule
\emph{ImageNet-1k --> CIFAR-100} & \textbf{57.5\%$\pm$0.2\%} & 55.6\%$\pm$0.2\% & \textbf{53.0\% $\pm$ 0.2\%} & 52.3\% $\pm$ 0.2\% \\
\midrule
\emph{ImageNet-1k --> Caltech-101} & \textbf{59.9\%$\pm$0.8\%} & 57.4\%$\pm$1.2\% & \textbf{50.9\% $\pm$ 1.6\%} & 49.5\% $\pm$ 1.5\% \\
\bottomrule  
\end{tabular}
}
}
\vskip -0.05in
\end{table*}

\begin{table*}[h]
\vskip -0.05in
\centering
	\caption{transferability of IES-2nd Pretrained in ImageNet-1k. Fine-tuning 5 epochs. (mean±std) }
 \vskip -0.05in
	\label{tabapp4}
\resizebox{1\textwidth}{!}{
\setlength{\tabcolsep}{3.5mm}{
\begin{tabular}{c|cc|cc}
\toprule
 & \multicolumn{2}{c|}{\emph{DenseNet-121}} & \multicolumn{2}{c}{\emph{ResNet-101}}\\
\cmidrule(lr){2-3}\cmidrule(lr){4-5}
Transfer Task & IES (Ours) & No Removal & IES (Ours) & No Removal \\
\midrule
\emph{ImageNet-1k --> CIFAR-10} & \textbf{82.6\%$\pm$0.1\%} & 81.7\%$\pm$0.1\% & \textbf{85.6\% $\pm$ 0.1\%} & 84.6\% $\pm$ 0.1\% \\
\midrule
\emph{ImageNet-1k --> CIFAR-100} & \textbf{61.6\%$\pm$0.2\%} & 60.8\%$\pm$0.2\% & \textbf{66.0\% $\pm$ 0.1\%} & 64.4\% $\pm$ 0.2\% \\
\midrule
\emph{ImageNet-1k --> Caltech-101} & \textbf{91.2\%$\pm$0.2\%} & 90.6\%$\pm$0.3\% & \textbf{92.7\% $\pm$ 0.2\%} & 92.5\% $\pm$ 0.3\% \\
\bottomrule  
\end{tabular}
}
}
\vskip -0.05in
\end{table*}


\subsection{Experiments in Figure \ref{fig3}}
\label{appb6}

\textbf{Setup}:
\begin{itemize}[leftmargin=*,nosep, label=$\circ$]
\item The main experimental settings were consistent with those described in Appendix \ref{appa}.
\item 5 runs, mean±std.
\item Batch size: The batch size is 128.
\item Number of epochs: The models are trained for 150 epochs.
\item $\delta = 1e^{-4}$.
\end{itemize}
\textbf{Evaluation Metrics}:
\begin{itemize}[leftmargin=*,nosep, label=$\circ$]

\item \textbf{SAM (Sharpness-Aware Minimization):}

The SAM value is defined as the difference between the perturbed loss and the original loss \cite{foret2020sharpness}.
The important hyperparameter is rho, which represents the magnitude of the perturbation. In this work, rho is set to 0.05.

\vspace{0.5em}
\item \textbf{Gradient Norm:}

In the training loop, for each batch, calculates the gradient norm.
For each parameter p, its gradient norm is calculated as \texttt{p.grad.data.norm(2).item() $**$ 2}.
The total gradient norm is the square root of the sum of squares of all parameter gradient norms.
Gradient Norm in Figure is the average of gradient norms for all batches in an epoch.

\vspace{0.5em}
\item \textbf{Maximum Eigenvalue of the Hessian Matrix:}

The maximum eigenvalue is estimated using the power iteration \cite{mises1929praktische} method to estimate the largest eigenvalue of the Hessian matrix.

The important hyperparameters include:

- n\_iters: The number of iterations for the power iteration method, set to 20.

- epsilon: A small positive number for numerical stability, set to $1e^{-10}$.
\vspace{0.5em}
\item \textbf{Training Loss:} The average cross-entropy loss on the training set.
\vspace{0.5em}
\item \textbf{Test Error:} The percentage of misclassified samples in the test set.
\end{itemize}

\clearpage
\renewcommand{\thesection}{C}
\section{Coefficient of Variation}
\label{appc}



To further investigate the properties of different orders of loss differences as potential \emph{mastered} criteria, we conducted experiments to compare their coefficient of variation (CV). The CV is a standardized measure of dispersion, calculated as the ratio of the standard deviation to the mean:

$$CV = \frac{\sigma}{\mu},$$

where $\sigma$ is the standard deviation and $\mu$ is the mean of the data.
We compute the CV values for the zero-order (loss value), first-order, second-order and third-order differences of each sample's loss during training. A lower CV value indicates that the data points are clustered more closely around the mean, while a higher CV suggests greater dispersion.
Figure \ref{figc1} presents the CV values for different orders of loss differences over the course of training when using the Adam optimizer. The results show that the second-order difference and the third-order difference generally maintains lower CV values compared to the zero-order and first-order differences throughout the training process.

\begin{figure}[h]
\centering
\includegraphics[width=0.35\textwidth]{Unknown-78_rebuttal_iclr.png}
\caption{Coefficient of variation (CV) of different orders of loss differences during training. Using Adam optimizer, learning rate = 0.001.}
\label{figc1}
\end{figure}

Although the CV values do not converge to a low level in the later stages of training as observed with the SGD optimizer (results on SGD shown in Figure \ref{fignew}), the second-order difference and the third-order difference still exhibits significantly smaller CV values compared to the other orders. This suggests that the second-order difference provides a relatively more consistent measure of an instance's learning status across different samples, even when the CV values do not converge.
The lower CV values of the second-order difference and the third-order difference throughout the training process support the use of a unified threshold $\delta$ to determine the \emph{mastered} instances. This property simplifies the implementation and management of the mastered criterion in the IES method, as it allows for a more consistent approach to identifying \emph{mastered} instances across the entire dataset. 
Using the second-order difference ($N=2$) as the mastered criterion achieves good performance in most cases, as shown in Figure \ref{fig4}. $N=2$ outperformed other configurations (including $N=3$) in most scenarios. Given the satisfactory performance of $N=2$, the potential benefits of exploring higher-order differences ($N>3$) may be limited. The additional computational complexity introduced by higher-order differences may not yield significant improvements in the effectiveness of the IES method.

These experimental results provide evidence for the effectiveness of using the second-order difference as the \emph{mastered} criterion in the IES method, enabling a more efficient and generalizable approach to instance-dependent early stopping. 

\clearpage
\renewcommand{\thesection}{D}
\section{Compare with Varying Methods and Criteria}
\label{appd}

To evaluate the effectiveness of the proposed IES method and its different criteria, we conducted experiments comparing IES with other sample selection methods under various hyperparameter settings. Figure \ref{fig4} presents the results of these experiments on CIFAR-10 and CIFAR-100 datasets. It is worth noting that the hyperparameters were fine-tuned to manually set the methods and criteria  to have similar total backpropagation sample savings rates, making the methods comparable.

\subsubsection{Experiment: IES with Different Criteria, Hyperparameters and Comparison Methods}
\begin{itemize}[leftmargin=*,nosep]
    \item \textbf{Setup}:
    \begin{itemize}[leftmargin=*,nosep]
        \item Models: ResNet-18 for CIFAR-10, ResNet-34 for CIFAR-100
        \item Optimizers: SGD with momentum and exponential decay, the initial learning rate is set to 0.1, and the gamma parameter is set to 0.96
        \item Training Epoch: 200 for CIFAR
        \item Batch Size: 64 for CIFAR
        \vspace{0.4cm}
        \item Comparison Methods (CIFAR-10 and CIFAR-100):
        \begin{itemize}[leftmargin=*,nosep]
            \item \emph{Random Remove}: Randomly excludes a fixed proportion of samples from backpropagation in each training epoch. Removal rates: 10\%, 20\%, 30\%, 40\%, 50\%.
            \item \emph{Small Loss \& Reweight} \citep{qin2023infobatch}: Randomly removes samples with smaller loss values and amplifies the gradients of the remaining small-loss samples. To focus on the core idea of the method and ensure a simple and direct comparison with the proposed IES method, we removed the annealing and other additional operations from the original implementation. This modification allows us to evaluate the effectiveness of removing small-loss samples and amplifying their gradients in isolation, providing a clearer understanding of the differences between the two methods. Removal ratios: 10\% - 50\%. A comparison of the wall-time between IES method and InfoBatch method is provided in Figure \ref{figwalltime}.
        \end{itemize}
    \vspace{0.4cm}
    \end{itemize}
    \item \textbf{Results}:
    \begin{itemize}[leftmargin=*,nosep]
        \item IES with $N=2$ (2nd order difference) outperforms other criteria and sample selection methods in most cases, achieving a good balance between computational efficiency and model performance.
        \item The performance of IES is relatively stable across a wide range of $\delta$ values for each criterion.
        \item \emph{Random Remove} significantly reduces model performance, confirming the effectiveness of the IES method in selecting not-yet \emph{mastered} samples.
        \item \emph{Small Loss \& Rescale} improves results compared to \emph{Random Remove} but still falls behind IES.
    \end{itemize}
\end{itemize}
\vspace{0.3cm}
Figure \ref{fig4} visualizes the results of these experiments, comparing the test accuracy of different methods and criteria under varying Total Excluded Samples ratios.

\begin{figure*}[h]
 \vskip -0.02in
 \begin{subfigure}[b]{0.408\textwidth}
     \includegraphics[width=\textwidth]{figwalltime_cifar10.png}
     \vskip -0.02in
     \caption{\emph{CIFAR-10}}
     \label{figwalltime:sub1}
 \end{subfigure}
 \hfill
 \begin{subfigure}[b]{0.408\textwidth}
     \includegraphics[width=\textwidth]{figwalltime_cifar100.png}
     \vskip -0.02in
     \caption{\emph{CIFAR-100}}
     \label{figwalltime:sub2}
 \end{subfigure}
 \vskip -0.08in
\caption{Comparison of the wall-time between IES method and InfoBatch method}
 \label{figwalltime}
\vskip -0.02in
\end{figure*}



\renewcommand{\thesection}{E}
\section{High-Level Tasks}
\label{appg}

To further validate the general applicability of our perspective and method, we provide a comprehensive evaluation of our proposed Instance-dependent Early Stopping (IES) method across two distinct but equally important high-level vision tasks: object detection and image segmentation, providing a broader perspective on its potential applications in the field of computer vision.
Our experimental approach centered on integrating our proposed IES method into established baseline models for each task. Here's a detailed look at the experimental setup and result for each task:

We selected Faster R-CNN \citep{ren2015faster} as our baseline for object detection. Faster R-CNN is a two-stage detector that has shown remarkable performance in accurately identifying and localizing multiple objects within an image. 
For this experiment, we utilized the PASCAL VOC2007 \citep{pascal-voc-2007} dataset, and we implemented VGG-16 \citep{simonyan2014very} as the backbone network for feature extraction. 
Both the baseline method and the IES method were run for 30 epochs. We evaluate the best mAP value of the trained model and report the proportion of back-propagation mini-batches saved by the IES method.

\begin{table}[h]
\centering
\begin{tabular}{lcc}
\toprule
\multicolumn{3}{c}{\textbf{Object Detection}} \\
\midrule
& mAP (\%) & Mini-Batch Saved (\%) \\
\midrule
Baseline & 70.2 $\pm$ 0.2 & $\backslash$ \\
InfoBatch \citep{qin2023infobatch} & 69.9 $\pm$ 0.2 & 18.7 \\
IES (Ours) & 70.2 $\pm$ 0.1 & 20.0 \\
\bottomrule
\end{tabular}
\end{table}

For the task of image segmentation, we chose DeepLab v3 \citep{chen2017rethinking} as our baseline. DeepLab v3 is a state-of-the-art model for semantic segmentation, allowing the model to capture multi-scale contextual information effectively.
We employed the PASCAL VOC2012 \citep{pascal-voc-2012} dataset for this experiment, and we used ResNet-50 \citep{he2016deep} as the backbone network. Both the baseline method and the IES method were run for 50 epochs. We evaluate the best mIoU value of the trained model and report the proportion of back-propagation mini-batches saved by the IES method.

\begin{table}[h]
\centering
\begin{tabular}{lcc}
\toprule
\multicolumn{3}{c}{\textbf{Image Segmentation}} \\
\midrule
& mIoU (\%) & Mini-Batch Saved (\%) \\
\midrule
Baseline & 76.2 $\pm$ 0.2 & $\backslash$ \\
InfoBatch \citep{qin2023infobatch} & 76.0 $\pm$ 0.3 & 12.0 \\
IES (Ours) & 76.1 $\pm$ 0.2 & 14.0 \\
\bottomrule
\end{tabular}
\end{table}

\renewcommand{\thesection}{F}
\section{More Baseline Methods}
\label{appf}

We further compare the IES method with several other data efficient methods, including:
\begin{enumerate}
\item The conventional early stopping method.
\item The importance sampling method \citep{jiang2019accelerating}.
\item Curriculum learning methods \citep{zhou2020curriculum, wang2024efficienttrain++}.
\end{enumerate}

To evaluate the applicability of the IES method in scenarios where efficiency is the primary objective, we conducted comparisons using the same training parameters as the IES method (detailed in Section \ref{appd}). To further demonstrate the ability of these methods to accelerate training while tolerating a certain degree of model performance degradation, we reduced the total training epochs by half to 100 and set the target computational speedup to approximately 2.0 and 3.0 times. Under higher speedup ratios, we evaluate the loss on the full training set at five-epoch intervals to reduce the computational overhead of loss evaluation, thereby enabling more efficient training. The comparison is made based on the test accuracy achieved by each method's trained model.

\begin{table}[h]
\centering
\caption{Comparison of IES and other data efficiency methods. (3 runs, mean±std)}
\vskip -0.12in
\resizebox{1\textwidth}{!}{
\setlength{\tabcolsep}{3.5mm}
\renewcommand{\arraystretch}{0.9}
\begin{tabular}{c|c|cc}
\toprule
Computation Speedup & Methods & CIFAR-10 & CIFAR-100 \\
\midrule
$1.0\times$ & Baseline (No Removal) & 94.3\%$\pm$0.3\% & 77.0\%$\pm$0.4\% \\
\midrule
\multirow{5}{*}{$\sim2.0\times$} & Conventional Early Stopping & 90.4\%$\pm$0.5\% & 68.7\%$\pm$0.5\% \\
& SB \citep{jiang2019accelerating} & 93.0\%$\pm$0.1\% & 70.6\%$\pm$0.5\% \\
& DIHCL \citep{zhou2020curriculum} & 93.4\%$\pm$0.2\% & 74.3\%$\pm$0.2\% \\
& EfficientTrain \citep{wang2024efficienttrain++} & 91.5\%$\pm$0.2\% & \textbf{75.0\%$\pm$0.1\%} \\
\cmidrule(lr){2-2}\cmidrule(lr){3-4}
& IES (Ours) & \textbf{93.7\%$\pm$0.4\%} & \textbf{74.9\%$\pm$0.5\%} \\
\midrule

\midrule
\multirow{5}{*}{$\sim3.0\times$} & Conventional Early Stopping & 88.1\%$\pm$0.3\% & 63.9\%$\pm$1.0\% \\
& SB \citep{jiang2019accelerating} & 91.1\%$\pm$0.5\% & 65.8\%$\pm$0.3\% \\
& DIHCL \citep{zhou2020curriculum} & 92.7\%$\pm$0.1\% & 72.6\%$\pm$0.1\% \\
& EfficientTrain \citep{wang2024efficienttrain++} & 92.5\%$\pm$0.2\% & 70.6\%$\pm$0.7\% \\
\cmidrule(lr){2-2}\cmidrule(lr){3-4}
& IES (Ours) & \textbf{93.2\%$\pm$0.1\%} & \textbf{73.0\%$\pm$0.5\%} \\
\bottomrule
\end{tabular}
}
\vskip -0.0in
\label{tab4app}
\end{table}

As shown in Table \ref{tab4app}, these comparisons further demonstrate that IES, while not specifically designed for scenarios where efficient training is the primary objective, still performs effectively in accelerating model training while maintaining model performance. This can be attributed to its adaptively identifying and excluding \emph{mastered} samples during the training process.

\renewcommand{\thesection}{G}
\section{Label Noise}
\label{applabelnoise}

An analysis of learning with noisy labels \citep{yao2020dual, yao2021instance, wei2022self, yuan2023late, lin2024cs, linlearning, wumitigating} is crucial to evaluate the robustness and practicality of our proposed IES method.
To address this, we attempt to discuss this issue under Typical Learning with Noisy Label scenarios and Epoch-wise Double Descent scenarios, respectively.

\paragraph{Typical Learning with Noisy Labels.}

We validate the performance of the IES method and the baseline method (without removal) under typical learning with noisy labels settings, specifically, on the CIFAR-10/CIFAR-100 datasets with 20\% and 40\% symmetric and instance-dependent \citep{xia2020part} label noise.

\begin{table}[h]
\caption{Performance comparison on CIFAR-10 dataset with different noise settings.}
\label{tab:cifar10_noise}
\centering
\begin{tabular}{l c c c c}
\toprule
\multirow{2}{*}{Noise Ratio} & \multirow{2}{*}{Type} & \multicolumn{2}{c}{Best Accuracy [\text{Early Stopping Epoch}]} & \multirow{2}{*}{Mini-batch Saved} \\
\cmidrule(lr){3-4}
& & Baseline & IES & \\
\midrule
20\% & Symmetric & 87.81\% [21] & 87.81\% [21] & 0\% \\
40\% & Symmetric & 81.29\% [13] & 81.29\% [13] & 0\% \\
20\% & Instance & 87.09\% [22] & 87.09\% [22] & 0\% \\
40\% & Instance & 83.49\% [20] & 83.49\% [20] & 0\% \\
\bottomrule
\end{tabular}
\end{table}

\begin{table}[h]
\caption{Performance comparison on CIFAR-100 dataset with different noise settings.}
\label{tab:cifar100_noise}
\centering
\begin{tabular}{l c c c c}
\toprule
\multirow{2}{*}{Noise Ratio} & \multirow{2}{*}{Type} & \multicolumn{2}{c}{Best Accuracy [\text{Early Stopping Epoch}]} & \multirow{2}{*}{Mini-batch Saved} \\
\cmidrule(lr){3-4}
& & Baseline & IES & \\
\midrule
20\% & Symmetric & 55.39\% [17] & 55.39\% [17] & 0\% \\
40\% & Symmetric & 43.87\% [15] & 43.87\% [15] & 0\% \\
20\% & Instance & 57.30\% [18] & 57.30\% [18] & 0\% \\
40\% & Instance & 47.67\% [18] & 47.67\% [18] & 0\% \\
\bottomrule
\end{tabular}
\end{table}

The experimental results indicate that the IES method degenerates to the baseline method (without removal) across all tested label noise rates, noise types, and datasets. This suggests that during the training process, no training sample satisfies the master criterion before the model overfits to the noisy labels and its performance declines.

The core idea behind the IES method is that once a model has mastered a sample, it should stop training on that sample. However, when a certain proportion of label noise exists in the dataset, memorization of mislabeled samples may affect the model's ability to learn stable patterns, making it difficult for the model to truly master any samples before the early stopping point.

\paragraph{Epoch-wise Double Descent.}

Epoch-wise Double Descent refers to the phenomenon where, when the training samples contain a certain amount (usually low) of label noise, as training progresses, the model's generalization performance first rises, then falls, and then rises again, with the generalization performance after the second rise being superior to the first peak. In this label noise scenario, the model needs to prolong training to achieve better generalization performance compared to conventional early stopping. We validate the performance of the IES method and the baseline method (without removal) under typical Epoch-wise Double Descent settings, specifically, on the CIFAR-100 datasets with 10\% symmetric and instance-dependent label noise.

\begin{table}[h]
\caption{Performance comparison under Epoch-wise Double Descent settings on CIFAR-100.}
\label{tab:double_descent}
\centering
\begin{tabular}{l c c c c}
\toprule
\multirow{2}{*}{Noise Ratio} & \multirow{2}{*}{Type} & \multicolumn{2}{c}{Best Accuracy [\text{Epoch}]} & \multirow{2}{*}{Mini-batch Saved} \\
\cmidrule(lr){3-4}
& & Baseline & IES & \\
\midrule
10\% & Symmetric & 61.9\% [190] & \textbf{62.0\%} [191] & 14.2\% \\
10\% & Instance & 58.9\% [151] & \textbf{59.2\%} [199] & 11.0\% \\
\bottomrule
\end{tabular}
\end{table}

The experimental results show that the IES method can achieve lossless efficient training under the Epoch-wise Double Descent scenario. In the later stages of training, the model inevitably ``well-learn'' some instances due to the memorization effect. However, this does not affect the generalization performance of the final model (even slightly better).

This behavior can potentially be explained by the fact that although 'well-learned' instances may be forgotten as the model training overfits the mislabeled samples, the IES method allows these samples to adaptively re-include in training, thereby mitigating the negative impact of mislabeled samples. Furthermore, as shown in the Figure \ref{fig3}, the IES method can more targetedly reduce steepness in these sharp directions of the loss landscape, and therefore may be able to train a model with better generalization performance even in the presence of label noise.

Consequently, in the typical scenarios of learning with noisy labels and scenarios of Epoch-wise Double Descent, the IES method appears to have no negative impact on model performance compared to the baseline.

\renewcommand{\thesection}{H}
\section{Catastrophic Forgetting}
\label{apph}


We define ``early removed examples'' as the first 5\% of samples that are removed. We conducted experiments in a typical IES training environment with CIFAR-10, ResNet18, and SGD optimizer, which saves approximately 43\% of the backpropagation samples in total 200 training epoch.

We tracked the average training loss and accuracy of these ``early removed examples'' during the training process and compared them with the corresponding values of the entire training set. The experimental results are as follows:

\begin{table}[t]
\caption{Comparison of training loss and accuracy between full training set and early removed examples across different epochs.}
\label{tab:training_metrics}
\centering
\begin{tabular}{c cccc}
\toprule
\multirow{2}{*}{Epoch} & \multicolumn{2}{c}{Training Loss} & \multicolumn{2}{c}{Training Accuracy} \\
\cmidrule(lr){2-3} \cmidrule(lr){4-5}
& Full Set & Early Removed & Full Set & Early Removed \\
\midrule
50  & 0.120135 & 0.001003 & 95.96\% & 100.00\% \\
100 & 0.001448 & 0.000920 & 99.99\% & 99.98\% \\
150 & 0.000914 & 0.000806 & 100.00\% & 100.00\% \\
200 & 0.000883 & 0.000833 & 100.00\% & 100.00\% \\
\bottomrule
\end{tabular}
\end{table}

The results demonstrate that the ``early removed examples'' are well learned (even better) by the model, and their training accuracy and loss are on par with other samples in the end of training. This implies that the model isn't catastrophically forgetting these ``early removed examples''.

Furthermore, we investigated the reasons why our IES method does not lead to catastrophic forgetting. Notably, the IES is a reversible method, which means that the removed samples have a chance to re-include in the training process if their second-order loss difference exceeds the threshold. Therefore, we tracked the average number of times the ``early removed examples'' were re-included in the training process, as shown in the following table:

\begin{table}[t]
\caption{Statistics of sample re-inclusion during training.}
\label{tab:reinclude_stats}
\centering
\begin{tabular}{l c}
\toprule
Metric & Value \\
\midrule
Average Times Re-included & 13.14 \\
Maximum Times Re-included & 26.00 \\
\bottomrule
\end{tabular}
\end{table}

Considering that our method allows these ``early removed examples'' to re-include in training for an average of about 13 times, with the most frequently replaced samples experiencing 26 training replays, we propose that this adaptive dynamic training mechanism contributes to the IES method's ability to effectively prevent ``early removed examples'' from being catastrophically forgotten during model training.

\renewcommand{\thesection}{I}
\section{Fairness}
\label{appi}

We conducted a preliminary assessment of the fairness of training using the IES method in sensitive environments.
We utilized the CelebA face dataset as an adversarial dataset to investigate whether the IES method would introduce new biases during training when using male as the sensitive attribute and attractiveness as the target label, thereby affecting the model's fairness.

We compared the baseline method (without sample removal) and the IES method on the ResNet-18 model for the attractiveness classification task, evaluating the accuracy, recall (True Positive Rate), and Demographic Parity Difference (DPD) metrics on the male and female validation subsets. The results are as follows:

\begin{table}[h]
\caption{Fairness evaluation on CelebA dataset using gender as the sensitive attribute. Metrics include overall accuracy, gender-specific accuracy and recall rates, and Demographic Parity Difference (DPD). Lower DPD indicates better fairness. Best results are shown in \textbf{bold}.}
\label{tab:fairness_metrics}
\centering
\begin{tabular}{l cccccc}
\toprule
\multirow{2}{*}{Method} & Overall & \multicolumn{2}{c}{Male} & \multicolumn{2}{c}{Female} & \multirow{2}{*}{DPD} \\
\cmidrule(lr){3-4} \cmidrule(lr){5-6}
& Acc. & Acc. & Recall & Acc. & Recall & \\
\midrule
Baseline & \textbf{82.5} & \textbf{83.8} & \textbf{68.2} & 81.6 & \textbf{90.6} & 0.4613 \\
IES (Ours) & 82.4 & 83.4 & 58.9 & \textbf{81.8} & 87.0 & \textbf{0.4544} \\
\bottomrule
\end{tabular}
\end{table}

From the Demographic Parity Difference (DPD) metric, which evaluates fairness (the closer to 0, the better), the IES method is slightly lower than the baseline method (0.4544 vs 0.4613), indicating that its prediction results have slightly less disparity between the two gender subsets. 

These results provide a preliminary indication that the IES method may introduce or amplify certain biases to some extent, negatively impacting the classification performance for different population subsets. However, since IES allows excluded samples to adaptively re-participate in training, the overall fairness is slightly improved.


\end{document}
