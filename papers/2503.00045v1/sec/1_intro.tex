\section{Introduction}
 
Autonomous driving tasks are usually data-intensive, relying on vast amounts of data to learn informed models. In recent years, autonomous driving has achieved considerable progress, particularly in the field of Bird's Eye View (BEV) perception~\citep{huang2021bevdet,liu2022petr,li2022bevformer,liu2023petrv2,li2023bevdepth,liao2022maptr} and end-to-end planning~\citep{shi2016uniad,jiang2023vad,chen2024vadv2}, thanks to the availability of public datasets such as nuScenes~\citep{nuscenes2019}, CARLA~\citep{dosovitskiy2017carla}, Waymo~\citep{ettinger2021large}, and ONCE~\citep{mao2021one}. However, these real-world driving datasets exist several limitations. On one hand, it is expensive and labor-intensive to collect large-scale real-world driving data. On the other hand, although corner-cases comprise only a small portion of dataset, they are more important for evaluating the safety of autonomous driving. Therefore, the scale and diversity of real-world datasets constrain the further development of autonomous driving.

\section{Introduction}
Backdoor attacks pose a concealed yet profound security risk to machine learning (ML) models, for which the adversaries can inject a stealth backdoor into the model during training, enabling them to illicitly control the model's output upon encountering predefined inputs. These attacks can even occur without the knowledge of developers or end-users, thereby undermining the trust in ML systems. As ML becomes more deeply embedded in critical sectors like finance, healthcare, and autonomous driving \citep{he2016deep, liu2020computing, tournier2019mrtrix3, adjabi2020past}, the potential damage from backdoor attacks grows, underscoring the emergency for developing robust defense mechanisms against backdoor attacks.

To address the threat of backdoor attacks, researchers have developed a variety of strategies \cite{liu2018fine,wu2021adversarial,wang2019neural,zeng2022adversarial,zhu2023neural,Zhu_2023_ICCV, wei2024shared,wei2024d3}, aimed at purifying backdoors within victim models. These methods are designed to integrate with current deployment workflows seamlessly and have demonstrated significant success in mitigating the effects of backdoor triggers \cite{wubackdoorbench, wu2023defenses, wu2024backdoorbench,dunnett2024countering}.  However, most state-of-the-art (SOTA) backdoor purification methods operate under the assumption that a small clean dataset, often referred to as \textbf{auxiliary dataset}, is available for purification. Such an assumption poses practical challenges, especially in scenarios where data is scarce. To tackle this challenge, efforts have been made to reduce the size of the required auxiliary dataset~\cite{chai2022oneshot,li2023reconstructive, Zhu_2023_ICCV} and even explore dataset-free purification techniques~\cite{zheng2022data,hong2023revisiting,lin2024fusing}. Although these approaches offer some improvements, recent evaluations \cite{dunnett2024countering, wu2024backdoorbench} continue to highlight the importance of sufficient auxiliary data for achieving robust defenses against backdoor attacks.

While significant progress has been made in reducing the size of auxiliary datasets, an equally critical yet underexplored question remains: \emph{how does the nature of the auxiliary dataset affect purification effectiveness?} In  real-world  applications, auxiliary datasets can vary widely, encompassing in-distribution data, synthetic data, or external data from different sources. Understanding how each type of auxiliary dataset influences the purification effectiveness is vital for selecting or constructing the most suitable auxiliary dataset and the corresponding technique. For instance, when multiple datasets are available, understanding how different datasets contribute to purification can guide defenders in selecting or crafting the most appropriate dataset. Conversely, when only limited auxiliary data is accessible, knowing which purification technique works best under those constraints is critical. Therefore, there is an urgent need for a thorough investigation into the impact of auxiliary datasets on purification effectiveness to guide defenders in  enhancing the security of ML systems. 

In this paper, we systematically investigate the critical role of auxiliary datasets in backdoor purification, aiming to bridge the gap between idealized and practical purification scenarios.  Specifically, we first construct a diverse set of auxiliary datasets to emulate real-world conditions, as summarized in Table~\ref{overall}. These datasets include in-distribution data, synthetic data, and external data from other sources. Through an evaluation of SOTA backdoor purification methods across these datasets, we uncover several critical insights: \textbf{1)} In-distribution datasets, particularly those carefully filtered from the original training data of the victim model, effectively preserve the model’s utility for its intended tasks but may fall short in eliminating backdoors. \textbf{2)} Incorporating OOD datasets can help the model forget backdoors but also bring the risk of forgetting critical learned knowledge, significantly degrading its overall performance. Building on these findings, we propose Guided Input Calibration (GIC), a novel technique that enhances backdoor purification by adaptively transforming auxiliary data to better align with the victim model’s learned representations. By leveraging the victim model itself to guide this transformation, GIC optimizes the purification process, striking a balance between preserving model utility and mitigating backdoor threats. Extensive experiments demonstrate that GIC significantly improves the effectiveness of backdoor purification across diverse auxiliary datasets, providing a practical and robust defense solution.

Our main contributions are threefold:
\textbf{1) Impact analysis of auxiliary datasets:} We take the \textbf{first step}  in systematically investigating how different types of auxiliary datasets influence backdoor purification effectiveness. Our findings provide novel insights and serve as a foundation for future research on optimizing dataset selection and construction for enhanced backdoor defense.
%
\textbf{2) Compilation and evaluation of diverse auxiliary datasets:}  We have compiled and rigorously evaluated a diverse set of auxiliary datasets using SOTA purification methods, making our datasets and code publicly available to facilitate and support future research on practical backdoor defense strategies.
%
\textbf{3) Introduction of GIC:} We introduce GIC, the \textbf{first} dedicated solution designed to align auxiliary datasets with the model’s learned representations, significantly enhancing backdoor mitigation across various dataset types. Our approach sets a new benchmark for practical and effective backdoor defense.




Instead of collecting large-scale real-world datasets, researchers have explored generating synthetic street-view data. Some approaches focus on leveraging Neural Radiance Fields (NeRF)~\citep{yang2023emernerf,yan2024oasim,yang2023unisim,tonderski2023neurad} and 3D Gaussian Splatting (3DGS)~\citep{yan2024street} for rendering the novel views in driving scenes. These cutting-edge approaches offer a powerful tool for rendering highly realistic and detailed images from various viewpoints. However, as shown in Fig. \ref{fig:comp_arch}(a), these approaches struggle to reconstruct the non-seen streets when the source trajectory and simulated trajectory are different. Recently, some approaches~\citep{gao2023magicdrive,yang2023bevcontrol,swerdlow2024street,li2023drivingdiffusion,jia2023adriver,hu2023gaia} attempt to employ Stable Diffusion~\citep{rombach2022high} to generate the synthetic data of driving scenes. For instance, Panacea~\citep{wen2023panacea} and DriveDreamer-2~\citep{zhao2024drivedreamer} have delved into the ability of Stable Diffusion to generate synthetic video data from a given initial frame. However, as shown in Fig.~\ref{fig:comp_arch}(b),  these diffusion models (e.g., Panacea) are limited to offline video generation and require high memory consumption. Moreover, it fails to dynamically generate video data in response to the changes in simulated trajectory. For a flexible generator, it requires the ability of generating dynamic scenes corresponding to the simulated trajectory.

To address the issues mentioned above, we propose \modelname, a simple yet effective framework that generates and forecasts synthetic video data  in an online (i.e., frame-by-frame) manner. Our  \modelname is based on Stable Diffusion, and introduces the latent variable propagation (LVP) to maintain temporal consistency when performing frame-by-frame generation. In LVP, the latent features of previous frame are viewed as the  noise prior injected into the latent features of current frame. In addition, we introduce a streaming data sampler (SDS), which aims to keep the consistency between training and inference, and present an efficient data sampling for training. In SDS, we sample original frames in video clip one by one at continuous iterations, and save the generated latent features in cache used for the noise prior of next iteration.

With such streaming generation manner, our  \modelname is capable of generating videos of arbitrary lengths in theory. Specifically, it can generate the videos of the novel scene from the noise when serving as the data generator. In addition, given the reference frame, it can produce the video data of specific scene. We perform the experiments on the public autonomous driving dataset nuScenes, which demonstrates the efficacy of our \modelname. The contributions and merits can be summarized as follows:

\begin{itemize}
\item 
A simple yet effective framework, named \modelname, is proposed to generate  video data in an online  manner, instead of offline fixed-length manner. Theoretically, our proposed \modelname is able to generate videos of arbitrary lengths, showcasing substantial potential in data generation and simulation.

\item A latent variable propagation strategy is introduced to ensure the temporal consistency for frame-by-frame video generation, where the latent features of previous frame are viewed as the noise fed to the latent features of current frame.


\item We further design a streaming data sampler to provide efficient training. In the streaming data sampler, we sample the frame in video clip one by one at several continuous iterations, and save the latent features in cache for next iteration.

\item The experiments are performed on the widely-used dataset nuScenes. Our \modelname  is able to generate high-quality video data  as generator. Further, our generated video data can significantly improve the perception, tracking, and HD map construction performance. 
\end{itemize}