\documentclass{article}

\newcommand{\CG}{\mathcal{G}\xspace}
\newcommand{\CV}{\mathcal{V}\xspace}
\newcommand{\CE}{\mathcal{E}\xspace}
\newcommand{\CA}{\mathcal{A}\xspace}
\newcommand{\CF}{\mathcal{F}\xspace}
\newcommand{\CR}{\mathcal{R}\xspace}
\newcommand{\CB}{\mathcal{B}\xspace}
\newcommand{\CX}{\mathcal{X}\xspace}
\newcommand{\CK}{\mathcal{K}\xspace}
\newcommand{\CM}{\mathcal{M}\xspace}
\newcommand{\CC}{\mathcal{C}\xspace}
\newcommand{\CL}{\mathcal{L}\xspace}
\newcommand{\CI}{\mathcal{I}\xspace}
\newcommand{\CQ}{\mathcal{Q}\xspace}
\newcommand{\CO}{\mathcal{O}\xspace}
\newcommand{\CP}{\mathcal{P}\xspace}
\newcommand{\CS}{\mathcal{S}\xspace}
\newcommand{\CT}{\mathcal{T}\xspace}
\newcommand{\CJ}{\mathcal{J}\xspace}
\usepackage[para]{footmisc}
\usepackage{subfig}
% \usepackage{subcaption}
% \usepackage{array}
% \usepackage{colortbl}


\usepackage[preprint]{cpal_2025}
\newcommand{\thought}[1]{{\color[rgb]{0.2,0.39,0.66}(#1)}}
\newcommand{\todo}[1]{{\color[rgb]{1.0,0.0,0.0}(#1)}}
\newcommand{\hsh}[1]{{\color{green!50!black} Henrik: #1}}
\newcommand{\st}[1]{{\color{red!50!black} Sebastian: #1}}

\newcommand{\ulm}[1]{_{\scaleto{\mathrm{#1}}{3pt}}}
\newcommand\at[2]{\left.#1\right|_{#2}}











\newtheorem{assumption}{Assumption}

\DeclareMathOperator*{\argmax}{arg\,max}
\DeclareMathOperator*{\argmin}{arg\,min}

\newcommand{\swname}[1]{\texttt{#1}}
\newcommand{\ie}{i\/.\/e\/.,\/~}
\newcommand{\eg}{e\/.\/g\/.,\/~}
\newcommand{\cf}{cf\/.\/~}

\newcommand{\fig}{Fig\/.\/~}
\newcommand{\defn}{Def\/.\/~}
\newcommand{\sect}{Sec\/.\/~}
\newcommand{\tabl}{Tab\/.\/~}
\newcommand{\algo}{Algorithm~}
\newcommand{\theo}{Theorem~}

\newcommand{\bnnl}{3 hidden layers}
\newcommand{\bnnn}{50 neurons}
\newcommand{\bnna}{tanh activations}

\newcommand{\capt}[1]{\mdseries{\emph{#1}}}

\newcommand{\videolink}{at \url{https://youtu.be/_d7AqTRjz6g}}
\newcommand{\codelink}{\url{https://github.com/wheelbot/mini-wheelbot}}

\newcommand{\fakepar}[1]{\vspace{0mm}\noindent\textbf{#1.}}

\newcommand{\needref}{\textcolor{red}{[REF]}}

\newcommand{\plotfontsize}{9pt}


\begin{document}
\pagecolor{anthracite}\afterpage{\nopagecolor}


\definecolor{poste_color}{RGB}{222, 184, 65}

\begin{center}
\color{white}
\begin{minipage}{0.99\linewidth}
    \thispagestyle{empty}
    \centering
    \tikz[remember picture,overlay] \node[opacity=0.3,inner sep=0pt] at (current page.center){\includegraphics[width=\paperwidth,height=\paperheight]{assets/topographic2.png}};

    \vspace{-1cm}
    

    \vspace{1cm}
    
    {\Huge {\textbf{Sparks of Explainability}}\\ Recent Advancements in Explaining Large Vision Models\\[0.5em]}

    \vspace{2cm}
    
    {
    Presented by\\
    {\textbf{\Large Thomas Fel}}\\[1em]

    
    Supervised by\\
    \textbf{Prof. Thomas Serre}\\[4em]
    }

    \vspace{1cm}
    
    {
    \raggedright
    Presented and publicly defended on July 25, 2024\\
    }
    \vspace{0em}
    \begin{tabbing}
        \hspace{8cm} \= \kill
        Prof. \textbf{George A. Alvarez} \> Reviewer\\
        \textit{\textcolor{poste_color}{Professor, Harvard University}} \\[0.5em]
        
        Prof. \textbf{Céline Hudelot} \> Reviewer\\
        \textit{\textcolor{poste_color}{Professor, Centrale Paris}} \\[0.5em]

        
        Dr. \textbf{Robert Geirhos} \> Examiner\\
        \textit{\textcolor{poste_color}{Research Scientist, Google}} \\[0.5em]
        
        Prof. \textbf{Ruth Fong} \> Examiner\\
        \textit{\textcolor{poste_color}{Professor, Princeton University}} \\[0.5em]
        
        Prof. \textbf{Rufin Van Rullen} \> Examiner\\
        \textit{\textcolor{poste_color}{Professor, Cerco, CNRS}} \\[0.5em]

        \\
        
        \textbf{Prof. Thomas Serre} \> Thesis Director\\
        \textit{\textcolor{poste_color}{Professor, Brown University \& ANITI}} \\[0.5em]
    \end{tabbing}

    \vspace{1.5cm}
    
    {\Large\textbf{DOCTORAL THESIS}\\[0.5em]}
    {\large Doctoral School of Mathematics, Computer Science, and Telecommunications of Toulouse}\\[2em]
    
\end{minipage}
\end{center}

\maketitle
% \mm{Thanks everyone for the comments. I am going to rewrite most of the paper on Sunday with some new results and integrating the feedback. If you could have a look at the paper again sometime on Monday that would be great! New deadline is Monday midnight.}
% \zx{We probably want to work on a better title -- revisit after reading the whole paper.}

% \begin{abstract}
% The impressive capabilities of Large Language Models (LLMs) in Natural Language Processing (NLP) comes at a cost of substantial computing resources. One line of work to address this issue is model compression that effectively reduces the model's size, while maintaining its performance. In particular, a novel compression scheme is to decompose the model's dense weights into a sum of sparse plus low-rank matrices. Despite recent works on such matrix decompositions in LLMs, designing principled one-shot methods for this type of compression remains partially addressed. We propose a framework coined \ourmethod for (semi-structured) sparse plus low-rank matrix decomposition of LLMs, that minimizes a well-posed optimization problem, solves it using principled algorithms, and achieves state-of-the-art performance on a wide-range of LLMs evaluation benchmarks, for different compression regimes: e.g. N:M sparse + Low-Rank, a regime for which highly-specialized CUDA kernels have recently been developed for runtime speedups and memory efficiency.
% \end{abstract}

\begin{abstract}
The impressive capabilities of large foundation models come at a cost of substantial computing resources to serve them. Compressing these pre-trained models is of practical interest as it can democratize deploying them to the machine learning community at large by lowering the costs associated with inference.
A promising compression scheme is to decompose foundation models' dense weights into a sum of sparse plus low-rank matrices.
In this paper, we design a unified framework coined \ourframework for (semi-structured) sparse plus low-rank matrix decomposition of foundation models.
Our framework introduces the local layer-wise reconstruction error objective for this decomposition, we demonstrate that prior work solves a relaxation of this optimization problem; and we provide efficient and scalable methods to minimize the \textit{exact} introduced optimization problem. 
\ourframework substantially outperforms state-of-the-art methods in terms of the introduced objective and a wide range of LLM evaluation benchmarks. For the Llama3-8B model with a 2:4 sparsity component plus a 64-rank component decomposition, a compression scheme for which recent work shows important inference acceleration on GPUs, \ourframework reduces the test perplexity by $12\%$ for the WikiText-2 dataset and reduces the gap (compared to the dense model) of the average of eight popular zero-shot tasks by $15\%$ compared to existing methods.
\end{abstract}


\vspace{-5pt}
\section{Introduction}
\label{sec:introduction}
\vspace{-5pt}
%!TEX root = main.tex


\section{Introduction}

Games play a central role in AI research. In the early $20^{th}$ century, \cite{zermelo1913} showed that perfect information games in extensive form can be solved by a bottom-up traversal of the game tree. Despite the fact that this does not readily provide efficient ways to solve large games such as Chess or Go in practice, this has indeed 
laid the foundation for the dramatic progress in the field of perfect information games, with computer programs being able to challenge human experts. Solving games becomes more intricate when the players (agents) have incomplete information about the state of the game -- Poker for instance, where a player does not know the cards of the others. One of the remarkable imperfect information games where computer programs have been able to defeat professional human players is Texas Hold'em Poker~\cite{libratus-poker,deepstack,pluribus-poker}. A main technique used in these algorithms is the abstraction of large games into smaller \emph{imperfect recall} games. 

Perfect recall is the ability of a player to remember her own actions. Poker is an imperfect information game played by several players. However, ideally one would assume that the players have a perfect recall of their actions. An imperfect recall player does not remember the sequence of her own actions. Imperfect recall allows for a structured mechanism to forget the information history and as \cite{ijcai2024p332} argues, it is particularly suited for AI agents.

From a modeling perspective, imperfect recall has been used to describe teams of agents, where each team can be represented as a single agent with imperfect recall~\cite{VONSTENGEL1997309,DBLP:conf/aaai/Celli018} or to describe agents modeling multiple nodes which do not share information between each other due to privacy reasons~\cite{DBLP:conf/aaai/Conitzer19}. Moreover~\cite{LAMBERT2019164} argues that imperfect recall is a model of bounded rationality. Given the limited memory of players, it is not realistic to assume that the players remember all their actions. We refer the reader to~\cite{ijcai2024p332} for an excellent introduction to different uses of imperfect recall.  
From a practical perspective, the most prominent use of imperfect recall is in abstracting games~\cite{PracticalUseImperfect,DBLP:conf/aaai/GanzfriedS14,DBLP:conf/atal/BrownGS15, CERMAK2020103248}. The state space generated by usual games is typically very large and abstractions are crucial for solving such games. Abstractions that preserve perfect recall force a player to distinguish the current information gained, in all later rounds, even if it is not relevant. Abstractions using players with imperfect recall have been shown to outperform those using players with perfect recall~\cite{ DBLP:conf/sara/WaughZJKSB09, johanson2013evaluating, DBLP:conf/atal/BrownGS15,DBLP:conf/ijcai/CermakBL17}.



From a complexity perspective, imperfect recall games are known to be $\NP$-hard

~\cite{KollerMegiddo::1992,Cermak::2018} even when there is a single player, whereas perfect recall games can be solved in polynomial-time~\cite{KollerMegiddo::1992,vonStengel::1996}. Recent studies have aligned the complexity of different solution concepts for imperfect recall games to the modern complexity classes~\cite{GPS20,tewolde-et-al:2023,ijcai2024p332}. The hardness of imperfect recall games has motivated the search for subclasses which are polynomial-time solvable~\cite{kline2002minimum,kaneko1995behavior}, or where algorithms similar to the perfect recall case can be applied~\cite{DBLP:conf/icml/LanctotGBB12,DBLP:conf/sigecom/KroerS16}. The class of \emph{A-loss recall}~\cite{kline2002minimum,kaneko1995behavior} is a special kind of imperfect recall, where the loss of information can be traced back to a player forgetting her own action at a point in the past -- the player remembers \emph{where} it was played, but forgets \emph{what} was played. We consider A-loss recall games to be \emph{simple} since there are polynomial-time algorithms for solving them. To the best of our knowledge, A-loss recall games are the biggest known class of imperfect recall with a polynomial-time solution. This has led to research towards finding A-loss recall abstractions~\cite{Cermak::2018}. 

\emph{Contributions.} Our broad goal in this work is to find efficient ways to solve imperfect recall games in extensive-form. We do so by simplifying them into A-loss recall games. We focus on games where the players are not absent-minded: a player is absent-minded if she even forgets whether a decision point was previously seen or not. Here are our major contributions.
\begin{enumerate}\item We first identify a class of one-player games where the player's information structure is more complex than A-loss recall, but shuffling the order of actions results in an equivalent A-loss recall game. This leads to a new $\mathsf{PTIME}$ solvable class of imperfect recall games, that extends A-loss recall (\cref{thm:1p-shuffle-ptime}, \cref{cor:effic-solv-class}, \cref{cor:2-effic-solv-class}). Furthermore, these classes themselves can be tested in $\mathsf{PTIME}$. 

\item We show that every game with \emph{non-absentminded} players can be transformed into an equivalent A-loss recall game (\cref{thm:existence-alr-span}). We present an algorithm to generate an equivalent A-loss recall game with the smallest size.
\end{enumerate}
















 


The caveat in the second result above is that the resulting A-loss recall game could be exponentially bigger. This is expected, since solving imperfect recall games is $\NP$-hard, whereas A-loss recall games can be solved in polynomial-time. The result however shows that in order to solve imperfect recall games, one could either use a worst-case exponential-time algorithm on the original game, or apply our transformation to a worst-case exponential-sized game and run a polynomial-time algorithm on it. From a conceptual point of view, our result shows that as long as there is no absentmindedness, imperfect recall can be transformed into one where the information loss can be attributed to forgetting own actions at a past point.


\emph{Organization of the document.} Section~\ref{sec:an-example} introduces a modification of the popular matching pennies game that will be used as a running example to illustrate our results. Section~\ref{sec:background} recalls necessary preliminaries on extensive-form games. Section~\ref{sec:shuffled-loss-recall} presents the new polynomial-time class of shuffled A-loss recall. Section~\ref{sec:span} generalizes the idea of shuffling to incorporate a ``linear combination'' of action sequences, and presents the second result mentioned above. Section~\ref{sec:two-player} extends the results to the two-player setting. 

  






  



\section{An example}
\label{sec:an-example}



Let us start with a one-player game called the \emph{single team matching-unmatching pennies game}, which will be used as a running example. A team of players with the same goal can be interpreted as a single player. 
In this case, the team consists of two players Alice and Bob, each possessing a coin with two sides, Head (H) and Tail (T) and each of them must choose a side for their respective coins independently. 
The game unfolds in the following manner : a fair $n$-faced die with outcomes from $\{0, \dots, n-1 \}$ is rolled; then Alice chooses a side from $\{H,T\}$, followed by Bob choosing from $\{H,T\}$. Winning or losing depends on the parity of the die outcome. If the outcome of the die is even, then they win if and only if they match their sides. If the outcome is odd, they win if and only if their sides do not match. We consider three variants depending on what Alice and Bob can observe, and model it in extensive form in \cref{fig:match-penny-3-die} for $n=3$. An informal description of the figures follows after this paragraph.
\begin{description}
  \item[I.] Both Alice and Bob observe nothing (\cref{fig:match-penny-3-die-a}).
  \item[II.] Alice can't distinguish between die outcome $2i$ and $2i+1$ for $i \geq 0$,  but Bob observes nothing (\cref{fig:match-penny-3-die-b}).
   \item[III.] Alice can't distinguish between die outcome $2i$ and $2i+1$ for $i \geq 0$, Bob only observes coin of Alice but not outcome of die (\cref{fig:match-penny-3-die-c}). 
 \end{description} 
Alice and Bob want to maximize their \emph{expected payoff}. We will see their possible strategies in Section~\ref{sec:background}.  
Later, we will see that game \textbf{I} falls under the simple class of A-loss recall. 
In Section~\ref{sec:shuffled-loss-recall} and \cref{sec:span} we will see how to simplify games \textbf{II} and \textbf{III} respectively. 
%!TEX root = ../main.tex

\begin{figure}

\begin{subfigure}{0.45\columnwidth}
\centering
\tikzset{
triangle/.style = {regular polygon,regular polygon sides=3,draw,inner sep = 2},
circ/.style = {circle,fill=cyan!10,draw,inner sep = 3},
term/.style = {circle,draw,inner sep = 1.5,fill=black},
sq/.style = {rectangle,fill=gray!20, draw, inner sep = 4}
}
\begin{tikzpicture}[scale=0.8]

\tikzstyle{level 1}=[level distance=18mm,sibling distance=24mm]
\tikzstyle{level 2}=[level distance=11mm,sibling distance=12mm]
\tikzstyle{level 3}=[level distance=11mm,sibling distance=5mm]

\begin{scope}[->, >=stealth]
 \node(0)[triangle]{}
    child{  
    node(00)[circ,draw=black]{}
        child{
        node(000)[circ]{}
            child{
            node(0000)[term,label=below:{\scriptsize $1$}]{} 
            edge from parent node[left,pos = 0.3,inner sep=1.5]{\scriptsize H}
            }
            child{
            node(0001)[term,label=below:{\scriptsize $0$}]{}
            edge from parent node[right,pos = 0.3,inner sep=1.5]{\scriptsize T}                
            }
        edge from parent node[left,pos = 0.2]{{\scriptsize H}}                
        }
        child{
        node(001)[circ]{}
            child{
            node(0010)[term,label=below:{\scriptsize $0$}]{} 
            edge from parent node[left,pos = 0.3,inner sep=1.5]{\scriptsize H}
            }
            child{
            node(0011)[term,label=below:{\scriptsize $1$}]{}
            edge from parent node[right,pos = 0.3,inner sep=1.5]{\scriptsize T}                
            }
        edge from parent node[right,pos = 0.2]{{\scriptsize T}}
        }
    edge from parent node[above,pos = 0.8]{0}
    edge from parent node[above,pos = 0.4]{\scriptsize $\frac{1}{3}$} 
    }
    child{
    node(01)[circ]{}
        child{
        node(010)[circ]{}
            child{
            node(0100)[term,label=below:{\scriptsize $0$}]{}
            edge from parent node[left,pos = 0.3,inner sep=1.5]{\scriptsize H}
            }
            child{
            node(0101)[term,label=below:{\scriptsize $1$}]{}
            edge from parent node[right,pos = 0.3,inner sep=1.5]{\scriptsize T}                
            }
        edge from parent node[left,pos = 0.2]{\scriptsize H}
        }
        child{
        node(011)[circ]{}
            child{
            node(0110)[term,label=below:{\scriptsize $1$}]{} 
            edge from parent node[left,pos = 0.3,inner sep=1.5]{\scriptsize H}
            }
            child{
            node(0111)[term,label=below:{\scriptsize $0$}]{}
            edge from parent node[right,pos = 0.3,inner sep=1.5]{\scriptsize T}                
            } 
        edge from parent node[right,pos = 0.2]{\scriptsize T}
        }
    edge from parent node[right,pos = 0.8]{1}
    edge from parent node[left,pos = 0.4]{\scriptsize $\frac{1}{3}$}
    }
    child{
    node(02)[circ]{}
        child{
        node(020)[circ]{}
            child{
            node(0200)[term,label=below:{\scriptsize $1$}]{}
            edge from parent node[left,pos = 0.3,inner sep=1.5]{\scriptsize H}
            }
            child{
            node(0201)[term,label=below:{\scriptsize $0$}]{}
            edge from parent node[right,pos = 0.3,inner sep=1.5]{\scriptsize T}                
            }
        edge from parent node[left,pos = 0.2]{\scriptsize H}
        }
        child{
        node(021)[circ]{}
            child{
            node(0210)[term,label=below:{\scriptsize $0$}]{} 
            edge from parent node[left,pos = 0.3,inner sep=1.5]{\scriptsize H}
            }
            child{
            node(0211)[term,label=below:{\scriptsize $1$}]{}
            edge from parent node[right,pos = 0.3,inner sep=1.5]{\scriptsize T}                
            } 
        edge from parent node[right,pos = 0.2]{\scriptsize T}
        }
    edge from parent node[above,pos = 0.8]{2}
    edge from parent node[above,pos = 0.4]{\scriptsize $\frac{1}{3}$}
    }
    ;

\end{scope}
\draw [dashed,thick,blue,out=22,in=158] (00) to (01); 
\draw [dashed,thick,blue,out=22,in=158] (01) to (02); 
\draw [dashed,ForestGreen,thick,out=22,in=158] (001) to (010);
\draw [dashed,ForestGreen,thick,out=22,in=158] (000) to (001);
\draw [dashed,ForestGreen,thick,out=22,in=158] (010) to (011);
\draw [dashed,ForestGreen,thick,out=22,in=158] (011) to (020);
\draw [dashed,ForestGreen,thick,out=22,in=158] (020) to (021);

%\node[fit=(1),dashed,thick,blue, draw, circle,inner sep=1pt] {};


%\node[fit=(2),dashed,thick,black, draw, circle,inner sep=1pt] {};

\end{tikzpicture}
\caption{Alice and Bob, both observe nothing}
\label{fig:match-penny-3-die-a}
\end{subfigure}
\vspace{6mm}
\begin{subfigure}{0.48\columnwidth}
\centering
\tikzset{
triangle/.style = {regular polygon,regular polygon sides=3,draw,inner sep = 2},
circ/.style = {circle,fill=cyan!10,draw,inner sep = 3},
term/.style = {circle,draw,inner sep = 1.5,fill=black},
sq/.style = {rectangle,fill=gray!20, draw, inner sep = 4}
}
\begin{tikzpicture}[scale=0.8]

\tikzstyle{level 1}=[level distance=18mm,sibling distance=24mm]
\tikzstyle{level 2}=[level distance=11mm,sibling distance=12mm]
\tikzstyle{level 3}=[level distance=11mm,sibling distance=5mm]

\begin{scope}[->, >=stealth]
 \node(0)[triangle]{}
    child{  
    node(00)[circ,draw=black]{}
        child{
        node(000)[circ]{}
            child{
            node(0000)[term,label=below:{\scriptsize $1$}]{} 
            edge from parent node[left,pos = 0.3,inner sep=1.5]{\scriptsize H}
            }
            child{
            node(0001)[term,label=below:{\scriptsize $0$}]{}
            edge from parent node[right,pos = 0.3,inner sep=1.5]{\scriptsize T}                
            }
        edge from parent node[left,pos = 0.2]{{\scriptsize H}}                
        }
        child{
        node(001)[circ]{}
            child{
            node(0010)[term,label=below:{\scriptsize $0$}]{} 
            edge from parent node[left,pos = 0.3,inner sep=1.5]{\scriptsize H}
            }
            child{
            node(0011)[term,label=below:{\scriptsize $1$}]{}
            edge from parent node[right,pos = 0.3,inner sep=1.5]{\scriptsize T}                
            }
        edge from parent node[right,pos = 0.2]{{\scriptsize T}}
        }
    edge from parent node[above,pos = 0.8]{0} 
    edge from parent node[above,pos = 0.4]{\scriptsize $\frac{1}{3}$}
    }
    child{
    node(01)[circ]{}
        child{
        node(010)[circ]{}
            child{
            node(0100)[term,label=below:{\scriptsize $0$}]{}
            edge from parent node[left,pos = 0.3,inner sep=1.5]{\scriptsize H}
            }
            child{
            node(0101)[term,label=below:{\scriptsize $1$}]{}
            edge from parent node[right,pos = 0.3,inner sep=1.5]{\scriptsize T}                
            }
        edge from parent node[left,pos = 0.2]{\scriptsize H}
        }
        child{
        node(011)[circ]{}
            child{
            node(0110)[term,label=below:{\scriptsize $1$}]{} 
            edge from parent node[left,pos = 0.3,inner sep=1.5]{\scriptsize H}
            }
            child{
            node(0111)[term,label=below:{\scriptsize $0$}]{}
            edge from parent node[right,pos = 0.3,inner sep=1.5]{\scriptsize T}                
            } 
        edge from parent node[right,pos = 0.2]{\scriptsize T}
        }
    edge from parent node[right,pos = 0.8]{1}
    edge from parent node[left,pos = 0.4]{\scriptsize $\frac{1}{3}$}
    }
    child{
    node(02)[circ]{}
        child{
        node(020)[circ]{}
            child{
            node(0200)[term,label=below:{\scriptsize $1$}]{}
            edge from parent node[left,pos = 0.3,inner sep=1.5]{\scriptsize H}
            }
            child{
            node(0201)[term,label=below:{\scriptsize $0$}]{}
            edge from parent node[right,pos = 0.3,inner sep=1.5]{\scriptsize T}                
            }
        edge from parent node[left,pos = 0.2]{\scriptsize H}
        }
        child{
        node(021)[circ]{}
            child{
            node(0210)[term,label=below:{\scriptsize $0$}]{} 
            edge from parent node[left,pos = 0.3,inner sep=1.5]{\scriptsize H}
            }
            child{
            node(0211)[term,label=below:{\scriptsize $1$}]{}
            edge from parent node[right,pos = 0.3,inner sep=1.5]{\scriptsize T}                
            } 
        edge from parent node[right,pos = 0.2]{\scriptsize T}
        }
    edge from parent node[above,pos = 0.8]{2}
    edge from parent node[above,pos = 0.4]{\scriptsize $\frac{1}{3}$}
    }
    ;

\end{scope}
\node[fit=(02),dashed,thick,red, draw, circle,inner sep=1pt] {};
\draw [dashed,thick,blue,out=22,in=158] (00) to (01); 
\draw [dashed,ForestGreen,thick,out=22,in=158] (001) to (010);
\draw [dashed,ForestGreen,thick,out=22,in=158] (000) to (001);
\draw [dashed,ForestGreen,thick,out=22,in=158] (010) to (011);
\draw [dashed,ForestGreen,thick,out=22,in=158] (011) to (020);
\draw [dashed,ForestGreen,thick,out=22,in=158] (020) to (021);
%\node[fit=(1),dashed,thick,blue, draw, circle,inner sep=1pt] {};


%\node[fit=(2),dashed,thick,black, draw, circle,inner sep=1pt] {};

\end{tikzpicture}
\caption{Alice can't distinguish between $2i$ and $2i+1$ for $i \geq 0$, Bob observes nothing}
\label{fig:match-penny-3-die-b}
\end{subfigure}

\begin{subfigure}{\columnwidth}
\centering
\tikzset{
triangle/.style = {regular polygon,regular polygon sides=3,draw,inner sep = 2},
circ/.style = {circle,fill=cyan!10,draw,inner sep = 3},
term/.style = {circle,draw,inner sep = 1.5,fill=black},
sq/.style = {rectangle,fill=gray!20, draw, inner sep = 4}
}
\begin{tikzpicture}[scale=0.8]

\tikzstyle{level 1}=[level distance=18mm,sibling distance=24mm]
\tikzstyle{level 2}=[level distance=11mm,sibling distance=12mm]
\tikzstyle{level 3}=[level distance=11mm,sibling distance=5mm]

\begin{scope}[->, >=stealth]
 \node(0)[triangle]{}
    child{  
    node(00)[circ,draw=black]{}
        child{
        node(000)[circ]{}
            child{
        node(0000)[term,label=below:{\scriptsize $1$}]{} 
            edge from parent node[left,pos = 0.2]{\scriptsize H}
            }
            child{
            node(0001)[term,label=below:{\scriptsize $0$}]{}
            edge from parent node[right,pos = 0.2]{\scriptsize T}                
            }
        edge from parent node[left,pos = 0.2]{{\scriptsize H}}                
        }
        child{
        node(001)[circ]{}
            child{
            node(0010)[term,label=below:{\scriptsize $0$}]{} 
            edge from parent node[left,pos = 0.2]{\scriptsize H}
            }
            child{
            node(0011)[term,label=below:{\scriptsize $1$}]{}
            edge from parent node[right,pos = 0.2]{\scriptsize T}                
            }
        edge from parent node[right,pos = 0.2]{{\scriptsize T}}
        }
    edge from parent node[above,pos = 0.8]{0} 
    edge from parent node[above,pos = 0.4]{\scriptsize $\frac{1}{3}$}
    }
    child{
    node(01)[circ]{}
        child{
        node(010)[circ]{}
            child{
            node(0100)[term,label=below:{\scriptsize $0$}]{}
            edge from parent node[left,pos = 0.2]{\scriptsize H}
            }
            child{
            node(0101)[term,label=below:{\scriptsize $1$}]{}
            edge from parent node[right,pos = 0.2]{\scriptsize T}                
            }
        edge from parent node[left,pos = 0.2]{\scriptsize H}
        }
        child{
        node(011)[circ]{}
            child{
            node(0110)[term,label=below:{\scriptsize $1$}]{} 
            edge from parent node[left,pos = 0.2]{\scriptsize H}
            }
            child{
            node(0111)[term,label=below:{\scriptsize $0$}]{}
            edge from parent node[right,pos = 0.2]{\scriptsize T}                
            } 
        edge from parent node[right,pos = 0.2]{\scriptsize T}
        }
    edge from parent node[right,pos = 0.8]{1}
    edge from parent node[left,pos = 0.4]{\scriptsize $\frac{1}{3}$}
    }
    child{
    node(02)[circ]{}
        child{
        node(020)[circ]{}
            child{
            node(0200)[term,label=below:{\scriptsize $1$}]{}
            edge from parent node[left,pos = 0.2]{\scriptsize H}
            }
            child{
            node(0201)[term,label=below:{\scriptsize $0$}]{}
            edge from parent node[right,pos = 0.2]{\scriptsize T}                
            }
        edge from parent node[left,pos = 0.2]{\scriptsize H}
        }
        child{
        node(021)[circ]{}
            child{
            node(0210)[term,label=below:{\scriptsize $0$}]{} 
            edge from parent node[left,pos = 0.2]{\scriptsize H}
            }
            child{
            node(0211)[term,label=below:{\scriptsize $1$}]{}
            edge from parent node[right,pos = 0.2]{\scriptsize T}                
            } 
        edge from parent node[right,pos = 0.2]{\scriptsize T}
        }
    edge from parent node[above,pos = 0.8]{2}
    edge from parent node[above,pos = 0.4]{\scriptsize $\frac{1}{3}$}
    }
    ;

\end{scope}
\node[fit=(02),dashed,thick,red, draw, circle,inner sep=1pt] {};
\draw [dashed,thick,blue,out=22,in=158] (00) to (01);
\draw [dashed,thick,ForestGreen,out=28,in=152] (000) to (010);
\draw [dashed,ForestGreen,thick,out=28,in=152] (010) to (020);
\draw [dashed,brown,thick,out=28,in=152] (001) to (011);
\draw [dashed,brown,thick,out=28,in=152] (011) to (021);
%\node[fit=(1),dashed,thick,blue, draw, circle,inner sep=1pt] {};


%\node[fit=(2),dashed,thick,black, draw, circle,inner sep=1pt] {};

\end{tikzpicture}
\caption{Bob only observes Alice's coin}
\label{fig:match-penny-3-die-c}
\end{subfigure}


% \begin{subfigure}{.48\columnwidth}
% \centering
% \tikzset{
% triangle/.style = {regular polygon,regular polygon sides=3,draw,inner sep = 2},
% circ/.style = {circle,fill=cyan!10,draw,inner sep = 3},
% term/.style = {circle,draw,inner sep = 1.5,fill=black},
% sq/.style = {rectangle,fill=gray!20, draw, inner sep = 4}
% }
% \begin{tikzpicture}[scale=0.8]

% \tikzstyle{level 1}=[level distance=15mm,sibling distance=23mm]
% \tikzstyle{level 2}=[level distance=10mm,sibling distance=12mm]
% \tikzstyle{level 3}=[level distance=12mm,sibling distance=4mm]

% \begin{scope}[->, >=stealth]
%  \node(0)[triangle]{}
%     child{  
%     node(00)[circ,draw=black]{}
%         child{
%         node(000)[circ]{}
%             child{
%             node(0000)[term,label=below:{\scriptsize $1$}]{} 
%             edge from parent node[left,pos = 0.2]{\scriptsize H}
%             }
%             child{
%             node(0001)[term,label=below:{\scriptsize $0$}]{}
%             edge from parent node[right,pos = 0.2]{\scriptsize T}                
%             }
%         edge from parent node[left,pos = 0.2]{{\scriptsize H}}                
%         }
%         child{
%         node(001)[circ]{}
%             child{
%             node(0010)[term,label=below:{\scriptsize $0$}]{} 
%             edge from parent node[left,pos = 0.2]{\scriptsize H}
%             }
%             child{
%             node(0011)[term,label=below:{\scriptsize $1$}]{}
%             edge from parent node[right,pos = 0.2]{\scriptsize T}                
%             }
%         edge from parent node[right,pos = 0.2]{{\scriptsize T}}
%         }
%     edge from parent node[above,pos = 0.8]{0} 
%     }
%     child{
%     node(01)[circ]{}
%         child{
%         node(010)[circ]{}
%             child{
%             node(0100)[term,label=below:{\scriptsize $0$}]{}
%             edge from parent node[left,pos = 0.2]{\scriptsize H}
%             }
%             child{
%             node(0101)[term,label=below:{\scriptsize $1$}]{}
%             edge from parent node[right,pos = 0.2]{\scriptsize T}                
%             }
%         edge from parent node[left,pos = 0.2]{\scriptsize H}
%         }
%         child{
%         node(011)[circ]{}
%             child{
%             node(0110)[term,label=below:{\scriptsize $1$}]{} 
%             edge from parent node[left,pos = 0.2]{\scriptsize H}
%             }
%             child{
%             node(0111)[term,label=below:{\scriptsize $0$}]{}
%             edge from parent node[right,pos = 0.2]{\scriptsize T}                
%             } 
%         edge from parent node[right,pos = 0.2]{\scriptsize T}
%         }
%     edge from parent node[right,pos = 0.8]{1}
%     }
%     child{
%     node(02)[circ]{}
%         child{
%         node(020)[circ]{}
%             child{
%             node(0200)[term,label=below:{\scriptsize $1$}]{}
%             edge from parent node[left,pos = 0.2]{\scriptsize H}
%             }
%             child{
%             node(0201)[term,label=below:{\scriptsize $0$}]{}
%             edge from parent node[right,pos = 0.2]{\scriptsize T}                
%             }
%         edge from parent node[left,pos = 0.2]{\scriptsize H}
%         }
%         child{
%         node(021)[circ]{}
%             child{
%             node(0210)[term,label=below:{\scriptsize $0$}]{} 
%             edge from parent node[left,pos = 0.2]{\scriptsize H}
%             }
%             child{
%             node(0211)[term,label=below:{\scriptsize $1$}]{}
%             edge from parent node[right,pos = 0.2]{\scriptsize T}                
%             } 
%         edge from parent node[right,pos = 0.2]{\scriptsize T}
%         }
%     edge from parent node[above,pos = 0.8]{2}
%     }
%     ;

% \end{scope}
% \draw [dashed,thick,blue,out=22,in=158] (00) to (01); 
% \draw [dashed,ForestGreen,thick,out=22,in=158] (000) to (021);
% \draw [dashed,brown,thick,out=22,in=158] (001) to (011);
% \draw [dashed,brown,thick,out=22,in=158] (010) to (020);
% %\node[fit=(1),dashed,thick,blue, draw, circle,inner sep=1pt] {};


% %\node[fit=(2),dashed,thick,black, draw, circle,inner sep=1pt] {};

% \end{tikzpicture}
% \caption{Bob can't distinguish between $(i,\text{C})$ and $(i+1 \mod n,\text{C})$ for $i \geq 0 $, $C \in \{H,T\}$ and $n=3$}
% \label{fig:match-penny-3-die-d}
% \end{subfigure}

\caption{Three versions of the single team matching-unmatching pennies game for $n=3$}
\label{fig:match-penny-3-die}
\end{figure}

Before we delve into the background and results, here is a description of the extensive-form model. 
The root node, marked with a triangle, is the event of rolling the die. The triangle nodes are called $\chance$ nodes, and the
edges out of them associate probabilities to each of the outcomes. For this game, the distribution is uniform. The circle nodes denote decision nodes of the team. The nodes in the second level (root being the first level) belong to Alice whereas the nodes in the third level belong to Bob. The actions labelled in edges out of these nodes denote the actions available to the corresponding players. 
A leaf node indicates an end state, and a path from root to leaf denotes
a play from start to end. The number associated with a leaf gives the
payoff that the team receives at the end of the corresponding play. E.g., in \cref{fig:match-penny-3-die-a} in the play resulting from the path $0, H, T$ the payoff is $0$ because the team loses. It is $1$ when they win. 

Imperfect information is expressed using a dotted line: a player cannot distinguish between two nodes joined by a dotted line.
For e.g., in \cref{fig:match-penny-3-die-a} the dotted red line joining all of Alice's nodes indicates that Alice cannot observe the die outcome. Similarly, the blue dotted line for Bob indicates, he neither observes the outcome of the die, nor the side of the coin chosen by Alice. These sets of indistinguishable nodes are called \emph{information sets}.





\section{Background and notations}
\label{sec:background}

%!TEX root = ../main.tex

\begin{figure}
%\begin{center}
\tikzset{
triangle/.style = {regular polygon,regular polygon sides=3,draw,inner sep = 2},
circ/.style = {circle,fill=cyan!10,draw,inner sep = 3},
term/.style = {circle,draw,inner sep = 1.5,fill=black},
sq/.style = {rectangle,fill=gray!20, draw, inner sep = 4}
}

\begin{subfigure}{.45\columnwidth}
\centering
\begin{tikzpicture}[scale=0.9]
\tikzstyle{level 1}=[level distance=9mm,sibling distance = 22mm]
\tikzstyle{level 2}=[level distance=7mm,sibling distance=10mm]
\tikzstyle{level 3}=[level distance=7mm,sibling distance=6mm]
\tikzstyle{level 4}=[level distance=7mm,sibling distance=5mm]

%node (ij) is the j th node in i th level

\begin{scope}[->, >=stealth]
\node (0) [circ] {}
child {
  node (00) [triangle] {}
  child {
    node (000) [circ] {}
    child {
      node (0000) [term, label=below:{}] {}
      edge from parent node [left] {\scriptsize $c$}
    }
    child {
      node (0001) [term, label=below:{}] {}
      edge from parent node [right] {\scriptsize $d$}
      }
    edge from parent node [left] {}
  }
  child {
    node (001) [circ] {}
    child {
      node (0010) [term, label=below:{}] {}
      edge from parent node [left] {\scriptsize $c$}
    }
    child {
      node (0011) [term, label=below:{}] {}
      edge from parent node [right] {\scriptsize $d$}
      }
    edge from parent node [right] {} 
  }
  edge from parent node [above] {\scriptsize$a$}
}
child {
  node (01) [triangle] {}
   child {
     node (010) [circ] {}
     child {
      node (0100) [term, label=below:{}] {}
      edge from parent node [left] {\scriptsize $e$}
    }
    child {
      node (0101) [term, label=below:{}] {}
      edge from parent node [right] {\scriptsize $f$}
      }
    edge from parent node [left] {}
  }
  child {
    node (011) [circ] {}
    child {
      node (0110) [term, label=below:{}] {}
      edge from parent node [left] {\scriptsize $e$}
    }
    child {
      node (0111) [term, label=below:{}] {}
      edge from parent node [right] {\scriptsize $f$}
      }
    edge from parent node [right] {} 
  }
  edge from parent node [above] {\scriptsize$b$}
}
;
\end{scope}

%observations
%\draw [dashed, thick, red, in=150,out=30](00) to (01) ;

  \node[fit=(0),dashed,thick,red, draw, circle,inner sep=1pt] {};
\draw [dashed, thick, blue, in=150,out=30] (000) to (001) ;
\draw [dashed, thick, ForestGreen, in=150,out=30] (010) to (011);

%node labels
\node [black] at (0,0.35) {\scriptsize $r$};
\node [black] at (-1,-0.55) {\scriptsize $u_1$};
\node [black] at (1, -0.55) {\scriptsize $u_2$};
\node [black] at (-2, -1.5) {\scriptsize $u_3$};
\node [black] at (-.25, -1.5) {\scriptsize $u_4$};

\node [black] at (0.25, -1.5) {\scriptsize $u_5$};
\node [black] at (2, -1.5) {\scriptsize $u_6$};

%obs labels
\node [red] at (0,-.5) {\scriptsize $I_1$};
\node [blue] at (-1.1,-1.6) {\scriptsize $I_2$};
\node [ForestGreen] at (1.1,-1.6) {\scriptsize $I_3$};



\end{tikzpicture}

\caption{$\Max$ with perfect recall}
\label{fig-allexmp-pftrec}
\end{subfigure}
\quad
\begin{subfigure}{.45\columnwidth}
\centering
\begin{tikzpicture}
\tikzstyle{level 1}=[level distance=7mm,sibling distance = 10mm]
\tikzstyle{level 2}=[level distance=7mm,sibling distance=10mm]
\tikzstyle{level 3}=[level distance=7mm,sibling distance=15mm]
\tikzstyle{level 4}=[level distance=7mm,sibling distance=8mm]

%\draw [help lines, step=0.5] (-3,-3) grid (3,0);

\begin{scope}[->, >=stealth]
\node (0) [circ] {}
child{
  node (1) [circ] {}
  child{
    node (3) [term, label=below:{}] {}
    edge from parent node [left] {\scriptsize $a$}
  }
  child{
    node (4) [term,label=below:{}] {}
    edge from parent node [right] {\scriptsize $b$}
  }
  edge from parent node [left] {\scriptsize $a$}
}
child{
  node (2) [term, label=below:{}] {}
  edge from parent node [right] {\scriptsize $b$}
}
;
\end{scope}

\draw [dashed, thick, blue, in=10,out=-100] (0) to (1);



\node [black] at (0,0.25) {\scriptsize $r$};
\node [black] at (-.9,-0.6) {\scriptsize $u_1$};



\node [blue] at (.1,-.6) {\scriptsize $I_1$};

\end{tikzpicture}
\caption{$\Max$ with absentmindedness}
\label{fig-allexmp-absentm}
\end{subfigure}


%\end{center}
\caption{Recalls of $\Max$}
\label{fig:recall-examples}
\end{figure}

This section presents the formal definitions. The single team matching-unmatching pennies game has only one player and chance nodes, but in general we will talk about zero-sum two player games. As in \cref{fig:2-p-shuffle}, there are two players $\Max$ (circle nodes) and $\Min$ (square nodes). The payoff at the leaf, is the amount $\Min$ loses and $\Max$ gains. The goal of $\Max$ is to maximize the expected payoff whereas $\Min$ wishes to minimize it. In \cref{fig:match-penny-3-die} $\Max$ was the team consisting of Alice and Bob.







In this paper, we mainly work with \emph{game-structures} and not games
themselves. Game-structures are essentially games sans the numerical
quantities. Any game on a game structure can be represented symbolically as in shown \cref{fig:alossSpan-a} with symbolic payoffs $z_i$s and symbolic chance probabilities $p_i$s (with constraints on $p_i$'s). An extensive
form game can be obtained from a game structure by plugging in values for $z_i$s and $p_i$s.  We work with game structures because the
notions of perfect recall and imperfect recall can be determined
simply by looking at the game-structure.


Formally, a game-structure $\Tt$ is a tuple $(V, L, r, A, E, \Ii)$
where $V$ is a finite set of non-terminal nodes partitioned as
$V_{\Max}$, $V_{\Min}$ and $V_{\chance}$; $L$ is a finite set of leaves;
$r \in V$ is a root node; $A = A_{\Max} \cup A_{\Min}$ is a finite set
of actions; $E \incl V \times (V \cup L)$ is an edge
relation that induces a directed tree; edges originating from $V_{\Max} \cup V_{\Min}$ are labelled with actions from $A$; we write $u \xra{a} v$ if
$(u, v)$ is labelled with $a$, and assume that there is no incoming edge
$u \xra{} r$ to the root node $r$; $\Ii = \Ii_{\Max} \cup \Ii_{\Min}$
is a set of information sets for $i \in \{ \Max, \Min \}$, each
information set $I \in \Ii_i$ is a subset of vertices belonging to
$i$, i.e. $I \incl V_i$, and moreover, the set of information sets
$\Ii_i$ partitions $V_i$. E.g., in \cref{fig-allexmp-pftrec},
$\Ii_{\Max} = \{I_1, I_2, I_3\}$ and $I_1 = \{r\}, I_2 = \{u_3, u_4\}$
and $I_3 = \{u_5, u_6\}$. We can understand these information sets as a signal that the player receives when she reaches a node in it. On receiving the signal, the player knows the actions that are available to play at that position. 

An information set models the fact that a player cannot distinguish
between the nodes within it. Therefore, the set of outgoing actions
from each node in an information set is required to be the same. This
allows us to define $\act(I)$ as the set of actions available at
information set $I$. E.g., in \cref{fig-allexmp-pftrec},
$\act(I_2) = \{c, d\}$. For technical convenience, we make a second
assumption: for all $I, I' \in \Ii$ with $I \neq I'$, we have
$\act(I) \cap \act(I') = \emptyset$. Therefore, the actions identify
the information sets. With this assumption, in \cref{fig:match-penny-3-die}, the actions of Alice should be seen as $H_A, T_A$ and those of Bob's as $H_B, T_B$. But we omit the subscripts in the figure for clarity. 
\begin{definition}[Extensive form games]\label{def:ext-form-games}
  A two-player zero-sum game in extensive form is a tuple
  $(\Tt,\d, \Uu)$ where $\Tt$ is a game-structure, $\d$ is the
  \emph{chance probability} associating to each $\chance$ node, a
  probability distribution on the outgoing actions, and
  $\Uu : L \mapsto \Rat $ is the utility function associating a payoff
  to each leaf.
\end{definition}

The \emph{size} of a game is the sum of the bit-lengths of all chance probabilities and leaf
payoffs in it. A \emph{behavioral strategy} for player $\Max$ ($\Min$ resp.) assigns a probability
distribution to $\act(I)$ for each $I \in \Ii_{\Max}$ ($\Ii_{\Min}$ resp.). Once we fix behavioral strategies $\sigma$ and $\tau$ for $\Max$ and $\Min$ respectively,
each edge in the game has an associated probability of being taken,
given by the corresponding strategy or $\chance$. The probability of reaching a leaf $u \in L$ is given by the product of all the numbers along the path to the leaf. Consider \cref{fig:shuffle-a}.
Let $\sigma$ assign $\frac{1}{4}$ to $b$ and $\frac{3}{4}$ to
$\bar{b}$; $0$ and $1$ to $c$ and $\bar{c}$, and $\frac{1}{3}$ to $a$
and $\frac{2}{3}$ to $\bar{a}$. The probability of reaching the leaf $b \bar{a}$ is 
then: $p_1 \times \frac{1}{4} \times \frac{2}{3}$. For a leaf $u$, we denote this quantity by
$\prob_{\sigma, \tau}(u)$. The \emph{expected payoff} $\Ee(\s, \t)$
when $\Max$ plays $\sigma$ and $\Min$ plays $\tau$, then equals
$\sum_{u \in L} \prob_{\s, \t} (u) \Uu(u)$. The solution concept that we
will consider in this paper is the notion of maxmin.
The \emph{maxmin value} of a game is the
following: \[\max\limits_{\s}\min\limits_{\t}\Ee(\s,\t)\] where
$\s,\t$ are behavioral strategies of $\Max$ and $\Min$ respectively. A
strategy of $\Max$ which provides the maxmin value is called a
\emph{maxmin strategy}. In one-player games, we only have $\Max$ player and the maxmin value of the game is $\max\limits_{\s}\Ee(\s)$. For one-player non-absentminded games, the maxmin value can be in fact obtained by a \emph{pure strategy} -- pure strategies are special cases of behavioural strategies which assign either $0$ or $1$ to each action~\cite{KollerMegiddo::1992}.

The maxmin value of the game in \cref{fig:match-penny-3-die-a} is $\frac{2}{3}$ since Alice and Bob can win at most in 2 of the 3 die rolls by playing matching sides. Another way to see this is to consider the four possible pure strategies $HH, HT, TH, TT$, which induce payoffs $\frac{2}{3}$, $\frac{1}{3}$, $\frac{1}{3}$ and $\frac{2}{3}$ respectively. Now since, in the rest of the following two versions, the team has more information \footnote{This can be observed by the fact that information sets in each version are refinements of the previous versions.} they can guarantee at least $\frac{2}{3}$ by playing the same strategy. Interestingly, one can observe (by enumerating all pure strategies) that they cannot do better than that in any version. 
\paragraph*{Histories and recalls.} We now move on to describing the
various types of imperfect information, based on what the player
remembers about her history. A node $w \in V$ is reached by a unique
path from the root: $r = v_0 \xra{} v_1 \xra{} \cdots \xra{} v_n =
w$. Let $v_{i_1}, v_{i_2}, \dots, v_{i_k}$ be the vertices in this
sequence which do not belong to $\chance$. Then,
$\his(w) = a_{1} a_{2} \cdots a_{k-1}$, where $v_{i_j} \xra{a_j} v_{i_{j + 1}}$.
For a player $i \in \{\Max, \Min\}$ the history of $i$ at $w$, denoted
by $\his_i(w)$, is the sequence of player $i$'s actions in the path to
$w$, which is simply the sub-sequence of $\his(w)$ restricted to
actions from $A_i$. E.g.: in \cref{fig-allexmp-pftrec},
$\his_{\Max}(u_3) = \his_{\Max}(u_4) = a$; in
\cref{fig:shuffle-a}, $\his_{\Max}(u_3) = b$ and $\his_{\Max}(u_2) = \epsilon$, the empty sequence. It is important to remark that this definition uses the assumption that actions determine information sets -- otherwise, we would need to incorporate the information sets that were visited along the way, into the history.



Let $\Hh$ denote the set of all histories and $\Hh_i$ be the set of
all histories of player $i$. For an information set $I \in \Ii_i$ let
$\Hh(I) = \{ \his(u) \mid u \in I\}$ be the set of histories of all
nodes in $I$. Similarly, we can define $\Hh_i(I)$ with respect to
$\Hh_i$. Let $\Hh(L)$ denote the set of all leaf histories.
When $\Hh_i(I)$ has multiple histories, at a node $v \in I$ the player
does not remember which history she traversed to reach $v$. Hence the
player loses information. For two
nodes $u$ and $v$ in $I$, comparing $\his_i(u)$ and $\his_i(v)$
reveals the loss or retention of previously withheld information at
the respective nodes. To capture this there are different notions of
\emph{recall}.

\emph{Perfect recall.} Player $i$ is said to have \emph{perfect
  recall} ($\pfr$) if for every $I \in \Ii_i$, and every pair of
distinct vertices $u, v \in I$, we have $\his_i(u) = \his_i(v)$,
i.e. $|\Hh_i(I)| = 1$.  Otherwise, the player is said to have imperfect
recall.  \cref{fig-allexmp-pftrec} is an example of a perfect recall
game.
\emph{Imperfect recall.} \cref{fig:shuffle-a} gives an example of
a game-structure that has imperfect recall. Notice that states $u_3$
and $u_4$ lie in the same information set $I_3$, but the sequence of
the player's actions leading to these states is different: history at
$u_3$ is $b$, whereas at $u_4$ it is $\bar{b}$. 
Within imperfect recall,
there are distinctions. The imperfect recall in
~\cref{fig:shuffle-b} and the one in ~\cref{fig:shuffle-a}
are in some sense different: in ~\cref{fig:shuffle-b}, the
inability to distinguish between the two nodes in $I_1$ can be traced back to
a point in the past where she forgets her own action from some
information set ($I_3$ in this case), whereas in \cref{fig:shuffle-a}, the player has been able to
distinguish between the two outcomes of the $\chance$ node, but later
forgets at $I_3$ where she started from, leading to four histories $b,\bar{b},c$ and $\bar{c}$ at $I_3$.


\emph{A-loss recall.} Game-structures as in \cref{fig:shuffle-b} are said to have \emph{A-loss
  recall}. A consequence of having A-loss recall is that a player always remembers any new information
gained from $\chance$ outcomes, which is not the case
in~\cref{fig:shuffle-a}. Player $i$ has \emph{A-loss recall}
($\alr$) if for all $I \in \Ii_i$, and every pair of distinct vertices
$u, v \in I$, either $\his_i(u) = \his_i(v)$, or $\his_i(u)$ is of the
form $s a s_1$, and $\his_i(v)$ of the form $s b s_2$, where
$a, b \in \act(I')$ for some $I' \in \Ii$, with $a \neq b$. The game in \cref{fig:match-penny-3-die-a} has A-loss recall, whereas the others, \cref{fig:match-penny-3-die-b} and \cref{fig:match-penny-3-die-c} do not.  

Finally, player $i$ is said to be
\emph{non-absentminded} ($\nam$) if $\forall u, v \in V_i$ with $u$
lying on the path to $v$, the information set that $u$ belongs to is
different from the information set that $v$ belongs
to, i.e. all nodes of $i$ on a path from $r$ to leaf node lie in distinct
information sets. \cref{fig-allexmp-absentm} is an example where
$\Max$ is absentminded, since both $r$ and $u_1$ lie in the same
information set. Notice that $\pfr$ implies $\alr$, which in turn implies implies $\nam$. 

When $\Max$ and $\Min$ have recalls $R_{\Max}, R_{\Min} \in \{ $\pfr$,~$\alr$,~$\nam$ \} $
respectively we will denote the game as a
$(R_{\Max}, R_{\Min})$-game. A one-player game with recall $R$ is denoted as $R$-game. In this paper we are only concerned with one-player $\nam$-games and two-player $(\nam,\nam)$-games. Let us now recall some known
results.
\begin{itemize}\item A maxmin solution in a $(\pfr,\alr)$-game can be computed in
  polynomial- time~\\\cite{KollerMegiddo::1992,vonStengel::1996,kaneko1995behavior}. As a corollary, an optimal solution in a one-player $\alr$-game can be computed in
  polynomial-time ~\cite{kaneko1995behavior}.

\item The maxmin decision problem for $(\nam,\nam)$-games is both
  $\NP$-hard~\cite{KollerMegiddo::1992} and

  
  $\sqsum$-hard~\cite{GPS20} \footnote{$\sqsum$ is the decision problem of checking
    if the sum of the square roots of $k$ positive integers is less
    than another positive number}.
   The $\NP$-hardness and the $\sqsum$-hardness hold even for
  $(\alr,\pfr)$-games~\cite{Cermak::2018,GPS20}. The maxmin decision problem for one-player $\nam$-games is
  $\NP$-complete~\cite{KollerMegiddo::1992}.
\end{itemize}


Our core idea is to view game structures through the polynomials they generate. 
\paragraph*{Leaf monomials.} In a game structure, assigning variable $x_a$ to each action
$a$, the monomial obtained by taking the product of all $x_a$ along the path to each leaf $t$ is called
a \emph{leaf monomial}, and denoted as $\mu(t)$. E.g., the leaf
monomials of the game-structure in \cref{fig-allexmp-pftrec} are
$\{ x_ax_c, x_ax_d, x_bx_e, x_bx_f\}$. For a game structure $\Tt$, we will write $X(\Tt)$ for the set of leaf monomials. For a game $G$, let
$\prob_{\chance}(t)$ denote the product of $\chance$ probabilities in
the path to $t$. The polynomial given by
$\sum\limits_{t \in L} \prob_{\chance}(t) \cdot \Uu(t) \cdot \mu(t)$
is called the \emph{payoff polynomial} of a game. 
A constraint of the form $\sum\limits_{a \in \act(I)} x_a = 1$ for an
information set $I$ will be called a \emph{strategy constraint}. Any non-negative valuation satisfying these constraints gives a behavioral strategy to the players.
The maxmin value in a game can be given by the maxmin of the payoff polynomial over all possible values satisfying the strategy constraints.


\paragraph*{Overview of our work}
In this work, our mantra for simplifying games is to find simpler games with same payoff polynomials (upto renaming of variables). Leaf monomials are the building blocks of payoff polynomials. We give methods to generate from a given game-structure $\Tt$, a transformed game-structure $\Tt'$ with A-loss recall such that: either $\Tt'$ has the same set of leaf monomials (Section~\ref{sec:shuffled-loss-recall}), or each leaf monomial of $\Tt$ is a linear combination of the leaf monomials of $\Tt'$ (Section~\ref{sec:span}).


\endinput



















%!TEX root = ../main.tex

\begin{figure}
\begin{subfigure}{0.5\columnwidth}
\centering
\tikzset{
triangle/.style = {regular polygon,regular polygon sides=3,draw,inner sep = 2},
circ/.style = {circle,fill=cyan!10,draw,inner sep = 3},
term/.style = {circle,draw,inner sep = 1.5,fill=black},
sq/.style = {rectangle,fill=gray!20, draw, inner sep = 4}
}

\begin{tikzpicture}[scale=0.85]
\tikzstyle{level 1}=[level distance=9mm,sibling distance = 22mm]
\tikzstyle{level 2}=[level distance=7mm,sibling distance=10mm]
\tikzstyle{level 3}=[level distance=7mm,sibling distance=6mm]
\tikzstyle{level 4}=[level distance=7mm,sibling distance=5mm]

%node (ij) is the j th node in i th level

\begin{scope}[->, >=stealth]
\node (0) [triangle] {}
child {
  node (00) [circ] {}
  child {
    node (000) [circ] {}
    child {
      node (0000) [term, label=below:{\scriptsize $z_1$}] {}
      edge from parent node [left] {\scriptsize $a$}
    }
    child {
      node (0001) [term, label=below:{\scriptsize $z_2$}] {}
      edge from parent node [right] {\scriptsize $\bar{a}$}
      }
    edge from parent node [left] {\scriptsize $b$}
  }
  child {
    node (001) [circ] {}
    child {
      node (0010) [term, label=below:{\scriptsize $z_3$}] {}
      edge from parent node [left] {\scriptsize $a$}
    }
    child {
      node (0011) [term, label=below:{\scriptsize $z_4$}] {}
      edge from parent node [right] {\scriptsize $\bar{a}$}
      }
    edge from parent node [right] {\scriptsize $\bar{b}$} 
  }
  edge from parent node [above] {\scriptsize $p_1$}
}
child {
  node (01) [circ] {}
   child {
     node (010) [circ] {}
     child {
      node (0100) [term, label=below:{\scriptsize $z_5$}] {}
      edge from parent node [left] {\scriptsize $a$}
    }
    child {
      node (0101) [term, label=below:{\scriptsize $z_6$}] {}
      edge from parent node [right] {\scriptsize $\bar{a}$}
      }
    edge from parent node [left] {\scriptsize $c$}
  }
  child {
    node (011) [circ] {}
    child {
      node (0110) [term, label=below:{\scriptsize $z_7$}] {}
      edge from parent node [left] {\scriptsize $a$}
    }
    child {
      node (0111) [term, label=below:{\scriptsize $z_8$}] {}
      edge from parent node [right] {\scriptsize $\bar{a}$}
      }
    edge from parent node [right] {\scriptsize $\bar{c}$} 
  }
  edge from parent node [above] {\scriptsize $p_2$}
}
;
 \node[fit=(00),dashed,thick,blue, draw, circle,inner sep=1pt] {};
  \node[fit=(01),dashed,thick,red, draw, circle,inner sep=1pt] {};
\end{scope}

\draw [dashed, thick, ForestGreen, in=150,out=30] (000) to (001);
\draw [dashed, thick, ForestGreen, in=150,out=30] (001) to (010);
\draw [dashed, thick, ForestGreen, in=150,out=30] (010) to (011);
%\draw [dashed, thick, blue, in=150,out=30] (000) to (001);
%\draw [dashed, thick, red, in=150,out=30] (010) to (011);

\node [black] at (0,0.35) {\scriptsize $r$};
\node [black] at (-1.5,-0.55) {\scriptsize $u_1$};
\node [black] at (1.5, -0.55) {\scriptsize $u_2$};
\node [black] at (-2, -1.6) {\scriptsize $u_3$};
\node [black] at (-.25, -1.7) {\scriptsize $u_4$};

\node [black] at (0.25, -1.7) {\scriptsize $u_5$};
\node [black] at (2, -1.6) {\scriptsize $u_6$};

%obs labels
\node [ForestGreen] at (0,-1.1) {\scriptsize $I_3$};
\node [blue] at (-.55,-.9) {\scriptsize $I_1$};
\node [red] at (.55,-.9) {\scriptsize $I_2$};

\end{tikzpicture}
\caption{$\Max$ without $\alr$ but has $\salr$}
\label{fig:shuffle-a}
\end{subfigure}
\begin{comment}
\begin{subfigure}{0.45\columnwidth}
%\centering
\tikzset{
triangle/.style = {regular polygon,regular polygon sides=3,draw,inner sep = 2},
circ/.style = {circle,fill=cyan!10,draw,inner sep = 3},
term/.style = {circle,draw,inner sep = 1.5,fill=black},
sq/.style = {rectangle,fill=gray!20, draw, inner sep = 4}
}

\begin{tikzpicture}[scale=0.85]
\tikzstyle{level 1}=[level distance=9mm,sibling distance = 22mm]
\tikzstyle{level 2}=[level distance=7mm,sibling distance=10mm]
\tikzstyle{level 3}=[level distance=7mm,sibling distance=6mm]
\tikzstyle{level 4}=[level distance=7mm,sibling distance=5mm]

%node (ij) is the j th node in i th level

\begin{scope}[->, >=stealth]
\node (0) [circ] {}
child {
  node (00) [triangle] {}
  child {
    node (000) [circ] {}
    child {
      node (0000) [term, label=below:{}] {}
      edge from parent node [left] {\scriptsize $b$}
    }
    child {
      node (0001) [term, label=below:{}] {}
      edge from parent node [right] {\scriptsize $\bar{b}$}
      }
    edge from parent node [left] {}
  }
  child {
    node (001) [circ] {}
    child {
      node (0010) [term, label=below:{}] {}
      edge from parent node [left] {\scriptsize $b$}
    }
    child {
      node (0011) [term, label=below:{}] {}
      edge from parent node [right] {\scriptsize $\bar{b}$}
      }
    edge from parent node [right] {} 
  }
  edge from parent node [above] {\scriptsize $a$}
}
child {
  node (01) [triangle] {}
   child {
     node (010) [circ] {}
     child {
      node (0100) [term, label=below:{}] {}
      edge from parent node [left] {\scriptsize $c$}
    }
    child {
      node (0101) [term, label=below:{}] {}
      edge from parent node [right] {\scriptsize $\bar{c}$}
      }
    edge from parent node [left] {}
  }
  child {
    node (011) [circ] {}
    child {
      node (0110) [term, label=below:{}] {}
      edge from parent node [left] {\scriptsize $c$}
    }
    child {
      node (0111) [term, label=below:{}] {}
      edge from parent node [right] {\scriptsize $\bar{c}$}
      }
    edge from parent node [right] {} 
  }
  edge from parent node [above] {\scriptsize $\bar{a}$}
}
;
\end{scope}

%\draw [dashed, thick, ForestGreen, in=150,out=30] (00) to (01);

\node[fit=(0),dashed,thick,ForestGreen, draw, circle,inner sep=1pt] {};
\draw [dashed, thick, blue, in=150,out=30] (000) to (010);
\draw [dashed, thick, red, in=150,out=30] (001) to (011);


\node [black] at (0,0.45) {\scriptsize $r$};
\node [black] at (-1.1,-0.45) {\scriptsize $u_1$};
\node [black] at (1.1, -0.45) {\scriptsize $u_2$};
\node [black] at (-2, -1.65) {\scriptsize $u_3$};
\node [black] at (-.2, -1.65) {\scriptsize $u_4$};

\node [black] at (0.25, -1.65) {\scriptsize $u_5$};
\node [black] at (2, -1.65) {\scriptsize $u_6$};

%obs labels
\node [ForestGreen] at (0,-0.6) {\scriptsize $I_3$};
\node [blue] at (-0.35,-1) {\scriptsize $I_1$};
\node [red] at (0.35,-1) {\scriptsize $I_2$};

\end{tikzpicture}
\caption{}
\label{fig:shuffle-c}
\end{subfigure}%
\end{comment}
\begin{subfigure}{0.48\columnwidth}
\centering
\tikzset{
triangle/.style = {regular polygon,regular polygon sides=3,draw,inner sep = 2},
circ/.style = {circle,fill=cyan!10,draw,inner sep = 3},
term/.style = {circle,draw,inner sep = 1.5,fill=black},
sq/.style = {rectangle,fill=gray!20, draw, inner sep = 4}
}

\begin{tikzpicture}[scale=0.85]
\tikzstyle{level 1}=[level distance=9mm,sibling distance = 22mm]
\tikzstyle{level 2}=[level distance=7mm,sibling distance=10mm]
\tikzstyle{level 3}=[level distance=7mm,sibling distance=6mm]
\tikzstyle{level 4}=[level distance=7mm,sibling distance=5mm]

%node (ij) is the j th node in i th level

\begin{scope}[->, >=stealth]
\node (0) [circ] {}
child {
  node (00) [triangle] {}
  child {
    node (000) [circ] {}
    child {
      node (0000) [term, label=below:{\scriptsize $z_1$}] {}
      edge from parent node [left] {\scriptsize $b$}
    }
    child {
      node (0001) [term, label=below:{\scriptsize $z_3$}] {}
      edge from parent node [right] {\scriptsize $\bar{b}$}
      }
    edge from parent node [left] {}
  }
  child {
    node (001) [circ] {}
    child {
      node (0010) [term, label=below:{\scriptsize $z_5$}] {}
      edge from parent node [left] {\scriptsize $c$}
    }
    child {
      node (0011) [term, label=below:{\scriptsize $z_7$}] {}
      edge from parent node [right] {\scriptsize $\bar{c}$}
      }
    edge from parent node [right] {} 
  }
  edge from parent node [above] {\scriptsize $a$}
}
child {
  node (01) [triangle] {}
   child {
     node (010) [circ] {}
     child {
      node (0100) [term, label=below:{\scriptsize $z_2$}] {}
      edge from parent node [left] {\scriptsize $b$}
    }
    child {
      node (0101) [term, label=below:{\scriptsize $z_4$}] {}
      edge from parent node [right] {\scriptsize $\bar{b}$}
      }
    edge from parent node [left] {}
  }
  child {
    node (011) [circ] {}
    child {
      node (0110) [term, label=below:{\scriptsize $z_6$}] {}
      edge from parent node [left] {\scriptsize $c$}
    }
    child {
      node (0111) [term, label=below:{\scriptsize $z_8$}] {}
      edge from parent node [right] {\scriptsize $\bar{c}$}
      }
    edge from parent node [right] {} 
  }
  edge from parent node [above] {\scriptsize $\bar{a}$}
}
;
\end{scope}

%\draw [dashed, thick, ForestGreen, in=150,out=30] (00) to (01);

\node[fit=(0),dashed,thick,ForestGreen, draw, circle,inner sep=1pt] {};
\draw [dashed, thick, blue, in=150,out=30] (000) to (010);
\draw [dashed, thick, red, in=150,out=30] (001) to (011);

%\node [black] at (0,0.45) {\scriptsize $r$};
%\node [black] at (-1.1,-0.45) {\scriptsize $u_1$};
%\node [black] at (1.1, -0.45) {\scriptsize $u_2$};
%\node [black] at (-2, -1.65) {\scriptsize $u_3$};
%\node [black] at (-.2, -1.65) {\scriptsize $u_4$};
%
%\node [black] at (0.25, -1.65) {\scriptsize $u_5$};
%\node [black] at (2, -1.65) {\scriptsize $u_6$};

%obs labels
\node [ForestGreen] at (0,-0.6) {\scriptsize $I_3$};
\node [blue] at (-0.35,-1) {\scriptsize $I_1$};
\node [red] at (0.35,-1) {\scriptsize $I_2$};

\node[black] at (-1.5,-.95) {\scriptsize $p_1$};
\node[black] at (-.73,-.95) {\scriptsize $p_2$};

\node[black] at (1.5,-.95) {\scriptsize $p_2$};
\node[black] at (.73,-.95) {\scriptsize $p_1$};
\end{tikzpicture}
\caption{$\Max$ with $\alr$}
\label{fig:shuffle-b}
\end{subfigure}
\caption{Equivalent $\alr$ game using $\salr$ for game without $\alr$ }
\label{fig:shuffle}
\end{figure}


\paragraph*{Shuffled A-loss recall}
The game-structure in \cref{fig:shuffle-a} does not have A-loss recall. This is because the player knows about $\chance$ outcomes $I_1$ and $I_2$ which she forgets at $I_3$.  Now, consider the game-structure in~\cref{fig:shuffle-c}, obtained by \emph{shuffling} the actions ($a$ goes above $b$ and $c$). This game-structure has A-loss recall. The crucial observation is that both the game-structures, \cref{fig:shuffle-a} and \cref{fig:shuffle-c}, lead to the same ``leaf monomials'': on assigning variable $x_a$ to an action labeled $a$, the product of the variables along the path to each leaf produces a leaf monomial. For instance, the leaf monomials for the game-structures in ~\cref{fig:shuffle-a} and ~\cref{fig:shuffle-c} respectively are $\{x_ax_b,x_ax_{\bar{b}},x_{\bar{a}}x_b,x_{\bar{a}}x_{\bar{b}},x_ax_c, x_ax_{\bar{c}},x_{\bar{a}}x_c,x_{\bar{a}}x_{\bar{c}} \}$.
We say that the game-structure of ~\cref{fig:shuffle-a} has \emph{shuffled A-loss recall}. Even though the game originally does not have A-loss recall, it can be shuffled in some way to get an A-recall structure. 
Not every game-structure has shuffled A-loss recall.

Our results:
\begin{itemize}
\item We provide a polynomial-time algorithm to identify whether a game-structure has shuffled A-loss recall. If the answer is yes, the algorithm also computes the shuffled game-structure.

\item As a result, we are able to show that one-player shuffled A-loss recall games can be solved in polynomial-time. Similarly, we deduce that two player games between a perfect recall player and a shuffled A-loss recall player can be solved in polynomial-time.
\end{itemize}






%!TEX root = ../main.tex

\begin{figure}
\centering

\begin{subfigure}{.3\columnwidth}
%\centering
\tikzset{
triangle/.style = {regular polygon,regular polygon sides=3,draw,inner sep = 2},
circ/.style = {circle,fill=cyan!10,draw,inner sep = 3},
term/.style = {circle,draw,inner sep = 1.5,fill=black},
sq/.style = {rectangle,fill=gray!20, draw, inner sep = 4}
}

\begin{tikzpicture}[scale=0.85]
\tikzstyle{level 1}=[level distance=9mm,sibling distance = 22mm]
\tikzstyle{level 2}=[level distance=7mm,sibling distance=10mm]
\tikzstyle{level 3}=[level distance=7mm,sibling distance=6mm]
\tikzstyle{level 4}=[level distance=7mm,sibling distance=5mm]

%node (ij) is the j th node in i th level

\begin{scope}[->, >=stealth]
\node (0) [triangle] {}
child {
  node (00) [circ] {}
  child {
    node (000) [circ] {}
    child {
      node (0000) [term, label=below:{\scriptsize $z_1$}] {}
      edge from parent node [left] {\scriptsize $c$}
    }
    child {
      node (0001) [term, label=below:{\scriptsize $z_2$}] {}
      edge from parent node [right] {\scriptsize $\bar{c}$}
      }
    edge from parent node [left] {\scriptsize $a$}
  }
  child {
    node (001) [circ] {}
    child {
      node (0010) [term, label=below:{\scriptsize $z_3$}] {}
      edge from parent node [left] {\scriptsize $d$}
    }
    child {
      node (0011) [term, label=below:{\scriptsize $z_4$}] {}
      edge from parent node [right] {\scriptsize $\bar{d}$}
      }
    edge from parent node [right] {\scriptsize $\bar{a}$} 
  }
  edge from parent node [above] {\scriptsize $p_1$}
}
child {
  node (01) [circ] {}
   child {
     node (010) [circ] {}
     child {
      node (0100) [term, label=below:{\scriptsize $z_5$}] {}
      edge from parent node [left] {\scriptsize $c$}
    }
    child {
      node (0101) [term, label=below:{\scriptsize $z_6$}] {}
      edge from parent node [right] {\scriptsize $\bar{c}$}
      }
    edge from parent node [left] {\scriptsize $b$}
  }
  child {
    node (011) [circ] {}
    child {
      node (0110) [term, label=below:{\scriptsize $z_7$}] {}
      edge from parent node [left] {\scriptsize $d$}
    }
    child {
      node (0111) [term, label=below:{\scriptsize $z_8$}] {}
      edge from parent node [right] {\scriptsize $\bar{d}$}
      }
    edge from parent node [right] {\scriptsize $\bar{b}$} 
  }
  edge from parent node [above] {\scriptsize $p_2$}
}
;
 \node[fit=(00),dashed,thick,blue, draw, circle,inner sep=1pt] {};
  \node[fit=(01),dashed,thick,red, draw, circle,inner sep=1pt] {};
\end{scope}

\draw [dashed, thick, ForestGreen, in=150,out=30] (000) to (010);
\draw [dashed, thick, brown, in=150,out=30] (001) to (011);
\node [black] at (0,0.35) {\scriptsize $r$};
\node [black] at (-1.5,-0.55) {\scriptsize $u_1$};
\node [black] at (1.5, -0.55) {\scriptsize $u_2$};
\node [black] at (-2, -1.6) {\scriptsize $u_3$};
\node [black] at (-.25, -1.7) {\scriptsize $u_4$};

\node [black] at (0.25, -1.7) {\scriptsize $u_5$};
\node [black] at (2, -1.6) {\scriptsize $u_6$};

%obs labels

\node [blue] at (-1.7,-.9) {\scriptsize $I_1$};
\node [red] at (1.7,-.9) {\scriptsize $I_2$};
\node [ForestGreen] at (-0.2,-1) {\scriptsize $I_3$};
\node [brown] at (0.3,-1) {\scriptsize $I_4$};

\end{tikzpicture}
\caption{$\Max$ without $\salr$}
\label{fig:alossSpan-a}
\end{subfigure}

\begin{subfigure}{.6\columnwidth}
%\centering
\tikzset{
triangle/.style = {regular polygon,regular polygon sides=3,draw,inner sep = 2},
circ/.style = {circle,fill=cyan!10,draw,inner sep = 3},
term/.style = {circle,draw,inner sep = 1.5,fill=black},
sq/.style = {rectangle,fill=gray!20, draw, inner sep = 4}
}

\begin{tikzpicture}[scale=0.8]
\tikzstyle{level 1}=[level distance=9mm,sibling distance = 50mm]
\tikzstyle{level 2}=[level distance=5mm,sibling distance=25mm]
\tikzstyle{level 3}=[level distance=9mm,sibling distance=12mm]
\tikzstyle{level 4}=[level distance=10mm,sibling distance=6mm]

%node (ij) is the j th node in i th level

\begin{scope}[->, >=stealth]
\node (0) [circ] {}
child {
  node (00) [circ] {}
  child {
  node (000) [triangle] {}
   child {
     node (0000) [circ] {}
     child {
      node (00000) [term, label=below:{\scriptsize $w_1$}] {}
      edge from parent node [left] {\scriptsize $a$}
    }
    child {
      node (00001) [term, label=below:{\scriptsize $w_2$}] {}
      edge from parent node [right] {\scriptsize $\bar{a}$}
      }
    edge from parent node [left,pos=0.2] {\scriptsize $\frac{1}{2}$}
  }
  child {
    node (0001) [circ] {}
    child {
      node (00010) [term, label=below:{\scriptsize $w_3$}] {}
      edge from parent node [left] {\scriptsize $b$}
    }
    child {
      node (00011) [term, label=below:{\scriptsize $w_4$}] {}
      edge from parent node [right] {\scriptsize $\bar{b}$}
      }
    edge from parent node [right,pos=0.2] {\scriptsize $\frac{1}{2}$} 
     }
  edge from parent node [above] {\scriptsize $d$}
  }
  child {
  node (001) [triangle] {}
   child {
     node (0010) [circ] {}
     child {
      node (00100) [term, label=below:{\scriptsize $w_5$}] {}
      edge from parent node [left] {\scriptsize $a$}
    }
    child {
      node (00101) [term, label=below:{\scriptsize $w_6$}] {}
      edge from parent node [right] {\scriptsize $\bar{a}$}
      }
    edge from parent node [left,pos=0.2] {\scriptsize $\frac{1}{2}$}
  }
  child {
    node (0011) [circ] {}
    child {
      node (00110) [term, label=below:{\scriptsize $w_7$}] {}
      edge from parent node [left] {\scriptsize $b$}
    }
    child {
      node (00111) [term, label=below:{\scriptsize $w_8$}] {}
      edge from parent node [right] {\scriptsize $\bar{b}$}
      }
    edge from parent node [right,pos=0.2] {\scriptsize $\frac{1}{2}$} 
     }
  edge from parent node [above] {\scriptsize $\bar{d}$}
  }
  edge from parent node [above] {\scriptsize $c$}
  }
child {
  node (01) [circ] {}
  child {
  node (010) [triangle] {}
   child {
     node (0100) [circ] {}
     child {
      node (01000) [term, label=below:{\scriptsize $w_9$}] {}
      edge from parent node [left] {\scriptsize $a$}
    }
    child {
      node (01001) [term, label=below:{\scriptsize $w_{10}$}] {}
      edge from parent node [right] {\scriptsize $\bar{a}$}
      }
    edge from parent node [left,pos=0.2] {\scriptsize $\frac{1}{2}$}
  }
  child {
    node (0101) [circ] {}
    child {
      node (01010) [term, label=below:{\scriptsize $w_{11}$}] {}
      edge from parent node [left] {\scriptsize $b$}
    }
    child {
      node (01011) [term, label=below:{\scriptsize $w_{12}$}] {}
      edge from parent node [right] {\scriptsize $\bar{b}$}
      }
    edge from parent node [right,pos=0.2] {\scriptsize $\frac{1}{2}$} 
  }
  edge from parent node [above] {\scriptsize $d$}
}
  child {
  node (011) [triangle] {}
   child {
     node (0110) [circ] {}
     child {
      node (01100) [term, label=below:{\scriptsize $w_{13}$}] {}
      edge from parent node [left] {\scriptsize $a$}
    }
    child {
      node (01101) [term, label=below:{\scriptsize $w_{14}$}] {}
      edge from parent node [right] {\scriptsize $\bar{a}$}
      }
    edge from parent node [left,pos=0.2] {\scriptsize $\frac{1}{2}$}
  }
  child {
    node (0111) [circ] {}
    child {
      node (01110) [term, label=below:{\scriptsize $w_{15}$}] {}
      edge from parent node [left] {\scriptsize $b$}
    }
    child {
      node (01111) [term, label=below:{\scriptsize $w_{16}$}] {}
      edge from parent node [right] {\scriptsize $\bar{b}$}
      }
    edge from parent node [right,pos=0.2,pos=0.2] {\scriptsize $\frac{1}{2}$} 
     }
  edge from parent node [above] {\scriptsize $\bar{d}$}
  }
  edge from parent node [above] {\scriptsize $\bar{c}$}
}
;
\end{scope}

%\draw [dashed, thick, ForestGreen, in=150,out=30] (00) to (01);

\node[fit=(0),dashed,thick,ForestGreen, draw, circle,inner sep=1pt] {};
\draw [dashed, thick, brown, in=165,out=15] (00) to (01);
\draw [dashed, thick, blue, in=150,out=30] (0000) to (0010);
\draw [dashed, thick, blue, in=150,out=30] (0010) to (0100);
\draw [dashed, thick, blue, in=150,out=30] (0100) to (0110);
\draw [dashed, thick, red, in=150,out=30] (0001) to (0011);
\draw [dashed, thick, red, in=150,out=30] (0101) to (0111);
\draw [dashed, thick, red, in=150,out=30] (0011) to (0101);

%\node [black] at (0,0.45) {\scriptsize $r$};
%\node [black] at (-1.1,-0.45) {\scriptsize $u_1$};
%\node [black] at (1.1, -0.45) {\scriptsize $u_2$};
%\node [black] at (-2, -1.65) {\scriptsize $u_3$};
%\node [black] at (-.2, -1.65) {\scriptsize $u_4$};
%
%\node [black] at (0.25, -1.65) {\scriptsize $u_5$};
%\node [black] at (2, -1.65) {\scriptsize $u_6$};

%obs labels
\node [ForestGreen] at (0.55,0.1) {\scriptsize $I_3$};
\node [brown] at (0,-.8) {\scriptsize $I_4$};
\node [blue] at (-2.8,-1.6) {\scriptsize $I_1$};
\node [red] at (2.8,-1.6) {\scriptsize $I_2$};

%\node[black] at (-1.5,-.95) {\scriptsize $p_1$};
%\node[black] at (-.73,-.95) {\scriptsize $p_2$};

%\node[black] at (1.5,-.95) {\scriptsize $p_2$};
%\node[black] at (.73,-.95) {\scriptsize $p_1$};
\end{tikzpicture}
\caption{$\Max$ with $\alr$}
\label{fig:alossSpan-b}
\end{subfigure}
\caption{Equivalent $\alr$ game using $\alr$-span for game without $\salr$}
\label{fig:span}
\end{figure}


\paragraph*{Span}
We move on to another way of simplifying game-structures. The game-structure $\Tt_1$ \cref{fig:alossSpan-a} neither has perfect recall, nor A-loss recall. Using the characterization obtained in Section~\ref{}, we can show that it does not have  shuffled A-loss recall either. Now, consider the game-structure $\Tt'_1$ in \cref{fig:alossSpan-c}. It has A-loss recall. Each leaf monomial of $\Tt_1$ can be written as a linear combination of the monomials of $\Tt'_1$: for example, the leaf monomial $x_ax_{\bar{b}}$ or $\Tt_1$ is equal to $x_a x_{\bar{b}}x_c + x_a x_{\bar{b} \bar{c}}$, the sum of two leaf monomials of $\Tt'_1$.  The game-structure $\Tt_1$ is said to be \emph{spanned by} $\Tt'_1$. This property allows to solve games derived from the structure $\Tt_1$ by converting them into a game on $\Tt'_1$, and solving the resulting A-loss recall game. 
Our results:
\begin{itemize}
\item We show that every imperfect recall game without absent-mindedness~\cite{} is spanned by an A-loss recall game.

\item The caveat is that the smallest A-loss recall span may be of exponential size: we exhibit a family of game structures where this happens. 

\item Finally, we provide an algorithm to compute an A-loss recall span of smallest size. 
\end{itemize}

From a conceptual point of view, we provide the following novel outlook.
\begin{itemize}
	\item Solving every non-absentminded game is \emph{equivalent} to solving an A-loss recall game. 
\end{itemize}

Recall that imperfect recall games are $\NP$-hard in general. The above results show that in order to solve an imperfect recall game, one could either use an exponential-time algorithm on the game directly, or apply the above transformation into a potentially exponential-sized game, on which a polynomial-time algorithm can be used. 









\vspace{-5pt}
\section{Related Work}
\label{sec:related-work}
\vspace{-5pt}
\section{Related Work}

We review related literature on pre-training, cross-domain transfer learning, and multi-domain pre-training for graph data.

\stitle{Graph pre-training.}
Graph pre-training methods aim to extract inherent properties of graphs, often utilizing self-supervised learning approaches, which can be either generative \cite{hu2020gpt,li2023s,hou2022graphmae,jiang2023incomplete} or contrastive \cite{velivckovic2018deep,xia2022simgrace,xu2021self,li2022mining}. The pre-trained model is then employed to address downstream tasks through fine-tuning \cite{you2020graph,velivckovic2018deep,qiu2020gcc} or parameter-efficient adaptation methods, notably prompt-based learning \cite{sun2022gppt,liu2023graphprompt,yu2023generalized,fang2022universal}. However, these methods typically assume that the pre-training and downstream graphs originate from the same domain, such as different subgraphs of a large graph \cite{you2020graph,yu2023hgprompt} or collections of similar graphs within the same dataset \cite{hu2020gpt,qiu2020gcc}, failing to account for multiple domains in either pre-training or downstream graphs.

\stitle{Graph cross-domain transfer.}
This line of work aims to transfer single-source domain knowledge to a different target domain by leveraging domain-invariant properties across domains~\cite{ding2021cross,hassani2022cross,wang2021pre,wang2023cross}. However, they rely exclusively on a single source domain, failing to harness the extensive knowledge available across multiple domains. Additionally, these approaches are often tailored to specific tasks or domains \cite{ding2021cross,hassani2022cross,wang2021pre,wang2023cross}, limiting their generalization.

\stitle{Multi-domain graph pre-training.}
In the context of graphs from multiple domains, recent works \cite{liu2023one,tang2024higpt,xia2024opengraph} utilize large language models to align node features from different domains through textual descriptions, thereby limiting their applicability to text-attributed graphs \cite{zhaolearning,wen2023prompt,zhang2024text}. For graphs without textual attributes, GraphControl \cite{zhu2024graphcontrol} applies ControlNet~\cite{zhang2023adding} to incorporate target domain node features with the pre-trained model, while neglecting the alignment among multiple source domains. Another recent study proposes GCOPE~\cite{zhao2024all}, which employs domain-specific virtual nodes that interconnect nodes across domains, facilitating the alignment of feature distribution and homophily patterns. Meanwhile, MDGPT~\cite{yu2024text} pre-trains domain-specific tokens to align feature semantics across various domains. However, these studies do not account for structural variance across different domains, hindering their effectiveness in integrating multi-domain knowledge. On a related front, multi-task pre-training techniques \cite{wang2022multi,yu2023multigprompt} employ pretext tokens for each pre-training task. It is important to note that they address a distinct problem, aiming to overcome potential interference among multiple tasks within a single domain, rather than interference across multiple domains. 
%In our work, we propose structure tokens and dual prompts to overcome the limitations of current multi-domain graph pre-training methods. 
%Note that multi-task pre-training aims to reduce interference among different pre-training tasks within a single domain, distinct from our objective of multi-domain pre-training.



\vspace{-5pt}
\section{Problem Formulation}
\label{sec:optimization-formulation}
\vspace{-5pt}
% !TEX root = ../main.tex

Before discussing our method, let us briefly introduce some important notations that we will use throughout the paper.

\textbf{Notation.\,\,\,\,} For a general matrix $\bfZ \in \R^{m \times n}$, 
$\rk{\bfZ}$ denotes the rank of $\bfZ$.
For a given rank $r \in \N$, $C_r(\mathbf{Z}) = \bfU_r \mathbf{\Sigma}_r \bfV_r^T$, corresponding to the matrices formed by retaining only the top-$r$ singular vectors and singular values from the full SVD of $\mathbf{Z}$. The \citet{eckart1936approximation} theorem shows that $C_r(\bfZ) = \argmin_\bfM \norm{\bfZ - \bfM},\,\, \rk{\bfM} \leq r.$

For a square matrix $\bfZ \in \R^{n \times n}$,
$\Tr{\bfZ} = \sum_{i \in [n]}\bfZ_{ii}$ denotes the trace of $\bfZ$,
$\diag{\bfZ}$ denotes the diagonal matrix $\bfD \in \R^{n \times n}$ such that $\bfD_{ii} = \bfZ_{ii}$ for any $i \in [n]$, and $\bfD_{ij} = 0$ for any $i \neq j, \, i,j \in [n]$.
We also note $\mathbf{1}_n$ and $\mathbf{0}_n$ the vector of entries all ones and all zeros respectively of size $n \in \N$.

\textbf{Layer-wise Reconstruction Error.\,\,\,\,} 
A common approach in post-training LLM compression is to decompose the full-model compression problem into layer-wise subproblems. The quality of the solution for each subproblem is assessed by measuring the $\ell_2$ error between the output of the dense layer and that of the compressed one, given a set of input activations.\\
More formally, let $\bfWold \in \R^{\Nin \times \Nout}$ denote the (dense) weight matrix of layer $\ell$, where $\Nin$ and $\Nout$ denote the input and output dimension of the layer, respectively. Given a set of $N$ calibration samples, the input activation matrix can be represented as $\bfX \in \R^{NL \times \Nin}$, where $L$ is the sequence length of an LLM. It corresponds to the output of the previous layer $\ell - 1$.  The goal of the matrix decomposition algorithm is to find a sum of a sparse weight matrix $\bfWS$ and a low-rank weight matrix $\bfM$ that minimizes the reconstruction error between the original and new layer outputs, while satisfying a target sparsity constraint and a low-rank constraint. The optimization problem is given by
\begin{equation}\label{eq:matrix-decomposition}
       \min\nolimits_{\bfWS, \bfM} \,\, \left\|\bfX \bfWold-\bfX \pr{\bfWS + \bfM}\right\|_F^2 ~~~~\text{s.t. } ~~~ \bfWS\in \calCS, ~~~\rk{\bfM} \leq r.
\end{equation}
where $\bfWS, \bfM \in \R^{\Nin \times \Nout}$, $\calCS$ denotes the the sparsity-pattern constraint set.

\vspace{-5pt}
\section{Algorithm Design}
\label{sec:algorithm-design}
\vspace{-5pt}
% !TEX root = ../main.tex

The Optimization of Problem \eqref{eq:matrix-decomposition} is challenging: we need to jointly find a support for $\bfWS$ so that it is feasible for the set $\calCS$, a subspace of dimension $r$ where the low-rank matrix $\mathbf{M}$ lies, and optimal weights, within these constrained sets, for both matrices to minimize the layerwise-reconstruction error. While such formulation relates to the Robust-PCA literature \cite{chandrasekaran2011rank, candes2011robust, hintermuller2015robust}, the size of parameters in $\bfWS$ and $\bfM$ can reach over 100 million in the LLM setting. For instance, the size of a down projection in a FFN of a Llama3-405b \cite{dubey2024llama} has more than 800 million parameters. 
Classical methods fail to be deployed at this scale, which makes designing novel algorithms that are more computationally efficient a necessity to solve the layerwise-reconstruction matrix decomposition problem \eqref{eq:matrix-decomposition}.

In this paper, we propose to optimize problem \eqref{eq:matrix-decomposition} using an Alternating-Minimization approach \cite{hintermuller2015robust, zhou2011godec}. We aim to decompose the problem into two 'friendlier' subproblems and iteratively minimize each one of them. In particular, we would like to iteratively solve, at iteration $t$, the subproblem \Pone, which pertains to the sparse component of the matrix decomposition.
\begin{align}\label{eq:pone-pruning}
       \bfWS^{(t+1)} &\in \argmin\nolimits_{\bfWS} \,\, \left\|\bfX \bfWold-\bfX \pr{\bfWS + \bfM^{(t)}}\right\|_F^2 ~~~~\text{s.t. } ~~~ \bfWS\in \calCS\\
       &= \argmin\nolimits_{\bfWS} \,\, \left\|\bfX \pruningW^{(t)} - \bfX \bfWS\right\|_F^2 ~~~~\text{s.t. } ~~~ \bfWS\in \calCS \tag{$\pruningW^{(t)} := \bfWold - \bfM^{(t)}$}.
\end{align}
The second subproblem to be solved, at iteration $t$, which pertains to the low-rank component of the matrix decomposition problem, is \Ptwo.
\begin{align}\label{eq:ptwo-low-rank}
       \bfM^{(t+1)} &\in \argmin\nolimits_{\bfM} \,\, \left\|\bfX \bfWold-\bfX \pr{\bfWS^{(t+1)} + \bfM}\right\|_F^2 ~~~~\text{s.t. } ~~~\rk{\bfM} \leq r\\
        &= \argmin\nolimits_{\bfM} \,\, \left\|\bfX \lowrankW^{(t+1)} -\bfX \bfM\right\|_F^2 ~~~~\text{s.t. } ~~~\rk{\bfM} \leq r \tag{$\lowrankW^{(t+1)} := \bfWold - \bfWS^{(t+1)}$}.
\end{align}
Before proceeding to discuss algorithms that solve different variations of \Pone and \Ptwo, and draw connections between existing methods in the literature of \textit{model compression}, we remove the dependence on the iteration $t$ and study \eqref{eq:pone-pruning} rewritten as follows.
\begin{align}\label{eq:general-pruning}
       \bfWS^\star 
       &\in \argmin\nolimits_{\bfWS} \,\, \left\|\bfX \pruningW - \bfX \bfWS\right\|_F^2 ~~~~\text{s.t. } ~~~ \bfWS\in \calCS\\
       &= \argmin\nolimits_{\bfWS} \,\, \Tr{(\pruningW - \bfWS)^\top \bfH (\pruningW - \bfWS)}~~~~\text{s.t. } ~~~ \bfWS\in \calCS \tag{$\bfH = \bfXTX$}. 
\end{align}

Similarly, we study \eqref{eq:ptwo-low-rank} rewritten as follows.
\begin{align}\label{eq:general-low-rank}
   \bfM^\star &\in \argmin\nolimits_{\bfM} \,\, \left\|\bfX \lowrankW -\bfX \bfM\right\|_F^2 ~~~~\text{s.t. } ~~~\rk{\bfM} \leq r\\
   &= \argmin\nolimits_{\bfM} \,\, \Tr{(\lowrankW - \bfM)^\top \bfH (\lowrankW - \bfM)} ~~~~\text{s.t. } ~~~\rk{\bfM} \leq r \notag{}.
\end{align}




\subsection{Minimizing Subproblem \Pone}\label{section-pone}
\vspace{-2pt}
To solve \eqref{eq:general-pruning}, one can consider multiple variations for $\bfH$, the Hessian of the local layer-wise reconstruction error.
\vspace{-3pt}
\subsubsection{Data-Free version: $\bfX = \bfI_{\Nin \times \Nin} \implies \bfH = \bfI_{\Nin \times \Nin}$} 
\vspace{-3pt}
A data-free pruning method (without a calibration dataset) considers $\bfX$ to be an identity matrix in \eqref{eq:general-pruning}. When $\bfH$ is an identity matrix, equation \eqref{eq:general-pruning} can be solved to optimality and an optimal solution is obtained with Magnitude Pruning (MP, \cite{han2015learning, sze2020efficient}) using a simple Hard-Thresholding operator on the dense weight $\pruningW$ -- keeping the largest values and setting the remaining values to zero. Note that MP can be applied to unstructured \cite{han2015learning}, semi-structured N:M sparsity \cite{zhou2021learning}, and structured pruning \cite{meng2024alps}. This accommodates most sparsity sets $\calCS$ in the pruning literature.

\subsubsection{Diagonal-approximation: $\bfH = \diagn{\bfXTX}$}\label{subsection-pruning-diagonal-approximation}
\vspace{-3pt}
An efficient way to approach problem \eqref{eq:general-pruning} is to approximate the Hessian of the local layer-wise reconstruction error by its diagonal. An optimal solution in this case, can be obtained by Hard-Thresholding $\bfD \pruningW$, where $\bfD = \sqrt{\diagn{\bfXTX}}$. Note that this approximation results in the state-of-the-art pruning algorithm Wanda \cite{sun2023simple}. In fact, the importance metric, $S_{ij}$ introduced in Wanda for each entry $\pruningW_{ij}$ reads as follows. Here $\bfX_j$ denotes the $j^{th}$ column of the input activation matrix $\bfX$.
\begin{equation}\label{eq:wanda-metric}
    S_{ij} = \abs{\pruningW_{ij}} \cdot \norm{\bfX_j}_2 = \abs{\bfD \pruningW}_{ij}. \tag{$\bfD = \sqrt{\diagn{\bfXTX}}$}
\end{equation}
\citet{sun2023simple} show impressive results with this approximation for unstructured and semi-structured sparsity. OATS \cite{zhang2024oats} is inspired by Wanda and decomposes model weights into sparse plus low-rank using Alternating-Minimization; their sparse update reduces to this approximation (diagonal of the local layer-wise objective's Hessian).
\vspace{-3pt}
\subsubsection{Full Hessian: $\bfH = \bfXTX + \lambda \bfI$}\label{full-hessian-pruning}
\vspace{-3pt}
This approach aims to directly minimize \eqref{eq:general-pruning}. \citet{frantar2023sparsegpt} are the first to use the full Hessian of the local layer-wise reconstruction objective \Pone at the scale of LLMs for pruning using approximations at the algorithmic level (as opposed to an approximation at the optimization formulation level). \citet{meng2024alps} extend this formulation using the operator splitting technique ADMM \cite{boyd2011distributed} and show impressive results for unstructured sparsity and N:M sparsity. \citet{meng2024osscar} extend the formulation for structured sparsity by leveraging combinatorial optimization techniques.

Our framework works as a plug-in method for any pruning algorithm to minimize \Pone at iteration $t$. Since we aim to minimize \cref{eq:matrix-decomposition} in an approximation-free manner, we select methods that use the entire Hessian as they tend to give better performance for high compression ratios. SparseGPT \cite{frantar2023sparsegpt} is a popular pruning method that considers the entire Hessian. In particular, for our numerical results, we use SparseGPT to minimize \Pone.
\vspace{-3pt}
\subsection{Minimizing Subproblem \Ptwo}
\vspace{-3pt}
As in the previous section \ref{section-pone}, we discuss algorithms and related work for different variations of $\bfH$.
\vspace{-3pt}
\subsubsection{Data-Free version: $\bfX = \bfI_{\Nin \times \Nin}$} 
\vspace{-3pt}
Drawing a line from the pruning literature, a data-free version is introduced that does not require a calibration dataset. In this case, a closed-form solution of the minimizer is given by the Truncated-SVD $C_r(\lowrankW)$. This corresponds to the best rank-$r$ approximation of $\lowrankW$. \citet{li2023loftq} use SVD on the full matrix during their low-rank minimization step for quantization plus low-rank matrix decomposition. \citet{guo2023lq} use a Randomized-SVD \cite{halko2011finding} approach for the same problem (quantization plus low-rank decomposition) instead of the full SVD since it is reduces runtime significantly while maintaining the minimization performance.
\vspace{-3pt}
\subsubsection{Diagonal-approximation: $\bfH = \diagn{\bfXTX}$}\label{subsection-diagonal-approximation}
\vspace{-3pt}
The diagonal approximation of $\bfH$ has been made popular in the pruning literature thanks to Wanda \cite{sun2023simple}. We analyze the minimizing \Ptwo with this approximation. Similar to \ref{subsection-pruning-diagonal-approximation}, we introduce $\bfD = \sqrt{\diagn{\bfXTX}}$ in equation \eqref{eq:general-low-rank}. Here we use the fact that $\bfD$ is symmetric.
\begin{align*}
   \bfM^\star 
   &\in \argmin\nolimits_{\bfM} \,\, \Tr{(\lowrankW - \bfM)^\top \bfD^2 (\lowrankW - \bfM)} ~~~~\text{s.t. } ~~~\rk{\bfM} \leq r\\
   &= \argmin\nolimits_{\bfM} \,\, \left\|\bfD \lowrankW -\bfD \bfM\right\|_F^2 ~~~~\text{s.t. } ~~~\rk{\bfM} \leq r.
\end{align*}

\begin{assumption}\label{ass:full-rank-diagonal}
The input activations matrix $\bfX$ satisfies $\diag{\bfXTX}$ is full-rank. Equivalently, no column of $\bfX$ is identically $\mathbf{0}_{N \cdot L}$.
\end{assumption}
\begin{theorem}\label{theorem:low-rank-closed-form-diag-approx}
    If \cref{ass:full-rank-diagonal} holds, then the closed-form minimizer of \eqref{eq:general-low-rank} is given by
    \begin{equation*}
        \bfM^\star = \bfD^{-1} C_r(\bfD \lowrankW).
    \end{equation*}
\end{theorem}
The proof of theorem \ref{theorem:low-rank-closed-form-diag-approx} is obtained by introducing the auxialiary variable $\tilde{\bfM} = \bfD \bfM$ and noting that $\rkn{\tilde{\bfM}} = \rkn{\bfM}$, when \cref{ass:full-rank-diagonal} holds.

Interestingly, OATS \cite{zhang2024oats} uses the same operation in the Low-Rank update of the Alternating-Minimization approach (for sparse plus low rank matrix decomposition). 
%OATS uses a full SVD on the matrix $\bfD \lowrankW$. 
% \zx{Do we know what exactly OATS did? It looks to me we can do $\bfM^\star = C_r(\lowrankW)$ as $\bfD$ is diagonal? Computationally, it probably does not matter much as $\bfD$ is diagonal.}
% \mm{Actually it happens that $\bfD^{-1} C_r(\bfD \lowrankW) \neq C_r(\lowrankW)$ in the general case, that's why OATS is an interesting approach.}
\begin{corollary}
OATS \cite{zhang2024oats} exactly minimizes \eqref{eq:matrix-decomposition} with a diagonal approximation of the Hessian of the local layer-wise reconstruction error, since they minimize \Pone and \Ptwo with the same diagonal approximation $\bfH = \diagn{\bfXTX}$.
\end{corollary}
\vspace{-2pt}
\subsubsection{Full Hessian: $\bfH = \bfXTX + \lambda \bfI$}
\vspace{-3pt}
The state-of-the-art pruning algorithms in terms of retaining compressed LLMs performance on multiple benchmarks are the ones that use the full Hessian \ref{full-hessian-pruning} \cite{meng2024alps,frantar2023sparsegpt}. This motivates minimizing equation \eqref{eq:general-low-rank} using the full Hessian as well.
Dealing with low-rank constraints can be challenging, we therefore propose to reparametrize the low-rank matrix $\bfM \in \R^{\Nin \times \Nout}$ by  $\bfUVt$, with $\bfU \in \R^{\Nin \times r}, \bfV \in \R^{\Nout \times r}$. We can therefore use more computationally efficient first-order optimization methods to minimize the layer-wise reconstruction objective, which can be rewritten as follows.
\begin{equation}\label{eq:uvt-general-low-rank}
    \bfM^\star = \bfU^\star \bfV^{\star^\top}, \quad \bfU^\star, \bfV^\star \in \argmin\nolimits_{\bfU, \bfV} \,\, \Tr{\pr{\lowrankW - \bfUVt}^\top \bfH \pr{\lowrankW - \bfUVt}}.
\end{equation}
\textbf{Diagonal Scaling for \Ptwo Minimization Stability.\,\,\,\,}\label{scaling-low-rank}
Our initial experiments to minimize equation \eqref{eq:uvt-general-low-rank}, using Gradient-Descent type methods on $\bfU$ and $\bfV$, have shown that the optimization problem can be ill-conditioned in some tranformer layers. This can lead to numerical instability in the optimization procedure. To address this, we follow a similar rescaling approach proposed by \citet{meng2024alps}. Define (similar to \ref{subsection-diagonal-approximation}) the matrix $\bfD = \sqrt{\diagn{\bfXTX}}$ and reformulate the optimization equation \eqref{eq:uvt-general-low-rank} as follows (when \cref{ass:full-rank-diagonal} holds).
\begin{equation}\label{eq:scaled-uvt-general-low-rank}
    \bfM^\star = \bfD\bfU^\star \bfV^{\star^\top}, \quad \bfU^\star, \bfV^\star \in \argmin\nolimits_{\bfU, \bfV} \,\, \Tr{\pr{\bfD\lowrankW - \bfUVt}^\top \bfD^{-1}\bfH\bfD^{-1} \pr{\bfD\lowrankW - \bfUVt}}.
\end{equation}
It is important to note that the minimization problems in equations \eqref{eq:scaled-uvt-general-low-rank} and \eqref{eq:uvt-general-low-rank} are equivalent, in terms of objective minimization and feasibility of $\bfM^\star$, which has rank at most $r$. This scaling, which sets the diagonal of the new Hessian to $\mathbf{1}_{\Nin}$, only modifies the steps of Gradient Descent and leads to faster convergence in practice. See the \Cref{fig:reconstruction-error} for an objective minimization comparison of the effect of this Diagonal scaling.
It is also worth noting that this scaling allows to use the same learning rate $\eta$ for Gradient-Descent type methods for all layers and all models.

\subsection{Our Proposed Approach}
\vspace{-5pt}
Our goal is to minimize \eqref{eq:matrix-decomposition} using Alternating-Minimization with the full Hessian approach $\bfH = \bfXTX$. Our numerical results show that leveraging the entire Hessian outperforms OATS \cite{zhang2024oats}, which minimizes \eqref{eq:matrix-decomposition} with the diagonal approximation of the Hessian approach $\bfH = \diagn{\bfXTX}$, on a wide-range of LLM benchmarks and compression ratios. 

In our numerical experiments, we show results with the SparseGPT \cite{frantar2023sparsegpt} algorithm to minimize \Pone and the Adam algorithm \cite{kingma2014adam} to minimize \Ptwo reparametrized and rescaled as in \cref{eq:scaled-uvt-general-low-rank}. 

\textbf{Computational Efficiency.\,\,\,\,} Note that for a given layer $\ell$, the Hessian of the local layer-wise reconstruction problem $\bfXTX$ in \eqref{eq:general-pruning} as well as the rescaled version $\bfD^{-1}\bfXTX\bfD^{-1}$ in \eqref{eq:general-low-rank} are invariant throughout iterations. 
This is very important as pruning algorithms that use the entire Hessian information \cite{frantar2023sparsegpt, meng2024alps} need the Hessian inverse in their algorithm update. This inversion and associated costs of Hessian construction are done only once and then \textit{amortized} throughout iterations. In the \cref{algo:low-rank-gd}, we use $\bfU^{(t-1)}$ and $\bfV^{(t-1)}$ as initializations for the optimizer, as they are close to the minimizers of \Ptwo at iteration $t$. This accelerates the convergence in practice.

% We are now ready to present our proposed algorithm \ourframework in Algorithm \ref{algo:gdprune}.

\begin{algorithm}[h]
    \caption{\texttt{Low-Rank-GD}
    % \zx{Maybe inline this two line algorithm?}\mm{I like to keep it this way, to show (i) warm-up variables Uinit which makes optimization much faster and (ii) that scaling and no scaling can be solved using 'same' way.}
    }
    \label{algo:low-rank-gd}
    \begin{algorithmic}[]
        \State \textbf{Input} \texttt{Optimizer} (optimization algorithm, e.g. Adam), $\bfH$ (Hessian), $\bfW$ (Weights), $\bfU_\text{init}$, $\bfV_\text{init}$ (warm-up initialization for the joint minimization of $\bfU, \bfV$), $T_{\text{LR}}$ (\# iterations), $\eta$ (learning rate).
        \State $\text{Obj}(\bfU, \bfV) \gets \Tr{\prn{\bfW - \bfUVt}^\top \bfH \prn{\bfW - \bfUVt}}$        \vspace*{0.4em}
        \State $\bfU^\star, \bfV^\star \gets \texttt{Optimizer}_{\bfU, \bfV}\pr{\text{Obj}, \bfU_\text{init}, \bfV_\text{init}, N, \eta}$
        \vspace*{0.4em}
        \State \textbf{Output} $\bfU^\star, \bfV^\star$.
    \end{algorithmic}
\end{algorithm}
\vspace{-\baselineskip}
\begin{algorithm}[h]
    \caption{\ourframework}
    \label{algo:gdprune}
    \begin{algorithmic}[]
        \State \textbf{Input} for a given layer $\ell$: $\mathbf{H} = (\bfXTX + \lambda \mathbf{I})$ (Hessian of \eqref{eq:matrix-decomposition}, plus a regularization term for numerical stability), $\widehat{\bfW}$ (dense pre-trained weights), $T_{\text{AM}}$ (\# iterations of Alternating-Minimization), $T_{\text{LR}}$ (\# iterations of \texttt{Low-Rank-GD}), $\eta$ (learning rate for $\bfU, \bfV$), $\calCS$ (sparsity pattern), $r$ (rank of low-rank components), \texttt{Prune} (any pruning algorithm, e.g. SparseGPT), \texttt{Optimizer} (any first-order algorithm, e.g. Adam), \textbf{is\_scaled} (bool to apply scaling \ref{scaling-low-rank}).
        \vspace*{0.1em}
        \State $\bfD \gets \sqrt{\diag{\bfH}}$ \quad \Comment{Diagonal of the Hessian.}
        % \vspace*{0.1em}
        \State $\mathbf{H}^{-1} \gets \texttt{inv}\pr{\mathbf{H}}$ \quad \Comment{Inverse the Hessian.}
        % \vspace*{0.1em}
        \State $\bfWS \gets \mathbf{0}_{\Nin \times \Nout}$
        % \vspace*{0.1em}
        \State $\bfU \gets \mathbf{0}_{\Nin \times r}$
        % \vspace*{0.1em}
        \State $\bfV \gets \mathcal{N}_{\Nout \times r}$ \quad  \Comment{element-wise independent gaussian initialization.}
        % \vspace*{0.1em}
        \For{$t = 1 \dots T$}
            % \vspace*{0.1em}
            \State $\bfWS \gets \texttt{Prune}\pr{\mathbf{H}^{-1}, \widehat{\bfW} - \bfUVt, \calCS}$
            \State \Comment{$\bfWS \approx \widehat{\bfW} - \bfUVt$, satisfies $\calCS$ sparsity pattern \& minimizes \Pone.}
            \vspace*{0.2em}
            \State $\eta_t \gets \text{get\_lr}(t, \eta)$ \quad \Comment{In practice, $\eta_t = \eta / (t + 10)$.}
            \vspace*{0.2em}
            \If{\textbf{is\_scaled}}
                % \vspace*{0.1em}
                \State $\bfU, \bfV \gets \texttt{Low-Rank-GD}\pr{\texttt{Optimizer}, \bfD^{-1}\mathbf{H}\bfD^{-1}, \bfD\pr{\widehat{\bfW} - \bfWS}, \bfD\bfU, \bfV, T_{\text{LR}}, \eta_t}$
                \State $\bfU \gets \bfD \bfU$ \quad \Comment{Rescale $\bfU$ back.}
            \Else
                \State $\bfU, \bfV \gets \texttt{Low-Rank-GD}\pr{\texttt{Optimizer}, \mathbf{H}, \widehat{\bfW} - \bfWS, \bfU, \bfV, T_{\text{LR}}, \eta_t}$ 
                % \zx{Do we still need to keep this unscaled version?} \mm{I'm thinking of keeping it and include a very brief ablation study, I think it's nice to show that both work but scaled requires much less iterations in practice}
            \EndIf
            % \State $\bfU, \bfV \gets \texttt{Low-Rank-GD}\pr{\texttt{Optimizer}, \bfD^{-1}\mathbf{H}\bfD^{-1}, \bfD\pr{\widehat{\bfW} - \bfWS}, \eta_t}$
            % \vspace*{0.3em}
            % \State $\bfU \gets \bfD \bfU$ \quad \Comment{Rescale $\bfU$ back.}
            \State \Comment{$\bfUVt \approx \widehat{\bfW} - \bfWS$, has rank at most $r$ \& minimizes \Ptwo.}
        \EndFor
        % \vspace*{0.2em}
        \State $\bfM \gets \bfUVt$
        % \vspace*{0.1em}
        \State \textbf{Output} for a given layer $\ell$: $\bfWS, \bfM$.
    \end{algorithmic}
\end{algorithm}


\vspace{-5pt}
\section{Experimental Results}
\label{sec:experimental-results}
\vspace{-5pt}
% !TEX root = ../main.tex

\subsection{Experiment Setup}
\vspace{-3pt}
\textbf{Models and datasets} We evaluate our proposed method \ourframework on two families of large language models: Llama-3 and Llama-3.2 \cite{dubey2024llama} with sizes ranging from 1 to 8 billion parameters. 
To construct the Hessian $\bfXTX$, we follow the approch of \citet{frantar2023sparsegpt}: we use 128 segments of 2048 each, randomly sampled from the first shard of the C4 training dataset \cite{JMLR:v21:20-074}. To ensure consistency, we utilize the same calibration data for all pruning algorithms we benchmark. We also consider one-shot compression results, without retraining. 
We assess the performance using perplexity and zero-shot evaluation benchmarks, with perplexity calculated according to the procedure described by HuggingFace \cite{Perplexity}, using full stride. For perplexity evaluations, we use the test sets of raw-WikiText2 \cite{merity2017pointer}, PTB \cite{Marcus1994}, and a subset of the C4 validation data, which are popular benchmarks in LLM pruning literature \cite{frantar2023sparsegpt,meng2024alps,meng2024osscar}. Additionally, we evaluate the following zero-shot tasks using LM Harness by \citet{gao10256836framework}: PIQA \cite{bisk2020piqa}, ARC-Easy (ARC-E) \& ARC-Challenge (ARC-C) \cite{clark2018think}, Hellaswag (HS) \cite{zellers2019hellaswag}, Winogrande (WG) \cite{sakaguchi2021winogrande}, RTE \cite{poliak2020survey}, OpenbookQA (OQA) \cite{banerjee2019careful} and BoolQ \cite{clark2019boolq}. The average of the eight zero-shot tasks is also reported.

\vspace{-5pt}
\subsection{Results}
\vspace{-10pt}
In order to benchmark the performance of our matrix decomposition algorithm, \ourframework uses the same number of Alternating-Minimization steps as OATS \cite{zhang2024oats} which is $80$. We report results for the scaled version of \ourframework, which uses the same learning rate $\eta = 1e^{-2}$ for all layers and considered models. We consider the following two settings.

\textbf{N:M Sparsity + Fixed Rank:}
We impose the sparsity pattern $\calCS$ to be $N:M$ sparsity and we fix the target rank $r = 64$ of the low-rank component for all layers. We benchmark our method with OATS \cite{zhang2024oats}. The results are reported in \cref{tab:slr-fixed-rank}.

\textbf{N:M Sparsity + Fixed Compression Ratio:}
This is similar to the setting described by \citet{zhang2024oats} for N:M sparsity evaluations. Each layer, with dense weight matrix $\widehat{\bfW}$, is compressed to a prefixed compression ratio $\rho$ (e.g. $50\%$) so that $\widehat{\bfW} \approx \bfW_{N:M} + \bfM$, and the target rank is given by\\[0.2em] 
$r = \left\lfloor {(1 - \rho - \frac{N}{M}) \cdot(\Nout \cdot \Nin)} / \pr{\Nout + \Nin} \right\rfloor$.\\[0.2em]
Note that the effective number of parameters stored is therefore\\[0.1em]
$\#\text{params } \bfW_{N:M} + \#\text{params } \bfU + \#\text{params } \bfV = \frac{N}{M} \cdot (\Nout \cdot \Nin) + r \Nin + r \Nout \leq (1 - \rho) \cdot \#\text{params } \widehat{\bfW}$,\\[0.5em]
hence the comparison to other pruning methods matched at the same compression ratio $\rho$. The results are reported for the Llama3-8B model in \cref{tab:slr-fixed-compression} for \ourframework, \texttt{OATS}, and different N:M pruning algorithms (SparseGPT \cite{frantar2023sparsegpt}, Wanda \cite{sun2023simple}, DSNoT \cite{zhang2023dynamic}) compressed at $\rho = 50\%$. The results are expanded for \ourframework and OATS in \cref{supp:experiments-supp}. 
% We vary the compression ratio $\rho$, we then select some N:M sparsity values that result in a more aggressive compression, we then match this compression ratio thanks to the low-rank component.

% \begin{table}[h!]
% \centering
% \resizebox{1.0\textwidth}{!}{%
% \renewcommand{\arraystretch}{1.3}
% \begin{tabular}{cccccccccccccc}
% \toprule
% \multirow{2.25}{*}{\textbf{Model}} & \multirow{2.25}{*}{\textbf{Algorithm}} & \multicolumn{3}{c}{\textbf{Perplexity ($\downarrow$)}} & \multicolumn{9}{c}{\textbf{Zero-shot ($\uparrow$)}} \\ 
% \cmidrule(rl){3-5} \cmidrule(rl){6-14}
% & & \textbf{C4} & \textbf{WT2} & \textbf{PTB} & \textbf{PIQA} & \textbf{HS} & \textbf{ARC-E} & \textbf{ARC-C} & \textbf{WG} & \textbf{RTE} & \textbf{OQA} & \textbf{BoolQ} & \textbf{Avg}\\ 
% \midrule
% \multirow{2}{*}{Llama3-8B} & \texttt{OATS-2:8+64LR}       & 368.24 & 858.90 & --.-- & 52.29 & 27.32 & \textbf{22.7} & 37.61 & -- & -- & -- & -- \\ 
%                           & \texttt{Ours-2:8+64LR}       & \textbf{90.46}  & \textbf{88.58} & --.-- & \textbf{54.52} & \textbf{31.44} & 20.73 & \textbf{40.93} & -- & -- & -- & -- \\ 
% \bottomrule
% \end{tabular}%
% }
% \caption{Evaluation results for Llama3-8B. PPL columns minimize ($\downarrow$), and accuracy columns maximize ($\uparrow$).}
% \label{tab:results}
% \end{table}

\vspace{-8pt}
\noindent
\begin{minipage}[t]{0.5\textwidth}
    \raggedleft
    \vspace{25pt}
    \captionof{table}{Performance analysis for one-shot N:M sparse plus a low-rank matrix decomposition of the Llama3-8b model. The compression ratio is fixed to be $\rho=0.5$. For Perplexity, $(\downarrow)$ lower values are preferred. For zero-shot tasks, $(\uparrow)$ higher values are preferred.\\
    Bolded values correspond to a comparison between sparse plus low-rank decomposition algorithms. Underlined values correspond to the overall best comopression scheme given a compression ratio $\rho = 50\%$.}
    \label{tab:slr-fixed-compression}
\end{minipage}%
\hspace{5pt}
\begin{minipage}[t]{0.5\textwidth}
\centering
\begin{table}[H]
\centering
\resizebox{1.0\textwidth}{!}{%
\renewcommand{\arraystretch}{1.3}
\begin{tabular}{ccccccccc}
\toprule
\multirow{2.25}{*}{\textbf{Algorithm}} && \multicolumn{3}{c}{\textbf{Perplexity ($\downarrow$)}} && \multicolumn{3}{c}{\textbf{Zero-shot ($\uparrow$)}} \\ 
\cmidrule(rl){3-5} \cmidrule(rl){7-9}
&&\textbf{C4} & \textbf{WT2} & \textbf{PTB} && \textbf{PIQA} & \textbf{ARC-E} & \textbf{ARC-C}\\ 
\midrule
\texttt{SparseGPT-4:8}     && \underline{14.94} & 12.40 & 17.90 && 73.20 & \underline{68.54} & 34.86 \\ 
\texttt{Wanda-4:8}         && 18.88 & 14.52 & 24.26 && 71.52 & 64.91 & 34.03 \\ 
\texttt{DSNoT-4:8}         && 18.89 & 14.76 & 23.90 && 71.49 & 65.65 & 33.57 \\ 
\cmidrule(rl){1-1}
\texttt{SparseGPT-2:4}     && 18.89 & 16.35 & 25.08 && 70.54 & 63.09 & 31.84 \\ 
\texttt{Wanda-2:4}         && 30.81 & 24.36 & 44.89 && 67.56 & 56.20 & 26.11 \\ 
\texttt{DSNoT-2:4}         && 28.78 & 23.09 & 40.95 && 67.70 & 56.46 & 25.68 \\
\cmidrule(rl){1-1}
\texttt{OATS-2:8+LR}       && 21.03 & \textbf{14.54} & 24.15 && 73.67  & 59.68 & \textbf{37.12}\\
\texttt{Ours-2:8+LR}       && \textbf{20.05} & 15.03 & \textbf{22.01} && \textbf{74.05} & \textbf{60.52} & 36.18\\
\cmidrule(rl){1-1}
\texttt{OATS-3:8+LR}       && 16.87 & 11.43 & 18.53 && 75.24 & 65.91 & 39.85 \\
\texttt{Ours-3:8+LR}       && \textbf{16.16} & \underline{\textbf{11.36}} & \underline{\textbf{16.71}} && \underline{\textbf{75.79}} & \textbf{67.55} & \underline{\textbf{41.04}} \\
\cmidrule(rl){1-1}
\texttt{dense}               && 9.44 & 6.14 & 11.18 && 80.79 & 77.69 & 53.33  \\
\bottomrule
\end{tabular}
}
\end{table}
\end{minipage}



\begin{table}[h!]
\centering
\resizebox{1.0\textwidth}{!}{%
\renewcommand{\arraystretch}{1.1}
\begin{tabular}{ccc@{\hskip 8pt}cccc@{\hskip 8pt}ccccccccc}
\toprule
\multirow{2.25}{*}{\textbf{Model}} & \multirow{2.25}{*}{\textbf{Algorithm}} && \multicolumn{3}{c}{\textbf{Perplexity ($\downarrow$)}} && \multicolumn{9}{c}{\textbf{Zero-shot ($\uparrow$)}} \\ 
\cmidrule(rl){4-6} \cmidrule(r){8-16}
&&& \textbf{C4} & \textbf{WT2} & \textbf{PTB} && \textbf{PIQA} & \textbf{HS} & \textbf{ARC-E} & \textbf{ARC-C} & \textbf{WG} & \textbf{RTE} & \textbf{OQA} & \textbf{BoolQ} & \textbf{Avg}\\ 
\midrule
\multirow{9.5}{*}{Llama3-8B} 
&\texttt{OATS-2:8+64LR}       && 368.24 & 416.14 & 565.46 && 52.29 & 28.03 & 27.53 & \textbf{22.70} & 49.17 & \textbf{52.71} & 26.40 & 42.08 & 37.61 \\
&\texttt{Ours-2:8+64LR}       && \textbf{90.46} & \textbf{92.59} & \textbf{108.80} && \textbf{54.52} & \textbf{30.85} & \textbf{31.44} & 20.73 & \textbf{50.20} & \textbf{52.71} & \textbf{26.60} & \textbf{60.37} & \textbf{40.93} \\
\cmidrule(rl){2-2}
&\texttt{OATS-3:8+64LR}       && 48.21 & 35.65 & 56.52 && 65.23 & 42.05 & 47.01 & 25.94 & 58.01 & 52.71 & 27.40 & 67.89 & 48.28 \\
&\texttt{Ours-3:8+64LR}       && \textbf{28.88} & \textbf{21.48} & \textbf{32.54} && \textbf{68.99} & \textbf{52.19} & \textbf{50.55} & \textbf{29.86} & \textbf{62.90} & \textbf{53.07} & \textbf{29.80} & \textbf{72.84} & \textbf{52.53} \\
\cmidrule(rl){2-2}
&\texttt{OATS-4:8+64LR}       && 15.97 & 10.52 & 16.71 && 75.14 & 68.69 & 66.67 & 40.87 & 69.69 & \textbf{54.87} & 39.40 & \textbf{79.76} & 61.89 \\
&\texttt{Ours-4:8+64LR}       && \textbf{14.67} & \textbf{9.93} & \textbf{15.28} && \textbf{76.39} & \textbf{70.48} & \textbf{68.48} & \textbf{42.58} & \textbf{70.32} & 54.15 & \textbf{39.80} & 79.48 & \textbf{62.71} \\
\cmidrule(rl){2-2}
&\texttt{OATS-2:4+64LR}       && 21.05 & 14.42 & 22.62 && 72.85 & 62.47 & 60.69 & 36.35 & 67.09 & 54.87 & 35.00 & 75.11 & 58.05 \\
&\texttt{Ours-2:4+64LR}       && \textbf{18.06} & \textbf{12.66} & \textbf{18.66} && \textbf{74.86} & \textbf{64.77} & \textbf{63.85} & \textbf{37.37} & \textbf{69.22} & \textbf{56.68} & \textbf{36.40} & \textbf{76.12} & \textbf{59.91} \\
\cmidrule(rl){2-2}
&\texttt{dense}               && 9.44 & 6.14 & 11.18 && 80.79 & 79.17 & 77.69 & 53.33 & 72.85 & 69.68 & 45.00 & 81.44 & 69.99 \\

\midrule
\multirow{9.5}{*}{Llama3.2-1B} 
&\texttt{OATS-2:8+64LR}       && 740.37 & 825.40 & 754.22 && 52.12 & 27.46 & 28.37 & \textbf{23.72} & 48.86 & 52.71 & 24.60 & 37.77 & 36.95 \\
&\texttt{Ours-2:8+64LR}       && \textbf{167.87} & \textbf{133.01} & \textbf{162.73} && \textbf{54.30} & \textbf{28.73} & \textbf{30.35} & 21.93 & \textbf{50.51} & \textbf{53.43} & \textbf{25.20} & \textbf{51.68} & \textbf{39.52} \\
\cmidrule(rl){2-2}
&\texttt{OATS-3:8+64LR}       && 96.32 & 74.10 & 93.70 && 59.52 & 33.51 & 36.41 & 22.70 & 50.99 & \textbf{52.71} & 25.80 & 62.14 & 42.97 \\
&\texttt{Ours-3:8+64LR}       && \textbf{45.79} & \textbf{34.15} & \textbf{52.20} && \textbf{62.08} & \textbf{38.24} & \textbf{41.04} & \textbf{23.63} & \textbf{54.54} & \textbf{52.71} & \textbf{30.40} & \textbf{62.20} & \textbf{45.60} \\
\cmidrule(rl){2-2}
&\texttt{OATS-4:8+64LR}       && 26.75 & 18.49 & 31.94 && 67.30 & 49.52 & 50.51 & 28.41 & 56.67 & \textbf{55.96} & \textbf{32.40} & \textbf{62.87} & 50.46 \\
&\texttt{Ours-4:8+64LR}       && \textbf{22.71} & \textbf{16.05} & \textbf{26.80} && \textbf{68.28} & \textbf{51.42} & \textbf{51.22} & \textbf{29.18} & \textbf{58.64} & 53.07 & 30.00 & 62.51 & \textbf{50.54} \\
\cmidrule(rl){2-2}
&\texttt{OATS-2:4+64LR}       && 36.89 & 26.26 & 42.35 && 64.36 & 43.35 & \textbf{47.77} & 26.45 & 55.80 & \textbf{52.71} & 30.40 & \textbf{62.66} & 47.94 \\
&\texttt{Ours-2:4+64LR}       && \textbf{27.09} & \textbf{19.57} & \textbf{31.73} && \textbf{67.03} & \textbf{47.53} & 47.43 & \textbf{28.16} & \textbf{58.64} & \textbf{52.71} & \textbf{30.60} & 62.60 & \textbf{49.34} \\
\cmidrule(rl){2-2}
&\texttt{dense}               && 14.01 & 9.75 & 17.59 && 74.59 & 63.66 & 60.48 & 36.26 & 60.69 & 56.68 & 37.20 & 63.98 & 56.69 \\

\midrule
\multirow{9.5}{*}{Llama3.2-3B} 
&\texttt{OATS-2:8+64LR}       && 444.37 & 543.53 & 851.16 && 52.56 & 27.54 & 27.99 & \textbf{23.46} & \textbf{50.43} & 51.99 & \textbf{26.60} & 37.86 & 37.30 \\
&\texttt{Ours-2:8+64LR}       && \textbf{122.14} & \textbf{114.74} & \textbf{165.78} && \textbf{54.57} & \textbf{28.93} & \textbf{30.09} & 21.08 & 49.49 & \textbf{52.71} & 26.20 & \textbf{62.14} & \textbf{40.65} \\
\cmidrule(rl){2-2}
&\texttt{OATS-3:8+64LR}       && 56.80 & 41.62 & 72.75 && 62.68 & 40.49 & 41.84 & 24.06 & 53.91 & 52.35 & 26.60 & 64.10 & 45.75 \\
&\texttt{Ours-3:8+64LR}       && \textbf{35.07} & \textbf{27.12} & \textbf{39.63} && \textbf{66.43} & \textbf{46.08} & \textbf{46.42} & \textbf{26.62} & \textbf{58.17} & \textbf{55.96} & \textbf{29.00} & \textbf{65.47} & \textbf{49.27} \\
\cmidrule(rl){2-2}
&\texttt{OATS-4:8+64LR}       && 18.52 & 12.85 & 20.69 && 72.85 & 61.68 & 62.42 & 36.01 & 64.17 & \textbf{60.29} & 36.40 & \textbf{72.75} & 58.32 \\
&\texttt{Ours-4:8+64LR}       && \textbf{17.19} & \textbf{12.15} & \textbf{19.24} && \textbf{73.99} & \textbf{63.59} & \textbf{62.92} & \textbf{36.26} & \textbf{67.48} & 57.76 & \textbf{39.20} & 71.90 & \textbf{59.14} \\
\cmidrule(rl){2-2}
&\texttt{OATS-2:4+64LR}       && 24.32 & 17.06 & 28.54 && \textbf{71.98} & 55.87 & 58.80 & 33.36 & 59.91 & 53.07 & \textbf{33.80} & \textbf{70.18} & 54.62 \\
&\texttt{Ours-2:4+64LR}       && \textbf{20.82} & \textbf{15.65} & \textbf{23.77} && 71.71 & \textbf{57.88} & \textbf{58.84} & \textbf{34.39} & \textbf{62.12} & \textbf{58.12} & 33.60 & 67.92 & \textbf{55.57} \\
\cmidrule(rl){2-2}
&\texttt{dense}               && 11.33 & 7.81 & 13.53 && 77.48 & 73.61 & 71.63 & 45.99 & 69.85 & 54.51 & 43.00 & 73.39 & 63.68 \\

\bottomrule
\end{tabular}
}
\vspace{3pt}
\caption{Performance analysis for one-shot N:M sparse plus a 64-rank low-rank matrix decomposition of Llama3 and Llama3.2 models. The rank of the low-rank component is fixed to be $r=64$. For Perplexity, $(\downarrow)$ lower values are preferred. For zero-shot tasks, $(\uparrow)$ higher values are preferred.}
\label{tab:slr-fixed-rank}
\end{table}
\subsection{Reconstruction error on a single Transformer block}
In order to show the performance of OATS and \ourframework on the layer-wise reconstruction objective \eqref{eq:matrix-decomposition}, we compute the error produced with the two algorithms (both after $80$ iterations--default value used in OATS \cite{zhang2024oats}), given by $\|\bfX \bfWold-\bfX \pr{\bfWS + \bfM}\|_F^2$, when applied to the model Llama-3-8B \cite{dubey2024llama}, and using the decomposition $\calCS$ corresponding to $2:4$ sparsity and a fixed rank $r = 64$. Results of the local layer-wisre error are reported in \cref{fig:reconstruction-error} for OATS, \ourframework scaled and \ourframework unscaled.

\vspace{-20pt}
\begin{minipage}[t]{0.49\textwidth}
    \raggedleft
    \vspace{30pt}
    \captionof{figure}{Local layer-wise reconstruction error $\downarrow$ (lower values are preferred) analysis of the decomposition of the layers of the \textbf{first} transformer block in Llama-3-8B into a 2:4 sparse component plus a 64-rank low-rank component. All methods use the same number of Alternating-Minimization steps $80$.}
    \label{fig:reconstruction-error}
\end{minipage}
\begin{minipage}[t]{0.49\textwidth}
    \vspace{0pt}
    \centering
    \includegraphics[width=\textwidth]{layerwise_error.pdf}
\end{minipage}
\vspace{-\baselineskip}

\vspace{-15pt}
\section{Conclusion}
\label{sec:conclusion}
\vspace{-5pt}
\section{Conclusion}

This work addresses the pressing need for enhanced security in the burgeoning blockchain ecosystem. We investigate the application of Large Language Models (LLMs) to smart contract vulnerability detection and repair, focusing on Solidity and Move. We introduce \textbf{Smartify}, a novel multi-agent framework that significantly improves LLM performance in this critical domain. The contributions of this work are: (1) \textbf{Smartify}, a novel multi-agent framework that enhances LLM-based smart contract vulnerability detection and repair; (2) a method for encoding language-specific knowledge, valuable for low-resource languages like Move; (3) a scalable, adaptable approach applicable to other programming languages and LLMs; (4) a demonstration of Smartify’s efficacy on generalized pre-trained LLMs; and (5) a detailed analysis of the challenges inherent in automated code repair.

\textbf{Smartify} represents a significant advancement in automating smart contract security, a crucial concern in the expanding blockchain landscape. Future work will refine the framework, expand its language coverage, particularly within the blockchain domain, and integrate it into real-world blockchain development workflows. This research lays the foundation for AI-powered tools that can bolster the security and reliability of decentralized applications, fostering a more robust and trustworthy blockchain ecosystem.

\section*{Acknowledgements}
This research is supported in part by grants from Google and the Office of Naval Research. We acknowledge the MIT SuperCloud~\cite{reuther2018interactive} for providing HPC resources that have contributed to the research results reported within this paper. We also acknowledge Google for providing us with Google Cloud Credits for computing. 

\clearpage


% Reference
% For natbib users:
\bibliography{reference}
\clearpage


%%%%%%%%%%%%%%%%%%%%%%%%%%%%%%%%%%%%%%%%%%%%%%%%%%%%%%%%%%%%
\appendix

\section{Experimental Details}
\label{supp:experiments-supp}
% !TEX root = ../main.tex

% \begin{proof}
%   Given $\bfX \in \R^{m \times p}, \bfW \in \R^{p \times n}$, and the compact singular value decomposition of $\bfX$ as $\bfX = \bfU_{m \times r} \bfSigma_{r \times r} \bfV_{p \times r}^\top$. If $r = p < m$, then
%   \begin{equation*}
%     \pr{\bfXTX}^{-1}\bfX^\top C_r\pr{\bfX \bfW} = \argmin\nolimits_{\bfM \in \R^{p \times n}} \,\, \left\|\bfX \bfW -\bfX \bfM\right\|_F^2 ~~~~\text{s.t. } ~~~\rk{\bfM} \leq r.
%   \end{equation*}
%   If we introduce $\bfY = \bfX \bfW$,  \citet{mazumder2020computing} have shown that under the assumption that $r = p < m$, the solution to the above problem is \textbf{unique} and given by
%   \begin{align*}
%     \bfM_r(\bfY) 
%     &= \pr{\bfXTX}^{-1}\bfX^\top C_r\pr{\bfU\bfU^\top\bfY}\\
%     &= \pr{\bfXTX}^{-1}\bfX^\top C_r\pr{\bfU\bfU^\top\bfX \bfW}\\
%     &= \pr{\bfXTX}^{-1}\bfX^\top C_r\pr{\bfU\bfU^\top\bfU \bfSigma \bfV^\top \bfW}\\
%     &= \pr{\bfXTX}^{-1}\bfX^\top C_r\pr{\bfU \bfSigma \bfV^\top \bfW} \tag{$\bfU$ is orthogonal}\\
%     &= \pr{\bfXTX}^{-1}\bfX^\top C_r\pr{\bfX \bfW}.\\
%   \end{align*}
% \end{proof}


\subsection{Experimental Setup}
Following the framework proposed by \citet{frantar2023sparsegpt} for one-shot pruning, we minimize \cref{eq:matrix-decomposition} sequentially, layer by layer. For a given layer $\ell$, the input activation matrix $\bfX$ introduced in \cref{sec:optimization-formulation} is the output of the previous $\ell - 1$ compressed layers (sparse plus low-rank) using $N$ calibration samples. 

\textbf{Implementation details.}
\begin{itemize}
    \item For the construction of the Hessian matrix $\bfH = \bfXTX$ introduced in \cref{sec:algorithm-design}, we use the same setup of SparseGPT \cite{frantar2023sparsegpt} and we use the author's implementation of SparseGPT---as a pruning plug-in method to minimize \Pone (codes available on GitHub).
    \item We utilize the author's implementation of OATS \cite{zhang2024oats} with the default hyperparameter settings to show LLM evaluation benchmarks and layer-wise reconstruction error in \cref{fig:reconstruction-error}.
    \item The LLM evaluation benchmarks reported in \cref{tab:slr-fixed-compression} are retrieved from the paper ALPS by \citet{meng2024alps} which uses the same evaluation strategy (and code) we do for the reported tasks [other zero-shot tasks are not reported in ALPS]. We report all zero-shot tasks results for OATS and \ourframework in \cref{tab:supp-slr-fixed-compression}.
\end{itemize}



\subsection{Hyperparameter Choice}
The hyperparameters used in \ourframework are the following: $\lambda = 0.01 \Tr{\bfH}$; default value in SparseGPT. $T_\text{AM}$ is set to be 80; default value in OATS. $T_\text{LR} = 50$; we propose this default value for all experiments. $\eta = 1e^{-2}$; we propose this default value for all experiments (only works well with the scaling introduced in \cref{scaling-low-rank}). $r$ is either set to $64$ and fixed for all layers, or is flexible and given by the formula $r = \left\lfloor {(1 - \rho - \frac{N}{M}) \cdot(\Nout \cdot \Nin)} / \pr{\Nout + \Nin} \right\rfloor$ introduced in \cref{sec:experimental-results}. \texttt{Prune}; we propose by default to use SparseGPT. \texttt{Optimizer}; we propose the Adam optimizer. \textbf{is\_scaled}; we propose to set this to True by default. It converges faster in practice and allows to skip the tuning of the learning rate $\eta$.

\subsection{Additional Experimental Resuls}
\textbf{N:M Sparsity + Fixed Compression Ratio:}
This is the same setting described in \cref{sec:experimental-results}. We extend the results reported in \cref{tab:slr-fixed-compression} to include the 8 zero-shot tasks and the Llama3.2 model. Results are reported in \cref{tab:supp-slr-fixed-compression}.
\begin{table}[h!]
\centering
\resizebox{1.0\textwidth}{!}{%
\renewcommand{\arraystretch}{1.3}
\begin{tabular}{cccccccccccccc}
\toprule
\multirow{2.25}{*}{\textbf{Model}} & \multirow{2.25}{*}{\textbf{Algorithm}} & \multicolumn{3}{c}{\textbf{Perplexity ($\downarrow$)}} & \multicolumn{9}{c}{\textbf{Zero-shot ($\uparrow$)}} \\ 
\cmidrule(rl){3-5} \cmidrule(rl){6-14}
& & \textbf{C4} & \textbf{WT2} & \textbf{PTB} & \textbf{PIQA} & \textbf{HS} & \textbf{ARC-E} & \textbf{ARC-C} & \textbf{WG} & \textbf{RTE} & \textbf{OQA} & \textbf{BoolQ} & \textbf{Avg}\\ 
\midrule
\multirow{5.5}{*}{Llama3-8B} 
&\texttt{OATS-2:8+LR}       & 21.03 & \textbf{14.54} & 24.15 & 73.67 & \textbf{62.42} & 59.68 & \textbf{37.12} & 65.43 & 55.23 & \textbf{36.40} & 73.98 & 57.99 \\
&\texttt{Ours-2:8+LR}       & \textbf{20.05} & 15.03 & \textbf{22.01} & \textbf{74.05} & 60.69 & \textbf{60.52} & 36.18 & \textbf{66.77} & \textbf{57.04} & 35.00 & \textbf{76.02} & \textbf{58.28} \\
\cmidrule(rl){2-2}
&\texttt{OATS-3:8+LR}       & 16.87 & 11.43 & 18.53 & 75.24 & 66.90 & 65.91 & 39.85 & 68.90 & 61.37 & 39.00 & 76.61 & 61.72 \\
&\texttt{Ours-3:8+LR}       & \textbf{16.16} & \textbf{11.36} & \textbf{16.71} & \textbf{75.79} & \textbf{67.33} & \textbf{67.55} & \textbf{41.04} & \textbf{69.53} & \textbf{58.48} & \textbf{39.20} & \textbf{79.91} & \textbf{62.35} \\
\cmidrule(rl){2-2}
&\texttt{dense}               & 9.44 & 6.14 & 11.18 & 80.79 & 79.17 & 77.69 & 53.33 & 72.85 & 69.68 & 45.00 & 81.44 & 69.99 \\

\midrule
\multirow{5.5}{*}{Llama3.2-1B} 
&\texttt{OATS-2:8+LR}       & 78.18 & 53.05 & 80.17 & 59.03 & 36.42 & 37.08 & 22.87 & 52.80 & 52.71 & 27.40 & 61.77 & 43.76 \\
&\texttt{Ours-2:8+LR}       & \textbf{41.08} & \textbf{30.92} & \textbf{48.85} & \textbf{63.22} & \textbf{39.07} & \textbf{42.55} & \textbf{25.77} & \textbf{55.17} & \textbf{53.07} & \textbf{28.00} & \textbf{62.11} & \textbf{46.12} \\
\cmidrule(rl){2-2}
&\texttt{OATS-3:8+LR}       & 42.81 & 29.35 & 47.58 & 63.49 & 42.25 & 43.43 & 25.09 & 54.85 & 52.35 & \textbf{29.60} & 62.05 & 46.64 \\
&\texttt{Ours-3:8+LR}       & \textbf{31.35} & \textbf{22.89} & \textbf{34.99} & \textbf{66.43} & \textbf{45.00} & \textbf{46.42} & \textbf{25.85} & \textbf{56.43} & \textbf{52.71} & 28.80 & \textbf{62.26} & \textbf{47.99} \\
\cmidrule(rl){2-2}
&\texttt{dense}               & 14.01 & 9.75 & 17.59 & 74.59 & 63.66 & 60.48 & 36.26 & 60.69 & 56.68 & 37.20 & 63.98 & 56.69 \\

\midrule
\multirow{5.5}{*}{Llama3.2-3B} 
&\texttt{OATS-2:8+LR}       & 30.73 & 22.65 & 36.31 & 68.55 & 51.76 & 54.46 & \textbf{31.14} & 61.17 & \textbf{58.48} & \textbf{30.80} & \textbf{70.43} & \textbf{53.35} \\
&\texttt{Ours-2:8+LR}       & \textbf{25.22} & \textbf{19.61} & \textbf{29.54} & \textbf{69.59} & \textbf{52.94} & \textbf{55.30} & 29.69 & \textbf{62.67} & 55.23 & 30.60 & 69.24 & 53.16\\
\cmidrule(rl){2-2}
&\texttt{OATS-3:8+LR}       & 21.96 & 15.84 & 26.22 & \textbf{72.69} & 58.61 & \textbf{58.92} & \textbf{34.13} & 63.14 & \textbf{58.12} & 33.60 & 67.22 & 55.80 \\
&\texttt{Ours-3:8+LR}       & \textbf{20.03} & \textbf{14.85} & \textbf{22.92} & 72.42 & \textbf{58.92} & 56.69 & 33.53 & \textbf{64.01} & 56.32 & \textbf{37.00} & \textbf{70.31} & \textbf{56.15} \\
\cmidrule(rl){2-2}
&\texttt{dense}               & 11.33 & 7.81 & 13.53 & 77.48 & 73.61 & 71.63 & 45.99 & 69.85 & 54.51 & 43.00 & 73.39 & 63.68 \\


\bottomrule
\end{tabular}
}
\vspace{3pt}
\caption{Performance analysis for one-shot N:M sparse plus a low-rank matrix decomposition of Llama3 and Llama3.2 models. The compression ratio is fixed to be $\rho=0.5$. For Perplexity, $(\downarrow)$ lower values are preferred. For zero-shot tasks, $(\uparrow)$ higher values are preferred.}
\label{tab:supp-slr-fixed-compression}
\end{table}

\textbf{Unstructured Sparsity + Fixed Rank Ratio:} This is the setting introduced in OATS \cite{zhang2024oats}. This scheme takes as inputs a compression ratio $\rho$ (e.g. $50\%$) and rank ratio $\kappa$ (e.g. $0.3$; default value in OATS for the Llama3-8B model). The rank of the low-rank component $r$ and the number of non-zeros $k$ in the unstructured sparsity are given by.
\begin{equation*}
    r = \left\lfloor \kappa \cdot (1 - \rho) \cdot \frac{\Nout \cdot \Nin}{\Nout + \Nin} \right\rfloor, \quad \quad k = \left\lfloor (1 - \kappa) \cdot (1 - \rho) \cdot \Nout \cdot \Nin \right\rfloor.
\end{equation*}
See OATS for a discussion on how to choose the rank ratio $\kappa$ for a given model. Note that OATS introduces OWL ratios--different sparsity budgets for different layers to reduce the utility drop. The results for this setting do not apply OWL and consider uniform unstructured sparsity throughout layers. Results for OATS and \ourframework are reported in \cref{tab:supp-slr-unstructured}.




\begin{table}[h!]
\centering
\resizebox{1.0\textwidth}{!}{%
\renewcommand{\arraystretch}{1.3}
\begin{tabular}{ccc@{\hskip 8pt}cccc@{\hskip 8pt}ccccccccc}
\toprule
\multirow{2.25}{*}{\textbf{Model}} & \multirow{2.25}{*}{\textbf{Algorithm}} && \multicolumn{3}{c}{\textbf{Perplexity ($\downarrow$)}} && \multicolumn{9}{c}{\textbf{Zero-shot ($\uparrow$)}} \\ 
\cmidrule(rl){4-6} \cmidrule(r){8-16}
&&& \textbf{C4} & \textbf{WT2} & \textbf{PTB} && \textbf{PIQA} & \textbf{HS} & \textbf{ARC-E} & \textbf{ARC-C} & \textbf{WG} & \textbf{RTE} & \textbf{OQA} & \textbf{BoolQ} & \textbf{Avg}\\ 

\midrule
\multirow{7.5}{*}{Llama3-8B}
&\texttt{OATS-60\%+LR}       && 23.61 & 16.52 & 25.85 && 72.91 & 59.65 & \textbf{60.10} & 33.36 & 65.35 & 53.07 & 31.60 & \textbf{75.96} & 56.50 \\
&\texttt{Ours-60\%+LR}       &&  \textbf{20.70} & \textbf{15.66} & \textbf{23.31} && \textbf{73.29} & \textbf{60.58} & 59.26 & \textbf{34.64} & \textbf{67.88} & \textbf{53.43} & \textbf{35.40} & 75.08 & \textbf{57.44} \\
\cmidrule(rl){2-2}
&\texttt{OATS-70\%+LR}       &&  106.98 & 81.77 & 110.44 && 55.60 & 30.30 & 32.45 & 20.05 & 49.96 & \textbf{52.71} & 27.00 & 62.35 & 41.30 \\
&\texttt{Ours-70\%+LR}       && \textbf{50.07} & \textbf{49.13} & \textbf{60.89} && \textbf{60.50} & \textbf{39.67} & \textbf{37.21} & \textbf{23.38} & \textbf{55.25} & \textbf{52.71} & \textbf{27.40} & \textbf{66.09} & \textbf{45.27} \\

\cmidrule(rl){2-2}
&\texttt{OATS-80\%+LR}       && 748.40 & 909.75 & 1601.02 && 52.29 & 27.25 & 26.81 & \textbf{24.40} & 47.59 & \textbf{52.71} & \textbf{26.60} & 37.83 & 36.93 \\
&\texttt{Ours-80\%+LR}       && \textbf{164.27} & \textbf{265.28} & \textbf{235.38} && \textbf{53.32} & \textbf{28.53} & \textbf{29.38} & 20.22 & \textbf{49.49} & \textbf{52.71} & \textbf{26.60} & \textbf{38.84} & \textbf{37.39} \\
\cmidrule(rl){2-2}
&\texttt{dense}               && 11.33 & 7.81 & 13.53 && 77.48 & 73.61 & 71.63 & 45.99 & 69.85 & 54.51 & 43.00 & 73.39 & 63.68 \\


\midrule
\multirow{7.5}{*}{Llama3.2-3B} 
&\texttt{OATS-60\%+LR}       && 34.57 & 24.94 & 41.51 && 67.79 & 48.40 & \textbf{52.57} & \textbf{30.38} & 57.70 & 54.15 & \textbf{30.80} & 65.66 & 50.93 \\
&\texttt{Ours-60\%+LR}       && \textbf{27.67} & \textbf{21.90} & \textbf{33.40} && \textbf{69.15} & \textbf{52.04} & 51.26 & 29.52 & \textbf{61.96} & \textbf{58.12} & 29.80 & \textbf{69.72} & \textbf{52.70} \\
\cmidrule(rl){2-2}
&\texttt{OATS-70\%+LR}       &&  155.48 & 121.76 & 167.60 && 54.57 & 29.83 & 30.43 & 21.42 & \textbf{49.64} & \textbf{52.71} & \textbf{28.20} & 60.43 & 40.90 \\
&\texttt{Ours-70\%+LR}       && \textbf{78.65} & \textbf{75.23} & \textbf{103.10} && \textbf{58.43} & \textbf{32.44} & \textbf{35.27} & \textbf{21.67} & 49.41 & \textbf{52.71} & 27.00 & \textbf{62.29} & \textbf{42.40} \\
\cmidrule(rl){2-2}
&\texttt{OATS-80\%+LR}       && 1085.27 & 1610.87 & 2546.29 && 50.60 & 26.60 & 26.68 & \textbf{24.40} & 47.67 & \textbf{52.71} & \textbf{26.60} & 37.83 & 36.64 \\
&\texttt{Ours-80\%+LR}       && \textbf{217.62} & \textbf{320.98} & \textbf{320.02} && \textbf{53.10} & \textbf{27.86} & \textbf{29.12} & 22.01 & \textbf{47.75} & 50.54 & \textbf{26.60} & \textbf{46.61} & \textbf{37.95} \\
\cmidrule(rl){2-2}
&\texttt{dense}               && 11.33 & 7.81 & 13.53 && 77.48 & 73.61 & 71.63 & 45.99 & 69.85 & 54.51 & 43.00 & 73.39 & 63.68 \\

\bottomrule
\end{tabular}
}
\vspace{2pt}
\caption{Performance analysis for one-shot unstructured sparsity plus a low-rank matrix decomposition of Llama3 and Llama3.2-3B model. The rank ratio of the low-rank component is fixed to be $\kappa=0.3$. For Perplexity, $(\downarrow)$ lower values are preferred. For zero-shot tasks, $(\uparrow)$ higher values are preferred.}
\label{tab:supp-slr-unstructured}
\end{table}


\end{document}
