\documentclass{article}

%
% --- inline annotations
%
\newcommand{\red}[1]{{\color{red}#1}}
\newcommand{\todo}[1]{{\color{red}#1}}
\newcommand{\TODO}[1]{\textbf{\color{red}[TODO: #1]}}
% --- disable by uncommenting  
% \renewcommand{\TODO}[1]{}
% \renewcommand{\todo}[1]{#1}



\newcommand{\VLM}{LVLM\xspace} 
\newcommand{\ours}{PeKit\xspace}
\newcommand{\yollava}{Yo’LLaVA\xspace}

\newcommand{\thisismy}{This-Is-My-Img\xspace}
\newcommand{\myparagraph}[1]{\noindent\textbf{#1}}
\newcommand{\vdoro}[1]{{\color[rgb]{0.4, 0.18, 0.78} {[V] #1}}}
% --- disable by uncommenting  
% \renewcommand{\TODO}[1]{}
% \renewcommand{\todo}[1]{#1}
\usepackage{slashbox}
% Vectors
\newcommand{\bB}{\mathcal{B}}
\newcommand{\bw}{\mathbf{w}}
\newcommand{\bs}{\mathbf{s}}
\newcommand{\bo}{\mathbf{o}}
\newcommand{\bn}{\mathbf{n}}
\newcommand{\bc}{\mathbf{c}}
\newcommand{\bp}{\mathbf{p}}
\newcommand{\bS}{\mathbf{S}}
\newcommand{\bk}{\mathbf{k}}
\newcommand{\bmu}{\boldsymbol{\mu}}
\newcommand{\bx}{\mathbf{x}}
\newcommand{\bg}{\mathbf{g}}
\newcommand{\be}{\mathbf{e}}
\newcommand{\bX}{\mathbf{X}}
\newcommand{\by}{\mathbf{y}}
\newcommand{\bv}{\mathbf{v}}
\newcommand{\bz}{\mathbf{z}}
\newcommand{\bq}{\mathbf{q}}
\newcommand{\bff}{\mathbf{f}}
\newcommand{\bu}{\mathbf{u}}
\newcommand{\bh}{\mathbf{h}}
\newcommand{\bb}{\mathbf{b}}

\newcommand{\rone}{\textcolor{green}{R1}}
\newcommand{\rtwo}{\textcolor{orange}{R2}}
\newcommand{\rthree}{\textcolor{red}{R3}}
\usepackage{amsmath}
%\usepackage{arydshln}
\DeclareMathOperator{\similarity}{sim}
\DeclareMathOperator{\AvgPool}{AvgPool}

\newcommand{\argmax}{\mathop{\mathrm{argmax}}}     


\usepackage[preprint]{cpal_2025}

%
\setlength\unitlength{1mm}
\newcommand{\twodots}{\mathinner {\ldotp \ldotp}}
% bb font symbols
\newcommand{\Rho}{\mathrm{P}}
\newcommand{\Tau}{\mathrm{T}}

\newfont{\bbb}{msbm10 scaled 700}
\newcommand{\CCC}{\mbox{\bbb C}}

\newfont{\bb}{msbm10 scaled 1100}
\newcommand{\CC}{\mbox{\bb C}}
\newcommand{\PP}{\mbox{\bb P}}
\newcommand{\RR}{\mbox{\bb R}}
\newcommand{\QQ}{\mbox{\bb Q}}
\newcommand{\ZZ}{\mbox{\bb Z}}
\newcommand{\FF}{\mbox{\bb F}}
\newcommand{\GG}{\mbox{\bb G}}
\newcommand{\EE}{\mbox{\bb E}}
\newcommand{\NN}{\mbox{\bb N}}
\newcommand{\KK}{\mbox{\bb K}}
\newcommand{\HH}{\mbox{\bb H}}
\newcommand{\SSS}{\mbox{\bb S}}
\newcommand{\UU}{\mbox{\bb U}}
\newcommand{\VV}{\mbox{\bb V}}


\newcommand{\yy}{\mathbbm{y}}
\newcommand{\xx}{\mathbbm{x}}
\newcommand{\zz}{\mathbbm{z}}
\newcommand{\sss}{\mathbbm{s}}
\newcommand{\rr}{\mathbbm{r}}
\newcommand{\pp}{\mathbbm{p}}
\newcommand{\qq}{\mathbbm{q}}
\newcommand{\ww}{\mathbbm{w}}
\newcommand{\hh}{\mathbbm{h}}
\newcommand{\vvv}{\mathbbm{v}}

% Vectors

\newcommand{\av}{{\bf a}}
\newcommand{\bv}{{\bf b}}
\newcommand{\cv}{{\bf c}}
\newcommand{\dv}{{\bf d}}
\newcommand{\ev}{{\bf e}}
\newcommand{\fv}{{\bf f}}
\newcommand{\gv}{{\bf g}}
\newcommand{\hv}{{\bf h}}
\newcommand{\iv}{{\bf i}}
\newcommand{\jv}{{\bf j}}
\newcommand{\kv}{{\bf k}}
\newcommand{\lv}{{\bf l}}
\newcommand{\mv}{{\bf m}}
\newcommand{\nv}{{\bf n}}
\newcommand{\ov}{{\bf o}}
\newcommand{\pv}{{\bf p}}
\newcommand{\qv}{{\bf q}}
\newcommand{\rv}{{\bf r}}
\newcommand{\sv}{{\bf s}}
\newcommand{\tv}{{\bf t}}
\newcommand{\uv}{{\bf u}}
\newcommand{\wv}{{\bf w}}
\newcommand{\vv}{{\bf v}}
\newcommand{\xv}{{\bf x}}
\newcommand{\yv}{{\bf y}}
\newcommand{\zv}{{\bf z}}
\newcommand{\zerov}{{\bf 0}}
\newcommand{\onev}{{\bf 1}}

% Matrices

\newcommand{\Am}{{\bf A}}
\newcommand{\Bm}{{\bf B}}
\newcommand{\Cm}{{\bf C}}
\newcommand{\Dm}{{\bf D}}
\newcommand{\Em}{{\bf E}}
\newcommand{\Fm}{{\bf F}}
\newcommand{\Gm}{{\bf G}}
\newcommand{\Hm}{{\bf H}}
\newcommand{\Id}{{\bf I}}
\newcommand{\Jm}{{\bf J}}
\newcommand{\Km}{{\bf K}}
\newcommand{\Lm}{{\bf L}}
\newcommand{\Mm}{{\bf M}}
\newcommand{\Nm}{{\bf N}}
\newcommand{\Om}{{\bf O}}
\newcommand{\Pm}{{\bf P}}
\newcommand{\Qm}{{\bf Q}}
\newcommand{\Rm}{{\bf R}}
\newcommand{\Sm}{{\bf S}}
\newcommand{\Tm}{{\bf T}}
\newcommand{\Um}{{\bf U}}
\newcommand{\Wm}{{\bf W}}
\newcommand{\Vm}{{\bf V}}
\newcommand{\Xm}{{\bf X}}
\newcommand{\Ym}{{\bf Y}}
\newcommand{\Zm}{{\bf Z}}

% Calligraphic

\newcommand{\Ac}{{\cal A}}
\newcommand{\Bc}{{\cal B}}
\newcommand{\Cc}{{\cal C}}
\newcommand{\Dc}{{\cal D}}
\newcommand{\Ec}{{\cal E}}
\newcommand{\Fc}{{\cal F}}
\newcommand{\Gc}{{\cal G}}
\newcommand{\Hc}{{\cal H}}
\newcommand{\Ic}{{\cal I}}
\newcommand{\Jc}{{\cal J}}
\newcommand{\Kc}{{\cal K}}
\newcommand{\Lc}{{\cal L}}
\newcommand{\Mc}{{\cal M}}
\newcommand{\Nc}{{\cal N}}
\newcommand{\nc}{{\cal n}}
\newcommand{\Oc}{{\cal O}}
\newcommand{\Pc}{{\cal P}}
\newcommand{\Qc}{{\cal Q}}
\newcommand{\Rc}{{\cal R}}
\newcommand{\Sc}{{\cal S}}
\newcommand{\Tc}{{\cal T}}
\newcommand{\Uc}{{\cal U}}
\newcommand{\Wc}{{\cal W}}
\newcommand{\Vc}{{\cal V}}
\newcommand{\Xc}{{\cal X}}
\newcommand{\Yc}{{\cal Y}}
\newcommand{\Zc}{{\cal Z}}

% Bold greek letters

\newcommand{\alphav}{\hbox{\boldmath$\alpha$}}
\newcommand{\betav}{\hbox{\boldmath$\beta$}}
\newcommand{\gammav}{\hbox{\boldmath$\gamma$}}
\newcommand{\deltav}{\hbox{\boldmath$\delta$}}
\newcommand{\etav}{\hbox{\boldmath$\eta$}}
\newcommand{\lambdav}{\hbox{\boldmath$\lambda$}}
\newcommand{\epsilonv}{\hbox{\boldmath$\epsilon$}}
\newcommand{\nuv}{\hbox{\boldmath$\nu$}}
\newcommand{\muv}{\hbox{\boldmath$\mu$}}
\newcommand{\zetav}{\hbox{\boldmath$\zeta$}}
\newcommand{\phiv}{\hbox{\boldmath$\phi$}}
\newcommand{\psiv}{\hbox{\boldmath$\psi$}}
\newcommand{\thetav}{\hbox{\boldmath$\theta$}}
\newcommand{\tauv}{\hbox{\boldmath$\tau$}}
\newcommand{\omegav}{\hbox{\boldmath$\omega$}}
\newcommand{\xiv}{\hbox{\boldmath$\xi$}}
\newcommand{\sigmav}{\hbox{\boldmath$\sigma$}}
\newcommand{\piv}{\hbox{\boldmath$\pi$}}
\newcommand{\rhov}{\hbox{\boldmath$\rho$}}
\newcommand{\upsilonv}{\hbox{\boldmath$\upsilon$}}

\newcommand{\Gammam}{\hbox{\boldmath$\Gamma$}}
\newcommand{\Lambdam}{\hbox{\boldmath$\Lambda$}}
\newcommand{\Deltam}{\hbox{\boldmath$\Delta$}}
\newcommand{\Sigmam}{\hbox{\boldmath$\Sigma$}}
\newcommand{\Phim}{\hbox{\boldmath$\Phi$}}
\newcommand{\Pim}{\hbox{\boldmath$\Pi$}}
\newcommand{\Psim}{\hbox{\boldmath$\Psi$}}
\newcommand{\Thetam}{\hbox{\boldmath$\Theta$}}
\newcommand{\Omegam}{\hbox{\boldmath$\Omega$}}
\newcommand{\Xim}{\hbox{\boldmath$\Xi$}}


% Sans Serif small case

\newcommand{\Gsf}{{\sf G}}

\newcommand{\asf}{{\sf a}}
\newcommand{\bsf}{{\sf b}}
\newcommand{\csf}{{\sf c}}
\newcommand{\dsf}{{\sf d}}
\newcommand{\esf}{{\sf e}}
\newcommand{\fsf}{{\sf f}}
\newcommand{\gsf}{{\sf g}}
\newcommand{\hsf}{{\sf h}}
\newcommand{\isf}{{\sf i}}
\newcommand{\jsf}{{\sf j}}
\newcommand{\ksf}{{\sf k}}
\newcommand{\lsf}{{\sf l}}
\newcommand{\msf}{{\sf m}}
\newcommand{\nsf}{{\sf n}}
\newcommand{\osf}{{\sf o}}
\newcommand{\psf}{{\sf p}}
\newcommand{\qsf}{{\sf q}}
\newcommand{\rsf}{{\sf r}}
\newcommand{\ssf}{{\sf s}}
\newcommand{\tsf}{{\sf t}}
\newcommand{\usf}{{\sf u}}
\newcommand{\wsf}{{\sf w}}
\newcommand{\vsf}{{\sf v}}
\newcommand{\xsf}{{\sf x}}
\newcommand{\ysf}{{\sf y}}
\newcommand{\zsf}{{\sf z}}


% mixed symbols

\newcommand{\sinc}{{\hbox{sinc}}}
\newcommand{\diag}{{\hbox{diag}}}
\renewcommand{\det}{{\hbox{det}}}
\newcommand{\trace}{{\hbox{tr}}}
\newcommand{\sign}{{\hbox{sign}}}
\renewcommand{\arg}{{\hbox{arg}}}
\newcommand{\var}{{\hbox{var}}}
\newcommand{\cov}{{\hbox{cov}}}
\newcommand{\Ei}{{\rm E}_{\rm i}}
\renewcommand{\Re}{{\rm Re}}
\renewcommand{\Im}{{\rm Im}}
\newcommand{\eqdef}{\stackrel{\Delta}{=}}
\newcommand{\defines}{{\,\,\stackrel{\scriptscriptstyle \bigtriangleup}{=}\,\,}}
\newcommand{\<}{\left\langle}
\renewcommand{\>}{\right\rangle}
\newcommand{\herm}{{\sf H}}
\newcommand{\trasp}{{\sf T}}
\newcommand{\transp}{{\sf T}}
\renewcommand{\vec}{{\rm vec}}
\newcommand{\Psf}{{\sf P}}
\newcommand{\SINR}{{\sf SINR}}
\newcommand{\SNR}{{\sf SNR}}
\newcommand{\MMSE}{{\sf MMSE}}
\newcommand{\REF}{{\RED [REF]}}

% Markov chain
\usepackage{stmaryrd} % for \mkv 
\newcommand{\mkv}{-\!\!\!\!\minuso\!\!\!\!-}

% Colors

\newcommand{\RED}{\color[rgb]{1.00,0.10,0.10}}
\newcommand{\BLUE}{\color[rgb]{0,0,0.90}}
\newcommand{\GREEN}{\color[rgb]{0,0.80,0.20}}

%%%%%%%%%%%%%%%%%%%%%%%%%%%%%%%%%%%%%%%%%%
\usepackage{hyperref}
\hypersetup{
    bookmarks=true,         % show bookmarks bar?
    unicode=false,          % non-Latin characters in AcrobatÕs bookmarks
    pdftoolbar=true,        % show AcrobatÕs toolbar?
    pdfmenubar=true,        % show AcrobatÕs menu?
    pdffitwindow=false,     % window fit to page when opened
    pdfstartview={FitH},    % fits the width of the page to the window
%    pdftitle={My title},    % title
%    pdfauthor={Author},     % author
%    pdfsubject={Subject},   % subject of the document
%    pdfcreator={Creator},   % creator of the document
%    pdfproducer={Producer}, % producer of the document
%    pdfkeywords={keyword1} {key2} {key3}, % list of keywords
    pdfnewwindow=true,      % links in new window
    colorlinks=true,       % false: boxed links; true: colored links
    linkcolor=red,          % color of internal links (change box color with linkbordercolor)
    citecolor=green,        % color of links to bibliography
    filecolor=blue,      % color of file links
    urlcolor=blue           % color of external links
}
%%%%%%%%%%%%%%%%%%%%%%%%%%%%%%%%%%%%%%%%%%%



\begin{document}
%% ----------------------------------------------------------------------------
% BIWI SA/MA thesis template
%
% Created 09/29/2006 by Andreas Ess
% Extended 13/02/2009 by Jan Lesniak - jlesniak@vision.ee.ethz.ch
% Updated 16/03/2023 by Danda Pani Paudel - paudel@vision.ee.ethz.ch
%% ----------------------------------------------------------------------------

\begin{titlepage}

\thispagestyle{empty}

\fancypagestyle{empty}{
\lhead{\includegraphics[height=1.5cm]{images/ethlogo_black}}
\renewcommand{\headrulewidth}{0.0pt}
\rhead{\vspace*{-0.2cm}\includegraphics[height=1.4cm]{images/cvl_logo}}
\fancyfoot{}
}



\vspace*{2cm}
\begin{center}
\LARGE{\textbf{NPSim: Nighttime Photorealistic Simulation From Daytime Images With Monocular Inverse Rendering and Ray Tracing
}\\}
% NPSim: Nighttime Photorealistic Simulation From Daytime Images With Monocular Inverse Rendering and Ray Tracing
% \LARGE{\textbf{Subtitle Subtitle Subtitle}\\[1cm]}
\vspace{5pt}
\large{Project Thesis\\[0.8cm]}
\LARGE{Shutong Zhang\\}
\normalsize{Department of Information Technology and Electrical Engineering}
\end{center}

\begin{center}
 


% \begin{center}
% \begin{tabular}{ll}
% \multirow{2}{*}{\includegraphics[height=1cm]{images/biwi_logo}} & Computer Vision Laboratory\\ 
% & ETH Zurich
% \end{tabular}
%  \end{center}

\end{center}


\vfill
\begin{center}
\begin{tabular}{ll}
\Large{\textbf Advisor:} & \Large{Dr.~Christos Sakaridis}\\
\Large{\textbf Supervisor:} & \Large{Prof.~Dr.~Luc Van Gool}\\
% 			    & \small{Computer Vision Laboratory}\\
% 			    & \small{Department of Information Technology and Electrical Engineering}\\
\end{tabular}
\end{center}

\begin{center}
% \today\\
August 18, 2023
\end{center}


\end{titlepage}


\maketitle
% \mm{Thanks everyone for the comments. I am going to rewrite most of the paper on Sunday with some new results and integrating the feedback. If you could have a look at the paper again sometime on Monday that would be great! New deadline is Monday midnight.}
% \zx{We probably want to work on a better title -- revisit after reading the whole paper.}

% \begin{abstract}
% The impressive capabilities of Large Language Models (LLMs) in Natural Language Processing (NLP) comes at a cost of substantial computing resources. One line of work to address this issue is model compression that effectively reduces the model's size, while maintaining its performance. In particular, a novel compression scheme is to decompose the model's dense weights into a sum of sparse plus low-rank matrices. Despite recent works on such matrix decompositions in LLMs, designing principled one-shot methods for this type of compression remains partially addressed. We propose a framework coined \ourmethod for (semi-structured) sparse plus low-rank matrix decomposition of LLMs, that minimizes a well-posed optimization problem, solves it using principled algorithms, and achieves state-of-the-art performance on a wide-range of LLMs evaluation benchmarks, for different compression regimes: e.g. N:M sparse + Low-Rank, a regime for which highly-specialized CUDA kernels have recently been developed for runtime speedups and memory efficiency.
% \end{abstract}

\begin{abstract}
The impressive capabilities of large foundation models come at a cost of substantial computing resources to serve them. Compressing these pre-trained models is of practical interest as it can democratize deploying them to the machine learning community at large by lowering the costs associated with inference.
A promising compression scheme is to decompose foundation models' dense weights into a sum of sparse plus low-rank matrices.
In this paper, we design a unified framework coined \ourframework for (semi-structured) sparse plus low-rank matrix decomposition of foundation models.
Our framework introduces the local layer-wise reconstruction error objective for this decomposition, we demonstrate that prior work solves a relaxation of this optimization problem; and we provide efficient and scalable methods to minimize the \textit{exact} introduced optimization problem. 
\ourframework substantially outperforms state-of-the-art methods in terms of the introduced objective and a wide range of LLM evaluation benchmarks. For the Llama3-8B model with a 2:4 sparsity component plus a 64-rank component decomposition, a compression scheme for which recent work shows important inference acceleration on GPUs, \ourframework reduces the test perplexity by $12\%$ for the WikiText-2 dataset and reduces the gap (compared to the dense model) of the average of eight popular zero-shot tasks by $15\%$ compared to existing methods.
\end{abstract}


\vspace{-5pt}
\section{Introduction}
\label{sec:introduction}
\vspace{-5pt}
% !TEX root = ../main.tex

Large Language Models (LLMs) have shown remarkable capabilities on numerous tasks in Natural Language Processing (NLP), 
ranging from language understanding to generation \cite{bubeck2023sparks, achiam2023gpt,team2023gemini, dubey2024llama}. The huge success of LLMs comes with important challenges to deploy them due to their massive size and computational costs. For instance,  Llama-3-405B \cite{dubey2024llama} requires 780GB of storage in half precision (FP16) and hence multiple high-end GPUs are needed just for inference. \textit{Model compression} has emerged as an important line of research to reduce the costs associated with deploying these foundation models. In particular, neural network pruning \cite{obd, hassibi1992second, benbaki2023fast}, where model weights are made to be sparse after training, has garnered significant attention. Different sparsity structures (Structured, Semi-Structured and Unstructured) obtained after neural network pruning result in different acceleration schemes. \textit{Structured pruning} removes entire structures such as channels, filters, or attention heads \cite{lebedev2016fast,wen2016learning,voita2019analyzing,el2022data} and readily results in acceleration as model weights dimensions are reduced. \textit{Semi-Structured pruning}, also known as, N:M sparsity \cite{zhou2021learning} requires that at most $N$ out of $M$ consecutive elements are non-zero elements. Modern NVIDIA GPUs provide support for 2:4 sparsity acceleration. \textit{Unstructured pruning} removes individual weights \cite{han2015learning, guo2016dynamic} from the model's weights and requires specialized hardware for acceleration. For instance, DeepSparse \cite{kurtic2022optimal, pmlr-v119-kurtz20a, DBLP:journals/corr/abs-2111-13445} provide CPU inference acceleration for unstructured sparsity.\\
Specializing to LLMs, one-shot pruning~\cite{meng2024alps, frantar2023sparsegpt, sun2023simple, zhang2023dynamic}, where one does a single forward pass on a small amount of calibration data, and prunes the model without expensive fine-tuning/retraining, is of particular interest. This setup requires less hardware requirements. For instance, \citet{meng2024alps} show how to prune an OPT-30B \cite{opt} using a single consumer-level V100 GPU with 32GB of CUDA memory, whereas full fine-tuning of such model using Adam \cite{kingma2014adam} at half-precision requires more than 220GB of CUDA memory.

Although one-shot pruning has desirable computational properties, it can degrade models' predictive and generative performance. To this end, recent work has studied extensions of model pruning to achieve smaller utility drop of model performance from compression. 
% Multiple one-shot methods have been developed in quantization \cite{frantar2022gptq, frantar2023sparsegpt, lin2024awq, behdin2023quantease, dettmers2023spqr} and neural network pruning \cite{frantar2023sparsegpt, meng2024alps, zhang2024oats}, which is closer to this paper's line of research. These one-shot methods do not require retraining--which is extremely expensive for models of the size of Llama-3-405B-- and work as resource-saving techniques that retain the model's performance. 

An interesting compression mechanism in the field of \textit{model compression} is the Sparse plus Low-Rank Matrix-Decomposition problem which aims to approximate model's weights by a sparse component plus a low-rank component~\cite{hintermuller2015robust, candes2011robust, lin2011linearized, 5394889, zhou2011godec, JMLR:v24:21-1130, NIPS2014_443cb001, yu2017compressing, li2023losparse}. Specializing to LLMs,~\citet{zhang2024oats} propose OATS 
%that addresses this type of %compression and 
that outperforms pruning methods for the same compression ratio (number of non-zero elements) on a wide range of LLM evaluation benchmarks (e.g. perplexity in Language generation). 

OATS \cite{zhang2024oats} is however a matrix decomposition algorithm inspired from a pruning algorithm Wanda \cite{sun2023simple}. Wanda has been designed as a relaxation/approximation of another state-of-the-art pruning algorithm SparseGPT \cite{frantar2023sparsegpt}. While Wanda has been found to be extremely useful and efficient in practice, recent work \cite{meng2024alps} show results where Wanda fails for high-sparsity regimes. In this paper, we provide a unified optimization framework to decompose pre-trained model weights into sparse plus low-rank components based on a layer-wise loss function. Our framework is modular and can incorporate different pruning and matrix-decomposition algorithms (developed independently in different contexts).
%under the umbrella of the local %layer-wise reconstruction error; 
Similar to~\cite{meng2024alps} we observe that our optimization-based framework results in models with better model utility-compression tradeoffs. The difference is particularly pronounced for higher compression regimes. 
%especially for higher compression %budgets, where SOTA methods 
% Our numerical results also show similar findings to \citet{meng2024alps} where high-sparsity significantly degrades the performance of approximation-based optimization methods like OATS.

Concurrently, in a different and complementary line of work,~\citet{mozaffari2024slope} have open-sourced highly-specialized CUDA kernels designed for N:M sparse \cite{zhou2021learning} plus low-rank matrix decompositions that result in significant acceleration and memory reduction for the pre-training of LLMs.
We note that our focus here is on improved algorithms for one-shot sparse plus low-rank matrix decompositions for foundation models with billions of parameters which is different from the work of \citet{mozaffari2024slope} that focuses on accelerating the pre-training of LLMs. The designed CUDA kernels \cite{mozaffari2024slope} can be exploited in our setting for faster acceleration and reduced memory footprint during inference.





% \textbf{Summary of approach and contributions:} We propose \ourmethod: an accurate and scalable framework for Sparse plus Low-Rank Matrix Decomposition for LLMs. Following the previous work on one-shot pruning and model compression, we pursue a layerwise approach. In particular, the reconstruction error resulting from compression in the output of each layer is minimized, under the compression constraints (i.e., sparsity and low-rank constraints).

\textbf{Summary of approach.\,\,\,\,} Our framework is coined \ourframework: \underline{H}ardware-\underline{A}ware (Semi-\underline{S}tructured) \underline{S}parse plus \underline{L}ow-rank \underline{E}fficient \& approximation-\underline{free} matrix decomposition for foundation models.

Hardware-aware refers to the fact that we mostly focus on a N:M sparse \cite{zhou2021learning} plus low-rank decomposition, for which acceleration on GPUs is possible, although \ourframework supports any type of sparsity pattern (unstructured, semi-structured, structured) in the sparsity constraint. Approximation-free refers to the fact that we directly minimize the local layer-wise reconstruction error introduced in \cref{eq:matrix-decomposition}, whereas we show prior work minimizes an approximation of this objective.

%Our unified framework introduces a well-posed 
%%As a part of our proposed framework, we consider an 
%%optimization form
We formulate the compression/decomposition task as a clear optimization problem; we minimize a local layer-wise reconstruction objective where the weights are given by the sum of a sparse and low-rank component. 
%%%of dense model weights under the  
%This optimization problem is decoupled into a sparse minimization subproblem and a low-rank minimization subproblem. 
We propose an efficient Alternating-Minimization approach that scales to models with billions of parameters relying on 
two key components: one involving sparse minimization (weight sparsity) and the other involving a low-rank optimization. 
For each of these subproblems 
we discuss how approximations to the optimization task can retrieve prior algorithms.
%the introduced subproblems, 
%we consider approximations to the minimization objective and retrieve different algorithms from related works given different %approximations.

% We provide an efficient and scalable algorithm based on Alternating-Minimization that does not rely on any approximation at the objective minimization level. 
% While \ourframework supports any sparsity pattern (unstructured, semi-structured, structured) in the sparsity constraint, we mostly focus on N:M sparsity \cite{zhou2021learning}, to make the decomposition Hardware-aware, as \citet{mozaffari2024slope} show how to get acceleration on modern GPUs for N:M sparse plus low-rank decomposition.

We note that \ourframework~differs from prior one-shot (sparse) pruning methods~\cite{frantar2023sparsegpt, meng2024alps, benbaki2023fast} as we seek a sparse plus low-rank decompositon of weights.
%%%%%introducing the low-rank component. 
Additionally, it differs from prior one-shot sparse plus low-rank matrix decomposition methods~\cite{zhang2024oats}
%by considering an approximation-free minimization approach of the 
as we directly minimize the local layer-wise reconstruction objective introduced in \cref{eq:matrix-decomposition}.

Our main \textbf{contributions} can be summarized as follows.
\begin{compactitem}
    \item We introduce \ourframework a unified one-shot LLM compression framework that scales to models with billions of parameters where we directly minimize the local layer-wise reconstruction error subject to  a sparse plus low-rank matrix decomposition of the pre-trained dense weights. 
    %    formulates a sparse plus low-rank matrix decomposition as an optimization problem with a local layer-wise reconstruction objective. We discuss approximations of this objective and show that OATS a popular method is recovered in a particular approximation.

    
    \item \ourframework uses an Alternating-Minimization approach that iteratively minimizes a Sparse and a Low-Rank component. \ourframework uses a given pruning method as a plug-in for the subproblem pertaining to the sparse component. Additionally, it uses Gradient-Descent type methods for the subproblem pertaining to the Low-Rank component.
    
    % \item In the subproblem pertaining to the sparse component, a rewrite of the optimization formulation shows that one can use any pruning algorithm, that minimizes the layer-wise reconstruction error, as a plug-in to sparsify the weights. We choose to show results for the algorithm SparseGPT.
    
    % In this pruning subproblem, we also enhance the performance of \ourmethod by exploiting the invariance of the Hessian--of the layer-wise reconstruction error--in each subproblem of the Alternating Minimization procedure, for a given layer. In particular, we use a pre-processing step that computes and stores the Hessian inverse--of the objective--, which is then passed to the deployed pruning algorithm (e.g. SparseGPT). 
    % \item In the subproblem  pertaining to the Low-Rank component, we give a theoretical closed form solution to the subproblem.
    % which does not scale to problems with billions of parameters. 
    % We also present a more tractable first-order optimization method for a reparametrization of the the low-rank problem, which is scalable to models with billions of parameters.
    
    % as $\bfUVt$ and use first-order optimization methods to minimize the layer-wise reconstruction objective.

    \item We discuss how special cases of our framework relying on specific approximations of the objective retrieve popular methods such as OATS, Wanda and MP --- \cite{zhang2024oats, sun2023simple,han2015learning, sze2020efficient}. This provides valuable insights into the underlying connections across different methods. 

    \item \ourframework improves upon state-of-the-art methods for one-shot sparse plus low-rank matrix decomposition. 
    For the Llama3-8B model with a 2:4 sparsity component plus a 64-rank component decomposition, \ourframework reduces the test perplexity by $12\%$ for the WikiText-2 dataset and reduces the gap (compared to the dense model) of the average of eight popular zero-shot tasks by $15\%$ compared to existing methods.
\end{compactitem}





\vspace{-5pt}
\section{Related Work}
\label{sec:related-work}
\vspace{-5pt}
\section{Related Work}
\label{sec:related-work}

We give a brief review of data compression for volumetric data. 
We then discuss the use of contour trees in topological data analysis, followed by related work for topology-preserving compression techniques.   

\para{Lossy compression.} 
Lossless compression techniques allow the original data to be perfectly reconstructed, 
but they usually suffer from limited compression ratios (less than $2\times$ according to~\cite{son2014data}) in scientific data and thus are not practical. 
Lossy compression is an alternative way to reduce the unprecedented  size of scientific data. 
Traditional lossy techniques such as JPEG/JPEG2000 leverage wavelet theories and bit plane encoding to compress image data, but they are not adept at dealing with multidimensional scientific data in floating-point format. 
Recently, there has been an increasing trend to leverage deep learning techniques, such as the autoencoder~\cite{le2023hierarchical} and implicit neural representation (INR)~\cite{lu2021compressive}, for data compression.
An autoencoder is a neural network composed of two components: an encoder and a decoder. 
The encoder is trained to produce low-dimensional representations of the input data, whereas the decoder is trained to reconstruct the original input data from the output of the encoder. 
An INR model trains a small neural network that can be used to recreate the ground truth. 
The neural network itself is shipped as a compressed file, and to decompress it, one must simply evaluate the network on an appropriate input. 
One notable INR model for volumetric scalar fields is Neurcomp~\cite{lu2021compressive}.
Recently, spatial super-resolution (SSR) models have employed neural networks to accurately upscale low-resolution representation of data as a form of interpolation. 
Several volumetric scalar field compressors incorporate SSR models, such as SSR-TVD~\cite{han2020ssr} and the deep hierarchical model~\cite{wurster2022deep}.
Unfortunately, these general lossy techniques lack precise error control on the data, which limits their use on scientific data.


Error-controlled lossy compressors~\cite{lindstrom2014fixed,ballester2019tthresh,zhao2021optimizing,lakshminarasimhan2013isabela} have been proposed and leveraged by the scientific computing community to reduce the data size while controlling the distortion in the decompressed data. 
In general, such compressors can be categorized into transform-based and prediction-based. 
Transform-based lossy compressors rely on domain transforms for data decorrelation. 
For instance, ZFP~\cite{lindstrom2014fixed} divides data into small blocks and then compresses each block independently. The compression procedure inside each block includes exponent alignment for fixed point conversion, a near-orthogonal domain transform, and embedded encoding. 
TTHRESH~\cite{ballester2019tthresh} is another transform-based compressor that leverages singular value decomposition (SVD) to improve the decorrelation efficiency for high-dimensional data.

Prediction-based compressors employ prediction methods such as interpolation to approximate the ground truth. The differences between original and predicted data are quantized and then encoded using entropy encoding and lossless techniques. 
ISABELA~\cite{lakshminarasimhan2013isabela}, as one of the pioneering error-controlled prediction-based compressors, uses B-splines to predict data. 
SZ3~\cite{liang2022sz3,zhao2021optimizing,liang2018error}, the most recent general release in the SZ compressor family, uses a combination of a Lorenzo predictor~\cite{ibarria2003out}, cubic spline interpolation, and linear interpolation. 
In addition, AE-SZ~\cite{liu2021exploring} is proposed as a variation of SZ that incorporates autoencoders in the prediction pipeline.

\para{Contour trees.} Our augmented compressors aim to preserve the contour tree of an input scalar field. 
Contour trees capture the relationships among contours of scalar fields. 
They have been used to support data analysis and visualization tasks across diverse disciplines, such as astronomy \cite{rosen2021using}, fluid dynamics \cite{aydogan2014characterization}, and medicine \cite{aydogan2013analysis, wang2018fast, szymczak2010categorical}. 
They have also been incorporated into algorithms in computer vision \cite{mizuta2005description} and visualization \cite{zhou2009automatic, kopp2022temporal} for interactive exploration of contours.  

\para{Topology-preserving compression.} 
To the best of our knowledge, only three compressors have been developed for scalar field compression with topological guarantees. 
The first compressor was developed by Soler et al. \cite{soler2018topologically}. We shall refer to it as TopoQZ. 
TopoQZ allows the user to specify a single parameter $\varepsilon$. 
It preserves all critical point pairs with finite persistence greater than $\varepsilon$ and eliminates all critical points with persistence less than $\varepsilon$. 
TopoQZ is not designed to perfectly preserve the contour tree. Therefore, the locations of preserved critical points may shift slightly during compression, and the connectivity of the critical points in the contour tree may be altered. 
TopoQZ can also guarantee that the reconstructed values differ from the ground truth at most by a user-specified error bound $\xi$. It is required that $\xi > \varepsilon$.
TopoQZ is currently implemented in the Topology Toolkit \cite{TiernyFavelierLevine2017, MasoodBudinFalk2021, leguillou_tvcg24}. That implementation couples TopoQZ with ZFP \cite{lindstrom2014fixed}, which improves the smoothness of the data but introduces additional pointwise error.

Another topology-preserving compressor is TopoSZ \cite{yan2023toposz}. 
TopoSZ modifies the classic SZ pipeline to perfectly preserve the contour tree of the ground truth data up to the persistence threshold of $\varepsilon$. 
That is, the contour tree of the output of TopoSZ will be equal to that of the ground truth after both datasets have been topologically simplified with a persistence threshold of $\varepsilon$. 
TopoSZ also allows the user to impose a strict error-bound $\xi$ (and allows $\xi \leq \varepsilon$). When compared with TopoQZ, TopoSZ yields generally higher compression ratios and reconstruction quality, although the algorithm takes longer to execute. 
Our general framework borrows some elements from the TopoSZ pipeline. However, our framework differs significantly from TopoSZ due to two technical innovations: progressive bound tightening and logarithmic-scaling quantization (see \cref{sec:method} for details).  

Most recently, Li et al. developed mSZ~\cite{li2024msz} that augments an existing lossy compressor to compress a 2D/3D scalar field while preserving its piecewise-linear (PL) Morse--Smale segmentation \cite{edelsbrunner2001hierarchical,edelsbrunner2003morse},~i.e.,~a partition of the data domain based on the Morse--Smale complex. 
In comparison to the contour tree that is based on the level sets of a scalar field, a Morse--Smale complex is a different topological descriptor based on the gradient behavior of a scalar field.   
Because our framework instead preserves contour trees and does not consider the gradients in its pipeline, mSZ is not directly comparable to our work. 

Finally, even though it does not preserve any common topological descriptor, cpSZ \cite{liang2022toward}---a variation of SZ---preserves the critical points of a vector field. cpSZ also introduces a log-scale quantization technique to store different error bounds for individual points.


\vspace{-5pt}
\section{Problem Formulation}
\label{sec:optimization-formulation}
\vspace{-5pt}
% !TEX root = ../main.tex

Before discussing our method, let us briefly introduce some important notations that we will use throughout the paper.

\textbf{Notation.\,\,\,\,} For a general matrix $\bfZ \in \R^{m \times n}$, 
$\rk{\bfZ}$ denotes the rank of $\bfZ$.
For a given rank $r \in \N$, $C_r(\mathbf{Z}) = \bfU_r \mathbf{\Sigma}_r \bfV_r^T$, corresponding to the matrices formed by retaining only the top-$r$ singular vectors and singular values from the full SVD of $\mathbf{Z}$. The \citet{eckart1936approximation} theorem shows that $C_r(\bfZ) = \argmin_\bfM \norm{\bfZ - \bfM},\,\, \rk{\bfM} \leq r.$

For a square matrix $\bfZ \in \R^{n \times n}$,
$\Tr{\bfZ} = \sum_{i \in [n]}\bfZ_{ii}$ denotes the trace of $\bfZ$,
$\diag{\bfZ}$ denotes the diagonal matrix $\bfD \in \R^{n \times n}$ such that $\bfD_{ii} = \bfZ_{ii}$ for any $i \in [n]$, and $\bfD_{ij} = 0$ for any $i \neq j, \, i,j \in [n]$.
We also note $\mathbf{1}_n$ and $\mathbf{0}_n$ the vector of entries all ones and all zeros respectively of size $n \in \N$.

\textbf{Layer-wise Reconstruction Error.\,\,\,\,} 
A common approach in post-training LLM compression is to decompose the full-model compression problem into layer-wise subproblems. The quality of the solution for each subproblem is assessed by measuring the $\ell_2$ error between the output of the dense layer and that of the compressed one, given a set of input activations.\\
More formally, let $\bfWold \in \R^{\Nin \times \Nout}$ denote the (dense) weight matrix of layer $\ell$, where $\Nin$ and $\Nout$ denote the input and output dimension of the layer, respectively. Given a set of $N$ calibration samples, the input activation matrix can be represented as $\bfX \in \R^{NL \times \Nin}$, where $L$ is the sequence length of an LLM. It corresponds to the output of the previous layer $\ell - 1$.  The goal of the matrix decomposition algorithm is to find a sum of a sparse weight matrix $\bfWS$ and a low-rank weight matrix $\bfM$ that minimizes the reconstruction error between the original and new layer outputs, while satisfying a target sparsity constraint and a low-rank constraint. The optimization problem is given by
\begin{equation}\label{eq:matrix-decomposition}
       \min\nolimits_{\bfWS, \bfM} \,\, \left\|\bfX \bfWold-\bfX \pr{\bfWS + \bfM}\right\|_F^2 ~~~~\text{s.t. } ~~~ \bfWS\in \calCS, ~~~\rk{\bfM} \leq r.
\end{equation}
where $\bfWS, \bfM \in \R^{\Nin \times \Nout}$, $\calCS$ denotes the the sparsity-pattern constraint set.

\vspace{-5pt}
\section{Algorithm Design}
\label{sec:algorithm-design}
\vspace{-5pt}
% !TEX root = ../main.tex

The Optimization of Problem \eqref{eq:matrix-decomposition} is challenging: we need to jointly find a support for $\bfWS$ so that it is feasible for the set $\calCS$, a subspace of dimension $r$ where the low-rank matrix $\mathbf{M}$ lies, and optimal weights, within these constrained sets, for both matrices to minimize the layerwise-reconstruction error. While such formulation relates to the Robust-PCA literature \cite{chandrasekaran2011rank, candes2011robust, hintermuller2015robust}, the size of parameters in $\bfWS$ and $\bfM$ can reach over 100 million in the LLM setting. For instance, the size of a down projection in a FFN of a Llama3-405b \cite{dubey2024llama} has more than 800 million parameters. 
Classical methods fail to be deployed at this scale, which makes designing novel algorithms that are more computationally efficient a necessity to solve the layerwise-reconstruction matrix decomposition problem \eqref{eq:matrix-decomposition}.

In this paper, we propose to optimize problem \eqref{eq:matrix-decomposition} using an Alternating-Minimization approach \cite{hintermuller2015robust, zhou2011godec}. We aim to decompose the problem into two 'friendlier' subproblems and iteratively minimize each one of them. In particular, we would like to iteratively solve, at iteration $t$, the subproblem \Pone, which pertains to the sparse component of the matrix decomposition.
\begin{align}\label{eq:pone-pruning}
       \bfWS^{(t+1)} &\in \argmin\nolimits_{\bfWS} \,\, \left\|\bfX \bfWold-\bfX \pr{\bfWS + \bfM^{(t)}}\right\|_F^2 ~~~~\text{s.t. } ~~~ \bfWS\in \calCS\\
       &= \argmin\nolimits_{\bfWS} \,\, \left\|\bfX \pruningW^{(t)} - \bfX \bfWS\right\|_F^2 ~~~~\text{s.t. } ~~~ \bfWS\in \calCS \tag{$\pruningW^{(t)} := \bfWold - \bfM^{(t)}$}.
\end{align}
The second subproblem to be solved, at iteration $t$, which pertains to the low-rank component of the matrix decomposition problem, is \Ptwo.
\begin{align}\label{eq:ptwo-low-rank}
       \bfM^{(t+1)} &\in \argmin\nolimits_{\bfM} \,\, \left\|\bfX \bfWold-\bfX \pr{\bfWS^{(t+1)} + \bfM}\right\|_F^2 ~~~~\text{s.t. } ~~~\rk{\bfM} \leq r\\
        &= \argmin\nolimits_{\bfM} \,\, \left\|\bfX \lowrankW^{(t+1)} -\bfX \bfM\right\|_F^2 ~~~~\text{s.t. } ~~~\rk{\bfM} \leq r \tag{$\lowrankW^{(t+1)} := \bfWold - \bfWS^{(t+1)}$}.
\end{align}
Before proceeding to discuss algorithms that solve different variations of \Pone and \Ptwo, and draw connections between existing methods in the literature of \textit{model compression}, we remove the dependence on the iteration $t$ and study \eqref{eq:pone-pruning} rewritten as follows.
\begin{align}\label{eq:general-pruning}
       \bfWS^\star 
       &\in \argmin\nolimits_{\bfWS} \,\, \left\|\bfX \pruningW - \bfX \bfWS\right\|_F^2 ~~~~\text{s.t. } ~~~ \bfWS\in \calCS\\
       &= \argmin\nolimits_{\bfWS} \,\, \Tr{(\pruningW - \bfWS)^\top \bfH (\pruningW - \bfWS)}~~~~\text{s.t. } ~~~ \bfWS\in \calCS \tag{$\bfH = \bfXTX$}. 
\end{align}

Similarly, we study \eqref{eq:ptwo-low-rank} rewritten as follows.
\begin{align}\label{eq:general-low-rank}
   \bfM^\star &\in \argmin\nolimits_{\bfM} \,\, \left\|\bfX \lowrankW -\bfX \bfM\right\|_F^2 ~~~~\text{s.t. } ~~~\rk{\bfM} \leq r\\
   &= \argmin\nolimits_{\bfM} \,\, \Tr{(\lowrankW - \bfM)^\top \bfH (\lowrankW - \bfM)} ~~~~\text{s.t. } ~~~\rk{\bfM} \leq r \notag{}.
\end{align}




\subsection{Minimizing Subproblem \Pone}\label{section-pone}
\vspace{-2pt}
To solve \eqref{eq:general-pruning}, one can consider multiple variations for $\bfH$, the Hessian of the local layer-wise reconstruction error.
\vspace{-3pt}
\subsubsection{Data-Free version: $\bfX = \bfI_{\Nin \times \Nin} \implies \bfH = \bfI_{\Nin \times \Nin}$} 
\vspace{-3pt}
A data-free pruning method (without a calibration dataset) considers $\bfX$ to be an identity matrix in \eqref{eq:general-pruning}. When $\bfH$ is an identity matrix, equation \eqref{eq:general-pruning} can be solved to optimality and an optimal solution is obtained with Magnitude Pruning (MP, \cite{han2015learning, sze2020efficient}) using a simple Hard-Thresholding operator on the dense weight $\pruningW$ -- keeping the largest values and setting the remaining values to zero. Note that MP can be applied to unstructured \cite{han2015learning}, semi-structured N:M sparsity \cite{zhou2021learning}, and structured pruning \cite{meng2024alps}. This accommodates most sparsity sets $\calCS$ in the pruning literature.

\subsubsection{Diagonal-approximation: $\bfH = \diagn{\bfXTX}$}\label{subsection-pruning-diagonal-approximation}
\vspace{-3pt}
An efficient way to approach problem \eqref{eq:general-pruning} is to approximate the Hessian of the local layer-wise reconstruction error by its diagonal. An optimal solution in this case, can be obtained by Hard-Thresholding $\bfD \pruningW$, where $\bfD = \sqrt{\diagn{\bfXTX}}$. Note that this approximation results in the state-of-the-art pruning algorithm Wanda \cite{sun2023simple}. In fact, the importance metric, $S_{ij}$ introduced in Wanda for each entry $\pruningW_{ij}$ reads as follows. Here $\bfX_j$ denotes the $j^{th}$ column of the input activation matrix $\bfX$.
\begin{equation}\label{eq:wanda-metric}
    S_{ij} = \abs{\pruningW_{ij}} \cdot \norm{\bfX_j}_2 = \abs{\bfD \pruningW}_{ij}. \tag{$\bfD = \sqrt{\diagn{\bfXTX}}$}
\end{equation}
\citet{sun2023simple} show impressive results with this approximation for unstructured and semi-structured sparsity. OATS \cite{zhang2024oats} is inspired by Wanda and decomposes model weights into sparse plus low-rank using Alternating-Minimization; their sparse update reduces to this approximation (diagonal of the local layer-wise objective's Hessian).
\vspace{-3pt}
\subsubsection{Full Hessian: $\bfH = \bfXTX + \lambda \bfI$}\label{full-hessian-pruning}
\vspace{-3pt}
This approach aims to directly minimize \eqref{eq:general-pruning}. \citet{frantar2023sparsegpt} are the first to use the full Hessian of the local layer-wise reconstruction objective \Pone at the scale of LLMs for pruning using approximations at the algorithmic level (as opposed to an approximation at the optimization formulation level). \citet{meng2024alps} extend this formulation using the operator splitting technique ADMM \cite{boyd2011distributed} and show impressive results for unstructured sparsity and N:M sparsity. \citet{meng2024osscar} extend the formulation for structured sparsity by leveraging combinatorial optimization techniques.

Our framework works as a plug-in method for any pruning algorithm to minimize \Pone at iteration $t$. Since we aim to minimize \cref{eq:matrix-decomposition} in an approximation-free manner, we select methods that use the entire Hessian as they tend to give better performance for high compression ratios. SparseGPT \cite{frantar2023sparsegpt} is a popular pruning method that considers the entire Hessian. In particular, for our numerical results, we use SparseGPT to minimize \Pone.
\vspace{-3pt}
\subsection{Minimizing Subproblem \Ptwo}
\vspace{-3pt}
As in the previous section \ref{section-pone}, we discuss algorithms and related work for different variations of $\bfH$.
\vspace{-3pt}
\subsubsection{Data-Free version: $\bfX = \bfI_{\Nin \times \Nin}$} 
\vspace{-3pt}
Drawing a line from the pruning literature, a data-free version is introduced that does not require a calibration dataset. In this case, a closed-form solution of the minimizer is given by the Truncated-SVD $C_r(\lowrankW)$. This corresponds to the best rank-$r$ approximation of $\lowrankW$. \citet{li2023loftq} use SVD on the full matrix during their low-rank minimization step for quantization plus low-rank matrix decomposition. \citet{guo2023lq} use a Randomized-SVD \cite{halko2011finding} approach for the same problem (quantization plus low-rank decomposition) instead of the full SVD since it is reduces runtime significantly while maintaining the minimization performance.
\vspace{-3pt}
\subsubsection{Diagonal-approximation: $\bfH = \diagn{\bfXTX}$}\label{subsection-diagonal-approximation}
\vspace{-3pt}
The diagonal approximation of $\bfH$ has been made popular in the pruning literature thanks to Wanda \cite{sun2023simple}. We analyze the minimizing \Ptwo with this approximation. Similar to \ref{subsection-pruning-diagonal-approximation}, we introduce $\bfD = \sqrt{\diagn{\bfXTX}}$ in equation \eqref{eq:general-low-rank}. Here we use the fact that $\bfD$ is symmetric.
\begin{align*}
   \bfM^\star 
   &\in \argmin\nolimits_{\bfM} \,\, \Tr{(\lowrankW - \bfM)^\top \bfD^2 (\lowrankW - \bfM)} ~~~~\text{s.t. } ~~~\rk{\bfM} \leq r\\
   &= \argmin\nolimits_{\bfM} \,\, \left\|\bfD \lowrankW -\bfD \bfM\right\|_F^2 ~~~~\text{s.t. } ~~~\rk{\bfM} \leq r.
\end{align*}

\begin{assumption}\label{ass:full-rank-diagonal}
The input activations matrix $\bfX$ satisfies $\diag{\bfXTX}$ is full-rank. Equivalently, no column of $\bfX$ is identically $\mathbf{0}_{N \cdot L}$.
\end{assumption}
\begin{theorem}\label{theorem:low-rank-closed-form-diag-approx}
    If \cref{ass:full-rank-diagonal} holds, then the closed-form minimizer of \eqref{eq:general-low-rank} is given by
    \begin{equation*}
        \bfM^\star = \bfD^{-1} C_r(\bfD \lowrankW).
    \end{equation*}
\end{theorem}
The proof of theorem \ref{theorem:low-rank-closed-form-diag-approx} is obtained by introducing the auxialiary variable $\tilde{\bfM} = \bfD \bfM$ and noting that $\rkn{\tilde{\bfM}} = \rkn{\bfM}$, when \cref{ass:full-rank-diagonal} holds.

Interestingly, OATS \cite{zhang2024oats} uses the same operation in the Low-Rank update of the Alternating-Minimization approach (for sparse plus low rank matrix decomposition). 
%OATS uses a full SVD on the matrix $\bfD \lowrankW$. 
% \zx{Do we know what exactly OATS did? It looks to me we can do $\bfM^\star = C_r(\lowrankW)$ as $\bfD$ is diagonal? Computationally, it probably does not matter much as $\bfD$ is diagonal.}
% \mm{Actually it happens that $\bfD^{-1} C_r(\bfD \lowrankW) \neq C_r(\lowrankW)$ in the general case, that's why OATS is an interesting approach.}
\begin{corollary}
OATS \cite{zhang2024oats} exactly minimizes \eqref{eq:matrix-decomposition} with a diagonal approximation of the Hessian of the local layer-wise reconstruction error, since they minimize \Pone and \Ptwo with the same diagonal approximation $\bfH = \diagn{\bfXTX}$.
\end{corollary}
\vspace{-2pt}
\subsubsection{Full Hessian: $\bfH = \bfXTX + \lambda \bfI$}
\vspace{-3pt}
The state-of-the-art pruning algorithms in terms of retaining compressed LLMs performance on multiple benchmarks are the ones that use the full Hessian \ref{full-hessian-pruning} \cite{meng2024alps,frantar2023sparsegpt}. This motivates minimizing equation \eqref{eq:general-low-rank} using the full Hessian as well.
Dealing with low-rank constraints can be challenging, we therefore propose to reparametrize the low-rank matrix $\bfM \in \R^{\Nin \times \Nout}$ by  $\bfUVt$, with $\bfU \in \R^{\Nin \times r}, \bfV \in \R^{\Nout \times r}$. We can therefore use more computationally efficient first-order optimization methods to minimize the layer-wise reconstruction objective, which can be rewritten as follows.
\begin{equation}\label{eq:uvt-general-low-rank}
    \bfM^\star = \bfU^\star \bfV^{\star^\top}, \quad \bfU^\star, \bfV^\star \in \argmin\nolimits_{\bfU, \bfV} \,\, \Tr{\pr{\lowrankW - \bfUVt}^\top \bfH \pr{\lowrankW - \bfUVt}}.
\end{equation}
\textbf{Diagonal Scaling for \Ptwo Minimization Stability.\,\,\,\,}\label{scaling-low-rank}
Our initial experiments to minimize equation \eqref{eq:uvt-general-low-rank}, using Gradient-Descent type methods on $\bfU$ and $\bfV$, have shown that the optimization problem can be ill-conditioned in some tranformer layers. This can lead to numerical instability in the optimization procedure. To address this, we follow a similar rescaling approach proposed by \citet{meng2024alps}. Define (similar to \ref{subsection-diagonal-approximation}) the matrix $\bfD = \sqrt{\diagn{\bfXTX}}$ and reformulate the optimization equation \eqref{eq:uvt-general-low-rank} as follows (when \cref{ass:full-rank-diagonal} holds).
\begin{equation}\label{eq:scaled-uvt-general-low-rank}
    \bfM^\star = \bfD\bfU^\star \bfV^{\star^\top}, \quad \bfU^\star, \bfV^\star \in \argmin\nolimits_{\bfU, \bfV} \,\, \Tr{\pr{\bfD\lowrankW - \bfUVt}^\top \bfD^{-1}\bfH\bfD^{-1} \pr{\bfD\lowrankW - \bfUVt}}.
\end{equation}
It is important to note that the minimization problems in equations \eqref{eq:scaled-uvt-general-low-rank} and \eqref{eq:uvt-general-low-rank} are equivalent, in terms of objective minimization and feasibility of $\bfM^\star$, which has rank at most $r$. This scaling, which sets the diagonal of the new Hessian to $\mathbf{1}_{\Nin}$, only modifies the steps of Gradient Descent and leads to faster convergence in practice. See the \Cref{fig:reconstruction-error} for an objective minimization comparison of the effect of this Diagonal scaling.
It is also worth noting that this scaling allows to use the same learning rate $\eta$ for Gradient-Descent type methods for all layers and all models.

\subsection{Our Proposed Approach}
\vspace{-5pt}
Our goal is to minimize \eqref{eq:matrix-decomposition} using Alternating-Minimization with the full Hessian approach $\bfH = \bfXTX$. Our numerical results show that leveraging the entire Hessian outperforms OATS \cite{zhang2024oats}, which minimizes \eqref{eq:matrix-decomposition} with the diagonal approximation of the Hessian approach $\bfH = \diagn{\bfXTX}$, on a wide-range of LLM benchmarks and compression ratios. 

In our numerical experiments, we show results with the SparseGPT \cite{frantar2023sparsegpt} algorithm to minimize \Pone and the Adam algorithm \cite{kingma2014adam} to minimize \Ptwo reparametrized and rescaled as in \cref{eq:scaled-uvt-general-low-rank}. 

\textbf{Computational Efficiency.\,\,\,\,} Note that for a given layer $\ell$, the Hessian of the local layer-wise reconstruction problem $\bfXTX$ in \eqref{eq:general-pruning} as well as the rescaled version $\bfD^{-1}\bfXTX\bfD^{-1}$ in \eqref{eq:general-low-rank} are invariant throughout iterations. 
This is very important as pruning algorithms that use the entire Hessian information \cite{frantar2023sparsegpt, meng2024alps} need the Hessian inverse in their algorithm update. This inversion and associated costs of Hessian construction are done only once and then \textit{amortized} throughout iterations. In the \cref{algo:low-rank-gd}, we use $\bfU^{(t-1)}$ and $\bfV^{(t-1)}$ as initializations for the optimizer, as they are close to the minimizers of \Ptwo at iteration $t$. This accelerates the convergence in practice.

% We are now ready to present our proposed algorithm \ourframework in Algorithm \ref{algo:gdprune}.

\begin{algorithm}[h]
    \caption{\texttt{Low-Rank-GD}
    % \zx{Maybe inline this two line algorithm?}\mm{I like to keep it this way, to show (i) warm-up variables Uinit which makes optimization much faster and (ii) that scaling and no scaling can be solved using 'same' way.}
    }
    \label{algo:low-rank-gd}
    \begin{algorithmic}[]
        \State \textbf{Input} \texttt{Optimizer} (optimization algorithm, e.g. Adam), $\bfH$ (Hessian), $\bfW$ (Weights), $\bfU_\text{init}$, $\bfV_\text{init}$ (warm-up initialization for the joint minimization of $\bfU, \bfV$), $T_{\text{LR}}$ (\# iterations), $\eta$ (learning rate).
        \State $\text{Obj}(\bfU, \bfV) \gets \Tr{\prn{\bfW - \bfUVt}^\top \bfH \prn{\bfW - \bfUVt}}$        \vspace*{0.4em}
        \State $\bfU^\star, \bfV^\star \gets \texttt{Optimizer}_{\bfU, \bfV}\pr{\text{Obj}, \bfU_\text{init}, \bfV_\text{init}, N, \eta}$
        \vspace*{0.4em}
        \State \textbf{Output} $\bfU^\star, \bfV^\star$.
    \end{algorithmic}
\end{algorithm}
\vspace{-\baselineskip}
\begin{algorithm}[h]
    \caption{\ourframework}
    \label{algo:gdprune}
    \begin{algorithmic}[]
        \State \textbf{Input} for a given layer $\ell$: $\mathbf{H} = (\bfXTX + \lambda \mathbf{I})$ (Hessian of \eqref{eq:matrix-decomposition}, plus a regularization term for numerical stability), $\widehat{\bfW}$ (dense pre-trained weights), $T_{\text{AM}}$ (\# iterations of Alternating-Minimization), $T_{\text{LR}}$ (\# iterations of \texttt{Low-Rank-GD}), $\eta$ (learning rate for $\bfU, \bfV$), $\calCS$ (sparsity pattern), $r$ (rank of low-rank components), \texttt{Prune} (any pruning algorithm, e.g. SparseGPT), \texttt{Optimizer} (any first-order algorithm, e.g. Adam), \textbf{is\_scaled} (bool to apply scaling \ref{scaling-low-rank}).
        \vspace*{0.1em}
        \State $\bfD \gets \sqrt{\diag{\bfH}}$ \quad \Comment{Diagonal of the Hessian.}
        % \vspace*{0.1em}
        \State $\mathbf{H}^{-1} \gets \texttt{inv}\pr{\mathbf{H}}$ \quad \Comment{Inverse the Hessian.}
        % \vspace*{0.1em}
        \State $\bfWS \gets \mathbf{0}_{\Nin \times \Nout}$
        % \vspace*{0.1em}
        \State $\bfU \gets \mathbf{0}_{\Nin \times r}$
        % \vspace*{0.1em}
        \State $\bfV \gets \mathcal{N}_{\Nout \times r}$ \quad  \Comment{element-wise independent gaussian initialization.}
        % \vspace*{0.1em}
        \For{$t = 1 \dots T$}
            % \vspace*{0.1em}
            \State $\bfWS \gets \texttt{Prune}\pr{\mathbf{H}^{-1}, \widehat{\bfW} - \bfUVt, \calCS}$
            \State \Comment{$\bfWS \approx \widehat{\bfW} - \bfUVt$, satisfies $\calCS$ sparsity pattern \& minimizes \Pone.}
            \vspace*{0.2em}
            \State $\eta_t \gets \text{get\_lr}(t, \eta)$ \quad \Comment{In practice, $\eta_t = \eta / (t + 10)$.}
            \vspace*{0.2em}
            \If{\textbf{is\_scaled}}
                % \vspace*{0.1em}
                \State $\bfU, \bfV \gets \texttt{Low-Rank-GD}\pr{\texttt{Optimizer}, \bfD^{-1}\mathbf{H}\bfD^{-1}, \bfD\pr{\widehat{\bfW} - \bfWS}, \bfD\bfU, \bfV, T_{\text{LR}}, \eta_t}$
                \State $\bfU \gets \bfD \bfU$ \quad \Comment{Rescale $\bfU$ back.}
            \Else
                \State $\bfU, \bfV \gets \texttt{Low-Rank-GD}\pr{\texttt{Optimizer}, \mathbf{H}, \widehat{\bfW} - \bfWS, \bfU, \bfV, T_{\text{LR}}, \eta_t}$ 
                % \zx{Do we still need to keep this unscaled version?} \mm{I'm thinking of keeping it and include a very brief ablation study, I think it's nice to show that both work but scaled requires much less iterations in practice}
            \EndIf
            % \State $\bfU, \bfV \gets \texttt{Low-Rank-GD}\pr{\texttt{Optimizer}, \bfD^{-1}\mathbf{H}\bfD^{-1}, \bfD\pr{\widehat{\bfW} - \bfWS}, \eta_t}$
            % \vspace*{0.3em}
            % \State $\bfU \gets \bfD \bfU$ \quad \Comment{Rescale $\bfU$ back.}
            \State \Comment{$\bfUVt \approx \widehat{\bfW} - \bfWS$, has rank at most $r$ \& minimizes \Ptwo.}
        \EndFor
        % \vspace*{0.2em}
        \State $\bfM \gets \bfUVt$
        % \vspace*{0.1em}
        \State \textbf{Output} for a given layer $\ell$: $\bfWS, \bfM$.
    \end{algorithmic}
\end{algorithm}


\vspace{-5pt}
\section{Experimental Results}
\label{sec:experimental-results}
\vspace{-5pt}
% !TEX root = ../main.tex

\subsection{Experiment Setup}
\vspace{-3pt}
\textbf{Models and datasets} We evaluate our proposed method \ourframework on two families of large language models: Llama-3 and Llama-3.2 \cite{dubey2024llama} with sizes ranging from 1 to 8 billion parameters. 
To construct the Hessian $\bfXTX$, we follow the approch of \citet{frantar2023sparsegpt}: we use 128 segments of 2048 each, randomly sampled from the first shard of the C4 training dataset \cite{JMLR:v21:20-074}. To ensure consistency, we utilize the same calibration data for all pruning algorithms we benchmark. We also consider one-shot compression results, without retraining. 
We assess the performance using perplexity and zero-shot evaluation benchmarks, with perplexity calculated according to the procedure described by HuggingFace \cite{Perplexity}, using full stride. For perplexity evaluations, we use the test sets of raw-WikiText2 \cite{merity2017pointer}, PTB \cite{Marcus1994}, and a subset of the C4 validation data, which are popular benchmarks in LLM pruning literature \cite{frantar2023sparsegpt,meng2024alps,meng2024osscar}. Additionally, we evaluate the following zero-shot tasks using LM Harness by \citet{gao10256836framework}: PIQA \cite{bisk2020piqa}, ARC-Easy (ARC-E) \& ARC-Challenge (ARC-C) \cite{clark2018think}, Hellaswag (HS) \cite{zellers2019hellaswag}, Winogrande (WG) \cite{sakaguchi2021winogrande}, RTE \cite{poliak2020survey}, OpenbookQA (OQA) \cite{banerjee2019careful} and BoolQ \cite{clark2019boolq}. The average of the eight zero-shot tasks is also reported.

\vspace{-5pt}
\subsection{Results}
\vspace{-10pt}
In order to benchmark the performance of our matrix decomposition algorithm, \ourframework uses the same number of Alternating-Minimization steps as OATS \cite{zhang2024oats} which is $80$. We report results for the scaled version of \ourframework, which uses the same learning rate $\eta = 1e^{-2}$ for all layers and considered models. We consider the following two settings.

\textbf{N:M Sparsity + Fixed Rank:}
We impose the sparsity pattern $\calCS$ to be $N:M$ sparsity and we fix the target rank $r = 64$ of the low-rank component for all layers. We benchmark our method with OATS \cite{zhang2024oats}. The results are reported in \cref{tab:slr-fixed-rank}.

\textbf{N:M Sparsity + Fixed Compression Ratio:}
This is similar to the setting described by \citet{zhang2024oats} for N:M sparsity evaluations. Each layer, with dense weight matrix $\widehat{\bfW}$, is compressed to a prefixed compression ratio $\rho$ (e.g. $50\%$) so that $\widehat{\bfW} \approx \bfW_{N:M} + \bfM$, and the target rank is given by\\[0.2em] 
$r = \left\lfloor {(1 - \rho - \frac{N}{M}) \cdot(\Nout \cdot \Nin)} / \pr{\Nout + \Nin} \right\rfloor$.\\[0.2em]
Note that the effective number of parameters stored is therefore\\[0.1em]
$\#\text{params } \bfW_{N:M} + \#\text{params } \bfU + \#\text{params } \bfV = \frac{N}{M} \cdot (\Nout \cdot \Nin) + r \Nin + r \Nout \leq (1 - \rho) \cdot \#\text{params } \widehat{\bfW}$,\\[0.5em]
hence the comparison to other pruning methods matched at the same compression ratio $\rho$. The results are reported for the Llama3-8B model in \cref{tab:slr-fixed-compression} for \ourframework, \texttt{OATS}, and different N:M pruning algorithms (SparseGPT \cite{frantar2023sparsegpt}, Wanda \cite{sun2023simple}, DSNoT \cite{zhang2023dynamic}) compressed at $\rho = 50\%$. The results are expanded for \ourframework and OATS in \cref{supp:experiments-supp}. 
% We vary the compression ratio $\rho$, we then select some N:M sparsity values that result in a more aggressive compression, we then match this compression ratio thanks to the low-rank component.

% \begin{table}[h!]
% \centering
% \resizebox{1.0\textwidth}{!}{%
% \renewcommand{\arraystretch}{1.3}
% \begin{tabular}{cccccccccccccc}
% \toprule
% \multirow{2.25}{*}{\textbf{Model}} & \multirow{2.25}{*}{\textbf{Algorithm}} & \multicolumn{3}{c}{\textbf{Perplexity ($\downarrow$)}} & \multicolumn{9}{c}{\textbf{Zero-shot ($\uparrow$)}} \\ 
% \cmidrule(rl){3-5} \cmidrule(rl){6-14}
% & & \textbf{C4} & \textbf{WT2} & \textbf{PTB} & \textbf{PIQA} & \textbf{HS} & \textbf{ARC-E} & \textbf{ARC-C} & \textbf{WG} & \textbf{RTE} & \textbf{OQA} & \textbf{BoolQ} & \textbf{Avg}\\ 
% \midrule
% \multirow{2}{*}{Llama3-8B} & \texttt{OATS-2:8+64LR}       & 368.24 & 858.90 & --.-- & 52.29 & 27.32 & \textbf{22.7} & 37.61 & -- & -- & -- & -- \\ 
%                           & \texttt{Ours-2:8+64LR}       & \textbf{90.46}  & \textbf{88.58} & --.-- & \textbf{54.52} & \textbf{31.44} & 20.73 & \textbf{40.93} & -- & -- & -- & -- \\ 
% \bottomrule
% \end{tabular}%
% }
% \caption{Evaluation results for Llama3-8B. PPL columns minimize ($\downarrow$), and accuracy columns maximize ($\uparrow$).}
% \label{tab:results}
% \end{table}

\vspace{-8pt}
\noindent
\begin{minipage}[t]{0.5\textwidth}
    \raggedleft
    \vspace{25pt}
    \captionof{table}{Performance analysis for one-shot N:M sparse plus a low-rank matrix decomposition of the Llama3-8b model. The compression ratio is fixed to be $\rho=0.5$. For Perplexity, $(\downarrow)$ lower values are preferred. For zero-shot tasks, $(\uparrow)$ higher values are preferred.\\
    Bolded values correspond to a comparison between sparse plus low-rank decomposition algorithms. Underlined values correspond to the overall best comopression scheme given a compression ratio $\rho = 50\%$.}
    \label{tab:slr-fixed-compression}
\end{minipage}%
\hspace{5pt}
\begin{minipage}[t]{0.5\textwidth}
\centering
\begin{table}[H]
\centering
\resizebox{1.0\textwidth}{!}{%
\renewcommand{\arraystretch}{1.3}
\begin{tabular}{ccccccccc}
\toprule
\multirow{2.25}{*}{\textbf{Algorithm}} && \multicolumn{3}{c}{\textbf{Perplexity ($\downarrow$)}} && \multicolumn{3}{c}{\textbf{Zero-shot ($\uparrow$)}} \\ 
\cmidrule(rl){3-5} \cmidrule(rl){7-9}
&&\textbf{C4} & \textbf{WT2} & \textbf{PTB} && \textbf{PIQA} & \textbf{ARC-E} & \textbf{ARC-C}\\ 
\midrule
\texttt{SparseGPT-4:8}     && \underline{14.94} & 12.40 & 17.90 && 73.20 & \underline{68.54} & 34.86 \\ 
\texttt{Wanda-4:8}         && 18.88 & 14.52 & 24.26 && 71.52 & 64.91 & 34.03 \\ 
\texttt{DSNoT-4:8}         && 18.89 & 14.76 & 23.90 && 71.49 & 65.65 & 33.57 \\ 
\cmidrule(rl){1-1}
\texttt{SparseGPT-2:4}     && 18.89 & 16.35 & 25.08 && 70.54 & 63.09 & 31.84 \\ 
\texttt{Wanda-2:4}         && 30.81 & 24.36 & 44.89 && 67.56 & 56.20 & 26.11 \\ 
\texttt{DSNoT-2:4}         && 28.78 & 23.09 & 40.95 && 67.70 & 56.46 & 25.68 \\
\cmidrule(rl){1-1}
\texttt{OATS-2:8+LR}       && 21.03 & \textbf{14.54} & 24.15 && 73.67  & 59.68 & \textbf{37.12}\\
\texttt{Ours-2:8+LR}       && \textbf{20.05} & 15.03 & \textbf{22.01} && \textbf{74.05} & \textbf{60.52} & 36.18\\
\cmidrule(rl){1-1}
\texttt{OATS-3:8+LR}       && 16.87 & 11.43 & 18.53 && 75.24 & 65.91 & 39.85 \\
\texttt{Ours-3:8+LR}       && \textbf{16.16} & \underline{\textbf{11.36}} & \underline{\textbf{16.71}} && \underline{\textbf{75.79}} & \textbf{67.55} & \underline{\textbf{41.04}} \\
\cmidrule(rl){1-1}
\texttt{dense}               && 9.44 & 6.14 & 11.18 && 80.79 & 77.69 & 53.33  \\
\bottomrule
\end{tabular}
}
\end{table}
\end{minipage}



\begin{table}[h!]
\centering
\resizebox{1.0\textwidth}{!}{%
\renewcommand{\arraystretch}{1.1}
\begin{tabular}{ccc@{\hskip 8pt}cccc@{\hskip 8pt}ccccccccc}
\toprule
\multirow{2.25}{*}{\textbf{Model}} & \multirow{2.25}{*}{\textbf{Algorithm}} && \multicolumn{3}{c}{\textbf{Perplexity ($\downarrow$)}} && \multicolumn{9}{c}{\textbf{Zero-shot ($\uparrow$)}} \\ 
\cmidrule(rl){4-6} \cmidrule(r){8-16}
&&& \textbf{C4} & \textbf{WT2} & \textbf{PTB} && \textbf{PIQA} & \textbf{HS} & \textbf{ARC-E} & \textbf{ARC-C} & \textbf{WG} & \textbf{RTE} & \textbf{OQA} & \textbf{BoolQ} & \textbf{Avg}\\ 
\midrule
\multirow{9.5}{*}{Llama3-8B} 
&\texttt{OATS-2:8+64LR}       && 368.24 & 416.14 & 565.46 && 52.29 & 28.03 & 27.53 & \textbf{22.70} & 49.17 & \textbf{52.71} & 26.40 & 42.08 & 37.61 \\
&\texttt{Ours-2:8+64LR}       && \textbf{90.46} & \textbf{92.59} & \textbf{108.80} && \textbf{54.52} & \textbf{30.85} & \textbf{31.44} & 20.73 & \textbf{50.20} & \textbf{52.71} & \textbf{26.60} & \textbf{60.37} & \textbf{40.93} \\
\cmidrule(rl){2-2}
&\texttt{OATS-3:8+64LR}       && 48.21 & 35.65 & 56.52 && 65.23 & 42.05 & 47.01 & 25.94 & 58.01 & 52.71 & 27.40 & 67.89 & 48.28 \\
&\texttt{Ours-3:8+64LR}       && \textbf{28.88} & \textbf{21.48} & \textbf{32.54} && \textbf{68.99} & \textbf{52.19} & \textbf{50.55} & \textbf{29.86} & \textbf{62.90} & \textbf{53.07} & \textbf{29.80} & \textbf{72.84} & \textbf{52.53} \\
\cmidrule(rl){2-2}
&\texttt{OATS-4:8+64LR}       && 15.97 & 10.52 & 16.71 && 75.14 & 68.69 & 66.67 & 40.87 & 69.69 & \textbf{54.87} & 39.40 & \textbf{79.76} & 61.89 \\
&\texttt{Ours-4:8+64LR}       && \textbf{14.67} & \textbf{9.93} & \textbf{15.28} && \textbf{76.39} & \textbf{70.48} & \textbf{68.48} & \textbf{42.58} & \textbf{70.32} & 54.15 & \textbf{39.80} & 79.48 & \textbf{62.71} \\
\cmidrule(rl){2-2}
&\texttt{OATS-2:4+64LR}       && 21.05 & 14.42 & 22.62 && 72.85 & 62.47 & 60.69 & 36.35 & 67.09 & 54.87 & 35.00 & 75.11 & 58.05 \\
&\texttt{Ours-2:4+64LR}       && \textbf{18.06} & \textbf{12.66} & \textbf{18.66} && \textbf{74.86} & \textbf{64.77} & \textbf{63.85} & \textbf{37.37} & \textbf{69.22} & \textbf{56.68} & \textbf{36.40} & \textbf{76.12} & \textbf{59.91} \\
\cmidrule(rl){2-2}
&\texttt{dense}               && 9.44 & 6.14 & 11.18 && 80.79 & 79.17 & 77.69 & 53.33 & 72.85 & 69.68 & 45.00 & 81.44 & 69.99 \\

\midrule
\multirow{9.5}{*}{Llama3.2-1B} 
&\texttt{OATS-2:8+64LR}       && 740.37 & 825.40 & 754.22 && 52.12 & 27.46 & 28.37 & \textbf{23.72} & 48.86 & 52.71 & 24.60 & 37.77 & 36.95 \\
&\texttt{Ours-2:8+64LR}       && \textbf{167.87} & \textbf{133.01} & \textbf{162.73} && \textbf{54.30} & \textbf{28.73} & \textbf{30.35} & 21.93 & \textbf{50.51} & \textbf{53.43} & \textbf{25.20} & \textbf{51.68} & \textbf{39.52} \\
\cmidrule(rl){2-2}
&\texttt{OATS-3:8+64LR}       && 96.32 & 74.10 & 93.70 && 59.52 & 33.51 & 36.41 & 22.70 & 50.99 & \textbf{52.71} & 25.80 & 62.14 & 42.97 \\
&\texttt{Ours-3:8+64LR}       && \textbf{45.79} & \textbf{34.15} & \textbf{52.20} && \textbf{62.08} & \textbf{38.24} & \textbf{41.04} & \textbf{23.63} & \textbf{54.54} & \textbf{52.71} & \textbf{30.40} & \textbf{62.20} & \textbf{45.60} \\
\cmidrule(rl){2-2}
&\texttt{OATS-4:8+64LR}       && 26.75 & 18.49 & 31.94 && 67.30 & 49.52 & 50.51 & 28.41 & 56.67 & \textbf{55.96} & \textbf{32.40} & \textbf{62.87} & 50.46 \\
&\texttt{Ours-4:8+64LR}       && \textbf{22.71} & \textbf{16.05} & \textbf{26.80} && \textbf{68.28} & \textbf{51.42} & \textbf{51.22} & \textbf{29.18} & \textbf{58.64} & 53.07 & 30.00 & 62.51 & \textbf{50.54} \\
\cmidrule(rl){2-2}
&\texttt{OATS-2:4+64LR}       && 36.89 & 26.26 & 42.35 && 64.36 & 43.35 & \textbf{47.77} & 26.45 & 55.80 & \textbf{52.71} & 30.40 & \textbf{62.66} & 47.94 \\
&\texttt{Ours-2:4+64LR}       && \textbf{27.09} & \textbf{19.57} & \textbf{31.73} && \textbf{67.03} & \textbf{47.53} & 47.43 & \textbf{28.16} & \textbf{58.64} & \textbf{52.71} & \textbf{30.60} & 62.60 & \textbf{49.34} \\
\cmidrule(rl){2-2}
&\texttt{dense}               && 14.01 & 9.75 & 17.59 && 74.59 & 63.66 & 60.48 & 36.26 & 60.69 & 56.68 & 37.20 & 63.98 & 56.69 \\

\midrule
\multirow{9.5}{*}{Llama3.2-3B} 
&\texttt{OATS-2:8+64LR}       && 444.37 & 543.53 & 851.16 && 52.56 & 27.54 & 27.99 & \textbf{23.46} & \textbf{50.43} & 51.99 & \textbf{26.60} & 37.86 & 37.30 \\
&\texttt{Ours-2:8+64LR}       && \textbf{122.14} & \textbf{114.74} & \textbf{165.78} && \textbf{54.57} & \textbf{28.93} & \textbf{30.09} & 21.08 & 49.49 & \textbf{52.71} & 26.20 & \textbf{62.14} & \textbf{40.65} \\
\cmidrule(rl){2-2}
&\texttt{OATS-3:8+64LR}       && 56.80 & 41.62 & 72.75 && 62.68 & 40.49 & 41.84 & 24.06 & 53.91 & 52.35 & 26.60 & 64.10 & 45.75 \\
&\texttt{Ours-3:8+64LR}       && \textbf{35.07} & \textbf{27.12} & \textbf{39.63} && \textbf{66.43} & \textbf{46.08} & \textbf{46.42} & \textbf{26.62} & \textbf{58.17} & \textbf{55.96} & \textbf{29.00} & \textbf{65.47} & \textbf{49.27} \\
\cmidrule(rl){2-2}
&\texttt{OATS-4:8+64LR}       && 18.52 & 12.85 & 20.69 && 72.85 & 61.68 & 62.42 & 36.01 & 64.17 & \textbf{60.29} & 36.40 & \textbf{72.75} & 58.32 \\
&\texttt{Ours-4:8+64LR}       && \textbf{17.19} & \textbf{12.15} & \textbf{19.24} && \textbf{73.99} & \textbf{63.59} & \textbf{62.92} & \textbf{36.26} & \textbf{67.48} & 57.76 & \textbf{39.20} & 71.90 & \textbf{59.14} \\
\cmidrule(rl){2-2}
&\texttt{OATS-2:4+64LR}       && 24.32 & 17.06 & 28.54 && \textbf{71.98} & 55.87 & 58.80 & 33.36 & 59.91 & 53.07 & \textbf{33.80} & \textbf{70.18} & 54.62 \\
&\texttt{Ours-2:4+64LR}       && \textbf{20.82} & \textbf{15.65} & \textbf{23.77} && 71.71 & \textbf{57.88} & \textbf{58.84} & \textbf{34.39} & \textbf{62.12} & \textbf{58.12} & 33.60 & 67.92 & \textbf{55.57} \\
\cmidrule(rl){2-2}
&\texttt{dense}               && 11.33 & 7.81 & 13.53 && 77.48 & 73.61 & 71.63 & 45.99 & 69.85 & 54.51 & 43.00 & 73.39 & 63.68 \\

\bottomrule
\end{tabular}
}
\vspace{3pt}
\caption{Performance analysis for one-shot N:M sparse plus a 64-rank low-rank matrix decomposition of Llama3 and Llama3.2 models. The rank of the low-rank component is fixed to be $r=64$. For Perplexity, $(\downarrow)$ lower values are preferred. For zero-shot tasks, $(\uparrow)$ higher values are preferred.}
\label{tab:slr-fixed-rank}
\end{table}
\subsection{Reconstruction error on a single Transformer block}
In order to show the performance of OATS and \ourframework on the layer-wise reconstruction objective \eqref{eq:matrix-decomposition}, we compute the error produced with the two algorithms (both after $80$ iterations--default value used in OATS \cite{zhang2024oats}), given by $\|\bfX \bfWold-\bfX \pr{\bfWS + \bfM}\|_F^2$, when applied to the model Llama-3-8B \cite{dubey2024llama}, and using the decomposition $\calCS$ corresponding to $2:4$ sparsity and a fixed rank $r = 64$. Results of the local layer-wisre error are reported in \cref{fig:reconstruction-error} for OATS, \ourframework scaled and \ourframework unscaled.

\vspace{-20pt}
\begin{minipage}[t]{0.49\textwidth}
    \raggedleft
    \vspace{30pt}
    \captionof{figure}{Local layer-wise reconstruction error $\downarrow$ (lower values are preferred) analysis of the decomposition of the layers of the \textbf{first} transformer block in Llama-3-8B into a 2:4 sparse component plus a 64-rank low-rank component. All methods use the same number of Alternating-Minimization steps $80$.}
    \label{fig:reconstruction-error}
\end{minipage}
\begin{minipage}[t]{0.49\textwidth}
    \vspace{0pt}
    \centering
    \includegraphics[width=\textwidth]{layerwise_error.pdf}
\end{minipage}
\vspace{-\baselineskip}

\vspace{-15pt}
\section{Conclusion}
\label{sec:conclusion}
\vspace{-5pt}
\section{Conclusion and Discussion}
\label{sec:conclusion}

We introduce a novel framework for augmenting \emph{any} lossy compressor to preserve the contour tree of a volumetric dataset while maintaining a user-specified global error bound. 
To do this, our framework first imposes topology-informed upper and lower bounds on each data point. 
It then progressively tightens those bounds until the contour tree is preserved. 
We also introduce a novel encoding scheme that efficiently stores individual points with variable precision and maintains these upper and lower bounds. 
When our framework is used to augment state-of-the-art lossy compressors, it is shown to preserve the contour trees of various scientific datasets.
Our augmented compressors also achieve higher compression ratios and reconstruction quality than those obtained by existing topology-preserving compressors in comparable or faster time.
Our framework will benefit from any advancement with lossy compression since it can be used to augment increasingly effective lossy compressors to achieve better topology-preserving compression. 

Our framework is not without limitations. The compression times are longer than the base compressors. This difference gets worse as the topological complexity of the data increases.
However, in some use-cases, topological preservation is preferable to run time.
Regardless, our framework would benefit from more efficient or parallel implementations for the contour/merge tree computation and the encoding scheme. 


\section*{Acknowledgements}
This research is supported in part by grants from Google and the Office of Naval Research. We acknowledge the MIT SuperCloud~\cite{reuther2018interactive} for providing HPC resources that have contributed to the research results reported within this paper. We also acknowledge Google for providing us with Google Cloud Credits for computing. 

\clearpage


% Reference
% For natbib users:
\bibliography{reference}
\clearpage


%%%%%%%%%%%%%%%%%%%%%%%%%%%%%%%%%%%%%%%%%%%%%%%%%%%%%%%%%%%%
\appendix

\section{Experimental Details}
\label{supp:experiments-supp}
% !TEX root = ../main.tex

% \begin{proof}
%   Given $\bfX \in \R^{m \times p}, \bfW \in \R^{p \times n}$, and the compact singular value decomposition of $\bfX$ as $\bfX = \bfU_{m \times r} \bfSigma_{r \times r} \bfV_{p \times r}^\top$. If $r = p < m$, then
%   \begin{equation*}
%     \pr{\bfXTX}^{-1}\bfX^\top C_r\pr{\bfX \bfW} = \argmin\nolimits_{\bfM \in \R^{p \times n}} \,\, \left\|\bfX \bfW -\bfX \bfM\right\|_F^2 ~~~~\text{s.t. } ~~~\rk{\bfM} \leq r.
%   \end{equation*}
%   If we introduce $\bfY = \bfX \bfW$,  \citet{mazumder2020computing} have shown that under the assumption that $r = p < m$, the solution to the above problem is \textbf{unique} and given by
%   \begin{align*}
%     \bfM_r(\bfY) 
%     &= \pr{\bfXTX}^{-1}\bfX^\top C_r\pr{\bfU\bfU^\top\bfY}\\
%     &= \pr{\bfXTX}^{-1}\bfX^\top C_r\pr{\bfU\bfU^\top\bfX \bfW}\\
%     &= \pr{\bfXTX}^{-1}\bfX^\top C_r\pr{\bfU\bfU^\top\bfU \bfSigma \bfV^\top \bfW}\\
%     &= \pr{\bfXTX}^{-1}\bfX^\top C_r\pr{\bfU \bfSigma \bfV^\top \bfW} \tag{$\bfU$ is orthogonal}\\
%     &= \pr{\bfXTX}^{-1}\bfX^\top C_r\pr{\bfX \bfW}.\\
%   \end{align*}
% \end{proof}


\subsection{Experimental Setup}
Following the framework proposed by \citet{frantar2023sparsegpt} for one-shot pruning, we minimize \cref{eq:matrix-decomposition} sequentially, layer by layer. For a given layer $\ell$, the input activation matrix $\bfX$ introduced in \cref{sec:optimization-formulation} is the output of the previous $\ell - 1$ compressed layers (sparse plus low-rank) using $N$ calibration samples. 

\textbf{Implementation details.}
\begin{itemize}
    \item For the construction of the Hessian matrix $\bfH = \bfXTX$ introduced in \cref{sec:algorithm-design}, we use the same setup of SparseGPT \cite{frantar2023sparsegpt} and we use the author's implementation of SparseGPT---as a pruning plug-in method to minimize \Pone (codes available on GitHub).
    \item We utilize the author's implementation of OATS \cite{zhang2024oats} with the default hyperparameter settings to show LLM evaluation benchmarks and layer-wise reconstruction error in \cref{fig:reconstruction-error}.
    \item The LLM evaluation benchmarks reported in \cref{tab:slr-fixed-compression} are retrieved from the paper ALPS by \citet{meng2024alps} which uses the same evaluation strategy (and code) we do for the reported tasks [other zero-shot tasks are not reported in ALPS]. We report all zero-shot tasks results for OATS and \ourframework in \cref{tab:supp-slr-fixed-compression}.
\end{itemize}



\subsection{Hyperparameter Choice}
The hyperparameters used in \ourframework are the following: $\lambda = 0.01 \Tr{\bfH}$; default value in SparseGPT. $T_\text{AM}$ is set to be 80; default value in OATS. $T_\text{LR} = 50$; we propose this default value for all experiments. $\eta = 1e^{-2}$; we propose this default value for all experiments (only works well with the scaling introduced in \cref{scaling-low-rank}). $r$ is either set to $64$ and fixed for all layers, or is flexible and given by the formula $r = \left\lfloor {(1 - \rho - \frac{N}{M}) \cdot(\Nout \cdot \Nin)} / \pr{\Nout + \Nin} \right\rfloor$ introduced in \cref{sec:experimental-results}. \texttt{Prune}; we propose by default to use SparseGPT. \texttt{Optimizer}; we propose the Adam optimizer. \textbf{is\_scaled}; we propose to set this to True by default. It converges faster in practice and allows to skip the tuning of the learning rate $\eta$.

\subsection{Additional Experimental Resuls}
\textbf{N:M Sparsity + Fixed Compression Ratio:}
This is the same setting described in \cref{sec:experimental-results}. We extend the results reported in \cref{tab:slr-fixed-compression} to include the 8 zero-shot tasks and the Llama3.2 model. Results are reported in \cref{tab:supp-slr-fixed-compression}.
\begin{table}[h!]
\centering
\resizebox{1.0\textwidth}{!}{%
\renewcommand{\arraystretch}{1.3}
\begin{tabular}{cccccccccccccc}
\toprule
\multirow{2.25}{*}{\textbf{Model}} & \multirow{2.25}{*}{\textbf{Algorithm}} & \multicolumn{3}{c}{\textbf{Perplexity ($\downarrow$)}} & \multicolumn{9}{c}{\textbf{Zero-shot ($\uparrow$)}} \\ 
\cmidrule(rl){3-5} \cmidrule(rl){6-14}
& & \textbf{C4} & \textbf{WT2} & \textbf{PTB} & \textbf{PIQA} & \textbf{HS} & \textbf{ARC-E} & \textbf{ARC-C} & \textbf{WG} & \textbf{RTE} & \textbf{OQA} & \textbf{BoolQ} & \textbf{Avg}\\ 
\midrule
\multirow{5.5}{*}{Llama3-8B} 
&\texttt{OATS-2:8+LR}       & 21.03 & \textbf{14.54} & 24.15 & 73.67 & \textbf{62.42} & 59.68 & \textbf{37.12} & 65.43 & 55.23 & \textbf{36.40} & 73.98 & 57.99 \\
&\texttt{Ours-2:8+LR}       & \textbf{20.05} & 15.03 & \textbf{22.01} & \textbf{74.05} & 60.69 & \textbf{60.52} & 36.18 & \textbf{66.77} & \textbf{57.04} & 35.00 & \textbf{76.02} & \textbf{58.28} \\
\cmidrule(rl){2-2}
&\texttt{OATS-3:8+LR}       & 16.87 & 11.43 & 18.53 & 75.24 & 66.90 & 65.91 & 39.85 & 68.90 & 61.37 & 39.00 & 76.61 & 61.72 \\
&\texttt{Ours-3:8+LR}       & \textbf{16.16} & \textbf{11.36} & \textbf{16.71} & \textbf{75.79} & \textbf{67.33} & \textbf{67.55} & \textbf{41.04} & \textbf{69.53} & \textbf{58.48} & \textbf{39.20} & \textbf{79.91} & \textbf{62.35} \\
\cmidrule(rl){2-2}
&\texttt{dense}               & 9.44 & 6.14 & 11.18 & 80.79 & 79.17 & 77.69 & 53.33 & 72.85 & 69.68 & 45.00 & 81.44 & 69.99 \\

\midrule
\multirow{5.5}{*}{Llama3.2-1B} 
&\texttt{OATS-2:8+LR}       & 78.18 & 53.05 & 80.17 & 59.03 & 36.42 & 37.08 & 22.87 & 52.80 & 52.71 & 27.40 & 61.77 & 43.76 \\
&\texttt{Ours-2:8+LR}       & \textbf{41.08} & \textbf{30.92} & \textbf{48.85} & \textbf{63.22} & \textbf{39.07} & \textbf{42.55} & \textbf{25.77} & \textbf{55.17} & \textbf{53.07} & \textbf{28.00} & \textbf{62.11} & \textbf{46.12} \\
\cmidrule(rl){2-2}
&\texttt{OATS-3:8+LR}       & 42.81 & 29.35 & 47.58 & 63.49 & 42.25 & 43.43 & 25.09 & 54.85 & 52.35 & \textbf{29.60} & 62.05 & 46.64 \\
&\texttt{Ours-3:8+LR}       & \textbf{31.35} & \textbf{22.89} & \textbf{34.99} & \textbf{66.43} & \textbf{45.00} & \textbf{46.42} & \textbf{25.85} & \textbf{56.43} & \textbf{52.71} & 28.80 & \textbf{62.26} & \textbf{47.99} \\
\cmidrule(rl){2-2}
&\texttt{dense}               & 14.01 & 9.75 & 17.59 & 74.59 & 63.66 & 60.48 & 36.26 & 60.69 & 56.68 & 37.20 & 63.98 & 56.69 \\

\midrule
\multirow{5.5}{*}{Llama3.2-3B} 
&\texttt{OATS-2:8+LR}       & 30.73 & 22.65 & 36.31 & 68.55 & 51.76 & 54.46 & \textbf{31.14} & 61.17 & \textbf{58.48} & \textbf{30.80} & \textbf{70.43} & \textbf{53.35} \\
&\texttt{Ours-2:8+LR}       & \textbf{25.22} & \textbf{19.61} & \textbf{29.54} & \textbf{69.59} & \textbf{52.94} & \textbf{55.30} & 29.69 & \textbf{62.67} & 55.23 & 30.60 & 69.24 & 53.16\\
\cmidrule(rl){2-2}
&\texttt{OATS-3:8+LR}       & 21.96 & 15.84 & 26.22 & \textbf{72.69} & 58.61 & \textbf{58.92} & \textbf{34.13} & 63.14 & \textbf{58.12} & 33.60 & 67.22 & 55.80 \\
&\texttt{Ours-3:8+LR}       & \textbf{20.03} & \textbf{14.85} & \textbf{22.92} & 72.42 & \textbf{58.92} & 56.69 & 33.53 & \textbf{64.01} & 56.32 & \textbf{37.00} & \textbf{70.31} & \textbf{56.15} \\
\cmidrule(rl){2-2}
&\texttt{dense}               & 11.33 & 7.81 & 13.53 & 77.48 & 73.61 & 71.63 & 45.99 & 69.85 & 54.51 & 43.00 & 73.39 & 63.68 \\


\bottomrule
\end{tabular}
}
\vspace{3pt}
\caption{Performance analysis for one-shot N:M sparse plus a low-rank matrix decomposition of Llama3 and Llama3.2 models. The compression ratio is fixed to be $\rho=0.5$. For Perplexity, $(\downarrow)$ lower values are preferred. For zero-shot tasks, $(\uparrow)$ higher values are preferred.}
\label{tab:supp-slr-fixed-compression}
\end{table}

\textbf{Unstructured Sparsity + Fixed Rank Ratio:} This is the setting introduced in OATS \cite{zhang2024oats}. This scheme takes as inputs a compression ratio $\rho$ (e.g. $50\%$) and rank ratio $\kappa$ (e.g. $0.3$; default value in OATS for the Llama3-8B model). The rank of the low-rank component $r$ and the number of non-zeros $k$ in the unstructured sparsity are given by.
\begin{equation*}
    r = \left\lfloor \kappa \cdot (1 - \rho) \cdot \frac{\Nout \cdot \Nin}{\Nout + \Nin} \right\rfloor, \quad \quad k = \left\lfloor (1 - \kappa) \cdot (1 - \rho) \cdot \Nout \cdot \Nin \right\rfloor.
\end{equation*}
See OATS for a discussion on how to choose the rank ratio $\kappa$ for a given model. Note that OATS introduces OWL ratios--different sparsity budgets for different layers to reduce the utility drop. The results for this setting do not apply OWL and consider uniform unstructured sparsity throughout layers. Results for OATS and \ourframework are reported in \cref{tab:supp-slr-unstructured}.




\begin{table}[h!]
\centering
\resizebox{1.0\textwidth}{!}{%
\renewcommand{\arraystretch}{1.3}
\begin{tabular}{ccc@{\hskip 8pt}cccc@{\hskip 8pt}ccccccccc}
\toprule
\multirow{2.25}{*}{\textbf{Model}} & \multirow{2.25}{*}{\textbf{Algorithm}} && \multicolumn{3}{c}{\textbf{Perplexity ($\downarrow$)}} && \multicolumn{9}{c}{\textbf{Zero-shot ($\uparrow$)}} \\ 
\cmidrule(rl){4-6} \cmidrule(r){8-16}
&&& \textbf{C4} & \textbf{WT2} & \textbf{PTB} && \textbf{PIQA} & \textbf{HS} & \textbf{ARC-E} & \textbf{ARC-C} & \textbf{WG} & \textbf{RTE} & \textbf{OQA} & \textbf{BoolQ} & \textbf{Avg}\\ 

\midrule
\multirow{7.5}{*}{Llama3-8B}
&\texttt{OATS-60\%+LR}       && 23.61 & 16.52 & 25.85 && 72.91 & 59.65 & \textbf{60.10} & 33.36 & 65.35 & 53.07 & 31.60 & \textbf{75.96} & 56.50 \\
&\texttt{Ours-60\%+LR}       &&  \textbf{20.70} & \textbf{15.66} & \textbf{23.31} && \textbf{73.29} & \textbf{60.58} & 59.26 & \textbf{34.64} & \textbf{67.88} & \textbf{53.43} & \textbf{35.40} & 75.08 & \textbf{57.44} \\
\cmidrule(rl){2-2}
&\texttt{OATS-70\%+LR}       &&  106.98 & 81.77 & 110.44 && 55.60 & 30.30 & 32.45 & 20.05 & 49.96 & \textbf{52.71} & 27.00 & 62.35 & 41.30 \\
&\texttt{Ours-70\%+LR}       && \textbf{50.07} & \textbf{49.13} & \textbf{60.89} && \textbf{60.50} & \textbf{39.67} & \textbf{37.21} & \textbf{23.38} & \textbf{55.25} & \textbf{52.71} & \textbf{27.40} & \textbf{66.09} & \textbf{45.27} \\

\cmidrule(rl){2-2}
&\texttt{OATS-80\%+LR}       && 748.40 & 909.75 & 1601.02 && 52.29 & 27.25 & 26.81 & \textbf{24.40} & 47.59 & \textbf{52.71} & \textbf{26.60} & 37.83 & 36.93 \\
&\texttt{Ours-80\%+LR}       && \textbf{164.27} & \textbf{265.28} & \textbf{235.38} && \textbf{53.32} & \textbf{28.53} & \textbf{29.38} & 20.22 & \textbf{49.49} & \textbf{52.71} & \textbf{26.60} & \textbf{38.84} & \textbf{37.39} \\
\cmidrule(rl){2-2}
&\texttt{dense}               && 11.33 & 7.81 & 13.53 && 77.48 & 73.61 & 71.63 & 45.99 & 69.85 & 54.51 & 43.00 & 73.39 & 63.68 \\


\midrule
\multirow{7.5}{*}{Llama3.2-3B} 
&\texttt{OATS-60\%+LR}       && 34.57 & 24.94 & 41.51 && 67.79 & 48.40 & \textbf{52.57} & \textbf{30.38} & 57.70 & 54.15 & \textbf{30.80} & 65.66 & 50.93 \\
&\texttt{Ours-60\%+LR}       && \textbf{27.67} & \textbf{21.90} & \textbf{33.40} && \textbf{69.15} & \textbf{52.04} & 51.26 & 29.52 & \textbf{61.96} & \textbf{58.12} & 29.80 & \textbf{69.72} & \textbf{52.70} \\
\cmidrule(rl){2-2}
&\texttt{OATS-70\%+LR}       &&  155.48 & 121.76 & 167.60 && 54.57 & 29.83 & 30.43 & 21.42 & \textbf{49.64} & \textbf{52.71} & \textbf{28.20} & 60.43 & 40.90 \\
&\texttt{Ours-70\%+LR}       && \textbf{78.65} & \textbf{75.23} & \textbf{103.10} && \textbf{58.43} & \textbf{32.44} & \textbf{35.27} & \textbf{21.67} & 49.41 & \textbf{52.71} & 27.00 & \textbf{62.29} & \textbf{42.40} \\
\cmidrule(rl){2-2}
&\texttt{OATS-80\%+LR}       && 1085.27 & 1610.87 & 2546.29 && 50.60 & 26.60 & 26.68 & \textbf{24.40} & 47.67 & \textbf{52.71} & \textbf{26.60} & 37.83 & 36.64 \\
&\texttt{Ours-80\%+LR}       && \textbf{217.62} & \textbf{320.98} & \textbf{320.02} && \textbf{53.10} & \textbf{27.86} & \textbf{29.12} & 22.01 & \textbf{47.75} & 50.54 & \textbf{26.60} & \textbf{46.61} & \textbf{37.95} \\
\cmidrule(rl){2-2}
&\texttt{dense}               && 11.33 & 7.81 & 13.53 && 77.48 & 73.61 & 71.63 & 45.99 & 69.85 & 54.51 & 43.00 & 73.39 & 63.68 \\

\bottomrule
\end{tabular}
}
\vspace{2pt}
\caption{Performance analysis for one-shot unstructured sparsity plus a low-rank matrix decomposition of Llama3 and Llama3.2-3B model. The rank ratio of the low-rank component is fixed to be $\kappa=0.3$. For Perplexity, $(\downarrow)$ lower values are preferred. For zero-shot tasks, $(\uparrow)$ higher values are preferred.}
\label{tab:supp-slr-unstructured}
\end{table}


\end{document}
