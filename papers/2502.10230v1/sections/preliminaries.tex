\section{Preliminary Concepts}\label{sec:prel}
\vspace{-0.5em}
In this section, we introduce the necessary concepts before presenting \texttt{ProReco} in Sec.~\ref{sec:ProReco}. 

\subsubsection{Event log}
The starting point of process mining is the event log where each event refers to a case (an instance of the process), an activity, and a point in time. 
The existence of these three attributes is the minimal requirement for an event log, whereas more attributes can be recorded and/or extracted. 
Event data can be extracted from various sources such as a database, a transaction log, a business suite/ERP system, etc. 
An event log can be seen as a collection of cases, whereas a case is a trace/sequence of events. 
Fig.~\ref{fig:preli-log} shows a synthetic event log for the purchasing process of an online retail site. 
Each row corresponds to an event. 
% An event is associated with a case (an instance of the process), an activity name, and a timestamp. 
% It's crucial to associate each event with its corresponding case so that process mining tools can compare multiple process executions effectively. 
% Activity names refer to different process steps or status changes in the process. 
% Lastly, the timestamp column indicates when each of the events took place. 

\begin{figure}[h!]
    % \vspace{-1.5em}
    \centering
        \begin{subfigure}[b]{.33\linewidth}
            \includegraphics[width=\linewidth]{figs/preli-log.pdf}
            \caption{An example event log}\label{fig:preli-log}
        \end{subfigure}
        % \hspace{0.5em} % for more space between subfigures
        \begin{subfigure}[b]{.65\linewidth}
            \includegraphics[width=\linewidth]{figs/preli-model.pdf}
            \caption{A process model represented using Petri net}\label{fig:preli-model}
        \end{subfigure}
    \caption{An example of an event log and the corresponding process model.}
    \label{fig:preli-log-model}
    \vspace{-1.5em}
\end{figure}

\subsubsection{Process model}

A process model is a structured representation of the activities and their relationships within a business process. 
It plays a crucial role in understanding, analyzing, and improving organizational workflows. 
Various process modeling notations exist such as Petri nets, BPMNs, BPEL models, or UML Activity Diagrams~\cite{Aalst16PMbook}.  
% Many fundamental discovery approaches use Petri nets as their representation of choice while other approaches leverage transition systems, process trees, and BPMN diagrams as the output representations. 
In \texttt{ProReco}, we focus on Petri net since it is one of the simplest formalisms that explicitly model concurrency. 
Moreover, it is trivial to convert process models in other notations into Petri nets.  

% A Petri net is a bipartite model consisting of places (circles), transitions (squares), and directed arcs. 
% Places can hold tokens (black dots) while transitions produce and/or consume tokens. 
% A transition is enabled if each input place contains a token. 
% An enabled transition can fire by consuming one token from each input place and producing a token for each output place. 
% By such a simple mechanism, Petri nets are used to represent and analyze the behavior of complex concurrent systems. 
Fig.~\ref{fig:preli-model} shows the corresponding process model (in the form of a Petri net) for the event log in Fig.~\ref{fig:preli-log}. 
The process starts with the activity \textit{``place order''} followed by the concurrent executions of activity \textit{``pay''} and \textit{``send invoice''}, where activity \textit{``pay''} is optional. 
Then, the process might be either \textit{``cancel''} or followed by a delivery procedure. 

\subsubsection{Process discovery}
Process discovery aims at constructing process models to describe the observed behaviors of information systems from event logs. 
In general, the problem of process discovery can be defined as follows: 
A process discovery algorithm is a function that maps an event log $L$ onto a process model $N$ such that the model $N$ is representative of the behaviors seen in the log $L$. 
Despite the development of process discovery algorithms, manually finding the most appropriate algorithm is a challenging and error-prone task. 
To assist users with identifying the most prominent discovery algorithm, we present \texttt{ProReco} in the next section. 


% \begin{table}[]
% \centering
% \caption{}
% \label{tab:event-log}
% \begin{tabular}{|c|c|c|}
% \hline
% Case ID & Activity         & Timestamp       \\ \hline
% 4268    & place order      & 2/13/2023 14:29 \\ \hline
% 1968    & place order      & 2/13/2023 16:17 \\ \hline
% 7426    & place order      & 2/13/2023 17:53 \\ \hline
% 7426    & send invoice     & 2/19/2023 9:20  \\ \hline
% 1968    & send invoice     & 2/19/2023 16:08 \\ \hline
% 4268    & send invoice     & 2/21/2023 9:38  \\ \hline
% 4268    & pay              & 3/2/2023 12:39  \\ \hline
% 7426    & pay              & 3/5/2023 15:46  \\ \hline
% 1968    & cancel order     & 3/6/2023 10:17  \\ \hline
% 4268    & prepare delivery & 3/7/2023 13:50  \\ \hline
% 4268    & make delivery    & 3/7/2023 16:41  \\ \hline
% 4268    & confirm payment  & 3/7/2023 16:53  \\ \hline
% 7426    & prepare delivery & 3/7/2023 17:05  \\ \hline
% 7426    & confirm payment  & 3/7/2023 17:59  \\ \hline
% 7426    & make delivery    & 3/8/2023 9:54   \\ \hline
% \end{tabular}
% \end{table}