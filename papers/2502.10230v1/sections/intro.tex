\section{Introduction}\label{sec:inro}

Process discovery~\cite{Aalst16PMbook} is a discipline that aims to automatically obtain formal representation through models of the operating mechanisms in a process. The input of such methods is a collection of data related to the historical execution of a process, often in the form of discrete \emph{events}. 
Discovery algorithms read events and their \emph{attributes} from a dataset (often called an \emph{event log}), and output a process model, to provide a representation as close as possible to the real process operations.

Since the inception of the discipline in the early 2000s, many discovery algorithms have been proposed~\cite{DBLP:books/sp/22/Augusto0022}, as well as numerous metrics to assess their desirability and quality. 
Nevertheless, the systematic review and benchmark~\cite{AugustoCDRMMMS19PDreviewbenchmark} show no algorithm dominating all other methods in terms of model quality measures. 
% Many discovery approaches use Petri nets as their representation of choice, since it is one of the simplest formalisms that explicitly model concurrency. 
% Other approaches leverage transition systems, process trees, and BPMN diagrams as their output representations. 
Moreover, producing a satisfactory process model is still an open challenge, although there exists extensive literature dedicated to measuring the quality of models obtained through discovery. 
This is because (i) some of the most widely adopted quality measures are competing (i.e., there exist trade-offs between them), and (ii) depending on the final use of the discovered model, different (and sometimes opposite) characteristics are desirable. 
Under such a circumstance, users are left with the task of manually selecting the most prominent process discovery algorithm for the event log at hand. 
The procedure is time-consuming and error-prone even for process mining experts, let alone inexperienced users. 

To address the aforementioned problems and to assist process mining users, previous works~\cite{RibeiroCMS14Recommend,TavaresJD22MetaRecommend} proposed using recommender systems for process discovery. 
% However, the algorithm portfolio and feature extraction are not up-to-date with the recent developments in process discovery technology. 
The approaches~\cite{RibeiroCMS14Recommend,TavaresJD22MetaRecommend} abstract from the actual values of model quality by calculating the final score based on the rankings. 
Also, the approach in~\cite{TavaresJD22MetaRecommend} does not incorporate user preferences for the recommendation, assuming every user wants to maximize all measures simultaneously. 
Lastly, the recommendations offered by both works lack accompanying explanations. 
Intransparent recommendations could hamper the acceptance of a recommender system~\cite{ZhangC20XRecommendation}. 

This paper proposes \texttt{ProReco}, a \textbf{Pro}cess discovery \textbf{Reco}mmender system. 
Given an event log and user preferences regarding model quality measures, \texttt{ProReco} recommends the most appropriate process discovery algorithm tailored to the users' needs. 
Internally, \texttt{ProReco} utilizes machine learning models to predict the values for each quality measure before computing the weighted (user preferences) sum of the final score. 
The scores are then used to rank and recommend the discovery algorithm. 
\texttt{ProReco} not only expands the features pool from previous work but also includes state-of-the-art process discovery algorithms.  
Last but not least, for every recommendation made by \texttt{ProReco}, explanations are available to the user thanks to the incorporation of the eXplainable AI (XAI) technique~\cite{LundbergL17SHAP} in \texttt{ProReco}. 

The remainder of the paper is structured as follows. Section~\ref{sec:prel} illustrates some preliminary notions. Section~\ref{sec:ProReco} describes the components and mechanics of \texttt{ProReco}. Lastly, Section~\ref{sec:conclusion} concludes the paper and indicates directions for future research. 