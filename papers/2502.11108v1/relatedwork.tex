\section{Related Work}
KRAGEN~\cite{kragen_2024} is a framework that combines knowledge graphs with retrieval-augmented generation to address intricate issues in the biomedical field. The study highlights advanced prompting methods, including graph-of-thoughts, to systematically break down tasks and reduce hallucinations in large language model outputs.

Polat et al.~\cite{pub.1182771521} have examined different prompt engineering techniques to extract knowledge. The findings indicate that straightforward instructions coupled with task demonstrations significantly boost extraction performance in various large language models, particularly when examples are chosen using retrieval methods.

Muntean et al.~\cite{diagnostics14141468} investigated the performance of LLMs in a specific ophthalmological domain, that is, age-related macular degeneration. The study reveals that ChatGPT4 and PaLM2 are valuable instruments for patient information and education based on the evaluation methodology proposed by Singhal et al. \cite{singhal2023large}. However, since there are still some limitations to these models, a fine-tuned model tailored for age-related macular degeneration has been proposed. Nevertheless, this approach can be adapted to other fields by following the same steps.

Additional research further supports the integration of structured knowledge with generative models. Lewis et al.\cite{Lewis_NEURIPS2020} introduced the Retrieval-Augmented Generation framework, demonstrating that grounding LLM outputs in external data significantly enhances factual accuracy. Wei et al.\cite{wei2022chain} showed that chain-of-thought prompting can guide LLMs through multistep reasoning processes, a capability essential for complex biomedical queries. Yang et al.~\cite{yang2024kgllm} further emphasized that merging knowledge graphs with LLMs leads to more reliable and interpretable outcomes.