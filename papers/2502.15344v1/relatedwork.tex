\section{Related Work}
\label{related_work}

%We discuss prior works, including temporary property analysis, termination analysis, encoding temporal properties using different kinds of Datalog semantics, and techniques for model repair. 


%-based program analysis for temporal, 

\vspace{-2mm}
\noindent\paragraph{\textbf{Program analyses for Temporal properties}} 

Most existing approaches for proving
program properties expressed in CTL have limitations upon the type of the infinite-state programs, such as pushdown systems \cite{DBLP:journals/iandc/EsparzaKS03}, parameterized
systems \cite{DBLP:conf/cav/EmersonN96}, timed systems \cite{alur1994theory}. 
Other approaches for proving CTL properties do not reliably support CTL formulas with arbitrary nesting of universal and existential path quantifiers \cite{DBLP:conf/cav/GurfinkelWC06,DBLP:conf/cav/CookKV11}, 
or support existential path quantifiers only indirectly by building upon the prior works 
for proving non-termination \cite{DBLP:conf/popl/GuptaHMRX08}, or by considering their universal dual (\terminator) \cite{DBLP:conf/fmcad/CookKP14}. In particular, the latter approach is problematic: since the universal dual of an existential ``until'' formula is non-trivial to define, \citet{DBLP:conf/sas/UrbanU018} demonstrated that the current implementation of \terminator does not support such formulas. 

\function~\cite{DBLP:conf/sas/UrbanU018} presents a CTL properties analyser via abstract interpretation. It deploys a backward analysis to propagate the weakest preconditions, which make the program satisfy the property. While being the first work to deal with a full class of CTL properties, it has several sources of the loss of precision, such as the \emph{dual widening} \cite{DBLP:conf/tacas/CourantU17} technique for proving the termination of loops. On top of that, to achieve a sound analysis, it alternatively applies the over/under approximation operators for dealing with existential/universal quantifiers, which also cause imprecision when mixed abstract domains occur. 
Notably, the experimental result of \function shows that none of the temporal analysis tools (among T2, \ultimate, and \function) subsumes the others. 
This observation highlights the collaborative power of their combination, which proves to be more potent than any of the tools used in isolation. 



Apart from the temporal property analysis with CTL, an earlier approach  \cite{DBLP:conf/popl/CookGPRV07} reduces the LTL model checking problem to fair termination checking. For the same purpose, \ultimate \cite{DBLP:conf/cav/DietschHLP15} presents a new approach exploiting the fact that constructing a proof of unsatisfiability is less costly than synthesizing ranking functions. 
Specifically, it selects finite prefixes of a path and checks their infeasibility before considering the full infinite path, verifying the liveness property with finite prefixes without the construction ranking functions. 
Compared to \cite{DBLP:conf/popl/CookGPRV07}, \ultimateshort can be seen as an improvement as it only uses fair termination checking when necessary, which avoids many (more costly) termination checks. 
Following the spirit, \toolName~ decouples the termination analysis from the CTL analysis and leverages the continuously advancing techniques upon ranking function synthesis  \cite{DBLP:conf/tacas/ColonS01,DBLP:conf/vmcai/PodelskiR04,bradley2005polyranking,DBLP:conf/concur/NeumannO020}, thus contributing to a more efficient and precise CTL analysis. 


\vspace{-1mm}
\noindent\paragraph{\textbf{Loop Summarization and Termination Analysis}}
A loop summary captures the relationship between the inputs and outputs of a loop as a set of symbolic constraints.
Loop summarization has been used widely in static analysis \cite{DBLP:conf/tacas/TsitovichSWK11,DBLP:journals/tse/XieCZLLL19}, which primarily computes the terminating program fragments, as well as dynamic test generation \cite{DBLP:conf/issta/GodefroidL11}, which discovers the loop structure of the program on the fly via detecting induction variables. However, little attention has been put on summarizing non-terminating program executions. 

On the other hand, termination is an important correctness property in itself and is a sub-problem of proving total correctness and liveness properties. Deciding on the termination of programs is probably the most famous problem in computer science. Synthesizing ranking functions for programs is a standard way to prove the termination of programs. 
Similar representations to our $\omegaRE$ have been deployed in prior works \cite{DBLP:conf/pldi/CookPR06,DBLP:conf/cav/HeizmannHP14,DBLP:conf/pldi/0001K21} to serve intermediate steps, such as computing the ranking function, in proving termination. 



Unlike the prior approach, instead of synthesizing a ranking function, we assume the ranking function is given and propagate summaries for both terminating and non-terminating behaviors, represented in $\omegaRE$.  
The direct benefit of our approach is that the CTL analysis stands alone away from the termination analysis and takes advantage of existing complete algorithms for various classes of ranking functions, such as linear, linear-lexicographic, nested, multi-phase, etc. 
In addition, it can further deploy the tools for more complicated ranking functions, such as for polynomial loops; thus, it handles more real-world programs extensively. 


%Furthermore, to ease the pain of providing the ranking functions, we also formalize a simple (but sound) procedure for inferring them when possible. 
%Note that the soundness of our approach does not reply to the precision of the provided/inferred ranking function, yet it does affect the completeness, i.e.,  there exists a patch, but we cannot find it. 
%Our approach also draws similarities to prior works \cite{DBLP:conf/pldi/0001K21,DBLP:conf/pldi/CookPR06}, we use $\omega$-regular expressions as the intermediate representation of the infinite behaviors. However, one of the differences is we summarise the behaviors for finite time trace repetitions instead of the Kleene star $\star$ commonly deployed previously. 









\vspace{-1mm}
\noindent\paragraph{\textbf{Datalog Temporal Analysis}}

To enable the expressivity for CTL properties using Datalog, prior works \cite{gottlob2002datalog,gottlob2000linear}
present Datalog LITE, a new deductive query language. Datalog LITE is a variant of Datalog that uses stratified negation, restricted variable occurrences, and a limited form of universal quantification in rule bodies. 
We borrow their encoding of the AF operator, 
%(cf. \figref{fig:AF_encoding}), 
which requires the finiteness of the input Kripke structure. 
This encoding also follows from the facts that, over finite structures, CTL can be embedded into transitive closure logic \cite{DBLP:conf/cav/ImmermanV97} and that transitive closure logic has the same expressive power as stratified linear Datalog programs \cite{DBLP:journals/tcs/ConsensM93,DBLP:conf/csl/Gradel91}. 
In parallel with the above mentioned works, \cite{DBLP:journals/tcs/GuessarianFAA03}
provides a direct and modular translation from CTL and the Modal $\mu$-calculus to Monadic inf-Datalog, with built-in predicates. Inf-Datalog has slightly different semantics from the conventional Datalog least fixed point semantics in that some recursive rules (corresponding to least fixed points) are allowed to unfold only finitely many times, whereas others (corresponding to greatest fixed points) are allowed to unfold infinitely many times. 



%a domain-specific
%language that makes it possible to write a range of SMT-based static analyses in a way that is both close to
%their formal specifications and amenable to high-level optimizations and efficient evaluation.
%Analyzers based on Datalog also gained popularity recently; for example, prior work \cite{DBLP:journals/pacmpl/Bembenek0C20} proposes Formulog, which extends the logic programming language Datalog with a first-order functional language and mechanisms for representing and reasoning about SMT formulas; a novel type system supports the construction of expressive formulas while ensuring that neither normal evaluation nor SMT solving goes wrong. Their 
%case studies demonstrate that a range of SMT-based analyses can naturally and concisely be encoded in Formulog, and high-level Datalog-style optimizations can be automatically and advantageously applied to these analyses. 


%Datalog-Defined Program Repair and 
%Prior work \cite{DBLP:conf/sigsoft/LiuMSR23} proposes to integrate program repair with Datalog-based analysis, which enables a general-purpose static program repair framework that targets a wide range of properties and programming languages. 
%It repairs the program by modifying the corresponding Datalog facts via a symbolic execution of Datalog. It is useful for properties that only involve the least fixpoints, such as null pointer exceptions or data leaks, without supporting the properties that involve a combination of the least and greatest fixed points, such as CTL. Our repair strategy in this work supports repairs guided by CTL properties by supporting the negations that can arbitrarily occur in the rules.   


\vspace{-1mm}
\noindent\paragraph{\textbf{Model Repair}}




\citet{DBLP:conf/ecai/DingZ06} proposed a CTL model update algorithm based on four primitive operations and a minimal change criterion by using Kripke structure models; the algorithm obtains a new model as similar as possible to the original one. 
\citet{DBLP:journals/ai/CarrilloR14} introduced the concept of protection for dealing with the loss of satisfiability problem of a property while introducing changes in the new model. 
%\citet{DBLP:conf/nfm/ChatzieleftheriouBSK12} use an abstract model repair framework, under a 3-valued semantics of CTL. 
\citet{martinez2015ctl} present a CTL model repair solution for bounded and deadlock-free Petri nets. 
This approach repairs the model guided by CTL specifications via two basic repair operations: adding/removing transitions and enabling/disabling variables. Both these operations can be reflected in our approach via adding/deleting transition facts and facts for atomic propositions for program variables. 
However, it requires a re-analysis after the ``repaired'' model is generated, which means that produced patches are not always safe; 
in contrast, our approach produces patches without any re-analysis, if any. 

%such a model repair could produce patches that still fail the target specification, which is called ``inconsistency''; 

%The difference is that our repair does not modify the model's skeleton, i.e., the transition flows are dependent on constraints of program variables, as they are encoded as rules instead of facts. 


Prior work, \cite{DBLP:conf/memocode/AttieCBSS15} maps an instance of the repair program, i.e., a Kripke structure model and a CTL property to a Boolean formula, and the satisfiability resulted from the SAT solver indicates a patch exist or not. When satisfiable, the returned model will be mapped back to a patch solution to remove transitions/states, 
without any addition of transitions/states, to avoid introducing new behaviors. 
This approach is also deployed for repairing model in the concurrent programs context~\cite{DBLP:conf/fossacs/AttieC23}.  
In \cite{DBLP:journals/ai/BuccafurriEGL99} the repair problem for CTL is considered and solved using adductive reasoning. The method generates repair suggestions that must then be
verified by model checking, one at a time. In contrast, we fix all faults at
once. 
Program Repair as a Game~\cite{DBLP:conf/cav/JobstmannGB05}, 
considers only one repair at a time, and their method is complete only for invariants. 
In 
\cite{DBLP:conf/charme/StaberJB05} , the approach of \cite{DBLP:conf/cav/JobstmannGB05} is extended so that multiple faults are considered at once, and 
it assumes that the specification is given in linear-time
temporal logic and state the localization and correction problem as a game that
is won if there is a correction that is valid for all possible inputs. 
To cope with more realistic systems, 
our approach differs from the above-mentioned model repair in a way that model repair targets models -- either Kripke structure or Petri nets. In contrast, our approach targets 
%(possibly infinite state) 
real-world programs, which may involve different language features, especially loops. 
To the best of the authors' knowledge, this is the first work to provide the CTL repair solution at the source code level. We prove its usefulness using real-world benchmarks, which no prior work has achieved. 
%Instead of enforcing invariant annotations, we provide a way to compute summaries of loops based on an iterative process of refining the ranking functions. Thus, t
%Moreover, while this paper focuses on supporting Datalog repair with negations, we leave the repair for Datalog rules as a future work. 

\vspace{-1mm}