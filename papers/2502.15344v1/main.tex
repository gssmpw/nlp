\documentclass[10pt,journal,compsoc]{IEEEtran}



\let\Bbbk\relax
\usepackage{amsmath,amssymb,amsfonts}
\usepackage{graphicx}
\usepackage{hyperref}
\usepackage{comment}
\usepackage{import} % subimport
\usepackage{xspace} % xspace
\usepackage{numprint} % numprint
\usepackage{wrapfig}
\usepackage{balance}
\usepackage{ifthen}
\usepackage{listings}
\usepackage{subcaption} % subfigure
\usepackage{centernot} % for the comparison
\renewcommand\theequation{\alph{equation}}
\usepackage[nameinlink]{cleveref}
\usepackage{caption}
\usepackage{framed}
\usepackage{tikz}
\usepackage{tikz-qtree}
\usetikzlibrary{shapes.geometric, arrows, positioning}
\usepackage[epsilon,altpo]{backnaur}
\usepackage{bussproofs}
\usepackage{booktabs}
\usepackage{tabularx}
\usepackage{pifont}
\usepackage{wasysym}
\usepackage{makecell}
\usepackage{mathtools}
\usepackage{xcolor}
\usepackage{mdframed} % for creating framed boxes that can break across pages


\usepackage{enumitem}
\usepackage{adjustbox}
\usepackage{amsthm}
\usepackage[utf8]{inputenc}
\usepackage{textcomp}
\usepackage{proof}
\usepackage{pmboxdraw}

\usepackage{algorithm}
\usepackage{algorithmic}

% \usepackage[noend]{algpseudocode}

% \usepackage[linesnumbered,ruled,vlined]{algorithm2e}

\usepackage{stmaryrd}
\usepackage{titlecaps}% http://ctan.org/pkg/titlecaps
\usepackage{tikz}
\usepackage{xspace}
\usepackage{pifont}% http://ctan.org/pkg/pifont
\usepackage[normalem]{ulem}


\usepackage{ifthen} % for todos
% TODOs

\usepackage{stmaryrd}
\usepackage{multirow}
\usepackage{longtable}
\usepackage{makecell, boldline}

\usepackage{xparse}


\usepackage{thm-restate}
%\usepackage[capitalise]{cleveref}


\usepackage{color,latexsym,graphics,wrapfig}
\usepackage{listings}
\usepackage{lineno}
\usepackage{mathpartir}
\usepackage{apxproof}
\usepackage{makecell}


% \newif\iftechreport
% \techreporttrue
% \newcommand*{\techreportonly}[1]{}


% \algrenewcommand\algorithmicindent{1em} % or any other value as you wish
%\usepackage[bottom]{footmisc}

\let\appendix\relax


\usepackage{booktabs}
\usepackage{multirow}
\usepackage{graphicx}
\usepackage{amsmath}
\usepackage{xspace}
\DeclareMathOperator*{\argmax}{arg\,max}
\DeclareMathOperator*{\stmx}{Softmax}
\allowdisplaybreaks
 
\newcommand{\ours}{\textsc{suPreMe}\xspace}
\newcommand{\ITC}{image-text consistency\xspace}
\newcommand{\BPG}{biased prompts generation\xspace}
\usepackage{colortbl}

\usepackage{amsthm}

\newtheorem*{remark}{Remark}
\newtheorem{theorem}{Theorem}
\newtheorem{assumption}{Assumption}

\usepackage{makecell}

\usepackage{amsbsy}
 % TODOs, English, etc


\hyphenation{op-tical net-works semi-conduc-tor IEEE-Xplore}

\begin{document}

\title{Computation Tree Logic Guided Program Repair}

\author{
\IEEEauthorblockN{Yu Liu*, Yahui Song*, Martin Mirchev, and Abhik Roychoudhury}
    \thanks{
        \begin{tabbing}
        $\bullet$ \= Yu Liu is with the National University of Singapore \\ 
        \> E-mail: liuyu@comp.nus.edu.sg \vspace{6pt} \\

        $\bullet$ \= Yahui Song is with the National University of Singapore \\ 
        \> E-mail: yahui\_s@nus.edu.sg \vspace{6pt} \\

        $\bullet$ \= Martin Mirchev is with the National University of Singapore \\ 
        \> E-mail: mmirchev@comp.nus.edu.sg \vspace{6pt} \\

        $\bullet$ \= Abhik Roychoudhury is with the National University of Singapore \\ 
        \> E-mail: abhik@comp.nus.edu.sg \vspace{6pt} \\

        \end{tabbing}
    }
}





\IEEEtitleabstractindextext{%
\begin{abstract}
  Temporal logics like Computation Tree Logic (CTL) have been widely used as expressive formalisms to capture rich behavioral specifications. CTL can express properties such as reachability, termination, invariants and responsiveness, which are {difficult to test}.
  This paper suggests a mechanism for the automated repair of infinite-state programs guided by CTL properties. 
  Our produced patches avoid the overfitting issue that occurs in test-suite-guided repair, where the repaired code may not pass tests outside the given test suite. 
  To realize this vision, we propose a repair framework based on Datalog, a widely used domain-specific language for program analysis, which readily supports nested fixed-point semantics of CTL via stratified negation. Specifically, our framework encodes the program and CTL properties into Datalog facts and rules and performs the repair by modifying the facts to pass the analysis rules. Previous research proposed a generic 
  repair mechanism for Datalog-based analysis in the form of Symbolic Execution of Datalog (SEDL). 
  However, SEDL only supports positive Datalog, which is insufficient for expressing CTL properties. Thus, we extended SEDL to make it applicable to stratified Datalog. 
  Moreover, liveness property violations involve infinite computations, which we handle via a novel loop summarization. Our approach achieves analysis accuracy of \ourToolSmallBenchmark\,  on a small-scale benchmark and \ourToolRealBenchmark\, on a real-world benchmark, outperforming the best baseline performances of \bestBaseLineSmall\, and \bestBaseLineReal.
  Our approach repairs all detected bugs, which is not achieved by existing tools.  
\end{abstract}
\begin{IEEEkeywords}
Program Analysis and Automated Repair, Datalog, Loop Summarization
\end{IEEEkeywords}
}

\maketitle

\subimport{sections/}{introduction.tex}
\subimport{sections/}{overview.tex}
\subimport{sections/}{background.tex}
\subimport{sections/}{ctl.tex}
\subimport{sections/}{repair.tex}
\subimport{sections/}{eval.tex}
\subimport{sections/}{related_work.tex}
\subimport{sections/}{conclusion.tex}


% \section{Data Availability}
% The software that supports \apiRepairTool is available at \reproductionURL.

 
 % argument is your BibTeX string definitions and bibliography database(s)

\bibliographystyle{IEEEtran}

\bibliography{IEEEabrv,bibliography}%


\newpage





\vfill

\end{document}
