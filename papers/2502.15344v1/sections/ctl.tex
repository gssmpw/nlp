
\begin{figure*}[!h]
\centering
\begin{gather*}
%%%%%%%%%%%%%%%%%%%%%%%%%%%%%%%%%%
%%%%%%%%%%%%%%% AP %%%%%%%%%%%%%%%
%%%%%%%%%%%%%%%%%%%%%%%%%%%%%%%%%%
\frac{
\begin{matrix}
\drule {=} \nm(S) \datalogarrow \pi(S)
\end{matrix}
}{\CTLToD{(\nm,\pi)}{\nm}{[\drule]}
}~[\CTLtoDKey\m{AP}]
\qquad 
%%%%%%%%%%%%%%%%%%%%%%%%%%%%%%%%%%
%%%%%%%%%%%%%%% Neg %%%%%%%%%%%%%%%
%%%%%%%%%%%%%%%%%%%%%%%%%%%%%%%%%%
\frac{
\begin{matrix}
\CTLToD{\phi}{\nm}{\drule^*} 
\qquad 
\nm_{\mathtt{new}} {=} ``NOT\_" \concat \nm  \qquad 
\drule^\prime {=} \nm_{\mathtt{new}}(S) \datalogarrow\shortNeg\,\nm(S)
\end{matrix}
}{ 
\CTLToD{\neg\phi}{\nm_{\mathtt{new}}}{\drule^* \concat [\drule^\prime]}
}~ 
[\CTLtoDKey\m{Neg}]
\\[1.5em]
%%%%%%%%%%%%%%%%%%%%%%%%%%%%%%%%%%
%%%%%%%%%%%%%%% AF %%%%%%%%%%%%%%%
%%%%%%%%%%%%%%%%%%%%%%%%%%%%%%%%%%
\frac{
\begin{matrix}
\CTLToD{\phi}{\nm}{\drule^*_1} \qquad
\nm_{\mathtt{new}} {=} ``AF\_" \concat \nm \qquad
\nm_{\mathtt{s}} {=} ``AFS\_" \concat \nm \qquad
\nm_{\mathtt{t}} {=} ``AFT\_" \concat \nm
\\
\drule^* {=} 
\left[\begin{matrix} 
\nm_{\mathtt{t}}(S,S') {\datalogarrow} \shortNeg\, \nm(S), \predFlow(S,S') 
\qquad\quad 
\nm_{\mathtt{t}}(S,S') {\datalogarrow}\nm_{\mathtt{t}}(S,S"), \shortNeg\ \nm(S"), \predFlow(S",S') 
\\
\nm_{\mathtt{s}}(S) \datalogarrow\nm_{t}(S,S)
\qquad\nm_{\mathtt{s}}(S) \datalogarrow\shortNeg \,\nm(S) , \predFlow(S,S'), \nm_{\mathtt{s}}(S') 
\qquad\nm_{\mathtt{new}}(S) \datalogarrow\shortNeg \,\nm_{\mathtt{s}}(S)
\end{matrix} \right]
\end{matrix}
}{\CTLToD{AF\,\phi}{\nm_{\mathtt{new}}}{\drule^*_1 \concat \drule^*}
}~[\CTLtoDKey\m{AF}]
\hide{
\\[0.2em] 
%%%%%%%%%%%%%%%%%%%%%%%%%%%%%%%%%%
%%%%%%%%%%%%%%% EF %%%%%%%%%%%%%%%
%%%%%%%%%%%%%%%%%%%%%%%%%%%%%%%%%%
\frac{
\begin{matrix}
[\CTLtoDKey\m{EF}] \\
\CTLToD{\phi}{\nm}{\drule^*_1} \quad\nm_{\mathtt{new}} {=} "EF\_" {\concat} \nm\\
\drule^* {=} \left[
\begin{matrix}
\nm_{\mathtt{new}}(S) \datalogarrow\ \nm(S) \\
\nm_{\mathtt{new}}(S) \datalogarrow\predFlow(S,S'), \nm_{\mathtt{new}}(S')
\end{matrix}
\right]
\end{matrix}
}{\CTLToD{EF\,\phi}{\nm_{\mathtt{new}}}{\drule^*_1 \concat \drule^*}
}
\quad 
%%%%%%%%%%%%%%%%%%%%%%%%%%%%%%%%%%
%%%%%%%%%%%%%%% Disj %%%%%%%%%%%%%%%
%%%%%%%%%%%%%%%%%%%%%%%%%%%%%%%%%%
\frac{
\begin{matrix}
[\CTLtoDKey\m{Disj}] \\
\CTLToD{\phi_{1}}{\nm_{1}}{\drule^*_1} \quad \CTLToD{\phi_{2}}{\nm_{2}}{\drule^*_2}\\ 
\nm_{\mathtt{new}} {=} \nm_{1} \concat \ensuremath{"\_OR\_"} \concat \nm_{2}
\\
\drule^* {=}[\nm_{\mathtt{new}}(S) \datalogarrow \nm_{1}(S) \qquad \nm_{\mathtt{new}}(S) \datalogarrow \nm_{2}(S)]
\end{matrix}
}{ 
\CTLToD{\phi_{1} \lor \phi_{2} }{\nm_{\mathtt{new}}}{\drule^*_1 \concat \drule^*_2 \concat \drule^*}
}
 \\[0.2em] 
 \quad 
%%%%%%%%%%%%%%%%%%%%%%%%%%%%%%%%%%
%%%%%%%%%%%%%%% hided EU %%%%%%%%%%%%%%%
%%%%%%%%%%%%%%%%%%%%%%%%%%%%%%%%%%
\frac{
\begin{matrix}
[\CTLtoDKey\m{EU}]\\
\CTLToD{\phi_{1}}{\nm_{1}}{\Datalog_1} \quad \CTLToD{\phi_{2}}{\nm_{2}}{\Datalog_2}\quad
\nm_{\mathtt{new}} {=} \nm_{1} \concat \ensuremath{``\_EU\_"} \concat \nm_{2}\\ 
\Datalog^\prime {=}
\left[ 
 \begin{matrix} 
\nm_{\mathtt{new}}(S) \datalogarrow\nm_{2}(S) \qquad 
\nm_{\mathtt{new}}(S) \datalogarrow\nm_{1}(S), \predFlow(S,S'), \nm_{\mathtt{new}}(S')
 \end{matrix} \right]
\end{matrix}
}{\CTLToD{E ( \phi_{1} \,U\, \phi_{2}) }{\nm_{\mathtt{new}}}{\Datalog_1 {\concat} \Datalog_2 {\concat} \Datalog^\prime}
}} 
\end{gather*}
\caption{Selected CTL-to-Datalog Encoding (The positive predicate ${\tt{State}}(S)$ is implicitly inserted to ground the variables)}
\label{fig:ctl-datalog-translation-table-ctl}
\end{figure*}


 
\section{CTL Analysis using Datalog}

In this section, we outline the essential steps for conducting a CTL analysis using Datalog. 


\subsection{From CTL Properties to Datalog Rules} 
\label{subsec:fromCTL2Datalog}

Given any CTL formula \code{\phi}, the relation ``\code{\CTLToD{\phi}{\nm}{\drule^*}}" holds if \code{\phi} can be translated into a set of Datalog rules  \code{\drule^*}. 
The validity of \code{\phi} against a program is then indicated by the presence of the IDB predicate \code{\m{\nm}} after executing \code{\drule^*} against the program facts. 
Since most of the encoding rules are standard \cite{rocca2014asp}, we selectively present them in \figref{fig:ctl-datalog-translation-table-ctl}. 
%, while the complete set of encoding can be found in \app{II} of the supplementary material \cite{tech_report}. 
For instance, in \code{[\CTLtoDKey\m{AP}]}, given a CTL formula containing one atomic proposition  
\code{(\nm, \pi)}, it produces the Datalog rule ``$\nm(S) \datalogarrow \pi(S)$", which generates the predicate \code{\nm} for all states that satisfy the pure constraint \code{\pi}. 


%fig:first_Example_ctl_rules
%In particular, the encoding of the $AF$ operators is adopted from \cite{gottlob2002datalog}, which enables the greatest fixpoint encoding using the least fixpoint semantics of Datalog. 




Different from prior work \cite{rocca2014asp}, which relies on the logical impurity -- the ``findall" operator --  to encode the $AF$ operator. 
However, since ``findall" operators introduce logical impurity and are not supported by Datalog, we adopt the encoding from the work of~\cite{gottlob2002datalog}, which enables the greatest fixed point encoding using the least fixed point semantics of Datalog while maintaining the Datalog purity. 
Intuitively, given a CTL formula \code{AF\,\phi}, 
the resulting Datalog rules from \code{[\CTLtoDKey\m{AF}]} are to prove the absence of the \emph{lasso-shaped} \cite{DBLP:conf/cav/HeizmannHP14} 
counterexamples, \ie ``$\nm_{\mathtt{s}}$" -- a stem path to a particular program location followed by a cycle that returns to the same program location, \ie the ``$\nm_{\mathtt{t}}(S,S)$", and the property \code{\phi} does not hold along the stem and the cycle. 

One example for \code{AF(y{=}5)} is shown in \figref{fig:first_Example_ctl_rules}. 
The positive predicate ${\tt{State}(S)}$ is implicitly inserted to ground the variables. 
We use ``$\concat$" for
both string and list concatenation. 





\subsection{From Programs to Guarded Omega-Regular Expressions} 
\label{subsec:progam2WRE}



As defined in \figref{fig:CFG_language}, 
any program \code{\mathcal{P}} consists of a set of functions, which is identified by a name $\nm$, and represented by a CFG with a starting node \code{N} and a transition function \code{\E}, which  
returns all the immediate successors of a given node. 
Each node is parameterized with a unique (integer) state identifier \code{s}. 
We assume that the sets of nodes in different functions are pairwise disjoint. 
Apart from the \code{\m{Start}} and \code{\m{Exit}}, 
\code{\m{Join}} is used to connect disjunctive branches (usually created by conditionals and loops), while \code{\m{Prune}} is used to specify each branch based on a path constraint. 
%given arithmetic formula. 
%For example, a \code{\m{Join}} node's successors can be two \code{\m{Prune}} nodes with disjoint constraints. 
Finally, \code{\m{Stmt}} stores statements such as assignments, returns, and function calls where the return value is explicitly denoted by \code{r}.  



\begin{figure}[!b]
{
$
\begin{array}{lrcl}
\m{(Program)}  & {\mathcal{P}} &{::=}& \m{func}^* \\
\m{(Func.~Def.)}  & \m{func} &{::=}& \nm = (N, \text{\code{\E}}) \\[0.2em]
\m{(Statements)}  & \m{e} &{::=}& x{:=} t 
              \mid \m{return}(x)  \mid \nm({x^*, r}) \\[0.2em]
  
\m{(Trans.~Fun.)} & \text{\code{\E}} &{::=}& N \,{\rightarrow}\, N^* \\[0.2em]
\m{(Nodes)} & {N} &{::=}& \m{Start}(s)  
\mid \m{Exit}(s) \mid \m{Join}(s)  \mid   \\[0.2em]
&  &&  \m{Prune}(\pi, s) \mid \m{Stmt}(e, s)
\end{array}$
\vspace{1mm}
\caption{CFG Structure of Target Programs} 
\label{fig:CFG_language}
}
\end{figure}

\begin{figure}[!b]
\[\effect {::=}   \bot \mid \epsilon \mid \pi\loc{@s} \mid [\pi]\loc{@s} \mid \effect_1 \cdot \effect_2 
 \mid \effect_1 \vee \effect_2 \mid \effect^\omega\]
\caption{The Syntax of Guarded \code{\omegaRE}}
\label{fig:Syntax_of_Omega_RE}
\end{figure}



As shown in \figref{fig:Syntax_of_Omega_RE},  
the Guarded \code{\omegaRE} formulas are similar to those in 
the classic \emph{Omega Regular Expressions},  
containing \code{\bot} for \emph{false}, \code{\epsilon} for empty traces, sequence concatenations (\code{\effect_1 \,{\cdot}\, \effect_2}), disjunctions (\code{\effect_1 \,{\vee}\,  \effect_2}), and infinite repetitions of a trace by \code{\effect^\omega}. 
The main difference is, in Guarded \code{\omegaRE}, singleton events can be either \code{\pi} or \code{[\pi]}. 
The former updates the program states with respect to the $\pi$ formula, while the latter serves as program guards. 
Program guards are the sole means of introducing and controlling non-determinism. 
For the formula ``\code{[\pi] \cdot \effect}", the guarded trace \code{\effect} is executed only when the guard does not fail. 
When checking a guard, a Boolean expression is evaluated. If it denotes false, the guard fails. If it denotes true, it allows the execution to proceed without affecting the program states. 
Moreover, Guarded~\code{\omegaRE} does not contain Kleene stars as it aims to eliminate the unknown number of repetitions using either fixed-number-length finite traces or $\omega$ formulas for infinite executions. 

\begin{algorithm}[!b]
  \caption{CFG2GWRE}
  \label{alg:cfg2GuardedomegaRE}
  \begin{algorithmic}[1]

  \REQUIRE A node $N$, and a transition function $\E$
  \ENSURE The final Guarded~\code{\omegaRE}, $\effect$

  \IF{$N$ matches $\m{Exit(s)}$ or $\m{Stmt(return(x), s)}$}
      \RETURN $\nodeEv(N)$
  \ELSIF{$N$ matches $\m{Join(s)}$}
      \STATE \textit{cycleInfo} $\gets$ \textsc{existsCycle} $(N, \E)$
      \IF{\textit{cycleInfo} is None}
          \RETURN \textsc{moveForward} $(N, \E)$
      \ELSE
          \STATE Extract $(\pi_g, \effect_{\m{cycle}}, N_{\m{nonCycleSucc}})$ from \textit{cycleInfo}
          \STATE $\effect_{\m{rest}} \gets \textsc{moveForward}(N_{\m{nonCycleSucc}}, \E)$
          \RETURN $\loopsummary(\pi_g, \effect_{\m{cycle}}, \effect_{\m{rest}})$
      \ENDIF
  \ELSE
      \RETURN \textsc{moveForward} $(N, \E)$
  \ENDIF

  \STATE

  \STATE \textbf{Function} \textsc{moveForward}$(N, \E)$
  \IF{$\E(N) = []$}
      \RETURN $\nodeEv(N)$
  \ELSE
      \STATE $\effect_{\m{acc}} \gets \bot$
      \FORALL{$N^\prime \in \E(N)$}
          \STATE $\effect_{\m{acc}} \gets \effect_{\m{acc}} \vee \textsc{CFG2GWRE}(N^\prime, \E)$
      \ENDFOR
      \RETURN $(\nodeEv(N) \cdot \effect_{\m{acc}})$
  \ENDIF
  

  \end{algorithmic}
\end{algorithm}


Each function is converted to a Guarded \code{\omegaRE} formula via \algoref{alg:cfg2GuardedomegaRE}, 
which enumerates all paths from the CFG and replaces the cycles using the loop summaries. 
If the current node is an \emph{exiting} node (line 2), it returns its corresponding \code{\effect} formula. 
Here, \code{\nodeEv}(N) maps from nodes to \code{\effect}.    
If the current node is a \code{\m{Join}}, 
it firstly checks if it leads to a cycle via calling \textsc{existsCycle}  (line 4), which 
%The \textsc{existsCycle} function 
returns ``None" if none of its successors directly leads to cycles; otherwise, it returns ``Some (\code{\pi_g}, \code{\effect_{\m{cycle}}}, \code{N_{\m{nonCycleSucc}}})", indicating that one successor of $N$ leads to a path with guard \code{\pi_g} that comes back to itself, and $\effect_{\m{cycle}}$ describes the behaviour of one iteration of the cycle, 
and $N_{\m{nonCycleSucc}}$ is the other successor, which does not directly lead to a cycle. 
The \textsc{existsCycle} is implemented using a 
depth-first search algorithm. 
%\appref{appsec:existsCycle}. 
The cycle behaviours are sent to a summary calculus (line 8).   
In all other cases (lines 5 and 9), it uses an auxiliary function  \textsc{Moveforward} to accumulate the effects of the current node and 
recursively combine the behaviours of all the following nodes. If there are no successors, \textsc{Moveforward} terminates; otherwise, it 
goes through successors and applies \textsc{CFG2GWRE} 
to calculate the formulas for the remainder of the path. It then disjunctively combines the outcomes. 

{
\begin{definition}[{{CFG Nodes to Guarded~\code{\omegaRE}}}]
\label{def:NodeS2Singleton_Events}
{{Given any \code{N}, its \code{\Phi} formula is defined as follows:}} 
{
\begin{gather*}
%\nodeEv{(\m{Start}(s))} {\nodeEvop} (\m{Start}())\loc{@s} \qquad     
\nodeEv{(\m{Stmt}(x{:=}t,s))} {\nodeEvop} (x{=}t)\loc{@s} 
% \quad\
\\  
\nodeEv{(\m{Stmt}(\m{return}(\_),s))}{\nodeEvop} 
((\m{Exit}())\loc{@s})^\omega
% \quad\ 
\\ 
\nodeEv{(\m{Prune}(\pi,s))}{\nodeEvop} [\pi]\loc{@s} 
%\qquad \nodeEv{(\m{Join}(s))} {\nodeEvop} \epsilon 
\qquad 
\nodeEv{(\m{Exit}(s))}{\nodeEvop} (\m{Exit}())\loc{@s}
% \quad\  
\\
\nodeEv{(\m{Stmt}(\nm(x^*,r),s))}{\nodeEvop}
\begin{cases}
(\nm(x^*))\loc{@s} \cdot(r{=}*)\loc{@s} \\
 \qquad \m{when}\    ({\text{{$\nm\,{\not\in}\,\mathcal{P}$}}})
\\
\effect_{\nm}[x^*/y^*, r/\m{ret}]   \\
 \qquad \m{when} \    ({\text{{$\effect_{\nm}(y^*,\m{ret})\,{\in}\,\mathcal{P}$}}})
\end{cases}
\end{gather*}
}
\end{definition}
}



The function \code{\nodeEv}, as described in  \defref{def:NodeS2Singleton_Events}, maps the \code{\m{Prune}} nodes into guards and excludes the not mentioned node constructs (\code{\m{Start}} and \code{\m{Join}}), using \code{\epsilon}. 
Function calls with undefined callees are modelled as non-deterministic choices, denoted by \code{r{=}*}. When the callee is defined, we retrieve its summary \code{\effect_{\nm}} and instantiate it by substituting formal arguments with actual arguments, denoted by \code{\effect_{\nm}[x^*/y^*, r/\m{ret}]}. %In all cases, these summaries over-approximate the actual behaviours observable. 
%during execution, providing a conservative estimate of the program's dynamics. 














{\subsection{From CFG Cycles to Guarded Omega-Regular Expressions }
\label{subsec:loop2wRE}

Targeting sequential non-recursive infinite-state programs, summaries are constructed from the innermost loop. In the case of nested loops, inner loops are expected to be 
replaced with summaries first. 
At any point during the analysis, the problem is therefore reduced to the analysis of a single loop. 
Intuitively, our loop summaries aim to replace terminating behaviours using their final states and convert non-terminating behaviours using $\omega$ constructs over finite traces. 




%invariants, and wrap them in the $\omega$ construct. 
%A loop terminates if a ranking function exists, which is non-negative at the entry point and proven to be decreasing at each iteration of the loop. 

\begin{algorithm}[!b]
\caption{Loop Summary Computation}
\label{alg:loopsummary}
\begin{algorithmic}
\REQUIRE $\pi_g$, $\effect_{\text{cycle}}$, $\effect_{\text{rest}}$
\ENSURE A loop summary $\effect$ or Unknown
\FORALL{$\text{rf} \in \text{RF}(\pi_g)$ \COMMENT{Obtain CRFs}}
\STATE $\pi_{\text{wpc}}^T \gets \Delta \text{rf}(\effect_{\text{cycle}}) \geq 1$
\STATE $\pi_{\text{wpc}}^{\text{NT}} \gets \Delta \text{rf}(\effect_{\text{cycle}}) < 1$
\IF{$\pi_g \Rightarrow \left(\pi_{\text{wpc}}^{T} \vee \pi_{\text{wpc}}^{\text{NT}}\right)$ \COMMENT{Full Path Conclusive}}
\STATE $\effect_{\text{term}}^1 \gets [\neg \pi_g] \cdot \effect_{\text{rest}}$
\STATE $\effect_{\text{term}}^2 \gets [\pi_g \wedge \pi_{\text{wpc}}^T] \cdot (\text{rf} < 0) \cdot \effect_{\text{rest}}$
\STATE $\effect_{\text{nonTerm}} \gets [\pi_g \wedge \pi_{\text{wpc}}^{\text{NT}}] \cdot (\text{rf} \geq 0)^\omega$
%\IF{$\pi_{\text{wpc}}^{\text{NT}} \not= \m{F}$} 
%\RETURN $\effect_{\text{term}}^1 \vee \loopsummary(\pi_{\text{wpc}}^{\text{NT}}, (\text{rf} < 0) \cdot \effect_{\text{cycle}} , (\text{rf} {<} 0) \cdot \effect_{\text{rest}})$
%\ELSE{
\RETURN $\effect_{\text{term}}^1 \vee \effect_{\text{term}}^2 \vee \effect_{\text{nonTerm}}$
%}\ENDIF
\ENDIF
\ENDFOR
\RETURN Unknown \COMMENT{Inconclusive Result}
\end{algorithmic}
\end{algorithm}


The \loopsummary~ function is detailed in 
\algoref{alg:loopsummary}, which takes three arguments: a loop guard \code{\pi_g}, the repetitive behaviours of the loop $\effect_{\m{cycle}}$, and the behaviours after the loop $\effect_{\m{rest}}$. 
Instead of being directly dedicated to the ranking function synthesis problem, the algorithm obtains a set of \emph{candidate ranking functions} (CRFs) from the guard of the repetitive case \code{\pi_g}, denoted by \code{\RF(\pi_g)}. 




  

For each CRF, we compute the weakest precondition (WPC) for termination and non-termination, denoted by 
\code{\pi_{\m{wpc}}^T} and \code{\pi_{\m{wpc}}^\m{NT}}, respectively. 
Here, the difference between the initial value of $\m{rf}$ and the updated value $\m{rf}^\prime$ after the transition \code{\effect} is denoted by \code{\Delta \m{rf}(\effect)}, \ie \code{\Delta \m{rf}(\effect) \,{=}\, \m{rf} \,\text{-}\, \m{rf}^\prime}. 
Intuitively, under \code{\pi_{\m{wpc}}^T}, \code{\m{rf}} can be proven to be strictly decreasing for all the transitions in the loop body; thus, it leads to terminating executions. 
Likewise, under \code{\pi_{\m{wpc}}^\m{NT}}, \code{\m{rf}} can be proven to be strictly not-decreasing for all the transitions in the loop body; thus, it leads to non-terminating executions. 




To achieve a sound CTL analysis, we only obtain conclusive results when the union of these two WPCs covers the full path upon entering the loop (checked in line 4). 
This means that the summary should not contain any paths for which we cannot determine whether they lead to termination or non-termination. 
Otherwise, we report ``Unknown" and quit the analysis. In the loop summary, \code{\effect_{\m{term}}^1} denotes the case when the execution did not enter into the loop at all,  \code{\effect_{\m{term}}^2} denotes the case when entering the loop and terminates eventually, and \code{\effect_{\m{nonTerm}}} denotes the case when entering the loop and getting into an infinite execution where \code{\m{rf}{\geq}0} is the recurrent state \cite{DBLP:conf/sas/Ben-AmramDG19} that witnesses non-termination. 
The status of other (non-CRF) program variables can be included in the summary relative to terminating and non-terminating cases, respectively. 



\paragraph*{\textbf{Example 1: }} 
\label{sec:example1}
Revisiting the loop shown in \figref{fig:first_Example}, when triggering \loopsummary, the arguments are \code{\pi_g{=}(x{=}y)}, \code{\effect_{\m{cycle}}{=}\epsilon} and  \code{\effect_{\m{rest}}{=}{(y{=}5)}}. 
By \defref{def:rankingFunction}, we obtain: 
\code{\m{rf}{=}(x\text{-}y)} and \code{\m{rf}{\,\in\,}\RF(x{=}y)}}. 
Based on \code{\m{rf}}, we compute 
\code{\pi_{\m{wpc}}^T{=}F} and \code{\pi_{\m{wpc}}^\m{NT}{=}T}, as \code{\m{rf}} never decreases. Thus, we conclude the final summary to be \code{([x{\not=}y]\cdot(y{=}5)) \vee ([x{=}y{\,\wedge\,}F]\cdot (\m{rf}{<}0) \cdot(y{=}5)) \vee
([x{=}y{\,\wedge\,}T]\cdot (\m{rf}{\geq}0)^\omega)
}; which reduces to the summary concluded in \figref{fig:omegaRE_first_Example}, \ie (\code{[x{\not=}y]  \cdot (y{=}5) \vee 
[x{=}y] \cdot (x{\geq}y)^\omega}). 


\begin{figure*}[!h]
\begin{gather*}
%%%%%%%%%%%%%%%%%%%%%%%%%%%%%%%%%
%%%%%%%%%%%% BASE %%%%%%%%%%%%%%%
%%%%%%%%%%%%%%%%%%%%%%%%%%%%%%%%%
\frac{
\begin{matrix}
\m{fst}(\effect){=}\s{} 
\quad\  
\Datalog{=}
[\predFlow(\prevS, \prevS)]%\datalogarrow T.]
\end{matrix}
}{
\RETOD{\prevS}{\pathPure}{\effect}{\Datalog}
}~[\toDatalogRule\m{Base}]
\qquad  
%%%%%%%%%%%%%%%%%%%%%%%%%%%%%%%%%
%%%%%%%%%%%% IND %%%%%%%%%%%%%%%
%%%%%%%%%%%%%%%%%%%%%%%%%%%%%%%%%
\frac{
\begin{matrix}
\m{F}{=}\m{fst}(\effect) 
\quad\  
(\forall f_i \,{\in}\, \m{F}). ~ 
\RETODHelper{\Pi}{f_i}{\prevS}{\pathPure}{\deri_{f_i}(\effect)}{\Datalog_i}
\end{matrix}
}{
\RETOD{\prevS}{\pathPure}{\effect}{\bigcup \Datalog_i}
}~[\toDatalogRule\m{Ind}] 
\\[1.2em] 
%%%%%%%%%%%%%%%%%%%%%%%%%%%%%%%%%
%%%%%%%%%%%% OMEGA %%%%%%%%%%%%%%%
%%%%%%%%%%%%%%%%%%%%%%%%%%%%%%%%%
\frac{
\begin{matrix}
[\toDatalogRule\m{Omega}]\\[0.2em] 
\Datalog_1{=} \m{tailToHeadFlows}(\effect)
\\[0.2em] 
\RETOD{\prevS}{\pathPure}{\effect}{\Datalog_2}
\end{matrix}
}{
\RETODHelper{\Pi}{\effect^\omega}{\prevS}{\pathPure}{\_}{\Datalog_1 {\concat} \Datalog_2}
}
\qquad    
%%%%%%%%%%%%%%%%%%%%%%%%%%%%%%%%%
%%%%%%%%%%%% GUARD %%%%%%%%%%%%%%%
%%%%%%%%%%%%%%%%%%%%%%%%%%%%%%%%%
\frac{
\begin{matrix}
%\pathPure^\prime = \pathPure {\,\wedge\,} \pi \qquad 
[\toDatalogRule\m{Guard}]
\\[0.2em] 
\Datalog_1{=} [\predFlow (\prevS, \currentS) \datalogarrow \pi(\prevS)] % ^\prime_{\m{path}}
\\[0.2em] 
\RETOD{\currentS}{\pathPure^\prime}{\effect}{\Datalog_2}
\end{matrix}
}{
\RETODHelper{\Pi}{[\pi]\loc{@s}}{\prevS}{\pathPure}{\effect}{\Datalog_1{\concat}\Datalog_2}
}
\qquad  
%%%%%%%%%%%%%%%%%%%%%%%%%%%%%%%%%
%%%%%%%%%%%% PURE %%%%%%%%%%%%%%%
%%%%%%%%%%%%%%%%%%%%%%%%%%%%%%%%%
\frac{
\begin{matrix}
\Datalog_1{=} [\predFlow (\prevS, \currentS)]
\\ 
\Datalog_2{=} \s{\pi'(\currentS)\mid \forall \pi' {\,\in\,} \Pi~.~ \pi\loc{@s}{\Rightarrow} \pi')}
\\
\RETOD{\currentS}{\pathPure}{\effect}{\Datalog_3}
\end{matrix}
}{
\RETODHelper{\Pi}{\pi\loc{@s}}{\prevS}{\pathPure}{\effect}{\Datalog_1{\concat} \Datalog_2{\concat} \Datalog_3}
} ~[\toDatalogRule\m{Pure}]
\end{gather*}
\caption{Translating a Guarded \code{\omegaRE} to a Datalog Program} 
\label{fig:WRE_2_Datalog}
\end{figure*}




{
\begin{definition}[Generating CRFs from Pure]
\label{def:rankingFunction}
Given any loop guard \code{\pi} on CFG, we propagate a set of terms which are candidate ranking functions: ({\code{\emptyset} for unmentioned constructs})
{
%\vspace{-5mm}
\begin{gather*}
\RF(t_1 {\geq}t_2){=}\s{t_1\text{-}t_2}
\qquad   
\RF(t_1{\leq}t_2){=}\s{t_2\text{-}t_1}
\\    
\RF(t_1{>}t_2){=}\s{t_1\text{-}t_2\text{-}1}
\qquad   
\RF(t_1{<} t_2){=}\s{t_2\text{-}t_1\text{-}1} 
\\   
%\RF(\pi_1 {\wedge} \pi_2) {=} \RF(\pi_1) \cup \RF(\pi_2)
%\qquad 
\RF(t_1{=}t_2){=}\s{(t_1\text{-}t_2);(t_2\text{-}t_1)}
%RF((t_1{\geq}t_ t_2\text{-}t_12;)\,{\wedge}\,(t_1{\leq}t_2)) 
\end{gather*}}





\end{definition}
}


\begin{theorem}[Soundness of the Generation of CRFs]
\label{theorem:Soundness_CRFs}
If the generated CRFs, from \defref{def:rankingFunction}, decreases at each iteration of the cycle, the cycle does terminate. 
\begin{proofsketch} 
By case analysis of the possible loop guard \code{\pi}. 
For example, when \code{\pi{=} (t_1{\geq}t_2)}, and
\code{\m{rf}{=}t_1\text{-}t_2}: to enter the loop, the state must satisfy \code{\m{rf}{\geq}0}, if \code{\m{rf}} is decreasing at each iteration, it will finally reach the state \code{\m{rf}{<}0}, \ie  
\code{t_1\text{-}t_2{<}0}, which no longer satisfy the loop guard; thus, the loop is terminating. 
Similar proofs for the rest of the cases. 
\hide{\begin{enumerate}
\item When \code{\pi{=} (t_1{\geq}t_2)}, and \code{\m{rf}{=}t_1\text{-}t_2}: to enter the loop, the state must satisfy \code{\m{rf}{\geq}0}, if \code{\m{rf}} is deceasing at each iteration, it will finally reach the state \code{\m{rf}{<}0}, \ie  
\code{t_1\text{-}t_2{<}0}, which no longer satisfy the loop guard; thus, the loop is terminating. 
\item When \code{\pi{=} (t_1{\leq}t_2)}, and \code{\m{rf}{=}t_2\text{-}t_1}: similar to (1). 

\item When \code{\pi{=} (t_1{>}t_2)}, and \code{\m{rf}{=}t_1\text{-}t_2\text{-}1}: to enter the loop, the state must satisfy \code{\m{rf}{\geq}0}, if \code{\m{rf}} is deceasing at each iteration, it will finally reach the state \code{\m{rf}{<}0}, \ie  
\code{t_1\text{-}t_2\text{-}1{<}0}, or \code{t_1{\leq}t_2},  which no longer satisfy the loop guard; thus, the loop is terminating. 

\item When \code{\pi{=} (t_1{<}t_2)}, and \code{\m{rf}{=}t_2\text{-}t_1\text{-}1}: similar to (3). 

\item When \code{\pi{=} (t_1{\not=}t_2)}, and 
\code{\m{rf}\in \s{(t_1\text{-}t_2\text{-}1); (t_2\text{-}t_1\text{-}1)}}: 
to enter the loop, the state must satisfy \code{t_1{\not=}t_2}, 
the exit condition of a loop with such a guard \code{\pi} is either (\code{t_1{\geq}t_2}) or (\code{t_1{\leq}t_2}); then if either the candidate ranking function is decreasing, it will reach either of the 
exit conditions and fall into cases (3) or (4); thus, the loop is terminating. 

\item When \code{\pi{=} (\pi_1\,{\wedge}\,\pi_2)}, and \code{\m{rf}\in \RF(\pi_1) \,{\cup}\, \RF(\pi_2)}: Similar to (5), the exit condition of a loop is either (\code{\neg\pi_1}) or (\code{\neg\pi_2}); then if any candidate ranking function is decreasing, it will reach either of the exit conditions; thus, the loop is terminating. 


\item When \code{\pi{=} (t_1{=}t_2)}, and \code{\m{rf}\in \s{(t_1\text{-}t_2); (t_2\text{-}t_1)}}: to enter the loop, the state must satisfy \code{t_1{=}t_2}, 
the exit condition of a loop with such a guard \code{\pi} is either (\code{t_1{>}t_2}) or (\code{t_1{<}t_2}); then if either the candidate ranking function is decreasing, it will reach either of the 
exit conditions and fall into cases (1) or (2); thus, the loop is terminating. 
\end{enumerate}}
\end{proofsketch}
\end{theorem}



Since all the conjunctions and disjunctions in arithmetic constraints are systematically decomposed by the CFG construction, we soundly over-approximate the set of CRFs using  \defref{def:rankingFunction}, meaning that if one of them concludes termination, the loop must be terminating. The soundness is defined in \theoref{theorem:Soundness_CRFs}. 
However, this approach may lack completeness, as we focus on loops that can be proven terminating through \emph{linear ranking functions} (LRFs) \cite{DBLP:conf/cav/Ben-AmramG17}, where loops are ranked linearly. 
As a result, the actual ranking functions -- when there exist leaking branches in the cycle or those that progress through phases -- may not be generated adequately. 
These situations will be classified as Unknown in the current setup. Computing loop summaries for other types of ranking functions is considered future work. %Next, we demonstrate how our approach handles MLRFs. 
%for loops with multiple paths 
%multiphase-



\begin{comment}
\paragraph*{\textbf{Example 2: Handling MLRFs}} 
\label{sec:example:Infinite_Loops_1}


\begin{figure}[!h]%[R]{0.5\linewidth}
\begin{lstlisting}[xleftmargin=4em,numbersep=6pt,basicstyle=\footnotesize\ttfamily] 
while (x>=0) { x = x - y; y = y + 1; }
\end{lstlisting} 
\vspace{-2mm}
\caption{MLRF Example} 
\label{fig:multiphase}
\vspace{-1mm}
\end{figure}

\figref{fig:multiphase} illustrates how our approach manages loops that have more than one phrase of ranking functions. 
The initial call to \loopsummary, is with 
the arguments \code{\pi_g{=}(x{\geq}0)}, \code{\effect_{\m{cycle}}{=}(x'{=}x\text{-}y)\cdot (y'{=}y{+}1)} and  \code{\effect_{\m{rest}}{=}\epsilon}. 
By \defref{def:rankingFunction}, we obtain: 
\code{\RF(x{\geq}0){=}\s{x}}. 
When \code{\m{rf}_1{=}x}, we compute 
\code{\pi_{\m{wpc}}^T{=}(y{>}0)} and \code{\pi_{\m{wpc}}^\m{NT}{=}(y{\leq}0)}, as  \code{\m{rf}_1} always decreases by \code{y}. Thus, we conclude the first summary to be: 
% \[\Phi_1 = ([x{<}0]\cdot \epsilon) \vee \uwave{([x{\geq}0{\wedge}y{>}0]\cdot (x{<}0)) \vee
% ([x{\geq}0{\wedge}y{\leq}0]\cdot ((x{\geq}0) \cdot (y'{=}y{+}1))^\omega)}. \]

\begin{align*}
  \Phi_1 = & \big([x < 0] \cdot \epsilon\big) \vee \big([x \geq 0 \wedge y > 0] \cdot (x < 0)\big) \notag \vee \\
           &  \big([x \geq 0 \wedge y \leq 0] \cdot ((x \geq 0) \cdot (y' = y + 1))^\omega \big)
\end{align*}
  


\noindent The last two cases of \code{\Phi_1} can trigger the \loopsummary~ function once more, with the arguments \code{\pi_g{=}(y{\leq}0)}, \code{\effect_{\m{cycle}}{=} (x{\geq}0) \cdot (y'{=}y{+}1)} and  \code{\effect_{\m{rest}}{=}(x{<}0)}. 
We then obtain: 
\code{\RF(y{\leq}0){=}\s{\text{-}y}}. 
When \code{\m{rf}_2{=}\text{-}y}, we compute 
\code{\pi_{\m{wpc}}^T{=}T} and \code{\pi_{\m{wpc}}^\m{NT}{=}F}, as \code{\m{rf}_2} always decreases by 1. 
Thus, we further summarize the \uwave{underlined} formula to be \code{\Phi_2}, and when plugged back in, the final summary is presented by  \code{\Phi_{\m{final}}}: 
{
\begin{align*}
\Phi_2 = & ([y{>}0]\cdot (x{<}0)) \vee 
([y{\leq}0{\,\wedge\,}T]\cdot (y{>}0)\cdot (x{<}0)) \vee \\
         & ([y{\leq}0{\,\wedge\,}F]\cdot (y{\leq}0)^\omega).
\\
\Phi_{\m{final}} = & ([x{<}0]\cdot \epsilon) \vee 
([x{\geq}0{\,\wedge\,}y{>}0] \cdot (x{<}0)) \vee \\
& ([x{\geq}0{\,\wedge\,}y{\leq}0] \cdot (y{>}0)\cdot (x{<}0)).
\end{align*}
}


At this point, we can observe that the loop terminates, as all paths in the summary are terminating. This summary is established by discovering the two-phrase linear ranking function, \ie \code{\langle \text{-}y, x \rangle}, which indicates that the loop decreases the value of \code{\text{-}y} to negative first and then decreases the value of \code{x} to negative, which leads to the exit from the loop. 
Intuitively, loops are handled by iteratively computing the guarded terminating and repetitive cases. The guards of the repetitive cases can then discover the earlier phrases' CRFs. 
\toolName next obtains a (non-)termination condition with respect to the CRFs and constructs the final loop summary with the (non-)termination condition. 

\end{comment}



\subsection{From Guarded~\code{\omegaRE} to Datalog Programs} 
\label{sec:GuardedomegaREtoDatalog}
There are two tasks for generating a Datalog program given a Guarded~\code{\omegaRE} $\effect$, and a CTL property $\phi$: 
produce the rules for conditional flows, and map concrete program states into abstract predicates in the form of facts. 
%and generate the corresponding facts. 
First, we provide the definitions of the deployed auxiliary functions. 
%\emph{Nullable}(\code{\delta}), 
%\emph{First}(\code{\m{fst}}), and \emph{Derivative}(\code{\deri}). 
Informally, the \emph{Nullable} function \code{\delta(\effect)} 
returns a Boolean value indicating whether \code{\effect} 
contains the empty trace; 
the \emph{First} function \code{\m{fst}( \effect)} computes a set of 
possible initial trace segments from \code{\effect};
the \emph{Derivative} function \code{\deri_{f}(\effect)} eliminates a segment \code{f} from the head of \code{\effect} and returns what remains. 


{
\begin{definition}[Nullable]\label{Nullable}
Given any \code{\effect}, we define \code{\delta(\effect)} as follows: ({\code{\m{false}} for unmentioned constructs})
{ 
\vspace{-1mm}
\begin{gather*}
\delta(\epsilon) {=}\m{true}
\qquad\quad 
\delta(\effect_1 {\,\cdot\,} \effect_2) {=} \delta(\effect_1) {\,\wedge\,} \delta(\effect_2)
\\ 
\delta(\effect_1 {\,\vee\,} \effect_2) {=} \delta(\effect_1) {\,\vee\,} \delta(\effect_2)
\\[-1.2em]
\end{gather*}}
\end{definition}}


{
\begin{definition}[First]\label{First1}
We define \code{\m{fst}(\effect)} to be the set of 
initial segments derivable from a   
\code{\effect} formula. 
{
\vspace{-1mm}
\begin{gather*} 
 \m{fst}(\pi\loc{@s}) {=} \{ \pi\loc{@s}\}
\qquad
 \m{fst}([\pi]\loc{@s}) {=} \{ [\pi]\loc{@s}\}
\\
\m{fst}(\effect_1 {\,\vee\,} \effect_2) {=} \m{fst}({\effect_1}) \cup \m{fst}({\effect_2}) 
\qquad
\m{fst}(\effect^\omega) {=} \{ \effect^\omega \}
\\ 
\m{fst}(\effect_1 {\,\cdot\,} \effect_2) {=} 
\begin{cases}
\m{fst}({\effect_1}) \,{\cup}\, \m{fst}({\effect_2}) & \m{when}\quad  \delta(\effect_1) {=} \m{true}
\\
\m{fst}(\effect_1) & \m{when}\quad  \delta(\effect_1) {=} \m{false}
\end{cases} 
\end{gather*}
}
\end{definition}}




{
\begin{definition}[Derivative]\label{Derivative}
The derivative \code{\deri_{f}(\effect)}
subtracts a trace segment \code{f} from the head of \code{\effect} and returns what remains, %in the form of $\Phi$, 
%denoted by \code{f^{\text{-}1} \effect}, 
defined as follows: 
({\code{\bot} for unmentioned constructs})
%of a trace \code{\effect} 
%wrt a segment \code{f} computes the 
%rest of the trace of 
%effects for the \syh{left quotient}, , 
{
\vspace{-1mm}
\begin{gather*}
\deri_{\pi\loc{@s}}(\pi\loc{@s})  {=} \epsilon 
\qquad 
\deri_{[\pi]\loc{@s}}([\pi]\loc{@s})  {=} \epsilon 
\\ 
\deri_{\effect^\omega}(\effect^\omega)  {=} \epsilon
\qquad
\deri_{f}(\effect_1 {\,\vee\,} \effect_2) {=}
\deri_{f}(\effect_1) {\,\vee\,} \deri_{f}(\effect_2) 
\\
\deri_{f}(\effect_1 \cdot \effect_2) =
\begin{cases}
    (\deri_{f}(\effect_1) \cdot \effect_2) \vee \deri_{f}(\effect_2) \\
    \quad \m{when}\, \delta(\effect_1) = \m{true} \\
    \deri_{f}(\effect_1) \cdot \effect_2 \\
    \quad \m{when}\, \delta(\effect_1) = \m{false}
\end{cases}
\\[-1.2em]
\end{gather*}}

\end{definition}}

%To begin with, w

%all the guards from \code{\effect}, while the  function gathers all predicates that $\phi$ is looking for. 







As shown in 
\figref{fig:WRE_2_Datalog}, 
translation rules are in the form of \code{\RETOD{\prevS}{\pathPure}{\effect}{\Datalog}}, where \code{\Pi} is a context containing a set of abstract predicates, \code{\prevS} is the preceding state, 
and the formula \code{\effect} will be converted into the Datalog program \code{\Datalog}. 
The translation is initially invoked with \code{\Pi{=}\m{Pure}(\effect) \,{\cup}\, \m{Pure}(\phi)},  \code{\prevS{=}\text{-}1} and \code{\pathPure{=}T}. 
We use \code{\m{Pure}(\effect)} to extract the predicates from the guards in \code{\effect}, and use 
\code{\m{Pure}(\phi)} to extract the atomic propositions in \code{\phi}. 
In total, \code{\Pi} gathers all the abstract predicates of interest. % in the analysis. 
%The implementations of \code{\m{Pure}} are detailed in \app{VII}   \cite{tech_report}.   


When the given \code{\effect} contains no \emph{initial} elements, \ie it is already the end of the trace, 
\code{[\toDatalogRule\m{Base}]}  adds a self-transition flow. 
Otherwise, \code{[\toDatalogRule\m{Ind}]} %disjunctively combine 
unions the Datalog programs generated from each initial segment and their derivatives via the relation \code{\RETODHelper{\Pi}{f}{\prevS}{\pathPure}{\effect}{\Datalog}}. 
%shown in \defref{First1}, 
There are three kinds of initial segments: 
When \code{f{=}\effect^\omega}, apart from the Datalog program generated from \code{\effect}, \code{[\toDatalogRule\m{Omega}]}  generates the flow facts 
connecting end and start states of \code{\effect}; 
When \code{f{=}\pi\loc{@s}}, \code{[\toDatalogRule\m{Pure}]} generates a flow rule from the previous state %\code{\prevS} 
to the current state %\code{S} 
and 
generates facts for predicates, which are entailed by the current state, 
denoted by \code{\s{\pi'(\currentS) \mid \forall \pi' {\,\in\,} \Pi~.~ \pi\loc{@s}{\Rightarrow} \pi')}} 
where the implication of 
\code{\pi\loc{@s}{\Rightarrow}\pi} is 
solved by a SMT solver \cite{DBLP:conf/tacas/MouraB08}. 
For example, if the concrete state is \code{y{=}1} at state \code{s} and \code{\Pi} includes \code{y{\geq}1} and \code{y{<}1}, then it generates one fact \lstinline|GtEq("y",1,|\loc{s}\lstinline|)|, since \code{(y{=}1){\Rightarrow}(y{\geq}1)} and \code{(y{=}1){\not\Rightarrow}(y{<}1)}. 
Lastly, when \code{f{=}[\pi]\loc{@s}}, \code{[\toDatalogRule\m{Guard}]} generates a conditional flow from the previous state to the current state, with the premise to be \code{\pi} holds at the previous state. It then continues to generate the Datalog program for the remainder of the trace. 


