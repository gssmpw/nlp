\vspace{-1.5mm}
\section{Overview and Illustrative Examples}

% and patches do exist. 
As shown in \figref{fig:Verification_overview},  
\toolName\ takes the program's control-flow graph (CFG) and a CTL specification and produces safe patches if the property does not hold. 
The main steps are highlighted in the rounded boxes and the arrows around them. 
In particular, we deploy an SMT solver \cite{DBLP:conf/tacas/MouraB08} and an ASP solver \cite{DBLP:books/sp/Lifschitz19} for solving linear arithmetic constraints and deciding the truth assignments 
for Datalog facts. %, respectively. 
The workflow of \toolName\ is as follows: 



\begin{enumerate}[wide]
%,leftmargin=2em
%,itemsep=0.1ex
\item Given any CFG, it is converted 
to our intermediate representation, \ie the guarded \code{\omega}-regular expression (or guarded \code{\omega}-RE), demonstrated in \secref{subsec:progam2WRE}. 
In this process, loops are managed using a summary calculus, as detailed in \secref{subsec:loop2wRE}. Given the undecidable nature of loop termination analysis, the framework outputs ``Unknown" when there exists a path for which we cannot conclusively prove either termination or non-termination. 

\item Given any guarded \code{\omega}-RE, it is translated into a Datalog program, shown in \secref{sec:GuardedomegaREtoDatalog}. 

\item For any CTL property, we convert it into stratified Datalog rules, detailed in \secref{subsec:fromCTL2Datalog}. 
%In particular, we deploy an encoding from \cite{gottlob2002datalog} for the ``AF'' operator (detailed in \secref{subsec:fromCTL2Datalog}), which enables a greatest fixpoint encoding using the least fixpoint semantics of Datalog.


\item The SEDL execution checks CTL properties precisely. 
%In contrast to the recursive labelling process in CTL model checking, Datalog engines solve the queries using a handful of declarative rules, leaving the labelling process implicit. 



\item 
When a given property does not hold, the erroneous Datalog program is sent for repair, as outlined in \secref{sec:program_repair}. The relationship between facts modifications and the source code is as follows: 
Adding facts leads to additional assignments along existing paths.
Updating facts modifies the current assignments on those paths.
Deleting facts necessitates the inclusion of conditional statements for early exits on problematic paths.
Collectively, these modifications form patches that consist of iteratively adding or revising assignments and conditional statements. 


\end{enumerate}


\begin{figure}[!b]
\centering
\vspace{-5mm}
\centering
\begin{lstlisting}[xleftmargin=9em,numbersep=6pt, name=intro,basicstyle=\footnotesize\ttfamily]
void main(){ //AF(y=5)
  int y = 1; 
  int i = *; 
  int x = *;
  if (i > 10) {x = 1;}
  while (x == y) {} 
  y = 5; 
  return;}
\end{lstlisting} 
\vspace{-1mm}
\caption{A Program Fails the CTL Specification}
\label{fig:first_Example}
%\vspace{-2mm}
\end{figure}


\paragraph*{\textbf{CTL Analysis using Datalog}}
\label{sec:example:mixed_abstract_domain}




{\begin{figure*}
\qquad\qquad\qquad 
$(y{=}1)\loc{@1} \cdot (i{=}*)\loc{@2} \cdot (x{=}*)\loc{@3}\, \cdot $ 
$\Biggl(\vee\begin{matrix}
[i{>}10]\loc{@4} \cdot (x{=}1)\loc{@5} 
\cdot
\begin{matrix}
\Bigl(\vee 
\begin{array}{l}
[x{\not=}y]\loc{@7}  \cdot (y{=}5)\loc{@11}
\\[0em]
[x{=}y]\loc{@8} \cdot ((x{\geq}y)\loc{@12})^\omega
\end{array}
\Bigr)
\end{matrix}
\\[1em]
\qquad\ \  
[i{\leq}10]\loc{@6}
\cdot \qquad \ \  
\begin{matrix}
\Bigl(\vee 
\begin{array}{l}
[x{\not=}y]\loc{@9}  \cdot (y{=}5)\loc{@11}
\\[0em]
[x{=}y]\loc{@10} \cdot ((x{\geq}y)\loc{@12})^\omega
\end{array}
\Bigr)
\end{matrix}
\end{matrix}\Biggr)$
\caption{The Guarded \code{\omega}-RE Representation, $\Phi_{\m{main}}$  (\loc{@n} are uniquely assigned state numbers)}
\label{fig:omegaRE_first_Example}
\vspace{-1mm}
\end{figure*}}




The program depicted in \figref{fig:first_Example} initiates by assigning the value \code{y{=}1} while allowing the variables \code{i} and \code{x} to assume any values. Here, the symbol \code{*} makes all the nondeterminism explicit. Following this initialization, the program executes a conditional statement, which includes an assignment of \code{x{=}1}. Subsequently, the program enters a while loop, and once the while loop is entered, it results in infinite execution. Finally, before returning, it assigns the value \code{y{=}5}. 
There are three symbolic paths: (1) When \code{i{>}10}, it enters the infinite loop; (2) When \code{i{\leq}10 \,{\wedge}\, x{=}y}, it also enters the infinite loop; and (3) When \code{i{\leq}10 \,{\wedge}\, x{\not=}y}, it terminates normally with \code{y{=}5}.  
The CTL property $\phi$ of interest is ``\code{AF(y{=}5)}'', stating that \emph{``for all paths, finally \code{y}'s value is 5''}. However, the current implementation fails to satisfy $\phi$ as the first two paths (1) and (2) never reach the state where \code{y{=}5}. 




To achieve a more precise CTL analysis for such infinite-state programs, we propose representing the program using Datalog facts and rules and leveraging the Datalog execution for sound and complete CTL checking. 
Specifically, we first convert programs into an intermediate representation, i.e., $\Phi_{\m{main}}$,  shown in \figref{fig:omegaRE_first_Example}, where \code{[\pi]} denotes a guard upon pure constraints. These guards are derived from conditional, loop guards and assertions. For example, the loop in line 5 is summarized into the following disjunction \code{([x{\not=}y] \cdot \epsilon) \vee ([x{=}y] \cdot (x{\geq}y)^\omega)}, where ``$\epsilon$'' denotes an empty trace while ``$\omega$'' denotes an infinite trace.  
This summary over-approximates the behaviours of the loop symbolically: when \code{x{\not=}y} it terminates, and otherwise it infinitely repeats the state \code{x{\geq}y}. 




\begin{figure}[!b]
\vspace{-3mm}
\begin{lstlisting}[xleftmargin=1em,numbers=none,basicstyle=\footnotesize\ttfamily]
// Abstract Predicates 
Gt(i,10,(*@\loc{2}@*)). LtEq(i,10,(*@\loc{2}@*)). Eq(y,5,(*@\loc{11}@*)). 
EqVar(x,y,(*@\loc{3}@*)). NotEqVar(x,y,(*@\loc{3}@*)). EqVar(x,y,(*@\loc{5}@*)).
// Persistent Transitions
flow((*@\loc{1}@*),(*@\loc{2}@*)). flow((*@\loc{2}@*),(*@\loc{3}@*)). flow((*@\loc{4}@*),(*@\loc{5}@*)). 
flow((*@\loc{7}@*),(*@\loc{11}@*)). flow((*@\loc{8}@*),(*@\loc{12}@*)). flow((*@\loc{9}@*),(*@\loc{11}@*)). 
flow((*@\loc{10}@*),(*@\loc{12}@*)). flow((*@\loc{11}@*),(*@\loc{11}@*)). flow((*@\loc{12}@*),(*@\loc{12}@*)). 
// Conditional Transitions
flow((*@\loc{3}@*),(*@\loc{4}@*)) :- Gt(i,10,(*@\loc{3}@*)).
flow((*@\loc{3}@*),(*@\loc{6}@*)) :- LtEq(i,2,(*@\loc{3}@*)).
flow((*@\loc{5}@*),(*@\loc{7}@*)) :- NotEqVar(x,y,(*@\loc{5}@*)).
flow((*@\loc{5}@*),(*@\loc{8}@*)) :- EqVar(x,y,(*@\loc{5}@*)).
flow((*@\loc{6}@*),(*@\loc{9}@*)) :- NotEqVar(x,y,(*@\loc{6}@*)).
flow((*@\loc{6}@*),(*@\loc{10}@*)):- EqVar(x,y,(*@\loc{6}@*)).
\end{lstlisting}
\caption{The (Simplified) Datalog Representation}
\label{fig:Datalog_first_Example}
\end{figure}

\begin{figure}[!b]
\begin{lstlisting}[xleftmargin=1em,numbers=none,basicstyle=\footnotesize\ttfamily]
yEQ5(S) :- Eq(y,5,S).

AFT_yEQ5(S,S1) :- !yEQ5(S), flow(S,S1). 
AFT_yEQ5(S,S1) :- AFT_yEQ5(S,S2), !yEQ5(S2), 
                  flow(S2,S1).
                   
AFS_yEQ5(S) :- AFT_yEQ5(S,S).
AFS_yEQ5(S) :- !yEQ5(S),flow(S,S1),AFS_yEQ5(S1).
               
AF_yEQ5(S):- State(S), !AFS_yEQ5(S).
\end{lstlisting} 
\caption{Datalog Rules for ``\code{AF(y{=}5)}''}
\label{fig:first_Example_ctl_rules}
\end{figure}

Next, from $\Phi_{\m{main}}$, the generated Datalog program is outlined in \figref{fig:Datalog_first_Example}. 
The facts are generated concerning the abstract states which lead to different paths, namely, whether \code{i{>}10} and whether \code{x{=}y}. 
Also, since $\phi$ concerns the {reachability} of \code{y{=}5}, we generate Datalog facts \wrt the truth values of the following predicates: \code{i{>}10}, \code{i{\leq}10}, \code{x{=}y}, \code{x{\not=}y}, and \code{y{=}5}, which are abstracted using facts
\lstinline|Gt|, \lstinline|LtEq|, \lstinline|EqVar|, \lstinline|NotEqVar|, \lstinline|Eq|, respectively.  
Abstract predicates are over program variables, constants, and state numbers (marked in {\loc{red}}). 
The predicate \lstinline|flow| represents the control flows; 
and some of them are persistent such as \text{\lstinline|flow(|\loc{1}, \loc{2}\lstinline|)|}, 
whereas some of them only exist if certain promises are satisfied, such as \text{\lstinline|flow(|\loc{3}, \loc{4}\lstinline|)|}, which only occurs when (transitivity) \code{i} is greater than \code{10} at state \loc{3}. 
Moreover, as we target infinite-states programs, it is standard to 
encode finite traces into infinite traces by adding self-transition flows at last states, such as
\text{\lstinline|flow(|\loc{11}, \loc{11}\lstinline|)|}. 
The Datalog query rules generated for $\phi$ are shown in \figref{fig:first_Example_ctl_rules}, and after executing the Datalog engine, the expected output fact $\relation$ is \lstinline|AF_yEQ5(|\loc{1}\lstinline|)|, which would indicate that $\phi$ holds at state \loc{1} -- the starting point of the program.  

Since the variables \code{i} and \code{x} can assume any values, all possible execution paths (1), (2), and (3) are enabled. Consequently, after executing the Datalog engine, it fails to produce $\relation$. This result indicates that the program does not satisfy the specified property. At this stage, \toolName successfully \emph{disproves} the property, paving the way for subsequent repairs to be implemented. 

%, following the encoding in \cite{gottlob2002datalog}. 


\paragraph*{\bf \em Need for Negation}
The semantics of Datalog is defined by least fixed point semantics, while greatest fixed point properties can be represented using negation over Datalog predicates. CTL properties involve a combination of least and greatest fixed point properties, \eg consider \code{AGAF\varphi}, which can be encoded in Datalog through stratified negation. 

\begin{figure}[!b]
\vspace{-1mm}
\centering
% First figure
\begin{lstlisting}[xleftmargin=0.3em,numbers=none,basicstyle=\footnotesize\ttfamily,mathescape]
$\xi_1\,$Gt(i,10,(*@\loc{2}@*)). $\xi_2\,$LtEq(i,10,(*@\loc{2}@*)).
$\xi_3\,$Eq(y,5,(*@\loc{11}@*)). $\xi_4\,$EqVar(x,y,(*@\loc{3}@*)). 
$\xi_5\,$NotEqVar(x,y,(*@\loc{3}@*)). $\xi_6\,$EqVar(x,y,(*@\loc{5}@*)). $\xi_7\,$Eq($\alpha_1$,$\alpha_2$,(*@\loc{$\alpha_3$}@*)).
\end{lstlisting}
\caption{Symbolic Facts}
\label{fig:symbolicEDBexample}
\vspace{-1mm}
\end{figure}

 
  % Second figure
\begin{figure}[!b]
\begin{align*}
\psi \triangleq& 
  (\neg \xi_1 \wedge \xi_2 \wedge \xi_3 \wedge \xi_4 \wedge \xi_5 \wedge \neg \xi_6 \wedge \neg \xi_7) \ \vee\, 
  \\
  & (\xi_1 {\wedge\,} \xi_2 {\wedge\,} \xi_3 {\wedge\,} \xi_4 {\wedge\,} \xi_5 {\wedge\,} \xi_6 {\wedge\,} \xi_7 {\wedge\,} \alpha_1{=}y {\wedge\,} \alpha_2 {=} 5 {\wedge\,} \alpha_3 {=} 12) \ \vee\\
  &  \dots
  \end{align*}
\vspace{-1mm}
\caption{Logical Constraints Enabling \lstinline|AF_yEQ5(|\loc{1}\lstinline|)| }
\label{fig:SEDLLogical_Constraints}
\end{figure}






\paragraph*{\textbf{Repairing Property with Negation via Extended SEDL}}
\label{sec:example:protocol}
Continuing with the earlier example, given that the expected output fact \code{\relation} {=} \lstinline|AF_yEQ5(|\loc{1}\lstinline|)|, \toolName converts input facts to \emph{symbolic input facts} by injecting symbolic constants \code{\alpha_1, 
\alpha_2,\alpha_3}, and symbolic signs \code{\xi_1, \dots \xi_7}. 
A symbolic constant, \code{\alpha}, represents an unknown value from its domain (\eg, integers, strings), while a symbolic sign, \code{\xi}, represents an unknown Boolean value indicating the presence of its associated fact. 
After injecting symbols, those facts that contain symbolic constants or signs are shown in \figref{fig:symbolicEDBexample}.
% The complete set of symbolic input facts is the union of those shown in \figref{fig:symbolicEDBexample} and other facts without symbolic constants or signs.
By applying the rules outlined in \figref{fig:Datalog_first_Example} and \figref{fig:first_Example_ctl_rules} to these symbolic facts and the unchanged facts, our SEDL generates the \emph{logical constraints} \code{\psi} for \code{\relation}, as shown in \figref{fig:SEDLLogical_Constraints}. 
Each disjunctive case corresponds to a patch that enables the generation of \code{\relation}. 
The satisfying assignment of case (b) introduces the fact \lstinline|Eq(y,5,|\locref{12}\lstinline|)|, which effectively adds the assignment \plaincode{y=5} along the existing path, specifically within the body of the while loop. This patch successfully allows the program to pass the CTL analysis. 
%, shown in \figref{fig:pathed-prorgam},  


% \begin{figure}[!t]
% \centering
% \begin{minipage}{0.37\textwidth}
% \begin{lstlisting}[xleftmargin=0.2em,numbers=none,basicstyle=\footnotesize\ttfamily,mathescape]
% $\xi_1\,$Gt(i,10,(*@\loc{2}@*)). $\xi_2\,$LtEq(i,10,(*@\loc{2}@*)). 
% $\xi_3\,$Eq(y,5,(*@\loc{11}@*)). $\xi_4\,$EqVar(x,y,(*@\loc{3}@*)). 
% $\xi_5\,$NotEqVar(x,y,(*@\loc{3}@*)). 
% $\xi_6\,$EqVar(x,y,(*@\loc{5}@*)). $\xi_7\,$Eq($\alpha_1$,$\alpha_2$,(*@\loc{$\alpha_3$}@*)).
% \end{lstlisting}
% \vspace{-3mm}
% \caption{Symbolic Facts}
% \label{fig:symbolicEDBexample}
% \end{minipage}
% \begin{minipage}{0.62\textwidth}
% \small 
% \begin{align}
% \psi \triangleq& 
% (\neg \xi_1 \wedge \xi_2 \wedge \xi_3 \wedge \xi_4 \wedge \xi_5 \wedge \neg \xi_6 \wedge \neg \xi_7) \ \vee\, 
% \\
% & (\xi_1 {\wedge\,} \xi_2 {\wedge\,} \xi_3 {\wedge\,} \xi_4 {\wedge\,} \xi_5 {\wedge\,} \xi_6 {\wedge\,} \xi_7 {\wedge\,} \alpha_1{=}y {\wedge\,} \alpha_2 {=} 5 {\wedge\,} \alpha_3 {=} 12)\\
% \vee\, \dots
% \end{align}
% \caption{Logical Constraints Enabling \lstinline|AF_yEQ5(|\loc{1}\lstinline|)| }
% \label{fig:SEDLLogical_Constraints}
% \end{minipage}
% \vspace{-3mm}
% \end{figure}




\begin{comment}
\begin{figure}[!h]
\vspace{-4mm}
\begin{lstlisting}[firstnumber=1, xleftmargin=2em,numbersep=10pt,basicstyle=\footnotesize\ttfamily]
void main(){ //AF(y=5)
  int y = 1; 
  int i = *; 
  int x = *;
  // Patch (a) 
  (*@\repaircode{if (i>10 || x{=}=y) {\{y = 5; return;\}}}@*) 
  if (i > 10) {x = 1;}
  while (x == y) {
    (*@\repaircode{y = 5;}@*) } // Patch (b)
  y = 5; return; }
\end{lstlisting} 
\vspace{-2mm}
\caption{Patch Options, Revised from \figref{fig:first_Example}}
\label{fig:Patched-program}
\vspace{-2mm}
\end{figure}
\end{comment}


The satisfying assignment of (a) drops the newly introduced symbolic fact \lstinline[mathescape]|Eq($\alpha_1$,$\alpha_2$,|\locref{$\alpha_3$}\lstinline|)|, and deletes the existing facts: \lstinline|Gt(i,10,|\locref{2}\lstinline|)| and \lstinline|EqVar(x,y,|\locref{3}\lstinline|)|, 
indicating that neither should \code{i} be greater than 10 at state 2 nor should \code{x} equal \code{y} at state 3. 
The deleted facts suggest that when the condition \code{i{>}10{\,\vee\,}x{=}y} is met, the program fails to satisfy the intended property. As a result, during the first iteration of the repair process, a conditional statement is added: \plaincode{if (i>10||x==y) return;}. This line is placed before the main logic of the program to prevent it from following the problematic execution path represented by the removed facts. 
However, the currently patched program still does not satisfy the condition \code{AF(y{=}5)} because the newly added path does not reach the state where \code{y{=}5}. Therefore, during the second iteration, the analysis shows that the property does not hold. As a result, the repair inserts the statement \code{y{=}5} before the return statement, similar to the generation of (a).  
These iterative processes ultimately lead to the correct conditional patch being added. Both source code level patches, referred to as Patch (a) and Patch (b), are illustrated in \figref{fig:Patched-program}.  


\begin{figure}[!h]
\vspace{-2mm}
\begin{lstlisting}[firstnumber=4, xleftmargin=5em,numbersep=10pt,basicstyle=\footnotesize\ttfamily]
// Patch (a) 
(*@\repaircode{if (i>10 || x{=}=y) {\{y = 5; return;\}}}@*) 
if (i > 10) {x = 1;}
while (x == y) { (*@\repaircode{y = 5;}@*)}  // Patch (b)
y = 5; return; }
\end{lstlisting} 
%\vspace{-1mm}
\caption{Patch Options, Revised from \figref{fig:first_Example}}
\label{fig:Patched-program}
\end{figure}



\noindent
{\bf \em Remark.} While simple, this example showcases our technical contributions in the following two aspects: loop summarization for both terminating and non-terminating behaviours and achieving the SEDL that accommodates stratified negations. 
Moreover, we define how to interpret fact modifications at the Datalog level into meaningful and general source-code level patches. 
%We also focus on selecting patches that result in the fewest modifications to the program. 
To our knowledge, this work is the first to generate repairs based on a novel CTL analysis. 
%These are the main concerns of this paper and have not been addressed by any prior works. 
