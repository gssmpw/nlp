\section{Introduction}
\label{sec:intro}

\vspace{-3mm}
%In real-world code, \syh{CTL can in principle be used to specify many properties that real-world projects care about} many instances of CTL properties are commonly specified to prevent bugs 
Computational Tree Logic (CTL) is based on a branching notion of time -- at each moment, there may be several different possible futures -- which is sufficiently expressive to formulate a rich set of properties for infinite-state programs, such as reactive systems. 
CTL can specify many properties that real-world projects care about to prevent bugs, such as non-termination \cite{DBLP:conf/sigsoft/ShiXLZCL22}, in the form of \code{\m{AF}(Exit())}; and unresponsive behaviours \cite{DBLP:conf/icse/MengDLBR22}, in the form of \code{\m{AG}(\phi_1{\rightarrow} \m{AF}\,\phi_2)}, \eg whenever an error occurs, the server will eventually respond with the corresponding error code. 
Here, \code{\phi_1} and \code{\phi_2} are the CTL sub-formulae; \code{A} and \code{E} are universal/existential quantifiers over the execution paths, and \code{F} and \code{G} stand for \emph{finally} and \emph{globally}, respectively.  

Typically, when a program fails to satisfy a CTL property, developers must examine counterexample traces identified by a model checker and manually fix them iteratively. 
Here, we propose a mechanism that, instead of requiring iterative fixes, deals with all counterexamples at once and automatically. 
To realize this vision, we propose a Datalog-based framework that encodes the given program's control flow into Datalog rules, the abstract program states into Datalog facts, and the CTL checking into Datalog query rules.
The presence of the expected output fact indicates whether the program satisfies the specified property. 
The repair is then achieved by modifying the input facts to allow query rules to output the expected fact. 
We chose Datalog for its inherent purity, which sufficiently captures the entire spectrum of CTL properties. Additionally, its symbolic execution capability (SEDL)~\cite{DBLP:conf/sigsoft/LiuMSR23} can identify potential modifications to the input facts that influence the output facts, thereby repairing the code to satisfy the given property. 



Specifically, symbols denoting unknown constants and the truthfulness of facts are injected into the input facts. 
The outcome of SEDL on these symbolic inputs summarizes the logical constraints over the symbols that enable the output.
Then, any valuation of symbols that produces the desired query output corresponds to a patch.
This allows us to repair all the violations of the property at once and find all possible repairs within the defined search space.
%In contrast, existing \emph{model repair} approaches for CTL bugs do not deal with real-life code directly, and 
%either require re-analysis~\cite{DBLP:journals/ai/BuccafurriEGL99, martinez2015ctl}, or are limited to specific repair templates~\cite{DBLP:conf/fossacs/AttieC23, DBLP:conf/ecai/DingZ06}. 
Besides, the relation between CTL and Datalog is longstanding~\cite{gottlob2002datalog}, showing that the semantics of CTL properties -- nested least and greatest fixed points -- can be readily supported by Datalog with stratified negation without introducing approximation.
However, the challenges are: the current SEDL supports only positive Datalog programs, and it is unclear how to precisely handle loops in a CTL analysis. 

\begin{figure*}
%\vspace{-2mm}
\centering
\includegraphics[width=0.75\linewidth]{pics/system_overview.png}
\caption{\label{fig:Verification_overview}System Overview of \toolName}
\vspace{-1mm}
\end{figure*}


To address the first challenge, we extend SEDL to support stratified negation. SEDL operates by over-approximating the domain of symbolic constants, then using delta-debugging~\cite{zeller1999yesterday} to select the dependent input facts that enable the expected output fact. 
There are two reasons why this is limited to positive Datalog: (i) its over-approximation only handles positive rules; and (ii) delta debugging requires monotonicity, which is only applicable to positive rules.
In this work, we propose a new solution in which the over-approximation takes account of stratified negations. It uses an Answer Set Programming (ASP) solver that works with non-monotonic rules, and the generated fact modifications are sound by construction, \ie they lead to the expected output results. 
%%the presence of 




To address the second challenge, we propose summarizing programs, including the loops, using an intermediate representation, denoted by $\effect$, which is a \emph{guarded \code{\omega}-regular language}. 
Unlike existing loop summarization techniques, which do not explicitly capture non-terminating behaviours, $\effect$ enables us to 
capture both terminating and non-terminating behaviours in a (guarded) disjunctive form.  
Existing CTL analyzers suffer from imprecision in their termination analysis, primarily due to limitations of ranking function synthesis techniques~\cite{DBLP:conf/fmcad/CookKP14, DBLP:conf/sas/UrbanU018}. 
%Unlike existing loop summarization techniques, which do not explicitly capture non-terminating behaviours, our loop summarization summarizes both terminating and non-terminating behaviours. 
We show that our novel loop summarization enables an effective \emph{linear ranking function} \cite{DBLP:conf/sas/Ben-AmramDG19} generation by inspecting the guards of the disjunctive summaries. 
Next, the control flow and abstract program states represented by $\effect$ can be encoded using a Datalog program, leveraging the Datalog execution for CTL property queries. 
Experimental results show that our tool \toolName can significantly improve the precision of the CTL checking. 
%due to our loop summarization.
%Overall, we consider the following repair problem \cite{DBLP:conf/memocode/AttieCBSS15}: given a finite structure but infinite-states  $\effect$ and a CTL specification $\phi$, determine if $\effect$ can be modified to  $\effect^\prime$ that satisfies $\phi$. %Since $\effect$ describes transitions among abstract program states, we offer two generic and composable repair strategies: (i) modifying the facts that influence the transitions so that they choose the transitions that satisfy $\phi$; and (ii) modifying the facts that indicate the program states so that the updated state transitions satisfy $\phi$. When mapped to the source level, they correspond to the modification of if-conditions and the insertion/deletion of assignments, respectively. 
%these modifications
%However, repair by adding the control flow rules is not primarily supported, as doing so does not seem to be any better than adding to the structure $\effect$ and then regenerating the repair solutions using the existing strategies. 
Our contributions are: 




\begin{itemize}[leftmargin=0.35cm]

\item 
We define a find-and-fix process which takes in a program and a CTL property and returns the repaired program if the property is violated.



\item We extend SEDL to support stratified negations, enabling repair guided by CTL properties involving a mixture of the least and greatest fixpoint-defined analysis. 


\item We develop a new approach for handling loops by calculating both termination and non-termination summaries to prove both safety and liveness properties, which improves the precision of the analysis. 



%Our novel loop summaries show advantages in finding ranking functions by stages. 
%and makes it possible to \syh{adjust} the precision of the CTL analysis. 

\item We prototype our proposal and evaluate our tool \toolName on two benchmarks, including a set of small-scale examples with complex loops and an extensive real-world benchmark with widely used C library repositories and protocol implementations. 
Our experimental results show that \toolName outperforms state-of-the-art tools in detecting and fixing CTL bugs. 

\end{itemize}
%benchmark consisting of 


