\section{Preliminary}
In this section, we introduce the background knowledge of the techniques we use in this paper.

\subsection{Computational Tree Logic and Datalog}


\begin{figure}[!b]
{
\centering
\renewcommand{\arraystretch}{1}
$
\arraycolsep=1.6pt
\begin{array}{lrll}
\m{(CTL)}  & \phi &{::=}&   
ap 
{~\mid~} \neg\phi
{~\mid~} \phi_1 {\wedge} \phi_2
{~\mid~} \phi_1 {\vee} \phi_2
{~\mid~} EF\phi 
{~\mid~}  
\\
&&& \m{EX}\phi 
{~\mid~} AF\phi 
{~\mid~} E(\phi_1 {U} \phi_2)
{~\mid~} \phi_1 {\rightarrow} \phi_2 
{~\mid~} 
%}
\\
&&& 
\m{AX}\phi 
{~\mid~} AG\phi 
{~\mid~} EG\phi 
{~\mid~} A(\phi_1 {U} \phi_2) 
\\ 
\m{(AP)}  & ap &{::=}&  \ 
(\nm, \pi)
\\
\m{(Pure)} &    \pi &{::=}&   
{ \m{T}}
{\,\mid\,}  \m{F}
 %\mid  {\m{bop}(}{x, t}{)}
{\,\mid\,} bop(t_1, t_2)
 %\mid Lt(t_1, t_2)
 %\mid GtEq(t_1, t_2)
 %\mid LtEq(t_1, t_2)
 %\mid Eq(t_1, t_2)
{\,\mid\,} \relation 
{\,\mid\,}   {{\pi_1}}{\wedge}  \pi_2
{\,\mid\,}  {{\pi_1}} {\vee} \pi_2 
\\    
\m{(Terms)} & t  &{::=}&  \ 
v
{~\mid~} \_ 
{~\mid~}  t_1{\text{\ttfamily +}}t_2
{~\mid~}  \text{-}t 
\\
\m{(Relation)}&  \relation &{::=}& 
\nm\,(v^*)
\end{array}$
\caption{CTL Syntax}
\label{fig:Syntax_of_CTL}
}
\end{figure}

Computational Tree Logic (CTL) is a branching-time logic for reasoning about multiple possible execution paths. As shown in \figref{fig:Syntax_of_CTL}, we capture each atomic proposition as a pure formula $\pi$ with a unique identifier $\nm$. 
%simplified and decidable set of Presburger 
Pure formulas are quantifier-free arithmetic predicates over program variables and constants. 
%The Boolean values of \emph{true} and \emph{false} are respectively indicated by \code{T} and \code{F}. 
Binary operators are represented using predicates, where \emph{bop}\code{\,{\in}\,\s{>, <, \geq, \leq, =}}. 
Other uninterpreted relations are represented using relational predicates \code{\relation}. 
A predicate has a name (\code{\nm}) and a set of arguments (\code{v^*}). 
%where \code{\nm} is an arbitrary distinct identifier. 
Terms consist of simple values, 
the wild card \_, 
and simple computations of terms. 
%, such as \code{t_1{+}t_2} and \code{t_1\text{-}t_2}. 
Each temporal operator contains a quantifier over paths: 
``\code{A}" means \emph{for all the paths}, while ``\code{E}" means \emph{there exists a path}; and a linear temporal logic \cite{DBLP:conf/focs/Pnueli77} operator: ``\code{F}" for finally, 
``\code{G}" for globally, ``\code{U}" for until, and 
``\code{X}" for next time. 
We use the $*$ superscript to denote a finite set of items. 




\begin{figure}[!h]
{
\centering
\renewcommand{\arraystretch}{1}
$
\arraycolsep=3pt
\begin{array}{lrll}
\m{(Datalog)} &   \Datalog & 
{::=}& \relation^*  \concat    \drule^* 
\\
\m{(Rule)}&     \m{\drule} &{::=}&  
\relation ~\datalogarrow~ L^*
\\
\m{(Literal)} &    \m{L} &{::=}& 
\relation
{~\mid~}  %{\tt{Neg}}
\shortNeg\,\relation 
\\
\m{(Relation)} & \m{\relation} &{::=}& p(v^*) 
\\
\m{(Arguments)}&  v &{::=}& 
c  {~\mid~} X
\end{array}$
\caption{A Core Syntax of Datalog}
\label{fig:Syntax_of_Datalog}
}
\end{figure}

The core syntax of Datalog is shown in \figref{fig:Syntax_of_Datalog}. 
A Datalog program consists of a set of facts (\code{R^*}) and rules (\code{\m{\drule^*}}). 
Arguments are constants ($c$), or program variables ($X$).
% An atom consists of a predicate (p) and a list of arguments.
% \code{p} is a predicate (relation name).
% Facts are represented as a relations.   
% A Datalog rule is a Horn clause that comprises a head predicate (atom) and a set of body predicates (literals) placed on the left and right side of the arrow symbol (\datalogarrow).
A Datalog rule is a Horn clause that comprises a head literal (an atom) and a set of body literals, with the head on the left side and the body on the right side of the arrow symbol (\datalogarrow).
A fact is a rule with an empty body, \ie it is unconditionally true. 
A Datalog query is executed against a database of facts, known as the \emph{extensional database} (EDB), and produces a set of derived facts, known as the \emph{intensional database} (IDB). 
Unification in Datalog is a pattern-matching operation determining whether two arguments can be made identical through substitution. Arguments unify if they are identical constants or if one is a variable. 
The semantics of Datalog is based on the least fixed point computation, where facts are iteratively derived by applying rules to existing facts until no new facts can be generated. 
Datalog with stratified negation can be partitioned into a finite number of Datalog programs, capturing the different strata. 
If the rule producing the \code{\relation} contains \code{\relation^\prime} negated in the body, then \code{\relation} and \code{\relation^\prime} are in different partitions, and \code{\relation} is in a higher strata than \code{\relation^\prime}. 
The least fixed point of the lower strata is computed first and then used to compute the least fixed point in the higher strata. 


\subsection{Symbolic execution of Datalog}
Symbolic Execution of Datalog (SEDL) aims to identify how varying database values impact query results by executing Datalog queries on a symbolic database, representing a range of concrete databases. 
SEDL uses symbolic constants and signs. 
Symbolic constants represent unknown arguments in facts, such as \lstinline[mathescape]{flow($\alpha$,$\beta$)}, where \code{\alpha} and \code{\beta} are symbolic constants over a finite domain of state numbers. 
Symbolic signs, denoted as $\xi$ are Boolean values indicating the presence of facts. 
Collectively, these symbols form \code{\Sigma{\,\triangleq\,} \{\sigma_1, {\dots}, \sigma_n\}} over domains \code{D_1, ..., D_n}, ranging from  integers, Booleans, strings. 
Any concrete \emph{valuation} of \code{\Sigma}, is represented as \code{\s{\sigma_1{\,=\,}v_1, ..., \sigma_n{\,=\,}v_n}}, where \code{
\forall\,i{\,\in\,}\s{1{\dots}n}.\  
v_i\,{\in}\, D_i}. 
We use \code{\psi^*} to represent all the \emph{logical constraints} for \code{\Sigma}. 
Given any \code{\psi}, its satisfying assignment is a concrete valuation of \code{\Sigma}. 
A symbolic EDB, denoted by \code{\SE}, is a set of input facts that contain symbolic constants and signs. 
The concretization of a symbolic EDB is obtained by applying a concrete valuation. 
The symbolic execution upon any symbolic EDB produces a set of pairs, and each pair contains an output fact \code{\relation} and the corresponding logical constraints enabling the generation of \code{\relation}. 
More specifically, it takes \code{\SE} and returns \code{{(\relation\times\psi^*)^*}}.

\subsubsection{Implementation}
SEDL is implemented in \Symlog~\cite{DBLP:conf/sigsoft/LiuMSR23}, which uses a meta-programming approach. Specifically, given a Datalog query and a symbolic EDB, it converts them into a meta-query and a meta-EDB so that executing the meta-query on the meta-EDB with standard Datalog semantics produces the symbolic execution output of the original query on the original symbolic database.
Consider the first rule in \figref{fig:first_Example_ctl_rules}: \lstinline[mathescape]`AF_yEQ5(S) :- Eq(y,5,S).` When symbolic constants \code{\alpha_1, ..., \alpha_n} are injected into the original EDB, \Symlog converts this rule into the following meta-query: (where \code{C_1, ..., C_n} are auxiliary variables)
\begin{lstlisting}[mathescape, xleftmargin=0em,numbers=none,basicstyle=\footnotesize\ttfamily]
AF_yEQ5(S, $C_1$, ..., $C_n$) :- Eq(y,5,S, $C_1$, ..., $C_n$).
\end{lstlisting}
\noindent Each concrete instantiation of \code{C_1, ..., C_n} is a  concrete valuation of \code{\alpha_1, ..., \alpha_n}.
% \ly{\Symlog is designed to run different assignments of symbolic constants simultaneously, instead of running the query on each EDB with a single assignment individually.
% So, the auxiliary variables are used to distinguish the facts derived from different assignments.}
Given a symbolic EDB fact \lstinline[mathescape]`Eq(y,5,$\alpha_1$)`, it is transformed into the following meta-EDB rule: 
(where the predicate \lstinline[mathescape]`domain_alpha_$i$` is true for all values from the domain of \code{\alpha_i})
%(which essentially is a rule) as follows: 
\begin{lstlisting}[mathescape, xleftmargin=0em,numbers=none,basicstyle=\footnotesize\ttfamily]
Eq(y,5,$C_1$, $C_1$, ..., $C_n$) :- domain_alpha_1($C_1$),...,domain_alpha_n($C_n$).
\end{lstlisting}


\Symlog computes the constraints in two steps:
(1) Constraints over Symbolic Constants.
When the meta-query is evaluated on the meta-EDB, each value of \code{C_i} represents a possible instantiation of the corresponding symbolic constant \code{\alpha_i}, collectively forming the constraints over the symbolic constants.
The output facts with the same assignments of auxiliary variables share the same constraints over the symbolic constants.
While a symbolic constant could theoretically take any value from an infinite domain (such as integers or strings), in practice, we only need to consider values that could potentially match with terms in the existing facts through unification.
\Symlog overapproximates the domain of a symbolic constant by analyzing how values flow through the Datalog program, tracking dependencies between predicates and constants to determine which values each symbolic constant could potentially take during execution.
This over-approximation ensures that all possible instantiations of a symbolic constant are considered. 
(2) Constraints over symbolic signs.
For output facts sharing the same symbolic constant constraints, \Symlog uses delta-debugging~\cite{zeller1999yesterday} to identify the sets of input facts (annotated with symbolic signs) that are necessary to derive each output fact.
Delta-debugging is a divide-and-conquer technique identifying the minimal subset of inputs necessary to produce a target output.
The symbolic signs of the identified input facts must be true, while the values of other symbolic signs are false. 


Given each output fact associated with concrete instantiation of the auxiliary variables, the constraint \code{\psi} %for the corresponding actual output fact 
is constructed by the conjunction of two types of constraints: those over symbolic signs and those over symbolic constants.
The final constraints for the expected output fact are formed by taking the disjunction of all such \code{\psi}, resulting in the pair \code{(R, \psi^*)}.


\subsubsection{Datalog Repair}
Datalog is widely used in program analysis, where EDB represents the analyzed program, and IDB represents the analysis results. 
SEDL guides program repair in several ways. For instance, if SEDL produces  \code{(\relation, \psi^*)}, and \code{\relation} indicates the expected results, then a satisfying assignment of \code{\psi^*} suggests a patch that enables the desired output. Conversely, if \code{\relation} indicates a bug, the satisfying assignment of \code{\neg\psi^*} points to a patch that can eliminate the bug. 


\subsubsection{Limitation of the Existing SEDL}
\Symlog only supports positive Datalog. 
% , where monotonicity holds, meaning that when input facts increase, the output of a rule must not decrease. 
For symbolic constants, its domain approximation cannot handle negation.
We resolve this by removing negations from the rules and then applying its method.
For symbolic signs, its deployed delta-debugging requires the program to be monotonic, \ie when input facts increase, the output of a rule must not decrease. 
We resolve this using an ASP solver to compute truth assignments for the symbolic signs. 
Overall, we achieve a SEDL that accommodates stratified negations and thereby supports the repair for CTL-defined analysis, detailed in \secref{sec:program_repair}. 


% \subsubsection{Limitation}
% \Symlog generates logical constraints in two steps: first, it generates constraints over symbolic constants, then it computes constraints over symbolic signs. 
% However, \Symlog's method for computing constraints over symbolic constants does not address how to handle negations. We resolve this by removing negations from the rules and then applying its method.
% This modification is an over-approximation for possible constraints over symbolic constants.
% Additionally, \Symlog computes constraints over symbolic signs using delta-debugging~\cite{zeller1999yesterday}, which requires the Datalog program to be monotonic (i.e., free of negations). 
% This requirement is not satisfied in this example.
% We address this limitation using an ASP solver to compute truth assignments for the symbolic signs. 
% We demonstrate our approach in \secref{sec:program_repair}. 




% to eliminate a target fact (\eg when \code{\relation} indicates a bug, and by eliminating \code{\relation}, we eliminate the bug), 


% by solving the negation of \code{\psi^*} for all \code{\relation} {matching} the fact, we can find an instantiation of \code{\Sigma}, \ie, a patch.
% Here, matching refers to the \code{\mathrm{SO}} being identical to the target fact or becoming identical by substituting symbolic constants with the corresponding concrete constants in the target fact.