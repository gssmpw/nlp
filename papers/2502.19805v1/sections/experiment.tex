\section{Experiments}
\subsection{Setup}

\paragraph{Baselines}
We compare our model with three Transformer models proposed in \citet{ruoss2024grandmaster}: State-action model (\texttt{S-A}) which learns to predict next move via behavioral cloning; State-value model (\texttt{S-V}) which predicts next move via comparing the value of next states; and Action-value model (\texttt{SA-V}) which predicts next move via comparing the value of each legal actions at the current state. We also integrate the trained \texttt{S-A} and \texttt{S-V} models into MCTS following AlphaZero~\citep{silver2017alphazero}.


\paragraph{Data}
\section{Dataset Generation}
\label{sec:dataset}
\revise{
To train the proposed GNN, we constructed a dataset of building structures and a subset of these structures were subjected to fire simulations using FEA. The dataset generation process is illustrated in \figref{fig:dataset_generation_procedure}. Initially, a total of 33,000 building structures with geometrical details, material properties, and gravity loads were created. Due to randomness in generating these structures, a filter is applied to remove unreasonable data after gravity load simulation, which included 15,377 structures. A trade-off between computational feasibility and model performance is made among the remaining 17,623 structures. As further labeling structures with MIDR requires resource-intensive fire simulations via OpenSeesRT, a large proportion of 16,050 structures is selected as unlabeled dataset. On the other hand, each of the other 1,573 structures was further subjected to 30 different fire simulations, forming the labeled dataset containing $1,573\times 30 = 47,190$ fire cases.} This section details the step-by-step process for generating the dataset, including geometry creation, material property assignment, and simulations due to gravity loads and fire scenarios. 
% To train the proposed neural network, we constructed a dataset comprising building structure data and a subset of fire scenario data. The dataset generation process is illustrated in \figref{fig:dataset_generation_procedure}. 
% A total of 33,000 building structures with geometric details, material properties, and gravity loads were initially created. Out of these, 3,000 structures were selected as labeled data, and the remaining 30,000 were designated as unlabeled data. Further, about half of them filtered out due to instability under gravity loads only. 
\begin{figure*}[h!]
    \centering
    \includegraphics[width=0.8\linewidth]{figures/dataset_filter_procedure.pdf}
    \caption{Workflow for dataset generation (geometry, material property, gravity loads, and fire scenarios).}
    \label{fig:dataset_generation_procedure}
\end{figure*}

\subsection{Geometry Generation}
\label{subsec:geometry_generation}
The geometry of the building structures forms the foundation of the dataset. Regular 
\revise{3D structures} resembling multi-story parking structures or shopping malls were generated, with parameters such as building floor dimensions and story heights selected randomly. Each building structure is composed of multiple rooms, which serve as the basic unit in this study. A room herein is a cuboid space defined by specific length, width, and height. Within a structure, rooms of the same dimensions are uniformly arranged along the length, width, and height, corresponding to the $x$-, $y$-, and $z$-axes, respectively. Structures vary in room size and number of rooms along each axis. Specifically, the room length, width, and height are independently sampled from a uniform distribution within the interval $[2, 5]$ meters along the three directions of the structure. Similarly, the room number along each axis is uniformly sampled independently as an integer within the interval $[2, 7]$, i.e., the maximum number of stories of the buildings simulated in this study is 7.

To introduce variability and simulate real-world scenarios, approximately $8\%$ of structural elements (beams or columns) are randomly removed after initial geometry creation. 
\revise{Such removal is not fire-induced damage, but reflects functional diversity often observed in real buildings, such as open spaces designed for activities in shopping malls, e.g., ice skating rinks. Examples of the generated geometries are illustrated in \figref{fig:example_generated_geometry}, showcasing the diversity and realism of the dataset. This element removal does not affect the definition of room's geometry in the structure and nor does it affect the number of considered fire scenarios.} 

\revise{A range of coefficient of variation values ($3.3\%$ to $17.5\%$) was derived from prior studies that investigated the statistics of geometrical and material properties of structural components of buildings (e.g., \cite{mirza1979variations, lee2004probabilistic}). These studies provide empirical data on the natural variability in parameters such as Young's modulus, yield strength, and dimensions of structural elements due to manufacturing tolerances and material inconsistencies. By selecting $8\%$ for the removal of structural elements in our database, we aimed to maintain a level of variability that is representative of real-world uncertainties while ensuring computational feasibility. This choice ensures that the database captures realistic deviations without introducing extreme cases that may not be commonly encountered in practice.}

\begin{figure*}[h!]
    \centering
    \includegraphics[width=\linewidth]{figures/example_generated_geometry.pdf}
    \caption{Examples of generated structural geometry of different sizes (all dimensions in meters).}
    \label{fig:example_generated_geometry} 
\end{figure*}

{\blockRevise

In this study, we opted for a deterministic square, dimension of $0.1$ m, solid cross-sectional steel elements due to their simplicity in modeling and analysis. Square sections exhibit uniform geometrical properties in all directions, simplifying the computation of structural responses and avoiding complications associated with more complex shapes, such as wide-flange sections, facilitating the computational efficiency and scalability to generate a large dataset. This choice also helps to mitigate issues related to stress concentrations and facilitates a more straightforward representation of structural behavior under thermal loads. 

\textit{Remark:} The selected cross-section provides a comparable flexural rigidity to a $W 130 \times 130 \times 28.1$ wide-flange section (metric units), albeit with significantly higher axial rigidity. This cross-section is acceptable for gravity-load-designed frames under service loading conditions where the models assume fully rigid, moment-resisting beam-column connections for the evaluation of the IDR under thermal loading. This assumption is reasonable in this computational study where the primary interest is to understand the global deformation response of frames under fire conditions. The selection of uniform square cross-sections for both beams and columns, rather than adherence to standard capacity design principles, was made here primarily for computational efficiency and to reduce design parameters in the database generation process. This choice allows for simplified and scalable approach to analyze the fire-induced response of generic steel frames without the need for large section variations, where this study mainly focuses on the fire vulnerability assessment using ML-based predictions. However, if additional loading conditions, e.g., seismic or wind loads, were to be considered, larger sections, strong-column/weak-beam principle, and ductile detailing would be required in the generated buildings for realistic structural behavior under combined loading conditions. Future studies may also consider investigating the influence of variable cross-sectional dimensions and semi-rigid connections on the structural performance under fire conditions. 
} % blockRevise

\subsection{Material Properties}
Steel is chosen as the material for the structures. To reflect real-world variations, we randomly assign one of five slightly different steel material types to each structural element. \revise{
The ranges of material properties are provided in \tabref{tab:material_property_ranges} and the properties are sampled from uniform distributions of the corresponding ranges. These variations simulate differences arising from manufacturing batches or regional material properties. That these properties are at ambient temperature and change when the temperature rises due to a fire. The selection of materials with varying properties is aimed at increasing the diversity of the data. Our goal is to represent as wide a range of data as possible with a limited amount of building structure data, thereby enhancing the generalization ability of the GNN. Our assumed material property ranges are expected to be wider than the real-world conditions based on findings in \cite{mirza1979variations, lee2004probabilistic}. Therefore, we are essentially tackling a more challenging and general task. If we can solve this problem, we are confident that our method will perform equally well or even better in real-world scenarios.
}
\begin{table}[h!]
    \centering
    \caption{Material properties ranges for considered steel structures.}
    \begin{tabular}{lc}
        \toprule
        Property & Range \\
        \midrule
        Young's modulus & [168, 252] GPa \\
        Yield strength & [220, 330] MPa \\
        Strain-hardening ratio & [0.8, 1.2] \% \\
        \bottomrule
    \end{tabular}
    \label{tab:material_property_ranges}
\end{table}

\subsection{Gravity Loads}
Gravity loads are applied to columns and beams based on their \revise{influence (tributary) areas as typically conducted in structural analysis. The considered ``service'' load conditions include the column self-weight and the additional loads directly supported on the beams from their self-weight and weights of the reinforced concrete slabs, people as live load, and building content. An edge beam typically carries approximately half the gravity load supported by a parallel interior beam}. The ranges of gravity loads are listed in \tabref{tab:gravity_load_ranges}. \revise{The loads are sampled from uniform distributions of the corresponding ranges.} Structures that failed to meet an MIDR threshold of $1\%$ under gravity loads were deemed unacceptable designs and filtered out, as such configurations of randomly chosen geometry, material, and gravity load combinations were considered unrealistic from a regulatory and practicality points of view.
\begin{table}[h!]
    \centering
    \caption{Gravity load ranges for considered beams and columns.}
    \begin{tabular}{lc}
        \toprule
        Element & Range (kN/m)  \\
        \midrule
        Column & [0.5, 1.0]  \\
        Edge beam & [1.5, 4.5]  \\
        Interior beam & [3.0, 7.5]  \\
        \bottomrule
    \end{tabular}
    \label{tab:gravity_load_ranges}
\end{table} 

\subsection{Rule-based Thermal Load Generation}
\label{subsec:thermal_load_generation}
To evaluate a building's structural response during a fire event, we employed a simplified rule-based approach for thermal load generation. 
% Previous studies \cite{nan_structuralfire_2023} have demonstrated that steel structures rapidly equilibrate with surrounding gases temperatures due to efficient heat exchange. Consequently, gas temperatures can be directly used as inputs for FEA tools, e.g., OpenSees, simplifying the process of modeling thermal loads. 
% Accurately simulating temperature fields in fire scenarios poses significant challenges. Advanced thermodynamic simulations, such as those performed using Fire Dynamics Simulator (FDS) \cite{mcgrattan_fire_2000}, provide precise temperature predictions. However, these methods are hindered by high computational costs, prolonging execution times, and limited scalability, making them impractical for generating large datasets. Additionally, real-world fire loads often display substantial spatial variability across different rooms \cite{dundar_fire_2023}, resulting in scenario-specific temperature fields with limited generalizability. For example, studies on bridge fires \cite{he_study_2024} have demonstrated that environmental factors, such as wind speeds, can significantly influence temperature distributions. Furthermore, even within identical scenarios, variations in fire modeling methodologies can produce distinctly different temperature fields \cite{zhang_temperature_2020, du_new_2012}. These challenges emphasize the need for efficient and adaptable methods to generate fire temperature data.
% To address these issues, we adopted a rule-based approach to model temperature variations. 
According to \cite{spearpoint_fire_2008}, a typical fire development follows a predictable pattern. During the {\em{growth stage}}, the temperature rises slowly and approximately linearly after ignition. This is followed by the {\em{flashover stage}}, where temperatures increase rapidly to peak values. After reaching the peak, the temperature either stabilizes or continues to rise slowly until the {\em{decay stage}} begins. Inspired by this fire development pattern, we describe the temperature evolution in time, $t$, prior to the decay stage in two distinct stages:
\begin{enumerate}
    \item {\bf{Initial linear increase stage}}: For $t \in [0, t_1)$, temperature increases gradually and linearly as the fire spreads through the building. This stage represents the time before the fire directly affects a structural element.  
    \item {\bf{ISO 834 fire curve stage}}: For $t \in [t_1, t_{\thre}]$, temperature rises rapidly following the ISO 834 curve \cite{ISO834}, modeling the direct impact of the fire on the structural element. 
\end{enumerate}
The slope of the linear temperature increase, $c$, and the transition time, $t_1$, are influenced by the spatial relationship between the fire source and the structural element. For the second stage of temperature evolution, we utilize the ISO 834 curve, a widely accepted standard for fire resistance testing. This standardized fire curve describes the temperature rise over time, enabling rapid and consistent thermal fields across various scenarios. The duration of fire simulation in this study is set to $t_{\thre}=60$ minutes. This value represents the upper limit for the temperature evolution of each structural element, providing a consistent basis for analyzing the structural response to fire.

Let $(x, y, z)$ represents the midpoint of a structural element and $(x_{\subfire}, y_{\subfire}, z_{\subfire})$ the fire source point. \revise{Integer parameters $h$ and $h_{\subfire}$ correspond to the respective floor levels of the element and the fire source}. The temperature evolution for each element is expressed as follows:
\begin{enumerate}
    \item Linear increase stage ($0 < t < t_1$):
    \begin{equation}
    T(t) = c \cdot t,
    \end{equation}
    where $c$, the rate of temperature increase ($^\circ\mathrm{C}/\mathrm{min}$), depends on the height difference between the element, $h$, and the fire source, $h_{\subfire}$:
    \begin{equation}
        c = 
        \begin{cases} 
        5\left/\left(h - h_{\subfire} + 1\right)\right., & h \geq h_{\subfire}, \\
        2\left/\left(h_{\subfire} - h\right)\right., & h < h_{\subfire}.
        \end{cases}
    \end{equation}
     \item ISO 834 stage ($t \geq t_1$):
\begin{equation}
    T(t) = c \cdot t_1 + 345 \log_{10} \left(8 \left(t - t_1\right) + 1\right).
\end{equation}
\end{enumerate}

The transition (arrival) time $t_1$, marking the end of the linear stage, depends on the spatial distance between the fire source and the element. We define the following two Euclidean distances $L_p$ in the $xy$ plane and $L_s$ in the $xyz$ space:
\begin{eqnarray}
L_p & \triangleq & \sqrt{(x - x_{\subfire})^2 + (y - y_{\subfire})^2}, \\
\label{eq:Lp}
L_s & \triangleq & \sqrt{(x - x_{\subfire})^2 + (y - y_{\subfire})^2 + (z - z_{\subfire})^2}.
\label{eq:Ls}
\end{eqnarray}
Accordingly, the transition time, $t_1$, is expressed as follows:
\begin{equation}
    t_1 = 
    \begin{cases}
    \beta_{1} \cdot \left(1 - \exp\left\{- L_s\left/\alpha_{1}\right.\right\}\right), & h > h_{\subfire}, \\
    \beta_{2} \cdot \left(1 - \exp\left\{- L_p\left/\alpha_{2}\right.\right\}\right), & h = h_{\subfire}, \\
    \beta_{3} \cdot \left(1 - \exp\left\{- L_s\left/\alpha_{3}\right.\right\}\right), & h < h_{\subfire} .
    \end{cases}
    \label{eq:t1}
\end{equation}
The parameters $\beta_i$ and $\alpha_i$ for determining $t_1$ are summarized in Table~\ref{tab:fire_spread_parameters}. In this study, we take $r_{\mathrm{up}}=0.95$ and $r_{\mathrm{down}}=0.97$.
\begin{table}[ht]
    \centering
    \caption{Fire spread parameters for $t_1$ calculations.}
    \begin{tabular}{lcc}
        \toprule
        Case  & $\beta_i$ & $\alpha_i$  \\
        \midrule
        $i=1$, Upward spread & $16 \left.\left(1-r_{\mathrm{up}}^{\left|h-h_{\subfire}\right|}\right)\right/\left(1-r_{\mathrm{up}}\right)$ & $10$  \\
        $i=2$, Horizontal spread & $18$ & $18$  \\
        $i=3$, Downward spread & $30 \left.\left(1-r_{\mathrm{down}}^{\left|h-h_{\subfire}\right|}\right)\right/\left(1-r_{\mathrm{down}}\right)$ & $5$  \\
        \bottomrule
    \end{tabular}
    \label{tab:fire_spread_parameters}
\end{table}

\figref{fig:t1_curve} illustrates the $t_1$ curves for various fire scenarios: (1) fire originating on the lower floor, $h-h_{\subfire}=1$ with rapid upward spread, (2) fire on the same floor, $h=h_{\subfire}$ with the fastest spread, and (3) fire on the upper floor, $h_{\subfire}-h=1$ with slow downward spread. The exponential decay in $t_1$ reflects the accelerating fire propagation speed as the distance increases. \figref{fig:t1_curve} also indicates that the employed simplified model is consistent with the Markov chain-based dynamic model given by \cite{cheng_dynamic_2011}, where the rooms at the same floor of the fire point start flashover slightly before the corresponding upper floors. Additionally, $\beta_{1}$ and $\beta_{3}$ are the summation of a geometric sequence, where story level $h$ is the index. The common ratios $r_{\mathrm{up}}<1$ in $\beta_{1}$ and $r_{\mathrm{down}}<1$ in $\beta_{3}$ indicate that the fire speeds up to spread through the next story, which is consistent with the real-world fire spread mechanism given in \cite{hokugo_mechanism_2000}. The temperature profile within the range $t \in [0, t_{\thre}]$ is subsequently used as the thermal load in OpenSeesRT simulations to compute displacements at each structural node at time $t_{\thre}$.
\begin{figure}[h!]
    \centering
    \includegraphics[width=0.8\linewidth]{figures/m204_t1_curve.pdf}
    \caption{Three examples for the $t_1$ curve.}
    \label{fig:t1_curve}
\end{figure}

\revise{
\textit{Remark:} The effects of structural elements, such as concrete floor slabs and partitions, are not explicitly modeled in our approach. Instead, their influence is implicitly captured through the careful selection of the parameters $ \alpha, \beta, r_\mathrm{up} $, and $ r_\mathrm{down} $. This parameterization provides a unified framework for generating temperature fields. Indeed, fire propagation is governed by a multitude of factors and remains an open research question. For instance, if the fire resistance of a floor slab is enhanced by fire protective coating, the corresponding model can account for this by decreasing $\alpha_1$ \& $\alpha_3$, increasing $\beta_1$ \& $\beta_3$, and adopting larger values for $r_\mathrm{up}$ \& $r_\mathrm{down}$, which collectively slow down the vertical spread of fire. Conversely, scenarios involving higher amounts of combustible materials would warrant the opposite adjustments. This flexible and integrated approach avoids the need to design separate models for different fire propagation scenarios while still capturing the essential effects.
}

\revise{
In conclusion, our rule-based approach is a computationally efficient method for approximating fire temperature fields, enabling large-scale dataset generation to train predictive models. By combining ISO 834 fire curves with spatial considerations and embedding structural effects through parameter calibration, the method achieves a balanced trade-off between accuracy and scalability, making it a practical solution for thermal load modeling in fire scenarios. After generating the temperature of each beam or column according to the middle point, the temperature is applied as uniform thermal load to the elements of the structure in question using OpenSeesRT. 
}

% In conclusion, this rule-based approach is a computationally efficient method to approximate fire temperature fields, enabling large-scale dataset generation to train predictive models. By combining ISO 834 fire curves with spatial considerations, the method balances accuracy and scalability, making it a practical solution for thermal load modeling in fire scenarios.

% \subsection{Interstory Drift Ratio}
\subsection{OpenSeesRT Simulation}
\label{subsec:opensees_simulation}

The thermal and mechanical responses of 3D frame structures under combined fire and gravity loads are simulated using OpenSeesRT \cite{perez2024openseesrt}. \revise{In the simulation, the IDR of each node at $t_{\thre}$ is computed using the computed nodal displacements. Each structural model features six degrees of freedom per node (3 translational  and 3 rotational), with linear geometrical transformations (\texttt{geomTransf: Linear}) defining how the element local coordinate systems are mapped to the global coordinate system and assuming small displacements and rotations. Although OpenSeesRT allows a variety of options for modeling finite deformations, in the present simulations and mainly for simplicity, we did not consider large deformations. All bottom nodes (nodes on the ground) are fully constrained in all six degrees of freedom, while degrees of freedom os all other nodes are free.} Material behavior is temperature-dependent and modeled with \texttt{Steel01Thermal}, while fiber-based sections (\texttt{FiberThermal}) capture nonlinear interactions between thermal and mechanical responses at the cross-section level. \revise{Structural elements are represented as displacement-based Euler-Bernoulli beam-columns (\texttt{dispBeamColumnThermal}). This element  formulation accounts for thermal strains (temperature gradients) in the section, which is discretized into fibers. Numerical integration is used along the length of each element using three integration (Gauss) points, one at each end and the third in the middle of the element.}

{\revise{Thermal expansion of steel members plays a crucial role in IDR development. In reality, reinforced concrete floor slabs heat at a different rate than steel members due to their higher thermal mass and lower thermal conductivity. This differential heating can lead to restrained thermal expansion, introducing axial compression in beams and affecting the overall structural response. In this study, explicit {\em{composite action}} between steel members and concrete slabs is not modeled. Instead, our approach focuses on isolating the response of the steel structural frame, which is often the critical load-bearing component in fire scenarios. This assumption aligns with prior studies \cite{Possidente_2024} demonstrating that steel structures reach thermal equilibrium with surrounding gases quickly, allowing the use of uniform thermal loading in fire analysis. Future work could enhance this framework by incorporating slab-beam interaction effects, through a refined FEA for an extended dataset where constraints imposed by floor slabs are explicitly considered.}

The analysis begins with the application of gravity loads, followed by incremental thermal loads simulating the fire exposure. A static nonlinear solver using  \texttt{ExpressNewton} algorithm ensures convergence, while the \texttt{NormDispIncr} test maintains accuracy. An incremental \texttt{LoadControl} scheme with small step sizes is employed to guarantee numerical stability, using 10\% for gravity loads and 1\% for thermal loads. 

\revise{
In the thermal load analysis, uniform thermal load is applied to each beam or column, i.e., the temperature of each element is set to be that at the middle point, according to \secref{subsec:thermal_load_generation}. The \texttt{Steel01Thermal} material allows the properties (e.g., Young's modulus and yield strength) to be adjusted at increasing temperatures according to \cite{EN1993} using its Table 3.1: Reduction factors for the stress-strain relationship of carbon steel at elevated temperatures. For example, if the Young’s modulus at ambient temperature is $E_0$, then as the temperature ($T$) increases, the modulus changes as $E(T) = \eta (T) \times E_0$. \cite{EN1993} directly provides the values of $\eta(T) \in \left[0,1\right] $ at every $100 ^\circ\mathrm{C}$ interval and recommends using linear interpolation to obtain $\eta(T)$ for intermediate values of $T$.
} OpenSeesRT documentation \cite{OpenSeesThermalExamples} provides several examples of thermal analyses.

This modeling framework accommodates variations in material properties, cross-sectional geometries, and temperature profiles, providing robust simulations of structural behavior under fire conditions. The primary settings and configurations for the OpenSeesRT simulations are summarized in \tabref{tab:ops_detail}.
\begin{table}[h!]
    \centering
        \caption{Key settings of OpenSeesRT simulations.}
    \begin{tabular}{l|>{\raggedright\arraybackslash}p{0.6\linewidth}} %
    \toprule
    Modeling Aspect     & Details \\
    \midrule
    Geometry            & 3D models; 6 degrees of freedom per node \\
    Transformation      & geomTransf: Linear \\ 
    Material            & Steel01Thermal \\
    Section             & FiberThermal; Cross-section: $0.1$ m $\times$ $0.1$ m \\ 
    Element type        & {dispBeamColumnThermal} \\ 
    Loading             & Gravity loads: {beamUniform}; Thermal loads: {beamThermal} \\
    Integration scheme  & Incremental {LoadControl}; Step size: $10\%$ (gravity analysis), $1\%$ (thermal analysis) \\
    Nonlinear solver    & {ExpressNewton} algorithm; {UmfPack} solver; Convergence test: {NormDispIncr} tolerance: $10^{-8}$; Maximum \# iterations per step: $1000$. \\ 
    \bottomrule
    \end{tabular}
    \label{tab:ops_detail}
\end{table}

For each structure in the labeled dataset, 30 fire points are selected using a dual-granularity approach, \revise{i.e., two-stage sampling strategy,} to ensure they are well-distributed. Specifically, rooms are sequentially selected, with one fire point randomly chosen within each selected room. If a building is large and contains more than 30 rooms, we randomly select 30 rooms without replacement, i.e., ensuring that no more than one fire point is located in the same room. Conversely, if the building is small and has fewer than 30 rooms, all rooms are initially selected, with one fire point randomly assigned to each room. Additionally, rooms are then selected with replacement until a total of 30 fire points are assigned. \revise{The room-level sampling prioritizes selecting distinct rooms to avoid spatial clustering of fire points, while the point-level sampling ensures intra-room variability. This approach aligns with stratified sampling principles commonly used for efficient spatial representation, where multi-stage sampling strategies optimize coverage and variability, e.g., \cite{arunachalam_generalized_2023}, and enables a more comprehensive characterizing of how the structures respond under fire conditions.}
% This selection method prevents fire points from clustering too closely while maintaining an element of randomness. By distributing fire points in this manner, the 30 fire scenarios are effectively utilized, enabling a more comprehensive characterizing of how the structures respond under fire conditions.

\subsection{Summary of the Dataset Generation}
As discussed in this section and related to  \figref{fig:dataset_generation_procedure}, three key steps were considered in the development of the dataset: 
\begin{enumerate}
    \item {\bf{Filtering process}}: Structures with MIDR exceeding $1\%$ under gravity loads were excluded,  resulting in $1,573$ labeled structures retained for fire simulation and $16,050$ unlabeled structures for training the MFSP predictor.
    \item {\bf{Fire simulations}}: For each retained labeled structure, 30 fire scenarios were simulated using OpenSeesRT, yielding $47,190$ fire cases.
    \item {\bf{Data distribution check}}: MIDR distributions for labeled and unlabeled data under gravity loads were highly similar, because both datasets were generated using the same method. Under fire conditions, the MIDR distribution shifted, reflecting significant structural deformation with values reaching a maximum of about 6\%, an average of 1.70\%, and a standard deviation of 1.12\%. This step ensured a diverse and comprehensive dataset for the proposed predictive framework.
\end{enumerate}
The statistical distribution histograms for MIDR (after applying the $1\%$ filtering threshold \revise{for gravity load responses}) under different loading conditions are plotted in \figref{fig:histogram_mdr}. Figures \ref{fig:histogram_mdr}(a) and \ref{fig:histogram_mdr}(b) show the MIDR distributions of the labeled and unlabeled data, respectively, under gravity loads only. \figref{fig:histogram_mdr}(c) shows the MIDR distribution of the labeled data under the combined effects of gravity and fire loads. Fire load causes the structures to significantly deform, leading to a noticeably \revise{right-skewed} MIDR distribution.

\begin{figure*}[h!]
    \centering
    \includegraphics[width=\linewidth]{figures/histogram_mdr.pdf}
    \caption{Histograms of MIDR for labeled and unlabeled structures with gravity loads and fire cases.}
    \label{fig:histogram_mdr}
\end{figure*}

\revise{
This dataset provides the basis for training and testing the performance of the GNN-based framework. Although we employed a simplified rule-based thermal load generation method compared with conventional CFD-based simulations, the temperature field, the changes of the material properties, and the response of the structures, are all still highly nonlinear and complex. Therefore, it is still a challenging task for the NN to predict the MIDRs based on this dataset.
}
We construct a dataset for supervised training by downloading games from \href{https://lichess.org}{lichess} recorded in February 2023. When analyzing the scaling behavior, we use up to 100k games, while reverting to the default 10k games for other experiments due to resource constraints. We show the data statistics in Table~\ref{tab:dataset}. Following~\citet{ruoss2024grandmaster}, we convert the centipawns returned by Stockfish to the win percentage and then discretize it into 128 bins to represent value in \texttt{S-V} and \texttt{SA-V}. We encode the state as a fixed-length FEN string with 77 characters by padding with `.' if needed. Actions are stored in UCI notation with 1968 possible moves in total. \reb{We provide example training data for each paradigm in Appendix~\ref{appendix:training-case}.}

\paragraph{Implementation Details}
For all the neural models in this paper, we use the same decoder-only GPT-2 transformer architecture~\citep{vaswani2017attention,radford2019language} for a rigorous comparison. For \ourmodel, we convert casual attention into full attention without introducing additional learned parameters. We train all baseline models until convergence and set a maximum of 200 epochs for diffusion models due to their slow convergence. We use the Adam optimizer~\citep{kingma2015adam}, a learning rate of 3e-4, and a batch size of 1024 for all models. By default, we set the horizon $h$ to be 4, the number of network layers to be 8 (with a total parameter size of 7M), the diffusion timesteps to be 20, and an absorbing noise type. By default, 100 simulations are utilized in MCTS-enhanced policy, and its impact is analyzed in Figure~\ref{fig:implicit-vs-explicit}. We adjust $c_{\text{puct}}$ and $\tau$, constants determining the level of exploration in MCTS, on a held-out set and set them to $c_{\text{puct}}=0.1$ and $\tau=1$ for its superior performance.
All experiments are done on 8 NVIDIA V100 32G GPUs.


\paragraph{Evaluation Metrics}
We mainly consider three metrics to evaluate the policies following ~\citep{ruoss2024grandmaster}: 1) \textbf{Action Accuracy}: the percentage of the test set in which the model selects the same action as the ground truth; 2) \textbf{Puzzle Accuracy}: the percentage of puzzles where the
policy’s action sequence exactly matches the known
solution action sequence and we use 10k puzzles with difficulty rated by Elo from
399 to 2867 provided by ~\citep{ruoss2024grandmaster}; 3) 
\textbf{Tournament Elo}: the Elo ratings calculated using BayesElo~\citep{coulom208whole} in an internal tournament involving all policies, where each pair of policies played 400 games, resulting in a total of 6000 games.

\subsection{Main Results}
\label{sec:main}
\begin{table}[t]
\caption{Comparison of the proposed ColonTCN approach with baseline architectures. For each model, the F1 score for each class, the wF1, the wJacc and the WMAPE computed on all the 60 REAL-Colon dataset videos, and the number of parameters and GFLOPs is reported.}
\centering
\begin{center}
\resizebox{\textwidth}{!}{
\begin{tabular}{@{}|ccc|ccccccccc|ccc|@{}}
\toprule
\multicolumn{15}{|c|}{5-fold REAL-Colon} \\
\midrule
Model & Params [M] & GFLOPs & Outside & Insertion & Ceacum & Ileum & Ascending & Transverse & Descending & Sigmoid & Rectum &  wF1 & wJacc & WMAPE \\
\midrule
TCN (TeCNO \cite{czempiel2020tecno}) & 0.5 & 2.809 & \(81.9\) & \(85.8\) & \(40.3\) & \(4.1\) & \(34.8\) & \(58.5\) & \(35.1\) & \(58.4\) & \(63.4\) & \(63.0\) & \(49.0\) & \(13.5\)  \\
MS-TCN (TeCNO \cite{czempiel2020tecno}) & 1.7 & 9.351 & \(87.9\) & \(81.9\) & \(43.7\) & \(27.4\) & \(33.3\) & \(5.04\) & \(19.7\) & \(45.7\) & \(66.0\)  & \(58.5\)& \(44.5\)  & \(21.4\) \\
ASFormer \cite{yi2021asformer} & 1.2 & 6.376 & \(\textbf{96.2}\) & \(95.1\) & \(63.8\) & \(52.1\) & \(\textbf{59.1}\) & \(68.9\) & \(\textbf{42.8}\) & \(\textbf{67.0}\) & \(\textbf{77.4} \) & \(75.2\) & \(63.4\) & \(3.3\)  \\
ColonTCN (Ours) & 0.9 & 4.386 & \(95.1\) & \(\textbf{95.8}\) & \(\textbf{67.3}\) & \(\textbf{56.0}\) & \(49.9\) & \(\textbf{74.8}\) & \(47.3\) & \(65.9\) & \(72.5\) & \(\textbf{76.2}\) & \(\textbf{64.7}\) & \(\textbf{3.1}\) \\

\midrule
\midrule
\multicolumn{15}{|c|}{4-fold REAL-Colon} \\
\midrule
Model & Params [M] & GFLOPs & Outside & Insertion & Ceacum & Ileum & Ascending & Transverse & Descending & Sigmoid & Rectum &  wF1 & wJacc & WMAPE \\
\midrule
TCN (TeCNO \cite{czempiel2020tecno}) & 0.5 & 2.809 &  \(73.6\) & \(71.6\) & \(32.0\) & \(0.0\) & \(24.9\) & \(35.8\) & \(22.6\) & \(34.4\) & \(60.5\) & \(47.8\) & \(33.8\) & \(24.8\) \\
MS-TCN (TeCNO \cite{czempiel2020tecno}) & 1.7 & 9.351 & \(87.5\) & \(83.4\) & \(39.5\) & \(10.9\) & \(32.9\) & \(44.5\) & \(18.1\) & \(49.5\) & \(60.8\) & \(56.5\) & \(43.1\) & \(18.6\) \\
ASFormer \cite{yi2021asformer} & 1.2 & 6.376 & \(\textbf{83.9}\) & \(\textbf{86.7}\) & \(45.9\) & \(30.8\) & \(41.7\) & \(54.2\) & \(27.6\) & \(40.0\) & \(\textbf{63.4}\) & \(61.0\) & \(47.4\) & \(\textbf{16.8}\)  \\
ColonTCN (Ours) & 0.8 & 4.099 & \(87.9\) & \(84.9\) & \(\textbf{49.2}\) & \(\textbf{30.1}\) & \(\textbf{41.6}\) & \(\textbf{56.7}\) & \(\textbf{26.2}\) & \(\textbf{53.0}\) & \(65.7\) & \(\textbf{62.7}\) & \(\textbf{48.6}\) & \(\textbf{16.8}\) \\
\bottomrule
\end{tabular}}
\end{center}
\label{tbl:maintab}
\end{table}


We report the prediction and playing strength comparison for our model against baselines in Table~\ref{tab:main}. Additionally, we report the performance of Stockfish 16 with a time limit of 0.05s per legal move, which stands as the oracle used to generate our dataset.
We find \ourmodel significantly outperforms the \texttt{S-A} model by 653 Elo and 19\% action accuracy, indicating the effectiveness of \ourmodel in improving next action prediction through future prediction. Remarkably, despite utilizing 20 times fewer data records than the \texttt{SA-V} model, our model demonstrates superior performance with approximately 10\% higher action accuracy. Our model demonstrates superior performance over the MCTS-based agent by achieving a higher Elo difference of 542 and an increased action accuracy of 14\%. This highlights the effectiveness of \ourmodel in modeling multi-step simulations when compared with the step-by-step MCTS-enhanced policy, which relies on a robust value model and necessitates a careful balance between the policy and value models.


\subsection{Ablations}
\label{sec:ablation}

\paragraph{Future paradigm matters}
% Please add the following required packages to your document preamble:
\begin{wraptable}{r}{0.45\textwidth}
\vspace{-15pt}
\caption{Action accuracy comparison of baselines and different future paradigms.}
\centering
\scalebox{0.9}{
\begin{tabular}{lcc}
\toprule
\textbf{Paradigms} & \textbf{Transformer} & \textbf{\ourmodel} \\
\hline
S-A & 22.10 & - \\
S-V & 21.45 & - \\
SA-V & 31.50 & - \\
S-AA & 26.62 & 15.07 \\
S-ASA & 27.39 & 41.31 \\
S-ASS & 24.93 & 41.19 \\
S-AVAV & 25.92 & 17.63 \\
S-AVSAV & 25.59 & 40.69 \\
\bottomrule
\end{tabular}}
\vspace{-5pt}
\label{tab:future_paradigm}
\end{wraptable}

We compare baselines and different future paradigms during the training of \ourmodel with horizon $h=4$ in Table~\ref{tab:future_paradigm}. For each future paradigm, we compare training with autoregressive Transformer and \ourmodel. We find that directly performing future action prediction (\texttt{S-AA})~\citep{chi2023diffusion} with \ourmodel hurts performance compared to \texttt{S-A} due to the difficulty of future move prediction in chess. \reb{However, after we integrate future states, we observe significant performance improvements when comparing \texttt{S-AA} (15.07)  to \texttt{S-ASA} (41.31), and also when comparing \texttt{S-AVAV} (17.63) to \texttt{S-AVSAV}  (40.69).}
No further improvement is observed when integrating the values in \ourmodel, which may be attributed to training on the optimal future trajectory rather than all possible trajectories. For training using autoregressive Transformer, we observe that the overall performance hovers around 26\%. This performance level is superior to that achieved by \texttt{S-A} (22.1\%), attributed to the utilization of more (S, A) pairs (e.g., each \texttt{S-ASA} record containing $h$ (S, A) pairs). However, the performance falls short when compared to \ourmodel, which underscores the importance of modeling bidirectional context to leverage future information for subsequent action prediction.

\begin{wraptable}{r}{0.37\textwidth}
\vspace{-10pt}
\caption{Future world quality in supervising the model for the S-ASA paradigm.}
\centering
\begin{tabular}{lr}
\toprule
\textbf{Future Quality} & \textbf{Acc.} \\
\hline
Without future & 22.10 \\
 + Random world+policy & 22.69 \\
 + Random policy & 39.47 \\
 + Stockfish policy & 41.31 \\
\bottomrule
\end{tabular}
\vspace{-5pt}
\label{tab:future_quality}
\end{wraptable}
\paragraph{Ensuring the validity of future world dynamics is crucial}
After we discuss the future paradigm, we now investigate the effect of future quality on performance, as shown in Table~\ref{tab:future_quality}. 
\reb{For better illustration, denote a sequence of future horizon 2 as $[s_1=f(s_0,a_0),a_1=g(s_1),s_2=f(s_1,a_1),a_2=g(s_2)]$, where $f$ is a world dynamic function and $g$ is a policy function. $s_0$ is the current state and $a_0$ is the move suggested by Stockfish.}
\reb{We first utilize random state-action sequences for future steps, where both actions and states were randomly selected (i.e., random world $f$ and random policy $g$)}. This methodology did not yield performance enhancements. Subsequently, we explore selecting random actions and incorporating the resulting state from executing those actions \reb{(i.e., random policy $g$ but an oracle world $f$)}, which notably outperforms the initial strategy. This underscores the significance of aligning states with corresponding actions, mirroring the dynamics of the world. Finally, we investigate incorporating high-quality future actions suggested by Stockfish \reb{(i.e., Stockfish $g$ and oracle world 
 $f$)} and observe additional performance improvements compared to the random action selection approach.

\begin{wraptable}{r}{0.4\textwidth}
\vspace{-10pt}
\caption{Comparison of training methods. Direct: train the model to predict the entire future sequence at once.}
\centering
\scalebox{1}{
\begin{tabular}{lc}
\toprule
\textbf{Method} & \textbf{Acc.} \\
\hline
Direct & 20.61 \\
Auto-regressive & 27.39 \\
Gaussian  & 31.91 \\
Absorbing, $\lambda_t=1$ & 39.66 \\
Absorbing, $\lambda_t=1/t$ & 39.07 \\
Absorbing, $\lambda_t=1-\frac{t-1}{T}$ & 41.31 \\
Multinomial, $\lambda_t=1-\frac{t-1}{T}$ & 40.08 \\
\bottomrule
\end{tabular}}
\label{tab:training_alg}
\vspace{-5pt}
\end{wraptable}
\paragraph{Proper discrete diffusion modeling helps}
Given the dataset $\mathcal{D}$ annotated with future states and actions, we investigate alternative ways to train the model, as presented in Table~\ref{tab:training_alg}. We first observe it is hard to teach the model to directly output the entire future sequence, leading to lower performance compared to auto-regressive training. Secondly, we employ continuous Gaussian diffusion VDM~\citep{kingma2021variational} and observe its superior performance compared to the Direct and auto-regressive methods, but inferior compared to discrete approaches. The absorbing diffusion with reciprocal $\lambda_t=1/t$ obtained by setting $\alpha_t=1-\frac{t}{T}$ in Eq.(\ref{eq:dm_simple}) is a simplified expression from D3PM~\citep{austin2021structured}, which we find significantly outperforms continuous diffusion. Finally, we discover a linear $\lambda_t$~\citep{bond2022unleashing, Zheng2023ARD} further exceeds the constant and reciprocal ones, as well as the multinomial counterpart.

\begin{figure*}[!t]
    \centering
    \includegraphics[width=0.98\linewidth]{figs/future-analysis.pdf}
    \caption{\textbf{(Left)} Prediction quality analysis for \ourmodel at different future steps. \textbf{(Middle)} Action accuracy when scaling self-attention layers. \textbf{(Right)} Action accuracy when increasing diffusion timesteps.}
    \label{fig:future-analysis}
    \vspace{-5pt}
\end{figure*}

\subsection{Analysis}
\label{exp:analysis}
\paragraph{Does \ourmodel predict accurate future information?}
We analyze the percentage of valid actions and the optimal action recommended by Stockfish for each predicted action. \reb{The best $a_0$ metric is exactly the action accuracy by definition.}. Additionally, we assess whether each predicted state is a valid representation and if $s_i$ corresponds to the resulting state when action $a_{i-1}$ is taken at $s_{i-1}$. The initial state $s_0$ provided as input is excluded, and the results are presented in the left figure of Figure \ref{fig:future-analysis}.
We observe that the first action, denoted as $a_0$, are almost 100\% valid. As we progress through future steps, both the valid rate and the optimal rate decline. However, even at $i=3$, where the valid rate stands at 50\%, it surpasses the random move baseline of approximately 1.6\% (calculated as the average number of legal actions per move, 31, divided by the total number of moves, 1968). This indicates that the model retains a certain level of predictive capability for future moves, albeit with reduced performance. A similar pattern appears in the evaluation of states, where the accuracy is perfect for the first predicted state $s_1$ but diminishes in subsequent predictions. \reb{In Appendix Table 8, we demonstrate that further increasing the training data enhances the effectiveness of the world model within \ourmodel, achieving over 90\% accuracy in predicting valid and matched future states corresponding to the preceding action.}

\paragraph{How does \ourmodel leverage future information?}
We attributes the future-aware ability of \ourmodel mainly to self-attention and iterative decoding process, as shown in the middle and right figures of Figure~\ref{fig:future-analysis}, respectively. When employing a single self-attention layer, our model exhibits inferior performance compared to the \texttt{S-A} model, yet surpasses it with two layers. Moreover, its performance steadily enhances as we augment the number of layers. This suggests that with additional layers, there is more chance for the subsequent actions and future to interact reciprocally, akin to the enhancement in the action prediction with increased MCTS simulations. We do not observe a similar upward trend in the performance of \texttt{S-A} model when increasing attention layers as in \citep{ruoss2024grandmaster}, possibly indicating that the available data (10k) does not necessitate the integration of more layers. In the right figure of Figure~\ref{fig:future-analysis}, it is evident that employing an appropriate decoding strategy (such as likelihood-based) further enhances next-action prediction as the number of iterations grows. However, the overall improvement is relatively modest compared to increasing the attention layers.



\paragraph{Explicit search vs. Implicit search}
Based on our previous analysis, we can consider \ourmodel as performing implicit search through the inner self-attention layers and the multi-step diffusion process. Now, we aim to evaluate the efficiency and effectiveness of this implicit search in comparison to explicit search using MCTS when conducting deeper searches. In \ourmodel, deeper search is realized by increasing the context length (80 tokens per search depth), whereas in MCTS, it is achieved through running more simulations.
In the left figure of Figure~\ref{fig:implicit-vs-explicit}, it is evident that \ourmodel exhibits significant enhancement when increasing search depth, while MCTS becomes stagnant after 50 simulations at a search depth of around 4. This could be attributed to the accumulated errors caused by the value network due to a limited horizon. In the middle figure of Figure~\ref{fig:implicit-vs-explicit}, we measure the latency per move for Transformer with MCTS and \ourmodel on a single V100 GPU with batch size 1. The performance of Transformer combined with MCTS is notably affected by the necessity of invoking the value network for every simulation. In contrast, \ourmodel experiences only a slight rise in latency as it requires just one call for greater depth. 


\begin{figure*}[!t]
    \centering
    \includegraphics[width=0.98\linewidth]{figs/search-compare.pdf}
    \caption{\textbf{(Left)} Action accuracy when increasing average search depth in MCTS through more simulations and \ourmodel through context length extension. \textbf{(Middle)} Latency measured by ms per second when increasing search depth. \textbf{(Right)} Action accuracy when scaling data size.}
    \label{fig:implicit-vs-explicit}
    \vspace{-5pt}
\end{figure*}


\paragraph{\reb{Scaling}}
In Figure~\ref{fig:future-analysis}, the effectiveness of model scaling in \ourmodel has been observed. Here we explore the impact of increasing the dataset size on the performance. Specifically, we conduct experiments training the \ourmodel \texttt{S-ASA} model with a horizon of 4 and the Transformer \texttt{S-A} using game sizes ranging from 5k to 100k, as shown in the right figure of Figure~\ref{fig:implicit-vs-explicit}. Both the  Transformer and \ourmodel models exhibit a log-2 scaling behavior, showing that doubling the training data results in a linear increase in accuracy. Scaling also enhances future prediction significantly, leading to a more valid and accurate representation of future actions and states, as well as a near-perfect level of capturing the state-action transition dynamics, as detailed in Appendix~\ref{app-sec:scaling}.



\paragraph{Case study}
We sample several challenging puzzles from Lichess (with Elo ratings above 1800) to compare the predictions of \ourmodel and Transformer (\texttt{S-A}). Two instances are shown in Figure~\ref{fig:case-study}, with additional cases provided in Appendix~\ref{appendix:case_study}. 
\ourmodel demonstrates superior foresight, accurately predicting critical exchanges and piece sacrifices that lead to long-term strategic advantages. In the left puzzle, \ourmodel strategically sacrifices the rook to set up a long-term checkmate situation against the opponent. This maneuver compels the opponent to defend and creates an opportunity to capture the queen, facilitating valuable piece exchanges. The \texttt{S-A} model, unfortunately, makes a critical error by focusing on achieving direct checkmate without considering the possibility of the opponent's queen launching a counterattack. \reb{Similarly, in the right puzzle, \ourmodel anticipates an exchange sacrifice, correctly valuing the long-term positional benefits of opening lines by sacrificing the rook for its queen.} Conversely, the \texttt{S-A} model misjudges the value of this exchange, leading to suboptimal moves.
These findings highlight the effectiveness of \ourmodel in long-term planning without relying on explicit search. 

\begin{figure*}[!t]
    \centering
    \includegraphics[width=\linewidth]{figs/case_study/case_crop.pdf}
    \caption{Two examples of Transformer (\texttt{S-A}) 
 and \ourmodel solving challenging puzzles. The predicted next move is in blue for both policies. The predicted future actions from \ourmodel are in light blue and red representing the two players, respectively, along with the numerical counters 1, 2, and 3 indicating future steps.}
    \label{fig:case-study}
    \vspace{-5pt}
\end{figure*}