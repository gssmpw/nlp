%%%%%%%% ICML 2025 EXAMPLE LATEX SUBMISSION FILE %%%%%%%%%%%%%%%%%

\documentclass{article}

% Recommended, but optional, packages for figures and better typesetting:
\usepackage{microtype}
\usepackage{graphicx}
\usepackage{subfig}
\usepackage{booktabs} % for professional tables

% hyperref makes hyperlinks in the resulting PDF.
% If your build breaks (sometimes temporarily if a hyperlink spans a page)
% please comment out the following usepackage line and replace
% \usepackage{icml2025} with \usepackage[nohyperref]{icml2025} above.
\usepackage{hyperref}


% Attempt to make hyperref and algorithmic work together better:
\newcommand{\theHalgorithm}{\arabic{algorithm}}

% Use the following line for the initial blind version submitted for review:
% \usepackage{icml2025}

% If accepted, instead use the following line for the camera-ready submission:
\usepackage[accepted]{icml2025}

% For theorems and such
\usepackage{amsmath}
\usepackage{amssymb}
\usepackage{mathtools}
\usepackage{amsthm}

% \usepackage[dvipsnames]{xcolor}
\usepackage[T1]{fontenc}
\usepackage{pifont}
\usepackage{multirow}
\usepackage{makecell}
\usepackage{bm}

% if you use cleveref..
\usepackage[capitalize,noabbrev]{cleveref}

%%%%%%%%%%%%%%%%%%%%%%%%%%%%%%%%
% THEOREMS
%%%%%%%%%%%%%%%%%%%%%%%%%%%%%%%%
\theoremstyle{plain}
\newtheorem{theorem}{Theorem}[section]
\newtheorem{proposition}[theorem]{Proposition}
\newtheorem{lemma}[theorem]{Lemma}
\newtheorem{corollary}[theorem]{Corollary}
\theoremstyle{definition}
\newtheorem{definition}[theorem]{Definition}
\newtheorem{assumption}[theorem]{Assumption}
\theoremstyle{remark}
\newtheorem{remark}[theorem]{Remark}

\newcommand{\cmark}{\textcolor{LimeGreen}{\ding{51}}}
\newcommand{\xmark}{\textcolor{red}{\ding{55}}}

\newcommand{\yc}[1] {\textcolor{purple}{[yc: #1]}}
\newcommand{\xzl}[1] {\textcolor{blue}{[xzl: #1]}}
\newcommand{\zrz}[1] {\textcolor{olive}{[zrz: #1]}}
% \newcommand{\todo} {\textcolor{red}{[TODO]}}
\newcommand{\done} {\textcolor{red}{[Done]}}


% Todonotes is useful during development; simply uncomment the next line
%    and comment out the line below the next line to turn off comments
%\usepackage[disable,textsize=tiny]{todonotes}
\usepackage[textsize=tiny]{todonotes}


% The \icmltitle you define below is probably too long as a header.
% Therefore, a short form for the running title is supplied here:
\icmltitlerunning{VolleyBots: A Testbed for Multi-Drone Volleyball Game Combining Motion Control and Strategic Play}

\begin{document}


\twocolumn[
% \icmltitle{Submission and Formatting Instructions for \\
           % International Conference on Machine Learning (ICML 2025)}
\icmltitle{VolleyBots: A Testbed for Multi-Drone Volleyball Game \\ Combining Motion Control and Strategic Play}

% It is OKAY to include author information, even for blind
% submissions: the style file will automatically remove it for you
% unless you've provided the [accepted] option to the icml2025
% package.

% List of affiliations: The first argument should be a (short)
% identifier you will use later to specify author affiliations
% Academic affiliations should list Department, University, City, Region, Country
% Industry affiliations should list Company, City, Region, Country

% You can specify symbols, otherwise they are numbered in order.
% Ideally, you should not use this facility. Affiliations will be numbered
% in order of appearance and this is the preferred way.
\icmlsetsymbol{equal}{*}

\begin{icmlauthorlist}
\icmlauthor{Zelai  Xu}{thu}
\icmlauthor{Chao Yu}{thu}
\icmlauthor{Ruize Zhang}{thu}
\icmlauthor{Huining Yuan}{buaa}
\icmlauthor{Xiangmin Yi}{bupt}
\icmlauthor{Shilong Ji}{thu}
\icmlauthor{Chuqi Wang}{thu}
\icmlauthor{Wenhao Tang}{thu}
\icmlauthor{Yu Wang}{thu}
\end{icmlauthorlist}

\icmlaffiliation{thu}{Department of Electronic Engineering, Tsinghua University, Beijing, China}
\icmlaffiliation{buaa}{School of Astronautics, Beihang University, Beijing, China}
\icmlaffiliation{bupt}{School of Artificial Intelligence, Beijing University of Posts and Telecommunications, Beijing, China}


\icmlcorrespondingauthor{Chao Yu}{yuchao@tsinghua.edu.cn}
\icmlcorrespondingauthor{Yu Wang}{yu-wang@tsinghua.edu.cn}

% You may provide any keywords that you
% find helpful for describing your paper; these are used to populate
% the "keywords" metadata in the PDF but will not be shown in the document
% \icmlkeywords{Machine Learning, ICML}

\vskip 0.3in
]

% this must go after the closing bracket ] following \twocolumn[ ...

% This command actually creates the footnote in the first column
% listing the affiliations and the copyright notice.
% The command takes one argument, which is text to display at the start of the footnote.
% The \icmlEqualContribution command is standard text for equal contribution.
% Remove it (just {}) if you do not need this facility.

\printAffiliationsAndNotice{}  % leave blank if no need to mention equal contribution
% \printAffiliationsAndNotice{\icmlEqualContribution} % otherwise use the standard text.

\begin{abstract}
Multi-agent reinforcement learning (MARL) has made significant progress, largely fueled by the development of specialized testbeds that enable systematic evaluation of algorithms in controlled yet challenging scenarios. However, existing testbeds often focus on purely virtual simulations or limited robot morphologies such as robotic arms, quadrupeds, and humanoids, leaving high-mobility platforms with real-world physical constraints like drones underexplored. To bridge this gap, we present \textbf{\textit{VolleyBots}}, a new MARL testbed where multiple drones cooperate and compete in the sport of volleyball under physical dynamics. VolleyBots features a turn-based interaction model under volleyball rules, a hierarchical decision-making process that combines motion control and strategic play, and a high-fidelity simulation for seamless sim-to-real transfer. We provide a comprehensive suite of tasks ranging from single-drone drills to multi-drone cooperative and competitive tasks, accompanied by baseline evaluations of representative MARL and game-theoretic algorithms. Results in simulation show that while existing algorithms handle simple tasks effectively, they encounter difficulty in complex tasks that require both low-level control and high-level strategy. We further demonstrate zero-shot deployment of a simulation-learned policy to real-world drones, highlighting VolleyBots’ potential to propel MARL research involving agile robotic platforms. The project page is at \url{https://sites.google.com/view/thu-volleybots/home}.
\end{abstract}

\section{Introduction}
\label{sec:intro}
% Multi-agent reinforcement learning (MARL) has seen significant success in a range of domains, including competitive board games such as Go~\cite{xxx}, cooperative card games like Hanabi~\cite{bard2020hanabi}, and real-time strategy challenges exemplified by the StarCraft Multi-Agent Challenge (SMAC)~\cite{samvelyan2019starcraft,ellis2024smacv2} and Google Research Football (GRF)~\cite{kurach2020google}. These testbeds have collectively pushed MARL research forward by illustrating how agents can learn cooperation, competition, and the capacity to adapt to complex, multi-agent interactions.

% Yet, many of these established MARL platforms are either purely virtual or place limited emphasis on the physical realism often found in robotic systems. An alternative direction has emerged in the form of robot sports, which can offer rich strategic and physical control requirements in tasks like multi-agent MuJoCo~\cite{peng2021facmac}, robotic table tennis~\cite{d2024achieving}, humanoid football~\cite{liu2022motor,haarnoja2024learning}, and Multi-agent Quadruped Environment (MQE)~\cite{xiong2024mqe}. Sports-based scenarios, in particular, naturally integrate aspects of teamwork, adversarial play, and complex motion control, making them powerful testbeds for studying agent decision-making in both simulation and real-world deployments.

Multi-agent reinforcement learning (MARL) has demonstrated remarkable success across diverse domains, including competitive board games such as Go~\cite{silver2016mastering}, cooperative card games like Hanabi~\cite{bard2020hanabi}, real-time strategy challenges such as the StarCraft Multi-Agent Challenge (SMAC)~\cite{samvelyan2019starcraft,ellis2024smacv2} and Google Research Football (GRF)~\cite{kurach2020google}, as well as human-AI cooperative games like Overcooked~\cite{carroll2019utility}. These testbeds have collectively propelled MARL research forward by showcasing how agents can effectively learn to cooperate, compete, and adapt within complex multi-agent interactions.

\begin{figure*}[t]
    \centering
    % \includegraphics[width=0.8\linewidth]{figs/overview.pdf}
    \includegraphics[width=0.9\linewidth]{figs/overview_b.pdf}
    \caption{Overview of the VolleyBots Testbed. VolleyBots comprises four key components: (1) Environment, supported by Isaac Sim and PyTorch, which defines entities, observations, actions, and reward functions; (2) Tasks, including 3 single-agent tasks, 3 multi-agent cooperative tasks and 2 multi-agent competitive tasks; (3) Algorithms, encompassing RL, MARL, game-theoretic algorithms; and (4) Sim-to-Real Transfer, enabling zero-shot deployment from simulation to real-world environments. }
    \label{fig:overview}
\end{figure*}

Despite the progress in MARL, most established platforms are fully simulated games that do not account for physical-world interactions. To address this gap, several MARL testbeds based on robotic platforms have been developed, such as Multi-Agent MuJoCo(MAMuJoCo)~\cite{peng2021facmac}, Bimanual Dexterous Hands (Bi-DexHands)~\cite{chen2022humanlevelbimanualdexterousmanipulation}, and Multi-agent Quadruped Environment (MQE)~\cite{xiong2024mqe}. Among these, robot sports stand out as a typical task that integrates strategic decision-making under sport-specific rules—such as teamwork and adversarial play—with complex, dynamic motion control constrained by real-world physical factors. Notable examples include robot table tennis~\cite{d2024achieving}, quadrupedal robot football~\cite{ji2022hierarchicalreinforcementlearningprecise}, and humanoid football~\cite{liu2022motor,haarnoja2024learning}. However, robot sports often lack open-source simulation environments and real-world support, limiting their effectiveness as research testbeds. Additionally, these platforms mainly focus on robotic arms, quadrupedal, and humanoid, with limited exploration of high-mobility, agile platforms like drones.

In this work, we introduce a novel MARL testbed named \textbf{\textit{VolleyBots}}, where multiple drones engage in the popular sport of volleyball. This testbed addresses gaps in existing robot sports platforms by incorporating: 
(i) a turn-based interaction model governed by volleyball rules, effectively capturing discrete offensive and defensive phases for MARL studies; 
(ii) a hierarchical decision-making process that combines low-level motion control with high-level strategic play to execute complex flight maneuvers; and
(iii) a high-fidelity simulation that models drone dynamics and interactions between the drones and the ball, and supports flexible parameter randomization, enabling seamless sim-to-real transfer.

% Fig.~\ref{fig:overview} provides an overview of the VolleyBots testbed.
The overview of the VolleyBots testbed is shown in Fig.~\ref{fig:overview}.
Built on Nvidia Isaac Sim~\cite{Mittal_2023}, VolleyBots supports GPU-based rapid data collection, making it highly suitable for RL research. Inspired by the way humans progressively learn the rules of volleyball, we designed a series of tasks ranging from single-drone drills to multi-drone cooperative plays and competitive matchups. In addition, we have implemented baseline MARL and game-theoretic algorithms and provided benchmark results for the proposed tasks. 
% The results reveal that existing algorithms struggle to solve the proposed tasks, especially the multi-agent competitive tasks that require both low-level motion control and high-level strategic play.
The simulation results reveal that existing algorithms perform competently on simple tasks like single-drone control, but struggle to solve complex tasks like multi-drone competitions that require both low-level motion control and high-level strategic play.
% \yc{check this.} 
To demonstrate real-world deployment ability, we show a policy trained to bump volleyball can be deployed on an open-source quadrotor equipped with a racket in a zero-shot manner. We envision VolleyBots as a valuable platform for studying MARL and game-theoretic algorithms with complex drone control tasks.

% In this work, we propose a new testbed named \textbf{VolleyBots}, fills a gap in existing robot sports contexts by featuring (i) a turn-based interaction (ii) a hierarchical decision process, and (iii) a focus on sim-to-real transfer. These attributes underscore the environment’s ability to capture discrete offensive and defensive turns, incorporate layered high-level strategies and low-level drone control, and align closely with real-world drone specifications and dynamics.

% Our VolleyBots framework is built on a high-fidelity simulation stack and encompasses a variety of tasks ranging from single-drone drills to multi-drone cooperative plays and competitive matchups. Figure~\ref{fig:overview} illustrates the overall setup, including the environment entities, the categories of tasks (single-agent, multi-agent cooperation, and multi-agent competition), the algorithmic baselines considered in our benchmark, and the eventual pathway to real hardware. By combining realistic flight dynamics, ball manipulation, and volleyball rules, VolleyBots aims to bridge the gap between purely virtual MARL testbeds and practical robotics applications.

Our main contributions are summarized as follows:
\begin{enumerate}
    \item We introduce VolleyBots, a multi-agent drone environment that operates under turn-based interactions, combines hierarchical decision-making, and is compatible with sim-to-real deployments.
    % \item We provide a high-fidelity simulation encompassing drone flight and ball dynamics, with multiple control interfaces ranging from thrust inputs to collective thrust and body rates.
    \item We release diverse benchmark tasks and baseline evaluations of representative MARL and game-theoretic algorithms, thereby facilitating reproducible research and comparative assessments.
    \item We show the sim-to-real feasibility of transferring learned policies to physical drones, highlighting the environment’s alignment with real-world parameters.
\end{enumerate}

% Reinforcement learning has demonstrated remarkable advances in single-agent settings across a wide variety of domains, including arcade-style tasks, continuous control benchmarks, board games, and complex esports environments. However, numerous real-world scenarios, such as autonomous driving fleets, multi-robot coordination, and collaborative drone operations, inherently involve multiple learning agents. These situations have brought attention to multi-agent reinforcement learning (MARL), which addresses the interplay of cooperation, competition, and scalability among distributed autonomous agents.

% Despite considerable progress in MARL, existing testbeds commonly exhibit limitations in three areas: (i) turn-based interactions, (ii) hierarchical decision processes, and (iii) sim-to-real transfer. Most platforms adopt mechanics based on concurrent rather than turn-based actions, which may obscure temporal dependencies and dilute clear transitions between offense and defense. In addition, although hierarchical structures are crucial in real-world tasks—where agents must handle low-level control like flight maneuvers and high-level strategy like offensive or defensive roles—many benchmarks do not explicitly support such layered decision processes. Furthermore, sim-to-real validation, which is vital for practical deployments, is often overlooked when platforms rely solely on virtual or simplified physics.

% To address these shortcomings, this work introduces Drone Volleyball, a novel MARL testbed that combines turn-based gameplay, hierarchical control requirements, and a high-fidelity physical simulator suitable for sim-to-real transfer. In this environment, multiple drones cooperate or compete under volleyball rules, navigating three-dimensional space, passing or spiking the ball, and scoring by placing the ball in the opponent’s court. Unlike concurrent-action simulations, volleyball is explicitly turn-based: once one side completes its sequence of hits, control shifts to the opposing side. This structure offers distinct intervals of offense and defense, allowing agents to anticipate opposing strategies and develop countermeasures. From a control perspective, the platform demands low-level drone maneuvers—such as motor thrust and orientation control—while also rewarding high-level tactical decisions—such as coordinating with teammates for effective spikes. By integrating carefully calibrated quadrotor dynamics and realistic collision modeling, Drone Volleyball bridges the gap between purely virtual MARL environments and real-world flight experiments, making it possible to train in simulation and subsequently deploy policies on physical drones.

% The main contributions of this work are summarized as follows:
% \begin{enumerate}
%     \item A new multi-agent testbed named Drone Volleyball that is turn-based, hierarchically challenging, and aligned with sim-to-real objectives, thus filling a critical gap in MARL environments.
%     \item A high-fidelity simulation encompassing drone flight and ball dynamics, including interfaces for low-level thrust control, trajectory-based maneuvers, and high-level volleyball tactics.
%     \item Demonstration of the feasibility of sim-to-real transfer through alignment with physical drone parameters, enabling seamless migration of trained policies to real hardware.
%     \item Release of diverse benchmark scenarios, an open-source application programming interface, and baseline performance assessments for several representative MARL algorithms to facilitate reproducible and comparable research.
% \end{enumerate}


\section{Related Work}
\label{sec:related}
\begin{table*}[t]
    \centering
    
% \begin{wraptable}{r}{0.65\textwidth}
% \centering
% \caption{The comparison of resource requirements between Eurus-2-7B-PRIME and Qwen2.5-Math-7B-Instruct.}
% \label{tab:comparision}
% \resizebox{0.65\textwidth}{!}{
% \begin{tabular}{l >{\columncolor[HTML]{D7E8E8}}l l}
% \toprule
% \textbf{Model} & \textbf{Eurus-2-7B-PRIME} & \textbf{Qwen2.5-Math-7B-Instruct} \\ \midrule
% Base Model     & Qwen2.5-Math-7B           & Qwen2.5-Math-7B                  \\
% SFT Data       & 230K (open-source)        & 2.5M (open-source and in-house)  \\
% RM Data        & 0                         & 618K (in-house)                 \\
% RM             & Eurus-2-7B-SFT            & Qwen2.5-Math-RM (72B)           \\
% RL Data        & 150K queries $\times$ 4 samples & 66K queries $\times$ 32 samples \\ \bottomrule
% \end{tabular}
% }
% \end{wraptable}

\begin{wraptable}{r}{0.65\textwidth}
\centering
\caption{The comparison of resource requirements between Eurus-2-7B-PRIME and Qwen2.5-Math-7B-Instruct.}
\label{tab:comparision}
% \resizebox{0.65\textwidth}{!}{
\resizebox{\linewidth}{!}{
\begin{tabular}{l >{\columncolor[HTML]{D7E8E8}}l l}
\toprule
\textbf{Model} & \textbf{Eurus-2-7B-PRIME} & \textbf{Qwen2.5-Math-7B-Instruct} \\ \midrule
Base Model     & Qwen2.5-Math-7B           & Qwen2.5-Math-7B                  \\
SFT Data       & 230K (open-source)        & 2.5M (open-source \& in-house)  \\
RM Data        & 0                         & 618K (in-house)                 \\
RM             & Eurus-2-7B-SFT            & Qwen2.5-Math-RM (72B)           \\
RL Data        & 150K queries $\times$ 4 samples & 66K queries $\times$ 32 samples \\ \bottomrule
\end{tabular}
}
\end{wraptable}





% 字体标橙色
% \begin{wraptable}{r}{0.65\textwidth}  % r表示表格在右侧,0.5\textwidth表示表格宽度为文本宽度的50%
% \centering
% \caption{The comparison of resource requirements between Eurus-2-7B-PRIME and Qwen2.5-Math-7B-Instruct.}
% \label{tab:comparision}
% \resizebox{0.65\textwidth}{!}{
% \begin{tabular}{lll}
% \toprule
% \textbf{Model} & {\color[HTML]{F8A102}\textbf{Eurus-2-7B-PRIME}}                             & \textbf{Qwen2.5-Math-7B-Instruct}            \\ \midrule
% Base Model     & {\color[HTML]{F8A102}Qwen2.5-Math-7B}                                       & Qwen2.5-Math-7B                              \\
% SFT Data       & {\color[HTML]{F8A102}\textbf{230K (open-source)}}                           & 2.5M (open-source and in-house)              \\
% RM Data        & {\color[HTML]{F8A102}\textbf{0}}                                            & 618K (in-house)                              \\
% RM             & {\color[HTML]{F8A102}\textbf{Eurus-2-7B-SFT}}                               & Qwen2.5-Math-RM (72B)                        \\
% RL Data        & {\color[HTML]{F8A102}\textbf{150K queries $\times$ 4 samples}} & 66K queries $\times$ 32 samples \\ \bottomrule
% \end{tabular}

% }
% \end{wraptable}




% \begin{table}[]
% \centering
% \caption{The comparison of resource requirements between Eurus-2-7B-PRIME and Qwen2.5-Math-7B-Instruct.\hanbin{Embed into text}}
% \label{tab:comparision}
% \resizebox{0.8\textwidth}{!}{
% \begin{tabular}{lll}
% \midrule
% \textbf{Model} & \textbf{Eurus-2-7B-PRIME}                             & \textbf{Qwen2.5-Math-7B-Instruct}            \\ \midrule
% Base Model     & Qwen2.5-Math-7B                                       & Qwen2.5-Math-7B                              \\
% SFT Data       & \textbf{230K (open-source)}                           & 2.5M (open-source and in-house)              \\
% RM Data        & \textbf{0}                                            & 618K (in-house)                              \\
% RM             & \textbf{Eurus-2-7B-SFT}                               & Qwen2.5-Math-RM (72B)                        \\
% RL Data        & \textbf{150K queries $\times$ 4 samples} & 66K queries $\times$ 32 samples \\ \midrule
% \end{tabular}
% }
% \end{table}


    \caption{Comparison of VolleyBots and existing representative MARL testbeds.}
    \label{tab:comparison}
\end{table*}

\subsection{Reinforcement Learning for Drone Control Task}
Executing precise and agile flight maneuvers is essential for drones, which has driven the development of diverse control strategies~\cite{bouabdallah2004design,7989202,hwangbo2017control}. Among these, RL has shown significant promise, offering flexibility and efficiency in drone control. 
% Unlike traditional methods, RL-based policies can directly map observations to actions, eliminating the need for detailed system dynamics modeling or strict actuation constraints.
Drone racing is a notable single-drone control task where RL has achieved human-level performance~\cite{kaufmann2023champion}, showcasing near-time-optimal decision-making capabilities. Beyond racing, researchers also leveraged RL for executing aggressive flight maneuvers~\cite{sun2022aggressive} and achieving hovering stabilization under highly challenging conditions~\cite{hwangbo2017control}. As for multi-drone tasks, RL has been applied to cooperative tasks such as formation maintenance~\cite{swarm-formation}, as well as more complex scenarios like multi-drone pursuit-evasion tasks~\cite{chen2024multiuavpursuitevasiononlineplanning}, further showcasing its potential to jointly optimize task-level planning and control.
In this paper, we present VolleyBots, an MARL testbed designed to study the novel drone control task of drone volleyball. This task introduces unique challenges, requiring drones to learn both cooperative and competitive strategies at the task level while maintaining agile and precise control. Additionally, VolleyBots provides a comprehensive platform with baseline implementations of (MA)RL and game-theoretic algorithms, as well as support for sim-to-real transfer, facilitating the development and evaluation of advanced drone control strategies.


% difference: multi, strategic hard
\subsection{MARL Testbeds}
Many existing MARL testbeds focus on fully simulated environments without real-world physical interactions. For instance, AlphaGo~\cite{silver2016mastering} explores the turn-based game of Go, sparking deep RL research. Subsequent environments like SMAC~\cite{samvelyan2019starcraft,ellis2024smacv2}, Overcooked~\cite{carroll2019utility}, and Hanabi~\cite{bard2020hanabi} simulate cooperative scenarios using video games or card games. Multi-agent Particle Environments (MPE)~\cite{lowe2020multiagentactorcriticmixedcooperativecompetitive} and GRF~\cite{kurach2020google} provide a variety of cooperative, competitive, and mixed cooperative-competitive tasks. These testbeds emphasize high-level decision-making and largely focus on discrete action spaces, overlooking real-world continuous control tasks with physical constraints.

Recently, several MARL environments have been proposed to incorporate real-world physics and interactions. For example, MAMuJoCo~\cite{peng2021facmac} offers multi-agent continuous control tasks, and Bi-DexHands~\cite{chen2022humanlevelbimanualdexterousmanipulation} focuses on dexterous bimanual manipulation with robotic hands. While these testbeds explore continuous action spaces, they are limited to cooperative tasks. Competitive sports, much like their societal role in humans, provide a standard way to evaluate decision-making under dynamic, rule-constrained, and physically realistic conditions. Some works have explored this direction, such as Robot Table Tennis~\cite{d2024achieving}, achieving near human-level performance in ping-pong, and Humanoid Football~\cite{liu2022motor,haarnoja2024learning}, which models 2 vs 2 football with humanoid robots. Despite bringing MARL closer to real-world applications, these platforms often lack open-source simulations or real-world deployment support. Open-source testbeds like SMPLOlympics~\cite{luo2024smplolympicssportsenvironmentsphysically} offer physically simulated environments for humanoids to compete in Olympic sports, and MQE~\cite{xiong2024mqe} proposes multi-agent locomotion tasks with quadrupeds, including 1 vs 1 and 2 vs 2 football competitions.
However, existing testbeds primarily focus on robotic arms, quadrupeds, and humanoids, leaving high-mobility, agile robot platforms like drones underexplored.

To address these gaps, we introduce VolleyBots, a turn-based, drone-focused sports environment featuring high-level decision-making and low-level continuous control. A comprehensive comparison between existing MARL testbeds and VolleyBots is shown in Table~\ref{tab:comparison}. Built on a realistic physical simulator, VolleyBots supports real-world deployment and offers a complementary testbed for advancing MARL research with agile robotic platforms.

% \subsection{MARL \yc{Testbeds}}
% \yc{Many existing MARL testbeds focus on fully simulated environments without real-world physical interactions. For instance, AlphaGo~\cite{silver2016mastering} explores the turn-based game of Go, sparking deep RL research. Subsequent environments like SMAC~\cite{samvelyan2019starcraft,ellis2024smacv2}, Overcooked~\cite{carroll2019utility}, and Hanabi~\cite{bard2020hanabi} simulate cooperative scenarios using video games or card games. Multi-agent Particle Environments (MPE)~\cite{lowe2020multiagentactorcriticmixedcooperativecompetitive} and GRF~\cite{kurach2020google} provide a variety of cooperative, competitive, and mixed cooperative-competitive tasks. These testbeds emphasize high-level decision-making and largely focus on discrete action spaces, overlooking real-world continuous control tasks.}

% \yc{Recently, several MARL environments have been proposed to incorporate real-world physics and interactions. For example, MAMuJoCo~\cite{peng2021facmac} offers multi-agent continuous control tasks, and Bi-DexHands~\cite{chen2022humanlevelbimanualdexterousmanipulation} focuses on dexterous bimanual manipulation with robotic hands. While these testbeds explore continuous action spaces, they are limited to cooperative tasks. Competitive sports, much like their societal role in humans, provide a standard way to evaluate decision-making under dynamic, rule-constrained, and physically realistic conditions. Some works have explored this direction, such as Robot Table Tennis~\cite{d2024achieving}, achieving near human-level performance in ping-pong, and Humanoid Football~\cite{liu2022motor,haarnoja2024learning}, which models 2 vs. 2 football with humanoid robots. Despite bringing MARL closer to real-world applications, these platforms often lack open-source simulations or real-world deployment support. Open-source testbeds like SMPLOlympics~\cite{luo2024smplolympicssportsenvironmentsphysically} offer physically simulated environments for humanoids to compete in Olympic sports, and MQE~\cite{xiong2024mqe} proposes multi-agent locomotion tasks with quadrupeds, including 1 vs. 1 and 2 vs. 2 football competitions.
% However, existing testbeds primarily focus on humanoids and quadrupeds, leaving high-mobility, agile robots like drones underexplored.}

% \yc{To address these gaps, we introduce VolleyBots, a turn-based, drone-focused sports environment featuring high-level decision-making and low-level continuous control. A comprehensive comparison between existing MARL testbeds and VolleyBots is shown in Table~\ref{tab:comparison}. Built on a realistic physical simulator, VolleyBots supports real-world deployment and offers a complementary testbed for advancing MARL research with agile robotic platforms.}

% A number of popular multi-agent reinforcement learning benchmarks focus on purely simulated environments without real-world physical interactions. 
% AlphaGo~\cite{silver2016mastering} considers the ancient turn-based competitive game of Go and sparked a wave of deep reinforcement learning research.
% Several MARL environments are then proposed to advance research on multi-agent cooperation. 
% StarCraft Multi-Agent Challenge (SMAC)~\cite{samvelyan2019starcraft,ellis2024smacv2} considers the well-known real-time strategy game, Overcooked~\cite{carroll2019utility} simulates the popular video game, and Hanabi uses a turn-based card game as environments for study on multi-agent cooperation. Multi-agent Particle Environment (MPE)~\cite{lowe2020multiagentactorcriticmixedcooperativecompetitive} is an intuitive and accessible testbed designed for studying MARL algorithms in a simplified particle-based setting.
% Google Research Football (GRF)~\cite{kurach2020google} models after popular football video games and proposes a variety of cooperative, competitive, and mixed cooperative-competitive tasks.
% While these benchmarks have propelled MARL research through their accessibility and controlled complexity, most of them remain purely simulated without considering real-world dynamics and physical interactions.
% In addition, these environments mainly consider discrete action spaces, making them unsuitable for studying continuous control tasks.


% In contrast, several MARL environments integrate real-world physics and interactions into their design. For instance, MAMuJoCo~\cite{peng2021facmac} provides continuous control challenges using a multi-agent MuJoCo simulator, and Bi-DexHands~\cite{chen2022humanlevelbimanualdexterousmanipulation} emphasizes dexterous bimanual manipulation on robotic hands. A growing line of work explores robot sports, where robotic agents cooperate and compete in sports-like tasks.
% Robot Table Tennis~\cite{d2024achieving} achieves near human-level performance in real-world ping-pong using a robotic arm, while Humanoid Football~\cite{liu2022motor,haarnoja2024learning} models 2 vs 2 football with humanoid robots in simulation, showcasing human-like movement and behavior. The Multi-agent Quadruped Environment (MQE)~\cite{xiong2024mqe} further expands on multi-agent quadrupedal tasks, including 1 vs 1 and 2 vs 2 football competitions. Although these sports-oriented platforms bring MARL closer to actual robotic deployment, they often lack open-source simulation or real-world deployment. In addition, existing works have limited exploration of tasks with high-mobility and agile robots like drones.
% In response, our proposed VolleyBots environment introduces a turn-based, drone-centered sports setting with continuous control and real-world deployment support, offering a complementary testbed for high-mobility, agile platforms in multi-agent research.
% A comprehensive comparison between existing environments and our VolleyBots environment is shown in Table~\ref{tab:comparison}.



\section{VolleyBots Environment}
\label{sec:env}
% \yc{In this section, we introduce the environment components of the VolleyBots testbed. The environment is built upon the high-throughput and GPU-parallelized OmniDrones~\cite{xu2024omnidrones} simulator, which itself relies on Isaac~Sim~\cite{Mittal_2023} to facilitate rapid data collection. We further configure OmniDrones to simulate realistic flight dynamics and interaction between the drone and ball, then overlay standard volleyball rules and gameplay mechanics to create a challenging domain for multi-agent reinforcement learning. We will describe the simulation entity, observation space, action space, and reward functions in the following subsections.}

In this section, we introduce the environment design of the VolleyBots testbed. The environment is built upon the high-throughput and GPU-parallelized OmniDrones~\cite{xu2024omnidrones} simulator, which relies on Isaac~Sim~\cite{Mittal_2023} to facilitate rapid data collection. We further configure OmniDrones to simulate realistic flight dynamics and interaction between the drones and the ball, then implement standard volleyball rules and gameplay mechanics to create a challenging domain for drone control tasks. We will describe the simulation entity, observation space, action space, and reward functions in the following subsections.
% \yc{check}
% This environment offers flexible control modes, ranging from low-level to high-level commands and supports seamless integration of multi-drone tasks, making it well-suited for simulating volleyball-inspired scenarios. 
% In particular, we configure OmniDrones to simulate realistic flight dynamics and collision responses, then overlay standard volleyball rules and gameplay mechanics to create a challenging domain for multi-agent reinforcement learning.

% \subsection{\yc{Simulation Entity}}
\subsection{Simulation Entity}

Our environment simulates real-world physics dynamics and interactions of three key components including the drones, the ball, and the court.
We provide a flexible configuration of each entity's model and parameters to enable a wide range of task designs. For the default configuration, we adopt the \textit{Iris} quadrotor model~\cite{furrer2016rotors} as the primary drone platform, augmented with a virtual ``racket'' of radius $0.2\,\text{m}$ and coefficient of restitution $0.8$ for ball striking. The ball is modeled as a sphere with a radius of $0.1\,\text{m}$, a mass of $5\,\text{g}$, and a coefficient of restitution of $0.8$, enabling realistic bounces and interactions with both drones and the environment. The court follows standard volleyball dimensions of $9\,\text{m} \times 18\,\text{m}$ with a net height of $2.43\,\text{m}$. All these models and parameters can be easily modified and randomized to facilitate sim-to-real transfer, and a zero-shot real-world deployment example will be presented in Sec.~\ref{sec:sim2real}.

% Our simulation mimics a standard volleyball court measuring \(9\,\text{m} \times 18\,\text{m}\) with a net height of \(2.43\,\text{m}\). We employ an \textit{Iris} quadrotor model \yc{cite?} as the primary drone platform, augmented with a virtual ``racket'' of radius \(0.2\,\text{m}\) and coefficient of restitution \(X\) for ball striking. The ball itself is modeled as a sphere of radius \(0.1\,\text{m}\), mass \(5\,\text{g}\), and coefficient of restitution \(X\), enabling realistic bounces and interactions with both drones and the environment. 
% % By adhering closely to real-world parameters, this setup facilitates potential sim-to-real transfers in future applications.
% \yc{Note that these parameters are configured for algorithmic studies. Our simulation environment supports flexible physical configurations, allowing researchers to programmatically generate new drone models based on existing ones and modify ball parameters for sim-to-real transfer. A zero-shot deployment example will be presented in Sec.~\ref{sec:sim2real}.}
% (TODO) 

\subsection{Observation Space}
% \yc{for MARL, state space?}
% \xzl{our tasks are partially observable, so I think observation space is more appropriate.}

To align with the feature of partial observability in real-world volleyball games, we adopt a state-based observation space where each drone can fully observe its own physical state and partially observe the ball's state and other drones' state. 
More specifically, each drone has full observability of its position, rotation, velocity, angular velocity, and other physical states. For ball observation, each drone can only partially observe the ball's position and velocity. In multi-agent tasks, each drone can also partially observe other drones' position and velocity. Minor variations in the observation space may be required for different tasks, such as the ID of each drone in multi-agent tasks. Detailed observation configurations for each task are provided in the Appendix~\ref{app:task}.

% We adopt a state-based observation scheme aimed at providing each drone with a comprehensive understanding of its own physical state while offering only fundamental information about the ball and other drones.
% \yc{Specifically, per-drone observations include the drone’s position, velocity, orientation, angular velocity, and other relevant proprioceptive measurements; the ball’s position and velocity; and, if applicable, partial or relative data about other drones, such as the positions of teammates or opponents. Minor variations in the observation specification may be required for different tasks, such as incorporating additional team-based cues. Detailed observation configurations for each task are provided in the Appendix \ref{app:task}.}

% In particular, the per-drone observations include: 
% Drone states like position, velocity, orientation, angular velocity, and other relevant proprioceptive measurements;
% Ball states like position and velocity of the ball;
% Other drones states (if applicable) like partial or relative data such as positions of teammates or opponents. 
% Different tasks may require minor variations in the observation specification (e.g., additional team-based cues). Full details are provided in the Appendix.


\subsection{Action Space}

We provide two types of continuous action spaces that differ in their level of control, with Collective Thrust and Body Rates (CTBR) offering a higher-level abstraction and Per-Rotor Thrust (PRT) offering a more fine-grained manipulation of individual rotors.

\textbf{Collective Thrust and Body Rates.}
A typical mode of drone control is to specify a single collective thrust command along with body rates for roll, pitch, and yaw. This higher-level abstraction hides many hardware-specific parameters of the drone, often leading to more stable training. It also simplifies sim-to-real transfer by reducing the reliance on precise modeling of individual rotor dynamics.

\textbf{Per-Rotor Thrust.}
Alternatively, the drone can directly control each rotor’s thrust. This fine-grained control allows the policy to fully exploit the drone’s agility and maneuverability. However, it typically demands a more accurate model of the drone’s hardware and may increase the difficulty of sim-to-real deployment. 


% The action space is \yc{continuous and} configurable to accommodate multiple levels of drone control, ranging from low-level thrust commands to higher-level abstractions:
% \paragraph{Thrust} The drone receives normalized motor thrust values, enabling a high degree of fine-grained control over flight dynamics. This approach is useful for testing agility or investigating learned control strategies that map directly to rotor outputs.
% \paragraph{CTBR} The drone can alternatively be controlled via collective thrust and body rates (CTBR). This higher-level representation abstracts away individual rotor management and is often more stable to train\yc{unsure about this, the experiments can support this? }, particularly for tasks focusing on tactical or strategic decision-making. \yc{CTBR is more robust for sim2real gap.}

\subsection{Reward Functions}

The reward function for each task consists of three parts, including the misbehave penalty for general motion control, the task reward for task completion, and the shaping reward to accelerate training.
% Across all tasks, we employ a three-part reward design. \yc{The detailed reward function of each task can be found in Appendix \ref{app:task}.}

\textbf{Misbehave Penalty.}
This term is consistent across all tasks and penalizes undesirable behaviors related to general drone motion control, such as crashes, collisions, and invalid hits. By imposing penalties for misbehavior, the drones are guided to maintain physically plausible trajectories and avoid actions that could lead to control failure.

\textbf{Task Reward.}
Each task features a primary objective-based reward that encourages the successful completion of the task. For example, in solo bump tasks, the drone will get a reward of $1$ for each successful hit of the ball. Since the task rewards are typically sparse, agents must rely on effective exploration to learn policies that complete the task.

\textbf{Shaping Reward.}
Due to the sparse nature of many task rewards, relying solely on the misbehave penalty and the task reward can make it difficult for agents to successfully complete the tasks. To address this challenge, we introduce additional shaping rewards in multi-agent tasks to help steer the learning process. For example, the drone's movement toward the ball is rewarded when a hit is required. By providing additional guidance, the shaping rewards significantly accelerate learning in complex tasks.

We note that while our reward and observation design is effective for algorithm testing and yields sensible results, it is not optimized for real-world deployment scenarios. To encourage further exploration, we provide a flexible user API to enable experiments with improved designs.

% \yc{We note that while our initial reward and observation design is effective for algorithm testing and yields sensible results, it is not optimized for real-world deployment scenarios. To encourage further exploration, we provide a flexible user API, enabling researchers to experiment with improved designs.}



\section{VolleyBots Tasks}
\label{sec:task}
\begin{figure*}[t]
    \centering
    \includegraphics[width=0.9\linewidth]{figs/tasks.pdf}
    % \caption{\yc{Proposed tasks in the VolleyBots testbed, inspired by the process of human learning in volleyball. Single-agent tasks evaluate low-level control, while multi-agent cooperative and competitive tasks integrate high-level decision-making with low-level control.}}
    \caption{Proposed tasks in the VolleyBots testbed, inspired by the process of human learning in volleyball. Single-agent tasks evaluate low-level control, while multi-agent cooperative and competitive tasks integrate high-level decision-making with low-level control.}
    \label{fig:tasks}
\end{figure*}

% \yc{eval metric}
% \yc{Inspired by the way humans progressively learn to play volleyball, we introduce a series of tasks that systematically assess both low-level flight control and higher-level coordination, as shown in Fig.~\ref{fig:tasks}.}
Inspired by the way humans progressively learn to play volleyball, we introduce a series of tasks that systematically assess both low-level motion control and high-level strategic play, as shown in Fig.~\ref{fig:tasks}.
% Building on the described environment and setup, we introduce a suite of volleyball-inspired tasks that systematically evaluate both low-level flight control and higher-level coordination. 
These tasks are organized into three categories: single-agent, multi-agent cooperative, and multi-agent competitive. Each category aligns with standard volleyball drills or match settings commonly adopted in human training, ranging from basic ball control, through cooperative play, to full competitive games. Evaluation metrics vary across tasks to assess performance in motion control, cooperative teamwork, and strategic competition. The detailed configuration and reward design of each task can be found in Appendix~\ref{app:task}.
% \yc{when will the task be done? the sim duration is? the max episode length?}

\subsection{Single-Agent Tasks}

Single-agent tasks are designed to follow typical solo training drills used in human volleyball practice, including \textit{Back and Forth}, \textit{Hit the Ball}, and \textit{Solo Bump}. These tasks evaluate the drone's basic capabilities such as flight stability, motion control, and ball-handling proficiency.

\textbf{Back and Forth.}
The drone sprints between two designated points to complete as many round trips as possible within the time limit. This task is analogous to the back-and-forth sprints in volleyball practice and requires basic motion control of the drone. The performance is evaluated by the number of completed round trips within the time limit.


\textbf{Hit the Ball.}
The ball is initialized directly above the drone, and the drone hits the ball once to make it land as far as possible. This task is analogous to the typical hitting drill in volleyball and requires both motion control and ball-handling proficiency. The performance is evaluated by the distance of the ball's landing position from the initial position.

\textbf{Solo Bump.}
The ball is initialized directly above the drone, and the drone bumps the ball in place to a specific height as many times as possible within the time limit. This task is analogous to the solo bump drill in human practice and requires motion control, ball-handling proficiency, and stability. The performance is evaluated by the number of successful bumps within the time limit.

% \subsection{Multi-Agent Cooperative Tasks}

% Cooperative tasks focus on pairwise or group-based collaboration to maintain or direct the ball. These tasks parallel common two-player routines in volleyball coaching, such as back-and-forth passing and setting–spiking drills. Agents must coordinate their movements and share roles to complete each objective efficiently.

\begin{table*}[t]
    \centering
    \begin{tabular}{ccccccc}
\toprule
 & \multicolumn{2}{c}{\textit{Back and Forth}} & \multicolumn{2}{c}{\textit{Hit the Ball}} & \multicolumn{2}{c}{\textit{Solo Bump}} \\
       & CTBR & PRT & CTBR & PRT & CTBR & PRT \\
\midrule
% DDPG &  $0.81_{(0.06)}$ &  $0.89_{(0.10)}$  & $3.48_{(0.55)}$ & $2.89_{(1.07)}$ & $0.00_{(0.00)}$ & $0.02_{(0.03)}$ \\
% PPO  & $9.10_{(0.22)}$ & $9.90_{(0.05)}$  & $10.46_{(0.07)}$ & $11.43_{(0.04)}$  & $11.19_{(0.31)}$ & $11.23_{(0.86)}$  \\
DDPG &  $0.81 \pm 0.06$ &  $0.89 \pm 0.10$  & $3.48 \pm 0.55$ & $2.89 \pm 1.07$ & $0.00 \pm 0.00$ & $0.02 \pm 0.03$ \\
PPO  & $\bm{9.10 \pm 0.22}$ & $\bm{9.90 \pm 0.05}$  & $\bm{10.46 \pm 0.07}$ & $\bm{11.43 \pm 0.04}$  & $\bm{11.19 \pm 0.31}$ & $\bm{11.23 \pm 0.86}$  \\
% SAC  &      &         &      &         &      &         \\
\bottomrule 
\end{tabular}

    \caption{Benchmark result of single-agent tasks with different action spaces including Collective Thrust and Body Rates (CTBR) and Per-Rotor Thrust (PRT). \textit{Back and Forth} is evaluated by the number of target points reached, \textit{Hit the Ball} is evaluated by the hitting distance, and \textit{Solo Bump} is evaluated by the number of bumps achieving a certain height.}
    \label{tab:single}
\end{table*}

\subsection{Multi-Agent Cooperative Tasks}

Multi-agent cooperative tasks are inspired by standard two-player training drills used in volleyball teamwork, including \textit{Bump and Pass}, \textit{Set and Spike (Easy)}, and \textit{Set and Spike (Easy)}. Besides basic motion control and ball handling, these tasks also require teamwork and cooperation.

\textbf{Bump and Pass.}
Two drones work together to bump and pass the ball to each other back and forth as many times as possible within the time limit. This task is analogous to the two-player bumping practice in volleyball training and requires homogeneous multi-agent cooperation. The performance is evaluated by the number of successful bumps within the time limit.

\textbf{Set and Spike (Easy).}
Two drones take on the role of a setter and an attacker. The setter passes the ball to the attacker, and the attacker then spikes the ball downward to the target region on the opposing side. This task is analogous to the setter-attacker offensive drills in volleyball training and requires heterogeneous multi-agent cooperation. The performance is evaluated by the success rate of the downward spike to the target region.

\textbf{Set and Spike (Hard).}
Similar to \textit{Set and Spike (Easy)} task, two drones act as a setter and an attacker to set and spike the ball to the opposing side. The difference is that there is a rule-based defense board on the opposing side to intercept the attacker's spike. The presence of the defense board further improves the difficulty of the task, requiring the drones to optimize their speed, precision, and cooperation to defeat the defense board. The performance is evaluated by the success rate of the downward spike that defeats the defense racket.

\subsection{Multi-Agent Competitive Tasks}

Multi-agent competitive tasks follow the standard volleyball match rules, including the competitive \textit{1 vs 1} task and the mixed cooperative-competitive \textit{3 vs 3} task. These tasks evaluate both the low-level motion control and the high-level strategic play of the drone policy. 
% Detailed discussion of tasks and evaluation metrics can be found in Appendix~\ref{app:1v1}.

\textbf{1 vs 1.}
One drone on each side competes against the other in a volleyball match and wins by hitting the ball in the opponent's court. When the ball is on its side, the drone is allowed only one hit to return the ball to the opponent's court. This two-player zero-sum setting creates a purely competitive environment that requires both precise flight control and strategic gameplay.
To evaluate the performance of the learned policy, we consider three typical metrics including the exploitability, the average win rate against other learned policies, and the Elo rating~\cite{elo1978rating}. More specifically, the exploitability is approximated by the gap between the learned best response’s win rate against the evaluated policy and its expected win rate at Nash equilibrium, and the Elo rating is computed by running a round-robin tournament between the evaluated policy and a fixed population of policies.

% One drone on each side competes to win by landing the ball in the opponent’s court, alternating possessions according to volleyball rules. The minimal team size highlights individual skill, reaction speed, and strategic decision-making. We evaluate policies through multiple metrics, including exploitability, win rates against a built-in defense board, and cross-play performance in round-robin tournaments that yield overall win rates and Elo ratings~\cite{elo1978rating}, which are typical metrics to evaluate the performance in competitive games.
% \yc{cite? which is a typical metric to evaluate competitive game?}.

\textbf{3 vs 3.}
Three drones on each side form a team to compete against the other team in a volleyball match. The drones in the same team cooperate to serve, pass, spike, and defend within the standard rule of three hits per side. This is a challenging mixed cooperative-competitive game that requires both cooperation within the same team and competition between the opposing teams.
Moreover, the drones are required to excel at both low-level motion control and high-level game play.  
We evaluated the policy performance using approximate exploitability, the average win rate against other learned policies, and the Elo rating of the policy.



\section{Benchmark Results}
\label{sec:benchmark}
\begin{table*}[t]
    \centering
    \begin{tabular}{ccccccc}
\toprule
 & \multicolumn{2}{c}{\textit{Bump and Pass}} & \multicolumn{2}{c}{\textit{Set and Spike (Easy)}} & \multicolumn{2}{c}{\textit{Set and Spike (Hard)}} \\
       & w.o. shaping & w. shaping & w.o. shaping & w. shaping & w.o. shaping & w. shaping \\
\midrule
MADDPG & $0.88 \pm 0.07$ & $0.90 \pm 0.01$ & $0.23 \pm 0.01$ & $0.24 \pm 0.01$  & $0.24 \pm 0.00$ & $0.23 \pm 0.00$ \\
MAPPO  & $\bm{11.17 \pm 1.03}$ & $\bm{14.05 \pm 0.36}$ & $\bm{0.25 \pm 0.00}$ & $\bm{0.99 \pm 0.00}$ & $\bm{0.25 \pm 0.00}$ & $0.75 \pm 0.01$ \\
HAPPO  & $6.65 \pm 3.59$ & $12.35 \pm 0.67$ & $\bm{0.25 \pm 0.00}$ & $0.98 \pm 0.00$ & $\bm{0.25 \pm 0.00}$ & $\bm{0.82 \pm 0.11}$ \\
MAT    & $7.68 \pm 4.9$ & $13.41 \pm 0.14$ & $\bm{0.25 \pm 0.00}$ & $\bm{0.99 \pm 0.00}$ & $\bm{0.25 \pm 0.00}$ & $0.75 \pm 0.00$ \\
\bottomrule 
\end{tabular}

% \begin{tabular}{ccccccc}
%     \toprule
%     & back-and-forth & bumping & hit & bump and pass & attack (easy) & attack (hard) \\
%     \midrule
%     MADDPG  &  &  &  &  &  &  \\
%     MAPPO   &  &  &  &  &  &  \\
%     HAPPO   &  &  &  &  &  &  \\
%     MAT  &  &  &  &  &  &  \\
%     \bottomrule 
% \end{tabular}

% \begin{tabular}{ccccc}
%     \toprule
%     & MADDPG & MAPPO & HAPPO & MAT \\
%     \midrule
%     back-and-forth  &  &  &  & \\
%     juggle   &  &  &  & \\
%     hit   &  &  &  & \\
%     dual juggle   &  &  &  & \\
%     attack easy  &  &  &  & \\
%     attack hard  &  &  &  & \\
%     \bottomrule 
% \end{tabular}

    \caption{Benchmark result of multi-agent cooperative tasks with different reward settings including without and with shaping reward. \textit{Bump and Pass} is evaluated by the number of bumps, \textit{Set the Spike (Easy)} and \textit{Set the Spike (Hard)} are evaluated by the success rate.}
    \label{tab:coop}
\end{table*}

\begin{table*}[t]
    \centering
    \begin{tabular}{ccccccc}
\toprule
& \multicolumn{3}{c}{\textit{1 vs 1}}      & \multicolumn{3}{c}{\textit{3 vs 3}}     \\
& Exploitability $\downarrow$ & Win Rate $\uparrow$ & Elo $\uparrow$ & Exploitability $\downarrow$ & Win Rate $\uparrow$ & Elo $\uparrow$ \\
\midrule
SP     & $48.63$ & $0.55$ & $1072$ & $\bm{25.76}$ & $0.59$ & $1077$ \\
FSP    & $30.41$ & $\bm{0.63}$ & $927$ & $38.86$ & $0.52$ & $906$ \\
PSRO$_\text{Uniform}$ & $18.51$ & $0.35$ & $854$ & $49.48$ & $0.28$ & $750$ \\
PSRO$_\text{Nash}$ & $\bm{10.74}$ & $0.47$ & $\bm{1147}$ & $35.83$ & $\bm{0.61}$ & $\bm{1268}$ \\
\bottomrule 
\end{tabular}

    \caption{Benchmark result of multi-agent competitive tasks including \textit{1 vs 1} and \textit{3 vs 3} with different evaluation metrics.}
    \label{tab:comp}
\end{table*}

We present extensive experiments to benchmark representative (MA)RL and game-theoretic algorithms in our VolleyBots testbed.
Specifically, for single-agent tasks, we benchmark two RL algorithms and compare their performance under different action space configurations. For multi-agent cooperative tasks, we evaluate four MARL algorithms and compare their performance with and without reward shaping. For multi-agent competitive tasks, we evaluate four game-theoretic algorithms and provide a comprehensive analysis across multiple evaluation metrics. We identify a key challenge in VolleyBots is the hierarchical decision-making process that requires both low-level motion control and high-level strategic play. We further show the potential of hierarchical policy in our VolleyBots testbed by implementing a simple yet effective baseline for the challenging \textit{3 vs 3} task.
Detailed discussion about the benchmark algorithms and more experiment results can be found in Appendix~\ref{app:alg} and \ref{app:exp}.

% We present extensive experiments to benchmark existing algorithms in our VolleyBots testbed. Specifically, for single-agent tasks, we benchmark RL algorithms and compare their performance under different action space configurations. For multi-agent cooperative tasks, we evaluate the performance of different MARL algorithms and compare their performance with and without reward shaping. For multi-agent competitive tasks, we evaluate a range of game-theoretic algorithms and provide a comprehensive analysis across multiple evaluation metrics. We identify a key challenge in VolleyBots is the hierarchical decision-making process that requires both low-level motion control and high-level strategic play. We further show the potential of hierarchical policy in our VolleyBots testbed by implementing a simple yet effective baseline for the challenging \textit{3 vs 3} task.
% More experiment and implementation details can be found in Appendix~\ref{app:exp}.

% \yc{it seems we have enough space, so we could add simulation performance here. fps of different tasks or we can put curves here.}
% \yc{the training hyperparameters can be found in Appendix.}

% \subsection{Results of Environment Setting}
\subsection{Results of Single-Agent Tasks}
% \xzl{@xiangmin check this subsection} \done

We evaluate two RL algorithms including Deterministic Policy Gradient (DDPG)~\cite{lillicrap2015continuous} and Proximal Policy Optimization (PPO)~\cite{schulman2017proximal} in three single-agent tasks. We also compare their performance under different action spaces including CTBR and PRT. The averaged results over three seeds are shown in Table~\ref{tab:single}.

Using the same number of training frames, the performance of PPO and DDPG shows a clear distinction in all three tasks. PPO consistently achieves high task performance in all tasks, while DDPG struggles to learn effective policies that complete these tasks meaningfully. This disparity can be attributed to PPO's stable on-policy updates, which facilitate efficient exploration and robust learning, 
whereas DDPG's deterministic policy and reliance on off-policy updates result in different learning dynamics.
% whereas DDPG's deterministic policy and reliance on off-policy updates are less effective. \yc{I would suggest not to say DDPG is less effective.}

Comparing different action spaces, the final results indicate that PRT slightly outperforms CTBR in most tasks. This outcome is likely due to PRT providing more granular control over each motor's thrust, enabling the drone to maximize task-specific performance with precise adjustments. On the other hand, CTBR demonstrates a slightly faster learning speed in some tasks, as its higher-level abstraction simplifies the control process and reduces the learning complexity. For optimal task performance, we use PRT as the default action space in subsequent experiments. More experiment results and learning curves are provided in Appendix~\ref{app:single}.

% \subsection{Results of MARL Baselines}
\subsection{Results of Multi-Agent Cooperative Tasks}
% \xzl{@xiangmin check this subsection} \done

We evaluate four MARL algorithms including Multi-Agent DDPG (MADDPG)~\cite{lowe2017multi}, Multi-Agent PPO (MAPPO)~\cite{yu2022surprising}, Heterogeneous-Agent PPO (HAPPO)~\cite{kuba2021trust}, Multi-Agent Transformer (MAT)~\cite{wen2022multi} in three multi-agent cooperative tasks. We also compare their performance with and without reward shaping. The averaged results over three seeds are shown in Table~\ref{tab:coop}.

Comparing the MARL algorithms, on-policy methods like MAPPO, HAPPO, and MAT successfully complete all three cooperative tasks and exhibit comparable performance, while off-policy method like MADDPG fails to complete these tasks. These results are consistent with the observation in single-agent experiments, and we use MAPPO as the default algorithm in subsequent experiments for its consistently strong performance and efficiency.

As for different reward functions, it is clear that using reward shaping leads to better performance, especially in more complex tasks like \textit{Set and Spike (Hard)}. This is because the misbehave penalty and task reward alone are usually sparse and make exploration in continuous space particularly challenging. Such sparse setups can serve as benchmarks to evaluate the exploration ability of MARL algorithms. On the other hand, shaping rewards provide intermediate feedback that guides agents toward task-specific objectives more efficiently, and we use shaping rewards in subsequent experiments for efficient learning. More experimental results and learning curves are provided in Appendix~\ref{app:multi}.


% We benchmark several representative algorithms, including MADDPG, MAPPO, HAPPO, and MAT, on both single-agent and multi-agent cooperative tasks. Single-agent scenarios (e.g., \emph{Bumping} \yc{bump or bumping?} and \emph{Hit}) help establish a baseline for each algorithm's capacity to handle flight control and basic ball interaction. In cooperative tasks, such as \emph{Bump and Pass} and \emph{Attack (Easy)}, \yc{use the right name.} we assess how well these methods learn coordinated maneuvers and shared objectives. 

% [Analyze exp result].  

\subsection{Results of Multi-Agent Competitive Tasks}
We evaluate four game-theoretic algorithms including self-play (SP), Fictitious Self-Play (FSP)~\cite{heinrich2015fictitious}, Policy-Space Response Oracles (PSRO)~\cite{lanctot2017unified} with a uniform meta-solver (PSRO$_\text{Uniform}$), and a Nash meta-solver (PSRO$_\text{Nash}$) in multi-agent competitive tasks. Their performance is evaluated by approximate exploitability, the average win rate against other learned policies, and Elo rating. 
The results are shown in Table~\ref{tab:comp}. 
More results and implementation details are provided in Appendix~\ref{app:mix}.
% The averaged results over three seeds are shown in Table~\ref{tab:comp}. 
% More experimental results and implementation details are provided in Appendix~\ref{app:mix}.

In the \textit{1 vs 1} task, all algorithms manage to learn basic behaviors like returning the ball and positioning for subsequent volleys. However, the exploitability metrics indicate that the learned policies are still far from achieving a Nash equilibrium, suggesting that the strategies lack robustness in this two-player zero-sum game. This performance gap highlights the inherent challenge of hierarchical decision-making in this task, where drones must simultaneously execute precise low-level motion control and engage in high-level strategic gameplay under volleyball rules. This challenge presents new opportunities for designing algorithms that can better integrate hierarchical decision-making capabilities.

% \zrz{In the \textit{3 vs 3} task, algorithms may exhibit minimal progress, such as learning to serve the ball, but fail to produce other meaningful behaviors.}
In the \textit{3 vs 3} task, algorithms exhibit minimal progress, such as learning to serve the ball, but fail to produce other strategic behaviors.
% In the \textit{3 vs 3} task, only SP and FSP exhibit minimal progress, such as learning to serve the ball, while other algorithms fail to produce meaningful behaviors. 
This outcome underscores the compounded challenges in this scenario, where each team of three drones must not only cooperate internally but also compete against the opposing team. The increased difficulty of achieving high-level strategic play in such a mixed cooperative-competitive environment further amplifies the hierarchical challenges observed in \textit{1 vs 1}. As a result, the \textit{3 vs 3} task serves as a highly demanding benchmark that highlights the need for new approaches for learning in complex multi-agent settings with a hierarchical decision process.


% \begin{figure}[t]
%     \centering
%     \includegraphics[width=0.8\linewidth]{figs/crossplay.png}
%     \caption{(TODO) Competitive crossplay result.}
%     \label{fig:crossplay}
% \end{figure}

% For the competitive \emph{1-vs-1} scenario, we incorporate methods from self-play and population-based research. Specifically, we consider Self-play, Fictitious Self-Play (FSP), PSRO\_uniform, and PSRO\_Nash. Each approach addresses the turn-based adversarial nature of our environment by mixing or iterating over different strategies in pursuit of robust policies. 

% [Analyze exp result].
% \yc{@ruize, 3v3}

\subsection{Hierarchical Policy}

We further investigate hierarchical policies as a promising approach. Using the \textit{3 vs 3} task as an example, we first employ the PPO algorithm to develop a set of low-level skill policies, including \textit{Hover}, \textit{Serve}, \textit{Pass}, \textit{Set}, and \textit{Attack}. The details of low-level skills can be found in Appendix~\ref{app:low}. Next, we design a rule-based high-level strategic policy to assign low-level skills to each drone. For the \textit{Serve} and \textit{Attack} skills, the high-level policy also determines whether to hit the ball to the left or right, with an equal probability of 0.5 for each direction. Fig.~\ref{fig:hier} illustrates two typical demonstrations of the hierarchical policy attaching the \textit{Serve} skill to drone 1 in a serve scenario and the \textit{Attack} skill to drone 3 in a rally scenario.
In accordance with volleyball rules, the high-level policy uses an event-driven mechanism, triggering decisions whenever the ball is hit. 
We evaluate the average win rate of 1000 episodes where the hierarchical policy competes against the SP policy. The results show that the hierarchical policy achieves a significantly higher win rate of 86\%. While the current design of the hierarchical policy is in its early stages, it demonstrates substantial potential and offers valuable inspiration for future developments.
% By dividing strategy into lower-level motion control and higher-level tactical reasoning, hierarchical frameworks may better exploit the structure of volleyball dynamics. 
% We decompose the strategy into lower-level motion control skills and higher-level tactical reasoning policy for the \textit{3 vs 3} task, as demonstrated in Fig.~\ref{fig:hier}. Low-level control policies such as Hover, Serve, First Pass, Second Pass, and Attack are each trained using single-agent RL, with the same training budget as other competitive algorithm baselines. Moreover, we design a fixed high-level policy that is event-driven: when a drone hits the ball, the policy determines which low-level skill each drone should use. 
 % This hierarchical policy is evaluated by playing 1000 games against the SP policy that are unseen during the training of the hierarchical policy, demonstrating a significantly higher win rate, as shown in Tab.~\ref{tab:hier}. Although the current hierarchical policy design is still in its early stages, it has already shown promising results for this type of task, providing valuable insights that could inspire future solutions.

\begin{figure}[t]
    \centering
    \subfloat[\textit{1 vs 1}]{
        \includegraphics[width=0.47\linewidth]{figs/1v1_crossplay.pdf}} 
    % \hspace{0.05\linewidth}
    \hfill
    \subfloat[\textit{3 vs 3}]{
        \includegraphics[width=0.47\linewidth]{figs/3v3_crossplay.pdf}}
    \vspace{-2mm}
    \caption{Cross-play heatmap of multi-agent competitive tasks.}
    \label{fig:crossplay}
    \vspace{-2mm}
\end{figure}

% \begin{table}[h]
%     \centering
%     \begin{tabular}{cc}
\toprule
& SP Policy\\
\midrule
Hierarchical Policy  & $0.86\pm0.06$ \\
\bottomrule 
\end{tabular}

%     \caption{Average win rate of the hierarchical policy versus the SP policy in the \textit{3 vs 3} task.}
%     \label{tab:hier}
% \end{table}


% \subsection{Action space}
% thrust v.s. CTBR

% \subsection{MARL Single / Coop / Comp}
% MADDPG, MAPPO, HAPPO, MAT

% SP, FSP, PSRO\_Unif, PSRO\_Nash

% \subsection{Hierarchical}

% \subsection{Sim2Real}


\section{Sim-to-Real}
\label{sec:sim2real}
\begin{figure}[t]
    \centering
    \subfloat[Serve]{
        \includegraphics[width=0.45\linewidth]{figs/hier_serve.pdf}} 
    % \hspace{0.05\linewidth}
    \hfill
    \subfloat[Attack]{
        \includegraphics[width=0.45\linewidth]{figs/hier_rally.pdf}}
    \vspace{-2mm}
    \caption{Demonstration of the hierarchical policy selecting \textit{Serve} and \textit{Attack} skills in the \textit{3 vs 3} task.}
    \label{fig:hier}
    \vspace{-4mm}
\end{figure}

We use the \textit{Solo Bump} task as a demonstration of the policy’s ability to zero-shot transfer to the real world. We use a quadrotor with a rigidly mounted badminton racket, as shown in Fig.~\ref{fig:overview}. The state of both the drone and the ball is captured using a motion capture system. The drone is modeled as a rigid body, with its position and orientation provided by the motion capture system. The drone's velocity is estimated using an Extended Kalman Filter (EKF) that fuses pose data from the motion capture system and IMU data from the PX4 Autopilot. The ball is modeled as a point mass, with its position sent by the motion capture system and its velocity indirectly computed through a Kalman Filter. The drone's dynamics parameters and the ball’s properties are determined through system identification.

To simulate real-world noise and imperfect execution of actions, small randomizations are introduced in the ball’s initial position, coefficient of restitution, and the ball's rebound velocity after each collision with the drone. Inspired by~\cite{chen2024matterslearningzeroshotsimtoreal}, we also add a smoothness reward to encourage smooth actions. The policy uses CTBR as output and is deployed on the onboard Nvidia Orin processor.
Experiment results show that the drone successfully performs bump tasks multiple times, providing initial evidence of sim-to-real transfer capability. To support further research, we make the drone configuration, model checkpoint, and real-world deployment videos publicly available on our website. We hope this will accelerate progress in this field.

\section{Conclusion}
\label{sec:conclusion}
\section{Call to Action: Benchmarking 101} \label{section:conclusion}

% \jbgcomment{I think this ``call to action'' could be a little stronger and more specific.}

If you want to make the best benchmark ever, where do you begin?
First, define the ability you want to test and decide if \mcqa is the right format (\cref{section:format}).
If the ability matches a downstream task (e.g. coding), just use that task \cite{saxon2024benchmarks}.
If the ability is fundamental (e.g. knowledge), consult education research to weigh any alternative formats (e.g. \cref{section:fixing_format}).

If \mcqa is the best format, find a data source~to curb leakage---one with fresh content (\cref{subsection:test_set_leakage}).
When curating MCQs from your source,~follow educators' rubrics to ensure answerability (\cref{subsection:quality}), and release the rubric as a data card to record errors \cite{pushkarna2022data}.
Consistent design choices will limit shortcuts (\cref{subsection:artifacts}), and providing a contrast set could help researchers check if their models over-rely on shortcuts \cite{Gardner2020EvaluatingML}.
As another safeguard, your benchmark can use calibration scoring beyond accuracy to discourage guessing~(\cref{subsubsection:shortcut_scoring}).

Post-release, models will hill-climb and saturate your data over time (\cref{subsection:saturation}).
If you want to delay~saturation, you may restart with an obscure knowledge source, but if you want your data to better diagnose errors, aim for interpretability.
Use IRT (\cref{subsection:irt}) to find which of your MCQs are hard and why, then design an engaging, adversarial dataset collection protocol (\cref{subsection:hitl}) guided by these insights, yielding a new dataset hard for models but easy for humans.

By using even some of educators' insights, we can refine the utility of \mcqa---or any task.
This approach takes more effort than the simple \mcqa practices that initially attracted researchers, but if we do not address the flaws of \mcqa, \textbf{what~model abilities can our \mcqa benchmarks even test?}


\section*{Acknowledgment}
This work was funded by the Deutsche Forschungsgemeinschaft (DFG, German Research Foundation) — 456666331, 321892712.

\section*{Impact Statement}
\section*{Impact statement}

This paper proposes that machine learning can and should be used to maximize social welfare. In principle, and by construction, the impact of our proposed framework on society aims to be positive. But our paper also points to the inherent difficulties of identifying, and making formal, what `good for society' is. We lean on the field of welfare economics, which has for decades contended with this challenge, for ideas on how the learning community can begin to approach this daunting task.
However, even if these ideas are conceptually appealing,
the path to practical welfare improvement presents many challenges---%
some expected, others unforseen.
% and will likely include many ups and downs.
For example, we may specify incorrect social welfare functions;
or we may specify them correctly but be unable to optimize them appropriately;
or we may be able to optimize but find that 
our assumptions are wrong, that theory differs from practice,
or that there were other considerations and complexities that we did not take into account.
For this we can look to other related fields---%
such as fairness, privacy, and alignment in machine learning---%
which have taken (and are still taking) similar journeys,
and learn from both their success and mistakes.
% and hope that ours will be similar.

Any discipline that seeks to affect policy should do so with much deliberation and care. Whereas welfare economics was designed with the explicit purpose of supporting (and influencing) policymakers,
machine learning has found itself in a similar position, but likely without any planned intent.
On the one hand, adjusting machine learning to support notions, such as social welfare,
that it was not designed to support initially can prove challenging.
However, and as we argue throughout, we believe that building on top of existing machinery is a more practical approach than to begin from scratch.
The necessity of confronting with welfare consideration can also
be an opportunity---as we can leverage these novel challenges
to make machine learning practice more informed, transparent, responsible, and socially aware.


% At the same time, the novelty of the challenges that welfare considerations present to the field make this an opportunity---%
% for chaning the role of machine learning in society for the better in a manner that is informed, transparent, and aware.





% In the unusual situation where you want a paper to appear in the
% references without citing it in the main text, use \nocite
\nocite{langley00}

\bibliography{reference}
\bibliographystyle{icml2025}


%%%%%%%%%%%%%%%%%%%%%%%%%%%%%%%%%%%%%%%%%%%%%%%%%%%%%%%%%%%%%%%%%%%%%%%%%%%%%%%
%%%%%%%%%%%%%%%%%%%%%%%%%%%%%%%%%%%%%%%%%%%%%%%%%%%%%%%%%%%%%%%%%%%%%%%%%%%%%%%
% APPENDIX
%%%%%%%%%%%%%%%%%%%%%%%%%%%%%%%%%%%%%%%%%%%%%%%%%%%%%%%%%%%%%%%%%%%%%%%%%%%%%%%
%%%%%%%%%%%%%%%%%%%%%%%%%%%%%%%%%%%%%%%%%%%%%%%%%%%%%%%%%%%%%%%%%%%%%%%%%%%%%%%
\newpage
\appendix
\onecolumn
\section{Details of VolleyBots Environment}
\label{app:env}
\subsection{Court}
The volleyball court in our environment is depicted in Fig.~\ref{fig:app:court}. The court is divided into two equal halves by the $y$-axis, which serves as the dividing line separating the two teams. The coordinate origin is located at the midpoint of the dividing line, and the $x$-axis extends along the length of the court. The total court length is $18\,{m}$, with $x = -9$ and $x = 9$ marking the ends of the court. The $y$-axis extends across the width of the court, with a total width of $9\,{m}$, spanning from $y = -4.5$ to $y = 4.5$. The net is positioned at the center of the court along the $y$-axis, with a height of $2.43\,{m}$, and spans horizontally from $(0, -4.5)$ to $(0, 4.5)$.

\begin{figure}[h]
    \centering
    \includegraphics[width=0.5\linewidth]{figs/court.pdf}
    \caption{Volleyball court layout in our environment with coordinates.}
    \label{fig:app:court}
\end{figure}


\subsection{Drone}
We use the \textit{Iris} quadrotor model~\cite{furrer2016rotors} as the primary drone platform, augmented with a virtual ``racket'' of radius $0.2\,{m}$ and coefficient of restitution $0.8$ for ball striking. The drone's root state is a vector with dimension $23$, including its position, rotation, linear velocity, angular velocity, forward orientation, upward orientation, and normalized rotor speeds.

The control dynamics of a multi-rotor drone are governed by its physical configuration and the interaction of various forces and torques. The system's dynamics can be described as follows:

\begin{align}
\bm{\dot{x}}_W = \bm{v}_W, \quad \bm{\dot{v}}_W = \bm{R_{WB}}f + \bm{g} + \bm{F} \\
\bm{\dot{q}} = \frac{1}{2} \bm{q} \otimes \bm{\omega}, \quad \bm{\dot{\omega}} = \bm{J}^{-1} (\bm{\eta} - \bm{\omega} \times \bm{J} \bm{\omega})
\end{align}

where $\bm{x}_W$ and $\bm{v}_W$ represent the position and velocity of the drone in the world frame, $\bm{R}_{WB}$ is the rotation matrix converting from the body frame to the world frame, $\bm{J}$ is the diagonal inertia matrix, $\bm{g}$ denotes gravity, $\bm{q}$ is the orientation represented by quaternions, and $\bm{\omega}$ is the angular velocity. The quaternion multiplication operator is denoted by $\otimes$. External forces $\bm{F}$, including aerodynamic drag and downwash effects, are also considered. The collective thrust $\bm{f}$ and body rate $\bm{\eta}$ are computed based on per-rotor thrusts $\bm{f}_i$ as:

\begin{align}
\bm{f} &= \sum_{i} \bm{R}^{(i)}_B \bm{f}_i \\
\bm{\eta} &= \sum_{i} \bm{T}^{(i)}_B \times \bm{f}_i + k_i \bm{f}_i
\end{align}

where $\bm{R}^{(i)}_B$ and $\bm{T}^{(i)}_B$ are the local orientation and translation of the $i$-th rotor in the body frame, and $k_i$ is the force-to-moment ratio.


\subsection{Defense Racket}
% \xzl{@chuqi briefly describe the implementation of the defense racket.}
% \done

We assume a thin cylindrical racket to mimic a human-held racket for adversarial interactions with a drone. 
%The restitution coefficient of the racket is set to match that of the ball, ensuring collision simulations align as closely as possible with reality. 
When the ball is hit toward the racket’s half of the court, the racket is designed to intercept the ball at a predefined height $h_{pre}$. Since the ball’s position and velocity data can be directly acquired, the descent time $t_{pre}$, landing point $\bm{p}_{ball\_land}$, and pre-collision velocity $\bm{v}_{ball\_pre}$ can be calculated using projectile motion equations. Additionally, to ensure the ball is returned to a designated position $\boldsymbol{p_b{}_{des}}$ and crosses the net, the post-collision motion duration $t_{post}$ of the ball is set to a sufficiently large value. This allows the projectile motion equations to similarly determine the post-collision velocity $\bm{v}_{ball\_post}$. Based on these conditions, the required collision position $\bm{p}_{collision}$, orientation $\bm{\theta}_{collision}$ and velocity $\bm{v}_{collision}$ of the racket can be derived  as follows:

\begin{align}
\bm{p}_{collision} &= \bm{p}_{ball\_land} \\
\bm{n}_{collision} &= \frac{\bm{v}_{ball\_post} - \bm{v}_{ball\_pre}}{\| \bm{v}_{ball\_post} - \bm{v}_{ball\_pre} \|} = [\sin p\cos r, -\sin r, \cos p\cos r] \\
\bm{\theta}_{collision} &= [-\arcsin {\bm{n}_{collision}}(2), \arctan {\frac{\bm{n}_{collision}(1)}{\bm{n}_{collision}(3)}}, 0] \\
\bm{v}_{collision} &= \frac{1}{1+\beta}(\beta \bm{v}_{ball\_pre} + \bm{v}_{ball\_post})
\end{align}

where $\bm{n}_{collision}$ represents the normal vector of the racket during impact, $r$ denotes the roll angle of the racket, $p$ denotes the pitch angle, while the yaw angle remains fixed at 0, and $\beta$ represents the restitution coefficient. To simulate the adversarial interaction as realistically as possible, we impose direct constraints on the racket’s linear velocity and angular velocity. Based on the simulation time step $t_{step}$ and the descent time $t_{post}$ of the ball, we can calculate the required displacement $\boldsymbol{d} = \frac{\boldsymbol{p_b{}_{des}}-\bm{p}_{ball\_land}}{t_{post}}t_{step}$ and rotation angle $\boldsymbol{\theta} = \frac{\bm{\theta}_{collision}}{t_{post}}t_{step}$ that the racket must achieve within each time step. If both $\boldsymbol{d}$ and $\boldsymbol{\theta}$ do not exceed their respective limits ($\boldsymbol{d}_{max}$ and $\boldsymbol{\theta}_{max}$), the racket moves with linear velocity $\boldsymbol{d}$ and angular velocity $\boldsymbol{\theta}$. Otherwise, the values are set to their corresponding limits $\boldsymbol{d}_{max}$ and $\boldsymbol{\theta}_{max}$.


\section{Details of Task Design}
\label{app:task}
\begin{table}[t]
    \centering
    % \xzl{@xiangmin add reward of back-and-forth task}
\begin{tabular}{ccccc} 
\toprule 
Type & Name & Sparse & Value Range & Description \\ 
\midrule 
\multirow{2}{*}{\begin{tabular}[c]{@{}c@{}}Misbehave\\ Penalty\end{tabular}} & \multirow{2}{*}{drone\_misbehave} & \multirow{2}{*}{\cmark} & \multirow{2}{*}{$\{0, -10\}$} & \multirow{2}{*}{drone too low or drone too remote} \\ 
& & & & \\ 
% Misbehave Penalty & drone\_misbehave & \cmark & $\{0, -10\}$ & \\
\midrule 
\multirow{2}{*}{\begin{tabular}[c]{@{}c@{}}Task\\ Reward\end{tabular}} 
& dist\_to\_target & \xmark & $[0, 0.5] \times $ \# step & $\propto$ drone's distance to the current target \\ 
& target\_stay & \cmark & $\{0, 2.5\} \times $ \# in\_target & drone stays in the current target region \\ 
\bottomrule 
\end{tabular}

    \caption{Reward of single-agent \textit{Back and Forth} task.}
    \label{tab:app:bnf_reward}
\end{table}

\begin{table}[t]
    \centering
    \begin{tabular}{ccccc}
\toprule
Type                                                                         & Name           & Sparse & Value Range & Description \\
\midrule
\multirow{3}{*}{\begin{tabular}[c]{@{}c@{}}Misbehave\\ Penalty\end{tabular}} & ball\_misbehave & \cmark  & $\{0, -10\}$ & ball too low or touch the net or out of court \\
& drone\_misbehave & \cmark & $\{0, -10\}$ & drone too low or touches the net \\
& wrong\_hit       & \cmark & $\{0, -10\}$ & drone does not use the racket to hit the ball \\
\midrule
\multirow{3}{*}{\begin{tabular}[c]{@{}c@{}}Task\\ Reward\end{tabular}}       & success\_hit    & \cmark & $\{0, 1\}$ & drone hits the ball \\
& distance  & \cmark & $[0, +\infty]$ & $\propto$ the landing position's distance to the anchor \\
& dist\_to\_anchor  & \xmark & $[-\infty, 0]$ & $\propto$ drone's distance to the anchor \\
\bottomrule
\end{tabular}
    \caption{Reward of single-agent \textit{Hit the Ball} task.}
    \label{tab:app:hit_reward}
\end{table}

\begin{table}[t]
    \centering
    \begin{tabular}{ccccc}
\toprule
Type                                                                         & Name           & Sparse & Value Range & Description \\
\midrule
\multirow{3}{*}{\begin{tabular}[c]{@{}c@{}}Misbehave\\ Penalty\end{tabular}} & ball\_misbehave & \cmark  & $\{0, -10\}$ & ball too low or touch the net or out of court \\
& drone\_misbehave & \cmark & $\{0, -10\}$ & drone too low or touches the net \\
& wrong\_hit       & \cmark & $\{0, -10\}$ & drone does not use the racket to hit the ball \\
\midrule
\multirow{3}{*}{\begin{tabular}[c]{@{}c@{}}Task\\ Reward\end{tabular}}       & success\_hit    & \cmark & $\{0, 1\} \times$ \# hit & drone hits the ball \\
& success\_height  & \cmark & $\{0, 1\} \times$ \# hit & ball reaches the minimum height \\
& dist\_to\_anchor  & \xmark & $[-\infty, 0]$ & $\propto$ drone's distance to the anchor \\
\bottomrule
\end{tabular}
    \caption{Reward of single-agent \textit{Solo Bump} task.}
    \label{tab:app:bump_reward}
\end{table}

\begin{table}[t]
    \centering
    \begin{tabular}{cccccc}
\toprule
Type                                                                         & Name           & Sparse & Shared & Value Range & Description \\
\midrule
\multirow{3}{*}{\begin{tabular}[c]{@{}c@{}}Misbehave\\ Penalty\end{tabular}} & ball\_misbehave & \cmark & \cmark & $\{0, -10\}$ & ball too low or touches the net or out of court \\
& drone\_misbehave & \cmark & \xmark & $\{0, -10\}$ & drone too low or touches the net \\
& wrong\_hit       & \cmark & \xmark & $\{0, -10\}$ & drone hits in the wrong turn \\
\midrule
\multirow{3}{*}{\begin{tabular}[c]{@{}c@{}}Task\\ Reward\end{tabular}}       & success\_hit    & \cmark & \cmark & $\{0, 1\} \times$ \# hit & drone hits the ball \\
& success\_cross  & \cmark & \cmark & $\{0, 1\} \times$ \# hit & ball crosses the height \\
& dist\_to\_anchor  & \xmark  & \cmark & $[-\infty, 0]$ & $\propto$ drone' distance to its anchor \\
\midrule
\multirow{2}{*}{\begin{tabular}[c]{@{}c@{}}Shaping\\ Reward\end{tabular}}       & hit\_direction    & \cmark & \xmark & $\{0, 1\} \times$ \# hit & drone hits the ball towards the other drone \\
& dist\_to\_ball  & \xmark  & \xmark & $[0, 0.05] \times$ \# step & $\propto$ drone's distance to the ball \\
\bottomrule
\end{tabular}
    \caption{Reward of multi-agent \textit{Bump and Pass} task.}
    \label{tab:app:bnp_reward}
\end{table}

\begin{table}[t]
    \centering
    \begin{tabular}{cccccc}
\toprule
Type                                                                         & Name           & Sparse & Shared & Value Range & Description \\
\midrule
\multirow{3}{*}{\begin{tabular}[c]{@{}c@{}}Misbehave\\ Penalty\end{tabular}} & ball\_misbehave & \cmark & \cmark & $\{0, -10\}$ & ball too low or touches the net or out of court\\
& drone\_misbehave & \cmark & \xmark & $\{0, -10\}$ & drone too low or touches the net \\
& wrong\_hit       & \cmark & \xmark & $\{0, -10\}$ & drone hits in the wrong turn \\
\midrule
\multirow{4}{*}{\begin{tabular}[c]{@{}c@{}}Task\\ Reward\end{tabular}}       & success\_hit    & \cmark & \cmark & $\{0, 5\} \times$ \# hit & drone hits the ball \\
& downward\_spike  & \cmark & \cmark & $\{0, 5\} \times$ \# spike & ball's velocity is downward after spike \\
& success\_cross  & \cmark & \cmark & $\{0, 5\}$ & ball crosses the net \\
& in\_target  & \cmark & \cmark & $\{0, 5\}$ & ball lands in the target region \\
& dist\_to\_anchor  & \xmark  & \cmark & $[-\infty, 0]$ & $\propto$ drone's distance its anchor \\
\midrule
\multirow{3}{*}{\begin{tabular}[c]{@{}c@{}}Shaping\\ Reward\end{tabular}}       & hit\_direction    & \cmark & \xmark & $\{0, 1\} \times$ \# hit & drone hits the ball towards its target \\
& spike\_velociy  & \cmark  & \cmark & $[0, +\infty] \times$ \# spike & $\propto$ ball's downward velocity after spike \\
& dist\_to\_ball  & \xmark  & \xmark & $[0, 0.05] \times$ \# step & $\propto$ drone's distance to the ball \\
& dist\_to\_target  & \xmark  & \cmark & $[0, 2]$ & $\propto$ ball's landing position to the target \\
\bottomrule
\end{tabular}
    \caption{Reward of multi-agent \textit{Set and Spike (Easy)} task.}
    \label{tab:app:sns_easy}
\end{table}

\begin{table}[t]
    \centering
    \begin{tabular}{cccccc}
\toprule
Type                                                                         & Name           & Sparse & Shared & Value Range & Description \\
\midrule
\multirow{3}{*}{\begin{tabular}[c]{@{}c@{}}Misbehave\\ Penalty\end{tabular}} & ball\_misbehave & \cmark & \cmark & $\{0, -10\}$ & ball too low or touches the net or out of court\\
& drone\_misbehave & \cmark & \xmark & $\{0, -10\}$ & drone too low or touches the net \\
& wrong\_hit       & \cmark & \xmark & $\{0, -10\}$ & drone hits in the wrong turn \\
\midrule
\multirow{4}{*}{\begin{tabular}[c]{@{}c@{}}Task\\ Reward\end{tabular}}       & success\_hit    & \cmark & \cmark & $\{0, 5\} \times$ \# hit & drone hits the ball\\
& downward\_spike  & \cmark & \cmark & $\{0, 5\} \times$ \# spike & ball's velocity is downward after spike \\
& success\_cross  & \cmark & \cmark & $\{0, 5\}$ & ball crosses the net \\
& success\_spike  & \cmark & \cmark & $\{0, 5\}$ & defense racket fails to intercept \\
& dist\_to\_anchor  & \xmark  & \cmark & $[-\infty, 0]$ & $\propto$ drone's distance to its anchor \\
\midrule
\multirow{3}{*}{\begin{tabular}[c]{@{}c@{}}Shaping\\ Reward\end{tabular}}       & hit\_direction    & \cmark & \xmark & $\{0, 1\} \times$ \# hit & drone hits the ball towards their targets \\
& spike\_velociy  & \cmark  & \cmark & $[0, +\infty] \times$ \# spike & $\propto$ ball's downward velocity after spike \\
& dist\_to\_ball  & \xmark  & \xmark & $[0, 0.05] \times$ \# step & $\propto$ drone's distance to the ball \\
\bottomrule
\end{tabular}
    \caption{Reward of multi-agent \textit{Set and Spike (Hard)} task.}
    \label{tab:app:sns_hard}
\end{table}

\begin{table}[t]
    \centering
    \begin{tabular}{cccccc}
\toprule
Type                                                                         & Name           & Sparse & Shared & Value Range & Description \\
\midrule
\multirow{2}{*}{\begin{tabular}[c]{@{}c@{}}Misbehave\\ Penalty\end{tabular}} & drone\_misbehave & \cmark & \xmark & $\{0, -100\}$ & drone too low or touches the net \\
& drone\_out\_of\_court & \xmark & \xmark & $[0, 0.2] \times$ \# step & $\propto$ drone's distance out of its court \\
\midrule
Task Reward       & win\_or\_lose    & \cmark & \xmark & $\{-100, 0, 100\}$ & drone wins or loses the game \\
\midrule
\multirow{2}{*}{\begin{tabular}[c]{@{}c@{}}Shaping\\ Reward\end{tabular}}       & success\_hit    & \cmark & \xmark & $\{0, 5\} \times$ \# hit & drone hits the ball \\
& dist\_to\_ball  & \xmark  & \xmark & $[0, 0.5] \times$ \# step & $\propto$ drone's distance to the ball\\
\bottomrule
\end{tabular}
    \caption{Reward of multi-agent \textit{1 vs 1} task.}
    \label{tab:app:1v1_reward}
\end{table}

\begin{table}[t]
    \centering
    \begin{tabular}{cccccc}
\toprule
Type                                                                         & Name           & Sparse & Shared & Value Range & Description \\
\midrule
\multirow{2}{*}{\begin{tabular}[c]{@{}c@{}}Misbehave\\ Penalty\end{tabular}} & drone\_misbehave & \cmark & \xmark & $\{0, -100\}$ & drone too low or touches the net. \\
& drone\_collision & \cmark & \xmark & $\{0, -100\}$ & drone collides with its teammate. \\
\midrule
Task Reward       & win\_or\_lose    & \cmark & \cmark & $\{-100, 0, 100\}$ & drones win or lose the game\\
\midrule
\multirow{3}{*}{\begin{tabular}[c]{@{}c@{}}Shaping\\ Reward\end{tabular}}       & success\_hit    & \cmark & \cmark & $\{0, 10\} \times$ \# hit & drone hits the ball \\
& dist\_to\_anchor  & \xmark  & \xmark & $[0, 0.05] \times$ \# step & $\propto$ drone's distance to its anchor \\
& dist\_to\_ball  & \xmark  & \xmark & $[0, 0.5] \times$ \# step & $\propto$ drone's distance to the ball \\
\bottomrule
\end{tabular}
    \caption{Reward of multi-agent \textit{3 vs 3} task.}
    \label{tab:app:3v3_reward}
\end{table}

\subsection{Back and Forth}
% \xzl{@xiangmin write this subsection. check other tasks for reference.} \done

\paragraph{Task Definition.}
The drone is initialized at an anchor position $(4.5, 0, 2)$, i.e., the center of the red court with a height of $2\,m$. The other anchor position is $(9.0, 4.5, 2)$, with the target points switching between two designated anchor positions. The drone is required to sprint between two designated anchors to complete as many round trips as possible. $5$ steps within a sphere with a $0.6\,m$ radius near the anchor position are required for each stay. The maximum episode length is $800$ steps.
\paragraph{Observation and Reward.}
When the action space is Per-Rotor Thrust(PRT), the observation is a vector of dimension $26$, which includes the drone's root state and its relative position to the target anchor. When the action space is Collective Thrust and Body Rates (CTBR), the observation dimension is reduced to $22$, excluding the drone’s throttle. The detailed description of the reward function of this task is listed in Table~\ref{tab:app:bnf_reward}.
\paragraph{Evaluation Metric.}
This task is evaluated by the number of target points reached within the time limit. A successful stay is defined as the drone staying $5$ steps within a sphere with a $0.6\,m$ radius near the target anchor.


\subsection{Hit the Ball}
% \xzl{@xiangmin check for errors} \done

\paragraph{Task Definition.}
The drone is initialized randomly around an anchor position $(4.5, 0, 2)$, i.e., the center of the red court with a height of $2\,m$. The drone's initial position is sampled uniformly random from $[4, -0.5, 1.8]$ to $[5, 0.5, 2.2]$. The ball is initialized at $(4.5, 0, 5)$, i.e., $3\,m$ above the anchor position. The ball starts with zero velocity and falls freely. The drone is required to perform a single hit to strike the ball toward the opponent’s court, i.e., in the negative direction of the x-axis, aiming for maximum distance. The maximum episode length is $800$ steps.

\paragraph{Observation and Reward.}
When the action space is Per-Rotor Thrust(PRT), the observation is a vector of dimension $32$, which includes the drone's root state, the drone's relative position to the anchor, the ball's relative position to the drone, and the ball's velocity. When the action space is Collective Thrust and Body Rates (CTBR), the observation dimension is reduced to $28$, excluding the drone’s throttle. The detailed description of the reward function of this task is listed in Table~\ref{tab:app:hit_reward}.

\paragraph{Evaluation Metric.}
This task is evaluated by the distance between the ball's landing position and the anchor position. The ball’s landing position is defined as the intersection of its trajectory with the plane $z = 2$.


\subsection{Solo Bump}
% \xzl{@xiangmin fill in the blank} \done

\paragraph{Task Definition.}
The drone is initialized randomly around an anchor position $(4.5, 0, 2)$, i.e., the center of the red court with a height of $2\,m$. The drone's initial position is sampled uniformly random from $[4, -0.5, 1.8]$ to $[5, 0.5, 2.2]$. The ball is initialized at $(4.5, 0, 4)$, i.e., $2\,m$ above the anchor position. The ball starts with zero velocity and falls freely. The drone is required to stay within a sphere with $1\,m$ radius near the anchor position and bump the ball as many times as possible. A minimum height of $4\,m$ is required for each bump. The maximum episode length is $800$ steps.

\paragraph{Observation and Reward.}
When the action space is Per-Rotor Thrust (PRT), the observation is a vector of dimension $32$, which includes the drone's root state, the drone's relative position to the anchor, the ball's relative position to the drone, and the ball's velocity. When the action space is Collective Thrust and Body Rates (CTBR), the observation dimension is reduced to $28$, excluding the drone’s throttle. The detailed description of the reward function of this task is listed in Table~\ref{tab:app:bump_reward}.

\paragraph{Evaluation Metric.}
This task is evaluated by the number of successful consecutive bumps performed by the drone. A successful bump is defined as the drone hitting the ball such that the ball’s highest height exceeds $4\,m$.

\subsection{Bump and Pass}
% \xzl{@xiangmin check for errors} \done

\paragraph{Task Definition.}
Drone 1 is initialized randomly around anchor 1 with position $(4.5, -2.5, 2)$, and Drone 2 is initialized randomly around anchor 2 with position $(4.5, 2.5, 2)$. The initial position of drone 1 is sampled uniformly random from $(4, -3, 1.8)$ to $(5, -2, 2.2)$, and the initial position of drone 2 is sampled uniformly random from $(4, 2, 1.8)$ to $(5, 3, 2.2)$. The ball is initialized at $(4.5, -2.5, 4)$, i.e., $2\,m$ above anchor 1. The ball starts with zero velocity and falls freely. The drones are required to stay within a sphere with $0.5\,m$ radius near their anchors and bump the ball to pass it to each other in turns as many times as possible. A minimum height of $4\,m$ is required for each bump. The maximum episode length is 800 steps.

\paragraph{Observation and Reward.}
The drone's observation is a vector of dimension $39$ including the drone's root state, the drone's relative position to the anchor, the drone's id, the current turn (which drone should hit the ball), the ball's relative position to the drone, the ball's velocity, and the other drone's relative position to the drone. The detailed description of the reward function of this task is listed in Table~\ref{tab:app:bnp_reward}.

\paragraph{Evaluation Metric.}
This task is evaluated by the number of successful consecutive bumps performed by the drones. A successful bump is defined as the drone hitting the ball such that the ball’s highest height exceeds  $4\,m$ and lands near the other drone.

\subsection{Set and Spike (Easy)}
% \xzl{@xiangmin check for errors} \done

\paragraph{Task Definition.}
Drone 1 (setter) is initialized randomly around anchor 1 with position $(2, -2.5, 2.5)$, and Drone 2 (attacker) is initialized randomly around anchor 2 with position $(2, 2.5, 3.5)$. The initial position of drone 1 is sampled uniformly random from $(1.5, -3, 2.3)$ to $(2.5, -2, 2.7)$, and the initial position of drone 2 is sampled uniformly random from $(1.5, 2, 3.3)$ to $(2.5, 3, 3.7)$. The ball is initialized at $(2, -2.5, 4.5)$, i.e., $2\,m$ above anchor 1. The ball starts with zero velocity and falls freely. 
The drones are required to stay within a sphere with $0.5\,m$ radius near their anchors. The setter is required to pass the ball to the attacker, and the attacker then spikes the ball downward to the target region in the opposing side. The target region is a circular area on the ground, centered at $(4.5, 0)$ with a radius of $1\,m$. The maximum episode length is 800 steps.

\paragraph{Observation and Reward.}
The drone's observation is a vector of dimension $40$ including the drone's root state, the drone's relative position to the anchor, the drone's id, the current turn (how many times the ball has been hit), the ball's relative position to the drone, the ball's velocity, and the other drone's relative position to the drone. The detailed description of the reward function of this task is listed in Table~\ref{tab:app:sns_easy}.

\paragraph{Evaluation Metric.}
This task is evaluated by the success rate of set and spike. A successful set and spike consist of four parts, (1) setter\_hit: the setter hits the ball; (2) attacker\_hit: the attacker hits the ball; (3) downward\_spike: the velocity of the ball after the attacker hit is downward, i.e., $v_z < 0$; (4) in\_target: the ball's landing position is within the target region. The success rate is computed as $1/4\times(\text{setter\_hit}+\text{attacker\_hit}+\text{downward\_spike}+\text{in\_target})$.


\subsection{Set and Spike (Hard)}
% \xzl{@xiangmin task definition and eval metric is not correct. It should be something related to the defense board, not in target.}

\paragraph{Task Definition.}
Drone 1 (setter) is initialized randomly around anchor 1 with position $(2, -2.5, 2.5)$, and Drone 2 (attacker) is initialized randomly around anchor 2 with position $(2, 2.5, 3.5)$. The initial position of drone 1 is sampled uniformly random from $(1.5, -3, 2.3)$ to $(2.5, -2, 2.7)$, and the initial position of drone 2 is sampled uniformly random from $(1.5, 2, 3.3)$ to $(2.5, 3, 3.7)$. The ball is initialized at $(2, -2.5, 4.5)$, i.e., $2\,m$ above anchor 1. The ball starts with zero velocity and falls freely. The racket is initialized at $(-4, 0, 0.5)$, i.e., the center of the opposing side.
The drones are required to stay within a sphere with $0.5\,m$ radius near their anchors. The setter is required to pass the ball to the attacker, and the attacker then spikes the ball downward to the opponent's court without being intercepted by the defense racket. The maximum episode length is 800 steps.

\paragraph{Observation and Reward.}
The drone's observation is a vector of dimension $40$ including the drone's root state, the drone's relative position to the anchor, the drone's id, the current turn (how many times the ball has been hit), the ball's relative position to the drone, the ball's velocity, and the other drone's relative position to the drone. The detailed description of the reward function of this task is listed in Table~\ref{tab:app:sns_hard}.

\paragraph{Evaluation Metric.}
This task is evaluated by the success rate of set and spike. A successful set and spike consist of four parts, (1) setter\_hit: the setter hits the ball; (2) attacker\_hit: the attacker hits the ball; (3) downward\_spike: the velocity of the ball after the attacker hit is downward, i.e., $v_z < 0$; (4) success\_spike: the ball's landing position is within the opponent's court without being intercepted by the defense racket. The success rate is computed as $1/4\times(\text{setter\_hit}+\text{attacker\_hit}+\text{downward\_spike}+\text{success\_spike})$.
% \yc{remember the script policy}


\subsection{1 vs 1}
\label{app:1v1}
% \xzl{@huining write this subsection. check other tasks for reference.}

\paragraph{Task Definition.}
Two drones are required to play 1-vs-1 volleyball in a court of $6\,m \times 3\,m$. Drone 1 is initialized randomly around anchor 1 with position $(1.5, 0.0, 2.0)$, i.e., the center of the red court with height $2\,m$, and Drone 2 is initialized randomly around anchor 2 with position $(-1.5, 0.0, 2.0)$, i.e., the center of the blue court with height $2\,m$. The initial position of drone 1 is sampled uniformly random from $(1.4, -0.1, 1.9)$ to $(1.6, 0.1, 2.1)$, and the initial position of drone 2 is sampled uniformly random from $(-1.4, -0.1, 1.9)$ to $(-1.6, 0.1, 2.1)$. At the start of a game (i.e. an episode), one of the two drones is randomly chosen to serve the ball, which is initialized $1.5\,m$ above the drone. The ball starts with zero velocity and falls freely. The game ends when one of the drones wins the game or one of the drones crashes. The maximum episode length is 800 steps.

\paragraph{Observation and Reward.}
The drone's observation is a vector of dimension $39$ including the drone's root state, the drone's relative position to the anchor, the drone's id, the current turn (which drone should hit the ball), the ball's relative position to the drone, the ball's velocity, and the other drone's relative position to the drone. The detailed description of the reward function of this task is listed in Table~\ref{tab:app:1v1_reward}.

\paragraph{Evaluation Metric.}
The drone wins the game by landing the ball in the opponent's court or causing the opponent to commit a violation. These violations include (1) crossing the net, (2) hitting the ball on the wrong turn, (3) hitting the ball with part of the drone body instead of the racket, (4) hitting the ball out of court, and (5) hitting the ball into the net.

To comprehensively evaluate the performance of strategies in the \textit{1 vs 1} task, we consider three evaluation metrics: exploitability, win rate, and Elo rating. These metrics provide complementary insights into the quality and robustness of the learned policies.

\begin{itemize}
    \item \textbf{Exploitability:} 
    Exploitability is a fundamental measure of how close a strategy is to a Nash equilibrium. It is defined as the difference between the payoff of a best response (BR) against the strategy and the payoff of the strategy itself. Mathematically, for a strategy \(\pi\), the exploitability is given by:
    \[
    \text{Exploitability}(\pi) = \max_{\pi'} U(\pi', \pi) - U(\pi, \pi),
    \]
    where \(U(\pi_1, \pi_2)\) represents the utility obtained by \(\pi_1\) when playing against \(\pi_2\). The meaning of exploitability is that smaller values indicate a strategy closer to Nash equilibrium, where it becomes increasingly difficult to exploit. Since exact computation of exploitability is often infeasible in real-world tasks, we instead use approximate exploitability. In this task, we fix the strategy on one side and train an approximate best response on the other side to maximize its utility, i.e., win rate. The difference between the BR's win rate and the evaluated policy's win rate then serves as the approximate exploitability.

    \item \textbf{Win Rate:} 
    Since exact exploitability is challenging to compute, a practical alternative is to evaluate the win rate through cross-play with other learned policy populations. 
    Specifically, we compute the average win rate of the evaluated policy when matched against other learned policies. Higher average win rates typically suggest stronger strategies. However, due to the transitive nature of zero-sum games~\cite{czarnecki2020real}, a high win rate against specific opponent populations does not necessarily imply overall mastery of the game. Thus, while win rate is a useful reference metric, it cannot be the sole criterion for assessing strategy strength.

    \item \textbf{Elo Rating:} 
    Elo rating is a widely used metric for evaluating the relative strength of strategies within a population. It is computed based on head-to-head match results, where the expected win probability between two strategies is determined by their Elo difference. After each match, the Elo ratings of the strategies are updated based on the match outcome. While a higher Elo rating indicates better performance within the given population, it does not necessarily imply proximity to Nash equilibrium. A strategy with a higher Elo might simply be more effective against the specific population, rather than being universally robust. Therefore, Elo complements exploitability by capturing population-specific relative performance.
\end{itemize}

% \xzl{one paragraph for winning condition, one or more paragraph for exploitability, win rate, and elo}
% \yc{remember the script policy}

\subsection{3 vs 3}
% \xzl{@ruize write this subsection. check other tasks for reference.} \done

\paragraph{Task Definition.}
The task involves two teams of drones competing in a 3-vs-3 volleyball match within a court of $9\,m \times 4.5\,m$.
Drone 1, Drone 2, and Drone 3 belong to \textit{Team 1} and are initialized at positions $(3.0, -1.5, 2.0)$, $(3.0, 1.5, 2.0)$ and $(6.0, 0.0, 2.0)$ respectively. Similarly, Drone 4, Drone 5, and Drone 6 belong to \textit{Team 2} and are initialized at positions $(-3.0, -1.5, 2.0)$, $(-3.0, 1.5, 2.0)$ and $(-6.0, 0.0, 2.0)$ respectively. At the start of a game (i.e., an episode), one of the two teams is randomly selected to serve the ball. The ball is initialized at a position $3\,m$ directly above the serving drone. The ball starts with zero velocity and falls freely. The game ends when one of the teams wins the game or one of the drones crashes. The maximum episode length is 500 steps.

\paragraph{Observation and Reward.}
The drone's observation is a vector of dimension $57$ including the drone's root state, the drone's relative position to the anchor, the ball's relative position to the drone, the ball's velocity, the current turn (which team should hit the ball), the drone's id, a flag indicating whether the drone is allowed to hit the ball, and the other drone's positions. The detailed description of the reward function of this task is listed in Table~\ref{tab:app:3v3_reward}.

\paragraph{Evaluation Metric.}
Similar to the \textit{1 vs 1} task, either of the two teams wins the game by landing the ball in the opponent's court or causing the opponent to commit a violation. These violations include (1) crossing the net, (2) hitting the ball on the wrong turn, (3) hitting the ball with part of the drone body, rather than the racket, (4) hitting the ball out of court, and (5) hitting the ball into the net. The task performance is also evaluated by the three metric metrics including exploitability, win rate, and Elo as described in the \textit{1 vs 1} task.


% \xzl{one paragraph for winning condition, one or more paragraph for exploitability, win rate, and elo}

\section{Discussion of Benchmark Algorithms}
\label{app:alg}
\subsection{Reinforcement Learning Algorithms}

To explore the capabilities of our testbed while also providing baseline results, we implement and benchmark a spectrum of popular RL and game-theoretic algorithms on the proposed tasks.

\paragraph{Single-Agent RL.}
In single-agent scenarios, we consider two commonly used algorithms in continuous control tasks. Deep Deterministic Policy Gradient (DDPG)~\cite{lillicrap2015continuous} is an off-policy actor-critic approach relying on a deterministic policy and an experience replay buffer to handle continuous actions. Proximal Policy Optimization (PPO)~\cite{schulman2017proximal} adopts a clipped objective to stabilize on-policy learning updates by constraining policy changes. 
% Soft Actor-Critic (SAC)~\cite{haarnoja2018soft} augments an off-policy actor-critic framework with a maximum-entropy objective, facilitating efficient exploration. 
Overall, these methods provide contrasting paradigms for tackling single-agent continuous tasks.

\paragraph{Multi-Agent RL.}
For tasks with multiple drones, we evaluate four representative multi-agent algorithms. Multi-Agent DDPG (MADDPG)~\cite{lowe2017multi} extends DDPG with a centralized critic for each agent, while policies remain decentralized. Multi-Agent PPO (MAPPO)~\cite{yu2022surprising} incorporates a shared value function to improve both coordination and sample efficiency. Heterogeneous-Agent PPO (HAPPO)~\cite{kuba2021trust} adapts PPO techniques to handle distinct roles or capabilities among agents. Multi-Agent Transformer (MAT)~\cite{wen2022multi} leverages a transformer-based architecture to enable attention-driven collaboration. Taken together, these algorithms offer a diverse set of baselines for multi-agent cooperation.

\paragraph{Game-Theoretic Algorithms.}
For multi-agent competitive tasks, we consider several representative game-theoretic algorithms in the literature~\cite{zhang2024survey}. Self-play (SP) trains agents against the current version of themselves, allowing a single policy to evolve efficiently. Fictitious Self-Play (FSP)~\cite{heinrich2015fictitious} trains agents against the average policy by maintaining a pool of past checkpoints. Policy-Space Response Oracles (PSRO)~\cite{lanctot2017unified} iteratively add the best responses to the mixture of a growing policy population. The mixture policy is determined by a meta-solver. PSRO$_\text{uniform}$ uses a uniform meta-solver that samples policies with equal probability, while PSRO$_\text{Nash}$ uses a Nash meta-solver that samples policies according to the Nash equilibrium. These methods provide an extensive benchmark for game-theoretic algorithms in multi-agent competition with both motion control and strategic play. There are also some algorithms like Team-PSRO~\cite{mcaleer2023team} and Fictitious Cross-Play (FXP)~\cite{xu2023fictitious} that are designed specifically for mixed cooperative-competitive games and can be integrated in our testbed in future work.



\section{Details of Benchmark Experiments}
\label{app:exp}
\begin{table}[t]
    \centering
    \begin{tabular}{l c | l c | l c}
    \toprule
    hyperparameters & value & hyperparameters & value & hyperparameters & value \\
    \midrule
    critic lr & $5\times10^{-4}$ & actor lr & $5\times10^{-4}$ & share actor & true \\
    optimizer & Adam & actor and critic network & MLP & MLP hidden sizes & $[256,128,128]$ \\
    max grad norm & $10$ & buffer length & $64$ & buffer size & $6\times10^6$\\
    batch size & $16384$ & gamma & $0.95$ & exploration noise & $0.1$ \\
    tau & $0.005$ & target update interval & $4$ & critic loss & smooth L1 \\
    max episode length & $800$ & num envs & $4096$ & train steps & $1\times10^{9}$ \\
    \bottomrule
\end{tabular}
\caption{Hyperparameters used for (MA)DDPG in single-agent tasks and multi-agent tasks.}
    \label{tab:app:maddpg_hyperparameters}
\end{table}

\subsection{Hyperparameters of Benchmarking Algorithms}

\subsubsection{Single-Agent Tasks.}
% \yc{@xiangmin} \done
The hyperparameters adopted for DDPG and PPO in the single-agent tasks are listed in Table~\ref{tab:app:maddpg_hyperparameters}-\ref{tab:app:multi_different_hyperparameters}. All algorithms are trained for $5\times10^{8}$ environment steps in each task. 

\subsubsection{Multi-Agent Cooperative Tasks}
% \yc{@xiangmin} \done
The hyperparameters adopted for different algorithms in multi-agent cooperative tasks are listed in Table~\ref{tab:app:maddpg_hyperparameters}-\ref{tab:app:multi_different_hyperparameters}. All algorithms are trained for $1\times10^{9}$ environment steps in each task. 

\begin{table}[t]
    \centering
    \begin{tabular}{l c | l c | l c}
    \toprule
    hyperparameters & value & hyperparameters & value & hyperparameters & value \\
    \midrule
    optimizer & Adam & max grad norm & $10$ & entropy coef & $0.001$\\
    buffer length & $64$ & num minibatches & $16$ & ppo epochs & $4$ \\
    value norm & ValueNorm1 & clip param & $0.1$ & normalize advantages & True\\
    use huber loss & True & huber delta & $10$ & gae lambda & $0.95$ \\
    use orthogonal & True & gain & $0.01$ & gae gamma & $0.995$ \\
    max episode length & $800$ & num envs & $4096$ & train steps & $1\times10^{9}$ \\
    \bottomrule
\end{tabular}
\caption{Common hyperparameters used for (MA)PPO, HAPPO, and MAT in single-agent tasks and the multi-agent tasks.}
    \label{tab:app:multi_hyperparameters}
\end{table}

\begin{table}[t]
    \centering
    \begin{tabular}{l | c c c}
    \toprule
    Algorithms & (MA)PPO & HAPPO & MAT \\
    \midrule
        actor lr & $5\times10^{-4}$ & $5\times10^{-4}$ & $3\times10^{-5}$ \\
        critic lr & $5\times10^{-4}$ & $5\times10^{-4}$ & $3\times10^{-5}$  \\
        share actor & True & False & / \\
        hidden sizes & $[256,128,128]$ & $[256,128]$ & $[256,256,256]$ \\
        num blocks & / & / & $3$ \\
        num head & / & / & $8$ \\

    \bottomrule
\end{tabular}
\caption{Different hyperparameters used for (MA)PPO, HAPPO, and MAT in the single-agent tasks and multi-agent tasks.}


    \label{tab:app:multi_different_hyperparameters}
\end{table}

\subsubsection{Multi-Agent Competitive Tasks.}
% \yc{@huining, ruize} \done

\paragraph{Training.}
For self-play (SP) in \textit{1 vs 1} and \textit{3 vs 3} competitive tasks, we adopt the MAPPO algorithm with shared actor networks and shared critic networks between two teams, in order to make sure two teams utilize the same policy. Also, we transform the samples from both sides into symmetric ones and then use these symmetric samples to update the network together. The hyperparameters employed here are the same as those used in the MAPPO algorithm for multi-agent cooperative tasks.

The PSRO algorithm for \textit{1 vs 1} competitive task instantiates a PPO agent for training one of the two drones while the other drone maintains a fixed policy. Similarly, the PSRO algorithm for the \textit{3 vs 3} task assigns each team to be controlled by MAPPO. We adopt the same set of hyperparameters listed in Table \ref{tab:app:multi_different_hyperparameters} for the (MA)PPO agent. In each iteration, the (MA)PPO agent is trained against the current population. Here, we offer two versions of meta-strategy solver, PSRO\textsubscript{Uniform} and PSRO\textsubscript{Nash}. Training is considered converged when the agent achieves over 90\% win rate with a standard deviation below 0.05. The iteration ends when the agent reaches convergence or reaches a maximum of iteration steps of 5000. The trained actor is then added to the population for the next iteration. 

For Fictitious Self-Play (FSP) in competitive tasks, we slightly modify PSRO\textsubscript{Uniform} so that in each iteration, the (MA)PPO agent inherits the learned policy from the previous iteration as initialization. Naturally, other hyperparameters and settings remain the same for a fair comparison.

In both \textit{1 vs 1} and \textit{3 vs 3} tasks, the algorithm leverages 2048 parallel environments to accelerate the training process. In this work, we report the results of different algorithms given a total budget of $1\times 10^{9}$ environmental steps.

% \paragraph{Evaluation in Multi-Agent Competitive Tasks}
% \yc{@huining, ruize} \done
\paragraph{Evaluation.}
The evaluation of exploitability requires evaluating the payoff of the best response (BR) over the trained policy or population from different algorithms. Here, we approximate the BR to each policy or population by learning an additional RL agent against the trained policy or population. In practice, this is done by performing an additional iteration of PSRO, where the opponent is fixed as the trained policy/population. In order to approximate the ideal BR as closely as possible, we initialize the BR policy with the latest FSP policy, given that FSP yields the best empirical performance in our experiments. We train the BR policy for $5000$ training step with $2048$ parallel environments. We disable the convergence condition for early termination and report the evaluated win rate to calculate the approximate exploitability. Importantly, to approximate the BR of the trained SP policy in the \textit{3 vs 3} task, we employ two distinct BR policies for the serve and rally scenarios, respectively. For the BR to serve, we directly use the latest FSP policy without further training, while for the BR to rally, we train a dedicated policy against the SP policy. The overall win rate of this BR is then computed as the average win rate across these two scenarios, given that each side has an equal serve probability of $0.5$.

% \xzl{@ruize evaluation parameters of cross-play.} \done
We run 1,000 games for each pair of policies to generate the cross-play win rate heatmap, covering 6 matchup scenarios, resulting in a total of 6,000 games. In each game, both policies are sampled from their respective policy populations based on the meta-strategy and play until a winner is determined.

Moreover, we use an open-source Elo implementation~\cite{HankSheehan_EloPy}. The coefficient $K$ is set to $168$, and the initial Elo rating for all policies is $1000$. We conduct $12000$ games among four policies. The number of games played between any two policies is guaranteed to be the same. Specifically, in each round, $6$ different matchups are played. Each policy participates in 3 matchups, competing against different opponent policies. A total of $2000$ rounds are carried out, amounting to $12000$ games in total. The game results are sampled and generated based on the cross-play results.

\subsection{Results of Single-Agent Tasks}
\label{app:single}
% \yc{@xiangmin, put training curves here.} \done 

% \yc{@xiangmin add some analysis.} \done

The training curves of DDPG and PPO in single-agent tasks are shown in Fig.~\ref{fig:single_tasks}. It can be seen that PPO outperforms DDPG in all tasks. For example, in Back and Forth, using PPO, the number of times the agent reaches the target anchor stabilizes around $9-10$, and the task is successfully completed. In contrast, with DDPG, the agent only completes half of the return motion, as the drone ends up flying out of bounds. In Hit the Ball, with PPO, the hitting distance stabilizes around $10-11\,m$, and the ball is almost always hit in a straight line. However, with DDPG, the landing spot of the ball is less controllable. In Solo Bump, with PPO, the ball juggles smoothly, reaching a height of $4\,m$, while DDPG almost fails to juggle properly and can only manage a single hit. CTBR and PRT are comparable. in Back and Forth and Hit the Ball PRT's final result is better, in Solo Bump is comparable but CTBR learns faster.

% \xzl{e.g. (1) PPO is much better than DDPG in all tasks, list some number in each task. describe the behaviors of different algorithms in the video.
% (2) CTBR and PRT are comparable. in Back and Forth and Hit the Ball PRT's result is better, in xxx is comparable but CTBR learns faster, balabala. describe the behaviors of different action space in the video}


\begin{figure}[t]
    \centering
    \subfloat[\textit{Back and Forth}]{
        \centering
        \includegraphics[width=0.3\linewidth]{figs/single_tasks/bnf.pdf}}
    % \hspace{0.03\linewidth}
    \subfloat[\textit{Hit the Ball}]{
        \centering
        \includegraphics[width=0.3\linewidth]{figs/single_tasks/hit.pdf}} 
    \subfloat[\textit{Solo Bump}]{
        \centering
        \includegraphics[width=0.3\linewidth]{figs/single_tasks/bump.pdf}}
    % \hspace{0.03\linewidth}
    \caption{Training curves of single-agent tasks over three seeds.}
    \label{fig:single_tasks}
\end{figure}

\begin{figure}[t]
    \centering
    \subfloat[\textit{Bump and Pass} w.o. shaping]{
        \centering
        \includegraphics[width=0.3\linewidth]{figs/multi_tasks/bmp_wo_shaping.pdf}}
    \subfloat[\textit{Set and Spike (Easy)} w.o. shaping]{
        \centering
        \includegraphics[width=0.3\linewidth]{figs/multi_tasks/sns_easy_wo_shaping.pdf}}
    \subfloat[\textit{Set and Spike (Hard)} w.o. shaping]{
        \centering
        \includegraphics[width=0.3\linewidth]{figs/multi_tasks/sns_hard_wo_shaping.pdf}}
    % 第二行子图
    \par
    \subfloat[\textit{Bump and Pass} w. shaping]{
        \centering
        \includegraphics[width=0.3\linewidth]{figs/multi_tasks/bmp_w_shaping.pdf}}
    \subfloat[\textit{Set and Spike (Easy)} w. shaping]{
        \centering
        \includegraphics[width=0.3\linewidth]{figs/multi_tasks/sns_easy_w_shaping.pdf}} 
    \subfloat[\textit{Set and Spike (Hard)} w. shaping]{
        \centering
        \includegraphics[width=0.3\linewidth]{figs/multi_tasks/sns_hard_w_shaping.pdf}} 
    \caption{Training curves of multi-agent cooperative tasks over three seeds.}
    \label{fig:multi_tasks}
\end{figure}


\subsection{Results of Multi-Agent Cooperative Tasks}
\label{app:multi}
% \yc{@xiangmin, put training curves here.} \done
The training curves of different algorithms in multi-agent tasks are shown in Fig.~\ref{fig:multi_tasks}. It can be seen that MAPPO, HAPPO, and MAT are able to complete the tasks relatively well, while MADDPG can only make the setter complete one hit, with the attacker unable to receive the ball. In Bump and Pass w.o. shaping, MAPPO maintains a fast training process and outperforms HAPPO and MAT in terms of final results. The same is true in Set and Spike (Easy and Hard) w.o. shaping. In Set and Spike (Easy) w. shaping, MAPPO, and HAPPO perform similarly, with slightly faster learning speeds than MAT. In Set and Spike (Hard) w. shaping task, HAPPO only succeeds in overcoming the defense racket in one seed, resulting in a success rate of approximately $0.82$. At this point, the ball can be quickly spiked into the opponent's court and the attacker doesn't touch the net. Overall, the success rate is slightly higher than MAPPO and MAT, as these two algorithms almost never successfully overcome the defense racket.

Additionally, we can observe that the presence of shaping rewards has a significant impact on task results. Adding shaping rewards clearly improves the performance and accelerates the learning process. In Bump and Pass, the task learns slower without shaping rewards because the policy must explore which direction to hit the ball, requiring many more steps. The hit direction reward in shaping rewards accelerates this process. In Set and Spike (Easy), all algorithms without shaping rewards have a success rate of only $0.25$ because they only learn to make the setter hit the ball, but not toward the hitter. As a result, the attacker fails to hit the ball. The hit direction reward in shaping rewards also helps accelerate this process. In Set and Spike (Hard), HAPPO achieves a success rate of about $0.82$, while MAPPO and MAT are slightly worse, at $0.75$, because the defense racket is strong and successfully intercepts their attacks.

% \xzl{one paragraph for each task, in this paragraph (1) compare different algorithms (which is better or comparable...) (2) discuss the difference between w. and w.o. shaping, and their behaviors in the video. for example, in Bump and Pass, why do all algorithms w.o. shaping learn slower than w. shaping (because w.o. shaping reward the policy has to explore which direction to hit the ball, this will take many more steps. But the hit direction reward in shaping reward can help accelerate this process). In Set and Spike (Easy), why do all algorithms w.o. shaping reward only has a success rate of 0.25? (because they only learn to let the setter hit the ball, but not towards the hitter.) In Set and Spike (Hard), why do all algorithms with reward shaping only have a success rate of 0.75? (Because the defense board is strong and successfully intercepts their attack), ...}

% \yc{@xiangmin add some analysis.} \done


% \begin{figure}[t]
%     \centering
%     \subfloat[Bump and Pass w.o. shaping Reward]{
%         \centering
%         \includegraphics[width=0.45\linewidth]{figs/multi_tasks/Bump and Pass w.o. shaping Reward.png}}
%     \subfloat[Bump and Pass w. shaping Reward]{
%         \centering
%         \includegraphics[width=0.45\linewidth]{figs/multi_tasks/Bump and Pass w. shaping Reward.png}} 
%     % 第二行子图
%     \par
%     \subfloat[Set and Spike (Easy) w.o. shaping Reward]{
%         \centering
%         \includegraphics[width=0.45\linewidth]{figs/multi_tasks/Set and Spike (Easy) w.o. shaping Reward.png}}
%     \subfloat[Set and Spike (Easy) w. shaping Reward]{
%         \centering
%         \includegraphics[width=0.45\linewidth]{figs/multi_tasks/Set and Spike (Easy) w. shaping Reward.png}} 
%     % 第三行子图
%     \par
%     \subfloat[Set and Spike (Hard) w.o. shaping Reward]{
%         \centering
%         \includegraphics[width=0.45\linewidth]{figs/multi_tasks/Set and Spike (Hard) w.o. shaping Reward.png}}
%     \subfloat[Set and Spike (Hard) w. shaping Reward]{
%         \centering
%         \includegraphics[width=0.45\linewidth]{figs/multi_tasks/Set and Spike (Hard) w. shaping Reward.png}} 
%     \caption{Training curves of multi-agent tasks.}
%     \label{fig:multi_tasks}
% \end{figure}

\subsection{Results of Multi-Agent Competitive Tasks}
\label{app:mix}
We provide a more detailed win rate evaluation of the PSRO populations from the \textit{1 vs 1} task in Fig.~\ref{fig:1v1heatmap}, where each policy in the PSRO population is evaluated against all other policies. In these heatmaps, the ordinate and abscissa represent the policy for drone 1 and drone 2 respectively. The heat of cells represents the evaluated win rate of drone 1, i.e. red means a higher win rate and blue means a lower win rate. Intuitively, each row represents a policy's performance against each policy of the population while playing as drone 1. A red cell indicates that the drone 1 policy outperforms the specific drone 2 policy. A full red row means that the policy outperforms all other policies.

% \xzl{@huining add more description to help the reader understand this map, for example, red means higher win rate and blue means lower win rate. For each row, more red cells mean outperforming more policies. A full red row means the policy outperforms all other policies... }\done

Evidently, FSP attains more iterations than PSRO\textsubscript{Uniform} and PSRO\textsubscript{Nash} given a budget of $1\times10^9$ steps, which yields a faster convergence speed. This advantage comes from the fact that FSP inherits the learned policy from the previous iteration, which serves as an advantageous initialization for the current iteration. In contrast, PSRO\textsubscript{Uniform} and PSRO\textsubscript{Nash} start from scratch in each iteration, which poses a challenge for the algorithm to converge and introduces more variance in the training process.

Moreover, in PSRO algorithms, as the learned policy gradually improves with each iteration, the most recent policy of the population naturally poses greater difficulty for subsequent iterations. Therefore, PSRO\textsubscript{Nash} tends to put more weight on the most recent policy in the meta-strategy. 
% \yc{explain how we can see this?}\done 
This in turn has an effect on the learning of new policies. We can observe the outcomes in the heatmaps: for each row, the win rate against the most recent policy is often higher than the others. In FSP, on the other hand, the win rate against each policy is more evenly distributed, indicating that the population is potentially more balanced and stable.


\begin{figure}[t]
    \centering
    % 第一行子图
    \subfloat[FSP]{
        \centering
        \includegraphics[width=0.25\textwidth]{figs/1v1_heatmap/fsp_seed_300.png}
    }
    \subfloat[PSRO\textsubscript{Uniform}]{
        \centering
        \includegraphics[width=0.25\textwidth]{figs/1v1_heatmap/uniform_seed_300.png}
    }
    \subfloat[PSRO\textsubscript{Nash}]{
        \centering
        \includegraphics[width=0.25\textwidth]{figs/1v1_heatmap/nash_seed_300.png}
    }
    % % 第二行子图
    % \par
    % \subfloat[PSRO\textsubscript{Uniform} seed=1]{
    %     \centering
    %     \includegraphics[width=0.25\textwidth]{figs/1v1_heatmap/uniform_seed_100.png}
    % }
    % \subfloat[PSRO\textsubscript{Uniform} seed=2]{
    %     \centering
    %     \includegraphics[width=0.25\textwidth]{figs/1v1_heatmap/uniform_seed_200.png}
    % }
    % \subfloat[PSRO\textsubscript{Uniform} seed=3]{
    %     \centering
    %     \includegraphics[width=0.25\textwidth]{figs/1v1_heatmap/uniform_seed_300.png}
    % }
    % % 第三行子图
    % \par
    % \subfloat[PSRO\textsubscript{Nash} seed=1]{
    %     \centering
    %     \includegraphics[width=0.25\textwidth]{figs/1v1_heatmap/nash_seed_100.png}
    % }
    % \subfloat[PSRO\textsubscript{Nash} seed=2]{
    %     \centering
    %     \includegraphics[width=0.25\textwidth]{figs/1v1_heatmap/nash_seed_200.png}
    % }
    % \subfloat[PSRO\textsubscript{Nash} seed=3]{
    %     \centering
    %     \includegraphics[width=0.25\textwidth]{figs/1v1_heatmap/nash_seed_300.png}
    % }
    \caption{Win rate heatmaps of the population in the \textit{1 vs 1} task.}
    \label{fig:1v1heatmap}
\end{figure}

\subsection{Low-level Skills of Hierarchical Policy}
\label{app:low}
% \yc{@ruize, write low-level skills} \done

Low-level skills are derived through PPO training, while the high-level skill is implemented as a rule-based, event-driven policy that determines which drone utilizes which skill in response to the current game state. In accordance with the \textit{3 vs 3} task setting, each team consists of three drones positioned as front-left, front-right, and backward within their half of the court. Below, we describe each low-level skill and explain when it is utilized by the high-level policy. 

\paragraph{Hover.}
The \textit{Hover} skill is designed to enable the drone to hover around a specified target position. This skill takes a three-dimensional target position as input. The skill is frequently utilized by the high-level policy. For instance, in the serve scenario, only the serving drone uses the \textit{Serve} skill, while the other two teammates use the \textit{Hover} skill to remain at their respective anchor points.

\paragraph{Serve.}
The \textit{Serve} skill is designed to enable the drone to serve the ball towards the opponent’s side of the court. This skill includes a one-hot target input that determines whether to serve the ball to the left side or the right side of the opponent’s court.
In accordance with the \textit{3 vs 3} task setting, for the \textit{Serve} skill, the ball is initialized at a position $3\,\mathrm{m}$ directly above the serving drone, with zero initial velocity. 
This skill is exclusively utilized by the high-level policy during the serve scenario, during which the designated serving drone employs the \textit{Serve} skill at the start of a match.

\paragraph{Pass.}
The \textit{Pass} skill is designed to handle the opponent’s serve or attack by allowing the drone to make the first contact of the team’s turn and pass the ball to a teammate. Here, this skill is exclusively used by the backward drone, which is responsible for hitting the ball to the front-left teammate. The high-level policy designates the backward drone to utilize this skill whenever the opponent hits the ball.

\paragraph{Set.}
The \textit{Set} skill is designed to transfer the ball from the passing drone to the attacking drone, serving as the second contact in the team’s turn. In our design, the front-left drone utilizes the \textit{Set} skill to pass the ball from the backward drone to the front-right drone. The high-level policy designates the front-left drone to utilize this skill whenever the backward drone successfully makes contact with the ball.

\paragraph{Attack.}
The \textit{Attack} skill is designed to hit the ball towards the opponent’s court, serving as the third and final contact in the team’s turn. This skill includes a one-hot target input that specifies whether to direct the ball to the left side or the right side of the opponent’s court. In our design, the front-right drone uses the \textit{Attack} skill to strike the ball after receiving it from the front-left drone. The high-level policy assigns the front-right drone to utilize this skill whenever the front-left drone successfully hits the ball.


\subsection{Sim-to-Real}
% \yc{@shilong} \done

The sim-to-real gap presents a significant challenge in RL-based robotic control, as policies trained in simulation often underperform when deployed in the real world. We utilize the \textit{Solo Bump} task as an initial demonstration to showcase the potential of transferring trained policies to real-world scenarios. To bridge the sim-to-real gap, we apply system identification to accurately model the ball's behavior during its impact with the racket.

The impact between the ball and the racket is modeled as an impulse acting in the direction of the normal of the racket $\bm{n}_c$, which means that the tangential velocity of the ball relative to the racket remains constant but the normal velocity changes, determined by the restitution coefficient. Since the mass of the racket and vehicle (rigidly mounted together) is much larger than that of the ball, it is assumed that the velocity of the racket remains unaffected. 
Denoting the ball's pre- and post-impact normal velocities relative to the racket as $\bm{v}_{ball\_pre\_n}$ and $\bm{v}_{ball\_post\_n}$, the restitution coefficient can be derived as follows:

\begin{align}
\beta &= -\frac{\bm{v}_{ball\_post\_n}^T\bm{n}_c}{\bm{v}_{ball\_pre\_n}^T\bm{n}_c}
\end{align}

Specifically, the restitution coefficient was measured through multiple experiments in which the racket and vehicle were fixed on the flat ground and the ball started a free fall from different heights right above the center of the racket. The pre-impact and post-impact velocity of the ball was collected using a motion capture system and the restitution coefficient was calculated using the above equation. We used the same ball and racket in all experiments.

The average restitution coefficient was $0.8$, which we used in the simulation. The highest and lowest were $0.85$ and $0.75$, respectively, and the deviation was relatively small and acceptable. Experiments also showed that the ball’s tangential velocity did change during impact, which was likely due to the friction between the ball and the racket. In addition, the asymmetric relative position from racket strings to the ball could lead to asymmetric force upon the ball, causing it to move tangentially. We model the change of tangential velocity of the ball during impact as a stochastic progress and utilize small randomization in the ball's rebound velocity after each collision in the simulation.


%%%%%%%%%%%%%%%%%%%%%%%%%%%%%%%%%%%%%%%%%%%%%%%%%%%%%%%%%%%%%%%%%%%%%%%%%%%%%%%
%%%%%%%%%%%%%%%%%%%%%%%%%%%%%%%%%%%%%%%%%%%%%%%%%%%%%%%%%%%%%%%%%%%%%%%%%%%%%%%


\end{document}


% This document was modified from the file originally made available by
% Pat Langley and Andrea Danyluk for ICML-2K. This version was created
% by Iain Murray in 2018, and modified by Alexandre Bouchard in
% 2019 and 2021 and by Csaba Szepesvari, Gang Niu and Sivan Sabato in 2022.
% Modified again in 2023 and 2024 by Sivan Sabato and Jonathan Scarlett.
% Previous contributors include Dan Roy, Lise Getoor and Tobias
% Scheffer, which was slightly modified from the 2010 version by
% Thorsten Joachims & Johannes Fuernkranz, slightly modified from the
% 2009 version by Kiri Wagstaff and Sam Roweis's 2008 version, which is
% slightly modified from Prasad Tadepalli's 2007 version which is a
% lightly changed version of the previous year's version by Andrew
% Moore, which was in turn edited from those of Kristian Kersting and
% Codrina Lauth. Alex Smola contributed to the algorithmic style files.
