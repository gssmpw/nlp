\subsection{Court}
The volleyball court in our environment is depicted in Fig.~\ref{fig:app:court}. The court is divided into two equal halves by the $y$-axis, which serves as the dividing line separating the two teams. The coordinate origin is located at the midpoint of the dividing line, and the $x$-axis extends along the length of the court. The total court length is $18\,{m}$, with $x = -9$ and $x = 9$ marking the ends of the court. The $y$-axis extends across the width of the court, with a total width of $9\,{m}$, spanning from $y = -4.5$ to $y = 4.5$. The net is positioned at the center of the court along the $y$-axis, with a height of $2.43\,{m}$, and spans horizontally from $(0, -4.5)$ to $(0, 4.5)$.

\begin{figure}[h]
    \centering
    \includegraphics[width=0.5\linewidth]{figs/court.pdf}
    \caption{Volleyball court layout in our environment with coordinates.}
    \label{fig:app:court}
\end{figure}


\subsection{Drone}
We use the \textit{Iris} quadrotor model~\cite{furrer2016rotors} as the primary drone platform, augmented with a virtual ``racket'' of radius $0.2\,{m}$ and coefficient of restitution $0.8$ for ball striking. The drone's root state is a vector with dimension $23$, including its position, rotation, linear velocity, angular velocity, forward orientation, upward orientation, and normalized rotor speeds.

The control dynamics of a multi-rotor drone are governed by its physical configuration and the interaction of various forces and torques. The system's dynamics can be described as follows:

\begin{align}
\bm{\dot{x}}_W = \bm{v}_W, \quad \bm{\dot{v}}_W = \bm{R_{WB}}f + \bm{g} + \bm{F} \\
\bm{\dot{q}} = \frac{1}{2} \bm{q} \otimes \bm{\omega}, \quad \bm{\dot{\omega}} = \bm{J}^{-1} (\bm{\eta} - \bm{\omega} \times \bm{J} \bm{\omega})
\end{align}

where $\bm{x}_W$ and $\bm{v}_W$ represent the position and velocity of the drone in the world frame, $\bm{R}_{WB}$ is the rotation matrix converting from the body frame to the world frame, $\bm{J}$ is the diagonal inertia matrix, $\bm{g}$ denotes gravity, $\bm{q}$ is the orientation represented by quaternions, and $\bm{\omega}$ is the angular velocity. The quaternion multiplication operator is denoted by $\otimes$. External forces $\bm{F}$, including aerodynamic drag and downwash effects, are also considered. The collective thrust $\bm{f}$ and body rate $\bm{\eta}$ are computed based on per-rotor thrusts $\bm{f}_i$ as:

\begin{align}
\bm{f} &= \sum_{i} \bm{R}^{(i)}_B \bm{f}_i \\
\bm{\eta} &= \sum_{i} \bm{T}^{(i)}_B \times \bm{f}_i + k_i \bm{f}_i
\end{align}

where $\bm{R}^{(i)}_B$ and $\bm{T}^{(i)}_B$ are the local orientation and translation of the $i$-th rotor in the body frame, and $k_i$ is the force-to-moment ratio.


\subsection{Defense Racket}
% \xzl{@chuqi briefly describe the implementation of the defense racket.}
% \done

We assume a thin cylindrical racket to mimic a human-held racket for adversarial interactions with a drone. 
%The restitution coefficient of the racket is set to match that of the ball, ensuring collision simulations align as closely as possible with reality. 
When the ball is hit toward the racket’s half of the court, the racket is designed to intercept the ball at a predefined height $h_{pre}$. Since the ball’s position and velocity data can be directly acquired, the descent time $t_{pre}$, landing point $\bm{p}_{ball\_land}$, and pre-collision velocity $\bm{v}_{ball\_pre}$ can be calculated using projectile motion equations. Additionally, to ensure the ball is returned to a designated position $\boldsymbol{p_b{}_{des}}$ and crosses the net, the post-collision motion duration $t_{post}$ of the ball is set to a sufficiently large value. This allows the projectile motion equations to similarly determine the post-collision velocity $\bm{v}_{ball\_post}$. Based on these conditions, the required collision position $\bm{p}_{collision}$, orientation $\bm{\theta}_{collision}$ and velocity $\bm{v}_{collision}$ of the racket can be derived  as follows:

\begin{align}
\bm{p}_{collision} &= \bm{p}_{ball\_land} \\
\bm{n}_{collision} &= \frac{\bm{v}_{ball\_post} - \bm{v}_{ball\_pre}}{\| \bm{v}_{ball\_post} - \bm{v}_{ball\_pre} \|} = [\sin p\cos r, -\sin r, \cos p\cos r] \\
\bm{\theta}_{collision} &= [-\arcsin {\bm{n}_{collision}}(2), \arctan {\frac{\bm{n}_{collision}(1)}{\bm{n}_{collision}(3)}}, 0] \\
\bm{v}_{collision} &= \frac{1}{1+\beta}(\beta \bm{v}_{ball\_pre} + \bm{v}_{ball\_post})
\end{align}

where $\bm{n}_{collision}$ represents the normal vector of the racket during impact, $r$ denotes the roll angle of the racket, $p$ denotes the pitch angle, while the yaw angle remains fixed at 0, and $\beta$ represents the restitution coefficient. To simulate the adversarial interaction as realistically as possible, we impose direct constraints on the racket’s linear velocity and angular velocity. Based on the simulation time step $t_{step}$ and the descent time $t_{post}$ of the ball, we can calculate the required displacement $\boldsymbol{d} = \frac{\boldsymbol{p_b{}_{des}}-\bm{p}_{ball\_land}}{t_{post}}t_{step}$ and rotation angle $\boldsymbol{\theta} = \frac{\bm{\theta}_{collision}}{t_{post}}t_{step}$ that the racket must achieve within each time step. If both $\boldsymbol{d}$ and $\boldsymbol{\theta}$ do not exceed their respective limits ($\boldsymbol{d}_{max}$ and $\boldsymbol{\theta}_{max}$), the racket moves with linear velocity $\boldsymbol{d}$ and angular velocity $\boldsymbol{\theta}$. Otherwise, the values are set to their corresponding limits $\boldsymbol{d}_{max}$ and $\boldsymbol{\theta}_{max}$.
