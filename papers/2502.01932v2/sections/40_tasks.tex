\begin{figure*}[t]
    \centering
    \includegraphics[width=0.9\linewidth]{figs/tasks.pdf}
    % \caption{\yc{Proposed tasks in the VolleyBots testbed, inspired by the process of human learning in volleyball. Single-agent tasks evaluate low-level control, while multi-agent cooperative and competitive tasks integrate high-level decision-making with low-level control.}}
    \caption{Proposed tasks in the VolleyBots testbed, inspired by the process of human learning in volleyball. Single-agent tasks evaluate low-level control, while multi-agent cooperative and competitive tasks integrate high-level decision-making with low-level control.}
    \label{fig:tasks}
\end{figure*}

% \yc{eval metric}
% \yc{Inspired by the way humans progressively learn to play volleyball, we introduce a series of tasks that systematically assess both low-level flight control and higher-level coordination, as shown in Fig.~\ref{fig:tasks}.}
Inspired by the way humans progressively learn to play volleyball, we introduce a series of tasks that systematically assess both low-level motion control and high-level strategic play, as shown in Fig.~\ref{fig:tasks}.
% Building on the described environment and setup, we introduce a suite of volleyball-inspired tasks that systematically evaluate both low-level flight control and higher-level coordination. 
These tasks are organized into three categories: single-agent, multi-agent cooperative, and multi-agent competitive. Each category aligns with standard volleyball drills or match settings commonly adopted in human training, ranging from basic ball control, through cooperative play, to full competitive games. Evaluation metrics vary across tasks to assess performance in motion control, cooperative teamwork, and strategic competition. The detailed configuration and reward design of each task can be found in Appendix~\ref{app:task}.
% \yc{when will the task be done? the sim duration is? the max episode length?}

\subsection{Single-Agent Tasks}

Single-agent tasks are designed to follow typical solo training drills used in human volleyball practice, including \textit{Back and Forth}, \textit{Hit the Ball}, and \textit{Solo Bump}. These tasks evaluate the drone's basic capabilities such as flight stability, motion control, and ball-handling proficiency.

\textbf{Back and Forth.}
The drone sprints between two designated points to complete as many round trips as possible within the time limit. This task is analogous to the back-and-forth sprints in volleyball practice and requires basic motion control of the drone. The performance is evaluated by the number of completed round trips within the time limit.


\textbf{Hit the Ball.}
The ball is initialized directly above the drone, and the drone hits the ball once to make it land as far as possible. This task is analogous to the typical hitting drill in volleyball and requires both motion control and ball-handling proficiency. The performance is evaluated by the distance of the ball's landing position from the initial position.

\textbf{Solo Bump.}
The ball is initialized directly above the drone, and the drone bumps the ball in place to a specific height as many times as possible within the time limit. This task is analogous to the solo bump drill in human practice and requires motion control, ball-handling proficiency, and stability. The performance is evaluated by the number of successful bumps within the time limit.

% \subsection{Multi-Agent Cooperative Tasks}

% Cooperative tasks focus on pairwise or group-based collaboration to maintain or direct the ball. These tasks parallel common two-player routines in volleyball coaching, such as back-and-forth passing and setting–spiking drills. Agents must coordinate their movements and share roles to complete each objective efficiently.

\begin{table*}[t]
    \centering
    \renewcommand{\arraystretch}{1.5}
\setlength{\tabcolsep}{5pt} % Default is 6pt
\captionsetup[table]{justification=raggedright, singlelinecheck=false}
\begin{table*}[!t]
    \centering
    \small
    \begin{tabular}{ccccccccccc}  % Defines 5 columns
         \hline
         &  & \multicolumn{4}{c}{\textbf{Train}} & \multicolumn{4}{c}{\textbf{Test}} & \textbf{} \\
        \cline{3-6} \cline{7-10} 
        \textbf{Model} & \textbf{Method} & \multicolumn{2}{c}{\textbf{ASR@10}} & \multicolumn{2}{c}{\textbf{ASR@1}} & \multicolumn{2}{c}{\textbf{ASR@10}} & \multicolumn{2}{c}{\textbf{ASR@1}} & \textbf{PPL} \\
        \cline{3-4} \cline{5-6} \cline{7-8} \cline{9-10} 
         &  & S & LG & S & LG & S & LG & S & LG &  \\
        \hline
        \multirow{2}{*}{Llama2-7b} & \textbf{AdvPrompter} & 18.3 & 12.8 & 11.5 & 6.4 & 7.7 & 5.8 & 2.9 & 1.9 & 160.107 \\ %\cline{2-15}
            % & \textbf{AutoDAN} & \underline{42.3} & \underline{34.9} & \textbf{19.2} & \textbf{13.8} & \underline{37.5} & \underline{27.9} & \underline{11.5} & \underline{8.7} & 251.687 \\
            % & \textbf{GPTFuzzer} & 32.4 & 31.4 & 3.2 & 1.0 & 26.9 & \underline{27.9} & 2.9 & 1.9 & \textbf{16.272} \\
            & \textbf{BEAST}-univ & --- & --- & \textbf{55.1} & \textbf{11.2} & --- & --- & \textbf{43.3} & \textbf{6.7} & \textbf{129.983} \\ 
            % & \textbf{JUMP++} & \textbf{64.4} & \textbf{51.0} & \underline{18.3} & \underline{12.8} & \textbf{55.8} & \textbf{50.0} & \textbf{15.4} & \textbf{12.5} & 119.245 \\ 
        \hline
        \multirow{2}{*}{Llama3-8b} & \textbf{AdvPrompter} & 66.7 & 42.9 & \textbf{38.8} & \textbf{18.6} & 46.2 & 26.0 & \textbf{8.7} & \textbf{4.8} & 116.354 \\ %\cline{2-15}
            % & \textbf{AutoDAN} & 22.8 & 14.7 & 6.4 & 2.6 & 15.4 & 11.5 & 4.8 & 2.9 & 301.689 \\
            % & \textbf{GPTFuzzer} & 45.8 & \underline{49.4} & 8.3 & 8.7 & 39.4 & \underline{42.3} & 4.8 & \underline{6.7} & \textbf{12.285} \\
            & \textbf{BEAST}-univ & --- & --- & 2.9 & 0.3 & --- & --- & 1.0 & 1.0 & \textbf{52.951} \\
            % & \textbf{JUMP++} & \textbf{76.6} & \textbf{62.5} & \textbf{39.1} & \textbf{26.0} & \textbf{82.7} & \textbf{64.4} & \textbf{33.7} & \textbf{24.0} & 82.427 \\
        \hline
    \end{tabular}
    \caption{Universal jailbreak results without handcrafted assistance. We compare the BEAST-univ setting, which attacks with a single prompt, with other baselines. The results show that this setting finds it difficult to perform equally well on all models. Data in \textbf{bold} font represent the best results.}
    \label{tab:single}
\end{table*}
    \caption{Benchmark result of single-agent tasks with different action spaces including Collective Thrust and Body Rates (CTBR) and Per-Rotor Thrust (PRT). \textit{Back and Forth} is evaluated by the number of target points reached, \textit{Hit the Ball} is evaluated by the hitting distance, and \textit{Solo Bump} is evaluated by the number of bumps achieving a certain height.}
    \label{tab:single}
\end{table*}

\subsection{Multi-Agent Cooperative Tasks}

Multi-agent cooperative tasks are inspired by standard two-player training drills used in volleyball teamwork, including \textit{Bump and Pass}, \textit{Set and Spike (Easy)}, and \textit{Set and Spike (Easy)}. Besides basic motion control and ball handling, these tasks also require teamwork and cooperation.

\textbf{Bump and Pass.}
Two drones work together to bump and pass the ball to each other back and forth as many times as possible within the time limit. This task is analogous to the two-player bumping practice in volleyball training and requires homogeneous multi-agent cooperation. The performance is evaluated by the number of successful bumps within the time limit.

\textbf{Set and Spike (Easy).}
Two drones take on the role of a setter and an attacker. The setter passes the ball to the attacker, and the attacker then spikes the ball downward to the target region on the opposing side. This task is analogous to the setter-attacker offensive drills in volleyball training and requires heterogeneous multi-agent cooperation. The performance is evaluated by the success rate of the downward spike to the target region.

\textbf{Set and Spike (Hard).}
Similar to \textit{Set and Spike (Easy)} task, two drones act as a setter and an attacker to set and spike the ball to the opposing side. The difference is that there is a rule-based defense board on the opposing side to intercept the attacker's spike. The presence of the defense board further improves the difficulty of the task, requiring the drones to optimize their speed, precision, and cooperation to defeat the defense board. The performance is evaluated by the success rate of the downward spike that defeats the defense racket.

\subsection{Multi-Agent Competitive Tasks}

Multi-agent competitive tasks follow the standard volleyball match rules, including the competitive \textit{1 vs 1} task and the mixed cooperative-competitive \textit{3 vs 3} task. These tasks evaluate both the low-level motion control and the high-level strategic play of the drone policy. 
% Detailed discussion of tasks and evaluation metrics can be found in Appendix~\ref{app:1v1}.

\textbf{1 vs 1.}
One drone on each side competes against the other in a volleyball match and wins by hitting the ball in the opponent's court. When the ball is on its side, the drone is allowed only one hit to return the ball to the opponent's court. This two-player zero-sum setting creates a purely competitive environment that requires both precise flight control and strategic gameplay.
To evaluate the performance of the learned policy, we consider three typical metrics including the exploitability, the average win rate against other learned policies, and the Elo rating~\cite{elo1978rating}. More specifically, the exploitability is approximated by the gap between the learned best response’s win rate against the evaluated policy and its expected win rate at Nash equilibrium, and the Elo rating is computed by running a round-robin tournament between the evaluated policy and a fixed population of policies.

% One drone on each side competes to win by landing the ball in the opponent’s court, alternating possessions according to volleyball rules. The minimal team size highlights individual skill, reaction speed, and strategic decision-making. We evaluate policies through multiple metrics, including exploitability, win rates against a built-in defense board, and cross-play performance in round-robin tournaments that yield overall win rates and Elo ratings~\cite{elo1978rating}, which are typical metrics to evaluate the performance in competitive games.
% \yc{cite? which is a typical metric to evaluate competitive game?}.

\textbf{3 vs 3.}
Three drones on each side form a team to compete against the other team in a volleyball match. The drones in the same team cooperate to serve, pass, spike, and defend within the standard rule of three hits per side. This is a challenging mixed cooperative-competitive game that requires both cooperation within the same team and competition between the opposing teams.
Moreover, the drones are required to excel at both low-level motion control and high-level game play.  
We evaluated the policy performance using approximate exploitability, the average win rate against other learned policies, and the Elo rating of the policy.

