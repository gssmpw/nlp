This work introduces VolleyBots, an MARL testbed specifically designed to push the boundaries of multi-agent reinforcement learning involving high-mobility robotic platforms such as drones. 
% The broader impacts of this research include advancing the intersection of MARL and robotics, enhancing the decision-making capabilities of drones in complex scenarios, and laying the groundwork for the development of innovative applications such as robotic sports systems.
The broader impacts of this research include advancing the intersection of MARL and robotics, enhancing the decision-making capabilities of drones in complex scenarios.
By bridging MARL with real-world robotic challenges, this work aims to inspire future breakthroughs in both robotics and multi-agent AI systems.

% This work introduces VolleyBots, an MARL testbed designed to advance the field of MARL, particularly in settings involving high-mobility robotic platforms like drones. The potential broader impacts of this work include contributions to the development of more robust and adaptive AI systems capable of handling real-world physical and strategic challenges. These advancements could benefit fields such as robotics, autonomous systems, and cooperative AI. 
% The focus on sim-to-real transfer aligns with efforts to bridge the gap between research and practical applications, enabling safer, more efficient deployment of drones in industries such as logistics and disaster response.


% While this research is primarily technical, the deployment of drones raises ethical considerations, such as privacy concerns and potential misuse. Researchers and practitioners should ensure adherence to ethical guidelines and regulations when utilizing the outcomes of this work. We believe this work contributes positively to the broader machine learning and robotics communities, offering a challenging and flexible testbed to foster innovation while promoting responsible and ethical applications of MARL technologies.