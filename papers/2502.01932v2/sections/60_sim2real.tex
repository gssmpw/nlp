\begin{figure}[t]
    \centering
    \subfloat[Serve]{
        \includegraphics[width=0.45\linewidth]{figs/hier_serve.pdf}} 
    % \hspace{0.05\linewidth}
    \hfill
    \subfloat[Attack]{
        \includegraphics[width=0.45\linewidth]{figs/hier_rally.pdf}}
    \vspace{-2mm}
    \caption{Demonstration of the hierarchical policy selecting \textit{Serve} and \textit{Attack} skills in the \textit{3 vs 3} task.}
    \label{fig:hier}
    \vspace{-4mm}
\end{figure}

We use the \textit{Solo Bump} task as a demonstration of the policy’s ability to zero-shot transfer to the real world. We use a quadrotor with a rigidly mounted badminton racket, as shown in Fig.~\ref{fig:overview}. The state of both the drone and the ball is captured using a motion capture system. The drone is modeled as a rigid body, with its position and orientation provided by the motion capture system. The drone's velocity is estimated using an Extended Kalman Filter (EKF) that fuses pose data from the motion capture system and IMU data from the PX4 Autopilot. The ball is modeled as a point mass, with its position sent by the motion capture system and its velocity indirectly computed through a Kalman Filter. The drone's dynamics parameters and the ball’s properties are determined through system identification.

To simulate real-world noise and imperfect execution of actions, small randomizations are introduced in the ball’s initial position, coefficient of restitution, and the ball's rebound velocity after each collision with the drone. Inspired by~\cite{chen2024matterslearningzeroshotsimtoreal}, we also add a smoothness reward to encourage smooth actions. The policy uses CTBR as output and is deployed on the onboard Nvidia Orin processor.
Experiment results show that the drone successfully performs bump tasks multiple times, providing initial evidence of sim-to-real transfer capability. To support further research, we make the drone configuration, model checkpoint, and real-world deployment videos publicly available on our website. We hope this will accelerate progress in this field.