\section{Related Work}
\paragraph{Long-form Text Generation}
Recent advances in long-form generation have focused on improving models through architectural enhancements and specialized training techniques~\cite{salemi2025experteffectiveexplainableevaluation,que2024hellobenchevaluatinglongtext,liu-etal-2023-task,Li2023TeachLT}. Approaches like Re3~\cite{yang-etal-2022-re3} use recursive reprompting for extended story generation, while DOC~\cite{yang-etal-2023-doc} and hierarchical outlining~\cite{wang2024generatinglongformstoryusing} improve narrative coherence through structured task decomposition. Personalized long-form generation has also gained attention~\cite{salemi2025experteffectiveexplainableevaluation,wang-etal-2024-learning-personalized}, with methods like LongLaMP~\cite{kumar2024longlampbenchmarkpersonalizedlongform} and reasoning-enhanced techniques~\cite{salemi2025reasoningenhancedselftraininglongformpersonalized} adapting models to meet user-specific needs. Similarly, long-form question answering focuses on producing detailed responses to complex queries~\cite{dasigi-etal-2021-dataset,stelmakh-etal-2022-asqa,pmlr-v202-lee23n,tan-etal-2024-proxyqa}. While these methods have improved generation capabilities~\cite{wu2024longgenbench,que2024hellobenchevaluatinglongtext}, our work addresses a critical gap by examining long-form generation through the lens of cognitive writing theory.


\begin{figure}[t!]
\centering
  \includegraphics[width=0.99\linewidth]{figs/instruction_performance_comparison.pdf} 
  \caption {Comparison of Instruction Type Performance.}
  \label{compt}
\end{figure}

\paragraph{Multi-agent Writing}
Multi-agent writing has made notable progress in recent years~\cite{hu-etal-2025-debate,pichlmair2024dramaengineframeworknarrative}, showing how agents can collaborate on diverse writing tasks~\cite{10.1007/s11704-024-40231-1,hong2024metagpt}. Research has explored heterogeneous agent integration\cite{chen2025internet} and educational applications\cite{10410857}. In academic writing, frameworks like SciAgents~\cite{Ghafarollahi2024SciAgentsAS} demonstrate collaboration among specialized agents for complex writing tasks~\cite{2024autosurvey,DArcy2024MARGMR,su2024headsbetteronemultiagent}, while the Agents’ Room approach~\cite{huot2024agentsroomnarrativegeneration} highlights the value of task decomposition in narrative writing.
Beyond academic contexts, multi-agent methods have been applied to creative and informational writing, such as Wikipedia-style articles~\cite{shao-etal-2024-assisting} and poetry~\cite{zhang2024llmbasedmultiagentpoetrygeneration,chen-etal-2024-evaluating-diversity}. 
While these methods focus on collaboration, our work applies cognitive writing principles with agents for planning, monitoring, and revisions, enabling flexible adaptation without task-specific training.