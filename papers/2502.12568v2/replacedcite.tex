\section{Related Work}
\paragraph{Long-form Text Generation}
Recent advances in long-form generation have focused on improving models through architectural enhancements and specialized training techniques____. Approaches like Re3____ use recursive reprompting for extended story generation, while DOC____ and hierarchical outlining____ improve narrative coherence through structured task decomposition. Personalized long-form generation has also gained attention____, with methods like LongLaMP____ and reasoning-enhanced techniques____ adapting models to meet user-specific needs. Similarly, long-form question answering focuses on producing detailed responses to complex queries____. While these methods have improved generation capabilities____, our work addresses a critical gap by examining long-form generation through the lens of cognitive writing theory.


\begin{figure}[t!]
\centering
  \includegraphics[width=0.99\linewidth]{figs/instruction_performance_comparison.pdf} 
  \caption {Comparison of Instruction Type Performance.}
  \label{compt}
\end{figure}

\paragraph{Multi-agent Writing}
Multi-agent writing has made notable progress in recent years____, showing how agents can collaborate on diverse writing tasks____. Research has explored heterogeneous agent integration____ and educational applications____. In academic writing, frameworks like SciAgents____ demonstrate collaboration among specialized agents for complex writing tasks____, while the Agents’ Room approach____ highlights the value of task decomposition in narrative writing.
Beyond academic contexts, multi-agent methods have been applied to creative and informational writing, such as Wikipedia-style articles____ and poetry____. 
While these methods focus on collaboration, our work applies cognitive writing principles with agents for planning, monitoring, and revisions, enabling flexible adaptation without task-specific training.