%%%% ijcai25.tex

\typeout{IJCAI--25 Instructions for Authors}

% These are the instructions for authors for IJCAI-25.

\documentclass{article}
\pdfpagewidth=8.5in
\pdfpageheight=11in

% The file ijcai25.sty is a copy from ijcai22.sty
% The file ijcai22.sty is NOT the same as previous years'
\usepackage{ijcai25}

% Use the postscript times font!
\usepackage{times}
\usepackage{soul}
\usepackage{url}
\usepackage[hidelinks]{hyperref}
\usepackage[utf8]{inputenc}
\usepackage[small]{caption}
\usepackage{graphicx}
\usepackage{amsmath}
\usepackage{amsthm}
\usepackage{booktabs}
\usepackage{algorithm}
\usepackage{algorithmic}
\usepackage[switch]{lineno}
\usepackage{mydef}

% Comment out this line in the camera-ready submission
% \linenumbers

\urlstyle{same}
\newtheorem{example}{Example}
\newtheorem{theorem}{Theorem}

\pdfinfo{
/TemplateVersion (IJCAI.2025.0)
}

\title{Graph Foundation Models for Recommendation: A Comprehensive Survey}


\author{
Bin Wu$^1$\footnote{Work done during internship at Baidu Inc.}\and
Yihang Wang$^1$\footnotemark[1]\and
Yuanhao Zeng$^1$\and
Jiawei Liu$^1$\and
Jiashu Zhao$^{2, 3}$\and\\
Cheng Yang$^1$\and
Yawen Li$^1$\and
Long Xia$^2$\and
Dawei Yin$^2$\and
Chuan Shi$^1$\footnote{Corresponding authors.}\\
\affiliations
$^1$Beijing University of Posts and Telecommunications\\
$^2$Baidu Inc.\\
$^3$Wilfrid Laurier University\\
}

\begin{document}

\maketitle

% \begin{abstract}
%    As the mainstream method for exploring user interests in vast amounts of data, recommender systems (RS) have recently been significantly influenced by graph neural networks (GNNs) and large language models (LLMs). GNN-based RS extracts higher-order structural information from the data, and the textual information within the data can be efficiently processed and comprehended by LLMs. The integration of GNNs and LLMs has led to the emergence of graph foundation models (GFMs), igniting a wave of innovation in the field of RS.
%     % In RS, structural and textual information within the data is intricately intertwined, making it an ideal scenario for the application of GFMs.
%     % GFM-based RS can harmoniously integrate these information, heralding a new era in the field of recommendations.
%     In this survey, we provide a comprehensive overview of GFM-based RS, discussing how they improve recommendation performance by integrating graph structural information with textual information from LLM.
%     Firstly, we present a clear taxonomy, then discuss each category in detail, and finally highlight the challenges and promising prospects in this field.
% \end{abstract}
\begin{abstract}
    Recommender systems (RS) serve as a fundamental tool for navigating the vast expanse of online information, with deep learning advancements playing an increasingly important role in improving ranking accuracy. Among these, graph neural networks (GNNs) excel at extracting higher-order structural information, while large language models (LLMs) are designed to process and comprehend natural language, making both approaches highly effective and widely adopted. Recent research has focused on graph foundation models (GFMs), which integrate the strengths of GNNs and LLMs to model complex RS problems more efficiently by leveraging the graph-based structure of user-item relationships alongside textual understanding. In this survey, we provide a comprehensive overview of GFM-based RS technologies by introducing a clear taxonomy of current approaches, diving into methodological details, and highlighting key challenges and future directions. By synthesizing recent advancements, we aim to offer valuable insights into the evolving landscape of GFM-based recommender systems.
\end{abstract}
\section{Introduction}
Recommender systems are essential components of contemporary digital landscape, enabling personalized services across a diverse range of fields, including e-commerce, social media, and entertainment \cite{zhang2023robust}. The data in RS generally consist of both structural information (\textit{e.g.}, user-item interactions) and textual information (\textit{e.g.}, user attributes and item descriptions).
% Based on technology and utilization of data, RS can be divided into three categories: content-based RS, collaborative filtering RS and hybrid filtering RS~\cite{adomavicius2005toward}.
\begin{figure}[ht]
    \centering
    \includegraphics[width=\linewidth]{figures/overview-yuanhao.pdf}
    \caption{An overview of GFM-based RS. Compared with GNN-based or LLM-based RS, GFM-based RS are positioned as integrating both approaches to create more comprehensive recommendations.}
    \label{fig:overview}
\end{figure}
With the rapid development of graph learning, GNN-based methods have emerged as an important technology in RS, which can further enhance the collaborative signals of collaborative filtering and extend the signals to higher-order structures and external knowledge~\cite{wu2022graph}. However, due to the inherent structural bias, they struggle to handle textual information.
This is where the powerful capabilities of large language models, which have made significant impacts in the field of natural language processing (NLP) and come into play in the realm of RS~\cite{yang2023palr,zhai2024actions}. Leveraging the advanced text capabilities of LLM, these methods efficiently capture user and item textual information while integrating world knowledge for improved recommendations. However, their reasoning limitations restrict the collaborative signals they can comprehend.
Inspired by the success of LLM in the NLP field, the graph domain has also been undergoing transformation, leading to the emergence of graph foundation models (GFMs)~\cite{liu2023towards}. By integrating GNN and LLM technologies, GFM-based RS can efficiently utilize data to align user preferences and make more precise recommendations with minimized bias, as depicted in Figure~\ref{fig:overview}. By appropriately integrating key information from both graph structures and text, GFM-based RS hold significant potential to emerge as a new paradigm in RS.
% \section{Taxonomy}

% As illustrated by Fig. \ref{}, the typical process of vision models based time series analysis has five components: (1) normalization/scaling; (2) time series to image transformation; (3) image modeling; (4) image to time series recovery; and (5) task processing. In the rest of this paper, we will discuss the typical methods for each of these components. The detailed taxonomy of the methods are summarized in Table \ref{tab.taxonomy}.

%Typical step: normalization/scaling, transformation, vision modeling, task-specific head, inverse transformation (for tasks that output time series, e.g., forecasting, generation, imputation, anomaly detection). Normalization is to fit the arbitrary range of time series values to RGB representation.

\begin{figure*}[!t]
\centering
\includegraphics[width=1.0\textwidth]{fig/fig_3.pdf}
% \vspace{-1em}
\caption{An illustration of different methods for imaging time series with a sample (length=336) from the \textit{Electricity} benchmark dataset \protect\cite{nie2023time}. (a)(c)(d)(e)(f) %are univariate methods.
visualize the same variate. (b) visualizes all 321 variates. Filterbank is omitted due to its %high
similarity to STFT.}\label{fig.tsimage}
\vspace{-0.2cm}
\end{figure*}

\begin{table*}[t]
\centering
\scriptsize
\setlength{\tabcolsep}{2.7pt}{
% \begin{tabular}{llllllllllll}
\begin{tabular}{llcccccccccl}
\toprule[1pt]
\multirow{2}{*}{Method} & \multirow{2}{*}{TS-Type} & \multirow{2}{*}{Imaging} & \multicolumn{5}{c}{Imaged Time Series Modeling} & \multirow{2}{*}{TS-Recover} & \multirow{2}{*}{Task} & \multirow{2}{*}{Domain} & \multirow{2}{*}{Code}\\ \cmidrule{4-8}
 & & & Multi-modal & Model & Pre-trained & Fine-tune & Prompt & & & & \\ \midrule
\cite{silva2013time} & UTS & RP & \xmark & \texttt{K-NN} & \xmark & \xmark & \xmark & \xmark & Classification & General & \xmark\\
\cite{wang2015encoding} & UTS & GAF & \xmark & \texttt{CNN} & \xmark & \cmark$^{\flat}$ & \xmark & \cmark & Classification & General & \xmark\\
\cite{wang2015imaging} & UTS & GAF & \xmark & \texttt{CNN} & \xmark & \cmark$^{\flat}$ & \xmark & \cmark & Multiple & General & \xmark\\
% \multirow{2}{*}{\cite{wang2015imaging}} & \multirow{2}{*}{UTS} & \multirow{2}{*}{GAF} & \multirow{2}{*}{\xmark} & \multirow{2}{*}{\texttt{CNN}} & \multirow{2}{*}{\xmark} & \multirow{2}{*}{\cmark$^{\flat}$} & \multirow{2}{*}{\xmark} & \multirow{2}{*}{\cmark} & Classification & \multirow{2}{*}{General} & \multirow{2}{*}{\xmark}\\
% & & & & & & & & & \& Imputation & & \\
\cite{ma2017learning} & MTS & Heatmap & \xmark & \texttt{CNN} & \xmark & \cmark$^{\flat}$ & \xmark & \cmark & Forecasting & Traffic & \xmark\\
\cite{hatami2018classification} & UTS & RP & \xmark & \texttt{CNN} & \xmark & \cmark$^{\flat}$ & \xmark & \xmark & Classification & General & \xmark\\
\cite{yazdanbakhsh2019multivariate} & MTS & Heatmap & \xmark & \texttt{CNN} & \xmark & \cmark$^{\flat}$ & \xmark & \xmark & Classification & General & \cmark\textsuperscript{\href{https://github.com/SonbolYb/multivariate_timeseries_dilated_conv}{[1]}}\\
MSCRED \cite{zhang2019deep} & MTS & Other ($\S$\ref{sec.othermethod}) & \xmark & \texttt{ConvLSTM} & \xmark & \cmark$^{\flat}$ & \xmark & \xmark & Anomaly & General & \cmark\textsuperscript{\href{https://github.com/7fantasysz/MSCRED}{[2]}}\\
\cite{li2020forecasting} & UTS & RP & \xmark & \texttt{CNN} & \cmark & \cmark & \xmark & \xmark & Forecasting & General & \cmark\textsuperscript{\href{https://github.com/lixixibj/forecasting-with-time-series-imaging}{[3]}}\\
\cite{cohen2020trading} & UTS & LinePlot & \xmark & \texttt{Ensemble} & \xmark & \cmark$^{\flat}$ & \xmark & \xmark & Classification & Finance & \xmark\\
% \cite{du2020image} & UTS & Spectrogram & \xmark & \texttt{CNN} & \xmark & \cmark$^{\flat}$ & \xmark & \xmark & Classification & Finance & \xmark\\
\cite{barra2020deep} & UTS & GAF & \xmark & \texttt{CNN} & \xmark & \cmark$^{\flat}$ & \xmark & \xmark & Classification & Finance & \xmark\\
% \cite{barra2020deep} & UTS & GAF & \xmark & \texttt{VGG-16} & \xmark & \cmark$^{\flat}$ & \xmark & \xmark & Classification & Finance & \xmark\\
% \cite{cao2021image} & UTS & RP & \xmark & \texttt{CNN} & \xmark & \cmark$^{\flat}$ & \xmark & \xmark & Classification & General & \xmark\\
VisualAE \cite{sood2021visual} & UTS & LinePlot & \xmark & \texttt{CNN} & \xmark & \cmark$^{\flat}$ & \xmark & \cmark & Forecasting & Finance & \xmark\\
% VisualAE \cite{sood2021visual} & UTS & LinePlot & \xmark & \texttt{CNN} & \xmark & \cmark$^{\flat}$ & \xmark & \xmark & Img-Generation & Finance & \xmark\\
\cite{zeng2021deep} & MTS & Heatmap & \xmark & \texttt{CNN,LSTM} & \xmark & \cmark$^{\flat}$ & \xmark & \cmark & Forecasting & Finance & \xmark\\
% \cite{zeng2021deep} & MTS & Heatmap & \xmark & \texttt{SRVP} & \xmark & \cmark$^{\flat}$ & \xmark & \cmark & Forecasting & Finance & \xmark\\
AST \cite{gong2021ast} & UTS & Spectrogram & \xmark & \texttt{DeiT} & \cmark & \cmark & \xmark & \xmark & Classification & Audio & \cmark\textsuperscript{\href{https://github.com/YuanGongND/ast}{[4]}}\\
TTS-GAN \cite{li2022tts} & MTS & Heatmap & \xmark & \texttt{ViT} & \xmark & \cmark$^{\flat}$ & \xmark & \cmark & Ts-Generation & Health & \cmark\textsuperscript{\href{https://github.com/imics-lab/tts-gan}{[5]}}\\
SSAST \cite{gong2022ssast} & UTS & Spectrogram & \xmark & \texttt{ViT} & \cmark$^{\natural}$ & \cmark & \xmark & \xmark & Classification & Audio & \cmark\textsuperscript{\href{https://github.com/YuanGongND/ssast}{[6]}}\\
MAE-AST \cite{baade2022mae} & UTS & Spectrogram & \xmark & \texttt{MAE} & \cmark$^{\natural}$ & \cmark & \xmark & \xmark & Classification & Audio & \cmark\textsuperscript{\href{https://github.com/AlanBaade/MAE-AST-Public}{[7]}}\\
AST-SED \cite{li2023ast} & UTS & Spectrogram & \xmark & \texttt{SSAST,GRU} & \cmark & \cmark & \xmark & \xmark & EventDetection & Audio & \xmark\\
\cite{jin2023classification} & UTS & %Multiple
LinePlot & \xmark & \texttt{CNN} & \cmark & \cmark & \xmark & \xmark & Classification & Physics & \xmark\\
ForCNN \cite{semenoglou2023image} & UTS & LinePlot & \xmark & \texttt{CNN} & \xmark & \cmark$^{\flat}$ & \xmark & \xmark & Forecasting & General & \xmark\\
Vit-num-spec \cite{zeng2023pixels} & UTS & Spectrogram & \xmark & \texttt{ViT} & \xmark & \cmark$^{\flat}$ & \xmark & \xmark & Forecasting & Finance & \xmark\\
% \cite{wimmer2023leveraging} & MTS & LinePlot & \xmark & \texttt{CLIP,LSTM} & \cmark & \cmark & \xmark & \xmark & Classification & Finance & \xmark\\
ViTST \cite{li2023time} & MTS & LinePlot & \xmark & \texttt{Swin} & \cmark & \cmark & \xmark & \xmark & Classification & General & \cmark\textsuperscript{\href{https://github.com/Leezekun/ViTST}{[8]}}\\
MV-DTSA \cite{yang2023your} & UTS\textsuperscript{*} & LinePlot & \xmark & \texttt{CNN} & \xmark & \cmark$^{\flat}$ & \xmark & \cmark & Forecasting & General & \cmark\textsuperscript{\href{https://github.com/IkeYang/machine-vision-assisted-deep-time-series-analysis-MV-DTSA-}{[9]}}\\
TimesNet \cite{wu2023timesnet} & MTS & Heatmap & \xmark & \texttt{CNN} & \xmark & \cmark$^{\flat}$ & \xmark & \cmark & Multiple & General & \cmark\textsuperscript{\href{https://github.com/thuml/TimesNet}{[10]}}\\
ITF-TAD \cite{namura2024training} & UTS & Spectrogram & \xmark & \texttt{CNN} & \cmark & \xmark & \xmark & \xmark & Anomaly & General & \xmark\\
\cite{kaewrakmuk2024multi} & UTS & GAF & \xmark & \texttt{CNN} & \cmark & \cmark & \xmark & \xmark & Classification & Sensing & \xmark\\
HCR-AdaAD \cite{lin2024hierarchical} & MTS & RP & \xmark & \texttt{CNN,GNN} & \xmark & \cmark$^{\flat}$ & \xmark & \xmark & Anomaly & General & \xmark\\
FIRTS \cite{costa2024fusion} & UTS & Other ($\S$\ref{sec.othermethod}) & \xmark & \texttt{CNN} & \xmark & \cmark$^{\flat}$ & \xmark & \xmark & Classification & General & \cmark\textsuperscript{\href{https://sites.google.com/view/firts-paper}{[11]}}\\
% \multirow{2}{*}{FIRTS \cite{costa2024fusion}} & \multirow{2}{*}{UTS} & Spectrogram & \multirow{2}{*}{\xmark} & \multirow{2}{*}{\texttt{CNN}} & \multirow{2}{*}{\xmark} & \multirow{2}{*}{\cmark$^{\flat}$} & \multirow{2}{*}{\xmark} & \multirow{2}{*}{\xmark} & \multirow{2}{*}{Classification} & \multirow{2}{*}{General} & \multirow{2}{*}{\cmark\textsuperscript{\href{https://sites.google.com/view/firts-paper}{[2]}}}\\
%  & & \& GAF,RP,MTF & & & & & & & & & \\
% \cite{homenda2024time} & UTS\textsuperscript{*} & Multiple & \xmark & \texttt{CNN} & \xmark & \cmark$^{\flat}$ & \xmark & \xmark & Classification & General & \xmark\\
CAFO \cite{kim2024cafo} & MTS & RP & \xmark & \texttt{CNN,ViT} & \xmark & \cmark$^{\flat}$ & \xmark & \xmark & Explanation & General & \cmark\textsuperscript{\href{https://github.com/eai-lab/CAFO}{[12]}}\\
% \multirow{2}{*}{CAFO \cite{kim2024cafo}} & \multirow{2}{*}{MTS} & \multirow{2}{*}{RP} & \multirow{2}{*}{\xmark} & \texttt{ShuffleNet,ResNet} & \multirow{2}{*}{\cmark} & \multirow{2}{*}{\cmark} & \multirow{2}{*}{\xmark} & \multirow{2}{*}{\xmark} & Classification & \multirow{2}{*}{General} & \multirow{2}{*}{\cmark}\\
%  & & & & \texttt{MLP-Mixer,ViT} & & & & & \& Explanation & & \\
ViTime \cite{yang2024vitime} & UTS\textsuperscript{*} & LinePlot & \xmark & \texttt{ViT} & \cmark$^{\natural}$ & \cmark & \xmark & \cmark & Forecasting & General & \cmark\textsuperscript{\href{https://github.com/IkeYang/ViTime}{[13]}}\\
ImagenTime \cite{naiman2024utilizing} & MTS & Other ($\S$\ref{sec.othermethod}) & \xmark & %\texttt{Diffusion}
\texttt{CNN} & \xmark & \cmark$^{\flat}$ & \xmark & \cmark & Ts-Generation & General & \cmark\textsuperscript{\href{https://github.com/azencot-group/ImagenTime}{[14]}}\\
TimEHR \cite{karami2024timehr} & MTS & Heatmap & \xmark & \texttt{CNN} & \xmark & \cmark$^{\flat}$ & \xmark & \cmark & Ts-Generation & Health & \cmark\textsuperscript{\href{https://github.com/esl-epfl/TimEHR}{[15]}}\\
VisionTS \cite{chen2024visionts} & UTS\textsuperscript{*} & Heatmap & \xmark & \texttt{MAE} & \cmark & \cmark & \xmark & \cmark & Forecasting & General & \cmark\textsuperscript{\href{https://github.com/Keytoyze/VisionTS}{[16]}}\\ \midrule
InsightMiner \cite{zhang2023insight} & UTS & LinePlot & \cmark & \texttt{LLaVA} & \cmark & \cmark & \cmark & \xmark & Txt-Generation & General & \xmark\\
\cite{wimmer2023leveraging} & MTS & LinePlot & \cmark & \texttt{CLIP,LSTM} & \cmark & \cmark & \xmark & \xmark & Classification & Finance & \xmark\\
% \cite{dixit2024vision} & UTS & Spectrogram & \cmark & \texttt{GPT4o,Gemini} & \cmark & \xmark & \cmark & \xmark & Classification & Audio & \xmark\\
\multirow{2}{*}{\cite{dixit2024vision}} & \multirow{2}{*}{UTS} & \multirow{2}{*}{Spectrogram} & \multirow{2}{*}{\cmark} & \texttt{GPT4o,Gemini} & \multirow{2}{*}{\cmark} & \multirow{2}{*}{\xmark} & \multirow{2}{*}{\cmark} & \multirow{2}{*}{\xmark} & \multirow{2}{*}{Classification} & \multirow{2}{*}{Audio} & \multirow{2}{*}{\xmark}\\
 & & & & \& \texttt{Claude3} & & & & & & & \\
\cite{daswani2024plots} & MTS & LinePlot & \cmark & \texttt{GPT4o,Gemini} & \cmark & \xmark & \cmark & \xmark & Multiple & General & \xmark\\
TAMA \cite{zhuang2024see} & UTS & LinePlot & \cmark & \texttt{GPT4o} & \cmark & \xmark & \cmark & \xmark & Anomaly & General & \xmark\\
\cite{prithyani2024feasibility} & MTS & LinePlot & \cmark & \texttt{LLaVA} & \cmark & \cmark & \cmark & \xmark & Classification & General & \cmark\textsuperscript{\href{https://github.com/vinayp17/VLM_TSC}{[17]}}\\
\bottomrule[1pt]
\end{tabular}}
\vspace{-0.25cm}
\caption{Taxonomy of vision models on time series. The top panel includes single-modal models. The bottom panel includes multi-modal models. {\bf TS-Type} denotes type of time series. {\bf TS-Recover} denotes %whether time series recovery ($\S$\ref{sec.processing}) has been performed.
recovering time series from predicted images ($\S$\ref{sec.processing}). \textsuperscript{*}: %the model has been %applied on MTSs by %processing %modeling the individual UTSs of each MTS.
the method has been used to model the individual UTSs of an MTS. $^{\natural}$: a new pre-trained model was proposed in the work. $^{\flat}$: %without using a pre-trained model, fine-tune means training from scratch.
when pre-trained models were unused, ``Fine-tune'' refers to train a task-specific model from scratch. %In the
{\bf Model} column: \texttt{CNN} could be regular CNN, ResNet, VGG-Net, %U-Net,
{\em etc.}}\label{tab.taxonomy}
% The code only include verified official code from the authors.
\vspace{-0.3cm}
\end{table*}

\begin{table*}[t]
\centering
\small
\setlength{\tabcolsep}{2.9pt}{
\begin{tabular}{l|l|l|l}\hline
% \toprule[1pt]
\rowcolor{gray!20}
{\bf Method} & {\bf TS-Type} & {\bf Advantages} & {\bf Limitations}\\ \hline
Line Plot ($\S$\ref{sec.lineplot}) & UTS, MTS & matches human perception of time series & limited to MTSs with a small number of variates\\ \hline
Heatmap ($\S$\ref{sec.heatmap}) & UTS, MTS & straightforward for both UTSs and MTSs & the order of variates may affect their correlation learning\\ \hline
Spectrogram ($\S$\ref{sec.spectrogram}) & UTS & encodes the time-frequency space & limited to UTSs; needs a proper choice of window/wavelet\\ \hline
GAF ($\S$\ref{sec.gaf}) & UTS & encodes the temporal correlations in a UTS & limited to UTSs; $O(T^{2})$ time and space complexity\\ \hline% for long time series\\ \hline
% RP ($\S$\ref{sec.rp}) & UTS & flexibility in image size by tuning $m$ and $\tau$ & limited to UTSs; the pattern has a threshold-dependency\\ \hline
RP ($\S$\ref{sec.rp}) & UTS & flexibility in image size by tuning $m$ and $\tau$ & limited to UTSs; information loss after thresholding\\ \hline
% \bottomrule[1pt]
\end{tabular}}
\vspace{-0.2cm}
\caption{Summary of the five primary methods for transforming time series to images. {\bf TS-Type} denotes type of time series.}\label{tab.tsimage}
\vspace{-0.2cm}
\end{table*}

\section{Time Series To Image Transformation}\label{sec.tsimage}

% This section summarizes 5 major methods for imaging time series ($\S$\ref{sec.lineplot}-$\S$\ref{sec.rp}). We also discuss some other methods ($\S$\ref{sec.othermethod}) and how to model MTS with these methods ($\S$\ref{sec.modelmts}).
This section summarizes the methods for imaging time series ($\S$\ref{sec.lineplot}-$\S$\ref{sec.othermethod}) and their extensions to encode MTSs ($\S$\ref{sec.modelmts}).

% This section summarizes 5 major methods for transforming time series to images, including Line Plot, Heatmap, Spetrogram, GAF and RP, and several minor methods. We discuss their pros and cons and how to deal with MTS.

% This section discusses the advantages and limitations of different methods for time series to image transformation (invertible, efficiency, information preservation, MTS, long-range time series, parametric, etc.).

%\subsection{Methods}

\vspace{-0.08cm}

\subsection{Line Plot}\label{sec.lineplot}

Line Plot is a straightforward way for visualizing UTSs for human analysis ({\em e.g.}, stocks, power consumption, {\em etc.}). As illustrated by Fig. \ref{fig.tsimage}(a), the simplest approach is to draw a 2D image with x-axis representing %the time horizon
time steps and y-axis representing %the magnitude of the normalized time series.
time-wise values, %A line is used to connect all values of the series over time.
with a line connecting all values of the series over time. This image can be %represented by either three-channel pixels or single-channel pixels
either three-channel ({\em i.e.}, RGB) or single-channel as the colors may not %provide additional information
be informative %\cite{cohen2020trading,sood2021visual,jin2023classification,zhang2023insight,zhuang2024see}.
\cite{cohen2020trading,sood2021visual,jin2023classification,zhang2023insight}. ForCNN \cite{semenoglou2023image} even uses a single 8-bit integer to represent each pixel for black-white images. So far, there is no consensus on whether other graphical components, such as legend, grids and tick labels, could provide extra benefits in any task. For example, ViTST \cite{li2023time} finds these components are superfluous in a classification task, while TAMA \cite{zhuang2024see} finds grid-like auxiliary lines help enhance anomaly detection.

In addition to the regular Line Plot, MV-DTSA \cite{yang2023your} and ViTime \cite{yang2024vitime} divide an image into $h\times L$ grids, %where $h$ is the number of rows and $L$ is the number of columns,
and %introduced
define a function to map each time step of a UTS to a grid, producing a grid-like Line Plot. Also, we include methods that use Scatter Plot \cite{daswani2024plots,prithyani2024feasibility} in this category because %the only difference between a Scatter Plot and a Line Plot is whether the time-wise values are connected by lines.
a Scatter Plot resembles a Line Plot but doesn't connect %time-wise values
data points with a line. By comparing them, \cite{prithyani2024feasibility} finds a Line Plot could induce better time series classification.

For MTSs, we defer the discussion on Line Plot to $\S$\ref{sec.modelmts}.

% For MTS, some methods use the channel-independence assumption proposed in \cite{nie2023time} and represent each variate in MTS with an individual Line Plot \cite{yang2023your,yang2024vitime}. ViTST \cite{li2023time} also uses an individual Line Plot per variate, but colors different lines and assembles all plots to form a bigger image. The method in \cite{wimmer2023leveraging} plots %the time series of
% all variates in a single Line Plot and distinguish them by %use different
% types of lines ({\em e.g.}, solid, dashed, dotted, {\em etc.}). %to distinguish them.
% However, these methods only work for a small number of variates. For example, in \cite{wimmer2023leveraging}, there are only 4 variates in its financial MTSs.

%\cite{li2023time} space-costly because of blank pixels. scatter plot.

%Invertible with a numeric prediction head \cite{sood2021visual}. It fits tasks such as forecasting, imputation, etc.

\vspace{-0.08cm}

\subsection{Heatmap}\label{sec.heatmap}

As shown in Fig. \ref{fig.tsimage}(b), Heatmap is a 2D visualization of the magnitude of the values in a matrix using color. %The variation of color represents the intensity of each value. %Therefore,
It has been used to %directly
represent the matrix of an MTS, {\em i.e.}, $\mat{X} \in \mathbb{R}^{d\times T}$, as a one-channel $d\times T$ image \cite{li2022tts,yazdanbakhsh2019multivariate}. Similarly, TimEHR \cite{karami2024timehr} represents an {\em irregular} MTS, where the intervals between time steps are uneven, as a $d\times H$ Heatmap image by grouping the uneven time steps into $H$ even time bins. In \cite{zeng2021deep}, a different method is used for visualizing a 9-variate financial %time series.
MTS. It reshapes the 9 variates at each time step to a $3\times 3$ Heatmap image, and uses the sequence of images to forecast future %image
frames, achieving %time series
%MTS
time series forecasting. In contrast, VisionTS \cite{chen2024visionts} uses Heatmap to visualize UTSs. %instead.
Similar to TimesNet \cite{wu2023timesnet}, it first segments a length-$T$ UTS into $\lfloor T/P\rfloor$ length-$P$ subsequences, where $P$ is a parameter representing a periodicity of the UTS. Then the subsequences are stacked into a $P\times \lfloor T/P\rfloor$ matrix, %and duplicated 3 times to produce a 3-channel
with 3 duplicated channels, to produce a grayscale image %which serves as an
input to %a vision foundation model.
an LVM. To encode MTSs, VisionTS adopts the channel independence assumption \cite{nie2023time} and individually models each variate in an MTS.

\vspace{0.2cm}

\noindent{\bf Remark.} Heatmap can be used to visualize matrices of various forms. It is also used for matrices generated by the subsequent methods ({\em e.g.}, Spectrogram, GAF, RP) in this section. In this paper, the name Heatmap refers specifically to images that use color to visualize the (normalized) values in UTS $\mat{x}$ or MTS $\mat{X}$ without performing other transformations.

%\cite{chen2024visionts,karami2024timehr} bin version of TSH \cite{karami2024timehr}, DE and STFT \cite{naiman2024utilizing} (DE can be used for constructing RP), rearrange variates for video version of TSH \cite{zeng2021deep}.

%\vspace{0.2cm}

\subsection{Spectrogram}\label{sec.spectrogram}

A {\em spectrogram} is a visual representation of the spectrum of frequencies of a signal as it varies with time, which are extensively used for analyzing audio signals \cite{gong2021ast}. Since audio signals are a type of UTS, spectrogram can be considered as a method for imaging a UTS. As shown in Fig. \ref{fig.tsimage}(c), a common format is a 2D heatmap image with x-axis representing time steps and y-axis representing frequency, {\em a.k.a.} a time-frequency space. %The color at each point
Each pixel in the image represents the (logarithmic) amplitude of a specific frequency at a specific time point. Typical methods for %transforming a UTS to
producing a spectrogram include {\bf Short-Time Fourier Transform (STFT)} \cite{griffin1984signal}, {\bf Wavelet Transform} \cite{daubechies1990wavelet}, and {\bf Filterbank} \cite{vetterli1992wavelets}.

\vspace{0.2cm}

\noindent{\bf STFT.} %Discrete Fourier transform (DFT) can be used to represent a UTS signal %$\mat{x}=[x_{1}, ..., x_{T}]$
%$\mat{x}\in\mathbb{R}^{1\times T}$ as a sum of sinusoidal components. The output of the transform is a function of frequency $f(w)$, describing the intensity of each constituent frequency $w$ of the entire UTS. 
Discrete Fourier transform (DFT) can be used to describe the intensity $f(w)$ of each constituent frequency $w$ of a UTS signal $\mat{x}\in\mathbb{R}^{1\times T}$. However, $f(w)$ has no time dependency. It cannot provide dynamic information such as when a specific frequency appear in the UTS. STFT addresses this deficiency by sliding a window function $g(t)$ over the time steps in %the UTS,
$\mat{x}$, and computing the DFT within each window by
\begin{equation}\label{eq.stft}
\small
\begin{aligned}
f(w,\tau) = \sum_{t=1}^{T}x_{t}g(t - \tau)e^{-iwt}
\end{aligned}
\end{equation}
where $w$ is frequency, $\tau$ is the position of the window, $f(w,\tau)$ describes the intensity of frequency $w$ at time step $\tau$.

%With a proper selection of the
By selecting a proper window function $g(\cdot)$ ({\em e.g.}, Gaussian/Hamming/Bartlett window), %({\em e.g.}, Gaussian window, Hamming window, Bartlett window), %{\em etc.}),
a 2D spectrogram ({\em e.g.}, Fig. \ref{fig.tsimage}(c)) can be drawn via a heatmap on the squared values $|f(w,\tau)|^{2}$, with $w$ as the y-axis, and $\tau$ as the x-axis. For example, \cite{dixit2024vision} uses STFT based spectrogram as an input to LMMs %\hh{do you mean LVMs? check}
for time series classification.

%Fourier transform is a powerful data analysis tool that represents any complex signal as a sum of sines and cosines and transforms the signal from the time domain to the frequency domain. However, Fourier transform can only show which frequencies are present in the signal, but not when these frequencies appear. The STFT divides original signal into several parts using a sliding window to fix this problem. STFT involves a sliding window for extracting frequency components within the window.

\vspace{0.2cm}

\noindent{\bf Wavelet Transform.} %Like Fourier transform, %\hh{this paragraph needs a citation}
Continuous Wavelet Transform (CWT) uses the inner product to measure the similarity between a signal function $x(t)$ and an analyzing function. %In STFT (Eq.~\eqref{eq.stft}), the analyzing function is a windowed exponential $g(t - \tau)e^{-iwt}$.
%In CWT,
The analyzing function is a {\em wavelet} $\psi(t)$, where the typical choices include Morse wavelet, Morlet wavelet, %Daubechies wavelet, %Beylkin wavelet, 
{\em etc.} %The
CWT compares $x(t)$ to the shifted and scaled ({\em i.e.}, stretched or shrunk) versions of the wavelet, and output a CWT coefficient by
\begin{equation}\label{eq.cwt}
\small
\begin{aligned}
c(s,\tau) = \int_{-\infty}^{\infty}x(t)\frac{1}{s}\psi^{*}(\frac{t - \tau}{s})dt
\end{aligned}
\end{equation}
where $*$ denotes complex conjugate, $\tau$ is the time step to shift, and $s$ represents the scale. In practice, a discretized version of CWT in Eq.~\eqref{eq.cwt} is implemented for UTS $[x_{1}, ..., x_{T}]$.

It is noteworthy that the scale $s$ controls the frequency encoded in a wavelet -- a larger $s$ leads to a stretched wavelet with a lower frequency, and vice versa. As such, by varying $s$ and $\tau$, a 2D spectrogram ({\em e.g.}, Fig. \ref{fig.tsimage}(d)) can be drawn %, often with a heatmap
on $|c(s,\tau)|$, where $s$ is the y-axis and $\tau$ is the x-axis. Compared to STFT, which uses a fixed window size, Wavelet Transform allows variable wavelet sizes -- a larger size %region
for more precise low frequency information. 
%Usually, $s$ and $\tau$ vary dependently -- a larger $s$ leads to a stretched wavelet that shifts slowly, {\em i.e.}, a smaller $\tau$. This property %of CWT
%yields a spectrogram that balances the resolutions of frequency %$s$
%and time, %$\tau$,
%which is an advantage over the fixed time resolution in STFT.
% Thus, both of the methods in %\cite{du2020image}
% \cite{namura2024training} and \cite{zeng2023pixels} choose CWT (with Morlet wavelet) to generate the spectrogram.
Thus, the methods in \cite{du2020image,namura2024training,zeng2023pixels} choose CWT (with Morlet wavelet) to generate the spectrogram.

%A wavelet is a wave-like oscillation that has zero mean and is localized in both time and frequency space.

\vspace{0.2cm}

\noindent{\bf Filterbank.} This method %is relevant to
resembles STFT and is often used in processing audio signals. Given an audio signal, it firstly goes through a {\em pre-emphasis filter} to boost high frequencies, which helps improve the clarity of the signal. Then, STFT is applied on the signal. %with a sliding window $g(t)$ of size $k$ that shifts in a fixed stride $\tau$. %where the adjacent windows may overlap in $k$ time length.
%Finally, filterbank features are computed by applying multiple ``triangle-shaped'' filters spaced on the Mel-scale to the STFT output $f(w, \tau)$. %where Mel-scale is a method to make the filters more discriminative on lower frequencies, %than higher frequencies,
%imitating the non-linear human ear perception of sound.
Finally, multiple ``triangle-shaped'' filters spaced on a Mel-scale are applied to the STFT power spectrum $|f(w, \tau)|^{2}$ to extract frequency bands. The outcome filterbank features $\hat{f}(w, \tau)$ can be used to yield a spectrogram with $w$ as the y-axis, and $\tau$ as the x-axis.

%Filterbank was introduced in AST \cite{gong2021ast} with %$k$=25ms
Filterbank was adopted in AST \cite{gong2021ast} with 
a 25ms Hamming window that shifts every 10ms for classifying audio signals using Vision Transformer (ViT). It then becomes widely used in the follow-up works such as SSAST \cite{gong2022ssast}, MAE-AST \cite{baade2022mae}, and AST-SED \cite{li2023ast}, as summarized in Table \ref{tab.taxonomy}.



%Use MLP to predict TS directly \cite{zeng2023pixels}.

%\vspace{0.2cm}

% \vspace{0.2cm}

\subsection{Gramian Angular Field (GAF)}\label{sec.gaf}

GAF was introduced for classifying UTSs using CNNs %using %image based CNNs
by \cite{wang2015encoding}. It was then extended %with an extension
to an imputation task in \cite{wang2015imaging}. Similarly, \cite{barra2020deep} applied GAF for financial time series forecasting.

Given a UTS $\mat{x}\in\mathbb{R}^{1\times T}$, %$[x_{1}, ..., x_{T}]$,
the first step %before GAF
is to rescale each $x_{t}$ to a value $\tilde{x}_{t}$ %in the interval of
within $[0, 1]$ (or $[-1, 1]$). %by min-max normalization.
This range enables mapping $\tilde{x}_{t}$ to polar coordinates by $\phi_{t}=\text{arccos}(\tilde{x}_{i})$, with a radius $r=t/N$ encoding the time stamp, where $N$ is a constant factor to regularize the span of the polar coordinates. %system. Then,
Two types of GAF, Gramian Sum Angular Field (GASF) and Gramian Difference Angular Field (GADF) are defined as
\begin{equation}\label{eq.gaf}
\small
\begin{aligned}
&\text{GASF:}~~\text{cos}(\phi_{t} + \phi_{t'})=x_{t}x_{t'} - \sqrt{1 - x_{t}^{2}}\sqrt{1 - x_{t'}^{2}}\\
&\text{GADF:}~~\text{sin}(\phi_{t} - \phi_{t'})=x_{t'}\sqrt{1 - x_{t}^{2}} - x_{t}\sqrt{1 - x_{t'}^{2}}
\end{aligned}
\end{equation}
which exploits the pairwise temporal correlations in the UTS. Thus, the outcome is a $T\times T$ matrix $\mat{G}$ with $\mat{G}_{t,t'}$ specified by either type in Eq.~\eqref{eq.gaf}. A GAF image is a heatmap on $\mat{G}$ with both axes representing time, as illustrated by Fig. \ref{fig.tsimage}(e).

% Invertible.

% \vspace{0.2cm}

\subsection{Recurrence Plot (RP)}\label{sec.rp}

%RP \cite{eckmann1987recurrence} is a method to encode a UTS into an image that aims to capture the periodic patterns in the UTS by using its reconstructed {\em phase space}. The phase space of a UTS $[x_{1}, ..., x_{T}]$ can be reconstructed by {\em time delay embedding}, which is a set of new vectors $\mat{v}_{1}$, ..., $\mat{v}_{l}$ with

RP \cite{eckmann1987recurrence} encodes a UTS into an image that captures its periodic patterns by using its reconstructed {\em phase space}. The phase space of %a UTS %$[x_{1}, ..., x_{T}]$
$\mat{x}\in\mathbb{R}^{1\times T}$ can be reconstructed by {\em time delay embedding} -- a set of new vectors $\mat{v}_{1}$, ..., $\mat{v}_{l}$ with
\begin{equation}\label{eq.de}
\small
\begin{aligned}
\mat{v}_{t}=[x_{t}, x_{t+\tau}, x_{t+2\tau}, ..., x_{t+(m-1)\tau}]\in\mathbb{R}^{m\tau},~~~1\le t \le l
\end{aligned}
\end{equation}
where $\tau$ is the time delay, $m$ is the dimension of the phase space, both %of which
are hyperparameters. Hence, $l=T-(m-1)\tau$. With vectors $\mat{v}_{1}$, ..., $\mat{v}_{l}$, an RP image %is constructed by measuring
measures their pairwise distances, results in an $l\times l$ image whose element
\begin{equation}\label{eq.rp}
\small
\begin{aligned}
\text{RP}_{i,j}=\Theta(\varepsilon - \|\mat{v}_{i} - \mat{v}_{j}\|),~~~1\le i,j\le l
\end{aligned}
\end{equation}
where $\Theta(\cdot)$ is the Heaviside step function, $\varepsilon$ is a threshold, and $\|\cdot\|$ is a norm function such as $\ell_{2}$ norm. Eq.~\eqref{eq.rp} %states RP produces a heatmap image on a binary matrix with $\text{RP}_{i,j}=1$ if $\mat{v}_{i}$ and $\mat{v}_{j}$ are sufficiently similar.
generates a binary matrix with $\text{RP}_{i,j}=1$ if $\mat{v}_{i}$ and $\mat{v}_{j}$ are sufficiently similar, producing a black-white image ({\em e.g.}, Fig. \ref{fig.tsimage}(f)).% ({\em e.g.}, a periodic pattern).

An advantage of RP is its flexibility in image size by tuning $m$ and $\tau$. Thus it has been used for time series classification %\cite{cao2021image},
\cite{silva2013time,hatami2018classification}, forecasting \cite{li2020forecasting}, anomaly detection \cite{lin2024hierarchical} and %feature-wise
explanation \cite{kim2024cafo}. Moreover, the method in \cite{hatami2018classification}, and similarly in HCR-AdaAD \cite{lin2024hierarchical}, omit the thresholding in Eq.~\eqref{eq.rp} and uses $\|\mat{v}_{i} - \mat{v}_{j}\|$ to produce continuously valued images %in a classification task
to avoid information loss.


% \vspace{0.2cm}

\subsection{Other Methods}\label{sec.othermethod}

%There are some less commonly used methods. For example, in
Additionally, %there are some peripheral methods. %In addition to GAF,
\cite{wang2015encoding} introduces Markov Transition Field (MTF) for imaging a UTS. %$\mat{x}\in\mathbb{R}^{1\times T}$. 
%MTF first assigns each $x_{t}$ to one of $Q$ quantile bins, then builds a $Q\times Q$ Markov transition matrix $\mat{M}$ {\em s.t.} $\mat{M}_{i,j}$ represents the frequency %with which
%of the case when a point $x_{t}$ in the $i$-th bin is followed by a point $x_{t'}$ in the $j$-th bin, {\em i.e.}, $t=t'+1$. Matrix $\mat{M}$ serves as the input of a heatmap image.
MTF is a matrix $\mat{M}\in\mathbb{R}^{Q\times Q}$ encoding the transition probabilities over time segments, where $Q$ is the number of segments. %Moreover,
ImagenTime \cite{naiman2024utilizing} stacks the delay embeddings $\mat{v}_{1}$, ..., $\mat{v}_{l}$ in Eq.~\eqref{eq.de} to an $l\times m\tau$ matrix for visualizing UTSs. %It also uses a variant of STFT.
% The method in \cite{homenda2024time} introduces five different 2D images by counting, rearranging, replicating the values in a UTS. 
MSCRED \cite{zhang2019deep} uses heatmaps on the $d\times d$ correlation matrices of MTSs with $d$ variates for anomaly detection. 
Furthermore, some methods use a mixture of imaging methods by stacking different transformations. \cite{wang2015imaging} stacks GASF, GADF, MTF to a 3-channel image. %Similarly,
FIRTS \cite{costa2024fusion} builds a 3-channel image by stacking GASF, MTF and RP. %the GASF, MTF, RP representations of each UTS.
%\cite{jin2023classification} combines Line Plot with Constant-Q Transform (CQT) \cite{brown1991calculation}, a method related to wavelet transform ($\S$\ref{sec.spectrogram}), to generate 2-channel images.
The mixture methods encode a UTS with multiple views and were found more robust than single-view images in these works for %time series
classification tasks.

\subsection{How to Model MTS}\label{sec.modelmts}

In the above methods, Heatmap ($\S$\ref{sec.heatmap}) can be %directly
used to visualize the %2D
variate-time matrices, $\mat{X}$, of MTSs ({\em e.g.}, Fig. \ref{fig.structure}(b)), where correlated variates %are better to
should be spatially close to each other. Line Plot ($\S$\ref{sec.lineplot}) can be used to visualize MTSs by plotting all variates in the same image \cite{wimmer2023leveraging,daswani2024plots} or combining all univariate images to compose a bigger %1-channel
image \cite {li2023time}, but these methods only work for a small number of variates. Spectrogram ($\S$\ref{sec.spectrogram}), GAF ($\S$\ref{sec.gaf}), and RP ($\S$\ref{sec.rp}) were designed specifically for UTSs. For these methods and Line Plot, which are not straightforward %for MTS transformation,
in imaging MTSs, the general approaches %to use them %for MTS
include using channel independence assumption to model each variate individually \cite{nie2023time}, %like VisionTS \cite{chen2024visionts},
or stacking the images of $d$ variates to form a $d$-channel image %as did by
\cite{naiman2024utilizing,kim2024cafo}. %\cite{prithyani2024feasibility,naiman2024utilizing,kim2024cafo}.
However, the latter does not fit some vision models pre-trained on RGB images which requires 3-channel inputs (more discussions are deferred to $\S$\ref{sec.processing}).

\vspace{0.2cm}

\noindent{\bf Remark.} As a summary, Table \ref{tab.tsimage} recaps the salient advantages and limitations of the five primary imaging methods that are introduced in this section.

% \hh{can we have a table (e.g., rows are different imaging methods and columns are a few desirable propoerties) or a short paragraph to discuss/summarize/compare the strenths and weakness of different imaging methods for ts? This might bring some structure/comprehension to this section (as opposed to, e.g., some reviewer might complain that what we do here is a laundry list)}

\section{Imaged Time Series Modeling}\label{sec.model}

With image representations, time series analysis can be readily performed with vision models. This section discusses such solutions from %traditional vision models %($\S$\ref{sec.cnns})
%to the recent large vision models %($\S$\ref{sec.lvms})
%and large multimodal models.% ($\S$\ref{sec.lmms}).
the traditional models to the SOTA models.

\begin{figure*}[!t]
\centering
\includegraphics[width=0.9\textwidth]{fig/fig_2.pdf}
% \vspace{-1em}
\caption{An illustration of different modeling strategies on imaged time series in (a)(b)(c) and task-specific heads in (d).}\label{fig.models}
\vspace{-0.2cm}
\end{figure*}

\subsection{Conventional Vision Models}\label{sec.cnns}

%Similar to
Following traditional %methods on
image classification, \cite{silva2013time} applies a K-NN classifier on the RPs of time series, \cite{cohen2020trading} applies an ensemble of fundamental classifiers such as %linear regression, SVM, Ada Boost, {\em etc.}
SVM and AdaBoost on the Line Plots %images
for time series classification. As an image encoder, %a typical encoder, %of images,
CNNs have been %extensively
widely used for learning image representations. %\cite{he2016deep}.
Different from using 1D CNNs on sequences %UTS or MTS
\cite{bai2018empirical}, %regular
2D or 3D CNNs can be applied on imaged time series as shown in Fig. \ref{fig.models}(a). %to learn time series representations by encoding their image transformations.
For example, %standard
regular CNNs have been used on Spectrograms \cite{du2020image}, tiled CNNs have been used on GAF images \cite{wang2015encoding,wang2015imaging}, dilated CNNs have been used on Heatmap images \cite{yazdanbakhsh2019multivariate}. More frequently, ResNet \cite{he2016deep}, Inception-v1 \cite{szegedy2015going}, and VGG-Net \cite{simonyan2014very} have been used on Line Plots \cite{jin2023classification,semenoglou2023image}, Heatmap images \cite{zeng2021deep}, RP images \cite{li2020forecasting,kim2024cafo}, GAF images \cite{barra2020deep,kaewrakmuk2024multi}, 
% Heatmaps \cite{zeng2021deep}, RPs \cite{li2020forecasting,kim2024cafo}, GAFs \cite{barra2020deep,kaewrakmuk2024multi},
and even a mixture of GAF, MTF and RP images \cite{costa2024fusion}. In particular, for time series generation tasks, %a diffusion model with U-Nets \cite{naiman2024utilizing} and GAN frameworks of CNNs \cite{li2022tts,karami2024timehr} have also been explored.%investigated.
GAN frameworks of CNNs \cite{li2022tts,karami2024timehr} and a diffusion model with U-Nets \cite{naiman2024utilizing} have also been explored.

Due to their small to medium sizes, these models are often trained from scratch using task-specific training data. %per task using the task's training set. %of time series images.
Meanwhile, fine-tuning {\em pre-trained vision models}  %such as those pre-trained on ImageNet, %\cite{deng2009imagenet}, 
have already been found promising in cross-modality knowledge transfer for time series anomaly detection \cite{namura2024training}, forecasting \cite{li2020forecasting} and classification \cite{jin2023classification}.

% \cite{li2020forecasting} uses ImageNet pretrained CNNs.

\subsection{Large Vision Models (LVMs)}\label{sec.lvms}

Vision Transformer (ViT) \cite{dosovitskiy2021image} has %given birth to
inspired the development of %some
modern LVMs %large vision models (LVMs)
such as %DeiT \cite{touvron2021training}, 
Swin \cite{liu2021swin}, BEiT \cite{bao2022beit}, and MAE \cite{he2022masked}. %Given an input image, ViT splits it
As Fig. \ref{fig.models}(b) shows, ViT splits an %input
image into {\em patches} of fixed size, then embeds each patch and augments it with a positional embedding. The %resulting
vectors of patches are processed by a Transformer %encoder
as if they were token embeddings. Compared to CNNs, ViTs are less data-efficient, but have higher capacity. %Consequently,
Thus, %the
{\em pre-trained} ViTs have been explored for modeling %the images of time series.
imaged time series. For example, AST \cite{gong2021ast} fine-tunes DeiT \cite{touvron2021training} on the filterbank spetrogram of audios %signals
for classification tasks and finds %using
ImageNet-pretrained DeiT is remarkably effective in knowledge transfer. The fine-tuning paradigm has also been %similarly
adopted in \cite{zeng2023pixels,li2023time} but with different pre-trained models %initializations
such as Swin by \cite{li2023time}. 
VisionTS \cite{chen2024visionts} %explains
attributes %the superiority of LVMs
LVMs' superiority over LLMs in knowledge transfer %over LLMs %as an outcome of
to the small gap between the pre-trained images and imaged time series. %the patterns learned from the large-scale pre-trained images and the patterns in the images of time series.
It %also
finds that with one-epoch fine-tuning, MAE becomes the SOTA time series forecasters on %many
some benchmark datasets.

Similar to %build
time series foundation models %\cite{das2024decoder,goswami2024moment,ansari2024chronos,shi2024time}, %such as TimesFM \cite{das2024decoder}, MOMENT \cite{goswami2024moment}, Chronos \cite{ansari2024chronos} and Time-MoE \cite{shi2024time},
such as TimesFM \cite{das2024decoder}, %and MOMENT \cite{goswami2024moment}, 
there are some initial efforts in pre-training ViT architectures with imaged time series. Following AST, SSAST \cite{gong2022ssast} introduced a %joint discriminative and generative
%masked spectrogram patch prediction self-supervised learning framework
masked spectrogram patch prediction framework for pre-training ViT on a large dataset -- AudioSet-2M. Then it becomes a backbone of some follow-up works such as AST-SED \cite{li2023ast} for sound event detection. %To be effective for UTSs,
For UTSs, ViTime \cite{yang2024vitime} generates a large set of Line Plots of synthetic UTSs for pre-training ViT, which was found superior over TimesFM in zero-shot forecasting tasks on benchmark datasets.

\subsection{Large Multimodal Models (LMMs)}\label{sec.lmms}

%As Large Multimodal Models (LMMs)
As LMMs %are getting
get growing attentions, some %of the
notable LMMs, such as LLaVA \cite{liu2023visual}, Gemini \cite{team2023gemini}, GPT-4o \cite{achiam2023gpt} and Claude-3 \cite{anthropic2024claude}, have been explored to consolidate the power of LLMs %on time series
and LVMs in time series analysis. 
Since LMMs support multimodal input via prompts, methods in this thread typically prompt LMMs with the textual and imaged representations of time series, %textual representation of time series and their %image transformations, transformed images,
%then instruct LMMs
and instructions on what tasks to perform ({\em e.g.}, Fig. \ref{fig.models}(c)).

InsightMiner \cite{zhang2023insight} is a pioneer work that uses the LLaVA architecture to generate %textual descriptions about
texts describing the trend of each input UTS. It extracts the trend of a UTS by Seasonal-Trend decomposition, encodes the Line Plot of the trend, and concatenates the embedding of the Line Plot with the embeddings of a textual instruction, which includes a sequence of numbers representing the UTS, {\em e.g.}, ``[1.1, 1.7, ..., 0.3]''. The concatenated embeddings are taken by a language model for generating trend descriptions. %It also fine-tunes a few layers with the generated texts to align LLaVA checkpoints with time series domain.
Similarly, \cite{prithyani2024feasibility} adopts the LLaVA architecture, but for MTS classification. An MTS is encoded by %a sequence of
the visual %token
embeddings of the stacked Line Plots of all variates. %meanwhile
%The method also stacks
%The time series of all variate are also stacked in a prompt % of all variates in a prompt
The matrix of the MTS is also verbalized in a prompt 
as the textual modality. %By manipulating token embeddings,
By integrating token embeddings, both %of these %works propose to
methods fine-tune some layers of the LMMs with some synthetic data.

Moreover, zero-shot and in-context learning performance of several commercial LMMs have been evaluated for audio classification \cite{dixit2024vision}, anomaly detection \cite{zhuang2024see}, and some synthetic tasks \cite{daswani2024plots}, where the image %({\em e.g.}, spectrograms, Line Plots)
and textual representations of a query %UTS or MTS
time series are integrated into a prompt. For in-context learning, these methods inject the images of a few example time series and their labels ({\em e.g.}, classes) %({\em e.g.}, classes, normal status)
into an instruction to prompt LMMs for assisting the prediction of the query time series.

\subsection{Task-Specific Heads}\label{sec.task}

%With the image embedding of a time series, the next step is to produce its prediction.
For classification tasks, most of the methods in Table \ref{tab.taxonomy} adopt a fully connected (FC) layer or multilayer perceptron (MLP) to transform an embedding into a probability distribution over all classes. For forecasting tasks, there are two approaches: (1) using a $d_{e}\times W$ MLP/FC layer to directly predict (from the $d_{e}$-dimensional embedding) the time series values in a future time window of size $W$ \cite{li2020forecasting,semenoglou2023image}; (2) predicting the pixel values that represent the future part of the time series and then recovering the time series from the predicted image \cite{yang2023your,chen2024visionts,yang2024vitime} ($\S$\ref{sec.processing} discusses the recovery methods). Imputation and generation tasks resemble forecasting %in the sense of predicting
as they also predict time series values. Thus approach (2) has been used for imputation \cite{wang2015imaging} and generation \cite{naiman2024utilizing,karami2024timehr}. %LMMs have been used for classification, text generation, and anomaly detection. For these tasks,
When using LMMs for classification, text generation, and anomaly detection, most of the methods prompt LMMs to produce the desired outputs in textual answers, circumventing task-specific heads \cite{zhang2023insight,dixit2024vision,zhuang2024see}.

%Forecasting: MLP, FC to predict numerical values using embeddings. Imputation of images (TSH). Classification: MLP, FC using embeddings.

\section{Pre-Processing and Post-Processing}\label{sec.processing}

To be successful in using vision models, some subtle design desiderata %to be considered
include {\bf time series normalization}, {\bf image alignment} and {\bf time series recovery}.

\vspace{0.2cm}

\noindent{\bf Time Series Normalization.} Vision models are usually trained on %images after Gaussian normalization (GN).
standardized images. To be aligned, the images introduced in $\S$\ref{sec.tsimage} should be normalized with a controlled mean and standard deviation, as did by \cite{gong2021ast} on spectrograms. In particular, as Heatmap is built on raw time series values, the commonly used Instance Normalization (IN) \cite{kim2022reversible} can be applied on the time series as suggested by VisionTS \cite{chen2024visionts} since IN share similar merits as Standardization. %although min-max normalization was used by \cite{karami2024timehr,zeng2021deep}.
Using Line Plot requires a proper range of y-axis. In addition to rescaling time series %by min-max or GN
\cite{zhuang2024see}, ViTST \cite{li2023time} introduced several methods to remove extreme values from the plot. GAF requires min-max normalization on its input, as it transforms time series values withtin $[0, 1]$ to polar coordinates ({\em i.e.}, arccos). In contrast, input to RP is usually normalization-free as an $\ell_{2}$ norm is involved in Eq.~\eqref{eq.rp} before thresholding.%for a comparison with a threshold.

\vspace{0.2cm}

\noindent{\bf Image Alignment.} When using pre-trained models, it is imperative to fit the image size to the input requirement of the models. This is especially true for Transformer based models as they use a fixed number of positional embeddings to encode the spacial information of image patches. For 3-channel RGB images such as Line Plot, it is straightforward to meet a pre-defined size by adjusting the resolution when producing the image. For images built upon matrices such as Heatmap, Spectrogram, GAF, RP, the number of channels and matrix size need adjustment. For the channels, one method is to duplicate a matrix to 3 channels \cite{chen2024visionts}, another way is to average the weights of the 3-channel patch embedding layer into a 1-channel layer \cite{gong2021ast}. For the image size, bilinear interpolation is a common method to resize input images \cite{chen2024visionts}. Alternatively, AST \cite{gong2021ast} %use cut and bilinear interpolation on
resizes the positional embeddings instead of the images to fit the model to a desired input size. However, the interpolation in these methods may either alter the time series or the spacial information in positional embeddings.

% single-channel (UTS), RGB channel (UTS), duplicate channels (UTS), multi-channel (MTS).

%Bilinear interpolation.

%Correlated variates are better to be spatially close to each other.

%\subsection{Pre-training}

\vspace{0.2cm}

\noindent{\bf Time Series Recovery.} As stated in $\S$\ref{sec.task}, tasks such as forecasting, imputation and generation requires predicting time series values. For models that predict pixel values of images, post-processing involves recovering time series from the predicted images. Recovery from Line Plots is tricky, it requires locating pixels that %correspond to
represent time series and mapping them back to the original values. This can be done by manipulating a grid-like Line Plot as introduced in \cite{yang2023your,yang2024vitime}, which has a recovery function. In contrast, recovery from Heatmap is straightforward as it directly stores the predicted time series values \cite{zeng2021deep,chen2024visionts}. Spectrogram is underexplored in these tasks and it remains open on how to recover time series from it. The existing work \cite{zeng2023pixels} uses Spectrogram for forecasting only with an MLP head that directly predicts time series. %predicts time series values.
GAF supports accurate recovery by an inverse mapping from polar coordinates to normalized time series \cite{wang2015imaging}. However, RP lost time series information during thresholding (Eq.~\ref{eq.rp}), thus may not fit recovery-demanded tasks without using an {\em ad-hoc} prediction head.


% Line Plot was regarded as matrices with rows and columns for mapping in \cite{sood2021visual}.


%\section{Tasks and Time Series Recovery}

%\subsection{Task-Specific Head}

% \subsection{Time Series Recovery}




The GFM-based RS effectively utilize the technological complementarity of GNN and LLM. GNNs struggle to model textual information, while the reasoning capabilities of LLMs do not support their comprehension of higher-order structural information. These two technologies complement each other's shortcomings in GFM, which emerges as a future opportunity in the field of recommendations. For example, LLMGR~\cite{guo2024integrating} injects the embeddings learned by GNN into the token embedding sequence of LLM, and adapts the GFM to the recommendation task through two-stage fine-tuning. LLMRG~\cite{wang2023enhancing} constructs inference graphs and divergence graphs based on user interaction history using LLM, which are then encoded by GNN for recommendations. DALR~\cite{peng2024denoising} aligns the embeddings encoded by GNN and those encoded by LLM in various ways, using the aligned embeddings for subsequent recommendations.

In this survey, we comprehensively investigate the relevant work of GFM-based RS, and provide a clear taxonomy based on the synergistic relationship between the graph and LLM in GFM: \textbf{Graph-augmented LLM}, \textbf{LLM-augmented graph} and \textbf{graph-LLM harmonization}.
Graph-augmented LLM methods can be viewed as utilizing the structural information of the graph to aid the knowledge obtained from LLM pre-training for recommendations. LLM-augmented graph methods, on the other hand, is led by the structural information of the graph, with the world knowledge of LLM serving as auxiliary information. Graph-LLM harmonization methods involve the equal transformation of these two types of information in the representation space. 

% As an evergreen topic in both academia and industry, numerous surveys have been conducted on recommender systems \cite{gao2023survey,wu2024survey}. The former provides a comprehensive review of graph-based recommender systems, representing traditional methodologies, while the latter offers an overview of LLM based recommender systems, representing a new paradigm. While the previous two surveys offer detailed insights into the respective technologies, they were unaware of the rapid development of GFM in the field of recommendations. Therefore, our survey offers a broader perspective for extensive research related to recommendations.

As an evergreen topic in both academia and industry, RS have been the subject of numerous surveys (e.g., \cite{gao2023survey,wu2024survey,liu2023towards,li2023survey}). \cite{gao2023survey,wu2024survey} focus on specific methodologies, such as GNN-based RS or the more recent LLM-based RS. \cite{li2023survey} concentrates on utilizing LLM to enhance graphs for tackling tasks related to graphs. However, the field is rapidly evolving with GFMs emerging as a crucial technique of the RS research. \cite{liu2023towards} systematically outlines the existing GFMs from the perspectives of pre-training and adaptation, while overlooking the recommendation which is one of the significant downstream tasks for GFM. This survey provides a timely and comprehensive overview that covers the landscape of GFM-based recommender systems.

The contributions of this survey can be summarized in the following aspects:\textbf{1)} \textit{Pioneering overview}: Our survey fills the blank in comprehensive work in the field of GFM-based RS. \textbf{2)} \textit{Clear taxonomy}: The comprehensive survey presents a well-structured taxonomy of GFM-based RS, allowing future work to be easily categorized within the corresponding branches. \textbf{3)} \textit{Promising outlook}: We present the challenges and future research directions in this field, which can serve as a valuable reference for research in this rapidly evolving area.
% This survey provides the first systematic review of graph foundation models for recommendation, offering several key contributions to the field:  

% 1. \textbf{A Novel Classification Framework}: We propose a comprehensive framework to categorize the GFM into three paradigms: Graph-augmented LLMs, LLM-augmented graphs, and LLM-graph harmonization in recommendation. This taxonomy provides a clear roadmap for understanding the field and guiding future research.

% 2. \textbf{Methodological Review}: We conduct an in-depth analysis of methodologies within each paradigm, discussing their theoretical foundations, design strategies, and real-world applications. Representative studies are examined to highlight their contributions to solving key recommendation challenges.

% 3. \textbf{Challenges and Future Directions}: Through meticulous literature synthesis, we unveil major challenges in this field, such as alignment of representations, computational efficiency, scalability, and integration complexity. Simultaneously, we spotlight prospective avenues for future research, including adaptive integration methods, cross-modal fusion, and efficient large-scale deployment strategies. Our analysis and insights aim to both address these current challenges and inspire future innovation, guiding researchers to unlock the full potential of integrating graph and LLM technologies.

\begin{figure*}[t]
    \centering
    \includegraphics[width=\linewidth]{figures/metric_intro.pdf}
    \caption{Illustration of the proposed categories of step-by-step reasoning evaluation criteria, \textit{i.e.} groundedness, validity, coherence, and utility. 
    The left shows an example of a query and a reasoning trace. The other four blocks demonstrate examples that fail to suffice the respective metric. Red filled rectangles indicate the error's location, and the outlined boxes and arrows show the cause of the error.}
    \label{fig:metric_intro}
\end{figure*}

\section{Taxonomy}
\label{sec:taxonomy}

% \newcommand{\ResultEightByEightCrossbarOverheadkGE}{13.1}
\newcommand{\ResultEightByEightCrossbarOverheadPercent}{9}
\newcommand{\ResultSixteenBySixteenCrossbarOverheadkGE}{45.4}
\newcommand{\ResultSixteenBySixteenCrossbarOverheadPercent}{12}
\newcommand{\ResultAsymptoticOverheadPercent}{21.6}
\newcommand{\ResultSixteenBySixteenCrossbarFrequencyOverheadPercent}{6}
\newcommand{\ResultThirtyTwoClusterEightKiBParallelFraction}{97}
\newcommand{\ResultThirtyTwoClusterTwoKiBSpeedup}{13.5}
\newcommand{\ResultThirtyTwoClusterThirtyTwoKiBSpeedup}{16.2}
\newcommand{\ResultThirtyTwoClusterGeometricMeanSpeedup}{5.6}
\newcommand{\ResultBaselineTileNOperationalIntensity}{1.9}
\newcommand{\ResultBaselineTileNPerformanceGFLOPS}{114.4}
\newcommand{\ResultBaselineTileNPerformancePercentage}{92}
\newcommand{\ResultHybridTileNOperationalIntensityIncrease}{3.7}
\newcommand{\ResultHybridTileNPerformanceIncrease}{2.6}
\newcommand{\ResultMulticastTileNOperationalIntensityIncrease}{16.5}
\newcommand{\ResultMulticastTileNPerformanceIncrease}{3.4}
\newcommand{\ResultMulticastTileNPerformanceIncreaseOverHybridPercentage}{29}
\newcommand{\ResultMulticastTileNPerformanceGFLOPS}{391.4}
 -> appendix

This section aims to provide a clear taxonomy of criteria for evaluating step-by-step reasoning. Existing criteria can be seen as falling into one of the four categories, namely \textbf{Groundedness}, \textbf{Validity}, \textbf{Coherence}, and \textbf{Utility}. These definitions are \textit{independent} (aim at different objectives -- Section \ref{sec:comparison-ours}), but \textit{not mutually exclusive} (a step can fail to suffice multiple criteria at once).

\subsection{Groundedness}

\textbf{Groundedness} evaluates if the \textit{step is factually true} according to the query \citep{NEURIPS2020_6b493230, gao2024retrievalaugmentedgenerationlargelanguage}. A step can be ungrounded to any part of the query, \textit{e.g.} the question (Figure \ref{fig:metric_intro}-Groundedness) or evidence (\textit{e.g.} falsely stating that \textit{Buddy Rich was born in Chicago}, where the retrieved document states that he was born in New York). 

\subsection{Validity}

\textbf{Validity} evaluates if a reasoning step contains no errors.

The validity of a reasoning step can be defined in terms of \textit{entailment} \citep{bowman-etal-2015-large}, which is widely accepted in factual/commonsense reasoning. Under this definition, a step is considered valid if it can be directly entailed from previous steps \citep{tafjord-etal-2021-proofwriter, dalvi-etal-2021-explaining, PrOntoQA} or at least does not contradict them \citep{DBLP:conf/iclr/GolovnevaCPCZFC23, prasad-etal-2023-receval, zhu2024deductivebeamsearchdecoding}.

The notion of validity often used in symbolic tasks is \textit{correctness}, \textit{e.g.} performing accurate calculations in math reasoning \citep{DBLP:conf/iclr/LightmanKBEBLLS24, jacovi-etal-2024-chain, zheng2024processbenchidentifyingprocesserrors} or inferring the correct logical conclusion based on the provided premises \citep{wu2024cofcastepwisecounterfactualmultihop, jacovi-etal-2024-chain, song2025prmbenchfinegrainedchallengingbenchmark}.

% While early works do not identify the type of fallacies by applying binary or ternary label, recent works tend to include fine-grained error types \citep{song2025prmbenchfinegrainedchallengingbenchmark} or human-written explanations \citep{jacovi-etal-2024-chain} to improve the explainability of the evaluation process.

% Tyen(2024) Direct mistake prompting

\subsection{Coherence}
\label{sec:coherence}

\textbf{Coherence} measures if a reasoning step's \textit{preconditions are satisfied} by the previous steps \citep{wang-etal-2023-towards}. For instance, if a trace includes the reasoning step \textit{"Next, we add 42 to 16."} but the origin of the value 42 was never explained in the previous steps, this step is considered incoherent. An intuitive way to obtain an incoherent trace is randomly shuffling a coherent trace \citep{wang-etal-2023-towards, nguyen-etal-2024-direct}, as the premise of some steps will not appear anywhere in the previous steps even though it can be eventually deduced (\textit{valid}).

Note that coherence judgment is inherently subjective and pragmatic compared to other criteria. For instance, seemingly trivial steps like \textit{"A part of something is present in that something"} in WorldTree V2 \citep{xie-etal-2020-worldtree} is annotated as necessary in \citet{dalvi-etal-2021-explaining} but not necessary in \citet{Ott_2023}.

% The same concept is also referred to as \textit{broad validity} (as opposed to strict validity) \citep{PrOntoQA} and \textit{prerequisite sensitivity} \citep{song2025prmbenchfinegrainedchallengingbenchmark}.

% \textbf{Symbolic solutions.} \hspace{0.1cm} Coherence can be clearly defined in \textit{symbolic} reasoning tasks, where the dependency between steps can be symbolically defined. \citep{PrOntoQA} defined steps that require applying two inference rules as valid but incoherent (\textit{broadly valid}). However, as they used a simple synthetic dataset that only includes \textit{Modus ponens}, observed instances of broadly valid steps can be well seen as coherent in common sense. \citet{nguyen-etal-2024-direct}, where a reasoning trace corresponds to a directed path in knowledge graphs (KGs), defines a coherent ERU as a directed KG edge where its source node was already introduced as a target node.

\subsection{Utility}
\label{sec:utility}

\textbf{Utility} measures whether a reasoning step contributes to getting the correct final answer (\textit{answer correctness}).

One interpretation of utility is \textit{progress}, or whether the step is correctly following the ground truth solution. For instance, in Game of 24 (making the number 24 using 4 natural numbers and basic arithmetic operations) \citep{NEURIPS2023_271db992}, a solution can be defined as a sequence of operations (\textit{e.g.} 5+7=12$\rightarrow$12-6=6$\rightarrow$6*4=24.). In this task, the utility of a step (making $5+7=12$ from $5$ and $7$) can be directly assessed by checking if it is a part of a correct solution.
% \footnote{Value function is also often described as \textit{progress} under the policy (\textit{i.e.} LLM) \citep{setlur2024rewardingprogressscalingautomated}. However, for clarity, we restrict the term to \textit{discrete} progress.}

Utility can also be interpreted as \textit{value function} (estimated reward), which is proportional to the probability of reaching the correct answer starting from the step \citep{hao-etal-2023-reasoning, wang-etal-2024-math, xie2024montecarlotreesearch, chen-etal-2023-rev}. This black-box interpretation of utility offers high scalability as it only requires the gold answer, without any human annotation or ground-truth solutions \citep{wang-etal-2024-math, lai2024stepdpostepwisepreferenceoptimization}.
% \section{Methods for Integrating Graphs and Large Language Models in Recommender Systems}

% \section{Graph-Augmented Large Language Models}
% \label{sec:graph-llm}

% Large Language Models (LLMs) excel at understanding and generating text but struggle with the complex relational structures inherent in recommender systems. While pre-training corpora and in-context learning provide some information, they lack an explicit mechanism to model the intricate relationships between users, items, and their attributes that are naturally represented as graphs. Recent research bridges this gap by integrating Graph Neural Networks (GNNs) with LLMs, focusing on the core challenge: \textbf{How to design cross-modal interfaces that effectively bridge graph structures to language models?} Current methods can be broadly categorized into token-level and context-level infusion. Our work will concentrate on these two primary categories, leaving aside for now the investigation of more complex architectures like Wang et al.'s~\cite{wang2024llm} LLM4REC, which integrates graph information by modifying the LLM's attention mechanism with GNN-learned edge representations. These advanced approaches, while promising, are still in early stages of development and fall outside the current scope.

% \subsection{Token-Level Infusion}
% \label{subsec:token-injection}

% This strategy integrates graph information directly into the LLM's input at the token level. Nodes or subgraphs are represented as special tokens, allowing the LLM to process graph information alongside text.

% \begin{equation}
% \mathbf{e}_{v} = f(\mathbf{h}_v)
% \end{equation}
% where $G = (\mathcal{V}, \mathcal{E})$ is a graph, a GNN learns node representations $\mathbf{h}_v \in \mathbb{R}^d$ for each $v \in \mathcal{V}$, and these are projected into the LLM's embedding space via a mapping $f: \mathbb{R}^d \rightarrow \mathbb{R}^{d_{\text{LLM}}}$, creating special token embeddings $\mathbf{e}_{v}$.

% \textbf{Syntax-Integrated Injection}
% This approach embeds special tokens as syntactic components within the LLM's input sequence. For instance, Ma~\cite{ma2024triple} introduce \texttt{[ACTION]} tokens like \texttt{[view]} or \texttt{[purchase]} to represent user actions within an interaction sequence. The embeddings for these actions are learned from a multi-behavior graph, enabling the LLM to process complex semantics like ``user \texttt{[views]} item". Building on this, Wang~\cite{wang2024enhancing} generate a GCN-based embedding $\mathbf{h}_i$ for each item $i$ and add it as a correction term to the item's original text embedding $\mathbf{e}_i$, resulting in a refined embedding $\mathbf{e'}_i = \mathbf{e}_i + \mathbf{h}_i$. This blends textual and structural information, improving item representation.

% A natural progression from here is to consider whether the LLM can directly output these special tokens. Guo~\cite{guo2024integrating} explore this by not only including special tokens in the input but also modifying the LLM's output layer. They introduce a special token for each item, allowing the model to directly generate item tokens as recommendations, effectively creating a tighter link between graph-based recommendations and the LLM's output. Further advancing this line of thought, Mei~\cite{mei2023lightlm} propose a hierarchical indexing scheme. Instead of treating user/item IDs as atomic units, they decompose them into multiple special tokens based on a user-item graph-derived index. Each component token encodes a different attribute or function, moving away from opaque numerical IDs towards more semantically meaningful representations.

% \textbf{Syntax-Decoupled Injection}
% This method appends graph embeddings as prefixes or suffixes, separating graph information from the main textual prompt. Ma~\cite{ma2024xrec} prepend GNN-learned embeddings that represent user-item relationships to the prompt. These embeddings are trained to capture high-level semantic concepts, such as preference similarity between users. Qiu~\cite{qiu2024unveiling} combine user queries with knowledge graph embeddings. A GNN generates embeddings that encapsulate both the user's query and relevant knowledge graph entities, which are then used as a prefix to guide the LLM towards more informed recommendation rationales.

% Token-level infusion offers a fine-grained way to integrate graph information, allowing for nuanced interactions between textual and structural data. However, it often requires modifications to the LLM's architecture or careful prompt engineering.

% \subsection{Context-Level Infusion}
% \label{subsec:prompt-injection}

% This strategy provides graph information as context to the LLM, either through text descriptions or implicit retrieval, avoiding modifications to the LLM's architecture.

% \begin{align}
%     \texttt{user A} &\rightarrow \texttt{purchase} \rightarrow \texttt{item B} \notag \\
%     &\Rightarrow \texttt{user A purchased item B.} \notag \\
%     C(v_q) &= \text{TopK}\left(\{\text{sim}(\mathbf{h}_{v_q}, \mathbf{h}_v) | v \in \mathcal{V}\}\right)
% \end{align}
    
% Where the first example illustrate explicit graph-to-text mapping. The second equation represents implicit graph retrieval, where $C(v_q)$ represents content for a query or subgraph $v_q$, $\text{sim}$ is a similarity function (e.g., cosine similarity) and $\mathcal{V}$ is the set of all nodes in the graph.

% \textbf{Explicit Graph-to-Text Mapping}
% This method involves converting localized graph structures into natural language descriptions, essentially translating graph relationships into text. The simplest form of this is exemplified by Jia~\cite{jia2025hetgcot}. They extract multi-hop neighbor nodes of a target node from a heterogeneous graph and concatenate their attributes to create a natural language description. This description becomes the context for the LLM, informing its recommendations. Similarly, Abu-Salih~\cite{abu2024knowledge} extract one-hop and two-hop related nodes from a knowledge graph and insert them into predefined prompt templates. These templates, now populated with graph information, guide the LLM in generating explainable recommendation rationales.

% Further refinements involve pre-processing the graph information before mapping it to text. Guan~\cite{guan2024enhancing} maintain a dynamic queue of negative samples, items the user is known to dislike, based on knowledge graph insights and LLM feedback. Providing both positive and negative samples as context for the LLM allows for more nuanced recommendations. The processing of graph information can also occur after its initial conversion to text. Wu~\cite{wu2024exploring} construct natural language descriptions of node paths in a job information graph. Each path represents a sequence of related job attributes or skills. During prompt construction, these paths are assigned different weights based on relevance and dependency strength, offering a more refined, contextually rich input to the LLM.

% \textbf{Implicit Graph Retrieval}
% When explicit mapping is difficult, this approach uses GNN embeddings to retrieve relevant information from the graph semantically. Chen~\cite{chen2024leverage} apply this in the legal domain. They encode a legal knowledge graph using a GNN, and retrieve relevant legal provisions based on the similarity between a user's case description embedding and the provision embeddings. These provisions are then concatenated into the prompt context. Shen~\cite{shen2024exploring} retrieve a user's historical interactions from an item-attribute graph, focusing on neighbor interactions. These records are added to the prompt, providing the LLM with cross-domain preference information.

% Context-level infusion provides a flexible way to incorporate graph knowledge without altering the LLM's architecture. It leverages the LLM's ability to understand and reason over natural language, making it suitable for scenarios where graph structures can be effectively verbalized.

% % \subsection{Graph-Integrated Architectures}
% % \label{subsec:advanced-architectures}

% % Beyond basic infusion techniques, new methods are emerging that fundamentally alter how LLMs process graph data.

% % \textbf{Hierarchical Graph-Indexed Tokenization} Mei~\cite{mei2023lightlm} propose LightLM, which uses spectral and graph index learned from GNN to create hierarchical IDs for users and items. These IDs, based on graph structure, are used to build a Trie structure that optimizes the beam search process for generating recommendations. LightLM also features a "deep and narrow" FFN to improve efficiency for recommendation tasks.

% % \textbf{Graph-Enhanced Attention Mechanism} Wang et al.~\cite{wang2024llm} introduce LLM4REC, which modifies the attention mechanism in LLMs (specifically GPT-2) by incorporating GNN-learned representations of edge relationships. This allows the model to directly consider graph proximity when processing tokens, leading to more informed recommendations. Specifically, they introduce adjacency and path relation correction term derived from the graph into the attention's query-key product, guiding the attention mechanism based on the graph structure.

% \subsection{Discussion}
% \label{subsec:discussion}

% The ``Graph $\rightarrow$ GNN $\rightarrow$ LLM" paradigm presents a promising avenue for enhancing recommender systems by bridging the gap between the static world knowledge encapsulated within LLMs and the dynamic nature of user-item interactions captured in graphs. This approach not only allows for a more nuanced understanding of user preferences and item relationships but also potentially mitigates the limitations of LLMs in handling evolving data patterns inherent in recommendation scenarios. By encoding rich structural and relational information from graphs, GNNs empower LLMs with a deeper, more context-aware understanding that can significantly improve recommendation accuracy, diversity, and explainability. Moreover, this graph-augmented approach naturally facilitates the incorporation of heterogeneous data sources, such as user interaction histories, item knowledge graphs, and textual content, paving the way for more holistic and sophisticated recommender systems. Recent innovations like hierarchical graph-indexed tokenization and graph-enhanced attention further demonstrate the potential of this approach by fundamentally altering how LLMs interact with graph data, paving the way for even more sophisticated and effective recommendation models.



\section{Graph-Augmented LLM}
\label{sec:graph-llm}

LLMs excel at understanding and generating text but struggle with the complex relational structures inherent in recommender systems. While pre-training corpora and in-context learning provide some information, they lack an explicit mechanism to model the intricate relationships between users and items that are naturally represented as graphs.
\begin{figure}[t]
    \centering
    \includegraphics[width=\linewidth]{figures/Graph-Augmented_LLM-jiawei.pdf}
    \caption{The illustration of graph-augmented LLM methods: 
    a) \textbf{Token-Level Infusion}, where nodes or subgraphs are represented as special tokens, integrating into LLM's input.
    b) \textbf{Context-Level Infusion}, where graph information is converted into context by translating graph into text or retrieving relevant text.}
    \label{fig:graph-augmented LLM}
\end{figure}
Recent research bridges this gap by integrating graphs with LLMs, focusing on the core challenge: \textbf{How to design cross-modal interfaces that effectively bridge graph structures to language models?} We categorize current methods into \textbf{token-level infusion} and \textbf{context-level infusion} (as illustrated in Figure~\ref{fig:graph-augmented LLM}), based on where the cross-modal interface is implemented.

% Our work will concentrate on these two primary categories, leaving aside for now the investigation of more complex architectures like \cite{wang2024llm}'s LLM4REC, which integrates graph information by modifying the LLM's attention mechanism with GNN-learned edge representations. These advanced approaches, while promising, are still in early stages of development and fall outside the current scope.



\subsection{Token-Level Infusion}
\label{subsec:token-injection}

This strategy integrates structural information directly into the LLM's input at the token level. Nodes or subgraphs are represented as special tokens, allowing the LLM to process structural information alongside text.
% Formally, we have:

% \begin{equation}
% \mathbf{e}_{v} = f(\mathbf{h}_v)
% \end{equation}
% where GNN learns node representations $\mathbf{h}_v \in \mathbb{R}^d$ for each vertex $v$, and these are projected into the LLM's embedding space via a mapping $f: \mathbb{R}^d \rightarrow \mathbb{R}^{d_{\text{LLM}}}$, creating special token embeddings $\mathbf{e}_{v}$.
\paragraph{Syntax-Integrated Injection.} This approach embeds special tokens as syntactic components within the LLM's input sequence. For example, TMF~\cite{ma2024triple} introduces \texttt{[ACTION]} tokens like \texttt{[view]} or \texttt{[purchase]} to represent user actions within an interaction sequence. The embeddings for these actions are learned from a multi-behavior graph, enabling the LLM to process complex semantics like ``user \texttt{[views]} item". Building on this, ELMRec~\cite{wang2024enhancing} generates a GCN-based embedding $\mathbf{h}_i$ for each item $i$ and adds it as a correction term to the item's original text embedding $\mathbf{e}_i$, resulting in a refined embedding $\mathbf{e'}_i = \mathbf{e}_i + \mathbf{h}_i$. This operation blends textual and structural information, improving item representation.

A natural progression from here is to consider whether the LLM can directly output these special tokens. LLMGR~\cite{guo2024integrating} explores this by not only including special tokens in the input but also modifying the LLM's output layer. They introduce a special token for each item, allowing the model to directly generate item tokens as recommendations, effectively creating a tighter link between graph-based recommendations and the LLM's output. Further advancing this line of thought, LightLM~\cite{mei2023lightlm} proposes a hierarchical indexing scheme, which decomposes user/item IDs into multiple special tokens based on a user-item graph-derived index. Each component token encodes a different attribute or function, moving away from opaque numerical IDs towards more semantically meaningful representations.

\paragraph{Syntax-Decoupled Injection.}
This method appends graph embeddings as prefixes or suffixes, separating structural information from the main textual prompt. XRec~\cite{ma2024xrec} prepends GNN-learned embeddings that represent user-item relationships to the prompt. These embeddings are trained to capture high-level semantic concepts, such as preference similarity between users. COMPASS~\cite{qiu2024unveiling} combines user queries with knowledge graph embeddings. A GNN generates embeddings that encapsulate both the user's query and relevant knowledge graph entities, which are then used as a prefix to guide the LLM for recommendation.

Token-level infusion offers a fine-grained way to integrate structural information, allowing for natural interactions between textual and structural data. However, it often requires modifications to the LLM's architecture or careful prompt engineering.

\subsection{Context-Level Infusion}

This strategy provides structural information as context to the LLM, either through text descriptions or implicit retrieval, avoiding modifications to the LLM's architecture.

\paragraph{Explicit Graph-to-Text Mapping.} 
This method involves converting localized graph structures into natural language descriptions, essentially translating graph relationships into text. The following example illustrates explicit graph-to-text mapping: \texttt{user A} $\rightarrow$ \texttt{purchase} $\rightarrow$ \texttt{item B} $\rightarrow$ \texttt{payment} $\rightarrow$ \texttt{credit card} $\Rightarrow$ \texttt{A purchased B with a credit card.} The simplest form of this is exemplified by HetGCoT-Rec~\cite{jia2025hetgcot}. They extract multihop neighbors of a target node from a heterogeneous graph and concatenate their attributes to create a natural language description, which becomes the context for the LLM and informs its recommendations. Similarly, KGRec~\cite{abu2024knowledge} extracts one-hop and two-hop related nodes from a knowledge graph and inserts them into predefined prompt templates, which guides the LLM in generating explainable recommendations.

Further refinements involve pre-processing the structural information before mapping it to text. GAL-Rec~\cite{guan2024enhancing} maintains a dynamic queue of negative samples, items the user is known to dislike, based on knowledge graph insights and LLM feedback. Providing both positive and negative samples as context for the LLM allows for more nuanced recommendations. The processing of structural information can also occur after its initial conversion to text. GLRec~\cite{wu2024exploring} constructs natural language descriptions of node paths in a job information graph. Each path represents a sequence of related job attributes or skills. During prompt construction, these paths are assigned different weights based on relevance and dependency strength, offering a more refined, contextually rich input to the LLM.

\paragraph{Implicit Graph Retrieval.} 
When explicit mapping is difficult, this approach uses GNN embeddings to retrieve relevant information from the graph semantically. For example, CLAKG~\cite{chen2024leverage} encodes a legal knowledge graph using a GNN, and retrieve relevant legal provisions based on the similarity between a user's case description embedding and the provision embeddings. These provisions are then concatenated into the prompt context. URLLM~\cite{shen2024exploring} retrieves a user's historical interactions from an item-attribute graph, focusing on neighbor interactions. These records are added to the prompt, providing the LLM with cross-domain preference information.

Context-level infusion provides a flexible way to incorporate graph knowledge without altering the LLM's architecture. It leverages the LLM's ability to understand and reason over natural language, making it suitable for scenarios where graph structures can be effectively verbalized.

\subsection{Discussion}
\label{subsec:discussion}

Graph-augmented LLM methods enhance recommendations by encoding rich relational information from graphs, typically through token-level or context-level infusion. This couples the benefits of graph-structured data with the power of LLMs: the graph provides valuable relational context, while the LLM leverages its pre-trained knowledge to interpret it. Furthermore, since the LLM is the central component, this approach augments the recommender system's ability to extrapolate and make inferences in scenarios where interaction data is limited. However, this heavy reliance on the LLM also introduces inherent biases unsuitable for recommendation, such as a lack of diversity and distributional mismatch with user preferences, potentially limiting its scalability and generalizability. The alternative methods, by shifting the focus to enriching or harmonizing the graph itself, effectively mitigate these issues and offer different trade-offs.
\section{LLM-Augmented Graph}
% GNNs can achieve accurate recommendations by capturing higher-order structural information. Additionally, by integrating textual information into collaborative signals with a graph-centric approach supported by LLM, the diversity of recommendations will be significantly improved.
Shifting the focus to the graph, the core idea of the LLM-augmented graph methods is to augment the data within graphs using LLM, thereby improving the effectiveness of various GNNs employed for recommendation tasks. Such methods can be categorized into \textbf{topology augmentation} and \textbf{feature augmentation} (as illustrated in Figure~\ref{fig:LLM-augmented graph}), based on the aspects of information enhanced in the text-attribute graph according to LLMs.
\begin{figure}[t]
    \centering
    \includegraphics[width=\linewidth]{figures/LLM-Augmented.pdf}
    \caption{The illustration of LLM-augmented graph methods:  
    a) \textbf{Topology Augmentation}, where LLMs extract structural information from data to alter and augment the topological structure of the graphs.;  
    b) \textbf{Feature Augmentation}, where LLMs processe the textual information in the data, augment the node text or embedding in the graph without changing the topological structure.}
    \label{fig:LLM-augmented graph}
\end{figure}
\subsection{Topology Augmentation}
Topology augmentation refers to the processes where the LLM restructures data, utilizing its world knowledge and contextual understanding capabilities to convert specific textual information into a structured format. Due to the introduction of new structural information, the topological structure of the graph constructed from the data is modified, thereby affecting the subsequent processes to achieve more accurate recommendations.
% The graph is composed of two fundamental elements: nodes and edges. Therefore, 
Intuitively, we categorize topology augmentation into two types: \textbf{edge-level expansion} and \textbf{node-level expansion}, based on whether new nodes are introduced.
% Throughout the process of graph structure expansion, the extensive global knowledge of LLM is utilized in the form of text or embeddings.
% Leveraging the extensive global knowledge and robust contextual comprehension abilities of LLMs, explicit or implicit expansion can be made to enrich the graph structural information present in the data.
% The terms explicit and implicit refer to whether the world knowledge of LLM is utilized to augment the graph structure in the form of text or in the form of embeddings. 

\paragraph{Edge-level Expansion.}
This method refers to the process where LLMs introduce new relationships between nodes in the data, such as complementary and substitutable relationships, which are two primary types of relationships of interest in RS.

To directly utilize the versatile capabilities of LLMs and the vast world knowledge, it is intuitive to adopt text-centric approaches for adding edges in the graph. These approaches typically rely on prior knowledge to guide LLMs in making relationship judgments and constructions, either at a superficial or deeper level. As a straightforward example, LLM-KERec~\cite{zhao2024breaking} employs LLMs to assess the complementarity between pairs of items, thereby establishing complementary relationships among item pairs and constructing a complementary item graph. From a deeper perspective, in addition to leveraging complementary relationships between items, subjective user-generated reviews can also be utilized. SAGCN~\cite{liu2023understanding} and FineRec~\cite{zhang2024finerec} utilize LLMs to extract user opinions on items at varying levels of granularity across multiple item attributes (\textit{e.g.}, price, comfort, \textit{etc.}), using this information as edges to construct distinct graphs for each attribute.
% Similarly, LLMRG~\cite{wang2023enhancing} directly provides user interaction history and user attributes to the LLM, enabling it to generate item reasoning chains and subsequently construct an item reasoning graph.

Compared to the aforementioned methods, the more refined expansion delves into the relationships within the embedding space, subsequently constructing graphs based on implicit relationships between nodes at a certain level. For example, CSRec~\cite{yang2024common} first employs an LLM to generate complementary or substitutable category nodes based on existing classified nodes, and then utilizes other pre-trained language models to map the pairs of newly generated nodes and existing nodes into the node set within the semantic space. The relationships generated by the LLM are also mapped into the node set within the embedding space. In addition to mapping edges based on semantic similarity, connections can also be established based on the similarity. Several works ~\cite{yang2024sequential,cui2024comprehending} employ an LLM to encode the textual information of nodes into embeddings, and then measure potential relationships between these embeddings through carefully designed methods. These potential relationships serve as edges for item nodes in the graph within the embedding space.

\paragraph{Node-level Expansion.} 
This method further leverages the world knowledge and contextual reasoning capabilities of LLMs, utilizing auxiliary information as new nodes to augment the information of existing nodes. Such method often directly utilizes the condensed information generated by LLM as new nodes to be introduced into the graph, or extracts the requisite information from the LLM’s output through particular approaches to serve as new nodes in the graph.
% Formalization is as follows:
% \begin{equation}
% \Delta V,\Delta E = P_\text{process}(\text{LLM}(G)) 
% \end{equation}
% The formula here is similar to the mentioned earlier, with the only difference being that in addition to expanding the edge set $E$, the node set $V$ is also expanded.

Several works~\cite{jeon2024topic,hu2024bridging} directly employ LLMs to generate auxiliary information nodes (\textit{e.g.}, user interests, item categories) for corresponding users or items based on existing textual information. Introducing  such auxiliary information nodes can be viewed as a supplementation of information in the textual space. This approach can, to some extent, assist subsequent GNN in modeling better user or item representations. Furthermore, this type of information supplementation can also be performed in the embedding space. For example, AutoGraph~\cite{shan2024automatic} utilizes LLMs to encode the textual information of users and items, followed by quantizing the semantic embeddings of users and items. By quantizing these semantic embeddings, fine-grained auxiliary information embeddings for users and items can be derived, which are then used as new nodes to supplement information for users and items.

Topology augmentation represents effectively utilizing the knowledge obtained from LLM pre-training to influence the training of the GNN. The key structural information introduced into the graph during this process enables subsequent GNN to learn more comprehensive representations of users and items.

\subsection{Feature Augmentation}
Enhancing the topological structure of the graph using LLMs may introduce biases, as they are not particularly adept at extracting structural information from text. In contrast, directly improving the node features in the graph without altering the topological structure is an task where LLMs truly excel.
This method focuses on leveraging the natural language processing capabilities of LLMs to augment data at the textual or embedding level, thereby influencing subsequent recommendations.
% This type of method can be formally described as:
% \begin{multline}
% C' = \text{LLM}(C) \quad \text{or} \quad 
% \mathbf{e}_\text{LLM} = E(\text{LLM}(C))
% \end{multline}
% where $C$ represents the original text information of the nodes, $C'$ represents the enhanced text information, $\mathbf{e}_\text{LLM}$ denotes the encoded semantic embeddings, LLM stands for the enhancement of the text, and $E$ represents an encoder (which can also be an LLM).

Some works~\cite{li2024learning,chen2024prompting} enhance the textual information of nodes by constructing appropriate prompts for input into LLMs. The enhanced textual information is subsequently encoded into embeddings by language models such as BERT. Unlike the aforementioned methods, GaCLLM~\cite{du2024large}  integrates the stages of LLM and GNN. In this approach, the LLM assumes the role responsible for message passing within the GNN, performing textual message passing and aggregation for each node in the graph, which is subsequently encoded by BERT. As a special case, LIKR~\cite{sakurai2024llm} employs LLMs to analyze user interaction histories to derive user-preferred item attributes. The nodes corresponding to these attributes in the graph serve as rewards for Markov walk-based reinforcement learning within the graph.

The textual information of users and items, being one of the abundant types of information in RS, significantly impacts the performance of RS when effective utilized. Consequently, LLM is increasingly becoming the preferred technology for feature augmentation of graphs based on textual information.

\subsection{Discussion}

The LLM-augmented graph methods, by incorporating the world knowledge of LLM into graph data, can enhance the capabilities of RS at the data level. This makes these methods more competitive in cold start scenarios with sparse interactions. Furthermore, the LLM in these methods is a plug-and-play component, allowing for flexible choices. Moreover, the data processing of the LLM can be performed offline in advance, and online recommendations based on statistical rules or neural models (\textit{e.g.}, GNN) can be made afterwards, significantly reducing time consumption. This has led to the increasing popularity of these methods in industry. However, these methods also have some drawbacks. First, graph learning methods do not fully exploit the world knowledge learned and utilized by LLM, which is a natural shortcoming of such plug-and-play methods. And LLMs have the potential to introduce extraneous noise in both dimensions (topology and feature) of topology augmentation, which may consequently give rise to certain biases. Furthermore, they have poor scalability, as the main body of these methods are shallow GNNs whose model depth is affected by the over-smoothing problem.

\section{LLM-Graph Harmonization}

Graph-augmented LLM methods leverage external graph structures to enhance LLMs but often suffer from inefficiencies in real-time adaptability and increased computational overhead. Conversely, LLM-augmented graph methods attempt to incorporate LLMs into graph-based learning but struggle with scalability and effective knowledge utilization. To address these limitations, this section introduces a novel framework that optimally balances computational efficiency, adaptability, and reasoning capabilities.
\begin{figure}[t]
    \centering
    \includegraphics[width=\linewidth]{figures/LLM-Graph_Harmonization.pdf}
    \caption{The illustration of LLM-graph harmonization methods:  
    a) \textbf{Embedding Fusion}, where LLM-derived semantic embeddings and GNN-learned structural embeddings are combined into a unified representation space through fusion mechanisms such as concatenation or attention-based integration;  
    b) \textbf{Embedding Alignment}, where embeddings from both modalities are mapped into a shared space using techniques like contrastive learning or MLP-based transformation to enhance consistency and coherence.  }
    \label{fig:LLM-graph harmonization}
\end{figure}

LLMs excel in capturing rich semantic information from unstructured textual data (e.g., item descriptions and user attributes), while GNNs are adept at modeling the topological structure of graphs (e.g., user-item interactions, social connections). As shown in Figure~\ref{fig:LLM-graph harmonization}, harmonizing these two paradigms effectively can significantly enhance recommendation performance. Existing methods can be broadly categorized into two mainstream strategies: \textbf{embedding fusion} and \textbf{embedding alignment}, based on transformations of embeddings.
% Below, we discuss these approaches in detail, highlighting their methodologies and contributions.

\subsection{Embedding Fusion}

The embedding fusion approach aims to combine LLM-derived textual representations with graph-learned structural embeddings, creating a unified feature space that leverages complementary information. This strategy emphasizes the synergy between textual semantics and graph-based connectivity.


% \begin{equation}
% \mathbf{z} = g_{\text{fusion}}(\mathbf{e}_{\text{LLM}}, \mathbf{e}_{\text{GL}})
% \end{equation}
% where $\mathbf{e}_{\text{LLM}} \in \mathbb{R}^{d_{\text{LLM}}}$ represents the semantic embeddings generated by the LLM (e.g., item captions, user-generated content), $\mathbf{e}_{\text{GL}} \in \mathbb{R}^{d_{\text{GL}}}$ denotes the structural embeddings produced by the graph learning method (e.g., user-item interaction embeddings), and $g_{\text{fusion}}$ is a fusion function (e.g., concatenation, attention-based fusion) that combines these embeddings into a shared latent space $\mathbf{z} \in \mathbb{R}^{d_{\text{fusion}}}$.

A notable framework in this domain is DynLLM~\cite{zhao2024dynllm}, which incorporates graph structures into LLMs through dynamic memory-enhanced fusion. DynLLM addresses the limitation of static embeddings by using a dual-flow interaction mechanism: one flow learns from GNN-updated dynamic embeddings reflecting user-item interactions, while the other adapts LLM-generated embeddings based on real-time textual content. These embeddings are fused in a shared latent space, enhancing both semantic and structural understanding. Such a dynamic fusion approach not only captures the temporal evolution of recommendation data but also integrates high-quality textual semantics, leading to more accurate and adaptive recommendations. Another example is LKPNR~\cite{runfeng2023lkpnr}, which combines LLMs with knowledge graphs to enhance personalized news recommendation. LKPNR leverages the semantic richness of LLMs to generate high-quality news representations and uses knowledge graphs to capture the relational structure of news entities. By integrating these modalities, LKPNR effectively addresses the long-tail problem in news recommendation.

The embedding fusion paradigm is particularly effective because it exploits the contextual richness of LLMs alongside the relational structures captured by GNNs. By fusing these modalities, models like DynLLM and LKPNR enable direct, dynamic, and efficient utilization of both textual and graph data, significantly improving representation learning in recommendation scenarios.

\subsection{Embedding Alignment}

Embedding alignment takes a different route by focusing on reconciling the heterogeneity between LLM-generated textual embeddings and GNN-learned structural representations. This strategy ensures that embeddings from both modalities can operate coherently within a unified representational space, reducing information loss and noise. The refined embeddings can provide more valuable information for recommendations.

% \begin{equation}
% \mathcal{L}_{\text{align}} = \sum_{i=1}^N \mathcal{L}\left(f(\mathbf{e}_{\text{LLM}}^{(i)}, \mathbf{e}_{\text{GNN}}^{(i)})\right)
% \end{equation}
% where $\mathbf{e}_{\text{LLM}}^{(i)}$ and $\mathbf{e}_{\text{GNN}}^{(i)}$ denote the LLM and GNN embeddings for instance $i$, respectively. The function $f$ represents a general alignment function, which can be any measure or network (such as contrastive loss, MLP-based mapping, or others) that compares the embeddings from both modalities. The function $\mathcal{L}$ is a loss function applied to the output of $f$, aiming to promote alignment between the two embeddings in the shared space.

The DALR framework~\cite{peng2024denoising} serves as a representative example, where structural embeddings from GNNs (e.g., user-item graph representations) and semantic embeddings from LLMs (e.g., product descriptions, user reviews) are aligned through contrastive learning paradigms. This alignment mitigates the semantic gaps and noise introduced by the inherently different data sources. Similarly, methods such as LLMRec~\cite{wei2024llmrec} and RLMRec~\cite{ren2024representation} also adopt multimodal alignment techniques, such as contrastive learning and MLP-based alignment, to unify embeddings from diverse modalities. For instance, LLMRec employs a denoised data robustification mechanism to enhance the reliability of augmented recommendation data, while RLMRec leverages contrastive alignment strategies to bridge the semantic space of LLMs with the collaborative relational signals from GNNs, thereby improving the overall quality of the learned representations.

Embedding alignment excels in its ability to unify heterogeneous modalities, ensuring consistent representation learning that facilitates better downstream recommendation tasks, such as personalized ranking or user preference clustering.

\subsection{Discussion}

The two approaches offer distinct advantages in integrating LLMs with graph learning for recommender systems. Embedding fusion directly combines textual semantics and structural relationships, leveraging their complementarity to enhance representation learning and personalization. Dynamic fusion methods, such as DynLLM, further enable real-time adaptation to evolving user preferences. Embedding alignment, in contrast, ensures coherence between textual and structural embeddings by mapping them into a shared space, mitigating inconsistencies. Methods like DALR leverages various contrastive learning approaches to enhance alignment robustness. However, embedding fusion may introduce redundant or conflicting information, and increase computational costs, while embedding alignment which is sensitive to noise depends on high-quality training data.
In these methods, the LLM primarily serves as an encoder, and due to the constraints of the scenario, it cannot fully utilize its contextual understanding and language generation capabilities.
% Moreover, in these methods, the involvement of the LLM occurs at a relatively later stage, making it difficult to ensure whether noise from the original data has been introduced into the graph learning process.
% Future work could explore hybrid strategies to balance fusion and alignment, maximizing their strengths.


\section{Challenges and Future Directions}

GFMs have demonstrated great potential in recommender systems by incorporating graph structural information  with external world knowledge of LLM. However, several challenges hinder their widespread adoption and effectiveness.

\paragraph{High Computational Cost and Scalability Issues.}
Existing GFM-based RS require substantial computational resources, posing challenges for large-scale deployment \cite{zhai2024actions}. The integration of graph-based reasoning and LLM inference results in high memory consumption and slow inference speed, particularly when processing dense user-item graphs or generating personalized recommendations in real-time \cite{10.1145/3640457.3688161}. Unlike traditional recommendation models, which can be efficiently pruned or quantized, GFMs face unique scalability constraints due to their reliance on long-range graph dependencies and LLM-generated representations. Addressing these limitations requires advancements in efficient model compression strategies tailored for graph-enhanced LLMs, and adaptive graph sparsification techniques to maintain performance while reducing overhead.

\paragraph{Robustness Against Noisy and Adversarial Data.}
Real-world user interactions are inherently noisy, exhibiting short-term fluctuations, incomplete preferences, and adversarial perturbations \cite{zhang2023robust}. Traditional recommendation models rely on explicit feedback signals, making them susceptible to biased or manipulated data. In contrast, GFMs integrate graph-based user-item relationships and LLM-generated contextual representations, which introduces additional sources of noise from both structured and unstructured data. Ensuring robustness requires advancements in self-supervised denoising techniques, adversarial training tailored for multimodal representations, and uncertainty-aware modeling to mitigate the impact of unreliable signals while preserving recommendation accuracy.

\paragraph{Multi-Modal Information Fusion.}
Modern recommendation scenarios involve a diverse range of data modalities, including text, structured graphs, images, audio, and video \cite{tao2020mgat}. While existing GFMs primarily focus on textual and structural embeddings, effectively incorporating rich multi-modal signals remains an open challenge. Different modalities exhibit varying levels of granularity, semantic gaps, and computational costs, making seamless integration nontrivial. Future research should explore adaptive fusion frameworks, cross-modal alignment mechanisms, and lightweight multi-modal representation learning to balance efficiency and accuracy in large-scale recommender systems.

\paragraph{Lack of End-to-End Optimization.}  
% Many existing methods rely on a multi-stage optimization pipeline, where different components—such as graph-based prompt construction, LLM inference, and downstream task adaptation—are optimized separately. For example, in graph-enhanced prompting strategies, the first stage typically focuses on designing optimal prompt generation rules, which are then fixed when fine-tuning the LLM for response generation. Such staged optimization does not necessarily lead to a globally optimal solution, as errors or biases introduced in earlier stages may propagate through the system. Future research should explore unified training frameworks that allow joint optimization across all components, ensuring a more synergistic integration of graph structures and LLM reasoning.
The concept of end-to-end recommender system is not unfamiliar. When deep learning was introduced into the field of recommendation, a process encapsulation was essentially performed~\cite{covington2016deep}. However, the early neural model RS have gradually fallen behind the times. The process of such RS can be roughly divided into three stages: matching, ranking, and re-ranking~\cite{gao2023survey}. In reality, this process is often more refined in industrial applications. Such a meticulous process naturally results in better recommendation performance. However, multi-stage model optimization requires a significant investment of time and manpower. Contrarily, an end-to-end generative RS, different from the one mentioned above, encapsulates multiple stages together for optimization. This significantly reduces complexity and can potentially lead to better performance. Gradually, similar endeavors are being pursued in the industrial field. HSTU~\cite{zhai2024actions} simplifies the internal structure of the LLM and fully implements it through serial modeling. Moreover, \cite{wang2024llm} takes into account both structural and textual information. Such an integrated generative recommendation that combines matching and ranking may likely be a hotspot in the future.

\paragraph{Knowledge-Preference Gap.}
While GFMs leverage external knowledge to alleviate data sparsity, a fundamental misalignment persists between globally pre-trained world knowledge and personalized user preferences \cite{10.1145/3640457.3688161}. Unlike embedding alignment, which focuses on bridging modality gaps (e.g., between graph structures and textual representations), this discrepancy stems from differences in how LLMs interpret knowledge and how users express preferences. For instance, LLM-based recommendation models may naturally generate factually coherent but overly neutral item descriptions, whereas users often respond more favorably to engaging or sensationalized content (e.g., ``Shocking! You won’t believe this..."). Addressing this challenge requires advancing preference-aware knowledge adaptation, dynamic refinement techniques, and contrastive learning strategies tailored to user-specific interests.

\paragraph{Summary}
Our findings provide significant insights into the influence of correctness, explanations, and refinement on evaluation accuracy and user trust in AI-based planners. 
In particular, the findings are three-fold: 
(1) The \textbf{correctness} of the generated plans is the most significant factor that impacts the evaluation accuracy and user trust in the planners. As the PDDL solver is more capable of generating correct plans, it achieves the highest evaluation accuracy and trust. 
(2) The \textbf{explanation} component of the LLM planner improves evaluation accuracy, as LLM+Expl achieves higher accuracy than LLM alone. Despite this improvement, LLM+Expl minimally impacts user trust. However, alternative explanation methods may influence user trust differently from the manually generated explanations used in our approach.
% On the other hand, explanations may help refine the trust of the planner to a more appropriate level by indicating planner shortcomings.
(3) The \textbf{refinement} procedure in the LLM planner does not lead to a significant improvement in evaluation accuracy; however, it exhibits a positive influence on user trust that may indicate an overtrust in some situations.
% This finding is aligned with prior works showing that iterative refinements based on user feedback would increase user trust~\cite{kunkel2019let, sebo2019don}.
Finally, the propensity-to-trust analysis identifies correctness as the primary determinant of user trust, whereas explanations provided limited improvement in scenarios where the planner's accuracy is diminished.

% In conclusion, our results indicate that the planner's correctness is the dominant factor for both evaluation accuracy and user trust. Therefore, selecting high-quality training data and optimizing the training procedure of AI-based planners to improve planning correctness is the top priority. Once the AI planner achieves a similar correctness level to traditional graph-search planners, strengthening its capability to explain and refine plans will further improve user trust compared to traditional planners.

\paragraph{Future Research} Future steps in this research include expanding user studies with larger sample sizes to improve generalizability and including additional planning problems per session for a more comprehensive evaluation. Next, we will explore alternative methods for generating plan explanations beyond manual creation to identify approaches that more effectively enhance user trust. 
Additionally, we will examine user trust by employing multiple LLM-based planners with varying levels of planning accuracy to better understand the interplay between planning correctness and user trust. 
Furthermore, we aim to enable real-time user-planner interaction, allowing users to provide feedback and refine plans collaboratively, thereby fostering a more dynamic and user-centric planning process.


% \section{Style and Format}

% \LaTeX{} and Word style files that implement these instructions
% can be retrieved electronically. (See Section~\ref{stylefiles} for
% instructions on how to obtain these files.)

% \subsection{Layout}

% Print manuscripts two columns to a page, in the manner in which these
% instructions are printed. The exact dimensions for pages are:
% \begin{itemize}
%     \item left and right margins: .75$''$
%     \item column width: 3.375$''$
%     \item gap between columns: .25$''$
%     \item top margin---first page: 1.375$''$
%     \item top margin---other pages: .75$''$
%     \item bottom margin: 1.25$''$
%     \item column height---first page: 6.625$''$
%     \item column height---other pages: 9$''$
% \end{itemize}

% All measurements assume an 8-1/2$''$ $\times$ 11$''$ page size. For
% A4-size paper, use the given top and left margins, column width,
% height, and gap, and modify the bottom and right margins as necessary.

% \subsection{Format of Electronic Manuscript}

% For the production of the electronic manuscript, you must use Adobe's
% {\em Portable Document Format} (PDF). A PDF file can be generated, for
% instance, on Unix systems using {\tt ps2pdf} or on Windows systems
% using Adobe's Distiller. There is also a website with free software
% and conversion services: \url{http://www.ps2pdf.com}. For reasons of
% uniformity, use of Adobe's {\em Times Roman} font is strongly suggested.
% In \LaTeX2e{} this is accomplished by writing
% \begin{quote}
%     \mbox{\tt $\backslash$usepackage\{times\}}
% \end{quote}
% in the preamble.\footnote{You may want to also use the package {\tt
%             latexsym}, which defines all symbols known from the old \LaTeX{}
%     version.}

% Additionally, it is of utmost importance to specify the {\bf
%         letter} format (corresponding to 8-1/2$''$ $\times$ 11$''$) when
% formatting the paper. When working with {\tt dvips}, for instance, one
% should specify {\tt -t letter}.

% \subsection{Papers Submitted for Review vs. Camera-ready Papers}
% In this document, we distinguish between papers submitted for review (henceforth, submissions) and camera-ready versions, i.e., accepted papers that will be included in the conference proceedings. The present document provides information to be used by both types of papers (submissions / camera-ready). There are relevant differences between the two versions. Find them next.

% \subsubsection{Anonymity}
% For the main track and some of the special tracks, submissions must be anonymous; for other special tracks they must be non-anonymous. The camera-ready versions for all tracks are non-anonymous. When preparing your submission, please check the track-specific instructions regarding anonymity.

% \subsubsection{Submissions}
% The following instructions apply to submissions:
% \begin{itemize}
% \item If your track requires submissions to be anonymous, they must be fully anonymized as discussed in the Modifications for Blind Review subsection below; in this case, Acknowledgements and Contribution Statement sections are not allowed.

% \item If your track requires non-anonymous submissions, you should provide all author information at the time of submission, just as for camera-ready papers (see below); Acknowledgements and Contribution Statement sections are allowed, but optional.

% \item Submissions must include line numbers to facilitate feedback in the review process . Enable line numbers by uncommenting the command {\tt \textbackslash{}linenumbers} in the preamble.

% \item The limit on the number of  content pages is \emph{strict}. All papers exceeding the limits will be desk rejected.
% \end{itemize}

% \subsubsection{Camera-Ready Papers}
% The following instructions apply to camera-ready papers:

% \begin{itemize}
% \item Authors and affiliations are mandatory. Explicit self-references are allowed. It is strictly forbidden to add authors not declared at submission time.

% \item Acknowledgements and Contribution Statement sections are allowed, but optional.

% \item Line numbering must be disabled. To achieve this, comment or disable {\tt \textbackslash{}linenumbers} in the preamble.

% \item For some of the tracks, you can exceed the page limit by purchasing extra pages.
% \end{itemize}

% \subsection{Title and Author Information}

% Center the title on the entire width of the page in a 14-point bold
% font. The title must be capitalized using Title Case. For non-anonymous papers, author names and affiliations should appear below the title. Center author name(s) in 12-point bold font. On the following line(s) place the affiliations.

% \subsubsection{Author Names}

% Each author name must be followed by:
% \begin{itemize}
%     \item A newline {\tt \textbackslash{}\textbackslash{}} command for the last author.
%     \item An {\tt \textbackslash{}And} command for the second to last author.
%     \item An {\tt \textbackslash{}and} command for the other authors.
% \end{itemize}

% \subsubsection{Affiliations}

% After all authors, start the affiliations section by using the {\tt \textbackslash{}affiliations} command.
% Each affiliation must be terminated by a newline {\tt \textbackslash{}\textbackslash{}} command. Make sure that you include the newline after the last affiliation, too.

% \subsubsection{Mapping Authors to Affiliations}

% If some scenarios, the affiliation of each author is clear without any further indication (\emph{e.g.}, all authors share the same affiliation, all authors have a single and different affiliation). In these situations you don't need to do anything special.

% In more complex scenarios you will have to clearly indicate the affiliation(s) for each author. This is done by using numeric math superscripts {\tt \$\{\^{}$i,j, \ldots$\}\$}. You must use numbers, not symbols, because those are reserved for footnotes in this section (should you need them). Check the authors definition in this example for reference.

% \subsubsection{Emails}

% This section is optional, and can be omitted entirely if you prefer. If you want to include e-mails, you should either include all authors' e-mails or just the contact author(s)' ones.

% Start the e-mails section with the {\tt \textbackslash{}emails} command. After that, write all emails you want to include separated by a comma and a space, following the order used for the authors (\emph{i.e.}, the first e-mail should correspond to the first author, the second e-mail to the second author and so on).

% You may ``contract" consecutive e-mails on the same domain as shown in this example (write the users' part within curly brackets, followed by the domain name). Only e-mails of the exact same domain may be contracted. For instance, you cannot contract ``person@example.com" and ``other@test.example.com" because the domains are different.


% \subsubsection{Modifications for Blind Review}
% When submitting to a track that requires anonymous submissions,
% in order to make blind reviewing possible, authors must omit their
% names, affiliations and e-mails. In place
% of names, affiliations and e-mails, you can optionally provide the submission number and/or
% a list of content areas. When referring to one's own work,
% use the third person rather than the
% first person. For example, say, ``Previously,
% Gottlob~\shortcite{gottlob:nonmon} has shown that\ldots'', rather
% than, ``In our previous work~\cite{gottlob:nonmon}, we have shown
% that\ldots'' Try to avoid including any information in the body of the
% paper or references that would identify the authors or their
% institutions, such as acknowledgements. Such information can be added post-acceptance to be included in the camera-ready
% version.
% Please also make sure that your paper metadata does not reveal
% the authors' identities.

% \subsection{Abstract}

% Place the abstract at the beginning of the first column 3$''$ from the
% top of the page, unless that does not leave enough room for the title
% and author information. Use a slightly smaller width than in the body
% of the paper. Head the abstract with ``Abstract'' centered above the
% body of the abstract in a 12-point bold font. The body of the abstract
% should be in the same font as the body of the paper.

% The abstract should be a concise, one-paragraph summary describing the
% general thesis and conclusion of your paper. A reader should be able
% to learn the purpose of the paper and the reason for its importance
% from the abstract. The abstract should be no more than 200 words long.

% \subsection{Text}

% The main body of the text immediately follows the abstract. Use
% 10-point type in a clear, readable font with 1-point leading (10 on
% 11).

% Indent when starting a new paragraph, except after major headings.

% \subsection{Headings and Sections}

% When necessary, headings should be used to separate major sections of
% your paper. (These instructions use many headings to demonstrate their
% appearance; your paper should have fewer headings.). All headings should be capitalized using Title Case.

% \subsubsection{Section Headings}

% Print section headings in 12-point bold type in the style shown in
% these instructions. Leave a blank space of approximately 10 points
% above and 4 points below section headings.  Number sections with
% Arabic numerals.

% \subsubsection{Subsection Headings}

% Print subsection headings in 11-point bold type. Leave a blank space
% of approximately 8 points above and 3 points below subsection
% headings. Number subsections with the section number and the
% subsection number (in Arabic numerals) separated by a
% period.

% \subsubsection{Subsubsection Headings}

% Print subsubsection headings in 10-point bold type. Leave a blank
% space of approximately 6 points above subsubsection headings. Do not
% number subsubsections.

% \paragraph{Titled paragraphs.} You should use titled paragraphs if and
% only if the title covers exactly one paragraph. Such paragraphs should be
% separated from the preceding content by at least 3pt, and no more than
% 6pt. The title should be in 10pt bold font and to end with a period.
% After that, a 1em horizontal space should follow the title before
% the paragraph's text.

% In \LaTeX{} titled paragraphs should be typeset using
% \begin{quote}
%     {\tt \textbackslash{}paragraph\{Title.\} text} .
% \end{quote}

% \subsection{Special Sections}

% \subsubsection{Appendices}
% You may move some of the contents of the paper into one or more appendices that appear after the main content, but before references. These appendices count towards the page limit and are distinct from the supplementary material that can be submitted separately through CMT. Such appendices are useful if you would like to include highly technical material (such as a lengthy calculation) that will disrupt the flow of the paper. They can be included both in papers submitted for review and in camera-ready versions; in the latter case, they will be included in the proceedings (whereas the supplementary materials will not be included in the proceedings).
% Appendices are optional. Appendices must appear after the main content.
% Appendix sections must use letters instead of Arabic numerals. In \LaTeX,  you can use the {\tt \textbackslash{}appendix} command to achieve this followed by  {\tt \textbackslash section\{Appendix\}} for your appendix sections.

% \subsubsection{Ethical Statement}

% Ethical Statement is optional. You may include an Ethical Statement to discuss  the ethical aspects and implications of your research. The section should be titled \emph{Ethical Statement} and be typeset like any regular section but without being numbered. This section may be placed on the References pages.

% Use
% \begin{quote}
%     {\tt \textbackslash{}section*\{Ethical Statement\}}
% \end{quote}

% \subsubsection{Acknowledgements}

% Acknowledgements are optional. In the camera-ready version you may include an unnumbered acknowledgments section, including acknowledgments of help from colleagues, financial support, and permission to publish. This is not allowed in the anonymous submission. If present, acknowledgements must be in a dedicated, unnumbered section appearing after all regular sections but before references.  This section may be placed on the References pages.

% Use
% \begin{quote}
%     {\tt \textbackslash{}section*\{Acknowledgements\}}
% \end{quote}
% to typeset the acknowledgements section in \LaTeX{}.


% \subsubsection{Contribution Statement}

% Contribution Statement is optional. In the camera-ready version you may include an unnumbered Contribution Statement section, explicitly describing the contribution of each of the co-authors to the paper. This is not allowed in the anonymous submission. If present, Contribution Statement must be in a dedicated, unnumbered section appearing after all regular sections but before references.  This section may be placed on the References pages.

% Use
% \begin{quote}
%     {\tt \textbackslash{}section*\{Contribution Statement\}}
% \end{quote}
% to typeset the Contribution Statement section in \LaTeX{}.

% \subsubsection{References}

% The references section is headed ``References'', printed in the same
% style as a section heading but without a number. A sample list of
% references is given at the end of these instructions. Use a consistent
% format for references. The reference list should not include publicly unavailable work.

% \subsubsection{Order of Sections}
% Sections should be arranged in the following order:
% \begin{enumerate}
%     \item Main content sections (numbered)
%     \item Appendices (optional, numbered using capital letters)
%     \item Ethical statement (optional, unnumbered)
%     \item Acknowledgements (optional, unnumbered)
%     \item Contribution statement (optional, unnumbered)
%     \item References (required, unnumbered)
% \end{enumerate}

% \subsection{Citations}

% Citations within the text should include the author's last name and
% the year of publication, for example~\cite{gottlob:nonmon}.  Append
% lowercase letters to the year in cases of ambiguity.  Treat multiple
% authors as in the following examples:~\cite{abelson-et-al:scheme}
% or~\cite{bgf:Lixto} (for more than two authors) and
% \cite{brachman-schmolze:kl-one} (for two authors).  If the author
% portion of a citation is obvious, omit it, e.g.,
% Nebel~\shortcite{nebel:jair-2000}.  Collapse multiple citations as
% follows:~\cite{gls:hypertrees,levesque:functional-foundations}.
% \nocite{abelson-et-al:scheme}
% \nocite{bgf:Lixto}
% \nocite{brachman-schmolze:kl-one}
% \nocite{gottlob:nonmon}
% \nocite{gls:hypertrees}
% \nocite{levesque:functional-foundations}
% \nocite{levesque:belief}
% \nocite{nebel:jair-2000}

% \subsection{Footnotes}

% Place footnotes at the bottom of the page in a 9-point font.  Refer to
% them with superscript numbers.\footnote{This is how your footnotes
%     should appear.} Separate them from the text by a short
% line.\footnote{Note the line separating these footnotes from the
%     text.} Avoid footnotes as much as possible; they interrupt the flow of
% the text.

% \section{Illustrations}

% Place all illustrations (figures, drawings, tables, and photographs)
% throughout the paper at the places where they are first discussed,
% rather than at the end of the paper.

% They should be floated to the top (preferred) or bottom of the page,
% unless they are an integral part
% of your narrative flow. When placed at the bottom or top of
% a page, illustrations may run across both columns, but not when they
% appear inline.

% Illustrations must be rendered electronically or scanned and placed
% directly in your document. They should be cropped outside \LaTeX{},
% otherwise portions of the image could reappear during the post-processing of your paper.
% When possible, generate your illustrations in a vector format.
% When using bitmaps, please use 300dpi resolution at least.
% All illustrations should be understandable when printed in black and
% white, albeit you can use colors to enhance them. Line weights should
% be 1/2-point or thicker. Avoid screens and superimposing type on
% patterns, as these effects may not reproduce well.

% Number illustrations sequentially. Use references of the following
% form: Figure 1, Table 2, etc. Place illustration numbers and captions
% under illustrations. Leave a margin of 1/4-inch around the area
% covered by the illustration and caption.  Use 9-point type for
% captions, labels, and other text in illustrations. Captions should always appear below the illustration.

% \section{Tables}

% Tables are treated as illustrations containing data. Therefore, they should also appear floated to the top (preferably) or bottom of the page, and with the captions below them.

% \begin{table}
%     \centering
%     \begin{tabular}{lll}
%         \hline
%         Scenario  & $\delta$ & Runtime \\
%         \hline
%         Paris     & 0.1s     & 13.65ms \\
%         Paris     & 0.2s     & 0.01ms  \\
%         New York  & 0.1s     & 92.50ms \\
%         Singapore & 0.1s     & 33.33ms \\
%         Singapore & 0.2s     & 23.01ms \\
%         \hline
%     \end{tabular}
%     \caption{Latex default table}
%     \label{tab:plain}
% \end{table}

% \begin{table}
%     \centering
%     \begin{tabular}{lrr}
%         \toprule
%         Scenario  & $\delta$ (s) & Runtime (ms) \\
%         \midrule
%         Paris     & 0.1          & 13.65        \\
%                   & 0.2          & 0.01         \\
%         New York  & 0.1          & 92.50        \\
%         Singapore & 0.1          & 33.33        \\
%                   & 0.2          & 23.01        \\
%         \bottomrule
%     \end{tabular}
%     \caption{Booktabs table}
%     \label{tab:booktabs}
% \end{table}

% If you are using \LaTeX, you should use the {\tt booktabs} package, because it produces tables that are better than the standard ones. Compare Tables~\ref{tab:plain} and~\ref{tab:booktabs}. The latter is clearly more readable for three reasons:

% \begin{enumerate}
%     \item The styling is better thanks to using the {\tt booktabs} rulers instead of the default ones.
%     \item Numeric columns are right-aligned, making it easier to compare the numbers. Make sure to also right-align the corresponding headers, and to use the same precision for all numbers.
%     \item We avoid unnecessary repetition, both between lines (no need to repeat the scenario name in this case) as well as in the content (units can be shown in the column header).
% \end{enumerate}

% \section{Formulas}

% IJCAI's two-column format makes it difficult to typeset long formulas. A usual temptation is to reduce the size of the formula by using the {\tt small} or {\tt tiny} sizes. This doesn't work correctly with the current \LaTeX{} versions, breaking the line spacing of the preceding paragraphs and title, as well as the equation number sizes. The following equation demonstrates the effects (notice that this entire paragraph looks badly formatted, and the line numbers no longer match the text):
% %
% \begin{tiny}
%     \begin{equation}
%         x = \prod_{i=1}^n \sum_{j=1}^n j_i + \prod_{i=1}^n \sum_{j=1}^n i_j + \prod_{i=1}^n \sum_{j=1}^n j_i + \prod_{i=1}^n \sum_{j=1}^n i_j + \prod_{i=1}^n \sum_{j=1}^n j_i
%     \end{equation}
% \end{tiny}%

% Reducing formula sizes this way is strictly forbidden. We {\bf strongly} recommend authors to split formulas in multiple lines when they don't fit in a single line. This is the easiest approach to typeset those formulas and provides the most readable output%
% %
% \begin{align}
%     x = & \prod_{i=1}^n \sum_{j=1}^n j_i + \prod_{i=1}^n \sum_{j=1}^n i_j + \prod_{i=1}^n \sum_{j=1}^n j_i + \prod_{i=1}^n \sum_{j=1}^n i_j + \nonumber \\
%     +   & \prod_{i=1}^n \sum_{j=1}^n j_i.
% \end{align}%

% If a line is just slightly longer than the column width, you may use the {\tt resizebox} environment on that equation. The result looks better and doesn't interfere with the paragraph's line spacing: %
% \begin{equation}
%     \resizebox{.91\linewidth}{!}{$
%             \displaystyle
%             x = \prod_{i=1}^n \sum_{j=1}^n j_i + \prod_{i=1}^n \sum_{j=1}^n i_j + \prod_{i=1}^n \sum_{j=1}^n j_i + \prod_{i=1}^n \sum_{j=1}^n i_j + \prod_{i=1}^n \sum_{j=1}^n j_i
%         $}.
% \end{equation}%

% This last solution may have to be adapted if you use different equation environments, but it can generally be made to work. Please notice that in any case:

% \begin{itemize}
%     \item Equation numbers must be in the same font and size as the main text (10pt).
%     \item Your formula's main symbols should not be smaller than {\small small} text (9pt).
% \end{itemize}

% For instance, the formula
% %
% \begin{equation}
%     \resizebox{.91\linewidth}{!}{$
%             \displaystyle
%             x = \prod_{i=1}^n \sum_{j=1}^n j_i + \prod_{i=1}^n \sum_{j=1}^n i_j + \prod_{i=1}^n \sum_{j=1}^n j_i + \prod_{i=1}^n \sum_{j=1}^n i_j + \prod_{i=1}^n \sum_{j=1}^n j_i + \prod_{i=1}^n \sum_{j=1}^n i_j
%         $}
% \end{equation}
% %
% would not be acceptable because the text is too small.

% \section{Examples, Definitions, Theorems and Similar}

% Examples, definitions, theorems, corollaries and similar must be written in their own paragraph. The paragraph must be separated by at least 2pt and no more than 5pt from the preceding and succeeding paragraphs. They must begin with the kind of item written in 10pt bold font followed by their number (e.g.: {\bf Theorem 1}),
% optionally followed by a title/summary between parentheses in non-bold font and ended with a period (in bold).
% After that the main body of the item follows, written in 10 pt italics font (see below for examples).

% In \LaTeX{} we strongly recommend that you define environments for your examples, definitions, propositions, lemmas, corollaries and similar. This can be done in your \LaTeX{} preamble using \texttt{\textbackslash{newtheorem}} -- see the source of this document for examples. Numbering for these items must be global, not per-section (e.g.: Theorem 1 instead of Theorem 6.1).

% \begin{example}[How to write an example]
%     Examples should be written using the example environment defined in this template.
% \end{example}

% \begin{theorem}
%     This is an example of an untitled theorem.
% \end{theorem}

% You may also include a title or description using these environments as shown in the following theorem.

% \begin{theorem}[A titled theorem]
%     This is an example of a titled theorem.
% \end{theorem}

% \section{Proofs}

% Proofs must be written in their own paragraph(s) separated by at least 2pt and no more than 5pt from the preceding and succeeding paragraphs. Proof paragraphs should start with the keyword ``Proof." in 10pt italics font. After that the proof follows in regular 10pt font. At the end of the proof, an unfilled square symbol (qed) marks the end of the proof.

% In \LaTeX{} proofs should be typeset using the \texttt{\textbackslash{proof}} environment.

% \begin{proof}
%     This paragraph is an example of how a proof looks like using the \texttt{\textbackslash{proof}} environment.
% \end{proof}


% \section{Algorithms and Listings}

% Algorithms and listings are a special kind of figures. Like all illustrations, they should appear floated to the top (preferably) or bottom of the page. However, their caption should appear in the header, left-justified and enclosed between horizontal lines, as shown in Algorithm~\ref{alg:algorithm}. The algorithm body should be terminated with another horizontal line. It is up to the authors to decide whether to show line numbers or not, how to format comments, etc.

% In \LaTeX{} algorithms may be typeset using the {\tt algorithm} and {\tt algorithmic} packages, but you can also use one of the many other packages for the task.

% \begin{algorithm}[tb]
%     \caption{Example algorithm}
%     \label{alg:algorithm}
%     \textbf{Input}: Your algorithm's input\\
%     \textbf{Parameter}: Optional list of parameters\\
%     \textbf{Output}: Your algorithm's output
%     \begin{algorithmic}[1] %[1] enables line numbers
%         \STATE Let $t=0$.
%         \WHILE{condition}
%         \STATE Do some action.
%         \IF {conditional}
%         \STATE Perform task A.
%         \ELSE
%         \STATE Perform task B.
%         \ENDIF
%         \ENDWHILE
%         \STATE \textbf{return} solution
%     \end{algorithmic}
% \end{algorithm}

% \section{\LaTeX{} and Word Style Files}\label{stylefiles}

% The \LaTeX{} and Word style files are available on the IJCAI--25
% website, \url{https://2025.ijcai.org/}.
% These style files implement the formatting instructions in this
% document.

% The \LaTeX{} files are {\tt ijcai25.sty} and {\tt ijcai25.tex}, and
% the Bib\TeX{} files are {\tt named.bst} and {\tt ijcai25.bib}. The
% \LaTeX{} style file is for version 2e of \LaTeX{}, and the Bib\TeX{}
% style file is for version 0.99c of Bib\TeX{} ({\em not} version
% 0.98i). .

% The Microsoft Word style file consists of a single file, {\tt
%         ijcai25.docx}. 
% %This template differs from the one used for IJCAI--23.

% These Microsoft Word and \LaTeX{} files contain the source of the
% present document and may serve as a formatting sample.

% Further information on using these styles for the preparation of
% papers for IJCAI--25 can be obtained by contacting {\tt
%         proceedings@ijcai.org}.

\appendix

% \section*{Ethical Statement}

% There are no ethical issues.
\clearpage
% \section*{Acknowledgments}

% The preparation of these instructions and the \LaTeX{} and Bib\TeX{}
% files that implement them was supported by Schlumberger Palo Alto
% Research, AT\&T Bell Laboratories, and Morgan Kaufmann Publishers.
% Preparation of the Microsoft Word file was supported by IJCAI.  An
% early version of this document was created by Shirley Jowell and Peter
% F. Patel-Schneider.  It was subsequently modified by Jennifer
% Ballentine, Thomas Dean, Bernhard Nebel, Daniel Pagenstecher,
% Kurt Steinkraus, Toby Walsh, Carles Sierra, Marc Pujol-Gonzalez,
% Francisco Cruz-Mencia and Edith Elkind.

% \newpage
%% The file named.bst is a bibliography style file for BibTeX 0.99c
\bibliographystyle{named}
\bibliography{ijcai25}

\end{document}

