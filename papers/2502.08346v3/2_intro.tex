\section{Introduction}
Recommender systems are essential components of contemporary digital landscape, enabling personalized services across a diverse range of fields, including e-commerce, social media, and entertainment \cite{zhang2023robust}. The data in RS generally consist of both structural information (\textit{e.g.}, user-item interactions) and textual information (\textit{e.g.}, user attributes and item descriptions).
% Based on technology and utilization of data, RS can be divided into three categories: content-based RS, collaborative filtering RS and hybrid filtering RS~\cite{adomavicius2005toward}.
\begin{figure}[ht]
    \centering
    \includegraphics[width=\linewidth]{figures/overview-yuanhao.pdf}
    \caption{An overview of GFM-based RS. Compared with GNN-based or LLM-based RS, GFM-based RS are positioned as integrating both approaches to create more comprehensive recommendations.}
    \label{fig:overview}
\end{figure}
With the rapid development of graph learning, GNN-based methods have emerged as an important technology in RS, which can further enhance the collaborative signals of collaborative filtering and extend the signals to higher-order structures and external knowledge~\cite{wu2022graph}. However, due to the inherent structural bias, they struggle to handle textual information.
This is where the powerful capabilities of large language models, which have made significant impacts in the field of natural language processing (NLP) and come into play in the realm of RS~\cite{yang2023palr,zhai2024actions}. Leveraging the advanced text capabilities of LLM, these methods efficiently capture user and item textual information while integrating world knowledge for improved recommendations. However, their reasoning limitations restrict the collaborative signals they can comprehend.
Inspired by the success of LLM in the NLP field, the graph domain has also been undergoing transformation, leading to the emergence of graph foundation models (GFMs)~\cite{liu2023towards}. By integrating GNN and LLM technologies, GFM-based RS can efficiently utilize data to align user preferences and make more precise recommendations with minimized bias, as depicted in Figure~\ref{fig:overview}. By appropriately integrating key information from both graph structures and text, GFM-based RS hold significant potential to emerge as a new paradigm in RS.
\section{Taxonomy of Research on SDN Software Security}\label{sec:tx}
To systematically extract insights and understand the current state-of-the-art in SDN software security, our SLR focuses on analyzing specific features of each publication. The primary outcome of this analysis is developing a novel, four-dimensional taxonomy. This taxonomy will structure the body of existing research and directly address the research questions outlined in Section\ref{sec:rqs}.
\subsection{Structure of the Taxonomy}
The proposed taxonomy is a four-dimensional model designed to categorize and analyze the research landscape on SDN software security. The dimensions and their defining features are as follows:
\begin{itemize}
    \item \textbf{Objectives (What):} This dimension identifies the security goals targeted by the research. Objectives include bug detection, fixing, localization, exploitation, mitigation, categorization, and hardening.
    %This dimension classifies the security goals research studies aim to achieve or address. Seven recurring objectives have been identified, including but not limited to bug detection, attack detection/prevention, and performance/scalability optimization.
    \item \textbf{Targets (Where):} This dimension focuses on the specific SDN software components subject to security analysis or investigation. Common targets encompass controllers, data planes, APIs, and SDN applications.
    \item \textbf{Methodology (How):}  This dimension categorizes the diverse research methodologies employed in the reviewed literature. These methodologies can be further subdivided into testing approaches (e.g., static analysis, dynamic testing), testing types (e.g., white box, black box, gray box), and specific analysis techniques (e.g., model checking, fuzzing, symbolic execution).
    \item \textbf{Representations (Which):} This dimension encompasses the various approaches used to represent and structure information related to the testing process. The choice of representation can significantly impact the efficiency, comprehensibility, and effectiveness of test execution.
\end{itemize}
Figure\ref{fig_txn} provides a visual representation of the proposed four-dimensional taxonomy.
\begin{figure}[ht!]
\centering
\begin{adjustbox}{width=\linewidth, center}
\includegraphics{Diagram2.png}
\end{adjustbox}
\caption{Taxonomy on Security of SDN Software.}
\label{fig_txn}
\end{figure}






The GFM-based RS effectively utilize the technological complementarity of GNN and LLM. GNNs struggle to model textual information, while the reasoning capabilities of LLMs do not support their comprehension of higher-order structural information. These two technologies complement each other's shortcomings in GFM, which emerges as a future opportunity in the field of recommendations. For example, LLMGR~\cite{guo2024integrating} injects the embeddings learned by GNN into the token embedding sequence of LLM, and adapts the GFM to the recommendation task through two-stage fine-tuning. LLMRG~\cite{wang2023enhancing} constructs inference graphs and divergence graphs based on user interaction history using LLM, which are then encoded by GNN for recommendations. DALR~\cite{peng2024denoising} aligns the embeddings encoded by GNN and those encoded by LLM in various ways, using the aligned embeddings for subsequent recommendations.

In this survey, we comprehensively investigate the relevant work of GFM-based RS, and provide a clear taxonomy based on the synergistic relationship between the graph and LLM in GFM: \textbf{Graph-augmented LLM}, \textbf{LLM-augmented graph} and \textbf{graph-LLM harmonization}.
Graph-augmented LLM methods can be viewed as utilizing the structural information of the graph to aid the knowledge obtained from LLM pre-training for recommendations. LLM-augmented graph methods, on the other hand, is led by the structural information of the graph, with the world knowledge of LLM serving as auxiliary information. Graph-LLM harmonization methods involve the equal transformation of these two types of information in the representation space. 

% As an evergreen topic in both academia and industry, numerous surveys have been conducted on recommender systems \cite{gao2023survey,wu2024survey}. The former provides a comprehensive review of graph-based recommender systems, representing traditional methodologies, while the latter offers an overview of LLM based recommender systems, representing a new paradigm. While the previous two surveys offer detailed insights into the respective technologies, they were unaware of the rapid development of GFM in the field of recommendations. Therefore, our survey offers a broader perspective for extensive research related to recommendations.

As an evergreen topic in both academia and industry, RS have been the subject of numerous surveys (e.g., \cite{gao2023survey,wu2024survey,liu2023towards,li2023survey}). \cite{gao2023survey,wu2024survey} focus on specific methodologies, such as GNN-based RS or the more recent LLM-based RS. \cite{li2023survey} concentrates on utilizing LLM to enhance graphs for tackling tasks related to graphs. However, the field is rapidly evolving with GFMs emerging as a crucial technique of the RS research. \cite{liu2023towards} systematically outlines the existing GFMs from the perspectives of pre-training and adaptation, while overlooking the recommendation which is one of the significant downstream tasks for GFM. This survey provides a timely and comprehensive overview that covers the landscape of GFM-based recommender systems.

The contributions of this survey can be summarized in the following aspects:\textbf{1)} \textit{Pioneering overview}: Our survey fills the blank in comprehensive work in the field of GFM-based RS. \textbf{2)} \textit{Clear taxonomy}: The comprehensive survey presents a well-structured taxonomy of GFM-based RS, allowing future work to be easily categorized within the corresponding branches. \textbf{3)} \textit{Promising outlook}: We present the challenges and future research directions in this field, which can serve as a valuable reference for research in this rapidly evolving area.
% This survey provides the first systematic review of graph foundation models for recommendation, offering several key contributions to the field:  

% 1. \textbf{A Novel Classification Framework}: We propose a comprehensive framework to categorize the GFM into three paradigms: Graph-augmented LLMs, LLM-augmented graphs, and LLM-graph harmonization in recommendation. This taxonomy provides a clear roadmap for understanding the field and guiding future research.

% 2. \textbf{Methodological Review}: We conduct an in-depth analysis of methodologies within each paradigm, discussing their theoretical foundations, design strategies, and real-world applications. Representative studies are examined to highlight their contributions to solving key recommendation challenges.

% 3. \textbf{Challenges and Future Directions}: Through meticulous literature synthesis, we unveil major challenges in this field, such as alignment of representations, computational efficiency, scalability, and integration complexity. Simultaneously, we spotlight prospective avenues for future research, including adaptive integration methods, cross-modal fusion, and efficient large-scale deployment strategies. Our analysis and insights aim to both address these current challenges and inspire future innovation, guiding researchers to unlock the full potential of integrating graph and LLM technologies.
