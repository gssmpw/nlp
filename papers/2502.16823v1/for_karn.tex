\subsubsection{Loihi-2}

Loihi 2, developed by Intel Labs as the successor to Loihi, is a cutting-edge neuromorphic research test chip introduced in 2022. It leverages an asynchronous spiking neural network (SNN) to enable adaptive, self-modifying, event-driven, fine-grained parallel computations, making it highly efficient for learning and inference processes. Fabricated on an Intel 4 process, this chip comprises 128 neuromorphic cores and is classified as a multi-core IC. One of its standout features is the inclusion of a unique programmable microcode learning engine, enabling on-chip SNN training. A board named Kapoho Point is built in a 4-inch by 4-inch form factor and houses eight Loihi 2 chips arranged in a 2-stacked configuration (double-sided PCB) with a 4x4 arrangement of chips. With this setup, Kapoho Point boasts an impressive 1,024 neuromorphic cores, 960,000,000 synapses, and 8.4 million neurons. In my research, I will be following the guidelines provided in documents that detail the setup and execution of the Lava extension for Loihi (lava-loihi) on the Intel Neuromorphic Research Cloud (vLab) systems \cite{WinNT, WinNT2}.


\subsubsection{Lava}

Lava is a freely accessible software library for formulating algorithms designed for neuromorphic computation. It simplifies the process of neuromorphic algorithm development by providing a user-friendly Python interface to construct the necessary components. Before deploying these algorithms on neuromorphic processors, such as the Intel Loihi 1/2, to exploit their speed and energy efficiencies, Lava enables users to run and test them on standard von Neumann hardware like CPUs. Moreover, Lava is built with adaptability, accommodating custom neuromorphic behavioral implementations and supporting new hardware backends. Lava can be bifurcated into two levels: One can utilize existing resources to build complex algorithms without a profound understanding of neuromorphic principles. Alternatively, Lava can be expanded readily to incorporate new behavior defined in Python and C for personalized requirements.

Lava can be modified and extended to work with third-party libraries like Nengo, ROS, YARP, etc. Additionally, users can profile the power and performance of workloads, visualize complex networks, or help convert floats to fixed points for low-precision devices like neuromorphic hardware, as illustrated in Figure \ref{Figure 8}.

At present, I am in possession of six distinct Python scripts, each pertaining to either NLP or Vision-based computations. My forthcoming initiative is to amalgamate these separate elements into a cohesive entity within the Lava framework, with an initial emphasis on integrating SNNs. Subsequent to this integration, my aim is to run these unified codes on the Loihi-2 neuromorphic hardware and analyze the ensuing results. To aid in the development of the vision-related components, I intend to leverage Intel's example as a blueprint \cite{WinNT4}. Notably, I am planning to transition the data set from MNIST to the more complex CIFAR-10. This step will increase the diversity and complexity of the data my model will train on, hence providing a more robust evaluation of the system's performance. I anticipate concluding these updates and modifications of at least SNNs to the relevant tables by the close of this week.
\vspace{5pt}
However, I foresee certain challenges in the design and integration of GNNs and Transformer architectures. Given the complexity of these designs, this phase may require more intensive efforts and specific expertise. To this end, I am considering reaching out to INRC Support to gain additional insights and assistance, which will be instrumental in overcoming these anticipated obstacles.



@misc{WinNT,
  title = {{Loihi-2}},
  howpublished = {\url{https://intel-ncl.atlassian.net/wiki/spaces/NAP/pages/1785856001/Get+Started+with+Intel+Loihi+2}},
  note = {Accessed: 2023-07-26}
}

@misc{WinNT2,
  title = {{Loihi-2}},
  howpublished = {\url{https://en.wikichip.org/wiki/intel/loihi_2}},
  note = {Accessed: 2023-07-26}
}


@misc{WinNT4,
  title = {{MNIST Digit Classification with Lava}},
  howpublished = {\url{https://lava-nc.org/lava/notebooks/end_to_end/tutorial01_mnist_digit_classification.html}},
  note = {Accessed: 2023-08-2}
}