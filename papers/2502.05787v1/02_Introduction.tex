%New addition: the latency is reduced to <3ns and the EDP/cell is reduced to 0.286EDP(fJ)/bit(under 64X64), harvesting 333X and 90.5X (both conservative) improvement for latency and EDP/cell respectively(comp. Nature comm.).


%\vspace{-1em}

\section{Introduction}
\label{sec:introduction}

In the era of advancing artificial intelligence, the computational demands on AI models are rapidly increasing. 
Training data volumes across various domains like computer vision (CV) \cite{CV}, natural language processing (NLP) \cite{neural}, and speech recognition \cite{speechrecognition} have surged, posing significant challenges to computing hardware and architectures, both at the edge and in data centers. The traditional von Neumann architecture, with its constant data movement between memory and processing units, exacerbates energy consumption and latency issues, intensifying the ``Memory Wall" problem.
To tackle this challenge, emerging computing paradigms, notably In-Memory Computing (IMC), have gained attention. 
IMC directly employs parallel data operations within the memory, enhancing core performance and efficiency while alleviating the ``Memory Wall" problem \cite{IMC2,  IMC3, yin2024deep, yin2024ferroelectric, yang2024energy, li2024febim}. 

Content Addressable Memory (CAM) emerges as a hardware solution of IMC, enabling parallel and efficient searching and similarity  measurement  within the memory. 
CAMs compare input data with all stored data simultaneously, and output the stored entry that matches with input or has the highest similarity to the input.
%avoiding power consumption and latency associated with data transfers. Therefore, CAMs 
Therefore, CAMs are viewed as a potential solution for accelerating 
%the processor-memory bottleneck. 
%Traditionally used in network routers and associative processors, CAMs are now gaining interest in 
various data-centric workloads like bioinformatics \cite{zhong2023asmcap,laguna2020seed}, machine learning \cite{2FeFETa, xu2024ferex, hu2021memory}, and neural language processing \cite{neural}.
%As a specialized solution to the Memory Wall problem, CAMs utilize the entire memory array to accelerate parallel search operations, showing great potential in today’s computing networks. 
Specifically, CAMs significantly speed up Hyperdimensional Computing (HDC), making this brain-inspired computing paradigm  efficient for tasks like image classification and speech recognition \cite{HDC1, HDC2, HDC3}. 
This effectiveness arises from CAMs' ability to transform sequential pattern matching into highly parallelizable computational tasks and simplify the complex distance measurements into Hamming distance \cite{kim2020geniehd}. 
The rapid search and matching capability of CAMs make them essential components in applications requiring efficient data access and retrieval.


Conventional  CMOS based CAM design consists of 10-16 transistors per cell, which results in large area overhead and high energy consumption \cite{16TCMOS}. 
%These problems severely limits their applicability \cite{16TCMOS} in various applications. 
To tackle the area and energy challenges,
%faced by CMOS CAM designs, 
researchers have proposed utilizing emerging non-volatile memory (NVM) devices to construct more compact and efficient CAM designs, as these  CAMs merge the storage and logic within the NVM devices, thus offering significant area and energy saving. 
%aiming to improve the  performance, area and energy efficiency. 
CAMs based on 2-terminal NVMs like resistive RAM (RRAM) \cite{li2021sapiens,Chang3t1r}, magnetic tunneling junction (MTJ) \cite{Matsunaga4t2mtj, zhuo2022design}, phase change memory (PCM) \cite{jing2t2r}, and 3-terminal ferroelectric field effect transistor (FeFET) \cite{2FeFET,4T2FeFET,1FeFET1R-transfer, yin2022ferroelectric, yin2023ultracompact, xu2023challenges, Huang2024, yin2020fecam, li2020scalable} have been explored. 
Among these devices, FeFETs stand out in constructing the compact and efficient CAM designs due to their unique hysteresis I-V characteristics, high current ON/OFF ratio, high off-state resistance, low write energy, and compatibility with CMOS technology \cite{Liu2022eva-cam}.  
While non-volatile storage can achieve high area efficiency and  mitigate the high energy consumption caused by CMOS technology, these CAMs still encounter limitations for data-intensive applications  due to their exact search functionality.
In the era of big data, as the amount of data for processing bursts and the chances of exact matching drop down,  these CAMs with limited array size fail to maintain the hardware utilization efficiency while consuming extra area and energy overheads.
%the application of approximate matching is increasingly common, offering potential hardware utilization efficiency improvements while maintaining an acceptable level of accuracy.
Many applications require approximate pattern search functions where entries with a similarity within a certain threshold distance to the search query are desired. 
%As a result, they fail to significantly enhance energy efficiency.
%\textbf{Secondly}, conventional CAMs only support exact match functions, returning the entry that precisely matches the search query. However, many applications require approximate pattern search functions where entries with a similarity within a certain threshold distance to the search query are desired. In the era of big data, the application of approximate matching is increasingly common, offering potential hardware utilization efficiency improvements while maintaining an acceptable level of accuracy.
%However, conventional CAM designs cannot satisfy the requirements of emerging applications primarily in two aspects. \textbf{Firstly}, conventional CAMs only support the exact match function, meaning only the stored entries that precisely match the search query can be returned. However, many applications require approximate pattern search functions where the stored entries do not necessarily match the search query exactly. 
%There are mainly two types of approximate match functions for CAM. Threshold match 
%Instead, entries with a similarity within a certain threshold distance to the search query
%As long as the similarity between the search query and stored entries is within a certain threshold, the entries 
%are desired to be returned. In the era centered around big data, the application of approximate matching is becoming increasingly widespread. This match pattern offers the potential to improve hardware utilization efficiency while maintaining an acceptable level of accuracy. 
%\textbf{Secondly}, the conventional CAMs are implemented by CMOS technology which requires 10-16 transistors per circuit cell, leading to large cell area and high energy consumption. The substantial overhead of CMOS CAM limits its applicability \cite{16TCMOS}. 
To address the challenge of limited CAM utilization efficiency, 
various CAM designs implementing approximate pattern search have been proposed. 
These approximate CAMs improve the utilization and overall energy efficiency by compensating the search accuracy within an acceptable range. 
For instance, HD-CAM \cite{conventionalCAM} introduced a 10T CMOS-based approximate CAM with a matchline (\textit{ML}) charge redistribution technique, but it suffers from a large cell area and lacks the support for wildcard (\textit{don’t care}) bits.
Moreover, the  design is unable to precisely control the degree of approximation, bit-by-bit.
MHCAM \cite{liu2023reconfigurable} presented an approximate CAM design based on FeFET with programmable thresholds, but it's tailored to applications requiring multi-state Hamming distance. 
\cite{MASC} implemented threshold matching by leveraging voltage scaling and controlling the precharge period, but its high energy consumption and inability to precisely control the threshold limit its applications.
\cite{2FeFETa} introduced approximate matching capabilities using 2FeFET TCAM. It computes the Hamming distance between search and stored vectors in a highly parallelized manner by monitoring \textit{ML} discharge rate. Despite achieving notable energy efficiency and density in TCAM, it lacks fine-grained control over approximate search precision.
%Both approximate CAM designs are unable to precisely control the level of approximation.
%To address the issue of implementing approximate pattern search, several CAM designs for approximate match functions have been proposed. HD-CAM~\cite{conventionalCAM} proposed a 10T CMOS-based approximate CAM with matchline (ML) charge redistribution technique. However, HD-CAM has a large cell area and does not support mask (\textit{don't care}) bits. MHCAM \cite{liu2023reconfigurable} presented an approximate CAM design providing programmable thresholds but it is specific to the applications that require multi-state Hamming distance.
% additional transistors


%To tackle the challenges of unit area and energy consumption faced by CMOS CAM designs, researchers have proposed utilizing emerging non-volatile memories (NVMs) devices to construct more compact and efficient CAM designs. This approach offers a novel means of enhancing system performance and energy efficiency. For example, CAMs based on 2-terminal NVMs encompass resistive RAM (RRAM)~\cite{li2021sapiens,Chang3t1r}, spin transfer torque magnetic RAM (STT-MRAM)~\cite{Matsunaga4t2mtj}, phase change memory (PCM)~\cite{jing2t2r}, and 3-terminal ferroelectric RAM (FeRAM)~\cite{2FeFET,4T2FeFET,1FeFET1R-transfer}. The 2-terminal NVMs typically require current-driven write schemes and large access transistors, resulting in high write energy consumption~\cite{Liu2022eva-cam}. CAMs based on 3-terminal FeFET devices have emerged as promising candidates for implementing NV-CAMs owing to their high current ON/OFF ratio, low write energy, and compatibility with CMOS technology. 

% address both approximate match function and NV-CAM
%In this work, we propose TAP-CAM, an approximate match engine featuring a tunable threshold match function. TAP-CAM utilizes a novel 2FeFET-2R  TCAM to achieve high density with supreme energy efficiency. The tunable threshold is set by the bias voltage of the evaluation transistor. We validate the basic storage and computing functionality of the TAP-CAM unit circuit and array circuit, and conduct extensive Monte Carlo simulations to explore the impact of device-to-device variation of FeFET. 
%We use the K-nearest neighbor search (KNN) as a representative application to investigate the application-level benefits of TAP-CAM. Simulation results demonstrate that TAP-CAM achieves 16.95$\times$ energy improvement and 3.06\% accuracy improvement compared with CMOS-based CAM with the exact match function. 

%These CAMs designed for approximate search are unable to precisely control the level of approximation. 
To address aforementioned challenges of existing approximate CAMs, in this work, we propose TAP-CAM, a general approximate matching engine featuring a bit-by-bit tunable threshold match function. 
We consider FeFET as a representative NVM device, and propose to utilize a novel 2FeFET-2R ternary CAM (TCAM) cell structure to store ternary value. 
An evaluation transistor is employed between the parallel connected TCAM cells and the CAM array sense amplifier to control the \textit{ML} discharge rate, and the tunable threshold of the approximate matching functionality is set by the bias voltage of the evaluation transistor. 
We validate the bit-wise XNOR logic and the tunable  threshold matching functionality of  TAP-CAM design at cell and array levels,  respectively,  and conduct extensive Monte Carlo simulations to examine the robustness against device-to-device variations.
We use the K-nearest neighbor search (KNN) as a representative application to investigate the benefits of TAP-CAM at application level.
Evaluation results demonstrate that TAP-CAM achieves a 16.95$\times$ energy improvement and 3.06\% accuracy improvement compared to 16T CMOS CAM with exact match function. Compared to 2FeFET TCAM with
approximate match functionality, TAP-CAM achieves a 6.78$\times$
energy improvement.



The rest of paper is organized as follows: Sec.~\ref{sec:background} reviews the FeFET device characteristics and existing CAM designs. Sec.~\ref{sec:proposed_work} introduces the proposed TAP-CAM. Sec.~\ref{sec:eval} presents the evaluation results and the KNN case study. Finally, Sec.~\ref{sec:conclusion} summarizes the paper.

%The rest of the paper is organized as follows. Sec.~\ref{sec:background} reviews the FeFET device characteristics and existing CAM designs. Sec.~\ref{sec:proposed_work} introduces the circuit design of TAP-CAM. Sec.~\ref{sec:eval} presents the evaluation results and the kNN case study. Finally, Sec.~\ref{sec:conclusion} summarizes the paper. 

% The  conventional Von Neumann architecture has been the predominant computing paradigm for decades, facilitating the blossom  of information society.
% However, this architecture faces the memory wall bottleneck when handling the vast amounts of data processed in next-generation artificial intelligence (AI) algorithms and models \cite{}.  Massive data transfer between the memory and the processor for computation has wasted a significant amount of energy and time, hindering the architecture from realizing fast and energy efficient  computing tasks. 
% To address these challenges,  the compute-in-memory (CiM) paradigm has emerged as a  promising alternative for data-intensive workloads. Unlike the traditional von Neumann architecture, CiM performs computations within the memory, eliminating the costly penalty associated with data movements. This approach has demonstrated efficacy across various domains, enabling efficient data processing with improved overall performance.

% As a special form of CiM, content addressable memory (CAM) performs parallel search functionality across the memory given an input query, and quickly identifies the stored entry that is identical to the input query. 
% Such efficient search capability enables the CAM to implement as an associative memory (AM) \cite{}, which meets the demands of data-intensive workloads and offers extraordinary performance for a number of emerging applications, such as machine learning \cite{},  cognitive learning \cite{}, genome analysis, etc.
% These applications generate a huge amount of redundant data, and storing the most frequent redundant data in CAMs can effectively reduce  redundant computations, therefore improving overall energy efficiency and performance.
% However, compared with the ever-growing amount of data, CAM based AMs can only accommodate a limited number of patterns that are exact matching with the redundant data. 
% Relaxing the matching degree of the CAM arrays from the exact matching function to the approximate matching, where a few mismatches between the stored entries and the input query still indicate a match output, would be much beneficial for aforementioned data-intensive applications.

% Typical CAM designs based on CMOS technology use digital error correction techniques to enable approximate matching capability 
% from low density and high energy consumption, various compact and energy efficient CAM designs based on emerging non-volatile  memories (NVMs) including resistive RAM (RRAM), spin-transfer torque RAM (STT-RAM), and ferroelectric FET RAM (FeFET-RAM), etc., have been proposed \cite{}. 
% %At the device level,  have been proposed for improving power, delay, area efficiency, and so on \cite{RRAM,ReRAM,STTMRAM,FeFET-RAM}.
% However, most of these CAM designs 
% In this work, ferroelectric field-effect transistor (FeFET) is exploited as a representative NV device due to its high energy efficiency, high density and low cost At the circuit level, CAM designs based on FeFET including 2FeFET,  1FeFET-1R, HD-CAM and so on. These designs focus on improving the power and delay efficiency of the nearest neighbor (NN) Hamming distance search. However, at the application level, finding the NN is typically not the best prediction for the query's label \cite{}. While prior arts mainly focus on the optimization of the exact-match CAM \cite{}, a CAM design that realizes tunable approximate matching (TAP) is highly desirable, enabling higher algorithmic accuracy in an application such as k-nearest neighbor (KNN). 

%\chekai{In this work, we propose TAP-CAM, a high-resolution FeFET-based TAP CAM. The cell-level functionality is first validated, and the TAP-CAM array is then evaluated to investigate the energy and delay trends. Results indicate that \design is able to achieve higher resolution compared and AAx energy improvements over the existing CMOS-based TAP design. In addition, Monte Carlo simulations are performed to justify the robustness of the proposed \design, showcasing the reliability of the design. Finally, an application study is performed to show the usefulness of \design. Results indicate BBx energy and CCx latency improvement over a GPU implementation, and on average 3.06\% algorithmic accuracy improvement over the existing exact-match CAM.}
% Content-Addressable Memory (CAM) is an effective solution for improving hardware performance. Unlike the conventional Von Neumann architecture, where data is transferred from memory to the processor for computation, the CAM's characteristic of storing and computing in one unit can shorten the time for data transfer to some extent. Its excellent features of low latency, high energy efficiency, and small area make it show tremendous potential for applications in fields such as deep learning, natural language processing, and computer vision.

% The design of Content-Addressable Memory (CAM) is based on new non-volatile devices, and scientists have conducted extensive research in the field of non-volatile memory. RRAM, STT-RAM, and FeFET-RAM have been proposed as powerful candidates for improving MOS chip storage, as seen in \cite{RRAM,ReRAM,STTMRAM,FeFET-RAM}. The CAM discussed in this article is designed based on ferroelectric transistors, whose unique polarized memory function significantly reduces the complexity of circuit structures and solves the high energy consumption and high cost issues of conventional CMOS processes.

% However, with the continuous development of artificial intelligence, the IoT industry's requirements for the processing speed and energy efficiency of memory are increasing. As a key part of data processing, search has become an important way to improve work efficiency. As the basic hardware core of Content-in-Memory (CiM), Ternary Content-Addressable Memory (TCAM) supports parallel search of given input vectors on the memory array and can give search results based on the stored content, providing a new mode to solve the processor-memory bottleneck problem in conventional digital machines. At the same time, CAM has also been widely used in dense big data. When processing large-scale data, approximate search is often used instead of exact search, allowing for a certain Hamming distance, which allows for a certain number of bit mismatches, thereby reducing waste of storage and computing resources. Although there exist Hamming distance-tolerant Content-Addressable Memories based on standard Complementary Metal-Oxide-Semiconductor (CMOS) technology, the energy efficiency and density of CAM still need to be improved due to the problems in area, cost, and leakage of the CMOS process.