% \begin{IEEEbiographynophoto}{Che-Kai Liu}
% is currently working toward B.Eng. degree with the College of Information Science and Electronic Engineering, Zhejiang University, China. He is also an undergraduate researcher at Zhejiang University and University of Notre Dame, USA, working with Dr. Xunzhao Yin and Dr. X. Sharon Hu, respectively. His research interests include brain-inspired computing and novel computing paradigms with both CMOS
% and emerging technologies.
% \end{IEEEbiographynophoto}

% \begin{IEEEbiographynophoto}{Haobang Chen}
% is currently a junior at Zhejiang University, Hangzhou, China. He  is working towards a B.Eng. degree of Electronics Science and Technology. His current research interests focus on Computation-in-Memory
% architectures and machine learning.

% \end{IEEEbiographynophoto}

% \begin{IEEEbiographynophoto}{Mohsen Imani}
% (S’14) received the B.Sc. and M.S. degrees from the School of Electrical and Computer Engineering, University of Tehran, Tehran, Iran, in 2011 and 2014, respectively, and the Ph.D. degree with the Department of Computer Science and Engineering, University of California at San Diego, La Jolla, CA, USA, in 2020. He is now an Assistant Professor with the University of California, Irvine. His current research interests include brain-inspired computing, approximation computing, and processing in-memory architectures.
% \end{IEEEbiographynophoto}

% \begin{IEEEbiographynophoto}{Kai Ni}
% received the B.S. degree in Electrical Engineering from University of Science and Technology of China, Hefei, China in 2011, and Ph.D. degree of Electrical Engineering from Vanderbilt University, Nashville, TN, USA in 2016 by working on characterization, modeling, and reliability of III-V MOSFETs. Since then, he became a postdoctoral associate at University of Notre Dame, working on ferroelectric devices for nonvolatile memory and novel computing paradigms. He is now an assistant professor in Electrical and Microelectronic Engineering at Rochester Institute of Technology. He has 80 publications in top journals and conference proceedings, including Nature Electronics, IEDM, VLSI Symposium, IRPS, EDL, etc. His current interests lie in nanoelectronic devices empowering unconventional computing, AI accelerator, 3D memory technology.
% \end{IEEEbiographynophoto}

% \begin{IEEEbiographynophoto}{Arman Kazemi}
% is a Ph. D. candidate at the University of Notre Dame and is co-advised by Dr. X. Sharon Hu and Dr. Michael Niemier. His research interests include hardware/software co-design, low-power hardware design, and in-memory computing. He is particularly interested in inventions leveraging emerging beyond-CMOS technologies e.g. ferroelectric materials. His research usually targets reducing computational resource requirements of machine learning applications using emerging circuits and architectures. He has been nominated for the best paper award at DATE 2021.
% \end{IEEEbiographynophoto}

% \begin{IEEEbiographynophoto}{Ann Franchesca Laguna}
% received her MS degree in Electrical Engineering and BS degree in Computer Engineering from University of the Philippines – Diliman. She is currently a Computer Science and Engineering PhD student at University of Notre Dame working with Dr. X. Sharon Hu and Dr. Michael Niemier. Her current research interests encompass hardware/software co-design of machine learning and bioinformatics algorithms using in-memory computing.
% \end{IEEEbiographynophoto}

% \begin{IEEEbiographynophoto}{Michael Niemier}
% (senior member, IEEE) is a Professor at the University of Notre Dame. His research interests include designing, benchmarking, and evaluating circuits and architectures based on emerging technologies. Currently, Niemier's research efforts are based on new transistor technologies, as well as devices based on alternative state variables such as spin. He is the recipient of an IBM Faculty Award, the Rev. Edmund P. Joyce, C.S.C. Award for Excellence in Undergraduate Teaching at the University of Notre Dame, as well as the Department of Computer Science and Engineering Teaching Award at the University of Notre Dame. Niemier has served on numerous technical program committees for design-related conferences (including DAC, DATE, ICCAD, etc.), and has chaired the emerging technologies track at DATE, DAC, and ICCAD. He is an associated editor for IEEE Transactions on Nanotechnology, as well as the ACM Journal of Emerging Technologies in Computing.
% \end{IEEEbiographynophoto}

% \begin{IEEEbiographynophoto}{Xiaobo Sharon Hu}
% (Fellow, IEEE) received a BS degree from Tianjin University, in 1982, an MS degree from the Polytechnic Institute of New York, in 1984, and a Ph.D. degree from Purdue University, in 1989. She is currently a professor with the Department of Computer Science and Engineering, University of Notre Dame. Her research interests include computing with beyond-CMOS technologies, low-power system design, and cyber-physical systems. She received the NSF CAREER Award in 1997, and the Best Paper Award from the DAC, in 2001, and the ACM/IEEE International Symposium on Low Power Electronics and Design, in 2018. She was the general chair of DAC'18. She served as an associate editor of the IEEE Transactions on VLSI Systems, ACM Transactions on Design Automation of Electronic Systems, and ACM Transactions on Embedded Computing. She is currently an associate editor of the ACM Transactions on Cyber-Physical Systems.
% \end{IEEEbiographynophoto}

% \begin{IEEEbiographynophoto}{Liang Zhao} is a tenure-track research professor with the College of Information Science and Electronic Engineering at Zhejiang University. He received his Ph.D. degree in Electrical Engineering from Stanford University in 2015 and B.S. degree in Microelectronics from Tsinghua University in 2010, respectively. His research interests include emerging non-volatile memories, semiconductor device modeling and simulations, as well as circuits and architectures for neuromorphic computing. He has received the best student paper award of VLSI-TSA 2014 and the best student paper nomination of VLSI Symposia 2013. He is a member of IEEE.
% \end{IEEEbiographynophoto}

% \begin{IEEEbiographynophoto}{Cheng Zhuo}
% (S’06-M’12-SM’16) received the B.S. and M.S. degrees from Zhejiang University, China, in 2005 and 2007, respectively, and the Ph.D. degree in computer science and engineering from the University of Michigan, Ann Arbor, MI, in 2010. He is currently a Professor with the College of Information Science Electronic Engineering, Zhejiang University. His research interests include low power optimization, 3-D integration, hardware acceleration, power and signal integrity. Dr. Zhuo has published over 100 technical papers and received 4 Best Paper Nominations in DAC'16, CSTIC'18, ICCAD'20, and VTS'21. He also received 2012 ACM SIGDA Technical Leadership Award and 2017 JSPS Invitation Fellowship. Dr. Zhuo has served on the technical program and organization committees of many international conferences, and as the Associate Editors of IEEE TCAD, ACM TODAES, and Elsevier {Integration}. He is a senior member of IEEE and a Fellow of IET.
% \end{IEEEbiographynophoto}

% \begin{IEEEbiographynophoto}{Xunzhao Yin}(S'16-M'19) is an assistant professor of the College of Information Science and Electronic Engineering at Zhejiang University. He received his Ph.D. degree in Computer Science and Engineering from University of Notre Dame in 2019 and B.S. degree in Electronic Engineering from Tsinghua University in 2013, respectively. His research interests include emerging circuit/architecture designs and novel computing paradigms with both CMOS and emerging technologies. He has received the best paper award nomination of ICCAD 2020, the Outstanding Research Assistant Award in the department of CSE at University of Notre Dame in 2017, and Bronze medal of Student Research Competition at ICCAD 2016, etc. He is a member of IEEE.
% \end{IEEEbiographynophoto}