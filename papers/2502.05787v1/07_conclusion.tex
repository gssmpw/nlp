
\section{Conclusion}
\label{sec:conclusion}


In this paper, we introduce TAP-CAM, a compact and energy-efficient TCAM design capable of threshold approximate matching. 
We propose a novel 2FeFET-2R TCAM design which employs an evaluation transistor  to adjust the ML discharge rate and measure the Hamming distance between the input query and the stored entries. 
Through gate bias voltage configuration, TAP-CAM achieves bit-by-bit tunable HD threshold matching functionality that is a crucial operation in many data-intensive applications. Evaluation results and application benchmarking suggest that our proposed 2FeFET-2R TAP-CAM array surpasses other advanced CAM technology in both energy efficiency and performance.



%In this article, we begin by explaining the structural characteristics and superior performance of FeFET devices, emphasizing the importance of the 1FeFET-1R structure in stabilizing current flow. We then review existing CAM designs and, considering their pros and cons, propose a new 2FeFET-2R TCAM structure akin to the established 2FeFET TCAM. We detail the structural design and operational principles of the unit circuit, validating its basic storage and computation functions. Next, we expand the unit circuit into an array circuit for a comprehensive performance analysis. An evaluation transistor is employed between the parallel connected TCAM cells and the CAM array ML to control the ML discharge rate, and the tunable threshold of the approximate matching functionality is set by the bias voltage of the evaluation transistor. Since different mismatch thresholds correspond to different evaluation voltages, we verify the approximate matching function by distinguishing between adjacent mismatches within 0-6 bits. This is achieved by adjusting the gate voltage of the evaluation transistors, observing the ML voltage variation, and determining the sensing time window.

%Furthermore, we scale up the 2FeFET-2R unit circuit to an m rows × n columns configuration and analyze changes in energy consumption and latency. We vary parameters such as $V_{DD}$, mismatch threshold, number of rows, and number of bits per row. We compare various performance indicators of the array circuit, including circuit area, search energy consumption, and search latency, with other TCAM designs. Finally, we verify the robustness of our proposed architecture through Monte Carlo simulations. Our results demonstrate that the 2FeFET-2R TCAM structure offers significantly better energy advantages than other designs, providing an innovative hardware solution for CAM design.


%In this article, we first elaborate on the structural characteristics and superior performance of FeFET devices, as well as the importance of the 1FeFET-1R structure in stabilizing current flow. We review existing CAM designs and based on their respective advantages and disadvantages, we propose a new 2FeFET-2R CAM structure similar to the established 2FeFET TCAM. We elucidate the structural design and working principle of the unit circuit and validate its basic storage and computation functions. Subsequently, we expand the unit circuit into an array circuit for a comprehensive analysis of its functionality and performance. Since different matching thresholds correspond to different evaluation voltages, to verify the normal operation of the approximate matching function, we distinguish between the situations where the adjacent number of mismatches within 0-6 bits occurs by adjusting the gate voltage of the evaluation transistors, obtaining the variation of the ML voltage and the corresponding sensing time window. We further scale up the 2FeFET-2R unit circuit to an m rows × n columns configuration and analyze the changes in energy consumption and latency by varying parameters such as $V_{DD}$, matching threshold, number of rows, and number of bits per row, and compare various performance indicators of the array circuit such as circuit area, search energy consumption, and search latency with other TCAM designs. Finally, we verify the robustness of our proposed architecture through Monte Carlo simulations. Our results demonstrate that the 2FeFET-2R TCAM structure exhibits significantly better energy advantages than other designs, providing an innovative hardware solution for CAM design. 
%\section*{Acknowledgements}
%\small
%This work was partially supported by the National Key R\&D Program of China (2018YFE0126300), Zhejiang Provincial Key R\&D program (2022C01232), NSF (LQ21F040006, LD21F040003), NSFC (62104213, 92164203), and Zhejiang Lab (2021MD0AB02). Niemier was supported in part by ASCENT, one of six centers in JUMP, a Semiconductor Research Corporation (SRC) program sponsored by DARPA. Imani was supported in part by National Science Foundation (NSF) \#2127780 and Semiconductor Research Corporation (SRC) \#2988.001.
