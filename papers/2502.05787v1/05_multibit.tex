%figure not finished yet


%\section{Multi-bit COSIME Implementation}

%\label{sec:multi-bit}
%{\color{red}So far, we've introduced vectors with binary values - '0' and '1'. However, application such as convolution neural network (CNN) that is mapped into vectors normally is presented in several weights rather than simply zeros and ones.
%Recent work for in-memory multi-bit multiplication approaches the canonical SRAM along with analog operations, the work uses either traditional 6T SRAMs in~\cite{sujan2018,IMAC,bitline_shifting}, or less-area efficient 8T or 10T cells for in-memory multi-bit multiplication~\cite{8T,biswas2018conv}. Besides, a fully-digital standard-cell-based CMOS in-memory multi-bit multiplication is presented in \cite{PPAC}.
%However, the peripherals in these works are complex as they introduce successive approximation analog-to-digital converter (SAR-ADC) and other digital peripherals, leading to higher search delay, and the conventional SRAM itself isn't area-efficient enough.

%Here, we propose an area and energy efficient way of analog-like in-memory multiplication. Similar to \cite{8T}, it's necessary to operate the input MOSFET of a cell in a particular voltage range such that the current is a linear function of the applied voltage $V_{in}$. But unlike prior work leveraging conventional SRAM, we make use of the same 1F1R structure as the binary {\cosi} discussed in \ref{sec:mem_array}, and a MOSFET that operates in the deep triode region (Eq.~\ref{eq:triode}) to serve as the input.
%\subsection{Schematic of 4-bit Cell}
%Let us consider a 4-bit precision for the weights. If the weight $W = w_3w_2w_1w_0$, where $w_i,i=0\sim 3$ corresponding to the 4-bit weight, the vector matrix dot product becomes:
% \begin{equation}\label{eq:4bit}
% \small
%     \begin{aligned}
%         &\sum(v_{in}\cdot W) = \sum[v_{in}\cdot(2^3w_3+2^2w_2+2^1w_1+2^0w_0)]\\
%         &=\sum(v_{in}\cdot2^3w_3)+\sum(v_{in}\cdot2^2w_2)+\sum(v_{in}\cdot2^1w_1)+\sum(v_{in}\cdot2^0w_0)
%     \end{aligned}
% \end{equation}
%Let us consider a cell with 4-bit weight $W = w_3w_2w_1w_0$. With a 4-bit input $V_{in}$, the output current should satisfy the following equation:
%\begin{equation}\label{eq:4bit}
%        I = V_{in} \cdot W = V_{in} \cdot (2^3w_3+2^2w_2+2^1w_1+2^0w_0)
%\end{equation}
%In {\cosi}, the weight, i.e. $2^3:2^2:2^1:2^0$, is achieved by the resistor applied in each 1F1R structure. The binary bits in Eq.~\ref{eq:4bit}, i.e. $w_3\sim w_0$, are stored in the FeFET in each 1F1R structure respectively. Finally, the input voltage $V_{in}$ is inputted through a MOSFET that operates in deep triode region. The $I_{DS}$ flows through a MOSFET/FeFET that operates in deep triode region can be characterized as follows:
%\begin{equation}\label{eq:triode}
%    \begin{aligned}
%    I_{DS} = \frac{\mu C_{ox}}{2}&\frac{W}{L}[2(V_{GS}-V_{TH})V_{DS}-V_{DS}^2],\\ &\text{with} V_{GS}>V_{TH}, V_{DS}<V_{GS}-V_{TH}
%    \end{aligned}
%\end{equation}
%where $\mu$ is the mobility of electrons, $C_{ox}$ is the silicon oxide, and $V_{TH}$ is the threshold voltage. In COSIME, the quadratic $V_{DS}$ term can be ignore since the operating current is small due to the large resistor applied in 1F1R structure. The MOSFET that serves as an input should operate in deep triode region, once it enters saturation, the multiplication can no longer be performed (see Fig.~\ref{fig:all}).

%In short, the stored value in multi-bit {\cosi} is binary, while the input value can be selected from 0 to 1.2V, depending on the weight of the input value, and for voltage that is higher than 1.2V, since the input MOSFET enters saturation, the mapping will be the same as 15. 



%\subsection{Experimental setup of Multi-bit Cell}

%In our SPICE experimental setup, each FeFETs is first programmed to either high/low-$V_{TH}$ state with 100ns $\pm4V$ pulse, as stated in Sec.~\ref{sec:background}. Then, the gate voltage of each FeFETs, $V[3]\sim V[0]$, is simply set to '1' (see Fig.~\ref{fig:cell_wave}(a)). Consequently, if the stored value in FeFET is zero, the output current $I_D$ of that certain 1F1R will be '0', on the contrary, will be '1'. With the proposed 1F1R structure, the ratio of the $I_D$ flows through each 1F1Rs' rails in a cell can reach 8:4:2:1. In {\cosi}, $R_0=2.5\times 10^8$, $R_1=1.25\times 10^8$, $R_2=0.625\times 10^8$, and $R_3=0.3125\times 10^8$ (see Fig.~\ref{fig:cell_wave}). The input MOSFET will be activated during the search. In Fig.~\ref{fig:cell_wave}(a), the 4-bit input is mapped into 16 different voltages.

%\subsection{Multi-function Crossbar COSIME}
%To this end, we proposed a dedicated in-memory search engine for cosine similarity. However, practical applications usually involve various functions~\cite{ReFeMAT}. To make {\cosi} supports other well-studied functions such as Hamming distance/binary convolution neural network (BCNN), another 1F1R is added on the basis of Fig.~\ref{fig:cell_wave}, which can be seen in Fig.~\ref{fig:cell}.
%\begin{figure}
%    \centering
%    \includegraphics[width=\linewidth]{Figures/new-multi-bit-cell.png}
%    \caption{{\color{red}{\cosi} cell supports Hamming distance and multi-bit multiplication on the basis of Fig.~\ref{fig:cell_wave}(a). The orange background color supports multi-bit multiplication while the green square supports Hamming distance.}}
%    \label{fig:cell}
%\end{figure}
%To support both multi-bit in-memory multiplication and Hamming distance, the crossbar array for the denominator of translinear circuit input, i.e. the $I_y$ in Eq.~\ref{eq:translinear}, is simply set to a constant. In multi-bit multiplication mode, the gate voltage of FeFET, i.e. $V[3]\sim V[0]$, is set to '1' when the query vector comes (see the waveform in Fig.~\ref{fig:cell_wave}). On the other hand, when {\cosi} operates in Hamming distance mode, $V[3]\sim V[1]$ will simply be set to '0' when the query comes.
%Consequently, {\cosi} consumes less search energy while functioning in these mode.
%}

