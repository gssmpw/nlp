%The model has been changed from bsim6 with no w\l specification and model card to PTM 45nm HP with all the length/width specified, resulting huge leap in latency and EDP/cell accordingly.


%\begin{figure*}%[hbt]
 %   \centering
%\includegraphics[width=1\linewidth]{Figures/top_1.png }
 %   \caption{(a) Equivalent RC circuit with mismatch of 4 bits. (b) Equivalent RC circuit with mismatch of 3 bits. (c) Conduction equivalent resistance of 1FeFET-1R. (d)The voltage difference between (a) and (b) varies with RC. (e) The structure of proposed 2FeFET-2R CAM array of 64 columms. (f)TIQ comparator. (g)Transient waveforms at V$_eval$=0.62V(Thrshold=3 bits)}
    %\vspace{-1.5em}
  %  \label{fig:overall}
%\end{figure*}
%\begin{figure*}%[hbt]
 %   \centering
  %  \includegraphics[width=1\linewidth]{Figures/top_2.png }
   % \caption{ (a) mxn 2FeFET-2R TCAM array. (b)Transient waveforms in different mismatch thresholds.}
    %\vspace{-1.5em}
%\label{fig:overall}
%\end{figure*}


\section{Proposed TAP-CAM Design}
\label{sec:proposed_work}
In this section, we present the TAP-CAM design with bit-by-bit tunable HD threshold match functionality, exploiting the 2FeFET-2R structure and incorporating a threshold-defined evaluation transistor. 
We first discuss the structure and operation principles of the cell, %particularly focusing on the 2FeFET-2R configuration. Subsequently, we
and then elucidate the threshold approximate match implementation at the array level.


\subsection{2FeFET-2R TCAM Cell} 

\begin{figure}
    \centering
    \includegraphics[width=\linewidth]{Figures/2F2R1.png}
    %\vspace{-0.4cm}
    \caption{\textbf{(a)} Structure of the proposed 2FeFET-2R TCAM cell; \textbf{(b)} Transient voltage waveforms of 2FeFET-2R CAM cell storing `1’.}
   % \vspace{-0.4cm}
\label{fig:2FeFET-2R Cell}

\end{figure}

\begin{table}[!t]
    \centering
    \caption{OPERATIONS OF 2FEFET-2R TCAM CELL}
\begin{adjustbox}{center}
\resizebox{1\columnwidth}{!}{
            \begin{tabular}{|c c |c|c|c|c|c|}
            
              \hline \hline
              $\textit{V}_\textit{write}$ = 4V & $\textit{V}_\textit{search}$ = 1V & \textit{BL}/$\overline{\textit{SL}}$ &  $\overline{\textit{BL}}$/$\textit{SL}$ & \textit{ScL} & $\textit{M}_\text{1}$ & $\textit{M}_\text{2}$   \\ 
              \hline
    \multirow{2}*{Write`1'} & Step1     &$\textit{V}_\textit{write}$ & 0 & 0 &`1' & hold\\ & Step2     &$\textit{V}_\textit{write}$ & 0 & $\textit{V}_\textit{write}$ & hold & `0'\\ 
    \hline
    \multirow{2}*{Write`0'} & Step1    & 0 & $\textit{V}_\textit{write}$ & $\textit{V}_\textit{write}$ &`0'  & hold\\
                               & Step2    &0 & $\textit{V}_\textit{write}$ & 0 & hold & `1'\\
                       \hline
              \multicolumn{2}{|c|}{Write \textit{don't care}}  &$\textit{V}_\textit{write}$ & $\textit{V}_\textit{write}$ & 0 & `1' & `1'\\
              \hline
              \hline
            \end{tabular}
             
      }
        \end{adjustbox}
    
    \label{tab:write opetarion}
\end{table}



\autoref{fig:2FeFET-2R Cell}(a) shows the structure of the proposed 2FeFET-2R TCAM Cell. It comprises a pair of parallel  1FeFET-1R structures, with the FeFET drain connected to the matchline (\textit{ML}), and the other end of the structure connected to the sourceline (\textit{ScL}), driven by either $V_{\textit{write}}$ or \textit{GND}.
The FeFET gate connects to the bitline and searchline (\textit{BL}/\textit{SL} and $\overline{\textit{BL}}$/$\overline{\textit{SL}}$). 
By adjusting the write gate input, the FeFET threshold aligns with different storage values. 
%$T_{1}$ precharges $ML$ before searching. 
The 2FeFET-2R structure can store logic `1', `0', and 
\textit{don't care} wildcard state.
%\autoref{fig:2FeFET-2R Cell}(a) shows the structure of the proposed 2FeFET-2R TCAM Cell. The 2FeFET-2R unit circuit comprises a pair of parallel 1FeFET-1R structures, where the drain of the FeFET is connected to the matchline ($ML$), and the other end of the resistor connected to the sourceline ($ScL$) can be driven to $V_{write}$ or $GND$. The gate of the FeFET is connected to the bitline and searchline ($BL$/$\overline{SL}$ and $\overline{BL}$/$SL$). By varying the input to the gate, the threshold of the FeFET can be regulated to correspond to different storage values. $T_{1}$ is connected to $ML$ for precharging $ML$ before the search. The 2FeFET-2R structure can store logic `1' and `0', and $don't\text{ $care$ } $cases. 
\autoref{tab:write opetarion} outlines the write operations of the 2FeFET-2R cell. 
Data bits are written in two steps, storing complementary logic states in each FeFET. 
To write logic `1', $V_{\textit{write}}$ is applied to \textit{BL}/\textit{SL}, while `0' to \textit{ScL} and $\overline{\textit{BL}}$/$\overline{\textit{SL}}$. This sets $V_{\textit{GS}}$ of $\textit{M}_\text{1}$ to 4V, writing logic `1' to $\textit{M}_\text{1}$. In the second step, $V_{\textit{write}}$ is applied to \textit{ScL}, while gate voltage remains the same, writing logic `0' to $\textit{M}_\text{2}$. Thus, the complementary stored  values represents logic `1'.
Similarly, to write logic `0' into the cell, `0' is written to $\textit{M}_\text{1}$ and `1' to $\textit{M}_\text{2}$, respectively. 
To write \textit{don't care} state, logic `1' is written to both $\textit{M}_\text{1}$ and $\textit{M}_\text{2}$. This sets both FeFETs to high-$\textit{V}_\textit{TH}$ state, matching regardless of the search value, aligning with the masking function of `\textit{don't care}' bits.
During writes, \textit{ML} is grounded to eliminate static current. \autoref{fig:fefet}(b) displays $\textit{I}_\textit{D}$-$\textit{V}_\textit{G}$ curves for FeFETs under different write pulses.


%\autoref{tab:write opetarion} summarizes the write operations of the 2FeFET-2R unit circuit. The data bits are written into the two FeFETs in the TCAM cell in two steps, storing opposite logic values in each FeFET. In order to write logic `1', in the first step, $V_{write}$ is applied to $BL$/$\overline{SL}$, while `0' is applied to $ScL$ and $\overline{BL}$/$SL$. Therefore, the gate-source voltage ($V_{GS}$) of $M_1$ is 4V, switching the FE polarization within FeFET and writing logic `1' to $M_1$. In the second step, $V_{write}$ is applied to $ScL$, while the gate voltage of the two FeFETs remains the same as in the first step (i.e. 4V at $BL$/$\overline{SL}$ and 0 at $\overline{BL}$/$SL$). As a result, the $V_{GS}$ of $M_2$ is -4V, and the logic `0' is written to $M_2$. Thus, the 2FeFET-2R unit as a whole represents the storage of logic '1'. 


%Similarly, to write logic `0' into the TCAM cell, we write `0' to $M_1$ and `1' to $M_2$, respectively. To write $don't\text{ $care$ }$into the TCAM cell, it only takes one step to write logic `0' to both $M_1$ and $M_2$. In this way, both FeFETs corresponding to `don't care' are in low-$V_{TH}$ state, resulting in a match regardless of the search value, which aligns with the masking function of `don't care' bits. During writing operations, the matchline $ML$ needs to be driven to ground to eliminate the influence of static current on the $ML$ voltage. \autoref{fig:fefet}(b) displays the $I_{D}$-$V_{G}$ curves corresponding to different write pulses of FeFETs.

During search, \textit{ML} voltage is precharged to high via a precharge transistor, and the search voltages are applied to searchlines ($\textit{SL}$/$\overline{\textit{SL}}$) according to the query data. For logic `1', \textit{SL} set to 1V, and 0 for logic `0', the  \textit{ML} voltage indicates the matching result. 
\autoref{fig:2FeFET-2R Cell}(b) validates the function of the 2FeFET-2R cell. 
\textit{ML} is first precharged by controlling $\textit{T}_\text{1}$'s gate voltage \textit{CLK}, and then left floating upon  search phase. 
When searching `1', \textit{ML} voltage stays high with  \textit{SL} = 1V, indicating a match. Conversely, searching `0' rapidly drops \textit{ML} voltage to 0, indicating a mismatch. %Results in \autoref{fig:2FeFET-2R Cell}(b) align with expectations, validating unit circuit's storage and computing functions.


%During the search operation, the voltage of $ML$ is previously precharged to a high level through a precharge transistor, and different search voltages are applied to the searchlines ($SL$/$\overline{SL}$) according to the input data. Here we set $SL$ to 1V for logic `1' and 0 for logic `0', and observe the voltage change of $ML$ to reflect the matching result. As shown in \autoref{fig:2FeFET-2R Cell}(b), by controlling the gate voltage $CLK$ of $T_{1}$ to precharge $ML$, the precharging is halted after entering the search phase. When searching for logic '1', the voltage on $ML$ remains almost unchanged at a high level when $SL$ voltage is 1V, indicating a match between the search query and the stored entry. Conversely, when searching for logic '0', the voltage on $ML$ rapidly drops to 0, indicating a mismatch between the search query and the stored entry. The operation results shown in \autoref{fig:2FeFET-2R Cell}(b) are consistent with expectations, validating the correctness of the unit circuit's storage and computing functions.


\subsection{2FeFET-2R TCAM Array}

\autoref{fig:1x64} demonstrates the schematic of the proposed 2FeFET-2R TAP-CAM array storing a 64-bit word with corresponding peripheral circuits. 
%This array stores a 64-bit word, sharing $ML$ and $ScL$ across all 64 cells.
PMOS $\textit{T}_\text{1}$ precharges \textit{ML} before the search operation, while an evaluation transistor $\textit{T}_\text{2}$ is connected between \textit{ML} and $\textit{V}_\text{o}$ to enable tunable threshold matching function. 
%$T_2$ evaluates the $ML$ voltage for threshold match function. 
Adjusting the gate voltage of the evaluation transistor controls the discharge rate of \textit{ML}, allowing varying mismatch bits to be sensed by the sense amplifier (SA) as a match case. 
%A sense amplifier (SA) detects the  output, enhancing signal stability, forming output waveform. 


%to mitigate $ML$ parasitic capacitance impact on sensing time.
%As shown in \autoref{fig:1x64}, the 2FeFET-2R unit circuit is expanded into a 1×64 array and accompanied by corresponding peripheral circuits. This array has the capability to store a 64-bit word, with all 64 units sharing the same $ML$ and $ScL$. The $ML$ is connected to two transisitors, where PMOS $T_1$ is used to precharge $ML$ before the matching operation begins, while NMOS $T_2$ is connected to ML as a transistor for voltage evaluation, enabling threshold-controlled approximate match functionality. By adjusting the gate voltage applied to NMOS, the rate of $ML$ voltage decrease can be controlled, thereby achieving control over the permissible number of mismatched unit bits. Additionally, a sense amplifier(SA) is serially connected at the output end to shape and amplify the output results, thereby enhancing the stability of the output signal, ultimately forming the output waveform. During the precharging phase, the PMOS control signal $CLK$ is driven to a low level, while $V_{eval}$ is kept at a high level to ensure the conduction of both $T_1$ and $T_2$. This allows $ML$ to be precharged to $V_{DD}$ to mitigate the impact of parasitic capacitance of $ML$ on the sensing time. 


\label{sec:2FeFET-2R TCAM Array}

\begin{figure}
    \centering    
    \includegraphics[width=\linewidth]{Figures/1x64.png}
    \caption{Structure of a 2FeFET-2R TCAM array with wordlength 64.}
\label{fig:1x64}
%\vspace{-0.4cm}
\end{figure}

During the precharge, \textit{CLK} is set to low, turning  $\textit{T}_\text{1}$ and $\textit{T}_\text{2}$ ON, and precharging \textit{ML} to \textit{VDD}.
During the search phase, setting the \textit{CLK} signal high turns $\textit{T}_\text{1}$ OFF and cutting the charging path. 
Pre-defined bias voltages are applied to the gate of evaluation transistor  $\textit{V}_\textit{eval}$ based on required mismatch thresholds. 
A mismatch between the stored entry and the search query forms a conduction path from $\textit{V}_\text{o}$ to \textit{GND}, discharging $\textit{V}_\text{o}$ and decreasing the voltage.
The rate of voltage decrease depends on the number of mismatched cells and $\textit{T}_\text{2}$'s gate voltage $\textit{V}_\textit{eval}$. 
This rate affect  the output of SA $\textit{SA}_\textit{out}$ which indicates the time for $\textit{SA}_\textit{out}$ to transition from high to low. 
With constant $\textit{V}_\textit{eval}$, more mismatched bits increase the discharge current from $\textit{V}_\text{o}$ to \textit{GND}, accelerating $\textit{SA}_\textit{out}$ voltage drop. 
Similarly, with constant mismatched bits, higher $\textit{V}_\textit{eval}$ boosts the conduction of $\textit{T}_\text{2}$, hastening $\textit{SA}_\textit{out}$ voltage drop. Hence, given the fixed SA sense time, decreasing the $\textit{V}_\textit{eval}$ allows for increasing the mismatch threshold.
%, keeping $ML$ voltage sensing time consistent.

%During the search phase, the $CLK$ signal is set to a high level, leading to the closure of $T_1$ and the interruption of the charging path. Different voltages are applied to $V_{eval}$ based on varying matching thresholds. In a 1x64 array, a mismatch between the stored entry and the search query results in the establishment of a conduction path from $V_o$ to $GND$, causing a decline in the voltage on $V_{o}$. The rate of the voltage decline is contingent upon both the number of mismatched bits and the gate voltage $V_{eval}$ of $T_2$. This decline rate is reflected in the output voltage $SA_{out}$, representing the time required for $SA_{out}$ to transition from a high to a low level.
%When $V_{eval}$ remains constant, an increase in the number of mismatched bits leads to a greater number of conduction paths between $V_o$ and $GND$, thereby accelerating the voltage drop on $SA_{out}$. Similarly, when the number of mismatched bits is constant, a higher $V_{eval}$ enhances the conduction state of $T_2$, resulting in a faster voltage drop on $SA_{out}$. Therefore, as the matching threshold increases, we can decrease the value of $V_{eval}$, so that the sensing time for discerning the voltage status on $ML$ under different matching thresholds remains within the same range.


%\begin{figure}
%    \centering
%    \includegraphics[width=\linewidth]{Figures/R_increase.pdf}
%    \caption{(a)The $I_{ds}$-$V_{gs}$ wave of 1FeFET-1R varies with $R_S$. (b) The Search latency varies with $R_S$.}
%\label{fig:R_increase}
%\end{figure}
%\label{sec:R}


%As previously discussed, the 1FeFET-1R configuration limits conduction current and enhances the robustness by suppressing current variability.
%Higher resistance series current limiter reduces the discharge current, and improves the conduction stability. 
%Additionally, $R_S$ presence results in a more significant $ML$ voltage drop due to greater mismatched bit count. 
Without loss of generality, 
for the TAP-CAM with n bits mismatch threshold (Th-n), i.e., $\leq$n mismatch bits are sensed as a match case, and $\ge$(n+1) bits mismatch indicates a mismatch, the sense margin between the n bits mismatch and (n+1) bits mismatch is determined by the equivalent resistance and associated \textit{ML} capacitance of the array $\textit{C}_\textit{M}$.
%, as illustrated in \autoref{fig:fefet}(c), $R_{ON}$ signifies FeFET's equivalent conduction resistance, while $C_M$ represents the circuit's equivalent capacitance.
%As previously articulated, the 1FeFET-1R configuration can limit the magnitude of the conduction current and enhance 
%the robustness of the circuit. Moreover, as the resistance value of the series current limiter increases, the current gradually decreases, leading to improved stability of the circuit. In addition, we also have observed that the presence of $R_S$ results in a more significant voltage drop of $ML$ due to greater mismatched bit count. Taking n bits mismatch and (n+1) bits mismatch as examples, as illustrated in \autoref{fig:fefet}(c), $R_{ON}$ represents the equivalent conduction resistance of the FeFET, while $C_M$ represents the equivalent capacitance of the circuit. 
The equivalent resistance for the two mismatch cases can be expressed as follows:
%We can express the relationship as follows:
\begin{equation}
\label{eq:R-C circuit1}
      \textit{R}_\textit{n} = \frac{\text{1}}{\textit{n}}\cdot (\textit{R}_\textit{ON}+\textit{R}_\textit{S})
\end{equation}
\begin{equation}
\label{eq:R-C circuit2}
   %   R_{n+1} = \frac{1}{n+1}\cdot (R_{ON}+R_S)
     \textit{R}_{\textit{n}\text{+1}} = \frac{\text{1}}{\textit{n}\text{ + 1}}\cdot (\textit{R}_\textit{ON}+\textit{R}_\textit{S})
\end{equation}
where $\textit{R}_\textit{n}$ represents the approximate equivalent resistance of array with n bits mismatch, and $\textit{R}_\text{n+1}$ represents the approximate equivalent resistance of array with (n+1) bits mismatch. %$C_M$ is the associated ML capacitance.
$\textit{R}_\textit{ON}$ represents the equivalent resistance of an ON FeFET, and $\textit{R}_\textit{S}$ represents the series resistance. 
From charging and discharging formula of RC circuit, we can approximately formulate the \textit{ML} voltage \textit{U}:
\begin{equation}
\label{eq:R-C circuit3}
      \textit{U}=\textit{U}_\text{0}\cdot \textit{e}^{-\frac{\textit{t}}{\textit{RC}_\textit{M}}}
\end{equation}

\begin{equation}
\label{eq:R-C circuit4}
      \frac{\textit{dU}}{\textit{dt}}=\textit{U}_\text{0}\cdot (-\frac{\text{1}}{\textit{RC}_\textit{M}})\textit{e}^{-\frac{\textit{t}}{\textit{RC}_\textit{M}}}
\end{equation}
where $\textit{U}_\text{0}$ represents the initial voltage of \textit{ML}. 
From \autoref{eq:R-C circuit4} we can conclude that the rate of \textit{ML} voltage drop will be faster as the equivalent resistance decreases. 
%Due to the fact that the parallel resistance of n identical resistors is greater than the parallel resistance of n+1 identical resistors,
From \autoref{eq:R-C circuit1} and \autoref{eq:R-C circuit2}, $\textit{R}_\textit{n}$ is larger than $\textit{R}_{\textit{n}\text{+1}}$. 
Therefore,  the voltage of \textit{ML} corresponding to (n+1) bits mismatch drops faster than that of n bits mismatch.
Upon the sensing, the sense margin of Th-n $\Delta U$ can be expressed as follows:
%$U_n$ represents the $ML$ voltage corresponding to  n  bits mismatch, and $U_{n+1}$ represents the $ML$ voltage corresponding to the circuit with (n+1) mismatched bits. As indicated by  \autoref{eq:R-C circuit3}, we can express this relationship as follows:
\begin{equation}
\label{eq:R-C circuit5}
     \Delta \textit{U}=\textit{U}_\textit{n}-\textit{U}_{\textit{n}\text{+1}}=\textit{U}_\text{0}\cdot (\textit{e}^{-\frac{\textit{t}}{\textit{R}_\textit{n}\textit{C}_\textit{M}}}-e^{-\frac{t}{\textit{R}_{\textit{n}\text{+1}}\textit{C}_\textit{M}}}) 
\vspace{0.1cm}
\end{equation}
where $\textit{U}_\textit{n}$ represents the \textit{ML} voltage corresponding to  n bits mismatch, and $\textit{U}_{\textit{n}\text{+1}}$ represents the \textit{ML} voltage corresponding to (n+1) bits mismatch.
%With an increase in $R_S$, $\Delta$$U$ also broadens, resulting in a wider sensing margin.
From \autoref{eq:R-C circuit5}, we observe that $\textit{R}_\textit{S}$ affects the magnitude of $\Delta$$U$ over time t, thus influencing the sense margin. Simultaneously, a larger $\textit{R}_\textit{S}$ value introduces larger search delay. Therefore, selecting an appropriate $\textit{R}_\textit{S}$ value is necessary to ensure that both sense margin and search delay remain within reasonable limits. 
We here select $\textit{R}_\textit{S}$ = 0.3M.% and \textit{VDD} = 0.6V



%As the series resistance $R_S$ increases, $\Delta$$U$ also experiences an augmentation, thereby resulting in a broader sensing margin. However, this enhancement in circuit performance is accompanied by an increase in latency. Therefore, it is necessary to carefully balance both performance and latency when determining the resistance value of $R_S$.Taking all factors into account, we ultimately opt for R=0.3M and $V_{DD}$=0.6V.

Another factor that affects the sense margin and the search time is the bias voltage at evaluation transistor gate. 
To implement the functionality of bit-by-bit tunable threshold approximate matching, we determine appropriate evaluation voltages $\textit{V}_\textit{eval}$ to distinguish different mismatch thresholds, taking the threshold ranging 0-6 bits as an example.
This involves adjusting the gate voltage of the evaluation transistor to differentiate between 0-bit and 1-bit mismatch (Th-0), 1-bit and 2-bit mismatch (Th-1), and so forth. 
%adjacent numbers of mismatched bits can cause the $ML$ voltage to exhibit either a high level (maintained at 1V) or a low level (dropped to 0V) within the same time window.
%Once this time window is established, the array's capability for approximate matching can be verified by controlling the magnitude of the evaluation voltage.%To verify the proper functioning of the approximate match capability, it is essential to determine the appropriate evaluation voltage $V_{eval}$ to distinguish adjacent numbers of mismatched bits within the range of 0-6 bits. This involves adjusting the gate voltage of the evaluation transistor to differentiate between 0-bit mismatch and 1-bit mismatch, 1-bit mismatch and 2-bit mismatch, and so forth. The method of differentiation entails ensuring that within the same time window, the adjacent number of mismatched bits causes the voltage on the ML to exhibit either a high level (maintained at 1V) or a low level (dropped to 0V). Once the time window for distinguishing adjacent numbers of mismatched bits is established, we can verify that the array can achieve approximate match functionality by controlling the threshold voltage. 
Increasing the number of mismatch bits and evaluation transistor gate voltage $\textit{V}_\textit{eval}$ lead to faster $\textit{SA}_\textit{out}$ voltage decrease. 
Hence, with increasing mismatch threshold, we decrease $\textit{V}_\textit{eval}$ to maintain consistent sense time window across different mismatch thresholds.
%Based on this, we conducted experiments, obtaining different $V_{eval}$ values for matching thresholds ranging from 0 to 5 bits. 
The evaluation voltages are therefore experimentally examined and configured as summarized in \autoref{tab:Threshold-V} to ensure that the sense time for distinguishing different mismatch thresholds falls within the same time window.
Different evaluation voltages correspond to different mismatch thresholds. 
%\autoref{tab:Threshold-V} presents the evaluation transistor gate voltage for differentiating adjacent mismatched bit numbers, corresponding to different matching thresholds. 
This evaluation voltage configuration lays the foundation for subsequent performance and latency analysis.

%Different evaluation voltages correspond to different matching thresholds. As the number of mismatched bits increases and the gate voltage of the evaluation transistor $V_{eval}$ increases, both contribute to a faster decrease in the output voltage $SA_{out}$. Consequently, as the matching threshold increases, we gradually decrease the value of $V_{eval}$ to ensure a consistent sensing time window across different match thresholds. Based on this, we have conducted experiments and obtained different $V_{eval}$ corresponding to matching thresholds ranging from 0 to 5 bits. \autoref{tab:Threshold-V} presents the evaluation transistor gate voltage corresponding to distinguishing adjacent numbers of mismatched bits, i.e., the evaluation voltage corresponding to different matching thresholds. These findings will serve as the foundation for our subsequent performance and latency analysis.




\begin{table}[!t]
    \centering
    \caption{$\textit{V}_\textit{eval}$ of different mismatch threshold}
    \begin{adjustbox}{center}
    \resizebox{1\columnwidth}{!}{
\begin{tabular}{|c|c|c|c|c|c|c|}
              \hline
            \makecell{Mismatch\\Threshold(bit)} & 0 & 1 & 2 & 3 & 4 & 5   \\
              \hline
              $\textit{V}_\textit{eval}$(V) & 1 & 0.75 & 0.63 & 0.52 & 0.43 & 0.37  \\ 
              \hline
\end{tabular}

}
\label{tab:Threshold-V}
\end{adjustbox}
\end{table}
\begin{figure}
    \centering
    \includegraphics[width=\linewidth]{Figures/threshold2.png}
    \caption{Transient waveforms of \textit{ML} under different mismatch thresholds. Solid and Dashed lines represent the match and mismatch cases corresponding to a certain mismatch threshold, respectively.}
\label{fig:threshold}
%\vspace{-0.4cm}
\end{figure}

The \textit{ML} transient waveforms corresponding to different mismatch thresholds in \autoref{fig:threshold} validate the bit-by-bit tunable threshold matching function.
%variations and the sensing time window during approximate matching under corresponding evaluation voltages and matching thresholds. 
Solid lines show the \textit{ML} voltage waveforms when the number of mismatched bits equals to the pre-defined mismatch threshold, while dashed lines show the \textit{ML} voltages when the number of mismatched bits  exceeds the pre-defined threshold. 
The sense margin of mismatch thresholds decreases as the threshold increases. 
%Notably, at Th-5, common window exists between 5-bits and 6-bits mismatch, with sensing time window endpoints serving as sensing margin. 
According to \autoref{fig:threshold}, the search latency for distinguishing adjacent mismatch threshold ranging from Th-0 to Th-5  is  1 ns.


%\autoref{fig:threshold} illustrates the transient voltage variations of the $ML$ and the sensing time window during approximate matching under corresponding evaluation voltages and matching thresholds. Solid lines represent the $ML$ voltage variations when the number of mismatched bits equals the matching threshold, while dashed lines represent the ML voltage variations when the number of mismatched bits just exceeds the matching threshold.  As the threshold increases, the margin between the solid and dashed lines diminishes. Notably, at a matching threshold of 5 (Th-5), there exists a common window between 5-bits mismatch and 6-bits mismatch, with the endpoints of the sensing time window serving as sensing margin. According to \autoref{fig:threshold}, the search latency for distinguishing adjacent numbers of mismatched bits within the range of 0 to 6 bits is 1 ns.

%\begin{figure}
%    \centering
%    \includegraphics[width=\linewidth]{Figures/mxn.pdf}
%    \caption{Structure of mxn 2FeFET-2R TCAM.}
%\label{fig:mxn}
%\end{figure}

%For energy and delay analysis, we expanded original 1x64 array of 2FeFET-2R TCAM cells into larger m rows and n columns array, as shown in \autoref{fig:mxn}. Expanded array accommodates m words, each of length n. Each cell in array possesses identical functionalities. Cells in same column share $SLs$, enabling parallel search operations, while cells in same row share $ML$ voltages, controlling $ML$ voltage variations simultaneously. Write/Search buffer inputs stored/search vectors into $SLs$ for matching operations, consistent with \autoref{tab:write opetarion}. During search, all rows compare same input query with stored entries. If mismatch occurs, $ML$ discharges. If $ML$ voltage drops below sense amplifier threshold within specified time window, corresponding SA output transitions to 0, recognized by encoder as mismatch. Conversely, if match occurs, address of stored entry matching search query is output.

%For the purpose of energy and delay analysis, we have expanded the original 1x64 array of 2FeFET-2R TCAM cells into a larger array of m rows and n columns, as illustrated in \autoref{fig:mxn}. This expanded array has the capability to accommodate m words, each with a length of n. Each individual cell within this array possesses identical functionalities and characteristics. All cells within the same column share the $SLs$, allowing for parallel execution of search operations, while cells within the same row share $ML$ voltages, thereby controlling the variations in $ML$ voltage simultaneously. To facilitate the matching operations, we employ Write/Search buffer to input the stored/search vectors into the $SLs$, thereby determining whether each $ML$ discharges or not. The operations during the writing process remain consistent with those outlined in \autoref{tab:write opetarion}. During the search period, all rows compare the same input query with their stored entries. If a mismatch occurs, the $ML$ discharges. If the $ML$ voltage drops below the threshold voltage of the sense amplifier within a specified time window, the corresponding output of the SA transitions to 0, thus being recognized by the encoder as a mismatch. Conversely, if there is a match, the address of the stored entry matching the search query is output.

%Compared to single-row 2FeFET-2R TCAM, expanded m×n array has larger capacity and efficiently executes matching operations during search. By sharing $SLs$ and $ML$ voltage, array achieves highly parallel search operations, improving overall performance and efficiency.


%Compared to a single-row 2FeFET-2R TCAM, the expanded m×n array has a larger capacity and can efficiently execute matching operations during the search period. By sharing $SLs$ and $ML$ voltage, the array can achieve highly parallel search operations, thereby improving overall performance and efficiency.






