
%\documentclass[sigconf,nonacm]{acmart}
\documentclass[conference]{IEEEtran}
\IEEEoverridecommandlockouts

%\settopmatter{printacmref=false} % Removes citation information below abstract
%\pagestyle{empty} % removes running headers


% Some Computer Society conferences also require the compsoc mode option,
% but others use the standard conference format.
%
% If IEEEtran.cls has not been installed into the LaTeX system files,
% manually specify the path to it like:
% \documentclass[conference]{../sty/IEEEtran}

\usepackage{graphicx}
  % declare the path(s) where your graphic files are
   \graphicspath{{./Figures/}}
\usepackage{hyperref}
\usepackage{threeparttable}
\usepackage{tabularx}
\usepackage{amsmath}    
\usepackage{algorithm}
\usepackage{algorithmic}
\usepackage{multirow}
\usepackage{cite}
\usepackage{svg}
\usepackage{color}
\usepackage{adjustbox}
\usepackage{multirow,tabularx}
\usepackage{makecell}
\DeclareGraphicsExtensions{.pdf,.jpeg,.png,.fig, .emf}
   \usepackage{subfigure}
    \usepackage{epstopdf}
    \usepackage{epsfig}
    \usepackage{subfigure}
    \usepackage{epstopdf}
    \usepackage{epsfig}
\usepackage{xspace}
\newcommand{\design}{TAP-CAM\xspace}    
 \usepackage{booktabs}


% Some very useful LaTeX packages include:
% (uncomment the ones you want to load)


% *** MISC UTILITY PACKAGES ***
%
%\usepackage{ifpdf}
% Heiko Oberdiek's ifpdf.sty is very useful if you need conditional
% compilation based on whether the output is pdf or dvi.
% usage:
% \ifpdf
%   % pdf code
% \else
%   % dvi code
% \fi
% The latest version of ifpdf.sty can be obtained from:
% http://www.ctan.org/pkg/ifpdf
% Also, note that IEEEtran.cls V1.7 and later provides a builtin
% \ifCLASSINFOpdf conditional that works the same way.
% When switching from latex to pdflatex and vice-versa, the compiler may
% have to be run twice to clear warning/error messages.






% *** CITATION PACKAGES ***
%
%\usepackage{cite}
% cite.sty was written by Donald Arseneau
% V1.6 and later of IEEEtran pre-defines the format of the cite.sty package
% \cite{} output to follow that of the IEEE. Loading the cite package will
% result in citation numbers being automatically sorted and properly
% "compressed/ranged". e.g., [1], [9], [2], [7], [5], [6] without using
% cite.sty will become [1], [2], [5]--[7], [9] using cite.sty. cite.sty's
% \cite will automatically add leading space, if needed. Use cite.sty's
% noadjust option (cite.sty V3.8 and later) if you want to turn this off
% such as if a citation ever needs to be enclosed in parenthesis.
% cite.sty is already installed on most LaTeX systems. Be sure and use
% version 5.0 (2009-03-20) and later if using hyperref.sty.
% The latest version can be obtained at:
% http://www.ctan.org/pkg/cite
% The documentation is contained in the cite.sty file itself.






% *** GRAPHICS RELATED PACKAGES ***
%
%\ifCLASSINFOpdf
  % \usepackage[pdftex]{graphicx}
  % declare the path(s) where your graphic files are
  % \graphicspath{{../pdf/}{../jpeg/}}
  % and their extensions so you won't have to specify these with
  % every instance of \includegraphics
  % \DeclareGraphicsExtensions{.pdf,.jpeg,.png}
%\else
  % or other class option (dvipsone, dvipdf, if not using dvips). graphicx
  % will default to the driver specified in the system graphics.cfg if no
  % driver is specified.
  % \usepackage[dvips]{graphicx}
  % declare the path(s) where your graphic files are
  % \graphicspath{{../eps/}}
  % and their extensions so you won't have to specify these with
  % every instance of \includegraphics
  % \DeclareGraphicsExtensions{.eps}
%\fi
% graphicx was written by David Carlisle and Sebastian Rahtz. It is
% required if you want graphics, photos, etc. graphicx.sty is already
% installed on most LaTeX systems. The latest version and documentation
% can be obtained at: 
% http://www.ctan.org/pkg/graphicx
% Another good source of documentation is "Using Imported Graphics in
% LaTeX2e" by Keith Reckdahl which can be found at:
% http://www.ctan.org/pkg/epslatex
%
% latex, and pdflatex in dvi mode, support graphics in encapsulated
% postscript (.eps) format. pdflatex in pdf mode supports graphics
% in .pdf, .jpeg, .png and .mps (metapost) formats. Users should ensure
% that all non-photo figures use a vector format (.eps, .pdf, .mps) and
% not a bitmapped formats (.jpeg, .png). The IEEE frowns on bitmapped formats
% which can result in "jaggedy"/blurry rendering of lines and letters as
% well as large increases in file sizes.
%
% You can find documentation about the pdfTeX application at:
% http://www.tug.org/applications/pdftex





% *** MATH PACKAGES ***
%
%\usepackage{amsmath}
% A popular package from the American Mathematical Society that provides
% many useful and powerful commands for dealing with mathematics.
%
% Note that the amsmath package sets \interdisplaylinepenalty to 10000
% thus preventing page breaks from occurring within multiline equations. Use:
%\interdisplaylinepenalty=2500
% after loading amsmath to restore such page breaks as IEEEtran.cls normally
% does. amsmath.sty is already installed on most LaTeX systems. The latest
% version and documentation can be obtained at:
% http://www.ctan.org/pkg/amsmath





% *** SPECIALIZED LIST PACKAGES ***
%
%\usepackage{algorithmic}
% algorithmic.sty was written by Peter Williams and Rogerio Brito.
% This package provides an algorithmic environment fo describing algorithms.
% You can use the algorithmic environment in-text or within a figure
% environment to provide for a floating algorithm. Do NOT use the algorithm
% floating environment provided by algorithm.sty (by the same authors) or
% algorithm2e.sty (by Christophe Fiorio) as the IEEE does not use dedicated
% algorithm float types and packages that provide these will not provide
% correct IEEE style captions. The latest version and documentation of
% algorithmic.sty can be obtained at:
% http://www.ctan.org/pkg/algorithms
% Also of interest may be the (relatively newer and more customizable)
% algorithmicx.sty package by Szasz Janos:
% http://www.ctan.org/pkg/algorithmicx




% *** ALIGNMENT PACKAGES ***
%
%\usepackage{array}
% Frank Mittelbach's and David Carlisle's array.sty patches and improves
% the standard LaTeX2e array and tabular environments to provide better
% appearance and additional user controls. As the default LaTeX2e table
% generation code is lacking to the point of almost being broken with
% respect to the quality of the end results, all users are strongly
% advised to use an enhanced (at the very least that provided by array.sty)
% set of table tools. array.sty is already installed on most systems. The
% latest version and documentation can be obtained at:
% http://www.ctan.org/pkg/array


% IEEEtran contains the IEEEeqnarray family of commands that can be used to
% generate multiline equations as well as matrices, tables, etc., of high
% quality.




% *** SUBFIGURE PACKAGES ***
%\ifCLASSOPTIONcompsoc
%  \usepackage[caption=false,font=normalsize,labelfont=sf,textfont=sf]{subfig}
%\else
%  \usepackage[caption=false,font=footnotesize]{subfig}
%\fi
% subfig.sty, written by Steven Douglas Cochran, is the modern replacement
% for subfigure.sty, the latter of which is no longer maintained and is
% incompatible with some LaTeX packages including fixltx2e. However,
% subfig.sty requires and automatically loads Axel Sommerfeldt's caption.sty
% which will override IEEEtran.cls' handling of captions and this will result
% in non-IEEE style figure/table captions. To prevent this problem, be sure
% and invoke subfig.sty's "caption=false" package option (available since
% subfig.sty version 1.3, 2005/06/28) as this is will preserve IEEEtran.cls
% handling of captions.
% Note that the Computer Society format requires a larger sans serif font
% than the serif footnote size font used in traditional IEEE formatting
% and thus the need to invoke different subfig.sty package options depending
% on whether compsoc mode has been enabled.
%
% The latest version and documentation of subfig.sty can be obtained at:
% http://www.ctan.org/pkg/subfig




% *** FLOAT PACKAGES ***
%
%\usepackage{fixltx2e}
% fixltx2e, the successor to the earlier fix2col.sty, was written by
% Frank Mittelbach and David Carlisle. This package corrects a few problems
% in the LaTeX2e kernel, the most notable of which is that in current
% LaTeX2e releases, the ordering of single and double column floats is not
% guaranteed to be preserved. Thus, an unpatched LaTeX2e can allow a
% single column figure to be placed prior to an earlier double column
% figure.
% Be aware that LaTeX2e kernels dated 2015 and later have fixltx2e.sty's
% corrections already built into the system in which case a warning will
% be issued if an attempt is made to load fixltx2e.sty as it is no longer
% needed.
% The latest version and documentation can be found at:
% http://www.ctan.org/pkg/fixltx2e


%\usepackage{stfloats}
% stfloats.sty was written by Sigitas Tolusis. This package gives LaTeX2e
% the ability to do double column floats at the bottom of the page as well
% as the top. (e.g., "\begin{figure*}[!b]" is not normally possible in
% LaTeX2e). It also provides a command:
%\fnbelowfloat
% to enable the placement of footnotes below bottom floats (the standard
% LaTeX2e kernel puts them above bottom floats). This is an invasive package
% which rewrites many portions of the LaTeX2e float routines. It may not work
% with other packages that modify the LaTeX2e float routines. The latest
% version and documentation can be obtained at:
% http://www.ctan.org/pkg/stfloats
% Do not use the stfloats baselinefloat ability as the IEEE does not allow
% \baselineskip to stretch. Authors submitting work to the IEEE should note
% that the IEEE rarely uses double column equations and that authors should try
% to avoid such use. Do not be tempted to use the cuted.sty or midfloat.sty
% packages (also by Sigitas Tolusis) as the IEEE does not format its papers in
% such ways.
% Do not attempt to use stfloats with fixltx2e as they are incompatible.
% Instead, use Morten Hogholm'a dblfloatfix which combines the features
% of both fixltx2e and stfloats:
%
% \usepackage{dblfloatfix}
% The latest version can be found at:
% http://www.ctan.org/pkg/dblfloatfix




% *** PDF, URL AND HYPERLINK PACKAGES ***
%
%\usepackage{url}
% url.sty was written by Donald Arseneau. It provides better support for
% handling and breaking URLs. url.sty is already installed on most LaTeX
% systems. The latest version and documentation can be obtained at:
% http://www.ctan.org/pkg/url
% Basically, \url{my_url_here}.




% *** Do not adjust lengths that control margins, column widths, etc. ***
% *** Do not use packages that alter fonts (such as pslatex).         ***
% There should be no need to do such things with IEEEtran.cls V1.6 and later.
% (Unless specifically asked to do so by the journal or conference you plan
% to submit to, of course. )


% correct bad hyphenation here
%\hyphenation{op-tical net-works semi-conduc-tor}
%\newcommand{\cg}{\color{red}}
%\newcommand{\andg}{{\tt{AND }}\xspace}

%\setlength{\itemsep}{-1pt}
%\linespread{0.98}
%\newcommand{\cosi}{COSIME}
%\setlength{\textfloatsep}{1.5pt plus 1.0pt minus 2.0pt}
%\abovedisplayskip=0.8pt
%\abovedisplayshortskip=0.8pt
%\belowdisplayskip=0.8pt
%\belowdisplayshortskip=0.8pt
%\copyrightyear{2022}
%\acmYear{2022}
%\setcopyright{acmcopyright}\acmConference[ICCAD '22]{IEEE/ACM International
%Conference on Computer-Aided Design}{October 30-November 3, 2022}{San Diego, CA, USA}
%\acmBooktitle{IEEE/ACM International Conference on Computer-Aided Design (ICCAD
%'22), October 30-November 3, 2022, San Diego, CA, USA}
%\acmPrice{15.00}
%\acmDOI{10.1145/3508352.3549412}
%\acmISBN{978-1-4503-9217-4/22/10}
\newcommand{\chekai}[1]{\textcolor{magenta}{#1}}

\begin{document}
\bstctlcite{IEEEexample:BSTcontrol}

% paper title
% Titles are generally capitalized except for words such as a, an, and, as,
% at, but, by, for, in, nor, of, on, or, the, to and up, which are usually
% not capitalized unless they are the first or last word of the title.
% Linebreaks \\ can be used within to get better formatting as desired.
% Do not put math or special symbols in the title.
% \title{Energy-Efficient Mismatch Detection in 64-Column Arrays Based on Ferroelectric Ternary Content Addressable Memories for 0-6 Bits}
\title{TAP-CAM: A Tunable Approximate Matching Engine based on Ferroelectric Content Addressable Memory}

\author{ 
\small
Chenyu Ni$^1$, Sijie Chen$^1$, Che-Kai Liu$^2$, Liu Liu$^3$, Mohsen Imani$^4$, Thomas Kämpfe$^5$, Kai Ni$^3$, \\
Michael Niemier$^3$, Xiaobo Sharon Hu$^3$,  Cheng Zhuo$^{1,6,*}$, Xunzhao Yin$^{1,6,*}$\\
$^1$Zhejiang University, Hangzhou, China; 
$^2$Georgia Institute of Technology, GA, USA\\
$^3$University of Notre Dame, IN, USA;
$^4$University of California Irvine, CA, USA\\
$^5$Fraunhofer IPMS, Dresden, Germany\\
$^6$Key Laboratory of Collaborative Sensing and Autonomous Unmanned Systems of Zhejiang Province, China\\
$^*$Corresponding authors, email: \{czhuo, xzyin1\}@zju.edu.cn

    }
% author names and affiliations
% use a multiple column layout for up to three different
% affiliations
% \author{\IEEEauthorblockN{Michael Shell}
% \IEEEauthorblockA{School of Electrical and\\Computer Engineering\\
% Georgia Institute of Technology\\
% Atlanta, Georgia 30332--0250\\
% Email: http://www.michaelshell.org/contact.html}
% \and
% \IEEEauthorblockN{Homer Simpson}
% \IEEEauthorblockA{Twentieth Century Fox\\
% Springfield, USA\\
% Email: homer@thesimpsons.com}
% \and
% \IEEEauthorblockN{James Kirk\\ and Montgomery Scott}
% \IEEEauthorblockA{Starfleet Academy\\
% San Francisco, California 96678--2391\\
% Telephone: (800) 555--1212\\
% Fax: (888) 555--1212}}

% conference papers do not typically use \thanks and this command
% is locked out in conference mode. If really needed, such as for
% the acknowledgment of grants, issue a \IEEEoverridecommandlockouts
% after \documentclass

% for over three affiliations, or if they all won't fit within the width
% of the page, use this alternative format:
% 
%\author{\IEEEauthorblockN{Michael Shell\IEEEauthorrefmark{1},
%Homer Simpson\IEEEauthorrefmark{2},
%James Kirk\IEEEauthorrefmark{3}, 
%Montgomery Scott\IEEEauthorrefmark{3} and
%Eldon Tyrell\IEEEauthorrefmark{4}}
%\IEEEauthorblockA{\IEEEauthorrefmark{1}School of Electrical and Computer Engineering\\
%Georgia Institute of Technology,
%Atlanta, Georgia 30332--0250\\ Email: see http://www.michaelshell.org/contact.html}
%\IEEEauthorblockA{\IEEEauthorrefmark{2}Twentieth Century Fox, Springfield, USA\\
%Email: homer@thesimpsons.com}
%\IEEEauthorblockA{\IEEEauthorrefmark{3}Starfleet Academy, San Francisco, California 96678-2391\\
%Telephone: (800) 555--1212, Fax: (888) 555--1212}
%\IEEEauthorblockA{\IEEEauthorrefmark{4}Tyrell Inc., 123 Replicant Street, Los Angeles, California 90210--4321}}




% use for special paper notices
%\IEEEspecialpapernotice{(Invited Paper)}
% \renewcommand{\bibfont}{\scriptsize}

% \let\oldbibliography\thebibliography
% \renewcommand{\thebibliography}[1]{\oldbibliography{#1}
% \setlength{\itemsep}{-0.5pt}} %Reducing spacing in the bibliography.


% % make the title area
% \maketitle

% As a general rule, do not put math, special symbols or citations
% in the abstract

%\vspace{-2ex}
% \begin{IEEEkeywords}
% Content addressable memory, associative search, nonvolatile memory, energy-aware scheme, FeFET.
% \end{IEEEkeywords}
% % no keywords


\maketitle
%\pagestyle{empty}
%\vspace{-3mm}

% For peer review papers, you can put extra information on the cover
% page as needed:
% \ifCLASSOPTIONpeerreview
% \begin{center} \bfseries EDICS Category: 3-BBND \end{center}
% \fi
%
% For peerreview papers, this IEEEtran command inserts a page break and
% creates the second title. It will be ignored for other modes.
% \IEEEpeerreviewmaketitle







\begin{abstract}

Pattern search is crucial in numerous analytic applications for retrieving data entries akin to the query. Content Addressable Memories (CAMs), an in-memory computing fabric, directly compare input queries with stored entries through embedded comparison logic, facilitating fast parallel pattern search in memory.
While conventional CAM designs offer exact match functionality, they are inadequate for meeting the approximate search needs of emerging data-intensive applications. 
Some recent CAM designs propose approximate matching functions, but they face limitations such as excessively large cell area or the inability to precisely control the degree of approximation. 
In this paper, we propose TAP-CAM, a novel ferroelectric field effect transistor (FeFET) based ternary CAM (TCAM) capable of both exact and tunable approximate matching. 
TAP-CAM employs a compact 2FeFET-2R cell structure as the entry storage unit, %for basic storage and computing functions at the unit circuit level, enabling a dense CAM array to enhance energy efficiency. 
and similarities in Hamming distances between input queries and stored entries are measured using an evaluation transistor associated with the matchline of CAM array. 
%Extensive Monte Carlo simulations assess the impact of FeFET device variation. 
The operation, robustness and performance of the proposed design at array level have been discussed and evaluated, respectively. 
We conduct a case study of K-nearest neighbor (KNN) search to benchmark the proposed TAP-CAM at application level.
%Results demonstrate that TAP-CAM achieves a 6.78× energy improvement compared to 2FeFET CAM implementing approximate match functionality, and 16.95× compared to 16T CMOS CAM implementing exact match functionality and 2FeFET CAM implementing approximate match functionality, along with a 3.06\% accuracy enhancement. 
Results demonstrate that compared to 16T CMOS CAM with exact match functionality, TAP-CAM achieves a 16.95$\times$ energy improvement, along with a 3.06\% accuracy enhancement. Compared to 2FeFET TCAM with approximate match functionality, TAP-CAM achieves a 6.78$\times$ energy improvement.


%Pattern search is a key operation in many analytic applications that retrieve the data entry similar to the query. Content Addressable Memories (CAMs), as a type of in-memory computing fabric, can directly compare input queries with all stored entries in the memory through embedded comparison logic to determine whether the input query matches a stored entry, thereby supporting fast parallel associative search. 
%Therefore, CAMs can provide high-performance and efficient hardware solutions in various applications such as data processing and high-speed searching. 
%However, the conventional CAM designs only provide exact match function which outputs the entry exactly matching with the input, thereby cannot satisfy the growing amount of approximate search requirements for emerging data-intensive applications.
%Recently, some CAM designs have been proposed to support approximate match function, but they are limited by either large  cell area overhead, lacking pattern masking capability, or only designed for specific applications. 
%In this paper, we propose TAP-CAM, a novel FeFET based ternary CAM (TCAM), that supports both exact and tunable threshold matching. TAP-CAM utilizes a 2FeFET-2R structure to achieve a dense CAM array based to enhance energy efficiency. The similarities in terms of Hamming distances between the query and stored entries are calculated by an evaluation transistor. Extensive Monte Carlo simulation is conducted to evaluate the impact of the device-to-device variation of FeFET. 
%We use K-nearest neighbor(kNN) search as a case study to benchmark the application-level improvement of TAP-CAM. 
%Results show that TAP-CAM achieves 16.95$\times$ energy improvement and 3.06\% accuracy improvement compared with CMOS-based CAM implementing the exact match function. 


%Advanced machine learning models such as hyperdimensional computing (HDC) and binary neural networks (BNNs) have been extensively studied for brain-inspired cognitive tasks. During their computations, cosine similarity has shown its great importance w9241990140832151214499here intensive inferences are performed based on the angles between the binary query vector and binary stored feature vectors.


%Cosine similarity measures the similarity between two vectors in an inner product space. It is widely used in a number of machine learning models such as hyperdimensional computing (HDC) and deep neural networks. More specifically, during the inference phase of these machine learning applications, a large number of cosine similarity-based searches (CSSs) are often needed.
%Specifically, for the prevalent binary neural networks (BNN) and hyperdimensional computing (HDC) models, cosine similarity plays a critical role in 
%\textcolor{red}{move the first paragraph to Intro}
%Cosine similarity measures the similarity between two vectors in an inner product space, and  has been widely used in a number of machine learning models, 
% Content addressable memory (CAM) has  been widely utilized as associative memory for data-intensive workloads, thanks to its  parallel in-memory pattern-matching capability. However, traditional CAM designs struggle to maintain their energy efficiency and performance advantages due to the limited number of exact matches between the stored entries and query patterns, especially considering the ever-growing amount of data. 
% %As such, efficient CAM designs and optimizations for approximate matching capability are highly desired.
% To address this challenge,
% we propose TAP-CAM, a novel tunable approximate matching engine based on a ferroelectric field effect transistor (FeFET) CAM.
% The FeFET CAM cell employs a low-power 1FeFET1R structure that integrates a series resistor current limiter into the intrinsic FeFET structure. 
% An evaluation transistor connected to the NOR-type matchline (ML) of the  CAM array  controls the ML discharge rate,  enabling TAP-CAM to perform approximate search operations. 
% By computing the Hamming distances (HDs) between the stored pattern entries and the input query,  TAP-CAM generates a match output for entries with an HD below a tunable matching threshold.
% We thoroghly analyze and validate the scalability and robustness of the proposed TAP-CAM array.
% Moreover, we evaluate the performance of TAP-CAM, which demonstrates a remarkable $16.95\times$ improvement in terms of energy efficiency compared to its CMOS based counterpart. 
% When TAP-CAM is deployed in the K-nearest neighbor (KNN) model for classification tasks, benchmarking results reveal significant improvements in  energy efficiency and delay (AA and BB, respectively). Furthermore, the application accuracy indicates an average improvement of 3.06 \%  compared to the existing exact match CAM, highlighting the value and effectiveness of our proposed \design design.
%for energy-efficient in-memory approximate matching applications, playing a significant role in artificial intelligence hardware acceleration and epidemic virus gene identification. Unlike existing CMOS-based structures, we designed a 2FeFET-2R RAM based on the novel FeFET device. The intrinsic polarization memory function of the FeFET greatly simplifies the circuit structure, reducing area and energy consumption overhead. In this paper, the 2FeFET-2R RAM can distinguishes adjacent mismatched bits by regulating the gate voltage of the evaluation transistor and comparing the voltage on the match line, achieving differentiation of 0-6 mismatch situations in a 64-bit long array. Through extensive Monte Carlo simulations, we analyzed the robustness of the circuit under different processes. Meanwhile, we evaluate the performance of the ferroelectric TCAM approach in a single search operation. Compared to the traditional CMOS-based structure, the energy efficiency has improved by 20 times, and the circuit area has been reduced by 6 times.
% \begin{IEEEkeywords}
% In-memory Computing, Cosine Similarity, FeFET, Associative Memory (AM)
% \end{IEEEkeywords}
\end{abstract}

%\vspace{-3ex}
\section{Introduction}
The transforming proximity operations and docking system (TPODS) is a conceptual 1U CubeSat module, developed at the Land, Air and Space Robotics (LASR) laboratory of Texas A\&M University \cite{TPODS_system,TPODS_estm,TPODS_GNC24}. The overall objective of the TPODS module is to enable servicing of a tumbling resident space object (RSO). The TPODS modules are stowed in a mothership, which has the necessary sensor suite to locate and approach an RSO. Once in the vicinity, the mothership analyzes the tumbling motion of the RSO and deploys multiple TPODS towards the object. TPODS then leverage non-adhesive attachment mechanisms to firmly affix with the RSO. 

This deployment strategy imparts a specific momentum change, resulting in significant reduction in the rotation rate of the RSO \cite{GNC_24_down}. However, for most practical applications, additional momentum transfer is required to completely detumble the object \cite{TPODS_detumble}. To perform a powered de-tumbling operation in a fuel efficient manner, it is often required to relocate the TPODS modules from their initial position on the body to achieve a better momentum lever. Since the RSO is still under substantial tumbling, the relocation process can be challenging, particularly with the uncertainty in pose estimates of each agent. If the uncertainties are not considered during the motion planning of relocation, it can result in catastrophic consequences due to the proximity of modules and the RSO. 

Once a stable rotation of the RSO is achieved, the TPODS modules can be rearranged to form various scaffolding structures to enable docking of a more capable servicing vehicle. Figure~\ref{fig:scafolding} presents one such example workflow. Since the TPODS modules now have to maneuver in a highly dynamical environment, a safety focused motion planning approach is necessary. Although more accurate pose determination via monocular vision sensors is available at shorter TPODS-to-TPODS distances, for the majority of the scaffolding generation process, the pose of each agent is driven by relative ranging and has significant associated uncertainties \cite{ICRA25,Ali_GNC24}. 

\begin{figure}[t!]
     \begin{subfigure}[b]{0.32\textwidth}
        \centering
         \includegraphics[width=\textwidth]{Figures/scafolding_start.png}
         \caption{Initial Position}\label{fig:1a}
     \end{subfigure}   
     \begin{subfigure}[b]{0.33\textwidth}
        \centering
         \includegraphics[width=\textwidth]{Figures/scafolding_1.png}
         \caption{Structure 1}\label{fig:1b}
     \end{subfigure}
     \centering
     \begin{subfigure}[b]{0.31\textwidth}
        \centering
         \includegraphics[width=\textwidth]{Figures/scafolding_2.png}
         \caption{Structure 2}\label{fig:1c}
    \end{subfigure}
    \caption{Scaffolding generation to enable servicing of RSO}
    \label{fig:scafolding}
\end{figure}

In considering safety of these autonomous systems, encoding notions of safety directly into an existing controller can be very useful. A popular approach to assuring safety in this manner is through the use of \textit{control barrier functions (CBFs)} \cite{ames_2017,ames2019control}, which provide sufficient conditions for forward invariance of safe sets. For control affine dynamics, these sufficient conditions become linear in control and often produce \textit{sets} of safe controls rather than a single one. Thus, a safe control signal can be found which both lies in such a set, and extremizes some cost function. Typically, the cost function is designed to minimize the deviation between some nominal or legacy controller, and the safe control signal. Because for affine dynamics these safe sets of controls are convex, the optimization problem can be solved very efficiently, making this solution especially appealing for spacecraft proximity operations where compute is scarce. Recognizing this fact, researchers have utilized CBF-based controllers for spacecraft docking \cite{dunlap2021comparing,Breeden_2022_docking}, spacecraft inspection \cite{dunlap2023RTA_inspection,vanWijk2024JAIS,dunlap2024run,hibbard_guaranteeing_2022}, safe reorientation \cite{breeden_attitude}, and generating safe trajectories in the presence of disturbances \cite{breeden_robust_2023,vanWijk_DRbCBF_24}.

The main contributions of the manuscript are threefold. First, we develop a guidance algorithm for relocation of multiple TPODS agents which use a multiplicative extended Kalman filter (MEKF) based estimator for state estimation. A differentiable collision detection and avoidance approach that considers the shape of the RSO is implemented to prevent collisions between TPODS agents and the RSO. Second, we design constraints enforced by CBFs to ensure multi-agent system safety informed by the MEKF, resulting in the safe operation of TPODS in close proximity. Lastly, a hybrid approach of TPODS-RSO and TPODS-TPODS collision avoidance is proposed and the efficacy of the approach is demonstrated with extensive simulation analysis of a pragmatic scenario of simultaneous relocation of two modules on a tumbling RSO.

% 1) ekf with sensor fusion + guidance
% 2) safety multi agent
% 3) sim and hardware demo

% {\color{red} David finish:
% \begin{itemize}
%     \item Main contributions
%     \item Organization of paper
% \end{itemize}}
% \vspace{-2ex}
% \vspace{-1em}
\section{Background}
\label{sec:background}

In this section, we discuss the structure and operational principles of FeFETs, and review existing CAM design works.

\begin{figure}%[H]
    \centering
    \includegraphics[width=1\linewidth]{Figures/1FRIV1.png}
   % \vspace{-0.4cm}
    \caption{\textbf{(a)} FeFET polarization directions and channel conditions after memory write operations;  \textbf{(b)} The FeFET $\textit{I}_\textit{D}$-$\textit{V}_\textit{G}$ characteristics after positive/negative gate write; % Source is grounded; 
    \textbf{(c)} 1FeFET-1R structure and equivalent circuit; \textbf{(d)} The 1FeFET-1R $\textit{I}_\textit{D}$-$\textit{V}_\textit{G}$ characteristics after positive/negative gate write.
    %Source is grounded.
    }
   
 
    \label{fig:fefet}
   %  \vspace{-0.4cm}
\end{figure}

%\vspace{-2ex}
%\vspace{-0cm}
\subsection{FeFET Basics}
\label{sec:device}
\setlength{\abovecaptionskip}{2pt}
\setlength{\belowcaptionskip}{2pt}


Recent advancements in ferroelectric material, particularly hafnium oxide ($\text{HfO}_\text{2}$), have spurred research interest in ferroelectric transistors  and the development of non-volatile circuit designs compatible with CMOS technology \cite{yin2020fecam}. 
FeFETs 
%belong to the subclass of metal-oxide-semiconductor field-effect transistors (MOSFETs) and 
incorporate a ferroelectric 
(FE) layer  within the gate stack. These devices exhibit unique electrical hysteresis characteristics, exhibiting reversible polarization states upon an applied voltage-driven electric field. 
%Integration of a ferroelectric capacitor with the MOSFET gate capacitor confers FeFETs with adjustable hysteresis characteristics. 
The FE layer induces a shift in the threshold voltage of the FeFET depending on the orientation of FE polarization \cite{FeFET-capacitor}, enabling non-volatile (NV) storage capabilities. 
By applying gate voltage pulses, such as -4V/+4V, to a FeFET device, as depicted in \autoref{fig:fefet}(a), it can be programmed to store low and high $\textit{V}_\textit{TH}$ states corresponding to logic ‘0’ and ‘1’, respectively. 
The associated hysteresis  $\textit{I}_\textit{D}$-$\textit{V}_\textit{G}$ transfer characteristics are shown in \autoref{fig:fefet}(b) \cite{transfer-characteristics}. FeFETs, being voltage-driven for read and write operations, exhibit superior energy efficiency compared to two-terminal current-driven NVMs.

%In recent years, with the continuous advancement of ferroelectric material hafnium oxide ($\rm{HfO_2}$),  there has been a growing focus among researchers on ferroelectric transistors and the exploration of non-volatile circuit structures compatible with CMOS technology \cite{yin2020fecam}. Ferroelectric gate field-effect transistors (FeFETs) represent a subclass of metal-oxide-semiconductor field-effect transistors (MOSFETs) that incorporate a ferroelectric layer (FE) within the gate stack, These devices possess distinctive electrical properties capable of reversible polarization under an applied electric field. The integration of a ferroelectric capacitor with the MOSFET gate capacitor grants FeFETs adjustable hysteresis characteristics. The ferroelectric layer introduces a shift in the threshold voltage contingent upon the orientation of ferroelectric polarization \cite{FeFET-capacitor}, resulting in non-volatile (NV) storage capabilities. 
%By applying gate voltage pulses, such as -4V/+4V, to a FeFET device, as illustrated in \autoref{fig:fefet}(a), the device can be programmed to exhibit low $V_{TH}$ and high $V_{TH}$ states corresponding to logic `0' and `1', respectively.
%FeFET have high energy efficiency and low energy consumption, making them a new type of device with great potential for application.
%We define that only two states of information exist in FeFET: logic '0' (high $ V_{TH}$) and logic '1' (low $ V_{TH}$), and logic '0' and logic '1' can be written by applying -4V/+4V pulses as shown in \autoref{fig:fefet}(a). 
 %The corresponding $I_D$-$V_{G}$ transfer characteristics  are depicted in \autoref{fig:fefet}(b) \cite{transfer-characteristics}.
%Given that FeFET read and write operations are voltage-driven, FeFETs demonstrate superior energy efficiency compared to two-terminal current-driven NVMs.

When the FeFET operates as a current source, its ON current gradually increases with the rise in gate voltage, as depicted in \autoref{fig:fefet}(b). Consequently, there's a certain variability in the conduction current regarding the gate read voltage. 
To ensure stable ON current during operation and enhance the design robustness, a current limiter is connected to the source of the FeFET, as shown in the equivalent circuit of \autoref{fig:fefet}(c).
Prior studies \cite{1FeFET1R-transfer, yin2023ultracompact} have shown that a series resistor on the drain/source of a FeFET can regulate the ON current, with 1FeFET-1R integration experimentally demonstrated \cite{area}. Such integration suppresses the ON current variability, making it independent of the $\textit{V}_\textit{TH}$ state and gate voltage when the series resistor is sufficiently large. 
The transfer characteristic curve of the 1FeFET-1R structure is depicted in \autoref{fig:fefet}(d).  
We adopt the 1FeFET-1R structure using a series resistor as a current limiter in this work. 
This approach mitigates the impact of ON current variability on \textit{ML} discharging in a CAM array 
%and reduces high energy consumption, 
achieving low power consumption and robust tunable approximate matching functionality.


%\autoref{fig:fefet}(b) demonstrates that even after entering the saturation region, the ferroelectric transistor behaves as a current source, with the ON current continuing to gradually increase with the rise in gate voltage. Consequently, there exists a certain variability in the conduction current concerning the gate read voltage. Given the requirement for approximate search functionality in the designed circuit, heightened demands are placed on the precision and stability of the current. 
%increases continuously with the increase of gate voltage. 
%After entering the saturation region, there will still be a slight increase in current. 
%Therefore, in order to ensure the stable ON current of the bitcell during operation and enhance the overall robustness of the circuit, a current limiter is connected to the source of the FeFET to constrain the variability of the ON current. In the equivalent circuit, this scenario presents a voltage division between the resistance of the ferroelectric transistor and the current limiter, as depicted in \autoref{fig:fefet}(c). 
%Previous studies \cite{1FeFET1R-transfer, yin2023ultracompact} have demonstrated that a series resistor on the drain/source of FeFETs can govern the ON current of FeFET devices, and such 1FeFET-1R integration has been experimentally verified \cite{area}.
%As a result, the variability in ON current is notably suppressed, and the ON current becomes independent of the $V_{TH}$ state, determined solely by the series resistor when the resistor is sufficiently large. The transfer characteristic curve of the 1FeFET-1R circuit is depicted in \autoref{fig:fefet}(d). 
%, which adds a limiter to control the change of the opening current, that is, a resistor with a larger resistance value is connected in series, and the resistance value of the resistor should be greater than the effective conductance resistance of the FeFET, so that the obtained opening current is not affected by $V_G$ and $V_{TH}$, but only controlled by $V_D$ and $R_S$ in \autoref{fig:fefet}(d).
%In this work, we adopt the 1FeFET-1R structure utilizing a series resistor as a current limiter. This approach not only mitigates the impact of ON current variability on the matchline discharging of a CAM array, but also reduces the high energy consumption caused by excessive ON current, thereby achieving low power consumption and tunable approximate matching functionality.

%This can improve the stability of the current \cite{stability}, and the obtained $ I_D$-$ V_{GS}$ curve is calibrated based on Preisach's FeFET model \cite{transfer-characteristics}.


%It's the voltage and the pulse width applied to gate that determine the memory window in FeFET devices\cite{Ni_2019},(Fig.\ref{FeFET}).
% \begin{figure}[H]
%     \centering
%     \begin{minipage}{0.47\linewidth}
%     \includegraphics[width=\linewidth]{Figures/FeFET.png}
%     \caption*{(a)}
%     \end{minipage}
%     \hspace{.1ex}
%     \begin{minipage}{0.47\linewidth}
%     \includegraphics[width=\linewidth]{Figures/i_v64.eps}
%     \caption*{(b)}
%     \end{minipage}
%     \caption{(a) FeFET polarization changes after applying a voltage pulse at the gate terminal; (b) FeFET I-V  curve we used in SPICE simulation.}
%     \label{FeFET}
    
% \end{figure}


   
%\vspace{-1ex}
\subsection{Existing CAM Designs}
\label{sec:existing_work}

\begin{figure}
    \centering
    \includegraphics[width=\linewidth]{Figures/bg4.png}
  %  \vspace{-0.4cm}
    \caption{Schematics of \textbf{(a)} 16T CMOS TCAM cell; \textbf{(b)} 2T-2ReRAM TCAM cell; \textbf{(c)} 20T-6MTJ TCAM cell; \textbf{(d)} 2FeFET TCAM cell.}
  %  \vspace{-0.4cm}
\label{fig:CAM}
\end{figure}


%\begin{figure}
%    \centering
%    \includegraphics[width=\linewidth]{Figures/conventional_CAM.pdf}
%    \caption{(a)Architecture of an M × N TCAM array.(b)NOR-type CAM bitcell.}
%\label{fig:array}
%\end{figure}


Various CAM designs have been proposed based on CMOS technology and NVM devices. A conventional 16T CMOS TCAM cell is shown in \autoref{fig:CAM}(a). CAMs leveraging NVM typically demonstrate enhanced performance over CMOS-based counterparts. For example, a 2T-2R TCAM design based on ReRAM was proposed in \cite{jing2t2r} for its compact structure, as shown in \autoref{fig:CAM}(b). While it consumes less area compared with conventional CMOS-based CAM designs, %issues arise primarily due to 
the low HRS/LRS ratio, low variable resistance and current-driven write-in mechanism associated with large access transistors  make the write and search energy significant concerns. 
\cite{20T6MTJ} proposed a 20T-6MTJ TCAM design as illustrated in \autoref{fig:CAM}(c), greatly enhancing the search speed and search performance. However, the reduced sense margin caused by the limited TMR ratio of STT-MRAM necessitates numerous transistors to address this issue, thus severely impacting area and power consumption.

Among NVM based CAM designs, utilizing FeFET stands out due to its high ON/OFF current ratio, efficient voltage-driven write mechanisms, low energy consumption, and cost-effectiveness, enabling significant performance improvements compared to conventional CMOS designs and other NVM-based designs. Building upon advanced FeFET models, researchers have proposed various FeFET CAM designs, particularly designs of TCAM. 
The 2FeFET TCAM design as depicted in \autoref{fig:CAM}(d) offers a compact alternative than CMOS counterparts \cite{2FeFET}. 2FeFET TCAM features a smaller cell area, reduced write and search energy consumption, and search delay. 
However, it faces limitations such as the lack of support for approximate matching functionality. 
%Our design will focus on addressing these issues.



%\autoref{fig:CAM}(a) illustrates a 4T-2FeFET cell, which incurs considerable energy and area overheads due to the presence of 4T cells. Furthermore, 
%Here we introduce a 2FeFET-2R CAM cell, depicted in \autoref{fig:CAM}(c), which incorporates the use of a drain resistor, similar to the 1FeFET-1R cell used for compute-in-memory applications \cite{power2}. The 2FeFET-2R cell extends beyond exact match functionality by employing a self-referenced sense amplifier to measure the Hamming distance (HD) between the input query and stored entries. Non-trivial sensing circuitry is required to effectively sample the limited HD.

%Various CAM designs have been proposed based on both CMOS and NVM devices so far. Utilizing NV technology such as FeFET distinguishes itself among various memory technologies due to its high ON/OFF current ratio, efficient voltage-driven write mechanisms, low energy consumption, and cost-effectiveness, thus enabling significant performance improvements compared to conventional CMOS designs. Building upon advanced FeFET models, researchers have proposed various FeFET CAM designs, particularly prevalent in the field of TCAM.  \autoref{fig:CAM}(a) shows a 4T-2FeFET cell, which incurs considerable energy and area overheads due to the presence of 4T cells. Furthermore, 2FeFET cells have been introduced as depicted in \autoref{fig:CAM}(b), which offer a more compact alternative than CMOS counterparts \cite{2FeFET}.
%Here we introduce a 2FeFET-2R CAM cell as shown in \autoref{fig:CAM}(c), which incorporates the use of a drain resistor previously studied in several works, similar to the 1FeFET-1R cell used for compute-in-memory applications \cite{power2}. 
%The 2FeFET-2R cell extends beyond the exact match functionality by employing a self-referenced sense amplifier to measure the Hamming distance (HD) between the input query and stored entries. Non-trivial sensing circuitry is required to effectively sample the limited HD.

\subsection{Threshold Matching Concepts and Related Works}
\label{sec:existing_work}

\begin{figure}
    \centering
    \includegraphics[width=\linewidth]{Figures/threshold_match.png}
  %  \vspace{-0.4cm}
    \caption{\textbf{(a) Exact match:} The stored entry that matches exactly with the query; \textbf{(b) Best match:} The stored entry that has the smallest distance to the query; \textbf{(c) Threshold match:} The stored entry whose distance to the query is below specified thresholds.}
 %   \vspace{-0.4cm}
\label{fig:Matchstyle}
\end{figure}


Most CMOS and NVM based CAM designs discussed earlier prioritize exact matching, as depicted in \autoref{fig:Matchstyle}(a), 
limiting their adaptability for data-intensive applications. 
%For data-centric applications,
In contrast, approximate matching gains favor due to its potential to enhance hardware utilization while maintaining acceptable accuracy. 
As a means to achieve approximate matching, best match CAMs, as illustrated in \autoref{fig:Matchstyle}(b), aim to output the stored entry with the highest similarity to the search query. 
For example, A-HAM \cite{AHAM} evaluates similarities across stored entries and identifies the closest Hamming distance to the input query. 
4T-2MTJ utilizing STT-MRAM \cite{STTMRAM} measures similarity between input query and stored entries in terms of \textit{ML} current and outputs the entry with the highest similarity. \cite{bestmatch} introduced a CAM design for minimum Hamming distance search using digital circuits for bit comparison. A Winner-Take-All (WTA) circuit at the output selects the entry with the highest degree of matching to the search query. However, CAMs designed for best matching may fail in applications requiring the output of multiple entries with specific similarities. Therefore, threshold matching CAMs were devised. 

Threshold matching CAMs, as illustrated in \autoref{fig:Matchstyle}(c), aim to provide multiple stored entries with similarity within a predefined Hamming distance (HD) threshold. 
For instance, the HD-CAM proposed in \cite{conventionalCAM} utilizes a 10T CMOS-based design incorporating \textit{ML} charge redistribution, enabling threshold matching with large HD tolerance, notably used in virus DNA classification. However, the SRAM based HD-CAM cell incurs substantial area and energy overheads. Furthermore, its effectiveness is limited in discerning patterns with substantial HDs due to the intricate tuning of \textit{ML} discharge current, making bit-by-bit tuning of HD thresholds impractical.
\cite{liu2023reconfigurable} introduced MHCAM, a multi-state CAM design encoding multiple CAM cells into distinct multi-states per dimension to perform both dimension-wise exact matching and reconfigurable threshold matching. However, additional transistors introduce fixed bit precisions (1-bit/2-bit/4-bit/8-bit per dimension), restricting fine-grained tunability in threshold matching and adaptability to applications demanding multi-state HD. The ReRAM-based CAM proposed in \cite{MASC} implements threshold matching by leveraging voltage scaling and controlling the precharge period. However, the current-driven mechanisms of ReRAMs result in high power consumption during operation and limited HD thresholds can be achieved due to the large \textit{ML} discharge current and non-trivial threshold-associated period sampling. \cite{2FeFETa} implements approximate matching functionality based on 2FeFET TCAM. It calculates the HD between search and stored vectors in a parallel manner by sensing the discharge rate of \textit{ML}. While achieving high energy efficiency and density in TCAM, it lacks precise control over the degree of approximate searching.

These threshold search CAMs all face a common issue, that they cannot precisely control the degree of approximate matching. Therefore, our design will focus on implementing bit-by-bit tuning of threshold to control the degree of approximate matching.



% To address these challenges, we propose a 2FeFET-2R CAM design, leveraging the advantages of 1FeFET-1R structure for compact, energy-efficient, and flexible bit-by-bit tunable HD threshold matching. We elaborate on our design in subsequent sections.




%Threshold matching CAMs illustrated in \autoref{fig:Matchstyle}(b) aim to provide multiple stored entries with similarity within a predefined Hamming distance (HD) threshold. 
%For instance, the HD-CAM proposed in \cite{conventionalCAM} employs a 10T CMOS-based design incorporating ML charge redistribution, enabling threshold matching with large HD tolerance, notably used in virus DNA classification.
%However, the SRAM based HD tolerant CAM cell incurs substantial area and energy overheads. Furthermore, its effectiveness diminishes in discerning patterns with substantial HDs due to the intricate tuning of ML discharge current, making bit-by-bit tuning of HD thresholds impractical.

%Most of aforementioned CMOS and NVM based CAM designs primarily focus on performing exact match as depicted in \autoref{fig:Matchstyle}(a), which restricts their adaptability to a wide range of emerging applications in the era of big data.
%However, in the context of data-centric applications,  approximate matching is increasingly favored. This approach offers the potential to significantly enhance hardware utilization while maintaining an acceptable level of  accuracy.
%In particular, threshold matching CAMs, illustrated in \autoref{fig:Matchstyle}(b), are designed to provide multiple stored entries that have a similarity within a specified Hamming distance (HD) threshold with the search query.



%An example is the HD tolerant CAM proposed in
%\cite{conventionalCAM}, which employs a 10T CMOS-based design incorporating a ML charge redistribution technique.
%This design implements threshold match with large HD tolerance, and is notably used in virus DNA classification. 
%Nevertheless, it's worth noting that the HD tolerant CAM, while functional, consumes substantial area and energy overheads. Moreover, its effectiveness is limited in discerning patterns  with substantial HDs  due to the intricate tuning of ML discharge current. This renders the bit-by-bit tuning of HD thresholds impractical. 

%The ReRAM-based CAM proposed in \cite{MASC} implements threshold matching by leveraging voltage scaling and controlling the precharge period.
%However, the current-driven mechanisms of ReRAMs result in high power consumption during operation, limiting achievable HD thresholds due to the large ML discharge current and non-trivial threshold-associated period sampling.

%\cite{liu2023reconfigurable} introduced MHCAM, a multi-state CAM design encoding multiple CAM cells into distinct multi-states per dimension to perform both dimension-wise exact matching and reconfigurable threshold matching. However, additional transistors introduce fixed bit precisions (1-bit/2-bit/4-bit/8-bit per dimension), restricting fine-grained tunability in threshold matching, and adaptability to applications demanding multi-state HD.

%\cite{2FeFETa} implements approximate matching functionality based on 2FeFET TCAM. It calculates the Hamming distance between search vectors and storage vectors in a massively parallel manner by sensing the discharge rate of ML. While achieving high energy efficiency and density in TCAM, it lacks precise control over the degree of approximate searching, which is the focus of our design breakthrough.

%The ReRAM-based CAM proposed in \cite{MASC} implements threshold matching by leveraging voltage scaling and controlling the precharge period. However, the current-driven mechanisms of ReRAMs result in high power consumption during operation, and limited HD thresholds can be achieved due to the large ML discharge current and non-trivial threshold associated period sampling.

%\cite{liu2023reconfigurable} proposed MHCAM, a multi-state CAM design that encodes multiple CAM cells into distinct multi-states per dimension to perform both dimension-wise exact matching and reconfigurable threshold matching. However, this approach introduces additional transistors to implement fixed bit precisions (i.e., 1-bit/2-bit/4-bit/8-bit per dimension), limiting its adaptability to specific applications demanding multi-state HD.




%The ReRAM-based CAM proposed in \cite{MASC} implements the threshold matching by leveraging voltage scaling and controlling the precharge period.
%However, the current-driven mechanisms of ReRAMs result in high power consumption during the operation and limited HD thresholds can be achieved due to the large ML discharge current and non-trivial threshold associated period sampling.
%to tolerant a few HD, thus  controls the search mode as either exact or approximate by selectively controlling the precharge period. 
%\cite{liu2023reconfigurable} proposed MHCAM, a multi-state CAM design that encodes multiple CAM cells into distinct multi-states per dimension  to perform both dimension-wise exact matching and reconfigurable threshold matching. However, additional transistors are introduced to implement fixed bit precisions (i.e., 1-bit/2-bit/4-bit/8-bit per dimension). Yet, this approach unavoidably restricts the extent of tunability in threshold matching, limiting its adaptability to specific applications demanding multi-state HD.
 

%The conventional n$\times$m CAM model is shown in the 
%CAM  searches  the input query across the stored entries in parallel, conducting a single input multiple output operation.
%\autoref{fig:array}(a) shows the conceptual schematic of a CAM array, which  performs bit-wise comparison between input query  and stored entries in parallel, conducting single input multiple output operations. 
%Each ML corresponds to the storage content in each row of the CAM, and all MLs are connected to a sense amplifier (SA). 
%Searchlines (SL/$\overline{SL}$) are placed vertically to write stored data and apply input data to the cells within the same column, while wordlines (WLs) are shared horizontally by the cells within a row.
%The storage in each column is connected to a pair of complementary search lines (SL/$\overline{SL}$), which control the writing and searching of rows. 
%Precharge transistors are used to precharge the MLs.
%The sense amplifier (SA) of each word measures the voltage of matchline (ML), which connects all the cells within the word, and is discharged depending on the accumulated bit-wise comparison results.
%and the voltage change of the ML is amplified and characterized by the SA \cite{conventionalCAM}.

%A single CAM unit stores content using a pair of cross-coupled inverters, and the MOS transistor on the row is activated by the writeline (WL) to drive SL and $\overline{SL}$ with complementary voltage values for writing. After the writing is completed, the SLs are precharged for reading and searching. The process and principle of searching are described as follows: first, the ML needs to be precharged to the high level of VDD, and during this process, the SLs need to be controlled at a low level to prevent ML from discharging. Then, turn off the precharge transistor and perform searching on the SL. If the search content on the SL matches the content stored in the unit (SL=D), the gate voltage of $M_{c3}$ is low, and ML cannot discharge, remaining at a high level. If it does not match (SL=$\overline{D}$), $M_{c1}$ and $M_{c2}$ control the gate voltage of $M_{c3}$ to be high, $M_{c3}$ opens, and ML discharges, lowering the voltage level. Therefore, any mismatched bit will be manifested as a decrease in ML voltage after amplification by the SA, resulting in an overall mismatch.

%The most basic content-addressable memory is a 16T CMOS CAM, which operates based on NOR and NAND cells. During the search process of the NAND cell, the transistor responsible for pre-charging needs to be charged to the power supply voltage. When searching for a storage match, all nMOS transistors are turned on, creating a path between the ML and ground, allowing the ML to discharge. If there is a bit miss, the ML will not discharge and remains at a high level. The disadvantage of the NAND matchline is that it depends on the quadratic delay of the number of cells, resulting in large parasitic capacitance and series resistance to ground, and low noise margin, so it is not widely used \cite{16TCMOS}.
%In comparison to the NAND cell, the NOR cell is more commonly used. The search cycle of the NOR cell is divided into three stages: searchline pre-charging, matchline pre-charging, and matchline evaluation. First, disconnect the matchline from ground, pre-charge the searchline to a low level, and charge the matchline to a high level. Then, drive the searchline with the content to be searched to evaluate the ML. The content to be searched is compared with the stored content based on the level change of the ML. The NOR cell has a fast evaluation speed. In the slowest 1-bit miss case, the critical matchline is composed of two series-connected transistors, but the number of transistors is high, resulting in high cost and power consumption.


%\vspace{-1ex}




%Correcting-Match Scheme to achieve soft-error tolerance, but typically only allow for a limited Hamming Distance of 1-4 bits \cite{softerror1}, \cite{softerror2}. 
%In \cite{conventionalCAM}, a dynamic and configurable search method was proposed, where the user can decide the threshold of the allowed mismatch, resulting in the design of a Hamming Distance Tolerant CAM (HD-CAM) whose parameters can be adjusted by the user. The schematic diagram of a bitcell in the HD-CAM is similar to \autoref{fig:accuracy}(a). It adds an evaluation transistor to the NOR-type CAM to control the evaluation voltage (i.e., the threshold voltage for mismatches). When Veval is set to the maximum voltage $V_{DD}$, it can perform a precise search. Alternatively, $V_{eval}$ can be set to other values less than $V_{DD}$ to perform an approximate search.

%However, the HD-CAM design using the 65nm CMOS technology has a large bitcell area and high cost, and consumes a large amount of power during search. It cannot distinguish between cases where the number of mismatched bits is close, which are all areas for future improvements of HD-CAM.


%\vspace{-1ex}
%\subsection{CAM Designs Based on FeFET}

%\label{sec:existing_work}
%To address the cost and power consumption issues, researchers have used new FeFET technology to design 4T-2FeFET TCAM \cite{4T2FeFET} and 2FeFET TCAM structure \cite{2FeFET}. Based on the material's non-volatility, FeFETs have the characteristic of storing and computing in one unit, thereby reducing the number of transistors used for writing and searching in the CMOS process, reducing the design area and cost. 

%First, we consider a 4T-2FeFET design in the context of a multi-domain Preisach model in Sec. \ref{sec:existing_work}(a). The design sets the write and read voltages to be 4V and 1V, respectively, and the write operation is divided into two steps. For example, when writing logic '1' to a TCAM cell, a high-level write voltage is used to drive the writeline (WL), which opens the $T_{3}$ and $T_{4}$ transistors. In the first step, $V_{write}$/0 is used to drive the gates of the $M_{1}$/$M_{2}$ FeFETs, polarizing $M_{1}$ and writing logic '1'. In the second step, 0/-$V_{write}$ is used to drive the gates of the $M_{1}$/$M_{2}$ FeFETs, writing logic '0' to $M_{2}$. During the search process, $V_{search}$ is used to drive the WL and two bitlines, thereby opening the FeFETs, while 0/$V_{search}$ is used to drive the searchline \cite{2FeFET}.

%For a fair comparison, we take 2FeFET design into consideration. The 2FeFET TCAM cell unit is shown in the Sec. \ref{sec:existing_work}(b), where a pair of parallel FeFETs connect the drain to the matchline (ML), the gate is connected to the bitline/searchline (BL/$\overline{SL}$ and $\overline{BL}$/SL), and the source is connected to the sourceline (ScL), driven by the write voltage through ScL.

%Taking writing logic '1' to the 2FeFET TCAM as an example, the specific writing process is described as follows:

%first, writing logic '1' to M1 requires adding $V_{write}$ to the BL/$\overline{SL}$ end, grounding the $\overline{BL}$/SL and ScL ends, ensuring that M1 $V_{GS}$ is high, and the ferroelectric is polarized, successfully writing logic '1' to $M_{1}$.Then, writing logic '0' to $M_{2}$ is similar to the first step, except that ScL is connected to $V_{write}$ to ensure that $M_{2}$ $V_{GS}$ is low, equivalent to writing logic '0' to $M_{2}$.
%The writing process for other content is shown in the table. 
%To perform a search, the ML voltage needs to be precharged and then apply the corresponding search voltage on SL/$\overline{SL}$. Logic '1' corresponds to a high voltage, and logic '0' corresponds to a low voltage. The search content is judged by the voltage change on the ML.

%Compared with conventional CMOS technology and 4T-2FeFET design, the 2FeFET TCAM structure has the advantages of low power consumption, low delay, and small area. However, it also has some limitations: the 2FeFET structure cannot select whether to match at a fixed voltage point, and there is no fixed sensetime to distinguish between cases where the number of mismatched bits is close. 






%In addition to threshold matching CAMs, many CMOS and NVM-based CAM designs incorporate approximate matching functionality through other means. 




%In this paper, we concentrate on CAMs with threshold matching functionality.
%To overcome the aforementioned challenges faced by these threshold matching CAMs, we propose a 2FeFET-2R CAM design, which implements compact, energy-efficient, and flexible bit-by-bit tunable HD threshold matching by exploiting the advantages of 1FeFET-1R structure, i.e., suppressed ON current with negligible variability, single transistor $AND$ logic, voltage-driven write and read mechanisms, etc. We elaborate on our design in the following sections.
%Therefore, considering the approximate search function and the high-energy efficient new devices based on FeFET comprehensively, we propose a 2FeFET-2R structure in Sec. \ref{sec:existing_work}(c), which inherits the advantages of FeFET's low power consumption, low delay, and low cost, while also being able to distinguish between cases where the number of mismatched bits is quite close. Compared with existing TCAM structures, it has richer functions and superior features.

%In addition to the previously discussed threshold matching CAMs, many CMOS and NVM based  CAM designs have also been developed to incorporate  approximate matching capabilities. 
%For instance, the A-HAM proposed in \cite{AHAM} evaluates the similarities across all stored entries and identifies the entry with the closest HD to the input query. 
%The PPAC in \cite{PPAC} calculates HD similarity by counting the number of `1's in  XNOR outputs of all CAM cells within a word. 
%The Hamming distance search CAM proposed in \cite{delayCAM} generates a delayed scoring signal when a bit mismatch occurs, and the Hamming distance is proportional to the time delay. 
%The STT-MRAM proposed in \cite{STTMRAM} measures the similarity between the input query and the stored entries in terms of $ML$ current. \cite{bestmatch} introduces a CAM design for minimum Hamming distance search, which utilizes digital circuits for bit comparison. Additionally, a Winner-Take-All (WTA) circuit is integrated at the end of the output voltage. It can be used to output the entry with the highest degree of matching to the search query, as depicted in \autoref{fig:Matchstyle}(c).
%However, these approximate matching CAMs are facing challenges such as high power consumption or constrained parallelism. 
%In this paper, we concentrate on the CAMs with threshold matching functionality.
%In addition, for the ReRAM-based MASC, the tolerance level for approximate Hamming distance search is low. Other structures use a digital search method, which results in low parallelism and so on.

% \vspace{-2ex}
%The model has been changed from bsim6 with no w\l specification and model card to PTM 45nm HP with all the length/width specified, resulting huge leap in latency and EDP/cell accordingly.


%\begin{figure*}%[hbt]
 %   \centering
%\includegraphics[width=1\linewidth]{Figures/top_1.png }
 %   \caption{(a) Equivalent RC circuit with mismatch of 4 bits. (b) Equivalent RC circuit with mismatch of 3 bits. (c) Conduction equivalent resistance of 1FeFET-1R. (d)The voltage difference between (a) and (b) varies with RC. (e) The structure of proposed 2FeFET-2R CAM array of 64 columms. (f)TIQ comparator. (g)Transient waveforms at V$_eval$=0.62V(Thrshold=3 bits)}
    %\vspace{-1.5em}
  %  \label{fig:overall}
%\end{figure*}
%\begin{figure*}%[hbt]
 %   \centering
  %  \includegraphics[width=1\linewidth]{Figures/top_2.png }
   % \caption{ (a) mxn 2FeFET-2R TCAM array. (b)Transient waveforms in different mismatch thresholds.}
    %\vspace{-1.5em}
%\label{fig:overall}
%\end{figure*}


\section{Proposed TAP-CAM Design}
\label{sec:proposed_work}
In this section, we present the TAP-CAM design with bit-by-bit tunable HD threshold match functionality, exploiting the 2FeFET-2R structure and incorporating a threshold-defined evaluation transistor. 
We first discuss the structure and operation principles of the cell, %particularly focusing on the 2FeFET-2R configuration. Subsequently, we
and then elucidate the threshold approximate match implementation at the array level.


\subsection{2FeFET-2R TCAM Cell} 

\begin{figure}
    \centering
    \includegraphics[width=\linewidth]{Figures/2F2R1.png}
    %\vspace{-0.4cm}
    \caption{\textbf{(a)} Structure of the proposed 2FeFET-2R TCAM cell; \textbf{(b)} Transient voltage waveforms of 2FeFET-2R CAM cell storing `1’.}
   % \vspace{-0.4cm}
\label{fig:2FeFET-2R Cell}

\end{figure}

\begin{table}[!t]
    \centering
    \caption{OPERATIONS OF 2FEFET-2R TCAM CELL}
\begin{adjustbox}{center}
\resizebox{1\columnwidth}{!}{
            \begin{tabular}{|c c |c|c|c|c|c|}
            
              \hline \hline
              $\textit{V}_\textit{write}$ = 4V & $\textit{V}_\textit{search}$ = 1V & \textit{BL}/$\overline{\textit{SL}}$ &  $\overline{\textit{BL}}$/$\textit{SL}$ & \textit{ScL} & $\textit{M}_\text{1}$ & $\textit{M}_\text{2}$   \\ 
              \hline
    \multirow{2}*{Write`1'} & Step1     &$\textit{V}_\textit{write}$ & 0 & 0 &`1' & hold\\ & Step2     &$\textit{V}_\textit{write}$ & 0 & $\textit{V}_\textit{write}$ & hold & `0'\\ 
    \hline
    \multirow{2}*{Write`0'} & Step1    & 0 & $\textit{V}_\textit{write}$ & $\textit{V}_\textit{write}$ &`0'  & hold\\
                               & Step2    &0 & $\textit{V}_\textit{write}$ & 0 & hold & `1'\\
                       \hline
              \multicolumn{2}{|c|}{Write \textit{don't care}}  &$\textit{V}_\textit{write}$ & $\textit{V}_\textit{write}$ & 0 & `1' & `1'\\
              \hline
              \hline
            \end{tabular}
             
      }
        \end{adjustbox}
    
    \label{tab:write opetarion}
\end{table}



\autoref{fig:2FeFET-2R Cell}(a) shows the structure of the proposed 2FeFET-2R TCAM Cell. It comprises a pair of parallel  1FeFET-1R structures, with the FeFET drain connected to the matchline (\textit{ML}), and the other end of the structure connected to the sourceline (\textit{ScL}), driven by either $V_{\textit{write}}$ or \textit{GND}.
The FeFET gate connects to the bitline and searchline (\textit{BL}/\textit{SL} and $\overline{\textit{BL}}$/$\overline{\textit{SL}}$). 
By adjusting the write gate input, the FeFET threshold aligns with different storage values. 
%$T_{1}$ precharges $ML$ before searching. 
The 2FeFET-2R structure can store logic `1', `0', and 
\textit{don't care} wildcard state.
%\autoref{fig:2FeFET-2R Cell}(a) shows the structure of the proposed 2FeFET-2R TCAM Cell. The 2FeFET-2R unit circuit comprises a pair of parallel 1FeFET-1R structures, where the drain of the FeFET is connected to the matchline ($ML$), and the other end of the resistor connected to the sourceline ($ScL$) can be driven to $V_{write}$ or $GND$. The gate of the FeFET is connected to the bitline and searchline ($BL$/$\overline{SL}$ and $\overline{BL}$/$SL$). By varying the input to the gate, the threshold of the FeFET can be regulated to correspond to different storage values. $T_{1}$ is connected to $ML$ for precharging $ML$ before the search. The 2FeFET-2R structure can store logic `1' and `0', and $don't\text{ $care$ } $cases. 
\autoref{tab:write opetarion} outlines the write operations of the 2FeFET-2R cell. 
Data bits are written in two steps, storing complementary logic states in each FeFET. 
To write logic `1', $V_{\textit{write}}$ is applied to \textit{BL}/\textit{SL}, while `0' to \textit{ScL} and $\overline{\textit{BL}}$/$\overline{\textit{SL}}$. This sets $V_{\textit{GS}}$ of $\textit{M}_\text{1}$ to 4V, writing logic `1' to $\textit{M}_\text{1}$. In the second step, $V_{\textit{write}}$ is applied to \textit{ScL}, while gate voltage remains the same, writing logic `0' to $\textit{M}_\text{2}$. Thus, the complementary stored  values represents logic `1'.
Similarly, to write logic `0' into the cell, `0' is written to $\textit{M}_\text{1}$ and `1' to $\textit{M}_\text{2}$, respectively. 
To write \textit{don't care} state, logic `1' is written to both $\textit{M}_\text{1}$ and $\textit{M}_\text{2}$. This sets both FeFETs to high-$\textit{V}_\textit{TH}$ state, matching regardless of the search value, aligning with the masking function of `\textit{don't care}' bits.
During writes, \textit{ML} is grounded to eliminate static current. \autoref{fig:fefet}(b) displays $\textit{I}_\textit{D}$-$\textit{V}_\textit{G}$ curves for FeFETs under different write pulses.


%\autoref{tab:write opetarion} summarizes the write operations of the 2FeFET-2R unit circuit. The data bits are written into the two FeFETs in the TCAM cell in two steps, storing opposite logic values in each FeFET. In order to write logic `1', in the first step, $V_{write}$ is applied to $BL$/$\overline{SL}$, while `0' is applied to $ScL$ and $\overline{BL}$/$SL$. Therefore, the gate-source voltage ($V_{GS}$) of $M_1$ is 4V, switching the FE polarization within FeFET and writing logic `1' to $M_1$. In the second step, $V_{write}$ is applied to $ScL$, while the gate voltage of the two FeFETs remains the same as in the first step (i.e. 4V at $BL$/$\overline{SL}$ and 0 at $\overline{BL}$/$SL$). As a result, the $V_{GS}$ of $M_2$ is -4V, and the logic `0' is written to $M_2$. Thus, the 2FeFET-2R unit as a whole represents the storage of logic '1'. 


%Similarly, to write logic `0' into the TCAM cell, we write `0' to $M_1$ and `1' to $M_2$, respectively. To write $don't\text{ $care$ }$into the TCAM cell, it only takes one step to write logic `0' to both $M_1$ and $M_2$. In this way, both FeFETs corresponding to `don't care' are in low-$V_{TH}$ state, resulting in a match regardless of the search value, which aligns with the masking function of `don't care' bits. During writing operations, the matchline $ML$ needs to be driven to ground to eliminate the influence of static current on the $ML$ voltage. \autoref{fig:fefet}(b) displays the $I_{D}$-$V_{G}$ curves corresponding to different write pulses of FeFETs.

During search, \textit{ML} voltage is precharged to high via a precharge transistor, and the search voltages are applied to searchlines ($\textit{SL}$/$\overline{\textit{SL}}$) according to the query data. For logic `1', \textit{SL} set to 1V, and 0 for logic `0', the  \textit{ML} voltage indicates the matching result. 
\autoref{fig:2FeFET-2R Cell}(b) validates the function of the 2FeFET-2R cell. 
\textit{ML} is first precharged by controlling $\textit{T}_\text{1}$'s gate voltage \textit{CLK}, and then left floating upon  search phase. 
When searching `1', \textit{ML} voltage stays high with  \textit{SL} = 1V, indicating a match. Conversely, searching `0' rapidly drops \textit{ML} voltage to 0, indicating a mismatch. %Results in \autoref{fig:2FeFET-2R Cell}(b) align with expectations, validating unit circuit's storage and computing functions.


%During the search operation, the voltage of $ML$ is previously precharged to a high level through a precharge transistor, and different search voltages are applied to the searchlines ($SL$/$\overline{SL}$) according to the input data. Here we set $SL$ to 1V for logic `1' and 0 for logic `0', and observe the voltage change of $ML$ to reflect the matching result. As shown in \autoref{fig:2FeFET-2R Cell}(b), by controlling the gate voltage $CLK$ of $T_{1}$ to precharge $ML$, the precharging is halted after entering the search phase. When searching for logic '1', the voltage on $ML$ remains almost unchanged at a high level when $SL$ voltage is 1V, indicating a match between the search query and the stored entry. Conversely, when searching for logic '0', the voltage on $ML$ rapidly drops to 0, indicating a mismatch between the search query and the stored entry. The operation results shown in \autoref{fig:2FeFET-2R Cell}(b) are consistent with expectations, validating the correctness of the unit circuit's storage and computing functions.


\subsection{2FeFET-2R TCAM Array}

\autoref{fig:1x64} demonstrates the schematic of the proposed 2FeFET-2R TAP-CAM array storing a 64-bit word with corresponding peripheral circuits. 
%This array stores a 64-bit word, sharing $ML$ and $ScL$ across all 64 cells.
PMOS $\textit{T}_\text{1}$ precharges \textit{ML} before the search operation, while an evaluation transistor $\textit{T}_\text{2}$ is connected between \textit{ML} and $\textit{V}_\text{o}$ to enable tunable threshold matching function. 
%$T_2$ evaluates the $ML$ voltage for threshold match function. 
Adjusting the gate voltage of the evaluation transistor controls the discharge rate of \textit{ML}, allowing varying mismatch bits to be sensed by the sense amplifier (SA) as a match case. 
%A sense amplifier (SA) detects the  output, enhancing signal stability, forming output waveform. 


%to mitigate $ML$ parasitic capacitance impact on sensing time.
%As shown in \autoref{fig:1x64}, the 2FeFET-2R unit circuit is expanded into a 1×64 array and accompanied by corresponding peripheral circuits. This array has the capability to store a 64-bit word, with all 64 units sharing the same $ML$ and $ScL$. The $ML$ is connected to two transisitors, where PMOS $T_1$ is used to precharge $ML$ before the matching operation begins, while NMOS $T_2$ is connected to ML as a transistor for voltage evaluation, enabling threshold-controlled approximate match functionality. By adjusting the gate voltage applied to NMOS, the rate of $ML$ voltage decrease can be controlled, thereby achieving control over the permissible number of mismatched unit bits. Additionally, a sense amplifier(SA) is serially connected at the output end to shape and amplify the output results, thereby enhancing the stability of the output signal, ultimately forming the output waveform. During the precharging phase, the PMOS control signal $CLK$ is driven to a low level, while $V_{eval}$ is kept at a high level to ensure the conduction of both $T_1$ and $T_2$. This allows $ML$ to be precharged to $V_{DD}$ to mitigate the impact of parasitic capacitance of $ML$ on the sensing time. 


\label{sec:2FeFET-2R TCAM Array}

\begin{figure}
    \centering    
    \includegraphics[width=\linewidth]{Figures/1x64.png}
    \caption{Structure of a 2FeFET-2R TCAM array with wordlength 64.}
\label{fig:1x64}
%\vspace{-0.4cm}
\end{figure}

During the precharge, \textit{CLK} is set to low, turning  $\textit{T}_\text{1}$ and $\textit{T}_\text{2}$ ON, and precharging \textit{ML} to \textit{VDD}.
During the search phase, setting the \textit{CLK} signal high turns $\textit{T}_\text{1}$ OFF and cutting the charging path. 
Pre-defined bias voltages are applied to the gate of evaluation transistor  $\textit{V}_\textit{eval}$ based on required mismatch thresholds. 
A mismatch between the stored entry and the search query forms a conduction path from $\textit{V}_\text{o}$ to \textit{GND}, discharging $\textit{V}_\text{o}$ and decreasing the voltage.
The rate of voltage decrease depends on the number of mismatched cells and $\textit{T}_\text{2}$'s gate voltage $\textit{V}_\textit{eval}$. 
This rate affect  the output of SA $\textit{SA}_\textit{out}$ which indicates the time for $\textit{SA}_\textit{out}$ to transition from high to low. 
With constant $\textit{V}_\textit{eval}$, more mismatched bits increase the discharge current from $\textit{V}_\text{o}$ to \textit{GND}, accelerating $\textit{SA}_\textit{out}$ voltage drop. 
Similarly, with constant mismatched bits, higher $\textit{V}_\textit{eval}$ boosts the conduction of $\textit{T}_\text{2}$, hastening $\textit{SA}_\textit{out}$ voltage drop. Hence, given the fixed SA sense time, decreasing the $\textit{V}_\textit{eval}$ allows for increasing the mismatch threshold.
%, keeping $ML$ voltage sensing time consistent.

%During the search phase, the $CLK$ signal is set to a high level, leading to the closure of $T_1$ and the interruption of the charging path. Different voltages are applied to $V_{eval}$ based on varying matching thresholds. In a 1x64 array, a mismatch between the stored entry and the search query results in the establishment of a conduction path from $V_o$ to $GND$, causing a decline in the voltage on $V_{o}$. The rate of the voltage decline is contingent upon both the number of mismatched bits and the gate voltage $V_{eval}$ of $T_2$. This decline rate is reflected in the output voltage $SA_{out}$, representing the time required for $SA_{out}$ to transition from a high to a low level.
%When $V_{eval}$ remains constant, an increase in the number of mismatched bits leads to a greater number of conduction paths between $V_o$ and $GND$, thereby accelerating the voltage drop on $SA_{out}$. Similarly, when the number of mismatched bits is constant, a higher $V_{eval}$ enhances the conduction state of $T_2$, resulting in a faster voltage drop on $SA_{out}$. Therefore, as the matching threshold increases, we can decrease the value of $V_{eval}$, so that the sensing time for discerning the voltage status on $ML$ under different matching thresholds remains within the same range.


%\begin{figure}
%    \centering
%    \includegraphics[width=\linewidth]{Figures/R_increase.pdf}
%    \caption{(a)The $I_{ds}$-$V_{gs}$ wave of 1FeFET-1R varies with $R_S$. (b) The Search latency varies with $R_S$.}
%\label{fig:R_increase}
%\end{figure}
%\label{sec:R}


%As previously discussed, the 1FeFET-1R configuration limits conduction current and enhances the robustness by suppressing current variability.
%Higher resistance series current limiter reduces the discharge current, and improves the conduction stability. 
%Additionally, $R_S$ presence results in a more significant $ML$ voltage drop due to greater mismatched bit count. 
Without loss of generality, 
for the TAP-CAM with n bits mismatch threshold (Th-n), i.e., $\leq$n mismatch bits are sensed as a match case, and $\ge$(n+1) bits mismatch indicates a mismatch, the sense margin between the n bits mismatch and (n+1) bits mismatch is determined by the equivalent resistance and associated \textit{ML} capacitance of the array $\textit{C}_\textit{M}$.
%, as illustrated in \autoref{fig:fefet}(c), $R_{ON}$ signifies FeFET's equivalent conduction resistance, while $C_M$ represents the circuit's equivalent capacitance.
%As previously articulated, the 1FeFET-1R configuration can limit the magnitude of the conduction current and enhance 
%the robustness of the circuit. Moreover, as the resistance value of the series current limiter increases, the current gradually decreases, leading to improved stability of the circuit. In addition, we also have observed that the presence of $R_S$ results in a more significant voltage drop of $ML$ due to greater mismatched bit count. Taking n bits mismatch and (n+1) bits mismatch as examples, as illustrated in \autoref{fig:fefet}(c), $R_{ON}$ represents the equivalent conduction resistance of the FeFET, while $C_M$ represents the equivalent capacitance of the circuit. 
The equivalent resistance for the two mismatch cases can be expressed as follows:
%We can express the relationship as follows:
\begin{equation}
\label{eq:R-C circuit1}
      \textit{R}_\textit{n} = \frac{\text{1}}{\textit{n}}\cdot (\textit{R}_\textit{ON}+\textit{R}_\textit{S})
\end{equation}
\begin{equation}
\label{eq:R-C circuit2}
   %   R_{n+1} = \frac{1}{n+1}\cdot (R_{ON}+R_S)
     \textit{R}_{\textit{n}\text{+1}} = \frac{\text{1}}{\textit{n}\text{ + 1}}\cdot (\textit{R}_\textit{ON}+\textit{R}_\textit{S})
\end{equation}
where $\textit{R}_\textit{n}$ represents the approximate equivalent resistance of array with n bits mismatch, and $\textit{R}_\text{n+1}$ represents the approximate equivalent resistance of array with (n+1) bits mismatch. %$C_M$ is the associated ML capacitance.
$\textit{R}_\textit{ON}$ represents the equivalent resistance of an ON FeFET, and $\textit{R}_\textit{S}$ represents the series resistance. 
From charging and discharging formula of RC circuit, we can approximately formulate the \textit{ML} voltage \textit{U}:
\begin{equation}
\label{eq:R-C circuit3}
      \textit{U}=\textit{U}_\text{0}\cdot \textit{e}^{-\frac{\textit{t}}{\textit{RC}_\textit{M}}}
\end{equation}

\begin{equation}
\label{eq:R-C circuit4}
      \frac{\textit{dU}}{\textit{dt}}=\textit{U}_\text{0}\cdot (-\frac{\text{1}}{\textit{RC}_\textit{M}})\textit{e}^{-\frac{\textit{t}}{\textit{RC}_\textit{M}}}
\end{equation}
where $\textit{U}_\text{0}$ represents the initial voltage of \textit{ML}. 
From \autoref{eq:R-C circuit4} we can conclude that the rate of \textit{ML} voltage drop will be faster as the equivalent resistance decreases. 
%Due to the fact that the parallel resistance of n identical resistors is greater than the parallel resistance of n+1 identical resistors,
From \autoref{eq:R-C circuit1} and \autoref{eq:R-C circuit2}, $\textit{R}_\textit{n}$ is larger than $\textit{R}_{\textit{n}\text{+1}}$. 
Therefore,  the voltage of \textit{ML} corresponding to (n+1) bits mismatch drops faster than that of n bits mismatch.
Upon the sensing, the sense margin of Th-n $\Delta U$ can be expressed as follows:
%$U_n$ represents the $ML$ voltage corresponding to  n  bits mismatch, and $U_{n+1}$ represents the $ML$ voltage corresponding to the circuit with (n+1) mismatched bits. As indicated by  \autoref{eq:R-C circuit3}, we can express this relationship as follows:
\begin{equation}
\label{eq:R-C circuit5}
     \Delta \textit{U}=\textit{U}_\textit{n}-\textit{U}_{\textit{n}\text{+1}}=\textit{U}_\text{0}\cdot (\textit{e}^{-\frac{\textit{t}}{\textit{R}_\textit{n}\textit{C}_\textit{M}}}-e^{-\frac{t}{\textit{R}_{\textit{n}\text{+1}}\textit{C}_\textit{M}}}) 
\vspace{0.1cm}
\end{equation}
where $\textit{U}_\textit{n}$ represents the \textit{ML} voltage corresponding to  n bits mismatch, and $\textit{U}_{\textit{n}\text{+1}}$ represents the \textit{ML} voltage corresponding to (n+1) bits mismatch.
%With an increase in $R_S$, $\Delta$$U$ also broadens, resulting in a wider sensing margin.
From \autoref{eq:R-C circuit5}, we observe that $\textit{R}_\textit{S}$ affects the magnitude of $\Delta$$U$ over time t, thus influencing the sense margin. Simultaneously, a larger $\textit{R}_\textit{S}$ value introduces larger search delay. Therefore, selecting an appropriate $\textit{R}_\textit{S}$ value is necessary to ensure that both sense margin and search delay remain within reasonable limits. 
We here select $\textit{R}_\textit{S}$ = 0.3M.% and \textit{VDD} = 0.6V



%As the series resistance $R_S$ increases, $\Delta$$U$ also experiences an augmentation, thereby resulting in a broader sensing margin. However, this enhancement in circuit performance is accompanied by an increase in latency. Therefore, it is necessary to carefully balance both performance and latency when determining the resistance value of $R_S$.Taking all factors into account, we ultimately opt for R=0.3M and $V_{DD}$=0.6V.

Another factor that affects the sense margin and the search time is the bias voltage at evaluation transistor gate. 
To implement the functionality of bit-by-bit tunable threshold approximate matching, we determine appropriate evaluation voltages $\textit{V}_\textit{eval}$ to distinguish different mismatch thresholds, taking the threshold ranging 0-6 bits as an example.
This involves adjusting the gate voltage of the evaluation transistor to differentiate between 0-bit and 1-bit mismatch (Th-0), 1-bit and 2-bit mismatch (Th-1), and so forth. 
%adjacent numbers of mismatched bits can cause the $ML$ voltage to exhibit either a high level (maintained at 1V) or a low level (dropped to 0V) within the same time window.
%Once this time window is established, the array's capability for approximate matching can be verified by controlling the magnitude of the evaluation voltage.%To verify the proper functioning of the approximate match capability, it is essential to determine the appropriate evaluation voltage $V_{eval}$ to distinguish adjacent numbers of mismatched bits within the range of 0-6 bits. This involves adjusting the gate voltage of the evaluation transistor to differentiate between 0-bit mismatch and 1-bit mismatch, 1-bit mismatch and 2-bit mismatch, and so forth. The method of differentiation entails ensuring that within the same time window, the adjacent number of mismatched bits causes the voltage on the ML to exhibit either a high level (maintained at 1V) or a low level (dropped to 0V). Once the time window for distinguishing adjacent numbers of mismatched bits is established, we can verify that the array can achieve approximate match functionality by controlling the threshold voltage. 
Increasing the number of mismatch bits and evaluation transistor gate voltage $\textit{V}_\textit{eval}$ lead to faster $\textit{SA}_\textit{out}$ voltage decrease. 
Hence, with increasing mismatch threshold, we decrease $\textit{V}_\textit{eval}$ to maintain consistent sense time window across different mismatch thresholds.
%Based on this, we conducted experiments, obtaining different $V_{eval}$ values for matching thresholds ranging from 0 to 5 bits. 
The evaluation voltages are therefore experimentally examined and configured as summarized in \autoref{tab:Threshold-V} to ensure that the sense time for distinguishing different mismatch thresholds falls within the same time window.
Different evaluation voltages correspond to different mismatch thresholds. 
%\autoref{tab:Threshold-V} presents the evaluation transistor gate voltage for differentiating adjacent mismatched bit numbers, corresponding to different matching thresholds. 
This evaluation voltage configuration lays the foundation for subsequent performance and latency analysis.

%Different evaluation voltages correspond to different matching thresholds. As the number of mismatched bits increases and the gate voltage of the evaluation transistor $V_{eval}$ increases, both contribute to a faster decrease in the output voltage $SA_{out}$. Consequently, as the matching threshold increases, we gradually decrease the value of $V_{eval}$ to ensure a consistent sensing time window across different match thresholds. Based on this, we have conducted experiments and obtained different $V_{eval}$ corresponding to matching thresholds ranging from 0 to 5 bits. \autoref{tab:Threshold-V} presents the evaluation transistor gate voltage corresponding to distinguishing adjacent numbers of mismatched bits, i.e., the evaluation voltage corresponding to different matching thresholds. These findings will serve as the foundation for our subsequent performance and latency analysis.




\begin{table}[!t]
    \centering
    \caption{$\textit{V}_\textit{eval}$ of different mismatch threshold}
    \begin{adjustbox}{center}
    \resizebox{1\columnwidth}{!}{
\begin{tabular}{|c|c|c|c|c|c|c|}
              \hline
            \makecell{Mismatch\\Threshold(bit)} & 0 & 1 & 2 & 3 & 4 & 5   \\
              \hline
              $\textit{V}_\textit{eval}$(V) & 1 & 0.75 & 0.63 & 0.52 & 0.43 & 0.37  \\ 
              \hline
\end{tabular}

}
\label{tab:Threshold-V}
\end{adjustbox}
\end{table}
\begin{figure}
    \centering
    \includegraphics[width=\linewidth]{Figures/threshold2.png}
    \caption{Transient waveforms of \textit{ML} under different mismatch thresholds. Solid and Dashed lines represent the match and mismatch cases corresponding to a certain mismatch threshold, respectively.}
\label{fig:threshold}
%\vspace{-0.4cm}
\end{figure}

The \textit{ML} transient waveforms corresponding to different mismatch thresholds in \autoref{fig:threshold} validate the bit-by-bit tunable threshold matching function.
%variations and the sensing time window during approximate matching under corresponding evaluation voltages and matching thresholds. 
Solid lines show the \textit{ML} voltage waveforms when the number of mismatched bits equals to the pre-defined mismatch threshold, while dashed lines show the \textit{ML} voltages when the number of mismatched bits  exceeds the pre-defined threshold. 
The sense margin of mismatch thresholds decreases as the threshold increases. 
%Notably, at Th-5, common window exists between 5-bits and 6-bits mismatch, with sensing time window endpoints serving as sensing margin. 
According to \autoref{fig:threshold}, the search latency for distinguishing adjacent mismatch threshold ranging from Th-0 to Th-5  is  1 ns.


%\autoref{fig:threshold} illustrates the transient voltage variations of the $ML$ and the sensing time window during approximate matching under corresponding evaluation voltages and matching thresholds. Solid lines represent the $ML$ voltage variations when the number of mismatched bits equals the matching threshold, while dashed lines represent the ML voltage variations when the number of mismatched bits just exceeds the matching threshold.  As the threshold increases, the margin between the solid and dashed lines diminishes. Notably, at a matching threshold of 5 (Th-5), there exists a common window between 5-bits mismatch and 6-bits mismatch, with the endpoints of the sensing time window serving as sensing margin. According to \autoref{fig:threshold}, the search latency for distinguishing adjacent numbers of mismatched bits within the range of 0 to 6 bits is 1 ns.

%\begin{figure}
%    \centering
%    \includegraphics[width=\linewidth]{Figures/mxn.pdf}
%    \caption{Structure of mxn 2FeFET-2R TCAM.}
%\label{fig:mxn}
%\end{figure}

%For energy and delay analysis, we expanded original 1x64 array of 2FeFET-2R TCAM cells into larger m rows and n columns array, as shown in \autoref{fig:mxn}. Expanded array accommodates m words, each of length n. Each cell in array possesses identical functionalities. Cells in same column share $SLs$, enabling parallel search operations, while cells in same row share $ML$ voltages, controlling $ML$ voltage variations simultaneously. Write/Search buffer inputs stored/search vectors into $SLs$ for matching operations, consistent with \autoref{tab:write opetarion}. During search, all rows compare same input query with stored entries. If mismatch occurs, $ML$ discharges. If $ML$ voltage drops below sense amplifier threshold within specified time window, corresponding SA output transitions to 0, recognized by encoder as mismatch. Conversely, if match occurs, address of stored entry matching search query is output.

%For the purpose of energy and delay analysis, we have expanded the original 1x64 array of 2FeFET-2R TCAM cells into a larger array of m rows and n columns, as illustrated in \autoref{fig:mxn}. This expanded array has the capability to accommodate m words, each with a length of n. Each individual cell within this array possesses identical functionalities and characteristics. All cells within the same column share the $SLs$, allowing for parallel execution of search operations, while cells within the same row share $ML$ voltages, thereby controlling the variations in $ML$ voltage simultaneously. To facilitate the matching operations, we employ Write/Search buffer to input the stored/search vectors into the $SLs$, thereby determining whether each $ML$ discharges or not. The operations during the writing process remain consistent with those outlined in \autoref{tab:write opetarion}. During the search period, all rows compare the same input query with their stored entries. If a mismatch occurs, the $ML$ discharges. If the $ML$ voltage drops below the threshold voltage of the sense amplifier within a specified time window, the corresponding output of the SA transitions to 0, thus being recognized by the encoder as a mismatch. Conversely, if there is a match, the address of the stored entry matching the search query is output.

%Compared to single-row 2FeFET-2R TCAM, expanded m×n array has larger capacity and efficiently executes matching operations during search. By sharing $SLs$ and $ML$ voltage, array achieves highly parallel search operations, improving overall performance and efficiency.


%Compared to a single-row 2FeFET-2R TCAM, the expanded m×n array has a larger capacity and can efficiently execute matching operations during the search period. By sharing $SLs$ and $ML$ voltage, the array can achieve highly parallel search operations, thereby improving overall performance and efficiency.







% \vspace{-2ex}


% \vspace{-1em}
\section{Evaluation}
\label{sec:eval}
In this section, we first evaluate the energy and performance of the proposed TAP-CAM design. We then benchmark the proposed TAP-CAM array in the context of K-nearest neighbor search tasks as tunable approximate matching engine.

\subsection{Evaluation Setup}

\begin{figure}
    \centering
    \includegraphics[width=\linewidth]{Figures/mxn.png}
    \caption{Schematic of m$\times$n TAP-CAM array.}
\label{fig:mxn}
\end{figure}

For the energy and performance evaluations, we conduct our experiments on
%original 1x64 array of 2FeFET-2R TCAM cells into 
a TAP-CAM array with m rows and n columns, as shown in \autoref{fig:mxn}. 
%Expanded array accommodates m words, each of length n.
The cells within the same row share the \textit{ML} and \textit{ScL}, and %the array possesses identical functionalities. 
the cells within the same column share \textit{SLs}, enabling parallel search operations. 
%while cells in the same row share $ML$ voltages, controlling $ML$ voltage variations simultaneously. 
Write/Search buffer drive stored/search vectors into \textit{SLs} for search operations, consistent with \autoref{tab:write opetarion}. 
During the search, all rows compare the same input query with stored entries.
If a mismatch occurs, \textit{ML} discharges. 
If \textit{ML} voltage drops below the sense amplifier threshold within the pre-defined sense time window, the corresponding SA output transitions to 0, recognized by the decoder as mismatch. Conversely, if a match occurs, the address of the stored entry matching the search query is output.

%Compared to a single-row 2FeFET-2R TCAM, the expanded m×n array has larger capacity and efficiently executes matching operations during search. By sharing $SLs$ and $ML$ voltage, the array achieves highly parallel search operations, thereby improving overall performance and efficiency.
The proposed 2FeFET-2R TAP-CAM array is evaluated using SPECTRE. The FeFETs are simulated based on the Preisach FeFET model \cite{transfer-characteristics}.
All MOSFETs are modeled using the 45nm PTM model and the 27°C TT process corner \cite{evaluation2}. The wordlength is set to 64 cells.
%, and the write voltage is ±4V. 

\subsection{Robustness Validation}

\begin{figure}
    \centering
    \includegraphics[width=\linewidth]{Figures/MC.png}
    \caption{100 Monte Carlo simulations considering device-to-device variations: \textbf{(a)} The output waveforms under \textit{VDD} = 0.6V; \textbf{(b)} The output waveforms under \textit{VDD} = 1V.}
\label{fig:MC}
%\vspace{-0.2cm}
\end{figure}

The robustness of the proposed TAP-CAM design under varying operating conditions is examined, specifically with \textit{VDD} = 0.6V and \textit{VDD} = 1V, respectively. 
%To control the experimental variability of the FeFETs,
The FeFETs are assumed to feature the stored low/high $\textit{V}_\textit{TH}$ threshold voltage states with a deviation $\sigma$ = 54mV, and 8$\%$   series resistor variability is considered \cite{area}.
100 Monte Carlo simulations have been conducted  to distinguish between 5-bits and 6-bits mismatches when the mismatch threshold is set to 5 bits (Th-5). 
\autoref{fig:V&N} consistently reveals that the time windows across the 100 runs can be identified.
This observation suggests that the proposed design effectively distinguishes between the adjacent numbers of mismatched bits by employing the evaluation transistor. 
%Furthermore, the design maintains consistent search performance under different experimental conditions. 
Based on these results, it can be inferred that the proposed TAP-CAM design  demonstrates the robustness, as it reliably achieves approximate threshold matching functionality given the variations in operating voltage and device variations.



\subsection{CAM Array Evaluation}
%Here we conduct the evaluations of our 
\begin{figure}
    \centering
    \includegraphics[width=\linewidth]{Figures/energy1.jpg}
    \caption{Energy and latency of the proposed 2FeFET-2R TAP-CAM array
with varying \textbf{(a)} \textit{VDD}; \textbf{(b)} mismatch thresholds; \textbf{(c)} number of rows and \textbf{(d)} number of bits per row.}
\label{fig:V&N}
%\vspace{-0.4cm}
\end{figure}

The search energy consumption of the proposed array mainly originates from precharging the \textit{ML} and SA energy consumption.
Precharging the \textit{ML}, primarily done by $\textit{T}_\text{1}$, depends heavily on \textit{VDD} and the associated \textit{ML} parasitic capacitance.
\autoref{fig:V&N}(a) demonstrates the impact of scaling \textit{VDD} on the search energy consumption and latency. 
As \textit{VDD} scales up, the precharging energy increases, leading to overall higher search energy consumption. At the same time, the amplitude of \textit{ML} dropping from high to low level when mismatch occurs increases, thereby increasing the search delay.  
\autoref{fig:V&N}(b) shows the sense time and sense margin for different mismatch thresholds at \textit{VDD} = 1V. The sense margin is the narrowest at the 5-bit mismatch threshold (Th-5), thus is selected as the sense margin for the SA sense time. 
\autoref{fig:V&N}(c) demonstrates how search energy and latency change with varying row numbers. Increased rows allow parallel search operations, linearly increasing the energy consumption with negligible latency change. 
Finally, \autoref{fig:V&N}(d) examines the wordlength's effect on the search latency and energy consumption per bit. 
Longer wordlengths associate more parasitic capacitance on the \textit{ML}, slowing down the discharge speed and thus increasing the search latency. The increase in capacitance 
leads to a rise in precharge energy per word. But increasing wordlength has minimal impact on the energy consumption of SA, so the search energy per bit decreases. The increasing latency and decreasing energy consumption per bit show trade-offs in the CAM array design optimization.

%The energy consumption in this process predominantly originates from two sources: the precharging of the $ML$ and the energy consumption by the SA. The precharging of the $ML$, primarily accomplished by $T_1$, is heavily contingent upon the magnitude of $V_{DD}$ and the parasitic capacitance. 

%\autoref{fig:V&N}(a) illustrates the impact of increasing $V_{DD}$ on energy consumption and latency. As $V_{DD}$ increases, precharging energy also increases, leading to overall higher energy consumption and latency. 
%\autoref{fig:V&N}(b) presents the sensing time and sensing margin for different matching thresholds at $V_{DD}$=1V. Notably, the sensing margin is narrowest at the 5-bit matching threshold, which determines the final sensing margin. \autoref{fig:V&N}(c) demonstrates how energy and latency vary with different numbers of rows. With increased row numbers, multiple rows can operate simultaneously, linearly increasing power consumption without affecting latency.
%Finally, \autoref{fig:V&N}(d) examines the effect of word length on latency and energy consumption per bit. Longer word lengths result in higher parasitic capacitance on the $ML$, slowing discharge speed and increasing latency. Conversely, longer word lengths lead to lower energy consumption per bit, highlighting the trade-offs in CAM array design optimization.


\section{Backup: compare with previous works}

\paragraph{Comparison with Theorem 1 of \cite{srikant2024rates}.} While the framework of our proof of Theorem \ref{thm:Srikant-generalize} is mainly inspired by the proof of Theorem 1 of \cite{srikant2024rates}, there are some noteworthy differences. Most importantly, we observe that in the equation beginning from the bottom of Page 7 and continuing to the start of Page 8, the right-most side contains a term
\begin{align}\label{eq:Srikant-error}
-\frac{1}{n-k+1} \mathsf{Tr}\left(\bm{\Sigma}_{\infty}^{-\frac{1}{2}}(\bm{\Sigma}_k - \bm{\Sigma}_{\infty})\bm{\Sigma}_{\infty}^{-\frac{1}{2}}\mathbb{E}[\nabla^2 f(\tilde{\bm{Z}}_k)]\right);
\end{align}
the author argued that ``by taking an expectation to remove conditioning, and defining $\bm{A}_k$ to be $\mathbb{E}[\nabla^2 f(\tilde{\bm{Z}}_k)]$'', this term can be transformed to the term
\begin{align}\label{eq:Srikant-wrong}
-\frac{1}{n-k+1} \mathsf{Tr}\left(\bm{A}_k \left(\bm{\Sigma}_{\infty}^{-\frac{1}{2}} \mathbb{E}[\bm{\Sigma}_k]\bm{\Sigma}_{\infty}^{-\frac{1}{2}}-\bm{I}\right)\right)
\end{align}
in the expression of Theorem 1. However, we note that the function $f(\cdot)$, as defined on Page 6 as the solution to the Stein's equation with respect to $\tilde{h}(\cdot)$, is \emph{dependent on} $\mathcal{F}_{k-1}$; in fact, $f$ corresponds to the function $f_k$ in our proof. Consequently, the terms $\bm{A}_k = \mathbb{E}[\nabla^2 f(\tilde{\bm{Z}}_k)]$ (which is actually a conditional expectation with respect to $\mathcal{F}_{k-1}$), and $\bm{\Sigma}_k$ (which corresponds to $\bm{V}_k$ in our proof), are confounded by $\mathcal{F}_{k-1}$ and hence \emph{not independent}. Therefore, taking expectation, with respect to $\mathcal{F}_0$, on \eqref{eq:Srikant-error} should yield
\begin{align}\label{eq:Srikant-right}
-\frac{1}{n-k+1} \mathbb{E}\left\{\mathsf{Tr}\left(\bm{A}_k \left(\bm{\Sigma}_{\infty}^{-\frac{1}{2}} \bm{\Sigma}_k\bm{\Sigma}_{\infty}^{-\frac{1}{2}}-\bm{I}\right)\right)\right\}
\end{align}
Notice that the expectation is taken over the trace as a whole, instead of only $\bm{\Sigma}_k$. However, also due to the confounding bewteen $\bm{A}_k$ and $\bm{\Sigma}_k$, there is no guarantee that the sum of \eqref{eq:Srikant-right} is bounded as shown in the proof of Theorem 2 in \cite{srikant2024rates} on page 10. In other words, the framework of the proof needs a substantial correction to obtain a meaningful Berry-Esseen bound. 

Our solution in the proof of Theorem \ref{thm:Srikant-generalize} is to replace the matrix $\bm{Q}=\sqrt{n-k+1}\bm{\Sigma}_{\infty}$, as defined on Page 6 of \cite{srikant2024rates}, with the matrix $\bm{P}_k$, following the precedent of \cite{JMLR2019CLT}. This essentially eliminates the term \eqref{eq:Srikant-right}, but would require $\bm{P}_k$ to be measurable with respect to $\mathcal{F}_{k-1}$. For this purpose, we impose the assumption that $\bm{P}_1 = n\bm{\Sigma}_n$ almost surely, also following the precedent of \cite{JMLR2019CLT}. The relaxation of this assumption would be addressed in Theorem \ref{thm:Berry-Esseen-mtg}. 

Another important improvement we made in Theroem \ref{thm:Srikant-generalize} is to tighten the upper bound through a closer scrutiny of the smoothness of the solution to the Stein's equation, as is indicated in Proposition \ref{prop:Stein-smooth}. This paves the way for Corollary \ref{cor:Wu}, the proof of which we present in the next subsection. 


\autoref{tab:compare} provides a comprehensive comparison of the proposed 2FeFET-2R TAP-CAM with other CAM designs, in terms of  device type, technology node, device count per cell, cell size, performance and normalized search energy. Cell size estimation is based on a 2$\times$2 layout of the 2FeFET-2R TAP-CAM array.
%\autoref{tab:compare} presents a comprehensive comparison of the 2FeFET-2R CAM with other CAM designs, including the utilized technology, node, device count per cell, cell size, search delay, and search energy per bit per search.  The cell size estimation is based on a 2x2 layout of the 2FeFET-2R CAM array. 
%Our design leverages innovative FeFET technology, enabling approximate matching through different thresholds, achieving significant breakthroughs in circuit functionality and energy consumption. 
Compared to the conventional  CMOS  CAM designs, our proposed  2FeFET-2R TAP-CAM design offers a much smaller cell size. 
The  comparisons highlight the significant advantages of the proposed 2FeFET-2R TAP-CAM design over other CAM designs in terms of energy consumption per bit per search.
The energy efficiency of 2FeFET-2R TAP-CAM is notably superior, being 16.95$\times$, 12.88$\times$, 9.49$\times$, and 6.78$\times$ more efficient compared to 16T TCAM, 10T CAM, 2T-2R TCAM, and 2FeFET TCAM, respectively. 
While some existing designs  achieve approximate search functionality, their energy consumption remains substantially higher than that of 2FeFET-2R structure. 
%Utilizing ferroelectric transistor technology, 2FeFET-2R design represents promising direction for CAM design innovation, providing practical solution for enhancing search efficiency, reducing energy consumption, and optimizing area costs.
Although our design incurs relatively high search delay, considering the search latency and energy  trade-offs and the substantial energy advantages of our proposed design, increased delay is deemed acceptable.


These findings validate the remarkable energy efficiency of 2FeFET-2R TAP-CAM  array, emphasizing its immense potential for data-intensive search applications. This suggests that 2FeFET-2R TAP-CAM architecture is well-positioned to address the evolving needs of modern computing environments, particularly those requiring efficient and high-performance solutions for processing large volumes of data in search-intensive applications.

%Our design leverages innovative FeFET technology, enabling approximate matching through different thresholds, and has achieved significant breakthroughs in both circuit functionality and energy consumption. Compared to CMOS technology, the 2FeFET-2R structure offers a smaller cell size, thus reducing area costs. Although our design incurs a relatively high search delay, considering the trade-off between latency and energy consumption, as well as the substantial energy advantages of the 2FeFET-2R structure, the increased search delay is deemed acceptable. The comparison provided highlights the significant advantages of the 2FeFET-2R CAM structure over several other CAM designs in terms of energy consumption per bit per search. The energy efficiency of the 2FeFET-2R CAM is notably superior, being 16.95$\times$, 12.88$\times$, 8.81$\times$, and 6.78$\times$ more efficient compared to 16T CAM, 10T CAM, 4T-2FeFET CAM, and 2FeFET CAM, respectively. Despite the capability of some alternative designs to achieve approximate search functionality, their energy consumption remains substantially higher than that of the 2FeFET-2R structure. These findings validate the remarkable energy efficiency of the 2FeFET-2R CAM circuit array, emphasizing its immense potential for data-intensive search applications. Utilizing ferroelectric transistor technology, the 2FeFET-2R design represents a promising direction for CAM design innovation. It presents a practical solution for enhancing search efficiency, reducing energy consumption, and optimizing area costs.








\subsection{Case Study: K-Nearest Neighbor Search}
\begin{figure*}
    \centering
    \includegraphics[width=\linewidth]{Figures/csj_benchmark.pdf}
    \caption{\textbf{(a)} KNN clustering accuracy under different \design thresholds, ranging from Th-1 to Th-6 (left to right); \textbf{(b)} Computational speedup and \textbf{(c)} energy efficiency improvement of \design with varying wordlengths compared to a GPU implementation. Datasets from left to right are Iris, Wine and Digits. }
    \label{fig:benchmark}
\end{figure*}
To demonstrate the efficiency of the proposed  design, we benchmark the proposed 2FeFET-2R TAP-CAM array in the context of  K-nearest neighbor (KNN) search framework. 
KNN, a fundamental algorithm in machine learning, embodies a non-parametric supervised model, particularly effective when $\textit{K = }1$, representing the nearest neighbor (NN) classification. 
This algorithm finds widespread use across various fields, including HDC  \cite{liu2022cosime, shou2023see}, reinforcement learning \cite{li2022associative}, and bioinformatics \cite{laguna2020seed}, etc.

At the core of the KNN approach lies the calculation of distances between the query instance, denoted as $x$, and the stored vectors, denoted as $y_i$, within the CAM array.
This process utilizes a distance function, typically denoted as $d(x, y_i)$, which quantifies the dissimilarity or similarity between the data points. 
When $\textit{K = }1$, i.e. NN classification, the class label attributed to the query instance $x$ corresponds to the category of the nearest stored vector $y_i$, identified by the smallest distance metric. This intuitive method allows for straightforward classification based on proximity, making it particularly suitable for scenarios with intricate decision boundaries or complex dataset patterns.
Conversely, when \textit{K} exceeds 1 instead of relying on the nearest neighbor, the algorithm considers the k closest neighbors of the query instance $x$. The class label assigned to $x$ is determined by a majority voting mechanism, where the most frequent class label among the k nearest neighbors prevails. 
This adaptive approach enables KNN to capture more nuanced relationships within the dataset, thereby enhancing its predictive capability and robustness in various applications.


%In the case of NN classification ($K=1$), the class label attributed to the query instance $x$ corresponds to the category of the nearest stored vector $y_i$, identified by the smallest distance metric. This intuitive method allows for straightforward classification based on proximity, making it particularly suitable for scenarios with intricate decision boundaries or complex dataset patterns.

%Expanding on this foundation, the KNN algorithm handles cases where K exceeds 1. In such situations, instead of relying on the nearest neighbor, the algorithm considers the K closest neighbors of the query instance $x$. The class label assigned to $x$ is determined by a majority voting mechanism, where the most frequent class label among the K nearest neighbors prevails. This adaptive approach enables KNN to capture more nuanced relationships within the dataset, thereby enhancing its predictive capability and robustness in various applications.

%In benchmarking our proposed 2FeFET-2R TAP-CAM architecture within the KNN framework, our goal is to showcase its versatility, efficiency, and applicability across various machine learning tasks. Through thorough evaluation and comparison with existing methodologies, we aim to highlight the potential of our design to advance CAM technology and contribute to machine learning research and development.

In benchmarking our proposed 2FeFET-2R TAP-CAM, for a given a function $d(x,y_i)$, which measures the distance between the query $x$ and the i-th stored vector $y_i$ in the CAM array, NN assigns the class label with the smallest distance value to $x$. Similarly, in KNN, given a query $x$, it assigns the most common class label of $x$'s k nearest neighbors to $x$ \cite{jiang2007survey}, as illustrated in \autoref{eq:knn}.
\begin{equation}
\label{eq:knn}
    c(x) = argmax\ \sum^k_{i=1} \delta(c,c(y_i))
\end{equation}
where $c(x)$ represents the class label of the query $x$, while $c(y_i)$ represents that of $y_i$. $y_i$ with $i$ ranges from 1 to k represent the k nearest neighbors. We have $\delta(c,c(y_i))=1$ when the query's label $c$ equals the label of $y_i$, otherwise $\delta(c,c(y_i))=0$.

\section{Dataset}
\label{sec:dataset}

\subsection{Data Collection}

To analyze political discussions on Discord, we followed the methodology in \cite{singh2024Cross-Platform}, collecting messages from politically-oriented public servers in compliance with Discord's platform policies.

Using Discord's Discovery feature, we employed a web scraper to extract server invitation links, names, and descriptions, focusing on public servers accessible without participation. Invitation links were used to access data via the Discord API. To ensure relevance, we filtered servers using keywords related to the 2024 U.S. elections (e.g., Trump, Kamala, MAGA), as outlined in \cite{balasubramanian2024publicdatasettrackingsocial}. This resulted in 302 server links, further narrowed to 81 English-speaking, politics-focused servers based on their names and descriptions.

Public messages were retrieved from these servers using the Discord API, collecting metadata such as \textit{content}, \textit{user ID}, \textit{username}, \textit{timestamp}, \textit{bot flag}, \textit{mentions}, and \textit{interactions}. Through this process, we gathered \textbf{33,373,229 messages} from \textbf{82,109 users} across \textbf{81 servers}, including \textbf{1,912,750 messages} from \textbf{633 bots}. Data collection occurred between November 13th and 15th, covering messages sent from January 1st to November 12th, just after the 2024 U.S. election.

\subsection{Characterizing the Political Spectrum}
\label{sec:timeline}

A key aspect of our research is distinguishing between Republican- and Democratic-aligned Discord servers. To categorize their political alignment, we relied on server names and self-descriptions, which often include rules, community guidelines, and references to key ideologies or figures. Each server's name and description were manually reviewed based on predefined, objective criteria, focusing on explicit political themes or mentions of prominent figures. This process allowed us to classify servers into three categories, ensuring a systematic and unbiased alignment determination.

\begin{itemize}
    \item \textbf{Republican-aligned}: Servers referencing Republican and right-wing and ideologies, movements, or figures (e.g., MAGA, Conservative, Traditional, Trump).  
    \item \textbf{Democratic-aligned}: Servers mentioning Democratic and left-wing ideologies, movements, or figures (e.g., Progressive, Liberal, Socialist, Biden, Kamala).  
    \item \textbf{Unaligned}: Servers with no defined spectrum and ideologies or opened to general political debate from all orientations.
\end{itemize}

To ensure the reliability and consistency of our classification, three independent reviewers assessed the classification following the specified set of criteria. The inter-rater agreement of their classifications was evaluated using Fleiss' Kappa \cite{fleiss1971measuring}, with a resulting Kappa value of \( 0.8191 \), indicating an almost perfect agreement among the reviewers. Disagreements were resolved by adopting the majority classification, as there were no instances where a server received different classifications from all three reviewers. This process guaranteed the consistency and accuracy of the final categorization.

Through this process, we identified \textbf{7 Republican-aligned servers}, \textbf{9 Democratic-aligned servers}, and \textbf{65 unaligned servers}.

Table \ref{tab:statistics} shows the statistics of the collected data. Notably, while Democratic- and Republican-aligned servers had a comparable number of user messages, users in the latter servers were significantly more active, posting more than double the number of messages per user compared to their Democratic counterparts. 
This suggests that, in our sample, Democratic-aligned servers attract more users, but these users were less engaged in text-based discussions. Additionally, around 10\% of the messages across all server categories were posted by bots. 

\subsection{Temporal Data} 

Throughout this paper, we refer to the election candidates using the names adopted by their respective campaigns: \textit{Kamala}, \textit{Biden}, and \textit{Trump}. To examine how the content of text messages evolves based on the political alignment of servers, we divided the 2024 election year into three periods: \textbf{Biden vs Trump} (January 1 to July 21), \textbf{Kamala vs Trump} (July 21 to September 20), and the \textbf{Voting Period} (after September 20). These periods reflect key phases of the election: the early campaign dominated by Biden and Trump, the shift in dynamics with Kamala Harris replacing Joe Biden as the Democratic candidate, and the final voting stage focused on electoral outcomes and their implications. This segmentation enables an analysis of how discourse responds to pivotal electoral moments.

Figure \ref{fig:line-plot} illustrates the distribution of messages over time, highlighting trends in total messages volume and mentions of each candidate. Prior to Biden's withdrawal on July 21, mentions of Biden and Trump were relatively balanced. However, following Kamala's entry into the race, mentions of Trump surged significantly, a trend further amplified by an assassination attempt on him, solidifying his dominance in the discourse. The only instance where Trump’s mentions were exceeded occurred during the first debate, as concerns about Biden’s age and cognitive abilities temporarily shifted the focus. In the final stages of the election, mentions of all three candidates rose, with Trump’s mentions peaking as he emerged as the victor.
To comprehensively evaluate the effectiveness and performance of the proposed TAP-CAM architecture, KNN clustering analysis is conducted under the three most frequently referenced datasets in the UCI Machine Learning Repository, as shown in \autoref{tab:benchmark}. The datasets include Iris, Wine, and Digits, representing a wide range of data types and complexities. In order to achieve a robust evaluation, we have partitioned these datasets into training sets and test sets at an 8:2 ratio to ensure accurate testing and comparison of TAP-CAM model's performance.


\autoref{fig:benchmark}(a) illustrates the effectiveness of the proposed \design architecture across different datasets. 
Among Iris, Wine, and Digits, the \textit{Wine} dataset exhibits the highest susceptibility to hardware device-level variations. This observation emphasizes the importance of robustness in hardware designs, particularly in applications where environmental factors introduce variability. Additionally, we have examined the accuracy performance of KNN search under different TAP-CAM thresholds. 
Interestingly, the results indicate that identifying the nearest neighbor may not always yield the optimal solution. 
For instance, the Iris, Wine, and Digits datasets achieve their respective maximum clustering accuracies at $\textit{K = }2$, $\textit{K = }6$, and $\textit{K = }3$, respectively. 
With the proposed tunable approximate matching scheme, an average 3.06 \% accuracy improvement is observed compared to existing exact-match CAM methods.

Power consumption is obtained via the \textit{Nvidia-smi} toolkit, with the study conducted on \textit{Nvidia 2080ti GPU}, and the \design operations are analyzed via the \textit{Pytorch profiler}. Assuming 256 \design rows, feasible in current manufacturing technology, the KNN clustering benchmark considers different \design wordlengths at the algorithmic level. Idling power is excluded from the results. \autoref{fig:benchmark}(b) illustrates that \design exhibits at least $1.95\times 10^3$ speedup compared to GPU implementation. 
In addition, the energy consumption in \design grows linearly with the number of cells per row, whereas GPU implementations show little increase with dimensionality increment.
Consequently, as dimensionality increases, energy efficiency improvement decreases as demonstrated in \autoref{fig:benchmark}(c). 
For the \textit{Digits} dataset,
\design energy increases with the large number of instances and features,
resulting in an average improvement of $3.15\times$ compared to GPU implementations.

These results illustrate the effectiveness of the proposed TAP-CAM architecture across multiple datasets and scenarios, confirming its feasibility and superiority in practical applications. Through evaluation and comparison with existing methodologies, we highlight the potential of our design to advance CAM technology and contribute to machine learning research and development.
% \vspace{-2ex}
\section{Conclusion}
We present live monitoring and mid-run interventions for multi-agent systems. We demonstrate that monitors based on simple statistical measures can effectively predict future agent failures, and these failures can be prevented by restarting the communication channel. Experiments across multiple environments and models show consistent gains of up to 17.4\%-20\% in system performance, with an addition in inference-time compute.
Our work also introduces \ourenv{}, a new environment for studying multi-agent cooperation.

\section*{Acknowledgements}
This work was supported in part by  NSFC (62104213, 92164203) and SGC Cooperation Project (Grant No. M-0612). 

\bibliographystyle{IEEEtran}
\bibliography{bib}
% that's all folks
\end{document}


