
%\documentclass[sigconf,nonacm]{acmart}
\documentclass[conference]{IEEEtran}
\IEEEoverridecommandlockouts

%\settopmatter{printacmref=false} % Removes citation information below abstract
%\pagestyle{empty} % removes running headers


% Some Computer Society conferences also require the compsoc mode option,
% but others use the standard conference format.
%
% If IEEEtran.cls has not been installed into the LaTeX system files,
% manually specify the path to it like:
% \documentclass[conference]{../sty/IEEEtran}

\usepackage{graphicx}
  % declare the path(s) where your graphic files are
   \graphicspath{{./Figures/}}
\usepackage{hyperref}
\usepackage{threeparttable}
\usepackage{tabularx}
\usepackage{amsmath}    
\usepackage{algorithm}
\usepackage{algorithmic}
\usepackage{multirow}
\usepackage{cite}
\usepackage{svg}
\usepackage{color}
\usepackage{adjustbox}
\usepackage{multirow,tabularx}
\usepackage{makecell}
\DeclareGraphicsExtensions{.pdf,.jpeg,.png,.fig, .emf}
   \usepackage{subfigure}
    \usepackage{epstopdf}
    \usepackage{epsfig}
    \usepackage{subfigure}
    \usepackage{epstopdf}
    \usepackage{epsfig}
\usepackage{xspace}
\newcommand{\design}{TAP-CAM\xspace}    
 \usepackage{booktabs}


% Some very useful LaTeX packages include:
% (uncomment the ones you want to load)


% *** MISC UTILITY PACKAGES ***
%
%\usepackage{ifpdf}
% Heiko Oberdiek's ifpdf.sty is very useful if you need conditional
% compilation based on whether the output is pdf or dvi.
% usage:
% \ifpdf
%   % pdf code
% \else
%   % dvi code
% \fi
% The latest version of ifpdf.sty can be obtained from:
% http://www.ctan.org/pkg/ifpdf
% Also, note that IEEEtran.cls V1.7 and later provides a builtin
% \ifCLASSINFOpdf conditional that works the same way.
% When switching from latex to pdflatex and vice-versa, the compiler may
% have to be run twice to clear warning/error messages.






% *** CITATION PACKAGES ***
%
%\usepackage{cite}
% cite.sty was written by Donald Arseneau
% V1.6 and later of IEEEtran pre-defines the format of the cite.sty package
% \cite{} output to follow that of the IEEE. Loading the cite package will
% result in citation numbers being automatically sorted and properly
% "compressed/ranged". e.g., [1], [9], [2], [7], [5], [6] without using
% cite.sty will become [1], [2], [5]--[7], [9] using cite.sty. cite.sty's
% \cite will automatically add leading space, if needed. Use cite.sty's
% noadjust option (cite.sty V3.8 and later) if you want to turn this off
% such as if a citation ever needs to be enclosed in parenthesis.
% cite.sty is already installed on most LaTeX systems. Be sure and use
% version 5.0 (2009-03-20) and later if using hyperref.sty.
% The latest version can be obtained at:
% http://www.ctan.org/pkg/cite
% The documentation is contained in the cite.sty file itself.






% *** GRAPHICS RELATED PACKAGES ***
%
%\ifCLASSINFOpdf
  % \usepackage[pdftex]{graphicx}
  % declare the path(s) where your graphic files are
  % \graphicspath{{../pdf/}{../jpeg/}}
  % and their extensions so you won't have to specify these with
  % every instance of \includegraphics
  % \DeclareGraphicsExtensions{.pdf,.jpeg,.png}
%\else
  % or other class option (dvipsone, dvipdf, if not using dvips). graphicx
  % will default to the driver specified in the system graphics.cfg if no
  % driver is specified.
  % \usepackage[dvips]{graphicx}
  % declare the path(s) where your graphic files are
  % \graphicspath{{../eps/}}
  % and their extensions so you won't have to specify these with
  % every instance of \includegraphics
  % \DeclareGraphicsExtensions{.eps}
%\fi
% graphicx was written by David Carlisle and Sebastian Rahtz. It is
% required if you want graphics, photos, etc. graphicx.sty is already
% installed on most LaTeX systems. The latest version and documentation
% can be obtained at: 
% http://www.ctan.org/pkg/graphicx
% Another good source of documentation is "Using Imported Graphics in
% LaTeX2e" by Keith Reckdahl which can be found at:
% http://www.ctan.org/pkg/epslatex
%
% latex, and pdflatex in dvi mode, support graphics in encapsulated
% postscript (.eps) format. pdflatex in pdf mode supports graphics
% in .pdf, .jpeg, .png and .mps (metapost) formats. Users should ensure
% that all non-photo figures use a vector format (.eps, .pdf, .mps) and
% not a bitmapped formats (.jpeg, .png). The IEEE frowns on bitmapped formats
% which can result in "jaggedy"/blurry rendering of lines and letters as
% well as large increases in file sizes.
%
% You can find documentation about the pdfTeX application at:
% http://www.tug.org/applications/pdftex





% *** MATH PACKAGES ***
%
%\usepackage{amsmath}
% A popular package from the American Mathematical Society that provides
% many useful and powerful commands for dealing with mathematics.
%
% Note that the amsmath package sets \interdisplaylinepenalty to 10000
% thus preventing page breaks from occurring within multiline equations. Use:
%\interdisplaylinepenalty=2500
% after loading amsmath to restore such page breaks as IEEEtran.cls normally
% does. amsmath.sty is already installed on most LaTeX systems. The latest
% version and documentation can be obtained at:
% http://www.ctan.org/pkg/amsmath





% *** SPECIALIZED LIST PACKAGES ***
%
%\usepackage{algorithmic}
% algorithmic.sty was written by Peter Williams and Rogerio Brito.
% This package provides an algorithmic environment fo describing algorithms.
% You can use the algorithmic environment in-text or within a figure
% environment to provide for a floating algorithm. Do NOT use the algorithm
% floating environment provided by algorithm.sty (by the same authors) or
% algorithm2e.sty (by Christophe Fiorio) as the IEEE does not use dedicated
% algorithm float types and packages that provide these will not provide
% correct IEEE style captions. The latest version and documentation of
% algorithmic.sty can be obtained at:
% http://www.ctan.org/pkg/algorithms
% Also of interest may be the (relatively newer and more customizable)
% algorithmicx.sty package by Szasz Janos:
% http://www.ctan.org/pkg/algorithmicx




% *** ALIGNMENT PACKAGES ***
%
%\usepackage{array}
% Frank Mittelbach's and David Carlisle's array.sty patches and improves
% the standard LaTeX2e array and tabular environments to provide better
% appearance and additional user controls. As the default LaTeX2e table
% generation code is lacking to the point of almost being broken with
% respect to the quality of the end results, all users are strongly
% advised to use an enhanced (at the very least that provided by array.sty)
% set of table tools. array.sty is already installed on most systems. The
% latest version and documentation can be obtained at:
% http://www.ctan.org/pkg/array


% IEEEtran contains the IEEEeqnarray family of commands that can be used to
% generate multiline equations as well as matrices, tables, etc., of high
% quality.




% *** SUBFIGURE PACKAGES ***
%\ifCLASSOPTIONcompsoc
%  \usepackage[caption=false,font=normalsize,labelfont=sf,textfont=sf]{subfig}
%\else
%  \usepackage[caption=false,font=footnotesize]{subfig}
%\fi
% subfig.sty, written by Steven Douglas Cochran, is the modern replacement
% for subfigure.sty, the latter of which is no longer maintained and is
% incompatible with some LaTeX packages including fixltx2e. However,
% subfig.sty requires and automatically loads Axel Sommerfeldt's caption.sty
% which will override IEEEtran.cls' handling of captions and this will result
% in non-IEEE style figure/table captions. To prevent this problem, be sure
% and invoke subfig.sty's "caption=false" package option (available since
% subfig.sty version 1.3, 2005/06/28) as this is will preserve IEEEtran.cls
% handling of captions.
% Note that the Computer Society format requires a larger sans serif font
% than the serif footnote size font used in traditional IEEE formatting
% and thus the need to invoke different subfig.sty package options depending
% on whether compsoc mode has been enabled.
%
% The latest version and documentation of subfig.sty can be obtained at:
% http://www.ctan.org/pkg/subfig




% *** FLOAT PACKAGES ***
%
%\usepackage{fixltx2e}
% fixltx2e, the successor to the earlier fix2col.sty, was written by
% Frank Mittelbach and David Carlisle. This package corrects a few problems
% in the LaTeX2e kernel, the most notable of which is that in current
% LaTeX2e releases, the ordering of single and double column floats is not
% guaranteed to be preserved. Thus, an unpatched LaTeX2e can allow a
% single column figure to be placed prior to an earlier double column
% figure.
% Be aware that LaTeX2e kernels dated 2015 and later have fixltx2e.sty's
% corrections already built into the system in which case a warning will
% be issued if an attempt is made to load fixltx2e.sty as it is no longer
% needed.
% The latest version and documentation can be found at:
% http://www.ctan.org/pkg/fixltx2e


%\usepackage{stfloats}
% stfloats.sty was written by Sigitas Tolusis. This package gives LaTeX2e
% the ability to do double column floats at the bottom of the page as well
% as the top. (e.g., "\begin{figure*}[!b]" is not normally possible in
% LaTeX2e). It also provides a command:
%\fnbelowfloat
% to enable the placement of footnotes below bottom floats (the standard
% LaTeX2e kernel puts them above bottom floats). This is an invasive package
% which rewrites many portions of the LaTeX2e float routines. It may not work
% with other packages that modify the LaTeX2e float routines. The latest
% version and documentation can be obtained at:
% http://www.ctan.org/pkg/stfloats
% Do not use the stfloats baselinefloat ability as the IEEE does not allow
% \baselineskip to stretch. Authors submitting work to the IEEE should note
% that the IEEE rarely uses double column equations and that authors should try
% to avoid such use. Do not be tempted to use the cuted.sty or midfloat.sty
% packages (also by Sigitas Tolusis) as the IEEE does not format its papers in
% such ways.
% Do not attempt to use stfloats with fixltx2e as they are incompatible.
% Instead, use Morten Hogholm'a dblfloatfix which combines the features
% of both fixltx2e and stfloats:
%
% \usepackage{dblfloatfix}
% The latest version can be found at:
% http://www.ctan.org/pkg/dblfloatfix




% *** PDF, URL AND HYPERLINK PACKAGES ***
%
%\usepackage{url}
% url.sty was written by Donald Arseneau. It provides better support for
% handling and breaking URLs. url.sty is already installed on most LaTeX
% systems. The latest version and documentation can be obtained at:
% http://www.ctan.org/pkg/url
% Basically, \url{my_url_here}.




% *** Do not adjust lengths that control margins, column widths, etc. ***
% *** Do not use packages that alter fonts (such as pslatex).         ***
% There should be no need to do such things with IEEEtran.cls V1.6 and later.
% (Unless specifically asked to do so by the journal or conference you plan
% to submit to, of course. )


% correct bad hyphenation here
%\hyphenation{op-tical net-works semi-conduc-tor}
%\newcommand{\cg}{\color{red}}
%\newcommand{\andg}{{\tt{AND }}\xspace}

%\setlength{\itemsep}{-1pt}
%\linespread{0.98}
%\newcommand{\cosi}{COSIME}
%\setlength{\textfloatsep}{1.5pt plus 1.0pt minus 2.0pt}
%\abovedisplayskip=0.8pt
%\abovedisplayshortskip=0.8pt
%\belowdisplayskip=0.8pt
%\belowdisplayshortskip=0.8pt
%\copyrightyear{2022}
%\acmYear{2022}
%\setcopyright{acmcopyright}\acmConference[ICCAD '22]{IEEE/ACM International
%Conference on Computer-Aided Design}{October 30-November 3, 2022}{San Diego, CA, USA}
%\acmBooktitle{IEEE/ACM International Conference on Computer-Aided Design (ICCAD
%'22), October 30-November 3, 2022, San Diego, CA, USA}
%\acmPrice{15.00}
%\acmDOI{10.1145/3508352.3549412}
%\acmISBN{978-1-4503-9217-4/22/10}
\newcommand{\chekai}[1]{\textcolor{magenta}{#1}}

\begin{document}
\bstctlcite{IEEEexample:BSTcontrol}

% paper title
% Titles are generally capitalized except for words such as a, an, and, as,
% at, but, by, for, in, nor, of, on, or, the, to and up, which are usually
% not capitalized unless they are the first or last word of the title.
% Linebreaks \\ can be used within to get better formatting as desired.
% Do not put math or special symbols in the title.
% \title{Energy-Efficient Mismatch Detection in 64-Column Arrays Based on Ferroelectric Ternary Content Addressable Memories for 0-6 Bits}
\title{TAP-CAM: A Tunable Approximate Matching Engine based on Ferroelectric Content Addressable Memory}

\author{ 
\small
Chenyu Ni$^1$, Sijie Chen$^1$, Che-Kai Liu$^2$, Liu Liu$^3$, Mohsen Imani$^4$, Thomas Kämpfe$^5$, Kai Ni$^3$, \\
Michael Niemier$^3$, Xiaobo Sharon Hu$^3$,  Cheng Zhuo$^{1,6,*}$, Xunzhao Yin$^{1,6,*}$\\
$^1$Zhejiang University, Hangzhou, China; 
$^2$Georgia Institute of Technology, GA, USA\\
$^3$University of Notre Dame, IN, USA;
$^4$University of California Irvine, CA, USA\\
$^5$Fraunhofer IPMS, Dresden, Germany\\
$^6$Key Laboratory of Collaborative Sensing and Autonomous Unmanned Systems of Zhejiang Province, China\\
$^*$Corresponding authors, email: \{czhuo, xzyin1\}@zju.edu.cn

    }
% author names and affiliations
% use a multiple column layout for up to three different
% affiliations
% \author{\IEEEauthorblockN{Michael Shell}
% \IEEEauthorblockA{School of Electrical and\\Computer Engineering\\
% Georgia Institute of Technology\\
% Atlanta, Georgia 30332--0250\\
% Email: http://www.michaelshell.org/contact.html}
% \and
% \IEEEauthorblockN{Homer Simpson}
% \IEEEauthorblockA{Twentieth Century Fox\\
% Springfield, USA\\
% Email: homer@thesimpsons.com}
% \and
% \IEEEauthorblockN{James Kirk\\ and Montgomery Scott}
% \IEEEauthorblockA{Starfleet Academy\\
% San Francisco, California 96678--2391\\
% Telephone: (800) 555--1212\\
% Fax: (888) 555--1212}}

% conference papers do not typically use \thanks and this command
% is locked out in conference mode. If really needed, such as for
% the acknowledgment of grants, issue a \IEEEoverridecommandlockouts
% after \documentclass

% for over three affiliations, or if they all won't fit within the width
% of the page, use this alternative format:
% 
%\author{\IEEEauthorblockN{Michael Shell\IEEEauthorrefmark{1},
%Homer Simpson\IEEEauthorrefmark{2},
%James Kirk\IEEEauthorrefmark{3}, 
%Montgomery Scott\IEEEauthorrefmark{3} and
%Eldon Tyrell\IEEEauthorrefmark{4}}
%\IEEEauthorblockA{\IEEEauthorrefmark{1}School of Electrical and Computer Engineering\\
%Georgia Institute of Technology,
%Atlanta, Georgia 30332--0250\\ Email: see http://www.michaelshell.org/contact.html}
%\IEEEauthorblockA{\IEEEauthorrefmark{2}Twentieth Century Fox, Springfield, USA\\
%Email: homer@thesimpsons.com}
%\IEEEauthorblockA{\IEEEauthorrefmark{3}Starfleet Academy, San Francisco, California 96678-2391\\
%Telephone: (800) 555--1212, Fax: (888) 555--1212}
%\IEEEauthorblockA{\IEEEauthorrefmark{4}Tyrell Inc., 123 Replicant Street, Los Angeles, California 90210--4321}}




% use for special paper notices
%\IEEEspecialpapernotice{(Invited Paper)}
% \renewcommand{\bibfont}{\scriptsize}

% \let\oldbibliography\thebibliography
% \renewcommand{\thebibliography}[1]{\oldbibliography{#1}
% \setlength{\itemsep}{-0.5pt}} %Reducing spacing in the bibliography.


% % make the title area
% \maketitle

% As a general rule, do not put math, special symbols or citations
% in the abstract

%\vspace{-2ex}
% \begin{IEEEkeywords}
% Content addressable memory, associative search, nonvolatile memory, energy-aware scheme, FeFET.
% \end{IEEEkeywords}
% % no keywords


\maketitle
%\pagestyle{empty}
%\vspace{-3mm}

% For peer review papers, you can put extra information on the cover
% page as needed:
% \ifCLASSOPTIONpeerreview
% \begin{center} \bfseries EDICS Category: 3-BBND \end{center}
% \fi
%
% For peerreview papers, this IEEEtran command inserts a page break and
% creates the second title. It will be ignored for other modes.
% \IEEEpeerreviewmaketitle







\begin{abstract}

Pattern search is crucial in numerous analytic applications for retrieving data entries akin to the query. Content Addressable Memories (CAMs), an in-memory computing fabric, directly compare input queries with stored entries through embedded comparison logic, facilitating fast parallel pattern search in memory.
While conventional CAM designs offer exact match functionality, they are inadequate for meeting the approximate search needs of emerging data-intensive applications. 
Some recent CAM designs propose approximate matching functions, but they face limitations such as excessively large cell area or the inability to precisely control the degree of approximation. 
In this paper, we propose TAP-CAM, a novel ferroelectric field effect transistor (FeFET) based ternary CAM (TCAM) capable of both exact and tunable approximate matching. 
TAP-CAM employs a compact 2FeFET-2R cell structure as the entry storage unit, %for basic storage and computing functions at the unit circuit level, enabling a dense CAM array to enhance energy efficiency. 
and similarities in Hamming distances between input queries and stored entries are measured using an evaluation transistor associated with the matchline of CAM array. 
%Extensive Monte Carlo simulations assess the impact of FeFET device variation. 
The operation, robustness and performance of the proposed design at array level have been discussed and evaluated, respectively. 
We conduct a case study of K-nearest neighbor (KNN) search to benchmark the proposed TAP-CAM at application level.
%Results demonstrate that TAP-CAM achieves a 6.78× energy improvement compared to 2FeFET CAM implementing approximate match functionality, and 16.95× compared to 16T CMOS CAM implementing exact match functionality and 2FeFET CAM implementing approximate match functionality, along with a 3.06\% accuracy enhancement. 
Results demonstrate that compared to 16T CMOS CAM with exact match functionality, TAP-CAM achieves a 16.95$\times$ energy improvement, along with a 3.06\% accuracy enhancement. Compared to 2FeFET TCAM with approximate match functionality, TAP-CAM achieves a 6.78$\times$ energy improvement.


%Pattern search is a key operation in many analytic applications that retrieve the data entry similar to the query. Content Addressable Memories (CAMs), as a type of in-memory computing fabric, can directly compare input queries with all stored entries in the memory through embedded comparison logic to determine whether the input query matches a stored entry, thereby supporting fast parallel associative search. 
%Therefore, CAMs can provide high-performance and efficient hardware solutions in various applications such as data processing and high-speed searching. 
%However, the conventional CAM designs only provide exact match function which outputs the entry exactly matching with the input, thereby cannot satisfy the growing amount of approximate search requirements for emerging data-intensive applications.
%Recently, some CAM designs have been proposed to support approximate match function, but they are limited by either large  cell area overhead, lacking pattern masking capability, or only designed for specific applications. 
%In this paper, we propose TAP-CAM, a novel FeFET based ternary CAM (TCAM), that supports both exact and tunable threshold matching. TAP-CAM utilizes a 2FeFET-2R structure to achieve a dense CAM array based to enhance energy efficiency. The similarities in terms of Hamming distances between the query and stored entries are calculated by an evaluation transistor. Extensive Monte Carlo simulation is conducted to evaluate the impact of the device-to-device variation of FeFET. 
%We use K-nearest neighbor(kNN) search as a case study to benchmark the application-level improvement of TAP-CAM. 
%Results show that TAP-CAM achieves 16.95$\times$ energy improvement and 3.06\% accuracy improvement compared with CMOS-based CAM implementing the exact match function. 


%Advanced machine learning models such as hyperdimensional computing (HDC) and binary neural networks (BNNs) have been extensively studied for brain-inspired cognitive tasks. During their computations, cosine similarity has shown its great importance w9241990140832151214499here intensive inferences are performed based on the angles between the binary query vector and binary stored feature vectors.


%Cosine similarity measures the similarity between two vectors in an inner product space. It is widely used in a number of machine learning models such as hyperdimensional computing (HDC) and deep neural networks. More specifically, during the inference phase of these machine learning applications, a large number of cosine similarity-based searches (CSSs) are often needed.
%Specifically, for the prevalent binary neural networks (BNN) and hyperdimensional computing (HDC) models, cosine similarity plays a critical role in 
%\textcolor{red}{move the first paragraph to Intro}
%Cosine similarity measures the similarity between two vectors in an inner product space, and  has been widely used in a number of machine learning models, 
% Content addressable memory (CAM) has  been widely utilized as associative memory for data-intensive workloads, thanks to its  parallel in-memory pattern-matching capability. However, traditional CAM designs struggle to maintain their energy efficiency and performance advantages due to the limited number of exact matches between the stored entries and query patterns, especially considering the ever-growing amount of data. 
% %As such, efficient CAM designs and optimizations for approximate matching capability are highly desired.
% To address this challenge,
% we propose TAP-CAM, a novel tunable approximate matching engine based on a ferroelectric field effect transistor (FeFET) CAM.
% The FeFET CAM cell employs a low-power 1FeFET1R structure that integrates a series resistor current limiter into the intrinsic FeFET structure. 
% An evaluation transistor connected to the NOR-type matchline (ML) of the  CAM array  controls the ML discharge rate,  enabling TAP-CAM to perform approximate search operations. 
% By computing the Hamming distances (HDs) between the stored pattern entries and the input query,  TAP-CAM generates a match output for entries with an HD below a tunable matching threshold.
% We thoroghly analyze and validate the scalability and robustness of the proposed TAP-CAM array.
% Moreover, we evaluate the performance of TAP-CAM, which demonstrates a remarkable $16.95\times$ improvement in terms of energy efficiency compared to its CMOS based counterpart. 
% When TAP-CAM is deployed in the K-nearest neighbor (KNN) model for classification tasks, benchmarking results reveal significant improvements in  energy efficiency and delay (AA and BB, respectively). Furthermore, the application accuracy indicates an average improvement of 3.06 \%  compared to the existing exact match CAM, highlighting the value and effectiveness of our proposed \design design.
%for energy-efficient in-memory approximate matching applications, playing a significant role in artificial intelligence hardware acceleration and epidemic virus gene identification. Unlike existing CMOS-based structures, we designed a 2FeFET-2R RAM based on the novel FeFET device. The intrinsic polarization memory function of the FeFET greatly simplifies the circuit structure, reducing area and energy consumption overhead. In this paper, the 2FeFET-2R RAM can distinguishes adjacent mismatched bits by regulating the gate voltage of the evaluation transistor and comparing the voltage on the match line, achieving differentiation of 0-6 mismatch situations in a 64-bit long array. Through extensive Monte Carlo simulations, we analyzed the robustness of the circuit under different processes. Meanwhile, we evaluate the performance of the ferroelectric TCAM approach in a single search operation. Compared to the traditional CMOS-based structure, the energy efficiency has improved by 20 times, and the circuit area has been reduced by 6 times.
% \begin{IEEEkeywords}
% In-memory Computing, Cosine Similarity, FeFET, Associative Memory (AM)
% \end{IEEEkeywords}
\end{abstract}

%\vspace{-3ex}
%New addition: the latency is reduced to <3ns and the EDP/cell is reduced to 0.286EDP(fJ)/bit(under 64X64), harvesting 333X and 90.5X (both conservative) improvement for latency and EDP/cell respectively(comp. Nature comm.).


%\vspace{-1em}

\section{Introduction}
\label{sec:introduction}

In the era of advancing artificial intelligence, the computational demands on AI models are rapidly increasing. 
Training data volumes across various domains like computer vision (CV) \cite{CV}, natural language processing (NLP) \cite{neural}, and speech recognition \cite{speechrecognition} have surged, posing significant challenges to computing hardware and architectures, both at the edge and in data centers. The traditional von Neumann architecture, with its constant data movement between memory and processing units, exacerbates energy consumption and latency issues, intensifying the ``Memory Wall" problem.
To tackle this challenge, emerging computing paradigms, notably In-Memory Computing (IMC), have gained attention. 
IMC directly employs parallel data operations within the memory, enhancing core performance and efficiency while alleviating the ``Memory Wall" problem \cite{IMC2,  IMC3, yin2024deep, yin2024ferroelectric, yang2024energy, li2024febim}. 

Content Addressable Memory (CAM) emerges as a hardware solution of IMC, enabling parallel and efficient searching and similarity  measurement  within the memory. 
CAMs compare input data with all stored data simultaneously, and output the stored entry that matches with input or has the highest similarity to the input.
%avoiding power consumption and latency associated with data transfers. Therefore, CAMs 
Therefore, CAMs are viewed as a potential solution for accelerating 
%the processor-memory bottleneck. 
%Traditionally used in network routers and associative processors, CAMs are now gaining interest in 
various data-centric workloads like bioinformatics \cite{zhong2023asmcap,laguna2020seed}, machine learning \cite{2FeFETa, xu2024ferex, hu2021memory}, and neural language processing \cite{neural}.
%As a specialized solution to the Memory Wall problem, CAMs utilize the entire memory array to accelerate parallel search operations, showing great potential in today’s computing networks. 
Specifically, CAMs significantly speed up Hyperdimensional Computing (HDC), making this brain-inspired computing paradigm  efficient for tasks like image classification and speech recognition \cite{HDC1, HDC2, HDC3}. 
This effectiveness arises from CAMs' ability to transform sequential pattern matching into highly parallelizable computational tasks and simplify the complex distance measurements into Hamming distance \cite{kim2020geniehd}. 
The rapid search and matching capability of CAMs make them essential components in applications requiring efficient data access and retrieval.


Conventional  CMOS based CAM design consists of 10-16 transistors per cell, which results in large area overhead and high energy consumption \cite{16TCMOS}. 
%These problems severely limits their applicability \cite{16TCMOS} in various applications. 
To tackle the area and energy challenges,
%faced by CMOS CAM designs, 
researchers have proposed utilizing emerging non-volatile memory (NVM) devices to construct more compact and efficient CAM designs, as these  CAMs merge the storage and logic within the NVM devices, thus offering significant area and energy saving. 
%aiming to improve the  performance, area and energy efficiency. 
CAMs based on 2-terminal NVMs like resistive RAM (RRAM) \cite{li2021sapiens,Chang3t1r}, magnetic tunneling junction (MTJ) \cite{Matsunaga4t2mtj, zhuo2022design}, phase change memory (PCM) \cite{jing2t2r}, and 3-terminal ferroelectric field effect transistor (FeFET) \cite{2FeFET,4T2FeFET,1FeFET1R-transfer, yin2022ferroelectric, yin2023ultracompact, xu2023challenges, Huang2024, yin2020fecam, li2020scalable} have been explored. 
Among these devices, FeFETs stand out in constructing the compact and efficient CAM designs due to their unique hysteresis I-V characteristics, high current ON/OFF ratio, high off-state resistance, low write energy, and compatibility with CMOS technology \cite{Liu2022eva-cam}.  
While non-volatile storage can achieve high area efficiency and  mitigate the high energy consumption caused by CMOS technology, these CAMs still encounter limitations for data-intensive applications  due to their exact search functionality.
In the era of big data, as the amount of data for processing bursts and the chances of exact matching drop down,  these CAMs with limited array size fail to maintain the hardware utilization efficiency while consuming extra area and energy overheads.
%the application of approximate matching is increasingly common, offering potential hardware utilization efficiency improvements while maintaining an acceptable level of accuracy.
Many applications require approximate pattern search functions where entries with a similarity within a certain threshold distance to the search query are desired. 
%As a result, they fail to significantly enhance energy efficiency.
%\textbf{Secondly}, conventional CAMs only support exact match functions, returning the entry that precisely matches the search query. However, many applications require approximate pattern search functions where entries with a similarity within a certain threshold distance to the search query are desired. In the era of big data, the application of approximate matching is increasingly common, offering potential hardware utilization efficiency improvements while maintaining an acceptable level of accuracy.
%However, conventional CAM designs cannot satisfy the requirements of emerging applications primarily in two aspects. \textbf{Firstly}, conventional CAMs only support the exact match function, meaning only the stored entries that precisely match the search query can be returned. However, many applications require approximate pattern search functions where the stored entries do not necessarily match the search query exactly. 
%There are mainly two types of approximate match functions for CAM. Threshold match 
%Instead, entries with a similarity within a certain threshold distance to the search query
%As long as the similarity between the search query and stored entries is within a certain threshold, the entries 
%are desired to be returned. In the era centered around big data, the application of approximate matching is becoming increasingly widespread. This match pattern offers the potential to improve hardware utilization efficiency while maintaining an acceptable level of accuracy. 
%\textbf{Secondly}, the conventional CAMs are implemented by CMOS technology which requires 10-16 transistors per circuit cell, leading to large cell area and high energy consumption. The substantial overhead of CMOS CAM limits its applicability \cite{16TCMOS}. 
To address the challenge of limited CAM utilization efficiency, 
various CAM designs implementing approximate pattern search have been proposed. 
These approximate CAMs improve the utilization and overall energy efficiency by compensating the search accuracy within an acceptable range. 
For instance, HD-CAM \cite{conventionalCAM} introduced a 10T CMOS-based approximate CAM with a matchline (\textit{ML}) charge redistribution technique, but it suffers from a large cell area and lacks the support for wildcard (\textit{don’t care}) bits.
Moreover, the  design is unable to precisely control the degree of approximation, bit-by-bit.
MHCAM \cite{liu2023reconfigurable} presented an approximate CAM design based on FeFET with programmable thresholds, but it's tailored to applications requiring multi-state Hamming distance. 
\cite{MASC} implemented threshold matching by leveraging voltage scaling and controlling the precharge period, but its high energy consumption and inability to precisely control the threshold limit its applications.
\cite{2FeFETa} introduced approximate matching capabilities using 2FeFET TCAM. It computes the Hamming distance between search and stored vectors in a highly parallelized manner by monitoring \textit{ML} discharge rate. Despite achieving notable energy efficiency and density in TCAM, it lacks fine-grained control over approximate search precision.
%Both approximate CAM designs are unable to precisely control the level of approximation.
%To address the issue of implementing approximate pattern search, several CAM designs for approximate match functions have been proposed. HD-CAM~\cite{conventionalCAM} proposed a 10T CMOS-based approximate CAM with matchline (ML) charge redistribution technique. However, HD-CAM has a large cell area and does not support mask (\textit{don't care}) bits. MHCAM \cite{liu2023reconfigurable} presented an approximate CAM design providing programmable thresholds but it is specific to the applications that require multi-state Hamming distance.
% additional transistors


%To tackle the challenges of unit area and energy consumption faced by CMOS CAM designs, researchers have proposed utilizing emerging non-volatile memories (NVMs) devices to construct more compact and efficient CAM designs. This approach offers a novel means of enhancing system performance and energy efficiency. For example, CAMs based on 2-terminal NVMs encompass resistive RAM (RRAM)~\cite{li2021sapiens,Chang3t1r}, spin transfer torque magnetic RAM (STT-MRAM)~\cite{Matsunaga4t2mtj}, phase change memory (PCM)~\cite{jing2t2r}, and 3-terminal ferroelectric RAM (FeRAM)~\cite{2FeFET,4T2FeFET,1FeFET1R-transfer}. The 2-terminal NVMs typically require current-driven write schemes and large access transistors, resulting in high write energy consumption~\cite{Liu2022eva-cam}. CAMs based on 3-terminal FeFET devices have emerged as promising candidates for implementing NV-CAMs owing to their high current ON/OFF ratio, low write energy, and compatibility with CMOS technology. 

% address both approximate match function and NV-CAM
%In this work, we propose TAP-CAM, an approximate match engine featuring a tunable threshold match function. TAP-CAM utilizes a novel 2FeFET-2R  TCAM to achieve high density with supreme energy efficiency. The tunable threshold is set by the bias voltage of the evaluation transistor. We validate the basic storage and computing functionality of the TAP-CAM unit circuit and array circuit, and conduct extensive Monte Carlo simulations to explore the impact of device-to-device variation of FeFET. 
%We use the K-nearest neighbor search (KNN) as a representative application to investigate the application-level benefits of TAP-CAM. Simulation results demonstrate that TAP-CAM achieves 16.95$\times$ energy improvement and 3.06\% accuracy improvement compared with CMOS-based CAM with the exact match function. 

%These CAMs designed for approximate search are unable to precisely control the level of approximation. 
To address aforementioned challenges of existing approximate CAMs, in this work, we propose TAP-CAM, a general approximate matching engine featuring a bit-by-bit tunable threshold match function. 
We consider FeFET as a representative NVM device, and propose to utilize a novel 2FeFET-2R ternary CAM (TCAM) cell structure to store ternary value. 
An evaluation transistor is employed between the parallel connected TCAM cells and the CAM array sense amplifier to control the \textit{ML} discharge rate, and the tunable threshold of the approximate matching functionality is set by the bias voltage of the evaluation transistor. 
We validate the bit-wise XNOR logic and the tunable  threshold matching functionality of  TAP-CAM design at cell and array levels,  respectively,  and conduct extensive Monte Carlo simulations to examine the robustness against device-to-device variations.
We use the K-nearest neighbor search (KNN) as a representative application to investigate the benefits of TAP-CAM at application level.
Evaluation results demonstrate that TAP-CAM achieves a 16.95$\times$ energy improvement and 3.06\% accuracy improvement compared to 16T CMOS CAM with exact match function. Compared to 2FeFET TCAM with
approximate match functionality, TAP-CAM achieves a 6.78$\times$
energy improvement.



The rest of paper is organized as follows: Sec.~\ref{sec:background} reviews the FeFET device characteristics and existing CAM designs. Sec.~\ref{sec:proposed_work} introduces the proposed TAP-CAM. Sec.~\ref{sec:eval} presents the evaluation results and the KNN case study. Finally, Sec.~\ref{sec:conclusion} summarizes the paper.

%The rest of the paper is organized as follows. Sec.~\ref{sec:background} reviews the FeFET device characteristics and existing CAM designs. Sec.~\ref{sec:proposed_work} introduces the circuit design of TAP-CAM. Sec.~\ref{sec:eval} presents the evaluation results and the kNN case study. Finally, Sec.~\ref{sec:conclusion} summarizes the paper. 

% The  conventional Von Neumann architecture has been the predominant computing paradigm for decades, facilitating the blossom  of information society.
% However, this architecture faces the memory wall bottleneck when handling the vast amounts of data processed in next-generation artificial intelligence (AI) algorithms and models \cite{}.  Massive data transfer between the memory and the processor for computation has wasted a significant amount of energy and time, hindering the architecture from realizing fast and energy efficient  computing tasks. 
% To address these challenges,  the compute-in-memory (CiM) paradigm has emerged as a  promising alternative for data-intensive workloads. Unlike the traditional von Neumann architecture, CiM performs computations within the memory, eliminating the costly penalty associated with data movements. This approach has demonstrated efficacy across various domains, enabling efficient data processing with improved overall performance.

% As a special form of CiM, content addressable memory (CAM) performs parallel search functionality across the memory given an input query, and quickly identifies the stored entry that is identical to the input query. 
% Such efficient search capability enables the CAM to implement as an associative memory (AM) \cite{}, which meets the demands of data-intensive workloads and offers extraordinary performance for a number of emerging applications, such as machine learning \cite{},  cognitive learning \cite{}, genome analysis, etc.
% These applications generate a huge amount of redundant data, and storing the most frequent redundant data in CAMs can effectively reduce  redundant computations, therefore improving overall energy efficiency and performance.
% However, compared with the ever-growing amount of data, CAM based AMs can only accommodate a limited number of patterns that are exact matching with the redundant data. 
% Relaxing the matching degree of the CAM arrays from the exact matching function to the approximate matching, where a few mismatches between the stored entries and the input query still indicate a match output, would be much beneficial for aforementioned data-intensive applications.

% Typical CAM designs based on CMOS technology use digital error correction techniques to enable approximate matching capability 
% from low density and high energy consumption, various compact and energy efficient CAM designs based on emerging non-volatile  memories (NVMs) including resistive RAM (RRAM), spin-transfer torque RAM (STT-RAM), and ferroelectric FET RAM (FeFET-RAM), etc., have been proposed \cite{}. 
% %At the device level,  have been proposed for improving power, delay, area efficiency, and so on \cite{RRAM,ReRAM,STTMRAM,FeFET-RAM}.
% However, most of these CAM designs 
% In this work, ferroelectric field-effect transistor (FeFET) is exploited as a representative NV device due to its high energy efficiency, high density and low cost At the circuit level, CAM designs based on FeFET including 2FeFET,  1FeFET-1R, HD-CAM and so on. These designs focus on improving the power and delay efficiency of the nearest neighbor (NN) Hamming distance search. However, at the application level, finding the NN is typically not the best prediction for the query's label \cite{}. While prior arts mainly focus on the optimization of the exact-match CAM \cite{}, a CAM design that realizes tunable approximate matching (TAP) is highly desirable, enabling higher algorithmic accuracy in an application such as k-nearest neighbor (KNN). 

%\chekai{In this work, we propose TAP-CAM, a high-resolution FeFET-based TAP CAM. The cell-level functionality is first validated, and the TAP-CAM array is then evaluated to investigate the energy and delay trends. Results indicate that \design is able to achieve higher resolution compared and AAx energy improvements over the existing CMOS-based TAP design. In addition, Monte Carlo simulations are performed to justify the robustness of the proposed \design, showcasing the reliability of the design. Finally, an application study is performed to show the usefulness of \design. Results indicate BBx energy and CCx latency improvement over a GPU implementation, and on average 3.06\% algorithmic accuracy improvement over the existing exact-match CAM.}
% Content-Addressable Memory (CAM) is an effective solution for improving hardware performance. Unlike the conventional Von Neumann architecture, where data is transferred from memory to the processor for computation, the CAM's characteristic of storing and computing in one unit can shorten the time for data transfer to some extent. Its excellent features of low latency, high energy efficiency, and small area make it show tremendous potential for applications in fields such as deep learning, natural language processing, and computer vision.

% The design of Content-Addressable Memory (CAM) is based on new non-volatile devices, and scientists have conducted extensive research in the field of non-volatile memory. RRAM, STT-RAM, and FeFET-RAM have been proposed as powerful candidates for improving MOS chip storage, as seen in \cite{RRAM,ReRAM,STTMRAM,FeFET-RAM}. The CAM discussed in this article is designed based on ferroelectric transistors, whose unique polarized memory function significantly reduces the complexity of circuit structures and solves the high energy consumption and high cost issues of conventional CMOS processes.

% However, with the continuous development of artificial intelligence, the IoT industry's requirements for the processing speed and energy efficiency of memory are increasing. As a key part of data processing, search has become an important way to improve work efficiency. As the basic hardware core of Content-in-Memory (CiM), Ternary Content-Addressable Memory (TCAM) supports parallel search of given input vectors on the memory array and can give search results based on the stored content, providing a new mode to solve the processor-memory bottleneck problem in conventional digital machines. At the same time, CAM has also been widely used in dense big data. When processing large-scale data, approximate search is often used instead of exact search, allowing for a certain Hamming distance, which allows for a certain number of bit mismatches, thereby reducing waste of storage and computing resources. Although there exist Hamming distance-tolerant Content-Addressable Memories based on standard Complementary Metal-Oxide-Semiconductor (CMOS) technology, the energy efficiency and density of CAM still need to be improved due to the problems in area, cost, and leakage of the CMOS process.
% \vspace{-2ex}
\subsection{Rapidly-Exploring Random Tree (RRT)}
RRT is a widely used sampling-based algorithm for efficiently exploring nonconvex, high-dimensional spaces in robotics and motion planning. It incrementally builds a tree, \(T\), rooted at an initial state \(q_\text{init}\), to explore the configuration space \(\mathcal{C}\). RRT leverages the property of voronoi bias \cite{kuffner2000efficient}, which naturally guides the exploration towards large unexplored regions in \(\mathcal{C}\), making it particularly effective for complex spaces.

The algorithm follows three main steps: \texttt{Sample}, \texttt{NearestNeighbor}, and \texttt{Extend}, iteratively growing the tree \(T\). In the \texttt{Sample} step, a random state \(q_\text{rand}\) is drawn from the configuration space \(\mathcal{C}\). Sampling is usually uniform but can be biased toward specific regions, such as \(q_\text{goal}\), to improve efficiency. voronoi bias ensures the tree naturally expands into less-explored regions, promoting uniform coverage of the space. Next, the \texttt{NearestNeighbor} step identifies the closest node \(q_\text{near}\) in the tree \(T\) to \(q_\text{rand}\) using a distance metric $d(q, q') = ||q - q'||$, often the Euclidean distance. Finally, in the \texttt{Extend} step, the algorithm attempts to connect \(q_\text{near}\) to \(q_\text{rand}\) with a straight-line path. If the path lies entirely within the collision-free configuration space \(\mathcal{C}_\text{free}\), \(q_\text{rand}\) is added to \(T\) as a new node. Otherwise, the step is discarded, and the process repeats.

The RRT iterates these steps until a termination condition is met, such as reaching the goal state \(q_\text{goal}\), exceeding a maximum number of iterations, or constructing a sufficiently dense tree. The complete algorithm is summarized in Algorithm~\ref{alg:rrt}.

\begin{algorithm}[H]
\caption{RRT Algorithm}
\label{alg:rrt}
\begin{algorithmic}[1]
\Require Initial state $q_\text{init}$, goal state $q_\text{goal}$, maximum iterations $N_\text{max}$
\State Initialize tree $\mathcal{T} \gets \{q_\text{init}\}$
\For{$i \gets 1$ to $N_\text{max}$}
    \State $q_\text{rand} \gets$ \texttt{Sample()} \Comment{Sample a random state}
    \State $q_\text{near} \gets \texttt{NearestNeighbor}(\mathcal{T}, q_\text{rand})$ \Comment{Find nearest node in $\mathcal{T}$}
    \State $q_\text{new} \gets \texttt{Extend}(q_\text{near}, q_\text{rand})$ \Comment{Generate new state}
    \If{$q_\text{new} \in \mathcal{C}_\text{free}$}
        \State Add $q_\text{new}$ to $\mathcal{T}$ with an edge to $q_\text{near}$
    \EndIf
    \If{$q_\text{new} = q_\text{goal}$}
        \State \Return Path from $q_\text{init}$ to $q_\text{goal}$ in $\mathcal{T}$
    \EndIf
\EndFor
\State \Return Failure
\end{algorithmic}
\end{algorithm}

% \subsection{Reinforcement Learning}

% \subsection{Imitation Learning}
% \vspace{-2ex}
%The model has been changed from bsim6 with no w\l specification and model card to PTM 45nm HP with all the length/width specified, resulting huge leap in latency and EDP/cell accordingly.


%\begin{figure*}%[hbt]
 %   \centering
%\includegraphics[width=1\linewidth]{Figures/top_1.png }
 %   \caption{(a) Equivalent RC circuit with mismatch of 4 bits. (b) Equivalent RC circuit with mismatch of 3 bits. (c) Conduction equivalent resistance of 1FeFET-1R. (d)The voltage difference between (a) and (b) varies with RC. (e) The structure of proposed 2FeFET-2R CAM array of 64 columms. (f)TIQ comparator. (g)Transient waveforms at V$_eval$=0.62V(Thrshold=3 bits)}
    %\vspace{-1.5em}
  %  \label{fig:overall}
%\end{figure*}
%\begin{figure*}%[hbt]
 %   \centering
  %  \includegraphics[width=1\linewidth]{Figures/top_2.png }
   % \caption{ (a) mxn 2FeFET-2R TCAM array. (b)Transient waveforms in different mismatch thresholds.}
    %\vspace{-1.5em}
%\label{fig:overall}
%\end{figure*}


\section{Proposed TAP-CAM Design}
\label{sec:proposed_work}
In this section, we present the TAP-CAM design with bit-by-bit tunable HD threshold match functionality, exploiting the 2FeFET-2R structure and incorporating a threshold-defined evaluation transistor. 
We first discuss the structure and operation principles of the cell, %particularly focusing on the 2FeFET-2R configuration. Subsequently, we
and then elucidate the threshold approximate match implementation at the array level.


\subsection{2FeFET-2R TCAM Cell} 

\begin{figure}
    \centering
    \includegraphics[width=\linewidth]{Figures/2F2R1.png}
    %\vspace{-0.4cm}
    \caption{\textbf{(a)} Structure of the proposed 2FeFET-2R TCAM cell; \textbf{(b)} Transient voltage waveforms of 2FeFET-2R CAM cell storing `1’.}
   % \vspace{-0.4cm}
\label{fig:2FeFET-2R Cell}

\end{figure}

\begin{table}[!t]
    \centering
    \caption{OPERATIONS OF 2FEFET-2R TCAM CELL}
\begin{adjustbox}{center}
\resizebox{1\columnwidth}{!}{
            \begin{tabular}{|c c |c|c|c|c|c|}
            
              \hline \hline
              $\textit{V}_\textit{write}$ = 4V & $\textit{V}_\textit{search}$ = 1V & \textit{BL}/$\overline{\textit{SL}}$ &  $\overline{\textit{BL}}$/$\textit{SL}$ & \textit{ScL} & $\textit{M}_\text{1}$ & $\textit{M}_\text{2}$   \\ 
              \hline
    \multirow{2}*{Write`1'} & Step1     &$\textit{V}_\textit{write}$ & 0 & 0 &`1' & hold\\ & Step2     &$\textit{V}_\textit{write}$ & 0 & $\textit{V}_\textit{write}$ & hold & `0'\\ 
    \hline
    \multirow{2}*{Write`0'} & Step1    & 0 & $\textit{V}_\textit{write}$ & $\textit{V}_\textit{write}$ &`0'  & hold\\
                               & Step2    &0 & $\textit{V}_\textit{write}$ & 0 & hold & `1'\\
                       \hline
              \multicolumn{2}{|c|}{Write \textit{don't care}}  &$\textit{V}_\textit{write}$ & $\textit{V}_\textit{write}$ & 0 & `1' & `1'\\
              \hline
              \hline
            \end{tabular}
             
      }
        \end{adjustbox}
    
    \label{tab:write opetarion}
\end{table}



\autoref{fig:2FeFET-2R Cell}(a) shows the structure of the proposed 2FeFET-2R TCAM Cell. It comprises a pair of parallel  1FeFET-1R structures, with the FeFET drain connected to the matchline (\textit{ML}), and the other end of the structure connected to the sourceline (\textit{ScL}), driven by either $V_{\textit{write}}$ or \textit{GND}.
The FeFET gate connects to the bitline and searchline (\textit{BL}/\textit{SL} and $\overline{\textit{BL}}$/$\overline{\textit{SL}}$). 
By adjusting the write gate input, the FeFET threshold aligns with different storage values. 
%$T_{1}$ precharges $ML$ before searching. 
The 2FeFET-2R structure can store logic `1', `0', and 
\textit{don't care} wildcard state.
%\autoref{fig:2FeFET-2R Cell}(a) shows the structure of the proposed 2FeFET-2R TCAM Cell. The 2FeFET-2R unit circuit comprises a pair of parallel 1FeFET-1R structures, where the drain of the FeFET is connected to the matchline ($ML$), and the other end of the resistor connected to the sourceline ($ScL$) can be driven to $V_{write}$ or $GND$. The gate of the FeFET is connected to the bitline and searchline ($BL$/$\overline{SL}$ and $\overline{BL}$/$SL$). By varying the input to the gate, the threshold of the FeFET can be regulated to correspond to different storage values. $T_{1}$ is connected to $ML$ for precharging $ML$ before the search. The 2FeFET-2R structure can store logic `1' and `0', and $don't\text{ $care$ } $cases. 
\autoref{tab:write opetarion} outlines the write operations of the 2FeFET-2R cell. 
Data bits are written in two steps, storing complementary logic states in each FeFET. 
To write logic `1', $V_{\textit{write}}$ is applied to \textit{BL}/\textit{SL}, while `0' to \textit{ScL} and $\overline{\textit{BL}}$/$\overline{\textit{SL}}$. This sets $V_{\textit{GS}}$ of $\textit{M}_\text{1}$ to 4V, writing logic `1' to $\textit{M}_\text{1}$. In the second step, $V_{\textit{write}}$ is applied to \textit{ScL}, while gate voltage remains the same, writing logic `0' to $\textit{M}_\text{2}$. Thus, the complementary stored  values represents logic `1'.
Similarly, to write logic `0' into the cell, `0' is written to $\textit{M}_\text{1}$ and `1' to $\textit{M}_\text{2}$, respectively. 
To write \textit{don't care} state, logic `1' is written to both $\textit{M}_\text{1}$ and $\textit{M}_\text{2}$. This sets both FeFETs to high-$\textit{V}_\textit{TH}$ state, matching regardless of the search value, aligning with the masking function of `\textit{don't care}' bits.
During writes, \textit{ML} is grounded to eliminate static current. \autoref{fig:fefet}(b) displays $\textit{I}_\textit{D}$-$\textit{V}_\textit{G}$ curves for FeFETs under different write pulses.


%\autoref{tab:write opetarion} summarizes the write operations of the 2FeFET-2R unit circuit. The data bits are written into the two FeFETs in the TCAM cell in two steps, storing opposite logic values in each FeFET. In order to write logic `1', in the first step, $V_{write}$ is applied to $BL$/$\overline{SL}$, while `0' is applied to $ScL$ and $\overline{BL}$/$SL$. Therefore, the gate-source voltage ($V_{GS}$) of $M_1$ is 4V, switching the FE polarization within FeFET and writing logic `1' to $M_1$. In the second step, $V_{write}$ is applied to $ScL$, while the gate voltage of the two FeFETs remains the same as in the first step (i.e. 4V at $BL$/$\overline{SL}$ and 0 at $\overline{BL}$/$SL$). As a result, the $V_{GS}$ of $M_2$ is -4V, and the logic `0' is written to $M_2$. Thus, the 2FeFET-2R unit as a whole represents the storage of logic '1'. 


%Similarly, to write logic `0' into the TCAM cell, we write `0' to $M_1$ and `1' to $M_2$, respectively. To write $don't\text{ $care$ }$into the TCAM cell, it only takes one step to write logic `0' to both $M_1$ and $M_2$. In this way, both FeFETs corresponding to `don't care' are in low-$V_{TH}$ state, resulting in a match regardless of the search value, which aligns with the masking function of `don't care' bits. During writing operations, the matchline $ML$ needs to be driven to ground to eliminate the influence of static current on the $ML$ voltage. \autoref{fig:fefet}(b) displays the $I_{D}$-$V_{G}$ curves corresponding to different write pulses of FeFETs.

During search, \textit{ML} voltage is precharged to high via a precharge transistor, and the search voltages are applied to searchlines ($\textit{SL}$/$\overline{\textit{SL}}$) according to the query data. For logic `1', \textit{SL} set to 1V, and 0 for logic `0', the  \textit{ML} voltage indicates the matching result. 
\autoref{fig:2FeFET-2R Cell}(b) validates the function of the 2FeFET-2R cell. 
\textit{ML} is first precharged by controlling $\textit{T}_\text{1}$'s gate voltage \textit{CLK}, and then left floating upon  search phase. 
When searching `1', \textit{ML} voltage stays high with  \textit{SL} = 1V, indicating a match. Conversely, searching `0' rapidly drops \textit{ML} voltage to 0, indicating a mismatch. %Results in \autoref{fig:2FeFET-2R Cell}(b) align with expectations, validating unit circuit's storage and computing functions.


%During the search operation, the voltage of $ML$ is previously precharged to a high level through a precharge transistor, and different search voltages are applied to the searchlines ($SL$/$\overline{SL}$) according to the input data. Here we set $SL$ to 1V for logic `1' and 0 for logic `0', and observe the voltage change of $ML$ to reflect the matching result. As shown in \autoref{fig:2FeFET-2R Cell}(b), by controlling the gate voltage $CLK$ of $T_{1}$ to precharge $ML$, the precharging is halted after entering the search phase. When searching for logic '1', the voltage on $ML$ remains almost unchanged at a high level when $SL$ voltage is 1V, indicating a match between the search query and the stored entry. Conversely, when searching for logic '0', the voltage on $ML$ rapidly drops to 0, indicating a mismatch between the search query and the stored entry. The operation results shown in \autoref{fig:2FeFET-2R Cell}(b) are consistent with expectations, validating the correctness of the unit circuit's storage and computing functions.


\subsection{2FeFET-2R TCAM Array}

\autoref{fig:1x64} demonstrates the schematic of the proposed 2FeFET-2R TAP-CAM array storing a 64-bit word with corresponding peripheral circuits. 
%This array stores a 64-bit word, sharing $ML$ and $ScL$ across all 64 cells.
PMOS $\textit{T}_\text{1}$ precharges \textit{ML} before the search operation, while an evaluation transistor $\textit{T}_\text{2}$ is connected between \textit{ML} and $\textit{V}_\text{o}$ to enable tunable threshold matching function. 
%$T_2$ evaluates the $ML$ voltage for threshold match function. 
Adjusting the gate voltage of the evaluation transistor controls the discharge rate of \textit{ML}, allowing varying mismatch bits to be sensed by the sense amplifier (SA) as a match case. 
%A sense amplifier (SA) detects the  output, enhancing signal stability, forming output waveform. 


%to mitigate $ML$ parasitic capacitance impact on sensing time.
%As shown in \autoref{fig:1x64}, the 2FeFET-2R unit circuit is expanded into a 1×64 array and accompanied by corresponding peripheral circuits. This array has the capability to store a 64-bit word, with all 64 units sharing the same $ML$ and $ScL$. The $ML$ is connected to two transisitors, where PMOS $T_1$ is used to precharge $ML$ before the matching operation begins, while NMOS $T_2$ is connected to ML as a transistor for voltage evaluation, enabling threshold-controlled approximate match functionality. By adjusting the gate voltage applied to NMOS, the rate of $ML$ voltage decrease can be controlled, thereby achieving control over the permissible number of mismatched unit bits. Additionally, a sense amplifier(SA) is serially connected at the output end to shape and amplify the output results, thereby enhancing the stability of the output signal, ultimately forming the output waveform. During the precharging phase, the PMOS control signal $CLK$ is driven to a low level, while $V_{eval}$ is kept at a high level to ensure the conduction of both $T_1$ and $T_2$. This allows $ML$ to be precharged to $V_{DD}$ to mitigate the impact of parasitic capacitance of $ML$ on the sensing time. 


\label{sec:2FeFET-2R TCAM Array}

\begin{figure}
    \centering    
    \includegraphics[width=\linewidth]{Figures/1x64.png}
    \caption{Structure of a 2FeFET-2R TCAM array with wordlength 64.}
\label{fig:1x64}
%\vspace{-0.4cm}
\end{figure}

During the precharge, \textit{CLK} is set to low, turning  $\textit{T}_\text{1}$ and $\textit{T}_\text{2}$ ON, and precharging \textit{ML} to \textit{VDD}.
During the search phase, setting the \textit{CLK} signal high turns $\textit{T}_\text{1}$ OFF and cutting the charging path. 
Pre-defined bias voltages are applied to the gate of evaluation transistor  $\textit{V}_\textit{eval}$ based on required mismatch thresholds. 
A mismatch between the stored entry and the search query forms a conduction path from $\textit{V}_\text{o}$ to \textit{GND}, discharging $\textit{V}_\text{o}$ and decreasing the voltage.
The rate of voltage decrease depends on the number of mismatched cells and $\textit{T}_\text{2}$'s gate voltage $\textit{V}_\textit{eval}$. 
This rate affect  the output of SA $\textit{SA}_\textit{out}$ which indicates the time for $\textit{SA}_\textit{out}$ to transition from high to low. 
With constant $\textit{V}_\textit{eval}$, more mismatched bits increase the discharge current from $\textit{V}_\text{o}$ to \textit{GND}, accelerating $\textit{SA}_\textit{out}$ voltage drop. 
Similarly, with constant mismatched bits, higher $\textit{V}_\textit{eval}$ boosts the conduction of $\textit{T}_\text{2}$, hastening $\textit{SA}_\textit{out}$ voltage drop. Hence, given the fixed SA sense time, decreasing the $\textit{V}_\textit{eval}$ allows for increasing the mismatch threshold.
%, keeping $ML$ voltage sensing time consistent.

%During the search phase, the $CLK$ signal is set to a high level, leading to the closure of $T_1$ and the interruption of the charging path. Different voltages are applied to $V_{eval}$ based on varying matching thresholds. In a 1x64 array, a mismatch between the stored entry and the search query results in the establishment of a conduction path from $V_o$ to $GND$, causing a decline in the voltage on $V_{o}$. The rate of the voltage decline is contingent upon both the number of mismatched bits and the gate voltage $V_{eval}$ of $T_2$. This decline rate is reflected in the output voltage $SA_{out}$, representing the time required for $SA_{out}$ to transition from a high to a low level.
%When $V_{eval}$ remains constant, an increase in the number of mismatched bits leads to a greater number of conduction paths between $V_o$ and $GND$, thereby accelerating the voltage drop on $SA_{out}$. Similarly, when the number of mismatched bits is constant, a higher $V_{eval}$ enhances the conduction state of $T_2$, resulting in a faster voltage drop on $SA_{out}$. Therefore, as the matching threshold increases, we can decrease the value of $V_{eval}$, so that the sensing time for discerning the voltage status on $ML$ under different matching thresholds remains within the same range.


%\begin{figure}
%    \centering
%    \includegraphics[width=\linewidth]{Figures/R_increase.pdf}
%    \caption{(a)The $I_{ds}$-$V_{gs}$ wave of 1FeFET-1R varies with $R_S$. (b) The Search latency varies with $R_S$.}
%\label{fig:R_increase}
%\end{figure}
%\label{sec:R}


%As previously discussed, the 1FeFET-1R configuration limits conduction current and enhances the robustness by suppressing current variability.
%Higher resistance series current limiter reduces the discharge current, and improves the conduction stability. 
%Additionally, $R_S$ presence results in a more significant $ML$ voltage drop due to greater mismatched bit count. 
Without loss of generality, 
for the TAP-CAM with n bits mismatch threshold (Th-n), i.e., $\leq$n mismatch bits are sensed as a match case, and $\ge$(n+1) bits mismatch indicates a mismatch, the sense margin between the n bits mismatch and (n+1) bits mismatch is determined by the equivalent resistance and associated \textit{ML} capacitance of the array $\textit{C}_\textit{M}$.
%, as illustrated in \autoref{fig:fefet}(c), $R_{ON}$ signifies FeFET's equivalent conduction resistance, while $C_M$ represents the circuit's equivalent capacitance.
%As previously articulated, the 1FeFET-1R configuration can limit the magnitude of the conduction current and enhance 
%the robustness of the circuit. Moreover, as the resistance value of the series current limiter increases, the current gradually decreases, leading to improved stability of the circuit. In addition, we also have observed that the presence of $R_S$ results in a more significant voltage drop of $ML$ due to greater mismatched bit count. Taking n bits mismatch and (n+1) bits mismatch as examples, as illustrated in \autoref{fig:fefet}(c), $R_{ON}$ represents the equivalent conduction resistance of the FeFET, while $C_M$ represents the equivalent capacitance of the circuit. 
The equivalent resistance for the two mismatch cases can be expressed as follows:
%We can express the relationship as follows:
\begin{equation}
\label{eq:R-C circuit1}
      \textit{R}_\textit{n} = \frac{\text{1}}{\textit{n}}\cdot (\textit{R}_\textit{ON}+\textit{R}_\textit{S})
\end{equation}
\begin{equation}
\label{eq:R-C circuit2}
   %   R_{n+1} = \frac{1}{n+1}\cdot (R_{ON}+R_S)
     \textit{R}_{\textit{n}\text{+1}} = \frac{\text{1}}{\textit{n}\text{ + 1}}\cdot (\textit{R}_\textit{ON}+\textit{R}_\textit{S})
\end{equation}
where $\textit{R}_\textit{n}$ represents the approximate equivalent resistance of array with n bits mismatch, and $\textit{R}_\text{n+1}$ represents the approximate equivalent resistance of array with (n+1) bits mismatch. %$C_M$ is the associated ML capacitance.
$\textit{R}_\textit{ON}$ represents the equivalent resistance of an ON FeFET, and $\textit{R}_\textit{S}$ represents the series resistance. 
From charging and discharging formula of RC circuit, we can approximately formulate the \textit{ML} voltage \textit{U}:
\begin{equation}
\label{eq:R-C circuit3}
      \textit{U}=\textit{U}_\text{0}\cdot \textit{e}^{-\frac{\textit{t}}{\textit{RC}_\textit{M}}}
\end{equation}

\begin{equation}
\label{eq:R-C circuit4}
      \frac{\textit{dU}}{\textit{dt}}=\textit{U}_\text{0}\cdot (-\frac{\text{1}}{\textit{RC}_\textit{M}})\textit{e}^{-\frac{\textit{t}}{\textit{RC}_\textit{M}}}
\end{equation}
where $\textit{U}_\text{0}$ represents the initial voltage of \textit{ML}. 
From \autoref{eq:R-C circuit4} we can conclude that the rate of \textit{ML} voltage drop will be faster as the equivalent resistance decreases. 
%Due to the fact that the parallel resistance of n identical resistors is greater than the parallel resistance of n+1 identical resistors,
From \autoref{eq:R-C circuit1} and \autoref{eq:R-C circuit2}, $\textit{R}_\textit{n}$ is larger than $\textit{R}_{\textit{n}\text{+1}}$. 
Therefore,  the voltage of \textit{ML} corresponding to (n+1) bits mismatch drops faster than that of n bits mismatch.
Upon the sensing, the sense margin of Th-n $\Delta U$ can be expressed as follows:
%$U_n$ represents the $ML$ voltage corresponding to  n  bits mismatch, and $U_{n+1}$ represents the $ML$ voltage corresponding to the circuit with (n+1) mismatched bits. As indicated by  \autoref{eq:R-C circuit3}, we can express this relationship as follows:
\begin{equation}
\label{eq:R-C circuit5}
     \Delta \textit{U}=\textit{U}_\textit{n}-\textit{U}_{\textit{n}\text{+1}}=\textit{U}_\text{0}\cdot (\textit{e}^{-\frac{\textit{t}}{\textit{R}_\textit{n}\textit{C}_\textit{M}}}-e^{-\frac{t}{\textit{R}_{\textit{n}\text{+1}}\textit{C}_\textit{M}}}) 
\vspace{0.1cm}
\end{equation}
where $\textit{U}_\textit{n}$ represents the \textit{ML} voltage corresponding to  n bits mismatch, and $\textit{U}_{\textit{n}\text{+1}}$ represents the \textit{ML} voltage corresponding to (n+1) bits mismatch.
%With an increase in $R_S$, $\Delta$$U$ also broadens, resulting in a wider sensing margin.
From \autoref{eq:R-C circuit5}, we observe that $\textit{R}_\textit{S}$ affects the magnitude of $\Delta$$U$ over time t, thus influencing the sense margin. Simultaneously, a larger $\textit{R}_\textit{S}$ value introduces larger search delay. Therefore, selecting an appropriate $\textit{R}_\textit{S}$ value is necessary to ensure that both sense margin and search delay remain within reasonable limits. 
We here select $\textit{R}_\textit{S}$ = 0.3M.% and \textit{VDD} = 0.6V



%As the series resistance $R_S$ increases, $\Delta$$U$ also experiences an augmentation, thereby resulting in a broader sensing margin. However, this enhancement in circuit performance is accompanied by an increase in latency. Therefore, it is necessary to carefully balance both performance and latency when determining the resistance value of $R_S$.Taking all factors into account, we ultimately opt for R=0.3M and $V_{DD}$=0.6V.

Another factor that affects the sense margin and the search time is the bias voltage at evaluation transistor gate. 
To implement the functionality of bit-by-bit tunable threshold approximate matching, we determine appropriate evaluation voltages $\textit{V}_\textit{eval}$ to distinguish different mismatch thresholds, taking the threshold ranging 0-6 bits as an example.
This involves adjusting the gate voltage of the evaluation transistor to differentiate between 0-bit and 1-bit mismatch (Th-0), 1-bit and 2-bit mismatch (Th-1), and so forth. 
%adjacent numbers of mismatched bits can cause the $ML$ voltage to exhibit either a high level (maintained at 1V) or a low level (dropped to 0V) within the same time window.
%Once this time window is established, the array's capability for approximate matching can be verified by controlling the magnitude of the evaluation voltage.%To verify the proper functioning of the approximate match capability, it is essential to determine the appropriate evaluation voltage $V_{eval}$ to distinguish adjacent numbers of mismatched bits within the range of 0-6 bits. This involves adjusting the gate voltage of the evaluation transistor to differentiate between 0-bit mismatch and 1-bit mismatch, 1-bit mismatch and 2-bit mismatch, and so forth. The method of differentiation entails ensuring that within the same time window, the adjacent number of mismatched bits causes the voltage on the ML to exhibit either a high level (maintained at 1V) or a low level (dropped to 0V). Once the time window for distinguishing adjacent numbers of mismatched bits is established, we can verify that the array can achieve approximate match functionality by controlling the threshold voltage. 
Increasing the number of mismatch bits and evaluation transistor gate voltage $\textit{V}_\textit{eval}$ lead to faster $\textit{SA}_\textit{out}$ voltage decrease. 
Hence, with increasing mismatch threshold, we decrease $\textit{V}_\textit{eval}$ to maintain consistent sense time window across different mismatch thresholds.
%Based on this, we conducted experiments, obtaining different $V_{eval}$ values for matching thresholds ranging from 0 to 5 bits. 
The evaluation voltages are therefore experimentally examined and configured as summarized in \autoref{tab:Threshold-V} to ensure that the sense time for distinguishing different mismatch thresholds falls within the same time window.
Different evaluation voltages correspond to different mismatch thresholds. 
%\autoref{tab:Threshold-V} presents the evaluation transistor gate voltage for differentiating adjacent mismatched bit numbers, corresponding to different matching thresholds. 
This evaluation voltage configuration lays the foundation for subsequent performance and latency analysis.

%Different evaluation voltages correspond to different matching thresholds. As the number of mismatched bits increases and the gate voltage of the evaluation transistor $V_{eval}$ increases, both contribute to a faster decrease in the output voltage $SA_{out}$. Consequently, as the matching threshold increases, we gradually decrease the value of $V_{eval}$ to ensure a consistent sensing time window across different match thresholds. Based on this, we have conducted experiments and obtained different $V_{eval}$ corresponding to matching thresholds ranging from 0 to 5 bits. \autoref{tab:Threshold-V} presents the evaluation transistor gate voltage corresponding to distinguishing adjacent numbers of mismatched bits, i.e., the evaluation voltage corresponding to different matching thresholds. These findings will serve as the foundation for our subsequent performance and latency analysis.




\begin{table}[!t]
    \centering
    \caption{$\textit{V}_\textit{eval}$ of different mismatch threshold}
    \begin{adjustbox}{center}
    \resizebox{1\columnwidth}{!}{
\begin{tabular}{|c|c|c|c|c|c|c|}
              \hline
            \makecell{Mismatch\\Threshold(bit)} & 0 & 1 & 2 & 3 & 4 & 5   \\
              \hline
              $\textit{V}_\textit{eval}$(V) & 1 & 0.75 & 0.63 & 0.52 & 0.43 & 0.37  \\ 
              \hline
\end{tabular}

}
\label{tab:Threshold-V}
\end{adjustbox}
\end{table}
\begin{figure}
    \centering
    \includegraphics[width=\linewidth]{Figures/threshold2.png}
    \caption{Transient waveforms of \textit{ML} under different mismatch thresholds. Solid and Dashed lines represent the match and mismatch cases corresponding to a certain mismatch threshold, respectively.}
\label{fig:threshold}
%\vspace{-0.4cm}
\end{figure}

The \textit{ML} transient waveforms corresponding to different mismatch thresholds in \autoref{fig:threshold} validate the bit-by-bit tunable threshold matching function.
%variations and the sensing time window during approximate matching under corresponding evaluation voltages and matching thresholds. 
Solid lines show the \textit{ML} voltage waveforms when the number of mismatched bits equals to the pre-defined mismatch threshold, while dashed lines show the \textit{ML} voltages when the number of mismatched bits  exceeds the pre-defined threshold. 
The sense margin of mismatch thresholds decreases as the threshold increases. 
%Notably, at Th-5, common window exists between 5-bits and 6-bits mismatch, with sensing time window endpoints serving as sensing margin. 
According to \autoref{fig:threshold}, the search latency for distinguishing adjacent mismatch threshold ranging from Th-0 to Th-5  is  1 ns.


%\autoref{fig:threshold} illustrates the transient voltage variations of the $ML$ and the sensing time window during approximate matching under corresponding evaluation voltages and matching thresholds. Solid lines represent the $ML$ voltage variations when the number of mismatched bits equals the matching threshold, while dashed lines represent the ML voltage variations when the number of mismatched bits just exceeds the matching threshold.  As the threshold increases, the margin between the solid and dashed lines diminishes. Notably, at a matching threshold of 5 (Th-5), there exists a common window between 5-bits mismatch and 6-bits mismatch, with the endpoints of the sensing time window serving as sensing margin. According to \autoref{fig:threshold}, the search latency for distinguishing adjacent numbers of mismatched bits within the range of 0 to 6 bits is 1 ns.

%\begin{figure}
%    \centering
%    \includegraphics[width=\linewidth]{Figures/mxn.pdf}
%    \caption{Structure of mxn 2FeFET-2R TCAM.}
%\label{fig:mxn}
%\end{figure}

%For energy and delay analysis, we expanded original 1x64 array of 2FeFET-2R TCAM cells into larger m rows and n columns array, as shown in \autoref{fig:mxn}. Expanded array accommodates m words, each of length n. Each cell in array possesses identical functionalities. Cells in same column share $SLs$, enabling parallel search operations, while cells in same row share $ML$ voltages, controlling $ML$ voltage variations simultaneously. Write/Search buffer inputs stored/search vectors into $SLs$ for matching operations, consistent with \autoref{tab:write opetarion}. During search, all rows compare same input query with stored entries. If mismatch occurs, $ML$ discharges. If $ML$ voltage drops below sense amplifier threshold within specified time window, corresponding SA output transitions to 0, recognized by encoder as mismatch. Conversely, if match occurs, address of stored entry matching search query is output.

%For the purpose of energy and delay analysis, we have expanded the original 1x64 array of 2FeFET-2R TCAM cells into a larger array of m rows and n columns, as illustrated in \autoref{fig:mxn}. This expanded array has the capability to accommodate m words, each with a length of n. Each individual cell within this array possesses identical functionalities and characteristics. All cells within the same column share the $SLs$, allowing for parallel execution of search operations, while cells within the same row share $ML$ voltages, thereby controlling the variations in $ML$ voltage simultaneously. To facilitate the matching operations, we employ Write/Search buffer to input the stored/search vectors into the $SLs$, thereby determining whether each $ML$ discharges or not. The operations during the writing process remain consistent with those outlined in \autoref{tab:write opetarion}. During the search period, all rows compare the same input query with their stored entries. If a mismatch occurs, the $ML$ discharges. If the $ML$ voltage drops below the threshold voltage of the sense amplifier within a specified time window, the corresponding output of the SA transitions to 0, thus being recognized by the encoder as a mismatch. Conversely, if there is a match, the address of the stored entry matching the search query is output.

%Compared to single-row 2FeFET-2R TCAM, expanded m×n array has larger capacity and efficiently executes matching operations during search. By sharing $SLs$ and $ML$ voltage, array achieves highly parallel search operations, improving overall performance and efficiency.


%Compared to a single-row 2FeFET-2R TCAM, the expanded m×n array has a larger capacity and can efficiently execute matching operations during the search period. By sharing $SLs$ and $ML$ voltage, the array can achieve highly parallel search operations, thereby improving overall performance and efficiency.







% \vspace{-2ex}


% \vspace{-1em}
\section{Evaluation}
\label{sec:eval}
In this section, we first evaluate the energy and performance of the proposed TAP-CAM design. We then benchmark the proposed TAP-CAM array in the context of K-nearest neighbor search tasks as tunable approximate matching engine.

\subsection{Evaluation Setup}

\begin{figure}
    \centering
    \includegraphics[width=\linewidth]{Figures/mxn.png}
    \caption{Schematic of m$\times$n TAP-CAM array.}
\label{fig:mxn}
\end{figure}

For the energy and performance evaluations, we conduct our experiments on
%original 1x64 array of 2FeFET-2R TCAM cells into 
a TAP-CAM array with m rows and n columns, as shown in \autoref{fig:mxn}. 
%Expanded array accommodates m words, each of length n.
The cells within the same row share the \textit{ML} and \textit{ScL}, and %the array possesses identical functionalities. 
the cells within the same column share \textit{SLs}, enabling parallel search operations. 
%while cells in the same row share $ML$ voltages, controlling $ML$ voltage variations simultaneously. 
Write/Search buffer drive stored/search vectors into \textit{SLs} for search operations, consistent with \autoref{tab:write opetarion}. 
During the search, all rows compare the same input query with stored entries.
If a mismatch occurs, \textit{ML} discharges. 
If \textit{ML} voltage drops below the sense amplifier threshold within the pre-defined sense time window, the corresponding SA output transitions to 0, recognized by the decoder as mismatch. Conversely, if a match occurs, the address of the stored entry matching the search query is output.

%Compared to a single-row 2FeFET-2R TCAM, the expanded m×n array has larger capacity and efficiently executes matching operations during search. By sharing $SLs$ and $ML$ voltage, the array achieves highly parallel search operations, thereby improving overall performance and efficiency.
The proposed 2FeFET-2R TAP-CAM array is evaluated using SPECTRE. The FeFETs are simulated based on the Preisach FeFET model \cite{transfer-characteristics}.
All MOSFETs are modeled using the 45nm PTM model and the 27°C TT process corner \cite{evaluation2}. The wordlength is set to 64 cells.
%, and the write voltage is ±4V. 

\subsection{Robustness Validation}

\begin{figure}
    \centering
    \includegraphics[width=\linewidth]{Figures/MC.png}
    \caption{100 Monte Carlo simulations considering device-to-device variations: \textbf{(a)} The output waveforms under \textit{VDD} = 0.6V; \textbf{(b)} The output waveforms under \textit{VDD} = 1V.}
\label{fig:MC}
%\vspace{-0.2cm}
\end{figure}

The robustness of the proposed TAP-CAM design under varying operating conditions is examined, specifically with \textit{VDD} = 0.6V and \textit{VDD} = 1V, respectively. 
%To control the experimental variability of the FeFETs,
The FeFETs are assumed to feature the stored low/high $\textit{V}_\textit{TH}$ threshold voltage states with a deviation $\sigma$ = 54mV, and 8$\%$   series resistor variability is considered \cite{area}.
100 Monte Carlo simulations have been conducted  to distinguish between 5-bits and 6-bits mismatches when the mismatch threshold is set to 5 bits (Th-5). 
\autoref{fig:V&N} consistently reveals that the time windows across the 100 runs can be identified.
This observation suggests that the proposed design effectively distinguishes between the adjacent numbers of mismatched bits by employing the evaluation transistor. 
%Furthermore, the design maintains consistent search performance under different experimental conditions. 
Based on these results, it can be inferred that the proposed TAP-CAM design  demonstrates the robustness, as it reliably achieves approximate threshold matching functionality given the variations in operating voltage and device variations.



\subsection{CAM Array Evaluation}
%Here we conduct the evaluations of our 
\begin{figure}
    \centering
    \includegraphics[width=\linewidth]{Figures/energy1.jpg}
    \caption{Energy and latency of the proposed 2FeFET-2R TAP-CAM array
with varying \textbf{(a)} \textit{VDD}; \textbf{(b)} mismatch thresholds; \textbf{(c)} number of rows and \textbf{(d)} number of bits per row.}
\label{fig:V&N}
%\vspace{-0.4cm}
\end{figure}

The search energy consumption of the proposed array mainly originates from precharging the \textit{ML} and SA energy consumption.
Precharging the \textit{ML}, primarily done by $\textit{T}_\text{1}$, depends heavily on \textit{VDD} and the associated \textit{ML} parasitic capacitance.
\autoref{fig:V&N}(a) demonstrates the impact of scaling \textit{VDD} on the search energy consumption and latency. 
As \textit{VDD} scales up, the precharging energy increases, leading to overall higher search energy consumption. At the same time, the amplitude of \textit{ML} dropping from high to low level when mismatch occurs increases, thereby increasing the search delay.  
\autoref{fig:V&N}(b) shows the sense time and sense margin for different mismatch thresholds at \textit{VDD} = 1V. The sense margin is the narrowest at the 5-bit mismatch threshold (Th-5), thus is selected as the sense margin for the SA sense time. 
\autoref{fig:V&N}(c) demonstrates how search energy and latency change with varying row numbers. Increased rows allow parallel search operations, linearly increasing the energy consumption with negligible latency change. 
Finally, \autoref{fig:V&N}(d) examines the wordlength's effect on the search latency and energy consumption per bit. 
Longer wordlengths associate more parasitic capacitance on the \textit{ML}, slowing down the discharge speed and thus increasing the search latency. The increase in capacitance 
leads to a rise in precharge energy per word. But increasing wordlength has minimal impact on the energy consumption of SA, so the search energy per bit decreases. The increasing latency and decreasing energy consumption per bit show trade-offs in the CAM array design optimization.

%The energy consumption in this process predominantly originates from two sources: the precharging of the $ML$ and the energy consumption by the SA. The precharging of the $ML$, primarily accomplished by $T_1$, is heavily contingent upon the magnitude of $V_{DD}$ and the parasitic capacitance. 

%\autoref{fig:V&N}(a) illustrates the impact of increasing $V_{DD}$ on energy consumption and latency. As $V_{DD}$ increases, precharging energy also increases, leading to overall higher energy consumption and latency. 
%\autoref{fig:V&N}(b) presents the sensing time and sensing margin for different matching thresholds at $V_{DD}$=1V. Notably, the sensing margin is narrowest at the 5-bit matching threshold, which determines the final sensing margin. \autoref{fig:V&N}(c) demonstrates how energy and latency vary with different numbers of rows. With increased row numbers, multiple rows can operate simultaneously, linearly increasing power consumption without affecting latency.
%Finally, \autoref{fig:V&N}(d) examines the effect of word length on latency and energy consumption per bit. Longer word lengths result in higher parasitic capacitance on the $ML$, slowing discharge speed and increasing latency. Conversely, longer word lengths lead to lower energy consumption per bit, highlighting the trade-offs in CAM array design optimization.


\begin{table}
\centering
% \footnotesize
\resizebox{\linewidth}{!}{
\setlength{\tabcolsep}{5pt}
\begin{tabular}[t]{l|ccc}
\toprule
 \makecell[c]{\textbf{Method}} & \makecell[c]{\textbf{Self}\\\textbf{Reflection}} & \makecell[c]{\textbf{Memory}} & \makecell[c]{\textbf{Length}\\\textbf{Generalization}} \\
\midrule
Revision~\cite{DBLP:journals/corr/abs-2408-03314} & \redcross & \greencheck & \redcross \\
Self-Refine~\cite{DBLP:conf/nips/MadaanTGHGW0DPY23} & \greencheck & \greencheck & \redcross \\
Best-of-N~\cite{DBLP:journals/corr/abs-2407-21787} & \redcross & \redcross & \greencheck \\
Beam Search~\cite{ow1988filtered} & \redcross & \redcross & \greencheck \\
Guided Beam Search~\cite{DBLP:conf/nips/XieKZZKHX23} & \greencheck & \redcross & \greencheck \\
\midrule
\textbf{FTTT (ours)} & \greencheck & \greencheck & \greencheck \\
\bottomrule
\end{tabular}
}
% \vspace{-5pt}
\caption{Comparing the advantages and drawbacks of FTTT and related works.}
\label{tab:compare}
% \vspace{-0.5cm}
\end{table}

\autoref{tab:compare} provides a comprehensive comparison of the proposed 2FeFET-2R TAP-CAM with other CAM designs, in terms of  device type, technology node, device count per cell, cell size, performance and normalized search energy. Cell size estimation is based on a 2$\times$2 layout of the 2FeFET-2R TAP-CAM array.
%\autoref{tab:compare} presents a comprehensive comparison of the 2FeFET-2R CAM with other CAM designs, including the utilized technology, node, device count per cell, cell size, search delay, and search energy per bit per search.  The cell size estimation is based on a 2x2 layout of the 2FeFET-2R CAM array. 
%Our design leverages innovative FeFET technology, enabling approximate matching through different thresholds, achieving significant breakthroughs in circuit functionality and energy consumption. 
Compared to the conventional  CMOS  CAM designs, our proposed  2FeFET-2R TAP-CAM design offers a much smaller cell size. 
The  comparisons highlight the significant advantages of the proposed 2FeFET-2R TAP-CAM design over other CAM designs in terms of energy consumption per bit per search.
The energy efficiency of 2FeFET-2R TAP-CAM is notably superior, being 16.95$\times$, 12.88$\times$, 9.49$\times$, and 6.78$\times$ more efficient compared to 16T TCAM, 10T CAM, 2T-2R TCAM, and 2FeFET TCAM, respectively. 
While some existing designs  achieve approximate search functionality, their energy consumption remains substantially higher than that of 2FeFET-2R structure. 
%Utilizing ferroelectric transistor technology, 2FeFET-2R design represents promising direction for CAM design innovation, providing practical solution for enhancing search efficiency, reducing energy consumption, and optimizing area costs.
Although our design incurs relatively high search delay, considering the search latency and energy  trade-offs and the substantial energy advantages of our proposed design, increased delay is deemed acceptable.


These findings validate the remarkable energy efficiency of 2FeFET-2R TAP-CAM  array, emphasizing its immense potential for data-intensive search applications. This suggests that 2FeFET-2R TAP-CAM architecture is well-positioned to address the evolving needs of modern computing environments, particularly those requiring efficient and high-performance solutions for processing large volumes of data in search-intensive applications.

%Our design leverages innovative FeFET technology, enabling approximate matching through different thresholds, and has achieved significant breakthroughs in both circuit functionality and energy consumption. Compared to CMOS technology, the 2FeFET-2R structure offers a smaller cell size, thus reducing area costs. Although our design incurs a relatively high search delay, considering the trade-off between latency and energy consumption, as well as the substantial energy advantages of the 2FeFET-2R structure, the increased search delay is deemed acceptable. The comparison provided highlights the significant advantages of the 2FeFET-2R CAM structure over several other CAM designs in terms of energy consumption per bit per search. The energy efficiency of the 2FeFET-2R CAM is notably superior, being 16.95$\times$, 12.88$\times$, 8.81$\times$, and 6.78$\times$ more efficient compared to 16T CAM, 10T CAM, 4T-2FeFET CAM, and 2FeFET CAM, respectively. Despite the capability of some alternative designs to achieve approximate search functionality, their energy consumption remains substantially higher than that of the 2FeFET-2R structure. These findings validate the remarkable energy efficiency of the 2FeFET-2R CAM circuit array, emphasizing its immense potential for data-intensive search applications. Utilizing ferroelectric transistor technology, the 2FeFET-2R design represents a promising direction for CAM design innovation. It presents a practical solution for enhancing search efficiency, reducing energy consumption, and optimizing area costs.








\subsection{Case Study: K-Nearest Neighbor Search}
\begin{figure*}
    \centering
    \includegraphics[width=\linewidth]{Figures/csj_benchmark.pdf}
    \caption{\textbf{(a)} KNN clustering accuracy under different \design thresholds, ranging from Th-1 to Th-6 (left to right); \textbf{(b)} Computational speedup and \textbf{(c)} energy efficiency improvement of \design with varying wordlengths compared to a GPU implementation. Datasets from left to right are Iris, Wine and Digits. }
    \label{fig:benchmark}
\end{figure*}
To demonstrate the efficiency of the proposed  design, we benchmark the proposed 2FeFET-2R TAP-CAM array in the context of  K-nearest neighbor (KNN) search framework. 
KNN, a fundamental algorithm in machine learning, embodies a non-parametric supervised model, particularly effective when $\textit{K = }1$, representing the nearest neighbor (NN) classification. 
This algorithm finds widespread use across various fields, including HDC  \cite{liu2022cosime, shou2023see}, reinforcement learning \cite{li2022associative}, and bioinformatics \cite{laguna2020seed}, etc.

At the core of the KNN approach lies the calculation of distances between the query instance, denoted as $x$, and the stored vectors, denoted as $y_i$, within the CAM array.
This process utilizes a distance function, typically denoted as $d(x, y_i)$, which quantifies the dissimilarity or similarity between the data points. 
When $\textit{K = }1$, i.e. NN classification, the class label attributed to the query instance $x$ corresponds to the category of the nearest stored vector $y_i$, identified by the smallest distance metric. This intuitive method allows for straightforward classification based on proximity, making it particularly suitable for scenarios with intricate decision boundaries or complex dataset patterns.
Conversely, when \textit{K} exceeds 1 instead of relying on the nearest neighbor, the algorithm considers the k closest neighbors of the query instance $x$. The class label assigned to $x$ is determined by a majority voting mechanism, where the most frequent class label among the k nearest neighbors prevails. 
This adaptive approach enables KNN to capture more nuanced relationships within the dataset, thereby enhancing its predictive capability and robustness in various applications.


%In the case of NN classification ($K=1$), the class label attributed to the query instance $x$ corresponds to the category of the nearest stored vector $y_i$, identified by the smallest distance metric. This intuitive method allows for straightforward classification based on proximity, making it particularly suitable for scenarios with intricate decision boundaries or complex dataset patterns.

%Expanding on this foundation, the KNN algorithm handles cases where K exceeds 1. In such situations, instead of relying on the nearest neighbor, the algorithm considers the K closest neighbors of the query instance $x$. The class label assigned to $x$ is determined by a majority voting mechanism, where the most frequent class label among the K nearest neighbors prevails. This adaptive approach enables KNN to capture more nuanced relationships within the dataset, thereby enhancing its predictive capability and robustness in various applications.

%In benchmarking our proposed 2FeFET-2R TAP-CAM architecture within the KNN framework, our goal is to showcase its versatility, efficiency, and applicability across various machine learning tasks. Through thorough evaluation and comparison with existing methodologies, we aim to highlight the potential of our design to advance CAM technology and contribute to machine learning research and development.

In benchmarking our proposed 2FeFET-2R TAP-CAM, for a given a function $d(x,y_i)$, which measures the distance between the query $x$ and the i-th stored vector $y_i$ in the CAM array, NN assigns the class label with the smallest distance value to $x$. Similarly, in KNN, given a query $x$, it assigns the most common class label of $x$'s k nearest neighbors to $x$ \cite{jiang2007survey}, as illustrated in \autoref{eq:knn}.
\begin{equation}
\label{eq:knn}
    c(x) = argmax\ \sum^k_{i=1} \delta(c,c(y_i))
\end{equation}
where $c(x)$ represents the class label of the query $x$, while $c(y_i)$ represents that of $y_i$. $y_i$ with $i$ ranges from 1 to k represent the k nearest neighbors. We have $\delta(c,c(y_i))=1$ when the query's label $c$ equals the label of $y_i$, otherwise $\delta(c,c(y_i))=0$.

\section{Dataset Generation}
\label{sec:dataset}
\revise{
To train the proposed GNN, we constructed a dataset of building structures and a subset of these structures were subjected to fire simulations using FEA. The dataset generation process is illustrated in \figref{fig:dataset_generation_procedure}. Initially, a total of 33,000 building structures with geometrical details, material properties, and gravity loads were created. Due to randomness in generating these structures, a filter is applied to remove unreasonable data after gravity load simulation, which included 15,377 structures. A trade-off between computational feasibility and model performance is made among the remaining 17,623 structures. As further labeling structures with MIDR requires resource-intensive fire simulations via OpenSeesRT, a large proportion of 16,050 structures is selected as unlabeled dataset. On the other hand, each of the other 1,573 structures was further subjected to 30 different fire simulations, forming the labeled dataset containing $1,573\times 30 = 47,190$ fire cases.} This section details the step-by-step process for generating the dataset, including geometry creation, material property assignment, and simulations due to gravity loads and fire scenarios. 
% To train the proposed neural network, we constructed a dataset comprising building structure data and a subset of fire scenario data. The dataset generation process is illustrated in \figref{fig:dataset_generation_procedure}. 
% A total of 33,000 building structures with geometric details, material properties, and gravity loads were initially created. Out of these, 3,000 structures were selected as labeled data, and the remaining 30,000 were designated as unlabeled data. Further, about half of them filtered out due to instability under gravity loads only. 
\begin{figure*}[h!]
    \centering
    \includegraphics[width=0.8\linewidth]{figures/dataset_filter_procedure.pdf}
    \caption{Workflow for dataset generation (geometry, material property, gravity loads, and fire scenarios).}
    \label{fig:dataset_generation_procedure}
\end{figure*}

\subsection{Geometry Generation}
\label{subsec:geometry_generation}
The geometry of the building structures forms the foundation of the dataset. Regular 
\revise{3D structures} resembling multi-story parking structures or shopping malls were generated, with parameters such as building floor dimensions and story heights selected randomly. Each building structure is composed of multiple rooms, which serve as the basic unit in this study. A room herein is a cuboid space defined by specific length, width, and height. Within a structure, rooms of the same dimensions are uniformly arranged along the length, width, and height, corresponding to the $x$-, $y$-, and $z$-axes, respectively. Structures vary in room size and number of rooms along each axis. Specifically, the room length, width, and height are independently sampled from a uniform distribution within the interval $[2, 5]$ meters along the three directions of the structure. Similarly, the room number along each axis is uniformly sampled independently as an integer within the interval $[2, 7]$, i.e., the maximum number of stories of the buildings simulated in this study is 7.

To introduce variability and simulate real-world scenarios, approximately $8\%$ of structural elements (beams or columns) are randomly removed after initial geometry creation. 
\revise{Such removal is not fire-induced damage, but reflects functional diversity often observed in real buildings, such as open spaces designed for activities in shopping malls, e.g., ice skating rinks. Examples of the generated geometries are illustrated in \figref{fig:example_generated_geometry}, showcasing the diversity and realism of the dataset. This element removal does not affect the definition of room's geometry in the structure and nor does it affect the number of considered fire scenarios.} 

\revise{A range of coefficient of variation values ($3.3\%$ to $17.5\%$) was derived from prior studies that investigated the statistics of geometrical and material properties of structural components of buildings (e.g., \cite{mirza1979variations, lee2004probabilistic}). These studies provide empirical data on the natural variability in parameters such as Young's modulus, yield strength, and dimensions of structural elements due to manufacturing tolerances and material inconsistencies. By selecting $8\%$ for the removal of structural elements in our database, we aimed to maintain a level of variability that is representative of real-world uncertainties while ensuring computational feasibility. This choice ensures that the database captures realistic deviations without introducing extreme cases that may not be commonly encountered in practice.}

\begin{figure*}[h!]
    \centering
    \includegraphics[width=\linewidth]{figures/example_generated_geometry.pdf}
    \caption{Examples of generated structural geometry of different sizes (all dimensions in meters).}
    \label{fig:example_generated_geometry} 
\end{figure*}

{\blockRevise

In this study, we opted for a deterministic square, dimension of $0.1$ m, solid cross-sectional steel elements due to their simplicity in modeling and analysis. Square sections exhibit uniform geometrical properties in all directions, simplifying the computation of structural responses and avoiding complications associated with more complex shapes, such as wide-flange sections, facilitating the computational efficiency and scalability to generate a large dataset. This choice also helps to mitigate issues related to stress concentrations and facilitates a more straightforward representation of structural behavior under thermal loads. 

\textit{Remark:} The selected cross-section provides a comparable flexural rigidity to a $W 130 \times 130 \times 28.1$ wide-flange section (metric units), albeit with significantly higher axial rigidity. This cross-section is acceptable for gravity-load-designed frames under service loading conditions where the models assume fully rigid, moment-resisting beam-column connections for the evaluation of the IDR under thermal loading. This assumption is reasonable in this computational study where the primary interest is to understand the global deformation response of frames under fire conditions. The selection of uniform square cross-sections for both beams and columns, rather than adherence to standard capacity design principles, was made here primarily for computational efficiency and to reduce design parameters in the database generation process. This choice allows for simplified and scalable approach to analyze the fire-induced response of generic steel frames without the need for large section variations, where this study mainly focuses on the fire vulnerability assessment using ML-based predictions. However, if additional loading conditions, e.g., seismic or wind loads, were to be considered, larger sections, strong-column/weak-beam principle, and ductile detailing would be required in the generated buildings for realistic structural behavior under combined loading conditions. Future studies may also consider investigating the influence of variable cross-sectional dimensions and semi-rigid connections on the structural performance under fire conditions. 
} % blockRevise

\subsection{Material Properties}
Steel is chosen as the material for the structures. To reflect real-world variations, we randomly assign one of five slightly different steel material types to each structural element. \revise{
The ranges of material properties are provided in \tabref{tab:material_property_ranges} and the properties are sampled from uniform distributions of the corresponding ranges. These variations simulate differences arising from manufacturing batches or regional material properties. That these properties are at ambient temperature and change when the temperature rises due to a fire. The selection of materials with varying properties is aimed at increasing the diversity of the data. Our goal is to represent as wide a range of data as possible with a limited amount of building structure data, thereby enhancing the generalization ability of the GNN. Our assumed material property ranges are expected to be wider than the real-world conditions based on findings in \cite{mirza1979variations, lee2004probabilistic}. Therefore, we are essentially tackling a more challenging and general task. If we can solve this problem, we are confident that our method will perform equally well or even better in real-world scenarios.
}
\begin{table}[h!]
    \centering
    \caption{Material properties ranges for considered steel structures.}
    \begin{tabular}{lc}
        \toprule
        Property & Range \\
        \midrule
        Young's modulus & [168, 252] GPa \\
        Yield strength & [220, 330] MPa \\
        Strain-hardening ratio & [0.8, 1.2] \% \\
        \bottomrule
    \end{tabular}
    \label{tab:material_property_ranges}
\end{table}

\subsection{Gravity Loads}
Gravity loads are applied to columns and beams based on their \revise{influence (tributary) areas as typically conducted in structural analysis. The considered ``service'' load conditions include the column self-weight and the additional loads directly supported on the beams from their self-weight and weights of the reinforced concrete slabs, people as live load, and building content. An edge beam typically carries approximately half the gravity load supported by a parallel interior beam}. The ranges of gravity loads are listed in \tabref{tab:gravity_load_ranges}. \revise{The loads are sampled from uniform distributions of the corresponding ranges.} Structures that failed to meet an MIDR threshold of $1\%$ under gravity loads were deemed unacceptable designs and filtered out, as such configurations of randomly chosen geometry, material, and gravity load combinations were considered unrealistic from a regulatory and practicality points of view.
\begin{table}[h!]
    \centering
    \caption{Gravity load ranges for considered beams and columns.}
    \begin{tabular}{lc}
        \toprule
        Element & Range (kN/m)  \\
        \midrule
        Column & [0.5, 1.0]  \\
        Edge beam & [1.5, 4.5]  \\
        Interior beam & [3.0, 7.5]  \\
        \bottomrule
    \end{tabular}
    \label{tab:gravity_load_ranges}
\end{table} 

\subsection{Rule-based Thermal Load Generation}
\label{subsec:thermal_load_generation}
To evaluate a building's structural response during a fire event, we employed a simplified rule-based approach for thermal load generation. 
% Previous studies \cite{nan_structuralfire_2023} have demonstrated that steel structures rapidly equilibrate with surrounding gases temperatures due to efficient heat exchange. Consequently, gas temperatures can be directly used as inputs for FEA tools, e.g., OpenSees, simplifying the process of modeling thermal loads. 
% Accurately simulating temperature fields in fire scenarios poses significant challenges. Advanced thermodynamic simulations, such as those performed using Fire Dynamics Simulator (FDS) \cite{mcgrattan_fire_2000}, provide precise temperature predictions. However, these methods are hindered by high computational costs, prolonging execution times, and limited scalability, making them impractical for generating large datasets. Additionally, real-world fire loads often display substantial spatial variability across different rooms \cite{dundar_fire_2023}, resulting in scenario-specific temperature fields with limited generalizability. For example, studies on bridge fires \cite{he_study_2024} have demonstrated that environmental factors, such as wind speeds, can significantly influence temperature distributions. Furthermore, even within identical scenarios, variations in fire modeling methodologies can produce distinctly different temperature fields \cite{zhang_temperature_2020, du_new_2012}. These challenges emphasize the need for efficient and adaptable methods to generate fire temperature data.
% To address these issues, we adopted a rule-based approach to model temperature variations. 
According to \cite{spearpoint_fire_2008}, a typical fire development follows a predictable pattern. During the {\em{growth stage}}, the temperature rises slowly and approximately linearly after ignition. This is followed by the {\em{flashover stage}}, where temperatures increase rapidly to peak values. After reaching the peak, the temperature either stabilizes or continues to rise slowly until the {\em{decay stage}} begins. Inspired by this fire development pattern, we describe the temperature evolution in time, $t$, prior to the decay stage in two distinct stages:
\begin{enumerate}
    \item {\bf{Initial linear increase stage}}: For $t \in [0, t_1)$, temperature increases gradually and linearly as the fire spreads through the building. This stage represents the time before the fire directly affects a structural element.  
    \item {\bf{ISO 834 fire curve stage}}: For $t \in [t_1, t_{\thre}]$, temperature rises rapidly following the ISO 834 curve \cite{ISO834}, modeling the direct impact of the fire on the structural element. 
\end{enumerate}
The slope of the linear temperature increase, $c$, and the transition time, $t_1$, are influenced by the spatial relationship between the fire source and the structural element. For the second stage of temperature evolution, we utilize the ISO 834 curve, a widely accepted standard for fire resistance testing. This standardized fire curve describes the temperature rise over time, enabling rapid and consistent thermal fields across various scenarios. The duration of fire simulation in this study is set to $t_{\thre}=60$ minutes. This value represents the upper limit for the temperature evolution of each structural element, providing a consistent basis for analyzing the structural response to fire.

Let $(x, y, z)$ represents the midpoint of a structural element and $(x_{\subfire}, y_{\subfire}, z_{\subfire})$ the fire source point. \revise{Integer parameters $h$ and $h_{\subfire}$ correspond to the respective floor levels of the element and the fire source}. The temperature evolution for each element is expressed as follows:
\begin{enumerate}
    \item Linear increase stage ($0 < t < t_1$):
    \begin{equation}
    T(t) = c \cdot t,
    \end{equation}
    where $c$, the rate of temperature increase ($^\circ\mathrm{C}/\mathrm{min}$), depends on the height difference between the element, $h$, and the fire source, $h_{\subfire}$:
    \begin{equation}
        c = 
        \begin{cases} 
        5\left/\left(h - h_{\subfire} + 1\right)\right., & h \geq h_{\subfire}, \\
        2\left/\left(h_{\subfire} - h\right)\right., & h < h_{\subfire}.
        \end{cases}
    \end{equation}
     \item ISO 834 stage ($t \geq t_1$):
\begin{equation}
    T(t) = c \cdot t_1 + 345 \log_{10} \left(8 \left(t - t_1\right) + 1\right).
\end{equation}
\end{enumerate}

The transition (arrival) time $t_1$, marking the end of the linear stage, depends on the spatial distance between the fire source and the element. We define the following two Euclidean distances $L_p$ in the $xy$ plane and $L_s$ in the $xyz$ space:
\begin{eqnarray}
L_p & \triangleq & \sqrt{(x - x_{\subfire})^2 + (y - y_{\subfire})^2}, \\
\label{eq:Lp}
L_s & \triangleq & \sqrt{(x - x_{\subfire})^2 + (y - y_{\subfire})^2 + (z - z_{\subfire})^2}.
\label{eq:Ls}
\end{eqnarray}
Accordingly, the transition time, $t_1$, is expressed as follows:
\begin{equation}
    t_1 = 
    \begin{cases}
    \beta_{1} \cdot \left(1 - \exp\left\{- L_s\left/\alpha_{1}\right.\right\}\right), & h > h_{\subfire}, \\
    \beta_{2} \cdot \left(1 - \exp\left\{- L_p\left/\alpha_{2}\right.\right\}\right), & h = h_{\subfire}, \\
    \beta_{3} \cdot \left(1 - \exp\left\{- L_s\left/\alpha_{3}\right.\right\}\right), & h < h_{\subfire} .
    \end{cases}
    \label{eq:t1}
\end{equation}
The parameters $\beta_i$ and $\alpha_i$ for determining $t_1$ are summarized in Table~\ref{tab:fire_spread_parameters}. In this study, we take $r_{\mathrm{up}}=0.95$ and $r_{\mathrm{down}}=0.97$.
\begin{table}[ht]
    \centering
    \caption{Fire spread parameters for $t_1$ calculations.}
    \begin{tabular}{lcc}
        \toprule
        Case  & $\beta_i$ & $\alpha_i$  \\
        \midrule
        $i=1$, Upward spread & $16 \left.\left(1-r_{\mathrm{up}}^{\left|h-h_{\subfire}\right|}\right)\right/\left(1-r_{\mathrm{up}}\right)$ & $10$  \\
        $i=2$, Horizontal spread & $18$ & $18$  \\
        $i=3$, Downward spread & $30 \left.\left(1-r_{\mathrm{down}}^{\left|h-h_{\subfire}\right|}\right)\right/\left(1-r_{\mathrm{down}}\right)$ & $5$  \\
        \bottomrule
    \end{tabular}
    \label{tab:fire_spread_parameters}
\end{table}

\figref{fig:t1_curve} illustrates the $t_1$ curves for various fire scenarios: (1) fire originating on the lower floor, $h-h_{\subfire}=1$ with rapid upward spread, (2) fire on the same floor, $h=h_{\subfire}$ with the fastest spread, and (3) fire on the upper floor, $h_{\subfire}-h=1$ with slow downward spread. The exponential decay in $t_1$ reflects the accelerating fire propagation speed as the distance increases. \figref{fig:t1_curve} also indicates that the employed simplified model is consistent with the Markov chain-based dynamic model given by \cite{cheng_dynamic_2011}, where the rooms at the same floor of the fire point start flashover slightly before the corresponding upper floors. Additionally, $\beta_{1}$ and $\beta_{3}$ are the summation of a geometric sequence, where story level $h$ is the index. The common ratios $r_{\mathrm{up}}<1$ in $\beta_{1}$ and $r_{\mathrm{down}}<1$ in $\beta_{3}$ indicate that the fire speeds up to spread through the next story, which is consistent with the real-world fire spread mechanism given in \cite{hokugo_mechanism_2000}. The temperature profile within the range $t \in [0, t_{\thre}]$ is subsequently used as the thermal load in OpenSeesRT simulations to compute displacements at each structural node at time $t_{\thre}$.
\begin{figure}[h!]
    \centering
    \includegraphics[width=0.8\linewidth]{figures/m204_t1_curve.pdf}
    \caption{Three examples for the $t_1$ curve.}
    \label{fig:t1_curve}
\end{figure}

\revise{
\textit{Remark:} The effects of structural elements, such as concrete floor slabs and partitions, are not explicitly modeled in our approach. Instead, their influence is implicitly captured through the careful selection of the parameters $ \alpha, \beta, r_\mathrm{up} $, and $ r_\mathrm{down} $. This parameterization provides a unified framework for generating temperature fields. Indeed, fire propagation is governed by a multitude of factors and remains an open research question. For instance, if the fire resistance of a floor slab is enhanced by fire protective coating, the corresponding model can account for this by decreasing $\alpha_1$ \& $\alpha_3$, increasing $\beta_1$ \& $\beta_3$, and adopting larger values for $r_\mathrm{up}$ \& $r_\mathrm{down}$, which collectively slow down the vertical spread of fire. Conversely, scenarios involving higher amounts of combustible materials would warrant the opposite adjustments. This flexible and integrated approach avoids the need to design separate models for different fire propagation scenarios while still capturing the essential effects.
}

\revise{
In conclusion, our rule-based approach is a computationally efficient method for approximating fire temperature fields, enabling large-scale dataset generation to train predictive models. By combining ISO 834 fire curves with spatial considerations and embedding structural effects through parameter calibration, the method achieves a balanced trade-off between accuracy and scalability, making it a practical solution for thermal load modeling in fire scenarios. After generating the temperature of each beam or column according to the middle point, the temperature is applied as uniform thermal load to the elements of the structure in question using OpenSeesRT. 
}

% In conclusion, this rule-based approach is a computationally efficient method to approximate fire temperature fields, enabling large-scale dataset generation to train predictive models. By combining ISO 834 fire curves with spatial considerations, the method balances accuracy and scalability, making it a practical solution for thermal load modeling in fire scenarios.

% \subsection{Interstory Drift Ratio}
\subsection{OpenSeesRT Simulation}
\label{subsec:opensees_simulation}

The thermal and mechanical responses of 3D frame structures under combined fire and gravity loads are simulated using OpenSeesRT \cite{perez2024openseesrt}. \revise{In the simulation, the IDR of each node at $t_{\thre}$ is computed using the computed nodal displacements. Each structural model features six degrees of freedom per node (3 translational  and 3 rotational), with linear geometrical transformations (\texttt{geomTransf: Linear}) defining how the element local coordinate systems are mapped to the global coordinate system and assuming small displacements and rotations. Although OpenSeesRT allows a variety of options for modeling finite deformations, in the present simulations and mainly for simplicity, we did not consider large deformations. All bottom nodes (nodes on the ground) are fully constrained in all six degrees of freedom, while degrees of freedom os all other nodes are free.} Material behavior is temperature-dependent and modeled with \texttt{Steel01Thermal}, while fiber-based sections (\texttt{FiberThermal}) capture nonlinear interactions between thermal and mechanical responses at the cross-section level. \revise{Structural elements are represented as displacement-based Euler-Bernoulli beam-columns (\texttt{dispBeamColumnThermal}). This element  formulation accounts for thermal strains (temperature gradients) in the section, which is discretized into fibers. Numerical integration is used along the length of each element using three integration (Gauss) points, one at each end and the third in the middle of the element.}

{\revise{Thermal expansion of steel members plays a crucial role in IDR development. In reality, reinforced concrete floor slabs heat at a different rate than steel members due to their higher thermal mass and lower thermal conductivity. This differential heating can lead to restrained thermal expansion, introducing axial compression in beams and affecting the overall structural response. In this study, explicit {\em{composite action}} between steel members and concrete slabs is not modeled. Instead, our approach focuses on isolating the response of the steel structural frame, which is often the critical load-bearing component in fire scenarios. This assumption aligns with prior studies \cite{Possidente_2024} demonstrating that steel structures reach thermal equilibrium with surrounding gases quickly, allowing the use of uniform thermal loading in fire analysis. Future work could enhance this framework by incorporating slab-beam interaction effects, through a refined FEA for an extended dataset where constraints imposed by floor slabs are explicitly considered.}

The analysis begins with the application of gravity loads, followed by incremental thermal loads simulating the fire exposure. A static nonlinear solver using  \texttt{ExpressNewton} algorithm ensures convergence, while the \texttt{NormDispIncr} test maintains accuracy. An incremental \texttt{LoadControl} scheme with small step sizes is employed to guarantee numerical stability, using 10\% for gravity loads and 1\% for thermal loads. 

\revise{
In the thermal load analysis, uniform thermal load is applied to each beam or column, i.e., the temperature of each element is set to be that at the middle point, according to \secref{subsec:thermal_load_generation}. The \texttt{Steel01Thermal} material allows the properties (e.g., Young's modulus and yield strength) to be adjusted at increasing temperatures according to \cite{EN1993} using its Table 3.1: Reduction factors for the stress-strain relationship of carbon steel at elevated temperatures. For example, if the Young’s modulus at ambient temperature is $E_0$, then as the temperature ($T$) increases, the modulus changes as $E(T) = \eta (T) \times E_0$. \cite{EN1993} directly provides the values of $\eta(T) \in \left[0,1\right] $ at every $100 ^\circ\mathrm{C}$ interval and recommends using linear interpolation to obtain $\eta(T)$ for intermediate values of $T$.
} OpenSeesRT documentation \cite{OpenSeesThermalExamples} provides several examples of thermal analyses.

This modeling framework accommodates variations in material properties, cross-sectional geometries, and temperature profiles, providing robust simulations of structural behavior under fire conditions. The primary settings and configurations for the OpenSeesRT simulations are summarized in \tabref{tab:ops_detail}.
\begin{table}[h!]
    \centering
        \caption{Key settings of OpenSeesRT simulations.}
    \begin{tabular}{l|>{\raggedright\arraybackslash}p{0.6\linewidth}} %
    \toprule
    Modeling Aspect     & Details \\
    \midrule
    Geometry            & 3D models; 6 degrees of freedom per node \\
    Transformation      & geomTransf: Linear \\ 
    Material            & Steel01Thermal \\
    Section             & FiberThermal; Cross-section: $0.1$ m $\times$ $0.1$ m \\ 
    Element type        & {dispBeamColumnThermal} \\ 
    Loading             & Gravity loads: {beamUniform}; Thermal loads: {beamThermal} \\
    Integration scheme  & Incremental {LoadControl}; Step size: $10\%$ (gravity analysis), $1\%$ (thermal analysis) \\
    Nonlinear solver    & {ExpressNewton} algorithm; {UmfPack} solver; Convergence test: {NormDispIncr} tolerance: $10^{-8}$; Maximum \# iterations per step: $1000$. \\ 
    \bottomrule
    \end{tabular}
    \label{tab:ops_detail}
\end{table}

For each structure in the labeled dataset, 30 fire points are selected using a dual-granularity approach, \revise{i.e., two-stage sampling strategy,} to ensure they are well-distributed. Specifically, rooms are sequentially selected, with one fire point randomly chosen within each selected room. If a building is large and contains more than 30 rooms, we randomly select 30 rooms without replacement, i.e., ensuring that no more than one fire point is located in the same room. Conversely, if the building is small and has fewer than 30 rooms, all rooms are initially selected, with one fire point randomly assigned to each room. Additionally, rooms are then selected with replacement until a total of 30 fire points are assigned. \revise{The room-level sampling prioritizes selecting distinct rooms to avoid spatial clustering of fire points, while the point-level sampling ensures intra-room variability. This approach aligns with stratified sampling principles commonly used for efficient spatial representation, where multi-stage sampling strategies optimize coverage and variability, e.g., \cite{arunachalam_generalized_2023}, and enables a more comprehensive characterizing of how the structures respond under fire conditions.}
% This selection method prevents fire points from clustering too closely while maintaining an element of randomness. By distributing fire points in this manner, the 30 fire scenarios are effectively utilized, enabling a more comprehensive characterizing of how the structures respond under fire conditions.

\subsection{Summary of the Dataset Generation}
As discussed in this section and related to  \figref{fig:dataset_generation_procedure}, three key steps were considered in the development of the dataset: 
\begin{enumerate}
    \item {\bf{Filtering process}}: Structures with MIDR exceeding $1\%$ under gravity loads were excluded,  resulting in $1,573$ labeled structures retained for fire simulation and $16,050$ unlabeled structures for training the MFSP predictor.
    \item {\bf{Fire simulations}}: For each retained labeled structure, 30 fire scenarios were simulated using OpenSeesRT, yielding $47,190$ fire cases.
    \item {\bf{Data distribution check}}: MIDR distributions for labeled and unlabeled data under gravity loads were highly similar, because both datasets were generated using the same method. Under fire conditions, the MIDR distribution shifted, reflecting significant structural deformation with values reaching a maximum of about 6\%, an average of 1.70\%, and a standard deviation of 1.12\%. This step ensured a diverse and comprehensive dataset for the proposed predictive framework.
\end{enumerate}
The statistical distribution histograms for MIDR (after applying the $1\%$ filtering threshold \revise{for gravity load responses}) under different loading conditions are plotted in \figref{fig:histogram_mdr}. Figures \ref{fig:histogram_mdr}(a) and \ref{fig:histogram_mdr}(b) show the MIDR distributions of the labeled and unlabeled data, respectively, under gravity loads only. \figref{fig:histogram_mdr}(c) shows the MIDR distribution of the labeled data under the combined effects of gravity and fire loads. Fire load causes the structures to significantly deform, leading to a noticeably \revise{right-skewed} MIDR distribution.

\begin{figure*}[h!]
    \centering
    \includegraphics[width=\linewidth]{figures/histogram_mdr.pdf}
    \caption{Histograms of MIDR for labeled and unlabeled structures with gravity loads and fire cases.}
    \label{fig:histogram_mdr}
\end{figure*}

\revise{
This dataset provides the basis for training and testing the performance of the GNN-based framework. Although we employed a simplified rule-based thermal load generation method compared with conventional CFD-based simulations, the temperature field, the changes of the material properties, and the response of the structures, are all still highly nonlinear and complex. Therefore, it is still a challenging task for the NN to predict the MIDRs based on this dataset.
}
To comprehensively evaluate the effectiveness and performance of the proposed TAP-CAM architecture, KNN clustering analysis is conducted under the three most frequently referenced datasets in the UCI Machine Learning Repository, as shown in \autoref{tab:benchmark}. The datasets include Iris, Wine, and Digits, representing a wide range of data types and complexities. In order to achieve a robust evaluation, we have partitioned these datasets into training sets and test sets at an 8:2 ratio to ensure accurate testing and comparison of TAP-CAM model's performance.


\autoref{fig:benchmark}(a) illustrates the effectiveness of the proposed \design architecture across different datasets. 
Among Iris, Wine, and Digits, the \textit{Wine} dataset exhibits the highest susceptibility to hardware device-level variations. This observation emphasizes the importance of robustness in hardware designs, particularly in applications where environmental factors introduce variability. Additionally, we have examined the accuracy performance of KNN search under different TAP-CAM thresholds. 
Interestingly, the results indicate that identifying the nearest neighbor may not always yield the optimal solution. 
For instance, the Iris, Wine, and Digits datasets achieve their respective maximum clustering accuracies at $\textit{K = }2$, $\textit{K = }6$, and $\textit{K = }3$, respectively. 
With the proposed tunable approximate matching scheme, an average 3.06 \% accuracy improvement is observed compared to existing exact-match CAM methods.

Power consumption is obtained via the \textit{Nvidia-smi} toolkit, with the study conducted on \textit{Nvidia 2080ti GPU}, and the \design operations are analyzed via the \textit{Pytorch profiler}. Assuming 256 \design rows, feasible in current manufacturing technology, the KNN clustering benchmark considers different \design wordlengths at the algorithmic level. Idling power is excluded from the results. \autoref{fig:benchmark}(b) illustrates that \design exhibits at least $1.95\times 10^3$ speedup compared to GPU implementation. 
In addition, the energy consumption in \design grows linearly with the number of cells per row, whereas GPU implementations show little increase with dimensionality increment.
Consequently, as dimensionality increases, energy efficiency improvement decreases as demonstrated in \autoref{fig:benchmark}(c). 
For the \textit{Digits} dataset,
\design energy increases with the large number of instances and features,
resulting in an average improvement of $3.15\times$ compared to GPU implementations.

These results illustrate the effectiveness of the proposed TAP-CAM architecture across multiple datasets and scenarios, confirming its feasibility and superiority in practical applications. Through evaluation and comparison with existing methodologies, we highlight the potential of our design to advance CAM technology and contribute to machine learning research and development.
% \vspace{-2ex}
\section{Discussion and Conclusion}
\label{sec:discuss}

We presented \bench, the first framework  and experimental platform to benchmark AI Agents for IT automation tasks. \bench strives to capture the complexity of modern IT systems and the diversity of IT tasks. The reproducibility of \bench ensures the community-driven effort despite inherent nondeterminism of large-scale IT systems. 

One of the key design principles of \bench is ensuring its flexibility to support diverse areas of different IT systems
and its extensibility to new scenarios. While current scope of \bench is comprehensive and representative, we plan to further enrich the benchmark suites by adding other important processes essential to modern IT automation. Furthermore, we plan to expand our benchmarking beyond event-triggered scenarios. 
We are actively working to expand scenario coverage for the supported processes and promote growth through open-community contributions.
 We invite the community to reproduce their real-world-inspired incidents in a synthetic sandboxed environment leveraging the \bench. We expect that everyone contributing can bring their expertise to the table.

We expect \bench to drive the innovations of AI agent-based techniques with a direct impact on the safety, efficiency, and intelligence of today’s IT infrastructures. 
With \bench, we are starting to explore many deep, exciting open problems: How to develop domain-specific AI agents that specialize in certain types of IT tasks? How to orchestrate multiple agents with various expertise to collaborate on bigger projects? How can we ensure safety of agent-driven solutions? How can we effectively use human-in-the-loop while developing diverse adaptive agents? We invite everyone to participate in answering these questions and realizing the vision of using AI agents to automate critical IT tasks.


\section*{Acknowledgements}
This work was supported in part by  NSFC (62104213, 92164203) and SGC Cooperation Project (Grant No. M-0612). 

\bibliographystyle{IEEEtran}
\bibliography{bib}
% that's all folks
\end{document}


