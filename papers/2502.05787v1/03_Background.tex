% \vspace{-1em}
\section{Background}
\label{sec:background}

In this section, we discuss the structure and operational principles of FeFETs, and review existing CAM design works.

\begin{figure}%[H]
    \centering
    \includegraphics[width=1\linewidth]{Figures/1FRIV1.png}
   % \vspace{-0.4cm}
    \caption{\textbf{(a)} FeFET polarization directions and channel conditions after memory write operations;  \textbf{(b)} The FeFET $\textit{I}_\textit{D}$-$\textit{V}_\textit{G}$ characteristics after positive/negative gate write; % Source is grounded; 
    \textbf{(c)} 1FeFET-1R structure and equivalent circuit; \textbf{(d)} The 1FeFET-1R $\textit{I}_\textit{D}$-$\textit{V}_\textit{G}$ characteristics after positive/negative gate write.
    %Source is grounded.
    }
   
 
    \label{fig:fefet}
   %  \vspace{-0.4cm}
\end{figure}

%\vspace{-2ex}
%\vspace{-0cm}
\subsection{FeFET Basics}
\label{sec:device}
\setlength{\abovecaptionskip}{2pt}
\setlength{\belowcaptionskip}{2pt}


Recent advancements in ferroelectric material, particularly hafnium oxide ($\text{HfO}_\text{2}$), have spurred research interest in ferroelectric transistors  and the development of non-volatile circuit designs compatible with CMOS technology \cite{yin2020fecam}. 
FeFETs 
%belong to the subclass of metal-oxide-semiconductor field-effect transistors (MOSFETs) and 
incorporate a ferroelectric 
(FE) layer  within the gate stack. These devices exhibit unique electrical hysteresis characteristics, exhibiting reversible polarization states upon an applied voltage-driven electric field. 
%Integration of a ferroelectric capacitor with the MOSFET gate capacitor confers FeFETs with adjustable hysteresis characteristics. 
The FE layer induces a shift in the threshold voltage of the FeFET depending on the orientation of FE polarization \cite{FeFET-capacitor}, enabling non-volatile (NV) storage capabilities. 
By applying gate voltage pulses, such as -4V/+4V, to a FeFET device, as depicted in \autoref{fig:fefet}(a), it can be programmed to store low and high $\textit{V}_\textit{TH}$ states corresponding to logic ‘0’ and ‘1’, respectively. 
The associated hysteresis  $\textit{I}_\textit{D}$-$\textit{V}_\textit{G}$ transfer characteristics are shown in \autoref{fig:fefet}(b) \cite{transfer-characteristics}. FeFETs, being voltage-driven for read and write operations, exhibit superior energy efficiency compared to two-terminal current-driven NVMs.

%In recent years, with the continuous advancement of ferroelectric material hafnium oxide ($\rm{HfO_2}$),  there has been a growing focus among researchers on ferroelectric transistors and the exploration of non-volatile circuit structures compatible with CMOS technology \cite{yin2020fecam}. Ferroelectric gate field-effect transistors (FeFETs) represent a subclass of metal-oxide-semiconductor field-effect transistors (MOSFETs) that incorporate a ferroelectric layer (FE) within the gate stack, These devices possess distinctive electrical properties capable of reversible polarization under an applied electric field. The integration of a ferroelectric capacitor with the MOSFET gate capacitor grants FeFETs adjustable hysteresis characteristics. The ferroelectric layer introduces a shift in the threshold voltage contingent upon the orientation of ferroelectric polarization \cite{FeFET-capacitor}, resulting in non-volatile (NV) storage capabilities. 
%By applying gate voltage pulses, such as -4V/+4V, to a FeFET device, as illustrated in \autoref{fig:fefet}(a), the device can be programmed to exhibit low $V_{TH}$ and high $V_{TH}$ states corresponding to logic `0' and `1', respectively.
%FeFET have high energy efficiency and low energy consumption, making them a new type of device with great potential for application.
%We define that only two states of information exist in FeFET: logic '0' (high $ V_{TH}$) and logic '1' (low $ V_{TH}$), and logic '0' and logic '1' can be written by applying -4V/+4V pulses as shown in \autoref{fig:fefet}(a). 
 %The corresponding $I_D$-$V_{G}$ transfer characteristics  are depicted in \autoref{fig:fefet}(b) \cite{transfer-characteristics}.
%Given that FeFET read and write operations are voltage-driven, FeFETs demonstrate superior energy efficiency compared to two-terminal current-driven NVMs.

When the FeFET operates as a current source, its ON current gradually increases with the rise in gate voltage, as depicted in \autoref{fig:fefet}(b). Consequently, there's a certain variability in the conduction current regarding the gate read voltage. 
To ensure stable ON current during operation and enhance the design robustness, a current limiter is connected to the source of the FeFET, as shown in the equivalent circuit of \autoref{fig:fefet}(c).
Prior studies \cite{1FeFET1R-transfer, yin2023ultracompact} have shown that a series resistor on the drain/source of a FeFET can regulate the ON current, with 1FeFET-1R integration experimentally demonstrated \cite{area}. Such integration suppresses the ON current variability, making it independent of the $\textit{V}_\textit{TH}$ state and gate voltage when the series resistor is sufficiently large. 
The transfer characteristic curve of the 1FeFET-1R structure is depicted in \autoref{fig:fefet}(d).  
We adopt the 1FeFET-1R structure using a series resistor as a current limiter in this work. 
This approach mitigates the impact of ON current variability on \textit{ML} discharging in a CAM array 
%and reduces high energy consumption, 
achieving low power consumption and robust tunable approximate matching functionality.


%\autoref{fig:fefet}(b) demonstrates that even after entering the saturation region, the ferroelectric transistor behaves as a current source, with the ON current continuing to gradually increase with the rise in gate voltage. Consequently, there exists a certain variability in the conduction current concerning the gate read voltage. Given the requirement for approximate search functionality in the designed circuit, heightened demands are placed on the precision and stability of the current. 
%increases continuously with the increase of gate voltage. 
%After entering the saturation region, there will still be a slight increase in current. 
%Therefore, in order to ensure the stable ON current of the bitcell during operation and enhance the overall robustness of the circuit, a current limiter is connected to the source of the FeFET to constrain the variability of the ON current. In the equivalent circuit, this scenario presents a voltage division between the resistance of the ferroelectric transistor and the current limiter, as depicted in \autoref{fig:fefet}(c). 
%Previous studies \cite{1FeFET1R-transfer, yin2023ultracompact} have demonstrated that a series resistor on the drain/source of FeFETs can govern the ON current of FeFET devices, and such 1FeFET-1R integration has been experimentally verified \cite{area}.
%As a result, the variability in ON current is notably suppressed, and the ON current becomes independent of the $V_{TH}$ state, determined solely by the series resistor when the resistor is sufficiently large. The transfer characteristic curve of the 1FeFET-1R circuit is depicted in \autoref{fig:fefet}(d). 
%, which adds a limiter to control the change of the opening current, that is, a resistor with a larger resistance value is connected in series, and the resistance value of the resistor should be greater than the effective conductance resistance of the FeFET, so that the obtained opening current is not affected by $V_G$ and $V_{TH}$, but only controlled by $V_D$ and $R_S$ in \autoref{fig:fefet}(d).
%In this work, we adopt the 1FeFET-1R structure utilizing a series resistor as a current limiter. This approach not only mitigates the impact of ON current variability on the matchline discharging of a CAM array, but also reduces the high energy consumption caused by excessive ON current, thereby achieving low power consumption and tunable approximate matching functionality.

%This can improve the stability of the current \cite{stability}, and the obtained $ I_D$-$ V_{GS}$ curve is calibrated based on Preisach's FeFET model \cite{transfer-characteristics}.


%It's the voltage and the pulse width applied to gate that determine the memory window in FeFET devices\cite{Ni_2019},(Fig.\ref{FeFET}).
% \begin{figure}[H]
%     \centering
%     \begin{minipage}{0.47\linewidth}
%     \includegraphics[width=\linewidth]{Figures/FeFET.png}
%     \caption*{(a)}
%     \end{minipage}
%     \hspace{.1ex}
%     \begin{minipage}{0.47\linewidth}
%     \includegraphics[width=\linewidth]{Figures/i_v64.eps}
%     \caption*{(b)}
%     \end{minipage}
%     \caption{(a) FeFET polarization changes after applying a voltage pulse at the gate terminal; (b) FeFET I-V  curve we used in SPICE simulation.}
%     \label{FeFET}
    
% \end{figure}


   
%\vspace{-1ex}
\subsection{Existing CAM Designs}
\label{sec:existing_work}

\begin{figure}
    \centering
    \includegraphics[width=\linewidth]{Figures/bg4.png}
  %  \vspace{-0.4cm}
    \caption{Schematics of \textbf{(a)} 16T CMOS TCAM cell; \textbf{(b)} 2T-2ReRAM TCAM cell; \textbf{(c)} 20T-6MTJ TCAM cell; \textbf{(d)} 2FeFET TCAM cell.}
  %  \vspace{-0.4cm}
\label{fig:CAM}
\end{figure}


%\begin{figure}
%    \centering
%    \includegraphics[width=\linewidth]{Figures/conventional_CAM.pdf}
%    \caption{(a)Architecture of an M × N TCAM array.(b)NOR-type CAM bitcell.}
%\label{fig:array}
%\end{figure}


Various CAM designs have been proposed based on CMOS technology and NVM devices. A conventional 16T CMOS TCAM cell is shown in \autoref{fig:CAM}(a). CAMs leveraging NVM typically demonstrate enhanced performance over CMOS-based counterparts. For example, a 2T-2R TCAM design based on ReRAM was proposed in \cite{jing2t2r} for its compact structure, as shown in \autoref{fig:CAM}(b). While it consumes less area compared with conventional CMOS-based CAM designs, %issues arise primarily due to 
the low HRS/LRS ratio, low variable resistance and current-driven write-in mechanism associated with large access transistors  make the write and search energy significant concerns. 
\cite{20T6MTJ} proposed a 20T-6MTJ TCAM design as illustrated in \autoref{fig:CAM}(c), greatly enhancing the search speed and search performance. However, the reduced sense margin caused by the limited TMR ratio of STT-MRAM necessitates numerous transistors to address this issue, thus severely impacting area and power consumption.

Among NVM based CAM designs, utilizing FeFET stands out due to its high ON/OFF current ratio, efficient voltage-driven write mechanisms, low energy consumption, and cost-effectiveness, enabling significant performance improvements compared to conventional CMOS designs and other NVM-based designs. Building upon advanced FeFET models, researchers have proposed various FeFET CAM designs, particularly designs of TCAM. 
The 2FeFET TCAM design as depicted in \autoref{fig:CAM}(d) offers a compact alternative than CMOS counterparts \cite{2FeFET}. 2FeFET TCAM features a smaller cell area, reduced write and search energy consumption, and search delay. 
However, it faces limitations such as the lack of support for approximate matching functionality. 
%Our design will focus on addressing these issues.



%\autoref{fig:CAM}(a) illustrates a 4T-2FeFET cell, which incurs considerable energy and area overheads due to the presence of 4T cells. Furthermore, 
%Here we introduce a 2FeFET-2R CAM cell, depicted in \autoref{fig:CAM}(c), which incorporates the use of a drain resistor, similar to the 1FeFET-1R cell used for compute-in-memory applications \cite{power2}. The 2FeFET-2R cell extends beyond exact match functionality by employing a self-referenced sense amplifier to measure the Hamming distance (HD) between the input query and stored entries. Non-trivial sensing circuitry is required to effectively sample the limited HD.

%Various CAM designs have been proposed based on both CMOS and NVM devices so far. Utilizing NV technology such as FeFET distinguishes itself among various memory technologies due to its high ON/OFF current ratio, efficient voltage-driven write mechanisms, low energy consumption, and cost-effectiveness, thus enabling significant performance improvements compared to conventional CMOS designs. Building upon advanced FeFET models, researchers have proposed various FeFET CAM designs, particularly prevalent in the field of TCAM.  \autoref{fig:CAM}(a) shows a 4T-2FeFET cell, which incurs considerable energy and area overheads due to the presence of 4T cells. Furthermore, 2FeFET cells have been introduced as depicted in \autoref{fig:CAM}(b), which offer a more compact alternative than CMOS counterparts \cite{2FeFET}.
%Here we introduce a 2FeFET-2R CAM cell as shown in \autoref{fig:CAM}(c), which incorporates the use of a drain resistor previously studied in several works, similar to the 1FeFET-1R cell used for compute-in-memory applications \cite{power2}. 
%The 2FeFET-2R cell extends beyond the exact match functionality by employing a self-referenced sense amplifier to measure the Hamming distance (HD) between the input query and stored entries. Non-trivial sensing circuitry is required to effectively sample the limited HD.

\subsection{Threshold Matching Concepts and Related Works}
\label{sec:existing_work}

\begin{figure}
    \centering
    \includegraphics[width=\linewidth]{Figures/threshold_match.png}
  %  \vspace{-0.4cm}
    \caption{\textbf{(a) Exact match:} The stored entry that matches exactly with the query; \textbf{(b) Best match:} The stored entry that has the smallest distance to the query; \textbf{(c) Threshold match:} The stored entry whose distance to the query is below specified thresholds.}
 %   \vspace{-0.4cm}
\label{fig:Matchstyle}
\end{figure}


Most CMOS and NVM based CAM designs discussed earlier prioritize exact matching, as depicted in \autoref{fig:Matchstyle}(a), 
limiting their adaptability for data-intensive applications. 
%For data-centric applications,
In contrast, approximate matching gains favor due to its potential to enhance hardware utilization while maintaining acceptable accuracy. 
As a means to achieve approximate matching, best match CAMs, as illustrated in \autoref{fig:Matchstyle}(b), aim to output the stored entry with the highest similarity to the search query. 
For example, A-HAM \cite{AHAM} evaluates similarities across stored entries and identifies the closest Hamming distance to the input query. 
4T-2MTJ utilizing STT-MRAM \cite{STTMRAM} measures similarity between input query and stored entries in terms of \textit{ML} current and outputs the entry with the highest similarity. \cite{bestmatch} introduced a CAM design for minimum Hamming distance search using digital circuits for bit comparison. A Winner-Take-All (WTA) circuit at the output selects the entry with the highest degree of matching to the search query. However, CAMs designed for best matching may fail in applications requiring the output of multiple entries with specific similarities. Therefore, threshold matching CAMs were devised. 

Threshold matching CAMs, as illustrated in \autoref{fig:Matchstyle}(c), aim to provide multiple stored entries with similarity within a predefined Hamming distance (HD) threshold. 
For instance, the HD-CAM proposed in \cite{conventionalCAM} utilizes a 10T CMOS-based design incorporating \textit{ML} charge redistribution, enabling threshold matching with large HD tolerance, notably used in virus DNA classification. However, the SRAM based HD-CAM cell incurs substantial area and energy overheads. Furthermore, its effectiveness is limited in discerning patterns with substantial HDs due to the intricate tuning of \textit{ML} discharge current, making bit-by-bit tuning of HD thresholds impractical.
\cite{liu2023reconfigurable} introduced MHCAM, a multi-state CAM design encoding multiple CAM cells into distinct multi-states per dimension to perform both dimension-wise exact matching and reconfigurable threshold matching. However, additional transistors introduce fixed bit precisions (1-bit/2-bit/4-bit/8-bit per dimension), restricting fine-grained tunability in threshold matching and adaptability to applications demanding multi-state HD. The ReRAM-based CAM proposed in \cite{MASC} implements threshold matching by leveraging voltage scaling and controlling the precharge period. However, the current-driven mechanisms of ReRAMs result in high power consumption during operation and limited HD thresholds can be achieved due to the large \textit{ML} discharge current and non-trivial threshold-associated period sampling. \cite{2FeFETa} implements approximate matching functionality based on 2FeFET TCAM. It calculates the HD between search and stored vectors in a parallel manner by sensing the discharge rate of \textit{ML}. While achieving high energy efficiency and density in TCAM, it lacks precise control over the degree of approximate searching.

These threshold search CAMs all face a common issue, that they cannot precisely control the degree of approximate matching. Therefore, our design will focus on implementing bit-by-bit tuning of threshold to control the degree of approximate matching.



% To address these challenges, we propose a 2FeFET-2R CAM design, leveraging the advantages of 1FeFET-1R structure for compact, energy-efficient, and flexible bit-by-bit tunable HD threshold matching. We elaborate on our design in subsequent sections.




%Threshold matching CAMs illustrated in \autoref{fig:Matchstyle}(b) aim to provide multiple stored entries with similarity within a predefined Hamming distance (HD) threshold. 
%For instance, the HD-CAM proposed in \cite{conventionalCAM} employs a 10T CMOS-based design incorporating ML charge redistribution, enabling threshold matching with large HD tolerance, notably used in virus DNA classification.
%However, the SRAM based HD tolerant CAM cell incurs substantial area and energy overheads. Furthermore, its effectiveness diminishes in discerning patterns with substantial HDs due to the intricate tuning of ML discharge current, making bit-by-bit tuning of HD thresholds impractical.

%Most of aforementioned CMOS and NVM based CAM designs primarily focus on performing exact match as depicted in \autoref{fig:Matchstyle}(a), which restricts their adaptability to a wide range of emerging applications in the era of big data.
%However, in the context of data-centric applications,  approximate matching is increasingly favored. This approach offers the potential to significantly enhance hardware utilization while maintaining an acceptable level of  accuracy.
%In particular, threshold matching CAMs, illustrated in \autoref{fig:Matchstyle}(b), are designed to provide multiple stored entries that have a similarity within a specified Hamming distance (HD) threshold with the search query.



%An example is the HD tolerant CAM proposed in
%\cite{conventionalCAM}, which employs a 10T CMOS-based design incorporating a ML charge redistribution technique.
%This design implements threshold match with large HD tolerance, and is notably used in virus DNA classification. 
%Nevertheless, it's worth noting that the HD tolerant CAM, while functional, consumes substantial area and energy overheads. Moreover, its effectiveness is limited in discerning patterns  with substantial HDs  due to the intricate tuning of ML discharge current. This renders the bit-by-bit tuning of HD thresholds impractical. 

%The ReRAM-based CAM proposed in \cite{MASC} implements threshold matching by leveraging voltage scaling and controlling the precharge period.
%However, the current-driven mechanisms of ReRAMs result in high power consumption during operation, limiting achievable HD thresholds due to the large ML discharge current and non-trivial threshold-associated period sampling.

%\cite{liu2023reconfigurable} introduced MHCAM, a multi-state CAM design encoding multiple CAM cells into distinct multi-states per dimension to perform both dimension-wise exact matching and reconfigurable threshold matching. However, additional transistors introduce fixed bit precisions (1-bit/2-bit/4-bit/8-bit per dimension), restricting fine-grained tunability in threshold matching, and adaptability to applications demanding multi-state HD.

%\cite{2FeFETa} implements approximate matching functionality based on 2FeFET TCAM. It calculates the Hamming distance between search vectors and storage vectors in a massively parallel manner by sensing the discharge rate of ML. While achieving high energy efficiency and density in TCAM, it lacks precise control over the degree of approximate searching, which is the focus of our design breakthrough.

%The ReRAM-based CAM proposed in \cite{MASC} implements threshold matching by leveraging voltage scaling and controlling the precharge period. However, the current-driven mechanisms of ReRAMs result in high power consumption during operation, and limited HD thresholds can be achieved due to the large ML discharge current and non-trivial threshold associated period sampling.

%\cite{liu2023reconfigurable} proposed MHCAM, a multi-state CAM design that encodes multiple CAM cells into distinct multi-states per dimension to perform both dimension-wise exact matching and reconfigurable threshold matching. However, this approach introduces additional transistors to implement fixed bit precisions (i.e., 1-bit/2-bit/4-bit/8-bit per dimension), limiting its adaptability to specific applications demanding multi-state HD.




%The ReRAM-based CAM proposed in \cite{MASC} implements the threshold matching by leveraging voltage scaling and controlling the precharge period.
%However, the current-driven mechanisms of ReRAMs result in high power consumption during the operation and limited HD thresholds can be achieved due to the large ML discharge current and non-trivial threshold associated period sampling.
%to tolerant a few HD, thus  controls the search mode as either exact or approximate by selectively controlling the precharge period. 
%\cite{liu2023reconfigurable} proposed MHCAM, a multi-state CAM design that encodes multiple CAM cells into distinct multi-states per dimension  to perform both dimension-wise exact matching and reconfigurable threshold matching. However, additional transistors are introduced to implement fixed bit precisions (i.e., 1-bit/2-bit/4-bit/8-bit per dimension). Yet, this approach unavoidably restricts the extent of tunability in threshold matching, limiting its adaptability to specific applications demanding multi-state HD.
 

%The conventional n$\times$m CAM model is shown in the 
%CAM  searches  the input query across the stored entries in parallel, conducting a single input multiple output operation.
%\autoref{fig:array}(a) shows the conceptual schematic of a CAM array, which  performs bit-wise comparison between input query  and stored entries in parallel, conducting single input multiple output operations. 
%Each ML corresponds to the storage content in each row of the CAM, and all MLs are connected to a sense amplifier (SA). 
%Searchlines (SL/$\overline{SL}$) are placed vertically to write stored data and apply input data to the cells within the same column, while wordlines (WLs) are shared horizontally by the cells within a row.
%The storage in each column is connected to a pair of complementary search lines (SL/$\overline{SL}$), which control the writing and searching of rows. 
%Precharge transistors are used to precharge the MLs.
%The sense amplifier (SA) of each word measures the voltage of matchline (ML), which connects all the cells within the word, and is discharged depending on the accumulated bit-wise comparison results.
%and the voltage change of the ML is amplified and characterized by the SA \cite{conventionalCAM}.

%A single CAM unit stores content using a pair of cross-coupled inverters, and the MOS transistor on the row is activated by the writeline (WL) to drive SL and $\overline{SL}$ with complementary voltage values for writing. After the writing is completed, the SLs are precharged for reading and searching. The process and principle of searching are described as follows: first, the ML needs to be precharged to the high level of VDD, and during this process, the SLs need to be controlled at a low level to prevent ML from discharging. Then, turn off the precharge transistor and perform searching on the SL. If the search content on the SL matches the content stored in the unit (SL=D), the gate voltage of $M_{c3}$ is low, and ML cannot discharge, remaining at a high level. If it does not match (SL=$\overline{D}$), $M_{c1}$ and $M_{c2}$ control the gate voltage of $M_{c3}$ to be high, $M_{c3}$ opens, and ML discharges, lowering the voltage level. Therefore, any mismatched bit will be manifested as a decrease in ML voltage after amplification by the SA, resulting in an overall mismatch.

%The most basic content-addressable memory is a 16T CMOS CAM, which operates based on NOR and NAND cells. During the search process of the NAND cell, the transistor responsible for pre-charging needs to be charged to the power supply voltage. When searching for a storage match, all nMOS transistors are turned on, creating a path between the ML and ground, allowing the ML to discharge. If there is a bit miss, the ML will not discharge and remains at a high level. The disadvantage of the NAND matchline is that it depends on the quadratic delay of the number of cells, resulting in large parasitic capacitance and series resistance to ground, and low noise margin, so it is not widely used \cite{16TCMOS}.
%In comparison to the NAND cell, the NOR cell is more commonly used. The search cycle of the NOR cell is divided into three stages: searchline pre-charging, matchline pre-charging, and matchline evaluation. First, disconnect the matchline from ground, pre-charge the searchline to a low level, and charge the matchline to a high level. Then, drive the searchline with the content to be searched to evaluate the ML. The content to be searched is compared with the stored content based on the level change of the ML. The NOR cell has a fast evaluation speed. In the slowest 1-bit miss case, the critical matchline is composed of two series-connected transistors, but the number of transistors is high, resulting in high cost and power consumption.


%\vspace{-1ex}




%Correcting-Match Scheme to achieve soft-error tolerance, but typically only allow for a limited Hamming Distance of 1-4 bits \cite{softerror1}, \cite{softerror2}. 
%In \cite{conventionalCAM}, a dynamic and configurable search method was proposed, where the user can decide the threshold of the allowed mismatch, resulting in the design of a Hamming Distance Tolerant CAM (HD-CAM) whose parameters can be adjusted by the user. The schematic diagram of a bitcell in the HD-CAM is similar to \autoref{fig:accuracy}(a). It adds an evaluation transistor to the NOR-type CAM to control the evaluation voltage (i.e., the threshold voltage for mismatches). When Veval is set to the maximum voltage $V_{DD}$, it can perform a precise search. Alternatively, $V_{eval}$ can be set to other values less than $V_{DD}$ to perform an approximate search.

%However, the HD-CAM design using the 65nm CMOS technology has a large bitcell area and high cost, and consumes a large amount of power during search. It cannot distinguish between cases where the number of mismatched bits is close, which are all areas for future improvements of HD-CAM.


%\vspace{-1ex}
%\subsection{CAM Designs Based on FeFET}

%\label{sec:existing_work}
%To address the cost and power consumption issues, researchers have used new FeFET technology to design 4T-2FeFET TCAM \cite{4T2FeFET} and 2FeFET TCAM structure \cite{2FeFET}. Based on the material's non-volatility, FeFETs have the characteristic of storing and computing in one unit, thereby reducing the number of transistors used for writing and searching in the CMOS process, reducing the design area and cost. 

%First, we consider a 4T-2FeFET design in the context of a multi-domain Preisach model in Sec. \ref{sec:existing_work}(a). The design sets the write and read voltages to be 4V and 1V, respectively, and the write operation is divided into two steps. For example, when writing logic '1' to a TCAM cell, a high-level write voltage is used to drive the writeline (WL), which opens the $T_{3}$ and $T_{4}$ transistors. In the first step, $V_{write}$/0 is used to drive the gates of the $M_{1}$/$M_{2}$ FeFETs, polarizing $M_{1}$ and writing logic '1'. In the second step, 0/-$V_{write}$ is used to drive the gates of the $M_{1}$/$M_{2}$ FeFETs, writing logic '0' to $M_{2}$. During the search process, $V_{search}$ is used to drive the WL and two bitlines, thereby opening the FeFETs, while 0/$V_{search}$ is used to drive the searchline \cite{2FeFET}.

%For a fair comparison, we take 2FeFET design into consideration. The 2FeFET TCAM cell unit is shown in the Sec. \ref{sec:existing_work}(b), where a pair of parallel FeFETs connect the drain to the matchline (ML), the gate is connected to the bitline/searchline (BL/$\overline{SL}$ and $\overline{BL}$/SL), and the source is connected to the sourceline (ScL), driven by the write voltage through ScL.

%Taking writing logic '1' to the 2FeFET TCAM as an example, the specific writing process is described as follows:

%first, writing logic '1' to M1 requires adding $V_{write}$ to the BL/$\overline{SL}$ end, grounding the $\overline{BL}$/SL and ScL ends, ensuring that M1 $V_{GS}$ is high, and the ferroelectric is polarized, successfully writing logic '1' to $M_{1}$.Then, writing logic '0' to $M_{2}$ is similar to the first step, except that ScL is connected to $V_{write}$ to ensure that $M_{2}$ $V_{GS}$ is low, equivalent to writing logic '0' to $M_{2}$.
%The writing process for other content is shown in the table. 
%To perform a search, the ML voltage needs to be precharged and then apply the corresponding search voltage on SL/$\overline{SL}$. Logic '1' corresponds to a high voltage, and logic '0' corresponds to a low voltage. The search content is judged by the voltage change on the ML.

%Compared with conventional CMOS technology and 4T-2FeFET design, the 2FeFET TCAM structure has the advantages of low power consumption, low delay, and small area. However, it also has some limitations: the 2FeFET structure cannot select whether to match at a fixed voltage point, and there is no fixed sensetime to distinguish between cases where the number of mismatched bits is close. 






%In addition to threshold matching CAMs, many CMOS and NVM-based CAM designs incorporate approximate matching functionality through other means. 




%In this paper, we concentrate on CAMs with threshold matching functionality.
%To overcome the aforementioned challenges faced by these threshold matching CAMs, we propose a 2FeFET-2R CAM design, which implements compact, energy-efficient, and flexible bit-by-bit tunable HD threshold matching by exploiting the advantages of 1FeFET-1R structure, i.e., suppressed ON current with negligible variability, single transistor $AND$ logic, voltage-driven write and read mechanisms, etc. We elaborate on our design in the following sections.
%Therefore, considering the approximate search function and the high-energy efficient new devices based on FeFET comprehensively, we propose a 2FeFET-2R structure in Sec. \ref{sec:existing_work}(c), which inherits the advantages of FeFET's low power consumption, low delay, and low cost, while also being able to distinguish between cases where the number of mismatched bits is quite close. Compared with existing TCAM structures, it has richer functions and superior features.

%In addition to the previously discussed threshold matching CAMs, many CMOS and NVM based  CAM designs have also been developed to incorporate  approximate matching capabilities. 
%For instance, the A-HAM proposed in \cite{AHAM} evaluates the similarities across all stored entries and identifies the entry with the closest HD to the input query. 
%The PPAC in \cite{PPAC} calculates HD similarity by counting the number of `1's in  XNOR outputs of all CAM cells within a word. 
%The Hamming distance search CAM proposed in \cite{delayCAM} generates a delayed scoring signal when a bit mismatch occurs, and the Hamming distance is proportional to the time delay. 
%The STT-MRAM proposed in \cite{STTMRAM} measures the similarity between the input query and the stored entries in terms of $ML$ current. \cite{bestmatch} introduces a CAM design for minimum Hamming distance search, which utilizes digital circuits for bit comparison. Additionally, a Winner-Take-All (WTA) circuit is integrated at the end of the output voltage. It can be used to output the entry with the highest degree of matching to the search query, as depicted in \autoref{fig:Matchstyle}(c).
%However, these approximate matching CAMs are facing challenges such as high power consumption or constrained parallelism. 
%In this paper, we concentrate on the CAMs with threshold matching functionality.
%In addition, for the ReRAM-based MASC, the tolerance level for approximate Hamming distance search is low. Other structures use a digital search method, which results in low parallelism and so on.
