





\begin{abstract}

Pattern search is crucial in numerous analytic applications for retrieving data entries akin to the query. Content Addressable Memories (CAMs), an in-memory computing fabric, directly compare input queries with stored entries through embedded comparison logic, facilitating fast parallel pattern search in memory.
While conventional CAM designs offer exact match functionality, they are inadequate for meeting the approximate search needs of emerging data-intensive applications. 
Some recent CAM designs propose approximate matching functions, but they face limitations such as excessively large cell area or the inability to precisely control the degree of approximation. 
In this paper, we propose TAP-CAM, a novel ferroelectric field effect transistor (FeFET) based ternary CAM (TCAM) capable of both exact and tunable approximate matching. 
TAP-CAM employs a compact 2FeFET-2R cell structure as the entry storage unit, %for basic storage and computing functions at the unit circuit level, enabling a dense CAM array to enhance energy efficiency. 
and similarities in Hamming distances between input queries and stored entries are measured using an evaluation transistor associated with the matchline of CAM array. 
%Extensive Monte Carlo simulations assess the impact of FeFET device variation. 
The operation, robustness and performance of the proposed design at array level have been discussed and evaluated, respectively. 
We conduct a case study of K-nearest neighbor (KNN) search to benchmark the proposed TAP-CAM at application level.
%Results demonstrate that TAP-CAM achieves a 6.78× energy improvement compared to 2FeFET CAM implementing approximate match functionality, and 16.95× compared to 16T CMOS CAM implementing exact match functionality and 2FeFET CAM implementing approximate match functionality, along with a 3.06\% accuracy enhancement. 
Results demonstrate that compared to 16T CMOS CAM with exact match functionality, TAP-CAM achieves a 16.95$\times$ energy improvement, along with a 3.06\% accuracy enhancement. Compared to 2FeFET TCAM with approximate match functionality, TAP-CAM achieves a 6.78$\times$ energy improvement.


%Pattern search is a key operation in many analytic applications that retrieve the data entry similar to the query. Content Addressable Memories (CAMs), as a type of in-memory computing fabric, can directly compare input queries with all stored entries in the memory through embedded comparison logic to determine whether the input query matches a stored entry, thereby supporting fast parallel associative search. 
%Therefore, CAMs can provide high-performance and efficient hardware solutions in various applications such as data processing and high-speed searching. 
%However, the conventional CAM designs only provide exact match function which outputs the entry exactly matching with the input, thereby cannot satisfy the growing amount of approximate search requirements for emerging data-intensive applications.
%Recently, some CAM designs have been proposed to support approximate match function, but they are limited by either large  cell area overhead, lacking pattern masking capability, or only designed for specific applications. 
%In this paper, we propose TAP-CAM, a novel FeFET based ternary CAM (TCAM), that supports both exact and tunable threshold matching. TAP-CAM utilizes a 2FeFET-2R structure to achieve a dense CAM array based to enhance energy efficiency. The similarities in terms of Hamming distances between the query and stored entries are calculated by an evaluation transistor. Extensive Monte Carlo simulation is conducted to evaluate the impact of the device-to-device variation of FeFET. 
%We use K-nearest neighbor(kNN) search as a case study to benchmark the application-level improvement of TAP-CAM. 
%Results show that TAP-CAM achieves 16.95$\times$ energy improvement and 3.06\% accuracy improvement compared with CMOS-based CAM implementing the exact match function. 


%Advanced machine learning models such as hyperdimensional computing (HDC) and binary neural networks (BNNs) have been extensively studied for brain-inspired cognitive tasks. During their computations, cosine similarity has shown its great importance w9241990140832151214499here intensive inferences are performed based on the angles between the binary query vector and binary stored feature vectors.


%Cosine similarity measures the similarity between two vectors in an inner product space. It is widely used in a number of machine learning models such as hyperdimensional computing (HDC) and deep neural networks. More specifically, during the inference phase of these machine learning applications, a large number of cosine similarity-based searches (CSSs) are often needed.
%Specifically, for the prevalent binary neural networks (BNN) and hyperdimensional computing (HDC) models, cosine similarity plays a critical role in 
%\textcolor{red}{move the first paragraph to Intro}
%Cosine similarity measures the similarity between two vectors in an inner product space, and  has been widely used in a number of machine learning models, 
% Content addressable memory (CAM) has  been widely utilized as associative memory for data-intensive workloads, thanks to its  parallel in-memory pattern-matching capability. However, traditional CAM designs struggle to maintain their energy efficiency and performance advantages due to the limited number of exact matches between the stored entries and query patterns, especially considering the ever-growing amount of data. 
% %As such, efficient CAM designs and optimizations for approximate matching capability are highly desired.
% To address this challenge,
% we propose TAP-CAM, a novel tunable approximate matching engine based on a ferroelectric field effect transistor (FeFET) CAM.
% The FeFET CAM cell employs a low-power 1FeFET1R structure that integrates a series resistor current limiter into the intrinsic FeFET structure. 
% An evaluation transistor connected to the NOR-type matchline (ML) of the  CAM array  controls the ML discharge rate,  enabling TAP-CAM to perform approximate search operations. 
% By computing the Hamming distances (HDs) between the stored pattern entries and the input query,  TAP-CAM generates a match output for entries with an HD below a tunable matching threshold.
% We thoroghly analyze and validate the scalability and robustness of the proposed TAP-CAM array.
% Moreover, we evaluate the performance of TAP-CAM, which demonstrates a remarkable $16.95\times$ improvement in terms of energy efficiency compared to its CMOS based counterpart. 
% When TAP-CAM is deployed in the K-nearest neighbor (KNN) model for classification tasks, benchmarking results reveal significant improvements in  energy efficiency and delay (AA and BB, respectively). Furthermore, the application accuracy indicates an average improvement of 3.06 \%  compared to the existing exact match CAM, highlighting the value and effectiveness of our proposed \design design.
%for energy-efficient in-memory approximate matching applications, playing a significant role in artificial intelligence hardware acceleration and epidemic virus gene identification. Unlike existing CMOS-based structures, we designed a 2FeFET-2R RAM based on the novel FeFET device. The intrinsic polarization memory function of the FeFET greatly simplifies the circuit structure, reducing area and energy consumption overhead. In this paper, the 2FeFET-2R RAM can distinguishes adjacent mismatched bits by regulating the gate voltage of the evaluation transistor and comparing the voltage on the match line, achieving differentiation of 0-6 mismatch situations in a 64-bit long array. Through extensive Monte Carlo simulations, we analyzed the robustness of the circuit under different processes. Meanwhile, we evaluate the performance of the ferroelectric TCAM approach in a single search operation. Compared to the traditional CMOS-based structure, the energy efficiency has improved by 20 times, and the circuit area has been reduced by 6 times.
% \begin{IEEEkeywords}
% In-memory Computing, Cosine Similarity, FeFET, Associative Memory (AM)
% \end{IEEEkeywords}
\end{abstract}
