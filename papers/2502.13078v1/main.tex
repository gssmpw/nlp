\documentclass[10pt,twocolumn,letterpaper]{article}

\usepackage[pagenumbers]{cvpr}

\usepackage[dvipsnames]{xcolor}
\usepackage[normalem]{ulem}
\usepackage{lipsum}
\usepackage{multirow}
\usepackage[skip=1 pt]{caption} 
\usepackage{tikz}
\usepackage{wrapfig}
\usetikzlibrary{positioning,calc}

\definecolor{cvprblue}{rgb}{0.21,0.49,0.74}
\usepackage[pagebackref,breaklinks,colorlinks,allcolors=cvprblue]{hyperref}

\definecolor{darkgreen}{RGB}{30,150,30}
\definecolor{darkblue}{RGB}{0,0,127}
\definecolor{darkyellow}{RGB}{171,133,0}
\definecolor{darkred}{RGB}{180,20,20}
\definecolor{darkmagenta}{RGB}{127,0,127}
\definecolor{darkcyan}{RGB}{0,127,127}
\definecolor{chromeyellow}{rgb}{1.0, 0.65, 0.0}
\definecolor{amber}{rgb}{1.0, 0.75, 0.0}

\newcommand{\methodName}{L4P\xspace}

\title{L4P: \uline{L}ow-Level \uline{4}D Vision \uline{P}erception Unified}

\newcommand{\mychar}{
  \begingroup\normalfont\centering
  \includegraphics[width=0.12\textwidth]{figures/nvidialogo.png}
  \endgroup
}

\author{
  Abhishek Badki$^*$ \quad  Hang Su$^*$ \quad Bowen Wen \quad Orazio Gallo \\ \vspace{-0.4em} \\
  \mychar{} \\
  \small\url{https://research.nvidia.com/labs/lpr/l4p}\\
}

\begin{document}

\maketitle

\newcommand\blfootnote[1]{
  \begingroup
  \renewcommand\thefootnote{}\footnote{#1}
  \addtocounter{footnote}{-1}
  \endgroup
}
\blfootnote{\hspace{-1em} $^*$ indicates equal contribution.}


\begin{abstract}
Retrieval-Augmented Generation (RAG) is often used with Large Language Models (LLMs) to infuse domain knowledge or user-specific information. In RAG, given a user query, a retriever extracts chunks of relevant text from a knowledge base. These chunks are sent to an LLM as part of the input prompt. Typically, any given chunk is repeatedly retrieved across user questions. However, currently, for every question, attention-layers in LLMs fully compute the key values (KVs) repeatedly for the input chunks, as state-of-the-art methods cannot reuse KV-caches when chunks appear at arbitrary locations with arbitrary contexts. Naive reuse leads to output quality degradation.  This leads to potentially redundant computations on expensive GPUs and increases latency. In this work, we propose \sys, a system for managing and reusing precomputed KVs corresponding to the text chunks (we call \textit{chunk-caches}) in RAG-based systems. We present how to identify \hl{\textit{chunk-caches} that are reusable}, how to efficiently perform a small fraction of recomputation to \textit{fix} the cache to maintain output quality, and how to efficiently store and evict \textit{chunk-caches} in the hardware for maximizing reuse while masking any overheads. With real production workloads as well as synthetic datasets, we show that \sys reduces redundant computation by \textbf{51\%} over SOTA prefix-caching and \textbf{75\%} over full recomputation.
\hl{Additionally, with continuous batching on a real production workload, we get a \textbf{1.6$\times$} speedup in throughput and a \textbf{2$\times$} reduction in end-to-end response latency over prefix-caching while maintaining quality, for both the \llama-3-8B and \llama-3-70B models. 
}
\end{abstract}




    
\section{Introduction}\label{sec:intro}
\vspace{-0.5cm} 
\section{Introduction}
% Event cameras are innovative bio-inspired sensors.
% Unlike traditional frame cameras, Event cameras do not operate at a fixed rate but asynchronously report pixel-wise intensity changes, known as events (\fig \ref{relatedwork}a). 
% With microsecond level resolution and an asynchronous, differential operating principle, event cameras excel at capturing high-speed motions that cause severe motion blur in frame cameras. 
% Additionally, Event cameras have a very high dynamic range (HDR) of 140dB compared to 60dB in frame cameras, performing well under varied illumination conditions. 
% Consequently, event cameras are considered an important sensing modality and are increasingly used for tasks like motion tracking and Simultaneous Localization and Mapping (SLAM).


% Event cameras are innovative bio-inspired sensors that report pixel-wise intensity changes as events asynchronously with microsecond sensing latency (\fig \ref{intro}a), rather than fixed interval frames \tocite. sensing latency is the time from a visual stimulus appearing to its sensor readout.
% Event cameras are innovative bio-inspired sensors that asynchronously report pixel brightness changes as events with \textit{millisecond latency} (\fig \ref{intro}a), instead of fixed interval frames \cite{he2024microsaccade, gehrig2024low}.  
% % sensing latency is the time from visual stimulus to sensor readout \cite{gehrig2024low}.
% With high temporal resolution and a high dynamic range, event cameras excel at capturing high-speed motions without blurring and perform well under varied illumination conditions \cite{falanga2020dynamic, xu2023taming}.
% Thus, event cameras are envisioned as an ideal solution for challenging 2D vision tasks, such as low latency and accurate object detection in \fig \ref{intro}b \cite{gallego2020event}.

Event cameras are innovative bio-inspired sensors that report changes in pixel brightness asynchronously as events with \textit{millisecond latency} (\fig \ref{intro}a), rather than at fixed-time
intervals \cite{he2024microsaccade, gehrig2024low}.  
% sensing latency is the time from visual stimulus to sensor readout \cite{gehrig2024low}.
With high temporal resolution and a high dynamic range, event cameras excel at capturing high-speed motions without blurring and perform well under varied illumination conditions \cite{falanga2020dynamic, xu2023taming}.
Thus, event cameras are envisioned as an ideal solution for 2D vision tasks, such as low latency and accurate object detection as shown in \fig \ref{intro}b \cite{gallego2020event}.

% Similar to frame cameras, event cameras encounter scale uncertainty (\aka, they struggle to accurately estimate object depth) \tocite.
% This challenge hinders event cameras from fully realizing their potential in 3D object localization \tocite. 
% Similar to frame cameras, event cameras encounter scale uncertainty (\aka, they struggle to accurately estimate object depth) \tocite.
% This challenge hinders event cameras from fully realizing their potential in 3D object localization \tocite. 
% 尽管事件相机在上述 2D vision tasks取得了不错的表现,
% However, event cameras struggle to fully realize their potential in low-latency 3D object localization, which has various potential applications (\eg, drone localization,入侵物体定位等)
% because they encounter scale uncertainty (\aka, they struggle to accurately estimate depth) \cite{zhang2022mobidepth}. 
% Although event cameras perform well in the aforementioned 2D vision tasks, they struggle to fully realize their potential in low-latency 3D object localization, which 对事件相机视野中出现的物体进行三维定位, has various potential applications (\eg, drone localization, intruding object detection, AR/MR) due to scale uncertainty (\aka, difficulty in accurately estimating depth) \cite{zhang2022mobidepth}.
% Although event cameras excel in the 2D vision tasks, they struggle to fully realize their potential in 3D vision tasks 以 low-latency 3D object localization为代表, which involves localizing the object within the camera's field of view in three dimensions (\fig \ref{intro}c). 
% Although event cameras excel in aforementioned 2D vision tasks, they struggle to fully realize their potential in 3D vision tasks, particularly in low-latency 3D object localization, which involves localizing objects within the camera's field of view in three dimensions (\fig \ref{intro}c) \cite{qin2019monogrnet}.
% The latency measures the time elapsed from visual stimulus to resulting localization output.
% This limitation, due to scale uncertainty (\ie, difficulty in accurately estimating depth) \cite{zhang2022mobidepth}, affects various potential vital applications of event cameras (\eg, drone localization \cite{wang2022micnest}, intruding object detection\cite{han2015twins}).

% Although event cameras excel in 2D vision tasks, they face fundamental challenges in 3D vision, preventing their full potential from being realized.
% Specifically, 3D object localization, identifying the location of objects within the camera's field of view in three dimensions, is a fundamental function for various vital 3D vision tasks (\eg, drone localization \cite{wang2022micnest}, AR/MR \cite{xu2021followupar}).
% However, event cameras, capturing per-pixel brightness changes in 2D without depth details, can't directly gauge object distance, causing scale uncertainty. 
% This limitation restricts event cameras in 3D object localization , hindering the exploitation of their low-latency advantage in 3D vision tasks.
% % To address this, 
% Two types of solutions are proposed:

However, event cameras face significant challenges when applied to more complex 3D vision tasks.
% which prevents their full potential from being realized. 
For instance, 3D object localization, which identifies the location of objects within the camera's field of view in three dimensions, is a fundamental block for various vital 3D vision tasks (\eg, drone navigation \cite{wang2022micnest}, augmented/mixed reality \cite{xu2021followupar}).
Event cameras only capture per-pixel brightness changes in 2D devoid of depth details, resulting in scale uncertainty that hinders their effectiveness in 3D object localization (\fig \ref{intro}c).
This limitation further restricts the exploitation of their potential in various 3D vision tasks.
To address this, two primary types of solutions are proposed to enhance event cameras:

% However, event cameras can only capture 2D images and lack depth information, making it impossible to directly measure the actual distance of the object. 
% This leads to scale uncertainty, preventing event cameras from performing 3D object localization (\fig \ref{intro}c).
% This prevents 3D object localization from leveraging the low-latency advantage of event cameras and hinders their use in various vital 3D vision applications.
% Two type solutions are proposed to augment event cameras:

% However, similar to frame cameras, event cameras face scale uncertainty (\aka, they cannot accurately estimate the depth of objects) \tocite.
% This is a fundamental challenge that prevents event cameras from fully realizing their potential in 3D object localization and tracking \tocite. 
% There are mainly two types of solutions proposed to address this issue, supplementing event cameras with depth information of objects:

\noindent $\bullet$ \textbf{Events only-based solutions.}
These methods rely solely on event data for object depth estimation and fall into two types. 
$(i)$ Incorporating known geometric information with observations to deduce depth. 
These methods rely heavily on prior knowledge, leading to poor performance in new scenes or with new objects \cite{falanga2020dynamic}.
$(ii)$ Employing machine learning algorithms that either stick events within a time window (\eg, $1ms$) into an image for DNN-based estimation \cite{guo2022low}, or devise event-oriented networks (\eg, SNN) for object localization \cite{zhou2023computational, barchid2023spiking}. 
These methods are computationally intensive during network inference \cite{diehl2015unsupervised, guo2021toward}, potentially entailing significant latency (\eg, tens to hundreds of milliseconds) in practice.

% However, they often entail significant delays (\eg, tens to hundreds of milliseconds) for network inference \cite{diehl2015unsupervised, guo2021toward}, negating low-latency benefits of event cameras.

% CNNs struggle to process event data directly due to its asynchronous nature \tocite. 
% Current practices

% These methods use only event data for object depth estimation, which can be categorized into two types.
% One type is machine learning algorithms. 
% Convolutional neural networks (CNNs) cannot directly process event data due to its asynchronous nature \tocite. 
% Current methods either stick events within a short time window (\eg, $<1ms$) into an image for CNN-based depth estimation \tocite or design event-oriented networks (\eg, spiking neural networks) for object localization \tocite. 
% However, these methods often require significant delays (\eg, tens to hundreds of milliseconds) for inference \cite{diehl2015unsupervised, guo2021toward}, negating the low-latency benefits of event cameras\tocite.
% The other type of methods incorporates known geometric information of the target object combined with observational data to infer depth, which heavily rely on prior knowledge, resulting in poor performance in new scenes and with unfamiliar objects.

\noindent $\bullet$ \textbf{Fusion-based solutions.}
These methods enhance event cameras for 3D object localization through sensor fusion, categorized into two types.
$(i)$ Involving dual event cameras \cite{zhou2021event, xu2023taming}. 
These methods often require meticulous calibration and feature matching between event cameras, which are time-consuming and sensitive to environmental noise \cite{falanga2020dynamic}.
$(ii)$ Introducing dedicated depth estimation sensors (\eg, depth cameras \cite{he2021fast}, LiDAR \cite{cui2022dense}) to provide event cameras with depth information \cite{li2022motion}. 
% However, these sensors typically operate at 10$Hz$ $\sim$ 30$Hz$ \tocite, requiring downsampling event cameras to synchronize, which nullify the low-latency benefits of event cameras \tocite.
However, these sensors typically operate with latencies ranging from 30$ms$ to 100$ms$ \cite{li2023leovr}, necessitating the downsampling of event data in the temporal domain for synchronization.
% requiring downsampling event cameras to synchronize, which nullify the low-latency benefits of event cameras.

\noindent \textbf{Remark.}
% In summary, current methods entail lengthy processing times or necessitate downsampling event cameras for synchronization with other sensors, presenting significant challenges in fully harnessing the potential of event cameras for low-latency 3D object localization.
% Inappropriate sensor choice for fusion and the absence of suitable algorithms negate the low-latency advantages of event cameras, posing challenges in fully leveraging their potential for low-latency 3D object localization.
In summary, the absence of efficient algorithms and the sensor with matched frequencies for depth estimation introduces substantial delays in 3D object localization. 
This limitation prevents the complete exploitation of the low-latency benefits of event cameras.

% "In summary, the lack of efficient algorithms and appropriately synchronized sensors for depth estimation causes significant delays in 3D object localization. This hurdle hinders the complete exploitation of the low-latency advantages offered by event cameras."
% event cameras’ potential.

% \noindent $\bullet$ \textbf{Dedicated depth sensors-based solution.}
% By introducing dedicated depth estimation sensors (\eg, depth cameras and LiDAR), these solutions provide event cameras with depth information of objects. 
% Specifically, these sensors emit light in a specific spectrum and calculate depth based on reflection time.
% However, they typically operate at frequencies of 10Hz $\sim$ 30Hz, much lower than the sampling frequency of event cameras, degrading localization performance.

% \noindent $\bullet$ \textbf{Learning-based solution.}
% Machine learning algorithms, such as convolutional neural networks (CNNs), cannot directly process event camera data because it consists of asynchronous events, not fixed-rate frames. 
% Current practices either $(i)$ stick all events within a short time window (\eg, $<1ms$) into an image for CNN-based depth estimation, or $(ii)$ design event-oriented networks (\eg, spiking neural networks) for object localization. 
% They rely heavily on labeled training data, leading to poor performance with new scenes and objects. Also, they introduce significant delays, negating the low-latency benefits of event cameras.

% \noindent $\bullet$ \textbf{Motion-based solution.}
% These methods combine information from inertial measurement units (IMUs) and use visual-inertial odometry to estimate 3D location of the target object. 
% Although they do not rely on dedicated sensors or training data, they require camera movement while the object remains stationary, which severely limits usage scenarios. 
% Additionally, current practices involve using dual event cameras with known pose relationships for 3D object localization. 
% However, these methods often require meticulous calibration and feature matching between cameras, which are highly sensitive to unexpected noise in the environment.

\begin{figure}[t]
    \setlength{\abovecaptionskip}{0.25cm} % height above Figure X caption
    \setlength{\belowcaptionskip}{-0.3cm}
    \setlength{\subfigcapskip}{-0.25cm}
    \centering
        \includegraphics[width=0.98\columnwidth]{Figs/intro_new.png}
        \vspace{-0.2cm}
    \caption{Illustration of events generation and applications of event cameras.}
    \label{intro}
    \vspace{-0.3cm}
\end{figure} 

\begin{figure*}[t]
    \setlength{\abovecaptionskip}{0.2cm} % height above Figure X caption
    \setlength{\belowcaptionskip}{-0.3cm}
    \setlength{\subfigcapskip}{-0.25cm}
    \centering
        \includegraphics[width=2\columnwidth]{evaFigs/relatedall_2.png}
        \vspace{-0.2cm}
    \caption{Benchmark study on drone localization and performance of existing solutions at different settings.}
    \label{relatedwork}
    \vspace{-0.2cm}
\end{figure*} 

\noindent \textbf{Enhance event camera with mmWave radar.}
% MmWave radar, utilizing frequency-modulated continuous waves (FM-CW) with microsecond level latency, measures relative angle and distance of moving objects, generating sparse point cloud \cite{woodford2023metasight, zheng2023neuroradar}. 
% with microsecond level latency
% MmWave radar, utilizing frequency-modulated continuous waves (FM-CW), has been widely employed to measure the relative angle and distance of moving objects, resulting in a sparse point cloud \cite{woodford2023metasight, zheng2023neuroradar}.
% The mmWave radar, utilizing frequency-modulated continuous waves (FM-CW), has been widely employed in detection and tracking of moving objects, resulting in a sparse point cloud \cite{woodford2023metasight, zheng2023neuroradar}.
% Inspired by achievements of mmWave-based sensing techniques, we observe that both event camera and mmWave radar share microsecond time resolution, making mmWave radar a promising modality to enhance the event camera in 3D object localization.
% This presents a significant opportunity for event-based accurate and low-latency localization.
mmWave signals, operating at high frequencies (30 $\sim$ 300 GHz) with wide bandwidth, offer high sensing sensitivity and precision \cite{fiandrino2019scaling, zhang2023survey}.
Endowed with fine-grained, directional sensing capability, and resistance to weather and illumination conditions, mmWave sensing has great advantages in object depth estimation \cite{sie2023batmobility, iizuka2023millisign, lu2020see, lu2020milliego}.
More importantly, both event cameras and mmWave radar feature \textit{millisecond latency} \cite{mmWaveUser}. These factors make mmWave a promising enhancement for event cameras in low-latency 3D object localization.
% Meanwhile, this fusion also holds potential in solving the issues of limited spatial resolution and scatter center drift faced by mmWave radar.

% mmWave signals, operating at high frequencies (30-300 GHz) with wide bandwidth, offer high sensing sensitivity. With fine-grained, directional sensing capability, mmWave sensing excels in object depth estimation. Both event cameras and mmWave radar share millisecond latency, making mmWave a promising enhancement for event cameras in low-latency 3D object localization. Additionally, this fusion can address the issues of limited spatial resolution and scatter center drift faced by mmWave radar.

% and resistance to weather and illumination conditions, 
% More importantly,尽管 mmWave 面临limited angular resolution和scatter center drift问题, both event cameras and mmWave radar share \textit{millisecond latency} \cite{mmWaveUser}, making mmWave a promising enhancement for event cameras in low-latency 3D object localization.
% Despite the issues of limited spatial resolution and scatter center drift faced by mmWave, both event cameras and mmWave radar share \textit{millisecond latency} \cite{mmWaveUser}. 
% This makes mmWave a promising enhancement for event cameras in low-latency 3D object localization, while the event camera also holds potential in solving mmWave radar issues.

% To better understand the potential of fusing the event camera and mmWave radar for low-latency and accurate localization, we conduct a benchmark study on landing drone localization at a real-world drone delivery airport (\fig \ref{relatedwork}a), as accurate and low-latency localization is essential for effective drone landing \cite{sun2023indoor}. 
% This is because landing is a critical phase where drones are most vulnerable \cite{wang2022micnest, xu2023taming, floreano2015science}, posing financial risks and safety threats \cite{Russiandrone}. 
% Higher accuracy improves landing success on designated platforms, while lower latency allows more reaction time to unexpected situations \cite{famili2022pilot, he2023acoustic, chi2022wi}.

To explore the potential of fusing the event camera and mmWave radar for improved 3D object localization, we conduct a benchmark study on drone localization during landing phase at a real-world drone delivery airport (\fig \ref{relatedwork}a). 
Accurate and low-latency localization is crucial for effective landing of the drone \cite{wang2022micnest, sun2023indoor}, as in this phase the drone is most vulnerable, posing financial risks and safety threats \cite{floreano2015science, Russiandrone}. 
Enhanced accuracy ensures successful landing on designated platforms, while reduced latency provides more reaction time for unexpected situations \cite{famili2022pilot, he2023acoustic, chi2022wi}.
Our benchmark study reveals that existing methods face fundamental challenges in 3D object localization, as elaborated below:

% \noindent $\bullet$ \textbf{C1: Millisecond sensing latency amplifies sensing noise, impairing drone detection.}
% \noindent $\bullet$ \textbf{C1: Differing noise distribution characteristics of both modalities hinder drone detection.}
% Unexpected environmental changes introduce irrelevant information as noise in sensing results \cite{xu2023taming}.
% Although both sensors have matched sensing latency, their noise distribution characteristics differ significantly due to their different mechanisms, hindering system's ability to identify signals changes caused by drone in both modalities \cite{zuo2024cross}, especially in millisecond latency (\fig \ref{relatedwork}b).
% However, traditional single modality-oriented noise filtering algorithms \cite{cao2024virteach, liu2024pmtrack, wang2021asynchronous, alzugaray2018asynchronous} achieve a low event and point cloud filtering rate (recall and precision < 65\% in \fig \ref{relatedwork}c) due to their rule-based pipelines struggle to distinguish drone-induced signal changes from scene dynamics.
% This results in detection precision bottlenecks, significantly diminishing the efficiency and accuracy of localization.

% 事件相机容易由于什么产生噪声,雷达容易由于什么产生噪声。对于同一个目标,这两种不同的传感器不仅产生异构的target-trigger的信息,也产生了空间上不同分布(dimentions,patterns)的噪声,而且这些噪声时间上也可能不同步,特别是在高时间分辨率的情况下。不幸的是,传统的方法要不就是针对单模态的滤波,要不就是对两个相似的信号进行滤波,不能应用到我们这个异构的高频场景下。
% Unexpected environmental changes introduce irrelevant information as noise in sensing results \cite{xu2023taming}.
% \noindent $\bullet$ \textbf{C1: Differing noise distribution characteristics of both modalities hinder drone detection.}
% Both sensor modalities yield not only heterogeneous information about the drone but also generate significantly heterogeneous noise. 
% Event cameras produce noise due to unexpected changes in brightness conditions, while mmWave radar struggles with noise caused by signal multipath effects.
% This noise differs greatly in dimensions and patterns, and it may lacks temporal synchronization, particularly under millisecond latency (\fig \ref{relatedwork}b). 
% These factors make noise filtering challenging, causing detection bottlenecks and reducing localization efficiency and accuracy \cite{xu2023taming}.
% Traditional noise filtering algorithms \cite{cao2024virteach, liu2024pmtrack, wang2021asynchronous, alzugaray2018asynchronous} target a specific modality, resulting in low noise event and point cloud filtering rates (recall and precision < 65\% in \fig \ref{relatedwork}c), limiting their effectiveness in our scenario.

% 一句背景,一句现象,一句结果,一句实验数据。
\noindent $\bullet$ \textbf{C1: Noise distribution characteristics of both modalities differ, hindering drone detection.}
% Both sensor modalities yield not only heterogeneous information about the drone but also generate significantly heterogeneous noise. 
These two sensor modalities not only provide different types of information but also generate significantly heterogeneous noise. 
Event cameras produce noise due to unexpected changes in brightness conditions, whereas mmWave radar struggles with noise caused by signal multipath effects.
These noises differ greatly in both dimension and pattern, which can also be asynchronous, especially under high temporal resolution (\fig \ref{relatedwork}b).
This spatial and temporal heterogeneity complicates noise filtering, causing detection bottlenecks \cite{xu2023taming}.
Unfortunately, existing traditional noise filtering algorithms \cite{cao2024virteach, liu2024pmtrack, wang2021asynchronous, alzugaray2018asynchronous} typically target a single modality, resulting in low noise event and point cloud filtering rates (recall and precision < 65\% in \fig \ref{relatedwork}c), limiting their effectiveness in our scenario.


% However, traditional noise filtering algorithms \cite{cao2024virteach, liu2024pmtrack, wang2021asynchronous, alzugaray2018asynchronous} are solely targeted at a specific modality and fail to exploit the consistency among different modalities, achieving a low event and point cloud filtering rate (recall and precision < 65\% in \fig \ref{relatedwork}c), which cannot be utilized in effective noise filtering in our scenario.
% This results in detection precision bottlenecks, significantly diminishing the efficiency and accuracy of localization.
% Although both sensors have matched sensing latency, their noise distribution characteristics differ significantly due to their different mechanisms, hindering system's ability to identify signals changes caused by drone in both modalities \cite{zuo2024cross}, especially in millisecond latency (\fig \ref{relatedwork}b).
% However, traditional single modality-oriented noise filtering algorithms \cite{cao2024virteach, liu2024pmtrack, wang2021asynchronous, alzugaray2018asynchronous} achieve a low event and point cloud filtering rate (recall and precision < 65\% in \fig \ref{relatedwork}c) due to their rule-based pipelines struggle to distinguish drone-induced signal changes from scene dynamics.

% \noindent $\bullet$ \textbf{C2: Ultra-large amount data burden the heterogeneous data fusion, delaying drone localization.}
% Once the drone is detected, accurate 3D spatial location estimation of it is essential, which is more time-consuming than detection due to additional processing (\eg, sensor fusion and optimization).
% The ultra-large amount of data generated by the millisecond latency further burdens the time consumption . 
% Although the localization accuracy is boosted, existing methods \cite{zhao20213d, falanga2020dynamic, mitrokhin2018event} introduces significant delays (\fig \ref{relatedwork}d).
% Moreover, asynchronous event streams and sparse point clouds from mmWave radar are heterogeneous in terms of precision, scale, and density. 
% Previous fusion methods (\eg, Extended kalman filter, particle filter, and graph optimization \cite{grisetti2010tutorial} in \fig \ref{relatedwork}d) suffer from severe cumulative drift error and lengthy processing latency, rendering them inadequate for accurate and low-latency localization.

\noindent $\bullet$ \textbf{C2: Ultra-large data volume burdens the heterogeneous data fusion, delaying drone localization.}
Accurately estimating 3D location of the drone after detection involves time-consuming processing steps, such as sensor fusion and optimization. 
% Once the drone is detected, we proceed to perform 3D localization on it.
% Accurately estimating 3D spatial location of drone involves several time-consuming processing steps, including the sensor fusion and optimization.
The ultra-large amount of data due to the high frequency further exacerbates the processing time, causing significant delays \cite{xu2021followupar}.
Meanwhile, the asynchronous event streams and sparse point clouds are heterogeneous in terms of precision, scale, and density, adding complexity to the sensor fusion.
Existing methods (\eg, Extended Kalman filter, particle filter, and graph optimization) suffer from cumulative drift error, heterogeneity issues, and lengthy processing latency, rendering them inadequate for accurate and low-latency localization as shown in \fig \ref{relatedwork}d \cite{zhao20213d, falanga2020dynamic, mitrokhin2018event, grisetti2010tutorial}.


\noindent \textbf{Our work.}
% We explore the sensing principles of the event camera and mmWave radar and propose EventLoc, an low latency-oriented event camera enhancement system that provides cm-level accurate 3D object localization with millisecond level latency to enable application of event camera in various 3D vision tasks.
We delve into the sensing principles of event cameras and mmWave radar, introducing EventLoc. 
This system enhances event camera functionality with a focus on low-latency 3D object localization, providing cm-level accuracy with millisecond latency on average. 
As a result, EventLoc broadens event camera application in diverse 3D vision tasks.
In detail, EventLoc features three key designs to fully unleash the potential of event camera and mmWave radar for 3D object localization, as elaborated below: \\
% and is implemented with adaptively acceleration algorithms to further improve accuracy and reduce latency, 
\noindent $\bullet$ \textbf{On system architecture front.}
By incorporating mmWave radar with millisecond latency, we enhance the performance of event camera and improve 3D localization performance at the data source.
EventLoc features a carefully designed system architecture that tightly couples event camera and mmWave radar. 
This integration spans from early-stage filtering to later-stage fusion and optimization, fully leveraging the unique advantages of both sensors (§\ref{3.2}). \\
\noindent $\bullet$ \textbf{On system algorithm front.}
We first introduce the Consi-stency-Instructed Collaborative Tracking (\textit{CCT}) algorithm to extract \textit{consistent information} in sensing data from both modalities to filter out environment-triggered noise with a low false positive rate, enhancing the detection performance with a low-latency (§\ref{4.1}). 
We then present the Graph-Informed Adaptive Joint Optimization (\textit{GAJO}) algorithm to fully fuse \textit{complementary information} from both modalities, accelerating the optimization in localizing the object (§\ref{4.2}). \\
\noindent $\bullet$ \textbf{On system implementation front.}
We further analyze the sources of latency and propose an Adaptive Optimization method for boosting the \textit{GAJO}. 
This method dynamically optimizes the set of locations rather than relying on a fixed sliding window, further enhancing the accuracy of localization and reducing latency (§\ref{5.1}).

\begin{figure*}[t]
    \setlength{\abovecaptionskip}{0.4cm} % height above Figure X caption
    \setlength{\belowcaptionskip}{-0.5cm}
    \setlength{\subfigcapskip}{-0.25cm}
    \centering
        \includegraphics[width=2\columnwidth]{Figs/overview2.png}
        \vspace{-0.2cm}
    \caption{System architecture of EventLoc.}
    \label{overview}
    % \vspace{-0.2cm}
\end{figure*} 

\noindent \textbf{Evaluation and Result.} 
We fully implement EventLoc with COTS event camera and mmWave radar.
Extensive experiments in indoor/outdoor environments are conducted with different drone flight conditions to comprehensively evaluate performance of EventLoc.
We compare the end-to-end drone localization accuracy and latency of EventLoc with three SOTA methods.
% Through over 30 hours of real-world experiments, we demonstrate that EventLoc enhances event camera with mmWave radar by achieving a localization accuracy of 1.01 $dm$, surpassing all baselines by >50\%. EventLoc further achieves localization latency of 5.15 $ms$, outperforming baselines by >50\% in average.
Through over 30 hours of experiments, we demonstrate that EventLoc enhances event camera with mmWave radar by achieving an average localization accuracy of 0.101 $m$ and latency of 5.15 $ms$, surpassing all baselines by >50\% on average.
Additionally, EventLoc is marginally affected by factors such as drone type and envir. conditions.\\
\textbf{Real-world deployment.}
We have deployed the sensor platform with EventLoc at a real-world drone delivery airport as shown in \fig \ref{relatedwork}a to demonstrate practicability of the system.
10 hours study shows that EventLoc meets drone landing demands within the constraints of available resources.

\noindent \textbf{Contributions.} This paper makes following contributions.

\noindent $(1)$ We propose EventLoc, a novel low latency-oriented event camera enhancement system. It tightly integrates asynchronous events and mmWave radar sparse point clouds, achieving accurate drone localization with millisecond latency.\\
\noindent $(2)$ We propose the $CCT$, a light-weight cross-modal noise filter to push the limit of detection accuracy by leveraging the \textit{consistent information} from both modalities. \\
\noindent $(3)$  We propose the $GAJO$, a factor graph-based optimization framework that fully harnessing \textit{complementary information} from both modalities to enhance localization performance.\\
% accuracy and latency
\noindent $(4)$ We implement and extensively evaluate EventLoc by comparing it with three SOTA methods, showing its effectiveness. We also deploy EventLoc in a real-world drone delivery airport, demonstrating feasibility of EventLoc.
% The remainder of the paper is organized as follows:
% §2 provides an overview of EventLoc, with detailed descriptions of the Consistency-Instructed Collaborative Tracking algorithm in §3.1 and the Graph-Informed Adaptive Joint Optimization algorithm in §3.2. §4 showcases the adaptive acceleration algorithms and implementation. §5 details our extensive indoor and outdoor experiments with EventLoc. §6 presents our experiments conducted at a real-world delivery airport. §7 discusses related work. §8 concludes the paper.


\section{Related Works}\label{sec:related}

\section{Related Work} \label{sec:related}

% \textbf{Adversarial Attack}
\textbf{Attacks on SLAM.} 
%With the rise of machine learning, 
The robustness of computer vision systems is being actively investigated. With the emergence of adversarial images in the digital domain by adding optimized noise directly to images~\cite{szegedy2013intriguing,carlini2017towards}, researchers find that such attacks also exist physically in the real world \cite{eykholt2018robust,song2018physical,zhao2019seeing}. To fill the gap between attacks in the digital and physical worlds, recent studies have demonstrated that attacks on real-world computer vision systems are practical \cite{eykholt2018robust,li2019adversarial,man2020ghostimage,sharif2016accessorize,zhao2019seeing,zhou2018invisible}. However, attacks on traditional computer vision methods such as SLAM are relatively less explored. \cite{yoshida2022adversarial} proposes an attack against the scan matching algorithm in LiDAR-based SLAM, while most SLAMs in AR/VR devices rely on different sensors like RGB/depth cameras and IMUs. \cite{ikram2022perceptual} and \cite{chen2024adversary} mislead visual SLAM by poisoning the images with special patterns, and \cite{wang2021can} causes the camera to fail using infrared light. In our work, we demonstrate attacks on Visual-Inertial SLAM (VI-SLAM) by perturbing the IMU readings, rather than cameras, and showing its impact on XR user experience. 

\textbf{Acoustic Injection Attacks.} Among various physical attacks, acoustic injection attacks are attractive due to their low cost. Son~\etal~\cite{son2015rocking} were the first to introduce acoustic attacks on MEMS gyroscopes, demonstrating how these attacks could lead to sensor denial-of-service and result in drone crashes. WALNUT~\cite{trippel2017walnut} expanded on this by developing output biasing and control attacks that enable precise manipulation of MEMS accelerometer outputs using modulated sound waves. Wang et al.~\cite{wang2017sonic} demonstrated a sonic gun, showcasing the vulnerability of various smart devices (\eg drones and self-balancing vehicles) to acoustic attacks. Tu et al. \cite{tu2018injected} designed side-swing and switching attacks to alter the outputs of MEMS gyroscopes and accelerometers. Furthermore, Ji et al. \cite{ji2021poltergeist} fool the object detectors by applying acoustic attack to the image stabilizers commonly used in modern cameras. However, none of the existing works study the relationship between the acoustic injections and SLAM outputs on recent XR devices. 

% \zijian{Do we need one session about security in AR/VR?}
% \yicheng{TODO}
%\jiasi{cite the AIVR paper (UMass Amherst?) paper is we have not already. They add IMU perturbation but w/o SLAM, iirc} \yicheng{Cited}

\textbf{XR Security and Privacy.} 
%Security and privacy concerns in XR systems have gained significant attention. 
For single-user XR systems, researchers have demonstrated various side-channel attacks to extract sensitive information (\eg keystrokes) through video feeds~\cite{ling2019know}, head movements~\cite{nair2023unique, slocum2023going}, architectural hints~\cite{zhang2023its,shang2020arspy}, power usage~\cite{li2024dangers}, and EM side-channel leakages~\cite{al2021vr}. In multi-user XR systems, Su et al.~\cite{su2024remote} use avatar motion data to infer keystrokes in shared VR environments. Slocum et al.~\cite{slocum2024doesn} reveal vulnerabilities in the shared state frameworks of multi-user AR. Similarly, Lebeck et al.~\cite{lebeck2017securing} highlight risks like deceptive virtual objects and emphasize access control for managing shared physical and virtual spaces. Ruth et al.~\cite{ruth2019secure} further propose a secure multi-user AR framework focusing on content sharing and permissions.
Chandio et al.~\cite{chandio2024stealthy} %introduced a multi-modal spatiotemporal attack that 
simultaneously manipulated visual and inertial sensors to disrupt XR pose estimation. However, their study evaluated the attack using offline datasets and assumed the attacker's capability to manipulate IMU data streams through acoustic means, without real experiments. Ours is the first to demonstrate acoustic injection attacks on recent XR devices, like the Hololens 2, in the real world.
 


\section{Method}\label{sec:method}
% \begin{figure}
%     \centering
%     \includegraphics[width=0.5\linewidth]{Move_teaser.pdf}
%     \caption{Comparison of different dynamic compute approaches. length of arrow indicates residual transformation per token while width indicates velocity of transformation.}
%     \label{fig:enter-label}
% \end{figure}

\section{Method}
\label{sec:method}
Residual connections play a crucial role in shaping token representations, yet their dynamics remain underexplored in the context of efficient decoding. In this work, we delve deeper into transformer residual dynamics and investigate how modulating residual transformation velocity can improve inference efficiency in token-level processing, optimizing both dense and sparse MoE transformers.


\subsection{Residual Dynamics and Motivation for Multi-rate Residuals} \label{sec:motivation}

To analyze how hidden representations evolve across different layers of a transformer architecture, it's crucial to consider the effect of residual connections. Each transformer decoder layer typically has residual connections across attention and MLP submodules. As the residual stream $h_i$ traverses from interval $E_j$ to $E_{j+1}$, it undergoes a residual transformation given by:  
% \begin{equation}
% \label{eq:slow_residual_transformation}
% H_{E_{j+1}} = H_{E_j} \prod_{i=E_j}^{E_{j+1}} \left( I + \mathcal{A}_i \right) \left( I + \mathcal{M}_i \right) \quad \text{where} \quad \mathcal{A}_i = f(c_i, h_{i}), \mathcal{M}_i = g(h_i)
% \end{equation}

\begin{equation} \label{eq:slow_residual_transformation}
h_{E_{j+1}} = h_{E_j} + \sum_{i=E_j}^{E_{j+1}-1} \left( \mathcal{A}_i(h_i) + \mathcal{M}_i(h_i + \mathcal{A}_i(h_i)) \right) \quad \text{where} \quad \mathcal{A}_i = f(c_i, h_{i}), \mathcal{M}_i = g(h_i). 
\end{equation}

Here, \( \mathcal{A}_i \) denotes the non-linear transformation introduced by the multi-head attention mechanism at layer \( i \), while \( \mathcal{M}_i \) corresponds to the non-linear transformation of the MLP block at the same layer. These transformations depend on the input residual stream \( h_i \) and, in the case of \( \mathcal{A}_i \), the previous contextual representation \( c_i \).\footnote{Normalization layers are typically applied in practice but are omitted here for simplicity of the argument.}


% For easy tokens, the magnitude and direction of this delta transformation become progressively smaller with each successive layer as shown in \cref{fig:delta_transformation}. Consequently, it is feasible to predict these tokens after only a few residual connections, whereas harder tokens necessitate more extensive processing through additional layers.

\begin{figure}[ht]
    \centering
    \begin{subfigure}{0.48\textwidth}
        \centering
        \includegraphics[width=\textwidth]{sections/figures/residual_change.pdf}
        \caption{}
        \label{fig:residual_change}
    \end{subfigure}%
    \hfill
    \begin{subfigure}{0.48\textwidth}
        \centering
        \includegraphics[width=\textwidth]{sections/figures/alignment_wrt_dedicated_model.pdf}
        \caption{}
    \label{fig:alignment_wrt_dedicated_model}
    \end{subfigure}
    \caption{(a) As residual streams propagate through the model, the directional shifts in the residuals become progressively smaller. (b) A dedicated model with $k$ layers achieves a faster rate of change in residual streams and higher alignment than base model leveraging early exit mechanisms at layer $k$.}
    \label{fig}
\end{figure}


To examine whether residual transformations can be accelerated across layers, we conducted experiments using a diverse set of prompts on a pre-trained Phi3 model~\cite{phi3_report}. As illustrated in \cref{fig:residual_change}, we measured the directional shift in residual states as \( 1 - \mathcal{C}(h_{i-1}, h_i) \), where \(\mathcal{C}\) denotes normalized cosine similarity. This shift is notably higher in the initial layers, gradually decreasing in subsequent layers. This behavior allows traditional early exit approaches to effectively accelerate decoding by enabling earlier exits for simpler tokens. However, these approaches typically rely on a distance-based approximation, where the full residual transformation of the model is approximated by the residual transformations of the initial layers. To gain deeper insights into the distance versus velocity aspects of residual transformation, we conducted a comparative study. Specifically, we trained an early exit head at layer $k$ of the Phi3 model, which consists of 32 layers, restricting the distance traveled by each token. To accelerate the residual transformation relative to number of layers, we trained a smaller model consisting of only $k$ layers, while keeping all other hyperparameters consistent. We then compared the next-token prediction accuracy of the early exit head of the base model with that of the smaller model. To ensure an equal number of trainable parameters, we inserted low-rank adapters into the smaller model and trained only these adapters, whereas, in the distance-based approach, we trained solely the early exit head. In addition, to accelerate the residual transformation in smaller model, we distilled the residual streams from the larger model by incorporating a distillation loss ~\cite{sanh2019distilbert} between the residual state at layer \(i\) of the smaller model and the residual state at layer \(4 \times i\) of the larger model. As shown in ~\cref{fig:alignment_wrt_dedicated_model} the smaller model demonstrates a significantly faster rate of change in residual streams, leading to higher next token prediction accuracy after $k$ layers compared to the base model that employs traditional early exit mechanisms after $k$ layers \cite{schuster2022confident, chen2023eellm, varshney-etal-2024-investigating}. This experimental setup, which modifies only the rate of change in residual streams while keeping other factors constant, suggests that dense transformers, trained with a fixed number of layers, may inherently possess a slow residual transformation bias.

This observation raises an intriguing question: if the rate of change in residual streams could be accelerated relative to the number of layers, is it possible to facilitate earlier alignment for a greater proportion of tokens? Earlier alignment would be beneficial to not only facilitate dynamic computation but also for generating speculative tokens efficiently with high acceptance rates in speculative decoding setups ~\cite{leviathan2023fast, chen2023accelerating}. 

%thereby enhancing the efficiency of early exiting? 
 % This bias likely constrains the effectiveness of early exiting, particularly for easier tokens. By addressing this limitation through accelerated residual transformations, we hypothesize that it is possible to substantially improve the efficiency and accuracy of early exit strategies in transformer models.

\subsection{Multi-Rate Residual Transformation} \label{m2r2_method}

To address the slow residual transformation bias described in ~\cref{sec:motivation}, we introduce \textit{accelerated residual streams} that operate at rate $R$ relative to original slow residual stream. We pair slow residual stream, $h$ with an accelerated residual stream, $p$, which has an intrinsic bias towards earlier alignment. Relative to ~\cref{eq:slow_residual_transformation}, accelerated residual transformation from interval $E_j$ to $E_{j+1}$ can be represented as: 

% \begin{equation}
% \label{eq:fast_residual_transformation}
% P_{E_{j+1}} = P_{E_j} \prod_{i=E_j}^{E_{j+1}} \left( I + \hat{\mathcal{A}_i} \right) \left( I + \hat{\mathcal{M}_i} \right) \quad \text{where} \quad \hat{\mathcal{A}_i} = \hat{f}(c_i, P_{i}), \hat{\mathcal{M}_i} = \hat{g}(P_{i})
% \end{equation}


\begin{equation} \label{eq:fast_residual_transformation}
p_{E_{j+1}} = p_{E_j} + \sum_{i=E_j}^{E_{j+1}-1} \left( \hat{\mathcal{A}_i}(p_i) + \hat{\mathcal{M}_i}(p_i + \hat{\mathcal{A}_i}(p_i)) \right) \quad \text{where} \quad \hat{\mathcal{A}_i} = \hat{f}(c_i, p_{i}), \hat{\mathcal{M}_i} = \hat{g}(h_i), 
\end{equation}



where $\hat{\mathcal{A}_i}$ and $\hat{\mathcal{M}_i}$ denote non-linear transformation added by layer $i$ to previous accelerated residual $p_{i}$. Similar to $\mathcal{A}_i$, non-linear transformation $\hat{\mathcal{A}_i}$ attends to same context $c_i$ but uses a different transformation $\hat{f}$ for accelerating $p_{E_j}$ relative to $h_{E_j}$. 

We integrate accelerated residual transformation directly into the base network using parallel accelerator adapters such that rank of accelerator adapters $R_p << d$ where $d$ denotes base model hidden dimension. This setup allows the slow residual stream $h_{E_j}$ to pass through the base model layers while the accelerated residual stream $p_{E_j}$ utilizes these parallel adapters as shown in ~\cref{fig:m2r2_main}. Both slow and accelerated residuals are processed in same forward pass via attention masking and incur negligible additional inference latency in memory bound decoding setups, while in compute bound decoding setups where FLOPs optimization is essential, accelerated residual stream utilizes a fraction of attention heads that of slow residual (see ~\cref{sec:flops_optimization}). Additionally, to maximize the utility of accelerated residual transformations without introducing dedicated KV caches, we propose a shared caching mechanism between the slow and accelerated streams which minimally impact alignment benefits of our approach while offering substantial memory savings (see ~\cref{fig:koala_alignment}). Specifically, the attention operation on the slow residuals \( \text{MHA}(h_t, h_{\leq t}, h_{\leq t}) \) is redefined for accelerated residuals as 
\[
\hat{\mathcal{A}} = MHA(p_t, h_{<t} \oplus p_t, h_{<t} \oplus p_t),
\]
where the accelerated residual at time-step $t$, \( p_t \) attends to the slow residual’s KV cache, facilitating the reuse of contextual information across both residual streams without incurring additional caching costs. Here, \(MHA(q, k, v) \) represents multi-head attention between query \( q \), key \( k \), and value \( v \).

\begin{figure}
    \centering
    \includegraphics[width=0.8\linewidth]{sections//figures/m2r2_main2.pdf}
    \caption{Multi-rate Residuals Framework: Slow residual stream of base model is accompanied by a faster stream that operates at a $2-(J+1)\times$ rate relative to the slow stream, undergoing transformations via accelerator adapters as detailed in \cref{m2r2_method}, where J denotes number of early exit intervals. Colors within the slow and fast residual streams indicate similarity, with matching colors representing the most closely aligned residual states. At the beginning of the forward pass and at each exit point, the accelerated residual state is initialized from the corresponding slow residual state to avoid gradient conflict during training (see ~\cref{sec:grad_conflict}). Early exiting decisions are informed by the Accelerated Residual Latent Attention (ARLA) mechanism, described in \cref{method_arla}, which evaluates residual dynamics across consecutive exit gates.}
    \label{fig:m2r2_main}
\end{figure}

% Furthermore. to maximize the benefits of fast residual transformations without using dedicated KV caches, we propose sharing the fast network’s cache with the slow network. Formally speaking, We modify attention operation on slow residuals $MHA(H_t, H_{<=t}, H_{<=t})$ as $MHA(P_{t}, H_{<t} \oplus P_t, H_{<t}  \oplus P_t)$ such that accelerated residuals attend to previous slow context KV cache, where $MHA(q,k,v)$ denotes multi head attention between query, $q$, key $k$ and value $v$.


\subsection{Enhanced Early Residual Alignment}
Early residual alignment is instrumental in optimizing early exiting, speculative decoding, and Mixture-of-Experts (MoE) inference mechanisms. In this section, we provide a detailed analysis of how accelerated residuals enhance these inference setups.

% By aligning the residual states of intermediate layers with the final output representations, the model can maintain high prediction accuracy even when computations are truncated at earlier layers. This enables more reliable early exiting, reducing the overall computational cost while preserving performance. Additionally, in speculative decoding, early residual alignment allows the model to make confident predictions using faster, partial computations, thereby accelerating inference without sacrificing output quality.


\subsubsection{Early Exiting} \label{method_early_exiting}

A prevalent strategy for enabling early exiting at an intermediate layer $E_{j}$ involves approximating the residual transformation between $E_{j}$ and the final layer $N-1$ using a linear, context independent mapping, $\mathcal{T}$, such that $H_{N-1} \approx \mathcal{T}(H_{E_{j}})$. This approximation has been extensively employed in conventional approaches ~\cite{schuster2022confident, chen2023eellm, varshney-etal-2024-investigating}, providing a computationally efficient means to project the output of deeper layers from intermediate states. Specifically, residual state of layer $N-1$ with this approximation can be expressed as:


% \begin{equation}
% \label{eq: vanila_ea_assumption}
% \Phi(H_{E_{j}}) \sim H_{E_{j}} \prod_{i=E_{j}}^{N}\left( I + \mathcal{A}_i \right) \left( I + \mathcal{M}_i \right) \quad \text{where} \quad \Phi \perp C
% \end{equation}

\begin{equation} \label{eq:early_exiting}
h_{E_j} + \sum_{i=E_j}^{N-1} \left( \mathcal{A}_i(h_i) + \mathcal{M}_i(h_i + \mathcal{A}_i(h_i)) \right) \sim \mathcal{T}(h_{E_{j}})  \quad \text{where} \quad \mathcal{T} \perp c. 
\end{equation}


Here, $\mathcal{A}_i$ and $\mathcal{M}_i$ represent the residual contributions of the multi-head attention and MLP layers, respectively, while $\mathcal{T}$ remains independent of $c$, the preceding context.

This approach is inherently limited by two major factors: first, the assumption of linearity between $h_{E_{j}}$ and $h_{N-1}$ may not hold uniformly for all tokens, particularly when $E_j \ll N$. Second, the linear transformation $\mathcal{T}$ disregards the influence of the context $c$ and fails to account for the latent representations of previous contextual states. In contrast, M2R2 accelerated residual states mitigate both of these challenges by approximating the slow residual transformation of all layers via a faster residual transformation of fewer layers as:
% \begin{equation}
% H_{E_j} \prod_{i=E_j}^{N}\left( I + \mathcal{A}_i \right) \left( I + \mathcal{M}_i \right) \sim P_{E_j} \prod_{i=E_j}^{E_j+1}\left( I + \hat{\mathcal{A}_i} \right) \left( I + \hat{\mathcal{M}_i} \right)
% \end{equation}


\begin{equation} \label{eq:m2r2_approximating_ea}
h_{E_j} + \sum_{i=E_j}^{N-1} \left( \mathcal{A}_i(h_i) + \mathcal{M}_i(h_i + \mathcal{A}_i(h_i)) \right) \sim p_{E_j} + \sum_{i=E_j}^{E_{j+1}-1} \left( \hat{\mathcal{A}_i}(p_i) + \hat{\mathcal{M}_i}(p_i + \hat{\mathcal{A}_i}(p_i)) \right), 
\end{equation}

% \begin{equation} \label{eq:fast_residual_transformation}
% p_{E_{j+1}} = p_{E_j} + \sum_{i=E_j}^{E_{j+1}-1} \left( \hat{\mathcal{A}_i}(p_i) + \hat{\mathcal{M}_i}(p_i + \hat{\mathcal{A}_i}(p_i)) \right) \quad \text{where} \quad \hat{\mathcal{A}_i} = \hat{f}(c_i, p_{i}), \hat{\mathcal{M}_i} = \hat{g}(h_i) 
% \end{equation}






where $p_{E_j}$ is initialized from the slow residual state $h_{E_j}$ at each early exit interval $E_j$ using an identity transformation (see ~\cref{fig:m2r2_main}). As shown in ~\cref{fig:m2r2_residual_sim}, accelerated residuals offer a smoother, more consistent shift in residual direction across layers, in contrast to the abrupt changes typically seen at early exit points in standard early exit methods. Moreover, the normalized cosine similarity between accelerated states at early exit intervals and final residual states is substantially higher compared to traditional early exit techniques, highlighting improved alignment with final layer representations. Traditional adaptive compute methods are constrained by two principal factors: the number of tokens eligible for early exit at intermediate layers and the precision of early exit decision. If residual streams fail to saturate early, the majority of tokens remain ineligible for exit, thereby diminishing potential speedups. Additionally, imprecise delineations between tokens suitable for early exit can lead to underthinking (premature exits that adversely affect accuracy) or overthinking (unnecessary processing that compromises efficiency) ~\cite{zhou2020self, dai2020dynamic}. Enhanced early alignment using ~\cref{eq:m2r2_approximating_ea} helps to address  first issue. To address the second issue we introduce Accelerated Residual Latent Attention, which dynamically assesses the saturation of the residual stream, allowing for a more precise differentiation between tokens that can exit early and those requiring further processing.

% This results in uniform change in residual direction    
% % We keep $\mathcal{A} = \hat{\mathcal{A}}$, while $\hat{\mathcal{M}}$ is accelerated by a factor of $2 - (N_{E}+1)X$ relative to the slower residual transformation $\mathcal{M}$, where $N_E$ represents number of early exiting intervals.
% Figure~\cref{fig:rate_change_comparison} illustrates the comparative rate of change between these transformation streams.



% fig:rate_change_comparison
% - grid plot x axis -> layer id (0, 8) , y axis -> layer id -> dark color cell for max similarity , lighter for lower 
% 
-------------------------------------------------------
Let's consider residual stream $h_i$ traverses through interval $E_j$ to $E_{j+1}$ and undergoes residual transformation given by 
\begin{equation}
h_{E_{j+1}} = h_{E_j} \prod_{i=E_j}^{E_{j+1}} \left( 1 + \delta_i \right)    
\end{equation}

where $\delta_i$ denotes non-linear transformation added by layer $i$. Each non-linear transformation of layer $i$ is a function of previous contextual representation, $c_i$ and input residual stream $h_i-1$ as
$\delta_i = f(c_i, h_{i-1})$ 

One way to exit early at exit $E_j+1$ is to assume that residual transformation from $E_j+1$ to final layer $N-1$ can be approximated by a linear function $\phi$ as $h_{N-1} \sim \Phi(h_{E_j+1})$ and most conventional approaches such as \todo{cite EA papers} use this approach. In other words, 

\begin{equation}
\Phi(h_{E_j+1} \sim h_{E_j+1} \prod_{i=E_j+1}^{N} \left( 1 + \delta_i \right)   
\end{equation}

This approach suffers from two primary issues, linearity assumption from $h_E_j+1$ to $H_N-1$ if often incorrect, particularly when $E_j << N$. More importantly, linear transformation $\Phi$ doesn't consider effect of context $C_i$. M2R2  effectively addresses these issues as accelerated residual stream at interval $E_j+1$ can be represented as 

\begin{equation}
r_{E_{j+1}} = r_{E_j} \prod_{i=E_j}^{E_{j+1}} \left( 1 + \gamma_i \right)    
\end{equation}

where $\gamma_i$ denotes non-linear transformation added by layer $i$ to previous accelerated residual $r_i-1$. Similar to $\delta_i$, non-linear transformation $\gamma_i$ considers context $C_i$ as 
$\gamma_i = g(c_i, r_{i-1})$. So in summary, slow residual transformation is approximated by accelerated residual as: 

\begin{equation}
h_{E_j} \prod_{i=E_j}^{N} \left( 1 + \delta_i \right) \sim h_{E_j} \prod_{i=E_j}^{E_j+1} \left( 1 + \gamma_i \right)
\end{equation}

It's worth noting that accelerated residual $r_i$ and slow residual $h_i$ are processed concurrently at layer $i$ by constructing proper attention mask such as attention of slow residual is represented as 

$MHA(H_it, H_{i<=t}, H_{i<=t}$ while attention of fast residual is computed as 

$MHA(r_it, H_{i<=t}, H_{i<=t}$ where $MHA(q,k,v$ denotes multi head attention between query, $q$, key $k$ and value $v$.


------------------------------------------------------------------

Vertical latent attention on accelerated residual is computed as 
$MHA(S_mt, S(Ej<=i<=m)t, S(Ej<=i<=m)t)$ where $Smt$ denotes query/key/value projection in latent domain at layer $m$ at time $t$. 
------------------------------------------------------------------

Gradient conflict Avoidance: 

Let's consider $w_j$ is a trainable parameter that belongs to a layer between $E_j$ and $E_j+1$. Consider early exit loss at gate $E_j+1$, $L_j+1$, gradient propagation of $w_j$ at another trainable parameter $w_j-n$ can be gives as 

$\sum_{k=E_j-n}^{E_j} \beta_k \frac{\partial L_{E_k}}{\partial w_k}$

where $\beta_j$ denotes backward transformation coefficient for weight $w_j$ to reach gate $E_j$. 
 
On the other hand, gradient propagation in proposed approach can be represented as 

\[
\frac{\partial L_{E_j}}{\partial w_j} = 
\begin{cases} 
\beta_j \frac{\partial L_{E_j}}{\partial w_j} & \text{if } E_j \leq w_j \leq E_{j+1} \\
0 & \text{otherwise}
\end{cases}
\]







% \begin{figure}[ht]
%     \centering
%     \includegraphics[width=0.8\textwidth, height=5cm]{rate_change_comparison.png}
%     \caption{Rate of change comparison between fast and slow residual streams.}
%     \label{fig:rate_change_comparison}
% \end{figure}

%vary k and and plot EA accuracy for larger and smaller models. 

% \begin{figure}[ht]
%     \centering
%     \includegraphics[width=0.5\textwidth,height=5cm]{sections/figures/alignment_comparison_dialogsum.pdf}
%     \caption{Alignment of exited tokens for different early exit layers using traditional early exiting heads, dedicated faster networks, and faster residuals.}
%     \label{fig:small_model_early_exiting}
% \end{figure}


\textbf{Accelerated Residual Latent Attention} \label{method_arla}

In the context of residual streams, we observe that the decision to exit at a given layer can be more effectively informed by analyzing the dynamics of residual stream transformations, instead of solely relying on a classification head applied at the early exit interval $E_j$. To capture the subtle dynamics of residual acceleration, we propose a \textit{Accelerated Residual Latent Attention} (ARLA) mechanism. This approach involves making the exit decision at gate $E_j$ by attending to the residuals spanning from gate $E_{j-1}$ to $E_j$, rather than considering only the residual at gate $E_j$. To minimize the computational overhead associated with exit decision-making, the attention mechanism operates within the latent domain as depicted in ~\cref{fig:arla_arch}. Formally, for each interval $[E_j, E_{j+1}]$, the accelerated residuals are projected into Query ($Q^s_{E_j}, \ldots, Q^s_{E_{j+1}}$), Key ($K^s_{E_j}, \ldots, K^s_{E_{j+1}}$), and Value ($V^s_{E_j}, \ldots, V^s_{E_{j+1}}$) vectors, with latent dimension $d^s$ for $Q^s$, $K^s$, and $V^s$ being significantly smaller than hidden dimension of $p$.\footnote{We use $d^s = 64$ for experiments described in ~\cref{sec:experiments}.} Notably, when the router is allowed to make exit decisions at gate $E_j$ based on residual change dynamics, we observe that the attention is not confined to the residual state at $E_j$ but is distributed across residual states from $E_{j-1}$ to $E_j$, %as illustrated in Figure~\ref{fig:vertical_latent_attention_dynamics}. 
This broader focus on residual dynamics significantly reduces decision ambiguity in early exits, as demonstrated in Figure~\ref{fig:roc_arla}, which contrasts routers based on the last hidden state, and the proposed ARLA router.

%show R -> S transformation. 
%show parameter and flop overhead as compared to adapter on last hidden state.

% \begin{figure}[ht]
%     \centering
%     \includegraphics[width=0.5\textwidth,height=5cm]{sections/figures/roc_arla.pdf}
%     \caption{ROC curves of early exit decision strategies: confidence-based methods (CALM/LITE), routers based on the accelerated hidden state, and latent attention routers.}
%     \label{fig:decision_making_comparison}
% \end{figure}

% \begin{figure}[ht]
%     \centering
%     \includegraphics[width=0.5\textwidth,height=5cm]{vertical_latent_attention.png}
%     \caption{Vertical latent attention mechanism for optimizing early exit decisions by considering residuals from gate \(M\) through \(M-1\).}
%     \label{fig:vertical_latent_attention}
% \end{figure}

\begin{figure}[ht]
    \centering
    \begin{subfigure}{0.52\textwidth}
        \centering
        \includegraphics[width=\textwidth, height = 4cm]{sections/figures/arla_arch.pdf}
        \caption{Accelerated Residual Latent Attention (ARLA): Accelerated residuals between early exit gates are projected into latent domain and attention over residual states within the interval is computed to capture residual dynamics and exit decision is made based on residual saturation.}
        \label{fig:arla_arch}
    \end{subfigure}%
    \hfill
    \begin{subfigure}{0.45\textwidth}
        \centering
        \includegraphics[width=\textwidth, height = 4.5cm]{sections/figures/vla_roc.pdf}
        \caption{ROC classification curves of early exit decision strategies using a linear router used on last residual state ~\cite{schuster2022confident, varshney-etal-2024-investigating, chen2023eellm}  and using ARLA approach that considers residual dynamics. }
        \label{fig:roc_arla}
    \end{subfigure}
    \caption{Effectiveness of ARLA in capturing residual dynamics for early exiting decisions.}


\end{figure}



% \begin{figure}[ht]
%     \centering
%     \includegraphics[width=1\textwidth,height=5cm]{sections/figures/arla.pdf}
%     \caption{fig that plots 32 rows 2 cols heatmap showing attention at each gate}
%     \label{fig:vertical_latent_attention_dynamics}
% \end{figure}

\subsubsection{Self Speculative Decoding} \label{method_self_speculative_decoding}

An alternative means to exploit the early alignment properties of our approach is through the use of accelerated residual states for speculative token sampling to accelerate autoregressive decoding. Speculative decoding aims to speed up memory-bound transformer inference by employing a lightweight draft model to predict candidate tokens, while verifying speculated tokens in parallel and advancing token generation by more than one token per full model invocation \cite{leviathan2023fast, chen2023accelerating, xia2023speculative, miao2023specinfer}. Despite its effectiveness in accelerating large language models (LLMs), speculative decoding introduces substantial complexity in both deployment and training. A separate draft model must be specifically trained and aligned with the target model for each application, which increases the training load and operational complexity ~\cite{chen2023accelerating}. Additionally, this approach is resource-inefficient, as it requires both the draft and target models to be simultaneously maintained in memory during inference \cite{leviathan2023fast, chen2023accelerating}. 

One strategy to address this inefficiency is to leverage the initial layers of the target model itself to generate speculative candidates, as depicted in ~\cite{Tang2024}. While this method reduces the autoregressive overhead associated with speculation, it suffers from suboptimal acceptance rates. This occurs because the linear transformation employed for translating hidden states from layer $k$ to the final layer $N$ is typically a poor approximation, as discussed in ~\cref{sec:motivation} and ~\cref{method_early_exiting}. Our approach resolves this limitation by utilizing accelerated residuals, which demonstrate higher fidelity to their slower counterparts. By utilizing accelerated residuals operating at a rate of $N/k$, where $k$ denotes the number of layers used for candidate speculation, we are able to efficiently generate speculative tokens for decoding.\footnote{We typically set $k = 4$ to balance the trade-off between autoregressive drafting overhead and acceptance rate, as discussed in~\cref{sec:experiments}.}
 This technique not only obviates the need for multiple models during inference but also improves the overall efficiency and effectiveness of speculative decoding.

\begin{figure}
    \centering    \includegraphics[width=1\linewidth]{sections/figures/m2r2_aot_loading.pdf}
    \caption{Ahead-of-Time Expert Loading: M2R2 accelerated residual stream predicts experts required for future layers, reducing reliance on on-demand lazy loading. Speculative pre-loading is efficiently overlapped with computation of multi-head attention (MHA) and MLP transformations. Only incorrectly speculated experts are loaded lazily, resulting in faster inference steps and improved computational efficiency. Here, H indicates LBM Host while D indicates HBM Device.}
    \label{fig:moe_expert_aot_loading}
\end{figure}


\subsubsection{Ahead of Time Expert Loading:} \label{method_aot_expert_loading}

Recent advancements in sparse Mixture-of-Experts (MoE) architectures ~\cite{shazeer2017outrageously, fedus2022switch, artetxe2019massively, lepikhin2020gshard, zoph2022designing} have introduced a paradigm shift in token generation by dynamically activating only a subset of experts per input, achieving superior efficiency in comparison to dense models, particularly under memory-bound constraints of autoregressive decoding \cite{fedus2022switch, zoph2022designing}. This sparse activation approach enables MoE-based language models to generate tokens more swiftly, leveraging the efficiency of selective expert usage and avoiding the overhead of full dense layer invocation. In dense transformer models, pre-loading layers is a common strategy to enhance throughput, as computations of current layer can be overlapped with pre-loading of next layer parameters ~\cite{narayanan2021efficient, shoeybi2020megatron}. However, MoE models face a unique challenge: expert selection occurs dynamically based on previous layer’s output, making it infeasible to preload next layer’s experts in parallel. This limitation results in inherent latency, as expert loading becomes a sequential, on-demand process ~\cite{lepikhin2020gshard, fedus2022switch}.

To address this inefficiency, our method introduces a mechanism with \textit{accelerated residuals}, which not only captures key characteristics of base slower residual states but also exhibit high cosine similarity with their final counterparts (as illustrated in \cref{fig:m2r2_residual_sim}). By employing accelerated residual streams, we can effectively predict the necessary experts for future layers well in advance of their actual invocation. Specifically, using a $2\times$ accelerated residual, the experts needed for layers $2i+2$ and $2i+3$ can be identified while still computing in layer $i$, thus overcoming the bottleneck of sequential, on-demand expert selection and mitigating latency in the decoding pipeline, as shown in \cref{fig:moe_expert_aot_loading}. Note that, we use fixed set of accelerator adapters for transforming accelerated residuals (as discussed in ~\cref{m2r2_method}) while slow residual is transformed via expert routing mechanism. 

Furthermore, our approach integrates a Least Recently Used (LRU) caching strategy, which enhances memory efficiency by replacing the least recently used experts with speculated experts that are anticipated to be needed in upcoming layers. This hybrid approach of preemptive expert loading with LRU caching yields substantial improvements over traditional on-demand loading or standalone caching strategies. By minimizing cache misses and efficiently managing memory, this approach addresses both compute and memory bottlenecks, leading to faster, more resource-efficient token generation in MoE architectures. A comprehensive evaluation of this strategy, in relation to state-of-the-art methods, is provided in \cref{experiments_aot}, and the compute and memory traces on an A100 GPU are detailed in \cref{fig:moe_aot_cuda_trace}.



% Recent advancements in sparse Mixture-of-Experts (MoE) architectures have introduced the concept of utilizing distinct computational paths for different tokens \cite{shazeer2017outrageously}. This approach, wherein only a subset of experts are activated per input, enables MoE-based language models to generate tokens more swiftly compared to their dense counterparts due to memory-bound nature of auto-regressive decoding. In dense models, pre-loading layers in advance is a common strategy to enhance computational efficiency. However, this technique is not applicable to MoE models, where expert selection occurs dynamically based on the outputs of previous layers, preventing parallel pre-fetching of experts.

% Our proposed method addresses this inefficiency. Accelerated residuals, which are highly similar to their slower counterparts (see \cref{fig:similarity}), can reliably predict the necessary experts ahead of time. For instance, by utilizing $2X$ accelerated residual stream, we can predict the experts needed for the layer $2i+1$ and $2i+3$ while carrying out computation in layer $i$. This enables us to commence expert loading significantly earlier, as illustrated in \cref{expert_loading}, effectively mitigating the delays observed with the naive on-demand expert loading. Additionally, our method benefits from incorporating a Least Recently Used (LRU) strategy, where speculated experts replace those that are least recently utilized, resulting in improved performance compared to using either strategy alone. For a comprehensive evaluation, refer to \cref{moe_trace}, which provides a CUDA compute and memory trace of our approach executed on <>.



% A naive solution involves using the residual state of the previous layer along with the gating function of the next layer to predict which experts need to be loaded, and initiating the expert loading process in parallel with the attention computation of the next layer. Yet, as shown in \cref{fig:MOE_attn_vs_loading_time}, the attention computation for medium to long contexts is considerably faster than the expert loading time, making this approach inefficient.




\subsection{Training} \label{method_training}
% This approach is feasible due to the absence of gradient conflicts, as discussed in \cref{sec:grad_conflict}.

To accelerate residual streams, we employ parallel accelerator adapters as described in \cref{m2r2_method}.  For the early exiting use-case outlined in \cref{method_early_exiting}, we define the training objective for these adapters using the following loss function, which combines cross-entropy loss at each exit $E_j$ with distillation loss at each layer $i$. Loss weights coefficients $\alpha_0$ and $\alpha_1$ are employed to balance contribution of corresponding losses.

\begin{align} \label{eq:mr_loss}
L_{\text{m2r2}} = \underbrace{-\alpha_0 \sum_{j=1}^{J} \sum_{t=1}^{T} \log p_{\theta} \left( \hat{y}_t^{E_j} \mid y_{<t}, x \right)}_{\text{cross-entropy loss}} 
+ \underbrace{\alpha_1\sum_{i=1}^{E_{J-1}} \sum_{t=1}^{T} \| \mathbf{p}_{t}^{i} - \mathbf{h}_{t}^{((i - E_{j(i)}) \cdot R_i) + E_{j(i)})} \|^2}_{\text{distillation loss}}.
\end{align}

where $\hat{y}_t^{E_j}$ denotes the predictions from the accelerated residual stream at layer $E_j$ and time step $t$, $y_t$ represents the corresponding ground truth tokens, and $x$ indicates previous context tokens. The distillation loss at each layer $i$ is computed by comparing accelerated residuals at layer $i$ with slow residuals at layer $(i - E_{j(i)}) \cdot R_i + E_{j(i)}$, where $R_i$ denotes the rate of accelerated residuals at layer $i$ while $E_{j(i)}$ represents the most recent gate layer index such that $E_{j(i)} <= i$. \( J \) represents the total number of early exit gates, N denotes number of hidden layers and $E_j$ denotes layer index corresponding to gate index $j$ and \( T \) denotes the sequence length. 

In dynamic compute settings, after training of accelerator adapters, we optimize the query, key, and value parameters governing the ARLA routers (see ~\cref{method_arla}) across all exits in parallel on binary cross entropy loss between predicted decision and ground truth exiting decision. The ground truth labels for the router are determined based on whether the application of the final logit head on $\hat{y}_t^{E_j}$ yields the correct next-token prediction. 


% The objective for this optimization is defined by the following loss function:


%TODO are equations required ? 
% \begin{equation} \label{eq:arla_loss_combined}\small
%     L_{\text{arla}} = -\frac{1}{N} \sum_{t=1}^{T} \left( \sum_{j=1}^{E_n} \left[ O_t^{E_j} \log(\hat{O}_t^{E_j}) + (1 - O_t^{E_j}) \log(1 - \hat{O}_t^{E_j}) \right] \right), \quad \text{where} \quad 
%     O_t^{E_j} = \begin{cases} 
%     1, & \text{if } L(\hat{y}_t^{E_j}) = y_t^{E_j} \\
%     0, & \text{otherwise}
%     \end{cases}
% \end{equation}

% where $\hat{O}_t^{E_j}$ represents the binary predicted logits produced by the vertical latent attention router, as described in \cref{sec:arla}, at gate $E_j$ and time step $t$, and $O_t^{E_j}$ denotes the corresponding ground truth labels. The ground truth labels for the router are determined based on whether the application of the logit head on $\hat{y}_t^{E_j}$ yields the correct next-token prediction. The parameters controlling vertical latent attention are trained concurrently to ensure consistency and efficient use of computational resources.

For self-speculative decoding, as described in \cref{method_self_speculative_decoding}, the training objective remains the same as \cref{eq:mr_loss}, but with the number of intervals set to $J = 1$ and the rate of residual transformation set to $R_n = N/k$, where the first $k$ layers generate speculative candidate tokens. In the context of Ahead-of-Time Expert Loading for Mixture-of-Experts (MoE) models (see \cref{method_aot_expert_loading}), setting the rate of residual transformation to $R_n = 2$ typically offers a good trade-off between the accuracy of expert speculation and AoT pre-loading of experts. 

% Thus, we set $J = 1$ and $E_1 = 16$.


~\subsection{FLOPs Optimization} \label{sec:flops_optimization}

Naively implemented, M2R2 incurs higher FLOP overhead compared to traditional speculative decoding and early exiting approaches such as ~\cite{medusa, schuster2022confident, Tang2024}. However, modern accelerators demonstrate compute bandwidth that exceeds memory access bandwidth by an order of magnitude or more~\cite{databricksLLMInference2023, jouppi2021ten}, meaning increased FLOPs do not necessarily translate to increased decoding latency. Nevertheless, to ensure fair comparison and efficiency in compute bound scenarios, we introduce targeted optimizations.

~\textbf{Attention FLOPs Optimization} For medium-to-long context lengths, attention computation dominates FLOPs in the self-attention layer, surpassing the contribution from MLP layers. Specifically, matrix multiplications involving queries, cached keys, and cached values scale with $l_{kv} * l_{q}$ where $l_{kv}$ denotes previous context length and $l_q$ denotes current query length. Since M2R2 pairs accelerated residuals with slow residuals, a naive implementation results in twice the FLOPs consumption compared to a standard attention layer. To address this, we limit the attention of accelerated residual stream to selectively attend to the top-k most relevant tokens, identified by the slow residual stream based on top attention coefficients\footnote{We set to k = 64 and attend to top 64 tokens as identified by the slow residual stream.}. This is possible since slow and accelerated residual streams are processed in same forward pass and accelerated streams have access to attention coefficients of slow stream. Note that, the faster residual stream still retains the flexibility to assign distinct attention coefficients to these tokens. Furthermore, we design the faster residual stream to employ only 8 attention heads, compared to the 32 heads used in the slow residual stream of the Phi-3 model, reducing query, key, value, and output projection FLOPs by a factor of 1/4. ~\cref{fig:m2r2_num_heads_ablation} indicates effect of using a slicker stream on alignment. As depicted, using $\hat{n}_h = 8$ offers a good trade-off between alignment and FLOPs overhead. 

~\textbf{MLP FLOPs Optimization} The accelerator adapters operating on the accelerated residual stream are intentionally designed with lower rank than their counterparts in the base model. This reduces FLOP overhead by a factor proportional to $hiddenSize / rank$. Additionally, since the faster residual stream uses only 8 attention heads (compared to 32 in the slow residual stream of Phi-3), the subsequent MLP layers process a smaller set of activations, further reducing FLOPs by another factor of 1/4.

These optimizations significantly reduce the FLOP overhead per speculative draft generation, as illustrated in ~\cref{fig:flops_optmization}. Notably, while traditional early-exiting speculative approaches such as DEED require propagating the full slow residual state through the initial layers, incurring substantial computational costs, M2R2 achieves efficient token generation via slimmer, low-rank faster residual streams. In contrast, Medusa introduces considerable FLOP overhead due to per-head computations scaling with $d^2+dv$\footnote{Here $d$ denotes hidden state dimension while $v$ denotes vocab size.}, whereas M2R2 employs low-rank layers for both MLP and language modeling heads, maintaining computational efficiency. All experiments involving the M2R2 approach, as detailed in ~\cref{sec:experiments}, are conducted using these FLOPs optimizations.









% \[
% O_t^{E_j} = 
% \begin{cases} 
% 1, & \text{if } L(\hat{y}_t^{E_j}) = y_t^{E_j} \\
% 0, & \text{otherwise}
% \end{cases}
% \]




%add distillation
% We train accelerator adapters described in \cref{m2r2_method} to accelerate residual streams on next token prediction all in parallel since there are no gradient conflict issues as described in \cref{sec:grad_conflict}.

% \begin{align} \label{eq:mr_loss}
% L_{mr} =  & -\sum_{j = 1}^{E_n} (\sum_{t=1}^{T}\log p_{\theta} (\hat{y}_t^{E_j} | \hat{y}_{<t}, x)) \nonumber
% \end{align}

% where $\hat{y_t^{E_j}}$ denotes predicted logits obtained from accelerated residual stream at gate $E_j$ and time-step $t$ while $y_t^{E_j}$ denotes corresponding truth tokens. 

% Upon training of adapters responsible for accelerating residual streams, we train query, key, value parameters responsible for vertical latent attention of all gates in parallel as

% \begin{equation} \label{eq:arla_loss}
%     L_{arla} = -\frac{1}{N} (\sum_{t=1}^{T}(1\sum_{j=1}^{E_n} \left[ O_t^{E_j} \log(\hat{O}_t^{E_j}) + (1 - o_t^{E_j}) \log(1 - \hat{o_t}_{E_j}) \right]))
% \end{equation}

% where $\hat{O_t^{E_j}}$ denotes binary predicted logits obtained from vertical latent attention router described in \cref{sec:arla} at gate $E_j$ and timestep $t$ while $O_t^{E_j}$ denotes corresponding truth label. Truth labels for router are obtained by computing whether logit head application on $\hat{y}_t^j$ results in true next token prediction. Formally speaking, 

% $O_t^{E_j} = 1 if L(\hat{y_t^{E_j}}) == y_t^{E_j} , 0 otherwise$. 

% Parameters responsible for vertical latent attention are also trained in parallel as well. 

%todo: training slow and fast residuals together and distillation can be two training mdoes. 
%Distillation can be an ablation. 




% Although transformer decoding is memory bound on most mainstream accelerators, there could be scenarios where flop savings are crucial. For instance, on on-device settings power consumption is directly correlated with flops per decoding step and reducing flops does help with overall energy consumption. Vanilla early exiting methods help with flop reduction but suffer from mismatch between training and inference due to early exited tokens. If token at decoding step $t$, $T_t$ exited at layer $E_i$, while token $T_{t+k}$ exits at layer $E_j$ such that $E_i < E_j$, hidden state $H_{t+k}l$ does not have corresponding hidden state $H_tl$ to attend to where $E_i < l <= E_j$. One solution that's often used in literature is to rely on last hidden state available, $H_t{E_j}$, however it tends to be sub-optimal and does affect generation quality \cite{ref}.  To alleviate this mismatch while reducing flops, we train router such that attention mask between token $T_{t+k}$ and token $T_{<t+k}$ is given by: 

% \begin{equation}
%     a_{T_{{t+k}{T_{<t+k}}} = 1 if  E_{T_{<t+k}} >= E{T_{t+k}}
%     else 0
% \end{equation}

% This attention mask enables router to account for exited tokens and get trained accordingly. Since attention mechanism during decoding remains exactly same as that during training, impact on generation quality tends to be minimal as noted in \cref{fig:gen_auality_with_and_without_recompute_attention_show_flops}.  Although MoD does not suffer from training and inference mismatch, we observe that it suffers from discountinuity between pre-training and super-vised fine-tuning resulting in sub-optimal perplexity. On the other hand, our method doesn't not require pre-training , doesn't suffer from discountinuity, and achieves much better perplexity in super-vised fine-tuning and instruction tuning setups as shown in \cref{fig:Mod_vs_m2r2_loss_curves}.






% Our techniques are directly applicable in such scenarios.    




%expert loading with cuda streams in experiments
\section{Experiments}\label{sec:expts}
%! TEX Root = ../main.tex

First, we provide details about our architecture and training. 
Then, for each task, we discuss the baselines, the evaluation metrics and datasets, and show quantitative and qualitative results.
We also present an ablation study that evaluates the contribution of different components of our approach.
Finally, we show a way to extend our approach to new tasks by using motion-based segmentation as an example.

\subsection{Implementation}\label{sec:impl}
\section{Implementation Environment}
\label{sec:implementation_environment}

Here we introduce the detailed implementation details and environment for reproducibility purpose. For our model, we choose hyperparameters based on the performance on validation set (Document classification task in the main paper explains how we split validation set). The results in the main paper are obtain by 5 independent runs. The standard deviations reported in the main paper are 1-sigma error bars and are obtained by calling its corresponding function in Excel library. All the experiments were done on Linux server with an NVIDIA A40 GPU with 46,068 MiB. Its operating system is CentOS Linux 7 (Core). We implemented our proposed model GTFormer using Python 3.10 as programming language and PyTorch 2.0.0 as deep learning library. Other frameworks include NumPy 1.23.1, sklearn 0.23.2, and scipy 1.5.2. We emphasize that the main focus of our model is effectiveness, instead of running efficiency. But for completeness, we still make a short comment on execution time. Our model is efficient, on the largest dataset Web, the training takes less than 40 hours to converge. We will release code and datasets upon publication.


%! TEX Root = ../main.tex
\begin{table}
\begin{center}
    \resizebox{\columnwidth}{!}{
    \begin{tabular}{rccccccccccc}
        \toprule
        & \multicolumn{2}{c}{Sintel ($\sim$50 frames)} & \multicolumn{2}{c}{ScanNet} (90 frames) & \multicolumn{2}{c}{KITTI} ($\sim$110 frames) & \multicolumn{2}{c}{Bonn} (110 frames) & \multicolumn{2}{c}{NYUv2} (1 frame)\\
        \cmidrule(lr{0.1em}){2-3}\cmidrule(lr{0.1em}){4-5}\cmidrule(lr{0.1em}){6-7}\cmidrule(lr{0.1em}){8-9}\cmidrule(lr{0.1em}){10-11}
        & AbsRel $\downarrow$ & $\delta_1$ $\uparrow$ & AbsRel $\downarrow$ & $\delta_1$ $\uparrow$ & AbsRel $\downarrow$ & $\delta_1$ $\uparrow$ & AbsRel $\downarrow$ & $\delta_1$ $\uparrow$ & AbsRel $\downarrow$ & $\delta_1$ $\uparrow$  \\
        \midrule
        Marigold~\cite{ke2024repurposing} & $0.532$ & $0.515$ & $0.166$ & $0.769$ & $0.149$ & $0.796$ & $0.091$ & $0.931$ & $0.070$ & $0.946$  \\
        DA~\cite{yang2024depthanything} & $0.325$ & $0.564$ & $0.130$ & $0.838$ & $0.142$ & $0.803$ & $0.078$ & $0.939$ & $\mathbf{0.042}$ & $\mathbf{0.981}$  \\
        DA-V2~\cite{yang2024depthanythingv2} & $0.367$ & $0.554$ & $0.135$ & $0.822$ & $0.140$ & $0.804$ & $0.106$ & $0.921$ & $\underline{0.043}$ & $\underline{0.978}$  \\
        \midrule
        NVDS~\cite{wang2023nvds} & $0.408$ & $0.483$ & $0.187$ & $0.677$ & $0.253$ & $0.588$ & $0.167$ & $0.766$ & $0.151$ & $0.780$  \\
        ChronoDepth~\cite{shao2024chronodepth}  & $0.587$ & $0.486$ & $0.159$ & $0.783$ & $0.167$ & $0.759$ & $0.100$ & $0.911$ & $0.073$ & $0.941$  \\
        DepthCrafter~\cite{hu2024depthcrafter} & $0.270$ & $\mathbf{0.697}$ & $0.123$ & $0.856$ & $0.104$ & $0.896$ & $0.071$ & $\underline{0.972}$ & $0.072$ & $0.948$  \\
        \midrule
        \methodName-depth (Ours) & $\underline{0.251}$ & $0.659$ & $0.102$ & $0.895$ & $0.099$ & $0.916$ & $0.061$ & $\underline{0.972}$ & $0.078$ & $0.932$  \\
        \methodName-depth* (Ours) & $0.267$ & $\underline{0.693}$ & $\mathbf{0.070}$ & $\mathbf{0.954}$ & $0.097$ & $0.903$ & $\mathbf{0.057}$ & $\underline{0.972}$ & $0.078$ & $0.932$  \\
        \methodName (Ours) & $0.263$ & $0.662$ & $0.103$ & $0.898$ & $\underline{0.093}$ & $\underline{0.924}$ & $0.060$ & $\mathbf{0.973}$ & $0.081$ & $0.925$  \\
        \methodName* (Ours) & $\mathbf{0.247}$ & $0.691$ & $\underline{0.072}$ & $\underline{0.951}$ & $\mathbf{0.090}$ & $\mathbf{0.928}$ & $\underline{0.058}$ & $\mathbf{0.973}$ & $0.081$ & $0.925$  \\
        \bottomrule
    \end{tabular}
    }
\end{center}
\caption{\textbf{Zero-shot depth estimation results.} 
We compare our methods against both single-image baselines (Row 1-3) and SOTA video depth estimation approaches (Row 4-6). 
\methodName-depth refers to our model trained specifically for depth estimation and \methodName refers to the version trained jointly for all tasks. 
Models marked * have predictions in overlapping windows aligned using the strategy described in Section~\ref{sec:method_dense}.
On video datasets (all except NYUv2), our model consistently performs better than DepthCrafter, the closest competition, and by a large margin on ScanNet and KITTI. 
\textbf{Best} and \underline{second best} results are highlighted.
}\label{tab:depth}
\end{table}



\begin{figure}
    \centering
    \includegraphics[width=0.99\columnwidth]{figures/depth_results/depth_figure.pdf}
    \caption{
        \textbf{Qualitative results for depth estimation.}
        We include one example each from Bonn, KITTI, and ScanNet. 
        Inference is conducted on 16-frame clips, but only 1 frame is shown. 
        }\label{fig:depth}
\end{figure}

\subsection{Video Depth Estimation}


We follow DepthCrafter~\cite{hu2024depthcrafter} and evaluate video depth estimation on a collection of five datasets.
We do not use any of the datasets for training our models or the baselines to better understand their generalization abilities.
There is an inherent scale-ambiguity in the estimated depthmaps. 
We follow the common practice of aligning linearly the estimation with the GT before calculating evaluation metrics.
The alignment is done for all the frames at once, and is carried out in \emph{disparity} space via least-square fitting.
For comparison on single image datasets, we repeat the single frame 16 times to compute our estimations.
We report two metrics: AbsRel ($\text{mean}(|\hat{\mathbf{d}}-\mathbf{d}| / \mathbf{d}))$ and $\delta_1$ (ratio of pixels satisfying $\max(\mathbf{d}/\hat{\mathbf{d}}, \hat{\mathbf{d}}/\mathbf{d})<1.25$), where $\mathbf{d}$ represents GT, and $\hat{\mathbf{d}}$ is depth estimation after alignment. 
We upsample our estimations from $224\times224$ to each dataset's original resolution for evaluation.

We consider video approaches including NVDS~\cite{wang2023nvds}, ChronoDepth~\cite{shao2024chronodepth}, DepthCrafter~\cite{hu2024depthcrafter}, 
as well as single-image ones, including Marigold~\cite{ke2024repurposing} and DepthAnything~\cite{yang2024depthanything,yang2024depthanythingv2}. 
Among them, DepthCrafter~\cite{hu2024depthcrafter} and DepthAnything~\cite{yang2024depthanything,yang2024depthanythingv2} each represent the SOTA respectively. 
Marigold and DepthCrafter are diffusion models, which afford impressive levels of details, but require an expensive iterative denoising process. 

Our results show consistent advantages over both SOTA single-image and video depth approaches on the four video datasets (Table~\ref{tab:depth}). 
Since \methodName is a video approach, applying it on single images from NYUv2 does not provide the necessary temporal context for it to perform well. 
DepthCrafter also similarly suffers on NYUv2. 
Figure~\ref{fig:depth} shows qualitative samples and comparison with select SOTA approaches. 
\methodName produces a level of details comparable to that of diffusion models such as DepthCrafter, while generally capturing more accurate relative scales.

\noindent
\textbf{Discussion.} Our final model performs on par with a specialized depth model (Table~\ref{tab:depth}), 
despite optimized jointly for all of our tasks. 
Scale alignment between windows for online inference makes a significant impact on ScanNet. 
This is due to the fast-paced view change in ScanNet samples making scale inconsistency between windows more prominent. 
It is also worth noting that \methodName performs competitively on KITTI, despite not fine-tuned on synthetic datasets that include driving scenarios.

%! TEX Root = ../main.tex



\begin{table}
    \begin{center}
        \resizebox{\columnwidth}{!}{
        \begin{tabular}{rccccccc}
            \toprule
            & \multicolumn{2}{c}{Kubric} & \multicolumn{2}{c}{Dynamic Replica} & \multicolumn{2}{c}{Spring} \\
            \cmidrule(lr{0.1em}){2-3}\cmidrule(lr{0.1em}){4-5}\cmidrule(lr{0.1em}){6-7}
            & $EPE\downarrow$ & $EPE<1\uparrow$ & $EPE\downarrow$ & $EPE<1\uparrow$ & $EPE\downarrow$ & $EPE<1\uparrow$ \\
            \midrule
            RAFT*~\cite{teed2020raft}  & $0.31$ & $94.6$ & $0.14$ & $98.7$ & $0.13$ & $98.4$ \\
            MemFlow~\cite{dong2024memflow} &  $0.27$ & $95.6$ & $0.11$ & $99.2$ & $0.13$ & $98.4$ \\
            Ours &  $(0.13)$ & $(97.6)$ & $(0.03)$ & $(99.9)$ & $\mathbf{0.10}$ & $\mathbf{98.5}$ \\
            \bottomrule
        \end{tabular}
        }
    \end{center}
    \caption{
        \textbf{Optical flow estimation results.} 
        We evaluate on validation sets of all datasets. 
        Our model has seen Kubric and Dynamic Replica during training (numbers in brackets). 
        All others show cross-dataset generalization.
        On Spring, we surpass both two-frame (marked with *) and multi-frame SOTA baselines, despite the latter having specially designed architectures and a complex memory mechanism. 
    }
    \label{tab:flow}
\end{table}




\begin{figure}
    \centering
    \includegraphics[width=0.99\columnwidth]{figures/flow_results/flow_figure.pdf}
    \caption{
        \textbf{Qualitative results for optical flow estimation on Spring.}
        Our results compare favorably to baselines in terms of both details and accuracy. 
        Inference is conducted on 16-frame clips, but only 1 frame is shown. 
        }\label{fig:flow}
\end{figure}

\subsection{Multi-Frame Optical Flow Estimation}

We use the Spring dataset~\cite{mehl2023spring} for evaluation.
We sample 289 16-frame clips from the \emph{train} split.
Spring is not used to train ours or other approaches we compare against, allowing us to evaluate generalization ability.
The input frames are resized to $224\times224$ for all evaluation.
We use the Endpoint Error (EPE), as well as a more robust metric, ratio of EPE $<1$, for the evaluation. 

We consider two baselines for comparison. 
RAFT~\cite{teed2020raft}, a competitive and widely used two-frame approach, creates dense pairwise pixel features and uses recurrent updates to estimate optical flow. 
MemFlow~\cite{dong2024memflow}, a recently published work, ranks among the top methods on the Spring benchmark.
It is a multi-frame approach that relies on a memory mechanism to leverage temporal context. 
Quantitatively, \methodName compares favorably to both RAFT and MemFlow on Spring (Table~\ref{tab:flow}).
Our model can capture well both small and large motions and presents more precise motion boundaries (Figure~\ref{fig:flow}). 
In addition, multi-frame approaches like MemFlow and ours generally have an edge in temporal stability (see Supplementary). 
Unlike many specialized approaches, our model currently only operates on low-resolution videos and further work is needed to enable efficient high-res estimation.

%! TEX Root = ../main.tex

\begin{table}
    \begin{center}
        
    \resizebox{\columnwidth}{!}{
        \begin{tabular}{rccccccc}
            \toprule
            & Aria & DriveTrack & PStudio & \multicolumn{3}{c}{\textbf{Overall}} \\
            \cmidrule(lr{0.1em}){2-4}\cmidrule(lr{0.1em}){5-7}
            & 2D-AJ $\uparrow$ & 2D-AJ $\uparrow$ & 2D-AJ $\uparrow$
            & 2D-AJ $\uparrow$ & APD $\uparrow$ & OA $\uparrow$ \\
            \midrule
            TAPIR~\cite{doersch2023tapir}         & $48.6$ & $57.2$ & $48.7$ & $53.2$ & $67.4$ & $80.5$ \\
            BootsTAPIR~\cite{doersch2024bootstap} & $54.7$ & $\mathbf{62.9}$ & $\mathbf{52.4}$ & $\mathbf{59.1}$ & $\mathbf{74.7}$  & $85.6$ \\
            CoTracker~\cite{karaev2023cotracker}  & $54.2$ & $59.8$ & $51.0$ & $57.2$ & $74.2$ & $84.5$ \\
            \midrule
            Ours (2D Only)                       & $\mathbf{56.7}$ & $54.2$ & $49.8$ & $53.5$ & $69.4$  & $88.6$  \\
            Ours (w/o Mem)                          & $36.8$ & $47.4$ & $41.1$ & $41.8$ & $62.9$ & $78.6$  \\
            Ours                                  & $53.0$ & $51.6$ & $48.8$ & $51.2$ & $67.0$ & $\mathbf{88.7}$  \\
            \toprule
        \end{tabular}
    }
    \end{center}
    \caption{\textbf{Evalution of 2D point tracking on TAPVid-3D.} 
    2D GT trajectories are obtained by projecting 3D GT trajectories onto 2D. 
    Though behind 2D SOTA approaches, our model performs competitively once trained specifically for 2D tracking (``2D Only").}
    \label{table:tapvid3d2dtrack}
\end{table}


%! TEX Root = ../main.tex

\begin{table}
    \begin{center}
        \resizebox{\columnwidth}{!}{
        \setlength{\tabcolsep}{3pt}
        \begin{tabular}{rccccccccccccc}
            \toprule
            & \multicolumn{3}{c}{Aria} & \multicolumn{3}{c}{DriveTrack} & \multicolumn{3}{c}{PStudio} & \multicolumn{3}{c}{\textbf{Overall}} \\
            \cmidrule(lr{0.1em}){2-4}\cmidrule(lr{0.1em}){5-7}\cmidrule(lr{0.1em}){8-10}\cmidrule(lr{0.1em}){11-13}
            & 3D-AJ $\uparrow$ & APD $\uparrow$ & OA $\uparrow$ & 3D-AJ $\uparrow$ & APD $\uparrow$ & OA $\uparrow$ & 3D-AJ $\uparrow$ & APD $\uparrow$ & OA $\uparrow$
            & 3D-AJ $\uparrow$ & APD $\uparrow$ & OA $\uparrow$ \\
            \midrule
            Static Baseline       & $4.9$ & $10.2$ & $55.4$ & $3.9$ & $6.5$ & $80.8$ & $5.9$ & $11.5$ & $75.8$ & $4.9$ & $9.4$ & $70.7$ \\
            TAPIR + CM        & $7.1$ & $11.9$ & $72.6$ & $8.9$ & $14.7$ & $80.4$ & $6.1$ & $10.7$ & $75.2$ & $7.4$ & $12.4$ & $76.1$ \\
            CoTracker + CM    & $8.0$ & $12.3$ & $78.6$ & $11.7$ & $\mathbf{19.1}$ & $81.7$ & $8.1$ & $13.5$ & $77.2$ & $9.3$ & $15.0$ & $79.1$ \\
            BootsTAPIR + CM   & $9.1$ & $14.5$ & $78.6$ & $\mathbf{11.8}$ & $18.6$ & $83.8$ & $6.9$ & $11.6$ & $81.8$ & $9.3$ & $14.9$ & $81.4$ \\
            \midrule
            TAPIR + ZD      & $9.0$ & $14.3$ & $79.7$ & $5.2$ & $8.8$ & $81.6$ & $10.7$ & $18.2$ & $78.7$ & $8.3$ & $13.8$ & $80.0$ \\
            CoTracker + ZD  & $10.0$ & $15.9$ & $87.8$ & $5.0$ & $9.1$ & $82.6$ & $11.2$ & $19.4$ & $80.0$ & $8.7$ & $14.8$ & $83.4$ \\
            BootsTAPIR + ZD & $9.9$ & $16.3$ & $86.5$ & $5.4$ & $9.2$ & $85.3$ & $11.3$ & $19.0$ & $82.7$ & $8.8$ & $14.8$ & $84.8$ \\
            TAPIR-3D              & $2.5$ & $4.8$  & $86.0$ & $3.2$ & $5.9$ & $83.3$ & $3.6$ & $7.0$ & $78.9$ & $3.1$ & $5.9$ & $82.8$ \\
            SpatialTracker        & $9.9$ & $16.1$ & $89.0$ & $6.2$ & $11.1$ & $83.7$ & $10.9$ & $19.2$ & $78.6$ & $9.0$ & $15.5$ & $83.7$ \\
            Ours (w/o Mem)          & 8.2 & 15.4  & 72.7  & 5.5   & 10.0 & 83.3 & $15.1$ & $25.2$ & $79.9$ & $9.6$ & $16.9$ & $78.6$ \\
            Ours                  & $\mathbf{11.2}$ & $\mathbf{17.7}$  & $\mathbf{90.3}$  & $6.6$   & $11.4$ & $\mathbf{88.1}$ & $\mathbf{18.6}$ & $\mathbf{28.2}$ & $\mathbf{87.6}$ & $\mathbf{12.1}$ & $\mathbf{19.1}$ & $\mathbf{88.7}$ \\
            \bottomrule
        \end{tabular}
        }
    \end{center}
    \caption{
        \textbf{Evaluation of 3D tracking on the \textit{full\_eval} split of TAPVid-3D.} 
        The top approaches combine 2D point tracking approaches with COLMAP (CM)~\cite{schoenberger2016sfm}, while the bottom ones, including ours are feedforward.
        Our approach consistently outperforms previous feedforward works, and also COLMAP baselines on average.
        We also show the impact of our memory mechanism (Ours vs. Ours w/o Mem). ``ZD" refers to ZoeDepth.}
    \label{table:tapvid3d3dtrack}
\end{table}



\begin{figure*}
    \centering
    \includegraphics[width=0.99\textwidth]{figures/track_results/track_figure.pdf}
    \caption{\textbf{Qualitative results of Sparse 2D/3D tracking on the TAPVid-3D benchmark.}
    Comparison with SpaTracker, a SOTA 3D tracking approach, demonstrates the superior quality of our 2D and 3D tracks.
    For joint visualization of depth and 3D tracks, we align them using median scaling.
    We use our depth maps for visualization of GT and for SpaTracker we use the ones used by their approach. 
    }\label{fig:track}
\end{figure*}

\subsection{Sparse 2D/3D Track Estimation}
We evaluate on TAPVid-3D~\cite{koppula2024tapvid3d}, a benchmark containing around 2.1M long-range 3D point trajectories from over 4000 real-world videos, covering a variety of objects, camera and object motion patterns, and indoor and outdoor environments.
It consists of three datasets: Aria~\cite{pan2023aria}, DriveTrack~\cite{sun2020waymodataset}, and PStudio~\cite{joo2015pstudio}.
It introduced several baselines by combining SOTA 2D point tracking approaches, such as TAPIR~\cite{doersch2023tapir}, BootsTAPIR~\cite{doersch2024bootstap}, and CoTracker~\cite{karaev2023cotracker}, with depth solutions like ZoeDepth~\cite{bhat2023zoedepth}, a monocular depth estimation approach, and COLMAP~\cite{schoenberger2016sfm,schoenberger2016mvs}, a structure-from-motion pipeline.
The top performing approach on the benchmark is SpaTracker~\cite{spatracker}.

The benchmark evaluates both 3D and 2D tracking approaches, and uses metrics that measure the ability to predict point visibility using an occlusion accuracy metric (OA), the accuracy of predicted trajectories in the visible regions (APD), and joint occlusion and geometric estimation (AJ).
To resolve the scale ambiguity in depth estimation, the benchmark uses global median scaling by computing the median of the depth ratios between the estimated and ground-truth 3D tracks over all the points and frames in a video.
We use the \textit{full\_eval} split evaluation numbers provided in the TAPVid-3D benchmark for comparing approaches.

On 3D tracking, we outperform previous approaches on average across all the metrics (Table~\ref{table:tapvid3d3dtrack}).
Among feedforward approaches, we perform better on all the datasets.
Approaches that combine 2D track estimation with COLMAP perform better on the DriveTrack~\cite{sun2020waymodataset} dataset.
This could be due to a relatively large bias of tracking mostly static vehicles, where COLMAP gives much more accurate depth.
Such COLMAP-based baselines, however, perform poorly on Aria~\cite{pan2023aria} and PStudio~\cite{joo2015pstudio}, which are mostly dynamic.
We show qualitative evaluation against the SOTA SpaTracker approach in Figure~\ref{fig:track}.

On 2D tracking, we are slightly behind the SOTA 2D tracking approaches (Table~\ref{table:tapvid3d2dtrack}).
Our approach becomes more competitive and performs better than TAPIR on average when we fine-tune our model only for the 2D tracking task.
We believe our reduced performance on 2D tracking comes from working at lower image resolution, $224\times224$ for us as compared to $384\times512$ for CoTracker and $256\times256$ for others, and a lack of task-specific tricks, like tracking multiple points together (CoTracker) or assuming access to all frames in the video and performing a global track-refinement (TAPIR and BootsTAPIR), both of which could also benefit our tracking head.
We also ablate our online tracking approach on both 2D and 3D tracking benchmarks, and show improved performance due to the use of memory mechanism when tracking points from one window to next.
Overall, we attribute our strong performance to our unified approach and carefully designed sparse head.



\subsection{Ablations}\label{sec:ablation}
To understand the contribution of different components of our approach, we perform an ablation study for depth, flow, 2D and 3D point tracking, as shown in Table~\ref{tab:ablation}.
For each of these tasks, we report average over the datasets not used in our training: for depth we use datasets in Table~\ref{tab:depth}, for optical flow we use the Spring dataset, and for tracking we use the \textit{minival} split from the TAPVid-3D~\cite{koppula2024tapvid3d} benchmark.
Our main contribution is to show how to leverage the priors of a pretrained VideoMAE for multiple dense and sparse low-level 4D perception tasks at once.
To show the usefulness of our end-to-end fine-tuning strategy, we compare against a pretrained and frozen VideoMAE, where we only fine-tune the task-specific heads. 
Table~\ref{tab:ablation} shows that our fine-tuned VideoMAE (row 3) produces better results than the pretrained and frozen VideoMAE across all tasks (row 2). 
A version trained end-to-end from scratch results in worse performance (row 1), which shows that our system leverages the pretraining of the VideoMAE.
Finally, by adding the proposed memory mechanism for the tracking head and using our two-stage training process, we obtain improvements in both 2D and 3D tracking tasks, while maintaining the performance on other tasks.

\begin{table}
    \centering
    \resizebox{.95\columnwidth}{!}{
      \begin{tabular}{l|cccc}
        \multicolumn{1}{c|}{} & \begin{tabular}[c]{@{}c@{}}Depth\\ AbsRel$\downarrow$ / $\delta_1\uparrow$\end{tabular} & \begin{tabular}[c]{@{}c@{}}Optical flow\\ EPE$\downarrow$ / EPE$<1\uparrow$\end{tabular} & \begin{tabular}[c]{@{}c@{}}2D Track\\ 2D-AJ$\uparrow$\end{tabular} & \multicolumn{1}{c}{\begin{tabular}[c]{@{}c@{}}3D Track\\ 3D-AJ$\uparrow$\end{tabular}} \\ \cline{1-5}
        \multicolumn{1}{l|}{From scratch} & 0.259 / 0.594 & 0.246 / 96.2 & 16.6 & \multicolumn{1}{c}{1.3} \\
        \multicolumn{1}{l|}{VideoMAE frozen} & 0.137 / 0.841 & 0.120 / 98.2 & 29.3 & \multicolumn{1}{c}{3.3} \\
        \multicolumn{1}{l|}{Ours (w/o Mem)} & \textbf{0.120 / 0.876} & \textbf{0.100 / 98.5} & 41.1 & \multicolumn{1}{c}{8.7} \\
        \multicolumn{1}{l|}{Ours} & \textbf{0.120 / 0.876} & \textbf{0.100 / 98.5} & \textbf{50.2} & \multicolumn{1}{c}{\textbf{10.8}} \\
      \end{tabular}
    }
    \caption{\textbf{Ablation study.} Training using pre-trained VideoMAE performs better than training from scratch (row 3 vs. 1), which shows our approach leverages VideoMAE priors. Our approach performs better than using a frozen VideoMAE and only fine-tuning the heads (row 3 vs. 2), which shows end-to-end fine-tuning helps. Adding memory mechanism and two-stage training strategy improves tracking performance while maintaining performance on other tasks (row 4 vs. 3).}
    \label{tab:ablation}
\end{table} 



\subsection{Additional Task: Motion-based Segmentation}
We use the motion-based segmentation task to show one way to add a new task to our network.
We do this simply by freezing our trained video encoder and fine-tuning our proposed dense head for this task.
We generate the ground-truth annotations for training and evaluation by using \emph{video} datasets that provide camera, depth and 3D-motion information.
For training, we use the Kubric~\cite{greff2022kubric} dataset and fine-tune using binary cross entropy loss.
For evaluation, we use the Virtual KITTI (VKITTI)~\cite{cabon2020vkitti2} and Spring~\cite{mehl2023spring} datasets.
We compare against RigidMask (RM)~\cite{yang2021rigidmotion}, a SOTA two-frame rigid-motion segmentation approach that combines dynamic motion signals from flow, optical-expansion~\cite{yang2020upgrading} and depth.
It is trained on the SceneFlowDatasets~\cite{mayer2016scenflow}; however, they also train a version for driving scenarios (RM-Drive).
To evaluate, we report foreground IoU (higher is better) on VKITTI and Spring.
\setlength{\columnsep}{2pt}
\setlength{\intextsep}{0pt}
\begin{wrapfigure}[3]{r}[5pt]{.41\columnwidth}
  \resizebox{0.4\columnwidth}{!}{
    \begin{tabular}{c|c|c}
      & VKITTI & Spring \\
    \hline
    RM & 32.6 & 16.5\\ 
    RM-Drive & 35.4 & 8.5\\ 
    Ours & {\bf 46.7} & {\bf 23.7}
    \end{tabular}
    }
\end{wrapfigure}
On both datasets, our video-based approach achieves better performance.
Note that while fine-tuning on driving scenes allows RigidMask (RM-Drive) to reduce the gap slightly on VKITTI, it significantly hurts performance on Spring, highlighting the benefit of our model's generalization ability.
As shown in Figure~\ref{fig:dyn}, for both the indoor scenarios with human-object interactions and the outdoor driving scenarios, our approach performs better and can detect small motions (see more comparisons in Supplementary).
Freezing the video encoder and fine-tuning a task-specific head is the simplest way to add a new task that does not affect the performance of other tasks we train for.
Better strategies may exist that allow for some fine-tuning of the video encoder without affecting the performance of other tasks, though the investigation is outside the scope of this paper.

\begin{figure}
  \centering
  \includegraphics[width=0.99\columnwidth]{figures/dynamic/dyn_seg.pdf}
  \caption{\textbf{Qualitative results of motion-based segmentation.} 
  Samples are chosen from the TAPVid-3D benchmark.
  Across various scenarios, ours show advantages in small motions, boundary accuracy, as well as temporal consistency (see Supp.).
  The GT masks overlaid on images are only provided to identify qualitatively which objects are moving, but are not pixel-accurate.
  }\label{fig:dyn}
\end{figure}
\section{Conclusions}\label{sec:conclusion}
\section*{Conclusion}
This paper aims to enhance our understanding of the computational complexity of computing various Shapley value variants. We found that for various ML models --- including decision trees, regression tree ensembles, weighted automata, and linear regression --- both local and global interventional and baseline SHAP can be computed in polynomial time under HMM modeled distributions. This extends popular algorithms, such as TreeSHAP, beyond their empirical distributional scope. We also establish strict complexity gaps between the various SHAP variants (baseline, interventional, and conditional) and prove the intractability of computing SHAP for tree ensembles and neural networks in simplified scenarios. Overall, we present SHAP as a versatile framework whose complexity depends on four key factors: \begin{inparaenum}[(i)] \item model type, \item SHAP variant, \item distribution modeling approach, \item and local vs. global explanations\end{inparaenum}. We believe this perspective provides deeper insight into the computational complexity of SHAP, paving the way for future work.




%We believe that our framework provides a more intricate understanding of SHAP computation complexity across different models, distributions, and variants, paving the way for further research.

Our work opens promising directions for future research. First, expanding our computational analysis to other SHAP-related metrics, such as asymmetric SHAP~\citep{frye20} and SAGE~\citep{covert2020understanding}, would be valuable. Additionally, we aim to explore more expressive distribution classes and relaxed assumptions beyond those in Section \ref{sec:tractable} while maintaining tractable SHAP computation. Finally, when exact computation is intractable (Section \ref{sec:intractable}), investigating the approximability of SHAP metrics through approximation and parameterized complexity theory~\citep{downey2012parameterized} is an important direction.

%Our work opens several promising avenues for future research on the computational properties of explainable AI methods, with a particular focus on SHAP. First, it would be interesting to broaden the computational analysis conducted in this work to include other popular SHAP-related metrics in the literature, such as asymmetric SHAP \cite{frye20} and SAGE \cite{covert2020understanding}. Also, in the future, we aim to explore more expressive distribution classes and relaxed distributional assumptions—extending beyond those examined in Section \ref{sec:tractable} —that still yield tractable SHAP computation. Finally, when exact computation proves intractable (Section \ref{sec:intractable}), it is worthwhile to theoretically investigate the question of the approximability of computing the SHAP metrics across various configurations, through the lens of approximation and parametrized complexity theory \cite{arora2009computational}.

%This paper aims to deepen our understanding of the computational complexity involved in obtaining different Shapley value variants. We found that for a variety of ML models, including decision trees, tree ensembles for regression, weighted automata, and linear regression models — computing both local and global interventional and baseline SHAP can be done in polynomial time when distributions are modeled by HMMs. This extends the distributional scope of popular algorithms like TreeSHAP, which is limited to empirical distributions. Additionally, we demonstrate a strict complexity gap between SHAP variants, showing that interventional and baseline SHAP can be strictly easier to compute than conditional SHAP. Despite these positive results, we uncovered intractability for various SHAP variants in neural networks and tree ensembles. Finally, we provided generalized complexity relations across SHAP variants. We believe that our framework offers a deeper understanding of the complexity involved in computing SHAP across various variants, models, distributions, as well as in both local and global computations, laying the groundwork for future research.
\section{Acknowledgements}
We would like to thank Jan Kautz for the continuous discussions and for reviewing an early draft of the paper, Zhiding Yu and Hongxu (Danny) Yin for the initial discussions on video models, and Yiqing Liang for help with the data. 

{
    \small
    \bibliographystyle{ieeenat_fullname}
    \bibliography{main}
}


\clearpage
% \setcounter{page}{1}
% \maketitlesupplementary
\begin{center}
Supplementary Material
\end{center}

% {
%     \onecolumn
%     \centering
%     \Large
%     \textbf{\thetitle}\\
%     \vspace{0.5em}Supplementary Material \\
%     \vspace{1.0em}
% }

\section{Proof of \cref{theorem:dr}}
We require some additional regularity assumptions:
\begin{assumption} 1) The number of classes $C$ is bounded w.r.t the number of samples $N$, 2) the missingness mechanism $P(A=1|Y,\theta)$, as well as its estimated counterpart $P(A=1|Y,\theta)$, are bounded below by some constant $\epsilon > 0$, 3) the quantities $P(Y|X,\theta)$ and $P(A|Y,\theta)$ are estimated using auxiliary samples independent of samples used for the sample averaging.
\label{assumption:extra}
\end{assumption}
Assumptions 1 and 2 are natural. For the missingness mechanism, the ground truth being bounded means that there is a non-vanishing proportion of samples for every class. The boundedness of the estimate can be enforced by clipping the estimate. Assumption 3 is called sample splitting in \cite{kennedy-dr}.

For convenience we use operator $\E_N$ to denote the average of $N$ samples i.e. $\frac{1}{N}\sum_{i=1}^N$. Note that this is by itself a random variable, in contrast to $\E$ which is a fixed number.

\begin{proof}[Proof of \cref{theorem:dr}] Because $C$ is bounded (assumption \ref{assumption:extra}), we can fix a class $c$ and prove the theorem.
Let us define the influence function $\phi$, parameterized by $\theta$, as
\begin{equation}
\phi(O | \theta)(c) = P(Y=c|X,\theta) + \frac{\one(A=1)}{P(A=1|Y,\theta)} (\one(Y=c) - P(Y=c|X,\theta)) - P(Y=c)
\end{equation}
As we have done in the main text, we use $\phi(O)$ to denote the same function but all estimated quantities are replaced with their truths. In other words, we use $\phi(O)$ for $\phi(O|\theta_0)$ where $\theta_0$ is the truth, given that our model contains $\theta_0$ e.g. when the model is consistent.

Recall that:
\begin{equation}
\begin{aligned}
\Psi_{dr}(\theta)(c) &= \frac{1}{N}\sum_{i=1}^N \left\{P(Y=c|X,\theta) + \frac{\one(A=1)}{P(A=1|Y,\theta)} (\one(Y=c) - P(Y=c|X,\theta))\right\}\\
&= \E_N [\phi(O|\theta)(c)] + P(Y=c)
\end{aligned}
\end{equation}

We will show that:
\begin{equation}
\Psi_{dr}(\theta)(c) - P(Y=c) = (\E_N - \E)[\phi(O)(c)] + o_P(N^{-1/2})
\label{eq:proof-linearity}
\end{equation}
To do that, we use the following decomposition
\begin{equation}
\begin{aligned}
\Psi_{dr}(\theta)(c) - P(Y=c) &= \E_N [\phi(O|\theta)(c)] \\
&= (\E_N - \E)[\phi(O)(c)] + (\E_N - \E)[\phi(O|\theta)(c) - \phi(O)(c)] + \E[\phi(O|\theta)(c)]
% &+ (\E_n - \E)[\phi(O;\theta) - \phi(O)]\\
% &+ \E[P(Y=c|X,\theta)] - \E[P(Y=c|X)] + \E[\phi(O,\theta)]
\end{aligned}
\end{equation}
and analyze the second and third term. The third term is:
\begin{equation}
\begin{aligned}
\E[\phi(O|\theta)(c)] &= \E[P(Y=c|X,\theta)] + \E\left[\frac{\one(A=1)}{P(A=1|Y,\theta)}(\one(Y=c) - P(Y=c|X,\theta))\right]- P(Y=c) \\
&= \E\left[P(Y=c|X,\theta) + \frac{P(A=1|Y)}{P(A=1|Y,\theta)}(P(Y=c|X) - P(Y=c|X,\theta))\right] - \E[P(Y=c|X)]\\
&= \E\left[(P(Y=c|X,\theta) - P(Y=c|X)) (P(A=1|Y,\theta) -P(A=1|Y)) \frac{1}{P(A=1|Y,\theta)}\right]\\
\end{aligned}
\end{equation}
by Cauchy-Schwarz inequality:
\begin{equation}
\begin{aligned}
\E[\phi(O|\theta)(c)] &\le \frac{1}{\epsilon} \|P(A=1|Y,\theta) - P(A=1|Y)\|_2 \|P(Y=c|X,\theta) - P(Y=c|X)\|_{L_2(P)}\\
&= \frac{1}{\epsilon} o_P(N^{-1/4} N^{-1/4}) = o_P(N^{-1/2})
\end{aligned}
\end{equation}
by assumption \ref{assumption:4th-root-n} and that $P(A=1|Y,\theta) > \epsilon$ (assumption \ref{assumption:extra}). The second term can be bounded by Chebyshev inequality
% \begin{equation}
% \begin{aligned}
% \E[\E_N[\phi(O|\theta)(c) - \phi(O)(c)]] &= \E[\phi(O|\theta)(c) - \phi(O)(c)]\\
% \var[\E_N[\phi(O|\theta)(c) - \phi(O)(c)]] &= \frac{1}{N}\var[\phi(O|\theta)(c) - \phi(O)(c)] \le 
% \end{aligned}
% \end{equation}
\begin{equation}
P(|(\E_N - \E)[\phi(O|\theta)(c) - \phi(O)(c)]| \ge t) \le \frac{\var[\E_N[\phi(O|\theta)(c) - \phi(O)(c)]]}{t^2} = \frac{\var[\phi(O|\theta)(c) - \phi(O)(c)]}{Nt^2}
\end{equation}
note here that $\theta$ is independent of the samples used for $\E_N$ by assumption \ref{assumption:extra}. For any $\varepsilon > 0$, by picking $t = \frac{1}{\sqrt{N\varepsilon}}$ we get
\begin{equation}
P\left(\left|\frac{(\E_N - \E)[\phi(O|\theta)(c) - \phi(O)(c)]}{N^{-1/2}}\right| \ge \frac{1}{\sqrt{\varepsilon}}\right) \le \varepsilon \var[\phi(O|\theta)(c) - \phi(O)(c)]
\end{equation}
by the definition of $O_P$, we then get
\begin{equation}
(\E_N - \E)[\phi(O|\theta)(c) - \phi(O)(c)] = O_P(N^{-1/2}\var[\phi(O|\theta)(c) - \phi(O)(c)])
\end{equation}
Because $\phi$ is a continuous function of $P(Y|X,\theta)$ and $P(A|Y,\theta)$ (given $P(A|Y,\theta) > \epsilon$, assumption \ref{assumption:extra}), by the continuous mapping theorem and the fact that $P(Y|X,\theta)$ and $P(A|Y,\theta)$ are convergent in probability (assumption \ref{assumption:4th-root-n}), we get $\var[\phi(O|\theta)(c) - \phi(O)(c)] = o_P(1)$. This gives
\begin{equation}
(\E_N - \E)[\phi(O|\theta)(c) - \phi(O)(c)] = o_P(N^{-1/2})
\end{equation}
Therefore, we have shown that the second and third term are both $o_P(N^{-1/2})$, proving \cref{eq:proof-linearity}. As the final step, multiply both sides of this equation by $\sqrt{N}$ we get:
\begin{equation}
\sqrt{N}(\Psi_{dr}(\theta)(c) - P(Y=c)) = \sqrt{N} (\E_N - \E)[\phi(O)(c)] + o_P(1) \rightsquigarrow \mathcal{N}(0, \var[\phi(O)(c)])
\end{equation}
by the central limit theorem, and $\var[\phi(O)(c)] = \E[\phi(O)(c)^2]$ because $\E[\phi(O)(c)] = 0$.
\end{proof}

While we started with the definition of $\phi$, \cref{eq:proof-linearity} shows that $\phi$ is indeed an influence function. Now we show that $\phi$ is also the efficient influence function, by using the characterization of the model's tangent space \cite{tsiatis-missingdata}. Note that the joint probability factorizes as $P(X,A,Y) = P(X)P(Y|X)P(A|Y)$, therefore the tangent space $\mathcal{T}$ factorizes as $\mathcal{T} = \mathcal{T}_{X} \oplus \mathcal{T}_{Y|X} \oplus \mathcal{T}_{A|Y}$ where $\mathcal{T}_X = \{h(X): \E[h] = 0\}$, $\mathcal{T}_{Y|X} = \{h(X,Y): \E[h|X] = 0\}$, $\mathcal{T}_{A|Y} = \{h(A,Y): \E[h|Y] = 0\}$, and the 3 subspaces are pairwise orthogonal. All influence functions are orthogonal to the tangent space, but the influence function that is also in the tangent space has the smallest variance and is called the efficient influence function. As $\phi$ is already an influence function, we need only show that $\phi$ is in $\mathcal{T}$. We write $\phi$ as
\begin{equation}
\phi(O)(c) = (P(Y=c|X) - P(Y=c)) + \left[\frac{\one(A=1)}{P(A=1|Y)} - 1\right](\one(Y=c) - P(Y=c|X)) + (\one(Y=c) - P(Y=c|X))
\end{equation}
and note that the first, second and third term are in $\mathcal{T}_X$, $\mathcal{T}_{A|Y}$ and $\mathcal{T}_{Y|X}$ respectively. Therefore, $\phi$ is indeed in $\mathcal{T}$. The efficient influence function has the smallest variance of all influence function, and therefore our estimator being asymptotically linear in $\phi$ (\cref{eq:proof-linearity}) has the smallest mean squared error in a local asymptotic minimax sense \cite{kennedy-dr, asymptoticstatistics}

\section{Further background and related work}
\paragraph{Discussion on semi-supervised EM.}
It appears that semi-supervised EM was first used for parameter estimation when the missingness mechanism is non-ignorable in \cite{ibrahim1996parameter}, but has not been used for label shift estimation.
Perhaps this is because the semi-supervised situation where additional unlabeled data is available during training is rarer than the test-time adaptation case. EM is well suited to take advantage of the extra unlabeled data to improve the classifier under very scarce and long-tailed labeled data. While the connection between pseudo-labeling and EM has been explored before \cite{entropyminimization}, the situation with label shift has not until recently \cite{simpro}. Here the application of EM is much more interesting, because other than simply giving pseudo-labeling a rigorous formulation, EM also estimates the missingness mechanism (equivalently the label distribution shift), which is important for shift correction and thus high-quality pseudo-labels \cite{acr}. The application of confidence thresholding can be seen as a sparse variant of EM \cite{neal1998view}.

\paragraph{The doubly-robust risk.} 
\label{subsec:dr-risk}
A technique that also derives from the theory of semi-parametric efficiency is orthogonal statistical learning \citep{foster2023orthogonal}. The idea is to minimize the doubly-robust risk:
\label{subsec:method-dr-risk}
\begin{equation}
\label{eq:dr-risk}
\mathcal{R}(\theta_2) = \frac{1}{N} \sum_{i=1}^N \Bigg[ l(x_i, \hat y_i|\theta_2) + \frac{\one(a_i=1)}{P(A=a_i|Y=y_i, \theta_1)} (l(x_i, y_i | \theta_2) - l(x_i, \hat y_i | \theta_2))\Bigg]
\end{equation}
where $l(x,y|\theta) = -\sum_{c=1}^C [y]_c \log P(Y=c|X=x,\theta)$ is the negative cross-entropy. 
The notation $[y]_c$ means that we are using the $c$-entry in a C-dimension probability vector $y$. 
Thus, $y_i$ denotes the one-hot label of observation $i$, while $\hat y_i$ denotes the pseudo-label, which can be one-hot or all-zero. 
Finally, we use $\theta_1$ to denote that $P(a|y,\theta_1)$ is an estimation from a previous stage, but it can be estimated with $\theta_2$ as well. 
The risk $\mathcal{R}(\theta_2)$ can be used as a training loss in a straightforward fashion. 
Similar to the doubly robust estimation of $P(Y)$, the doubly robust risk provides approximately unbiased estimation of the risk. 
This property has been used in \citep{arelabelsinformative, onnonrandommissinglabels, drst} also in the semi-supervised learning setting.
More broadly, it is at the heart of one of the core techniques in heterogenous treatment effect estimation in causal estimation \cite{kennedy2023towards, foster2023orthogonal, wager2018estimation}. 
The focus here is not the estimation of $\mathcal{R}(\theta_2)$ per se, but the quality of the learned model \cite{foster2023orthogonal}.
By using the doubly-robust risk, we can achieve an optimality result similar in spirit to our theorem \cref{theorem:dr}, but for the generalization error.
While this is appealing, in practice there are 2 problems with this approach. First, the inverse probability weight $P(A=a_i|Y=y_i,\theta_1)$ can be very large if the class ratio is highly unlabeled, making training unstable \cite{kallus2020deepmatch, pham2023stable}. 
This problem exists for our estimation as well. However, it is much easier to control for estimation than for training because of the iterative nature of model update. Secondly, we can further write $\mathcal{R}$ as:
\begin{equation}
\mathcal{R}(\theta_2) = \frac{1}{N}\sum_{i=1}^N l\left(x_i, \hat y_i + \frac{\one(a_i=1)}{P(A=a_i|Y=y_i,\theta_1)} (y_i - \hat y_i)\Bigg\vert\theta_2\right)
\end{equation}
which is a cross-entropy loss with new meta-pseudo-labels. However, these labels are not meant to be learned exactly, and furthermore they can be negative. Thus, theoretical works have to put stringent assumptions on the models. In \cref{subsec:ablation-1}, we show that experimentally that the instability problem makes doubly-robust risk performance worse than our 2-stage approach.

\section{Training and hyperparameter settings.}
\label{subsec:training-setting}
For neural network training, we follow the implementation and hyperparameter settings of \cite{simpro}. In particular, we adapt the core code of SimPro for Supervised, MLE and EM. For MLE, we update $P(A|Y)$ using the Adam optimizer with learning rate 1e-3, while for EM we use a momentum update similar to SimPro's update of $P(Y|A)$ because it has a a closed-form solution at each mini-batch. We use Wide ResNet-28-2 on all methods and all datasets in this section, including Imagenet-127, because we are motivated by the fact that stage-1's goal is not classification accuracy but the estimation of a finite-dimensional parameter. When using Wide ResNet-28-2 for Imagenet-127, we use the hyperparameters of CIFAR-100, except we lower the batch size of unlabeled data to 2 times that of labeled data instead of 8 for memory reason. We do not perform additional hyperparameter tuning. All experiments can be performed on 1 A6000 RTX GPU, and are run 3 times. We report the total variation distance between the estimated and the ground truth unlabeled class distribution, similar to its usage in Theorem 3.1 of \cite{lsc}, and the top-1 classification accuracy.

In the second stage of our algorithm, we freeze our estimation and plug it in SimPro and BOAT.
We keep exactly the same hyperparameter settings that SimPro and BOAT use. In particular, for Imagenet-127, we now use ResNet-50 and run each experiment once.
In SimPro, we set the unlabeled class distribution $P(Y|A=0)$ at the E-step;  however, we still keep a running estimate of the class distribution $P(Y)$ in the logit adjustment loss \cref{eq:simpro-la-loss}. While it is possible to use the first stage estimate in the logit adjustment loss, we observe that doing so results in lower accuracy than using the the running average. This is conceptually consistent with the role of the running average - serving not as an accurate estimate of $P(Y)$ but to make the classifier's class distribution uniform through the logit adjustment loss, which is good for the test set. Similarly, in BOAT, we only replace $\Delta_c = \log P(Y|A=1) - \log P(Y|A=0)$ in equation (4) of \cite{boat}, which is adjusting a classifier's predictions from the labeled to the unlabeled class distribution, with our SimPro + DR estimate instead of their on-the-fly estimate. 


% \section{Additional experiments}
% % \begin{table*}[t]
\centering
\caption{Total Variation Distance on CIFAR-10-LT ($N_l = 500$, $M_l = 4000$) with different class imbalance ratios $\gamma_l$ and $\gamma_u$ under five different unlabeled class distributions.}
\label{tab:cifar10-tv}
\resizebox{\textwidth}{!}{
\begin{tabular}{lccccccccccc}
\toprule
& & \multicolumn{2}{c}{consistent} & \multicolumn{2}{c}{uniform} & \multicolumn{2}{c}{reversed} & \multicolumn{2}{c}{middle} & \multicolumn{2}{c}{head-tail} \\
\cmidrule(lr){3-4} \cmidrule(lr){5-6} \cmidrule(lr){7-8} \cmidrule(lr){9-10} \cmidrule(lr){11-12}
& & $\gamma_l = 150$ & $\gamma_l = 100$ & $\gamma_l = 150$ & $\gamma_l = 100$ & $\gamma_l = 150$ & $\gamma_l = 100$ & $\gamma_l = 150$ & $\gamma_l = 100$ & $\gamma_l = 150$ & $\gamma_l = 100$ \\
Model & Estimator & $\gamma_u = 150$ & $\gamma_u = 100$ & $\gamma_u = 1$ & $\gamma_u = 1$ & $\gamma_u = 1/150$ & $\gamma_u = 1/100$ & $\gamma_u = 150$ & $\gamma_u = 100$ & $\gamma_u = 150$ & $\gamma_u = 100$ \\
\midrule
Supervised & MLLS & 0.269 ± 0.252 & 0.038 ± 0.006 & 0.251 ± 0.046 & 0.255 ± 0.060 & 0.429 ± 0.028 & 0.493 ± 0.050 & 0.333 ± 0.042 & 0.320 ± 0.009 & 0.457 ± 0.034 & 0.444 ± 0.043 \\
Supervised & RLLS & 0.043 ± 0.001 & 0.044 ± 0.010 & 0.348 ± 0.034 & 0.305 ± 0.068 & 0.769 ± 0.016 & 0.678 ± 0.028 & 0.430 ± 0.008 & 0.368 ± 0.013 & 0.539 ± 0.018 & 0.503 ± 0.020 \\
\midrule
MLE & IPW & 0.027 ± 0.001 & 0.027 ± 0.000 & 0.319 ± 0.072 & 0.243 ± 0.010 & 0.674 ± 0.020 & 0.646 ± 0.041 & 0.438 ± 0.020 & 0.454 ± 0.026 & 0.547 ± 0.049 & 0.491 ± 0.059 \\
MLE & OR & 0.045 ± 0.004 & 0.042 ± 0.000 & 0.215 ± 0.026 & 0.203 ± 0.032 & 0.433 ± 0.017 & 0.395 ± 0.033 & 0.193 ± 0.006 & 0.209 ± 0.037 & 0.307 ± 0.147 & 0.249 ± 0.130 \\
MLE & DR & 0.090 ± 0.002 & 0.079 ± 0.000 & 0.407 ± 0.027 & 0.360 ± 0.007 & 0.425 ± 0.007 & 0.421 ± 0.029 & 0.256 ± 0.001 & 0.286 ± 0.031 & 0.435 ± 0.136 & 0.362 ± 0.122 \\
\midrule
EM & IPW & 0.035 ± 0.002 & 0.040 ± 0.001 & 0.021 ± 0.001 & 0.029 ± 0.015 & 0.303 ± 0.187 & 0.091 ± 0.010 & 0.119 ± 0.011 & 0.105 ± 0.022 & 0.104 ± 0.026 & 0.104 ± 0.051 \\
EM & OR & 0.037 ± 0.003 & 0.042 ± 0.002 & 0.016 ± 0.001 & 0.024 ± 0.012 & 0.269 ± 0.183 & 0.090 ± 0.008 & 0.122 ± 0.012 & 0.103 ± 0.022 & 0.072 ± 0.012 & 0.073 ± 0.024 \\
EM & DR & 0.034 ± 0.004 & 0.037 ± 0.001 & 0.014 ± 0.001 & 0.027 ± 0.020 & 0.264 ± 0.191 & 0.092 ± 0.005 & 0.111 ± 0.019 & 0.097 ± 0.026 & 0.077 ± 0.016 & 0.073 ± 0.028 \\
\midrule
SimPro & IPW & 0.070 ± 0.011 & 0.058 ± 0.000 & 0.046 ± 0.001 & 0.049 ± 0.005 & 0.254 ± 0.074 & 0.223 ± 0.098 & 0.097 ± 0.025 & 0.067 ± 0.002 & 0.105 ± 0.066 & 0.110 ± 0.079 \\
SimPro & OR & 0.071 ± 0.012 & 0.058 ± 0.000 & 0.045 ± 0.001 & 0.049 ± 0.006 & 0.040 ± 0.003 & 0.059 ± 0.017 & 0.074 ± 0.006 & 0.075 ± 0.002 & 0.033 ± 0.003 & 0.033 ± 0.003 \\
SimPro & DR & 0.017 ± 0.004 & 0.026 ± 0.001 & 0.019 ± 0.002 & 0.018 ± 0.003 & 0.039 ± 0.003 & 0.058 ± 0.025 & 0.091 ± 0.007 & 0.031 ± 0.001 & 0.015 ± 0.003 & 0.019 ± 0.007 \\
\bottomrule
\end{tabular}
}
\end{table*}
% 

\begin{table*}[t]
\centering
\caption{Total Variation Distance on CIFAR-100-LT ($N_l = 50$, $M_l = 400$) with different class imbalance ratios $\gamma_l$ and $\gamma_u$ under five different unlabeled class distributions.}
\label{tab:cifar100-tv}
\resizebox{\textwidth}{!}{
\begin{tabular}{lccccccccccc}
\toprule
& & \multicolumn{2}{c}{consistent} & \multicolumn{2}{c}{uniform} & \multicolumn{2}{c}{reversed} & \multicolumn{2}{c}{middle} & \multicolumn{2}{c}{head-tail} \\
\cmidrule(lr){3-4} \cmidrule(lr){5-6} \cmidrule(lr){7-8} \cmidrule(lr){9-10} \cmidrule(lr){11-12}
& & $\gamma_l = 20$ & $\gamma_l = 10$ & $\gamma_l = 20$ & $\gamma_l = 10$ & $\gamma_l = 20$ & $\gamma_l = 10$ & $\gamma_l = 20$ & $\gamma_l = 10$ & $\gamma_l = 20$ & $\gamma_l = 10$ \\
Model & Estimator & $\gamma_u = 20$ & $\gamma_u = 10$ & $\gamma_u = 1$ & $\gamma_u = 1$ & $\gamma_u = 1/20$ & $\gamma_u = 1/10$ & $\gamma_u = 20$ & $\gamma_u = 10$ & $\gamma_u = 20$ & $\gamma_u = 10$ \\
\midrule
Supervised & MLLS & 0.707 ± 0.016 & 0.313 ± 0.100 & 0.445 ± 0.172 & 0.309 ± 0.119 & 0.383 ± 0.075 & 0.397 ± 0.006 & 0.570 ± 0.001 & 0.373 ± 0.107 & 0.543 ± 0.009 & 0.231 ± 0.057 \\
Supervised & RLLS & 0.520 ± 0.007 & 0.133 ± 0.003 & 0.337 ± 0.125 & 0.253 ± 0.082 & 0.424 ± 0.060 & 0.463 ± 0.003 & 0.454 ± 0.021 & 0.306 ± 0.074 & 0.460 ± 0.028 & 0.241 ± 0.040 \\
\midrule
MLE & IPW & 0.075 ± 0.000 & 0.071 ± 0.001 & 0.229 ± 0.001 & 0.167 ± 0.002 & 0.565 ± 0.005 & 0.443 ± 0.007 & 0.415 ± 0.000 & 0.311 ± 0.005 & 0.343 ± 0.000 & 0.280 ± 0.001 \\
MLE & OR & 0.065 ± 0.002 & 0.061 ± 0.001 & 0.200 ± 0.007 & 0.143 ± 0.001 & 0.526 ± 0.011 & 0.399 ± 0.023 & 0.360 ± 0.003 & 0.256 ± 0.012 & 0.328 ± 0.003 & 0.266 ± 0.005 \\
MLE & DR & 0.149 ± 0.019 & 0.145 ± 0.010 & 0.243 ± 0.004 & 0.214 ± 0.019 & 0.568 ± 0.005 & 0.464 ± 0.014 & 0.403 ± 0.014 & 0.309 ± 0.012 & 0.365 ± 0.007 & 0.320 ± 0.004 \\
\midrule
EM & IPW & 0.097 ± 0.008 & 0.092 ± 0.004 & 0.239 ± 0.007 & 0.179 ± 0.003 & 0.478 ± 0.012 & 0.329 ± 0.020 & 0.262 ± 0.016 & 0.202 ± 0.003 & 0.312 ± 0.002 & 0.227 ± 0.001 \\
EM & OR & 0.121 ± 0.007 & 0.108 ± 0.005 & 0.261 ± 0.007 & 0.189 ± 0.004 & 0.489 ± 0.013 & 0.335 ± 0.020 & 0.274 ± 0.016 & 0.211 ± 0.004 & 0.336 ± 0.003 & 0.235 ± 0.001 \\
EM & DR & 0.125 ± 0.005 & 0.111 ± 0.004 & 0.269 ± 0.007 & 0.194 ± 0.005 & 0.497 ± 0.010 & 0.336 ± 0.024 & 0.281 ± 0.019 & 0.219 ± 0.008 & 0.336 ± 0.007 & 0.233 ± 0.004 \\
\midrule
SimPro & IPW & 0.125 ± 0.001 & 0.100 ± 0.005 & 0.166 ± 0.007 & 0.141 ± 0.009 & 0.353 ± 0.023 & 0.261 ± 0.008 & 0.202 ± 0.003 & 0.158 ± 0.005 & 0.277 ± 0.009 & 0.197 ± 0.003 \\
SimPro & OR & 0.133 ± 0.005 & 0.100 ± 0.004 & 0.160 ± 0.007 & 0.138 ± 0.010 & 0.322 ± 0.014 & 0.253 ± 0.008 & 0.202 ± 0.003 & 0.156 ± 0.005 & 0.269 ± 0.006 & 0.191 ± 0.004 \\
SimPro & DR & 0.122 ± 0.003 & 0.106 ± 0.006 & 0.188 ± 0.009 & 0.149 ± 0.006 & 0.343 ± 0.023 & 0.257 ± 0.007 & 0.219 ± 0.010 & 0.172 ± 0.002 & 0.279 ± 0.007 & 0.198 ± 0.004 \\
\bottomrule
\end{tabular}
}
\end{table*}

\end{document}
