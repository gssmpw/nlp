\documentclass[10pt,twocolumn,letterpaper]{article}

\usepackage[pagenumbers]{cvpr}

\usepackage[dvipsnames]{xcolor}
\usepackage[normalem]{ulem}
\usepackage{lipsum}
\usepackage{multirow}
\usepackage[skip=1 pt]{caption} 
\usepackage{tikz}
\usepackage{wrapfig}
\usetikzlibrary{positioning,calc}

\definecolor{cvprblue}{rgb}{0.21,0.49,0.74}
\usepackage[pagebackref,breaklinks,colorlinks,allcolors=cvprblue]{hyperref}

\definecolor{darkgreen}{RGB}{30,150,30}
\definecolor{darkblue}{RGB}{0,0,127}
\definecolor{darkyellow}{RGB}{171,133,0}
\definecolor{darkred}{RGB}{180,20,20}
\definecolor{darkmagenta}{RGB}{127,0,127}
\definecolor{darkcyan}{RGB}{0,127,127}
\definecolor{chromeyellow}{rgb}{1.0, 0.65, 0.0}
\definecolor{amber}{rgb}{1.0, 0.75, 0.0}

\newcommand{\methodName}{L4P\xspace}

\title{L4P: \uline{L}ow-Level \uline{4}D Vision \uline{P}erception Unified}

\newcommand{\mychar}{
  \begingroup\normalfont\centering
  \includegraphics[width=0.12\textwidth]{figures/nvidialogo.png}
  \endgroup
}

\author{
  Abhishek Badki$^*$ \quad  Hang Su$^*$ \quad Bowen Wen \quad Orazio Gallo \\ \vspace{-0.4em} \\
  \mychar{} \\
  \small\url{https://research.nvidia.com/labs/lpr/l4p}\\
}

\begin{document}

\maketitle

\newcommand\blfootnote[1]{
  \begingroup
  \renewcommand\thefootnote{}\footnote{#1}
  \addtocounter{footnote}{-1}
  \endgroup
}
\blfootnote{\hspace{-1em} $^*$ indicates equal contribution.}


\begin{abstract}  
Test time scaling is currently one of the most active research areas that shows promise after training time scaling has reached its limits.
Deep-thinking (DT) models are a class of recurrent models that can perform easy-to-hard generalization by assigning more compute to harder test samples.
However, due to their inability to determine the complexity of a test sample, DT models have to use a large amount of computation for both easy and hard test samples.
Excessive test time computation is wasteful and can cause the ``overthinking'' problem where more test time computation leads to worse results.
In this paper, we introduce a test time training method for determining the optimal amount of computation needed for each sample during test time.
We also propose Conv-LiGRU, a novel recurrent architecture for efficient and robust visual reasoning. 
Extensive experiments demonstrate that Conv-LiGRU is more stable than DT, effectively mitigates the ``overthinking'' phenomenon, and achieves superior accuracy.
\end{abstract}      
\section{Introduction}\label{sec:intro}
%\bibliographystyle{plainnat}
%\begin{thebibliography}{9}

%\bibitem[Christiano et~al.(2017)]{Christiano2017DeepRLHF}
%Paul Christiano, Jan Leike, Tom Brown, et~al.
%\newblock Deep reinforcement learning from human preferences.
%\newblock In \emph{Advances in Neural Information Processing Systems (NeurIPS)}, 2017.

%\bibitem[Rafailov et~al.(2023)]{Rafailov2023DirectPreference}
%R~Rafailov, Mitchell Wortsman, Ludwig Schmidt, et~al.
%\newblock Direct preference optimization: Your language model is secretly a reward model.
%\newblock \emph{arXiv preprint arXiv:2305.17888}, 2023.

%\bibitem[Von Stackelberg(1934)]{VonStackelberg1934Marktform}
%Heinrich Von Stackelberg.
%\newblock \emph{Marktform und Gleichgewicht}.
%\newblock Julius Springer, 1934.

%\bibitem[Basar and Olsder(1999)]{Basar1999DynamicNG}
%Tamer Basar and Geert~Jan Olsder.
%\newblock \emph{Dynamic Noncooperative Game Theory}.
%\newblock SIAM, 1999.

%\bibitem[Villani(2008)]{Villani2008OptimalTW}
%Cédric Villani.
%\newblock \emph{Optimal transport: old and new}.
%\newblock Springer, 2008.

%\end{thebibliography}

\section{Introduction}
\label{sec:intro}
Recent breakthroughs in large language models (LLMs) have made it increasingly crucial to \emph{align} generated text with human preferences for both usability and safety \citep{Ouyang2022Training,Bai2022Training}. Traditional approaches such as Reinforcement Learning from Human Feedback (RLHF) \citep{Christiano2017Deep} and Direct Preference Optimization (DPO) \citep{Rafailov2023Direct} often require \emph{massive} amounts of meticulously curated preference data. Not only is gathering such a dataset expensive and time-consuming, but any mislabeling can propagate through iterative alignment stages \citep{Casper2023Open}, leading to suboptimal or even unsafe model behaviors. This raises a critical challenge: \emph{How can we achieve preference data-efficient alignment of language models while maintaining robustness to annotation noise?}

From the perspective of data efficiency and robustness, existing alignment approaches often suffer from two main issues:\emph{Self-annotation gaps} and \emph{Lack of equilibrium guarantees under noise}. Recent work explores self-annotation \citep{Lee2024Rlaif,Yuan2024Self,Kim2025Spread}, where an LLM generates labels for new prompt--response pairs instead of relying on humans. While this indeed lowers annotation cost, such methods often treat policy updates and preference annotation as disconnected processes. Consequently, once noisy synthetic preferences are generated, there is limited recourse if the LLM’s self-labels embed systematic biases or errors that can corrupt future training \citep{Chowdhury2024Provably}. Some adversarial training approaches \citep{Cheng2023Adversarial,Wu2024Towards} attempt to counter distributional shifts in preference data, but they often lack formal \emph{equilibrium} guarantees and can lead to unstable optimization cycles in practice. Furthermore, these adversarial approaches are not specifically tailored for data-scarce alignment regimes, thus limiting their applicability when human labels are extremely expensive. \emph{We delay a more thorough \textbf{related work} section in the Appendix~\ref{sec:related_supp}}.

To address these issues, we propose \textbf{Stackelberg Game Preference Optimization (SGPO)}, a framework that models alignment as a \emph{two-player} Stackelberg game between: a \emph{policy} (the leader), which aims to satisfy real human preferences, and an \emph{adversarial preference distribution} (the follower), which explores worst-case shifts within a defined Wasserstein ball of radius $\epsilon$. Drawing inspiration from Stackelberg dynamics \citep{Bacsar1998Dynamic} and optimal transport \citep{Villani2009Optimal}, SGPO ensures that the policy optimizes against the worst plausible shifts in preference data. Specifically, under $\epsilon$-bounded shifts (or annotation noise), we prove that the resulting policy’s regret is at most $\mathcal{O}(\epsilon)$ (see Section~\ref{sec:theory}), whereas standard DPO can incur \emph{linear} regret growth with respect to the magnitude of distribution mismatch. Although our analysis uses $\epsilon$-Wasserstein balls as a tractable model of moderate noise or mismatch, it remains relevant for practical alignment scenarios where annotation errors are not unbounded but still matter.

On top of SGPO, we develop the \textbf{Stackelberg Self-Annotated Preference Optimization (SSAPO)} (Section~\ref{sec:ssapo}) algorithm, a procedure aimed at drastically lowering human annotation needs. SSAPO starts with a small human-labeled seed (about $1/30$ of the usual scale in our experiments) and then: \emph{(1) Self-Annotates} newly sampled prompts by generating responses and extracting winner--loser pairs from the current policy's own comparisons. \emph{(2) Adversarially Reweights} these pairs within a Wasserstein ball of radius $\epsilon$, by solving a distributionally robust optimization (DRO) \citep{Esfahani2018Data} program,ensuring that potentially corrupted or unrepresentative synthetic preferences do not overwhelm the policy update.
By iterating these two steps, SSAPO instantiates the SGPO framework, preserving theoretical bounded-regret guarantees while yielding significant data-efficiency gains. In practice, we find that with only 1/30 of the usual human annotations (from the UltraFeedback dataset \citep{Cui2023Ultrafeedback}), SSAPO attains 35.82~\% GPT4 win-rate (24.44\% LC win-rate) on Mistral-7B, 40.12\% win-rate (33.33\% LC win-rate) on Llama3-8B-instruct. Which matches Mixtral Large (21.4\% win-rate and 32.7\% LC win-rate) and Llama3-70B-instruct (33.2\% winrate and 34.4\% LC win-rate) according to the AlphacaEval~2.0 leaderboard~\citep{dubois2024length}.% Concretely, within 3 self-annotation rounds, our model matches methods that use \textbf{30 times} more human-labeled data.

In summary, We formulate DPO-like alignment as a Stackelberg game, demonstrate \emph{existence} of an equilibrium, and establish that \emph{SGPO} achieves an $\mathcal{O}(\epsilon)$ regret bound under $\epsilon$-bounded noise, in contrast to the linear regret behavior of DPO when facing similarly scaled shifts.(Section~\ref{sec:theory}). We implement SGPO via \emph{SSAPO}, which combines self-annotation with distributionaly robust optimization.(Section~\ref{sec:ssapo}) %We tackle the concavity requirement of Standard DRO with piecewise-linear \emph{concave} envelope. We handle large-scale preference data via a \emph{uniform grouping} strategy for parallel subproblem solutions.  
 Extensive experiments show that SSAPO attains strong alignment performance with substantially fewer human labels, thereby showing its real-world viability for cost-effective preference alignment of LLMs (Section~\ref{sec:experiments}).

%Overall, our main contributions are: (1) \textbf{Theoretical Foundation} (Section~\ref{sec:theory}): We formulate alignment under limited labels as a Stackelberg game, demonstrate \emph{existence} of an equilibrium, and establish that \emph{SGPO} achieves an $\mathcal{O}(\epsilon)$ regret bound under $\epsilon$-bounded noise. This stands in contrast to the linear regret behavior of DPO when facing similarly scaled shifts.(Theorems~\ref{thm:existence_se}--\ref{thm:sgpo_regret}). (2) \textbf{Scalable Algorithm} (Section~\ref{sec:ssapo}):  We implement SGPO via \emph{SSAPO}, which combines self-annotation with distributionaly robust optimization (DRO). We tackle the concavity requirement of Standard DRO with piecewise-linear \emph{concave} envelope. We handle large-scale preference data via a \emph{uniform grouping} strategy for parallel subproblem solutions. (3) \textbf{Empirical Validation} (Section~\ref{sec:experiments}): Extensive experiments show that SSAPO attains strong alignment performance with substantially fewer human labels, thereby showing its real-world viability for cost-effective preference alignment of LLMs.
%By furnishing both formal guarantees and an effective, scalable method, we hope to advance the practice of LLM alignment under realistic labeling budgets.

\section{Related Works}\label{sec:related}
\putsec{related}{Related Work}

\noindent \textbf{Efficient Radiance Field Rendering.}
%
The introduction of Neural Radiance Fields (NeRF)~\cite{mil:sri20} has
generated significant interest in efficient 3D scene representation and
rendering for radiance fields.
%
Over the past years, there has been a large amount of research aimed at
accelerating NeRFs through algorithmic or software
optimizations~\cite{mul:eva22,fri:yu22,che:fun23,sun:sun22}, and the
development of hardware
accelerators~\cite{lee:cho23,li:li23,son:wen23,mub:kan23,fen:liu24}.
%
The state-of-the-art method, 3D Gaussian splatting~\cite{ker:kop23}, has
further fueled interest in accelerating radiance field
rendering~\cite{rad:ste24,lee:lee24,nie:stu24,lee:rho24,ham:mel24} as it
employs rasterization primitives that can be rendered much faster than NeRFs.
%
However, previous research focused on software graphics rendering on
programmable cores or building dedicated hardware accelerators. In contrast,
\name{} investigates the potential of efficient radiance field rendering while
utilizing fixed-function units in graphics hardware.
%
To our knowledge, this is the first work that assesses the performance
implications of rendering Gaussian-based radiance fields on the hardware
graphics pipeline with software and hardware optimizations.

%%%%%%%%%%%%%%%%%%%%%%%%%%%%%%%%%%%%%%%%%%%%%%%%%%%%%%%%%%%%%%%%%%%%%%%%%%
\myparagraph{Enhancing Graphics Rendering Hardware.}
%
The performance advantage of executing graphics rendering on either
programmable shader cores or fixed-function units varies depending on the
rendering methods and hardware designs.
%
Previous studies have explored the performance implication of graphics hardware
design by developing simulation infrastructures for graphics
workloads~\cite{bar:gon06,gub:aam19,tin:sax23,arn:par13}.
%
Additionally, several studies have aimed to improve the performance of
special-purpose hardware such as ray tracing units in graphics
hardware~\cite{cho:now23,liu:cha21} and proposed hardware accelerators for
graphics applications~\cite{lu:hua17,ram:gri09}.
%
In contrast to these works, which primarily evaluate traditional graphics
workloads, our work focuses on improving the performance of volume rendering
workloads, such as Gaussian splatting, which require blending a huge number of
fragments per pixel.

%%%%%%%%%%%%%%%%%%%%%%%%%%%%%%%%%%%%%%%%%%%%%%%%%%%%%%%%%%%%%%%%%%%%%%%%%%
%
In the context of multi-sample anti-aliasing, prior work proposed reducing the
amount of redundant shading by merging fragments from adjacent triangles in a
mesh at the quad granularity~\cite{fat:bou10}.
%
While both our work and quad-fragment merging (QFM)~\cite{fat:bou10} aim to
reduce operations by merging quads, our proposed technique differs from QFM in
many aspects.
%
Our method aims to blend \emph{overlapping primitives} along the depth
direction and applies to quads from any primitive. In contrast, QFM merges quad
fragments from small (e.g., pixel-sized) triangles that \emph{share} an edge
(i.e., \emph{connected}, \emph{non-overlapping} triangles).
%
As such, QFM is not applicable to the scenes consisting of a number of
unconnected transparent triangles, such as those in 3D Gaussian splatting.
%
In addition, our method computes the \emph{exact} color for each pixel by
offloading blending operations from ROPs to shader units, whereas QFM
\emph{approximates} pixel colors by using the color from one triangle when
multiple triangles are merged into a single quad.


\section{Method}\label{sec:method}
\section{Method}\label{sec:method}
\begin{figure}
    \centering
    \includegraphics[width=0.85\textwidth]{imgs/heatmap_acc.pdf}
    \caption{\textbf{Visualization of the proposed periodic Bayesian flow with mean parameter $\mu$ and accumulated accuracy parameter $c$ which corresponds to the entropy/uncertainty}. For $x = 0.3, \beta(1) = 1000$ and $\alpha_i$ defined in \cref{appd:bfn_cir}, this figure plots three colored stochastic parameter trajectories for receiver mean parameter $m$ and accumulated accuracy parameter $c$, superimposed on a log-scale heatmap of the Bayesian flow distribution $p_F(m|x,\senderacc)$ and $p_F(c|x,\senderacc)$. Note the \emph{non-monotonicity} and \emph{non-additive} property of $c$ which could inform the network the entropy of the mean parameter $m$ as a condition and the \emph{periodicity} of $m$. %\jj{Shrink the figures to save space}\hanlin{Do we need to make this figure one-column?}
    }
    \label{fig:vmbf_vis}
    \vskip -0.1in
\end{figure}
% \begin{wrapfigure}{r}{0.5\textwidth}
%     \centering
%     \includegraphics[width=0.49\textwidth]{imgs/heatmap_acc.pdf}
%     \caption{\textbf{Visualization of hyper-torus Bayesian flow based on von Mises Distribution}. For $x = 0.3, \beta(1) = 1000$ and $\alpha_i$ defined in \cref{appd:bfn_cir}, this figure plots three colored stochastic parameter trajectories for receiver mean parameter $m$ and accumulated accuracy parameter $c$, superimposed on a log-scale heatmap of the Bayesian flow distribution $p_F(m|x,\senderacc)$ and $p_F(c|x,\senderacc)$. Note the \emph{non-monotonicity} and \emph{non-additive} property of $c$. \jj{Shrink the figures to save space}}
%     \label{fig:vmbf_vis}
%     \vspace{-30pt}
% \end{wrapfigure}


In this section, we explain the detailed design of CrysBFN tackling theoretical and practical challenges. First, we describe how to derive our new formulation of Bayesian Flow Networks over hyper-torus $\mathbb{T}^{D}$ from scratch. Next, we illustrate the two key differences between \modelname and the original form of BFN: $1)$ a meticulously designed novel base distribution with different Bayesian update rules; and $2)$ different properties over the accuracy scheduling resulted from the periodicity and the new Bayesian update rules. Then, we present in detail the overall framework of \modelname over each manifold of the crystal space (\textit{i.e.} fractional coordinates, lattice vectors, atom types) respecting \textit{periodic E(3) invariance}. 

% In this section, we first demonstrate how to build Bayesian flow on hyper-torus $\mathbb{T}^{D}$ by overcoming theoretical and practical problems to provide a low-noise parameter-space approach to fractional atom coordinate generation. Next, we present how \modelname models each manifold of crystal space respecting \textit{periodic E(3) invariance}. 

\subsection{Periodic Bayesian Flow on Hyper-torus \texorpdfstring{$\mathbb{T}^{D}$}{}} 
For generative modeling of fractional coordinates in crystal, we first construct a periodic Bayesian flow on \texorpdfstring{$\mathbb{T}^{D}$}{} by designing every component of the totally new Bayesian update process which we demonstrate to be distinct from the original Bayesian flow (please see \cref{fig:non_add}). 
 %:) 
 
 The fractional atom coordinate system \citep{jiao2023crystal} inherently distributes over a hyper-torus support $\mathbb{T}^{3\times N}$. Hence, the normal distribution support on $\R$ used in the original \citep{bfn} is not suitable for this scenario. 
% The key problem of generative modeling for crystal is the periodicity of Cartesian atom coordinates $\vX$ requiring:
% \begin{equation}\label{eq:periodcity}
% p(\vA,\vL,\vX)=p(\vA,\vL,\vX+\vec{LK}),\text{where}~\vec{K}=\vec{k}\vec{1}_{1\times N},\forall\vec{k}\in\mathbb{Z}^{3\times1}
% \end{equation}
% However, there does not exist such a distribution supporting on $\R$ to model such property because the integration of such distribution over $\R$ will not be finite and equal to 1. Therefore, the normal distribution used in \citet{bfn} can not meet this condition.

To tackle this problem, the circular distribution~\citep{mardia2009directional} over the finite interval $[-\pi,\pi)$ is a natural choice as the base distribution for deriving the BFN on $\mathbb{T}^D$. 
% one natural choice is to 
% we would like to consider the circular distribution over the finite interval as the base 
% we find that circular distributions \citep{mardia2009directional} defined on a finite interval with lengths of $2\pi$ can be used as the instantiation of input distribution for the BFN on $\mathbb{T}^D$.
Specifically, circular distributions enjoy desirable periodic properties: $1)$ the integration over any interval length of $2\pi$ equals 1; $2)$ the probability distribution function is periodic with period $2\pi$.  Sharing the same intrinsic with fractional coordinates, such periodic property of circular distribution makes it suitable for the instantiation of BFN's input distribution, in parameterizing the belief towards ground truth $\x$ on $\mathbb{T}^D$. 
% \yuxuan{this is very complicated from my perspective.} \hanlin{But this property is exactly beautiful and perfectly fit into the BFN.}

\textbf{von Mises Distribution and its Bayesian Update} We choose von Mises distribution \citep{mardia2009directional} from various circular distributions as the form of input distribution, based on the appealing conjugacy property required in the derivation of the BFN framework.
% to leverage the Bayesian conjugacy property of von Mises distribution which is required by the BFN framework. 
That is, the posterior of a von Mises distribution parameterized likelihood is still in the family of von Mises distributions. The probability density function of von Mises distribution with mean direction parameter $m$ and concentration parameter $c$ (describing the entropy/uncertainty of $m$) is defined as: 
\begin{equation}
f(x|m,c)=vM(x|m,c)=\frac{\exp(c\cos(x-m))}{2\pi I_0(c)}
\end{equation}
where $I_0(c)$ is zeroth order modified Bessel function of the first kind as the normalizing constant. Given the last univariate belief parameterized by von Mises distribution with parameter $\theta_{i-1}=\{m_{i-1},\ c_{i-1}\}$ and the sample $y$ from sender distribution with unknown data sample $x$ and known accuracy $\alpha$ describing the entropy/uncertainty of $y$,  Bayesian update for the receiver is deducted as:
\begin{equation}
 h(\{m_{i-1},c_{i-1}\},y,\alpha)=\{m_i,c_i \}, \text{where}
\end{equation}
\begin{equation}\label{eq:h_m}
m_i=\text{atan2}(\alpha\sin y+c_{i-1}\sin m_{i-1}, {\alpha\cos y+c_{i-1}\cos m_{i-1}})
\end{equation}
\begin{equation}\label{eq:h_c}
c_i =\sqrt{\alpha^2+c_{i-1}^2+2\alpha c_{i-1}\cos(y-m_{i-1})}
\end{equation}
The proof of the above equations can be found in \cref{apdx:bayesian_update_function}. The atan2 function refers to  2-argument arctangent. Independently conducting  Bayesian update for each dimension, we can obtain the Bayesian update distribution by marginalizing $\y$:
\begin{equation}
p_U(\vtheta'|\vtheta,\bold{x};\alpha)=\mathbb{E}_{p_S(\bold{y}|\bold{x};\alpha)}\delta(\vtheta'-h(\vtheta,\bold{y},\alpha))=\mathbb{E}_{vM(\bold{y}|\bold{x},\alpha)}\delta(\vtheta'-h(\vtheta,\bold{y},\alpha))
\end{equation} 
\begin{figure}
    \centering
    \vskip -0.15in
    \includegraphics[width=0.95\linewidth]{imgs/non_add.pdf}
    \caption{An intuitive illustration of non-additive accuracy Bayesian update on the torus. The lengths of arrows represent the uncertainty/entropy of the belief (\emph{e.g.}~$1/\sigma^2$ for Gaussian and $c$ for von Mises). The directions of the arrows represent the believed location (\emph{e.g.}~ $\mu$ for Gaussian and $m$ for von Mises).}
    \label{fig:non_add}
    \vskip -0.15in
\end{figure}
\textbf{Non-additive Accuracy} 
The additive accuracy is a nice property held with the Gaussian-formed sender distribution of the original BFN expressed as:
\begin{align}
\label{eq:standard_id}
    \update(\parsn{}'' \mid \parsn{}, \x; \alpha_a+\alpha_b) = \E_{\update(\parsn{}' \mid \parsn{}, \x; \alpha_a)} \update(\parsn{}'' \mid \parsn{}', \x; \alpha_b)
\end{align}
Such property is mainly derived based on the standard identity of Gaussian variable:
\begin{equation}
X \sim \mathcal{N}\left(\mu_X, \sigma_X^2\right), Y \sim \mathcal{N}\left(\mu_Y, \sigma_Y^2\right) \Longrightarrow X+Y \sim \mathcal{N}\left(\mu_X+\mu_Y, \sigma_X^2+\sigma_Y^2\right)
\end{equation}
The additive accuracy property makes it feasible to derive the Bayesian flow distribution $
p_F(\boldsymbol{\theta} \mid \mathbf{x} ; i)=p_U\left(\boldsymbol{\theta} \mid \boldsymbol{\theta}_0, \mathbf{x}, \sum_{k=1}^{i} \alpha_i \right)
$ for the simulation-free training of \cref{eq:loss_n}.
It should be noted that the standard identity in \cref{eq:standard_id} does not hold in the von Mises distribution. Hence there exists an important difference between the original Bayesian flow defined on Euclidean space and the Bayesian flow of circular data on $\mathbb{T}^D$ based on von Mises distribution. With prior $\btheta = \{\bold{0},\bold{0}\}$, we could formally represent the non-additive accuracy issue as:
% The additive accuracy property implies the fact that the "confidence" for the data sample after observing a series of the noisy samples with accuracy ${\alpha_1, \cdots, \alpha_i}$ could be  as the accuracy sum  which could be  
% Here we 
% Here we emphasize the specific property of BFN based on von Mises distribution.
% Note that 
% \begin{equation}
% \update(\parsn'' \mid \parsn, \x; \alpha_a+\alpha_b) \ne \E_{\update(\parsn' \mid \parsn, \x; \alpha_a)} \update(\parsn'' \mid \parsn', \x; \alpha_b)
% \end{equation}
% \oyyw{please check whether the below equation is better}
% \yuxuan{I fill somehow confusing on what is the update distribution with $\alpha$. }
% \begin{equation}
% \update(\parsn{}'' \mid \parsn{}, \x; \alpha_a+\alpha_b) \ne \E_{\update(\parsn{}' \mid \parsn{}, \x; \alpha_a)} \update(\parsn{}'' \mid \parsn{}', \x; \alpha_b)
% \end{equation}
% We give an intuitive visualization of such difference in \cref{fig:non_add}. The untenability of this property can materialize by considering the following case: with prior $\btheta = \{\bold{0},\bold{0}\}$, check the two-step Bayesian update distribution with $\alpha_a,\alpha_b$ and one-step Bayesian update with $\alpha=\alpha_a+\alpha_b$:
\begin{align}
\label{eq:nonadd}
     &\update(c'' \mid \parsn, \x; \alpha_a+\alpha_b)  = \delta(c-\alpha_a-\alpha_b)
     \ne  \mathbb{E}_{p_U(\parsn' \mid \parsn, \x; \alpha_a)}\update(c'' \mid \parsn', \x; \alpha_b) \nonumber \\&= \mathbb{E}_{vM(\bold{y}_b|\bold{x},\alpha_a)}\mathbb{E}_{vM(\bold{y}_a|\bold{x},\alpha_b)}\delta(c-||[\alpha_a \cos\y_a+\alpha_b\cos \y_b,\alpha_a \sin\y_a+\alpha_b\sin \y_b]^T||_2)
\end{align}
A more intuitive visualization could be found in \cref{fig:non_add}. This fundamental difference between periodic Bayesian flow and that of \citet{bfn} presents both theoretical and practical challenges, which we will explain and address in the following contents.

% This makes constructing Bayesian flow based on von Mises distribution intrinsically different from previous Bayesian flows (\citet{bfn}).

% Thus, we must reformulate the framework of Bayesian flow networks  accordingly. % and do necessary reformulations of BFN. 

% \yuxuan{overall I feel this part is complicated by using the language of update distribution. I would like to suggest simply use bayesian update, to provide intuitive explantion.}\hanlin{See the illustration in \cref{fig:non_add}}

% That introduces a cascade of problems, and we investigate the following issues: $(1)$ Accuracies between sender and receiver are not synchronized and need to be differentiated. $(2)$ There is no tractable Bayesian flow distribution for a one-step sample conditioned on a given time step $i$, and naively simulating the Bayesian flow results in computational overhead. $(3)$ It is difficult to control the entropy of the Bayesian flow. $(4)$ Accuracy is no longer a function of $t$ and becomes a distribution conditioned on $t$, which can be different across dimensions.
%\jj{Edited till here}

\textbf{Entropy Conditioning} As a common practice in generative models~\citep{ddpm,flowmatching,bfn}, timestep $t$ is widely used to distinguish among generation states by feeding the timestep information into the networks. However, this paper shows that for periodic Bayesian flow, the accumulated accuracy $\vc_i$ is more effective than time-based conditioning by informing the network about the entropy and certainty of the states $\parsnt{i}$. This stems from the intrinsic non-additive accuracy which makes the receiver's accumulated accuracy $c$ not bijective function of $t$, but a distribution conditioned on accumulated accuracies $\vc_i$ instead. Therefore, the entropy parameter $\vc$ is taken logarithm and fed into the network to describe the entropy of the input corrupted structure. We verify this consideration in \cref{sec:exp_ablation}. 
% \yuxuan{implement variant. traditionally, the timestep is widely used to distinguish the different states by putting the timestep embedding into the networks. citation of FM, diffusion, BFN. However, we find that conditioned on time in periodic flow could not provide extra benefits. To further boost the performance, we introduce a simple yet effective modification term entropy conditional. This is based on that the accumulated accuracy which represents the current uncertainty or entropy could be a better indicator to distinguish different states. + Describe how you do this. }



\textbf{Reformulations of BFN}. Recall the original update function with Gaussian sender distribution, after receiving noisy samples $\y_1,\y_2,\dots,\y_i$ with accuracies $\senderacc$, the accumulated accuracies of the receiver side could be analytically obtained by the additive property and it is consistent with the sender side.
% Since observing sample $\y$ with $\alpha_i$ can not result in exact accuracy increment $\alpha_i$ for receiver, the accuracies between sender and receiver are not synchronized which need to be differentiated. 
However, as previously mentioned, this does not apply to periodic Bayesian flow, and some of the notations in original BFN~\citep{bfn} need to be adjusted accordingly. We maintain the notations of sender side's one-step accuracy $\alpha$ and added accuracy $\beta$, and alter the notation of receiver's accuracy parameter as $c$, which is needed to be simulated by cascade of Bayesian updates. We emphasize that the receiver's accumulated accuracy $c$ is no longer a function of $t$ (differently from the Gaussian case), and it becomes a distribution conditioned on received accuracies $\senderacc$ from the sender. Therefore, we represent the Bayesian flow distribution of von Mises distribution as $p_F(\btheta|\x;\alpha_1,\alpha_2,\dots,\alpha_i)$. And the original simulation-free training with Bayesian flow distribution is no longer applicable in this scenario.
% Different from previous BFNs where the accumulated accuracy $\rho$ is not explicitly modeled, the accumulated accuracy parameter $c$ (visualized in \cref{fig:vmbf_vis}) needs to be explicitly modeled by feeding it to the network to avoid information loss.
% the randomaccuracy parameter $c$ (visualized in \cref{fig:vmbf_vis}) implies that there exists information in $c$ from the sender just like $m$, meaning that $c$ also should be fed into the network to avoid information loss. 
% We ablate this consideration in  \cref{sec:exp_ablation}. 

\textbf{Fast Sampling from Equivalent Bayesian Flow Distribution} Based on the above reformulations, the Bayesian flow distribution of von Mises distribution is reframed as: 
\begin{equation}\label{eq:flow_frac}
p_F(\btheta_i|\x;\alpha_1,\alpha_2,\dots,\alpha_i)=\E_{\update(\parsnt{1} \mid \parsnt{0}, \x ; \alphat{1})}\dots\E_{\update(\parsn_{i-1} \mid \parsnt{i-2}, \x; \alphat{i-1})} \update(\parsnt{i} | \parsnt{i-1},\x;\alphat{i} )
\end{equation}
Naively sampling from \cref{eq:flow_frac} requires slow auto-regressive iterated simulation, making training unaffordable. Noticing the mathematical properties of \cref{eq:h_m,eq:h_c}, we  transform \cref{eq:flow_frac} to the equivalent form:
\begin{equation}\label{eq:cirflow_equiv}
p_F(\vec{m}_i|\x;\alpha_1,\alpha_2,\dots,\alpha_i)=\E_{vM(\y_1|\x,\alpha_1)\dots vM(\y_i|\x,\alpha_i)} \delta(\vec{m}_i-\text{atan2}(\sum_{j=1}^i \alpha_j \cos \y_j,\sum_{j=1}^i \alpha_j \sin \y_j))
\end{equation}
\begin{equation}\label{eq:cirflow_equiv2}
p_F(\vec{c}_i|\x;\alpha_1,\alpha_2,\dots,\alpha_i)=\E_{vM(\y_1|\x,\alpha_1)\dots vM(\y_i|\x,\alpha_i)}  \delta(\vec{c}_i-||[\sum_{j=1}^i \alpha_j \cos \y_j,\sum_{j=1}^i \alpha_j \sin \y_j]^T||_2)
\end{equation}
which bypasses the computation of intermediate variables and allows pure tensor operations, with negligible computational overhead.
\begin{restatable}{proposition}{cirflowequiv}
The probability density function of Bayesian flow distribution defined by \cref{eq:cirflow_equiv,eq:cirflow_equiv2} is equivalent to the original definition in \cref{eq:flow_frac}. 
\end{restatable}
\textbf{Numerical Determination of Linear Entropy Sender Accuracy Schedule} ~Original BFN designs the accuracy schedule $\beta(t)$ to make the entropy of input distribution linearly decrease. As for crystal generation task, to ensure information coherence between modalities, we choose a sender accuracy schedule $\senderacc$ that makes the receiver's belief entropy $H(t_i)=H(p_I(\cdot|\vtheta_i))=H(p_I(\cdot|\vc_i))$ linearly decrease \emph{w.r.t.} time $t_i$, given the initial and final accuracy parameter $c(0)$ and $c(1)$. Due to the intractability of \cref{eq:vm_entropy}, we first use numerical binary search in $[0,c(1)]$ to determine the receiver's $c(t_i)$ for $i=1,\dots, n$ by solving the equation $H(c(t_i))=(1-t_i)H(c(0))+tH(c(1))$. Next, with $c(t_i)$, we conduct numerical binary search for each $\alpha_i$ in $[0,c(1)]$ by solving the equations $\E_{y\sim vM(x,\alpha_i)}[\sqrt{\alpha_i^2+c_{i-1}^2+2\alpha_i c_{i-1}\cos(y-m_{i-1})}]=c(t_i)$ from $i=1$ to $i=n$ for arbitrarily selected $x\in[-\pi,\pi)$.

After tackling all those issues, we have now arrived at a new BFN architecture for effectively modeling crystals. Such BFN can also be adapted to other type of data located in hyper-torus $\mathbb{T}^{D}$.

\subsection{Equivariant Bayesian Flow for Crystal}
With the above Bayesian flow designed for generative modeling of fractional coordinate $\vF$, we are able to build equivariant Bayesian flow for each modality of crystal. In this section, we first give an overview of the general training and sampling algorithm of \modelname (visualized in \cref{fig:framework}). Then, we describe the details of the Bayesian flow of every modality. The training and sampling algorithm can be found in \cref{alg:train} and \cref{alg:sampling}.

\textbf{Overview} Operating in the parameter space $\bthetaM=\{\bthetaA,\bthetaL,\bthetaF\}$, \modelname generates high-fidelity crystals through a joint BFN sampling process on the parameter of  atom type $\bthetaA$, lattice parameter $\vec{\theta}^L=\{\bmuL,\brhoL\}$, and the parameter of fractional coordinate matrix $\bthetaF=\{\bmF,\bcF\}$. We index the $n$-steps of the generation process in a discrete manner $i$, and denote the corresponding continuous notation $t_i=i/n$ from prior parameter $\thetaM_0$ to a considerably low variance parameter $\thetaM_n$ (\emph{i.e.} large $\vrho^L,\bmF$, and centered $\bthetaA$).

At training time, \modelname samples time $i\sim U\{1,n\}$ and $\bthetaM_{i-1}$ from the Bayesian flow distribution of each modality, serving as the input to the network. The network $\net$ outputs $\net(\parsnt{i-1}^\mathcal{M},t_{i-1})=\net(\parsnt{i-1}^A,\parsnt{i-1}^F,\parsnt{i-1}^L,t_{i-1})$ and conducts gradient descents on loss function \cref{eq:loss_n} for each modality. After proper training, the sender distribution $p_S$ can be approximated by the receiver distribution $p_R$. 

At inference time, from predefined $\thetaM_0$, we conduct transitions from $\thetaM_{i-1}$ to $\thetaM_{i}$ by: $(1)$ sampling $\y_i\sim p_R(\bold{y}|\thetaM_{i-1};t_i,\alpha_i)$ according to network prediction $\predM{i-1}$; and $(2)$ performing Bayesian update $h(\thetaM_{i-1},\y^\calM_{i-1},\alpha_i)$ for each dimension. 

% Alternatively, we complete this transition using the flow-back technique by sampling 
% $\thetaM_{i}$ from Bayesian flow distribution $\flow(\btheta^M_{i}|\predM{i-1};t_{i-1})$. 

% The training objective of $\net$ is to minimize the KL divergence between sender distribution and receiver distribution for every modality as defined in \cref{eq:loss_n} which is equivalent to optimizing the negative variational lower bound $\calL^{VLB}$ as discussed in \cref{sec:preliminaries}. 

%In the following part, we will present the Bayesian flow of each modality in detail.

\textbf{Bayesian Flow of Fractional Coordinate $\vF$}~The distribution of the prior parameter $\bthetaF_0$ is defined as:
\begin{equation}\label{eq:prior_frac}
    p(\bthetaF_0) \defeq \{vM(\vm_0^F|\vec{0}_{3\times N},\vec{0}_{3\times N}),\delta(\vc_0^F-\vec{0}_{3\times N})\} = \{U(\vec{0},\vec{1}),\delta(\vc_0^F-\vec{0}_{3\times N})\}
\end{equation}
Note that this prior distribution of $\vm_0^F$ is uniform over $[\vec{0},\vec{1})$, ensuring the periodic translation invariance property in \cref{De:pi}. The training objective is minimizing the KL divergence between sender and receiver distribution (deduction can be found in \cref{appd:cir_loss}): 
%\oyyw{replace $\vF$ with $\x$?} \hanlin{notations follow Preliminary?}
\begin{align}\label{loss_frac}
\calL_F = n \E_{i \sim \ui{n}, \flow(\parsn{}^F \mid \vF ; \senderacc)} \alpha_i\frac{I_1(\alpha_i)}{I_0(\alpha_i)}(1-\cos(\vF-\predF{i-1}))
\end{align}
where $I_0(x)$ and $I_1(x)$ are the zeroth and the first order of modified Bessel functions. The transition from $\bthetaF_{i-1}$ to $\bthetaF_{i}$ is the Bayesian update distribution based on network prediction:
\begin{equation}\label{eq:transi_frac}
    p(\btheta^F_{i}|\parsnt{i-1}^\calM)=\mathbb{E}_{vM(\bold{y}|\predF{i-1},\alpha_i)}\delta(\btheta^F_{i}-h(\btheta^F_{i-1},\bold{y},\alpha_i))
\end{equation}
\begin{restatable}{proposition}{fracinv}
With $\net_{F}$ as a periodic translation equivariant function namely $\net_F(\parsnt{}^A,w(\parsnt{}^F+\vt),\parsnt{}^L,t)=w(\net_F(\parsnt{}^A,\parsnt{}^F,\parsnt{}^L,t)+\vt), \forall\vt\in\R^3$, the marginal distribution of $p(\vF_n)$ defined by \cref{eq:prior_frac,eq:transi_frac} is periodic translation invariant. 
\end{restatable}
\textbf{Bayesian Flow of Lattice Parameter \texorpdfstring{$\boldsymbol{L}$}{}}   
Noting the lattice parameter $\bm{L}$ located in Euclidean space, we set prior as the parameter of a isotropic multivariate normal distribution $\btheta^L_0\defeq\{\vmu_0^L,\vrho_0^L\}=\{\bm{0}_{3\times3},\bm{1}_{3\times3}\}$
% \begin{equation}\label{eq:lattice_prior}
% \btheta^L_0\defeq\{\vmu_0^L,\vrho_0^L\}=\{\bm{0}_{3\times3},\bm{1}_{3\times3}\}
% \end{equation}
such that the prior distribution of the Markov process on $\vmu^L$ is the Dirac distribution $\delta(\vec{\mu_0}-\vec{0})$ and $\delta(\vec{\rho_0}-\vec{1})$, 
% \begin{equation}
%     p_I^L(\boldsymbol{L}|\btheta_0^L)=\mathcal{N}(\bm{L}|\bm{0},\bm{I})
% \end{equation}
which ensures O(3)-invariance of prior distribution of $\vL$. By Eq. 77 from \citet{bfn}, the Bayesian flow distribution of the lattice parameter $\bm{L}$ is: 
\begin{align}% =p_U(\bmuL|\btheta_0^L,\bm{L},\beta(t))
p_F^L(\bmuL|\bm{L};t) &=\mathcal{N}(\bmuL|\gamma(t)\bm{L},\gamma(t)(1-\gamma(t))\bm{I}) 
\end{align}
where $\gamma(t) = 1 - \sigma_1^{2t}$ and $\sigma_1$ is the predefined hyper-parameter controlling the variance of input distribution at $t=1$ under linear entropy accuracy schedule. The variance parameter $\vrho$ does not need to be modeled and fed to the network, since it is deterministic given the accuracy schedule. After sampling $\bmuL_i$ from $p_F^L$, the training objective is defined as minimizing KL divergence between sender and receiver distribution (based on Eq. 96 in \citet{bfn}):
\begin{align}
\mathcal{L}_{L} = \frac{n}{2}\left(1-\sigma_1^{2/n}\right)\E_{i \sim \ui{n}}\E_{\flow(\bmuL_{i-1} |\vL ; t_{i-1})}  \frac{\left\|\vL -\predL{i-1}\right\|^2}{\sigma_1^{2i/n}},\label{eq:lattice_loss}
\end{align}
where the prediction term $\predL{i-1}$ is the lattice parameter part of network output. After training, the generation process is defined as the Bayesian update distribution given network prediction:
\begin{equation}\label{eq:lattice_sampling}
    p(\bmuL_{i}|\parsnt{i-1}^\calM)=\update^L(\bmuL_{i}|\predL{i-1},\bmuL_{i-1};t_{i-1})
\end{equation}
    

% The final prediction of the lattice parameter is given by $\bmuL_n = \predL{n-1}$.
% \begin{equation}\label{eq:final_lattice}
%     \bmuL_n = \predL{n-1}
% \end{equation}

\begin{restatable}{proposition}{latticeinv}\label{prop:latticeinv}
With $\net_{L}$ as  O(3)-equivariant function namely $\net_L(\parsnt{}^A,\parsnt{}^F,\vQ\parsnt{}^L,t)=\vQ\net_L(\parsnt{}^A,\parsnt{}^F,\parsnt{}^L,t),\forall\vQ^T\vQ=\vI$, the marginal distribution of $p(\bmuL_n)$ defined by \cref{eq:lattice_sampling} is O(3)-invariant. 
\end{restatable}


\textbf{Bayesian Flow of Atom Types \texorpdfstring{$\boldsymbol{A}$}{}} 
Given that atom types are discrete random variables located in a simplex $\calS^K$, the prior parameter of $\boldsymbol{A}$ is the discrete uniform distribution over the vocabulary $\parsnt{0}^A \defeq \frac{1}{K}\vec{1}_{1\times N}$. 
% \begin{align}\label{eq:disc_input_prior}
% \parsnt{0}^A \defeq \frac{1}{K}\vec{1}_{1\times N}
% \end{align}
% \begin{align}
%     (\oh{j}{K})_k \defeq \delta_{j k}, \text{where }\oh{j}{K}\in \R^{K},\oh{\vA}{KD} \defeq \left(\oh{a_1}{K},\dots,\oh{a_N}{K}\right) \in \R^{K\times N}
% \end{align}
With the notation of the projection from the class index $j$ to the length $K$ one-hot vector $ (\oh{j}{K})_k \defeq \delta_{j k}, \text{where }\oh{j}{K}\in \R^{K},\oh{\vA}{KD} \defeq \left(\oh{a_1}{K},\dots,\oh{a_N}{K}\right) \in \R^{K\times N}$, the Bayesian flow distribution of atom types $\vA$ is derived in \citet{bfn}:
\begin{align}
\flow^{A}(\parsn^A \mid \vA; t) &= \E_{\N{\y \mid \beta^A(t)\left(K \oh{\vA}{K\times N} - \vec{1}_{K\times N}\right)}{\beta^A(t) K \vec{I}_{K\times N \times N}}} \delta\left(\parsn^A - \frac{e^{\y}\parsnt{0}^A}{\sum_{k=1}^K e^{\y_k}(\parsnt{0})_{k}^A}\right).
\end{align}
where $\beta^A(t)$ is the predefined accuracy schedule for atom types. Sampling $\btheta_i^A$ from $p_F^A$ as the training signal, the training objective is the $n$-step discrete-time loss for discrete variable \citep{bfn}: 
% \oyyw{can we simplify the next equation? Such as remove $K \times N, K \times N \times N$}
% \begin{align}
% &\calL_A = n\E_{i \sim U\{1,n\},\flow^A(\parsn^A \mid \vA ; t_{i-1}),\N{\y \mid \alphat{i}\left(K \oh{\vA}{KD} - \vec{1}_{K\times N}\right)}{\alphat{i} K \vec{I}_{K\times N \times N}}} \ln \N{\y \mid \alphat{i}\left(K \oh{\vA}{K\times N} - \vec{1}_{K\times N}\right)}{\alphat{i} K \vec{I}_{K\times N \times N}}\nonumber\\
% &\qquad\qquad\qquad-\sum_{d=1}^N \ln \left(\sum_{k=1}^K \out^{(d)}(k \mid \parsn^A; t_{i-1}) \N{\ydd{d} \mid \alphat{i}\left(K\oh{k}{K}- \vec{1}_{K\times N}\right)}{\alphat{i} K \vec{I}_{K\times N \times N}}\right)\label{discdisc_t_loss_exp}
% \end{align}
\begin{align}
&\calL_A = n\E_{i \sim U\{1,n\},\flow^A(\parsn^A \mid \vA ; t_{i-1}),\N{\y \mid \alphat{i}\left(K \oh{\vA}{KD} - \vec{1}\right)}{\alphat{i} K \vec{I}}} \ln \N{\y \mid \alphat{i}\left(K \oh{\vA}{K\times N} - \vec{1}\right)}{\alphat{i} K \vec{I}}\nonumber\\
&\qquad\qquad\qquad-\sum_{d=1}^N \ln \left(\sum_{k=1}^K \out^{(d)}(k \mid \parsn^A; t_{i-1}) \N{\ydd{d} \mid \alphat{i}\left(K\oh{k}{K}- \vec{1}\right)}{\alphat{i} K \vec{I}}\right)\label{discdisc_t_loss_exp}
\end{align}
where $\vec{I}\in \R^{K\times N \times N}$ and $\vec{1}\in\R^{K\times D}$. When sampling, the transition from $\bthetaA_{i-1}$ to $\bthetaA_{i}$ is derived as:
\begin{equation}
    p(\btheta^A_{i}|\parsnt{i-1}^\calM)=\update^A(\btheta^A_{i}|\btheta^A_{i-1},\predA{i-1};t_{i-1})
\end{equation}

The detailed training and sampling algorithm could be found in \cref{alg:train} and \cref{alg:sampling}.




\section{Experiments}\label{sec:expts}
%! TEX Root = ../main.tex

First, we provide details about our architecture and training. 
Then, for each task, we discuss the baselines, the evaluation metrics and datasets, and show quantitative and qualitative results.
We also present an ablation study that evaluates the contribution of different components of our approach.
Finally, we show a way to extend our approach to new tasks by using motion-based segmentation as an example.

\subsection{Implementation}\label{sec:impl}
\chapter{Implementation}{\label{ch:implementation}}
In this chapter, we present the implementation of the final product. We start by discussing how the four steps introduced in \hyperref[ch:high_level_approach]{chapter \ref*{ch:high_level_approach}} are integrated. We then outline the main system components of our score follower, presenting each as an independent, self-contained module. We then combine this into an overall system architecture and finally introduce the open-source score renderer used to display the score and evaluate the score follower.       

% \section{Aims and Requirements}
% The overall aim of the score follower was to 


\section{Score Follower Framework Details}
Our score follower conforms to the high-level framework presented in \hyperref[section:score_follower_framework]{section \ref*{section:score_follower_framework}}. In step 1, two score features are extracted from a MIDI file (see \hyperref[subsection:midi]{subsection \ref*{subsection:midi}}), namely MIDI note numbers\footnote{\href{https://inspiredacoustics.com/en/MIDI_note_numbers_and_center_frequencies}{https://inspiredacoustics.com/en/MIDI\_note\_numbers\_and\_center\_frequencies}} (corresponding to pitch) and note onsets (corresponding to duration). In step 2, the audio is streamed (whether from a file or into a microphone) and audioframes that exceed some predefined energy threshold are extracted. Here, audioframes are groups of contiguous audio samples, whose length can be specified by the argument \verb|frame_length|, usually between 800 and 2000 samples. The period between consecutive audioframes can also be defined by the argument \verb|hop_length|, typically between 2000 and 5000 audio samples. In step 3, score following is performed via a `Windowed' Viterbi algorithm (see  \hyperref[subsection:adjusting_viterbi]{subsection \ref*{subsection:adjusting_viterbi}}) which uses the Gaussian Process (GP) log marginal likelihoods (LMLs) for emission probabilities (see \hyperref[section:state_duration_model]{section \ref*{section:state_duration_model}}) and a state duration model for transition probabilities (see \hyperref[section:state_duration_model]{section \ref*{section:state_duration_model}}). Finally, in step 4 we render our results using an adapted version of the open source user interface, \textit{Flippy Qualitative Testbench}.

\section{Following Modes}
Two modes are available to the user: Pre-recorded Mode and Live Mode. The former requires a pre-recorded $\verb|.wav|$ file, whereas the latter takes an input stream of audio via the device's microphone. Note that both modes are still forms of score following, as opposed to score alignment, since in each mode we receive audioframes at the sampling rate, not all at once.\\

Live Mode offers a practical example of a score follower, displaying a score and position marker which a musician can read off while playing. However, this mode is not suitable for evaluation because the input and results cannot be easily replicated. Even ignoring repeatability, Live Mode is not suitable for one-off testing since a musician using this application may be influenced by the movement of the marker. For instance, the performer may speed up if the score follower `gets ahead' or slow down if the position marker lags or `gets lost'. To avoid this, we use Pre-recorded Mode when evaluating the performance of our score follower. Furthermore, Pre-recorded Mode offers the advantage of testing away from the music room, providing the opportunity to evaluate a variety of recordings available online. 

\section{System Architecture}
Our guiding principle for development was to build modular code in order to create a streamlined system where each component performs a specific task independently. This structure facilitates easy testing and debugging. \hyperref[fig:black_box]{Figure \ref*{fig:black_box}} presents a high-level architecture diagram, where each black box abstracts a key component of the score follower. When operating in Pre-recorded Mode, there is the option to stream the recording during run-time, which outputs to the device's speakers (as indicated by the dashed lines).

\begin{figure}[H]
    \centering
    \includegraphics[width=1\textwidth]{figs/Part_4_Implementation_And_Results/black_box.png}
    \caption{Abstracted system architecture diagram displaying inputs in grey, the 4 main components of the score follower in black and the outputs in green.}
    \label{fig:black_box}
\end{figure}

\subsection{Score Preprocessor}
The architecture for the Score Preprocesor is given in \hyperref[fig:score_preprocessor]{Figure \ref*{fig:score_preprocessor}}. First, MIDI note number and note onset times are extracted from each MIDI event. Simultaneous notes can be gathered into states and returned as a time-sorted list of lists called \verb|score|, where each element of the outer list is a list of simultaneous note onsets. Similarly, a list of note durations calculated as the time difference between consecutive states is returned as \verb|times_to_next|. Finally, all covariance matrices are precalculated and stored in a dictionary, where the key of the dictionary is determined by the notes present. This is because the distribution of notes and chords in a score is not random: notes tend to belong to a home \gls{key} and melodies tend to be repeated or related (similar to subject fields in speech processing). Therefore, states tend to be reused often, allowing us to achieve amortised time and space savings (by avoiding repeated calculation of the same covariance matrices). 

\begin{figure}[H]
    \centering
    \includegraphics[width=1\textwidth]{figs/Part_3_Implementation/Stage_2_Alignment/score_preprocessor.png}
    \caption{System architecture diagram representing the Score Preprocessor with inputs in grey, processes in blue and objects in yellow.}
    \label{fig:score_preprocessor}
\end{figure}


\subsection{Audio Preprocessor}
The architecture for the Audio Preprocessor is illustrated in \hyperref[fig:audio_preprocessor]{Figure \ref*{fig:audio_preprocessor}}. In Pre-recorded Mode, the Slicer receives a $\verb|.wav|$ file and returns audioframes separated by the \verb|hop_length|. These audioframes are periodically added to a multiprocessing queue, \verb|AudioFramesQueue|, to simulate real-time score following. In Live Mode, we use the python module \verb|sounddevice| to receive a stream of audio, using a periodic callback function to place audioframes on \verb|AudioFramesQueue|. 

\begin{figure}[H]
    \centering
    \includegraphics[width=1\textwidth]{figs/Part_4_Implementation_And_Results/audio_preprocessor.png}
    \caption{System architecture diagram representing the Audio Preprocessor with inputs in grey, processes in blue and objects in yellow.}
    \label{fig:audio_preprocessor}
\end{figure}

\subsection{Follower and Backend}
The joint Follower and Backend architecture diagram is shown in \hyperref[fig:follwer_and_backend]{Figure \ref*{fig:follwer_and_backend}}. The Viterbi Follower (detailed in \hyperref[subsection:adjusting_viterbi]{section \ref*{subsection:adjusting_viterbi}}) calculates the most probable state in the score, given audioframes continually taken from \verb|AudioFramesQueue|. These states are placed on another multiprocessing queue, the \verb|FollowerOutputQueue|, for the Backend to process and send. This prevents any bottle-necking occurring at the Follower stage. The Backend first sets up a UDP connection and then reads off values from \verb|FollowerOutputQueue|, sending them via UDP packets to the score renderer.

\begin{figure}[H]
    \centering
    \includegraphics[width=1\textwidth]{figs/Part_4_Implementation_And_Results/follower_and_backend.png}
    \caption{System architecture diagram representing the Follower and Backend processes with processes in blue, objects in yellow and outputs in green.}
    \label{fig:follwer_and_backend}
\end{figure}

\subsection{Player}
In Pre-recorded Mode, the Player sets up a new process and begins streaming the recording once the Follower process begins. This provides a baseline for testing purposes, as a trained musician can observe the score position marker and judge how well it matches the music. 

\subsection{Overall System Architecture}
The overall system architecture is presented in \hyperref[fig:overall_system_architecture]{Figure \ref*{fig:overall_system_architecture}}. Since the Follower runs a real-time, time sensitive process, parallelism is employed to reduce the total system latency. We use two \verb|multiprocessing| queues\footnote{\href{https://docs.python.org/3/library/multiprocessing.html}{https://docs.python.org/3/library/multiprocessing.html}} to avoid bottle-necking, which allows us to run 4 concurrent processes (Audio Preprocessor, Follower, Backend, and Audio Player). Hence, this architecture allows the components to run independently of one another to avoid blocking. Furthermore, this allows the system to take advantage of the multiple cores and high computational power offered by most modern machines.  

\begin{figure}[H]
    \centering
    \includegraphics[width=1\textwidth]{figs/Part_4_Implementation_And_Results/overall_score_follower_2.png}
    \caption{System architecture diagram representing the overall score follower running in Pre-recorded mode, with inputs in grey, processes in blue, objects in yellow and outputs in green.}
    \label{fig:overall_system_architecture}
\end{figure}


\section{Rendering Results}{\label{section:renderer}}
To visualise the results of our score follower, we adapted an open source tool for testing different score followers.\footnote{\href{https://github.com/flippy-fyp/flippy-qualitative-testbench/blob/main/README.md}{https://github.com/flippy-fyp/flippy-qualitative-testbench/blob/main/README.md}} \hyperref[fig:flippy_example]{Figure \ref*{fig:flippy_example}} shows the user interface of the score position renderer, where the green bar indicates score position. 

\begin{figure}[H]
    \centering
    \includegraphics{figs/Part_4_Implementation_And_Results/example_renderer.png}
    \caption{Screenshot of the score renderer user interface which displays a score (here we show a keyboard arrangement of \textit{O Haupt voll Blut und Wunden} by Bach). The green marker represents the score follower position.}
    \label{fig:flippy_example}
\end{figure}






\begin{table*}[htbp]
  \centering
  \caption{Quantitative comparison of depth estimation with both specialized models and multi-task models on zero-shot datasets. Our visual generalist model can perform \textit{on par} with state-of-the-art models.
  We use the same evaluation protocal ($\dagger$) as Genpercept~\cite{xu2024diffusion}.
  }
  
  
\resizebox{.99\linewidth}{!}{%
  \begin{tabular}{@{}r|c|lr|lr|lr|lr|lr@{}}
    \toprule
	
	\multirow{2}{*}{Method} & Training & \multicolumn{2}{c|}{KITTI~\cite{kitti}}  & \multicolumn{2}{c|}{NYUv2~\cite{nyu}} & \multicolumn{2}{c|}{ScanNet~\cite{scannet}}
 & \multicolumn{2}{c|}{DIODE~\cite{diode}} & \multicolumn{2}{c}{ETH3D~\cite{eth3d}}\\
	
    \cline{3-12}
	
    & Samples &  AbsRel$\downarrow$ & $\delta_1$$\uparrow$ & AbsRel$\downarrow$ & $\delta_1$$\uparrow$ & AbsRel$\downarrow$ & $\delta_1$$\uparrow$ & AbsRel$\downarrow$ & $\delta_1$$\uparrow$ & AbsRel$\downarrow$ & $\delta_1$$\uparrow$ \\


    \hline
       
    MiDaS~\cite{midas}   & 2M	    & 0.236  & 0.630
     		& 0.111	& 0.885
                & 0.121 & 0.846
     		& 0.332	& 0.715
                & 0.184  & 0.752
     		\\
       
    Omnidata~\cite{omnidata}  & 12.2M	& 0.149  & 0.835
     		& 0.074	& 0.945
                & 0.075 & 0.936
     		& 0.339	& 0.742
                & 0.166  & 0.778
     		\\
       
    DPT-large~\cite{dptlarge}  & 1.4M	& 0.100  & 0.901
     		& 0.098	& 0.903
                & 0.082 & 0.934
     		& 0.182	& 0.758
                & 0.078 & 0.946
     		\\

    DepthAnything$^\dagger$~\cite{yang2024depth}  & 63.5M	& 0.080  & 0.946
     		& 0.043	& 0.980
                & 0.043  & 0.981
     		& 0.261	& 0.759
                & 0.058  & \textbf{0.984}
     		\\

    DepthAnything v2$^\dagger$~\cite{yang2024depth2}  & 62.6M	& 0.080  & 0.943
     		& 0.043	& 0.979
                & 0.042  & 0.979
     		& 0.321	& 0.758
                & 0.066  & 0.983
     		\\

    Depth Pro$^\dagger$~\cite{bochkovskii2024depth}  & -	& 0.055  & 0.974
     		& 0.042	& 0.977
                & 0.041  & 0.978
     		& 0.217	& 0.764
                & 0.043  & 0.974
     		\\

    Metric3D v2$^\dagger$~\cite{hu2024metric3d}   & 16M	& \textbf{0.052}  & \textbf{0.979}
     		& \textbf{0.039}	& \textbf{0.979}
                & \textbf{0.023}  & \textbf{0.989}
     		& \textbf{0.147}	& \textbf{0.892}
                & \textbf{0.040}  & 0.983
     		\\
    
    % \hline
    % \hline

    DiverseDepth~\cite{diversedepth}  & 320K 	& 0.190  & 0.704
     		& 0.117	& 0.875
                & 0.109 & 0.882
     		& 0.376	& 0.631
                & 0.228 & 0.694
     		\\
       
    LeReS~\cite{leres}  & 354K	    & 0.149  & 0.784
     		& 0.090	& 0.916
                & 0.091 & 0.917
     		& 0.271	& 0.766
                & 0.171 & 0.777
     		\\
       
    HDN~\cite{hdn}  & 300K	    & 0.115  & 0.867
     		& 0.069	& 0.948
                & 0.080 & 0.939
     		& 0.246	& 0.780
                & 0.121  & 0.833
     		\\

    GeoWizard~\cite{fu2024geowizard}  & 280K & 0.097  & 0.921
     		& 0.052	& 0.966
                & 0.061 & 0.953
     		& 0.297	& 0.792
                & 0.064  & 0.961
     		\\

    DepthFM~\cite{gui2024depthfm}  & 63K	& 0.083  & 0.934
     		& 0.065	& 0.956
                &  - & -
     		& 0.225 & 0.800
                & -  & -
     		\\
            
    % \hline

    Marigold$^\dagger$~\cite{ke2024repurposing}  & 74K	& 0.099  & 0.916
     		& 0.055	& 0.964
                & 0.064  & 0.951
     		& 0.308	& 0.773
                & 0.065  & 0.960
     		\\

    DMP Official$^\dagger$~\cite{lee2024exploiting}  & -  & 0.240  & 0.622
     		& 0.109	& 0.891
                & 0.146    & 0.814
     		& 0.361 	& 0.706
                & 0.128    &  0.857
     		\\

    GeoWizard$^\dagger$~\cite{fu2024geowizard}  & 280K & 0.129  & 0.851
     		& 0.059	& 0.959
                & 0.066  & 0.953
     		& 0.328	& 0.753
                & 0.077  & 0.940 
     		\\

    DepthFM$^\dagger$~\cite{gui2024depthfm}  & 63K	& 0.174  & 0.718
     		& 0.082	& 0.932
                & 0.095  & 0.903
     		& 0.334 	& 0.729
                & 0.101  & 0.902
     		\\
    Genpercept$^\dagger$~\cite{xu2024diffusion}  & 90K	& 0.094  & 0.923
     		& 0.091	& 0.932
                & 0.056  & 0.965
     		& 0.302	& 0.767
                & 0.066  & 0.957
     		\\
    \hline
    Painter$^\dagger$~\cite{wang2023images}  & 24K	& 0.324  & 0.393
        & \textbf{0.046}	& \textbf{0.979}
        & 0.083  & 0.927
        & 0.342	& 0.534
        & 0.203  & 0.644
        \\
    Unified-IO$^\dagger$~\cite{lu2022unified}  & 48K	& 0.188  & 0.699
        & 0.059	& 0.970
        & \textbf{0.063}  & \textbf{0.965}
        & 0.369	& 0.906
        & 0.103  & 0.906
        \\
    4M-XL$^\dagger$~\cite{mizrahi20234m}  & 759M	& 0.105 & 0.896
        & 0.068	& 0.951
        & 0.065  & 0.955
        & 0.331	& 0.734
        & 0.070  & 0.953
        \\
    OneDiffusion$^\dagger$ ~\cite{le2024diffusiongenerate} & 500K	& 0.101  & 0.908
        & 0.087	& 0.924
        & 0.094  & 0.906
        & 0.399	& 0.661
        & 0.072  & 0.949
        \\
            
    \hline
    \textcolor{gray}{Ours-single}$^\dagger$  & \textcolor{gray}{500K}	& \textcolor{gray}{0.081}  & \textcolor{gray}{0.942}
     		& \textcolor{gray}{0.068}	& \textcolor{gray}{0.949}
                & \textcolor{gray}{0.078}  & \textcolor{gray}{0.945}
     		& \textcolor{gray}{0.267}	& \textcolor{gray}{0.709}
                & \textcolor{gray}{0.059}  & \textcolor{gray}{0.969}
     		\\
    Ours$^\dagger$  & 500K	& \textbf{0.075}  & \textbf{0.945}
     		& 0.072	& 0.939
                & 0.075  & 0.938
     		& \textbf{0.243}	& \textbf{0.741}
                & \textbf{0.053}  & \textbf{0.967}
     		\\
     
    \bottomrule
  \end{tabular}
  }
  \label{tab:depth}
  %\vspace{-2 em}
\end{table*}



\begin{figure}
    \centering
    \includegraphics[width=0.99\columnwidth]{figures/depth_results/depth_figure.pdf}
    \caption{
        \textbf{Qualitative results for depth estimation.}
        We include one example each from Bonn, KITTI, and ScanNet. 
        Inference is conducted on 16-frame clips, but only 1 frame is shown. 
        }\label{fig:depth}
\end{figure}

\subsection{Video Depth Estimation}


We follow DepthCrafter~\cite{hu2024depthcrafter} and evaluate video depth estimation on a collection of five datasets.
We do not use any of the datasets for training our models or the baselines to better understand their generalization abilities.
There is an inherent scale-ambiguity in the estimated depthmaps. 
We follow the common practice of aligning linearly the estimation with the GT before calculating evaluation metrics.
The alignment is done for all the frames at once, and is carried out in \emph{disparity} space via least-square fitting.
For comparison on single image datasets, we repeat the single frame 16 times to compute our estimations.
We report two metrics: AbsRel ($\text{mean}(|\hat{\mathbf{d}}-\mathbf{d}| / \mathbf{d}))$ and $\delta_1$ (ratio of pixels satisfying $\max(\mathbf{d}/\hat{\mathbf{d}}, \hat{\mathbf{d}}/\mathbf{d})<1.25$), where $\mathbf{d}$ represents GT, and $\hat{\mathbf{d}}$ is depth estimation after alignment. 
We upsample our estimations from $224\times224$ to each dataset's original resolution for evaluation.

We consider video approaches including NVDS~\cite{wang2023nvds}, ChronoDepth~\cite{shao2024chronodepth}, DepthCrafter~\cite{hu2024depthcrafter}, 
as well as single-image ones, including Marigold~\cite{ke2024repurposing} and DepthAnything~\cite{yang2024depthanything,yang2024depthanythingv2}. 
Among them, DepthCrafter~\cite{hu2024depthcrafter} and DepthAnything~\cite{yang2024depthanything,yang2024depthanythingv2} each represent the SOTA respectively. 
Marigold and DepthCrafter are diffusion models, which afford impressive levels of details, but require an expensive iterative denoising process. 

Our results show consistent advantages over both SOTA single-image and video depth approaches on the four video datasets (Table~\ref{tab:depth}). 
Since \methodName is a video approach, applying it on single images from NYUv2 does not provide the necessary temporal context for it to perform well. 
DepthCrafter also similarly suffers on NYUv2. 
Figure~\ref{fig:depth} shows qualitative samples and comparison with select SOTA approaches. 
\methodName produces a level of details comparable to that of diffusion models such as DepthCrafter, while generally capturing more accurate relative scales.

\noindent
\textbf{Discussion.} Our final model performs on par with a specialized depth model (Table~\ref{tab:depth}), 
despite optimized jointly for all of our tasks. 
Scale alignment between windows for online inference makes a significant impact on ScanNet. 
This is due to the fast-paced view change in ScanNet samples making scale inconsistency between windows more prominent. 
It is also worth noting that \methodName performs competitively on KITTI, despite not fine-tuned on synthetic datasets that include driving scenarios.

%! TEX Root = ../main.tex



\begin{table}
    \begin{center}
        \resizebox{\columnwidth}{!}{
        \begin{tabular}{rccccccc}
            \toprule
            & \multicolumn{2}{c}{Kubric} & \multicolumn{2}{c}{Dynamic Replica} & \multicolumn{2}{c}{Spring} \\
            \cmidrule(lr{0.1em}){2-3}\cmidrule(lr{0.1em}){4-5}\cmidrule(lr{0.1em}){6-7}
            & $EPE\downarrow$ & $EPE<1\uparrow$ & $EPE\downarrow$ & $EPE<1\uparrow$ & $EPE\downarrow$ & $EPE<1\uparrow$ \\
            \midrule
            RAFT*~\cite{teed2020raft}  & $0.31$ & $94.6$ & $0.14$ & $98.7$ & $0.13$ & $98.4$ \\
            MemFlow~\cite{dong2024memflow} &  $0.27$ & $95.6$ & $0.11$ & $99.2$ & $0.13$ & $98.4$ \\
            Ours &  $(0.13)$ & $(97.6)$ & $(0.03)$ & $(99.9)$ & $\mathbf{0.10}$ & $\mathbf{98.5}$ \\
            \bottomrule
        \end{tabular}
        }
    \end{center}
    \caption{
        \textbf{Optical flow estimation results.} 
        We evaluate on validation sets of all datasets. 
        Our model has seen Kubric and Dynamic Replica during training (numbers in brackets). 
        All others show cross-dataset generalization.
        On Spring, we surpass both two-frame (marked with *) and multi-frame SOTA baselines, despite the latter having specially designed architectures and a complex memory mechanism. 
    }
    \label{tab:flow}
\end{table}




\begin{figure}
    \centering
    \includegraphics[width=0.99\columnwidth]{figures/flow_results/flow_figure.pdf}
    \caption{
        \textbf{Qualitative results for optical flow estimation on Spring.}
        Our results compare favorably to baselines in terms of both details and accuracy. 
        Inference is conducted on 16-frame clips, but only 1 frame is shown. 
        }\label{fig:flow}
\end{figure}

\subsection{Multi-Frame Optical Flow Estimation}

We use the Spring dataset~\cite{mehl2023spring} for evaluation.
We sample 289 16-frame clips from the \emph{train} split.
Spring is not used to train ours or other approaches we compare against, allowing us to evaluate generalization ability.
The input frames are resized to $224\times224$ for all evaluation.
We use the Endpoint Error (EPE), as well as a more robust metric, ratio of EPE $<1$, for the evaluation. 

We consider two baselines for comparison. 
RAFT~\cite{teed2020raft}, a competitive and widely used two-frame approach, creates dense pairwise pixel features and uses recurrent updates to estimate optical flow. 
MemFlow~\cite{dong2024memflow}, a recently published work, ranks among the top methods on the Spring benchmark.
It is a multi-frame approach that relies on a memory mechanism to leverage temporal context. 
Quantitatively, \methodName compares favorably to both RAFT and MemFlow on Spring (Table~\ref{tab:flow}).
Our model can capture well both small and large motions and presents more precise motion boundaries (Figure~\ref{fig:flow}). 
In addition, multi-frame approaches like MemFlow and ours generally have an edge in temporal stability (see Supplementary). 
Unlike many specialized approaches, our model currently only operates on low-resolution videos and further work is needed to enable efficient high-res estimation.

%! TEX Root = ../main.tex

\begin{table}
    \begin{center}
        
    \resizebox{\columnwidth}{!}{
        \begin{tabular}{rccccccc}
            \toprule
            & Aria & DriveTrack & PStudio & \multicolumn{3}{c}{\textbf{Overall}} \\
            \cmidrule(lr{0.1em}){2-4}\cmidrule(lr{0.1em}){5-7}
            & 2D-AJ $\uparrow$ & 2D-AJ $\uparrow$ & 2D-AJ $\uparrow$
            & 2D-AJ $\uparrow$ & APD $\uparrow$ & OA $\uparrow$ \\
            \midrule
            TAPIR~\cite{doersch2023tapir}         & $48.6$ & $57.2$ & $48.7$ & $53.2$ & $67.4$ & $80.5$ \\
            BootsTAPIR~\cite{doersch2024bootstap} & $54.7$ & $\mathbf{62.9}$ & $\mathbf{52.4}$ & $\mathbf{59.1}$ & $\mathbf{74.7}$  & $85.6$ \\
            CoTracker~\cite{karaev2023cotracker}  & $54.2$ & $59.8$ & $51.0$ & $57.2$ & $74.2$ & $84.5$ \\
            \midrule
            Ours (2D Only)                       & $\mathbf{56.7}$ & $54.2$ & $49.8$ & $53.5$ & $69.4$  & $88.6$  \\
            Ours (w/o Mem)                          & $36.8$ & $47.4$ & $41.1$ & $41.8$ & $62.9$ & $78.6$  \\
            Ours                                  & $53.0$ & $51.6$ & $48.8$ & $51.2$ & $67.0$ & $\mathbf{88.7}$  \\
            \toprule
        \end{tabular}
    }
    \end{center}
    \caption{\textbf{Evalution of 2D point tracking on TAPVid-3D.} 
    2D GT trajectories are obtained by projecting 3D GT trajectories onto 2D. 
    Though behind 2D SOTA approaches, our model performs competitively once trained specifically for 2D tracking (``2D Only").}
    \label{table:tapvid3d2dtrack}
\end{table}


%! TEX Root = ../main.tex

\begin{table}
    \begin{center}
        \resizebox{\columnwidth}{!}{
        \setlength{\tabcolsep}{3pt}
        \begin{tabular}{rccccccccccccc}
            \toprule
            & \multicolumn{3}{c}{Aria} & \multicolumn{3}{c}{DriveTrack} & \multicolumn{3}{c}{PStudio} & \multicolumn{3}{c}{\textbf{Overall}} \\
            \cmidrule(lr{0.1em}){2-4}\cmidrule(lr{0.1em}){5-7}\cmidrule(lr{0.1em}){8-10}\cmidrule(lr{0.1em}){11-13}
            & 3D-AJ $\uparrow$ & APD $\uparrow$ & OA $\uparrow$ & 3D-AJ $\uparrow$ & APD $\uparrow$ & OA $\uparrow$ & 3D-AJ $\uparrow$ & APD $\uparrow$ & OA $\uparrow$
            & 3D-AJ $\uparrow$ & APD $\uparrow$ & OA $\uparrow$ \\
            \midrule
            Static Baseline       & $4.9$ & $10.2$ & $55.4$ & $3.9$ & $6.5$ & $80.8$ & $5.9$ & $11.5$ & $75.8$ & $4.9$ & $9.4$ & $70.7$ \\
            TAPIR + CM        & $7.1$ & $11.9$ & $72.6$ & $8.9$ & $14.7$ & $80.4$ & $6.1$ & $10.7$ & $75.2$ & $7.4$ & $12.4$ & $76.1$ \\
            CoTracker + CM    & $8.0$ & $12.3$ & $78.6$ & $11.7$ & $\mathbf{19.1}$ & $81.7$ & $8.1$ & $13.5$ & $77.2$ & $9.3$ & $15.0$ & $79.1$ \\
            BootsTAPIR + CM   & $9.1$ & $14.5$ & $78.6$ & $\mathbf{11.8}$ & $18.6$ & $83.8$ & $6.9$ & $11.6$ & $81.8$ & $9.3$ & $14.9$ & $81.4$ \\
            \midrule
            TAPIR + ZD      & $9.0$ & $14.3$ & $79.7$ & $5.2$ & $8.8$ & $81.6$ & $10.7$ & $18.2$ & $78.7$ & $8.3$ & $13.8$ & $80.0$ \\
            CoTracker + ZD  & $10.0$ & $15.9$ & $87.8$ & $5.0$ & $9.1$ & $82.6$ & $11.2$ & $19.4$ & $80.0$ & $8.7$ & $14.8$ & $83.4$ \\
            BootsTAPIR + ZD & $9.9$ & $16.3$ & $86.5$ & $5.4$ & $9.2$ & $85.3$ & $11.3$ & $19.0$ & $82.7$ & $8.8$ & $14.8$ & $84.8$ \\
            TAPIR-3D              & $2.5$ & $4.8$  & $86.0$ & $3.2$ & $5.9$ & $83.3$ & $3.6$ & $7.0$ & $78.9$ & $3.1$ & $5.9$ & $82.8$ \\
            SpatialTracker        & $9.9$ & $16.1$ & $89.0$ & $6.2$ & $11.1$ & $83.7$ & $10.9$ & $19.2$ & $78.6$ & $9.0$ & $15.5$ & $83.7$ \\
            Ours (w/o Mem)          & 8.2 & 15.4  & 72.7  & 5.5   & 10.0 & 83.3 & $15.1$ & $25.2$ & $79.9$ & $9.6$ & $16.9$ & $78.6$ \\
            Ours                  & $\mathbf{11.2}$ & $\mathbf{17.7}$  & $\mathbf{90.3}$  & $6.6$   & $11.4$ & $\mathbf{88.1}$ & $\mathbf{18.6}$ & $\mathbf{28.2}$ & $\mathbf{87.6}$ & $\mathbf{12.1}$ & $\mathbf{19.1}$ & $\mathbf{88.7}$ \\
            \bottomrule
        \end{tabular}
        }
    \end{center}
    \caption{
        \textbf{Evaluation of 3D tracking on the \textit{full\_eval} split of TAPVid-3D.} 
        The top approaches combine 2D point tracking approaches with COLMAP (CM)~\cite{schoenberger2016sfm}, while the bottom ones, including ours are feedforward.
        Our approach consistently outperforms previous feedforward works, and also COLMAP baselines on average.
        We also show the impact of our memory mechanism (Ours vs. Ours w/o Mem). ``ZD" refers to ZoeDepth.}
    \label{table:tapvid3d3dtrack}
\end{table}



\begin{figure*}
    \centering
    \includegraphics[width=0.99\textwidth]{figures/track_results/track_figure.pdf}
    \caption{\textbf{Qualitative results of Sparse 2D/3D tracking on the TAPVid-3D benchmark.}
    Comparison with SpaTracker, a SOTA 3D tracking approach, demonstrates the superior quality of our 2D and 3D tracks.
    For joint visualization of depth and 3D tracks, we align them using median scaling.
    We use our depth maps for visualization of GT and for SpaTracker we use the ones used by their approach. 
    }\label{fig:track}
\end{figure*}

\subsection{Sparse 2D/3D Track Estimation}
We evaluate on TAPVid-3D~\cite{koppula2024tapvid3d}, a benchmark containing around 2.1M long-range 3D point trajectories from over 4000 real-world videos, covering a variety of objects, camera and object motion patterns, and indoor and outdoor environments.
It consists of three datasets: Aria~\cite{pan2023aria}, DriveTrack~\cite{sun2020waymodataset}, and PStudio~\cite{joo2015pstudio}.
It introduced several baselines by combining SOTA 2D point tracking approaches, such as TAPIR~\cite{doersch2023tapir}, BootsTAPIR~\cite{doersch2024bootstap}, and CoTracker~\cite{karaev2023cotracker}, with depth solutions like ZoeDepth~\cite{bhat2023zoedepth}, a monocular depth estimation approach, and COLMAP~\cite{schoenberger2016sfm,schoenberger2016mvs}, a structure-from-motion pipeline.
The top performing approach on the benchmark is SpaTracker~\cite{spatracker}.

The benchmark evaluates both 3D and 2D tracking approaches, and uses metrics that measure the ability to predict point visibility using an occlusion accuracy metric (OA), the accuracy of predicted trajectories in the visible regions (APD), and joint occlusion and geometric estimation (AJ).
To resolve the scale ambiguity in depth estimation, the benchmark uses global median scaling by computing the median of the depth ratios between the estimated and ground-truth 3D tracks over all the points and frames in a video.
We use the \textit{full\_eval} split evaluation numbers provided in the TAPVid-3D benchmark for comparing approaches.

On 3D tracking, we outperform previous approaches on average across all the metrics (Table~\ref{table:tapvid3d3dtrack}).
Among feedforward approaches, we perform better on all the datasets.
Approaches that combine 2D track estimation with COLMAP perform better on the DriveTrack~\cite{sun2020waymodataset} dataset.
This could be due to a relatively large bias of tracking mostly static vehicles, where COLMAP gives much more accurate depth.
Such COLMAP-based baselines, however, perform poorly on Aria~\cite{pan2023aria} and PStudio~\cite{joo2015pstudio}, which are mostly dynamic.
We show qualitative evaluation against the SOTA SpaTracker approach in Figure~\ref{fig:track}.

On 2D tracking, we are slightly behind the SOTA 2D tracking approaches (Table~\ref{table:tapvid3d2dtrack}).
Our approach becomes more competitive and performs better than TAPIR on average when we fine-tune our model only for the 2D tracking task.
We believe our reduced performance on 2D tracking comes from working at lower image resolution, $224\times224$ for us as compared to $384\times512$ for CoTracker and $256\times256$ for others, and a lack of task-specific tricks, like tracking multiple points together (CoTracker) or assuming access to all frames in the video and performing a global track-refinement (TAPIR and BootsTAPIR), both of which could also benefit our tracking head.
We also ablate our online tracking approach on both 2D and 3D tracking benchmarks, and show improved performance due to the use of memory mechanism when tracking points from one window to next.
Overall, we attribute our strong performance to our unified approach and carefully designed sparse head.



\subsection{Ablations}\label{sec:ablation}
To understand the contribution of different components of our approach, we perform an ablation study for depth, flow, 2D and 3D point tracking, as shown in Table~\ref{tab:ablation}.
For each of these tasks, we report average over the datasets not used in our training: for depth we use datasets in Table~\ref{tab:depth}, for optical flow we use the Spring dataset, and for tracking we use the \textit{minival} split from the TAPVid-3D~\cite{koppula2024tapvid3d} benchmark.
Our main contribution is to show how to leverage the priors of a pretrained VideoMAE for multiple dense and sparse low-level 4D perception tasks at once.
To show the usefulness of our end-to-end fine-tuning strategy, we compare against a pretrained and frozen VideoMAE, where we only fine-tune the task-specific heads. 
Table~\ref{tab:ablation} shows that our fine-tuned VideoMAE (row 3) produces better results than the pretrained and frozen VideoMAE across all tasks (row 2). 
A version trained end-to-end from scratch results in worse performance (row 1), which shows that our system leverages the pretraining of the VideoMAE.
Finally, by adding the proposed memory mechanism for the tracking head and using our two-stage training process, we obtain improvements in both 2D and 3D tracking tasks, while maintaining the performance on other tasks.

\begin{table}
    \centering
    \resizebox{.95\columnwidth}{!}{
      \begin{tabular}{l|cccc}
        \multicolumn{1}{c|}{} & \begin{tabular}[c]{@{}c@{}}Depth\\ AbsRel$\downarrow$ / $\delta_1\uparrow$\end{tabular} & \begin{tabular}[c]{@{}c@{}}Optical flow\\ EPE$\downarrow$ / EPE$<1\uparrow$\end{tabular} & \begin{tabular}[c]{@{}c@{}}2D Track\\ 2D-AJ$\uparrow$\end{tabular} & \multicolumn{1}{c}{\begin{tabular}[c]{@{}c@{}}3D Track\\ 3D-AJ$\uparrow$\end{tabular}} \\ \cline{1-5}
        \multicolumn{1}{l|}{From scratch} & 0.259 / 0.594 & 0.246 / 96.2 & 16.6 & \multicolumn{1}{c}{1.3} \\
        \multicolumn{1}{l|}{VideoMAE frozen} & 0.137 / 0.841 & 0.120 / 98.2 & 29.3 & \multicolumn{1}{c}{3.3} \\
        \multicolumn{1}{l|}{Ours (w/o Mem)} & \textbf{0.120 / 0.876} & \textbf{0.100 / 98.5} & 41.1 & \multicolumn{1}{c}{8.7} \\
        \multicolumn{1}{l|}{Ours} & \textbf{0.120 / 0.876} & \textbf{0.100 / 98.5} & \textbf{50.2} & \multicolumn{1}{c}{\textbf{10.8}} \\
      \end{tabular}
    }
    \caption{\textbf{Ablation study.} Training using pre-trained VideoMAE performs better than training from scratch (row 3 vs. 1), which shows our approach leverages VideoMAE priors. Our approach performs better than using a frozen VideoMAE and only fine-tuning the heads (row 3 vs. 2), which shows end-to-end fine-tuning helps. Adding memory mechanism and two-stage training strategy improves tracking performance while maintaining performance on other tasks (row 4 vs. 3).}
    \label{tab:ablation}
\end{table} 



\subsection{Additional Task: Motion-based Segmentation}
We use the motion-based segmentation task to show one way to add a new task to our network.
We do this simply by freezing our trained video encoder and fine-tuning our proposed dense head for this task.
We generate the ground-truth annotations for training and evaluation by using \emph{video} datasets that provide camera, depth and 3D-motion information.
For training, we use the Kubric~\cite{greff2022kubric} dataset and fine-tune using binary cross entropy loss.
For evaluation, we use the Virtual KITTI (VKITTI)~\cite{cabon2020vkitti2} and Spring~\cite{mehl2023spring} datasets.
We compare against RigidMask (RM)~\cite{yang2021rigidmotion}, a SOTA two-frame rigid-motion segmentation approach that combines dynamic motion signals from flow, optical-expansion~\cite{yang2020upgrading} and depth.
It is trained on the SceneFlowDatasets~\cite{mayer2016scenflow}; however, they also train a version for driving scenarios (RM-Drive).
To evaluate, we report foreground IoU (higher is better) on VKITTI and Spring.
\setlength{\columnsep}{2pt}
\setlength{\intextsep}{0pt}
\begin{wrapfigure}[3]{r}[5pt]{.41\columnwidth}
  \resizebox{0.4\columnwidth}{!}{
    \begin{tabular}{c|c|c}
      & VKITTI & Spring \\
    \hline
    RM & 32.6 & 16.5\\ 
    RM-Drive & 35.4 & 8.5\\ 
    Ours & {\bf 46.7} & {\bf 23.7}
    \end{tabular}
    }
\end{wrapfigure}
On both datasets, our video-based approach achieves better performance.
Note that while fine-tuning on driving scenes allows RigidMask (RM-Drive) to reduce the gap slightly on VKITTI, it significantly hurts performance on Spring, highlighting the benefit of our model's generalization ability.
As shown in Figure~\ref{fig:dyn}, for both the indoor scenarios with human-object interactions and the outdoor driving scenarios, our approach performs better and can detect small motions (see more comparisons in Supplementary).
Freezing the video encoder and fine-tuning a task-specific head is the simplest way to add a new task that does not affect the performance of other tasks we train for.
Better strategies may exist that allow for some fine-tuning of the video encoder without affecting the performance of other tasks, though the investigation is outside the scope of this paper.

\begin{figure}
  \centering
  \includegraphics[width=0.99\columnwidth]{figures/dynamic/dyn_seg.pdf}
  \caption{\textbf{Qualitative results of motion-based segmentation.} 
  Samples are chosen from the TAPVid-3D benchmark.
  Across various scenarios, ours show advantages in small motions, boundary accuracy, as well as temporal consistency (see Supp.).
  The GT masks overlaid on images are only provided to identify qualitatively which objects are moving, but are not pixel-accurate.
  }\label{fig:dyn}
\end{figure}
\section{Conclusions}\label{sec:conclusion}
\section{Conclusion}
In this work, we propose a simple yet effective approach, called SMILE, for graph few-shot learning with fewer tasks. Specifically, we introduce a novel dual-level mixup strategy, including within-task and across-task mixup, for enriching the diversity of nodes within each task and the diversity of tasks. Also, we incorporate the degree-based prior information to learn expressive node embeddings. Theoretically, we prove that SMILE effectively enhances the model's generalization performance. Empirically, we conduct extensive experiments on multiple benchmarks and the results suggest that SMILE significantly outperforms other baselines, including both in-domain and cross-domain few-shot settings.
\section{Acknowledgements}
We would like to thank Jan Kautz for the continuous discussions and for reviewing an early draft of the paper, Zhiding Yu and Hongxu (Danny) Yin for the initial discussions on video models, and Yiqing Liang for help with the data. 

{
    \small
    \bibliographystyle{ieeenat_fullname}
    \bibliography{main}
}

\clearpage
\pagenumbering{gobble}
\maketitlesupplementary

\section{Additional Results on Embodied Tasks}

To evaluate the broader applicability of our EgoAgent's learned representation beyond video-conditioned 3D human motion prediction, we test its ability to improve visual policy learning for embodiments other than the human skeleton.
Following the methodology in~\cite{majumdar2023we}, we conduct experiments on the TriFinger benchmark~\cite{wuthrich2020trifinger}, which involves a three-finger robot performing two tasks: reach cube and move cube. 
We freeze the pretrained representations and use a 3-layer MLP as the policy network, training each task with 100 demonstrations.

\begin{table}[h]
\centering
\caption{Success rate (\%) on the TriFinger benchmark, where each model's pretrained representation is fixed, and additional linear layers are trained as the policy network.}
\label{tab:trifinger}
\resizebox{\linewidth}{!}{%
\begin{tabular}{llcc}
\toprule
Methods       & Training Dataset & Reach Cube & Move Cube \\
\midrule
DINO~\cite{caron2021emerging}         & WT Venice        & 78.03     & 47.42     \\
DoRA~\cite{venkataramanan2023imagenet}          & WT Venice        & 81.62     & 53.76     \\
DoRA~\cite{venkataramanan2023imagenet}          & WT All           & 82.40     & 48.13     \\
\midrule
EgoAgent-300M & WT+Ego-Exo4D      & 82.61    & 54.21      \\
EgoAgent-1B   & WT+Ego-Exo4D      & \textbf{85.72}      & \textbf{57.66}   \\
\bottomrule
\end{tabular}%
}
\end{table}

As shown in Table~\ref{tab:trifinger}, EgoAgent achieves the highest success rates on both tasks, outperforming the best models from DoRA~\cite{venkataramanan2023imagenet} with increases of +3.32\% and +3.9\% respectively.
This result shows that by incorporating human action prediction into the learning process, EgoAgent demonstrates the ability to learn more effective representations that benefit both image classification and embodied manipulation tasks.
This highlights the potential of leveraging human-centric motion data to bridge the gap between visual understanding and actionable policy learning.



\section{Additional Results on Egocentric Future State Prediction}

In this section, we provide additional qualitative results on the egocentric future state prediction task. Additionally, we describe our approach to finetune video diffusion model on the Ego-Exo4D dataset~\cite{grauman2024ego} and generate future video frames conditioned on initial frames as shown in Figure~\ref{fig:opensora_finetune}.

\begin{figure}[b]
    \centering
    \includegraphics[width=\linewidth]{figures/opensora_finetune.pdf}
    \caption{Comparison of OpenSora V1.1 first-frame-conditioned video generation results before and after finetuning on Ego-Exo4D. Fine-tuning enhances temporal consistency, but the predicted pixel-space future states still exhibit errors, such as inaccuracies in the basketball's trajectory.}
    \label{fig:opensora_finetune}
\end{figure}

\subsection{Visualizations and Comparisons}

More visualizations of our method, DoRA, and OpenSora in different scenes (as shown in Figure~\ref{fig:supp pred}). For OpenSora, when predicting the states of $t_k$, we use all the ground truth frames from $t_{0}$ to $t_{k-1}$ as conditions. As OpenSora takes only past observations as input and neglects human motion, it performs well only when the human has relatively small motions (see top cases in Figure~\ref{fig:supp pred}), but can not adjust to large movements of the human body or quick viewpoint changes (see bottom cases in Figure~\ref{fig:supp pred}).

\begin{figure*}
    \centering
    \includegraphics[width=\linewidth]{figures/supp_pred.pdf}
    \caption{Retrieval and generation results for egocentric future state prediction. Correct and wrong retrieval images are marked with green and red boundaries, respectively.}
    \label{fig:supp pred}
\end{figure*}

\begin{figure*}[t]
    \centering
    \includegraphics[width=0.9\linewidth]{figures/motion_prediction.pdf}
    \vspace{-0.5mm}
    \caption{Motion prediction results in scenes with minor changes in observation.}
    \vspace{-1.5mm}
    \label{fig:motion_prediction}
\end{figure*}

\subsection{Finetuning OpenSora on Ego-Exo4D}

OpenSora V1.1~\cite{opensora}, initially trained on internet videos and images, produces severely inconsistent results when directly applied to infer future videos on the Ego-Exo4D dataset, as illustrated in Figure~\ref{fig:opensora_finetune}.
To address the gap between general internet content and egocentric video data, we fine-tune the official checkpoint on the Ego-Exo4D training set for 50 epochs.
OpenSora V1.1 proposed a random mask strategy during training to enable video generation by image and video conditioning. We adopted the default masking rate, which applies: 75\% with no masking, 2.5\% with random masking of 1 frame to 1/4 of the total frames, 2.5\% with masking at either the beginning or the end for 1 frame to 1/4 of the total frames, and 5\% with random masking spanning 1 frame to 1/4 of the total frames at both the beginning and the end.

As shown in Fig.~\ref{fig:opensora_finetune}, despite being trained on a large dataset, OpenSora struggles to generalize to the Ego-Exo4D dataset, producing future video frames with minimal consistency relative to the conditioning frame. While fine-tuning improves temporal consistency, the moving trajectories of objects like the basketball and soccer ball still deviate from realistic physical laws. Compared with our feature space prediction results, this suggests that training world models in a reconstructive latent space is more challenging than training them in a feature space.


\section{Additional Results on 3D Human Motion Prediction}

We present additional qualitative results for the 3D human motion prediction task, highlighting a particularly challenging scenario where egocentric observations exhibit minimal variation. This scenario poses significant difficulties for video-conditioned motion prediction, as the model must effectively capture and interpret subtle changes. As demonstrated in Fig.~\ref{fig:motion_prediction}, EgoAgent successfully generates accurate predictions that closely align with the ground truth motion, showcasing its ability to handle fine-grained temporal dynamics and nuanced contextual cues.

\section{OpenSora for Image Classification}

In this section, we detail the process of extracting features from OpenSora V1.1~\cite{opensora} (without fine-tuning) for an image classification task. Following the approach of~\cite{xiang2023denoising}, we leverage the insight that diffusion models can be interpreted as multi-level denoising autoencoders. These models inherently learn linearly separable representations within their intermediate layers, without relying on auxiliary encoders. The quality of the extracted features depends on both the layer depth and the noise level applied during extraction.


\begin{table}[h]
\centering
\caption{$k$-NN evaluation results of OpenSora V1.1 features from different layer depths and noising scales on ImageNet-100. Top1 and Top5 accuracy (\%) are reported.}
\label{tab:opensora-knn}
\resizebox{0.95\linewidth}{!}{%
\begin{tabular}{lcccccc}
\toprule
\multirow{2}{*}{Timesteps} & \multicolumn{2}{c}{First Layer} & \multicolumn{2}{c}{Middle Layer} & \multicolumn{2}{c}{Last Layer} \\
\cmidrule(r){2-3}   \cmidrule(r){4-5}  \cmidrule(r){6-7}  & Top1           & Top5           & Top1            & Top5           & Top1           & Top5          \\
\midrule
32        &  6.10           & 18.20             & 34.04               & 59.50             & 30.40             & 55.74             \\
64        & 6.12              & 18.48              & 36.04               & 61.84              & 31.80         & 57.06         \\
128       & 5.84             & 18.14             & 38.08               & 64.16              & 33.44       & 58.42 \\
256       & 5.60             & 16.58              & 30.34               & 56.38              &28.14          & 52.32        \\
512       & 3.66              & 11.70            & 6.24              & 17.62              & 7.24              & 19.44  \\ 
\bottomrule
\end{tabular}%
}
\end{table}

As shown in Table~\ref{tab:opensora-knn}, we first evaluate $k$-NN classification performance on the ImageNet-100 dataset using three intermediate layers and five different noise scales. We find that a noise timestep of 128 yields the best results, with the middle and last layers performing significantly better than the first layer.
We then test this optimal configuration on ImageNet-1K and find that the last layer with 128 noising timesteps achieves the best classification accuracy.

\section{Data Preprocess}
For egocentric video sequences, we utilize videos from the Ego-Exo4D~\cite{grauman2024ego} and WT~\cite{venkataramanan2023imagenet} datasets.
The original resolution of Ego-Exo4D videos is 1408×1408, captured at 30 fps. We sample one frame every five frames and use the original resolution to crop local views (224×224) for computing the self-supervised representation loss. For computing the prediction and action loss, the videos are downsampled to 224×224 resolution.
WT primarily consists of 4K videos (3840×2160) recorded at 60 or 30 fps. Similar to Ego-Exo4D, we use the original resolution and downsample the frame rate to 6 fps for representation loss computation.
As Ego-Exo4D employs fisheye cameras, we undistort the images to a pinhole camera model using the official Project Aria Tools to align them with the WT videos.

For motion sequences, the Ego-Exo4D dataset provides synchronized 3D motion annotations and camera extrinsic parameters for various tasks and scenes. While some annotations are manually labeled, others are automatically generated using 3D motion estimation algorithms from multiple exocentric views. To maximize data utility and maintain high-quality annotations, manual labels are prioritized wherever available, and automated annotations are used only when manual labels are absent.
Each pose is converted into the egocentric camera's coordinate system using transformation matrices derived from the camera extrinsics. These matrices also enable the computation of trajectory vectors for each frame in a sequence. Beyond the x, y, z coordinates, a visibility dimension is appended to account for keypoints invisible to all exocentric views. Finally, a sliding window approach segments sequences into fixed-size windows to serve as input for the model. Note that we do not downsample the frame rate of 3D motions.

\section{Training Details}
\subsection{Architecture Configurations}
In Table~\ref{tab:arch}, we provide detailed architecture configurations for EgoAgent following the scaling-up strategy of InternLM~\cite{team2023internlm}. To ensure the generalization, we do not modify the internal modules in InternML, \emph{i.e.}, we adopt the RMSNorm and 1D RoPE. We show that, without specific modules designed for vision tasks, EgoAgent can perform well on vision and action tasks.

\begin{table}[ht]
  \centering
  \caption{Architecture configurations of EgoAgent.}
  \resizebox{0.8\linewidth}{!}{%
    \begin{tabular}{lcc}
    \toprule
          & EgoAgent-300M & EgoAgent-1B \\
          \midrule
    Depth & 22    & 22 \\
    Embedding dim & 1024  & 2048 \\
    Number of heads & 8     & 16 \\
    MLP ratio &    8/3   & 8/3 \\
    $\#$param.  & 284M & 1.13B \\
    \bottomrule
    \end{tabular}%
    }
  \label{tab:arch}%
\end{table}%

Table~\ref{tab:io_structure} presents the detailed configuration of the embedding and prediction modules in EgoAgent, including the image projector ($\text{Proj}_i$), representation head/state prediction head ($\text{MLP}_i$), action projector ($\text{Proj}_a$) and action prediction head ($\text{MLP}_a$).
Note that the representation head and the state prediction head share the same architecture but have distinct weights.

\begin{table}[t]
\centering
\caption{Architecture of the embedding ($\text{Proj}_i$, $\text{Proj}_a$) and prediction ($\text{MLP}_i$, $\text{MLP}_a$) modules in EgoAgent. For details on module connections and functions, please refer to Fig.~2 in the main paper.}
\label{tab:io_structure}
\resizebox{\linewidth}{!}{%
\begin{tabular}{lcl}
\toprule
       & \multicolumn{1}{c}{Norm \& Activation} & \multicolumn{1}{c}{Output Shape}  \\
\midrule
\multicolumn{3}{l}{$\text{Proj}_i$ (\textit{Image projector})} \\
\midrule
Input image  & -          & 3$\times$224$\times$224 \\
Conv 2D (16$\times$16) & -       & Embedding dim$\times$14$\times$14    \\
\midrule
\multicolumn{3}{l}{$\text{MLP}_i$ (\textit{State prediction head} \& \textit{Representation head)}} \\
\midrule
Input embedding  & -          & Embedding dim \\
Linear & GELU       & 2048          \\
Linear & GELU       & 2048          \\
Linear & -          & 256           \\
Linear & -          & 65536     \\
\midrule
\multicolumn{3}{l}{$\text{Proj}_a$ (\textit{Action projector})} \\
\midrule
Input pose sequence  & -          & 4$\times$5$\times$17 \\
Conv 2D (5$\times$17) & LN, GELU   & Embedding dim$\times$1$\times$1    \\
\midrule
\multicolumn{3}{l}{$\text{MLP}_a$ (\textit{Action prediction head})} \\
\midrule
Input embedding  & -          & Embedding dim$\times$1$\times$1 \\
Linear & -          & 4$\times$5$\times$17     \\
\bottomrule
\end{tabular}%
}
\end{table}


\subsection{Training Configurations}
In Table~\ref{tab:training hyper}, we provide the detailed training hyper-parameters for experiments in the main manuscripts.

\begin{table}[ht]
  \centering
  \caption{Hyper-parameters for training EgoAgent.}
  \resizebox{0.86\linewidth}{!}{%
    \begin{tabular}{lc}
    \toprule
    Training Configuration & EgoAgent-300M/1B \\
    \midrule
    Training recipe: &  \\
    optimizer & AdamW~\cite{loshchilov2017decoupled} \\
    optimizer momentum & $\beta_1=0.9, \beta_2=0.999$ \\
    \midrule
    Learning hyper-parameters: &  \\
    base learning rate & 6.0E-04 \\
    learning rate schedule & cosine \\
    base weight decay & 0.04 \\
    end weight decay & 0.4 \\
    batch size & 1920 \\
    training iters & 72,000 \\
    lr warmup iters & 1,800 \\
    warmup schedule & linear \\
    gradient clip & 1.0 \\
    data type & float16 \\
    norm epsilon & 1.0E-06 \\
    \midrule
    EMA hyper-parameters: &  \\
    momentum & 0.996 \\
    \bottomrule
    \end{tabular}%
    }
  \label{tab:training hyper}%
\end{table}%

\clearpage


\end{document}
