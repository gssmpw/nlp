% This is samplepaper.tex, a sample chapter demonstrating the
% LLNCS macro package for Springer Computer Science proceedings;
% Version 2.21 of 2022/01/12
%
\documentclass[runningheads]{llncs}
%
\usepackage[T1]{fontenc}
% T1 fonts will be used to generate the final print and online PDFs,
% so please use T1 fonts in your manuscript whenever possible.
% Other font encondings may result in incorrect characters.
%
\usepackage{graphicx}
% Used for displaying a sample figure. If possible, figure files should
% be included in EPS format.
%
% If you use the hyperref package, please uncomment the following two lines
% to display URLs in blue roman font according to Springer's eBook style:
%\usepackage{color}
%\renewcommand\UrlFont{\color{blue}\rmfamily}
%\urlstyle{rm}
%
\usepackage{todonotes}
\usepackage{amssymb}
\usepackage{amsmath}
\usepackage{xspace}
\usepackage{enumitem}



%%%%%%% suggestions and todonotes personalizadas

\newcommand{\nina}[1]{\todo[size=\tiny, color=red!10]{Nina: #1}}
\newcommand{\ivo}[1]{\todo[size=\tiny, color=green!10]{Ivo: #1}}
\newcommand{\vini}[1]{\todo[size=\tiny, color=blue!10]{Vini: #1}}

\newcommand{\suggestion}[2]{{\color{red}\sout{#1} #2}}
%\newcommand{\suggestion}[2]{#2}

%%%%%%% suggestions and todonotes personalizadas

\newcommand{\ham}{\textsc{Hamiltonian Path}\xspace}
\newcommand{\domiset}{\textsc{Dominating Set}\xspace}
\newcommand{\delcactus}{\textsc{Deletion to cactus}\xspace}
\newcommand{\delcater}{\textsc{Deletion to caterpillar}\xspace}
\newcommand{\delcons}{\textsc{Deletion to constellations}\xspace}
\newcommand{\delpath}{\textsc{Deletion to path}\xspace}
\newcommand{\subgcomp}[1]{{\sc Subgraph complementation to} {#1}}%
\newcommand{\cbdfull}{\subgcomp{$\mathcal{G}_{\Delta \leq k}$}\xspace}%
%\newcommand{\cbdfull}{{\sc Complementation to bounded degree}\xspace}%
\newcommand{\cbdk}[1]{{\sc SC to $\Delta \leq {#1}$}\xspace}%
\newcommand{\cbd}{{\sc SC to $\Delta \leq k$}\xspace}%
\newcommand{\Ccbdfull}{\subgcomp{$\mathcal{G}_{\delta \geq k}$}\xspace}%
%\newcommand{\cbdfull}{{\sc ALTERNATIVE COMPLEMENTATION TO BOUNDED DEGREE to bounded degree}\xspace}%
\newcommand{\Ccbd}{{\sc SC to $\delta \geq k$}\xspace}%


\newcommand{\newproblem}[3]{\bigskip{\centering\fbox{\parbox{0.99\textwidth}{#1\\\textit{Input}: #2\\\textit{Question}: #3}}}\bigskip}


\begin{document}
%
\title{Exploring subgraph complementation to bounded degree graphs}
%
%\titlerunning{Abbreviated paper title}
% If the paper title is too long for the running head, you can set
% an abbreviated paper title here
%
\author{Ivo Koch\inst{1} \and
Nina Pardal\inst{2}\orcidID{0000-0002-5150-6947} \and
Vinicius F. dos Santos\inst{3}\orcidID{0000-0002-4608-4559}}
%
\authorrunning{I. Koch, N. Pardal, V.F. dos Santos}
% First names are abbreviated in the running head.
% If there are more than two authors, 'et al.' is used.
%
\institute{Universidad de General Sarmiento, Argentina\\ \email{ikoch@campus.ungs.edu.ar} \and
University of Huddersfield, UK \\
\email{n.pardal@hud.ac.uk} \\
%\url{http://www.springer.com/gp/computer-science/lncs} 
\and
Universidade Federal de Minas Gerais, Brazil \\
\email{viniciussantos@dcc.ufmg.br}}
%
\maketitle              % typeset the header of the contribution
%
\begin{abstract}
Graph modification problems are computational tasks where the goal is to change an input graph $G$ using operations from a fixed set, in order to make the resulting graph satisfy a target property, which usually entails membership to a desired graph class $\mathcal{C}$. Some well-known examples of operations include vertex-deletion, edge-deletion, edge-addition and edge-contraction. In this paper we address an operation known as \emph{subgraph complement}.

Given a graph $G$ and a subset $S$ of its vertices, the subgraph complement of $G \oplus S$ is the graph resulting of complementing the edge set of the subgraph induced by $S$ in $G$.
%obtained from $G$ by taking the complement of the subgraph defined by the set $S$, and leaving the rest of the graph unchanged.
%Given a graph $G$ and a set $S$ of vertices, the subgraph complement of $G \oplus S$ is the graph with the same vertex set as $G$, where an edge $uv$ belongs to $G \oplus S$ if and only if $u, v \in S$ and $uv$ is an not edge of $G$, or if $uv$ is an edge of $G$ and at most $u$ or $v$ do not belong to $S$.
We say that a graph $H$ is a \emph{subgraph complement} of $G$ if there is an $S$ such that $H$ is isomorphic to $G \oplus S$. 
For a graph class $\mathcal{C}$, the \subgcomp{$\mathcal{C}$} is the problem of deciding, for a given graph $G$, whether $G$ has a subgraph complement in $\mathcal{C}$. 
This problem has been studied and its complexity has been settled for many classes $\mathcal{C}$ such as $\mathcal{H}$-free graphs, for various families $\mathcal{H}$, and for classes of bounded degeneracy. In this work, we focus on classes graphs of minimum/maximum degree upper/lower bounded by some value $k$. In particular, we answer an open question of Antony et al. [Information Processing Letters 188, 106530 (2025)], by showing that \subgcomp{$\mathcal{C}$} is NP-complete when $\mathcal{C}$ is the class of graphs of minimum degree at least $k$, if $k$ is part of the input.
We also show that \subgcomp{$k$-\textsc{regular}} parameterized by $k$ is fixed-parameter tractable. 
%\nina{agregue esta ultima linea}

\keywords{Subgraph complementation \and Graph modification \and Minimum degree.}
\end{abstract}
%
%
%
%\section*{Notes}

%From WG website: Contributors are invited to submit an extended abstract of at most 12 pages including title and abstract, but excluding references. Proofs omitted due to space restrictions must be placed in an appendix, to be read by program committee members at their discretion.

%From the template: The abstract should briefly summarize the contents of the paper in
%150--250 words. (it seems to be a bit longer than that right now, but it should be fine)

%It seems that the command subgcomp is putting some extra space after the graph class.


\section{Introduction}

Graph modification problems consist of taking an input graph and performing a predefined modification operation to obtain a graph with a desired target property. The goal in these problems is usually to obtain a minimum or minimal set of nodes and/or edges involved in the modification. Graph modification problems
are typically divided into four main classes: completion, deletion, editing (where edges are added, deleted, or both, respectively), and vertex-deletion. These problems have numerous practical applications across various fields. In computational biology, for example, graph modification problems are used to model relationships between biological entities, such as protein interactions or gene networks. Moreover, these problems are relevant in areas like database theory, where inconsistencies in semi-structured databases may arise from missing or corrupted data, or errors in data processing.

The problem of subgraph complementation was introduced by Kaminski et al. \cite{kaminski2009recent} during their investigations of the clique-width of graphs. 
%
Subgraph complementation is a particular type of graph modification problem, where the modification operation consists of complementing the edges within a given induced subgraph of the input. Notably, both edge deletion and edge addition can be viewed as special cases of subgraph complementation. 
%
An interesting fact about this operation is that, when applied to a graph class with bounded clique-width, the resulting graphs also have bounded clique-width. This, in turn, allows us to conclude, due to Courcelle's theorem~\cite{Courcelle00lineartime}, that any property expressed in MSO within the input class will also be solvable in linear time in the resulting graph. 
%
Crespelle et al.~\cite{crespelle2023survey} gave a comprehensive survey on the complexity of graph modification problems and results on parameterized algorithms, we a special focus on edge modification problems.
%\textcolor{red}{survey on the complexity of edge modification problems by Crespelle et al.~\cite{crespelle2023survey}.}
%

Fomin et al. \cite{fomin2020subgraph} give a number of sufficient conditions for subgraph complementation to a graph class $\mathcal{C}$ to be polynomially solvable, such as $\mathcal{C}$ being a triangle-free class or having bounded degeneracy, as long as $\mathcal{C}$ is recognizable in polynomial time. They also show positive results on graphs of bounded clique-width and on classes defined by some partition matrices and, on the negative side, that the problem is NP-complete when $\mathcal{C}$ is the class of regular graphs. In \cite{antony2022subgraph}, the authors investigate the case when $\mathcal{C}$ are the $H$-free graphs, with $H$ a complete graph, a star, a path or a cycle. 
While the first case is polynomial-time solvable, the remaining three are NP-complete, except for a few cases with small number of vertices. In a later work~\cite{antony2024cutting}, the authors show the NP-completeness of the problem when $\mathcal{C}$ is the class of $H$-free graphs, where $H$ is a tree, with the exception of 41 trees of at most 13 vertices. The authors also find a polynomial-time algorithm when $H$ is a paw. For the hard cases in~\cite{antony2022subgraph,antony2024cutting}, the authors also prove that they do no admit subexponential-time algorithms (under assumption of the \emph{Exponential Time Hypothesis}). Antony et al. present in \cite{antony2023algorithms} a polynomial-time algorithm when $\mathcal{C}$ is the class of graphs with minimum degree at least $k$, with $k$ constant, by showing that there exist a kernelization algorithm that returns a kernel linear in $k$. In the same paper, they provide a polynomial-time algorithm for the case when $\mathcal{C}$ is the class of graphs without any induced copies of the star on $k + 1$ vertices (for $k \geq 3$ constant) and the diamond.

In this paper, we refine the results concerning minimum and maximum degree of the graph $G$. More precisely, we address the problems of making the minimum (resp.\ maximum) degree at least (resp.\ at most) a certain value $k$, which can be fixed or given in the input. We settle the complexity, classic and parameterized, for all cases not previously considered in the literature, including the one stated as an open problem in~\cite{antony2023algorithms}. We also show that \subgcomp{$k$-\textsc{regular}} parameterized by $k$ is fixed-parameter tractable, which can be interpreted as the intersection of the classes of graphs of maximum and minimum degree $k$. 
%\nina{agregue esta ultima linea}

\section{Preliminaries}

All graphs are finite, undirected and simple. A graph class $\mathcal{C}$ is a set of graphs with a specified property, such as ``being bipartite''. For a graph $G$, we denote by $V(G)$ its vertex set and by $E(G)$ its edge set. We write $\overline{G}$ to denote the complement of the graph $G$. The neighborhood of a vertex $v \in V(G)$ is denoted by $N_G(v)$ and its closed neighborhood $N_G(v) \cup \{v\}$ by $N_G[v]$. Similarly, we denote the neighborhood of a set $S\subseteq V$ by $N_G(S)$, and the closed neighborhood by $N_G\left[ S \right]$. The degree of $v$ is %the size of it neighborhood and 
denoted by $d(v)$. For any graph $G$, let $\Delta(G)$ be its maximum degree and $\delta(G)$ its minimum degree. A graph is \emph{$d$-degenerate} if every subgraph $H$ of $G$ satisfies $\delta(H) \leq d$. Let $V_{* k}(G) = \{ v \in V(G) \mid d(v) * k \}$, where $* \in \{<,>, =, \neq \}$. %For a graph $G$, let $V_{\neq k}(G)=\{v \in V(G) \mid d_G(v) \neq k\}$ and $V_{=k}(G) = V(G)\setminus V_{\neq k}$.
For all these concepts, we omit the reference to the graph when it is clear from the context. 

As per usual conventions, we denote by $K_n$ the complete graph on $n$ vertices.
%
The distance between two vertices $u$ and $v$ is the length of a shortest path connecting them. 
The ball of radius $r$ centered in $v$ is the set of all vertices at a distance smaller or equal to $r$ from $v$.
%


FPT stands for \emph{fixed-parameter tractable}: these are the parameterized problems for which there exists a computable function $f$ such that the problem can be solved in $f(k)x^{|\mathcal{O}(1)|}$-time.
The class of parameterized problems that can be solved in $f(k)n^{g(k)}$-time for some computable function $f$ and $g$ is called XP.
%\vini{we should provide the basic definition of parameterized complexity. We probably just need to define XP and FPT.}
%\nina{I added the defs}


We denote by $\oplus$ the subgraph complementation operation, that is, given a graph $G$ and a set of vertices $S \subset V(G)$, $G \oplus S$ is the graph with the same vertex set as $G$, where an edge $uv$ belongs to $G \oplus S$ if and only if $u, v \in S$ and $uv$ is not an edge of $G$, or if $uv$ is an edge of $G$ and at most one of $u, v$ belongs to $S$.
For a fixed graph class $\mathcal{C}$, we define the problem \subgcomp{$\mathcal{C}$} as follows.

\newproblem{\subgcomp{$\mathcal{C}$}}{A graph $G$.}{Is there is an $S \subseteq V(G)$ such that $G\oplus S \in \mathcal{C}$?}

In some cases, the definition of a graph class naturally depends on some parameter $k$, like the $k$-colorable graphs. In such cases, it could be convenient to denote it by $\mathcal{C}_k$ and to consider the more general problem where $k$ is part of the input.

\newproblem{\subgcomp{$\mathcal{C}_k$}}{A graph $G$ and $k \in \mathbb{N}$.}{Is there is an $S \subseteq V(G)$ such that $G\oplus S \in \mathcal{C}_k$?}

Let $\pi(G)$ be a graph parameter mapping graphs to naturals, and let $k \in \mathbb{N}$. 
We will denote by $\mathcal{G}_{\pi \leq k}$ the class of graphs $G$ satisfying  $\pi(G) \leq k$. The class $\mathcal{G}_{\pi \geq k}$ is defined analogously. In this paper we are mostly focused on the case when $\pi \in \{\delta, \Delta\}$.
%The search for a subgraph complementation such that $f(G \oplus S) \leq k$ (resp. $f(G \oplus S) \geq k$) is equivalent to considering a special kind of subgraph complementation to $\mathcal{C}$, with $\mathcal{C}$ the class of graphs such that $G \in \mathcal{C}$ if and only if $f(G) \leq k$ (resp. $f(G \oplus S) \geq k$). Considering $k$ as part of the input and $\mathcal{C}_k$ a parameterized graph class, we have an analogous problem, which we define next. %Also, for any parameter, we can consider the maximization or minimization version.  %
%In what follows, we consider this problem.

For a graph $G$, it is always possible to create an isolated (universal) vertex through a subgraph complementation, by taking $S=N[v]$ ($S=V(G)\setminus N(v)$) for any $v \in V(G)$, which makes 
\subgcomp{$\mathcal{G}_{\delta \leq k}$} (\subgcomp{$\mathcal{G}_{\Delta \geq k}$}, respectively) trivial.
%
In the next section we focus on the remaining cases.

\section{Subgraph complementation to bounded degree}

In this section, we will first present a hardness result concerning the complexity of \cbdfull (\cbd), the problem of deciding, given a graph $G$ and a natural number $k$, whether there is a set $S \subseteq V(G)$ such that $\Delta(G\oplus S) \leq k$.
This will then imply hardness results for the problem \Ccbdfull (\Ccbd), which consists of, given a graph $G$ and an natural number $k$, deciding if there exists a subgraph $S$ such that $\delta(G\oplus S) \geq k$.


%We now turn our attention to \subgcomp{$\mathcal{G}_{\delta \geq k}$} 
% and \subgcomp{$\mathcal{G}_{\Delta \leq k}$}.
It is known that \subgcomp{$\mathcal{C}$} is polynomial-time solvable if $\mathcal{C}$ is a class of $d$-degenerate graphs~\cite{fomin2020subgraph}, for some fixed $d$, which include \subgcomp{$\mathcal{G}_{\Delta \leq k}$} can be solved in polynomial time for any constant $k$. 
The case \subgcomp{$\mathcal{G}_{\delta \geq k}$} has also been shown to be solvable in polynomial time~\cite{antony2023algorithms} for fixed $k$, and in the same paper the case when $k$ is part of the input is stated as an open problem. 
%
In \cite{fomin2020subgraph} it was also shown that when $\mathcal{C}$ is the class of all regular graphs, the problem is NP-hard. 
It also follows from their construction that \subgcomp{$k$-{\sc regular}} is NP-hard; notice that in this problem we are given both $G$ and $k$ as inputs, and we want to decide if $G$ can be made $k$-regular. %, in which we want to decide if a graph $G$ can be made $k$-regular (for $k$ in the input) is also NP-hard. 
%\vini{should we elaborate on this?}

%It follows from their construction that \subgcomp{$k$-{\sc regular}}, in which we want to decide if a graph $G$ can be made $k$-regular (for $k$ in the input) is also NP-hard. \vini{should we elaborate on this?}
%It is possible that the same construction (or something inspired by it) implies something for the open cases.


\begin{theorem}\label{thm:min_deg_NPc}
     \cbdfull is NP-complete.
\end{theorem}
\begin{proof}
    We show this via a reduction from {\sc Clique on regular graphs (CRG)}, which given a graph $G$ and an integer $k$ asks whether there exists a clique of size $k$ in $G$. %\vini{we need to define the problem} 
    Let $(G,k)$ be an instance of {\sc CRG}, where $G$ has $n$ vertices, and $n > k$.  
    Let $r=\Delta(G)$. We will construct an instance $(G',k')$ of \cbdk{k'} as follows.

    Let $s=n$, $t = n-k+1$, $b = n+r-2k+1$, and $a=r+1$. %({\color{red}we can change values later if needed}). 
    We start with a copy of $G$ and the following sets (see Figure \ref{fig:reduction}), in this order:
    \begin{enumerate}[label=$\bullet$]
        \item Add (a copy of) $V(K_t)$, and all possible edges between $K_t$ and $G$;
        \item Add (a copy of) $V(K_s)$, and all possible edges between $K_s$ and $K_t$;
        \item For each $v \in V(K_t)$, add a complete graph $K_a^v$ of $a$ vertices, and all possible edges between $K_a^v$ and $v$;
        \item For each $v \in V(K_s)$, add a complete graph $K_b^v$ of $b$ vertices, and all possible edges between $K_b^v$ and $v$.
    \end{enumerate}
    Finally, we set $k'=n+r-k+1$. This completes the definition of the reduction. It is easy to see that the reduction can be done in polynomial time in the size of the input.

    Suppose $(G,k)$ is a {\sc yes} instance of {\sc CRG}. We will show that $(G',k')$ is a {\sc yes} instance of \cbdk{k'}. Let $C\subseteq V(G)$ be a clique with $|C|=k$. Let $S=C\cup V(K_t)\cup V(K_s)$. An inspection of the vertex degrees in $G' \oplus S$ shows that $S$ is a solution for \cbdk{k'}.
    
    Now for the other direction, let $(G',k')$ be a {\sc yes} instance of \cbdk{k'}, and $W = V(K_t)\cup V(K_s)$.
    First, notice that for any $v \in W$, $d(v) > k'$.
    Hence, for any $S$ with $\Delta(G'\oplus S) \leq k', W \subseteq S$. 
    If $W = S$, then for $v\in K_t$, $d_{G'\oplus S}(v) = n+r+1 > k'$. 
    It follows that every $v\in K_t$ must have additional neighbors in $S$. Assume that those neighbors belong to $K^v_a$. Then the vertices of $K^v_a$ would get a large degree, at least $t - 1 + s$. Since $n > 1$, this number is greater than $k'$. The same argument can be applied to every $v \in K_s$ with respect to the vertices of $K^v_b$. We conclude that the additional vertices we need for $S$ must stem from $G$, hence we have $S\cap V(G) \neq \emptyset$. Let $C = S\cap V(G)$. We will show that $C$ is a clique.
    For $v\in K_s, d_{G'\oplus S}(v) \geq b+|C| = n+r-2k+1 + |C|$. Since $d_{G'\oplus S}(v) \leq k' = n+r-k+1$, $|C| \leq k$ must hold.
    On the other hand, for $w \in C$, $d_{G'\oplus S}(w) \geq s+r-(|C|-1) = n+r-|C|+1$, with equality only if $w$ is universal to $C$. If $|C| < k$, $d_{G'\oplus S}(w) > k'$. Hence $|C| = k$ and all vertices in $C$ are universal to $C$. $C$ is therefore a clique.
\end{proof}

\begin{figure}
\centering
\includegraphics[width=0.6\textwidth]{reduction.eps}
\caption{Reduction used in the proof of Theorem \ref{thm:min_deg_NPc}.} \label{fig:reduction}
\end{figure}


\begin{remark}\label{rem:max_deg_NPc}
    Using the hardness of \cbd, %CBD (where, given $G,k$ we look for a subgraph $S$ such that $\Delta(G\oplus S) \leq k$), 
    we can also prove that \Ccbd
    %ALTERNATIVE COMPLEMENTATION TO BOUNDED DEGREE 
    is NP-complete, by taking the instance consisting of $\overline{G}, n-k-1$. This answers the problem left open in~\cite{antony2023algorithms}.
\end{remark}

\section{Natural parameterization and approximation}

We now turn our attention to %The last 
a natural question: \textit{are these problems FPT when parameterized by the (natural) parameter $k$?} For the problem \cbd,  %case $\Delta(G\oplus S) \leq k$ 
the degeneracy argument only gives an XP algorithm. On the other hand, the argument of~\cite{antony2023algorithms} gives a linear kernel for the \Ccbd problem, %$\delta(G\oplus S) \geq k$ case, 
implying it is in FPT. Then, the remaining problem is to decide if \cbd, 
%$\Delta(G\oplus S) \leq k$, 
parameterized by $k$ is %$W[1]$-hard/
FPT.
%\vini{update the notation/problem names} \nina{si es solo aca, ya lo hice} 
We answer this affirmatively.
In order to show this result, we start by understanding the effect of a subgraph complementation on the degree of a vertex.

\begin{proposition}\label{prop:degree}
    Let $G$ be a graph, and $S \subseteq V(G)$. Then the degree of $v$ in $G \oplus S$ is at least $\max\{|S|-d_G(v)-1,d_G(v) - (|S|-1)\}$ if $v \in S$, and exactly $d_G(v)$ otherwise. 
\end{proposition}
\begin{proof} Any vertex $v$ that is not part of $S$ retains its neighborhood intact in $G \oplus S$, that is, $d_{G \oplus S}(v) = d_G(v)$. If $v \in S$, it is easy to see that $d_{G \oplus S} = |N_G(v) \cup S'| - |N_G(v) \cap S'|$, where $S' = S  \setminus \{v\}$. This sum attains its minimum when $|N_G(v) \cap S'|$ reaches its maximum, i.e. when $N_G(v) \cap S' = N_G(v)$ or $N_G(v) \cap S' = S'$. In the first case, we have $d_{G \oplus S}(v) = |N_G(v)| - |S'| = d_G(v) - (|S| - 1)$ (since $S' \subseteq N_G(v)$), and in the second case $d_{G \oplus S}(v) = |S'| - |N_G(v)| = (|S| - 1) - d_G(v)$ (since $N_G(v) \subseteq S'$). The result follows.
\end{proof}

It follows from Proposition~\ref{prop:degree} that, if $\Delta(G \oplus S) \leq k$, then all the vertices in $V_{>k}$ must lie in $S$. %those vertices with degree greater than the desired bound must lie in $S$. 
It is easy to check whether $S = V_{>k}$ gives as a result the desired subgraph complementation, i.e.\, if $\Delta(G \oplus S) \leq k$. When this is not the case, we need a new solution. We explore the structure of such an $S$ %and $G$, as 
in the following lemma.

\begin{lemma}\label{lemma:smallS}
    Let $G$ be a graph, $k \in \mathbb{N}$, and $S \subseteq V(G)$ such that $\Delta(G\oplus S) \leq k$. If there is a vertex $v \in S$ with $d_G(v) \leq k$,
     then $|S| \leq 2k+1$ and $\Delta(G) \leq 3k$.
\end{lemma}
%\begin{lemma}
    %Let $G$ be a graph, $k \in \mathbb{N}$, and suppose there is a subset $S \subseteq V(G)$ such that $\Delta(G\oplus S) \leq k$.
%    If $\Delta(G\oplus V_{>k}) > k$, then $|S| \leq 2k+1$ and $\Delta(G) \leq 3k$.
%\end{lemma}
\begin{proof}
    Let $S \subseteq V(G)$ such that $\Delta(G\oplus S) \leq k$. Recall that, in that case, $V_{>k} \subseteq S$.
    %Moreover, there must exist a vertex $v \in S \setminus V_{>k}$, otherwise $\Delta(G\oplus V_{>k}) \leq k$. 
    Let $v \in S \setminus V_{>k}$.
    From Proposition~\ref{prop:degree}, we have that $|S|-d_G(v)-1 \leq d_{G\oplus S}(v) \leq \Delta(G\oplus S) \leq k$, which implies $|S| \leq k + d_G(v) + 1$. Since $v \notin V_{>k}$, we know that $d_G(v) \leq k$, and thus it follows that $|S| \leq 2k+1$.

    Furthermore, for any vertex $w \in V(G)$, we also know from Proposition~\ref{prop:degree} that $d_{G\oplus S}(w) \geq d_G(w) - (|S|-1)$, implying $d_G(w) \leq d_{G\oplus S}(w) + |S| - 1 \leq k + (2k+1) - 1 = 3k$ and thus finishing the proof of the lemma.
\end{proof}

    Lemma~\ref{lemma:smallS} provides a straightforward $3$-approximation algorithm for the minimization version of the problem, as stated in the following:

\begin{theorem}\label{thm:3approxMaxDeg}
    There is a 3-approximation algorithm for the problem of finding $S \subset V(G)$ that  minimizes $\Delta(G\oplus S)$.
\end{theorem}
\begin{proof} Consider the following procedure: from $k=0$ to $|V(G)|-1$, we first verify if $\Delta(G\oplus V_{>k}) \leq k$. If true, then we output $k$ as the minimum achievable maximum degree after a subgraph complementation. Otherwise, if $\Delta(G) > 3k$, we conclude by Lemma \ref{lemma:smallS} that $k$ is not achievable and proceed to the next value of $k$. Finally, if $\Delta(G) \leq 3k$, we output $3k$ as the answer. Note that in this case we may assume $S=\emptyset$.
\end{proof}


\begin{theorem}\label{thm:maxdegFPT}
    There is an FPT algorithm for deciding \cbd, %$\Delta(G\oplus S) \leq k$,
    parameterized by $k$.
\end{theorem}
\begin{proof}We will resort again to  Lemma~\ref{lemma:smallS}.
    We start by checking whether $\Delta(G\oplus V_{>k}) \leq k$ or not. 
    If $V_{>k}$ is not the desired $S$ and $\Delta(G) > 3k$, the algorithm returns ``\textsc{No}'' as an answer. 
    Otherwise, we initialize the set $S$ to $S = V_{>k}$, since $V_{>k} \subseteq S$ for every possible solution. We will branch now on the current vertices of $S$. %every vertex in $V_{>k}$ must also be in a desired $S$, if a solution exists.
    %, we start setting $S = V_{>k}$ and we branch on some possibilities. 
    Since the degree of all vertices that are not in $S$ remains unchanged, any $v$ with $d_{G\oplus S}(v) > k$ must lie in  $S$. For each vertex $v \in S$ such that $d_{G\oplus S}(v) > k$, we know that a vertex $w \in N_G(v)\setminus S$ must be added to $S$ in order to decrease the degree of $v$. 
    We choose one such vertex $v$, and branch on all possible $w$. Since, by Lemma~\ref{lemma:smallS}, a solution exists if it has size $|S|\leq 2k+1$, and we start with $|S| \geq 1$, then the depth of our branching algorithm is at most $2k$. Using the fact that $v$ has at most $\Delta(G) \leq 3k$ neighbors outside $S$ and that $|S|\leq 2k$ in each branching step, we can establish that there are at most $3k$ candidates to be added, thus resulting on a bound of $\mathcal{O}((3k)^{2k})$ subproblems, and therefore yielding an FPT algorithm of time bounded by $\mathcal{O}(9^k k^{2k} |V(G)|^c)$, for some constant $c$.
\end{proof}

The final case we consider is when one asks for a subgraph complement $H$ such that $\delta(H) = \Delta(H) = k$, i.e., $H$ is $k$-regular, which can also be seen as \subgcomp{$\mathcal{G}_{\delta \geq k} \cap \mathcal{G}_{\Delta \leq k}$} . It follows from~\cite{fomin2020subgraph} that \subgcomp{\sc $k$-regular} is polynomial time solvable if $k$ is constant, since $k$-regular graphs have degeneracy $k$, and NP-complete when $k$ is part of the input as a consequence of their NP-hardness reduction to \subgcomp{\sc regular}. The polynomial-time algorithm obtained using the framework of bounded degeneracy gives a time complexity XP in $k$, which leaves the question whether the problem can be solved in FPT time. We answer this affirmatively. In order to show that, we start analyzing the structure of the graph induced by the vertices being complemented.

%\vini{From here on, I used Lemma 1 repeatedley but I believe it was a slight abuse because the assumptions are not exactly the same (although the result hold). Not sure if the best solution is rewritting its statement of writing an analogous version for $k$-regular}
%\nina{me inclino mas por la segunda opcion?}


\begin{lemma}\label{lemma:regular-component}
    Let $G$ be a graph, $k \in \mathbb{N}$, let $S \subseteq V(G)$ be such that $G\oplus S$ is $k$-regular and let $C$ be set of vertices of a connected component of $G[S]$. If $C \subseteq V_{=k}(G)$, then $G[C]$ induces a $\big(\frac{|S|-1}{2}\big)$-regular subgraph.
\end{lemma}

\begin{proof}
    Let $C$ be as in the statement and $v \in C$. 
    Then 
    $$N_{G\oplus S}(v) = (N_G(v)\setminus S) \cup (S\setminus C) \cup (C \setminus (N_G(v) \cup \{v\}))$$
    and $|N_{G\oplus S}(v)| = k$. 
    On the other hand, 
    $$N_{G}(v) = (N_G(v)\setminus S) \cup (N_G(v)\cap S)$$
    and $|N_{G}(v)| = k$. 
    Putting things together, we have
    \begin{equation*}
    \begin{split}
    |N_G(v)\cap S| & = |S\setminus C| + |C \setminus (N_G(v) \cup \{v\})| \\
    & = (|S|-|C|) + (|C|-|N_G(v)\cap C|-1) \\
    & = |S|-|N_G(v)\cap C|-1.
    \end{split}      
    \end{equation*}
    Since $|N_G(v)\cap S| = |N_G(v)\cap C|$, the result follows.
    
\end{proof}

Notice that, if there is a connected component of $G[S]$ as in the statement of Lemma~\ref{lemma:regular-component}, then that component has more than $|S|/2$ vertices, thus implying that it is unique.
Moreover, it follows similarly as in the discussion preceding Lemma~\ref{lemma:smallS} that, if there is an $S$ such that $G\oplus S$ is $k$-regular, then $V_{\neq k} \subseteq S$.
Hence, from now on we assume that $S$ is the union of two disjoint sets, $S'$ and $C$, where $V_{\neq k } \subseteq S'$, and $C$ is either empty or induces an $\big(\frac{|S|-1}{2}\big)$-regular connected component of $G[S]$. Notice that each connected component of $G[S']$ has at least one vertex in $V_{\neq k}$.

In order to find such an $S$, we proceed as follows: 
%first we guess $S'$, 
we start with $S'=V_{\neq k }$, and proceed by adding vertices adjacent to a previously added vertex. For each candidate $S'$, we look for a suitable (non-empty) $C$. We now show that if such $C$ exists, it can be found in FPT time parameterized by $k$.

\begin{lemma}\label{lemma:regular-component-fpt}
    Let $S' \subseteq V(G)$ be such that $|S'| \leq k$, and each connected component of $G[S']$ has at least one vertex in $V_{\neq k }$. Then we can find, in time FPT in $k$, a set $C$ such that $N(S') \cap C = \emptyset$
    %\vini{is the neighborhood of a set $N(S)$ defined?} \nina{lo agregue como notacion donde deifnimos neighborhood de un vertice, pero en gral no estamos agregando definiciones basicas} 
    and $G\oplus(S'\cup C)$ is $k$-regular, if it exists.
\end{lemma}
\begin{proof}
    By Lemma~\ref{lemma:smallS}, we may assume that $\Delta(G) \leq 3k$, otherwise $G$ has no $k$-regular subgraph complement. Moreover, since we look for a $C$ such that ${N(S') \cap C = \emptyset}$ and each element of $C$ will be adjacent to each element of $S'$ in  $G\oplus(S'\cup C)$, we know that $|C| \leq k$. 
    Let $v \in V(G) \setminus N(S')$ and let $B_{k-1}(v)$ be the ball of radius $k-1$ around $v$. If $v \in C$, then $C \subseteq B_{k-1}(v)\setminus N(S')$ and since $\Delta(G) \leq 3k$, $|B_{k-1}(v)|$ is bounded by a function of $k$, which allows us to try all possible subsets of $B_{k-1}(v)$. By doing this for all possible $v \in V(G) \setminus N(S')$ we can find $C$, if it exists.
\end{proof}

\begin{theorem}
    \subgcomp{\sc $k$-regular} parameterized by $k$ is in FPT.
\end{theorem}
\begin{proof}
    %Let $G$ be the input graph and let $V_{\neq k}=\{v \in V(G) \mid d_G(v) \neq k \}$.  Then, we can begin testing if $G \oplus V_{\neq k}$ is $k$-regular. If not, if a solution exists, there must be a $v \in S$ such that $d_G(v) = k$. %But then, from Proposition~\ref{}, $S$ cannot be too large, since $d_{G\oplus S}(v) \geq |S|-k-1$, implying $|S|\leq 2k+1$ and $\Delta(G) \leq 3k$, as in Lemma~\ref{}. 
    %We then may proceed as in Theorem~\ref{}: we start with $S=V_{\neq k}$ and if $G \oplus S$ is not $k$-regular we branch on some possibilities.
    %If there is a $v \in S$ with $d_{G\oplus S} > k$, we use the same argument as in Theorem~\ref{}, branching in at most $3k$ cases.
    %If there is a $v \in S$ with $d_{G\oplus S} < k$, we still have to figure out what to do :-(
    It follows from Lemma~\ref{lemma:smallS} that, if a solution $S$ transforms $G\oplus S$ to a $k$-regular graph, then $|S| \leq 2k+1$. 
    Starting with $S' = V_{\neq k}$, we test whether $G\oplus S'$ is $k$-regular, which can be done in polynomial time. If not, and if $|S'| \leq k$, by using Lemma~\ref{lemma:regular-component-fpt} we can check whether there exists a $C$ such that $S=S'\cup C$ is a solution in FPT time. Otherwise, for each $v \in N(S')$ we branch on $S'\cup \{v\}$ checking either if $S = S' \cup \{v\}$ is a solution, or whether we can extend it to a solution by using Lemma~\ref{lemma:regular-component-fpt}. Since any solution satisfies $|S| \leq 2k+1$ and $|N(S')| = O(k^2)$, both the breadth and depth of the branch tree are bounded by a function of $k$, and the fact that all operations can be done in FPT time implies that the problem is in FPT. 
    
\end{proof}


\section{Conclusions and Future work}

In this work, we explored the problem of subgraph complementation to bounded-degree %\vini{I believe it is better to leave it as plain text, since it is not a problem, but a family of problems}
graphs, addressing both its classical and parameterized complexities. We demonstrated that the problem of subgraph complementation to graphs with maximum degree at most 
$k$ %(denoted by \cbd) 
is NP-complete when $k$ is part of the input. This result also implies the same NP-completeness for the problem of subgraph complementation to graphs with minimum degree at least $k$. %(denoted by \Ccbd).
%
We also settled the parameterized complexity of these problems for the previously open cases. Specifically, we showed that the parameterized version of \cbd is FPT. 
Additionally, we presented a $3$-approximation algorithm for the minimization version of this problem. Finally, building on previous work, we completed the parameterized complexity analysis for the problem of subgraph complementation to $k$-regular graphs, demonstrating that this problem can be solved in FPT time.

There remain many open questions in the study of modification problems that use complementation as the operation. For instance, the complexity of subgraph complementation to some well-known classes such as chordal or interval graphs remains open. Current ongoing work extends the analysis to other common graph parameters, such as the matching number, dominance number, and diameter. %\vini{should we say that we are already investigating this?} 
Another research avenue would be to restrict the input to specific graph classes, to try and obtain tractable versions of the problems. Finally, a reconfiguration version of these problems could be explored, focusing on whether a sequence of $k$ subgraph complementations can transform a given graph into one that satisfies the desired property.

\section*{Acknowledgements}

\noindent Nina Pardal was partially supported by DFG grant VI 1045/1-1. Vinicius dos Santos was partially supported by FAPEMIG and CNPq (Grants 312069/2021-9 406036/2021-7, and 404479/2023-5).



% In this work, we explored the problem of \subgcomp{\textsc{bounded degree}}, settling not only the classic complexity but also the parameterized complexity of the problem. 
% We show np-complete for \subgcomp{$\mathcal{G}_{\Delta \leq k}$} when $k$ is input. 
% This gives same classical complexity for \subgcomp{$\mathcal{G}_{\delta \geq k}$}.
% %
% We settled whether these problems are FPT or not when parameterized by the natural parameter $k$ for all remaining open cases.
% We show the parameterized version of \subgcomp{$\mathcal{G}_{\Delta \leq k}$} is FPT, and we find a $3$-approximation algorithm for the minimization version of this problem.
% %
% Finally, following the results in~\cite{fomin2020subgraph}, we complete the parameterized complexity picture for the problem of \subgcomp{\textsc{$k$-regular}} showing that the problem can be solved in FPT time.

% The picture of modification problems that use complementation as the operation has room for many questions; for instance, \textsc{Subgraph complementation} to chordal or interval graphs is still open. Similarly as we did in this work, one could focus on other commonly known graph parameters, such as the matching number, dominance number, and diameter.
% On the other hand, the results that we presented are for general graphs, leaving the possibility of restricting the input graph class as another way of finding tractable versions of the problems. 
% Finally, we could consider the problem of deciding whether the reconfiguration version of these problems and ask if the desired transformation can be done in $k$ steps, that is, finding a sequence of subgraphs $H_1, \ldots, H_m$ such that sequentially applying subgraph complementation by $H_i$ for $i=1, \dots, m$ gives as a final result a graph fulfilling the desired property.




%
% ---- Bibliography ----
%
% BibTeX users should specify bibliography style 'splncs04'.
% References will then be sorted and formatted in the correct style.
%
\bibliographystyle{splncs04}
\bibliography{cas-refs}
%
\end{document}
