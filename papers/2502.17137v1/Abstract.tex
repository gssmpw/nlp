
\thispagestyle{fancy}			% Supress header 
\setlength{\parskip}{0pt plus 3.0pt}
\section*{Abstract}
The aim of this thesis is to extend the applications of the Quantile Regression Forest (QRF) algorithm to handle mixed-frequency and longitudinal data. To this end, standard statistical approaches have been exploited to build two novel algorithms: the Mixed- Frequency Quantile Regression Forest (MIDAS-QRF) and the Finite Mixture Quantile Regression Forest (FM-QRF).
\vspace{0.15in}

\noindent The MIDAS-QRF combines the flexibility of QRF with the Mixed Data Sampling (MIDAS) approach, enabling non-parametric quantile estimation with variables observed at different frequencies. FM-QRF, on the other hand, extends random effects machine learning algorithms to a QR framework, allowing for conditional quantile estimation in a longitudinal data setting.
The contributions of this dissertation lie both methodologically and empirically. 

\vspace{0.15in}

\noindent Methodologically, the MIDAS-QRF and the FM-QRF represent two novel approaches for handling mixed-frequency and longitudinal data in QR machine learning framework. Empirically, the application of the proposed models in financial risk management and climate-change impact evaluation demonstrates their validity as accurate and flexible models to be applied in complex empirical settings.
