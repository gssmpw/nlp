\begin{dedication}
\small
    \textit{Quando ti metterai in viaggio per Itaca}\newline
\textit{devi augurarti che la strada sia lunga,}\newline
\textit{fertile in avventure e in esperienze.}\newline
\textit{I Lestrigoni e i Ciclopi}\newline
\textit{o la furia di Nettuno non temere,}\newline
\textit{non sarà questo il genere di incontri}\newline
\textit{se il pensiero resta alto e un sentimento}\newline
\textit{fermo guida il tuo spirito e il tuo corpo.}\newline
\textit{In Ciclopi e Lestrigoni, no certo,}\newline
\textit{nè nell’irato Nettuno incapperai}\newline
\textit{se non li porti dentro}\newline
\textit{se l’anima non te li mette contro.}\newline

\textit{Devi augurarti che la strada sia lunga.}\newline
\textit{Che i mattini d’estate siano tanti}\newline
\textit{quando nei porti - finalmente e con che gioia -}\newline
\textit{toccherai terra tu per la prima volta:}\newline
\textit{negli empori fenici indugia e acquista}\newline
\textit{madreperle coralli ebano e ambre}\newline
\textit{tutta merce fina, anche profumi}\newline
\textit{penetranti d’ogni sorta; più profumi inebrianti che puoi,}\newline
\textit{va in molte città egizie}\newline
\textit{impara una quantità di cose dai dotti.}\newline

\textit{Sempre devi avere in mente Itaca -}\newline
\textit{raggiungerla sia il pensiero costante.}\newline
\textit{Soprattutto, non affrettare il viaggio;}\newline
\textit{fa che duri a lungo, per anni, e che da vecchio}\newline
\textit{metta piede sull’isola, tu, ricco}\newline
\textit{dei tesori accumulati per strada}\newline
\textit{senza aspettarti ricchezze da Itaca.}\newline
\textit{Itaca ti ha dato il bel viaggio,}\newline
\textit{senza di lei mai ti saresti messo}\newline
\textit{sulla strada: che cos’altro ti aspetti?}\newline

\textit{E se la trovi povera, non per questo Itaca ti avrà deluso.}\newline
\textit{Fatto ormai savio, con tutta la tua esperienza addosso}\newline
\textit{già tu avrai capito ciò che Itaca vuole significare.}\newline
\vspace{0.3cm}

\textnormal{Konstantinos Kavafis, \textit{Itaca}, 1911}
\end{dedication}
