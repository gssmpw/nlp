\chapter{The Impact of the COVID-19 Pandemic on Risk Factors for Children's Mental Health: Evidence from the UK Household Longitudinal Study}\label{ch:SDQ}
\chaptermark{The impact of COVID-19 Pandemic}

\section{Introduction}

The recent COVID-19 pandemic sensibly impacted communities, healthcare systems and society, affecting the physical and mental health of people of all ages. Among these broader effects, social isolation and increased household stress particularly affected children, and part of the extensive literature on the pandemic has focused on investigating the effects of such disruptions on children's mental health \citep{de2020covid, ma2021prevalence, adegboye2021understanding,kauhanen2023systematic}.

\vspace{0.15in}

\noindent In this context, a variety of studies have used longitudinal data concerning the "SDQ score", a widely used measure in pediatric psychological research computed with answers to the Strengths and Difficulties Questionnaire (SDQ) \citep{panagi2024mental, miall2023inequalities, merlo2022quantile, ravens2021quality, bignardi2021longitudinal, waite2021did}. This metric has been introduced in \cite{goodman1997strengths} to evaluate children's emotional, social and behavioural characteristics. High SDQ scores indicate a higher prevalence of psychological issues related to conduct or peer-related issues, hyperactivity, and emotional symptoms, including anxiety and depression.
\vspace{0.15in}

\noindent Under a methodological point of view, given the longitudinal structure of the these data, part of the above-mentioned studies have implemented standard linear random-effects or mixed-effects models, which allow to estimate the expected value of the conditional distribution of the response variable in a linear modeling framework. However, in many empirical applications, the relationship between the outcome and the covariates is non-linear, and the outcome's distribution often violates the gaussianity assumptions of random-effects models. This is the case of the SDQ distribution, which is typically asymmetric. Moreover, results of previous contributions point out that the effect of risk factors such as socio-economic disadvantage and maternal depression changes across the SDQ conditional distribution, with a more significant impact on the right tail, which is related to abnormal levels of behavioural issues \citep{davis2010socioeconomic, flouri2008psychopathology, flouri2010modeling, merlo2022quantile, tzavidis2016longitudinal}.

\vspace{0.15in}

\noindent These findings highlight that a more complete picture of the SDQ conditional distribution is necessary to gain a deeper understanding of the phenomenon of interest. In this context, contributions such as \cite{tzavidis2016longitudinal} and \cite{merlo2022quantile} implemented QR mixed-effects models to obtain more robust and accurate results. These approaches allow to infer the entire conditional distribution of the outcome by modeling location parameters beyond the conditional expected value, such as quantiles, in a longitudinal data setting.

\vspace{0.15in}

\noindent To overcome the drawbacks of parametric formulations of these models, QR has been also extended to the non-parametric setting by developing QR machine learning algorithms, such as QR Neural Networks \citep{white1992nonparametric}, QR Support Vector Machines \citep{hwang2005simple, xu2015weighted}, QR kernel based algorithms \citep{christmann2008consistency}, QR Forests (QRF) \citep{meinshausen2006quantile} and Generalized QR Forests 
\citep{ athey2019generalized}. However, these models cannot handle longitudinal data in a mixed-effects framework, and for this reason the analysis carried out in this chapter employs the FM-QRF algorithm of Chapter \ref{ch:FM-QRF}. 

\vspace{0.15in}

\noindent The FM-QRF extends the traditional QR model to a non-linear setting and it also enhances the QRF algorithm by introducing an additional parameter represented by the random effects part of mixed-effects model, which allows to model unobserved heterogeneity across statistical units. These unique features make the FM-QRF well-suited for modeling phenomena characterized by complex non-linear relationships among the variables of interest, especially in empirical studies involving repeated measurements and longitudinal data, such as SDQ score data.

\vspace{0.15in}

\noindent Thanks to this approach, this study contributes to the strand of literature on children's mental health by uncovering relationships between children's mental health and risk factors during the pandemic that have not have been captured by the standard linear models applied in previous contributions \citep{walton2010contextual,goodnight2012quasi,bradley2002socioeconomic}.

\vspace{0.15in}

\noindent To this end, the data used in this study concern the SDQ scores of children that participated in at the UK Household Longitudinal Study (UKHLS), the largest household panel survey carried out in the United Kingdom since 2009. Data related to this study have been widely used in the literature concerning the effects of the pandemic on mental health \citep{miall2023inequalities,bayrakdar2023inequalities,metherell2022digital,daly2022psychological,thorn2022education,reimers2022primary,mendolia2022have}, and the aim of this study is to use the SDQ score as outcome variable and a set of variables selected according to previous studies findings as set of covariates.

\vspace{0.15in}

\noindent To compare the set of risk factors before and after the pandemic, the FM-QRF is trained with two datasets: the first contains observations from the pre-pandemic period (2016-2019), and the other from the pandemic period (2020-2021). Then, quantiles at five different probability levels are estimated for each period and the Variable Importance measure is extracted to select the most significant risk factors for children's mental health in the two periods of interest.

\vspace{0.15in}

\noindent The results from the pre-pandemic and pandemic periods are compared to identify patterns and shifts in the most important risk factors. Additionally, these findings are compared with the set of statistically significant variables obtained using the LQMM model.

\vspace{0.15in}

\noindent The empirical findings obtained with the FM-QRF indicate that the key risk factors vary based on the quantile level, justifying the adoption of a QR approach over a standard regression method, and that they differ between the pre-pandemic and pandemic periods. Results also show that relying solely on the LQMM model provides only a partial understanding of the phenomena under investigation, highlighting the relevance of a non-linear approach. In this sense, the FM-QRF proves to be a valuable choice to gain a deeper understanding of the complex relationship between the COVID-19 pandemic and children's mental well-being.

\vspace{0.15in}

\noindent The rest of the chapter is organized as follows. Section \ref{sec:SDQ-data} presents the data used in this empirical study, Section \ref{sec:SDQ-empirical} reports the results obtained with the FM-QRF and the comparison with the LQMM model and Section  \ref{sec:SDQ-conclusions} concludes.

\section{The Data}
\label{sec:SDQ-data}

The data used in this study concern the SDQ score as outcome variable and a set of  covariates selected according to previous results findings.

\vspace{0.15in}

\noindent In particular, the SDQ score data have been retrieved by the UKHLS, the largest household panel survey carried out in the United Kingdom since 2009 involving 40,000 households. The aim of the UKHLS is to capture a broad range of information concerning economic circumstances, employment, education, health, and mental well-being before and after the pandemic. The longitudinal structure of this dataset provides a unique set of information to understand the long-term impact of a variety of factors on households.

\vspace{0.15in}

\noindent The UKHLS collects also data of individual responses from both adults, youngsters and children. In the latter case, data concerning the assessment of children mental health are collected through the SDQ. The focus of the analysis presented in this chapter is the "total difficulties score" (\textit{chsdqtd}) which ranges from 0 to 40 and represents the sum of 25 scores related to five domains comprising emotional symptoms, peer problems, conduct problems, hyperactivity and prosocial behaviour. The response for each item is evaluated on a 3-points scale, where 0 represents a "not true" answer, 1 is given if the respondent partially agrees with the question statement, and 2 for a "true" answer. High scores indicate a higher prevalence of psychological issues related to conduct or peer-related issues, hyperactivity, inattention, and emotional symptoms, including anxiety and depression.


\vspace{0.15in}

\noindent The covariates have been selected based on previous studies on children's mental health risk factors, such as family poverty, ethnicity, and living area. In particular, the set of covariates is composed of variables whose effects might have changed between before and during the pandemic: social benefit income (\textit{fihhmnsben}), income from investements (\textit{fihhmninv}),  household size (\textit{hhsize}) and number of bedrooms in the house (\textit{hsbeds}), number of employed people in the household (\textit{nemp}) and employed people in the household that are not being paid (\textit{nue}), living area (urban or rural) (\textit{urban}), being up to date with bills payments (\textit{xphsdba}) and internet access (\textit{pcnet}).
The belonging to an ethnic minority (\textit{emboost}) is considered as an additional time-invariant variable.


\vspace{0.15in}

\noindent The aim of this chapter is to compare how the main risk factors for children's mental health changed before and during the pandemic period.
Thus, the dataset comprises observations from 2016 to 2019 for the pre-pandemic period and from 2020 to 2021 for the pandemic period.
As stated above, the outcome variable is the SDQ total score, and after eliminating statistical units with missing data, the final sample including both the pre-pandemic and pandemic periods is composed of 2401 children for a total of $n=3101$ observations. A total of 1729 and 1028 children are included in the pre-pandemic and pandemic sample, respectively. The barplot showing the number of children interviewed 1, 2, 3 or 4
times during sample period is represented in Figure \ref{fig:n_children}.

\vspace{0.15in}

\noindent The summary statistics reported in Tables \ref{tab:summary-PRECOVID} and \ref{tab:summary-COVID}, related to the pre-COVID and COVID periods respectively, highlight the asymmetry of the SDQ unconditional distribution, whose mean and median sensibly differ especially in the pre-COVID period.

% Please add the following required packages to your document preamble:
% \usepackage{booktabs}
% \usepackage{graphicx}
% \usepackage[table,xcdraw]{xcolor}
% Beamer presentation requires \usepackage{colortbl} instead of \usepackage[table,xcdraw]{xcolor}
\begin{table}[]
\centering
\resizebox{\textwidth}{!}{%
\begin{tabular}{@{}
>{\columncolor[HTML]{FFFFFF}}l 
>{\columncolor[HTML]{FFFFFF}}l 
>{\columncolor[HTML]{FFFFFF}}l 
>{\columncolor[HTML]{FFFFFF}}l 
>{\columncolor[HTML]{FFFFFF}}l 
>{\columncolor[HTML]{FFFFFF}}l 
>{\columncolor[HTML]{FFFFFF}}l 
>{\columncolor[HTML]{FFFFFF}}l @{}}
\toprule
\textit{}               & \textbf{Obs} & \textbf{Min} & \textbf{Max} & \textbf{Median} & \textbf{Mean} & \textbf{St.Dev} & \textbf{Null} \\ \midrule
\textit{chsdqtd}    & 2006         &              & 37.00        & 8.00            & 8.51          & 5.94            & 8856          \\
\textit{urban}      & 2006         &              & 2.00         & 1.00            & 1.23          & 0.42            & 14            \\
\textit{hhsize}     & 2006         &              & 13.00        & 4.00            & 4.32          & 1.18            & 8833          \\
\textit{hsbeds}     & 2006         &              & 8.00         & 3.00            & 3.29          & 0.87            & 8835          \\
\textit{fihhmnsben} & 2006         &              & 6096.14      & 235.67          & 605.84        & 758.22          & 0             \\
\textit{fihhmninv}  & 2006         &              & 19416.67     & 0.00            & 173.01        & 845.31          & 8             \\
\textit{nemp}       & 2006         &              & 7.00         & 2.00            & 1.63          & 0.71            & 0             \\
\textit{pcnet}      & 2006         &              & 2.00         & 1.00            & 1.01          & 0.10            & 6             \\
\textit{nue}        & 2006         &              & 6.00         & 0.00            & 0.45          & 0.72            & 0             \\
\textit{xphsdba}    & 2006         &              & 3.00         & 1.00            & 1.08          & 0.28            & 22            \\
\textit{emboost}    & 2006         &              & 1.00         & 0.00            & 0.10          & 0.30            & 0             \\  \bottomrule
\end{tabular}%
}
\caption{Summary statistics related to the pre-COVID period of the variables included in the sample after the data cleaning procedure. The Null column reports the number of null values (not applicable or missing items) before the data cleaning procedure.}
\label{tab:summary-PRECOVID}
\end{table}
% Please add the following required packages to your document preamble:
% \usepackage{booktabs}
% \usepackage{graphicx}
% \usepackage[table,xcdraw]{xcolor}
% Beamer presentation requires \usepackage{colortbl} instead of \usepackage[table,xcdraw]{xcolor}
\begin{table}[]
\centering
\resizebox{\textwidth}{!}{%
\begin{tabular}{@{}
>{\columncolor[HTML]{FFFFFF}}l 
>{\columncolor[HTML]{FFFFFF}}l 
>{\columncolor[HTML]{FFFFFF}}l 
>{\columncolor[HTML]{FFFFFF}}l 
>{\columncolor[HTML]{FFFFFF}}l 
>{\columncolor[HTML]{FFFFFF}}l 
>{\columncolor[HTML]{FFFFFF}}l 
>{\columncolor[HTML]{FFFFFF}}l @{}}
\toprule
\textit{}               & \textbf{Obs} & \textbf{Min} & \textbf{Max} & \textbf{Median} & \textbf{Mean} & \textbf{St.Dev} & \textbf{Null} \\ \midrule
\textit{chsdqtd}    & 1095         & 0            & 32.00        & 7               & 8.68          & 6.01            & 9155          \\
\textit{urban}      & 1095         & 1            & 2.00         & 1               & 1.23          & 0.42            & 74            \\
\textit{hhsize}     & 1095         & 2            & 12.00        & 4               & 4.36          & 1.21            & 9137          \\
\textit{hsbeds}     & 1095         & 0            & 8.00         & 3               & 3.33          & 0.92            & 9141          \\
\textit{fihhmnsben} & 1095         & 0            & 4901.66      & 208             & 568.05        & 760.12          & 0             \\
\textit{fihhmninv}  & 1095         & 0            & 9683.33      & 0               & 211.61        & 819.50          & 7             \\
\textit{nemp}       & 1095         & 0            & 7.00         & 2               & 1.62          & 0.72            & 0             \\
\textit{pcnet}      & 1095         & 1            & 2.00         & 1               & 1.01          & 0.07            & 12            \\
\textit{nue}        & 1095         & 0            & 6.00         & 0               & 0.50          & 0.83            & 0             \\
\textit{xphsdba}    & 1095         & 1            & 3.00         & 1               & 1.11          & 0.33            & 24            \\
\textit{emboost}    & 1095         & 0            & 1.00         & 0               & 0.10          & 0.30            & 0             \\ 
\bottomrule
\end{tabular}%
}
\caption{Summary statistics related to the COVID period of the variables included in the sample after the data cleaning procedure. The Null column reports the number of null values (not applicable or missing items) before the data cleaning procedure.}
\label{tab:summary-COVID}
\end{table}

\vspace{0.15in}

\noindent The features of the SDQ conditional distribution given the above-mentioned set of covariates are also investigated by fitting a model for the SDQ score with random-effects at child level. The Q-Q plot shown in Figure \ref{fig:qqplot} highlights a severe departure of the model's residuals from the Gaussian distribution, which is one of the key assumption of the linear random-effects model. Thus, it would be useful to estimate more robust measures of central tendency and conditional quantiles to better estimate the relation between the SDQ score and the covariates.

\begin{figure}[H]
    \centering
\includegraphics[width=0.7\textwidth]{SDQ/images/qqplot1.png}
    \caption{Normal probability plot of the linear mixed model residuals for SDQ total score with  with random-effects specified at child level}
    \label{fig:qqplot}
\end{figure}

\vspace{0.15in}

\noindent The validity of a non-linear QR approach over a standard linear one is assessed with an analysis of variance (ANOVA) test \citep{st1989analysis} to compare a spline QR model and standard QR at five quantile levels $\tau=0.1, 0.25, 0.5, 0.75, 0.9$. 
The spline QR model is an additive model represented by piecewise-defined polynomial functions. This approach allows to model non-linear relationships by means of knot points, at which each different polynomial segment originates. In particular, in this chapter the the QR spline model is represented by
a piecewise cubic polynomial with 3 knots for each covariate.

\vspace{0.15in}

\noindent The results obtained from the ANOVA test in Table \ref{tab:anova} reveal that, with the dataset of interest, a non-linear approach results more valid than the linear one. This finding supports the adoption of a non-linear QR approach, such as the FM-QRF, as a preferred choice over standard linear QR model, especially in scenarios where the underlying data exhibits non-linear relationships.



\begin{table}[h]
    \centering
    \caption{ANOVA test results for different values of $\tau$. The 'Tn' column reports the test statistic and the 'P-Value' column reports the level of significance of the test at 5\% significance level. }
    \label{tab:anova}
    \begin{tabularx}{\textwidth}{*{6}{>{\centering\arraybackslash}X}}
        \hline
        \textbf{$\tau$} & \textbf{0.1} & \textbf{0.25} & \textbf{0.5} & \textbf{0.75} & \textbf{0.9} \\ \hline
        \textit{Tn}  & 1.513                                              & 3.585                                              & 10.686                                             & 27.329                                             & 2.2                                                \\
\textit{P-Value}                                              & 0.001**                                            & 0***                                               & 0***                                               & 0***                                               & 0***                                               \\ \hline
    \end{tabularx}
\end{table}

% \section{Finite mixtures of Quantile Regression Forests}

% In this section we describe the FM-QRF, implemented to study the risk factors driving children's mental health before and during the pandemic.

% We denote with $Y_{it}, i=1,\dots,N$, $t=1,\dots,T_i$ the response variable observed for the $i$-th statistical unit at time $t$ with realisation $y_{it}$. We also denote with $\mathbf{x}_{it} \in \mathbb{R}^p$ the vector of observed explanatory variables with components $x_{it,j}, j=1, \dots, p$ and $x_{it,1} \equiv 1$. In the longitudinal and clustered data setting, the heterogeneity between individuals is modeled by  $\mathbf{b}_{\tau}=\{b_{1, \tau},\dots,b_{N, \tau}\}$, which is the vector of unit- and quantile-specific random coefficients at given quantile level $\tau \in (0, 1)$. 

% The standard LQMM estimates the location parameter $\mu_{it,\tau}$ as:
% \begin{equation}
%     	\label{eq:me-lqmm}   \mu_{it,\tau}=\mathbf{x}_{it}^{\prime}\bbeta_\tau+b_{i, \tau}  \;\;\;  \tau \in (0,1).
% 	\end{equation}
% where $b_{i, \tau}$ is the time-constant random parameter that varies across statistical units according to a distribution $f_b(\cdot)$ with support $\mathcal{B}$, where $E[b_{i, \tau}] = 0$ is used for parameter identifiability.

% In a Maximum Likelihood framework, the response $Y_{it}$ is assumed to be distributed as an Asymmetric Laplace distribution \citep[ALD - ][]{SDQ/images/yu2001bayesian}:
% \begin{equation}\label{eq:ald}
% f_{y|b}(y_{it}|b_{i, \tau}; \tau) = \frac{\tau(1- \tau)}{\sigma_\tau} \exp \Bigg\{ - \rho_\tau \left( \frac{y_{it} - \mu_{it,\tau}}{\sigma_\tau}\right) \Bigg\},
% \end{equation}
% where $\tau$, $\sigma_\tau > 0$ and $\mu_{it, \tau}$ represent, respectively, the skewness, the scale, and the location parameter of the distribution. In this setting, the term $\mu_{it, \tau}$ is also the quantile of the conditional distribution and the function $\rho_\tau (u) = u (\tau - \boldsymbol{1}_{\{u < 0\}})$ is the quantile loss function of \cite{koenker1978regression}.

% In \cite{alfo2017finite}, the LQMM model has been extended to a Finite Mixtures framework 
% by leaving the distribution $f_b(\cdot)$ unspecified, and the parameters $\mathbf{b}_i$ are estimated directly from the observed data via the Non-Parametric Maximum Likelihood approach \cite[NPML - ][]{SDQ/images/laird1978nonparametric}.
% In this setting, the distribution $f_b(\cdot)$ is approximated with a a discrete distribution on $K < N$ locations $\alpha_{k,\tau}$ so that:
% \begin{equation}
% \label{eq:bk_bi}
%     {\alpha}_{k,\tau} \sim \sum_{k=1}^K \pi_{k,\tau}\delta_{{\alpha}_{k,\tau}},
% \end{equation}
% where the probability $\pi_{k,\tau}$ is defined as $\pi_{k,\tau} = \mathbb{P}({b}_{i, \tau} = {\alpha}_{k ,\tau})$ with $i=1,\dots, N$ and $k=1,\dots,K$ and $\delta_{{\alpha}_{k,\tau}}$ is a one-point distribution putting a unit mass at ${\alpha}_{k,\tau}$. 

% The FM-QRF extends this approach to a machine learning framework using a non-parametric unknown function $g_{\tau}(\mathbf{x}_{it})$ instead of the traditional linear one $\mathbf{x}_{it}^{\prime}\bbeta_\tau$. 
% With this approach, the location parameter of the ALD in equation \eqref{eq:me-lqmm} becomes:

% \begin{equation}
%     	\label{eq:fmqrf}
%      \mu_{itk,\tau}=g_{\tau}(\mathbf{x}_{it})+\alpha_{k, \tau}  \;\;\;  \tau \in (0,1),
% 	\end{equation}
% with $b_{i, \tau}={\alpha}_{k,\tau}$ and $g_\tau: \mathbb{R}^p \rightarrow \mathbb{R}$.
% %is a non-parametric unknown function which, %in the FM-QRF, is estimated as in QRF \citep{meinshausen2006quantile}, 

% Differently from models proposed in previous contributions \citep[see][]{SDQ/images/geraci2007quantile, geraci2014linear,farcomeni2012quantile}, 
% the NPML approach makes the FM-QRF a valid data-driven approach resistant to misspecificationof the random effects distribution \citep{alfo2017finite, farcomeni2012quantile}.
% Moreover, the FM-QRF represents an extension of both the QRF \citep{meinshausen2006quantile}, since it adds a random effect part to the non-parametric unknown function of QRF, and the LQMM, because the linear fixed-part is substituted by the function $g_{\tau}(\mathbf{x}_{it})$.

% The simultaneous estimation of $g_{\tau}(\mathbf{x}_{it})$ and $b_{k, \tau}$ in a quantile regression setting is performed with an EM algorithm using the ALD as working likelihood.

\section{Analysis of the UK Household Longitudinal Study Data} \label{sec:SDQ-empirical}

This section reports the results of the analysis of the SDQ total score dataset for the selected UKHLS sample of children. Given previous literature results, the analysis concerns the risk factors for children’s emotional and behavioural problems related to family poverty, ethnicity, overcrowding in the household and internet access. 
\vspace{0.15in}

\noindent In particular, the analysis considers covariates related to such risk factors whose effects might have changed between before and during the pandemic: social benefit income, household size and number of bedrooms in the house, number of employed people in the household and employed people in the household that are not being paid, living area (urban or rural), being up to date with bills payments and internet access.
The belonging to an ethnic minority is included as time-invariant variable.

\vspace{0.15in}

\noindent The FM-QRF described in Chapter \ref{ch:FM-QRF} is used to model quantiles at five different levels $\tau=0.1, 0.25, 0.5, 0.75, 0.9$:

\begin{equation}
\hat{Q}_{it,\tau}=\widehat{g_{\tau}(\mathbf{x}_{it})}+\hat{\alpha}_{k, \tau}  \;\;\;  \tau \in (0,1),
	\end{equation}
where $g_\tau: \mathbb{R}^p \rightarrow \mathbb{R}$. In this case, $\hat{Q}_{it,\tau}$ is the estimated quantile for the $i-th$ individual at level $\tau$ and time $t$, $g_{\tau}(\mathbf{x}_{it})$ is the fixed-effects part of the model estimated with a QRF approach and $\alpha_{k, \tau}$ is the random effects part estimated with the finite mixture approach described in Section \ref{sec:FM-QRF-methodology}.

\noindent In this empirical application, the number of mixture components $K$
has been set to 10 via grid search.

\noindent For the sake of clarity, this chapter reports only results for $\tau=0.1, 0.5, 0.9$. Results for the remaining quantile levels, which are similar to the ones reported in this section, are available upon request.

\subsection{Risk Factors Analysis}

The risk factor analysis is run by extracting the Variable Importance measure from the FM-QRF in order to evaluate which covariate has a more relevant role in predicting the SDQ total score quantiles. Figures \ref{fig:quantile01}-\ref{fig:quantile09} report the bar graphs in which each bar represents the variable importance of each covariate at quantile levels $\tau=0.1, 0.5, 0.9$ before and during the COVID-19 pandemic. The variables have been ordered in each graph according to the average variable importance in both periods. This allows to understand both the ranking of the variables in each separate period and the overall ranking at quantile level. The variable importance values for the quantile levels $ \tau=0.25, 0.75$ are shown in Figures \ref{fig:quantile025}-\ref{fig:quantile075}.

\begin{figure}[H]
    \centering
\includegraphics[width=0.8\textwidth]{SDQ/images/quantile01.pdf}
    \caption{Bar plot showing the Variable Importance extracted from the FM-QRF for each covariate at quantile level $\tau=0.1$. The blue bars represent the Variable Importance in the Pre-pandemic period and the red bars are related to the pandemic period. Numbers at the top of the bars indicate the ranking position of each variable in terms of Variable Importance.}
    \label{fig:quantile01}
\end{figure}

\begin{figure}[H]
    \centering
    \includegraphics[width=0.8\textwidth]{SDQ/images/quantile05.pdf}
 \caption{Bar plot showing the Variable Importance extracted from the FM-QRF for each covariate at quantile level $\tau=0.5$. The blue bars represent the Variable Importance in the Pre-pandemic period and the red bars are related to the pandemic period. Numbers at the top of the bars indicate the ranking position of each variable in terms of Variable Importance.}
    \label{fig:quantile05}
\end{figure}

\begin{figure}[H]
    \centering
    \includegraphics[width=0.8\textwidth]{SDQ/images/quantile09.pdf}
     \caption{Bar plot showing the Variable Importance extracted from the FM-QRF for each covariate at quantile level $\tau=0.9$. The blue bars represent the Variable Importance in the pre-pandemic period and the red bars are related to the pandemic period. Numbers at the top of the bars indicate the ranking position of each variable in terms of Variable Importance.}
    \label{fig:quantile09}
\end{figure}


\vspace{0.15in}

\noindent At level $\tau=0.01$, the three most important variables in the pre-pandemic period are social benefit income, number of employed people not paid in the household, both proxies for family poverty, and household size.
During the COVID pandemic, social benefit income remains the most important variable, whereas the second most important one shifts from being the number of employed people not paid in the household to internet access, which is among the less important variables before the pandemic. This result highlights how during the pandemic the importance of internet access significantly increased for children with low levels of behavioural issues.

\vspace{0.15in}

\noindent At the median level $\tau=0.5$, before the pandemic the most important variable is social benefit income, the second important variable is household size and the third one is number of employed people not paid in the household. After the pandemic, the variable importance order changes only at the second place, which is covered by the number of employed people in the household.

\vspace{0.15in}

\noindent At the higher quantile level $\tau=0.9$, the first three most important variables before the pandemic are social benefit income, number of employed people in the household and household size. During the pandemic, the third most important variable is the number of employed people not being paid. Even if the order is quite similar before and during the pandemic, the main difference between these two periods is that the relevance of the three most important variables is sensibly higher during the pandemic. Being these three variables proxies for the level of family poverty, this result indicates that this factor became more important after the pandemic for children with a high grade of behavioural issues.

\vspace{0.15in}

\noindent In conclusion, in both periods, the most important variable for children with both high and low levels of behavioural issues is social benefit income. The result of this analysis highlights that during the pandemic the importance of internet access significantly increased during the pandemic in maintaining low levels of behavioural issues, whereas the relevance of family poverty increased for children with high levels of behavioural issues.
Moreover, it is also worth noticing that the variable importance changes across quantile levels, corroborating the use of the QR approach proposed in this chapter. 


\subsection{Comparison with the LQMM Results}

This section reports the results obtained by fitting the LQMM model of Equation \eqref{eq:me-lqmm} with random intercept at the five quantile levels $\tau=0.1, 0.25, 0.5, 0.75, 0.9$. The aim of this analysis is to investigate whether the non-parametric approach of the FM-QRF might represent an additional valuable approach to investigate the risk factors determining children behavioural issues. 

\vspace{0.15in}

\noindent To this end, the sets of significant variables obtained with the LQMM models fitted for each quantile level are compared with the set of most important variables obtained from the FM-QRF using the Variable Importance measure. 
\vspace{0.15in}

\noindent The LQMM results coefficients for the pre-pandemic and pandemic period are reported in Table \ref{tab:lqmm_summary}.

\vspace{0.15in}

\noindent At quantile level $\tau=0.1$, in the pre-pandemic period the set of significant variables comprises variables proxies of the overcrowding in the household (number of beds and household size), the number of employed people in the household and ethnic minority. During the pandemic, the set of significant variables gets larger and includes also social benefit income and investment income. Similarly to results shown in previous contributions, the significant variables with the higher coefficient during the pandemic period are the employment variable and the ethnic minority dummy variable.

\vspace{0.15in}

\noindent At quantile level $\tau=0.5$, the set of significant variables remains almost the same in the pre-pandemic and pandemic periods. In the pre-pandemic period the significant variables are household size, number of beds, social benefit income, employed people in the household and ethnic minority. The main difference with the pandemic period is that during the COVID-19 pandemic the set of significant variables includes the investment income and does not include social benefit income and number of beds. 

\vspace{0.15in}

\noindent At quantile level $\tau=0.9$, in the pre-pandemic period the significant variables are social benefit income, number of employed people and ethnic minority. During the pandemic, the set of significant variables includes also household size and number of employed people not paid.

\vspace{0.15in}

\noindent Additionally, the pseudo-$R^2$ measure is computed for both models to evaluate their goodness of fit. This measure has been proposed in \cite{koenker1999goodness} and implemented in the quantile regression literature (see, for instance, \cite{bianchi2018estimation, borgoni2024semiparametric}) as a valid alternative to the standard $R^2$. In particular, the pseudo-$R^2$ is computed as:

\begin{equation}
R^2_{\rho}(\tau)=1-\frac{\sum_{i=1}^{n}\rho_{\tau}(e_{it})}{\sum_{i=1}^{n}\rho_{\tau}(\Tilde{e}_{it})}
\end{equation}

\noindent where $\rho_{\tau}(\cdot)$ is the quantile loss function of \cite{koenker1978regression}, $e_{it}$ are the standardized residuals of the full model trained with the whole set of covariates and
$\Tilde{e}_{it}$ are the standardized residuals under the null model, which considers only the intercept (the coefficients of the covariates are set to 0). The results concerning the goodness of fit of the LQMM and FM-QRF model in the pre-COVID and COVID periods are reported separately in Tables \ref{tab:r2-precovid} and \ref{tab:r2-covid}.

\begin{table}[]
\centering
\resizebox{\columnwidth}{!}{%
\begin{tabularx}{\textwidth}{*{6}{>{\centering\arraybackslash}X}}
\toprule
\multicolumn{6}{c}{\textbf{Pre-Pandemic}}                                  \\ \midrule
$\tau$       & \textbf{0.10} & \textbf{0.25} & \textbf{0.50} & \textbf{0.75} & \textbf{0.90} \\
\textbf{LQMM}   & 2.85        & 3.58        & 4.70        & 6.64        & 8.41        \\
\textbf{FM-QRF} & 6.56        & 21.96       & 13.29       & 41.88       & 66.47       \\ \bottomrule
\end{tabularx}%
}
\caption{Pseudo-$R^2$ values related to the pre-pandemic period for the LQMM and the FM-QRF models. Values are expressed in percentages.}
\label{tab:r2-precovid}
\end{table}

% Please add the following required packages to your document preamble:
% \usepackage{booktabs}
% \usepackage{graphicx}
% \usepackage[table,xcdraw]{xcolor}
% Beamer presentation requires \usepackage{colortbl} instead of \usepackage[table,xcdraw]{xcolor}
\begin{table}[]
\centering
\resizebox{\columnwidth}{!}{%
\begin{tabularx}{\textwidth}{*{6}{>{\centering\arraybackslash}X}}
\toprule
\multicolumn{6}{c}{\cellcolor[HTML]{FFFFFF}\textbf{Pandemic}}                                  \\ \midrule
$\tau $  & \textbf{0.10} & \textbf{0.25} & \textbf{0.50} & \textbf{0.75} & \textbf{0.90} \\
\textbf{LQMM}   & 0.46        & 1.96        & 2.08        & 4.74        & 10.58       \\
\textbf{FM-QRF} & 0.86        & 16.64       & 8.57        & 71.82       & 35.07       \\ \bottomrule
\end{tabularx}%
}
\caption{Pseudo-$R^2$ values related to the pandemic period for the LQMM and the FM-QRF models. Values are expressed in percentages.}
\label{tab:r2-covid}
\end{table}

\noindent The values of the pseudo-$R^2$ highlight that the FM-QRF has a greater goodness of fit with respect to the LQMM at all quantile levels both in the pre-COVID and COVID periods. Moreover, as already noted in \cite{} for other kind of models, the pseudo-$R^2$ increases with the quantile level of interest for both the FM-QRF and the LQMM.


\vspace{0.15in}

\noindent From these results a variety of conclusions can be drawn. 
First, similarly to the FM-QRF analysis results, the significant variables set and the related coefficient values change across quantile levels. This result further justify the relevance of a QR approach. However, the ethnic minority variable and number of employed people in the household are significant at all quantile levels in the pandemic period and their coefficient values are similar across quantiles.

\vspace{0.15in}

\noindent Second, the set of significant variables obtained with the LQMM approach does not always coincide with the set of most important variables obtained with the FM-QRF.

\vspace{0.15in}

\noindent For instance, at quantile level $\tau=0.1$, the internet access variable is the second most important variable in the FM-QRF analysis, whereas in the LQMM analysis it is not significant neither in the pre-pandemic period nor during the pandemic. 

\vspace{0.15in}

\noindent These results highlight how a non-parametric approach might be useful to uncover meaningful non-linear relationships among variables that are being overlooked with a linear approach.

% Please add the following required packages to your document preamble:
% \usepackage{booktabs}
% \usepackage{multirow}
% \usepackage{graphicx}
\begin{table}[H]
\centering
\caption{LQMM results coefficients for the pre-pandemic and the pandemic period at five quantile levels $\tau=0.1, 0.5, 0.9$. The symbol '***' denotes significance at 1\% level and '**'  significance at 5\%.}
\label{tab:lqmm_summary}
\resizebox{\textwidth}{!}{%
\begin{tabular}{@{}ccccclcccccl@{}}
\toprule
 & \multicolumn{5}{c}{\textbf{Pre-Pandemic}} &  & \multicolumn{5}{c}{\textbf{Pandemic}} \\ \midrule
 & \textit{Variable} & \textit{Estimate} & \textit{Std. Error} & \multicolumn{2}{c}{\textit{P-Value}} &  & \textit{Variable} & \textit{Estimate} & \textit{Std. Error} & \multicolumn{2}{c}{\textit{P-Value}} \\ \cmidrule(lr){2-6} \cmidrule(l){8-12} 
\multirow{11}{*}{\textbf{0.1}} & \cellcolor[HTML]{D9D9D9} Intercept & 11.153 & (1.681) & 0.000& *** &  & Intercept & 5.447 & (4.517) & 0.234 &  \\
 & area & -0.036 & (0.173) & 0.838 &  &  & area & 0.055 & (0.136) & 0.690 &  \\
 & \cellcolor[HTML]{D9D9D9} hhsize & -0.887 & (0.155) & 0.000& *** &  & \cellcolor[HTML]{D9D9D9} hhsize & -0.966 & (0.25) & 0.000& *** \\
 & \cellcolor[HTML]{D9D9D9} n\_beds & -0.759 & (0.145) & 0.000& *** &  & \cellcolor[HTML]{D9D9D9} n\_beds & -0.437 & (0.209) & 0.042 & ** \\
 & soc\_ben\_inc & -0.001 & (0.001) & 0.106 &  &  & \cellcolor[HTML]{D9D9D9} soc\_ben\_inc & -0.002 & (0.001) & 0.021 & ** \\
 & invest\_inc & -0.001 & (0.001) & 0.254 &  &  & \cellcolor[HTML]{D9D9D9} invest\_inc & -0.002 & (0.001) & 0.002 & *** \\
 & empl\_not\_paid & -0.241 & (0.272) & 0.380 &  &  & empl\_not\_paid & 0.530 & (0.484) & 0.278 &  \\
 & internet\_access & -0.387 & (1.204) & 0.750 &  &  & internet\_access & 4.197 & (4.503) & 0.356 &  \\
 & \cellcolor[HTML]{D9D9D9} empl & 0.726 & (0.284) & 0.014 & ** &  & \cellcolor[HTML]{D9D9D9} empl & 1.337 & (0.413) & 0.002 & *** \\
 & bills & 0.325 & (0.499) & 0.519 &  &  & bills & 0.011 & (0.59) & 0.985 &  \\
 & \cellcolor[HTML]{D9D9D9} eth\_min & -1.696 & (0.288) & 0.000& *** &  & \cellcolor[HTML]{D9D9D9} eth\_min & -1.663 & (0.683) & 0.019 & ** \\
 \midrule
\multirow{11}{*}{\textbf{0.5}} & \cellcolor[HTML]{D9D9D9} Intercept & 11.236 & (1.778) & 0.000& *** &  & Intercept & 5.517 & (5.37) & 0.309 &  \\
 & area & 0.065 & (0.183) & 0.723 &  &  & area & 0.138 & (0.126) & 0.279 &  \\
 & \cellcolor[HTML]{D9D9D9} hhsize & -0.575 & (0.154) & 0.000& *** &  & \cellcolor[HTML]{D9D9D9} hhsize & -0.699 & (0.247) & 0.007 & *** \\
 & \cellcolor[HTML]{D9D9D9} n\_beds & -0.495 & (0.146) & 0.001 & *** &  & n\_beds & -0.204 & (0.237) & 0.392 &  \\
 & \cellcolor[HTML]{D9D9D9} soc\_ben\_inc & 0.001 & (0.000) & 0.000& *** &  & soc\_ben\_inc & 0.001 & (0.000) & 0.145 &  \\
 & invest\_inc & 0.000& (0.000) & 0.135 &  &  & \cellcolor[HTML]{D9D9D9} invest\_inc & 0.000& (0.000) & 0.026 & ** \\
 & empl\_not\_paid & -0.088 & (0.249) & 0.724 &  &  & empl\_not\_paid & 0.656 & (0.418) & 0.122 &  \\
 & internet\_access & -0.303 & (1.483) & 0.839 &  &  & internet\_access & 4.267 & (5.078) & 0.405 &  \\
 & \cellcolor[HTML]{D9D9D9} empl & 0.735 & (0.325) & 0.028 & ** &  & \cellcolor[HTML]{D9D9D9} empl & 1.346 & (0.386) & 0.001 & *** \\
 & bills & 0.409 & (0.41) & 0.324 &  &  & bills & 0.084 & (0.743) & 0.910 &  \\
 & \cellcolor[HTML]{D9D9D9} eth\_min & -1.693 & (0.4) & 0.000& *** &  & \cellcolor[HTML]{D9D9D9} eth\_min & -1.662 & (0.591) & 0.007 & *** \\
\midrule
\multirow{11}{*}{\textbf{0.9}} & \cellcolor[HTML]{D9D9D9} Intercept & 11.285 & (1.531) & 0.000& *** &  & Intercept & 5.542 & (5.511) & 0.320 &  \\
 & area & 0.129 & (0.22) & 0.560 &  &  & area & 0.168 & (0.15) & 0.268 &  \\
 & hhsize & -0.385 & (0.217) & 0.083 &  &  & \cellcolor[HTML]{D9D9D9} hhsize & -0.601 & (0.21) & 0.006 & *** \\
 & n\_beds & -0.335 & (0.196) & 0.093 &  &  & n\_beds & -0.122 & (0.203) & 0.550 &  \\
 & \cellcolor[HTML]{D9D9D9} soc\_ben\_inc & 0.007 & (0.001) & 0.000& *** &  & \cellcolor[HTML]{D9D9D9} soc\_ben\_inc & 0.008 & (0.001) & 0.000& *** \\
 & invest\_inc & 0.001 & (0.001) & 0.417 &  &  & invest\_inc & 0.001 & (0.001) & 0.151 &  \\
 & empl\_not\_paid & 0.006 & (0.283) & 0.983 &  &  & \cellcolor[HTML]{D9D9D9} empl\_not\_paid & 0.704 & (0.345) & 0.047 & ** \\
 & internet\_access & -0.253 & (1.433) & 0.860 &  &  & internet\_access & 4.292 & (5.184) & 0.412 &  \\
 & \cellcolor[HTML]{D9D9D9} empl & 0.740 & (0.325) & 0.027 & ** &  & \cellcolor[HTML]{D9D9D9} empl & 1.350 & (0.385) & 0.001 & *** \\
 & bills & 0.458 & (0.403) & 0.261 &  &  & bills & 0.110 & (0.727) & 0.880 &  \\
 & \cellcolor[HTML]{D9D9D9} eth\_min & -1.690 & (0.365) & 0.000& *** &  & \cellcolor[HTML]{D9D9D9} eth\_min & -1.660 & (0.662) & 0.015 & ** \\ \cmidrule(l){2-12} 
\end{tabular}%
}
\end{table}

\section{Conclusions}
\label{sec:SDQ-conclusions}

The COVID-19 pandemic has significantly affected different aspects of society, including children's mental health. This chapter investigates the risk factors driving children's mental health before and during the pandemic using data from the UK Household Longitudinal Study. The analysis employs the novel machine learning algorithm FM-QRF of Chapter \ref{ch:FM-QRF} to model the complex relationship between pandemic-related factors and children's mental health, measured with the well-known SDQ score.

\vspace{0.15in}

\noindent The empirical findings reveal that the drivers of children's mental health differ between those with low and high SDQ scores, and that these drivers also vary before and during the pandemic. Moreover, by studying the SDQ conditional distribution at different quantile levels, this study provides a deeper understanding of the impact of the pandemic on children's mental health outcomes with respect to standard linear approached.

\vspace{0.15in}

\noindent Key findings indicate that social benefit income variable remains a crucial factor across different quantile levels in both pre-pandemic and pandemic periods. Additionally, the importance of internet access significantly increased during the pandemic, especially for children with lower levels of behavioral issues. The study also highlights the higher relevance of family poverty during the pandemic for children with higher levels of behavioral issues.

\vspace{0.15in}

\noindent The comparison with the LQMM model results highlights the relevance of the non-parametric approach employed in this study. As a matter of fact, the FM-QRF  offers an additional approach to gain a more comprehensive understanding of the complex dynamics affecting children's mental health, since it reveals non-linear relationships among variables that are overlooked by linear models, such as the LQMM.

\vspace{0.15in}

\noindent In conclusion, this research contributes to the growing body of literature addressing the impact of the COVID-19 pandemic on children's mental health. The use of non-linear approaches like FM-QRF enhances the depth of the analysis, providing valuable insights for policymakers, educators, and healthcare professionals working to support and improve children's well-being.

