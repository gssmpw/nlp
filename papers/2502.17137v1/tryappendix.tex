\chapter{Mixed-Frequency Quantile
Regression Forests}\label{app:MIDAS-QRF}

This section reports additional Figures and Tables from Chapter \ref{ch:MIDAS-QRF}.


% Please add the following required packages to your document preamble:
% \usepackage{booktabs}
\begin{table}[h]
\centering
\begin{tabularx}{\textwidth}{*{8}{>{\centering\arraybackslash}X}}
\toprule
            & \textit{Obs}  & \textit{Min}     & \textit{Max}    & \textit{Mean}  & \textit{SD }   & \textit{Skew.}  & \textit{Kurt.}  \\ \midrule
WTI         & 2053 & -28.138 & 42.583 & 0.042 & 3.275 & 1.177  & 39.354 \\
BRENT       & 2053 & -25.639 & 41.202 & 0.036 & 2.922 & 1.107  & 39.192 \\
HEAT. & 2053 & -22.314 & 14.862 & 0.022 & 2.511 & -0.416 & 12.046 \\ 
SP500 & 2053 & -12.765 & 8.968 & 0.047 & 1.095 & -1.039 & 23.955  \\

DOLL & 92 &  4.429 & 4.747 & 4.599 & 0.090 & -0.439 & 1.638 \\

NATGAS & 31 &   -0.219 & 0.605 & 0.158 & 0.166 & 0.284 & 3.158 \\

SAUDI\_P. & 31 &   -1.313 & 1.775 & 0.0242 & 0.714 &  0.350 & 2.746 \\

\bottomrule
\end{tabularx}
\caption{Summary statistics of the variables included in the sample. The table reports the number of observations (Obs), the minimum (Min), maximum (Max) along with the mean, standard deviation (SD), skewness and excess kurtosis.}
\label{tab:sum-stats}
\end{table}

\begin{figure}[b]
    \centering
\includegraphics[width=\textwidth]{MIDAS-QRF/images/WTI.pdf}
    \caption{WTI index time series. the black line represents the training set and the red line the out-of-sample period.}
    \label{fig:wti}
\end{figure}

\begin{figure}[h]
    \centering \includegraphics[width=\textwidth]{MIDAS-QRF/images/BRENT.pdf}
    \caption{Brent index time series. the black line represents the training set and the red line the out-of-sample period.}
    \label{fig:brent}
\end{figure}

\begin{figure}[h]
    \centering
    \includegraphics[width=\textwidth]{MIDAS-QRF/images/HEAT_OIL.pdf}
    \caption{Heating oil index time series. the black line represents the training set and the red line the out-of-sample period.}
    \label{fig:heat}
\end{figure}

\begin{figure}[h]
    \centering
    \includegraphics[width=\textwidth]{MIDAS-QRF/images/brent_pred.pdf}
\includegraphics[width=\textwidth]{MIDAS-QRF/images/brent_pred_dyn.pdf}
    \caption{Brent index (black line) out-of-sample predictions at quantile levels $\tau= 0.01, 0.025, 0.05$. The top panel and the bottom panel show the predictions obtained with the dynamic MIDAS-QRF model, respectively.}
    \label{fig:brent-pred}
\end{figure}

\begin{figure}[h]
    \centering
    \includegraphics[width=\textwidth]{MIDAS-QRF/images/wti_pred.pdf}
\includegraphics[width=\textwidth]{MIDAS-QRF/images/wti_pred_dyn.pdf}
    \caption{WTI index (black line) out-of-sample predictions at quantile levels $\tau= 0.01, 0.025, 0.05$. The top panel and the bottom panel show the predictions obtained with the dynamic MIDAS-QRF model, respectively.}
    \label{fig:wti-pred}
\end{figure}

% Please add the following required packages to your document preamble:
% \usepackage{booktabs}
% \usepackage{multirow}
% \usepackage{graphicx}
\begin{table}[]
\centering
\resizebox{\columnwidth}{!}{%
\begin{tabularx}{\textwidth}{*{5}{>{\centering\arraybackslash}X}}
\toprule
\multicolumn{5}{c}{\textbf{WTI}}                                                                                               \\ \midrule
                                  & $\tau$         & \textit{\textbf{0.01}} & \textit{\textbf{0.025}} & \textit{\textbf{0.05}} \\
\multirow{2}{*}{\textit{Static}}  & \textit{0.01}  & \multicolumn{1}{c}{-}  & 0.21                    & 0.04                   \\
                                  & \textit{0.025} &                        & \multicolumn{1}{c}{-}   & 0.12                   \\ \midrule
\multirow{2}{*}{\textit{Dynamic}} & \textit{0.01}  & \multicolumn{1}{c}{-}  & 0.12                    & 0.01                   \\
                                  & \textit{0.025} &                        & \multicolumn{1}{c}{-}   & 0.07                   \\ \midrule
\multicolumn{5}{c}{\textbf{BRENT}}                                                                                             \\ \midrule
                                  & $\tau$         & \textit{\textbf{0.01}} & \textit{\textbf{0.025}} & \textit{\textbf{0.05}} \\
\multirow{2}{*}{\textit{Static}}  & \textit{0.01}  & \multicolumn{1}{c}{-}  & 0.08                    & 0.007                  \\
                                  & \textit{0.025} &                        & \multicolumn{1}{c}{-}   & 0.16                   \\ \midrule
\multirow{2}{*}{\textit{Dynamic}} & \textit{0.01}  & \multicolumn{1}{c}{-}  & 0.158                   & 0.01                   \\
                                  & \textit{0.025} &                        & \multicolumn{1}{c}{-}   & 0.24                   \\ \midrule
\multicolumn{5}{c}{\textbf{HEATING OIL}}                                                                                       \\ \midrule
                                  & $\tau$         & \textit{\textbf{0.01}} & \textit{\textbf{0.025}} & \textit{\textbf{0.05}} \\
\multirow{2}{*}{\textit{Static}}  & \textit{0.01}  & \multicolumn{1}{c}{-}  & 0.08                    & 0.01                   \\
                                  & \textit{0.025} &                        & \multicolumn{1}{c}{-}   & 0.11                   \\ \midrule
\multirow{2}{*}{\textit{Dynamic}} & \textit{0.01}  & \multicolumn{1}{c}{-}  & 0.14                    & 0.02                   \\
                                  & \textit{0.025} &                        & \multicolumn{1}{c}{-}   & 0.11                   \\ \bottomrule
\end{tabularx}%
}
\caption{Ratio between the number of times quantiles computed at level $\tau$ indicated in the rows are higher than those computed at level $\tau$ in the columns.}
\label{tab:quants}
\end{table}

\chapter{Finite mixtures of Quantile
Regression Forests and their
application to GDP growth-at-risk
from climate change}


This section reports the mean (in bold) and the standard error of the quantile estimates at level $\tau=0.01, 0.5, 0.99$ for each country for the SSP1 and SSP5 scenario. The estimates are obtained via bootstrapping with 500 iterations. Each table reports the values for the years 2030, 2050, 2100.

% Please add the following required packages to your document preamble:
% \usepackage[table,xcdraw]{xcolor}
% Beamer presentation requires \usepackage{colortbl} instead of \usepackage[table,xcdraw]{xcolor}
% \usepackage{lscape}
% \usepackage{longtable}
% Note: It may be necessary to compile the document several times to get a multi-page table to line up properly
\begin{landscape}
    
\footnotesize
\begin{longtable}[c]{
>{\columncolor[HTML]{D4D4D4}}l llllllllllllllllll}
\hline
\cellcolor[HTML]{B0B3B2}                     & \multicolumn{6}{c}{\cellcolor[HTML]{B0B3B2}\textbf{0.01}}                                                                                                                         & \multicolumn{6}{c}{\cellcolor[HTML]{B0B3B2}\textbf{0.5}}                                                                                                                          & \multicolumn{6}{c}{\cellcolor[HTML]{B0B3B2}\textbf{0.99}}                                                                                                                         \\ \hline
\endfirsthead
%
\multicolumn{19}{c}%
{{\bfseries Table \thetable\ continued from previous page}} \\
\hline
\cellcolor[HTML]{B0B3B2}                     & \multicolumn{6}{c}{\cellcolor[HTML]{B0B3B2}\textbf{0.01}}                                                                                                                         & \multicolumn{6}{c}{\cellcolor[HTML]{B0B3B2}\textbf{0.5}}                                                                                                                          & \multicolumn{6}{c}{\cellcolor[HTML]{B0B3B2}\textbf{0.99}}                                                                                                                         \\ \hline
\endhead
%
\hline
\endfoot
%
\endlastfoot
%
\multicolumn{1}{c}{\cellcolor[HTML]{B0B3B2}} & \multicolumn{2}{c}{\cellcolor[HTML]{B0B3B2}\textbf{2030}} & \multicolumn{2}{c}{\cellcolor[HTML]{B0B3B2}\textbf{2050}} & \multicolumn{2}{c}{\cellcolor[HTML]{B0B3B2}\textbf{2100}} & \multicolumn{2}{c}{\cellcolor[HTML]{B0B3B2}\textbf{2030}} & \multicolumn{2}{c}{\cellcolor[HTML]{B0B3B2}\textbf{2050}} & \multicolumn{2}{c}{\cellcolor[HTML]{B0B3B2}\textbf{2100}} & \multicolumn{2}{c}{\cellcolor[HTML]{B0B3B2}\textbf{2030}} & \multicolumn{2}{c}{\cellcolor[HTML]{B0B3B2}\textbf{2050}} & \multicolumn{2}{c}{\cellcolor[HTML]{B0B3B2}\textbf{2100}} \\
\textbf{AFG}                                 & \textbf{-4.56}                    & 0.39                  & \textbf{-4.32}                    & 0.4                   & \textbf{-4.83}                    & 0.51                  & \textbf{6.51}                    & 0.22                   & \textbf{6.44}                    & 0.24                   & \textbf{6.38}                    & 0.22                   & \textbf{21.25}                   & 0.72                   & \textbf{20.99}                   & 0.66                   & \textbf{21.1}                    & 0.69                   \\
\textbf{ALB}                                 & \textbf{-4.49}                    & 0.53                  & \textbf{-4.51}                    & 0.53                  & \textbf{-3.88}                    & 0.38                  & \textbf{3.99}                    & 0.2                    & \textbf{3.99}                    & 0.16                   & \textbf{4.26}                    & 0.17                   & \textbf{9.33}                    & 0.32                   & \textbf{9.7}                     & 0.35                   & \textbf{9.77}                    & 0.32                   \\
\textbf{DZA}                                 & \textbf{-4.1}                     & 0.3                   & \textbf{-4.12}                    & 0.29                  & \textbf{-3.85}                    & 0.24                  & \textbf{3.53}                    & 0.08                   & \textbf{3.35}                    & 0.08                   & \textbf{3.56}                    & 0.08                   & \textbf{8.75}                    & 0.25                   & \textbf{8.96}                    & 0.36                   & \textbf{8.71}                    & 0.23                   \\
\textbf{AGO}                                 & \textbf{-4}                       & 0.33                  & \textbf{-3.82}                    & 0.31                  & \textbf{-3.89}                    & 0.33                  & \textbf{5.49}                    & 0.21                   & \textbf{5.29}                    & 0.22                   & \textbf{5.32}                    & 0.22                   & \textbf{16.12}                   & 0.47                   & \textbf{15.81}                   & 0.45                   & \textbf{16.15}                   & 0.46                   \\
\textbf{ATG}                                 & \textbf{-16.1}                    & 0.55                  & \textbf{-17.38}                   & 0.89                  & \textbf{-17.17}                   & 0.91                  & \textbf{2.65}                    & 0.48                   & \textbf{2.35}                    & 0.48                   & \textbf{2.49}                    & 0.5                    & \textbf{13.03}                   & 0.41                   & \textbf{12.65}                   & 0.41                   & \textbf{12.97}                   & 0.43                   \\
\textbf{ARG}                                 & \textbf{-13.31}                   & 0.49                  & \textbf{-13.89}                   & 0.73                  & \textbf{-13.6}                    & 0.44                  & \textbf{3.69}                    & 0.37                   & \textbf{3.59}                    & 0.36                   & \textbf{3.59}                    & 0.38                   & \textbf{11.44}                   & 0.39                   & \textbf{11.25}                   & 0.35                   & \textbf{11.28}                   & 0.35                   \\
\textbf{ARM}                                 & \textbf{-18.12}                   & 0.55                  & \textbf{-16.49}                   & 0.47                  & \textbf{-17.1}                    & 0.52                  & \textbf{4.45}                    & 0.61                   & \textbf{6.6}                     & 0.61                   & \textbf{6.38}                    & 0.61                   & \textbf{16.65}                   & 0.45                   & \textbf{17}                      & 0.44                   & \textbf{17.08}                   & 0.45                   \\
\textbf{ABW}                                 & \textbf{-15.38}                   & 0.51                  & \textbf{-15.2}                    & 0.59                  & \textbf{-15.37}                   & 0.54                  & \textbf{1.14}                    & 0.27                   & \textbf{1.14}                    & 0.27                   & \textbf{0.93}                    & 0.28                   & \textbf{8.82}                    & 0.29                   & \textbf{9.05}                    & 0.31                   & \textbf{8.81}                    & 0.3                    \\
\textbf{AUS}                                 & \textbf{-4.35}                    & 0.64                  & \textbf{-4.92}                    & 0.78                  & \textbf{-3.68}                    & 0.28                  & \textbf{3.12}                    & 0.11                   & \textbf{3.06}                    & 0.12                   & \textbf{3.21}                    & 0.12                   & \textbf{8.64}                    & 0.23                   & \textbf{8.54}                    & 0.23                   & \textbf{8.75}                    & 0.27                   \\
\textbf{AUT}                                 & \textbf{-5.62}                    & 0.31                  & \textbf{-6.57}                    & 0.36                  & \textbf{-6.82}                    & 1.45                  & \textbf{1.16}                    & 0.14                   & \textbf{0.82}                    & 0.13                   & \textbf{0.75}                    & 0.12                   & \textbf{8.23}                    & 0.27                   & \textbf{8.15}                    & 0.27                   & \textbf{8.32}                    & 0.28                   \\
\textbf{AZE}                                 & \textbf{-6.3}                     & 0.76                  & \textbf{-5.21}                    & 0.41                  & \textbf{-5.02}                    & 0.35                  & \textbf{6.86}                    & 0.22                   & \textbf{7.12}                    & 0.2                    & \textbf{7.08}                    & 0.2                    & \textbf{29.95}                   & 0.9                    & \textbf{29.86}                   & 0.91                   & \textbf{30.71}                   & 0.89                   \\
\textbf{BHS}                                 & \textbf{-6.85}                    & 0.5                   & \textbf{-6.85}                    & 0.49                  & \textbf{-6.8}                     & 0.39                  & \textbf{1.24}                    & 0.17                   & \textbf{1.52}                    & 0.2                    & \textbf{1.32}                    & 0.21                   & \textbf{8.74}                    & 0.25                   & \textbf{8.91}                    & 0.27                   & \textbf{8.95}                    & 0.29                   \\
\textbf{BHR}                                 & \textbf{-4.05}                    & 0.26                  & \textbf{-3.69}                    & 0.31                  & \textbf{-4.32}                    & 0.53                  & \textbf{3.75}                    & 0.12                   & \textbf{4.77}                    & 0.14                   & \textbf{4.59}                    & 0.16                   & \textbf{8.75}                    & 0.2                    & \textbf{8.86}                    & 0.19                   & \textbf{9.04}                    & 0.23                   \\
\textbf{BGD}                                 & \textbf{-4.03}                    & 0.37                  & \textbf{-3.88}                    & 0.36                  & \textbf{-4.18}                    & 0.46                  & \textbf{5.22}                    & 0.17                   & \textbf{5.28}                    & 0.18                   & \textbf{5.3}                     & 0.16                   & \textbf{9.75}                    & 0.29                   & \textbf{9.74}                    & 0.27                   & \textbf{9.61}                    & 0.27                   \\
\textbf{BRB}                                 & \textbf{-6.39}                    & 0.4                   & \textbf{-6.64}                    & 0.4                   & \textbf{-6.4}                     & 0.29                  & \textbf{1.22}                    & 0.18                   & \textbf{0.99}                    & 0.2                    & \textbf{1.38}                    & 0.18                   & \textbf{8.45}                    & 0.27                   & \textbf{8.59}                    & 0.28                   & \textbf{8.8}                     & 0.27                   \\
\textbf{BLR}                                 & \textbf{-3.26}                    & 0.27                  & \textbf{-4.47}                    & 0.8                   & \textbf{-4.52}                    & 0.45                  & \textbf{6.23}                    & 0.22                   & \textbf{6.07}                    & 0.24                   & \textbf{5.93}                    & 0.24                   & \textbf{12.92}                   & 0.35                   & \textbf{12.33}                   & 0.34                   & \textbf{12.42}                   & 0.34                   \\
\textbf{BEL}                                 & \textbf{-4.47}                    & 0.65                  & \textbf{-5.68}                    & 0.57                  & \textbf{-5.4}                     & 0.99                  & \textbf{1.46}                    & 0.13                   & \textbf{0.65}                    & 0.13                   & \textbf{1.4}                     & 0.09                   & \textbf{8.22}                    & 0.28                   & \textbf{8.09}                    & 0.27                   & \textbf{8.4}                     & 0.28                   \\
\textbf{BLZ}                                 & \textbf{-3.91}                    & 0.3                   & \textbf{-4.1}                     & 0.28                  & \textbf{-4.28}                    & 0.34                  & \textbf{3.68}                    & 0.13                   & \textbf{3.46}                    & 0.13                   & \textbf{3.37}                    & 0.12                   & \textbf{12.95}                   & 0.4                    & \textbf{12.37}                   & 0.4                    & \textbf{12.47}                   & 0.42                   \\
\textbf{BEN}                                 & \textbf{-3.66}                    & 0.21                  & \textbf{-4}                       & 0.38                  & \textbf{-3.71}                    & 0.26                  & \textbf{4.08}                    & 0.12                   & \textbf{4.18}                    & 0.13                   & \textbf{4.13}                    & 0.13                   & \textbf{8.63}                    & 0.18                   & \textbf{8.93}                    & 0.22                   & \textbf{8.78}                    & 0.21                   \\
\textbf{BMU}                                 & \textbf{-8.68}                    & 0.34                  & \textbf{-7.86}                    & 0.33                  & \textbf{-8.63}                    & 0.33                  & \textbf{1.1}                     & 0.23                   & \textbf{1.01}                    & 0.25                   & \textbf{0.93}                    & 0.27                   & \textbf{10.23}                   & 0.42                   & \textbf{10.39}                   & 0.42                   & \textbf{10.15}                   & 0.42                   \\
\textbf{BTN}                                 & \textbf{-4.15}                    & 0.65                  & \textbf{-4.29}                    & 0.46                  & \textbf{-6.09}                    & 1.38                  & \textbf{6.76}                    & 0.23                   & \textbf{6.71}                    & 0.2                    & \textbf{6.54}                    & 0.19                   & \textbf{18.52}                   & 0.58                   & \textbf{18.13}                   & 0.56                   & \textbf{18.07}                   & 0.55                   \\
\textbf{BOL}                                 & \textbf{-3.9}                     & 0.23                  & \textbf{-4.78}                    & 0.55                  & \textbf{-4.51}                    & 0.34                  & \textbf{4.07}                    & 0.11                   & \textbf{3.7}                     & 0.11                   & \textbf{4.01}                    & 0.1                    & \textbf{9.83}                    & 0.33                   & \textbf{9.58}                    & 0.29                   & \textbf{10}                      & 0.3                    \\
\textbf{BIH}                                 & \textbf{-5.34}                    & 0.75                  & \textbf{-5.5}                     & 1.04                  & \textbf{-5.03}                    & 0.71                  & \textbf{3.14}                    & 0.22                   & \textbf{3.49}                    & 0.23                   & \textbf{3.1}                     & 0.22                   & \textbf{12.1}                    & 0.44                   & \textbf{12.43}                   & 0.4                    & \textbf{12.41}                   & 0.45                   \\
\textbf{BWA}                                 & \textbf{-9.59}                    & 0.45                  & \textbf{-9.86}                    & 0.54                  & \textbf{-9.5}                     & 0.52                  & \textbf{3.96}                    & 0.42                   & \textbf{4.05}                    & 0.33                   & \textbf{3.99}                    & 0.36                   & \textbf{12}                      & 0.33                   & \textbf{11.6}                    & 0.29                   & \textbf{11.41}                   & 0.34                   \\
\textbf{BRA}                                 & \textbf{-4.66}                    & 0.45                  & \textbf{-5.5}                     & 0.83                  & \textbf{-4.64}                    & 0.38                  & \textbf{3.41}                    & 0.13                   & \textbf{3.26}                    & 0.13                   & \textbf{3.43}                    & 0.1                    & \textbf{10.11}                   & 0.34                   & \textbf{10.41}                   & 0.38                   & \textbf{10.18}                   & 0.34                   \\
\textbf{BRN}                                 & \textbf{-4.91}                    & 0.31                  & \textbf{-4.79}                    & 0.31                  & \textbf{-4.98}                    & 0.29                  & \textbf{1.14}                    & 0.14                   & \textbf{1.14}                    & 0.19                   & \textbf{0.89}                    & 0.17                   & \textbf{8.74}                    & 0.29                   & \textbf{8.85}                    & 0.3                    & \textbf{8.75}                    & 0.3                    \\
\textbf{BGR}                                 & \textbf{-5.25}                    & 0.44                  & \textbf{-5.54}                    & 0.39                  & \textbf{-5.62}                    & 0.46                  & \textbf{4.23}                    & 0.24                   & \textbf{4.07}                    & 0.21                   & \textbf{3.81}                    & 0.21                   & \textbf{8.82}                    & 0.2                    & \textbf{8.43}                    & 0.2                    & \textbf{8.39}                    & 0.19                   \\
\textbf{BFA}                                 & \textbf{-4.03}                    & 0.33                  & \textbf{-3.96}                    & 0.51                  & \textbf{-3.74}                    & 0.42                  & \textbf{5.09}                    & 0.19                   & \textbf{5.28}                    & 0.18                   & \textbf{5.48}                    & 0.18                   & \textbf{9.4}                     & 0.23                   & \textbf{9.52}                    & 0.22                   & \textbf{9.79}                    & 0.27                   \\
\textbf{BDI}                                 & \textbf{-4.7}                     & 0.46                  & \textbf{-4.72}                    & 0.45                  & \textbf{-4.48}                    & 0.51                  & \textbf{3.51}                    & 0.2                    & \textbf{3.72}                    & 0.2                    & \textbf{3.61}                    & 0.18                   & \textbf{8.79}                    & 0.24                   & \textbf{8.61}                    & 0.21                   & \textbf{8.84}                    & 0.25                   \\
\textbf{CPV}                                 & \textbf{-4.97}                    & 0.65                  & \textbf{-4.55}                    & 0.66                  & \textbf{-5.08}                    & 0.55                  & \textbf{4}                       & 0.22                   & \textbf{3.97}                    & 0.2                    & \textbf{3.51}                    & 0.19                   & \textbf{16.47}                   & 0.5                    & \textbf{16.51}                   & 0.51                   & \textbf{16.52}                   & 0.54                   \\
\textbf{KHM}                                 & \textbf{-4.21}                    & 0.44                  & \textbf{-4.33}                    & 0.45                  & \textbf{-4.84}                    & 0.53                  & \textbf{7.06}                    & 0.2                    & \textbf{7.08}                    & 0.19                   & \textbf{6.55}                    & 0.23                   & \textbf{14.4}                    & 0.38                   & \textbf{14.57}                   & 0.42                   & \textbf{14.46}                   & 0.42                   \\
\textbf{CMR}                                 & \textbf{-4.46}                    & 0.4                   & \textbf{-4.47}                    & 0.56                  & \textbf{-4.01}                    & 0.42                  & \textbf{3.85}                    & 0.13                   & \textbf{3.78}                    & 0.1                    & \textbf{3.8}                     & 0.11                   & \textbf{9.24}                    & 0.27                   & \textbf{9.15}                    & 0.27                   & \textbf{9.24}                    & 0.26                   \\
\textbf{CAN}                                 & \textbf{-4.52}                    & 0.28                  & \textbf{-4.43}                    & 0.25                  & \textbf{-4.68}                    & 0.29                  & \textbf{3.42}                    & 0.16                   & \textbf{3.39}                    & 0.17                   & \textbf{3.34}                    & 0.16                   & \textbf{9.25}                    & 0.29                   & \textbf{9.28}                    & 0.29                   & \textbf{9.16}                    & 0.26                   \\
\textbf{CYM}                                 & \textbf{-9.07}                    & 0.39                  & \textbf{-10.12}                   & 0.98                  & \textbf{-9.02}                    & 0.34                  & \textbf{0.73}                    & 0.33                   & \textbf{1.13}                    & 0.31                   & \textbf{0.97}                    & 0.29                   & \textbf{8.6}                     & 0.28                   & \textbf{8.57}                    & 0.29                   & \textbf{8.39}                    & 0.26                   \\
\textbf{CAF}                                 & \textbf{-51.35}                   & 2.08                  & \textbf{-52.95}                   & 1.76                  & \textbf{-52.08}                   & 1.85                  & \textbf{3.2}                     & 0.64                   & \textbf{2.97}                    & 0.66                   & \textbf{2.84}                    & 0.67                   & \textbf{9.23}                    & 0.25                   & \textbf{9.17}                    & 0.27                   & \textbf{9.17}                    & 0.25                   \\
\textbf{TCD}                                 & \textbf{-4.31}                    & 0.42                  & \textbf{-4.21}                    & 0.48                  & \textbf{-3.9}                     & 0.3                   & \textbf{6.34}                    & 0.28                   & \textbf{6.42}                    & 0.31                   & \textbf{6.68}                    & 0.31                   & \textbf{29.19}                   & 0.84                   & \textbf{29.53}                   & 0.86                   & \textbf{29.47}                   & 0.88                   \\
\textbf{CHL}                                 & \textbf{-4.19}                    & 0.31                  & \textbf{-4.01}                    & 0.31                  & \textbf{-3.97}                    & 0.41                  & \textbf{4.5}                     & 0.14                   & \textbf{4.08}                    & 0.17                   & \textbf{4.4}                     & 0.17                   & \textbf{9.2}                     & 0.23                   & \textbf{8.83}                    & 0.23                   & \textbf{9.07}                    & 0.24                   \\
\textbf{CHN}                                 & \textbf{-4.32}                    & 0.67                  & \textbf{-3.98}                    & 0.29                  & \textbf{-4.54}                    & 0.49                  & \textbf{7.17}                    & 0.19                   & \textbf{7.17}                    & 0.21                   & \textbf{6.65}                    & 0.2                    & \textbf{14.42}                   & 0.37                   & \textbf{14.6}                    & 0.33                   & \textbf{14.21}                   & 0.37                   \\
\textbf{COL}                                 & \textbf{-4.36}                    & 0.64                  & \textbf{-4.17}                    & 0.41                  & \textbf{-5.19}                    & 0.97                  & \textbf{3.89}                    & 0.16                   & \textbf{3.72}                    & 0.13                   & \textbf{3.75}                    & 0.15                   & \textbf{8.69}                    & 0.19                   & \textbf{8.66}                    & 0.2                    & \textbf{8.75}                    & 0.23                   \\
\textbf{COM}                                 & \textbf{-4.19}                    & 0.41                  & \textbf{-4.23}                    & 0.64                  & \textbf{-4.38}                    & 0.4                   & \textbf{2.8}                     & 0.11                   & \textbf{2.87}                    & 0.12                   & \textbf{2.72}                    & 0.11                   & \textbf{11.4}                    & 0.41                   & \textbf{11.52}                   & 0.42                   & \textbf{11.25}                   & 0.41                   \\
\textbf{COD}                                 & \textbf{-8.79}                    & 0.77                  & \textbf{-8.26}                    & 0.39                  & \textbf{-8.18}                    & 0.46                  & \textbf{5.44}                    & 0.4                    & \textbf{5.5}                     & 0.36                   & \textbf{5.32}                    & 0.41                   & \textbf{11.18}                   & 0.33                   & \textbf{10.99}                   & 0.32                   & \textbf{11.12}                   & 0.34                   \\
\textbf{COG}                                 & \textbf{-8.9}                     & 0.43                  & \textbf{-8.41}                    & 0.37                  & \textbf{-8.43}                    & 0.44                  & \textbf{5.54}                    & 0.42                   & \textbf{5.65}                    & 0.43                   & \textbf{5.61}                    & 0.39                   & \textbf{13.37}                   & 0.43                   & \textbf{13.26}                   & 0.4                    & \textbf{13.38}                   & 0.44                   \\
\textbf{CRI}                                 & \textbf{-5.52}                    & 0.62                  & \textbf{-5.31}                    & 0.45                  & \textbf{-5.14}                    & 0.46                  & \textbf{3.71}                    & 0.13                   & \textbf{3.56}                    & 0.09                   & \textbf{3.62}                    & 0.11                   & \textbf{9.41}                    & 0.35                   & \textbf{9.42}                    & 0.34                   & \textbf{9.52}                    & 0.34                   \\
\textbf{CIV}                                 & \textbf{-7.13}                    & 0.28                  & \textbf{-6.96}                    & 0.34                  & \textbf{-6.68}                    & 0.35                  & \textbf{1.71}                    & 0.24                   & \textbf{1.84}                    & 0.25                   & \textbf{1.69}                    & 0.22                   & \textbf{10.82}                   & 0.39                   & \textbf{11.21}                   & 0.41                   & \textbf{10.89}                   & 0.38                   \\
\textbf{HRV}                                 & \textbf{-9.14}                    & 0.5                   & \textbf{-10.55}                   & 1.06                  & \textbf{-10.29}                   & 1.52                  & \textbf{1.92}                    & 0.25                   & \textbf{2.17}                    & 0.26                   & \textbf{1.22}                    & 0.25                   & \textbf{8.45}                    & 0.26                   & \textbf{8.73}                    & 0.27                   & \textbf{8.36}                    & 0.27                   \\
\textbf{CUW}                                 & \textbf{-5.3}                     & 0.44                  & \textbf{-4.99}                    & 0.49                  & \textbf{-5.32}                    & 0.44                  & \textbf{1.24}                    & 0.12                   & \textbf{1.28}                    & 0.14                   & \textbf{1.13}                    & 0.12                   & \textbf{8.4}                     & 0.26                   & \textbf{8.73}                    & 0.28                   & \textbf{8.55}                    & 0.29                   \\
\textbf{CYP}                                 & \textbf{-8.65}                    & 0.48                  & \textbf{-7.96}                    & 0.29                  & \textbf{-8.59}                    & 0.57                  & \textbf{2.74}                    & 0.31                   & \textbf{2.99}                    & 0.31                   & \textbf{2.93}                    & 0.32                   & \textbf{8.75}                    & 0.25                   & \textbf{8.44}                    & 0.21                   & \textbf{8.22}                    & 0.21                   \\
\textbf{CZE}                                 & \textbf{-6.64}                    & 0.43                  & \textbf{-7.2}                     & 0.36                  & \textbf{-7.57}                    & 0.91                  & \textbf{2.54}                    & 0.2                    & \textbf{2.25}                    & 0.21                   & \textbf{2.56}                    & 0.18                   & \textbf{7.92}                    & 0.22                   & \textbf{7.88}                    & 0.21                   & \textbf{8.33}                    & 0.23                   \\
\textbf{DNK}                                 & \textbf{-6.17}                    & 0.42                  & \textbf{-5.76}                    & 0.37                  & \textbf{-7.87}                    & 1.33                  & \textbf{1.19}                    & 0.17                   & \textbf{1.16}                    & 0.16                   & \textbf{1.32}                    & 0.19                   & \textbf{8.55}                    & 0.28                   & \textbf{8.13}                    & 0.26                   & \textbf{8.43}                    & 0.27                   \\
\textbf{DJI}                                 & \textbf{-4.38}                    & 0.64                  & \textbf{-4}                       & 0.42                  & \textbf{-4.39}                    & 0.54                  & \textbf{6.86}                    & 0.25                   & \textbf{6.71}                    & 0.25                   & \textbf{6.63}                    & 0.22                   & \textbf{9.04}                    & 0.17                   & \textbf{9.07}                    & 0.19                   & \textbf{8.86}                    & 0.15                   \\
\textbf{DMA}                                 & \textbf{-5.92}                    & 0.67                  & \textbf{-5.67}                    & 0.8                   & \textbf{-5.31}                    & 0.35                  & \textbf{1.5}                     & 0.15                   & \textbf{1.25}                    & 0.14                   & \textbf{1.46}                    & 0.15                   & \textbf{8.64}                    & 0.28                   & \textbf{8.57}                    & 0.28                   & \textbf{8.69}                    & 0.26                   \\
\textbf{DOM}                                 & \textbf{-4.31}                    & 0.28                  & \textbf{-4.9}                     & 0.46                  & \textbf{-4.86}                    & 0.35                  & \textbf{4.03}                    & 0.11                   & \textbf{3.95}                    & 0.12                   & \textbf{3.83}                    & 0.11                   & \textbf{10.91}                   & 0.37                   & \textbf{10.9}                    & 0.36                   & \textbf{10.5}                    & 0.34                   \\
\textbf{ECU}                                 & \textbf{-4.14}                    & 0.28                  & \textbf{-3.94}                    & 0.36                  & \textbf{-4.14}                    & 0.34                  & \textbf{3.84}                    & 0.12                   & \textbf{3.74}                    & 0.13                   & \textbf{3.79}                    & 0.15                   & \textbf{9.98}                    & 0.32                   & \textbf{9.75}                    & 0.34                   & \textbf{9.73}                    & 0.31                   \\
\textbf{EGY}                                 & \textbf{-4.61}                    & 0.71                  & \textbf{-4.73}                    & 0.81                  & \textbf{-3.66}                    & 0.32                  & \textbf{3.72}                    & 0.09                   & \textbf{3.56}                    & 0.1                    & \textbf{4.06}                    & 0.11                   & \textbf{9.67}                    & 0.34                   & \textbf{9.59}                    & 0.39                   & \textbf{9.53}                    & 0.29                   \\
\textbf{SLV}                                 & \textbf{-4.26}                    & 0.3                   & \textbf{-4.61}                    & 0.5                   & \textbf{-3.92}                    & 0.31                  & \textbf{1.26}                    & 0.17                   & \textbf{1.14}                    & 0.14                   & \textbf{1.39}                    & 0.14                   & \textbf{8.74}                    & 0.27                   & \textbf{8.53}                    & 0.27                   & \textbf{8.71}                    & 0.27                   \\
\textbf{GNQ}                                 & \textbf{-11.01}                   & 0.46                  & \textbf{-10.89}                   & 0.43                  & \textbf{-10.65}                   & 0.38                  & \textbf{6.4}                     & 0.51                   & \textbf{6.58}                    & 0.48                   & \textbf{6.5}                     & 0.46                   & \textbf{56.33}                   & 1.93                   & \textbf{54.96}                   & 1.9                    & \textbf{56.32}                   & 1.9                    \\
\textbf{EST}                                 & \textbf{-17.89}                   & 0.56                  & \textbf{-17.56}                   & 0.73                  & \textbf{-18.25}                   & 0.79                  & \textbf{5.3}                     & 0.72                   & \textbf{5.23}                    & 0.75                   & \textbf{5.38}                    & 0.77                   & \textbf{11.49}                   & 0.3                    & \textbf{10.96}                   & 0.34                   & \textbf{11.29}                   & 0.32                   \\
\textbf{SWZ}                                 & \textbf{-4.15}                    & 0.36                  & \textbf{-4.28}                    & 0.36                  & \textbf{-4.84}                    & 1.02                  & \textbf{3.4}                     & 0.14                   & \textbf{3.27}                    & 0.14                   & \textbf{3.63}                    & 0.12                   & \textbf{8.54}                    & 0.22                   & \textbf{8.79}                    & 0.25                   & \textbf{9.12}                    & 0.28                   \\
\textbf{ETH}                                 & \textbf{-4.32}                    & 0.38                  & \textbf{-4.12}                    & 0.39                  & \textbf{-4.14}                    & 0.42                  & \textbf{7}                       & 0.21                   & \textbf{7.23}                    & 0.23                   & \textbf{6.96}                    & 0.22                   & \textbf{14.7}                    & 0.37                   & \textbf{15.07}                   & 0.39                   & \textbf{14.53}                   & 0.39                   \\
\textbf{FJI}                                 & \textbf{-4.42}                    & 0.41                  & \textbf{-4.49}                    & 0.4                   & \textbf{-4.42}                    & 0.41                  & \textbf{1.33}                    & 0.15                   & \textbf{1.18}                    & 0.15                   & \textbf{1.45}                    & 0.15                   & \textbf{8.57}                    & 0.25                   & \textbf{8.72}                    & 0.27                   & \textbf{8.68}                    & 0.26                   \\
\textbf{FIN}                                 & \textbf{-10.84}                   & 0.57                  & \textbf{-12.73}                   & 1.03                  & \textbf{-12.46}                   & 1                     & \textbf{1.85}                    & 0.28                   & \textbf{1.52}                    & 0.29                   & \textbf{1.53}                    & 0.28                   & \textbf{8.78}                    & 0.28                   & \textbf{8.18}                    & 0.25                   & \textbf{8.11}                    & 0.25                   \\
\textbf{FRA}                                 & \textbf{-4.65}                    & 0.4                   & \textbf{-5.35}                    & 0.55                  & \textbf{-5.43}                    & 1.06                  & \textbf{1.23}                    & 0.13                   & \textbf{1.24}                    & 0.12                   & \textbf{1.21}                    & 0.14                   & \textbf{8.23}                    & 0.27                   & \textbf{8.24}                    & 0.28                   & \textbf{8.39}                    & 0.27                   \\
\textbf{GAB}                                 & \textbf{-6.81}                    & 0.43                  & \textbf{-6.53}                    & 0.43                  & \textbf{-6.73}                    & 0.42                  & \textbf{1.36}                    & 0.24                   & \textbf{1.54}                    & 0.2                    & \textbf{1.32}                    & 0.24                   & \textbf{8.59}                    & 0.25                   & \textbf{8.67}                    & 0.26                   & \textbf{8.44}                    & 0.26                   \\
\textbf{GMB}                                 & \textbf{-10.84}                   & 0.51                  & \textbf{-11.47}                   & 0.67                  & \textbf{-11.52}                   & 0.97                  & \textbf{3.55}                    & 0.35                   & \textbf{3.55}                    & 0.42                   & \textbf{3.59}                    & 0.36                   & \textbf{9.69}                    & 0.28                   & \textbf{9.95}                    & 0.32                   & \textbf{9.85}                    & 0.32                   \\
\textbf{GEO}                                 & \textbf{-6.18}                    & 0.49                  & \textbf{-5.42}                    & 0.35                  & \textbf{-5.4}                     & 0.41                  & \textbf{3.67}                    & 0.27                   & \textbf{5.02}                    & 0.26                   & \textbf{4.95}                    & 0.26                   & \textbf{12.56}                   & 0.37                   & \textbf{12.93}                   & 0.39                   & \textbf{12.88}                   & 0.38                   \\
\textbf{DEU}                                 & \textbf{-7.15}                    & 0.36                  & \textbf{-8.01}                    & 0.57                  & \textbf{-7.52}                    & 0.42                  & \textbf{1.09}                    & 0.19                   & \textbf{0.82}                    & 0.18                   & \textbf{1.3}                     & 0.18                   & \textbf{8.41}                    & 0.27                   & \textbf{8.06}                    & 0.26                   & \textbf{8.39}                    & 0.27                   \\
\textbf{GHA}                                 & \textbf{-4.02}                    & 0.3                   & \textbf{-3.86}                    & 0.48                  & \textbf{-3.63}                    & 0.47                  & \textbf{5.06}                    & 0.15                   & \textbf{5.21}                    & 0.16                   & \textbf{5.24}                    & 0.16                   & \textbf{15.37}                   & 0.54                   & \textbf{15.7}                    & 0.54                   & \textbf{15.59}                   & 0.53                   \\
\textbf{GRL}                                 & \textbf{-12.98}                   & 0.49                  & \textbf{-12.68}                   & 0.46                  & \textbf{-14.1}                    & 0.99                  & \textbf{1.06}                    & 0.28                   & \textbf{1.19}                    & 0.23                   & \textbf{0.93}                    & 0.3                    & \textbf{8.71}                    & 0.27                   & \textbf{8.95}                    & 0.27                   & \textbf{9.12}                    & 0.31                   \\
\textbf{GUM}                                 & \textbf{-9.19}                    & 0.37                  & \textbf{-9.04}                    & 0.42                  & \textbf{-8.64}                    & 0.44                  & \textbf{0.79}                    & 0.19                   & \textbf{1.01}                    & 0.21                   & \textbf{1}                       & 0.22                   & \textbf{12.71}                   & 0.54                   & \textbf{13.03}                   & 0.56                   & \textbf{12.91}                   & 0.51                   \\
\textbf{GIN}                                 & \textbf{-4.06}                    & 0.33                  & \textbf{-4.22}                    & 0.41                  & \textbf{-3.8}                     & 0.47                  & \textbf{3.39}                    & 0.11                   & \textbf{3.28}                    & 0.14                   & \textbf{3.44}                    & 0.1                    & \textbf{8.87}                    & 0.28                   & \textbf{8.98}                    & 0.28                   & \textbf{8.95}                    & 0.29                   \\
\textbf{GNB}                                 & \textbf{-3.78}                    & 0.42                  & \textbf{-4.94}                    & 0.57                  & \textbf{-4.54}                    & 0.58                  & \textbf{3.82}                    & 0.15                   & \textbf{3.16}                    & 0.17                   & \textbf{3.35}                    & 0.14                   & \textbf{8.64}                    & 0.21                   & \textbf{8.64}                    & 0.2                    & \textbf{8.7}                     & 0.19                   \\
\textbf{GUY}                                 & \textbf{-4.66}                    & 0.41                  & \textbf{-4.75}                    & 0.4                   & \textbf{-5.27}                    & 0.55                  & \textbf{2.77}                    & 0.18                   & \textbf{2.58}                    & 0.2                    & \textbf{2.65}                    & 0.17                   & \textbf{8.99}                    & 0.27                   & \textbf{9.3}                     & 0.32                   & \textbf{8.76}                    & 0.25                   \\
\textbf{HTI}                                 & \textbf{-4.56}                    & 0.32                  & \textbf{-4.8}                     & 0.35                  & \textbf{-4.87}                    & 0.32                  & \textbf{2.11}                    & 0.21                   & \textbf{2.06}                    & 0.22                   & \textbf{1.98}                    & 0.21                   & \textbf{8.77}                    & 0.26                   & \textbf{8.66}                    & 0.27                   & \textbf{8.62}                    & 0.26                   \\
\textbf{HND}                                 & \textbf{-8.39}                    & 0.48                  & \textbf{-8.72}                    & 0.59                  & \textbf{-8.84}                    & 0.56                  & \textbf{1.34}                    & 0.22                   & \textbf{1.26}                    & 0.19                   & \textbf{1.22}                    & 0.22                   & \textbf{8.67}                    & 0.26                   & \textbf{8.32}                    & 0.25                   & \textbf{8.49}                    & 0.25                   \\
\textbf{HKG}                                 & \textbf{-5.75}                    & 0.44                  & \textbf{-5.56}                    & 0.45                  & \textbf{-7.7}                     & 1.55                  & \textbf{3.43}                    & 0.24                   & \textbf{3.47}                    & 0.27                   & \textbf{3.41}                    & 0.2                    & \textbf{9.93}                    & 0.36                   & \textbf{10.18}                   & 0.41                   & \textbf{10.04}                   & 0.39                   \\
\textbf{HUN}                                 & \textbf{-3.99}                    & 0.41                  & \textbf{-5.17}                    & 0.45                  & \textbf{-4.88}                    & 0.97                  & \textbf{3.49}                    & 0.19                   & \textbf{1.17}                    & 0.47                   & \textbf{3.36}                    & 0.21                   & \textbf{10.18}                   & 0.37                   & \textbf{10.07}                   & 0.33                   & \textbf{10.08}                   & 0.35                   \\
\textbf{ISL}                                 & \textbf{-8.35}                    & 0.34                  & \textbf{-9.01}                    & 0.49                  & \textbf{-8.72}                    & 0.33                  & \textbf{3.13}                    & 0.28                   & \textbf{3.04}                    & 0.28                   & \textbf{3.17}                    & 0.25                   & \textbf{8.96}                    & 0.3                    & \textbf{8.63}                    & 0.25                   & \textbf{8.95}                    & 0.29                   \\
\textbf{IND}                                 & \textbf{-9.4}                     & 0.37                  & \textbf{-9.19}                    & 0.41                  & \textbf{-10.39}                   & 0.89                  & \textbf{2.32}                    & 0.34                   & \textbf{2.6}                     & 0.34                   & \textbf{2.28}                    & 0.38                   & \textbf{8.92}                    & 0.25                   & \textbf{9.04}                    & 0.26                   & \textbf{8.86}                    & 0.25                   \\
\textbf{IDN}                                 & \textbf{-4.26}                    & 0.65                  & \textbf{-4.03}                    & 0.4                   & \textbf{-4.21}                    & 0.47                  & \textbf{6.84}                    & 0.2                    & \textbf{6.89}                    & 0.21                   & \textbf{6.79}                    & 0.23                   & \textbf{9.94}                    & 0.27                   & \textbf{10.14}                   & 0.26                   & \textbf{10.13}                   & 0.27                   \\
\textbf{IRN}                                 & \textbf{-4.22}                    & 0.65                  & \textbf{-3.64}                    & 0.32                  & \textbf{-3.92}                    & 0.28                  & \textbf{4.63}                    & 0.12                   & \textbf{5.03}                    & 0.15                   & \textbf{5.09}                    & 0.13                   & \textbf{8.9}                     & 0.24                   & \textbf{8.71}                    & 0.22                   & \textbf{8.66}                    & 0.2                    \\
\textbf{IRL}                                 & \textbf{-50.85}                   & 2.02                  & \textbf{-55.14}                   & 1.75                  & \textbf{-53.94}                   & 1.58                  & \textbf{3.93}                    & 0.7                    & \textbf{3.64}                    & 0.75                   & \textbf{2.47}                    & 0.69                   & \textbf{53.34}                   & 2                      & \textbf{54.85}                   & 2.05                   & \textbf{54.6}                    & 2.04                   \\
\textbf{IMN}                                 & \textbf{-7.91}                    & 0.44                  & \textbf{-8.56}                    & 0.75                  & \textbf{-7.61}                    & 0.25                  & \textbf{3.74}                    & 0.29                   & \textbf{3.61}                    & 0.3                    & \textbf{3.98}                    & 0.27                   & \textbf{10.1}                    & 0.34                   & \textbf{10.01}                   & 0.34                   & \textbf{10.28}                   & 0.34                   \\
\textbf{ITA}                                 & \textbf{-4.39}                    & 0.59                  & \textbf{-3.81}                    & 0.41                  & \textbf{-4.1}                     & 0.42                  & \textbf{3.84}                    & 0.12                   & \textbf{4.13}                    & 0.13                   & \textbf{3.8}                     & 0.12                   & \textbf{10.03}                   & 0.3                    & \textbf{10.33}                   & 0.34                   & \textbf{10.13}                   & 0.32                   \\
\textbf{JAM}                                 & \textbf{-6.94}                    & 0.26                  & \textbf{-7.41}                    & 0.58                  & \textbf{-7.18}                    & 0.29                  & \textbf{1.07}                    & 0.23                   & \textbf{1.1}                     & 0.23                   & \textbf{0.87}                    & 0.24                   & \textbf{8.61}                    & 0.27                   & \textbf{8.74}                    & 0.27                   & \textbf{8.7}                     & 0.28                   \\
\textbf{JPN}                                 & \textbf{-8.06}                    & 0.47                  & \textbf{-7.57}                    & 0.39                  & \textbf{-8.02}                    & 0.44                  & \textbf{1.22}                    & 0.16                   & \textbf{1.3}                     & 0.17                   & \textbf{1.03}                    & 0.16                   & \textbf{8.76}                    & 0.28                   & \textbf{8.53}                    & 0.26                   & \textbf{8.81}                    & 0.32                   \\
\textbf{JOR}                                 & \textbf{-7.75}                    & 0.93                  & \textbf{-6.64}                    & 0.36                  & \textbf{-7.01}                    & 0.36                  & \textbf{0.71}                    & 0.11                   & \textbf{1.48}                    & 0.11                   & \textbf{1.35}                    & 0.12                   & \textbf{8.38}                    & 0.29                   & \textbf{8.49}                    & 0.29                   & \textbf{8.63}                    & 0.29                   \\
\textbf{KAZ}                                 & \textbf{-4.13}                    & 0.33                  & \textbf{-4.11}                    & 0.31                  & \textbf{-3.97}                    & 0.3                   & \textbf{4.42}                    & 0.14                   & \textbf{4.17}                    & 0.11                   & \textbf{4.37}                    & 0.14                   & \textbf{10.5}                    & 0.39                   & \textbf{10.21}                   & 0.38                   & \textbf{10.5}                    & 0.36                   \\
\textbf{KEN}                                 & \textbf{-3.67}                    & 0.39                  & \textbf{-3.58}                    & 0.39                  & \textbf{-4.18}                    & 0.39                  & \textbf{7.07}                    & 0.21                   & \textbf{7.2}                     & 0.23                   & \textbf{6.74}                    & 0.19                   & \textbf{14.16}                   & 0.35                   & \textbf{14.29}                   & 0.36                   & \textbf{13.71}                   & 0.33                   \\
\textbf{KIR}                                 & \textbf{-4.58}                    & 0.57                  & \textbf{-4.52}                    & 0.53                  & \textbf{-3.84}                    & 0.36                  & \textbf{3.73}                    & 0.11                   & \textbf{3.54}                    & 0.14                   & \textbf{3.78}                    & 0.13                   & \textbf{8.79}                    & 0.22                   & \textbf{8.64}                    & 0.21                   & \textbf{9.06}                    & 0.24                   \\
\textbf{PRK}                                 & \textbf{-5.4}                     & 1.07                  & \textbf{-5.84}                    & 1.14                  & \textbf{-6.89}                    & 1.82                  & \textbf{1.37}                    & 0.1                    & \textbf{1.04}                    & 0.12                   & \textbf{0.85}                    & 0.14                   & \textbf{8.61}                    & 0.27                   & \textbf{8.34}                    & 0.26                   & \textbf{8.47}                    & 0.3                    \\
\textbf{KWT}                                 & \textbf{-4}                       & 0.32                  & \textbf{-3.68}                    & 0.37                  & \textbf{-4.76}                    & 0.68                  & \textbf{4.32}                    & 0.13                   & \textbf{4.53}                    & 0.13                   & \textbf{3.98}                    & 0.11                   & \textbf{9.03}                    & 0.29                   & \textbf{9.32}                    & 0.26                   & \textbf{9.09}                    & 0.3                    \\
\textbf{KGZ}                                 & \textbf{-9.69}                    & 0.7                   & \textbf{-9.7}                     & 1.02                  & \textbf{-9.71}                    & 0.71                  & \textbf{3.91}                    & 0.33                   & \textbf{3.99}                    & 0.29                   & \textbf{3.75}                    & 0.29                   & \textbf{16.97}                   & 0.57                   & \textbf{16.25}                   & 0.54                   & \textbf{16.19}                   & 0.51                   \\
\textbf{LAO}                                 & \textbf{-4.1}                     & 0.45                  & \textbf{-4.17}                    & 0.45                  & \textbf{-5.13}                    & 0.76                  & \textbf{4.22}                    & 0.21                   & \textbf{4.4}                     & 0.24                   & \textbf{4.01}                    & 0.25                   & \textbf{11.46}                   & 0.37                   & \textbf{11.5}                    & 0.34                   & \textbf{11.56}                   & 0.37                   \\
\textbf{LVA}                                 & \textbf{-3.39}                    & 0.35                  & \textbf{-4.45}                    & 0.69                  & \textbf{-4.56}                    & 0.54                  & \textbf{6.99}                    & 0.21                   & \textbf{6.84}                    & 0.22                   & \textbf{7.04}                    & 0.2                    & \textbf{11.18}                   & 0.31                   & \textbf{10.65}                   & 0.29                   & \textbf{10.41}                   & 0.28                   \\
\textbf{LBN}                                 & \textbf{-17.7}                    & 0.41                  & \textbf{-17.67}                   & 0.83                  & \textbf{-17.21}                   & 0.39                  & \textbf{5.3}                     & 0.6                    & \textbf{5.64}                    & 0.57                   & \textbf{5.64}                    & 0.58                   & \textbf{12.5}                    & 0.39                   & \textbf{12.5}                    & 0.32                   & \textbf{12.6}                    & 0.35                   \\
\textbf{LSO}                                 & \textbf{-5.57}                    & 0.44                  & \textbf{-5.33}                    & 0.78                  & \textbf{-4.73}                    & 0.5                   & \textbf{3.17}                    & 0.11                   & \textbf{3.22}                    & 0.13                   & \textbf{3.66}                    & 0.14                   & \textbf{11.85}                   & 0.44                   & \textbf{11.97}                   & 0.45                   & \textbf{12.57}                   & 0.47                   \\
\textbf{LBR}                                 & \textbf{-4.4}                     & 0.64                  & \textbf{-4.14}                    & 0.25                  & \textbf{-4.27}                    & 0.64                  & \textbf{3.58}                    & 0.11                   & \textbf{3.53}                    & 0.1                    & \textbf{3.59}                    & 0.1                    & \textbf{8.91}                    & 0.26                   & \textbf{9.1}                     & 0.27                   & \textbf{8.89}                    & 0.24                   \\
\textbf{LBY}                                 & \textbf{-44.41}                   & 1.66                  & \textbf{-44.71}                   & 1.56                  & \textbf{-44.38}                   & 1.62                  & \textbf{4.88}                    & 0.73                   & \textbf{4.64}                    & 0.72                   & \textbf{4.96}                    & 0.73                   & \textbf{11.79}                   & 0.38                   & \textbf{12.2}                    & 0.43                   & \textbf{12.18}                   & 0.39                   \\
\textbf{LIE}                                 & \textbf{-69.49}                   & 1.67                  & \textbf{-68.92}                   & 2.11                  & \textbf{-69.93}                   & 1.44                  & \textbf{2.25}                    & 0.63                   & \textbf{2.31}                    & 0.6                    & \textbf{1.88}                    & 0.66                   & \textbf{62.14}                   & 2.01                   & \textbf{60.75}                   & 2.36                   & \textbf{62.64}                   & 2                      \\
\textbf{LUX}                                 & \textbf{-16.44}                   & 0.54                  & \textbf{-17.61}                   & 0.46                  & \textbf{-19}                      & 1.45                  & \textbf{4.05}                    & 0.67                   & \textbf{3.97}                    & 0.7                    & \textbf{3.94}                    & 0.69                   & \textbf{11.74}                   & 0.28                   & \textbf{12.33}                   & 0.33                   & \textbf{12.02}                   & 0.3                    \\
\textbf{MAC}                                 & \textbf{-6.41}                    & 0.45                  & \textbf{-6.22}                    & 0.47                  & \textbf{-8.79}                    & 1.68                  & \textbf{2.63}                    & 0.32                   & \textbf{2.73}                    & 0.24                   & \textbf{2.69}                    & 0.17                   & \textbf{9.31}                    & 0.38                   & \textbf{9.82}                    & 0.4                    & \textbf{9.47}                    & 0.4                    \\
\textbf{MDG}                                 & \textbf{-4.59}                    & 0.41                  & \textbf{-4.43}                    & 0.3                   & \textbf{-4.99}                    & 0.54                  & \textbf{6.73}                    & 0.32                   & \textbf{6.86}                    & 0.34                   & \textbf{6.7}                     & 0.28                   & \textbf{27.49}                   & 0.88                   & \textbf{27.7}                    & 0.9                    & \textbf{27.56}                   & 0.91                   \\
\textbf{MWI}                                 & \textbf{-15.65}                   & 0.38                  & \textbf{-15.93}                   & 0.36                  & \textbf{-16.98}                   & 0.48                  & \textbf{3.01}                    & 0.53                   & \textbf{3.12}                    & 0.51                   & \textbf{3}                       & 0.52                   & \textbf{10.42}                   & 0.31                   & \textbf{10.36}                   & 0.32                   & \textbf{10.53}                   & 0.35                   \\
\textbf{MYS}                                 & \textbf{-7.28}                    & 0.38                  & \textbf{-7.09}                    & 0.42                  & \textbf{-7.28}                    & 0.42                  & \textbf{4.44}                    & 0.22                   & \textbf{4.43}                    & 0.21                   & \textbf{4.27}                    & 0.23                   & \textbf{10.16}                   & 0.33                   & \textbf{10.32}                   & 0.33                   & \textbf{10.58}                   & 0.31                   \\
\textbf{MDV}                                 & \textbf{-4.44}                    & 0.31                  & \textbf{-4.17}                    & 0.44                  & \textbf{-4.23}                    & 0.35                  & \textbf{4.93}                    & 0.2                    & \textbf{15.73}                   & 1.25                   & \textbf{4.71}                    & 0.16                   & \textbf{10.09}                   & 0.31                   & \textbf{10.91}                   & 0.45                   & \textbf{9.72}                    & 0.31                   \\
\textbf{MLI}                                 & \textbf{-18.33}                   & 0.76                  & \textbf{-17.17}                   & 0.75                  & \textbf{-17.46}                   & 0.74                  & \textbf{5.94}                    & 0.67                   & \textbf{6.19}                    & 0.63                   & \textbf{6.03}                    & 0.61                   & \textbf{27.08}                   & 0.9                    & \textbf{27.67}                   & 0.87                   & \textbf{27.15}                   & 0.83                   \\
\textbf{MLT}                                 & \textbf{-4.56}                    & 0.66                  & \textbf{-4.39}                    & 0.34                  & \textbf{-5.18}                    & 0.91                  & \textbf{4.18}                    & 0.13                   & \textbf{4.25}                    & 0.13                   & \textbf{4.09}                    & 0.14                   & \textbf{17.19}                   & 0.53                   & \textbf{17.05}                   & 0.56                   & \textbf{16.63}                   & 0.56                   \\
\textbf{MHL}                                 & \textbf{-5.08}                    & 0.51                  & \textbf{-4.72}                    & 0.52                  & \textbf{-4.33}                    & 0.49                  & \textbf{3.1}                     & 0.23                   & \textbf{3.24}                    & 0.21                   & \textbf{3.61}                    & 0.18                   & \textbf{18.88}                   & 0.67                   & \textbf{18.91}                   & 0.72                   & \textbf{18.68}                   & 0.63                   \\
\textbf{MRT}                                 & \textbf{-8.32}                    & 0.55                  & \textbf{-7.58}                    & 0.3                   & \textbf{-8.44}                    & 0.54                  & \textbf{0.73}                    & 0.23                   & \textbf{1.04}                    & 0.23                   & \textbf{0.64}                    & 0.21                   & \textbf{8.63}                    & 0.28                   & \textbf{8.53}                    & 0.26                   & \textbf{8.67}                    & 0.27                   \\
\textbf{MUS}                                 & \textbf{-6.88}                    & 0.39                  & \textbf{-6.93}                    & 0.42                  & \textbf{-6.75}                    & 0.44                  & \textbf{3.19}                    & 0.25                   & \textbf{3.1}                     & 0.26                   & \textbf{3.2}                     & 0.27                   & \textbf{18.83}                   & 0.77                   & \textbf{18.97}                   & 0.76                   & \textbf{19.18}                   & 0.77                   \\
\textbf{MEX}                                 & \textbf{-4.15}                    & 0.41                  & \textbf{-4.2}                     & 0.43                  & \textbf{-4.1}                     & 0.42                  & \textbf{3.81}                    & 0.1                    & \textbf{3.67}                    & 0.13                   & \textbf{3.89}                    & 0.09                   & \textbf{8.84}                    & 0.22                   & \textbf{8.55}                    & 0.23                   & \textbf{8.87}                    & 0.21                   \\
\textbf{FSM}                                 & \textbf{-7.09}                    & 0.39                  & \textbf{-6.44}                    & 0.28                  & \textbf{-6.35}                    & 0.34                  & \textbf{1.97}                    & 0.26                   & \textbf{2.02}                    & 0.26                   & \textbf{2.09}                    & 0.23                   & \textbf{8.59}                    & 0.28                   & \textbf{8.88}                    & 0.3                    & \textbf{8.58}                    & 0.26                   \\
\textbf{MDA}                                 & \textbf{-6.82}                    & 0.58                  & \textbf{-6.26}                    & 0.3                   & \textbf{-7.73}                    & 1.49                  & \textbf{1.1}                     & 0.14                   & \textbf{1.24}                    & 0.17                   & \textbf{0.97}                    & 0.15                   & \textbf{8.19}                    & 0.27                   & \textbf{8.39}                    & 0.27                   & \textbf{8.02}                    & 0.27                   \\
\textbf{MCO}                                 & \textbf{-8.61}                    & 0.97                  & \textbf{-7.93}                    & 0.87                  & \textbf{-7.7}                     & 0.46                  & \textbf{5.51}                    & 0.33                   & \textbf{5.83}                    & 0.3                    & \textbf{5.51}                    & 0.3                    & \textbf{9.71}                    & 0.24                   & \textbf{10.05}                   & 0.26                   & \textbf{9.61}                    & 0.28                   \\
\textbf{MNE}                                 & \textbf{-4.34}                    & 0.4                   & \textbf{-5.12}                    & 0.94                  & \textbf{-3.98}                    & 0.39                  & \textbf{6.74}                    & 0.27                   & \textbf{6.89}                    & 0.3                    & \textbf{6.9}                     & 0.25                   & \textbf{16.97}                   & 0.46                   & \textbf{17.34}                   & 0.47                   & \textbf{17.19}                   & 0.46                   \\
\textbf{MAR}                                 & \textbf{-8.77}                    & 0.44                  & \textbf{-8.23}                    & 0.44                  & \textbf{-9.83}                    & 1.54                  & \textbf{3}                       & 0.3                    & \textbf{3.19}                    & 0.26                   & \textbf{2.84}                    & 0.31                   & \textbf{9.47}                    & 0.31                   & \textbf{9.67}                    & 0.31                   & \textbf{9.39}                    & 0.35                   \\
\textbf{MOZ}                                 & \textbf{-4.96}                    & 0.95                  & \textbf{-4.42}                    & 0.38                  & \textbf{-5.13}                    & 0.96                  & \textbf{3.96}                    & 0.1                    & \textbf{3.9}                     & 0.1                    & \textbf{3.87}                    & 0.11                   & \textbf{10.35}                   & 0.34                   & \textbf{10.06}                   & 0.33                   & \textbf{10.22}                   & 0.35                   \\
\textbf{MMR}                                 & \textbf{-3.98}                    & 0.44                  & \textbf{-4.33}                    & 0.65                  & \textbf{-4.02}                    & 0.67                  & \textbf{6.98}                    & 0.22                   & \textbf{7.11}                    & 0.22                   & \textbf{7.01}                    & 0.2                    & \textbf{13.32}                   & 0.36                   & \textbf{13.2}                    & 0.38                   & \textbf{13.31}                   & 0.4                    \\
\textbf{NAM}                                 & \textbf{-3.88}                    & 0.27                  & \textbf{-4.77}                    & 0.8                   & \textbf{-3.92}                    & 0.32                  & \textbf{6.96}                    & 0.19                   & \textbf{6.94}                    & 0.17                   & \textbf{6.9}                     & 0.2                    & \textbf{16.8}                    & 0.38                   & \textbf{17.09}                   & 0.4                    & \textbf{17.14}                   & 0.4                    \\
\textbf{NRU}                                 & \textbf{-3.88}                    & 0.25                  & \textbf{-4.18}                    & 0.38                  & \textbf{-4.7}                     & 0.49                  & \textbf{4.25}                    & 0.14                   & \textbf{4.14}                    & 0.21                   & \textbf{4.19}                    & 0.12                   & \textbf{12.22}                   & 0.39                   & \textbf{12.44}                   & 0.42                   & \textbf{12.5}                    & 0.39                   \\
\textbf{NPL}                                 & \textbf{-33.64}                   & 1.98                  & \textbf{-33.09}                   & 1.86                  & \textbf{-35.07}                   & 2                     & \textbf{6.06}                    & 0.76                   & \textbf{6.32}                    & 0.77                   & \textbf{5.59}                    & 1.16                   & \textbf{29.78}                   & 0.87                   & \textbf{28.97}                   & 0.9                    & \textbf{29.36}                   & 0.89                   \\
\textbf{NLD}                                 & \textbf{-3.7}                     & 0.3                   & \textbf{-4.96}                    & 0.52                  & \textbf{-4.89}                    & 1.04                  & \textbf{3.98}                    & 0.11                   & \textbf{3.16}                    & 0.09                   & \textbf{3.83}                    & 0.11                   & \textbf{9.45}                    & 0.3                    & \textbf{8.7}                     & 0.31                   & \textbf{9.34}                    & 0.28                   \\
\textbf{NCL}                                 & \textbf{-6.14}                    & 0.46                  & \textbf{-6.2}                     & 0.45                  & \textbf{-6.05}                    & 0.43                  & \textbf{1.18}                    & 0.16                   & \textbf{1.21}                    & 0.13                   & \textbf{1.16}                    & 0.15                   & \textbf{8.57}                    & 0.3                    & \textbf{8.57}                    & 0.25                   & \textbf{8.56}                    & 0.23                   \\
\textbf{NIC}                                 & \textbf{-4.61}                    & 0.38                  & \textbf{-4.51}                    & 0.39                  & \textbf{-4.37}                    & 0.4                   & \textbf{2.54}                    & 0.21                   & \textbf{2.55}                    & 0.19                   & \textbf{2.69}                    & 0.16                   & \textbf{8.77}                    & 0.29                   & \textbf{8.4}                     & 0.22                   & \textbf{8.45}                    & 0.24                   \\
\textbf{NER}                                 & \textbf{-5.32}                    & 0.35                  & \textbf{-4.59}                    & 0.47                  & \textbf{-4.43}                    & 0.34                  & \textbf{3.68}                    & 0.16                   & \textbf{3.77}                    & 0.18                   & \textbf{3.82}                    & 0.19                   & \textbf{8.43}                    & 0.2                    & \textbf{8.71}                    & 0.21                   & \textbf{8.87}                    & 0.22                   \\
\textbf{NGA}                                 & \textbf{-4.46}                    & 0.27                  & \textbf{-4.28}                    & 0.43                  & \textbf{-4.29}                    & 0.41                  & \textbf{4.43}                    & 0.25                   & \textbf{4.54}                    & 0.23                   & \textbf{4.59}                    & 0.23                   & \textbf{10.84}                   & 0.33                   & \textbf{11.14}                   & 0.34                   & \textbf{11.15}                   & 0.33                   \\
\textbf{MKD}                                 & \textbf{-4.06}                    & 0.37                  & \textbf{-4.07}                    & 0.39                  & \textbf{-4.23}                    & 0.64                  & \textbf{6.53}                    & 0.19                   & \textbf{6.34}                    & 0.19                   & \textbf{6.6}                     & 0.2                    & \textbf{17.18}                   & 0.56                   & \textbf{17.77}                   & 0.57                   & \textbf{17.71}                   & 0.59                   \\
\textbf{MNP}                                 & \textbf{-6.08}                    & 0.54                  & \textbf{-5.89}                    & 0.43                  & \textbf{-5.85}                    & 0.45                  & \textbf{3.31}                    & 0.18                   & \textbf{3.21}                    & 0.23                   & \textbf{3.22}                    & 0.24                   & \textbf{9.62}                    & 0.3                    & \textbf{10.28}                   & 0.39                   & \textbf{9.55}                    & 0.32                   \\
\textbf{OMN}                                 & \textbf{-3.87}                    & 0.48                  & \textbf{-3.67}                    & 0.45                  & \textbf{-3.86}                    & 0.43                  & \textbf{1.34}                    & 0.13                   & \textbf{1.51}                    & 0.12                   & \textbf{1.27}                    & 0.13                   & \textbf{8.28}                    & 0.25                   & \textbf{8.52}                    & 0.26                   & \textbf{8.14}                    & 0.25                   \\
\textbf{PAK}                                 & \textbf{-4.83}                    & 0.46                  & \textbf{-4.42}                    & 0.44                  & \textbf{-6.33}                    & 1.61                  & \textbf{3.56}                    & 0.2                    & \textbf{3.71}                    & 0.2                    & \textbf{3.37}                    & 0.2                    & \textbf{9.95}                    & 0.32                   & \textbf{9.92}                    & 0.32                   & \textbf{9.81}                    & 0.32                   \\
\textbf{PLW}                                 & \textbf{-4.94}                    & 0.53                  & \textbf{-4.52}                    & 0.45                  & \textbf{-5.09}                    & 0.79                  & \textbf{3.82}                    & 0.11                   & \textbf{3.92}                    & 0.11                   & \textbf{3.64}                    & 0.12                   & \textbf{9.47}                    & 0.29                   & \textbf{9.1}                     & 0.27                   & \textbf{9.39}                    & 0.31                   \\
\textbf{PAN}                                 & \textbf{-8.94}                    & 0.37                  & \textbf{-9.52}                    & 0.73                  & \textbf{-9.88}                    & 0.54                  & \textbf{1.14}                    & 0.32                   & \textbf{0.78}                    & 0.29                   & \textbf{0.95}                    & 0.31                   & \textbf{8.44}                    & 0.25                   & \textbf{8.89}                    & 0.3                    & \textbf{8.47}                    & 0.25                   \\
\textbf{PNG}                                 & \textbf{-4.5}                     & 0.79                  & \textbf{-4.11}                    & 0.41                  & \textbf{-3.89}                    & 0.29                  & \textbf{6.45}                    & 0.21                   & \textbf{6.42}                    & 0.2                    & \textbf{6.26}                    & 0.22                   & \textbf{12.89}                   & 0.37                   & \textbf{12.89}                   & 0.36                   & \textbf{12.81}                   & 0.32                   \\
\textbf{PRY}                                 & \textbf{-5.48}                    & 0.6                   & \textbf{-5.74}                    & 0.48                  & \textbf{-4.95}                    & 0.4                   & \textbf{3.53}                    & 0.23                   & \textbf{3.34}                    & 0.24                   & \textbf{3.51}                    & 0.2                    & \textbf{13.5}                    & 0.39                   & \textbf{13.26}                   & 0.42                   & \textbf{13.71}                   & 0.43                   \\
\textbf{PER}                                 & \textbf{-4.47}                    & 0.4                   & \textbf{-4.81}                    & 0.53                  & \textbf{-4.8}                     & 0.44                  & \textbf{3.75}                    & 0.2                    & \textbf{3.51}                    & 0.21                   & \textbf{3.56}                    & 0.27                   & \textbf{11.11}                   & 0.34                   & \textbf{11.19}                   & 0.34                   & \textbf{11.29}                   & 0.35                   \\
\textbf{PHL}                                 & \textbf{-4.54}                    & 0.46                  & \textbf{-4.71}                    & 0.63                  & \textbf{-4.56}                    & 0.42                  & \textbf{5.02}                    & 0.15                   & \textbf{5.06}                    & 0.2                    & \textbf{4.75}                    & 0.22                   & \textbf{9.64}                    & 0.26                   & \textbf{9.53}                    & 0.28                   & \textbf{9.64}                    & 0.29                   \\
\textbf{POL}                                 & \textbf{-4.11}                    & 0.37                  & \textbf{-5.54}                    & 0.6                   & \textbf{-5.31}                    & 0.69                  & \textbf{4.75}                    & 0.17                   & \textbf{4.17}                    & 0.14                   & \textbf{4.67}                    & 0.16                   & \textbf{8.66}                    & 0.22                   & \textbf{7.96}                    & 0.19                   & \textbf{8.69}                    & 0.2                    \\
\textbf{PRT}                                 & \textbf{-3.89}                    & 0.36                  & \textbf{-3.99}                    & 0.28                  & \textbf{-5.27}                    & 0.97                  & \textbf{3.66}                    & 0.1                    & \textbf{3.49}                    & 0.09                   & \textbf{3.44}                    & 0.09                   & \textbf{9.61}                    & 0.32                   & \textbf{9.53}                    & 0.32                   & \textbf{9.28}                    & 0.31                   \\
\textbf{PRI}                                 & \textbf{-5.64}                    & 0.41                  & \textbf{-5.97}                    & 0.37                  & \textbf{-5.49}                    & 0.29                  & \textbf{1.19}                    & 0.2                    & \textbf{1.29}                    & 0.15                   & \textbf{1.22}                    & 0.14                   & \textbf{8.81}                    & 0.29                   & \textbf{8.69}                    & 0.29                   & \textbf{8.58}                    & 0.28                   \\
\textbf{QAT}                                 & \textbf{-5.78}                    & 0.31                  & \textbf{-5.6}                     & 0.42                  & \textbf{-5.81}                    & 0.39                  & \textbf{0.84}                    & 0.17                   & \textbf{1.43}                    & 0.14                   & \textbf{1.28}                    & 0.15                   & \textbf{10.01}                   & 0.36                   & \textbf{10.11}                   & 0.37                   & \textbf{10.01}                   & 0.36                   \\
\textbf{ROU}                                 & \textbf{-3.61}                    & 0.31                  & \textbf{-5.07}                    & 0.95                  & \textbf{-3.77}                    & 0.48                  & \textbf{7.18}                    & 0.2                    & \textbf{6.77}                    & 0.21                   & \textbf{6.86}                    & 0.19                   & \textbf{26.96}                   & 0.76                   & \textbf{27.09}                   & 0.76                   & \textbf{26.71}                   & 0.76                   \\
\textbf{RUS}                                 & \textbf{-6.81}                    & 0.34                  & \textbf{-6.62}                    & 0.24                  & \textbf{-6.77}                    & 0.28                  & \textbf{3.47}                    & 0.3                    & \textbf{3.66}                    & 0.29                   & \textbf{3.66}                    & 0.29                   & \textbf{11.3}                    & 0.35                   & \textbf{11.08}                   & 0.35                   & \textbf{11.08}                   & 0.35                   \\
\textbf{RWA}                                 & \textbf{-10.42}                   & 0.47                  & \textbf{-11}                      & 0.99                  & \textbf{-11.16}                   & 0.97                  & \textbf{4.28}                    & 0.39                   & \textbf{4.29}                    & 0.35                   & \textbf{4.37}                    & 0.36                   & \textbf{11.47}                   & 0.39                   & \textbf{11.16}                   & 0.36                   & \textbf{10.99}                   & 0.35                   \\
\textbf{WSM}                                 & \textbf{-4.2}                     & 0.46                  & \textbf{-5.01}                    & 0.99                  & \textbf{-4.21}                    & 0.37                  & \textbf{6.81}                    & 0.26                   & \textbf{6.9}                     & 0.24                   & \textbf{7.06}                    & 0.23                   & \textbf{14.07}                   & 0.45                   & \textbf{13.83}                   & 0.36                   & \textbf{14.15}                   & 0.41                   \\
\textbf{SMR}                                 & \textbf{-6.71}                    & 0.52                  & \textbf{-6.1}                     & 0.45                  & \textbf{-5.77}                    & 0.46                  & \textbf{2.8}                     & 0.22                   & \textbf{2.76}                    & 0.19                   & \textbf{2.89}                    & 0.25                   & \textbf{7.94}                    & 0.18                   & \textbf{8.35}                    & 0.18                   & \textbf{8.54}                    & 0.22                   \\
\textbf{STP}                                 & \textbf{-15.08}                   & 0.48                  & \textbf{-14.81}                   & 0.54                  & \textbf{-15.58}                   & 0.89                  & \textbf{0.75}                    & 0.52                   & \textbf{0.69}                    & 0.43                   & \textbf{0.47}                    & 0.46                   & \textbf{8.93}                    & 0.3                    & \textbf{8.93}                    & 0.35                   & \textbf{9.03}                    & 0.31                   \\
\textbf{SAU}                                 & \textbf{-4.04}                    & 0.38                  & \textbf{-3.95}                    & 0.36                  & \textbf{-4.04}                    & 0.36                  & \textbf{4.14}                    & 0.11                   & \textbf{4.23}                    & 0.14                   & \textbf{4.24}                    & 0.12                   & \textbf{9.85}                    & 0.32                   & \textbf{10}                      & 0.34                   & \textbf{9.91}                    & 0.33                   \\
\textbf{SEN}                                 & \textbf{-5.4}                     & 0.36                  & \textbf{-5.62}                    & 0.36                  & \textbf{-6.15}                    & 0.78                  & \textbf{3.92}                    & 0.22                   & \textbf{3.68}                    & 0.21                   & \textbf{3.38}                    & 0.23                   & \textbf{14.25}                   & 0.48                   & \textbf{14.33}                   & 0.49                   & \textbf{14.49}                   & 0.54                   \\
\textbf{SRB}                                 & \textbf{-3.97}                    & 0.44                  & \textbf{-3.91}                    & 0.4                   & \textbf{-3.97}                    & 0.42                  & \textbf{3.63}                    & 0.12                   & \textbf{3.66}                    & 0.12                   & \textbf{3.44}                    & 0.12                   & \textbf{9.12}                    & 0.28                   & \textbf{9.12}                    & 0.32                   & \textbf{9.05}                    & 0.28                   \\
\textbf{SYC}                                 & \textbf{-5.83}                    & 0.97                  & \textbf{-5.95}                    & 0.85                  & \textbf{-5.53}                    & 0.53                  & \textbf{4.1}                     & 0.25                   & \textbf{3.95}                    & 0.25                   & \textbf{3.95}                    & 0.22                   & \textbf{9.15}                    & 0.22                   & \textbf{9.34}                    & 0.23                   & \textbf{9.17}                    & 0.24                   \\
\textbf{SLE}                                 & \textbf{-8.45}                    & 0.38                  & \textbf{-8.11}                    & 0.37                  & \textbf{-8.01}                    & 0.44                  & \textbf{1.32}                    & 0.2                    & \textbf{1.28}                    & 0.18                   & \textbf{1.53}                    & 0.19                   & \textbf{11.21}                   & 0.42                   & \textbf{11.62}                   & 0.49                   & \textbf{11.6}                    & 0.43                   \\
\textbf{SGP}                                 & \textbf{-8.27}                    & 0.5                   & \textbf{-7.81}                    & 0.63                  & \textbf{-8.3}                     & 0.95                  & \textbf{5.73}                    & 0.26                   & \textbf{5.71}                    & 0.25                   & \textbf{5.64}                    & 0.28                   & \textbf{27.13}                   & 0.9                    & \textbf{27.78}                   & 0.88                   & \textbf{26.81}                   & 0.89                   \\
\textbf{SXM}                                 & \textbf{-4.71}                    & 0.65                  & \textbf{-5.32}                    & 1.04                  & \textbf{-4.23}                    & 0.32                  & \textbf{4.79}                    & 0.28                   & \textbf{4.65}                    & 0.22                   & \textbf{4.8}                     & 0.22                   & \textbf{14.83}                   & 0.44                   & \textbf{14.53}                   & 0.42                   & \textbf{14.48}                   & 0.42                   \\
\textbf{SVK}                                 & \textbf{-3.9}                     & 0.34                  & \textbf{-4.94}                    & 0.44                  & \textbf{-4.59}                    & 0.53                  & \textbf{1.46}                    & 0.06                   & \textbf{1.01}                    & 0.07                   & \textbf{1.4}                     & 0.07                   & \textbf{8.26}                    & 0.26                   & \textbf{8.27}                    & 0.26                   & \textbf{8.55}                    & 0.27                   \\
\textbf{SVN}                                 & \textbf{-8.16}                    & 1.51                  & \textbf{-7.44}                    & 0.54                  & \textbf{-8.46}                    & 1.53                  & \textbf{3.68}                    & 0.26                   & \textbf{3.88}                    & 0.25                   & \textbf{3.14}                    & 0.26                   & \textbf{11.19}                   & 0.35                   & \textbf{11.27}                   & 0.33                   & \textbf{11.14}                   & 0.38                   \\
\textbf{SLB}                                 & \textbf{-9.58}                    & 0.25                  & \textbf{-9.72}                    & 0.28                  & \textbf{-9.31}                    & 0.35                  & \textbf{2.51}                    & 0.33                   & \textbf{2.62}                    & 0.31                   & \textbf{3.02}                    & 0.32                   & \textbf{8.76}                    & 0.27                   & \textbf{8.48}                    & 0.22                   & \textbf{8.76}                    & 0.23                   \\
\textbf{SOM}                                 & \textbf{-17.67}                   & 0.48                  & \textbf{-17.82}                   & 0.46                  & \textbf{-18.08}                   & 0.54                  & \textbf{3.45}                    & 0.67                   & \textbf{3.71}                    & 0.66                   & \textbf{3.14}                    & 0.64                   & \textbf{11.14}                   & 0.35                   & \textbf{11.49}                   & 0.37                   & \textbf{10.96}                   & 0.37                   \\
\textbf{ZAF}                                 & \textbf{-3.96}                    & 0.28                  & \textbf{-4.64}                    & 0.67                  & \textbf{-4.32}                    & 0.45                  & \textbf{6.93}                    & 0.26                   & \textbf{6.93}                    & 0.24                   & \textbf{6.87}                    & 0.21                   & \textbf{9.17}                    & 0.2                    & \textbf{9.08}                    & 0.18                   & \textbf{9.21}                    & 0.2                    \\
\textbf{SSD}                                 & \textbf{-4.38}                    & 0.43                  & \textbf{-4.33}                    & 0.66                  & \textbf{-4.32}                    & 0.43                  & \textbf{3.25}                    & 0.18                   & \textbf{3.43}                    & 0.14                   & \textbf{3.32}                    & 0.14                   & \textbf{8.59}                    & 0.23                   & \textbf{8.83}                    & 0.25                   & \textbf{8.3}                     & 0.22                   \\
\textbf{LKA}                                 & \textbf{-6.37}                    & 0.42                  & \textbf{-6.47}                    & 0.28                  & \textbf{-6.8}                     & 0.51                  & \textbf{1.79}                    & 0.16                   & \textbf{1.51}                    & 0.22                   & \textbf{1.69}                    & 0.16                   & \textbf{8.59}                    & 0.25                   & \textbf{8.56}                    & 0.26                   & \textbf{8.6}                     & 0.24                   \\
\textbf{KNA}                                 & \textbf{-4.07}                    & 0.42                  & \textbf{-5.09}                    & 1.61                  & \textbf{-4.09}                    & 0.38                  & \textbf{5.55}                    & 0.23                   & \textbf{5.24}                    & 0.22                   & \textbf{5.44}                    & 0.22                   & \textbf{11.87}                   & 0.37                   & \textbf{11.54}                   & 0.32                   & \textbf{11.59}                   & 0.3                    \\
\textbf{LCA}                                 & \textbf{-6.27}                    & 0.39                  & \textbf{-6.33}                    & 0.55                  & \textbf{-5.86}                    & 0.36                  & \textbf{2.84}                    & 0.21                   & \textbf{2.83}                    & 0.22                   & \textbf{2.85}                    & 0.24                   & \textbf{12.53}                   & 0.43                   & \textbf{12.33}                   & 0.41                   & \textbf{12.63}                   & 0.44                   \\
\textbf{MAF}                                 & \textbf{-5.89}                    & 0.64                  & \textbf{-6.62}                    & 1.02                  & \textbf{-5.53}                    & 0.33                  & \textbf{1.16}                    & 0.26                   & \textbf{1.01}                    & 0.18                   & \textbf{1.22}                    & 0.18                   & \textbf{8.73}                    & 0.3                    & \textbf{8.47}                    & 0.27                   & \textbf{8.61}                    & 0.28                   \\
\textbf{SDN}                                 & \textbf{-7.08}                    & 0.39                  & \textbf{-6.62}                    & 0.38                  & \textbf{-6.74}                    & 0.25                  & \textbf{1.17}                    & 0.16                   & \textbf{1.28}                    & 0.16                   & \textbf{1.31}                    & 0.17                   & \textbf{8.48}                    & 0.24                   & \textbf{8.6}                     & 0.26                   & \textbf{8.65}                    & 0.25                   \\
\textbf{SUR}                                 & \textbf{-18.87}                   & 0.68                  & \textbf{-19.52}                   & 0.93                  & \textbf{-19.23}                   & 0.78                  & \textbf{3.67}                    & 0.73                   & \textbf{3.68}                    & 0.69                   & \textbf{3.66}                    & 0.72                   & \textbf{9.75}                    & 0.32                   & \textbf{9.7}                     & 0.3                    & \textbf{9.41}                    & 0.31                   \\
\textbf{SWE}                                 & \textbf{-3.62}                    & 0.28                  & \textbf{-3.79}                    & 0.35                  & \textbf{-4.59}                    & 0.53                  & \textbf{4.1}                     & 0.16                   & \textbf{3.92}                    & 0.13                   & \textbf{3.98}                    & 0.14                   & \textbf{10.1}                    & 0.31                   & \textbf{9.35}                    & 0.29                   & \textbf{9.67}                    & 0.32                   \\
\textbf{CHE}                                 & \textbf{-8.35}                    & 0.56                  & \textbf{-8.42}                    & 0.5                   & \textbf{-9.55}                    & 1.57                  & \textbf{2.17}                    & 0.19                   & \textbf{2.26}                    & 0.22                   & \textbf{1.57}                    & 0.21                   & \textbf{8.61}                    & 0.24                   & \textbf{8.53}                    & 0.23                   & \textbf{8.41}                    & 0.23                   \\
\textbf{SYR}                                 & \textbf{-5.03}                    & 0.46                  & \textbf{-3.95}                    & 0.31                  & \textbf{-3.92}                    & 0.24                  & \textbf{0.9}                     & 0.19                   & \textbf{2.09}                    & 0.16                   & \textbf{2.22}                    & 0.18                   & \textbf{8.32}                    & 0.25                   & \textbf{8.54}                    & 0.26                   & \textbf{8.75}                    & 0.26                   \\
\textbf{TZA}                                 & \textbf{-3.83}                    & 0.34                  & \textbf{-3.96}                    & 0.64                  & \textbf{-4.18}                    & 0.41                  & \textbf{7.2}                     & 0.2                    & \textbf{7.05}                    & 0.2                    & \textbf{6.93}                    & 0.2                    & \textbf{13.14}                   & 0.33                   & \textbf{12.91}                   & 0.29                   & \textbf{12.91}                   & 0.35                   \\
\textbf{THA}                                 & \textbf{-4.14}                    & 0.45                  & \textbf{-4.45}                    & 0.65                  & \textbf{-5.04}                    & 0.76                  & \textbf{6.45}                    & 0.2                    & \textbf{6.4}                     & 0.22                   & \textbf{6.3}                     & 0.21                   & \textbf{10.99}                   & 0.31                   & \textbf{11.08}                   & 0.37                   & \textbf{11}                      & 0.3                    \\
\textbf{TLS}                                 & \textbf{-4.21}                    & 0.52                  & \textbf{-4.3}                     & 0.37                  & \textbf{-4.26}                    & 0.48                  & \textbf{3.93}                    & 0.12                   & \textbf{3.94}                    & 0.14                   & \textbf{3.44}                    & 0.12                   & \textbf{10.81}                   & 0.41                   & \textbf{10.59}                   & 0.36                   & \textbf{10.33}                   & 0.36                   \\
\textbf{TGO}                                 & \textbf{-8.08}                    & 0.42                  & \textbf{-8.08}                    & 0.39                  & \textbf{-8.1}                     & 0.5                   & \textbf{3.91}                    & 0.33                   & \textbf{3.92}                    & 0.35                   & \textbf{3.96}                    & 0.36                   & \textbf{16.66}                   & 0.48                   & \textbf{16.32}                   & 0.48                   & \textbf{16.51}                   & 0.5                    \\
\textbf{TON}                                 & \textbf{-6.33}                    & 0.4                   & \textbf{-6.21}                    & 0.36                  & \textbf{-7.7}                     & 1.54                  & \textbf{3.92}                    & 0.27                   & \textbf{3.87}                    & 0.27                   & \textbf{3.64}                    & 0.25                   & \textbf{9.12}                    & 0.24                   & \textbf{8.75}                    & 0.23                   & \textbf{8.92}                    & 0.22                   \\
\textbf{TTO}                                 & \textbf{-7.68}                    & 0.35                  & \textbf{-7.16}                    & 0.36                  & \textbf{-7.68}                    & 0.37                  & \textbf{1.12}                    & 0.21                   & \textbf{1.15}                    & 0.17                   & \textbf{0.97}                    & 0.18                   & \textbf{8.79}                    & 0.29                   & \textbf{8.82}                    & 0.28                   & \textbf{8.53}                    & 0.3                    \\
\textbf{TUN}                                 & \textbf{-6.37}                    & 0.62                  & \textbf{-6.01}                    & 0.43                  & \textbf{-6.3}                     & 0.29                  & \textbf{3.85}                    & 0.24                   & \textbf{3.83}                    & 0.25                   & \textbf{3.7}                     & 0.24                   & \textbf{15.02}                   & 0.48                   & \textbf{15.01}                   & 0.48                   & \textbf{15.11}                   & 0.49                   \\
\textbf{TUR}                                 & \textbf{-4.86}                    & 0.59                  & \textbf{-4.23}                    & 0.41                  & \textbf{-4.28}                    & 0.46                  & \textbf{3.49}                    & 0.18                   & \textbf{3.67}                    & 0.17                   & \textbf{3.57}                    & 0.16                   & \textbf{8.58}                    & 0.32                   & \textbf{9.11}                    & 0.28                   & \textbf{8.99}                    & 0.28                   \\
\textbf{TKM}                                 & \textbf{-8.35}                    & 0.47                  & \textbf{-8.1}                     & 0.43                  & \textbf{-8.28}                    & 0.43                  & \textbf{3.93}                    & 0.44                   & \textbf{5.91}                    & 0.39                   & \textbf{5.94}                    & 0.39                   & \textbf{12.36}                   & 0.33                   & \textbf{12.29}                   & 0.35                   & \textbf{12.33}                   & 0.37                   \\
\textbf{TCA}                                 & \textbf{-4.42}                    & 0.64                  & \textbf{-3.65}                    & 0.32                  & \textbf{-3.88}                    & 0.31                  & \textbf{7.06}                    & 0.22                   & \textbf{7.16}                    & 0.21                   & \textbf{7.19}                    & 0.26                   & \textbf{16.21}                   & 0.41                   & \textbf{15.86}                   & 0.42                   & \textbf{16.07}                   & 0.43                   \\
\textbf{TUV}                                 & \textbf{-5.01}                    & 0.37                  & \textbf{-4.83}                    & 0.35                  & \textbf{-4.79}                    & 0.4                   & \textbf{1.15}                    & 0.19                   & \textbf{1.07}                    & 0.16                   & \textbf{1.25}                    & 0.18                   & \textbf{9.39}                    & 0.39                   & \textbf{9.48}                    & 0.4                    & \textbf{9.05}                    & 0.33                   \\
\textbf{UGA}                                 & \textbf{-9.37}                    & 0.9                   & \textbf{-9.76}                    & 0.91                  & \textbf{-9.62}                    & 0.88                  & \textbf{1.19}                    & 0.23                   & \textbf{1.16}                    & 0.27                   & \textbf{1.23}                    & 0.19                   & \textbf{11.85}                   & 0.49                   & \textbf{11.61}                   & 0.46                   & \textbf{11.29}                   & 0.45                   \\
\textbf{UKR}                                 & \textbf{-4.06}                    & 0.43                  & \textbf{-4.18}                    & 0.4                   & \textbf{-3.53}                    & 0.42                  & \textbf{6.26}                    & 0.21                   & \textbf{6.51}                    & 0.23                   & \textbf{6.47}                    & 0.21                   & \textbf{12}                      & 0.34                   & \textbf{12.02}                   & 0.32                   & \textbf{11.75}                   & 0.31                   \\
\textbf{ARE}                                 & \textbf{-17.82}                   & 0.32                  & \textbf{-17.79}                   & 0.34                  & \textbf{-17.53}                   & 0.35                  & \textbf{3.65}                    & 0.68                   & \textbf{3.4}                     & 0.66                   & \textbf{3.59}                    & 0.68                   & \textbf{12.28}                   & 0.36                   & \textbf{12.21}                   & 0.35                   & \textbf{12.21}                   & 0.34                   \\
\textbf{GBR}                                 & \textbf{-6.25}                    & 0.29                  & \textbf{-6.77}                    & 0.37                  & \textbf{-7.48}                    & 0.98                  & \textbf{3.87}                    & 0.26                   & \textbf{3.92}                    & 0.23                   & \textbf{4.02}                    & 0.27                   & \textbf{12.92}                   & 0.39                   & \textbf{12.44}                   & 0.38                   & \textbf{13.01}                   & 0.4                    \\
\textbf{USA}                                 & \textbf{-6}                       & 0.27                  & \textbf{-5.57}                    & 0.3                   & \textbf{-5.64}                    & 0.26                  & \textbf{1.33}                    & 0.21                   & \textbf{1.68}                    & 0.22                   & \textbf{1.56}                    & 0.19                   & \textbf{8.41}                    & 0.25                   & \textbf{8.82}                    & 0.26                   & \textbf{8.67}                    & 0.25                   \\
\textbf{URY}                                 & \textbf{-4.65}                    & 0.41                  & \textbf{-6.6}                     & 1.57                  & \textbf{-4.54}                    & 0.35                  & \textbf{1.76}                    & 0.13                   & \textbf{1.56}                    & 0.12                   & \textbf{1.64}                    & 0.11                   & \textbf{8.52}                    & 0.25                   & \textbf{8.38}                    & 0.26                   & \textbf{8.59}                    & 0.24                   \\
\textbf{UZB}                                 & \textbf{-10.04}                   & 0.98                  & \textbf{-9.13}                    & 0.49                  & \textbf{-10.33}                   & 0.97                  & \textbf{3.62}                    & 0.28                   & \textbf{3.66}                    & 0.28                   & \textbf{3.63}                    & 0.27                   & \textbf{10.69}                   & 0.35                   & \textbf{10.23}                   & 0.34                   & \textbf{10.59}                   & 0.34                   \\
\textbf{VUT}                                 & \textbf{-4.26}                    & 0.41                  & \textbf{-4.42}                    & 0.42                  & \textbf{-4}                       & 0.41                  & \textbf{6.85}                    & 0.22                   & \textbf{6.9}                     & 0.2                    & \textbf{6.86}                    & 0.2                    & \textbf{11.23}                   & 0.32                   & \textbf{11.3}                    & 0.34                   & \textbf{11.12}                   & 0.29                   \\
\textbf{VEN}                                 & \textbf{-6.84}                    & 0.38                  & \textbf{-6.94}                    & 0.4                   & \textbf{-7.85}                    & 1.02                  & \textbf{2.99}                    & 0.29                   & \textbf{2.82}                    & 0.32                   & \textbf{2.74}                    & 0.3                    & \textbf{9.09}                    & 0.26                   & \textbf{9.1}                     & 0.26                   & \textbf{8.95}                    & 0.28                   \\
\textbf{VIR}                                 & \textbf{-4.11}                    & 0.42                  & \textbf{-4.24}                    & 0.39                  & \textbf{-4.08}                    & 0.44                  & \textbf{6.31}                    & 0.25                   & \textbf{6.31}                    & 0.22                   & \textbf{6.44}                    & 0.19                   & \textbf{9.27}                    & 0.2                    & \textbf{8.9}                     & 0.2                    & \textbf{9.22}                    & 0.2                    \\
\textbf{YEM}                                 & \textbf{-17.35}                   & 0.94                  & \textbf{-15.73}                   & 0.39                  & \textbf{-16.24}                   & 0.5                   & \textbf{4.31}                    & 0.68                   & \textbf{4.34}                    & 0.69                   & \textbf{4.02}                    & 0.68                   & \textbf{21.59}                   & 0.75                   & \textbf{21.45}                   & 0.73                   & \textbf{21.71}                   & 0.74                   \\
\textbf{ZWE}                                 & \textbf{-4.14}                    & 0.39                  & \textbf{-4.54}                    & 0.52                  & \textbf{-4.64}                    & 0.57                  & \textbf{6.7}                     & 0.2                    & \textbf{6.7}                     & 0.21                   & \textbf{6.66}                    & 0.2                    & \textbf{11.25}                   & 0.3                    & \textbf{11.11}                   & 0.3                    & \textbf{11.1}                    & 0.3                    \\ \hline
\caption{Mean (in bold) and standard error of the quantile estimates at level $\tau=0.01, 0.5, 0.99$ for the SSP1 scenario.}
\label{tab:se_table_SSP1}\\
\end{longtable}
\end{landscape}

% Please add the following required packages to your document preamble:
% \usepackage[table,xcdraw]{xcolor}
% Beamer presentation requires \usepackage{colortbl} instead of \usepackage[table,xcdraw]{xcolor}
% \usepackage{lscape}
% \usepackage{longtable}
% Note: It may be necessary to compile the document several times to get a multi-page table to line up properly
\begin{landscape}
\footnotesize
\begin{longtable}[c]{
>{\columncolor[HTML]{D4D4D4}}l llllllllllllllllll}
\hline
\cellcolor[HTML]{B0B3B2}                     & \multicolumn{6}{c}{\cellcolor[HTML]{B0B3B2}\textbf{0.01}}                                                                                                                         & \multicolumn{6}{c}{\cellcolor[HTML]{B0B3B2}\textbf{0.5}}                                                                                                                          & \multicolumn{6}{c}{\cellcolor[HTML]{B0B3B2}\textbf{0.99}}                                                                                                                         \\ \hline
\endfirsthead
%
\multicolumn{19}{c}%
{{\bfseries Table \thetable\ continued from previous page}} \\
\hline
\cellcolor[HTML]{B0B3B2}                     & \multicolumn{6}{c}{\cellcolor[HTML]{B0B3B2}\textbf{0.01}}                                                                                                                         & \multicolumn{6}{c}{\cellcolor[HTML]{B0B3B2}\textbf{0.5}}                                                                                                                          & \multicolumn{6}{c}{\cellcolor[HTML]{B0B3B2}\textbf{0.99}}                                                                                                                         \\ \hline
\endhead
%
\hline
\endfoot
%
\endlastfoot
%
\multicolumn{1}{c}{\cellcolor[HTML]{B0B3B2}} & \multicolumn{2}{c}{\cellcolor[HTML]{B0B3B2}\textbf{2030}} & \multicolumn{2}{c}{\cellcolor[HTML]{B0B3B2}\textbf{2050}} & \multicolumn{2}{c}{\cellcolor[HTML]{B0B3B2}\textbf{2100}} & \multicolumn{2}{c}{\cellcolor[HTML]{B0B3B2}\textbf{2030}} & \multicolumn{2}{c}{\cellcolor[HTML]{B0B3B2}\textbf{2050}} & \multicolumn{2}{c}{\cellcolor[HTML]{B0B3B2}\textbf{2100}} & \multicolumn{2}{c}{\cellcolor[HTML]{B0B3B2}\textbf{2030}} & \multicolumn{2}{c}{\cellcolor[HTML]{B0B3B2}\textbf{2050}} & \multicolumn{2}{c}{\cellcolor[HTML]{B0B3B2}\textbf{2100}} \\
\textbf{AFG}                                 & \textbf{-4.54}                    & 0.46                  & \textbf{-4.73}                    & 0.35                  & \textbf{-4.28}                    & 0.36                  & \textbf{6.49}                    & 0.23                   & \textbf{6.26}                    & 0.24                   & \textbf{6.27}                    & 0.24                   & \textbf{21.11}                   & 0.7                    & \textbf{21.76}                   & 0.77                   & \textbf{21.21}                   & 0.76                   \\
\textbf{ALB}                                 & \textbf{-3.96}                    & 0.4                   & \textbf{-4.42}                    & 0.47                  & \textbf{-5.01}                    & 1                     & \textbf{4.41}                    & 0.15                   & \textbf{4.2}                     & 0.17                   & \textbf{4.4}                     & 0.16                   & \textbf{9.68}                    & 0.33                   & \textbf{9.57}                    & 0.32                   & \textbf{9.49}                    & 0.33                   \\
\textbf{DZA}                                 & \textbf{-4.17}                    & 0.26                  & \textbf{-3.71}                    & 0.26                  & \textbf{-3.7}                     & 0.27                  & \textbf{3.45}                    & 0.09                   & \textbf{3.65}                    & 0.11                   & \textbf{3.37}                    & 0.08                   & \textbf{8.65}                    & 0.22                   & \textbf{8.88}                    & 0.24                   & \textbf{8.73}                    & 0.25                   \\
\textbf{AGO}                                 & \textbf{-3.58}                    & 0.25                  & \textbf{-4.07}                    & 0.29                  & \textbf{-3.8}                     & 0.3                   & \textbf{5.43}                    & 0.21                   & \textbf{5.2}                     & 0.24                   & \textbf{5.31}                    & 0.21                   & \textbf{15.96}                   & 0.44                   & \textbf{15.94}                   & 0.48                   & \textbf{16.07}                   & 0.48                   \\
\textbf{ATG}                                 & \textbf{-16.87}                   & 0.9                   & \textbf{-17.21}                   & 0.86                  & \textbf{-17.16}                   & 0.89                  & \textbf{2.51}                    & 0.49                   & \textbf{2.4}                     & 0.52                   & \textbf{2.26}                    & 0.47                   & \textbf{12.82}                   & 0.42                   & \textbf{12.82}                   & 0.41                   & \textbf{12.39}                   & 0.39                   \\
\textbf{ARG}                                 & \textbf{-13.77}                   & 0.55                  & \textbf{-13.44}                   & 0.45                  & \textbf{-13.79}                   & 0.49                  & \textbf{3.62}                    & 0.39                   & \textbf{3.61}                    & 0.36                   & \textbf{3.79}                    & 0.36                   & \textbf{11.18}                   & 0.35                   & \textbf{11.61}                   & 0.35                   & \textbf{11.48}                   & 0.36                   \\
\textbf{ARM}                                 & \textbf{-17.66}                   & 0.44                  & \textbf{-16.55}                   & 0.46                  & \textbf{-16.48}                   & 0.46                  & \textbf{6.26}                    & 0.61                   & \textbf{6.38}                    & 0.6                    & \textbf{5.94}                    & 0.58                   & \textbf{16.72}                   & 0.46                   & \textbf{17.05}                   & 0.47                   & \textbf{16.6}                    & 0.45                   \\
\textbf{ABW}                                 & \textbf{-15.21}                   & 0.56                  & \textbf{-15.61}                   & 0.62                  & \textbf{-15.94}                   & 0.67                  & \textbf{1.16}                    & 0.27                   & \textbf{0.89}                    & 0.27                   & \textbf{0.68}                    & 0.28                   & \textbf{8.71}                    & 0.27                   & \textbf{8.7}                     & 0.29                   & \textbf{8.71}                    & 0.29                   \\
\textbf{AUS}                                 & \textbf{-4.14}                    & 0.51                  & \textbf{-4.05}                    & 0.3                   & \textbf{-3.86}                    & 0.29                  & \textbf{3.26}                    & 0.11                   & \textbf{3.08}                    & 0.11                   & \textbf{3.15}                    & 0.11                   & \textbf{8.57}                    & 0.23                   & \textbf{8.65}                    & 0.22                   & \textbf{8.71}                    & 0.24                   \\
\textbf{AUT}                                 & \textbf{-5.42}                    & 0.28                  & \textbf{-6.2}                     & 0.5                   & \textbf{-5.63}                    & 0.26                  & \textbf{1.2}                     & 0.15                   & \textbf{1.06}                    & 0.15                   & \textbf{1.39}                    & 0.15                   & \textbf{8.63}                    & 0.27                   & \textbf{8.34}                    & 0.26                   & \textbf{8.45}                    & 0.27                   \\
\textbf{AZE}                                 & \textbf{-5.19}                    & 0.38                  & \textbf{-4.94}                    & 0.39                  & \textbf{-5.06}                    & 0.38                  & \textbf{6.93}                    & 0.21                   & \textbf{7.08}                    & 0.2                    & \textbf{6.7}                     & 0.18                   & \textbf{30.1}                    & 0.88                   & \textbf{30.28}                   & 0.88                   & \textbf{30.6}                    & 0.87                   \\
\textbf{BHS}                                 & \textbf{-6.97}                    & 0.7                   & \textbf{-7.39}                    & 0.67                  & \textbf{-7.82}                    & 0.97                  & \textbf{1.31}                    & 0.17                   & \textbf{1.05}                    & 0.18                   & \textbf{1.19}                    & 0.17                   & \textbf{8.55}                    & 0.25                   & \textbf{8.59}                    & 0.26                   & \textbf{8.73}                    & 0.26                   \\
\textbf{BHR}                                 & \textbf{-3.78}                    & 0.64                  & \textbf{-4.4}                     & 0.52                  & \textbf{-4.11}                    & 0.54                  & \textbf{4.46}                    & 0.12                   & \textbf{4.23}                    & 0.13                   & \textbf{4.27}                    & 0.12                   & \textbf{8.83}                    & 0.21                   & \textbf{9.08}                    & 0.21                   & \textbf{8.78}                    & 0.22                   \\
\textbf{BGD}                                 & \textbf{-3.89}                    & 0.38                  & \textbf{-4.11}                    & 0.42                  & \textbf{-3.98}                    & 0.36                  & \textbf{5.15}                    & 0.13                   & \textbf{5.26}                    & 0.15                   & \textbf{5.05}                    & 0.15                   & \textbf{9.77}                    & 0.29                   & \textbf{9.9}                     & 0.31                   & \textbf{9.72}                    & 0.27                   \\
\textbf{BRB}                                 & \textbf{-6.21}                    & 0.27                  & \textbf{-6.73}                    & 0.3                   & \textbf{-7.4}                     & 0.7                   & \textbf{1.16}                    & 0.22                   & \textbf{1.07}                    & 0.18                   & \textbf{1.2}                     & 0.18                   & \textbf{8.49}                    & 0.27                   & \textbf{8.63}                    & 0.28                   & \textbf{8.77}                    & 0.27                   \\
\textbf{BLR}                                 & \textbf{-4.64}                    & 0.78                  & \textbf{-4.12}                    & 0.28                  & \textbf{-4.37}                    & 0.38                  & \textbf{5.98}                    & 0.23                   & \textbf{6.02}                    & 0.26                   & \textbf{6.16}                    & 0.22                   & \textbf{12.97}                   & 0.37                   & \textbf{12.63}                   & 0.36                   & \textbf{12.69}                   & 0.33                   \\
\textbf{BEL}                                 & \textbf{-4.54}                    & 0.33                  & \textbf{-4.75}                    & 0.31                  & \textbf{-4.74}                    & 0.64                  & \textbf{1.41}                    & 0.12                   & \textbf{1.33}                    & 0.14                   & \textbf{1.45}                    & 0.1                    & \textbf{8.36}                    & 0.25                   & \textbf{8.5}                     & 0.27                   & \textbf{8.44}                    & 0.25                   \\
\textbf{BLZ}                                 & \textbf{-3.66}                    & 0.29                  & \textbf{-3.93}                    & 0.3                   & \textbf{-3.98}                    & 0.52                  & \textbf{3.71}                    & 0.14                   & \textbf{3.51}                    & 0.12                   & \textbf{3.43}                    & 0.15                   & \textbf{13.01}                   & 0.44                   & \textbf{13.07}                   & 0.42                   & \textbf{12.98}                   & 0.43                   \\
\textbf{BEN}                                 & \textbf{-3.69}                    & 0.26                  & \textbf{-4.05}                    & 0.26                  & \textbf{-4}                       & 0.25                  & \textbf{4.33}                    & 0.11                   & \textbf{4.11}                    & 0.1                    & \textbf{4.18}                    & 0.1                    & \textbf{8.78}                    & 0.19                   & \textbf{8.6}                     & 0.19                   & \textbf{8.73}                    & 0.2                    \\
\textbf{BMU}                                 & \textbf{-8.79}                    & 0.36                  & \textbf{-8.45}                    & 0.33                  & \textbf{-8.6}                     & 0.38                  & \textbf{0.97}                    & 0.24                   & \textbf{0.82}                    & 0.28                   & \textbf{1.06}                    & 0.25                   & \textbf{10.26}                   & 0.4                    & \textbf{10.17}                   & 0.4                    & \textbf{10.25}                   & 0.42                   \\
\textbf{BTN}                                 & \textbf{-4.72}                    & 0.64                  & \textbf{-4.17}                    & 0.42                  & \textbf{-4.18}                    & 0.39                  & \textbf{6.65}                    & 0.2                    & \textbf{6.66}                    & 0.21                   & \textbf{6.68}                    & 0.22                   & \textbf{18.07}                   & 0.55                   & \textbf{18.28}                   & 0.54                   & \textbf{18.04}                   & 0.53                   \\
\textbf{BOL}                                 & \textbf{-3.71}                    & 0.24                  & \textbf{-3.92}                    & 0.38                  & \textbf{-3.68}                    & 0.25                  & \textbf{4.06}                    & 0.09                   & \textbf{4.03}                    & 0.1                    & \textbf{4.05}                    & 0.1                    & \textbf{9.92}                    & 0.3                    & \textbf{9.83}                    & 0.31                   & \textbf{10.06}                   & 0.32                   \\
\textbf{BIH}                                 & \textbf{-4.71}                    & 0.44                  & \textbf{-6.12}                    & 1.09                  & \textbf{-5.08}                    & 0.8                   & \textbf{3.41}                    & 0.21                   & \textbf{3.19}                    & 0.19                   & \textbf{3.36}                    & 0.24                   & \textbf{12.55}                   & 0.45                   & \textbf{12.64}                   & 0.43                   & \textbf{12.24}                   & 0.41                   \\
\textbf{BWA}                                 & \textbf{-8.52}                    & 0.41                  & \textbf{-8.9}                     & 0.43                  & \textbf{-9.64}                    & 1.01                  & \textbf{4.16}                    & 0.36                   & \textbf{3.98}                    & 0.35                   & \textbf{4.23}                    & 0.39                   & \textbf{11.62}                   & 0.33                   & \textbf{11.37}                   & 0.32                   & \textbf{11.57}                   & 0.34                   \\
\textbf{BRA}                                 & \textbf{-4.3}                     & 0.31                  & \textbf{-4.39}                    & 0.33                  & \textbf{-4.17}                    & 0.35                  & \textbf{3.62}                    & 0.14                   & \textbf{3.32}                    & 0.13                   & \textbf{3.65}                    & 0.13                   & \textbf{10.34}                   & 0.36                   & \textbf{10.38}                   & 0.33                   & \textbf{10.12}                   & 0.31                   \\
\textbf{BRN}                                 & \textbf{-4.52}                    & 0.31                  & \textbf{-5.04}                    & 0.29                  & \textbf{-4.46}                    & 0.26                  & \textbf{1.21}                    & 0.16                   & \textbf{1.11}                    & 0.12                   & \textbf{1.21}                    & 0.17                   & \textbf{8.78}                    & 0.28                   & \textbf{8.57}                    & 0.25                   & \textbf{8.58}                    & 0.26                   \\
\textbf{BGR}                                 & \textbf{-5.84}                    & 0.48                  & \textbf{-5.95}                    & 0.49                  & \textbf{-5.42}                    & 0.42                  & \textbf{4.09}                    & 0.21                   & \textbf{3.91}                    & 0.2                    & \textbf{4.18}                    & 0.24                   & \textbf{8.9}                     & 0.21                   & \textbf{8.62}                    & 0.19                   & \textbf{8.85}                    & 0.25                   \\
\textbf{BFA}                                 & \textbf{-3.89}                    & 0.33                  & \textbf{-3.68}                    & 0.26                  & \textbf{-3.8}                     & 0.32                  & \textbf{5.28}                    & 0.18                   & \textbf{5.12}                    & 0.16                   & \textbf{5.25}                    & 0.17                   & \textbf{9.38}                    & 0.22                   & \textbf{9.5}                     & 0.24                   & \textbf{9.56}                    & 0.21                   \\
\textbf{BDI}                                 & \textbf{-3.87}                    & 0.24                  & \textbf{-4.48}                    & 0.36                  & \textbf{-4.09}                    & 0.3                   & \textbf{3.82}                    & 0.19                   & \textbf{3.67}                    & 0.18                   & \textbf{3.85}                    & 0.17                   & \textbf{8.67}                    & 0.2                    & \textbf{8.86}                    & 0.21                   & \textbf{8.64}                    & 0.2                    \\
\textbf{CPV}                                 & \textbf{-4.18}                    & 0.31                  & \textbf{-4.75}                    & 0.65                  & \textbf{-4.15}                    & 0.32                  & \textbf{4.01}                    & 0.23                   & \textbf{4.14}                    & 0.2                    & \textbf{4.03}                    & 0.22                   & \textbf{16.7}                    & 0.53                   & \textbf{16.36}                   & 0.51                   & \textbf{16.73}                   & 0.52                   \\
\textbf{KHM}                                 & \textbf{-4.12}                    & 0.65                  & \textbf{-5.49}                    & 1.6                   & \textbf{-3.82}                    & 0.31                  & \textbf{7.11}                    & 0.19                   & \textbf{7.08}                    & 0.19                   & \textbf{6.96}                    & 0.2                    & \textbf{14.6}                    & 0.41                   & \textbf{14.85}                   & 0.42                   & \textbf{14.38}                   & 0.38                   \\
\textbf{CMR}                                 & \textbf{-4}                       & 0.41                  & \textbf{-4.23}                    & 0.42                  & \textbf{-4.01}                    & 0.34                  & \textbf{3.87}                    & 0.1                    & \textbf{3.8}                     & 0.1                    & \textbf{4.01}                    & 0.08                   & \textbf{9.09}                    & 0.25                   & \textbf{9.13}                    & 0.25                   & \textbf{9.33}                    & 0.25                   \\
\textbf{CAN}                                 & \textbf{-4.61}                    & 0.25                  & \textbf{-4.58}                    & 0.27                  & \textbf{-4.55}                    & 0.32                  & \textbf{3.34}                    & 0.16                   & \textbf{3.36}                    & 0.15                   & \textbf{3.23}                    & 0.16                   & \textbf{9.23}                    & 0.27                   & \textbf{9.16}                    & 0.3                    & \textbf{9.11}                    & 0.29                   \\
\textbf{CYM}                                 & \textbf{-8.55}                    & 0.34                  & \textbf{-9.32}                    & 0.42                  & \textbf{-8.98}                    & 0.35                  & \textbf{0.91}                    & 0.28                   & \textbf{1}                       & 0.37                   & \textbf{0.88}                    & 0.29                   & \textbf{8.68}                    & 0.28                   & \textbf{8.91}                    & 0.34                   & \textbf{8.6}                     & 0.27                   \\
\textbf{CAF}                                 & \textbf{-52.03}                   & 1.78                  & \textbf{-51.7}                    & 1.76                  & \textbf{-52.08}                   & 1.82                  & \textbf{3.16}                    & 0.66                   & \textbf{3.01}                    & 0.71                   & \textbf{3.24}                    & 0.62                   & \textbf{9.39}                    & 0.27                   & \textbf{9.26}                    & 0.25                   & \textbf{9.34}                    & 0.26                   \\
\textbf{TCD}                                 & \textbf{-4.04}                    & 0.21                  & \textbf{-3.94}                    & 0.34                  & \textbf{-3.45}                    & 0.25                  & \textbf{6.48}                    & 0.27                   & \textbf{6.44}                    & 0.29                   & \textbf{6.86}                    & 0.29                   & \textbf{28.94}                   & 0.83                   & \textbf{29.11}                   & 0.83                   & \textbf{29.32}                   & 0.84                   \\
\textbf{CHL}                                 & \textbf{-4.13}                    & 0.3                   & \textbf{-4.28}                    & 0.63                  & \textbf{-3.91}                    & 0.42                  & \textbf{4.39}                    & 0.14                   & \textbf{4.43}                    & 0.15                   & \textbf{4.52}                    & 0.18                   & \textbf{9}                       & 0.23                   & \textbf{9.06}                    & 0.24                   & \textbf{9.04}                    & 0.23                   \\
\textbf{CHN}                                 & \textbf{-3.97}                    & 0.28                  & \textbf{-4.13}                    & 0.32                  & \textbf{-3.93}                    & 0.33                  & \textbf{7.14}                    & 0.2                    & \textbf{7.11}                    & 0.2                    & \textbf{7.18}                    & 0.19                   & \textbf{14.61}                   & 0.37                   & \textbf{14.88}                   & 0.38                   & \textbf{14.71}                   & 0.35                   \\
\textbf{COL}                                 & \textbf{-3.8}                     & 0.27                  & \textbf{-4.2}                     & 0.39                  & \textbf{-4.05}                    & 0.41                  & \textbf{3.93}                    & 0.11                   & \textbf{3.78}                    & 0.12                   & \textbf{3.81}                    & 0.15                   & \textbf{8.8}                     & 0.19                   & \textbf{8.61}                    & 0.19                   & \textbf{8.78}                    & 0.21                   \\
\textbf{COM}                                 & \textbf{-3.82}                    & 0.28                  & \textbf{-4.14}                    & 0.44                  & \textbf{-3.83}                    & 0.29                  & \textbf{2.96}                    & 0.12                   & \textbf{2.83}                    & 0.14                   & \textbf{2.82}                    & 0.1                    & \textbf{11.22}                   & 0.39                   & \textbf{11.52}                   & 0.41                   & \textbf{11.51}                   & 0.42                   \\
\textbf{COD}                                 & \textbf{-8.04}                    & 0.37                  & \textbf{-8.25}                    & 0.54                  & \textbf{-8.14}                    & 0.42                  & \textbf{5.48}                    & 0.41                   & \textbf{5.29}                    & 0.36                   & \textbf{5.57}                    & 0.36                   & \textbf{10.89}                   & 0.33                   & \textbf{10.8}                    & 0.3                    & \textbf{10.8}                    & 0.31                   \\
\textbf{COG}                                 & \textbf{-8.58}                    & 0.42                  & \textbf{-8.76}                    & 0.32                  & \textbf{-8.73}                    & 0.33                  & \textbf{5.48}                    & 0.42                   & \textbf{5.57}                    & 0.4                    & \textbf{5.74}                    & 0.38                   & \textbf{12.93}                   & 0.4                    & \textbf{12.98}                   & 0.4                    & \textbf{13.19}                   & 0.42                   \\
\textbf{CRI}                                 & \textbf{-4.65}                    & 0.45                  & \textbf{-5.52}                    & 0.84                  & \textbf{-5.59}                    & 0.74                  & \textbf{3.95}                    & 0.09                   & \textbf{3.68}                    & 0.11                   & \textbf{3.84}                    & 0.11                   & \textbf{9.34}                    & 0.31                   & \textbf{9.37}                    & 0.34                   & \textbf{9.32}                    & 0.32                   \\
\textbf{CIV}                                 & \textbf{-6.95}                    & 0.25                  & \textbf{-7.23}                    & 0.3                   & \textbf{-7.07}                    & 0.29                  & \textbf{1.81}                    & 0.26                   & \textbf{1.68}                    & 0.24                   & \textbf{1.95}                    & 0.22                   & \textbf{10.91}                   & 0.37                   & \textbf{10.77}                   & 0.4                    & \textbf{11.07}                   & 0.39                   \\
\textbf{HRV}                                 & \textbf{-8.75}                    & 0.44                  & \textbf{-9.51}                    & 0.64                  & \textbf{-8.6}                     & 0.4                   & \textbf{2.14}                    & 0.28                   & \textbf{1.83}                    & 0.25                   & \textbf{2.1}                     & 0.27                   & \textbf{8.74}                    & 0.29                   & \textbf{8.47}                    & 0.26                   & \textbf{8.8}                     & 0.28                   \\
\textbf{CUW}                                 & \textbf{-5.18}                    & 0.44                  & \textbf{-5.28}                    & 0.45                  & \textbf{-5.34}                    & 0.41                  & \textbf{1.42}                    & 0.11                   & \textbf{1.06}                    & 0.11                   & \textbf{1.01}                    & 0.1                    & \textbf{8.62}                    & 0.27                   & \textbf{8.56}                    & 0.27                   & \textbf{8.55}                    & 0.29                   \\
\textbf{CYP}                                 & \textbf{-8.51}                    & 0.72                  & \textbf{-7.82}                    & 0.25                  & \textbf{-8.11}                    & 0.74                  & \textbf{2.7}                     & 0.34                   & \textbf{3.07}                    & 0.31                   & \textbf{3.03}                    & 0.38                   & \textbf{8.44}                    & 0.22                   & \textbf{8.57}                    & 0.24                   & \textbf{8.78}                    & 0.23                   \\
\textbf{CZE}                                 & \textbf{-6.48}                    & 0.36                  & \textbf{-7.38}                    & 0.62                  & \textbf{-8.14}                    & 1.12                  & \textbf{2.69}                    & 0.2                    & \textbf{2.2}                     & 0.2                    & \textbf{2.67}                    & 0.21                   & \textbf{8.61}                    & 0.24                   & \textbf{8.28}                    & 0.19                   & \textbf{8.49}                    & 0.23                   \\
\textbf{DNK}                                 & \textbf{-6.47}                    & 0.62                  & \textbf{-6.55}                    & 0.41                  & \textbf{-6.36}                    & 0.36                  & \textbf{1.26}                    & 0.18                   & \textbf{1.11}                    & 0.2                    & \textbf{1.24}                    & 0.14                   & \textbf{8.56}                    & 0.26                   & \textbf{8.4}                     & 0.27                   & \textbf{8.38}                    & 0.26                   \\
\textbf{DJI}                                 & \textbf{-4.27}                    & 0.63                  & \textbf{-4.08}                    & 0.4                   & \textbf{-4.03}                    & 0.34                  & \textbf{6.72}                    & 0.24                   & \textbf{6.77}                    & 0.25                   & \textbf{6.77}                    & 0.26                   & \textbf{8.94}                    & 0.16                   & \textbf{9.01}                    & 0.19                   & \textbf{9.11}                    & 0.16                   \\
\textbf{DMA}                                 & \textbf{-5.57}                    & 0.44                  & \textbf{-5.81}                    & 0.66                  & \textbf{-6.49}                    & 0.86                  & \textbf{1.28}                    & 0.17                   & \textbf{1.41}                    & 0.12                   & \textbf{1.4}                     & 0.13                   & \textbf{8.51}                    & 0.27                   & \textbf{8.81}                    & 0.29                   & \textbf{8.65}                    & 0.27                   \\
\textbf{DOM}                                 & \textbf{-4.88}                    & 0.63                  & \textbf{-4.83}                    & 0.65                  & \textbf{-4.82}                    & 0.64                  & \textbf{3.97}                    & 0.1                    & \textbf{3.83}                    & 0.11                   & \textbf{3.91}                    & 0.09                   & \textbf{10.69}                   & 0.34                   & \textbf{10.75}                   & 0.33                   & \textbf{10.56}                   & 0.33                   \\
\textbf{ECU}                                 & \textbf{-4.11}                    & 0.3                   & \textbf{-4.17}                    & 0.44                  & \textbf{-3.92}                    & 0.4                   & \textbf{3.93}                    & 0.11                   & \textbf{3.83}                    & 0.11                   & \textbf{3.98}                    & 0.14                   & \textbf{9.69}                    & 0.31                   & \textbf{9.76}                    & 0.29                   & \textbf{9.75}                    & 0.31                   \\
\textbf{EGY}                                 & \textbf{-4.24}                    & 0.28                  & \textbf{-3.85}                    & 0.28                  & \textbf{-4.11}                    & 0.65                  & \textbf{3.69}                    & 0.12                   & \textbf{3.75}                    & 0.1                    & \textbf{3.97}                    & 0.11                   & \textbf{9.45}                    & 0.31                   & \textbf{9.62}                    & 0.28                   & \textbf{9.71}                    & 0.32                   \\
\textbf{SLV}                                 & \textbf{-4.28}                    & 0.31                  & \textbf{-3.87}                    & 0.27                  & \textbf{-3.89}                    & 0.25                  & \textbf{1.29}                    & 0.1                    & \textbf{1.36}                    & 0.12                   & \textbf{1.26}                    & 0.14                   & \textbf{8.7}                     & 0.26                   & \textbf{8.52}                    & 0.27                   & \textbf{8.47}                    & 0.26                   \\
\textbf{GNQ}                                 & \textbf{-11.22}                   & 0.42                  & \textbf{-11.28}                   & 0.42                  & \textbf{-10.97}                   & 0.4                   & \textbf{6.71}                    & 0.47                   & \textbf{6.47}                    & 0.45                   & \textbf{6.8}                     & 0.49                   & \textbf{56.36}                   & 1.92                   & \textbf{56.03}                   & 1.9                    & \textbf{54.94}                   & 1.89                   \\
\textbf{EST}                                 & \textbf{-19.32}                   & 1.07                  & \textbf{-18.92}                   & 0.91                  & \textbf{-17.15}                   & 0.52                  & \textbf{5.13}                    & 0.76                   & \textbf{5.59}                    & 0.74                   & \textbf{5.5}                     & 0.75                   & \textbf{11.31}                   & 0.33                   & \textbf{11.58}                   & 0.32                   & \textbf{11.53}                   & 0.31                   \\
\textbf{SWZ}                                 & \textbf{-3.97}                    & 0.25                  & \textbf{-4.05}                    & 0.36                  & \textbf{-3.6}                     & 0.35                  & \textbf{3.51}                    & 0.12                   & \textbf{3.54}                    & 0.11                   & \textbf{3.48}                    & 0.14                   & \textbf{8.52}                    & 0.22                   & \textbf{8.86}                    & 0.25                   & \textbf{8.59}                    & 0.23                   \\
\textbf{ETH}                                 & \textbf{-3.77}                    & 0.28                  & \textbf{-4.47}                    & 0.41                  & \textbf{-4.41}                    & 0.4                   & \textbf{7.07}                    & 0.25                   & \textbf{6.82}                    & 0.24                   & \textbf{7.27}                    & 0.23                   & \textbf{14.68}                   & 0.42                   & \textbf{14.74}                   & 0.35                   & \textbf{14.63}                   & 0.35                   \\
\textbf{FJI}                                 & \textbf{-4.41}                    & 0.42                  & \textbf{-4.6}                     & 0.42                  & \textbf{-4.52}                    & 0.42                  & \textbf{1.38}                    & 0.13                   & \textbf{1.18}                    & 0.14                   & \textbf{1.39}                    & 0.13                   & \textbf{8.61}                    & 0.26                   & \textbf{8.57}                    & 0.25                   & \textbf{8.51}                    & 0.26                   \\
\textbf{FIN}                                 & \textbf{-12.64}                   & 1.01                  & \textbf{-12.6}                    & 0.99                  & \textbf{-12.36}                   & 1                     & \textbf{1.26}                    & 0.29                   & \textbf{1.77}                    & 0.3                    & \textbf{1.6}                     & 0.28                   & \textbf{8.47}                    & 0.26                   & \textbf{8.28}                    & 0.24                   & \textbf{8.56}                    & 0.28                   \\
\textbf{FRA}                                 & \textbf{-5.73}                    & 1.05                  & \textbf{-4.77}                    & 0.38                  & \textbf{-5.61}                    & 1.06                  & \textbf{1.16}                    & 0.13                   & \textbf{1.39}                    & 0.14                   & \textbf{1.39}                    & 0.12                   & \textbf{8.34}                    & 0.26                   & \textbf{8.5}                     & 0.26                   & \textbf{8.41}                    & 0.26                   \\
\textbf{GAB}                                 & \textbf{-6.26}                    & 0.38                  & \textbf{-6.55}                    & 0.41                  & \textbf{-6.81}                    & 0.42                  & \textbf{1.6}                     & 0.18                   & \textbf{1.43}                    & 0.19                   & \textbf{1.74}                    & 0.19                   & \textbf{8.51}                    & 0.24                   & \textbf{8.56}                    & 0.24                   & \textbf{8.72}                    & 0.27                   \\
\textbf{GMB}                                 & \textbf{-10.73}                   & 0.47                  & \textbf{-11.48}                   & 0.97                  & \textbf{-11.14}                   & 0.99                  & \textbf{4.13}                    & 0.38                   & \textbf{3.68}                    & 0.35                   & \textbf{3.8}                     & 0.34                   & \textbf{9.87}                    & 0.3                    & \textbf{9.88}                    & 0.3                    & \textbf{9.98}                    & 0.29                   \\
\textbf{GEO}                                 & \textbf{-5.76}                    & 0.4                   & \textbf{-5.9}                     & 0.65                  & \textbf{-5.39}                    & 0.36                  & \textbf{5.1}                     & 0.26                   & \textbf{5.03}                    & 0.28                   & \textbf{4.56}                    & 0.26                   & \textbf{12.93}                   & 0.36                   & \textbf{13.02}                   & 0.4                    & \textbf{12.78}                   & 0.34                   \\
\textbf{DEU}                                 & \textbf{-7.55}                    & 0.38                  & \textbf{-7.74}                    & 0.41                  & \textbf{-7.62}                    & 0.41                  & \textbf{1.09}                    & 0.2                    & \textbf{1.32}                    & 0.18                   & \textbf{1.39}                    & 0.17                   & \textbf{8.48}                    & 0.26                   & \textbf{8.58}                    & 0.27                   & \textbf{8.45}                    & 0.26                   \\
\textbf{GHA}                                 & \textbf{-3.75}                    & 0.33                  & \textbf{-4.59}                    & 0.79                  & \textbf{-4.11}                    & 0.41                  & \textbf{5.36}                    & 0.18                   & \textbf{5.24}                    & 0.16                   & \textbf{5.24}                    & 0.16                   & \textbf{15.45}                   & 0.54                   & \textbf{15.43}                   & 0.53                   & \textbf{15.39}                   & 0.56                   \\
\textbf{GRL}                                 & \textbf{-13.1}                    & 0.41                  & \textbf{-12.9}                    & 0.46                  & \textbf{-13.83}                   & 0.95                  & \textbf{1.03}                    & 0.27                   & \textbf{1.25}                    & 0.3                    & \textbf{0.87}                    & 0.29                   & \textbf{8.7}                     & 0.28                   & \textbf{8.76}                    & 0.28                   & \textbf{8.58}                    & 0.26                   \\
\textbf{GUM}                                 & \textbf{-8.63}                    & 0.5                   & \textbf{-8.67}                    & 0.5                   & \textbf{-9.09}                    & 0.44                  & \textbf{1.21}                    & 0.2                    & \textbf{1.38}                    & 0.24                   & \textbf{1.04}                    & 0.22                   & \textbf{12.92}                   & 0.52                   & \textbf{12.97}                   & 0.51                   & \textbf{12.7}                    & 0.52                   \\
\textbf{GIN}                                 & \textbf{-3.77}                    & 0.42                  & \textbf{-4.45}                    & 0.61                  & \textbf{-3.99}                    & 0.63                  & \textbf{3.52}                    & 0.1                    & \textbf{3.22}                    & 0.1                    & \textbf{3.63}                    & 0.12                   & \textbf{8.94}                    & 0.28                   & \textbf{8.7}                     & 0.27                   & \textbf{9.15}                    & 0.28                   \\
\textbf{GNB}                                 & \textbf{-3.87}                    & 0.41                  & \textbf{-4.17}                    & 0.47                  & \textbf{-3.95}                    & 0.41                  & \textbf{3.81}                    & 0.17                   & \textbf{3.37}                    & 0.17                   & \textbf{3.6}                     & 0.16                   & \textbf{9.03}                    & 0.21                   & \textbf{8.55}                    & 0.19                   & \textbf{8.77}                    & 0.2                    \\
\textbf{GUY}                                 & \textbf{-4.69}                    & 0.4                   & \textbf{-4.3}                     & 0.41                  & \textbf{-4.71}                    & 0.43                  & \textbf{2.78}                    & 0.2                    & \textbf{2.73}                    & 0.17                   & \textbf{2.57}                    & 0.17                   & \textbf{9}                       & 0.28                   & \textbf{8.92}                    & 0.27                   & \textbf{9.01}                    & 0.27                   \\
\textbf{HTI}                                 & \textbf{-5.01}                    & 0.64                  & \textbf{-5.06}                    & 0.81                  & \textbf{-5.21}                    & 0.8                   & \textbf{2.05}                    & 0.2                    & \textbf{1.9}                     & 0.22                   & \textbf{1.87}                    & 0.21                   & \textbf{8.66}                    & 0.25                   & \textbf{8.51}                    & 0.26                   & \textbf{8.75}                    & 0.25                   \\
\textbf{HND}                                 & \textbf{-8.26}                    & 0.54                  & \textbf{-8.29}                    & 0.51                  & \textbf{-9.11}                    & 1                     & \textbf{1.37}                    & 0.19                   & \textbf{1.25}                    & 0.18                   & \textbf{1.12}                    & 0.22                   & \textbf{8.62}                    & 0.26                   & \textbf{8.66}                    & 0.28                   & \textbf{8.68}                    & 0.26                   \\
\textbf{HKG}                                 & \textbf{-5.54}                    & 0.34                  & \textbf{-5.22}                    & 0.35                  & \textbf{-5.39}                    & 0.42                  & \textbf{3.62}                    & 0.2                    & \textbf{3.71}                    & 0.21                   & \textbf{3.7}                     & 0.25                   & \textbf{10.04}                   & 0.39                   & \textbf{10.15}                   & 0.4                    & \textbf{10.16}                   & 0.41                   \\
\textbf{HUN}                                 & \textbf{-4.89}                    & 0.78                  & \textbf{-5.17}                    & 0.81                  & \textbf{-4.88}                    & 0.63                  & \textbf{3.54}                    & 0.19                   & \textbf{3.3}                     & 0.19                   & \textbf{3.16}                    & 0.19                   & \textbf{10.37}                   & 0.33                   & \textbf{10.33}                   & 0.32                   & \textbf{10.1}                    & 0.34                   \\
\textbf{ISL}                                 & \textbf{-8.61}                    & 0.36                  & \textbf{-8.68}                    & 0.44                  & \textbf{-8.2}                     & 0.31                  & \textbf{3.07}                    & 0.26                   & \textbf{3.15}                    & 0.28                   & \textbf{3.09}                    & 0.25                   & \textbf{8.69}                    & 0.23                   & \textbf{8.86}                    & 0.27                   & \textbf{8.73}                    & 0.27                   \\
\textbf{IND}                                 & \textbf{-9.76}                    & 0.87                  & \textbf{-10.13}                   & 0.85                  & \textbf{-10.47}                   & 0.88                  & \textbf{2.43}                    & 0.35                   & \textbf{2.4}                     & 0.36                   & \textbf{2.34}                    & 0.36                   & \textbf{9.13}                    & 0.26                   & \textbf{9.07}                    & 0.26                   & \textbf{9.12}                    & 0.25                   \\
\textbf{IDN}                                 & \textbf{-3.8}                     & 0.33                  & \textbf{-4.3}                     & 0.44                  & \textbf{-3.89}                    & 0.34                  & \textbf{7.06}                    & 0.19                   & \textbf{6.61}                    & 0.19                   & \textbf{6.8}                     & 0.21                   & \textbf{10.12}                   & 0.27                   & \textbf{9.9}                     & 0.26                   & \textbf{10.03}                   & 0.26                   \\
\textbf{IRN}                                 & \textbf{-3.78}                    & 0.25                  & \textbf{-4.55}                    & 0.64                  & \textbf{-4.09}                    & 0.35                  & \textbf{4.84}                    & 0.13                   & \textbf{4.77}                    & 0.11                   & \textbf{4.83}                    & 0.14                   & \textbf{8.55}                    & 0.2                    & \textbf{8.77}                    & 0.23                   & \textbf{8.86}                    & 0.26                   \\
\textbf{IRL}                                 & \textbf{-51.25}                   & 2.02                  & \textbf{-52.08}                   & 2.01                  & \textbf{-50.94}                   & 2.08                  & \textbf{3.92}                    & 0.66                   & \textbf{4.01}                    & 0.68                   & \textbf{4.11}                    & 0.67                   & \textbf{53}                      & 2.03                   & \textbf{53.93}                   & 2.09                   & \textbf{54.07}                   & 2.04                   \\
\textbf{IMN}                                 & \textbf{-7.45}                    & 0.45                  & \textbf{-7.69}                    & 0.26                  & \textbf{-8.34}                    & 0.52                  & \textbf{3.68}                    & 0.28                   & \textbf{3.91}                    & 0.29                   & \textbf{3.96}                    & 0.31                   & \textbf{9.98}                    & 0.35                   & \textbf{10.03}                   & 0.34                   & \textbf{10.31}                   & 0.35                   \\
\textbf{ITA}                                 & \textbf{-4.26}                    & 0.37                  & \textbf{-4.4}                     & 0.49                  & \textbf{-4.35}                    & 0.4                   & \textbf{3.7}                     & 0.14                   & \textbf{3.88}                    & 0.11                   & \textbf{3.97}                    & 0.13                   & \textbf{10.08}                   & 0.3                    & \textbf{10.21}                   & 0.33                   & \textbf{10.17}                   & 0.33                   \\
\textbf{JAM}                                 & \textbf{-6.52}                    & 0.31                  & \textbf{-7.45}                    & 0.44                  & \textbf{-7.13}                    & 0.61                  & \textbf{1.22}                    & 0.22                   & \textbf{0.94}                    & 0.23                   & \textbf{1.09}                    & 0.2                    & \textbf{8.71}                    & 0.27                   & \textbf{8.65}                    & 0.27                   & \textbf{8.55}                    & 0.27                   \\
\textbf{JPN}                                 & \textbf{-8.12}                    & 0.62                  & \textbf{-7.46}                    & 0.42                  & \textbf{-7.4}                     & 0.49                  & \textbf{1}                       & 0.19                   & \textbf{1.01}                    & 0.16                   & \textbf{1.26}                    & 0.18                   & \textbf{8.46}                    & 0.26                   & \textbf{8.33}                    & 0.25                   & \textbf{8.72}                    & 0.27                   \\
\textbf{JOR}                                 & \textbf{-8.27}                    & 0.98                  & \textbf{-7.52}                    & 0.92                  & \textbf{-7.68}                    & 0.89                  & \textbf{0.9}                     & 0.1                    & \textbf{1.18}                    & 0.11                   & \textbf{1.32}                    & 0.12                   & \textbf{8.49}                    & 0.29                   & \textbf{8.56}                    & 0.29                   & \textbf{8.5}                     & 0.3                    \\
\textbf{KAZ}                                 & \textbf{-4.13}                    & 0.26                  & \textbf{-3.87}                    & 0.29                  & \textbf{-3.99}                    & 0.31                  & \textbf{4.32}                    & 0.12                   & \textbf{4.58}                    & 0.13                   & \textbf{4.4}                     & 0.12                   & \textbf{10.37}                   & 0.38                   & \textbf{10.7}                    & 0.39                   & \textbf{10.58}                   & 0.36                   \\
\textbf{KEN}                                 & \textbf{-3.66}                    & 0.28                  & \textbf{-3.86}                    & 0.28                  & \textbf{-4.07}                    & 0.33                  & \textbf{7.01}                    & 0.21                   & \textbf{6.88}                    & 0.18                   & \textbf{7.08}                    & 0.22                   & \textbf{14.02}                   & 0.37                   & \textbf{13.73}                   & 0.34                   & \textbf{13.95}                   & 0.35                   \\
\textbf{KIR}                                 & \textbf{-4.16}                    & 0.64                  & \textbf{-4.42}                    & 0.99                  & \textbf{-4.5}                     & 0.99                  & \textbf{4.03}                    & 0.15                   & \textbf{3.97}                    & 0.15                   & \textbf{3.78}                    & 0.15                   & \textbf{9.08}                    & 0.21                   & \textbf{9.12}                    & 0.23                   & \textbf{8.78}                    & 0.2                    \\
\textbf{PRK}                                 & \textbf{-4.08}                    & 0.3                   & \textbf{-6}                       & 1.16                  & \textbf{-4.93}                    & 0.61                  & \textbf{1.4}                     & 0.12                   & \textbf{1.27}                    & 0.15                   & \textbf{1.37}                    & 0.11                   & \textbf{8.61}                    & 0.27                   & \textbf{8.47}                    & 0.26                   & \textbf{8.54}                    & 0.27                   \\
\textbf{KWT}                                 & \textbf{-4.11}                    & 0.3                   & \textbf{-3.72}                    & 0.41                  & \textbf{-4.08}                    & 0.36                  & \textbf{4.16}                    & 0.13                   & \textbf{4.52}                    & 0.14                   & \textbf{4.33}                    & 0.13                   & \textbf{8.85}                    & 0.25                   & \textbf{9.09}                    & 0.25                   & \textbf{9.03}                    & 0.27                   \\
\textbf{KGZ}                                 & \textbf{-9.31}                    & 0.47                  & \textbf{-9.79}                    & 0.56                  & \textbf{-9.68}                    & 0.63                  & \textbf{3.76}                    & 0.29                   & \textbf{3.86}                    & 0.3                    & \textbf{3.79}                    & 0.29                   & \textbf{16.48}                   & 0.55                   & \textbf{16.48}                   & 0.52                   & \textbf{16.39}                   & 0.55                   \\
\textbf{LAO}                                 & \textbf{-3.87}                    & 0.33                  & \textbf{-4.3}                     & 0.4                   & \textbf{-4.03}                    & 0.3                   & \textbf{4.23}                    & 0.26                   & \textbf{4.07}                    & 0.23                   & \textbf{4.29}                    & 0.26                   & \textbf{10.98}                   & 0.32                   & \textbf{11.56}                   & 0.35                   & \textbf{11.3}                    & 0.33                   \\
\textbf{LVA}                                 & \textbf{-4.08}                    & 0.32                  & \textbf{-3.82}                    & 0.31                  & \textbf{-4.56}                    & 0.48                  & \textbf{6.95}                    & 0.21                   & \textbf{7.28}                    & 0.21                   & \textbf{6.99}                    & 0.2                    & \textbf{11.3}                    & 0.31                   & \textbf{11.27}                   & 0.32                   & \textbf{11.05}                   & 0.31                   \\
\textbf{LBN}                                 & \textbf{-17.18}                   & 0.44                  & \textbf{-17.12}                   & 0.51                  & \textbf{-16.76}                   & 0.47                  & \textbf{5.34}                    & 0.56                   & \textbf{5.69}                    & 0.6                    & \textbf{5.56}                    & 0.55                   & \textbf{12.31}                   & 0.34                   & \textbf{12.87}                   & 0.39                   & \textbf{12.58}                   & 0.32                   \\
\textbf{LSO}                                 & \textbf{-4.24}                    & 0.31                  & \textbf{-5}                       & 0.82                  & \textbf{-4.28}                    & 0.5                   & \textbf{3.44}                    & 0.12                   & \textbf{3.24}                    & 0.13                   & \textbf{3.29}                    & 0.13                   & \textbf{12.38}                   & 0.44                   & \textbf{12.36}                   & 0.47                   & \textbf{12.04}                   & 0.44                   \\
\textbf{LBR}                                 & \textbf{-3.93}                    & 0.36                  & \textbf{-4.38}                    & 0.43                  & \textbf{-4.68}                    & 0.64                  & \textbf{3.77}                    & 0.13                   & \textbf{3.52}                    & 0.09                   & \textbf{3.63}                    & 0.1                    & \textbf{8.93}                    & 0.23                   & \textbf{8.84}                    & 0.22                   & \textbf{8.91}                    & 0.24                   \\
\textbf{LBY}                                 & \textbf{-43.72}                   & 1.73                  & \textbf{-43.55}                   & 1.59                  & \textbf{-44.53}                   & 1.8                   & \textbf{4.92}                    & 0.71                   & \textbf{5}                       & 0.73                   & \textbf{4.98}                    & 0.73                   & \textbf{11.95}                   & 0.41                   & \textbf{12}                      & 0.39                   & \textbf{11.89}                   & 0.36                   \\
\textbf{LIE}                                 & \textbf{-68.16}                   & 1.99                  & \textbf{-71.83}                   & 1                     & \textbf{-68.19}                   & 1.99                  & \textbf{2.34}                    & 0.67                   & \textbf{2.08}                    & 0.62                   & \textbf{2.42}                    & 0.65                   & \textbf{59.93}                   & 2.39                   & \textbf{61.83}                   & 1.97                   & \textbf{62.2}                    & 2.01                   \\
\textbf{LUX}                                 & \textbf{-17.41}                   & 0.59                  & \textbf{-17.61}                   & 0.7                   & \textbf{-16.99}                   & 0.57                  & \textbf{3.88}                    & 0.66                   & \textbf{3.88}                    & 0.67                   & \textbf{4.01}                    & 0.7                    & \textbf{12}                      & 0.34                   & \textbf{12.19}                   & 0.31                   & \textbf{12.11}                   & 0.29                   \\
\textbf{MAC}                                 & \textbf{-6.21}                    & 0.42                  & \textbf{-6.3}                     & 0.42                  & \textbf{-6.09}                    & 0.42                  & \textbf{2.87}                    & 0.18                   & \textbf{2.95}                    & 0.17                   & \textbf{3.02}                    & 0.2                    & \textbf{9.49}                    & 0.38                   & \textbf{9.6}                     & 0.39                   & \textbf{9.49}                    & 0.38                   \\
\textbf{MDG}                                 & \textbf{-4.28}                    & 0.33                  & \textbf{-4.5}                     & 0.32                  & \textbf{-4.87}                    & 0.86                  & \textbf{6.92}                    & 0.31                   & \textbf{6.88}                    & 0.29                   & \textbf{6.84}                    & 0.29                   & \textbf{27.24}                   & 0.86                   & \textbf{27.45}                   & 0.85                   & \textbf{27.95}                   & 0.91                   \\
\textbf{MWI}                                 & \textbf{-15.89}                   & 0.44                  & \textbf{-16.76}                   & 0.37                  & \textbf{-15.3}                    & 0.4                   & \textbf{3.21}                    & 0.55                   & \textbf{3}                       & 0.51                   & \textbf{3.15}                    & 0.51                   & \textbf{10.53}                   & 0.33                   & \textbf{10.72}                   & 0.32                   & \textbf{10.63}                   & 0.33                   \\
\textbf{MYS}                                 & \textbf{-7.12}                    & 0.41                  & \textbf{-7.13}                    & 0.39                  & \textbf{-6.79}                    & 0.39                  & \textbf{4.66}                    & 0.21                   & \textbf{4.32}                    & 0.21                   & \textbf{4.56}                    & 0.25                   & \textbf{10.42}                   & 0.33                   & \textbf{10.3}                    & 0.32                   & \textbf{10.41}                   & 0.33                   \\
\textbf{MDV}                                 & \textbf{-4.07}                    & 0.28                  & \textbf{-4.04}                    & 0.25                  & \textbf{-4.58}                    & 0.45                  & \textbf{4.96}                    & 0.16                   & \textbf{4.93}                    & 0.16                   & \textbf{4.85}                    & 0.19                   & \textbf{9.96}                    & 0.29                   & \textbf{9.89}                    & 0.3                    & \textbf{10}                      & 0.3                    \\
\textbf{MLI}                                 & \textbf{-17.01}                   & 0.37                  & \textbf{-16.32}                   & 0.42                  & \textbf{-17.25}                   & 0.76                  & \textbf{6.01}                    & 0.71                   & \textbf{6.17}                    & 0.6                    & \textbf{6.17}                    & 0.65                   & \textbf{26.93}                   & 0.88                   & \textbf{26.94}                   & 0.92                   & \textbf{27.08}                   & 0.83                   \\
\textbf{MLT}                                 & \textbf{-4.4}                     & 0.34                  & \textbf{-4.37}                    & 0.45                  & \textbf{-5.03}                    & 0.73                  & \textbf{4.02}                    & 0.15                   & \textbf{3.84}                    & 0.14                   & \textbf{4.1}                     & 0.13                   & \textbf{16.54}                   & 0.56                   & \textbf{16.65}                   & 0.56                   & \textbf{16.89}                   & 0.57                   \\
\textbf{MHL}                                 & \textbf{-4.63}                    & 0.48                  & \textbf{-4.16}                    & 0.38                  & \textbf{-3.87}                    & 0.4                   & \textbf{2.99}                    & 0.36                   & \textbf{3.25}                    & 0.22                   & \textbf{3.71}                    & 0.19                   & \textbf{18.79}                   & 0.7                    & \textbf{18.87}                   & 0.74                   & \textbf{18.48}                   & 0.65                   \\
\textbf{MRT}                                 & \textbf{-7.64}                    & 0.31                  & \textbf{-7.8}                     & 0.59                  & \textbf{-8.06}                    & 0.43                  & \textbf{1.01}                    & 0.25                   & \textbf{0.61}                    & 0.69                   & \textbf{0.99}                    & 0.23                   & \textbf{8.54}                    & 0.25                   & \textbf{8.68}                    & 0.28                   & \textbf{8.71}                    & 0.27                   \\
\textbf{MUS}                                 & \textbf{-6.83}                    & 0.41                  & \textbf{-6.81}                    & 0.41                  & \textbf{-6.54}                    & 0.41                  & \textbf{3.26}                    & 0.25                   & \textbf{3.13}                    & 0.25                   & \textbf{3.27}                    & 0.29                   & \textbf{19.01}                   & 0.75                   & \textbf{19.04}                   & 0.74                   & \textbf{19.35}                   & 0.73                   \\
\textbf{MEX}                                 & \textbf{-3.71}                    & 0.26                  & \textbf{-3.93}                    & 0.39                  & \textbf{-3.77}                    & 0.31                  & \textbf{4}                       & 0.1                    & \textbf{3.99}                    & 0.11                   & \textbf{3.97}                    & 0.1                    & \textbf{8.98}                    & 0.24                   & \textbf{8.89}                    & 0.23                   & \textbf{8.9}                     & 0.22                   \\
\textbf{FSM}                                 & \textbf{-6.52}                    & 0.31                  & \textbf{-6.42}                    & 0.28                  & \textbf{-6.49}                    & 0.33                  & \textbf{2.11}                    & 0.25                   & \textbf{2.34}                    & 0.24                   & \textbf{2.03}                    & 0.23                   & \textbf{9.11}                    & 0.38                   & \textbf{8.44}                    & 0.24                   & \textbf{8.78}                    & 0.27                   \\
\textbf{MDA}                                 & \textbf{-6.57}                    & 0.45                  & \textbf{-6.54}                    & 0.45                  & \textbf{-5.85}                    & 0.31                  & \textbf{1.32}                    & 0.15                   & \textbf{1.13}                    & 0.14                   & \textbf{1.2}                     & 0.17                   & \textbf{8.49}                    & 0.27                   & \textbf{8.25}                    & 0.26                   & \textbf{8.75}                    & 0.32                   \\
\textbf{MCO}                                 & \textbf{-8.3}                     & 0.69                  & \textbf{-8.01}                    & 0.9                   & \textbf{-7.62}                    & 0.43                  & \textbf{5.51}                    & 0.34                   & \textbf{5.51}                    & 0.32                   & \textbf{5.52}                    & 0.31                   & \textbf{9.66}                    & 0.26                   & \textbf{9.52}                    & 0.25                   & \textbf{9.52}                    & 0.27                   \\
\textbf{MNE}                                 & \textbf{-4.53}                    & 0.46                  & \textbf{-6.17}                    & 1.31                  & \textbf{-5.7}                     & 1.03                  & \textbf{6.76}                    & 0.23                   & \textbf{6.54}                    & 0.25                   & \textbf{6.96}                    & 0.23                   & \textbf{17.13}                   & 0.46                   & \textbf{16.97}                   & 0.45                   & \textbf{17.34}                   & 0.45                   \\
\textbf{MAR}                                 & \textbf{-8.24}                    & 0.42                  & \textbf{-8.34}                    & 0.45                  & \textbf{-8.74}                    & 0.43                  & \textbf{2.94}                    & 0.28                   & \textbf{2.94}                    & 0.27                   & \textbf{2.81}                    & 0.28                   & \textbf{9.44}                    & 0.31                   & \textbf{9.33}                    & 0.32                   & \textbf{9.4}                     & 0.3                    \\
\textbf{MOZ}                                 & \textbf{-4.11}                    & 0.38                  & \textbf{-4.5}                     & 0.45                  & \textbf{-3.8}                     & 0.39                  & \textbf{4.1}                     & 0.1                    & \textbf{3.79}                    & 0.12                   & \textbf{3.92}                    & 0.11                   & \textbf{10.27}                   & 0.32                   & \textbf{10.26}                   & 0.32                   & \textbf{10.19}                   & 0.31                   \\
\textbf{MMR}                                 & \textbf{-3.96}                    & 0.65                  & \textbf{-4.1}                     & 0.41                  & \textbf{-4.54}                    & 0.44                  & \textbf{6.9}                     & 0.21                   & \textbf{6.94}                    & 0.2                    & \textbf{6.73}                    & 0.22                   & \textbf{13.33}                   & 0.33                   & \textbf{13.52}                   & 0.39                   & \textbf{13.39}                   & 0.36                   \\
\textbf{NAM}                                 & \textbf{-4.14}                    & 0.8                   & \textbf{-4.02}                    & 0.52                  & \textbf{-3.93}                    & 0.64                  & \textbf{7.12}                    & 0.2                    & \textbf{6.96}                    & 0.19                   & \textbf{7.13}                    & 0.23                   & \textbf{16.97}                   & 0.42                   & \textbf{16.75}                   & 0.42                   & \textbf{17.23}                   & 0.44                   \\
\textbf{NRU}                                 & \textbf{-4.41}                    & 0.45                  & \textbf{-3.59}                    & 0.3                   & \textbf{-4.2}                     & 0.55                  & \textbf{4.31}                    & 0.16                   & \textbf{4.51}                    & 0.15                   & \textbf{4.52}                    & 0.16                   & \textbf{12.33}                   & 0.38                   & \textbf{12.85}                   & 0.46                   & \textbf{12.63}                   & 0.37                   \\
\textbf{NPL}                                 & \textbf{-31.32}                   & 1.94                  & \textbf{-31.2}                    & 1.84                  & \textbf{-32.04}                   & 1.79                  & \textbf{6.19}                    & 0.74                   & \textbf{6.4}                     & 0.75                   & \textbf{6.18}                    & 0.78                   & \textbf{28.54}                   & 0.82                   & \textbf{29.45}                   & 0.87                   & \textbf{28.81}                   & 0.9                    \\
\textbf{NLD}                                 & \textbf{-4.13}                    & 0.24                  & \textbf{-3.9}                     & 0.23                  & \textbf{-3.83}                    & 0.25                  & \textbf{3.88}                    & 0.09                   & \textbf{3.93}                    & 0.09                   & \textbf{3.99}                    & 0.11                   & \textbf{9.26}                    & 0.29                   & \textbf{9.33}                    & 0.29                   & \textbf{9.26}                    & 0.3                    \\
\textbf{NCL}                                 & \textbf{-6.25}                    & 0.46                  & \textbf{-6.1}                     & 0.46                  & \textbf{-5.94}                    & 0.35                  & \textbf{1.14}                    & 0.14                   & \textbf{1.3}                     & 0.12                   & \textbf{1.31}                    & 0.16                   & \textbf{8.48}                    & 0.26                   & \textbf{8.74}                    & 0.26                   & \textbf{8.54}                    & 0.24                   \\
\textbf{NIC}                                 & \textbf{-4.16}                    & 0.39                  & \textbf{-4.29}                    & 0.41                  & \textbf{-5.8}                     & 1.47                  & \textbf{2.86}                    & 0.17                   & \textbf{2.57}                    & 0.17                   & \textbf{2.52}                    & 0.16                   & \textbf{8.74}                    & 0.22                   & \textbf{8.57}                    & 0.23                   & \textbf{8.53}                    & 0.24                   \\
\textbf{NER}                                 & \textbf{-4.76}                    & 0.29                  & \textbf{-4.47}                    & 0.26                  & \textbf{-4.23}                    & 0.33                  & \textbf{3.85}                    & 0.18                   & \textbf{3.68}                    & 0.18                   & \textbf{4.15}                    & 0.18                   & \textbf{8.63}                    & 0.2                    & \textbf{8.68}                    & 0.2                    & \textbf{8.96}                    & 0.2                    \\
\textbf{NGA}                                 & \textbf{-4.68}                    & 0.63                  & \textbf{-4.71}                    & 0.63                  & \textbf{-4.73}                    & 0.64                  & \textbf{4.76}                    & 0.22                   & \textbf{4.5}                     & 0.23                   & \textbf{4.59}                    & 0.22                   & \textbf{10.98}                   & 0.32                   & \textbf{10.94}                   & 0.33                   & \textbf{11.02}                   & 0.32                   \\
\textbf{MKD}                                 & \textbf{-3.88}                    & 0.31                  & \textbf{-5.51}                    & 1.17                  & \textbf{-5.39}                    & 1.18                  & \textbf{6.75}                    & 0.21                   & \textbf{6.23}                    & 0.19                   & \textbf{6.5}                     & 0.19                   & \textbf{17.99}                   & 0.6                    & \textbf{17.54}                   & 0.56                   & \textbf{17.31}                   & 0.58                   \\
\textbf{MNP}                                 & \textbf{-5.58}                    & 0.42                  & \textbf{-5.3}                     & 0.44                  & \textbf{-5.66}                    & 0.43                  & \textbf{3.38}                    & 0.18                   & \textbf{3.5}                     & 0.25                   & \textbf{3.14}                    & 0.19                   & \textbf{9.65}                    & 0.31                   & \textbf{10.03}                   & 0.33                   & \textbf{9.46}                    & 0.32                   \\
\textbf{OMN}                                 & \textbf{-3.97}                    & 0.19                  & \textbf{-3.78}                    & 0.26                  & \textbf{-3.49}                    & 0.19                  & \textbf{1.35}                    & 0.13                   & \textbf{1.43}                    & 0.13                   & \textbf{1.57}                    & 0.13                   & \textbf{8.31}                    & 0.23                   & \textbf{8.6}                     & 0.28                   & \textbf{8.66}                    & 0.26                   \\
\textbf{PAK}                                 & \textbf{-4.39}                    & 0.33                  & \textbf{-4.64}                    & 0.35                  & \textbf{-4.55}                    & 0.36                  & \textbf{3.7}                     & 0.18                   & \textbf{3.65}                    & 0.19                   & \textbf{3.79}                    & 0.2                    & \textbf{9.92}                    & 0.32                   & \textbf{10.01}                   & 0.35                   & \textbf{10}                      & 0.33                   \\
\textbf{PLW}                                 & \textbf{-4.91}                    & 0.99                  & \textbf{-4.16}                    & 0.26                  & \textbf{-4.43}                    & 0.54                  & \textbf{3.98}                    & 0.12                   & \textbf{3.77}                    & 0.12                   & \textbf{3.78}                    & 0.11                   & \textbf{9.31}                    & 0.29                   & \textbf{9.27}                    & 0.28                   & \textbf{9.35}                    & 0.31                   \\
\textbf{PAN}                                 & \textbf{-9.65}                    & 0.71                  & \textbf{-9.7}                     & 0.56                  & \textbf{-9.84}                    & 0.9                   & \textbf{1.16}                    & 0.27                   & \textbf{0.96}                    & 0.26                   & \textbf{1.24}                    & 0.34                   & \textbf{8.52}                    & 0.27                   & \textbf{8.81}                    & 0.3                    & \textbf{8.65}                    & 0.27                   \\
\textbf{PNG}                                 & \textbf{-4.41}                    & 0.63                  & \textbf{-4.91}                    & 0.98                  & \textbf{-3.89}                    & 0.4                   & \textbf{6.27}                    & 0.2                    & \textbf{6.28}                    & 0.22                   & \textbf{6.51}                    & 0.23                   & \textbf{12.97}                   & 0.37                   & \textbf{12.74}                   & 0.32                   & \textbf{12.95}                   & 0.37                   \\
\textbf{PRY}                                 & \textbf{-4.83}                    & 0.41                  & \textbf{-5.42}                    & 0.46                  & \textbf{-5.18}                    & 0.39                  & \textbf{3.76}                    & 0.23                   & \textbf{3.46}                    & 0.2                    & \textbf{3.61}                    & 0.23                   & \textbf{13.78}                   & 0.44                   & \textbf{13.72}                   & 0.41                   & \textbf{13.99}                   & 0.46                   \\
\textbf{PER}                                 & \textbf{-4.37}                    & 0.41                  & \textbf{-4.59}                    & 0.5                   & \textbf{-4.38}                    & 0.41                  & \textbf{3.75}                    & 0.21                   & \textbf{3.58}                    & 0.23                   & \textbf{3.87}                    & 0.22                   & \textbf{11.09}                   & 0.35                   & \textbf{11.24}                   & 0.37                   & \textbf{11.14}                   & 0.32                   \\
\textbf{PHL}                                 & \textbf{-4.95}                    & 0.79                  & \textbf{-4.36}                    & 0.43                  & \textbf{-4.25}                    & 0.4                   & \textbf{5.2}                     & 0.19                   & \textbf{4.97}                    & 0.18                   & \textbf{5.12}                    & 0.18                   & \textbf{9.78}                    & 0.28                   & \textbf{9.48}                    & 0.27                   & \textbf{9.54}                    & 0.26                   \\
\textbf{POL}                                 & \textbf{-4.87}                    & 0.81                  & \textbf{-4.48}                    & 0.29                  & \textbf{-4.69}                    & 0.65                  & \textbf{4.75}                    & 0.16                   & \textbf{4.79}                    & 0.17                   & \textbf{4.82}                    & 0.16                   & \textbf{8.78}                    & 0.24                   & \textbf{8.94}                    & 0.22                   & \textbf{8.98}                    & 0.23                   \\
\textbf{PRT}                                 & \textbf{-4.64}                    & 0.64                  & \textbf{-4.42}                    & 0.43                  & \textbf{-4.45}                    & 0.8                   & \textbf{3.42}                    & 0.1                    & \textbf{3.48}                    & 0.12                   & \textbf{3.56}                    & 0.14                   & \textbf{9.47}                    & 0.31                   & \textbf{9.38}                    & 0.3                    & \textbf{9.59}                    & 0.32                   \\
\textbf{PRI}                                 & \textbf{-5.55}                    & 0.4                   & \textbf{-5.66}                    & 0.38                  & \textbf{-6}                       & 0.4                   & \textbf{1.36}                    & 0.13                   & \textbf{1.26}                    & 0.13                   & \textbf{1.24}                    & 0.16                   & \textbf{8.64}                    & 0.27                   & \textbf{8.78}                    & 0.27                   & \textbf{8.63}                    & 0.27                   \\
\textbf{QAT}                                 & \textbf{-5.69}                    & 0.45                  & \textbf{-6.7}                     & 0.62                  & \textbf{-5.89}                    & 0.59                  & \textbf{1.21}                    & 0.13                   & \textbf{0.73}                    & 0.11                   & \textbf{1.12}                    & 0.11                   & \textbf{10.15}                   & 0.39                   & \textbf{10.17}                   & 0.37                   & \textbf{9.97}                    & 0.38                   \\
\textbf{ROU}                                 & \textbf{-3.95}                    & 0.25                  & \textbf{-4.19}                    & 0.35                  & \textbf{-3.8}                     & 0.21                  & \textbf{6.92}                    & 0.2                    & \textbf{6.88}                    & 0.18                   & \textbf{6.6}                     & 0.22                   & \textbf{27.26}                   & 0.78                   & \textbf{26.93}                   & 0.75                   & \textbf{26.46}                   & 0.79                   \\
\textbf{RUS}                                 & \textbf{-7.1}                     & 0.26                  & \textbf{-6.66}                    & 0.31                  & \textbf{-6.77}                    & 0.27                  & \textbf{3.52}                    & 0.3                    & \textbf{3.62}                    & 0.28                   & \textbf{3.68}                    & 0.3                    & \textbf{11.01}                   & 0.35                   & \textbf{10.91}                   & 0.33                   & \textbf{10.92}                   & 0.33                   \\
\textbf{RWA}                                 & \textbf{-10.12}                   & 0.48                  & \textbf{-9.96}                    & 0.42                  & \textbf{-10.3}                    & 0.48                  & \textbf{4.29}                    & 0.37                   & \textbf{4.36}                    & 0.39                   & \textbf{4.48}                    & 0.33                   & \textbf{11.46}                   & 0.38                   & \textbf{11.37}                   & 0.35                   & \textbf{11.23}                   & 0.33                   \\
\textbf{WSM}                                 & \textbf{-4.31}                    & 0.5                   & \textbf{-4.23}                    & 0.35                  & \textbf{-4.32}                    & 0.32                  & \textbf{7.08}                    & 0.25                   & \textbf{6.94}                    & 0.24                   & \textbf{6.96}                    & 0.24                   & \textbf{13.86}                   & 0.37                   & \textbf{13.56}                   & 0.35                   & \textbf{14.38}                   & 0.47                   \\
\textbf{SMR}                                 & \textbf{-6.93}                    & 1                     & \textbf{-7.05}                    & 1.06                  & \textbf{-6.33}                    & 0.61                  & \textbf{2.89}                    & 0.2                    & \textbf{2.67}                    & 0.2                    & \textbf{3}                       & 0.2                    & \textbf{8.64}                    & 0.2                    & \textbf{8.59}                    & 0.23                   & \textbf{8.35}                    & 0.2                    \\
\textbf{STP}                                 & \textbf{-15.44}                   & 0.93                  & \textbf{-15.51}                   & 0.9                   & \textbf{-14.62}                   & 0.46                  & \textbf{0.96}                    & 0.45                   & \textbf{0.65}                    & 0.48                   & \textbf{0.8}                     & 0.48                   & \textbf{8.66}                    & 0.31                   & \textbf{8.73}                    & 0.33                   & \textbf{8.76}                    & 0.3                    \\
\textbf{SAU}                                 & \textbf{-3.7}                     & 0.35                  & \textbf{-4.54}                    & 0.94                  & \textbf{-3.98}                    & 0.94                  & \textbf{4.48}                    & 0.11                   & \textbf{4.4}                     & 0.14                   & \textbf{4.6}                     & 0.12                   & \textbf{9.97}                    & 0.33                   & \textbf{10.11}                   & 0.33                   & \textbf{10.28}                   & 0.33                   \\
\textbf{SEN}                                 & \textbf{-5.13}                    & 0.37                  & \textbf{-5.38}                    & 0.65                  & \textbf{-5.55}                    & 0.64                  & \textbf{4.16}                    & 0.23                   & \textbf{3.75}                    & 0.25                   & \textbf{3.89}                    & 0.23                   & \textbf{14.31}                   & 0.48                   & \textbf{14.32}                   & 0.48                   & \textbf{14.22}                   & 0.49                   \\
\textbf{SRB}                                 & \textbf{-4.11}                    & 0.33                  & \textbf{-4.9}                     & 0.81                  & \textbf{-4.09}                    & 0.43                  & \textbf{3.53}                    & 0.11                   & \textbf{3.3}                     & 0.11                   & \textbf{3.67}                    & 0.11                   & \textbf{9.03}                    & 0.29                   & \textbf{8.85}                    & 0.3                    & \textbf{8.79}                    & 0.31                   \\
\textbf{SYC}                                 & \textbf{-5.13}                    & 0.51                  & \textbf{-4.86}                    & 0.28                  & \textbf{-4.66}                    & 0.3                   & \textbf{4.21}                    & 0.26                   & \textbf{4.11}                    & 0.25                   & \textbf{4.12}                    & 0.23                   & \textbf{9.31}                    & 0.25                   & \textbf{9.24}                    & 0.24                   & \textbf{9.27}                    & 0.24                   \\
\textbf{SLE}                                 & \textbf{-8.07}                    & 0.4                   & \textbf{-8.14}                    & 0.3                   & \textbf{-8.14}                    & 0.37                  & \textbf{1.45}                    & 0.21                   & \textbf{1.31}                    & 0.15                   & \textbf{1.47}                    & 0.19                   & \textbf{11.85}                   & 0.43                   & \textbf{11.38}                   & 0.43                   & \textbf{11.57}                   & 0.44                   \\
\textbf{SGP}                                 & \textbf{-8.34}                    & 0.41                  & \textbf{-7.81}                    & 0.41                  & \textbf{-7.84}                    & 0.76                  & \textbf{5.55}                    & 0.31                   & \textbf{17.01}                   & 1.23                   & \textbf{5.85}                    & 0.28                   & \textbf{26.95}                   & 0.91                   & \textbf{27.26}                   & 0.86                   & \textbf{27.9}                    & 0.87                   \\
\textbf{SXM}                                 & \textbf{-3.93}                    & 0.25                  & \textbf{-4.39}                    & 0.4                   & \textbf{-4.58}                    & 0.49                  & \textbf{4.89}                    & 0.24                   & \textbf{4.72}                    & 0.24                   & \textbf{4.62}                    & 0.25                   & \textbf{15.07}                   & 0.45                   & \textbf{14.81}                   & 0.43                   & \textbf{14.65}                   & 0.43                   \\
\textbf{SVK}                                 & \textbf{-4.31}                    & 0.46                  & \textbf{-5.65}                    & 0.97                  & \textbf{-5.28}                    & 0.82                  & \textbf{1.62}                    & 0.06                   & \textbf{1.18}                    & 0.06                   & \textbf{1.4}                     & 0.08                   & \textbf{8.79}                    & 0.26                   & \textbf{8.42}                    & 0.24                   & \textbf{8.43}                    & 0.26                   \\
\textbf{SVN}                                 & \textbf{-6.74}                    & 0.48                  & \textbf{-7.8}                     & 0.64                  & \textbf{-7.27}                    & 0.34                  & \textbf{3.91}                    & 0.26                   & \textbf{3.57}                    & 0.3                    & \textbf{3.84}                    & 0.3                    & \textbf{11.62}                   & 0.37                   & \textbf{11.26}                   & 0.36                   & \textbf{11.56}                   & 0.37                   \\
\textbf{SLB}                                 & \textbf{-9.67}                    & 0.31                  & \textbf{-9.71}                    & 0.27                  & \textbf{-9.5}                     & 0.29                  & \textbf{2.66}                    & 0.36                   & \textbf{2.68}                    & 0.32                   & \textbf{2.69}                    & 0.3                    & \textbf{8.6}                     & 0.22                   & \textbf{8.61}                    & 0.23                   & \textbf{8.5}                     & 0.22                   \\
\textbf{SOM}                                 & \textbf{-18.12}                   & 0.48                  & \textbf{-18.25}                   & 0.5                   & \textbf{-17.41}                   & 0.51                  & \textbf{3.22}                    & 0.68                   & \textbf{3.37}                    & 0.65                   & \textbf{3.64}                    & 0.7                    & \textbf{11.14}                   & 0.39                   & \textbf{11.12}                   & 0.35                   & \textbf{11.34}                   & 0.38                   \\
\textbf{ZAF}                                 & \textbf{-3.84}                    & 0.32                  & \textbf{-3.83}                    & 0.27                  & \textbf{-3.89}                    & 0.44                  & \textbf{6.91}                    & 0.22                   & \textbf{6.92}                    & 0.22                   & \textbf{6.92}                    & 0.24                   & \textbf{8.98}                    & 0.18                   & \textbf{8.97}                    & 0.17                   & \textbf{8.89}                    & 0.17                   \\
\textbf{SSD}                                 & \textbf{-4.48}                    & 0.64                  & \textbf{-4.34}                    & 0.37                  & \textbf{-4.45}                    & 0.67                  & \textbf{3.28}                    & 0.16                   & \textbf{3.27}                    & 0.15                   & \textbf{3.34}                    & 0.16                   & \textbf{8.45}                    & 0.23                   & \textbf{8.85}                    & 0.25                   & \textbf{8.66}                    & 0.22                   \\
\textbf{LKA}                                 & \textbf{-6.32}                    & 0.29                  & \textbf{-6.44}                    & 0.41                  & \textbf{-6.21}                    & 0.63                  & \textbf{1.77}                    & 0.18                   & \textbf{1.55}                    & 0.21                   & \textbf{1.61}                    & 0.14                   & \textbf{8.59}                    & 0.27                   & \textbf{8.45}                    & 0.26                   & \textbf{8.56}                    & 0.26                   \\
\textbf{KNA}                                 & \textbf{-3.61}                    & 0.35                  & \textbf{-4.15}                    & 0.4                   & \textbf{-3.97}                    & 0.28                  & \textbf{5.48}                    & 0.2                    & \textbf{5.32}                    & 0.23                   & \textbf{5.23}                    & 0.2                    & \textbf{11.62}                   & 0.32                   & \textbf{11.66}                   & 0.3                    & \textbf{11.29}                   & 0.28                   \\
\textbf{LCA}                                 & \textbf{-6.34}                    & 0.41                  & \textbf{-6.04}                    & 0.41                  & \textbf{-6.87}                    & 0.58                  & \textbf{2.84}                    & 0.21                   & \textbf{2.85}                    & 0.21                   & \textbf{2.89}                    & 0.23                   & \textbf{12.49}                   & 0.45                   & \textbf{12.68}                   & 0.43                   & \textbf{12.73}                   & 0.49                   \\
\textbf{MAF}                                 & \textbf{-5.07}                    & 0.29                  & \textbf{-5.59}                    & 0.42                  & \textbf{-5.67}                    & 0.5                   & \textbf{1.27}                    & 0.2                    & \textbf{1.13}                    & 0.2                    & \textbf{1.09}                    & 0.19                   & \textbf{8.62}                    & 0.27                   & \textbf{8.69}                    & 0.28                   & \textbf{8.59}                    & 0.26                   \\
\textbf{SDN}                                 & \textbf{-6.42}                    & 0.37                  & \textbf{-6.53}                    & 0.39                  & \textbf{-6.4}                     & 0.3                   & \textbf{1.43}                    & 0.15                   & \textbf{1.24}                    & 0.14                   & \textbf{1.45}                    & 0.17                   & \textbf{8.45}                    & 0.25                   & \textbf{8.83}                    & 0.26                   & \textbf{8.6}                     & 0.25                   \\
\textbf{SUR}                                 & \textbf{-18.98}                   & 0.65                  & \textbf{-19.56}                   & 0.9                   & \textbf{-19.02}                   & 0.71                  & \textbf{3.8}                     & 0.67                   & \textbf{3.73}                    & 0.66                   & \textbf{3.44}                    & 0.68                   & \textbf{9.82}                    & 0.32                   & \textbf{9.9}                     & 0.31                   & \textbf{9.7}                     & 0.31                   \\
\textbf{SWE}                                 & \textbf{-4.33}                    & 0.36                  & \textbf{-3.85}                    & 0.23                  & \textbf{-4.89}                    & 0.62                  & \textbf{3.76}                    & 0.12                   & \textbf{4}                       & 0.13                   & \textbf{3.75}                    & 0.12                   & \textbf{9.68}                    & 0.33                   & \textbf{9.97}                    & 0.3                    & \textbf{9.91}                    & 0.3                    \\
\textbf{CHE}                                 & \textbf{-7.84}                    & 0.41                  & \textbf{-8.03}                    & 0.41                  & \textbf{-7.9}                     & 0.45                  & \textbf{2.06}                    & 0.24                   & \textbf{2.25}                    & 0.2                    & \textbf{2.22}                    & 0.22                   & \textbf{8.41}                    & 0.24                   & \textbf{8.41}                    & 0.21                   & \textbf{8.61}                    & 0.25                   \\
\textbf{SYR}                                 & \textbf{-4.26}                    & 0.24                  & \textbf{-3.88}                    & 0.26                  & \textbf{-4.04}                    & 0.25                  & \textbf{1.82}                    & 0.17                   & \textbf{1.99}                    & 0.17                   & \textbf{1.92}                    & 0.17                   & \textbf{8.44}                    & 0.26                   & \textbf{8.58}                    & 0.26                   & \textbf{8.57}                    & 0.27                   \\
\textbf{TZA}                                 & \textbf{-3.7}                     & 0.24                  & \textbf{-3.67}                    & 0.28                  & \textbf{-4.06}                    & 0.27                  & \textbf{7.06}                    & 0.2                    & \textbf{7.01}                    & 0.21                   & \textbf{6.99}                    & 0.22                   & \textbf{13.28}                   & 0.32                   & \textbf{13.54}                   & 0.39                   & \textbf{12.81}                   & 0.32                   \\
\textbf{THA}                                 & \textbf{-3.98}                    & 0.64                  & \textbf{-4.24}                    & 0.42                  & \textbf{-3.85}                    & 0.31                  & \textbf{6.46}                    & 0.22                   & \textbf{6.23}                    & 0.2                    & \textbf{6.49}                    & 0.23                   & \textbf{11.27}                   & 0.31                   & \textbf{11.25}                   & 0.34                   & \textbf{11.18}                   & 0.33                   \\
\textbf{TLS}                                 & \textbf{-4.85}                    & 0.6                   & \textbf{-4.48}                    & 0.64                  & \textbf{-4.25}                    & 0.45                  & \textbf{3.91}                    & 0.2                    & \textbf{4.13}                    & 0.13                   & \textbf{3.76}                    & 0.14                   & \textbf{10.44}                   & 0.39                   & \textbf{10.53}                   & 0.38                   & \textbf{10.46}                   & 0.38                   \\
\textbf{TGO}                                 & \textbf{-7.98}                    & 0.4                   & \textbf{-8.25}                    & 0.37                  & \textbf{-8.25}                    & 0.38                  & \textbf{3.91}                    & 0.36                   & \textbf{3.76}                    & 0.35                   & \textbf{3.88}                    & 0.35                   & \textbf{16.68}                   & 0.51                   & \textbf{16.4}                    & 0.46                   & \textbf{16.44}                   & 0.51                   \\
\textbf{TON}                                 & \textbf{-6.11}                    & 0.38                  & \textbf{-6.53}                    & 0.41                  & \textbf{-6.24}                    & 0.39                  & \textbf{3.8}                     & 0.25                   & \textbf{3.87}                    & 0.26                   & \textbf{3.89}                    & 0.26                   & \textbf{8.88}                    & 0.21                   & \textbf{8.95}                    & 0.22                   & \textbf{9.01}                    & 0.24                   \\
\textbf{TTO}                                 & \textbf{-7.21}                    & 0.38                  & \textbf{-7.65}                    & 0.6                   & \textbf{-7.23}                    & 0.35                  & \textbf{1.13}                    & 0.18                   & \textbf{1.17}                    & 0.17                   & \textbf{1.35}                    & 0.19                   & \textbf{8.71}                    & 0.29                   & \textbf{8.88}                    & 0.3                    & \textbf{8.85}                    & 0.32                   \\
\textbf{TUN}                                 & \textbf{-6.26}                    & 0.41                  & \textbf{-5.98}                    & 0.29                  & \textbf{-6.31}                    & 0.42                  & \textbf{3.77}                    & 0.24                   & \textbf{3.69}                    & 0.27                   & \textbf{3.8}                     & 0.24                   & \textbf{14.99}                   & 0.45                   & \textbf{15.25}                   & 0.51                   & \textbf{14.95}                   & 0.49                   \\
\textbf{TUR}                                 & \textbf{-4.72}                    & 0.46                  & \textbf{-4.42}                    & 0.42                  & \textbf{-4.11}                    & 0.41                  & \textbf{3.42}                    & 0.15                   & \textbf{3.73}                    & 0.17                   & \textbf{3.62}                    & 0.18                   & \textbf{8.88}                    & 0.28                   & \textbf{9.07}                    & 0.32                   & \textbf{8.99}                    & 0.36                   \\
\textbf{TKM}                                 & \textbf{-8.32}                    & 0.39                  & \textbf{-8.39}                    & 0.41                  & \textbf{-7.57}                    & 0.42                  & \textbf{5.77}                    & 0.41                   & \textbf{5.81}                    & 0.38                   & \textbf{6.08}                    & 0.41                   & \textbf{12.16}                   & 0.36                   & \textbf{12.21}                   & 0.33                   & \textbf{12.49}                   & 0.36                   \\
\textbf{TCA}                                 & \textbf{-3.83}                    & 0.25                  & \textbf{-3.78}                    & 0.26                  & \textbf{-4.88}                    & 0.99                  & \textbf{7.07}                    & 0.18                   & \textbf{6.84}                    & 0.25                   & \textbf{6.84}                    & 0.23                   & \textbf{16.08}                   & 0.38                   & \textbf{15.99}                   & 0.39                   & \textbf{15.75}                   & 0.38                   \\
\textbf{TUV}                                 & \textbf{-5.21}                    & 0.53                  & \textbf{-5.23}                    & 0.44                  & \textbf{-4.82}                    & 0.46                  & \textbf{1.29}                    & 0.16                   & \textbf{1.01}                    & 0.17                   & \textbf{1.25}                    & 0.15                   & \textbf{9.37}                    & 0.32                   & \textbf{9.72}                    & 0.38                   & \textbf{9.92}                    & 0.43                   \\
\textbf{UGA}                                 & \textbf{-8.88}                    & 0.4                   & \textbf{-8.84}                    & 0.36                  & \textbf{-9.68}                    & 0.86                  & \textbf{1.08}                    & 0.24                   & \textbf{0.99}                    & 0.23                   & \textbf{1.21}                    & 0.22                   & \textbf{11.34}                   & 0.48                   & \textbf{11.34}                   & 0.44                   & \textbf{11.44}                   & 0.47                   \\
\textbf{UKR}                                 & \textbf{-4.24}                    & 0.32                  & \textbf{-4}                       & 0.42                  & \textbf{-3.96}                    & 0.38                  & \textbf{6.29}                    & 0.19                   & \textbf{6.52}                    & 0.2                    & \textbf{6.37}                    & 0.2                    & \textbf{12}                      & 0.3                    & \textbf{11.87}                   & 0.31                   & \textbf{11.98}                   & 0.34                   \\
\textbf{ARE}                                 & \textbf{-17.46}                   & 0.35                  & \textbf{-17.65}                   & 0.3                   & \textbf{-17.71}                   & 0.34                  & \textbf{3.6}                     & 0.69                   & \textbf{3.67}                    & 0.65                   & \textbf{3.94}                    & 0.66                   & \textbf{12.27}                   & 0.36                   & \textbf{12.48}                   & 0.34                   & \textbf{12.37}                   & 0.35                   \\
\textbf{GBR}                                 & \textbf{-6.73}                    & 0.39                  & \textbf{-6.59}                    & 0.27                  & \textbf{-6.37}                    & 0.3                   & \textbf{4.01}                    & 0.25                   & \textbf{3.95}                    & 0.24                   & \textbf{4.09}                    & 0.23                   & \textbf{12.98}                   & 0.38                   & \textbf{13.07}                   & 0.42                   & \textbf{12.99}                   & 0.39                   \\
\textbf{USA}                                 & \textbf{-5.48}                    & 0.26                  & \textbf{-5.48}                    & 0.28                  & \textbf{-6.07}                    & 0.37                  & \textbf{1.64}                    & 0.22                   & \textbf{1.52}                    & 0.23                   & \textbf{1.42}                    & 0.21                   & \textbf{8.65}                    & 0.25                   & \textbf{8.71}                    & 0.25                   & \textbf{8.57}                    & 0.25                   \\
\textbf{URY}                                 & \textbf{-5.36}                    & 0.76                  & \textbf{-4.51}                    & 0.31                  & \textbf{-4.57}                    & 0.5                   & \textbf{1.8}                     & 0.12                   & \textbf{1.76}                    & 0.11                   & \textbf{1.6}                     & 0.11                   & \textbf{8.32}                    & 0.24                   & \textbf{8.68}                    & 0.24                   & \textbf{8.39}                    & 0.25                   \\
\textbf{UZB}                                 & \textbf{-9.6}                     & 0.4                   & \textbf{-9.35}                    & 0.39                  & \textbf{-9.42}                    & 0.98                  & \textbf{3.51}                    & 0.28                   & \textbf{3.58}                    & 0.25                   & \textbf{3.77}                    & 0.29                   & \textbf{10.49}                   & 0.37                   & \textbf{10.54}                   & 0.32                   & \textbf{10.81}                   & 0.35                   \\
\textbf{VUT}                                 & \textbf{-3.92}                    & 0.4                   & \textbf{-4.4}                     & 0.42                  & \textbf{-3.93}                    & 0.41                  & \textbf{7}                       & 0.25                   & \textbf{6.97}                    & 0.2                    & \textbf{6.91}                    & 0.2                    & \textbf{11.14}                   & 0.27                   & \textbf{11.21}                   & 0.29                   & \textbf{11.17}                   & 0.28                   \\
\textbf{VEN}                                 & \textbf{-6.77}                    & 0.39                  & \textbf{-6.77}                    & 0.42                  & \textbf{-7.1}                     & 0.39                  & \textbf{3}                       & 0.29                   & \textbf{2.8}                     & 0.24                   & \textbf{2.91}                    & 0.31                   & \textbf{8.96}                    & 0.26                   & \textbf{8.96}                    & 0.26                   & \textbf{9.22}                    & 0.27                   \\
\textbf{VIR}                                 & \textbf{-3.97}                    & 0.39                  & \textbf{-4.2}                     & 0.42                  & \textbf{-4.17}                    & 0.42                  & \textbf{6.29}                    & 0.18                   & \textbf{6.19}                    & 0.2                    & \textbf{6.43}                    & 0.19                   & \textbf{8.75}                    & 0.15                   & \textbf{8.92}                    & 0.18                   & \textbf{8.83}                    & 0.18                   \\
\textbf{YEM}                                 & \textbf{-16.61}                   & 0.37                  & \textbf{-16.91}                   & 0.44                  & \textbf{-16.37}                   & 0.48                  & \textbf{4.19}                    & 0.7                    & \textbf{3.94}                    & 0.67                   & \textbf{4.22}                    & 0.68                   & \textbf{21.35}                   & 0.75                   & \textbf{21.74}                   & 0.75                   & \textbf{21.82}                   & 0.77                   \\
\textbf{ZWE}                                 & \textbf{-4.14}                    & 0.44                  & \textbf{-4.44}                    & 0.46                  & \textbf{-4}                       & 0.43                  & \textbf{6.87}                    & 0.2                    & \textbf{6.52}                    & 0.23                   & \textbf{6.73}                    & 0.23                   & \textbf{11.33}                   & 0.32                   & \textbf{11.21}                   & 0.3                    & \textbf{11.21}                   & 0.33                   \\ \hline
\caption{Mean (in bold) and standard error of the quantile estimates at level $\tau=0.01, 0.5, 0.99$ for the SSP5 scenario.}
\label{tab:se_table_SSP5}\\
\end{longtable}
\end{landscape}


\chapter{The Impact of the COVID-19 Pandemic on Risk Factors for Children’s Mental Health}\label{app:SDQ}

This section reports additional Figures from Chapter \ref{ch:SDQ}.



\begin{figure}[h]
    \centering
\includegraphics[width=0.75\textwidth]{SDQ/images/n_children.png}
    \caption{Bar plot showing the number of children interviewed 1, 2, 3 or 4 times during the sample period. }
    \label{fig:n_children}
\end{figure}

\begin{figure}[h]
    \centering
\includegraphics[width=0.93\textwidth]{SDQ/images/quantile025.pdf}
    \caption{Bar plot showing the Variable Importance extracted from the FM-QRF for each covariate at quantile level $\tau=0.25$. }
    \label{fig:quantile025}
\end{figure}

\begin{figure}[h]
    \centering
\includegraphics[width=0.93\textwidth]{SDQ/images/quantile075.pdf}
    \caption{Bar plot showing the Variable Importance extracted from the FM-QRF for each covariate at quantile level $\tau=0.75$.}
    \label{fig:quantile075}
\end{figure}




