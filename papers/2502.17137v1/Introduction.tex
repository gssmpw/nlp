
\chapter{Introduction and Overview}
\thispagestyle{plain}
Quantile Regression (QR) has been introduced in \cite{koenker1978regression} as a powerful technique to model the entire conditional distribution of a response variable given a set of covariates. This approach is particularly useful when the standard regression models fail at correctly modeling the relationship between the response and the covariates, or when the gaussianity assumption on the outcome's distribution is untanable. In such scenarios, QR allows to obtain more robust and reliable results by modeling location parameters beyond the conditional mean, and a variety of fields, such as economics, finance, healthcare, and environmental science \citep{koenker2005quantile, koenker2017handbook}, have reaped the benefits of this approach. 
\vspace{0.15in}

\noindent The recent development of non-parametric QR models have further extended the applications of QR, including the machine learning realm. In this context, QR machine learning algorithms represent one of the main advancements in overcoming the limits of the parametric formulation of standard QR models. As a matter of fact, machine learning algorithms do not require any a-priori assumption on the functional form of the relationship between the outcome and the covariates, resulting more flexible and reliable than standard QR in empirical applications involving unknown and highly complex relationships among variables.

\vspace{0.15in}

\noindent Few examples of QR machine learning algorithms are the QR neural network model of \cite{white1992nonparametric}, QR Support Vector Machines \citep{hwang2005simple, xu2015weighted} and QR Random Forests \citep{meinshausen2006quantile, athey2019generalized}.
\vspace{0.15in}

\noindent Despite the benefits of these algorithms, their application is constrained to standard experimental designs, as they often falter in non-standard empirical settings, such as those involving mixed-frequency or longitudinal data.
\vspace{0.15in}

\noindent The former case is particularly relevant in time series analysis, in which information is often available at different temporal resolutions. Standard statistical and econometrics models, including QR, usually require the use of variables observed at the same frequency, causing potentially useful predictors to be excluded due to the temporal mismatch. One of the main contributions to address this issue is the Mixed Data Sampling (MIDAS) approach proposed in \cite{ghysels2007midas}. This model allows to include variables observed at different frequencies, allowing to expand the research methodology beyond the standard statistical setting. 

\vspace{0.15in}

\noindent Given its innovativity, the MIDAS approach has been extended to QR and applied in a variety of fields, such as finance and economics \citep{kuzin2009midas, andreani2021multivariate, candila2023mixed}, environmental sciences \citep{oloko2022climate, jiang2023carbon} and tourism \citep{bangwayo2015can, wen2021forecasting}.

\vspace{0.15in}

\noindent In the longitudinal data setting instead, previous contributions to the QR literature have exploited mixed-effects or random effects models to account for the potential association between dependent observations \citep{farcomeni2012quantile, smith2015multilevel, alfo2017finite, marino2018mixed, merlo2021forecasting,merlo2022quantile,merlo2022quantilets}. Although these models incorporate an individual-specific random intercept to model unobserved heterogeneity, their parametric formulation may not be suitable in various empirical applications, leading to potentially inaccurate inferences.

\vspace{0.15in}

\noindent Non-parametric approaches have been proposed, but in the machine learning realm, the algorithms that have been extended to handle mixed-frequency data \citep{xu2019artificial} and longitudinal data \citep{xiong2019mixed, luts2012mixed, hajjem2014mixed, hajjem2011mixed, sela2012re}  lack the capability to estimate conditional quantiles, as they have been developed only in a standard regression setting.

\vspace{0.15in}

\noindent Thus, the aim of this dissertation is to bridge this gap in the literature by introducing two novel machine learning algorithms that allow to estimate conditional quantiles in the mixed-frequency and longitudinal data frameworks.

\vspace{0.15in}

\noindent Both proposed models build upon the Quantile Regression Forest (QRF) algorithm, ensuring a high level of accuracy and computational efficiency. The choice of the QRF algorithm stems from its inherent flexibility, allowing for easy comparison with standard econometric models in terms of statistical adequacy, accuracy, interpretability, and computational effort. Empirical results presented in this thesis demonstrate that the proposed extensions outperform commonly used econometric and statistical models in the QR literature.

\vspace{0.15in}

\noindent The main contributions of this dissertation are twofold. Methodologically, it introduces two algorithms: the Mixed-Frequency Quantile Regression Forest (MIDAS-QRF, detailed in Chapter \ref{ch:MIDAS-QRF}) and the Finite Mixture Quantile Regression Forest (FM-QRF, presented in Chapter \ref{ch:FM-QRF}). The MIDAS-QRF is based on a novel methodology that merges the MIDAS approach and the QRF algorithm, enabling non-parametric estimation of quantiles using data observed at different frequencies. The FM-QRF, on the other hand, builds upon random effects machine learning algorithms and it is based on leaving the random effects distribution unspecified and on estimating the fixed part of the model with QRF. Quantile estimates are obtained using an iterative procedure based on the Expectation Maximization-type algorithm with the Asymmetric Laplace distribution as the working likelihood. This methodology extends the work of \cite{geraci2007quantile, geraci2014linear, alfo2017finite} to a non-linear and non-parametric framework and adapts the mixed-modeling approach presented in \cite{hajjem2014mixed} to a QR framework.

\vspace{0.15in}

\noindent The validity of the proposed models has been tested empirically in the financial and economics settings. As a matter of fact, the recent emerging risks concerning financial crises and climate change have required financial institutions, policymakers, and researchers to develop novel methodological approaches to capture complex relationships and manage such risks.

\vspace{0.15in}

\noindent The two innovative methodological approaches result particularly useful in this settings, where non-Gaussian characteristics and mixed-frequency or longitudinal data are common. The MIDAS-QRF is empirically applied in a financial risk management setting for computing the well-known financial risk measure Value-at-Risk. Empirical findings demonstrate that the MIDAS-QRF delivers statistically adequate forecasts that outperform popular models in terms of accuracy (refer to Section \ref{sec:MIDAS-QRF-empirical}).
\vspace{0.15in}


\noindent The FM-QRF is applied in a climate-change impact evaluation setting to predict the Growth-at-Risk (GaR) of GDP growth for 210 countries. Climate-related variables are used as covariates, revealing heterogeneous effects of unsustainable climate practices among countries. 
\vspace{0.15in}

\noindent In order to test the flexibility of the proposed FM-QRF, the model is applied on an additional longitudinal dataset concerning the effects of the COVID-19 pandemic on children's mental health to extend previous findings based on standard linear models (refer to Section \ref{sec:FM-QRF-empirical}).