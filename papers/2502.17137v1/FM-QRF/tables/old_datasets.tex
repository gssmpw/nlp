\section{Case studies}

In this section, the results of the application of the ME-QRF to data from two benchmark case studies are shown. In the first case study, repeated measures data on labour pain \citep{geraci2007quantile, farcomeni2012quantile} are analysed.  The second case study concerns the placebo-controlled, double-blind, randomised trial conducted by the Treatment of Lead- Exposed Children (TLC) Trial Group.
The performance of the ME-QRF model is compared to the Linear Quantile Mixed Effects model of \cite{geraci2014linear} in terms of quantile loss \citep{gonzalez2004forecasting}.



\subsection{Case study 1: Labor pain data}

This data set consists of a maximum of six repeated measures of self-reported amount of pain for 83 women in labor, 43 of them randomly assigned to a pain medication group and 40 to a placebo group. In total, the training set is composed of 357 observations.  The outcome variable is the amount of pain measured every 30 min on a 100-mm line, where 0 means no pain and 100 means extreme pain. The set of explanatory variables is composed of the Treatment dummy variable (treatment=1, placebo=0) and the measurement occasion Time, with the category (Time = 0) being been chosen as reference. These feature of this data is that it is severely skewed, with a skewness value that changes magnitude and sign over time.

The quantile is computed with the ME-QRF at three different probability levels, $\tau= 0.25, 0.5, 0.75$. The model considers treatment group, time and the interaction between time and treatment group as fixed effects, whereas the random part of the model includes a subject-specific intercept.
The formulation of the LQMM model is:

\begin{equation}
 M_{it}=\beta_1 R_{it} + \beta_2 t_{it} +\beta_3 R_{it} * t_{it} + b_i
\end{equation}

The results are presented in Table \ref{tab:painlabor}. Numbers in bold indicate the lowest quantile loss value for each probability level.

\begin{table}[H]
\centering
\begin{tabular}{@{}ccc@{}}
\toprule
\cellcolor[HTML]{FFFFFF}{\color[HTML]{333333} $\tau$} & \textbf{ME-QRF} & \textbf{LQMM} \\ \midrule
0.25                                                  & \textbf{4.29}        & 4.56       \\
0.5                                                   & 5.84        & \textbf{5.56}    \\
0.75                                                  & \textbf{5.23}    & 5.28      \\ \bottomrule
\end{tabular}
\caption{Results in terms of Quantile Loss for the Labor Pain dataset}
\label{tab:painlabor}
\end{table}

Results show that  outperforms the LQMM one for $\tau=0.25, 0.75$. However, due to the high grade of skewness of his dataset, it performs worse than the LQMM model in predicting the median. This might be due to the low number of observation and covariates of the dataset, that do not allow the QRF algorithm to model the skewness of the data.


\subsection{Case study 2: Data on treatment of lead-exposed children}

This paragraph concerns a dataset from a placebo-controlled, double-blind, randomised trial to study the effect of succimer (a chelating agent) on children with blood lead levels (BLL) of 20–44 $\mu$g/dL. This trial study was conducted by the Treatment of Lead-Exposed Children (TLC) Trial Group (2000) in order to assess whether the succimer treatment of children with blood lead levels $<$ 45$\mu$g/dL is beneficial and safe.

The dataset includes 4 weekly measurements of BLL for 100 children as outcome variable. The covariates are the dummy variable Treatment taking value 1 for children that have been treated and 0 otherwise, and Time, with the category (Time = 0) being
been chosen as reference.
The ME-QRF model with a subject-specific random intercept has been compared with the a LQMM with the following formulation: 

\begin{equation}
    M_{it}=\beta_1 R_{it} + \beta_2 t^2_{it}+ \beta_3 R_{it}*t^2_{it} + \beta_4 R_{it} * t_{it} + b_i
\end{equation}

where $M$ represents the measurement variable, $R$ the tratment dummy variable and $t$ the time variable.

The results are presented in Table \ref{tab:child}.


\begin{table}[H]
\centering
\begin{tabular}{@{}ccc@{}}
\toprule
\cellcolor[HTML]{FFFFFF}{\color[HTML]{333333} $\tau$} & \multicolumn{1}{c}{\textbf{ME-QRF}} & \multicolumn{1}{c}{\textbf{LQMM}} \\ \midrule
0.25                                                  & \textbf{1.22}                            & 1.61                          \\
0.5                                                   & \textbf{1.55}                             & 1.82                           \\
0.75                                                  & \textbf{1.31}                            & 1.59                          \\ \bottomrule
\end{tabular}
\caption{Results in terms of Quantile Loss for the treatment of lead-exposed children dataset}
\label{tab:child}
\end{table}

Results show that the ME-QRF model outperforms the LQMM at each quantile level.
Differently from the skewed dataset of the former paragraph, in this case the dataset is characterised by a lower skewness and the ME-QRF model outperforms the LQMM model also in predicting the median.
In Figure \ref{fig:TLCtrajectories} the estimated trajectories for the three levels of quantile are shown for each group identified with the mixtures and separately for the treatment and the control group.
The trajectories estimated with the ME-QRF model are coherent with the finding of \cite{alfo2017finite}.
\newpage


\begin{figure}[H]
     \begin{center}
%
        \subfigure{%
            \label{fig:first}
            \includegraphics[width=0.4\textwidth]{Latex/Templates/0.25_1-5.png}
        }%
        \subfigure{%
           \label{fig:second}
           \includegraphics[width=0.4\textwidth]{Latex/Templates/0.25_6-10.png}
        }\\ %  ------- End of the first row ----------------------%
        \subfigure{%
            \label{fig:third}
            \includegraphics[width=0.4\textwidth]{Latex/Templates/0.5_1-5.png}
        }%
        \subfigure{%
            \label{fig:fourth}
            \includegraphics[width=0.4\textwidth]{Latex/Templates/0.5_6-10.png}
        } \\%------- End of the second row ----------------------%
\subfigure{%
            \label{fig:third}
            \includegraphics[width=0.4\textwidth]{Latex/Templates/0.75_1-5.png}
        }%
        \subfigure{%
            \label{fig:fourth}
            \includegraphics[width=0.48\textwidth]{Latex/Templates/0.75_6-10.png}
        }
    \end{center}

    \caption{%
        TLC data. Estimated trajectories for ME-QRF for each mixture component and represented separately for the treatment (solid curves) and the control (dashed lines) groups. Lines are colour coded according to the mixture component. The number of units belonging to each mixture component is reported in the legend.
     }%
   \label{fig:TLCtrajectories}
\end{figure}

\newpage


%\subsection{Case study 3: Non-linear effects of climate change on economic productivity}

%In this paragraph the ME-QRF model is tested on the dataset of \cite{burke2015global} in order to study the effect of temperature and precipitation on economic productivity of 166 countries. In particular, the outcome variables is the first difference of the natural logarithm of annual real (inflation-adjusted) GDP per capita. The covariates are 0.5 degree gridded yearly temperature and precipitation weighted by population density of each country in the year 2000. As in \cite{burke2015global}, also the squared of both temperature and precipitation variables are included in the covariates set.

%The full dataset contains 6584 country-year observations between the years 1960-2010.

%The ME-QRF is used to compute quantiles at five different probability levels $\tau=0.01, 0.05, 0.25, 0.5, 0.75$ and its performance is compared to the one of a simple LQMM in terms of quantile loss. In particular, the quantile loss is computed on the entire set of in-sample data. The LQMM that has been considered in the analysis is:

%$$Y_{it}=\beta_1T_{it}+\beta_2T^2_{it}+\beta_3P_{it}+\beta_4P^2_{it}+\mathbf{b}_i $$

%where $Y_{it}$ represents the GDP growth of the $i$-th country in the $t$-th year and $T$ and $P$ represent the temperature and precipitation variables, respectively. \textcolor{red}{Thus, in this application, the random effect $\mathbf{b}_i$ models country-based clustering.}
%Table \ref{tab:resburke} reports the average results in terms of in-sample quantile loss across all countries of the two models for the five quantiles:

% Please add the following required packages to your document preamble:
% \usepackage{booktabs}
% \usepackage{graphicx}
%\begin{table}[H]
%\centering
%\small
%\begin{tabular}{@{}cccc@{}}
%\toprule
%$\tau$ & \textbf{ME-QRF} & \textbf{LQMM} & \textbf{Loss ME-QRF/Loss LQMM} \\ \midrule
%0.01   & \textbf{2.31}   & 2.91          & 80\%                           \\
%0.05   & \textbf{7.99}   & 8.87          & 90\%                           \\
%0.25   & \textbf{15.16}  & 18.31         & 83\%                           \\
%0.5    & \textbf{17.18}  & 18.06         & 95\%                           \\
%0.75   & \textbf{13.43}  & 15.96         & 84\%                           \\ \bottomrule
%\end{tabular}%

%\caption{\small Results in terms of Quantile Loss for the economic output and climate change dataset. Values have been multiplied by 1000. The third column shows the ratio between the ME-QRF and the LQMM quantile losses; values smaller than 100\% indicate that the ME-QRF model outperforms the LQMM model.}
%\label{tab:resburke}
%\end{table}

%Results show that the ME-QRF model significantly outperforms the LQMM, delivering a a maximum of 20\% increase of accuracy at the most extreme quantile ($\tau=0.01$).

%Moreover, the importance of each covariate is measured for each quantile. Results are shown in Figure \ref{fig:varimp} and Table \ref{tab:meanstdev}. The variable importance, measured as proposed in \cite{breiman2001random}, gives information regarding the predictive power of each covariate by measuring how much the prediction accuracy decreases when such covariate is excluded when fitting the model. In other words, if a variable is important, then it is expected that, after permuting the values of the variable, the model’s performance will decrease. The larger the change in the performance, the more relevant (or "important") the variable is.


% \begin{figure}[H]
%     \centering
%     \includegraphics[width=0.6\textwidth]{varimp}
%     \caption{Variable Importance values for each variable across quantiles. Each cluster represents the variable importance at different quantile levels.}
%     \label{fig:varimp}
% \end{figure}


% \begin{table}[H]
% \centering
% \resizebox{0.3\textwidth}{!}{%
% \begin{tabular}{@{}ccccc@{}}
% \toprule
%                   & \textbf{$P^2$} & \textbf{$P$} & \textbf{$T^2$} & \textbf{$T$} \\ \midrule
% \textbf{Mean}     & 2.186          & 2.214        & 1.949          & 1.957        \\
% \textbf{St. Dev.} & 0.328          & 0.334        & 0.205          & 0.203        \\ \bottomrule
% \end{tabular}%
% }
% \caption{Mean and standard deviation of each variable importance across quantiles}
% \label{tab:meanstdev}
% \end{table}

% The results of the variable importance measurements show that, on average, the most important variable across all quantile levels is the precipitation variable ($P$), with an average importance of $2.21$ and an average importance of $2.19$ for its square.  The temperature variable ($T$) and its square instead, have an average importance across quantiles equal to $1.96$ and $1.95$, respectively.

% Although the precipitation variable has on average the greatest importance, such importance changes across quantiles more than other variables. This result is corroborated by the value of the standard deviation, which is greater for the precipitation variable than the temperature variable. As a matter of fact, for $\tau=0.05$, temperature and its square have a greater importance with respect to the precipitation variable, whereas for $\tau=0.5$ and $0.75$ precipitation and temperature have nearly the same importance. The greatest difference between the importance of temperature and precipitation variables is observed for $\tau=0.01$ and $0.25$, for which precipitation importance is sensibly greater than temperature importance.

\subsection{Case study 3: Non-linear effects of climate change on economic productivity: ONLY RESULTS}

In this paragraph the ME-QRF model is tested on an extension of the original dataset of \cite{kahn2021long} in order to study the effect of temperature and precipitation on economic productivity of 172 countries. The original dataset consists on yearly observations for 174 countries over the years 1960-2014, but we extend the time span of such dataset by considering data from 1965 to 2021.

In particular, the updated dataset contains 7937 country-year observations for 172 countries.
The outcome variable is the first difference of the natural logarithm of annual real (inflation-adjusted) GDP per capita. The covariates set includes four lags of the outcome variable the current value of yearly temperature and precipitation deviation from their norm along with the related four lags.

The ME-QRF is used to compute quantiles at seven different probability levels $\tau=0.01, 0.05, 0.25, 0.5, 0.75, 0.95, 0.99$ and its performance is compared to the one of a simple LQMM in terms of quantile loss. In particular, the quantile loss is computed on the entire set of in-sample data. The ME-QRF that has been considered in the analysis is:

$$Y_{it}=f(\mathbb{T}^*, \mathbb{P}^*, \mathbb{Y})+\mathbf{b}_i $$

where $Y_{it}$ represents the GDP growth of the $i$-th country in the $t$-th year and $\mathbb{Y}^*=\{Y^*_{i,t-j}\}_{j=1}^{4}$, $\mathbb{T}^*=\{T^*_{i,t-j}\}_{j=0}^{4}$, $\mathbb{P}^*=\{P^*_{i,t-j}\}_{j=0}^{4}$, where $T^*_{it}=T_{it}-\bar{T}_i$ and $P^*_{it}=P_{it}-\bar{P}_i$ represent the deviation of temperature and precipitation from their norm $\bar{T}_i$ and $\bar{P}_i$, respectively. As in \cite{kahn2021long}, the norms are computed as the 30-years moving average of the temperature and precipitation variables. In this application, the random effect $\mathbf{b}_i$ models country-based clustering.
Table \ref{tab:reskahn} reports the average results in terms of in-sample quantile loss across all countries of the two models for the seven quantiles:

% \begin{table}[H]
% \centering
% %\resizebox{\textwidth}{!}
% {%
% \begin{tabular}{@{}cccc@{}}
% \toprule
% $\tau$ & \textbf{PLAIN}                        & \textbf{LQMM}                & \textbf{\%Loss} \\ \midrule
% 0.01   & {\color[HTML]{0F0009} \textbf{0.533}} & {\color[HTML]{0F0009} 2.285} & 23              \\
% 0.05   & {\color[HTML]{0F0009} \textbf{1.553}} & {\color[HTML]{0F0009} 2.492} & 62              \\
% 0.25   & {\color[HTML]{0F0009} \textbf{3.799}} & {\color[HTML]{0F0009} 4.106} & 92              \\
% 0.5    & {\color[HTML]{0F0009} \textbf{4.382}} & {\color[HTML]{0F0009} 4.511} & 97              \\
% 0.75   & {\color[HTML]{0F0009} \textbf{3.469}} & {\color[HTML]{0F0009} 3.854} & 90              \\
% 0.95   & {\color[HTML]{0F0009} \textbf{1.329}} & {\color[HTML]{0F0009} 1.785} & 74              \\
% 0.99   & {\color[HTML]{0F0009} \textbf{0.443}} & {\color[HTML]{0F0009} 1.852} & 24              \\ \bottomrule
% \end{tabular}%
% }
% \caption{\small Results in terms of Quantile Loss for the economic output and climate change dataset. Values in bold indicate the smallest quantile loss related to each quantile. The third column shows the ratio between the ME-QRF and the LQMM quantile losses; values smaller than 100\% indicate that the ME-QRF model outperforms the LQMM model.}
% \label{tab:reskahn}
% \end{table}


Table \ref{tab:heatmap} reports the variable importance for each variables for the seven quantiles.

% \begin{landscape}
% \begin{table}[]
% \centering
% \resizebox{1.5\textwidth}{!}{%
% \begin{tabular}{ccccccccccccccc}
% $\tau$                             & {\color[HTML]{0F0009} \textbf{GDP\_LAG1}}                                 & \textbf{GDP\_LAG2}                                 & \textbf{GDP\_LAG3}                                & \textbf{GDP\_LAG4}                                & \textbf{TMP\_LAG0}                                & \textbf{TMP\_LAG1}                                & \textbf{TMP\_LAG2}                                & \textbf{TMP\_LAG3}                                & \textbf{TMP\_LAG4}                                & \textbf{PRE\_LAG0}                                & {\color[HTML]{0F0009} \textbf{PRE\_LAG1}}                                & \textbf{PRE\_LAG2}                                & \textbf{PRE\_LAG3}                                & \textbf{PRE\_LAG4}                                \\ \cline{2-15} 
% \multicolumn{1}{c|}{\textbf{0.01}} & \multicolumn{1}{c|}{\cellcolor[HTML]{305496}{\color[HTML]{0F0009} 38.51}} & \multicolumn{1}{c|}{\cellcolor[HTML]{7390C7}13.37} & \multicolumn{1}{c|}{\cellcolor[HTML]{8AA5D8}4.54} & \multicolumn{1}{c|}{\cellcolor[HTML]{86A2D5}5.85} & \multicolumn{1}{c|}{\cellcolor[HTML]{92ACDD}2.60} & \multicolumn{1}{c|}{\cellcolor[HTML]{8CA7DA}3.64} & \multicolumn{1}{c|}{\cellcolor[HTML]{8EA9DB}2.86} & \multicolumn{1}{c|}{\cellcolor[HTML]{A8BCE3}1.84} & \multicolumn{1}{c|}{\cellcolor[HTML]{839FD3}6.95} & \multicolumn{1}{c|}{\cellcolor[HTML]{D9E1F2}0.08} & \multicolumn{1}{c|}{\cellcolor[HTML]{D5DEF1}{\color[HTML]{0F0009} 0.26}} & \multicolumn{1}{c|}{\cellcolor[HTML]{D0DAF0}0.42} & \multicolumn{1}{c|}{\cellcolor[HTML]{D0DBF0}0.41} & \multicolumn{1}{c|}{\cellcolor[HTML]{D1DBF0}0.39} \\ \cline{2-15} 
% \multicolumn{1}{c|}{\textbf{0.05}} & \multicolumn{1}{c|}{\cellcolor[HTML]{305496}{\color[HTML]{0F0009} 21.40}} & \multicolumn{1}{c|}{\cellcolor[HTML]{7A97CD}6.69}  & \multicolumn{1}{c|}{\cellcolor[HTML]{88A4D7}3.93} & \multicolumn{1}{c|}{\cellcolor[HTML]{87A3D6}4.07} & \multicolumn{1}{c|}{\cellcolor[HTML]{D9E1F2}1.63} & \multicolumn{1}{c|}{\cellcolor[HTML]{A7BCE3}2.29} & \multicolumn{1}{c|}{\cellcolor[HTML]{B4C5E7}2.13} & \multicolumn{1}{c|}{\cellcolor[HTML]{C0CEEB}1.97} & \multicolumn{1}{c|}{\cellcolor[HTML]{9DB5E0}2.42} & \multicolumn{1}{c|}{\cellcolor[HTML]{94AEDD}2.54} & \multicolumn{1}{c|}{\cellcolor[HTML]{8EA9DB}{\color[HTML]{0F0009} 2.72}} & \multicolumn{1}{c|}{\cellcolor[HTML]{8DA8DB}2.84} & \multicolumn{1}{c|}{\cellcolor[HTML]{8EA9DB}2.69} & \multicolumn{1}{c|}{\cellcolor[HTML]{BFCDEA}1.99} \\ \cline{2-15} 
% \multicolumn{1}{c|}{\textbf{0.5}}  & \multicolumn{1}{c|}{\cellcolor[HTML]{305496}{\color[HTML]{0F0009} 30.20}} & \multicolumn{1}{c|}{\cellcolor[HTML]{7E9BD0}6.81}  & \multicolumn{1}{c|}{\cellcolor[HTML]{86A2D5}4.58} & \multicolumn{1}{c|}{\cellcolor[HTML]{8AA5D8}3.32} & \multicolumn{1}{c|}{\cellcolor[HTML]{C4D1EC}1.24} & \multicolumn{1}{c|}{\cellcolor[HTML]{8AA5D8}3.37} & \multicolumn{1}{c|}{\cellcolor[HTML]{8EA9DB}1.97} & \multicolumn{1}{c|}{\cellcolor[HTML]{92ACDC}1.90} & \multicolumn{1}{c|}{\cellcolor[HTML]{87A2D6}4.29} & \multicolumn{1}{c|}{\cellcolor[HTML]{B0C3E6}1.49} & \multicolumn{1}{c|}{\cellcolor[HTML]{B8C9E8}{\color[HTML]{0F0009} 1.39}} & \multicolumn{1}{c|}{\cellcolor[HTML]{C0CFEB}1.28} & \multicolumn{1}{c|}{\cellcolor[HTML]{D3DDF1}1.03} & \multicolumn{1}{c|}{\cellcolor[HTML]{D9E1F2}0.96} \\ \cline{2-15} 
% \multicolumn{1}{c|}{\textbf{0.75}} & \multicolumn{1}{c|}{\cellcolor[HTML]{305496}{\color[HTML]{0F0009} 29.19}} & \multicolumn{1}{c|}{\cellcolor[HTML]{809CD1}6.36}  & \multicolumn{1}{c|}{\cellcolor[HTML]{86A2D5}4.67} & \multicolumn{1}{c|}{\cellcolor[HTML]{8BA7D9}3.04} & \multicolumn{1}{c|}{\cellcolor[HTML]{C1D0EB}1.43} & \multicolumn{1}{c|}{\cellcolor[HTML]{88A4D7}4.04} & \multicolumn{1}{c|}{\cellcolor[HTML]{8EA9DB}2.25} & \multicolumn{1}{c|}{\cellcolor[HTML]{94ADDD}2.09} & \multicolumn{1}{c|}{\cellcolor[HTML]{88A3D7}4.10} & \multicolumn{1}{c|}{\cellcolor[HTML]{B8C9E8}1.56} & \multicolumn{1}{c|}{\cellcolor[HTML]{CFDAEF}{\color[HTML]{0F0009} 1.23}} & \multicolumn{1}{c|}{\cellcolor[HTML]{D9E1F2}1.09} & \multicolumn{1}{c|}{\cellcolor[HTML]{D7DFF2}1.12} & \multicolumn{1}{c|}{\cellcolor[HTML]{D9E1F2}1.08} \\ \cline{2-15} 
% \multicolumn{1}{c|}{\textbf{0.95}} & \multicolumn{1}{c|}{\cellcolor[HTML]{305496}{\color[HTML]{0F0009} 29.89}} & \multicolumn{1}{c|}{\cellcolor[HTML]{84A0D4}5.38}  & \multicolumn{1}{c|}{\cellcolor[HTML]{84A0D4}5.23} & \multicolumn{1}{c|}{\cellcolor[HTML]{8CA7DA}2.98} & \multicolumn{1}{c|}{\cellcolor[HTML]{D9E1F2}1.02} & \multicolumn{1}{c|}{\cellcolor[HTML]{8AA5D8}3.62} & \multicolumn{1}{c|}{\cellcolor[HTML]{8EA9DB}2.29} & \multicolumn{1}{c|}{\cellcolor[HTML]{A0B6E1}1.95} & \multicolumn{1}{c|}{\cellcolor[HTML]{85A1D4}5.12} & \multicolumn{1}{c|}{\cellcolor[HTML]{93ADDD}2.15} & \multicolumn{1}{c|}{\cellcolor[HTML]{C9D5ED}{\color[HTML]{0F0009} 1.29}} & \multicolumn{1}{c|}{\cellcolor[HTML]{BDCCEA}1.48} & \multicolumn{1}{c|}{\cellcolor[HTML]{C9D5EE}1.28} & \multicolumn{1}{c|}{\cellcolor[HTML]{CCD7EE}1.24} \\ \cline{2-15} 
% \multicolumn{1}{c|}{\textbf{0.99}} & \multicolumn{1}{c|}{\cellcolor[HTML]{305496}{\color[HTML]{0F0009} 25.31}} & \multicolumn{1}{c|}{\cellcolor[HTML]{7F9CD0}5.43}  & \multicolumn{1}{c|}{\cellcolor[HTML]{7C98CE}6.42} & \multicolumn{1}{c|}{\cellcolor[HTML]{87A3D6}3.42} & \multicolumn{1}{c|}{\cellcolor[HTML]{B4C5E7}1.37} & \multicolumn{1}{c|}{\cellcolor[HTML]{8AA5D8}2.84} & \multicolumn{1}{c|}{\cellcolor[HTML]{9EB5E0}1.53} & \multicolumn{1}{c|}{\cellcolor[HTML]{C7D4ED}1.22} & \multicolumn{1}{c|}{\cellcolor[HTML]{8CA7DA}2.31} & \multicolumn{1}{c|}{\cellcolor[HTML]{ADC0E5}1.42} & \multicolumn{1}{c|}{\cellcolor[HTML]{8EA9DB}{\color[HTML]{0F0009} 1.70}} & \multicolumn{1}{c|}{\cellcolor[HTML]{96AFDE}1.59} & \multicolumn{1}{c|}{\cellcolor[HTML]{AEC1E5}1.41} & \multicolumn{1}{c|}{\cellcolor[HTML]{D9E1F2}1.08} \\ \cline{2-15} 
% \textbf{MEAN}                      & \cellcolor[HTML]{305496}{\color[HTML]{0F0009} 29.09}                      & \cellcolor[HTML]{7C99CE}7.34                       & \cellcolor[HTML]{84A0D4}4.89                      & \cellcolor[HTML]{88A4D7}3.78                      & \cellcolor[HTML]{B5C6E7}1.55                      & \cellcolor[HTML]{8AA5D8}3.30                      & \cellcolor[HTML]{8EA9DB}2.17                      & \cellcolor[HTML]{9DB4E0}1.83                      & \cellcolor[HTML]{87A3D6}4.20                      & \cellcolor[HTML]{B6C7E8}1.54                      & \cellcolor[HTML]{BFCEEA}{\color[HTML]{0F0009} 1.43}                      & \cellcolor[HTML]{BECDEA}1.45                      & \cellcolor[HTML]{C8D5ED}1.32                      & \cellcolor[HTML]{D9E1F2}1.12                     
% \end{tabular}%
% }
% \caption{Variable importance of each covariate. Darker shades of blue indicate a higher variable importance for each quantile.}
% \label{tab:heatmap}
% \end{table}
% \end{landscape}


