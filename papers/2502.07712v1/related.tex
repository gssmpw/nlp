\section[Related]{Related Work}

\label{sec:relatedwork}
\subsection{Unit Testing using Mocks}
Unit testing stands as a fundamental practice in software development,
aimed at evaluating each functionality of the software independently and uncovering bugs early in the development cycle.
Due to dependencies, testing the code in isolation becomes challenging.
To tackle this challenge, a technique known as mock objects have been proposed in the past by Mackinnon \etal~\cite{mackinnon2000endotesting} for unit testing, involving the replacement of dependencies with dummy implementations.
Their findings suggest that using mock objects for unit tests improves test quality and the structure of both domain and test code.
Prior studies have shown the benefits of using mock objects for unit testing various applications, including servlet~\cite{thomas2002mock}, multi-agent systems~\cite{coelho2006unit}, mobile apps~\cite{fazzini2020framework}, and database applications~\cite{taneja2010moda}.
However, the use of mock objects for testing DL applications has not been investigated before.


\subsection{Fault Localization and Bug Repair in DL Programs}
The rise in DL application usage has led researchers to adapt fault localization techniques to validate DL systems and identify faulty behaviors. In the past, various static and dynamic analysis approaches were proposed for DL programs.
\nlint~\cite{nikanjam2021neuralint} is a static analysis approach for automatic fault detection using predefined rules.
UMLAUT~\cite{schoop2021umlaut} combines static and dynamic analysis to examine program structure before training and model behavior during training.
DeepLocalize~\cite{wardat21DeepLocalize} is a dynamic fault localization approach that identifies 
numerical errors during training.
AutoTrainer~\cite{Zhang21Autotrainer} is a system designed to identify and repair 5 common training issues in DL models. 
DeepDiagnosis~\cite{wardat22DeepDiagnosis} is a dynamic 
technique that identifies various symptoms during training and suggests actionable fixes.
DeepFD~\cite{cao2022deepfd} is a learning-based framework for fault diagnosis.
TheDeepChecker \cite{BraiekDeepChecker} is a property-based debugging approach that detects bugs before, during, and after training.
deepmufl~\cite{ghanbari2023deepmufl} is a mutation-based fault localization approach that generates mutants of pre-trained models to detect bugs.
Prior works treat data preprocessing steps and the model as a comprehensive DL program, 
and identify and localize bugs by monitoring the training process.
In contrast, {\em KUnit} treats data and the model as independent entities and aims to detect bugs before integrating them.
