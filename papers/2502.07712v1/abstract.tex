
\begin{abstract}
While deep learning (DL) has permeated, and become an integral component of many critical software systems, today software engineering research hasn't explored how to separately test data and models that are integral for DL approaches to work effectively. 
The main challenge in independently testing these components arises from the tight dependency between data and models.
This research explores this gap, introducing our methodology of {\em mock deep testing}
for unit testing of DL applications. 
To enable unit testing, we introduce a design paradigm that decomposes the workflow into distinct, manageable components, minimizes sequential dependencies, and modularizes key stages of the DL, including data preparation and model design. For unit testing these components, we propose modeling their dependencies using mocks.
In the context of DL, mocks refer to mock data and mock model that mimic the behavior of the original data and model, respectively.
This modular approach facilitates independent development and testing of the components, ensuring comprehensive quality assurance throughout the development process.
We have developed {\em KUnit}, a framework for enabling mock deep testing
for the Keras library, a popular library for developing DL applications. We empirically evaluated {\em {KUnit}} to determine the effectiveness of mocks in independently testing data and models. Our assessment of 50 DL programs obtained from \sof and \textit{GitHub} shows that mocks effectively identified 10 issues in the data preparation stage 
and 53 issues in the model design stage.
We also conducted a user study with 36 participants using {\em KUnit} to perceive the effectiveness of our approach.
Participants using {\em KUnit} successfully resolved 25 issues in the data preparation stage
and 38 issues in the model design stage.
Our findings highlight that mock objects provide a lightweight emulation of the dependencies for unit testing, facilitating early bug detection.
Lastly, to evaluate the
usability of {\em KUnit}, we conducted a post-study survey. The results reveal that {\em KUnit} is helpful to DL application developers, enabling them to independently test each component (data and model) and resolve issues effectively in different stages.

\end{abstract}

