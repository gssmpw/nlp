%\uvp{Think about a better branding for our taxonomy citizens than `type'.} 

%\uvp{
%From Michael (we don't have to do everything):
%add reference to genre, register;
%to communicative goals (gamon and halliday); `communicative goals';
%small diagram opposite the abstract;
%shorten 3.4;
%rethink tables 4, 5, and accompanying text;
%for future work---use rhetorical structure to identify helpful reviews.
%}



A body of text is written with a purpose of conveying a message to its reader.
Information within the text is expressed in various formats and styles in order to fulfill this goal \citep{levy1979CommunicativeGoals}.
By identifying these methods of communication, a reader can focus on information of interest, or, conversely, obtain a clearer picture of the holistic intent of the full text.
In automatic applications, this is commonly accomplished by means of text classification that categorizes types of text spans.
%\rl{it reads as though a reader needs text classifiers for categorizing text spans}
%\uvp{I'm in favor of removing that last sentence altogether}

Types of text differ across domains and communicative objectives.
For example, media coverage or debate statements attempt to convince of an ideological stance by using different frame typologies like `morality' or `legality'~\citep{naderi2017newsFrames, ali2022framingSurvey}.
Scientific articles contain snippets of texts purposed for explaining the discussed matter, like `background' or `results'~\citep{dernoncourt2017pubmed, cohan2019articleClassification}.
Product reviews often differ in sentiment to express an opinion regarding an aspect of the product~\citep{serranoguerrero2015sentAna, Yadav2020sentAna}.
Additionally, the length of a text influences its allowance for verbosity and detail~\citep{louis2014summaryContent}, and its rhetorical structure allows for a coherent arrangement of ideas within the text~\citep{mann1988rst}.
These characteristics, and others, provide means for classifying and analyzing text, at the levels of words, sentences, or documents.

In the e-commerce domain, product reviews assist a potential customer in learning about a product.
Reviewers share their experience by providing opinions and describing different aspects of the product.
They can suggest how to use the product, or compare it to alternative products. 
While doing so, they may vary their writing style to emphasize their viewpoint or argue for it.
% Clearly, the variability of information types in product reviews is quite large.
We suggest that systematic identification of these facets can unlock novel explainable analyses of downstream applications.
%. We suggest that identifying these types opens the door to useful downstream applications and explainable analyses that unlock a wealth of new information from within reviews.

\begin{figure}[t]
  \centering
  \includegraphics[width=0.9\linewidth]{Figures/img_example.png}
  \caption{A sentence from a review, and the dominant information types communicated in it.}
  \label{fig_example}
\end{figure}


Content realization in product reviews has been extensively studied through, for example, aspect extraction and sentiment analysis~\citep{Tubishat2018aspExtr, xu2019bertSentAna}, but very little research has centered on \textit{types} of content within reviews, excluding `tips' and `arguments'~\citep{hirsch2021producttips, liu2017argsForHelpfullness}.
We introduce a \textit{broad} typology of sentence types for product reviews (\S{\ref{sec_taxonomy}}) consisting of 24 classes, including both functional types (e.g.,~`opinion', `product usage') as well as markers of linguistic style (e.g.,~`sarcasm', `imagery').
\autoref{fig_example} presents an illustration of an example review sentence with some of its plausible types.
More examples are available in \autoref{tab_example_annotations} in the appendix.
% The classes are non-independent, and a sentence can be labeled with multiple classes.
% For example, the sentence:
% \begin{table}[h]
%     \vspace{-2mm}
%     %\centering
%     \resizebox{\columnwidth}{!}{
%     \begin{tabular}{|l|}
%         \hline
%         \textsl{
%         \begin{tabular}[c]{@{}l@{}}For \$8, you cannot go wrong and this is a must\\have item if you are a frequent traveler.\end{tabular}} \\
%         \hline
%     \end{tabular}}
%     \vspace{-4mm}
% \end{table}

% \noindent
% discusses the \textit{price} and \textit{product usage}, poses a \textit{speculation}, and states an \textit{opinion}.
% See more examples in \autoref{tab_example_annotations}. 

%We automatically annotate the taxonomy by prompting a large language model~\citep[Flan-T5-xxl][]{wei2022flan} to effectively classify sentences to their relevant types (\S{\ref{sec_prediction}}) in a zero-shot setting. We designed and tuned this model using two small-scale high-quality sets of labeled sentences.
%We then apply the model to product reviews and their product-level summaries in order to analyze the behavior of information variation (\S{\ref{sec_analysis}}). By segmenting review data according to different fragmentations, like helpfulness and text length, we can perceive what types of content are more serviceable for potential customers. We conclude by discussing some implications of the knowledge acquired from the analysis facilitated by the taxonomy (\S{\ref{sec_conclusion}}), including use cases and applications.
%\uvp{`We will release our seed dataset of labeled examples upon publication?' Or are we doing this in an appendix? What about the prompts?}

We conduct large-scale experimentation and analysis of our typology on product reviews by leveraging zero-shot capabilities of a large language model, \texttt{flan-T5-xxl}~\citep{wei2022flan} (\S{\ref{sec_prediction}}).
We then demonstrate how sentence type labels alone, without further access to the text itself, provide a strong signal for major downstream tasks, including review helpfulness and sentiment analysis (\S{\ref{sec_experiments}}).
We show that revealing the types of information present in reviews and their summaries provides 
%a bounty of explainable insights
insight into the structure of subjective text and the nature of information readers find useful (\S{\ref{sec_analysis}}).
The enriched understanding of such texts can in turn lead to improved consumption and enhanced modeling of relevant applications.
%\rl{maybe say: several explainable insights such as <example>} 
%that assists the reader to better capitalize on these texts (\S{\ref{sec_analysis}}). \rl{are these insights for the reader directly or for improved understanding of such texts that may lead to improved modeling}


%- the zero-shot format facilitates quick and easy prediction of many classes with very little data resources, so why not use it to enrich the meta-information about reviews


%- Emphasize that the taxonomy is important on its own, and that the LLM predictor is just a means to an end. Of course we could test the amazing LLM on helpfulness and sentiment, but that's not the point. Assuming there was an oracle predictor for the taxonomy, we show what capabilities are made available.